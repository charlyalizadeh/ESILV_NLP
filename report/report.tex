\documentclass[a4paper, 12pt, one column]{article}

\usepackage[english]{babel}
\usepackage[utf8x]{inputenc}
\usepackage[T1]{fontenc}
\usepackage{tikz}
\usepackage{xcolor}
\usepackage{subfig}
\usepackage{caption}
\usepackage{float}
\usepackage{float}
\usepackage[top=1.3cm, bottom=2.0cm, outer=2.5cm, inner=2.5cm, heightrounded,
marginparwidth=1.5cm, marginparsep=0.4cm, margin=2.5cm]{geometry}
\usepackage{graphicx} 
\usepackage{hyperref} 
\usepackage{amsmath} 
\usepackage{amsfonts}
\usepackage{amssymb} 
\usepackage{multirow}
\usepackage{layouts}
\usepackage[nameinlink]{cleveref}
%\usepackage{listings}
\usepackage{listingsutf8}
\crefdefaultlabelformat{#2#1#3}
\graphicspath{{images/}}

\renewcommand{\partname}{}
\renewcommand{\thepart}{}

\lstset{basicstyle=\ttfamily, keywordstyle=\bfseries, inputencoding=utf8/latin1}

\begin{document}
\input{titlepage}
\newpage


\section{Data}

\subsection{Cleaning}

The only cleaning needed was to convert the \lstinline{date} into \lstinline{Datetime} object.
All the dates had the same format, for example the first row had the following string for the \lstinline{date} column: \textit{"06 septembre 2021 suite à une expérience en septembre 2021"}.
To convert it into \lstinline[language=python]{Datetime} object we first replace the french months by their respective two digit format (\lstinline{janvier: "01", fevrier: "02", etc}), then we strip the begining trailing whitespaces and only keep the first 10 characters. Then we use the \lstinline{pandas}' function \lstinline{to_datetime()} to convert the column.


\subsection{Exploration}

First we looked into the stars distribution for all the assureur (\cref{fig:distrib}).


\begin{figure}[H]
    \centering
    %% Creator: Matplotlib, PGF backend
%%
%% To include the figure in your LaTeX document, write
%%   \input{<filename>.pgf}
%%
%% Make sure the required packages are loaded in your preamble
%%   \usepackage{pgf}
%%
%% Also ensure that all the required font packages are loaded; for instance,
%% the lmodern package is sometimes necessary when using math font.
%%   \usepackage{lmodern}
%%
%% Figures using additional raster images can only be included by \input if
%% they are in the same directory as the main LaTeX file. For loading figures
%% from other directories you can use the `import` package
%%   \usepackage{import}
%%
%% and then include the figures with
%%   \import{<path to file>}{<filename>.pgf}
%%
%% Matplotlib used the following preamble
%%
\begingroup%
\makeatletter%
\begin{pgfpicture}%
\pgfpathrectangle{\pgfpointorigin}{\pgfqpoint{5.404013in}{4.449691in}}%
\pgfusepath{use as bounding box, clip}%
\begin{pgfscope}%
\pgfsetbuttcap%
\pgfsetmiterjoin%
\definecolor{currentfill}{rgb}{1.000000,1.000000,1.000000}%
\pgfsetfillcolor{currentfill}%
\pgfsetlinewidth{0.000000pt}%
\definecolor{currentstroke}{rgb}{1.000000,1.000000,1.000000}%
\pgfsetstrokecolor{currentstroke}%
\pgfsetdash{}{0pt}%
\pgfpathmoveto{\pgfqpoint{0.000000in}{0.000000in}}%
\pgfpathlineto{\pgfqpoint{5.404013in}{0.000000in}}%
\pgfpathlineto{\pgfqpoint{5.404013in}{4.449691in}}%
\pgfpathlineto{\pgfqpoint{0.000000in}{4.449691in}}%
\pgfpathlineto{\pgfqpoint{0.000000in}{0.000000in}}%
\pgfpathclose%
\pgfusepath{fill}%
\end{pgfscope}%
\begin{pgfscope}%
\pgfsetbuttcap%
\pgfsetmiterjoin%
\definecolor{currentfill}{rgb}{1.000000,1.000000,1.000000}%
\pgfsetfillcolor{currentfill}%
\pgfsetlinewidth{0.000000pt}%
\definecolor{currentstroke}{rgb}{0.000000,0.000000,0.000000}%
\pgfsetstrokecolor{currentstroke}%
\pgfsetstrokeopacity{0.000000}%
\pgfsetdash{}{0pt}%
\pgfpathmoveto{\pgfqpoint{0.654013in}{0.499691in}}%
\pgfpathlineto{\pgfqpoint{5.304013in}{0.499691in}}%
\pgfpathlineto{\pgfqpoint{5.304013in}{4.349691in}}%
\pgfpathlineto{\pgfqpoint{0.654013in}{4.349691in}}%
\pgfpathlineto{\pgfqpoint{0.654013in}{0.499691in}}%
\pgfpathclose%
\pgfusepath{fill}%
\end{pgfscope}%
\begin{pgfscope}%
\pgfpathrectangle{\pgfqpoint{0.654013in}{0.499691in}}{\pgfqpoint{4.650000in}{3.850000in}}%
\pgfusepath{clip}%
\pgfsetbuttcap%
\pgfsetmiterjoin%
\definecolor{currentfill}{rgb}{0.121569,0.466667,0.705882}%
\pgfsetfillcolor{currentfill}%
\pgfsetfillopacity{0.750000}%
\pgfsetlinewidth{1.003750pt}%
\definecolor{currentstroke}{rgb}{0.000000,0.000000,0.000000}%
\pgfsetstrokecolor{currentstroke}%
\pgfsetdash{}{0pt}%
\pgfpathmoveto{\pgfqpoint{0.865377in}{0.499691in}}%
\pgfpathlineto{\pgfqpoint{1.710831in}{0.499691in}}%
\pgfpathlineto{\pgfqpoint{1.710831in}{4.166358in}}%
\pgfpathlineto{\pgfqpoint{0.865377in}{4.166358in}}%
\pgfpathlineto{\pgfqpoint{0.865377in}{0.499691in}}%
\pgfpathclose%
\pgfusepath{stroke,fill}%
\end{pgfscope}%
\begin{pgfscope}%
\pgfpathrectangle{\pgfqpoint{0.654013in}{0.499691in}}{\pgfqpoint{4.650000in}{3.850000in}}%
\pgfusepath{clip}%
\pgfsetbuttcap%
\pgfsetmiterjoin%
\definecolor{currentfill}{rgb}{0.121569,0.466667,0.705882}%
\pgfsetfillcolor{currentfill}%
\pgfsetfillopacity{0.750000}%
\pgfsetlinewidth{1.003750pt}%
\definecolor{currentstroke}{rgb}{0.000000,0.000000,0.000000}%
\pgfsetstrokecolor{currentstroke}%
\pgfsetdash{}{0pt}%
\pgfpathmoveto{\pgfqpoint{1.710831in}{0.499691in}}%
\pgfpathlineto{\pgfqpoint{2.556286in}{0.499691in}}%
\pgfpathlineto{\pgfqpoint{2.556286in}{2.374124in}}%
\pgfpathlineto{\pgfqpoint{1.710831in}{2.374124in}}%
\pgfpathlineto{\pgfqpoint{1.710831in}{0.499691in}}%
\pgfpathclose%
\pgfusepath{stroke,fill}%
\end{pgfscope}%
\begin{pgfscope}%
\pgfpathrectangle{\pgfqpoint{0.654013in}{0.499691in}}{\pgfqpoint{4.650000in}{3.850000in}}%
\pgfusepath{clip}%
\pgfsetbuttcap%
\pgfsetmiterjoin%
\definecolor{currentfill}{rgb}{0.121569,0.466667,0.705882}%
\pgfsetfillcolor{currentfill}%
\pgfsetfillopacity{0.750000}%
\pgfsetlinewidth{1.003750pt}%
\definecolor{currentstroke}{rgb}{0.000000,0.000000,0.000000}%
\pgfsetstrokecolor{currentstroke}%
\pgfsetdash{}{0pt}%
\pgfpathmoveto{\pgfqpoint{2.556286in}{0.499691in}}%
\pgfpathlineto{\pgfqpoint{3.401740in}{0.499691in}}%
\pgfpathlineto{\pgfqpoint{3.401740in}{2.205188in}}%
\pgfpathlineto{\pgfqpoint{2.556286in}{2.205188in}}%
\pgfpathlineto{\pgfqpoint{2.556286in}{0.499691in}}%
\pgfpathclose%
\pgfusepath{stroke,fill}%
\end{pgfscope}%
\begin{pgfscope}%
\pgfpathrectangle{\pgfqpoint{0.654013in}{0.499691in}}{\pgfqpoint{4.650000in}{3.850000in}}%
\pgfusepath{clip}%
\pgfsetbuttcap%
\pgfsetmiterjoin%
\definecolor{currentfill}{rgb}{0.121569,0.466667,0.705882}%
\pgfsetfillcolor{currentfill}%
\pgfsetfillopacity{0.750000}%
\pgfsetlinewidth{1.003750pt}%
\definecolor{currentstroke}{rgb}{0.000000,0.000000,0.000000}%
\pgfsetstrokecolor{currentstroke}%
\pgfsetdash{}{0pt}%
\pgfpathmoveto{\pgfqpoint{3.401740in}{0.499691in}}%
\pgfpathlineto{\pgfqpoint{4.247195in}{0.499691in}}%
\pgfpathlineto{\pgfqpoint{4.247195in}{2.963130in}}%
\pgfpathlineto{\pgfqpoint{3.401740in}{2.963130in}}%
\pgfpathlineto{\pgfqpoint{3.401740in}{0.499691in}}%
\pgfpathclose%
\pgfusepath{stroke,fill}%
\end{pgfscope}%
\begin{pgfscope}%
\pgfpathrectangle{\pgfqpoint{0.654013in}{0.499691in}}{\pgfqpoint{4.650000in}{3.850000in}}%
\pgfusepath{clip}%
\pgfsetbuttcap%
\pgfsetmiterjoin%
\definecolor{currentfill}{rgb}{0.121569,0.466667,0.705882}%
\pgfsetfillcolor{currentfill}%
\pgfsetfillopacity{0.750000}%
\pgfsetlinewidth{1.003750pt}%
\definecolor{currentstroke}{rgb}{0.000000,0.000000,0.000000}%
\pgfsetstrokecolor{currentstroke}%
\pgfsetdash{}{0pt}%
\pgfpathmoveto{\pgfqpoint{4.247195in}{0.499691in}}%
\pgfpathlineto{\pgfqpoint{5.092649in}{0.499691in}}%
\pgfpathlineto{\pgfqpoint{5.092649in}{2.945480in}}%
\pgfpathlineto{\pgfqpoint{4.247195in}{2.945480in}}%
\pgfpathlineto{\pgfqpoint{4.247195in}{0.499691in}}%
\pgfpathclose%
\pgfusepath{stroke,fill}%
\end{pgfscope}%
\begin{pgfscope}%
\pgfsetbuttcap%
\pgfsetroundjoin%
\definecolor{currentfill}{rgb}{0.000000,0.000000,0.000000}%
\pgfsetfillcolor{currentfill}%
\pgfsetlinewidth{0.803000pt}%
\definecolor{currentstroke}{rgb}{0.000000,0.000000,0.000000}%
\pgfsetstrokecolor{currentstroke}%
\pgfsetdash{}{0pt}%
\pgfsys@defobject{currentmarker}{\pgfqpoint{0.000000in}{-0.048611in}}{\pgfqpoint{0.000000in}{0.000000in}}{%
\pgfpathmoveto{\pgfqpoint{0.000000in}{0.000000in}}%
\pgfpathlineto{\pgfqpoint{0.000000in}{-0.048611in}}%
\pgfusepath{stroke,fill}%
}%
\begin{pgfscope}%
\pgfsys@transformshift{1.288104in}{0.499691in}%
\pgfsys@useobject{currentmarker}{}%
\end{pgfscope}%
\end{pgfscope}%
\begin{pgfscope}%
\definecolor{textcolor}{rgb}{0.000000,0.000000,0.000000}%
\pgfsetstrokecolor{textcolor}%
\pgfsetfillcolor{textcolor}%
\pgftext[x=1.288104in,y=0.402469in,,top]{\color{textcolor}\rmfamily\fontsize{10.000000}{12.000000}\selectfont \(\displaystyle {1}\)}%
\end{pgfscope}%
\begin{pgfscope}%
\pgfsetbuttcap%
\pgfsetroundjoin%
\definecolor{currentfill}{rgb}{0.000000,0.000000,0.000000}%
\pgfsetfillcolor{currentfill}%
\pgfsetlinewidth{0.803000pt}%
\definecolor{currentstroke}{rgb}{0.000000,0.000000,0.000000}%
\pgfsetstrokecolor{currentstroke}%
\pgfsetdash{}{0pt}%
\pgfsys@defobject{currentmarker}{\pgfqpoint{0.000000in}{-0.048611in}}{\pgfqpoint{0.000000in}{0.000000in}}{%
\pgfpathmoveto{\pgfqpoint{0.000000in}{0.000000in}}%
\pgfpathlineto{\pgfqpoint{0.000000in}{-0.048611in}}%
\pgfusepath{stroke,fill}%
}%
\begin{pgfscope}%
\pgfsys@transformshift{2.133559in}{0.499691in}%
\pgfsys@useobject{currentmarker}{}%
\end{pgfscope}%
\end{pgfscope}%
\begin{pgfscope}%
\definecolor{textcolor}{rgb}{0.000000,0.000000,0.000000}%
\pgfsetstrokecolor{textcolor}%
\pgfsetfillcolor{textcolor}%
\pgftext[x=2.133559in,y=0.402469in,,top]{\color{textcolor}\rmfamily\fontsize{10.000000}{12.000000}\selectfont \(\displaystyle {2}\)}%
\end{pgfscope}%
\begin{pgfscope}%
\pgfsetbuttcap%
\pgfsetroundjoin%
\definecolor{currentfill}{rgb}{0.000000,0.000000,0.000000}%
\pgfsetfillcolor{currentfill}%
\pgfsetlinewidth{0.803000pt}%
\definecolor{currentstroke}{rgb}{0.000000,0.000000,0.000000}%
\pgfsetstrokecolor{currentstroke}%
\pgfsetdash{}{0pt}%
\pgfsys@defobject{currentmarker}{\pgfqpoint{0.000000in}{-0.048611in}}{\pgfqpoint{0.000000in}{0.000000in}}{%
\pgfpathmoveto{\pgfqpoint{0.000000in}{0.000000in}}%
\pgfpathlineto{\pgfqpoint{0.000000in}{-0.048611in}}%
\pgfusepath{stroke,fill}%
}%
\begin{pgfscope}%
\pgfsys@transformshift{2.979013in}{0.499691in}%
\pgfsys@useobject{currentmarker}{}%
\end{pgfscope}%
\end{pgfscope}%
\begin{pgfscope}%
\definecolor{textcolor}{rgb}{0.000000,0.000000,0.000000}%
\pgfsetstrokecolor{textcolor}%
\pgfsetfillcolor{textcolor}%
\pgftext[x=2.979013in,y=0.402469in,,top]{\color{textcolor}\rmfamily\fontsize{10.000000}{12.000000}\selectfont \(\displaystyle {3}\)}%
\end{pgfscope}%
\begin{pgfscope}%
\pgfsetbuttcap%
\pgfsetroundjoin%
\definecolor{currentfill}{rgb}{0.000000,0.000000,0.000000}%
\pgfsetfillcolor{currentfill}%
\pgfsetlinewidth{0.803000pt}%
\definecolor{currentstroke}{rgb}{0.000000,0.000000,0.000000}%
\pgfsetstrokecolor{currentstroke}%
\pgfsetdash{}{0pt}%
\pgfsys@defobject{currentmarker}{\pgfqpoint{0.000000in}{-0.048611in}}{\pgfqpoint{0.000000in}{0.000000in}}{%
\pgfpathmoveto{\pgfqpoint{0.000000in}{0.000000in}}%
\pgfpathlineto{\pgfqpoint{0.000000in}{-0.048611in}}%
\pgfusepath{stroke,fill}%
}%
\begin{pgfscope}%
\pgfsys@transformshift{3.824468in}{0.499691in}%
\pgfsys@useobject{currentmarker}{}%
\end{pgfscope}%
\end{pgfscope}%
\begin{pgfscope}%
\definecolor{textcolor}{rgb}{0.000000,0.000000,0.000000}%
\pgfsetstrokecolor{textcolor}%
\pgfsetfillcolor{textcolor}%
\pgftext[x=3.824468in,y=0.402469in,,top]{\color{textcolor}\rmfamily\fontsize{10.000000}{12.000000}\selectfont \(\displaystyle {4}\)}%
\end{pgfscope}%
\begin{pgfscope}%
\pgfsetbuttcap%
\pgfsetroundjoin%
\definecolor{currentfill}{rgb}{0.000000,0.000000,0.000000}%
\pgfsetfillcolor{currentfill}%
\pgfsetlinewidth{0.803000pt}%
\definecolor{currentstroke}{rgb}{0.000000,0.000000,0.000000}%
\pgfsetstrokecolor{currentstroke}%
\pgfsetdash{}{0pt}%
\pgfsys@defobject{currentmarker}{\pgfqpoint{0.000000in}{-0.048611in}}{\pgfqpoint{0.000000in}{0.000000in}}{%
\pgfpathmoveto{\pgfqpoint{0.000000in}{0.000000in}}%
\pgfpathlineto{\pgfqpoint{0.000000in}{-0.048611in}}%
\pgfusepath{stroke,fill}%
}%
\begin{pgfscope}%
\pgfsys@transformshift{4.669922in}{0.499691in}%
\pgfsys@useobject{currentmarker}{}%
\end{pgfscope}%
\end{pgfscope}%
\begin{pgfscope}%
\definecolor{textcolor}{rgb}{0.000000,0.000000,0.000000}%
\pgfsetstrokecolor{textcolor}%
\pgfsetfillcolor{textcolor}%
\pgftext[x=4.669922in,y=0.402469in,,top]{\color{textcolor}\rmfamily\fontsize{10.000000}{12.000000}\selectfont \(\displaystyle {5}\)}%
\end{pgfscope}%
\begin{pgfscope}%
\definecolor{textcolor}{rgb}{0.000000,0.000000,0.000000}%
\pgfsetstrokecolor{textcolor}%
\pgfsetfillcolor{textcolor}%
\pgftext[x=2.979013in,y=0.223457in,,top]{\color{textcolor}\rmfamily\fontsize{10.000000}{12.000000}\selectfont note}%
\end{pgfscope}%
\begin{pgfscope}%
\pgfsetbuttcap%
\pgfsetroundjoin%
\definecolor{currentfill}{rgb}{0.000000,0.000000,0.000000}%
\pgfsetfillcolor{currentfill}%
\pgfsetlinewidth{0.803000pt}%
\definecolor{currentstroke}{rgb}{0.000000,0.000000,0.000000}%
\pgfsetstrokecolor{currentstroke}%
\pgfsetdash{}{0pt}%
\pgfsys@defobject{currentmarker}{\pgfqpoint{-0.048611in}{0.000000in}}{\pgfqpoint{-0.000000in}{0.000000in}}{%
\pgfpathmoveto{\pgfqpoint{-0.000000in}{0.000000in}}%
\pgfpathlineto{\pgfqpoint{-0.048611in}{0.000000in}}%
\pgfusepath{stroke,fill}%
}%
\begin{pgfscope}%
\pgfsys@transformshift{0.654013in}{0.499691in}%
\pgfsys@useobject{currentmarker}{}%
\end{pgfscope}%
\end{pgfscope}%
\begin{pgfscope}%
\definecolor{textcolor}{rgb}{0.000000,0.000000,0.000000}%
\pgfsetstrokecolor{textcolor}%
\pgfsetfillcolor{textcolor}%
\pgftext[x=0.487346in, y=0.451466in, left, base]{\color{textcolor}\rmfamily\fontsize{10.000000}{12.000000}\selectfont \(\displaystyle {0}\)}%
\end{pgfscope}%
\begin{pgfscope}%
\pgfsetbuttcap%
\pgfsetroundjoin%
\definecolor{currentfill}{rgb}{0.000000,0.000000,0.000000}%
\pgfsetfillcolor{currentfill}%
\pgfsetlinewidth{0.803000pt}%
\definecolor{currentstroke}{rgb}{0.000000,0.000000,0.000000}%
\pgfsetstrokecolor{currentstroke}%
\pgfsetdash{}{0pt}%
\pgfsys@defobject{currentmarker}{\pgfqpoint{-0.048611in}{0.000000in}}{\pgfqpoint{-0.000000in}{0.000000in}}{%
\pgfpathmoveto{\pgfqpoint{-0.000000in}{0.000000in}}%
\pgfpathlineto{\pgfqpoint{-0.048611in}{0.000000in}}%
\pgfusepath{stroke,fill}%
}%
\begin{pgfscope}%
\pgfsys@transformshift{0.654013in}{1.003978in}%
\pgfsys@useobject{currentmarker}{}%
\end{pgfscope}%
\end{pgfscope}%
\begin{pgfscope}%
\definecolor{textcolor}{rgb}{0.000000,0.000000,0.000000}%
\pgfsetstrokecolor{textcolor}%
\pgfsetfillcolor{textcolor}%
\pgftext[x=0.279012in, y=0.955752in, left, base]{\color{textcolor}\rmfamily\fontsize{10.000000}{12.000000}\selectfont \(\displaystyle {1000}\)}%
\end{pgfscope}%
\begin{pgfscope}%
\pgfsetbuttcap%
\pgfsetroundjoin%
\definecolor{currentfill}{rgb}{0.000000,0.000000,0.000000}%
\pgfsetfillcolor{currentfill}%
\pgfsetlinewidth{0.803000pt}%
\definecolor{currentstroke}{rgb}{0.000000,0.000000,0.000000}%
\pgfsetstrokecolor{currentstroke}%
\pgfsetdash{}{0pt}%
\pgfsys@defobject{currentmarker}{\pgfqpoint{-0.048611in}{0.000000in}}{\pgfqpoint{-0.000000in}{0.000000in}}{%
\pgfpathmoveto{\pgfqpoint{-0.000000in}{0.000000in}}%
\pgfpathlineto{\pgfqpoint{-0.048611in}{0.000000in}}%
\pgfusepath{stroke,fill}%
}%
\begin{pgfscope}%
\pgfsys@transformshift{0.654013in}{1.508264in}%
\pgfsys@useobject{currentmarker}{}%
\end{pgfscope}%
\end{pgfscope}%
\begin{pgfscope}%
\definecolor{textcolor}{rgb}{0.000000,0.000000,0.000000}%
\pgfsetstrokecolor{textcolor}%
\pgfsetfillcolor{textcolor}%
\pgftext[x=0.279012in, y=1.460039in, left, base]{\color{textcolor}\rmfamily\fontsize{10.000000}{12.000000}\selectfont \(\displaystyle {2000}\)}%
\end{pgfscope}%
\begin{pgfscope}%
\pgfsetbuttcap%
\pgfsetroundjoin%
\definecolor{currentfill}{rgb}{0.000000,0.000000,0.000000}%
\pgfsetfillcolor{currentfill}%
\pgfsetlinewidth{0.803000pt}%
\definecolor{currentstroke}{rgb}{0.000000,0.000000,0.000000}%
\pgfsetstrokecolor{currentstroke}%
\pgfsetdash{}{0pt}%
\pgfsys@defobject{currentmarker}{\pgfqpoint{-0.048611in}{0.000000in}}{\pgfqpoint{-0.000000in}{0.000000in}}{%
\pgfpathmoveto{\pgfqpoint{-0.000000in}{0.000000in}}%
\pgfpathlineto{\pgfqpoint{-0.048611in}{0.000000in}}%
\pgfusepath{stroke,fill}%
}%
\begin{pgfscope}%
\pgfsys@transformshift{0.654013in}{2.012550in}%
\pgfsys@useobject{currentmarker}{}%
\end{pgfscope}%
\end{pgfscope}%
\begin{pgfscope}%
\definecolor{textcolor}{rgb}{0.000000,0.000000,0.000000}%
\pgfsetstrokecolor{textcolor}%
\pgfsetfillcolor{textcolor}%
\pgftext[x=0.279012in, y=1.964325in, left, base]{\color{textcolor}\rmfamily\fontsize{10.000000}{12.000000}\selectfont \(\displaystyle {3000}\)}%
\end{pgfscope}%
\begin{pgfscope}%
\pgfsetbuttcap%
\pgfsetroundjoin%
\definecolor{currentfill}{rgb}{0.000000,0.000000,0.000000}%
\pgfsetfillcolor{currentfill}%
\pgfsetlinewidth{0.803000pt}%
\definecolor{currentstroke}{rgb}{0.000000,0.000000,0.000000}%
\pgfsetstrokecolor{currentstroke}%
\pgfsetdash{}{0pt}%
\pgfsys@defobject{currentmarker}{\pgfqpoint{-0.048611in}{0.000000in}}{\pgfqpoint{-0.000000in}{0.000000in}}{%
\pgfpathmoveto{\pgfqpoint{-0.000000in}{0.000000in}}%
\pgfpathlineto{\pgfqpoint{-0.048611in}{0.000000in}}%
\pgfusepath{stroke,fill}%
}%
\begin{pgfscope}%
\pgfsys@transformshift{0.654013in}{2.516837in}%
\pgfsys@useobject{currentmarker}{}%
\end{pgfscope}%
\end{pgfscope}%
\begin{pgfscope}%
\definecolor{textcolor}{rgb}{0.000000,0.000000,0.000000}%
\pgfsetstrokecolor{textcolor}%
\pgfsetfillcolor{textcolor}%
\pgftext[x=0.279012in, y=2.468612in, left, base]{\color{textcolor}\rmfamily\fontsize{10.000000}{12.000000}\selectfont \(\displaystyle {4000}\)}%
\end{pgfscope}%
\begin{pgfscope}%
\pgfsetbuttcap%
\pgfsetroundjoin%
\definecolor{currentfill}{rgb}{0.000000,0.000000,0.000000}%
\pgfsetfillcolor{currentfill}%
\pgfsetlinewidth{0.803000pt}%
\definecolor{currentstroke}{rgb}{0.000000,0.000000,0.000000}%
\pgfsetstrokecolor{currentstroke}%
\pgfsetdash{}{0pt}%
\pgfsys@defobject{currentmarker}{\pgfqpoint{-0.048611in}{0.000000in}}{\pgfqpoint{-0.000000in}{0.000000in}}{%
\pgfpathmoveto{\pgfqpoint{-0.000000in}{0.000000in}}%
\pgfpathlineto{\pgfqpoint{-0.048611in}{0.000000in}}%
\pgfusepath{stroke,fill}%
}%
\begin{pgfscope}%
\pgfsys@transformshift{0.654013in}{3.021123in}%
\pgfsys@useobject{currentmarker}{}%
\end{pgfscope}%
\end{pgfscope}%
\begin{pgfscope}%
\definecolor{textcolor}{rgb}{0.000000,0.000000,0.000000}%
\pgfsetstrokecolor{textcolor}%
\pgfsetfillcolor{textcolor}%
\pgftext[x=0.279012in, y=2.972898in, left, base]{\color{textcolor}\rmfamily\fontsize{10.000000}{12.000000}\selectfont \(\displaystyle {5000}\)}%
\end{pgfscope}%
\begin{pgfscope}%
\pgfsetbuttcap%
\pgfsetroundjoin%
\definecolor{currentfill}{rgb}{0.000000,0.000000,0.000000}%
\pgfsetfillcolor{currentfill}%
\pgfsetlinewidth{0.803000pt}%
\definecolor{currentstroke}{rgb}{0.000000,0.000000,0.000000}%
\pgfsetstrokecolor{currentstroke}%
\pgfsetdash{}{0pt}%
\pgfsys@defobject{currentmarker}{\pgfqpoint{-0.048611in}{0.000000in}}{\pgfqpoint{-0.000000in}{0.000000in}}{%
\pgfpathmoveto{\pgfqpoint{-0.000000in}{0.000000in}}%
\pgfpathlineto{\pgfqpoint{-0.048611in}{0.000000in}}%
\pgfusepath{stroke,fill}%
}%
\begin{pgfscope}%
\pgfsys@transformshift{0.654013in}{3.525410in}%
\pgfsys@useobject{currentmarker}{}%
\end{pgfscope}%
\end{pgfscope}%
\begin{pgfscope}%
\definecolor{textcolor}{rgb}{0.000000,0.000000,0.000000}%
\pgfsetstrokecolor{textcolor}%
\pgfsetfillcolor{textcolor}%
\pgftext[x=0.279012in, y=3.477184in, left, base]{\color{textcolor}\rmfamily\fontsize{10.000000}{12.000000}\selectfont \(\displaystyle {6000}\)}%
\end{pgfscope}%
\begin{pgfscope}%
\pgfsetbuttcap%
\pgfsetroundjoin%
\definecolor{currentfill}{rgb}{0.000000,0.000000,0.000000}%
\pgfsetfillcolor{currentfill}%
\pgfsetlinewidth{0.803000pt}%
\definecolor{currentstroke}{rgb}{0.000000,0.000000,0.000000}%
\pgfsetstrokecolor{currentstroke}%
\pgfsetdash{}{0pt}%
\pgfsys@defobject{currentmarker}{\pgfqpoint{-0.048611in}{0.000000in}}{\pgfqpoint{-0.000000in}{0.000000in}}{%
\pgfpathmoveto{\pgfqpoint{-0.000000in}{0.000000in}}%
\pgfpathlineto{\pgfqpoint{-0.048611in}{0.000000in}}%
\pgfusepath{stroke,fill}%
}%
\begin{pgfscope}%
\pgfsys@transformshift{0.654013in}{4.029696in}%
\pgfsys@useobject{currentmarker}{}%
\end{pgfscope}%
\end{pgfscope}%
\begin{pgfscope}%
\definecolor{textcolor}{rgb}{0.000000,0.000000,0.000000}%
\pgfsetstrokecolor{textcolor}%
\pgfsetfillcolor{textcolor}%
\pgftext[x=0.279012in, y=3.981471in, left, base]{\color{textcolor}\rmfamily\fontsize{10.000000}{12.000000}\selectfont \(\displaystyle {7000}\)}%
\end{pgfscope}%
\begin{pgfscope}%
\definecolor{textcolor}{rgb}{0.000000,0.000000,0.000000}%
\pgfsetstrokecolor{textcolor}%
\pgfsetfillcolor{textcolor}%
\pgftext[x=0.223457in,y=2.424691in,,bottom,rotate=90.000000]{\color{textcolor}\rmfamily\fontsize{10.000000}{12.000000}\selectfont Count}%
\end{pgfscope}%
\begin{pgfscope}%
\pgfsetrectcap%
\pgfsetmiterjoin%
\pgfsetlinewidth{0.803000pt}%
\definecolor{currentstroke}{rgb}{0.000000,0.000000,0.000000}%
\pgfsetstrokecolor{currentstroke}%
\pgfsetdash{}{0pt}%
\pgfpathmoveto{\pgfqpoint{0.654013in}{0.499691in}}%
\pgfpathlineto{\pgfqpoint{0.654013in}{4.349691in}}%
\pgfusepath{stroke}%
\end{pgfscope}%
\begin{pgfscope}%
\pgfsetrectcap%
\pgfsetmiterjoin%
\pgfsetlinewidth{0.803000pt}%
\definecolor{currentstroke}{rgb}{0.000000,0.000000,0.000000}%
\pgfsetstrokecolor{currentstroke}%
\pgfsetdash{}{0pt}%
\pgfpathmoveto{\pgfqpoint{5.304013in}{0.499691in}}%
\pgfpathlineto{\pgfqpoint{5.304013in}{4.349691in}}%
\pgfusepath{stroke}%
\end{pgfscope}%
\begin{pgfscope}%
\pgfsetrectcap%
\pgfsetmiterjoin%
\pgfsetlinewidth{0.803000pt}%
\definecolor{currentstroke}{rgb}{0.000000,0.000000,0.000000}%
\pgfsetstrokecolor{currentstroke}%
\pgfsetdash{}{0pt}%
\pgfpathmoveto{\pgfqpoint{0.654013in}{0.499691in}}%
\pgfpathlineto{\pgfqpoint{5.304013in}{0.499691in}}%
\pgfusepath{stroke}%
\end{pgfscope}%
\begin{pgfscope}%
\pgfsetrectcap%
\pgfsetmiterjoin%
\pgfsetlinewidth{0.803000pt}%
\definecolor{currentstroke}{rgb}{0.000000,0.000000,0.000000}%
\pgfsetstrokecolor{currentstroke}%
\pgfsetdash{}{0pt}%
\pgfpathmoveto{\pgfqpoint{0.654013in}{4.349691in}}%
\pgfpathlineto{\pgfqpoint{5.304013in}{4.349691in}}%
\pgfusepath{stroke}%
\end{pgfscope}%
\end{pgfpicture}%
\makeatother%
\endgroup%

    \caption{Stars distribution}
    \label{fig:distrib}
\end{figure}

Then we looked into the stars distribution per assureur without and without scaling the y axis (\cref{fig:distrib_split_noscale} and \cref{fig:distrib_split_scale}).

\newgeometry{top=1cm, bottom=0cm}
\begin{figure}[H]
    \advance\leftskip-3cm
    %% Creator: Matplotlib, PGF backend
%%
%% To include the figure in your LaTeX document, write
%%   \input{<filename>.pgf}
%%
%% Make sure the required packages are loaded in your preamble
%%   \usepackage{pgf}
%%
%% Also ensure that all the required font packages are loaded; for instance,
%% the lmodern package is sometimes necessary when using math font.
%%   \usepackage{lmodern}
%%
%% Figures using additional raster images can only be included by \input if
%% they are in the same directory as the main LaTeX file. For loading figures
%% from other directories you can use the `import` package
%%   \usepackage{import}
%%
%% and then include the figures with
%%   \import{<path to file>}{<filename>.pgf}
%%
%% Matplotlib used the following preamble
%%
\begingroup%
\makeatletter%
\begin{pgfpicture}%
\pgfpathrectangle{\pgfpointorigin}{\pgfqpoint{7.998611in}{5.057558in}}%
\pgfusepath{use as bounding box, clip}%
\begin{pgfscope}%
\pgfsetbuttcap%
\pgfsetmiterjoin%
\definecolor{currentfill}{rgb}{1.000000,1.000000,1.000000}%
\pgfsetfillcolor{currentfill}%
\pgfsetlinewidth{0.000000pt}%
\definecolor{currentstroke}{rgb}{1.000000,1.000000,1.000000}%
\pgfsetstrokecolor{currentstroke}%
\pgfsetdash{}{0pt}%
\pgfpathmoveto{\pgfqpoint{0.000000in}{0.000000in}}%
\pgfpathlineto{\pgfqpoint{7.998611in}{0.000000in}}%
\pgfpathlineto{\pgfqpoint{7.998611in}{5.057558in}}%
\pgfpathlineto{\pgfqpoint{0.000000in}{5.057558in}}%
\pgfpathlineto{\pgfqpoint{0.000000in}{0.000000in}}%
\pgfpathclose%
\pgfusepath{fill}%
\end{pgfscope}%
\begin{pgfscope}%
\pgfsetbuttcap%
\pgfsetmiterjoin%
\definecolor{currentfill}{rgb}{1.000000,1.000000,1.000000}%
\pgfsetfillcolor{currentfill}%
\pgfsetlinewidth{0.000000pt}%
\definecolor{currentstroke}{rgb}{0.000000,0.000000,0.000000}%
\pgfsetstrokecolor{currentstroke}%
\pgfsetstrokeopacity{0.000000}%
\pgfsetdash{}{0pt}%
\pgfpathmoveto{\pgfqpoint{0.148611in}{4.306611in}}%
\pgfpathlineto{\pgfqpoint{0.973079in}{4.306611in}}%
\pgfpathlineto{\pgfqpoint{0.973079in}{4.768611in}}%
\pgfpathlineto{\pgfqpoint{0.148611in}{4.768611in}}%
\pgfpathlineto{\pgfqpoint{0.148611in}{4.306611in}}%
\pgfpathclose%
\pgfusepath{fill}%
\end{pgfscope}%
\begin{pgfscope}%
\pgfpathrectangle{\pgfqpoint{0.148611in}{4.306611in}}{\pgfqpoint{0.824468in}{0.462000in}}%
\pgfusepath{clip}%
\pgfsetbuttcap%
\pgfsetmiterjoin%
\definecolor{currentfill}{rgb}{0.121569,0.466667,0.705882}%
\pgfsetfillcolor{currentfill}%
\pgfsetfillopacity{0.500000}%
\pgfsetlinewidth{1.003750pt}%
\definecolor{currentstroke}{rgb}{0.000000,0.000000,0.000000}%
\pgfsetstrokecolor{currentstroke}%
\pgfsetdash{}{0pt}%
\pgfpathmoveto{\pgfqpoint{0.186087in}{4.306611in}}%
\pgfpathlineto{\pgfqpoint{0.335990in}{4.306611in}}%
\pgfpathlineto{\pgfqpoint{0.335990in}{4.574182in}}%
\pgfpathlineto{\pgfqpoint{0.186087in}{4.574182in}}%
\pgfpathlineto{\pgfqpoint{0.186087in}{4.306611in}}%
\pgfpathclose%
\pgfusepath{stroke,fill}%
\end{pgfscope}%
\begin{pgfscope}%
\pgfpathrectangle{\pgfqpoint{0.148611in}{4.306611in}}{\pgfqpoint{0.824468in}{0.462000in}}%
\pgfusepath{clip}%
\pgfsetbuttcap%
\pgfsetmiterjoin%
\definecolor{currentfill}{rgb}{0.121569,0.466667,0.705882}%
\pgfsetfillcolor{currentfill}%
\pgfsetfillopacity{0.500000}%
\pgfsetlinewidth{1.003750pt}%
\definecolor{currentstroke}{rgb}{0.000000,0.000000,0.000000}%
\pgfsetstrokecolor{currentstroke}%
\pgfsetdash{}{0pt}%
\pgfpathmoveto{\pgfqpoint{0.335990in}{4.306611in}}%
\pgfpathlineto{\pgfqpoint{0.485894in}{4.306611in}}%
\pgfpathlineto{\pgfqpoint{0.485894in}{4.520612in}}%
\pgfpathlineto{\pgfqpoint{0.335990in}{4.520612in}}%
\pgfpathlineto{\pgfqpoint{0.335990in}{4.306611in}}%
\pgfpathclose%
\pgfusepath{stroke,fill}%
\end{pgfscope}%
\begin{pgfscope}%
\pgfpathrectangle{\pgfqpoint{0.148611in}{4.306611in}}{\pgfqpoint{0.824468in}{0.462000in}}%
\pgfusepath{clip}%
\pgfsetbuttcap%
\pgfsetmiterjoin%
\definecolor{currentfill}{rgb}{0.121569,0.466667,0.705882}%
\pgfsetfillcolor{currentfill}%
\pgfsetfillopacity{0.500000}%
\pgfsetlinewidth{1.003750pt}%
\definecolor{currentstroke}{rgb}{0.000000,0.000000,0.000000}%
\pgfsetstrokecolor{currentstroke}%
\pgfsetdash{}{0pt}%
\pgfpathmoveto{\pgfqpoint{0.485894in}{4.306611in}}%
\pgfpathlineto{\pgfqpoint{0.635797in}{4.306611in}}%
\pgfpathlineto{\pgfqpoint{0.635797in}{4.616592in}}%
\pgfpathlineto{\pgfqpoint{0.485894in}{4.616592in}}%
\pgfpathlineto{\pgfqpoint{0.485894in}{4.306611in}}%
\pgfpathclose%
\pgfusepath{stroke,fill}%
\end{pgfscope}%
\begin{pgfscope}%
\pgfpathrectangle{\pgfqpoint{0.148611in}{4.306611in}}{\pgfqpoint{0.824468in}{0.462000in}}%
\pgfusepath{clip}%
\pgfsetbuttcap%
\pgfsetmiterjoin%
\definecolor{currentfill}{rgb}{0.121569,0.466667,0.705882}%
\pgfsetfillcolor{currentfill}%
\pgfsetfillopacity{0.500000}%
\pgfsetlinewidth{1.003750pt}%
\definecolor{currentstroke}{rgb}{0.000000,0.000000,0.000000}%
\pgfsetstrokecolor{currentstroke}%
\pgfsetdash{}{0pt}%
\pgfpathmoveto{\pgfqpoint{0.635797in}{4.306611in}}%
\pgfpathlineto{\pgfqpoint{0.785700in}{4.306611in}}%
\pgfpathlineto{\pgfqpoint{0.785700in}{4.746611in}}%
\pgfpathlineto{\pgfqpoint{0.635797in}{4.746611in}}%
\pgfpathlineto{\pgfqpoint{0.635797in}{4.306611in}}%
\pgfpathclose%
\pgfusepath{stroke,fill}%
\end{pgfscope}%
\begin{pgfscope}%
\pgfpathrectangle{\pgfqpoint{0.148611in}{4.306611in}}{\pgfqpoint{0.824468in}{0.462000in}}%
\pgfusepath{clip}%
\pgfsetbuttcap%
\pgfsetmiterjoin%
\definecolor{currentfill}{rgb}{0.121569,0.466667,0.705882}%
\pgfsetfillcolor{currentfill}%
\pgfsetfillopacity{0.500000}%
\pgfsetlinewidth{1.003750pt}%
\definecolor{currentstroke}{rgb}{0.000000,0.000000,0.000000}%
\pgfsetstrokecolor{currentstroke}%
\pgfsetdash{}{0pt}%
\pgfpathmoveto{\pgfqpoint{0.785700in}{4.306611in}}%
\pgfpathlineto{\pgfqpoint{0.935603in}{4.306611in}}%
\pgfpathlineto{\pgfqpoint{0.935603in}{4.720105in}}%
\pgfpathlineto{\pgfqpoint{0.785700in}{4.720105in}}%
\pgfpathlineto{\pgfqpoint{0.785700in}{4.306611in}}%
\pgfpathclose%
\pgfusepath{stroke,fill}%
\end{pgfscope}%
\begin{pgfscope}%
\pgfsetrectcap%
\pgfsetmiterjoin%
\pgfsetlinewidth{0.803000pt}%
\definecolor{currentstroke}{rgb}{0.000000,0.000000,0.000000}%
\pgfsetstrokecolor{currentstroke}%
\pgfsetdash{}{0pt}%
\pgfpathmoveto{\pgfqpoint{0.148611in}{4.306611in}}%
\pgfpathlineto{\pgfqpoint{0.148611in}{4.768611in}}%
\pgfusepath{stroke}%
\end{pgfscope}%
\begin{pgfscope}%
\pgfsetrectcap%
\pgfsetmiterjoin%
\pgfsetlinewidth{0.803000pt}%
\definecolor{currentstroke}{rgb}{0.000000,0.000000,0.000000}%
\pgfsetstrokecolor{currentstroke}%
\pgfsetdash{}{0pt}%
\pgfpathmoveto{\pgfqpoint{0.973079in}{4.306611in}}%
\pgfpathlineto{\pgfqpoint{0.973079in}{4.768611in}}%
\pgfusepath{stroke}%
\end{pgfscope}%
\begin{pgfscope}%
\pgfsetrectcap%
\pgfsetmiterjoin%
\pgfsetlinewidth{0.803000pt}%
\definecolor{currentstroke}{rgb}{0.000000,0.000000,0.000000}%
\pgfsetstrokecolor{currentstroke}%
\pgfsetdash{}{0pt}%
\pgfpathmoveto{\pgfqpoint{0.148611in}{4.306611in}}%
\pgfpathlineto{\pgfqpoint{0.973079in}{4.306611in}}%
\pgfusepath{stroke}%
\end{pgfscope}%
\begin{pgfscope}%
\pgfsetrectcap%
\pgfsetmiterjoin%
\pgfsetlinewidth{0.803000pt}%
\definecolor{currentstroke}{rgb}{0.000000,0.000000,0.000000}%
\pgfsetstrokecolor{currentstroke}%
\pgfsetdash{}{0pt}%
\pgfpathmoveto{\pgfqpoint{0.148611in}{4.768611in}}%
\pgfpathlineto{\pgfqpoint{0.973079in}{4.768611in}}%
\pgfusepath{stroke}%
\end{pgfscope}%
\begin{pgfscope}%
\definecolor{textcolor}{rgb}{0.000000,0.000000,0.000000}%
\pgfsetstrokecolor{textcolor}%
\pgfsetfillcolor{textcolor}%
\pgftext[x=0.560845in,y=4.851944in,,base]{\color{textcolor}\rmfamily\fontsize{11.000000}{13.200000}\selectfont Direct...}%
\end{pgfscope}%
\begin{pgfscope}%
\pgfsetbuttcap%
\pgfsetmiterjoin%
\definecolor{currentfill}{rgb}{1.000000,1.000000,1.000000}%
\pgfsetfillcolor{currentfill}%
\pgfsetlinewidth{0.000000pt}%
\definecolor{currentstroke}{rgb}{0.000000,0.000000,0.000000}%
\pgfsetstrokecolor{currentstroke}%
\pgfsetstrokeopacity{0.000000}%
\pgfsetdash{}{0pt}%
\pgfpathmoveto{\pgfqpoint{1.137973in}{4.306611in}}%
\pgfpathlineto{\pgfqpoint{1.962441in}{4.306611in}}%
\pgfpathlineto{\pgfqpoint{1.962441in}{4.768611in}}%
\pgfpathlineto{\pgfqpoint{1.137973in}{4.768611in}}%
\pgfpathlineto{\pgfqpoint{1.137973in}{4.306611in}}%
\pgfpathclose%
\pgfusepath{fill}%
\end{pgfscope}%
\begin{pgfscope}%
\pgfpathrectangle{\pgfqpoint{1.137973in}{4.306611in}}{\pgfqpoint{0.824468in}{0.462000in}}%
\pgfusepath{clip}%
\pgfsetbuttcap%
\pgfsetmiterjoin%
\definecolor{currentfill}{rgb}{0.121569,0.466667,0.705882}%
\pgfsetfillcolor{currentfill}%
\pgfsetfillopacity{0.500000}%
\pgfsetlinewidth{1.003750pt}%
\definecolor{currentstroke}{rgb}{0.000000,0.000000,0.000000}%
\pgfsetstrokecolor{currentstroke}%
\pgfsetdash{}{0pt}%
\pgfpathmoveto{\pgfqpoint{1.175449in}{4.306611in}}%
\pgfpathlineto{\pgfqpoint{1.325352in}{4.306611in}}%
\pgfpathlineto{\pgfqpoint{1.325352in}{4.411087in}}%
\pgfpathlineto{\pgfqpoint{1.175449in}{4.411087in}}%
\pgfpathlineto{\pgfqpoint{1.175449in}{4.306611in}}%
\pgfpathclose%
\pgfusepath{stroke,fill}%
\end{pgfscope}%
\begin{pgfscope}%
\pgfpathrectangle{\pgfqpoint{1.137973in}{4.306611in}}{\pgfqpoint{0.824468in}{0.462000in}}%
\pgfusepath{clip}%
\pgfsetbuttcap%
\pgfsetmiterjoin%
\definecolor{currentfill}{rgb}{0.121569,0.466667,0.705882}%
\pgfsetfillcolor{currentfill}%
\pgfsetfillopacity{0.500000}%
\pgfsetlinewidth{1.003750pt}%
\definecolor{currentstroke}{rgb}{0.000000,0.000000,0.000000}%
\pgfsetstrokecolor{currentstroke}%
\pgfsetdash{}{0pt}%
\pgfpathmoveto{\pgfqpoint{1.325352in}{4.306611in}}%
\pgfpathlineto{\pgfqpoint{1.475255in}{4.306611in}}%
\pgfpathlineto{\pgfqpoint{1.475255in}{4.397175in}}%
\pgfpathlineto{\pgfqpoint{1.325352in}{4.397175in}}%
\pgfpathlineto{\pgfqpoint{1.325352in}{4.306611in}}%
\pgfpathclose%
\pgfusepath{stroke,fill}%
\end{pgfscope}%
\begin{pgfscope}%
\pgfpathrectangle{\pgfqpoint{1.137973in}{4.306611in}}{\pgfqpoint{0.824468in}{0.462000in}}%
\pgfusepath{clip}%
\pgfsetbuttcap%
\pgfsetmiterjoin%
\definecolor{currentfill}{rgb}{0.121569,0.466667,0.705882}%
\pgfsetfillcolor{currentfill}%
\pgfsetfillopacity{0.500000}%
\pgfsetlinewidth{1.003750pt}%
\definecolor{currentstroke}{rgb}{0.000000,0.000000,0.000000}%
\pgfsetstrokecolor{currentstroke}%
\pgfsetdash{}{0pt}%
\pgfpathmoveto{\pgfqpoint{1.475255in}{4.306611in}}%
\pgfpathlineto{\pgfqpoint{1.625158in}{4.306611in}}%
\pgfpathlineto{\pgfqpoint{1.625158in}{4.451186in}}%
\pgfpathlineto{\pgfqpoint{1.475255in}{4.451186in}}%
\pgfpathlineto{\pgfqpoint{1.475255in}{4.306611in}}%
\pgfpathclose%
\pgfusepath{stroke,fill}%
\end{pgfscope}%
\begin{pgfscope}%
\pgfpathrectangle{\pgfqpoint{1.137973in}{4.306611in}}{\pgfqpoint{0.824468in}{0.462000in}}%
\pgfusepath{clip}%
\pgfsetbuttcap%
\pgfsetmiterjoin%
\definecolor{currentfill}{rgb}{0.121569,0.466667,0.705882}%
\pgfsetfillcolor{currentfill}%
\pgfsetfillopacity{0.500000}%
\pgfsetlinewidth{1.003750pt}%
\definecolor{currentstroke}{rgb}{0.000000,0.000000,0.000000}%
\pgfsetstrokecolor{currentstroke}%
\pgfsetdash{}{0pt}%
\pgfpathmoveto{\pgfqpoint{1.625158in}{4.306611in}}%
\pgfpathlineto{\pgfqpoint{1.775062in}{4.306611in}}%
\pgfpathlineto{\pgfqpoint{1.775062in}{4.696692in}}%
\pgfpathlineto{\pgfqpoint{1.625158in}{4.696692in}}%
\pgfpathlineto{\pgfqpoint{1.625158in}{4.306611in}}%
\pgfpathclose%
\pgfusepath{stroke,fill}%
\end{pgfscope}%
\begin{pgfscope}%
\pgfpathrectangle{\pgfqpoint{1.137973in}{4.306611in}}{\pgfqpoint{0.824468in}{0.462000in}}%
\pgfusepath{clip}%
\pgfsetbuttcap%
\pgfsetmiterjoin%
\definecolor{currentfill}{rgb}{0.121569,0.466667,0.705882}%
\pgfsetfillcolor{currentfill}%
\pgfsetfillopacity{0.500000}%
\pgfsetlinewidth{1.003750pt}%
\definecolor{currentstroke}{rgb}{0.000000,0.000000,0.000000}%
\pgfsetstrokecolor{currentstroke}%
\pgfsetdash{}{0pt}%
\pgfpathmoveto{\pgfqpoint{1.775062in}{4.306611in}}%
\pgfpathlineto{\pgfqpoint{1.924965in}{4.306611in}}%
\pgfpathlineto{\pgfqpoint{1.924965in}{4.746611in}}%
\pgfpathlineto{\pgfqpoint{1.775062in}{4.746611in}}%
\pgfpathlineto{\pgfqpoint{1.775062in}{4.306611in}}%
\pgfpathclose%
\pgfusepath{stroke,fill}%
\end{pgfscope}%
\begin{pgfscope}%
\pgfsetrectcap%
\pgfsetmiterjoin%
\pgfsetlinewidth{0.803000pt}%
\definecolor{currentstroke}{rgb}{0.000000,0.000000,0.000000}%
\pgfsetstrokecolor{currentstroke}%
\pgfsetdash{}{0pt}%
\pgfpathmoveto{\pgfqpoint{1.137973in}{4.306611in}}%
\pgfpathlineto{\pgfqpoint{1.137973in}{4.768611in}}%
\pgfusepath{stroke}%
\end{pgfscope}%
\begin{pgfscope}%
\pgfsetrectcap%
\pgfsetmiterjoin%
\pgfsetlinewidth{0.803000pt}%
\definecolor{currentstroke}{rgb}{0.000000,0.000000,0.000000}%
\pgfsetstrokecolor{currentstroke}%
\pgfsetdash{}{0pt}%
\pgfpathmoveto{\pgfqpoint{1.962441in}{4.306611in}}%
\pgfpathlineto{\pgfqpoint{1.962441in}{4.768611in}}%
\pgfusepath{stroke}%
\end{pgfscope}%
\begin{pgfscope}%
\pgfsetrectcap%
\pgfsetmiterjoin%
\pgfsetlinewidth{0.803000pt}%
\definecolor{currentstroke}{rgb}{0.000000,0.000000,0.000000}%
\pgfsetstrokecolor{currentstroke}%
\pgfsetdash{}{0pt}%
\pgfpathmoveto{\pgfqpoint{1.137973in}{4.306611in}}%
\pgfpathlineto{\pgfqpoint{1.962441in}{4.306611in}}%
\pgfusepath{stroke}%
\end{pgfscope}%
\begin{pgfscope}%
\pgfsetrectcap%
\pgfsetmiterjoin%
\pgfsetlinewidth{0.803000pt}%
\definecolor{currentstroke}{rgb}{0.000000,0.000000,0.000000}%
\pgfsetstrokecolor{currentstroke}%
\pgfsetdash{}{0pt}%
\pgfpathmoveto{\pgfqpoint{1.137973in}{4.768611in}}%
\pgfpathlineto{\pgfqpoint{1.962441in}{4.768611in}}%
\pgfusepath{stroke}%
\end{pgfscope}%
\begin{pgfscope}%
\definecolor{textcolor}{rgb}{0.000000,0.000000,0.000000}%
\pgfsetstrokecolor{textcolor}%
\pgfsetfillcolor{textcolor}%
\pgftext[x=1.550207in,y=4.851944in,,base]{\color{textcolor}\rmfamily\fontsize{11.000000}{13.200000}\selectfont L'oliv...}%
\end{pgfscope}%
\begin{pgfscope}%
\pgfsetbuttcap%
\pgfsetmiterjoin%
\definecolor{currentfill}{rgb}{1.000000,1.000000,1.000000}%
\pgfsetfillcolor{currentfill}%
\pgfsetlinewidth{0.000000pt}%
\definecolor{currentstroke}{rgb}{0.000000,0.000000,0.000000}%
\pgfsetstrokecolor{currentstroke}%
\pgfsetstrokeopacity{0.000000}%
\pgfsetdash{}{0pt}%
\pgfpathmoveto{\pgfqpoint{2.127335in}{4.306611in}}%
\pgfpathlineto{\pgfqpoint{2.951803in}{4.306611in}}%
\pgfpathlineto{\pgfqpoint{2.951803in}{4.768611in}}%
\pgfpathlineto{\pgfqpoint{2.127335in}{4.768611in}}%
\pgfpathlineto{\pgfqpoint{2.127335in}{4.306611in}}%
\pgfpathclose%
\pgfusepath{fill}%
\end{pgfscope}%
\begin{pgfscope}%
\pgfpathrectangle{\pgfqpoint{2.127335in}{4.306611in}}{\pgfqpoint{0.824468in}{0.462000in}}%
\pgfusepath{clip}%
\pgfsetbuttcap%
\pgfsetmiterjoin%
\definecolor{currentfill}{rgb}{0.121569,0.466667,0.705882}%
\pgfsetfillcolor{currentfill}%
\pgfsetfillopacity{0.500000}%
\pgfsetlinewidth{1.003750pt}%
\definecolor{currentstroke}{rgb}{0.000000,0.000000,0.000000}%
\pgfsetstrokecolor{currentstroke}%
\pgfsetdash{}{0pt}%
\pgfpathmoveto{\pgfqpoint{2.164810in}{4.306611in}}%
\pgfpathlineto{\pgfqpoint{2.314714in}{4.306611in}}%
\pgfpathlineto{\pgfqpoint{2.314714in}{4.746611in}}%
\pgfpathlineto{\pgfqpoint{2.164810in}{4.746611in}}%
\pgfpathlineto{\pgfqpoint{2.164810in}{4.306611in}}%
\pgfpathclose%
\pgfusepath{stroke,fill}%
\end{pgfscope}%
\begin{pgfscope}%
\pgfpathrectangle{\pgfqpoint{2.127335in}{4.306611in}}{\pgfqpoint{0.824468in}{0.462000in}}%
\pgfusepath{clip}%
\pgfsetbuttcap%
\pgfsetmiterjoin%
\definecolor{currentfill}{rgb}{0.121569,0.466667,0.705882}%
\pgfsetfillcolor{currentfill}%
\pgfsetfillopacity{0.500000}%
\pgfsetlinewidth{1.003750pt}%
\definecolor{currentstroke}{rgb}{0.000000,0.000000,0.000000}%
\pgfsetstrokecolor{currentstroke}%
\pgfsetdash{}{0pt}%
\pgfpathmoveto{\pgfqpoint{2.314714in}{4.306611in}}%
\pgfpathlineto{\pgfqpoint{2.464617in}{4.306611in}}%
\pgfpathlineto{\pgfqpoint{2.464617in}{4.581862in}}%
\pgfpathlineto{\pgfqpoint{2.314714in}{4.581862in}}%
\pgfpathlineto{\pgfqpoint{2.314714in}{4.306611in}}%
\pgfpathclose%
\pgfusepath{stroke,fill}%
\end{pgfscope}%
\begin{pgfscope}%
\pgfpathrectangle{\pgfqpoint{2.127335in}{4.306611in}}{\pgfqpoint{0.824468in}{0.462000in}}%
\pgfusepath{clip}%
\pgfsetbuttcap%
\pgfsetmiterjoin%
\definecolor{currentfill}{rgb}{0.121569,0.466667,0.705882}%
\pgfsetfillcolor{currentfill}%
\pgfsetfillopacity{0.500000}%
\pgfsetlinewidth{1.003750pt}%
\definecolor{currentstroke}{rgb}{0.000000,0.000000,0.000000}%
\pgfsetstrokecolor{currentstroke}%
\pgfsetdash{}{0pt}%
\pgfpathmoveto{\pgfqpoint{2.464617in}{4.306611in}}%
\pgfpathlineto{\pgfqpoint{2.614520in}{4.306611in}}%
\pgfpathlineto{\pgfqpoint{2.614520in}{4.403049in}}%
\pgfpathlineto{\pgfqpoint{2.464617in}{4.403049in}}%
\pgfpathlineto{\pgfqpoint{2.464617in}{4.306611in}}%
\pgfpathclose%
\pgfusepath{stroke,fill}%
\end{pgfscope}%
\begin{pgfscope}%
\pgfpathrectangle{\pgfqpoint{2.127335in}{4.306611in}}{\pgfqpoint{0.824468in}{0.462000in}}%
\pgfusepath{clip}%
\pgfsetbuttcap%
\pgfsetmiterjoin%
\definecolor{currentfill}{rgb}{0.121569,0.466667,0.705882}%
\pgfsetfillcolor{currentfill}%
\pgfsetfillopacity{0.500000}%
\pgfsetlinewidth{1.003750pt}%
\definecolor{currentstroke}{rgb}{0.000000,0.000000,0.000000}%
\pgfsetstrokecolor{currentstroke}%
\pgfsetdash{}{0pt}%
\pgfpathmoveto{\pgfqpoint{2.614520in}{4.306611in}}%
\pgfpathlineto{\pgfqpoint{2.764423in}{4.306611in}}%
\pgfpathlineto{\pgfqpoint{2.764423in}{4.376931in}}%
\pgfpathlineto{\pgfqpoint{2.614520in}{4.376931in}}%
\pgfpathlineto{\pgfqpoint{2.614520in}{4.306611in}}%
\pgfpathclose%
\pgfusepath{stroke,fill}%
\end{pgfscope}%
\begin{pgfscope}%
\pgfpathrectangle{\pgfqpoint{2.127335in}{4.306611in}}{\pgfqpoint{0.824468in}{0.462000in}}%
\pgfusepath{clip}%
\pgfsetbuttcap%
\pgfsetmiterjoin%
\definecolor{currentfill}{rgb}{0.121569,0.466667,0.705882}%
\pgfsetfillcolor{currentfill}%
\pgfsetfillopacity{0.500000}%
\pgfsetlinewidth{1.003750pt}%
\definecolor{currentstroke}{rgb}{0.000000,0.000000,0.000000}%
\pgfsetstrokecolor{currentstroke}%
\pgfsetdash{}{0pt}%
\pgfpathmoveto{\pgfqpoint{2.764423in}{4.306611in}}%
\pgfpathlineto{\pgfqpoint{2.914327in}{4.306611in}}%
\pgfpathlineto{\pgfqpoint{2.914327in}{4.376931in}}%
\pgfpathlineto{\pgfqpoint{2.764423in}{4.376931in}}%
\pgfpathlineto{\pgfqpoint{2.764423in}{4.306611in}}%
\pgfpathclose%
\pgfusepath{stroke,fill}%
\end{pgfscope}%
\begin{pgfscope}%
\pgfsetrectcap%
\pgfsetmiterjoin%
\pgfsetlinewidth{0.803000pt}%
\definecolor{currentstroke}{rgb}{0.000000,0.000000,0.000000}%
\pgfsetstrokecolor{currentstroke}%
\pgfsetdash{}{0pt}%
\pgfpathmoveto{\pgfqpoint{2.127335in}{4.306611in}}%
\pgfpathlineto{\pgfqpoint{2.127335in}{4.768611in}}%
\pgfusepath{stroke}%
\end{pgfscope}%
\begin{pgfscope}%
\pgfsetrectcap%
\pgfsetmiterjoin%
\pgfsetlinewidth{0.803000pt}%
\definecolor{currentstroke}{rgb}{0.000000,0.000000,0.000000}%
\pgfsetstrokecolor{currentstroke}%
\pgfsetdash{}{0pt}%
\pgfpathmoveto{\pgfqpoint{2.951803in}{4.306611in}}%
\pgfpathlineto{\pgfqpoint{2.951803in}{4.768611in}}%
\pgfusepath{stroke}%
\end{pgfscope}%
\begin{pgfscope}%
\pgfsetrectcap%
\pgfsetmiterjoin%
\pgfsetlinewidth{0.803000pt}%
\definecolor{currentstroke}{rgb}{0.000000,0.000000,0.000000}%
\pgfsetstrokecolor{currentstroke}%
\pgfsetdash{}{0pt}%
\pgfpathmoveto{\pgfqpoint{2.127335in}{4.306611in}}%
\pgfpathlineto{\pgfqpoint{2.951803in}{4.306611in}}%
\pgfusepath{stroke}%
\end{pgfscope}%
\begin{pgfscope}%
\pgfsetrectcap%
\pgfsetmiterjoin%
\pgfsetlinewidth{0.803000pt}%
\definecolor{currentstroke}{rgb}{0.000000,0.000000,0.000000}%
\pgfsetstrokecolor{currentstroke}%
\pgfsetdash{}{0pt}%
\pgfpathmoveto{\pgfqpoint{2.127335in}{4.768611in}}%
\pgfpathlineto{\pgfqpoint{2.951803in}{4.768611in}}%
\pgfusepath{stroke}%
\end{pgfscope}%
\begin{pgfscope}%
\definecolor{textcolor}{rgb}{0.000000,0.000000,0.000000}%
\pgfsetstrokecolor{textcolor}%
\pgfsetfillcolor{textcolor}%
\pgftext[x=2.539569in,y=4.851944in,,base]{\color{textcolor}\rmfamily\fontsize{11.000000}{13.200000}\selectfont Matmut}%
\end{pgfscope}%
\begin{pgfscope}%
\pgfsetbuttcap%
\pgfsetmiterjoin%
\definecolor{currentfill}{rgb}{1.000000,1.000000,1.000000}%
\pgfsetfillcolor{currentfill}%
\pgfsetlinewidth{0.000000pt}%
\definecolor{currentstroke}{rgb}{0.000000,0.000000,0.000000}%
\pgfsetstrokecolor{currentstroke}%
\pgfsetstrokeopacity{0.000000}%
\pgfsetdash{}{0pt}%
\pgfpathmoveto{\pgfqpoint{3.116696in}{4.306611in}}%
\pgfpathlineto{\pgfqpoint{3.941164in}{4.306611in}}%
\pgfpathlineto{\pgfqpoint{3.941164in}{4.768611in}}%
\pgfpathlineto{\pgfqpoint{3.116696in}{4.768611in}}%
\pgfpathlineto{\pgfqpoint{3.116696in}{4.306611in}}%
\pgfpathclose%
\pgfusepath{fill}%
\end{pgfscope}%
\begin{pgfscope}%
\pgfpathrectangle{\pgfqpoint{3.116696in}{4.306611in}}{\pgfqpoint{0.824468in}{0.462000in}}%
\pgfusepath{clip}%
\pgfsetbuttcap%
\pgfsetmiterjoin%
\definecolor{currentfill}{rgb}{0.121569,0.466667,0.705882}%
\pgfsetfillcolor{currentfill}%
\pgfsetfillopacity{0.500000}%
\pgfsetlinewidth{1.003750pt}%
\definecolor{currentstroke}{rgb}{0.000000,0.000000,0.000000}%
\pgfsetstrokecolor{currentstroke}%
\pgfsetdash{}{0pt}%
\pgfpathmoveto{\pgfqpoint{3.154172in}{4.306611in}}%
\pgfpathlineto{\pgfqpoint{3.304075in}{4.306611in}}%
\pgfpathlineto{\pgfqpoint{3.304075in}{4.746611in}}%
\pgfpathlineto{\pgfqpoint{3.154172in}{4.746611in}}%
\pgfpathlineto{\pgfqpoint{3.154172in}{4.306611in}}%
\pgfpathclose%
\pgfusepath{stroke,fill}%
\end{pgfscope}%
\begin{pgfscope}%
\pgfpathrectangle{\pgfqpoint{3.116696in}{4.306611in}}{\pgfqpoint{0.824468in}{0.462000in}}%
\pgfusepath{clip}%
\pgfsetbuttcap%
\pgfsetmiterjoin%
\definecolor{currentfill}{rgb}{0.121569,0.466667,0.705882}%
\pgfsetfillcolor{currentfill}%
\pgfsetfillopacity{0.500000}%
\pgfsetlinewidth{1.003750pt}%
\definecolor{currentstroke}{rgb}{0.000000,0.000000,0.000000}%
\pgfsetstrokecolor{currentstroke}%
\pgfsetdash{}{0pt}%
\pgfpathmoveto{\pgfqpoint{3.304075in}{4.306611in}}%
\pgfpathlineto{\pgfqpoint{3.453979in}{4.306611in}}%
\pgfpathlineto{\pgfqpoint{3.453979in}{4.481293in}}%
\pgfpathlineto{\pgfqpoint{3.304075in}{4.481293in}}%
\pgfpathlineto{\pgfqpoint{3.304075in}{4.306611in}}%
\pgfpathclose%
\pgfusepath{stroke,fill}%
\end{pgfscope}%
\begin{pgfscope}%
\pgfpathrectangle{\pgfqpoint{3.116696in}{4.306611in}}{\pgfqpoint{0.824468in}{0.462000in}}%
\pgfusepath{clip}%
\pgfsetbuttcap%
\pgfsetmiterjoin%
\definecolor{currentfill}{rgb}{0.121569,0.466667,0.705882}%
\pgfsetfillcolor{currentfill}%
\pgfsetfillopacity{0.500000}%
\pgfsetlinewidth{1.003750pt}%
\definecolor{currentstroke}{rgb}{0.000000,0.000000,0.000000}%
\pgfsetstrokecolor{currentstroke}%
\pgfsetdash{}{0pt}%
\pgfpathmoveto{\pgfqpoint{3.453979in}{4.306611in}}%
\pgfpathlineto{\pgfqpoint{3.603882in}{4.306611in}}%
\pgfpathlineto{\pgfqpoint{3.603882in}{4.580169in}}%
\pgfpathlineto{\pgfqpoint{3.453979in}{4.580169in}}%
\pgfpathlineto{\pgfqpoint{3.453979in}{4.306611in}}%
\pgfpathclose%
\pgfusepath{stroke,fill}%
\end{pgfscope}%
\begin{pgfscope}%
\pgfpathrectangle{\pgfqpoint{3.116696in}{4.306611in}}{\pgfqpoint{0.824468in}{0.462000in}}%
\pgfusepath{clip}%
\pgfsetbuttcap%
\pgfsetmiterjoin%
\definecolor{currentfill}{rgb}{0.121569,0.466667,0.705882}%
\pgfsetfillcolor{currentfill}%
\pgfsetfillopacity{0.500000}%
\pgfsetlinewidth{1.003750pt}%
\definecolor{currentstroke}{rgb}{0.000000,0.000000,0.000000}%
\pgfsetstrokecolor{currentstroke}%
\pgfsetdash{}{0pt}%
\pgfpathmoveto{\pgfqpoint{3.603882in}{4.306611in}}%
\pgfpathlineto{\pgfqpoint{3.753785in}{4.306611in}}%
\pgfpathlineto{\pgfqpoint{3.753785in}{4.603240in}}%
\pgfpathlineto{\pgfqpoint{3.603882in}{4.603240in}}%
\pgfpathlineto{\pgfqpoint{3.603882in}{4.306611in}}%
\pgfpathclose%
\pgfusepath{stroke,fill}%
\end{pgfscope}%
\begin{pgfscope}%
\pgfpathrectangle{\pgfqpoint{3.116696in}{4.306611in}}{\pgfqpoint{0.824468in}{0.462000in}}%
\pgfusepath{clip}%
\pgfsetbuttcap%
\pgfsetmiterjoin%
\definecolor{currentfill}{rgb}{0.121569,0.466667,0.705882}%
\pgfsetfillcolor{currentfill}%
\pgfsetfillopacity{0.500000}%
\pgfsetlinewidth{1.003750pt}%
\definecolor{currentstroke}{rgb}{0.000000,0.000000,0.000000}%
\pgfsetstrokecolor{currentstroke}%
\pgfsetdash{}{0pt}%
\pgfpathmoveto{\pgfqpoint{3.753785in}{4.306611in}}%
\pgfpathlineto{\pgfqpoint{3.903688in}{4.306611in}}%
\pgfpathlineto{\pgfqpoint{3.903688in}{4.540619in}}%
\pgfpathlineto{\pgfqpoint{3.753785in}{4.540619in}}%
\pgfpathlineto{\pgfqpoint{3.753785in}{4.306611in}}%
\pgfpathclose%
\pgfusepath{stroke,fill}%
\end{pgfscope}%
\begin{pgfscope}%
\pgfsetrectcap%
\pgfsetmiterjoin%
\pgfsetlinewidth{0.803000pt}%
\definecolor{currentstroke}{rgb}{0.000000,0.000000,0.000000}%
\pgfsetstrokecolor{currentstroke}%
\pgfsetdash{}{0pt}%
\pgfpathmoveto{\pgfqpoint{3.116696in}{4.306611in}}%
\pgfpathlineto{\pgfqpoint{3.116696in}{4.768611in}}%
\pgfusepath{stroke}%
\end{pgfscope}%
\begin{pgfscope}%
\pgfsetrectcap%
\pgfsetmiterjoin%
\pgfsetlinewidth{0.803000pt}%
\definecolor{currentstroke}{rgb}{0.000000,0.000000,0.000000}%
\pgfsetstrokecolor{currentstroke}%
\pgfsetdash{}{0pt}%
\pgfpathmoveto{\pgfqpoint{3.941164in}{4.306611in}}%
\pgfpathlineto{\pgfqpoint{3.941164in}{4.768611in}}%
\pgfusepath{stroke}%
\end{pgfscope}%
\begin{pgfscope}%
\pgfsetrectcap%
\pgfsetmiterjoin%
\pgfsetlinewidth{0.803000pt}%
\definecolor{currentstroke}{rgb}{0.000000,0.000000,0.000000}%
\pgfsetstrokecolor{currentstroke}%
\pgfsetdash{}{0pt}%
\pgfpathmoveto{\pgfqpoint{3.116696in}{4.306611in}}%
\pgfpathlineto{\pgfqpoint{3.941164in}{4.306611in}}%
\pgfusepath{stroke}%
\end{pgfscope}%
\begin{pgfscope}%
\pgfsetrectcap%
\pgfsetmiterjoin%
\pgfsetlinewidth{0.803000pt}%
\definecolor{currentstroke}{rgb}{0.000000,0.000000,0.000000}%
\pgfsetstrokecolor{currentstroke}%
\pgfsetdash{}{0pt}%
\pgfpathmoveto{\pgfqpoint{3.116696in}{4.768611in}}%
\pgfpathlineto{\pgfqpoint{3.941164in}{4.768611in}}%
\pgfusepath{stroke}%
\end{pgfscope}%
\begin{pgfscope}%
\definecolor{textcolor}{rgb}{0.000000,0.000000,0.000000}%
\pgfsetstrokecolor{textcolor}%
\pgfsetfillcolor{textcolor}%
\pgftext[x=3.528930in,y=4.851944in,,base]{\color{textcolor}\rmfamily\fontsize{11.000000}{13.200000}\selectfont Néolia...}%
\end{pgfscope}%
\begin{pgfscope}%
\pgfsetbuttcap%
\pgfsetmiterjoin%
\definecolor{currentfill}{rgb}{1.000000,1.000000,1.000000}%
\pgfsetfillcolor{currentfill}%
\pgfsetlinewidth{0.000000pt}%
\definecolor{currentstroke}{rgb}{0.000000,0.000000,0.000000}%
\pgfsetstrokecolor{currentstroke}%
\pgfsetstrokeopacity{0.000000}%
\pgfsetdash{}{0pt}%
\pgfpathmoveto{\pgfqpoint{4.106058in}{4.306611in}}%
\pgfpathlineto{\pgfqpoint{4.930526in}{4.306611in}}%
\pgfpathlineto{\pgfqpoint{4.930526in}{4.768611in}}%
\pgfpathlineto{\pgfqpoint{4.106058in}{4.768611in}}%
\pgfpathlineto{\pgfqpoint{4.106058in}{4.306611in}}%
\pgfpathclose%
\pgfusepath{fill}%
\end{pgfscope}%
\begin{pgfscope}%
\pgfpathrectangle{\pgfqpoint{4.106058in}{4.306611in}}{\pgfqpoint{0.824468in}{0.462000in}}%
\pgfusepath{clip}%
\pgfsetbuttcap%
\pgfsetmiterjoin%
\definecolor{currentfill}{rgb}{0.121569,0.466667,0.705882}%
\pgfsetfillcolor{currentfill}%
\pgfsetfillopacity{0.500000}%
\pgfsetlinewidth{1.003750pt}%
\definecolor{currentstroke}{rgb}{0.000000,0.000000,0.000000}%
\pgfsetstrokecolor{currentstroke}%
\pgfsetdash{}{0pt}%
\pgfpathmoveto{\pgfqpoint{4.143534in}{4.306611in}}%
\pgfpathlineto{\pgfqpoint{4.293437in}{4.306611in}}%
\pgfpathlineto{\pgfqpoint{4.293437in}{4.746611in}}%
\pgfpathlineto{\pgfqpoint{4.143534in}{4.746611in}}%
\pgfpathlineto{\pgfqpoint{4.143534in}{4.306611in}}%
\pgfpathclose%
\pgfusepath{stroke,fill}%
\end{pgfscope}%
\begin{pgfscope}%
\pgfpathrectangle{\pgfqpoint{4.106058in}{4.306611in}}{\pgfqpoint{0.824468in}{0.462000in}}%
\pgfusepath{clip}%
\pgfsetbuttcap%
\pgfsetmiterjoin%
\definecolor{currentfill}{rgb}{0.121569,0.466667,0.705882}%
\pgfsetfillcolor{currentfill}%
\pgfsetfillopacity{0.500000}%
\pgfsetlinewidth{1.003750pt}%
\definecolor{currentstroke}{rgb}{0.000000,0.000000,0.000000}%
\pgfsetstrokecolor{currentstroke}%
\pgfsetdash{}{0pt}%
\pgfpathmoveto{\pgfqpoint{4.293437in}{4.306611in}}%
\pgfpathlineto{\pgfqpoint{4.443340in}{4.306611in}}%
\pgfpathlineto{\pgfqpoint{4.443340in}{4.484155in}}%
\pgfpathlineto{\pgfqpoint{4.293437in}{4.484155in}}%
\pgfpathlineto{\pgfqpoint{4.293437in}{4.306611in}}%
\pgfpathclose%
\pgfusepath{stroke,fill}%
\end{pgfscope}%
\begin{pgfscope}%
\pgfpathrectangle{\pgfqpoint{4.106058in}{4.306611in}}{\pgfqpoint{0.824468in}{0.462000in}}%
\pgfusepath{clip}%
\pgfsetbuttcap%
\pgfsetmiterjoin%
\definecolor{currentfill}{rgb}{0.121569,0.466667,0.705882}%
\pgfsetfillcolor{currentfill}%
\pgfsetfillopacity{0.500000}%
\pgfsetlinewidth{1.003750pt}%
\definecolor{currentstroke}{rgb}{0.000000,0.000000,0.000000}%
\pgfsetstrokecolor{currentstroke}%
\pgfsetdash{}{0pt}%
\pgfpathmoveto{\pgfqpoint{4.443340in}{4.306611in}}%
\pgfpathlineto{\pgfqpoint{4.593244in}{4.306611in}}%
\pgfpathlineto{\pgfqpoint{4.593244in}{4.484155in}}%
\pgfpathlineto{\pgfqpoint{4.443340in}{4.484155in}}%
\pgfpathlineto{\pgfqpoint{4.443340in}{4.306611in}}%
\pgfpathclose%
\pgfusepath{stroke,fill}%
\end{pgfscope}%
\begin{pgfscope}%
\pgfpathrectangle{\pgfqpoint{4.106058in}{4.306611in}}{\pgfqpoint{0.824468in}{0.462000in}}%
\pgfusepath{clip}%
\pgfsetbuttcap%
\pgfsetmiterjoin%
\definecolor{currentfill}{rgb}{0.121569,0.466667,0.705882}%
\pgfsetfillcolor{currentfill}%
\pgfsetfillopacity{0.500000}%
\pgfsetlinewidth{1.003750pt}%
\definecolor{currentstroke}{rgb}{0.000000,0.000000,0.000000}%
\pgfsetstrokecolor{currentstroke}%
\pgfsetdash{}{0pt}%
\pgfpathmoveto{\pgfqpoint{4.593244in}{4.306611in}}%
\pgfpathlineto{\pgfqpoint{4.743147in}{4.306611in}}%
\pgfpathlineto{\pgfqpoint{4.743147in}{4.476436in}}%
\pgfpathlineto{\pgfqpoint{4.593244in}{4.476436in}}%
\pgfpathlineto{\pgfqpoint{4.593244in}{4.306611in}}%
\pgfpathclose%
\pgfusepath{stroke,fill}%
\end{pgfscope}%
\begin{pgfscope}%
\pgfpathrectangle{\pgfqpoint{4.106058in}{4.306611in}}{\pgfqpoint{0.824468in}{0.462000in}}%
\pgfusepath{clip}%
\pgfsetbuttcap%
\pgfsetmiterjoin%
\definecolor{currentfill}{rgb}{0.121569,0.466667,0.705882}%
\pgfsetfillcolor{currentfill}%
\pgfsetfillopacity{0.500000}%
\pgfsetlinewidth{1.003750pt}%
\definecolor{currentstroke}{rgb}{0.000000,0.000000,0.000000}%
\pgfsetstrokecolor{currentstroke}%
\pgfsetdash{}{0pt}%
\pgfpathmoveto{\pgfqpoint{4.743147in}{4.306611in}}%
\pgfpathlineto{\pgfqpoint{4.893050in}{4.306611in}}%
\pgfpathlineto{\pgfqpoint{4.893050in}{4.433980in}}%
\pgfpathlineto{\pgfqpoint{4.743147in}{4.433980in}}%
\pgfpathlineto{\pgfqpoint{4.743147in}{4.306611in}}%
\pgfpathclose%
\pgfusepath{stroke,fill}%
\end{pgfscope}%
\begin{pgfscope}%
\pgfsetrectcap%
\pgfsetmiterjoin%
\pgfsetlinewidth{0.803000pt}%
\definecolor{currentstroke}{rgb}{0.000000,0.000000,0.000000}%
\pgfsetstrokecolor{currentstroke}%
\pgfsetdash{}{0pt}%
\pgfpathmoveto{\pgfqpoint{4.106058in}{4.306611in}}%
\pgfpathlineto{\pgfqpoint{4.106058in}{4.768611in}}%
\pgfusepath{stroke}%
\end{pgfscope}%
\begin{pgfscope}%
\pgfsetrectcap%
\pgfsetmiterjoin%
\pgfsetlinewidth{0.803000pt}%
\definecolor{currentstroke}{rgb}{0.000000,0.000000,0.000000}%
\pgfsetstrokecolor{currentstroke}%
\pgfsetdash{}{0pt}%
\pgfpathmoveto{\pgfqpoint{4.930526in}{4.306611in}}%
\pgfpathlineto{\pgfqpoint{4.930526in}{4.768611in}}%
\pgfusepath{stroke}%
\end{pgfscope}%
\begin{pgfscope}%
\pgfsetrectcap%
\pgfsetmiterjoin%
\pgfsetlinewidth{0.803000pt}%
\definecolor{currentstroke}{rgb}{0.000000,0.000000,0.000000}%
\pgfsetstrokecolor{currentstroke}%
\pgfsetdash{}{0pt}%
\pgfpathmoveto{\pgfqpoint{4.106058in}{4.306611in}}%
\pgfpathlineto{\pgfqpoint{4.930526in}{4.306611in}}%
\pgfusepath{stroke}%
\end{pgfscope}%
\begin{pgfscope}%
\pgfsetrectcap%
\pgfsetmiterjoin%
\pgfsetlinewidth{0.803000pt}%
\definecolor{currentstroke}{rgb}{0.000000,0.000000,0.000000}%
\pgfsetstrokecolor{currentstroke}%
\pgfsetdash{}{0pt}%
\pgfpathmoveto{\pgfqpoint{4.106058in}{4.768611in}}%
\pgfpathlineto{\pgfqpoint{4.930526in}{4.768611in}}%
\pgfusepath{stroke}%
\end{pgfscope}%
\begin{pgfscope}%
\definecolor{textcolor}{rgb}{0.000000,0.000000,0.000000}%
\pgfsetstrokecolor{textcolor}%
\pgfsetfillcolor{textcolor}%
\pgftext[x=4.518292in,y=4.851944in,,base]{\color{textcolor}\rmfamily\fontsize{11.000000}{13.200000}\selectfont APRIL}%
\end{pgfscope}%
\begin{pgfscope}%
\pgfsetbuttcap%
\pgfsetmiterjoin%
\definecolor{currentfill}{rgb}{1.000000,1.000000,1.000000}%
\pgfsetfillcolor{currentfill}%
\pgfsetlinewidth{0.000000pt}%
\definecolor{currentstroke}{rgb}{0.000000,0.000000,0.000000}%
\pgfsetstrokecolor{currentstroke}%
\pgfsetstrokeopacity{0.000000}%
\pgfsetdash{}{0pt}%
\pgfpathmoveto{\pgfqpoint{5.095420in}{4.306611in}}%
\pgfpathlineto{\pgfqpoint{5.919888in}{4.306611in}}%
\pgfpathlineto{\pgfqpoint{5.919888in}{4.768611in}}%
\pgfpathlineto{\pgfqpoint{5.095420in}{4.768611in}}%
\pgfpathlineto{\pgfqpoint{5.095420in}{4.306611in}}%
\pgfpathclose%
\pgfusepath{fill}%
\end{pgfscope}%
\begin{pgfscope}%
\pgfpathrectangle{\pgfqpoint{5.095420in}{4.306611in}}{\pgfqpoint{0.824468in}{0.462000in}}%
\pgfusepath{clip}%
\pgfsetbuttcap%
\pgfsetmiterjoin%
\definecolor{currentfill}{rgb}{0.121569,0.466667,0.705882}%
\pgfsetfillcolor{currentfill}%
\pgfsetfillopacity{0.500000}%
\pgfsetlinewidth{1.003750pt}%
\definecolor{currentstroke}{rgb}{0.000000,0.000000,0.000000}%
\pgfsetstrokecolor{currentstroke}%
\pgfsetdash{}{0pt}%
\pgfpathmoveto{\pgfqpoint{5.132895in}{4.306611in}}%
\pgfpathlineto{\pgfqpoint{5.282799in}{4.306611in}}%
\pgfpathlineto{\pgfqpoint{5.282799in}{4.746611in}}%
\pgfpathlineto{\pgfqpoint{5.132895in}{4.746611in}}%
\pgfpathlineto{\pgfqpoint{5.132895in}{4.306611in}}%
\pgfpathclose%
\pgfusepath{stroke,fill}%
\end{pgfscope}%
\begin{pgfscope}%
\pgfpathrectangle{\pgfqpoint{5.095420in}{4.306611in}}{\pgfqpoint{0.824468in}{0.462000in}}%
\pgfusepath{clip}%
\pgfsetbuttcap%
\pgfsetmiterjoin%
\definecolor{currentfill}{rgb}{0.121569,0.466667,0.705882}%
\pgfsetfillcolor{currentfill}%
\pgfsetfillopacity{0.500000}%
\pgfsetlinewidth{1.003750pt}%
\definecolor{currentstroke}{rgb}{0.000000,0.000000,0.000000}%
\pgfsetstrokecolor{currentstroke}%
\pgfsetdash{}{0pt}%
\pgfpathmoveto{\pgfqpoint{5.282799in}{4.306611in}}%
\pgfpathlineto{\pgfqpoint{5.432702in}{4.306611in}}%
\pgfpathlineto{\pgfqpoint{5.432702in}{4.405232in}}%
\pgfpathlineto{\pgfqpoint{5.282799in}{4.405232in}}%
\pgfpathlineto{\pgfqpoint{5.282799in}{4.306611in}}%
\pgfpathclose%
\pgfusepath{stroke,fill}%
\end{pgfscope}%
\begin{pgfscope}%
\pgfpathrectangle{\pgfqpoint{5.095420in}{4.306611in}}{\pgfqpoint{0.824468in}{0.462000in}}%
\pgfusepath{clip}%
\pgfsetbuttcap%
\pgfsetmiterjoin%
\definecolor{currentfill}{rgb}{0.121569,0.466667,0.705882}%
\pgfsetfillcolor{currentfill}%
\pgfsetfillopacity{0.500000}%
\pgfsetlinewidth{1.003750pt}%
\definecolor{currentstroke}{rgb}{0.000000,0.000000,0.000000}%
\pgfsetstrokecolor{currentstroke}%
\pgfsetdash{}{0pt}%
\pgfpathmoveto{\pgfqpoint{5.432702in}{4.306611in}}%
\pgfpathlineto{\pgfqpoint{5.582605in}{4.306611in}}%
\pgfpathlineto{\pgfqpoint{5.582605in}{4.359715in}}%
\pgfpathlineto{\pgfqpoint{5.432702in}{4.359715in}}%
\pgfpathlineto{\pgfqpoint{5.432702in}{4.306611in}}%
\pgfpathclose%
\pgfusepath{stroke,fill}%
\end{pgfscope}%
\begin{pgfscope}%
\pgfpathrectangle{\pgfqpoint{5.095420in}{4.306611in}}{\pgfqpoint{0.824468in}{0.462000in}}%
\pgfusepath{clip}%
\pgfsetbuttcap%
\pgfsetmiterjoin%
\definecolor{currentfill}{rgb}{0.121569,0.466667,0.705882}%
\pgfsetfillcolor{currentfill}%
\pgfsetfillopacity{0.500000}%
\pgfsetlinewidth{1.003750pt}%
\definecolor{currentstroke}{rgb}{0.000000,0.000000,0.000000}%
\pgfsetstrokecolor{currentstroke}%
\pgfsetdash{}{0pt}%
\pgfpathmoveto{\pgfqpoint{5.582605in}{4.306611in}}%
\pgfpathlineto{\pgfqpoint{5.732509in}{4.306611in}}%
\pgfpathlineto{\pgfqpoint{5.732509in}{4.390059in}}%
\pgfpathlineto{\pgfqpoint{5.582605in}{4.390059in}}%
\pgfpathlineto{\pgfqpoint{5.582605in}{4.306611in}}%
\pgfpathclose%
\pgfusepath{stroke,fill}%
\end{pgfscope}%
\begin{pgfscope}%
\pgfpathrectangle{\pgfqpoint{5.095420in}{4.306611in}}{\pgfqpoint{0.824468in}{0.462000in}}%
\pgfusepath{clip}%
\pgfsetbuttcap%
\pgfsetmiterjoin%
\definecolor{currentfill}{rgb}{0.121569,0.466667,0.705882}%
\pgfsetfillcolor{currentfill}%
\pgfsetfillopacity{0.500000}%
\pgfsetlinewidth{1.003750pt}%
\definecolor{currentstroke}{rgb}{0.000000,0.000000,0.000000}%
\pgfsetstrokecolor{currentstroke}%
\pgfsetdash{}{0pt}%
\pgfpathmoveto{\pgfqpoint{5.732509in}{4.306611in}}%
\pgfpathlineto{\pgfqpoint{5.882412in}{4.306611in}}%
\pgfpathlineto{\pgfqpoint{5.882412in}{4.359715in}}%
\pgfpathlineto{\pgfqpoint{5.732509in}{4.359715in}}%
\pgfpathlineto{\pgfqpoint{5.732509in}{4.306611in}}%
\pgfpathclose%
\pgfusepath{stroke,fill}%
\end{pgfscope}%
\begin{pgfscope}%
\pgfsetrectcap%
\pgfsetmiterjoin%
\pgfsetlinewidth{0.803000pt}%
\definecolor{currentstroke}{rgb}{0.000000,0.000000,0.000000}%
\pgfsetstrokecolor{currentstroke}%
\pgfsetdash{}{0pt}%
\pgfpathmoveto{\pgfqpoint{5.095420in}{4.306611in}}%
\pgfpathlineto{\pgfqpoint{5.095420in}{4.768611in}}%
\pgfusepath{stroke}%
\end{pgfscope}%
\begin{pgfscope}%
\pgfsetrectcap%
\pgfsetmiterjoin%
\pgfsetlinewidth{0.803000pt}%
\definecolor{currentstroke}{rgb}{0.000000,0.000000,0.000000}%
\pgfsetstrokecolor{currentstroke}%
\pgfsetdash{}{0pt}%
\pgfpathmoveto{\pgfqpoint{5.919888in}{4.306611in}}%
\pgfpathlineto{\pgfqpoint{5.919888in}{4.768611in}}%
\pgfusepath{stroke}%
\end{pgfscope}%
\begin{pgfscope}%
\pgfsetrectcap%
\pgfsetmiterjoin%
\pgfsetlinewidth{0.803000pt}%
\definecolor{currentstroke}{rgb}{0.000000,0.000000,0.000000}%
\pgfsetstrokecolor{currentstroke}%
\pgfsetdash{}{0pt}%
\pgfpathmoveto{\pgfqpoint{5.095420in}{4.306611in}}%
\pgfpathlineto{\pgfqpoint{5.919888in}{4.306611in}}%
\pgfusepath{stroke}%
\end{pgfscope}%
\begin{pgfscope}%
\pgfsetrectcap%
\pgfsetmiterjoin%
\pgfsetlinewidth{0.803000pt}%
\definecolor{currentstroke}{rgb}{0.000000,0.000000,0.000000}%
\pgfsetstrokecolor{currentstroke}%
\pgfsetdash{}{0pt}%
\pgfpathmoveto{\pgfqpoint{5.095420in}{4.768611in}}%
\pgfpathlineto{\pgfqpoint{5.919888in}{4.768611in}}%
\pgfusepath{stroke}%
\end{pgfscope}%
\begin{pgfscope}%
\definecolor{textcolor}{rgb}{0.000000,0.000000,0.000000}%
\pgfsetstrokecolor{textcolor}%
\pgfsetfillcolor{textcolor}%
\pgftext[x=5.507654in,y=4.851944in,,base]{\color{textcolor}\rmfamily\fontsize{11.000000}{13.200000}\selectfont SantéVet}%
\end{pgfscope}%
\begin{pgfscope}%
\pgfsetbuttcap%
\pgfsetmiterjoin%
\definecolor{currentfill}{rgb}{1.000000,1.000000,1.000000}%
\pgfsetfillcolor{currentfill}%
\pgfsetlinewidth{0.000000pt}%
\definecolor{currentstroke}{rgb}{0.000000,0.000000,0.000000}%
\pgfsetstrokecolor{currentstroke}%
\pgfsetstrokeopacity{0.000000}%
\pgfsetdash{}{0pt}%
\pgfpathmoveto{\pgfqpoint{6.084781in}{4.306611in}}%
\pgfpathlineto{\pgfqpoint{6.909249in}{4.306611in}}%
\pgfpathlineto{\pgfqpoint{6.909249in}{4.768611in}}%
\pgfpathlineto{\pgfqpoint{6.084781in}{4.768611in}}%
\pgfpathlineto{\pgfqpoint{6.084781in}{4.306611in}}%
\pgfpathclose%
\pgfusepath{fill}%
\end{pgfscope}%
\begin{pgfscope}%
\pgfpathrectangle{\pgfqpoint{6.084781in}{4.306611in}}{\pgfqpoint{0.824468in}{0.462000in}}%
\pgfusepath{clip}%
\pgfsetbuttcap%
\pgfsetmiterjoin%
\definecolor{currentfill}{rgb}{0.121569,0.466667,0.705882}%
\pgfsetfillcolor{currentfill}%
\pgfsetfillopacity{0.500000}%
\pgfsetlinewidth{1.003750pt}%
\definecolor{currentstroke}{rgb}{0.000000,0.000000,0.000000}%
\pgfsetstrokecolor{currentstroke}%
\pgfsetdash{}{0pt}%
\pgfpathmoveto{\pgfqpoint{6.122257in}{4.306611in}}%
\pgfpathlineto{\pgfqpoint{6.272160in}{4.306611in}}%
\pgfpathlineto{\pgfqpoint{6.272160in}{4.746611in}}%
\pgfpathlineto{\pgfqpoint{6.122257in}{4.746611in}}%
\pgfpathlineto{\pgfqpoint{6.122257in}{4.306611in}}%
\pgfpathclose%
\pgfusepath{stroke,fill}%
\end{pgfscope}%
\begin{pgfscope}%
\pgfpathrectangle{\pgfqpoint{6.084781in}{4.306611in}}{\pgfqpoint{0.824468in}{0.462000in}}%
\pgfusepath{clip}%
\pgfsetbuttcap%
\pgfsetmiterjoin%
\definecolor{currentfill}{rgb}{0.121569,0.466667,0.705882}%
\pgfsetfillcolor{currentfill}%
\pgfsetfillopacity{0.500000}%
\pgfsetlinewidth{1.003750pt}%
\definecolor{currentstroke}{rgb}{0.000000,0.000000,0.000000}%
\pgfsetstrokecolor{currentstroke}%
\pgfsetdash{}{0pt}%
\pgfpathmoveto{\pgfqpoint{6.272160in}{4.306611in}}%
\pgfpathlineto{\pgfqpoint{6.422064in}{4.306611in}}%
\pgfpathlineto{\pgfqpoint{6.422064in}{4.404389in}}%
\pgfpathlineto{\pgfqpoint{6.272160in}{4.404389in}}%
\pgfpathlineto{\pgfqpoint{6.272160in}{4.306611in}}%
\pgfpathclose%
\pgfusepath{stroke,fill}%
\end{pgfscope}%
\begin{pgfscope}%
\pgfpathrectangle{\pgfqpoint{6.084781in}{4.306611in}}{\pgfqpoint{0.824468in}{0.462000in}}%
\pgfusepath{clip}%
\pgfsetbuttcap%
\pgfsetmiterjoin%
\definecolor{currentfill}{rgb}{0.121569,0.466667,0.705882}%
\pgfsetfillcolor{currentfill}%
\pgfsetfillopacity{0.500000}%
\pgfsetlinewidth{1.003750pt}%
\definecolor{currentstroke}{rgb}{0.000000,0.000000,0.000000}%
\pgfsetstrokecolor{currentstroke}%
\pgfsetdash{}{0pt}%
\pgfpathmoveto{\pgfqpoint{6.422064in}{4.306611in}}%
\pgfpathlineto{\pgfqpoint{6.571967in}{4.306611in}}%
\pgfpathlineto{\pgfqpoint{6.571967in}{4.327867in}}%
\pgfpathlineto{\pgfqpoint{6.422064in}{4.327867in}}%
\pgfpathlineto{\pgfqpoint{6.422064in}{4.306611in}}%
\pgfpathclose%
\pgfusepath{stroke,fill}%
\end{pgfscope}%
\begin{pgfscope}%
\pgfpathrectangle{\pgfqpoint{6.084781in}{4.306611in}}{\pgfqpoint{0.824468in}{0.462000in}}%
\pgfusepath{clip}%
\pgfsetbuttcap%
\pgfsetmiterjoin%
\definecolor{currentfill}{rgb}{0.121569,0.466667,0.705882}%
\pgfsetfillcolor{currentfill}%
\pgfsetfillopacity{0.500000}%
\pgfsetlinewidth{1.003750pt}%
\definecolor{currentstroke}{rgb}{0.000000,0.000000,0.000000}%
\pgfsetstrokecolor{currentstroke}%
\pgfsetdash{}{0pt}%
\pgfpathmoveto{\pgfqpoint{6.571967in}{4.306611in}}%
\pgfpathlineto{\pgfqpoint{6.721870in}{4.306611in}}%
\pgfpathlineto{\pgfqpoint{6.721870in}{4.312988in}}%
\pgfpathlineto{\pgfqpoint{6.571967in}{4.312988in}}%
\pgfpathlineto{\pgfqpoint{6.571967in}{4.306611in}}%
\pgfpathclose%
\pgfusepath{stroke,fill}%
\end{pgfscope}%
\begin{pgfscope}%
\pgfpathrectangle{\pgfqpoint{6.084781in}{4.306611in}}{\pgfqpoint{0.824468in}{0.462000in}}%
\pgfusepath{clip}%
\pgfsetbuttcap%
\pgfsetmiterjoin%
\definecolor{currentfill}{rgb}{0.121569,0.466667,0.705882}%
\pgfsetfillcolor{currentfill}%
\pgfsetfillopacity{0.500000}%
\pgfsetlinewidth{1.003750pt}%
\definecolor{currentstroke}{rgb}{0.000000,0.000000,0.000000}%
\pgfsetstrokecolor{currentstroke}%
\pgfsetdash{}{0pt}%
\pgfpathmoveto{\pgfqpoint{6.721870in}{4.306611in}}%
\pgfpathlineto{\pgfqpoint{6.871774in}{4.306611in}}%
\pgfpathlineto{\pgfqpoint{6.871774in}{4.308737in}}%
\pgfpathlineto{\pgfqpoint{6.721870in}{4.308737in}}%
\pgfpathlineto{\pgfqpoint{6.721870in}{4.306611in}}%
\pgfpathclose%
\pgfusepath{stroke,fill}%
\end{pgfscope}%
\begin{pgfscope}%
\pgfsetrectcap%
\pgfsetmiterjoin%
\pgfsetlinewidth{0.803000pt}%
\definecolor{currentstroke}{rgb}{0.000000,0.000000,0.000000}%
\pgfsetstrokecolor{currentstroke}%
\pgfsetdash{}{0pt}%
\pgfpathmoveto{\pgfqpoint{6.084781in}{4.306611in}}%
\pgfpathlineto{\pgfqpoint{6.084781in}{4.768611in}}%
\pgfusepath{stroke}%
\end{pgfscope}%
\begin{pgfscope}%
\pgfsetrectcap%
\pgfsetmiterjoin%
\pgfsetlinewidth{0.803000pt}%
\definecolor{currentstroke}{rgb}{0.000000,0.000000,0.000000}%
\pgfsetstrokecolor{currentstroke}%
\pgfsetdash{}{0pt}%
\pgfpathmoveto{\pgfqpoint{6.909249in}{4.306611in}}%
\pgfpathlineto{\pgfqpoint{6.909249in}{4.768611in}}%
\pgfusepath{stroke}%
\end{pgfscope}%
\begin{pgfscope}%
\pgfsetrectcap%
\pgfsetmiterjoin%
\pgfsetlinewidth{0.803000pt}%
\definecolor{currentstroke}{rgb}{0.000000,0.000000,0.000000}%
\pgfsetstrokecolor{currentstroke}%
\pgfsetdash{}{0pt}%
\pgfpathmoveto{\pgfqpoint{6.084781in}{4.306611in}}%
\pgfpathlineto{\pgfqpoint{6.909249in}{4.306611in}}%
\pgfusepath{stroke}%
\end{pgfscope}%
\begin{pgfscope}%
\pgfsetrectcap%
\pgfsetmiterjoin%
\pgfsetlinewidth{0.803000pt}%
\definecolor{currentstroke}{rgb}{0.000000,0.000000,0.000000}%
\pgfsetstrokecolor{currentstroke}%
\pgfsetdash{}{0pt}%
\pgfpathmoveto{\pgfqpoint{6.084781in}{4.768611in}}%
\pgfpathlineto{\pgfqpoint{6.909249in}{4.768611in}}%
\pgfusepath{stroke}%
\end{pgfscope}%
\begin{pgfscope}%
\definecolor{textcolor}{rgb}{0.000000,0.000000,0.000000}%
\pgfsetstrokecolor{textcolor}%
\pgfsetfillcolor{textcolor}%
\pgftext[x=6.497015in,y=4.851944in,,base]{\color{textcolor}\rmfamily\fontsize{11.000000}{13.200000}\selectfont Mercer}%
\end{pgfscope}%
\begin{pgfscope}%
\pgfsetbuttcap%
\pgfsetmiterjoin%
\definecolor{currentfill}{rgb}{1.000000,1.000000,1.000000}%
\pgfsetfillcolor{currentfill}%
\pgfsetlinewidth{0.000000pt}%
\definecolor{currentstroke}{rgb}{0.000000,0.000000,0.000000}%
\pgfsetstrokecolor{currentstroke}%
\pgfsetstrokeopacity{0.000000}%
\pgfsetdash{}{0pt}%
\pgfpathmoveto{\pgfqpoint{7.074143in}{4.306611in}}%
\pgfpathlineto{\pgfqpoint{7.898611in}{4.306611in}}%
\pgfpathlineto{\pgfqpoint{7.898611in}{4.768611in}}%
\pgfpathlineto{\pgfqpoint{7.074143in}{4.768611in}}%
\pgfpathlineto{\pgfqpoint{7.074143in}{4.306611in}}%
\pgfpathclose%
\pgfusepath{fill}%
\end{pgfscope}%
\begin{pgfscope}%
\pgfpathrectangle{\pgfqpoint{7.074143in}{4.306611in}}{\pgfqpoint{0.824468in}{0.462000in}}%
\pgfusepath{clip}%
\pgfsetbuttcap%
\pgfsetmiterjoin%
\definecolor{currentfill}{rgb}{0.121569,0.466667,0.705882}%
\pgfsetfillcolor{currentfill}%
\pgfsetfillopacity{0.500000}%
\pgfsetlinewidth{1.003750pt}%
\definecolor{currentstroke}{rgb}{0.000000,0.000000,0.000000}%
\pgfsetstrokecolor{currentstroke}%
\pgfsetdash{}{0pt}%
\pgfpathmoveto{\pgfqpoint{7.111619in}{4.306611in}}%
\pgfpathlineto{\pgfqpoint{7.261522in}{4.306611in}}%
\pgfpathlineto{\pgfqpoint{7.261522in}{4.746611in}}%
\pgfpathlineto{\pgfqpoint{7.111619in}{4.746611in}}%
\pgfpathlineto{\pgfqpoint{7.111619in}{4.306611in}}%
\pgfpathclose%
\pgfusepath{stroke,fill}%
\end{pgfscope}%
\begin{pgfscope}%
\pgfpathrectangle{\pgfqpoint{7.074143in}{4.306611in}}{\pgfqpoint{0.824468in}{0.462000in}}%
\pgfusepath{clip}%
\pgfsetbuttcap%
\pgfsetmiterjoin%
\definecolor{currentfill}{rgb}{0.121569,0.466667,0.705882}%
\pgfsetfillcolor{currentfill}%
\pgfsetfillopacity{0.500000}%
\pgfsetlinewidth{1.003750pt}%
\definecolor{currentstroke}{rgb}{0.000000,0.000000,0.000000}%
\pgfsetstrokecolor{currentstroke}%
\pgfsetdash{}{0pt}%
\pgfpathmoveto{\pgfqpoint{7.261522in}{4.306611in}}%
\pgfpathlineto{\pgfqpoint{7.411425in}{4.306611in}}%
\pgfpathlineto{\pgfqpoint{7.411425in}{4.374303in}}%
\pgfpathlineto{\pgfqpoint{7.261522in}{4.374303in}}%
\pgfpathlineto{\pgfqpoint{7.261522in}{4.306611in}}%
\pgfpathclose%
\pgfusepath{stroke,fill}%
\end{pgfscope}%
\begin{pgfscope}%
\pgfpathrectangle{\pgfqpoint{7.074143in}{4.306611in}}{\pgfqpoint{0.824468in}{0.462000in}}%
\pgfusepath{clip}%
\pgfsetbuttcap%
\pgfsetmiterjoin%
\definecolor{currentfill}{rgb}{0.121569,0.466667,0.705882}%
\pgfsetfillcolor{currentfill}%
\pgfsetfillopacity{0.500000}%
\pgfsetlinewidth{1.003750pt}%
\definecolor{currentstroke}{rgb}{0.000000,0.000000,0.000000}%
\pgfsetstrokecolor{currentstroke}%
\pgfsetdash{}{0pt}%
\pgfpathmoveto{\pgfqpoint{7.411425in}{4.306611in}}%
\pgfpathlineto{\pgfqpoint{7.561329in}{4.306611in}}%
\pgfpathlineto{\pgfqpoint{7.561329in}{4.393644in}}%
\pgfpathlineto{\pgfqpoint{7.411425in}{4.393644in}}%
\pgfpathlineto{\pgfqpoint{7.411425in}{4.306611in}}%
\pgfpathclose%
\pgfusepath{stroke,fill}%
\end{pgfscope}%
\begin{pgfscope}%
\pgfpathrectangle{\pgfqpoint{7.074143in}{4.306611in}}{\pgfqpoint{0.824468in}{0.462000in}}%
\pgfusepath{clip}%
\pgfsetbuttcap%
\pgfsetmiterjoin%
\definecolor{currentfill}{rgb}{0.121569,0.466667,0.705882}%
\pgfsetfillcolor{currentfill}%
\pgfsetfillopacity{0.500000}%
\pgfsetlinewidth{1.003750pt}%
\definecolor{currentstroke}{rgb}{0.000000,0.000000,0.000000}%
\pgfsetstrokecolor{currentstroke}%
\pgfsetdash{}{0pt}%
\pgfpathmoveto{\pgfqpoint{7.561329in}{4.306611in}}%
\pgfpathlineto{\pgfqpoint{7.711232in}{4.306611in}}%
\pgfpathlineto{\pgfqpoint{7.711232in}{4.311446in}}%
\pgfpathlineto{\pgfqpoint{7.561329in}{4.311446in}}%
\pgfpathlineto{\pgfqpoint{7.561329in}{4.306611in}}%
\pgfpathclose%
\pgfusepath{stroke,fill}%
\end{pgfscope}%
\begin{pgfscope}%
\pgfpathrectangle{\pgfqpoint{7.074143in}{4.306611in}}{\pgfqpoint{0.824468in}{0.462000in}}%
\pgfusepath{clip}%
\pgfsetbuttcap%
\pgfsetmiterjoin%
\definecolor{currentfill}{rgb}{0.121569,0.466667,0.705882}%
\pgfsetfillcolor{currentfill}%
\pgfsetfillopacity{0.500000}%
\pgfsetlinewidth{1.003750pt}%
\definecolor{currentstroke}{rgb}{0.000000,0.000000,0.000000}%
\pgfsetstrokecolor{currentstroke}%
\pgfsetdash{}{0pt}%
\pgfpathmoveto{\pgfqpoint{7.711232in}{4.306611in}}%
\pgfpathlineto{\pgfqpoint{7.861135in}{4.306611in}}%
\pgfpathlineto{\pgfqpoint{7.861135in}{4.325952in}}%
\pgfpathlineto{\pgfqpoint{7.711232in}{4.325952in}}%
\pgfpathlineto{\pgfqpoint{7.711232in}{4.306611in}}%
\pgfpathclose%
\pgfusepath{stroke,fill}%
\end{pgfscope}%
\begin{pgfscope}%
\pgfsetrectcap%
\pgfsetmiterjoin%
\pgfsetlinewidth{0.803000pt}%
\definecolor{currentstroke}{rgb}{0.000000,0.000000,0.000000}%
\pgfsetstrokecolor{currentstroke}%
\pgfsetdash{}{0pt}%
\pgfpathmoveto{\pgfqpoint{7.074143in}{4.306611in}}%
\pgfpathlineto{\pgfqpoint{7.074143in}{4.768611in}}%
\pgfusepath{stroke}%
\end{pgfscope}%
\begin{pgfscope}%
\pgfsetrectcap%
\pgfsetmiterjoin%
\pgfsetlinewidth{0.803000pt}%
\definecolor{currentstroke}{rgb}{0.000000,0.000000,0.000000}%
\pgfsetstrokecolor{currentstroke}%
\pgfsetdash{}{0pt}%
\pgfpathmoveto{\pgfqpoint{7.898611in}{4.306611in}}%
\pgfpathlineto{\pgfqpoint{7.898611in}{4.768611in}}%
\pgfusepath{stroke}%
\end{pgfscope}%
\begin{pgfscope}%
\pgfsetrectcap%
\pgfsetmiterjoin%
\pgfsetlinewidth{0.803000pt}%
\definecolor{currentstroke}{rgb}{0.000000,0.000000,0.000000}%
\pgfsetstrokecolor{currentstroke}%
\pgfsetdash{}{0pt}%
\pgfpathmoveto{\pgfqpoint{7.074143in}{4.306611in}}%
\pgfpathlineto{\pgfqpoint{7.898611in}{4.306611in}}%
\pgfusepath{stroke}%
\end{pgfscope}%
\begin{pgfscope}%
\pgfsetrectcap%
\pgfsetmiterjoin%
\pgfsetlinewidth{0.803000pt}%
\definecolor{currentstroke}{rgb}{0.000000,0.000000,0.000000}%
\pgfsetstrokecolor{currentstroke}%
\pgfsetdash{}{0pt}%
\pgfpathmoveto{\pgfqpoint{7.074143in}{4.768611in}}%
\pgfpathlineto{\pgfqpoint{7.898611in}{4.768611in}}%
\pgfusepath{stroke}%
\end{pgfscope}%
\begin{pgfscope}%
\definecolor{textcolor}{rgb}{0.000000,0.000000,0.000000}%
\pgfsetstrokecolor{textcolor}%
\pgfsetfillcolor{textcolor}%
\pgftext[x=7.486377in,y=4.851944in,,base]{\color{textcolor}\rmfamily\fontsize{11.000000}{13.200000}\selectfont Generali}%
\end{pgfscope}%
\begin{pgfscope}%
\pgfsetbuttcap%
\pgfsetmiterjoin%
\definecolor{currentfill}{rgb}{1.000000,1.000000,1.000000}%
\pgfsetfillcolor{currentfill}%
\pgfsetlinewidth{0.000000pt}%
\definecolor{currentstroke}{rgb}{0.000000,0.000000,0.000000}%
\pgfsetstrokecolor{currentstroke}%
\pgfsetstrokeopacity{0.000000}%
\pgfsetdash{}{0pt}%
\pgfpathmoveto{\pgfqpoint{0.148611in}{3.613611in}}%
\pgfpathlineto{\pgfqpoint{0.973079in}{3.613611in}}%
\pgfpathlineto{\pgfqpoint{0.973079in}{4.075611in}}%
\pgfpathlineto{\pgfqpoint{0.148611in}{4.075611in}}%
\pgfpathlineto{\pgfqpoint{0.148611in}{3.613611in}}%
\pgfpathclose%
\pgfusepath{fill}%
\end{pgfscope}%
\begin{pgfscope}%
\pgfpathrectangle{\pgfqpoint{0.148611in}{3.613611in}}{\pgfqpoint{0.824468in}{0.462000in}}%
\pgfusepath{clip}%
\pgfsetbuttcap%
\pgfsetmiterjoin%
\definecolor{currentfill}{rgb}{0.121569,0.466667,0.705882}%
\pgfsetfillcolor{currentfill}%
\pgfsetfillopacity{0.500000}%
\pgfsetlinewidth{1.003750pt}%
\definecolor{currentstroke}{rgb}{0.000000,0.000000,0.000000}%
\pgfsetstrokecolor{currentstroke}%
\pgfsetdash{}{0pt}%
\pgfpathmoveto{\pgfqpoint{0.186087in}{3.613611in}}%
\pgfpathlineto{\pgfqpoint{0.335990in}{3.613611in}}%
\pgfpathlineto{\pgfqpoint{0.335990in}{4.053611in}}%
\pgfpathlineto{\pgfqpoint{0.186087in}{4.053611in}}%
\pgfpathlineto{\pgfqpoint{0.186087in}{3.613611in}}%
\pgfpathclose%
\pgfusepath{stroke,fill}%
\end{pgfscope}%
\begin{pgfscope}%
\pgfpathrectangle{\pgfqpoint{0.148611in}{3.613611in}}{\pgfqpoint{0.824468in}{0.462000in}}%
\pgfusepath{clip}%
\pgfsetbuttcap%
\pgfsetmiterjoin%
\definecolor{currentfill}{rgb}{0.121569,0.466667,0.705882}%
\pgfsetfillcolor{currentfill}%
\pgfsetfillopacity{0.500000}%
\pgfsetlinewidth{1.003750pt}%
\definecolor{currentstroke}{rgb}{0.000000,0.000000,0.000000}%
\pgfsetstrokecolor{currentstroke}%
\pgfsetdash{}{0pt}%
\pgfpathmoveto{\pgfqpoint{0.335990in}{3.613611in}}%
\pgfpathlineto{\pgfqpoint{0.485894in}{3.613611in}}%
\pgfpathlineto{\pgfqpoint{0.485894in}{3.742280in}}%
\pgfpathlineto{\pgfqpoint{0.335990in}{3.742280in}}%
\pgfpathlineto{\pgfqpoint{0.335990in}{3.613611in}}%
\pgfpathclose%
\pgfusepath{stroke,fill}%
\end{pgfscope}%
\begin{pgfscope}%
\pgfpathrectangle{\pgfqpoint{0.148611in}{3.613611in}}{\pgfqpoint{0.824468in}{0.462000in}}%
\pgfusepath{clip}%
\pgfsetbuttcap%
\pgfsetmiterjoin%
\definecolor{currentfill}{rgb}{0.121569,0.466667,0.705882}%
\pgfsetfillcolor{currentfill}%
\pgfsetfillopacity{0.500000}%
\pgfsetlinewidth{1.003750pt}%
\definecolor{currentstroke}{rgb}{0.000000,0.000000,0.000000}%
\pgfsetstrokecolor{currentstroke}%
\pgfsetdash{}{0pt}%
\pgfpathmoveto{\pgfqpoint{0.485894in}{3.613611in}}%
\pgfpathlineto{\pgfqpoint{0.635797in}{3.613611in}}%
\pgfpathlineto{\pgfqpoint{0.635797in}{3.672201in}}%
\pgfpathlineto{\pgfqpoint{0.485894in}{3.672201in}}%
\pgfpathlineto{\pgfqpoint{0.485894in}{3.613611in}}%
\pgfpathclose%
\pgfusepath{stroke,fill}%
\end{pgfscope}%
\begin{pgfscope}%
\pgfpathrectangle{\pgfqpoint{0.148611in}{3.613611in}}{\pgfqpoint{0.824468in}{0.462000in}}%
\pgfusepath{clip}%
\pgfsetbuttcap%
\pgfsetmiterjoin%
\definecolor{currentfill}{rgb}{0.121569,0.466667,0.705882}%
\pgfsetfillcolor{currentfill}%
\pgfsetfillopacity{0.500000}%
\pgfsetlinewidth{1.003750pt}%
\definecolor{currentstroke}{rgb}{0.000000,0.000000,0.000000}%
\pgfsetstrokecolor{currentstroke}%
\pgfsetdash{}{0pt}%
\pgfpathmoveto{\pgfqpoint{0.635797in}{3.613611in}}%
\pgfpathlineto{\pgfqpoint{0.785700in}{3.613611in}}%
\pgfpathlineto{\pgfqpoint{0.785700in}{3.634290in}}%
\pgfpathlineto{\pgfqpoint{0.635797in}{3.634290in}}%
\pgfpathlineto{\pgfqpoint{0.635797in}{3.613611in}}%
\pgfpathclose%
\pgfusepath{stroke,fill}%
\end{pgfscope}%
\begin{pgfscope}%
\pgfpathrectangle{\pgfqpoint{0.148611in}{3.613611in}}{\pgfqpoint{0.824468in}{0.462000in}}%
\pgfusepath{clip}%
\pgfsetbuttcap%
\pgfsetmiterjoin%
\definecolor{currentfill}{rgb}{0.121569,0.466667,0.705882}%
\pgfsetfillcolor{currentfill}%
\pgfsetfillopacity{0.500000}%
\pgfsetlinewidth{1.003750pt}%
\definecolor{currentstroke}{rgb}{0.000000,0.000000,0.000000}%
\pgfsetstrokecolor{currentstroke}%
\pgfsetdash{}{0pt}%
\pgfpathmoveto{\pgfqpoint{0.785700in}{3.613611in}}%
\pgfpathlineto{\pgfqpoint{0.935603in}{3.613611in}}%
\pgfpathlineto{\pgfqpoint{0.935603in}{3.625099in}}%
\pgfpathlineto{\pgfqpoint{0.785700in}{3.625099in}}%
\pgfpathlineto{\pgfqpoint{0.785700in}{3.613611in}}%
\pgfpathclose%
\pgfusepath{stroke,fill}%
\end{pgfscope}%
\begin{pgfscope}%
\pgfsetrectcap%
\pgfsetmiterjoin%
\pgfsetlinewidth{0.803000pt}%
\definecolor{currentstroke}{rgb}{0.000000,0.000000,0.000000}%
\pgfsetstrokecolor{currentstroke}%
\pgfsetdash{}{0pt}%
\pgfpathmoveto{\pgfqpoint{0.148611in}{3.613611in}}%
\pgfpathlineto{\pgfqpoint{0.148611in}{4.075611in}}%
\pgfusepath{stroke}%
\end{pgfscope}%
\begin{pgfscope}%
\pgfsetrectcap%
\pgfsetmiterjoin%
\pgfsetlinewidth{0.803000pt}%
\definecolor{currentstroke}{rgb}{0.000000,0.000000,0.000000}%
\pgfsetstrokecolor{currentstroke}%
\pgfsetdash{}{0pt}%
\pgfpathmoveto{\pgfqpoint{0.973079in}{3.613611in}}%
\pgfpathlineto{\pgfqpoint{0.973079in}{4.075611in}}%
\pgfusepath{stroke}%
\end{pgfscope}%
\begin{pgfscope}%
\pgfsetrectcap%
\pgfsetmiterjoin%
\pgfsetlinewidth{0.803000pt}%
\definecolor{currentstroke}{rgb}{0.000000,0.000000,0.000000}%
\pgfsetstrokecolor{currentstroke}%
\pgfsetdash{}{0pt}%
\pgfpathmoveto{\pgfqpoint{0.148611in}{3.613611in}}%
\pgfpathlineto{\pgfqpoint{0.973079in}{3.613611in}}%
\pgfusepath{stroke}%
\end{pgfscope}%
\begin{pgfscope}%
\pgfsetrectcap%
\pgfsetmiterjoin%
\pgfsetlinewidth{0.803000pt}%
\definecolor{currentstroke}{rgb}{0.000000,0.000000,0.000000}%
\pgfsetstrokecolor{currentstroke}%
\pgfsetdash{}{0pt}%
\pgfpathmoveto{\pgfqpoint{0.148611in}{4.075611in}}%
\pgfpathlineto{\pgfqpoint{0.973079in}{4.075611in}}%
\pgfusepath{stroke}%
\end{pgfscope}%
\begin{pgfscope}%
\definecolor{textcolor}{rgb}{0.000000,0.000000,0.000000}%
\pgfsetstrokecolor{textcolor}%
\pgfsetfillcolor{textcolor}%
\pgftext[x=0.560845in,y=4.158944in,,base]{\color{textcolor}\rmfamily\fontsize{11.000000}{13.200000}\selectfont Allianz}%
\end{pgfscope}%
\begin{pgfscope}%
\pgfsetbuttcap%
\pgfsetmiterjoin%
\definecolor{currentfill}{rgb}{1.000000,1.000000,1.000000}%
\pgfsetfillcolor{currentfill}%
\pgfsetlinewidth{0.000000pt}%
\definecolor{currentstroke}{rgb}{0.000000,0.000000,0.000000}%
\pgfsetstrokecolor{currentstroke}%
\pgfsetstrokeopacity{0.000000}%
\pgfsetdash{}{0pt}%
\pgfpathmoveto{\pgfqpoint{1.137973in}{3.613611in}}%
\pgfpathlineto{\pgfqpoint{1.962441in}{3.613611in}}%
\pgfpathlineto{\pgfqpoint{1.962441in}{4.075611in}}%
\pgfpathlineto{\pgfqpoint{1.137973in}{4.075611in}}%
\pgfpathlineto{\pgfqpoint{1.137973in}{3.613611in}}%
\pgfpathclose%
\pgfusepath{fill}%
\end{pgfscope}%
\begin{pgfscope}%
\pgfpathrectangle{\pgfqpoint{1.137973in}{3.613611in}}{\pgfqpoint{0.824468in}{0.462000in}}%
\pgfusepath{clip}%
\pgfsetbuttcap%
\pgfsetmiterjoin%
\definecolor{currentfill}{rgb}{0.121569,0.466667,0.705882}%
\pgfsetfillcolor{currentfill}%
\pgfsetfillopacity{0.500000}%
\pgfsetlinewidth{1.003750pt}%
\definecolor{currentstroke}{rgb}{0.000000,0.000000,0.000000}%
\pgfsetstrokecolor{currentstroke}%
\pgfsetdash{}{0pt}%
\pgfpathmoveto{\pgfqpoint{1.175449in}{3.613611in}}%
\pgfpathlineto{\pgfqpoint{1.325352in}{3.613611in}}%
\pgfpathlineto{\pgfqpoint{1.325352in}{3.674451in}}%
\pgfpathlineto{\pgfqpoint{1.175449in}{3.674451in}}%
\pgfpathlineto{\pgfqpoint{1.175449in}{3.613611in}}%
\pgfpathclose%
\pgfusepath{stroke,fill}%
\end{pgfscope}%
\begin{pgfscope}%
\pgfpathrectangle{\pgfqpoint{1.137973in}{3.613611in}}{\pgfqpoint{0.824468in}{0.462000in}}%
\pgfusepath{clip}%
\pgfsetbuttcap%
\pgfsetmiterjoin%
\definecolor{currentfill}{rgb}{0.121569,0.466667,0.705882}%
\pgfsetfillcolor{currentfill}%
\pgfsetfillopacity{0.500000}%
\pgfsetlinewidth{1.003750pt}%
\definecolor{currentstroke}{rgb}{0.000000,0.000000,0.000000}%
\pgfsetstrokecolor{currentstroke}%
\pgfsetdash{}{0pt}%
\pgfpathmoveto{\pgfqpoint{1.325352in}{3.613611in}}%
\pgfpathlineto{\pgfqpoint{1.475255in}{3.613611in}}%
\pgfpathlineto{\pgfqpoint{1.475255in}{3.689660in}}%
\pgfpathlineto{\pgfqpoint{1.325352in}{3.689660in}}%
\pgfpathlineto{\pgfqpoint{1.325352in}{3.613611in}}%
\pgfpathclose%
\pgfusepath{stroke,fill}%
\end{pgfscope}%
\begin{pgfscope}%
\pgfpathrectangle{\pgfqpoint{1.137973in}{3.613611in}}{\pgfqpoint{0.824468in}{0.462000in}}%
\pgfusepath{clip}%
\pgfsetbuttcap%
\pgfsetmiterjoin%
\definecolor{currentfill}{rgb}{0.121569,0.466667,0.705882}%
\pgfsetfillcolor{currentfill}%
\pgfsetfillopacity{0.500000}%
\pgfsetlinewidth{1.003750pt}%
\definecolor{currentstroke}{rgb}{0.000000,0.000000,0.000000}%
\pgfsetstrokecolor{currentstroke}%
\pgfsetdash{}{0pt}%
\pgfpathmoveto{\pgfqpoint{1.475255in}{3.613611in}}%
\pgfpathlineto{\pgfqpoint{1.625158in}{3.613611in}}%
\pgfpathlineto{\pgfqpoint{1.625158in}{3.778747in}}%
\pgfpathlineto{\pgfqpoint{1.475255in}{3.778747in}}%
\pgfpathlineto{\pgfqpoint{1.475255in}{3.613611in}}%
\pgfpathclose%
\pgfusepath{stroke,fill}%
\end{pgfscope}%
\begin{pgfscope}%
\pgfpathrectangle{\pgfqpoint{1.137973in}{3.613611in}}{\pgfqpoint{0.824468in}{0.462000in}}%
\pgfusepath{clip}%
\pgfsetbuttcap%
\pgfsetmiterjoin%
\definecolor{currentfill}{rgb}{0.121569,0.466667,0.705882}%
\pgfsetfillcolor{currentfill}%
\pgfsetfillopacity{0.500000}%
\pgfsetlinewidth{1.003750pt}%
\definecolor{currentstroke}{rgb}{0.000000,0.000000,0.000000}%
\pgfsetstrokecolor{currentstroke}%
\pgfsetdash{}{0pt}%
\pgfpathmoveto{\pgfqpoint{1.625158in}{3.613611in}}%
\pgfpathlineto{\pgfqpoint{1.775062in}{3.613611in}}%
\pgfpathlineto{\pgfqpoint{1.775062in}{3.982994in}}%
\pgfpathlineto{\pgfqpoint{1.625158in}{3.982994in}}%
\pgfpathlineto{\pgfqpoint{1.625158in}{3.613611in}}%
\pgfpathclose%
\pgfusepath{stroke,fill}%
\end{pgfscope}%
\begin{pgfscope}%
\pgfpathrectangle{\pgfqpoint{1.137973in}{3.613611in}}{\pgfqpoint{0.824468in}{0.462000in}}%
\pgfusepath{clip}%
\pgfsetbuttcap%
\pgfsetmiterjoin%
\definecolor{currentfill}{rgb}{0.121569,0.466667,0.705882}%
\pgfsetfillcolor{currentfill}%
\pgfsetfillopacity{0.500000}%
\pgfsetlinewidth{1.003750pt}%
\definecolor{currentstroke}{rgb}{0.000000,0.000000,0.000000}%
\pgfsetstrokecolor{currentstroke}%
\pgfsetdash{}{0pt}%
\pgfpathmoveto{\pgfqpoint{1.775062in}{3.613611in}}%
\pgfpathlineto{\pgfqpoint{1.924965in}{3.613611in}}%
\pgfpathlineto{\pgfqpoint{1.924965in}{4.053611in}}%
\pgfpathlineto{\pgfqpoint{1.775062in}{4.053611in}}%
\pgfpathlineto{\pgfqpoint{1.775062in}{3.613611in}}%
\pgfpathclose%
\pgfusepath{stroke,fill}%
\end{pgfscope}%
\begin{pgfscope}%
\pgfsetrectcap%
\pgfsetmiterjoin%
\pgfsetlinewidth{0.803000pt}%
\definecolor{currentstroke}{rgb}{0.000000,0.000000,0.000000}%
\pgfsetstrokecolor{currentstroke}%
\pgfsetdash{}{0pt}%
\pgfpathmoveto{\pgfqpoint{1.137973in}{3.613611in}}%
\pgfpathlineto{\pgfqpoint{1.137973in}{4.075611in}}%
\pgfusepath{stroke}%
\end{pgfscope}%
\begin{pgfscope}%
\pgfsetrectcap%
\pgfsetmiterjoin%
\pgfsetlinewidth{0.803000pt}%
\definecolor{currentstroke}{rgb}{0.000000,0.000000,0.000000}%
\pgfsetstrokecolor{currentstroke}%
\pgfsetdash{}{0pt}%
\pgfpathmoveto{\pgfqpoint{1.962441in}{3.613611in}}%
\pgfpathlineto{\pgfqpoint{1.962441in}{4.075611in}}%
\pgfusepath{stroke}%
\end{pgfscope}%
\begin{pgfscope}%
\pgfsetrectcap%
\pgfsetmiterjoin%
\pgfsetlinewidth{0.803000pt}%
\definecolor{currentstroke}{rgb}{0.000000,0.000000,0.000000}%
\pgfsetstrokecolor{currentstroke}%
\pgfsetdash{}{0pt}%
\pgfpathmoveto{\pgfqpoint{1.137973in}{3.613611in}}%
\pgfpathlineto{\pgfqpoint{1.962441in}{3.613611in}}%
\pgfusepath{stroke}%
\end{pgfscope}%
\begin{pgfscope}%
\pgfsetrectcap%
\pgfsetmiterjoin%
\pgfsetlinewidth{0.803000pt}%
\definecolor{currentstroke}{rgb}{0.000000,0.000000,0.000000}%
\pgfsetstrokecolor{currentstroke}%
\pgfsetdash{}{0pt}%
\pgfpathmoveto{\pgfqpoint{1.137973in}{4.075611in}}%
\pgfpathlineto{\pgfqpoint{1.962441in}{4.075611in}}%
\pgfusepath{stroke}%
\end{pgfscope}%
\begin{pgfscope}%
\definecolor{textcolor}{rgb}{0.000000,0.000000,0.000000}%
\pgfsetstrokecolor{textcolor}%
\pgfsetfillcolor{textcolor}%
\pgftext[x=1.550207in,y=4.158944in,,base]{\color{textcolor}\rmfamily\fontsize{11.000000}{13.200000}\selectfont APRIL ...}%
\end{pgfscope}%
\begin{pgfscope}%
\pgfsetbuttcap%
\pgfsetmiterjoin%
\definecolor{currentfill}{rgb}{1.000000,1.000000,1.000000}%
\pgfsetfillcolor{currentfill}%
\pgfsetlinewidth{0.000000pt}%
\definecolor{currentstroke}{rgb}{0.000000,0.000000,0.000000}%
\pgfsetstrokecolor{currentstroke}%
\pgfsetstrokeopacity{0.000000}%
\pgfsetdash{}{0pt}%
\pgfpathmoveto{\pgfqpoint{2.127335in}{3.613611in}}%
\pgfpathlineto{\pgfqpoint{2.951803in}{3.613611in}}%
\pgfpathlineto{\pgfqpoint{2.951803in}{4.075611in}}%
\pgfpathlineto{\pgfqpoint{2.127335in}{4.075611in}}%
\pgfpathlineto{\pgfqpoint{2.127335in}{3.613611in}}%
\pgfpathclose%
\pgfusepath{fill}%
\end{pgfscope}%
\begin{pgfscope}%
\pgfpathrectangle{\pgfqpoint{2.127335in}{3.613611in}}{\pgfqpoint{0.824468in}{0.462000in}}%
\pgfusepath{clip}%
\pgfsetbuttcap%
\pgfsetmiterjoin%
\definecolor{currentfill}{rgb}{0.121569,0.466667,0.705882}%
\pgfsetfillcolor{currentfill}%
\pgfsetfillopacity{0.500000}%
\pgfsetlinewidth{1.003750pt}%
\definecolor{currentstroke}{rgb}{0.000000,0.000000,0.000000}%
\pgfsetstrokecolor{currentstroke}%
\pgfsetdash{}{0pt}%
\pgfpathmoveto{\pgfqpoint{2.164810in}{3.613611in}}%
\pgfpathlineto{\pgfqpoint{2.314714in}{3.613611in}}%
\pgfpathlineto{\pgfqpoint{2.314714in}{4.053611in}}%
\pgfpathlineto{\pgfqpoint{2.164810in}{4.053611in}}%
\pgfpathlineto{\pgfqpoint{2.164810in}{3.613611in}}%
\pgfpathclose%
\pgfusepath{stroke,fill}%
\end{pgfscope}%
\begin{pgfscope}%
\pgfpathrectangle{\pgfqpoint{2.127335in}{3.613611in}}{\pgfqpoint{0.824468in}{0.462000in}}%
\pgfusepath{clip}%
\pgfsetbuttcap%
\pgfsetmiterjoin%
\definecolor{currentfill}{rgb}{0.121569,0.466667,0.705882}%
\pgfsetfillcolor{currentfill}%
\pgfsetfillopacity{0.500000}%
\pgfsetlinewidth{1.003750pt}%
\definecolor{currentstroke}{rgb}{0.000000,0.000000,0.000000}%
\pgfsetstrokecolor{currentstroke}%
\pgfsetdash{}{0pt}%
\pgfpathmoveto{\pgfqpoint{2.314714in}{3.613611in}}%
\pgfpathlineto{\pgfqpoint{2.464617in}{3.613611in}}%
\pgfpathlineto{\pgfqpoint{2.464617in}{3.743544in}}%
\pgfpathlineto{\pgfqpoint{2.314714in}{3.743544in}}%
\pgfpathlineto{\pgfqpoint{2.314714in}{3.613611in}}%
\pgfpathclose%
\pgfusepath{stroke,fill}%
\end{pgfscope}%
\begin{pgfscope}%
\pgfpathrectangle{\pgfqpoint{2.127335in}{3.613611in}}{\pgfqpoint{0.824468in}{0.462000in}}%
\pgfusepath{clip}%
\pgfsetbuttcap%
\pgfsetmiterjoin%
\definecolor{currentfill}{rgb}{0.121569,0.466667,0.705882}%
\pgfsetfillcolor{currentfill}%
\pgfsetfillopacity{0.500000}%
\pgfsetlinewidth{1.003750pt}%
\definecolor{currentstroke}{rgb}{0.000000,0.000000,0.000000}%
\pgfsetstrokecolor{currentstroke}%
\pgfsetdash{}{0pt}%
\pgfpathmoveto{\pgfqpoint{2.464617in}{3.613611in}}%
\pgfpathlineto{\pgfqpoint{2.614520in}{3.613611in}}%
\pgfpathlineto{\pgfqpoint{2.614520in}{3.634282in}}%
\pgfpathlineto{\pgfqpoint{2.464617in}{3.634282in}}%
\pgfpathlineto{\pgfqpoint{2.464617in}{3.613611in}}%
\pgfpathclose%
\pgfusepath{stroke,fill}%
\end{pgfscope}%
\begin{pgfscope}%
\pgfpathrectangle{\pgfqpoint{2.127335in}{3.613611in}}{\pgfqpoint{0.824468in}{0.462000in}}%
\pgfusepath{clip}%
\pgfsetbuttcap%
\pgfsetmiterjoin%
\definecolor{currentfill}{rgb}{0.121569,0.466667,0.705882}%
\pgfsetfillcolor{currentfill}%
\pgfsetfillopacity{0.500000}%
\pgfsetlinewidth{1.003750pt}%
\definecolor{currentstroke}{rgb}{0.000000,0.000000,0.000000}%
\pgfsetstrokecolor{currentstroke}%
\pgfsetdash{}{0pt}%
\pgfpathmoveto{\pgfqpoint{2.614520in}{3.613611in}}%
\pgfpathlineto{\pgfqpoint{2.764423in}{3.613611in}}%
\pgfpathlineto{\pgfqpoint{2.764423in}{3.637235in}}%
\pgfpathlineto{\pgfqpoint{2.614520in}{3.637235in}}%
\pgfpathlineto{\pgfqpoint{2.614520in}{3.613611in}}%
\pgfpathclose%
\pgfusepath{stroke,fill}%
\end{pgfscope}%
\begin{pgfscope}%
\pgfpathrectangle{\pgfqpoint{2.127335in}{3.613611in}}{\pgfqpoint{0.824468in}{0.462000in}}%
\pgfusepath{clip}%
\pgfsetbuttcap%
\pgfsetmiterjoin%
\definecolor{currentfill}{rgb}{0.121569,0.466667,0.705882}%
\pgfsetfillcolor{currentfill}%
\pgfsetfillopacity{0.500000}%
\pgfsetlinewidth{1.003750pt}%
\definecolor{currentstroke}{rgb}{0.000000,0.000000,0.000000}%
\pgfsetstrokecolor{currentstroke}%
\pgfsetdash{}{0pt}%
\pgfpathmoveto{\pgfqpoint{2.764423in}{3.613611in}}%
\pgfpathlineto{\pgfqpoint{2.914327in}{3.613611in}}%
\pgfpathlineto{\pgfqpoint{2.914327in}{3.631329in}}%
\pgfpathlineto{\pgfqpoint{2.764423in}{3.631329in}}%
\pgfpathlineto{\pgfqpoint{2.764423in}{3.613611in}}%
\pgfpathclose%
\pgfusepath{stroke,fill}%
\end{pgfscope}%
\begin{pgfscope}%
\pgfsetrectcap%
\pgfsetmiterjoin%
\pgfsetlinewidth{0.803000pt}%
\definecolor{currentstroke}{rgb}{0.000000,0.000000,0.000000}%
\pgfsetstrokecolor{currentstroke}%
\pgfsetdash{}{0pt}%
\pgfpathmoveto{\pgfqpoint{2.127335in}{3.613611in}}%
\pgfpathlineto{\pgfqpoint{2.127335in}{4.075611in}}%
\pgfusepath{stroke}%
\end{pgfscope}%
\begin{pgfscope}%
\pgfsetrectcap%
\pgfsetmiterjoin%
\pgfsetlinewidth{0.803000pt}%
\definecolor{currentstroke}{rgb}{0.000000,0.000000,0.000000}%
\pgfsetstrokecolor{currentstroke}%
\pgfsetdash{}{0pt}%
\pgfpathmoveto{\pgfqpoint{2.951803in}{3.613611in}}%
\pgfpathlineto{\pgfqpoint{2.951803in}{4.075611in}}%
\pgfusepath{stroke}%
\end{pgfscope}%
\begin{pgfscope}%
\pgfsetrectcap%
\pgfsetmiterjoin%
\pgfsetlinewidth{0.803000pt}%
\definecolor{currentstroke}{rgb}{0.000000,0.000000,0.000000}%
\pgfsetstrokecolor{currentstroke}%
\pgfsetdash{}{0pt}%
\pgfpathmoveto{\pgfqpoint{2.127335in}{3.613611in}}%
\pgfpathlineto{\pgfqpoint{2.951803in}{3.613611in}}%
\pgfusepath{stroke}%
\end{pgfscope}%
\begin{pgfscope}%
\pgfsetrectcap%
\pgfsetmiterjoin%
\pgfsetlinewidth{0.803000pt}%
\definecolor{currentstroke}{rgb}{0.000000,0.000000,0.000000}%
\pgfsetstrokecolor{currentstroke}%
\pgfsetdash{}{0pt}%
\pgfpathmoveto{\pgfqpoint{2.127335in}{4.075611in}}%
\pgfpathlineto{\pgfqpoint{2.951803in}{4.075611in}}%
\pgfusepath{stroke}%
\end{pgfscope}%
\begin{pgfscope}%
\definecolor{textcolor}{rgb}{0.000000,0.000000,0.000000}%
\pgfsetstrokecolor{textcolor}%
\pgfsetfillcolor{textcolor}%
\pgftext[x=2.539569in,y=4.158944in,,base]{\color{textcolor}\rmfamily\fontsize{11.000000}{13.200000}\selectfont Cegema...}%
\end{pgfscope}%
\begin{pgfscope}%
\pgfsetbuttcap%
\pgfsetmiterjoin%
\definecolor{currentfill}{rgb}{1.000000,1.000000,1.000000}%
\pgfsetfillcolor{currentfill}%
\pgfsetlinewidth{0.000000pt}%
\definecolor{currentstroke}{rgb}{0.000000,0.000000,0.000000}%
\pgfsetstrokecolor{currentstroke}%
\pgfsetstrokeopacity{0.000000}%
\pgfsetdash{}{0pt}%
\pgfpathmoveto{\pgfqpoint{3.116696in}{3.613611in}}%
\pgfpathlineto{\pgfqpoint{3.941164in}{3.613611in}}%
\pgfpathlineto{\pgfqpoint{3.941164in}{4.075611in}}%
\pgfpathlineto{\pgfqpoint{3.116696in}{4.075611in}}%
\pgfpathlineto{\pgfqpoint{3.116696in}{3.613611in}}%
\pgfpathclose%
\pgfusepath{fill}%
\end{pgfscope}%
\begin{pgfscope}%
\pgfpathrectangle{\pgfqpoint{3.116696in}{3.613611in}}{\pgfqpoint{0.824468in}{0.462000in}}%
\pgfusepath{clip}%
\pgfsetbuttcap%
\pgfsetmiterjoin%
\definecolor{currentfill}{rgb}{0.121569,0.466667,0.705882}%
\pgfsetfillcolor{currentfill}%
\pgfsetfillopacity{0.500000}%
\pgfsetlinewidth{1.003750pt}%
\definecolor{currentstroke}{rgb}{0.000000,0.000000,0.000000}%
\pgfsetstrokecolor{currentstroke}%
\pgfsetdash{}{0pt}%
\pgfpathmoveto{\pgfqpoint{3.154172in}{3.613611in}}%
\pgfpathlineto{\pgfqpoint{3.304075in}{3.613611in}}%
\pgfpathlineto{\pgfqpoint{3.304075in}{4.053611in}}%
\pgfpathlineto{\pgfqpoint{3.154172in}{4.053611in}}%
\pgfpathlineto{\pgfqpoint{3.154172in}{3.613611in}}%
\pgfpathclose%
\pgfusepath{stroke,fill}%
\end{pgfscope}%
\begin{pgfscope}%
\pgfpathrectangle{\pgfqpoint{3.116696in}{3.613611in}}{\pgfqpoint{0.824468in}{0.462000in}}%
\pgfusepath{clip}%
\pgfsetbuttcap%
\pgfsetmiterjoin%
\definecolor{currentfill}{rgb}{0.121569,0.466667,0.705882}%
\pgfsetfillcolor{currentfill}%
\pgfsetfillopacity{0.500000}%
\pgfsetlinewidth{1.003750pt}%
\definecolor{currentstroke}{rgb}{0.000000,0.000000,0.000000}%
\pgfsetstrokecolor{currentstroke}%
\pgfsetdash{}{0pt}%
\pgfpathmoveto{\pgfqpoint{3.304075in}{3.613611in}}%
\pgfpathlineto{\pgfqpoint{3.453979in}{3.613611in}}%
\pgfpathlineto{\pgfqpoint{3.453979in}{3.642944in}}%
\pgfpathlineto{\pgfqpoint{3.304075in}{3.642944in}}%
\pgfpathlineto{\pgfqpoint{3.304075in}{3.613611in}}%
\pgfpathclose%
\pgfusepath{stroke,fill}%
\end{pgfscope}%
\begin{pgfscope}%
\pgfpathrectangle{\pgfqpoint{3.116696in}{3.613611in}}{\pgfqpoint{0.824468in}{0.462000in}}%
\pgfusepath{clip}%
\pgfsetbuttcap%
\pgfsetmiterjoin%
\definecolor{currentfill}{rgb}{0.121569,0.466667,0.705882}%
\pgfsetfillcolor{currentfill}%
\pgfsetfillopacity{0.500000}%
\pgfsetlinewidth{1.003750pt}%
\definecolor{currentstroke}{rgb}{0.000000,0.000000,0.000000}%
\pgfsetstrokecolor{currentstroke}%
\pgfsetdash{}{0pt}%
\pgfpathmoveto{\pgfqpoint{3.453979in}{3.613611in}}%
\pgfpathlineto{\pgfqpoint{3.603882in}{3.613611in}}%
\pgfpathlineto{\pgfqpoint{3.603882in}{3.672278in}}%
\pgfpathlineto{\pgfqpoint{3.453979in}{3.672278in}}%
\pgfpathlineto{\pgfqpoint{3.453979in}{3.613611in}}%
\pgfpathclose%
\pgfusepath{stroke,fill}%
\end{pgfscope}%
\begin{pgfscope}%
\pgfpathrectangle{\pgfqpoint{3.116696in}{3.613611in}}{\pgfqpoint{0.824468in}{0.462000in}}%
\pgfusepath{clip}%
\pgfsetbuttcap%
\pgfsetmiterjoin%
\definecolor{currentfill}{rgb}{0.121569,0.466667,0.705882}%
\pgfsetfillcolor{currentfill}%
\pgfsetfillopacity{0.500000}%
\pgfsetlinewidth{1.003750pt}%
\definecolor{currentstroke}{rgb}{0.000000,0.000000,0.000000}%
\pgfsetstrokecolor{currentstroke}%
\pgfsetdash{}{0pt}%
\pgfpathmoveto{\pgfqpoint{3.603882in}{3.613611in}}%
\pgfpathlineto{\pgfqpoint{3.753785in}{3.613611in}}%
\pgfpathlineto{\pgfqpoint{3.753785in}{3.613611in}}%
\pgfpathlineto{\pgfqpoint{3.603882in}{3.613611in}}%
\pgfpathlineto{\pgfqpoint{3.603882in}{3.613611in}}%
\pgfpathclose%
\pgfusepath{stroke,fill}%
\end{pgfscope}%
\begin{pgfscope}%
\pgfpathrectangle{\pgfqpoint{3.116696in}{3.613611in}}{\pgfqpoint{0.824468in}{0.462000in}}%
\pgfusepath{clip}%
\pgfsetbuttcap%
\pgfsetmiterjoin%
\definecolor{currentfill}{rgb}{0.121569,0.466667,0.705882}%
\pgfsetfillcolor{currentfill}%
\pgfsetfillopacity{0.500000}%
\pgfsetlinewidth{1.003750pt}%
\definecolor{currentstroke}{rgb}{0.000000,0.000000,0.000000}%
\pgfsetstrokecolor{currentstroke}%
\pgfsetdash{}{0pt}%
\pgfpathmoveto{\pgfqpoint{3.753785in}{3.613611in}}%
\pgfpathlineto{\pgfqpoint{3.903688in}{3.613611in}}%
\pgfpathlineto{\pgfqpoint{3.903688in}{3.613611in}}%
\pgfpathlineto{\pgfqpoint{3.753785in}{3.613611in}}%
\pgfpathlineto{\pgfqpoint{3.753785in}{3.613611in}}%
\pgfpathclose%
\pgfusepath{stroke,fill}%
\end{pgfscope}%
\begin{pgfscope}%
\pgfsetrectcap%
\pgfsetmiterjoin%
\pgfsetlinewidth{0.803000pt}%
\definecolor{currentstroke}{rgb}{0.000000,0.000000,0.000000}%
\pgfsetstrokecolor{currentstroke}%
\pgfsetdash{}{0pt}%
\pgfpathmoveto{\pgfqpoint{3.116696in}{3.613611in}}%
\pgfpathlineto{\pgfqpoint{3.116696in}{4.075611in}}%
\pgfusepath{stroke}%
\end{pgfscope}%
\begin{pgfscope}%
\pgfsetrectcap%
\pgfsetmiterjoin%
\pgfsetlinewidth{0.803000pt}%
\definecolor{currentstroke}{rgb}{0.000000,0.000000,0.000000}%
\pgfsetstrokecolor{currentstroke}%
\pgfsetdash{}{0pt}%
\pgfpathmoveto{\pgfqpoint{3.941164in}{3.613611in}}%
\pgfpathlineto{\pgfqpoint{3.941164in}{4.075611in}}%
\pgfusepath{stroke}%
\end{pgfscope}%
\begin{pgfscope}%
\pgfsetrectcap%
\pgfsetmiterjoin%
\pgfsetlinewidth{0.803000pt}%
\definecolor{currentstroke}{rgb}{0.000000,0.000000,0.000000}%
\pgfsetstrokecolor{currentstroke}%
\pgfsetdash{}{0pt}%
\pgfpathmoveto{\pgfqpoint{3.116696in}{3.613611in}}%
\pgfpathlineto{\pgfqpoint{3.941164in}{3.613611in}}%
\pgfusepath{stroke}%
\end{pgfscope}%
\begin{pgfscope}%
\pgfsetrectcap%
\pgfsetmiterjoin%
\pgfsetlinewidth{0.803000pt}%
\definecolor{currentstroke}{rgb}{0.000000,0.000000,0.000000}%
\pgfsetstrokecolor{currentstroke}%
\pgfsetdash{}{0pt}%
\pgfpathmoveto{\pgfqpoint{3.116696in}{4.075611in}}%
\pgfpathlineto{\pgfqpoint{3.941164in}{4.075611in}}%
\pgfusepath{stroke}%
\end{pgfscope}%
\begin{pgfscope}%
\definecolor{textcolor}{rgb}{0.000000,0.000000,0.000000}%
\pgfsetstrokecolor{textcolor}%
\pgfsetfillcolor{textcolor}%
\pgftext[x=3.528930in,y=4.158944in,,base]{\color{textcolor}\rmfamily\fontsize{11.000000}{13.200000}\selectfont LCL}%
\end{pgfscope}%
\begin{pgfscope}%
\pgfsetbuttcap%
\pgfsetmiterjoin%
\definecolor{currentfill}{rgb}{1.000000,1.000000,1.000000}%
\pgfsetfillcolor{currentfill}%
\pgfsetlinewidth{0.000000pt}%
\definecolor{currentstroke}{rgb}{0.000000,0.000000,0.000000}%
\pgfsetstrokecolor{currentstroke}%
\pgfsetstrokeopacity{0.000000}%
\pgfsetdash{}{0pt}%
\pgfpathmoveto{\pgfqpoint{4.106058in}{3.613611in}}%
\pgfpathlineto{\pgfqpoint{4.930526in}{3.613611in}}%
\pgfpathlineto{\pgfqpoint{4.930526in}{4.075611in}}%
\pgfpathlineto{\pgfqpoint{4.106058in}{4.075611in}}%
\pgfpathlineto{\pgfqpoint{4.106058in}{3.613611in}}%
\pgfpathclose%
\pgfusepath{fill}%
\end{pgfscope}%
\begin{pgfscope}%
\pgfpathrectangle{\pgfqpoint{4.106058in}{3.613611in}}{\pgfqpoint{0.824468in}{0.462000in}}%
\pgfusepath{clip}%
\pgfsetbuttcap%
\pgfsetmiterjoin%
\definecolor{currentfill}{rgb}{0.121569,0.466667,0.705882}%
\pgfsetfillcolor{currentfill}%
\pgfsetfillopacity{0.500000}%
\pgfsetlinewidth{1.003750pt}%
\definecolor{currentstroke}{rgb}{0.000000,0.000000,0.000000}%
\pgfsetstrokecolor{currentstroke}%
\pgfsetdash{}{0pt}%
\pgfpathmoveto{\pgfqpoint{4.143534in}{3.613611in}}%
\pgfpathlineto{\pgfqpoint{4.293437in}{3.613611in}}%
\pgfpathlineto{\pgfqpoint{4.293437in}{4.053611in}}%
\pgfpathlineto{\pgfqpoint{4.143534in}{4.053611in}}%
\pgfpathlineto{\pgfqpoint{4.143534in}{3.613611in}}%
\pgfpathclose%
\pgfusepath{stroke,fill}%
\end{pgfscope}%
\begin{pgfscope}%
\pgfpathrectangle{\pgfqpoint{4.106058in}{3.613611in}}{\pgfqpoint{0.824468in}{0.462000in}}%
\pgfusepath{clip}%
\pgfsetbuttcap%
\pgfsetmiterjoin%
\definecolor{currentfill}{rgb}{0.121569,0.466667,0.705882}%
\pgfsetfillcolor{currentfill}%
\pgfsetfillopacity{0.500000}%
\pgfsetlinewidth{1.003750pt}%
\definecolor{currentstroke}{rgb}{0.000000,0.000000,0.000000}%
\pgfsetstrokecolor{currentstroke}%
\pgfsetdash{}{0pt}%
\pgfpathmoveto{\pgfqpoint{4.293437in}{3.613611in}}%
\pgfpathlineto{\pgfqpoint{4.443340in}{3.613611in}}%
\pgfpathlineto{\pgfqpoint{4.443340in}{3.887125in}}%
\pgfpathlineto{\pgfqpoint{4.293437in}{3.887125in}}%
\pgfpathlineto{\pgfqpoint{4.293437in}{3.613611in}}%
\pgfpathclose%
\pgfusepath{stroke,fill}%
\end{pgfscope}%
\begin{pgfscope}%
\pgfpathrectangle{\pgfqpoint{4.106058in}{3.613611in}}{\pgfqpoint{0.824468in}{0.462000in}}%
\pgfusepath{clip}%
\pgfsetbuttcap%
\pgfsetmiterjoin%
\definecolor{currentfill}{rgb}{0.121569,0.466667,0.705882}%
\pgfsetfillcolor{currentfill}%
\pgfsetfillopacity{0.500000}%
\pgfsetlinewidth{1.003750pt}%
\definecolor{currentstroke}{rgb}{0.000000,0.000000,0.000000}%
\pgfsetstrokecolor{currentstroke}%
\pgfsetdash{}{0pt}%
\pgfpathmoveto{\pgfqpoint{4.443340in}{3.613611in}}%
\pgfpathlineto{\pgfqpoint{4.593244in}{3.613611in}}%
\pgfpathlineto{\pgfqpoint{4.593244in}{3.684962in}}%
\pgfpathlineto{\pgfqpoint{4.443340in}{3.684962in}}%
\pgfpathlineto{\pgfqpoint{4.443340in}{3.613611in}}%
\pgfpathclose%
\pgfusepath{stroke,fill}%
\end{pgfscope}%
\begin{pgfscope}%
\pgfpathrectangle{\pgfqpoint{4.106058in}{3.613611in}}{\pgfqpoint{0.824468in}{0.462000in}}%
\pgfusepath{clip}%
\pgfsetbuttcap%
\pgfsetmiterjoin%
\definecolor{currentfill}{rgb}{0.121569,0.466667,0.705882}%
\pgfsetfillcolor{currentfill}%
\pgfsetfillopacity{0.500000}%
\pgfsetlinewidth{1.003750pt}%
\definecolor{currentstroke}{rgb}{0.000000,0.000000,0.000000}%
\pgfsetstrokecolor{currentstroke}%
\pgfsetdash{}{0pt}%
\pgfpathmoveto{\pgfqpoint{4.593244in}{3.613611in}}%
\pgfpathlineto{\pgfqpoint{4.743147in}{3.613611in}}%
\pgfpathlineto{\pgfqpoint{4.743147in}{3.661179in}}%
\pgfpathlineto{\pgfqpoint{4.593244in}{3.661179in}}%
\pgfpathlineto{\pgfqpoint{4.593244in}{3.613611in}}%
\pgfpathclose%
\pgfusepath{stroke,fill}%
\end{pgfscope}%
\begin{pgfscope}%
\pgfpathrectangle{\pgfqpoint{4.106058in}{3.613611in}}{\pgfqpoint{0.824468in}{0.462000in}}%
\pgfusepath{clip}%
\pgfsetbuttcap%
\pgfsetmiterjoin%
\definecolor{currentfill}{rgb}{0.121569,0.466667,0.705882}%
\pgfsetfillcolor{currentfill}%
\pgfsetfillopacity{0.500000}%
\pgfsetlinewidth{1.003750pt}%
\definecolor{currentstroke}{rgb}{0.000000,0.000000,0.000000}%
\pgfsetstrokecolor{currentstroke}%
\pgfsetdash{}{0pt}%
\pgfpathmoveto{\pgfqpoint{4.743147in}{3.613611in}}%
\pgfpathlineto{\pgfqpoint{4.893050in}{3.613611in}}%
\pgfpathlineto{\pgfqpoint{4.893050in}{3.637395in}}%
\pgfpathlineto{\pgfqpoint{4.743147in}{3.637395in}}%
\pgfpathlineto{\pgfqpoint{4.743147in}{3.613611in}}%
\pgfpathclose%
\pgfusepath{stroke,fill}%
\end{pgfscope}%
\begin{pgfscope}%
\pgfsetrectcap%
\pgfsetmiterjoin%
\pgfsetlinewidth{0.803000pt}%
\definecolor{currentstroke}{rgb}{0.000000,0.000000,0.000000}%
\pgfsetstrokecolor{currentstroke}%
\pgfsetdash{}{0pt}%
\pgfpathmoveto{\pgfqpoint{4.106058in}{3.613611in}}%
\pgfpathlineto{\pgfqpoint{4.106058in}{4.075611in}}%
\pgfusepath{stroke}%
\end{pgfscope}%
\begin{pgfscope}%
\pgfsetrectcap%
\pgfsetmiterjoin%
\pgfsetlinewidth{0.803000pt}%
\definecolor{currentstroke}{rgb}{0.000000,0.000000,0.000000}%
\pgfsetstrokecolor{currentstroke}%
\pgfsetdash{}{0pt}%
\pgfpathmoveto{\pgfqpoint{4.930526in}{3.613611in}}%
\pgfpathlineto{\pgfqpoint{4.930526in}{4.075611in}}%
\pgfusepath{stroke}%
\end{pgfscope}%
\begin{pgfscope}%
\pgfsetrectcap%
\pgfsetmiterjoin%
\pgfsetlinewidth{0.803000pt}%
\definecolor{currentstroke}{rgb}{0.000000,0.000000,0.000000}%
\pgfsetstrokecolor{currentstroke}%
\pgfsetdash{}{0pt}%
\pgfpathmoveto{\pgfqpoint{4.106058in}{3.613611in}}%
\pgfpathlineto{\pgfqpoint{4.930526in}{3.613611in}}%
\pgfusepath{stroke}%
\end{pgfscope}%
\begin{pgfscope}%
\pgfsetrectcap%
\pgfsetmiterjoin%
\pgfsetlinewidth{0.803000pt}%
\definecolor{currentstroke}{rgb}{0.000000,0.000000,0.000000}%
\pgfsetstrokecolor{currentstroke}%
\pgfsetdash{}{0pt}%
\pgfpathmoveto{\pgfqpoint{4.106058in}{4.075611in}}%
\pgfpathlineto{\pgfqpoint{4.930526in}{4.075611in}}%
\pgfusepath{stroke}%
\end{pgfscope}%
\begin{pgfscope}%
\definecolor{textcolor}{rgb}{0.000000,0.000000,0.000000}%
\pgfsetstrokecolor{textcolor}%
\pgfsetfillcolor{textcolor}%
\pgftext[x=4.518292in,y=4.158944in,,base]{\color{textcolor}\rmfamily\fontsize{11.000000}{13.200000}\selectfont Afer}%
\end{pgfscope}%
\begin{pgfscope}%
\pgfsetbuttcap%
\pgfsetmiterjoin%
\definecolor{currentfill}{rgb}{1.000000,1.000000,1.000000}%
\pgfsetfillcolor{currentfill}%
\pgfsetlinewidth{0.000000pt}%
\definecolor{currentstroke}{rgb}{0.000000,0.000000,0.000000}%
\pgfsetstrokecolor{currentstroke}%
\pgfsetstrokeopacity{0.000000}%
\pgfsetdash{}{0pt}%
\pgfpathmoveto{\pgfqpoint{5.095420in}{3.613611in}}%
\pgfpathlineto{\pgfqpoint{5.919888in}{3.613611in}}%
\pgfpathlineto{\pgfqpoint{5.919888in}{4.075611in}}%
\pgfpathlineto{\pgfqpoint{5.095420in}{4.075611in}}%
\pgfpathlineto{\pgfqpoint{5.095420in}{3.613611in}}%
\pgfpathclose%
\pgfusepath{fill}%
\end{pgfscope}%
\begin{pgfscope}%
\pgfpathrectangle{\pgfqpoint{5.095420in}{3.613611in}}{\pgfqpoint{0.824468in}{0.462000in}}%
\pgfusepath{clip}%
\pgfsetbuttcap%
\pgfsetmiterjoin%
\definecolor{currentfill}{rgb}{0.121569,0.466667,0.705882}%
\pgfsetfillcolor{currentfill}%
\pgfsetfillopacity{0.500000}%
\pgfsetlinewidth{1.003750pt}%
\definecolor{currentstroke}{rgb}{0.000000,0.000000,0.000000}%
\pgfsetstrokecolor{currentstroke}%
\pgfsetdash{}{0pt}%
\pgfpathmoveto{\pgfqpoint{5.132895in}{3.613611in}}%
\pgfpathlineto{\pgfqpoint{5.282799in}{3.613611in}}%
\pgfpathlineto{\pgfqpoint{5.282799in}{4.053611in}}%
\pgfpathlineto{\pgfqpoint{5.132895in}{4.053611in}}%
\pgfpathlineto{\pgfqpoint{5.132895in}{3.613611in}}%
\pgfpathclose%
\pgfusepath{stroke,fill}%
\end{pgfscope}%
\begin{pgfscope}%
\pgfpathrectangle{\pgfqpoint{5.095420in}{3.613611in}}{\pgfqpoint{0.824468in}{0.462000in}}%
\pgfusepath{clip}%
\pgfsetbuttcap%
\pgfsetmiterjoin%
\definecolor{currentfill}{rgb}{0.121569,0.466667,0.705882}%
\pgfsetfillcolor{currentfill}%
\pgfsetfillopacity{0.500000}%
\pgfsetlinewidth{1.003750pt}%
\definecolor{currentstroke}{rgb}{0.000000,0.000000,0.000000}%
\pgfsetstrokecolor{currentstroke}%
\pgfsetdash{}{0pt}%
\pgfpathmoveto{\pgfqpoint{5.282799in}{3.613611in}}%
\pgfpathlineto{\pgfqpoint{5.432702in}{3.613611in}}%
\pgfpathlineto{\pgfqpoint{5.432702in}{3.799355in}}%
\pgfpathlineto{\pgfqpoint{5.282799in}{3.799355in}}%
\pgfpathlineto{\pgfqpoint{5.282799in}{3.613611in}}%
\pgfpathclose%
\pgfusepath{stroke,fill}%
\end{pgfscope}%
\begin{pgfscope}%
\pgfpathrectangle{\pgfqpoint{5.095420in}{3.613611in}}{\pgfqpoint{0.824468in}{0.462000in}}%
\pgfusepath{clip}%
\pgfsetbuttcap%
\pgfsetmiterjoin%
\definecolor{currentfill}{rgb}{0.121569,0.466667,0.705882}%
\pgfsetfillcolor{currentfill}%
\pgfsetfillopacity{0.500000}%
\pgfsetlinewidth{1.003750pt}%
\definecolor{currentstroke}{rgb}{0.000000,0.000000,0.000000}%
\pgfsetstrokecolor{currentstroke}%
\pgfsetdash{}{0pt}%
\pgfpathmoveto{\pgfqpoint{5.432702in}{3.613611in}}%
\pgfpathlineto{\pgfqpoint{5.582605in}{3.613611in}}%
\pgfpathlineto{\pgfqpoint{5.582605in}{3.698871in}}%
\pgfpathlineto{\pgfqpoint{5.432702in}{3.698871in}}%
\pgfpathlineto{\pgfqpoint{5.432702in}{3.613611in}}%
\pgfpathclose%
\pgfusepath{stroke,fill}%
\end{pgfscope}%
\begin{pgfscope}%
\pgfpathrectangle{\pgfqpoint{5.095420in}{3.613611in}}{\pgfqpoint{0.824468in}{0.462000in}}%
\pgfusepath{clip}%
\pgfsetbuttcap%
\pgfsetmiterjoin%
\definecolor{currentfill}{rgb}{0.121569,0.466667,0.705882}%
\pgfsetfillcolor{currentfill}%
\pgfsetfillopacity{0.500000}%
\pgfsetlinewidth{1.003750pt}%
\definecolor{currentstroke}{rgb}{0.000000,0.000000,0.000000}%
\pgfsetstrokecolor{currentstroke}%
\pgfsetdash{}{0pt}%
\pgfpathmoveto{\pgfqpoint{5.582605in}{3.613611in}}%
\pgfpathlineto{\pgfqpoint{5.732509in}{3.613611in}}%
\pgfpathlineto{\pgfqpoint{5.732509in}{3.709528in}}%
\pgfpathlineto{\pgfqpoint{5.582605in}{3.709528in}}%
\pgfpathlineto{\pgfqpoint{5.582605in}{3.613611in}}%
\pgfpathclose%
\pgfusepath{stroke,fill}%
\end{pgfscope}%
\begin{pgfscope}%
\pgfpathrectangle{\pgfqpoint{5.095420in}{3.613611in}}{\pgfqpoint{0.824468in}{0.462000in}}%
\pgfusepath{clip}%
\pgfsetbuttcap%
\pgfsetmiterjoin%
\definecolor{currentfill}{rgb}{0.121569,0.466667,0.705882}%
\pgfsetfillcolor{currentfill}%
\pgfsetfillopacity{0.500000}%
\pgfsetlinewidth{1.003750pt}%
\definecolor{currentstroke}{rgb}{0.000000,0.000000,0.000000}%
\pgfsetstrokecolor{currentstroke}%
\pgfsetdash{}{0pt}%
\pgfpathmoveto{\pgfqpoint{5.732509in}{3.613611in}}%
\pgfpathlineto{\pgfqpoint{5.882412in}{3.613611in}}%
\pgfpathlineto{\pgfqpoint{5.882412in}{3.648628in}}%
\pgfpathlineto{\pgfqpoint{5.732509in}{3.648628in}}%
\pgfpathlineto{\pgfqpoint{5.732509in}{3.613611in}}%
\pgfpathclose%
\pgfusepath{stroke,fill}%
\end{pgfscope}%
\begin{pgfscope}%
\pgfsetrectcap%
\pgfsetmiterjoin%
\pgfsetlinewidth{0.803000pt}%
\definecolor{currentstroke}{rgb}{0.000000,0.000000,0.000000}%
\pgfsetstrokecolor{currentstroke}%
\pgfsetdash{}{0pt}%
\pgfpathmoveto{\pgfqpoint{5.095420in}{3.613611in}}%
\pgfpathlineto{\pgfqpoint{5.095420in}{4.075611in}}%
\pgfusepath{stroke}%
\end{pgfscope}%
\begin{pgfscope}%
\pgfsetrectcap%
\pgfsetmiterjoin%
\pgfsetlinewidth{0.803000pt}%
\definecolor{currentstroke}{rgb}{0.000000,0.000000,0.000000}%
\pgfsetstrokecolor{currentstroke}%
\pgfsetdash{}{0pt}%
\pgfpathmoveto{\pgfqpoint{5.919888in}{3.613611in}}%
\pgfpathlineto{\pgfqpoint{5.919888in}{4.075611in}}%
\pgfusepath{stroke}%
\end{pgfscope}%
\begin{pgfscope}%
\pgfsetrectcap%
\pgfsetmiterjoin%
\pgfsetlinewidth{0.803000pt}%
\definecolor{currentstroke}{rgb}{0.000000,0.000000,0.000000}%
\pgfsetstrokecolor{currentstroke}%
\pgfsetdash{}{0pt}%
\pgfpathmoveto{\pgfqpoint{5.095420in}{3.613611in}}%
\pgfpathlineto{\pgfqpoint{5.919888in}{3.613611in}}%
\pgfusepath{stroke}%
\end{pgfscope}%
\begin{pgfscope}%
\pgfsetrectcap%
\pgfsetmiterjoin%
\pgfsetlinewidth{0.803000pt}%
\definecolor{currentstroke}{rgb}{0.000000,0.000000,0.000000}%
\pgfsetstrokecolor{currentstroke}%
\pgfsetdash{}{0pt}%
\pgfpathmoveto{\pgfqpoint{5.095420in}{4.075611in}}%
\pgfpathlineto{\pgfqpoint{5.919888in}{4.075611in}}%
\pgfusepath{stroke}%
\end{pgfscope}%
\begin{pgfscope}%
\definecolor{textcolor}{rgb}{0.000000,0.000000,0.000000}%
\pgfsetstrokecolor{textcolor}%
\pgfsetfillcolor{textcolor}%
\pgftext[x=5.507654in,y=4.158944in,,base]{\color{textcolor}\rmfamily\fontsize{11.000000}{13.200000}\selectfont Pacifica}%
\end{pgfscope}%
\begin{pgfscope}%
\pgfsetbuttcap%
\pgfsetmiterjoin%
\definecolor{currentfill}{rgb}{1.000000,1.000000,1.000000}%
\pgfsetfillcolor{currentfill}%
\pgfsetlinewidth{0.000000pt}%
\definecolor{currentstroke}{rgb}{0.000000,0.000000,0.000000}%
\pgfsetstrokecolor{currentstroke}%
\pgfsetstrokeopacity{0.000000}%
\pgfsetdash{}{0pt}%
\pgfpathmoveto{\pgfqpoint{6.084781in}{3.613611in}}%
\pgfpathlineto{\pgfqpoint{6.909249in}{3.613611in}}%
\pgfpathlineto{\pgfqpoint{6.909249in}{4.075611in}}%
\pgfpathlineto{\pgfqpoint{6.084781in}{4.075611in}}%
\pgfpathlineto{\pgfqpoint{6.084781in}{3.613611in}}%
\pgfpathclose%
\pgfusepath{fill}%
\end{pgfscope}%
\begin{pgfscope}%
\pgfpathrectangle{\pgfqpoint{6.084781in}{3.613611in}}{\pgfqpoint{0.824468in}{0.462000in}}%
\pgfusepath{clip}%
\pgfsetbuttcap%
\pgfsetmiterjoin%
\definecolor{currentfill}{rgb}{0.121569,0.466667,0.705882}%
\pgfsetfillcolor{currentfill}%
\pgfsetfillopacity{0.500000}%
\pgfsetlinewidth{1.003750pt}%
\definecolor{currentstroke}{rgb}{0.000000,0.000000,0.000000}%
\pgfsetstrokecolor{currentstroke}%
\pgfsetdash{}{0pt}%
\pgfpathmoveto{\pgfqpoint{6.122257in}{3.613611in}}%
\pgfpathlineto{\pgfqpoint{6.272160in}{3.613611in}}%
\pgfpathlineto{\pgfqpoint{6.272160in}{4.053611in}}%
\pgfpathlineto{\pgfqpoint{6.122257in}{4.053611in}}%
\pgfpathlineto{\pgfqpoint{6.122257in}{3.613611in}}%
\pgfpathclose%
\pgfusepath{stroke,fill}%
\end{pgfscope}%
\begin{pgfscope}%
\pgfpathrectangle{\pgfqpoint{6.084781in}{3.613611in}}{\pgfqpoint{0.824468in}{0.462000in}}%
\pgfusepath{clip}%
\pgfsetbuttcap%
\pgfsetmiterjoin%
\definecolor{currentfill}{rgb}{0.121569,0.466667,0.705882}%
\pgfsetfillcolor{currentfill}%
\pgfsetfillopacity{0.500000}%
\pgfsetlinewidth{1.003750pt}%
\definecolor{currentstroke}{rgb}{0.000000,0.000000,0.000000}%
\pgfsetstrokecolor{currentstroke}%
\pgfsetdash{}{0pt}%
\pgfpathmoveto{\pgfqpoint{6.272160in}{3.613611in}}%
\pgfpathlineto{\pgfqpoint{6.422064in}{3.613611in}}%
\pgfpathlineto{\pgfqpoint{6.422064in}{3.701611in}}%
\pgfpathlineto{\pgfqpoint{6.272160in}{3.701611in}}%
\pgfpathlineto{\pgfqpoint{6.272160in}{3.613611in}}%
\pgfpathclose%
\pgfusepath{stroke,fill}%
\end{pgfscope}%
\begin{pgfscope}%
\pgfpathrectangle{\pgfqpoint{6.084781in}{3.613611in}}{\pgfqpoint{0.824468in}{0.462000in}}%
\pgfusepath{clip}%
\pgfsetbuttcap%
\pgfsetmiterjoin%
\definecolor{currentfill}{rgb}{0.121569,0.466667,0.705882}%
\pgfsetfillcolor{currentfill}%
\pgfsetfillopacity{0.500000}%
\pgfsetlinewidth{1.003750pt}%
\definecolor{currentstroke}{rgb}{0.000000,0.000000,0.000000}%
\pgfsetstrokecolor{currentstroke}%
\pgfsetdash{}{0pt}%
\pgfpathmoveto{\pgfqpoint{6.422064in}{3.613611in}}%
\pgfpathlineto{\pgfqpoint{6.571967in}{3.613611in}}%
\pgfpathlineto{\pgfqpoint{6.571967in}{3.692811in}}%
\pgfpathlineto{\pgfqpoint{6.422064in}{3.692811in}}%
\pgfpathlineto{\pgfqpoint{6.422064in}{3.613611in}}%
\pgfpathclose%
\pgfusepath{stroke,fill}%
\end{pgfscope}%
\begin{pgfscope}%
\pgfpathrectangle{\pgfqpoint{6.084781in}{3.613611in}}{\pgfqpoint{0.824468in}{0.462000in}}%
\pgfusepath{clip}%
\pgfsetbuttcap%
\pgfsetmiterjoin%
\definecolor{currentfill}{rgb}{0.121569,0.466667,0.705882}%
\pgfsetfillcolor{currentfill}%
\pgfsetfillopacity{0.500000}%
\pgfsetlinewidth{1.003750pt}%
\definecolor{currentstroke}{rgb}{0.000000,0.000000,0.000000}%
\pgfsetstrokecolor{currentstroke}%
\pgfsetdash{}{0pt}%
\pgfpathmoveto{\pgfqpoint{6.571967in}{3.613611in}}%
\pgfpathlineto{\pgfqpoint{6.721870in}{3.613611in}}%
\pgfpathlineto{\pgfqpoint{6.721870in}{3.613611in}}%
\pgfpathlineto{\pgfqpoint{6.571967in}{3.613611in}}%
\pgfpathlineto{\pgfqpoint{6.571967in}{3.613611in}}%
\pgfpathclose%
\pgfusepath{stroke,fill}%
\end{pgfscope}%
\begin{pgfscope}%
\pgfpathrectangle{\pgfqpoint{6.084781in}{3.613611in}}{\pgfqpoint{0.824468in}{0.462000in}}%
\pgfusepath{clip}%
\pgfsetbuttcap%
\pgfsetmiterjoin%
\definecolor{currentfill}{rgb}{0.121569,0.466667,0.705882}%
\pgfsetfillcolor{currentfill}%
\pgfsetfillopacity{0.500000}%
\pgfsetlinewidth{1.003750pt}%
\definecolor{currentstroke}{rgb}{0.000000,0.000000,0.000000}%
\pgfsetstrokecolor{currentstroke}%
\pgfsetdash{}{0pt}%
\pgfpathmoveto{\pgfqpoint{6.721870in}{3.613611in}}%
\pgfpathlineto{\pgfqpoint{6.871774in}{3.613611in}}%
\pgfpathlineto{\pgfqpoint{6.871774in}{3.613611in}}%
\pgfpathlineto{\pgfqpoint{6.721870in}{3.613611in}}%
\pgfpathlineto{\pgfqpoint{6.721870in}{3.613611in}}%
\pgfpathclose%
\pgfusepath{stroke,fill}%
\end{pgfscope}%
\begin{pgfscope}%
\pgfsetrectcap%
\pgfsetmiterjoin%
\pgfsetlinewidth{0.803000pt}%
\definecolor{currentstroke}{rgb}{0.000000,0.000000,0.000000}%
\pgfsetstrokecolor{currentstroke}%
\pgfsetdash{}{0pt}%
\pgfpathmoveto{\pgfqpoint{6.084781in}{3.613611in}}%
\pgfpathlineto{\pgfqpoint{6.084781in}{4.075611in}}%
\pgfusepath{stroke}%
\end{pgfscope}%
\begin{pgfscope}%
\pgfsetrectcap%
\pgfsetmiterjoin%
\pgfsetlinewidth{0.803000pt}%
\definecolor{currentstroke}{rgb}{0.000000,0.000000,0.000000}%
\pgfsetstrokecolor{currentstroke}%
\pgfsetdash{}{0pt}%
\pgfpathmoveto{\pgfqpoint{6.909249in}{3.613611in}}%
\pgfpathlineto{\pgfqpoint{6.909249in}{4.075611in}}%
\pgfusepath{stroke}%
\end{pgfscope}%
\begin{pgfscope}%
\pgfsetrectcap%
\pgfsetmiterjoin%
\pgfsetlinewidth{0.803000pt}%
\definecolor{currentstroke}{rgb}{0.000000,0.000000,0.000000}%
\pgfsetstrokecolor{currentstroke}%
\pgfsetdash{}{0pt}%
\pgfpathmoveto{\pgfqpoint{6.084781in}{3.613611in}}%
\pgfpathlineto{\pgfqpoint{6.909249in}{3.613611in}}%
\pgfusepath{stroke}%
\end{pgfscope}%
\begin{pgfscope}%
\pgfsetrectcap%
\pgfsetmiterjoin%
\pgfsetlinewidth{0.803000pt}%
\definecolor{currentstroke}{rgb}{0.000000,0.000000,0.000000}%
\pgfsetstrokecolor{currentstroke}%
\pgfsetdash{}{0pt}%
\pgfpathmoveto{\pgfqpoint{6.084781in}{4.075611in}}%
\pgfpathlineto{\pgfqpoint{6.909249in}{4.075611in}}%
\pgfusepath{stroke}%
\end{pgfscope}%
\begin{pgfscope}%
\definecolor{textcolor}{rgb}{0.000000,0.000000,0.000000}%
\pgfsetstrokecolor{textcolor}%
\pgfsetfillcolor{textcolor}%
\pgftext[x=6.497015in,y=4.158944in,,base]{\color{textcolor}\rmfamily\fontsize{11.000000}{13.200000}\selectfont SwissLife}%
\end{pgfscope}%
\begin{pgfscope}%
\pgfsetbuttcap%
\pgfsetmiterjoin%
\definecolor{currentfill}{rgb}{1.000000,1.000000,1.000000}%
\pgfsetfillcolor{currentfill}%
\pgfsetlinewidth{0.000000pt}%
\definecolor{currentstroke}{rgb}{0.000000,0.000000,0.000000}%
\pgfsetstrokecolor{currentstroke}%
\pgfsetstrokeopacity{0.000000}%
\pgfsetdash{}{0pt}%
\pgfpathmoveto{\pgfqpoint{7.074143in}{3.613611in}}%
\pgfpathlineto{\pgfqpoint{7.898611in}{3.613611in}}%
\pgfpathlineto{\pgfqpoint{7.898611in}{4.075611in}}%
\pgfpathlineto{\pgfqpoint{7.074143in}{4.075611in}}%
\pgfpathlineto{\pgfqpoint{7.074143in}{3.613611in}}%
\pgfpathclose%
\pgfusepath{fill}%
\end{pgfscope}%
\begin{pgfscope}%
\pgfpathrectangle{\pgfqpoint{7.074143in}{3.613611in}}{\pgfqpoint{0.824468in}{0.462000in}}%
\pgfusepath{clip}%
\pgfsetbuttcap%
\pgfsetmiterjoin%
\definecolor{currentfill}{rgb}{0.121569,0.466667,0.705882}%
\pgfsetfillcolor{currentfill}%
\pgfsetfillopacity{0.500000}%
\pgfsetlinewidth{1.003750pt}%
\definecolor{currentstroke}{rgb}{0.000000,0.000000,0.000000}%
\pgfsetstrokecolor{currentstroke}%
\pgfsetdash{}{0pt}%
\pgfpathmoveto{\pgfqpoint{7.111619in}{3.613611in}}%
\pgfpathlineto{\pgfqpoint{7.261522in}{3.613611in}}%
\pgfpathlineto{\pgfqpoint{7.261522in}{4.053611in}}%
\pgfpathlineto{\pgfqpoint{7.111619in}{4.053611in}}%
\pgfpathlineto{\pgfqpoint{7.111619in}{3.613611in}}%
\pgfpathclose%
\pgfusepath{stroke,fill}%
\end{pgfscope}%
\begin{pgfscope}%
\pgfpathrectangle{\pgfqpoint{7.074143in}{3.613611in}}{\pgfqpoint{0.824468in}{0.462000in}}%
\pgfusepath{clip}%
\pgfsetbuttcap%
\pgfsetmiterjoin%
\definecolor{currentfill}{rgb}{0.121569,0.466667,0.705882}%
\pgfsetfillcolor{currentfill}%
\pgfsetfillopacity{0.500000}%
\pgfsetlinewidth{1.003750pt}%
\definecolor{currentstroke}{rgb}{0.000000,0.000000,0.000000}%
\pgfsetstrokecolor{currentstroke}%
\pgfsetdash{}{0pt}%
\pgfpathmoveto{\pgfqpoint{7.261522in}{3.613611in}}%
\pgfpathlineto{\pgfqpoint{7.411425in}{3.613611in}}%
\pgfpathlineto{\pgfqpoint{7.411425in}{3.900370in}}%
\pgfpathlineto{\pgfqpoint{7.261522in}{3.900370in}}%
\pgfpathlineto{\pgfqpoint{7.261522in}{3.613611in}}%
\pgfpathclose%
\pgfusepath{stroke,fill}%
\end{pgfscope}%
\begin{pgfscope}%
\pgfpathrectangle{\pgfqpoint{7.074143in}{3.613611in}}{\pgfqpoint{0.824468in}{0.462000in}}%
\pgfusepath{clip}%
\pgfsetbuttcap%
\pgfsetmiterjoin%
\definecolor{currentfill}{rgb}{0.121569,0.466667,0.705882}%
\pgfsetfillcolor{currentfill}%
\pgfsetfillopacity{0.500000}%
\pgfsetlinewidth{1.003750pt}%
\definecolor{currentstroke}{rgb}{0.000000,0.000000,0.000000}%
\pgfsetstrokecolor{currentstroke}%
\pgfsetdash{}{0pt}%
\pgfpathmoveto{\pgfqpoint{7.411425in}{3.613611in}}%
\pgfpathlineto{\pgfqpoint{7.561329in}{3.613611in}}%
\pgfpathlineto{\pgfqpoint{7.561329in}{3.744094in}}%
\pgfpathlineto{\pgfqpoint{7.411425in}{3.744094in}}%
\pgfpathlineto{\pgfqpoint{7.411425in}{3.613611in}}%
\pgfpathclose%
\pgfusepath{stroke,fill}%
\end{pgfscope}%
\begin{pgfscope}%
\pgfpathrectangle{\pgfqpoint{7.074143in}{3.613611in}}{\pgfqpoint{0.824468in}{0.462000in}}%
\pgfusepath{clip}%
\pgfsetbuttcap%
\pgfsetmiterjoin%
\definecolor{currentfill}{rgb}{0.121569,0.466667,0.705882}%
\pgfsetfillcolor{currentfill}%
\pgfsetfillopacity{0.500000}%
\pgfsetlinewidth{1.003750pt}%
\definecolor{currentstroke}{rgb}{0.000000,0.000000,0.000000}%
\pgfsetstrokecolor{currentstroke}%
\pgfsetdash{}{0pt}%
\pgfpathmoveto{\pgfqpoint{7.561329in}{3.613611in}}%
\pgfpathlineto{\pgfqpoint{7.711232in}{3.613611in}}%
\pgfpathlineto{\pgfqpoint{7.711232in}{3.660646in}}%
\pgfpathlineto{\pgfqpoint{7.561329in}{3.660646in}}%
\pgfpathlineto{\pgfqpoint{7.561329in}{3.613611in}}%
\pgfpathclose%
\pgfusepath{stroke,fill}%
\end{pgfscope}%
\begin{pgfscope}%
\pgfpathrectangle{\pgfqpoint{7.074143in}{3.613611in}}{\pgfqpoint{0.824468in}{0.462000in}}%
\pgfusepath{clip}%
\pgfsetbuttcap%
\pgfsetmiterjoin%
\definecolor{currentfill}{rgb}{0.121569,0.466667,0.705882}%
\pgfsetfillcolor{currentfill}%
\pgfsetfillopacity{0.500000}%
\pgfsetlinewidth{1.003750pt}%
\definecolor{currentstroke}{rgb}{0.000000,0.000000,0.000000}%
\pgfsetstrokecolor{currentstroke}%
\pgfsetdash{}{0pt}%
\pgfpathmoveto{\pgfqpoint{7.711232in}{3.613611in}}%
\pgfpathlineto{\pgfqpoint{7.861135in}{3.613611in}}%
\pgfpathlineto{\pgfqpoint{7.861135in}{3.637887in}}%
\pgfpathlineto{\pgfqpoint{7.711232in}{3.637887in}}%
\pgfpathlineto{\pgfqpoint{7.711232in}{3.613611in}}%
\pgfpathclose%
\pgfusepath{stroke,fill}%
\end{pgfscope}%
\begin{pgfscope}%
\pgfsetrectcap%
\pgfsetmiterjoin%
\pgfsetlinewidth{0.803000pt}%
\definecolor{currentstroke}{rgb}{0.000000,0.000000,0.000000}%
\pgfsetstrokecolor{currentstroke}%
\pgfsetdash{}{0pt}%
\pgfpathmoveto{\pgfqpoint{7.074143in}{3.613611in}}%
\pgfpathlineto{\pgfqpoint{7.074143in}{4.075611in}}%
\pgfusepath{stroke}%
\end{pgfscope}%
\begin{pgfscope}%
\pgfsetrectcap%
\pgfsetmiterjoin%
\pgfsetlinewidth{0.803000pt}%
\definecolor{currentstroke}{rgb}{0.000000,0.000000,0.000000}%
\pgfsetstrokecolor{currentstroke}%
\pgfsetdash{}{0pt}%
\pgfpathmoveto{\pgfqpoint{7.898611in}{3.613611in}}%
\pgfpathlineto{\pgfqpoint{7.898611in}{4.075611in}}%
\pgfusepath{stroke}%
\end{pgfscope}%
\begin{pgfscope}%
\pgfsetrectcap%
\pgfsetmiterjoin%
\pgfsetlinewidth{0.803000pt}%
\definecolor{currentstroke}{rgb}{0.000000,0.000000,0.000000}%
\pgfsetstrokecolor{currentstroke}%
\pgfsetdash{}{0pt}%
\pgfpathmoveto{\pgfqpoint{7.074143in}{3.613611in}}%
\pgfpathlineto{\pgfqpoint{7.898611in}{3.613611in}}%
\pgfusepath{stroke}%
\end{pgfscope}%
\begin{pgfscope}%
\pgfsetrectcap%
\pgfsetmiterjoin%
\pgfsetlinewidth{0.803000pt}%
\definecolor{currentstroke}{rgb}{0.000000,0.000000,0.000000}%
\pgfsetstrokecolor{currentstroke}%
\pgfsetdash{}{0pt}%
\pgfpathmoveto{\pgfqpoint{7.074143in}{4.075611in}}%
\pgfpathlineto{\pgfqpoint{7.898611in}{4.075611in}}%
\pgfusepath{stroke}%
\end{pgfscope}%
\begin{pgfscope}%
\definecolor{textcolor}{rgb}{0.000000,0.000000,0.000000}%
\pgfsetstrokecolor{textcolor}%
\pgfsetfillcolor{textcolor}%
\pgftext[x=7.486377in,y=4.158944in,,base]{\color{textcolor}\rmfamily\fontsize{11.000000}{13.200000}\selectfont MAAF}%
\end{pgfscope}%
\begin{pgfscope}%
\pgfsetbuttcap%
\pgfsetmiterjoin%
\definecolor{currentfill}{rgb}{1.000000,1.000000,1.000000}%
\pgfsetfillcolor{currentfill}%
\pgfsetlinewidth{0.000000pt}%
\definecolor{currentstroke}{rgb}{0.000000,0.000000,0.000000}%
\pgfsetstrokecolor{currentstroke}%
\pgfsetstrokeopacity{0.000000}%
\pgfsetdash{}{0pt}%
\pgfpathmoveto{\pgfqpoint{0.148611in}{2.920611in}}%
\pgfpathlineto{\pgfqpoint{0.973079in}{2.920611in}}%
\pgfpathlineto{\pgfqpoint{0.973079in}{3.382611in}}%
\pgfpathlineto{\pgfqpoint{0.148611in}{3.382611in}}%
\pgfpathlineto{\pgfqpoint{0.148611in}{2.920611in}}%
\pgfpathclose%
\pgfusepath{fill}%
\end{pgfscope}%
\begin{pgfscope}%
\pgfpathrectangle{\pgfqpoint{0.148611in}{2.920611in}}{\pgfqpoint{0.824468in}{0.462000in}}%
\pgfusepath{clip}%
\pgfsetbuttcap%
\pgfsetmiterjoin%
\definecolor{currentfill}{rgb}{0.121569,0.466667,0.705882}%
\pgfsetfillcolor{currentfill}%
\pgfsetfillopacity{0.500000}%
\pgfsetlinewidth{1.003750pt}%
\definecolor{currentstroke}{rgb}{0.000000,0.000000,0.000000}%
\pgfsetstrokecolor{currentstroke}%
\pgfsetdash{}{0pt}%
\pgfpathmoveto{\pgfqpoint{0.186087in}{2.920611in}}%
\pgfpathlineto{\pgfqpoint{0.335990in}{2.920611in}}%
\pgfpathlineto{\pgfqpoint{0.335990in}{3.360611in}}%
\pgfpathlineto{\pgfqpoint{0.186087in}{3.360611in}}%
\pgfpathlineto{\pgfqpoint{0.186087in}{2.920611in}}%
\pgfpathclose%
\pgfusepath{stroke,fill}%
\end{pgfscope}%
\begin{pgfscope}%
\pgfpathrectangle{\pgfqpoint{0.148611in}{2.920611in}}{\pgfqpoint{0.824468in}{0.462000in}}%
\pgfusepath{clip}%
\pgfsetbuttcap%
\pgfsetmiterjoin%
\definecolor{currentfill}{rgb}{0.121569,0.466667,0.705882}%
\pgfsetfillcolor{currentfill}%
\pgfsetfillopacity{0.500000}%
\pgfsetlinewidth{1.003750pt}%
\definecolor{currentstroke}{rgb}{0.000000,0.000000,0.000000}%
\pgfsetstrokecolor{currentstroke}%
\pgfsetdash{}{0pt}%
\pgfpathmoveto{\pgfqpoint{0.335990in}{2.920611in}}%
\pgfpathlineto{\pgfqpoint{0.485894in}{2.920611in}}%
\pgfpathlineto{\pgfqpoint{0.485894in}{3.077754in}}%
\pgfpathlineto{\pgfqpoint{0.335990in}{3.077754in}}%
\pgfpathlineto{\pgfqpoint{0.335990in}{2.920611in}}%
\pgfpathclose%
\pgfusepath{stroke,fill}%
\end{pgfscope}%
\begin{pgfscope}%
\pgfpathrectangle{\pgfqpoint{0.148611in}{2.920611in}}{\pgfqpoint{0.824468in}{0.462000in}}%
\pgfusepath{clip}%
\pgfsetbuttcap%
\pgfsetmiterjoin%
\definecolor{currentfill}{rgb}{0.121569,0.466667,0.705882}%
\pgfsetfillcolor{currentfill}%
\pgfsetfillopacity{0.500000}%
\pgfsetlinewidth{1.003750pt}%
\definecolor{currentstroke}{rgb}{0.000000,0.000000,0.000000}%
\pgfsetstrokecolor{currentstroke}%
\pgfsetdash{}{0pt}%
\pgfpathmoveto{\pgfqpoint{0.485894in}{2.920611in}}%
\pgfpathlineto{\pgfqpoint{0.635797in}{2.920611in}}%
\pgfpathlineto{\pgfqpoint{0.635797in}{2.920611in}}%
\pgfpathlineto{\pgfqpoint{0.485894in}{2.920611in}}%
\pgfpathlineto{\pgfqpoint{0.485894in}{2.920611in}}%
\pgfpathclose%
\pgfusepath{stroke,fill}%
\end{pgfscope}%
\begin{pgfscope}%
\pgfpathrectangle{\pgfqpoint{0.148611in}{2.920611in}}{\pgfqpoint{0.824468in}{0.462000in}}%
\pgfusepath{clip}%
\pgfsetbuttcap%
\pgfsetmiterjoin%
\definecolor{currentfill}{rgb}{0.121569,0.466667,0.705882}%
\pgfsetfillcolor{currentfill}%
\pgfsetfillopacity{0.500000}%
\pgfsetlinewidth{1.003750pt}%
\definecolor{currentstroke}{rgb}{0.000000,0.000000,0.000000}%
\pgfsetstrokecolor{currentstroke}%
\pgfsetdash{}{0pt}%
\pgfpathmoveto{\pgfqpoint{0.635797in}{2.920611in}}%
\pgfpathlineto{\pgfqpoint{0.785700in}{2.920611in}}%
\pgfpathlineto{\pgfqpoint{0.785700in}{3.014897in}}%
\pgfpathlineto{\pgfqpoint{0.635797in}{3.014897in}}%
\pgfpathlineto{\pgfqpoint{0.635797in}{2.920611in}}%
\pgfpathclose%
\pgfusepath{stroke,fill}%
\end{pgfscope}%
\begin{pgfscope}%
\pgfpathrectangle{\pgfqpoint{0.148611in}{2.920611in}}{\pgfqpoint{0.824468in}{0.462000in}}%
\pgfusepath{clip}%
\pgfsetbuttcap%
\pgfsetmiterjoin%
\definecolor{currentfill}{rgb}{0.121569,0.466667,0.705882}%
\pgfsetfillcolor{currentfill}%
\pgfsetfillopacity{0.500000}%
\pgfsetlinewidth{1.003750pt}%
\definecolor{currentstroke}{rgb}{0.000000,0.000000,0.000000}%
\pgfsetstrokecolor{currentstroke}%
\pgfsetdash{}{0pt}%
\pgfpathmoveto{\pgfqpoint{0.785700in}{2.920611in}}%
\pgfpathlineto{\pgfqpoint{0.935603in}{2.920611in}}%
\pgfpathlineto{\pgfqpoint{0.935603in}{3.014897in}}%
\pgfpathlineto{\pgfqpoint{0.785700in}{3.014897in}}%
\pgfpathlineto{\pgfqpoint{0.785700in}{2.920611in}}%
\pgfpathclose%
\pgfusepath{stroke,fill}%
\end{pgfscope}%
\begin{pgfscope}%
\pgfsetrectcap%
\pgfsetmiterjoin%
\pgfsetlinewidth{0.803000pt}%
\definecolor{currentstroke}{rgb}{0.000000,0.000000,0.000000}%
\pgfsetstrokecolor{currentstroke}%
\pgfsetdash{}{0pt}%
\pgfpathmoveto{\pgfqpoint{0.148611in}{2.920611in}}%
\pgfpathlineto{\pgfqpoint{0.148611in}{3.382611in}}%
\pgfusepath{stroke}%
\end{pgfscope}%
\begin{pgfscope}%
\pgfsetrectcap%
\pgfsetmiterjoin%
\pgfsetlinewidth{0.803000pt}%
\definecolor{currentstroke}{rgb}{0.000000,0.000000,0.000000}%
\pgfsetstrokecolor{currentstroke}%
\pgfsetdash{}{0pt}%
\pgfpathmoveto{\pgfqpoint{0.973079in}{2.920611in}}%
\pgfpathlineto{\pgfqpoint{0.973079in}{3.382611in}}%
\pgfusepath{stroke}%
\end{pgfscope}%
\begin{pgfscope}%
\pgfsetrectcap%
\pgfsetmiterjoin%
\pgfsetlinewidth{0.803000pt}%
\definecolor{currentstroke}{rgb}{0.000000,0.000000,0.000000}%
\pgfsetstrokecolor{currentstroke}%
\pgfsetdash{}{0pt}%
\pgfpathmoveto{\pgfqpoint{0.148611in}{2.920611in}}%
\pgfpathlineto{\pgfqpoint{0.973079in}{2.920611in}}%
\pgfusepath{stroke}%
\end{pgfscope}%
\begin{pgfscope}%
\pgfsetrectcap%
\pgfsetmiterjoin%
\pgfsetlinewidth{0.803000pt}%
\definecolor{currentstroke}{rgb}{0.000000,0.000000,0.000000}%
\pgfsetstrokecolor{currentstroke}%
\pgfsetdash{}{0pt}%
\pgfpathmoveto{\pgfqpoint{0.148611in}{3.382611in}}%
\pgfpathlineto{\pgfqpoint{0.973079in}{3.382611in}}%
\pgfusepath{stroke}%
\end{pgfscope}%
\begin{pgfscope}%
\definecolor{textcolor}{rgb}{0.000000,0.000000,0.000000}%
\pgfsetstrokecolor{textcolor}%
\pgfsetfillcolor{textcolor}%
\pgftext[x=0.560845in,y=3.465944in,,base]{\color{textcolor}\rmfamily\fontsize{11.000000}{13.200000}\selectfont Solly ...}%
\end{pgfscope}%
\begin{pgfscope}%
\pgfsetbuttcap%
\pgfsetmiterjoin%
\definecolor{currentfill}{rgb}{1.000000,1.000000,1.000000}%
\pgfsetfillcolor{currentfill}%
\pgfsetlinewidth{0.000000pt}%
\definecolor{currentstroke}{rgb}{0.000000,0.000000,0.000000}%
\pgfsetstrokecolor{currentstroke}%
\pgfsetstrokeopacity{0.000000}%
\pgfsetdash{}{0pt}%
\pgfpathmoveto{\pgfqpoint{1.137973in}{2.920611in}}%
\pgfpathlineto{\pgfqpoint{1.962441in}{2.920611in}}%
\pgfpathlineto{\pgfqpoint{1.962441in}{3.382611in}}%
\pgfpathlineto{\pgfqpoint{1.137973in}{3.382611in}}%
\pgfpathlineto{\pgfqpoint{1.137973in}{2.920611in}}%
\pgfpathclose%
\pgfusepath{fill}%
\end{pgfscope}%
\begin{pgfscope}%
\pgfpathrectangle{\pgfqpoint{1.137973in}{2.920611in}}{\pgfqpoint{0.824468in}{0.462000in}}%
\pgfusepath{clip}%
\pgfsetbuttcap%
\pgfsetmiterjoin%
\definecolor{currentfill}{rgb}{0.121569,0.466667,0.705882}%
\pgfsetfillcolor{currentfill}%
\pgfsetfillopacity{0.500000}%
\pgfsetlinewidth{1.003750pt}%
\definecolor{currentstroke}{rgb}{0.000000,0.000000,0.000000}%
\pgfsetstrokecolor{currentstroke}%
\pgfsetdash{}{0pt}%
\pgfpathmoveto{\pgfqpoint{1.175449in}{2.920611in}}%
\pgfpathlineto{\pgfqpoint{1.325352in}{2.920611in}}%
\pgfpathlineto{\pgfqpoint{1.325352in}{3.337351in}}%
\pgfpathlineto{\pgfqpoint{1.175449in}{3.337351in}}%
\pgfpathlineto{\pgfqpoint{1.175449in}{2.920611in}}%
\pgfpathclose%
\pgfusepath{stroke,fill}%
\end{pgfscope}%
\begin{pgfscope}%
\pgfpathrectangle{\pgfqpoint{1.137973in}{2.920611in}}{\pgfqpoint{0.824468in}{0.462000in}}%
\pgfusepath{clip}%
\pgfsetbuttcap%
\pgfsetmiterjoin%
\definecolor{currentfill}{rgb}{0.121569,0.466667,0.705882}%
\pgfsetfillcolor{currentfill}%
\pgfsetfillopacity{0.500000}%
\pgfsetlinewidth{1.003750pt}%
\definecolor{currentstroke}{rgb}{0.000000,0.000000,0.000000}%
\pgfsetstrokecolor{currentstroke}%
\pgfsetdash{}{0pt}%
\pgfpathmoveto{\pgfqpoint{1.325352in}{2.920611in}}%
\pgfpathlineto{\pgfqpoint{1.475255in}{2.920611in}}%
\pgfpathlineto{\pgfqpoint{1.475255in}{3.348981in}}%
\pgfpathlineto{\pgfqpoint{1.325352in}{3.348981in}}%
\pgfpathlineto{\pgfqpoint{1.325352in}{2.920611in}}%
\pgfpathclose%
\pgfusepath{stroke,fill}%
\end{pgfscope}%
\begin{pgfscope}%
\pgfpathrectangle{\pgfqpoint{1.137973in}{2.920611in}}{\pgfqpoint{0.824468in}{0.462000in}}%
\pgfusepath{clip}%
\pgfsetbuttcap%
\pgfsetmiterjoin%
\definecolor{currentfill}{rgb}{0.121569,0.466667,0.705882}%
\pgfsetfillcolor{currentfill}%
\pgfsetfillopacity{0.500000}%
\pgfsetlinewidth{1.003750pt}%
\definecolor{currentstroke}{rgb}{0.000000,0.000000,0.000000}%
\pgfsetstrokecolor{currentstroke}%
\pgfsetdash{}{0pt}%
\pgfpathmoveto{\pgfqpoint{1.475255in}{2.920611in}}%
\pgfpathlineto{\pgfqpoint{1.625158in}{2.920611in}}%
\pgfpathlineto{\pgfqpoint{1.625158in}{3.269510in}}%
\pgfpathlineto{\pgfqpoint{1.475255in}{3.269510in}}%
\pgfpathlineto{\pgfqpoint{1.475255in}{2.920611in}}%
\pgfpathclose%
\pgfusepath{stroke,fill}%
\end{pgfscope}%
\begin{pgfscope}%
\pgfpathrectangle{\pgfqpoint{1.137973in}{2.920611in}}{\pgfqpoint{0.824468in}{0.462000in}}%
\pgfusepath{clip}%
\pgfsetbuttcap%
\pgfsetmiterjoin%
\definecolor{currentfill}{rgb}{0.121569,0.466667,0.705882}%
\pgfsetfillcolor{currentfill}%
\pgfsetfillopacity{0.500000}%
\pgfsetlinewidth{1.003750pt}%
\definecolor{currentstroke}{rgb}{0.000000,0.000000,0.000000}%
\pgfsetstrokecolor{currentstroke}%
\pgfsetdash{}{0pt}%
\pgfpathmoveto{\pgfqpoint{1.625158in}{2.920611in}}%
\pgfpathlineto{\pgfqpoint{1.775062in}{2.920611in}}%
\pgfpathlineto{\pgfqpoint{1.775062in}{3.360611in}}%
\pgfpathlineto{\pgfqpoint{1.625158in}{3.360611in}}%
\pgfpathlineto{\pgfqpoint{1.625158in}{2.920611in}}%
\pgfpathclose%
\pgfusepath{stroke,fill}%
\end{pgfscope}%
\begin{pgfscope}%
\pgfpathrectangle{\pgfqpoint{1.137973in}{2.920611in}}{\pgfqpoint{0.824468in}{0.462000in}}%
\pgfusepath{clip}%
\pgfsetbuttcap%
\pgfsetmiterjoin%
\definecolor{currentfill}{rgb}{0.121569,0.466667,0.705882}%
\pgfsetfillcolor{currentfill}%
\pgfsetfillopacity{0.500000}%
\pgfsetlinewidth{1.003750pt}%
\definecolor{currentstroke}{rgb}{0.000000,0.000000,0.000000}%
\pgfsetstrokecolor{currentstroke}%
\pgfsetdash{}{0pt}%
\pgfpathmoveto{\pgfqpoint{1.775062in}{2.920611in}}%
\pgfpathlineto{\pgfqpoint{1.924965in}{2.920611in}}%
\pgfpathlineto{\pgfqpoint{1.924965in}{3.221052in}}%
\pgfpathlineto{\pgfqpoint{1.775062in}{3.221052in}}%
\pgfpathlineto{\pgfqpoint{1.775062in}{2.920611in}}%
\pgfpathclose%
\pgfusepath{stroke,fill}%
\end{pgfscope}%
\begin{pgfscope}%
\pgfsetrectcap%
\pgfsetmiterjoin%
\pgfsetlinewidth{0.803000pt}%
\definecolor{currentstroke}{rgb}{0.000000,0.000000,0.000000}%
\pgfsetstrokecolor{currentstroke}%
\pgfsetdash{}{0pt}%
\pgfpathmoveto{\pgfqpoint{1.137973in}{2.920611in}}%
\pgfpathlineto{\pgfqpoint{1.137973in}{3.382611in}}%
\pgfusepath{stroke}%
\end{pgfscope}%
\begin{pgfscope}%
\pgfsetrectcap%
\pgfsetmiterjoin%
\pgfsetlinewidth{0.803000pt}%
\definecolor{currentstroke}{rgb}{0.000000,0.000000,0.000000}%
\pgfsetstrokecolor{currentstroke}%
\pgfsetdash{}{0pt}%
\pgfpathmoveto{\pgfqpoint{1.962441in}{2.920611in}}%
\pgfpathlineto{\pgfqpoint{1.962441in}{3.382611in}}%
\pgfusepath{stroke}%
\end{pgfscope}%
\begin{pgfscope}%
\pgfsetrectcap%
\pgfsetmiterjoin%
\pgfsetlinewidth{0.803000pt}%
\definecolor{currentstroke}{rgb}{0.000000,0.000000,0.000000}%
\pgfsetstrokecolor{currentstroke}%
\pgfsetdash{}{0pt}%
\pgfpathmoveto{\pgfqpoint{1.137973in}{2.920611in}}%
\pgfpathlineto{\pgfqpoint{1.962441in}{2.920611in}}%
\pgfusepath{stroke}%
\end{pgfscope}%
\begin{pgfscope}%
\pgfsetrectcap%
\pgfsetmiterjoin%
\pgfsetlinewidth{0.803000pt}%
\definecolor{currentstroke}{rgb}{0.000000,0.000000,0.000000}%
\pgfsetstrokecolor{currentstroke}%
\pgfsetdash{}{0pt}%
\pgfpathmoveto{\pgfqpoint{1.137973in}{3.382611in}}%
\pgfpathlineto{\pgfqpoint{1.962441in}{3.382611in}}%
\pgfusepath{stroke}%
\end{pgfscope}%
\begin{pgfscope}%
\definecolor{textcolor}{rgb}{0.000000,0.000000,0.000000}%
\pgfsetstrokecolor{textcolor}%
\pgfsetfillcolor{textcolor}%
\pgftext[x=1.550207in,y=3.465944in,,base]{\color{textcolor}\rmfamily\fontsize{11.000000}{13.200000}\selectfont GMF}%
\end{pgfscope}%
\begin{pgfscope}%
\pgfsetbuttcap%
\pgfsetmiterjoin%
\definecolor{currentfill}{rgb}{1.000000,1.000000,1.000000}%
\pgfsetfillcolor{currentfill}%
\pgfsetlinewidth{0.000000pt}%
\definecolor{currentstroke}{rgb}{0.000000,0.000000,0.000000}%
\pgfsetstrokecolor{currentstroke}%
\pgfsetstrokeopacity{0.000000}%
\pgfsetdash{}{0pt}%
\pgfpathmoveto{\pgfqpoint{2.127335in}{2.920611in}}%
\pgfpathlineto{\pgfqpoint{2.951803in}{2.920611in}}%
\pgfpathlineto{\pgfqpoint{2.951803in}{3.382611in}}%
\pgfpathlineto{\pgfqpoint{2.127335in}{3.382611in}}%
\pgfpathlineto{\pgfqpoint{2.127335in}{2.920611in}}%
\pgfpathclose%
\pgfusepath{fill}%
\end{pgfscope}%
\begin{pgfscope}%
\pgfpathrectangle{\pgfqpoint{2.127335in}{2.920611in}}{\pgfqpoint{0.824468in}{0.462000in}}%
\pgfusepath{clip}%
\pgfsetbuttcap%
\pgfsetmiterjoin%
\definecolor{currentfill}{rgb}{0.121569,0.466667,0.705882}%
\pgfsetfillcolor{currentfill}%
\pgfsetfillopacity{0.500000}%
\pgfsetlinewidth{1.003750pt}%
\definecolor{currentstroke}{rgb}{0.000000,0.000000,0.000000}%
\pgfsetstrokecolor{currentstroke}%
\pgfsetdash{}{0pt}%
\pgfpathmoveto{\pgfqpoint{2.164810in}{2.920611in}}%
\pgfpathlineto{\pgfqpoint{2.314714in}{2.920611in}}%
\pgfpathlineto{\pgfqpoint{2.314714in}{3.044886in}}%
\pgfpathlineto{\pgfqpoint{2.164810in}{3.044886in}}%
\pgfpathlineto{\pgfqpoint{2.164810in}{2.920611in}}%
\pgfpathclose%
\pgfusepath{stroke,fill}%
\end{pgfscope}%
\begin{pgfscope}%
\pgfpathrectangle{\pgfqpoint{2.127335in}{2.920611in}}{\pgfqpoint{0.824468in}{0.462000in}}%
\pgfusepath{clip}%
\pgfsetbuttcap%
\pgfsetmiterjoin%
\definecolor{currentfill}{rgb}{0.121569,0.466667,0.705882}%
\pgfsetfillcolor{currentfill}%
\pgfsetfillopacity{0.500000}%
\pgfsetlinewidth{1.003750pt}%
\definecolor{currentstroke}{rgb}{0.000000,0.000000,0.000000}%
\pgfsetstrokecolor{currentstroke}%
\pgfsetdash{}{0pt}%
\pgfpathmoveto{\pgfqpoint{2.314714in}{2.920611in}}%
\pgfpathlineto{\pgfqpoint{2.464617in}{2.920611in}}%
\pgfpathlineto{\pgfqpoint{2.464617in}{3.065039in}}%
\pgfpathlineto{\pgfqpoint{2.314714in}{3.065039in}}%
\pgfpathlineto{\pgfqpoint{2.314714in}{2.920611in}}%
\pgfpathclose%
\pgfusepath{stroke,fill}%
\end{pgfscope}%
\begin{pgfscope}%
\pgfpathrectangle{\pgfqpoint{2.127335in}{2.920611in}}{\pgfqpoint{0.824468in}{0.462000in}}%
\pgfusepath{clip}%
\pgfsetbuttcap%
\pgfsetmiterjoin%
\definecolor{currentfill}{rgb}{0.121569,0.466667,0.705882}%
\pgfsetfillcolor{currentfill}%
\pgfsetfillopacity{0.500000}%
\pgfsetlinewidth{1.003750pt}%
\definecolor{currentstroke}{rgb}{0.000000,0.000000,0.000000}%
\pgfsetstrokecolor{currentstroke}%
\pgfsetdash{}{0pt}%
\pgfpathmoveto{\pgfqpoint{2.464617in}{2.920611in}}%
\pgfpathlineto{\pgfqpoint{2.614520in}{2.920611in}}%
\pgfpathlineto{\pgfqpoint{2.614520in}{3.054962in}}%
\pgfpathlineto{\pgfqpoint{2.464617in}{3.054962in}}%
\pgfpathlineto{\pgfqpoint{2.464617in}{2.920611in}}%
\pgfpathclose%
\pgfusepath{stroke,fill}%
\end{pgfscope}%
\begin{pgfscope}%
\pgfpathrectangle{\pgfqpoint{2.127335in}{2.920611in}}{\pgfqpoint{0.824468in}{0.462000in}}%
\pgfusepath{clip}%
\pgfsetbuttcap%
\pgfsetmiterjoin%
\definecolor{currentfill}{rgb}{0.121569,0.466667,0.705882}%
\pgfsetfillcolor{currentfill}%
\pgfsetfillopacity{0.500000}%
\pgfsetlinewidth{1.003750pt}%
\definecolor{currentstroke}{rgb}{0.000000,0.000000,0.000000}%
\pgfsetstrokecolor{currentstroke}%
\pgfsetdash{}{0pt}%
\pgfpathmoveto{\pgfqpoint{2.614520in}{2.920611in}}%
\pgfpathlineto{\pgfqpoint{2.764423in}{2.920611in}}%
\pgfpathlineto{\pgfqpoint{2.764423in}{3.232978in}}%
\pgfpathlineto{\pgfqpoint{2.614520in}{3.232978in}}%
\pgfpathlineto{\pgfqpoint{2.614520in}{2.920611in}}%
\pgfpathclose%
\pgfusepath{stroke,fill}%
\end{pgfscope}%
\begin{pgfscope}%
\pgfpathrectangle{\pgfqpoint{2.127335in}{2.920611in}}{\pgfqpoint{0.824468in}{0.462000in}}%
\pgfusepath{clip}%
\pgfsetbuttcap%
\pgfsetmiterjoin%
\definecolor{currentfill}{rgb}{0.121569,0.466667,0.705882}%
\pgfsetfillcolor{currentfill}%
\pgfsetfillopacity{0.500000}%
\pgfsetlinewidth{1.003750pt}%
\definecolor{currentstroke}{rgb}{0.000000,0.000000,0.000000}%
\pgfsetstrokecolor{currentstroke}%
\pgfsetdash{}{0pt}%
\pgfpathmoveto{\pgfqpoint{2.764423in}{2.920611in}}%
\pgfpathlineto{\pgfqpoint{2.914327in}{2.920611in}}%
\pgfpathlineto{\pgfqpoint{2.914327in}{3.360611in}}%
\pgfpathlineto{\pgfqpoint{2.764423in}{3.360611in}}%
\pgfpathlineto{\pgfqpoint{2.764423in}{2.920611in}}%
\pgfpathclose%
\pgfusepath{stroke,fill}%
\end{pgfscope}%
\begin{pgfscope}%
\pgfsetrectcap%
\pgfsetmiterjoin%
\pgfsetlinewidth{0.803000pt}%
\definecolor{currentstroke}{rgb}{0.000000,0.000000,0.000000}%
\pgfsetstrokecolor{currentstroke}%
\pgfsetdash{}{0pt}%
\pgfpathmoveto{\pgfqpoint{2.127335in}{2.920611in}}%
\pgfpathlineto{\pgfqpoint{2.127335in}{3.382611in}}%
\pgfusepath{stroke}%
\end{pgfscope}%
\begin{pgfscope}%
\pgfsetrectcap%
\pgfsetmiterjoin%
\pgfsetlinewidth{0.803000pt}%
\definecolor{currentstroke}{rgb}{0.000000,0.000000,0.000000}%
\pgfsetstrokecolor{currentstroke}%
\pgfsetdash{}{0pt}%
\pgfpathmoveto{\pgfqpoint{2.951803in}{2.920611in}}%
\pgfpathlineto{\pgfqpoint{2.951803in}{3.382611in}}%
\pgfusepath{stroke}%
\end{pgfscope}%
\begin{pgfscope}%
\pgfsetrectcap%
\pgfsetmiterjoin%
\pgfsetlinewidth{0.803000pt}%
\definecolor{currentstroke}{rgb}{0.000000,0.000000,0.000000}%
\pgfsetstrokecolor{currentstroke}%
\pgfsetdash{}{0pt}%
\pgfpathmoveto{\pgfqpoint{2.127335in}{2.920611in}}%
\pgfpathlineto{\pgfqpoint{2.951803in}{2.920611in}}%
\pgfusepath{stroke}%
\end{pgfscope}%
\begin{pgfscope}%
\pgfsetrectcap%
\pgfsetmiterjoin%
\pgfsetlinewidth{0.803000pt}%
\definecolor{currentstroke}{rgb}{0.000000,0.000000,0.000000}%
\pgfsetstrokecolor{currentstroke}%
\pgfsetdash{}{0pt}%
\pgfpathmoveto{\pgfqpoint{2.127335in}{3.382611in}}%
\pgfpathlineto{\pgfqpoint{2.951803in}{3.382611in}}%
\pgfusepath{stroke}%
\end{pgfscope}%
\begin{pgfscope}%
\definecolor{textcolor}{rgb}{0.000000,0.000000,0.000000}%
\pgfsetstrokecolor{textcolor}%
\pgfsetfillcolor{textcolor}%
\pgftext[x=2.539569in,y=3.465944in,,base]{\color{textcolor}\rmfamily\fontsize{11.000000}{13.200000}\selectfont AMV}%
\end{pgfscope}%
\begin{pgfscope}%
\pgfsetbuttcap%
\pgfsetmiterjoin%
\definecolor{currentfill}{rgb}{1.000000,1.000000,1.000000}%
\pgfsetfillcolor{currentfill}%
\pgfsetlinewidth{0.000000pt}%
\definecolor{currentstroke}{rgb}{0.000000,0.000000,0.000000}%
\pgfsetstrokecolor{currentstroke}%
\pgfsetstrokeopacity{0.000000}%
\pgfsetdash{}{0pt}%
\pgfpathmoveto{\pgfqpoint{3.116696in}{2.920611in}}%
\pgfpathlineto{\pgfqpoint{3.941164in}{2.920611in}}%
\pgfpathlineto{\pgfqpoint{3.941164in}{3.382611in}}%
\pgfpathlineto{\pgfqpoint{3.116696in}{3.382611in}}%
\pgfpathlineto{\pgfqpoint{3.116696in}{2.920611in}}%
\pgfpathclose%
\pgfusepath{fill}%
\end{pgfscope}%
\begin{pgfscope}%
\pgfpathrectangle{\pgfqpoint{3.116696in}{2.920611in}}{\pgfqpoint{0.824468in}{0.462000in}}%
\pgfusepath{clip}%
\pgfsetbuttcap%
\pgfsetmiterjoin%
\definecolor{currentfill}{rgb}{0.121569,0.466667,0.705882}%
\pgfsetfillcolor{currentfill}%
\pgfsetfillopacity{0.500000}%
\pgfsetlinewidth{1.003750pt}%
\definecolor{currentstroke}{rgb}{0.000000,0.000000,0.000000}%
\pgfsetstrokecolor{currentstroke}%
\pgfsetdash{}{0pt}%
\pgfpathmoveto{\pgfqpoint{3.154172in}{2.920611in}}%
\pgfpathlineto{\pgfqpoint{3.304075in}{2.920611in}}%
\pgfpathlineto{\pgfqpoint{3.304075in}{3.360611in}}%
\pgfpathlineto{\pgfqpoint{3.154172in}{3.360611in}}%
\pgfpathlineto{\pgfqpoint{3.154172in}{2.920611in}}%
\pgfpathclose%
\pgfusepath{stroke,fill}%
\end{pgfscope}%
\begin{pgfscope}%
\pgfpathrectangle{\pgfqpoint{3.116696in}{2.920611in}}{\pgfqpoint{0.824468in}{0.462000in}}%
\pgfusepath{clip}%
\pgfsetbuttcap%
\pgfsetmiterjoin%
\definecolor{currentfill}{rgb}{0.121569,0.466667,0.705882}%
\pgfsetfillcolor{currentfill}%
\pgfsetfillopacity{0.500000}%
\pgfsetlinewidth{1.003750pt}%
\definecolor{currentstroke}{rgb}{0.000000,0.000000,0.000000}%
\pgfsetstrokecolor{currentstroke}%
\pgfsetdash{}{0pt}%
\pgfpathmoveto{\pgfqpoint{3.304075in}{2.920611in}}%
\pgfpathlineto{\pgfqpoint{3.453979in}{2.920611in}}%
\pgfpathlineto{\pgfqpoint{3.453979in}{2.987998in}}%
\pgfpathlineto{\pgfqpoint{3.304075in}{2.987998in}}%
\pgfpathlineto{\pgfqpoint{3.304075in}{2.920611in}}%
\pgfpathclose%
\pgfusepath{stroke,fill}%
\end{pgfscope}%
\begin{pgfscope}%
\pgfpathrectangle{\pgfqpoint{3.116696in}{2.920611in}}{\pgfqpoint{0.824468in}{0.462000in}}%
\pgfusepath{clip}%
\pgfsetbuttcap%
\pgfsetmiterjoin%
\definecolor{currentfill}{rgb}{0.121569,0.466667,0.705882}%
\pgfsetfillcolor{currentfill}%
\pgfsetfillopacity{0.500000}%
\pgfsetlinewidth{1.003750pt}%
\definecolor{currentstroke}{rgb}{0.000000,0.000000,0.000000}%
\pgfsetstrokecolor{currentstroke}%
\pgfsetdash{}{0pt}%
\pgfpathmoveto{\pgfqpoint{3.453979in}{2.920611in}}%
\pgfpathlineto{\pgfqpoint{3.603882in}{2.920611in}}%
\pgfpathlineto{\pgfqpoint{3.603882in}{2.980071in}}%
\pgfpathlineto{\pgfqpoint{3.453979in}{2.980071in}}%
\pgfpathlineto{\pgfqpoint{3.453979in}{2.920611in}}%
\pgfpathclose%
\pgfusepath{stroke,fill}%
\end{pgfscope}%
\begin{pgfscope}%
\pgfpathrectangle{\pgfqpoint{3.116696in}{2.920611in}}{\pgfqpoint{0.824468in}{0.462000in}}%
\pgfusepath{clip}%
\pgfsetbuttcap%
\pgfsetmiterjoin%
\definecolor{currentfill}{rgb}{0.121569,0.466667,0.705882}%
\pgfsetfillcolor{currentfill}%
\pgfsetfillopacity{0.500000}%
\pgfsetlinewidth{1.003750pt}%
\definecolor{currentstroke}{rgb}{0.000000,0.000000,0.000000}%
\pgfsetstrokecolor{currentstroke}%
\pgfsetdash{}{0pt}%
\pgfpathmoveto{\pgfqpoint{3.603882in}{2.920611in}}%
\pgfpathlineto{\pgfqpoint{3.753785in}{2.920611in}}%
\pgfpathlineto{\pgfqpoint{3.753785in}{2.920611in}}%
\pgfpathlineto{\pgfqpoint{3.603882in}{2.920611in}}%
\pgfpathlineto{\pgfqpoint{3.603882in}{2.920611in}}%
\pgfpathclose%
\pgfusepath{stroke,fill}%
\end{pgfscope}%
\begin{pgfscope}%
\pgfpathrectangle{\pgfqpoint{3.116696in}{2.920611in}}{\pgfqpoint{0.824468in}{0.462000in}}%
\pgfusepath{clip}%
\pgfsetbuttcap%
\pgfsetmiterjoin%
\definecolor{currentfill}{rgb}{0.121569,0.466667,0.705882}%
\pgfsetfillcolor{currentfill}%
\pgfsetfillopacity{0.500000}%
\pgfsetlinewidth{1.003750pt}%
\definecolor{currentstroke}{rgb}{0.000000,0.000000,0.000000}%
\pgfsetstrokecolor{currentstroke}%
\pgfsetdash{}{0pt}%
\pgfpathmoveto{\pgfqpoint{3.753785in}{2.920611in}}%
\pgfpathlineto{\pgfqpoint{3.903688in}{2.920611in}}%
\pgfpathlineto{\pgfqpoint{3.903688in}{2.936467in}}%
\pgfpathlineto{\pgfqpoint{3.753785in}{2.936467in}}%
\pgfpathlineto{\pgfqpoint{3.753785in}{2.920611in}}%
\pgfpathclose%
\pgfusepath{stroke,fill}%
\end{pgfscope}%
\begin{pgfscope}%
\pgfsetrectcap%
\pgfsetmiterjoin%
\pgfsetlinewidth{0.803000pt}%
\definecolor{currentstroke}{rgb}{0.000000,0.000000,0.000000}%
\pgfsetstrokecolor{currentstroke}%
\pgfsetdash{}{0pt}%
\pgfpathmoveto{\pgfqpoint{3.116696in}{2.920611in}}%
\pgfpathlineto{\pgfqpoint{3.116696in}{3.382611in}}%
\pgfusepath{stroke}%
\end{pgfscope}%
\begin{pgfscope}%
\pgfsetrectcap%
\pgfsetmiterjoin%
\pgfsetlinewidth{0.803000pt}%
\definecolor{currentstroke}{rgb}{0.000000,0.000000,0.000000}%
\pgfsetstrokecolor{currentstroke}%
\pgfsetdash{}{0pt}%
\pgfpathmoveto{\pgfqpoint{3.941164in}{2.920611in}}%
\pgfpathlineto{\pgfqpoint{3.941164in}{3.382611in}}%
\pgfusepath{stroke}%
\end{pgfscope}%
\begin{pgfscope}%
\pgfsetrectcap%
\pgfsetmiterjoin%
\pgfsetlinewidth{0.803000pt}%
\definecolor{currentstroke}{rgb}{0.000000,0.000000,0.000000}%
\pgfsetstrokecolor{currentstroke}%
\pgfsetdash{}{0pt}%
\pgfpathmoveto{\pgfqpoint{3.116696in}{2.920611in}}%
\pgfpathlineto{\pgfqpoint{3.941164in}{2.920611in}}%
\pgfusepath{stroke}%
\end{pgfscope}%
\begin{pgfscope}%
\pgfsetrectcap%
\pgfsetmiterjoin%
\pgfsetlinewidth{0.803000pt}%
\definecolor{currentstroke}{rgb}{0.000000,0.000000,0.000000}%
\pgfsetstrokecolor{currentstroke}%
\pgfsetdash{}{0pt}%
\pgfpathmoveto{\pgfqpoint{3.116696in}{3.382611in}}%
\pgfpathlineto{\pgfqpoint{3.941164in}{3.382611in}}%
\pgfusepath{stroke}%
\end{pgfscope}%
\begin{pgfscope}%
\definecolor{textcolor}{rgb}{0.000000,0.000000,0.000000}%
\pgfsetstrokecolor{textcolor}%
\pgfsetfillcolor{textcolor}%
\pgftext[x=3.528930in,y=3.465944in,,base]{\color{textcolor}\rmfamily\fontsize{11.000000}{13.200000}\selectfont CNP As...}%
\end{pgfscope}%
\begin{pgfscope}%
\pgfsetbuttcap%
\pgfsetmiterjoin%
\definecolor{currentfill}{rgb}{1.000000,1.000000,1.000000}%
\pgfsetfillcolor{currentfill}%
\pgfsetlinewidth{0.000000pt}%
\definecolor{currentstroke}{rgb}{0.000000,0.000000,0.000000}%
\pgfsetstrokecolor{currentstroke}%
\pgfsetstrokeopacity{0.000000}%
\pgfsetdash{}{0pt}%
\pgfpathmoveto{\pgfqpoint{4.106058in}{2.920611in}}%
\pgfpathlineto{\pgfqpoint{4.930526in}{2.920611in}}%
\pgfpathlineto{\pgfqpoint{4.930526in}{3.382611in}}%
\pgfpathlineto{\pgfqpoint{4.106058in}{3.382611in}}%
\pgfpathlineto{\pgfqpoint{4.106058in}{2.920611in}}%
\pgfpathclose%
\pgfusepath{fill}%
\end{pgfscope}%
\begin{pgfscope}%
\pgfpathrectangle{\pgfqpoint{4.106058in}{2.920611in}}{\pgfqpoint{0.824468in}{0.462000in}}%
\pgfusepath{clip}%
\pgfsetbuttcap%
\pgfsetmiterjoin%
\definecolor{currentfill}{rgb}{0.121569,0.466667,0.705882}%
\pgfsetfillcolor{currentfill}%
\pgfsetfillopacity{0.500000}%
\pgfsetlinewidth{1.003750pt}%
\definecolor{currentstroke}{rgb}{0.000000,0.000000,0.000000}%
\pgfsetstrokecolor{currentstroke}%
\pgfsetdash{}{0pt}%
\pgfpathmoveto{\pgfqpoint{4.143534in}{2.920611in}}%
\pgfpathlineto{\pgfqpoint{4.293437in}{2.920611in}}%
\pgfpathlineto{\pgfqpoint{4.293437in}{3.360611in}}%
\pgfpathlineto{\pgfqpoint{4.143534in}{3.360611in}}%
\pgfpathlineto{\pgfqpoint{4.143534in}{2.920611in}}%
\pgfpathclose%
\pgfusepath{stroke,fill}%
\end{pgfscope}%
\begin{pgfscope}%
\pgfpathrectangle{\pgfqpoint{4.106058in}{2.920611in}}{\pgfqpoint{0.824468in}{0.462000in}}%
\pgfusepath{clip}%
\pgfsetbuttcap%
\pgfsetmiterjoin%
\definecolor{currentfill}{rgb}{0.121569,0.466667,0.705882}%
\pgfsetfillcolor{currentfill}%
\pgfsetfillopacity{0.500000}%
\pgfsetlinewidth{1.003750pt}%
\definecolor{currentstroke}{rgb}{0.000000,0.000000,0.000000}%
\pgfsetstrokecolor{currentstroke}%
\pgfsetdash{}{0pt}%
\pgfpathmoveto{\pgfqpoint{4.293437in}{2.920611in}}%
\pgfpathlineto{\pgfqpoint{4.443340in}{2.920611in}}%
\pgfpathlineto{\pgfqpoint{4.443340in}{3.123796in}}%
\pgfpathlineto{\pgfqpoint{4.293437in}{3.123796in}}%
\pgfpathlineto{\pgfqpoint{4.293437in}{2.920611in}}%
\pgfpathclose%
\pgfusepath{stroke,fill}%
\end{pgfscope}%
\begin{pgfscope}%
\pgfpathrectangle{\pgfqpoint{4.106058in}{2.920611in}}{\pgfqpoint{0.824468in}{0.462000in}}%
\pgfusepath{clip}%
\pgfsetbuttcap%
\pgfsetmiterjoin%
\definecolor{currentfill}{rgb}{0.121569,0.466667,0.705882}%
\pgfsetfillcolor{currentfill}%
\pgfsetfillopacity{0.500000}%
\pgfsetlinewidth{1.003750pt}%
\definecolor{currentstroke}{rgb}{0.000000,0.000000,0.000000}%
\pgfsetstrokecolor{currentstroke}%
\pgfsetdash{}{0pt}%
\pgfpathmoveto{\pgfqpoint{4.443340in}{2.920611in}}%
\pgfpathlineto{\pgfqpoint{4.593244in}{2.920611in}}%
\pgfpathlineto{\pgfqpoint{4.593244in}{3.017299in}}%
\pgfpathlineto{\pgfqpoint{4.443340in}{3.017299in}}%
\pgfpathlineto{\pgfqpoint{4.443340in}{2.920611in}}%
\pgfpathclose%
\pgfusepath{stroke,fill}%
\end{pgfscope}%
\begin{pgfscope}%
\pgfpathrectangle{\pgfqpoint{4.106058in}{2.920611in}}{\pgfqpoint{0.824468in}{0.462000in}}%
\pgfusepath{clip}%
\pgfsetbuttcap%
\pgfsetmiterjoin%
\definecolor{currentfill}{rgb}{0.121569,0.466667,0.705882}%
\pgfsetfillcolor{currentfill}%
\pgfsetfillopacity{0.500000}%
\pgfsetlinewidth{1.003750pt}%
\definecolor{currentstroke}{rgb}{0.000000,0.000000,0.000000}%
\pgfsetstrokecolor{currentstroke}%
\pgfsetdash{}{0pt}%
\pgfpathmoveto{\pgfqpoint{4.593244in}{2.920611in}}%
\pgfpathlineto{\pgfqpoint{4.743147in}{2.920611in}}%
\pgfpathlineto{\pgfqpoint{4.743147in}{2.978063in}}%
\pgfpathlineto{\pgfqpoint{4.593244in}{2.978063in}}%
\pgfpathlineto{\pgfqpoint{4.593244in}{2.920611in}}%
\pgfpathclose%
\pgfusepath{stroke,fill}%
\end{pgfscope}%
\begin{pgfscope}%
\pgfpathrectangle{\pgfqpoint{4.106058in}{2.920611in}}{\pgfqpoint{0.824468in}{0.462000in}}%
\pgfusepath{clip}%
\pgfsetbuttcap%
\pgfsetmiterjoin%
\definecolor{currentfill}{rgb}{0.121569,0.466667,0.705882}%
\pgfsetfillcolor{currentfill}%
\pgfsetfillopacity{0.500000}%
\pgfsetlinewidth{1.003750pt}%
\definecolor{currentstroke}{rgb}{0.000000,0.000000,0.000000}%
\pgfsetstrokecolor{currentstroke}%
\pgfsetdash{}{0pt}%
\pgfpathmoveto{\pgfqpoint{4.743147in}{2.920611in}}%
\pgfpathlineto{\pgfqpoint{4.893050in}{2.920611in}}%
\pgfpathlineto{\pgfqpoint{4.893050in}{2.951439in}}%
\pgfpathlineto{\pgfqpoint{4.743147in}{2.951439in}}%
\pgfpathlineto{\pgfqpoint{4.743147in}{2.920611in}}%
\pgfpathclose%
\pgfusepath{stroke,fill}%
\end{pgfscope}%
\begin{pgfscope}%
\pgfsetrectcap%
\pgfsetmiterjoin%
\pgfsetlinewidth{0.803000pt}%
\definecolor{currentstroke}{rgb}{0.000000,0.000000,0.000000}%
\pgfsetstrokecolor{currentstroke}%
\pgfsetdash{}{0pt}%
\pgfpathmoveto{\pgfqpoint{4.106058in}{2.920611in}}%
\pgfpathlineto{\pgfqpoint{4.106058in}{3.382611in}}%
\pgfusepath{stroke}%
\end{pgfscope}%
\begin{pgfscope}%
\pgfsetrectcap%
\pgfsetmiterjoin%
\pgfsetlinewidth{0.803000pt}%
\definecolor{currentstroke}{rgb}{0.000000,0.000000,0.000000}%
\pgfsetstrokecolor{currentstroke}%
\pgfsetdash{}{0pt}%
\pgfpathmoveto{\pgfqpoint{4.930526in}{2.920611in}}%
\pgfpathlineto{\pgfqpoint{4.930526in}{3.382611in}}%
\pgfusepath{stroke}%
\end{pgfscope}%
\begin{pgfscope}%
\pgfsetrectcap%
\pgfsetmiterjoin%
\pgfsetlinewidth{0.803000pt}%
\definecolor{currentstroke}{rgb}{0.000000,0.000000,0.000000}%
\pgfsetstrokecolor{currentstroke}%
\pgfsetdash{}{0pt}%
\pgfpathmoveto{\pgfqpoint{4.106058in}{2.920611in}}%
\pgfpathlineto{\pgfqpoint{4.930526in}{2.920611in}}%
\pgfusepath{stroke}%
\end{pgfscope}%
\begin{pgfscope}%
\pgfsetrectcap%
\pgfsetmiterjoin%
\pgfsetlinewidth{0.803000pt}%
\definecolor{currentstroke}{rgb}{0.000000,0.000000,0.000000}%
\pgfsetstrokecolor{currentstroke}%
\pgfsetdash{}{0pt}%
\pgfpathmoveto{\pgfqpoint{4.106058in}{3.382611in}}%
\pgfpathlineto{\pgfqpoint{4.930526in}{3.382611in}}%
\pgfusepath{stroke}%
\end{pgfscope}%
\begin{pgfscope}%
\definecolor{textcolor}{rgb}{0.000000,0.000000,0.000000}%
\pgfsetstrokecolor{textcolor}%
\pgfsetfillcolor{textcolor}%
\pgftext[x=4.518292in,y=3.465944in,,base]{\color{textcolor}\rmfamily\fontsize{11.000000}{13.200000}\selectfont MAIF}%
\end{pgfscope}%
\begin{pgfscope}%
\pgfsetbuttcap%
\pgfsetmiterjoin%
\definecolor{currentfill}{rgb}{1.000000,1.000000,1.000000}%
\pgfsetfillcolor{currentfill}%
\pgfsetlinewidth{0.000000pt}%
\definecolor{currentstroke}{rgb}{0.000000,0.000000,0.000000}%
\pgfsetstrokecolor{currentstroke}%
\pgfsetstrokeopacity{0.000000}%
\pgfsetdash{}{0pt}%
\pgfpathmoveto{\pgfqpoint{5.095420in}{2.920611in}}%
\pgfpathlineto{\pgfqpoint{5.919888in}{2.920611in}}%
\pgfpathlineto{\pgfqpoint{5.919888in}{3.382611in}}%
\pgfpathlineto{\pgfqpoint{5.095420in}{3.382611in}}%
\pgfpathlineto{\pgfqpoint{5.095420in}{2.920611in}}%
\pgfpathclose%
\pgfusepath{fill}%
\end{pgfscope}%
\begin{pgfscope}%
\pgfpathrectangle{\pgfqpoint{5.095420in}{2.920611in}}{\pgfqpoint{0.824468in}{0.462000in}}%
\pgfusepath{clip}%
\pgfsetbuttcap%
\pgfsetmiterjoin%
\definecolor{currentfill}{rgb}{0.121569,0.466667,0.705882}%
\pgfsetfillcolor{currentfill}%
\pgfsetfillopacity{0.500000}%
\pgfsetlinewidth{1.003750pt}%
\definecolor{currentstroke}{rgb}{0.000000,0.000000,0.000000}%
\pgfsetstrokecolor{currentstroke}%
\pgfsetdash{}{0pt}%
\pgfpathmoveto{\pgfqpoint{5.132895in}{2.920611in}}%
\pgfpathlineto{\pgfqpoint{5.282799in}{2.920611in}}%
\pgfpathlineto{\pgfqpoint{5.282799in}{3.360611in}}%
\pgfpathlineto{\pgfqpoint{5.132895in}{3.360611in}}%
\pgfpathlineto{\pgfqpoint{5.132895in}{2.920611in}}%
\pgfpathclose%
\pgfusepath{stroke,fill}%
\end{pgfscope}%
\begin{pgfscope}%
\pgfpathrectangle{\pgfqpoint{5.095420in}{2.920611in}}{\pgfqpoint{0.824468in}{0.462000in}}%
\pgfusepath{clip}%
\pgfsetbuttcap%
\pgfsetmiterjoin%
\definecolor{currentfill}{rgb}{0.121569,0.466667,0.705882}%
\pgfsetfillcolor{currentfill}%
\pgfsetfillopacity{0.500000}%
\pgfsetlinewidth{1.003750pt}%
\definecolor{currentstroke}{rgb}{0.000000,0.000000,0.000000}%
\pgfsetstrokecolor{currentstroke}%
\pgfsetdash{}{0pt}%
\pgfpathmoveto{\pgfqpoint{5.282799in}{2.920611in}}%
\pgfpathlineto{\pgfqpoint{5.432702in}{2.920611in}}%
\pgfpathlineto{\pgfqpoint{5.432702in}{3.028366in}}%
\pgfpathlineto{\pgfqpoint{5.282799in}{3.028366in}}%
\pgfpathlineto{\pgfqpoint{5.282799in}{2.920611in}}%
\pgfpathclose%
\pgfusepath{stroke,fill}%
\end{pgfscope}%
\begin{pgfscope}%
\pgfpathrectangle{\pgfqpoint{5.095420in}{2.920611in}}{\pgfqpoint{0.824468in}{0.462000in}}%
\pgfusepath{clip}%
\pgfsetbuttcap%
\pgfsetmiterjoin%
\definecolor{currentfill}{rgb}{0.121569,0.466667,0.705882}%
\pgfsetfillcolor{currentfill}%
\pgfsetfillopacity{0.500000}%
\pgfsetlinewidth{1.003750pt}%
\definecolor{currentstroke}{rgb}{0.000000,0.000000,0.000000}%
\pgfsetstrokecolor{currentstroke}%
\pgfsetdash{}{0pt}%
\pgfpathmoveto{\pgfqpoint{5.432702in}{2.920611in}}%
\pgfpathlineto{\pgfqpoint{5.582605in}{2.920611in}}%
\pgfpathlineto{\pgfqpoint{5.582605in}{2.974489in}}%
\pgfpathlineto{\pgfqpoint{5.432702in}{2.974489in}}%
\pgfpathlineto{\pgfqpoint{5.432702in}{2.920611in}}%
\pgfpathclose%
\pgfusepath{stroke,fill}%
\end{pgfscope}%
\begin{pgfscope}%
\pgfpathrectangle{\pgfqpoint{5.095420in}{2.920611in}}{\pgfqpoint{0.824468in}{0.462000in}}%
\pgfusepath{clip}%
\pgfsetbuttcap%
\pgfsetmiterjoin%
\definecolor{currentfill}{rgb}{0.121569,0.466667,0.705882}%
\pgfsetfillcolor{currentfill}%
\pgfsetfillopacity{0.500000}%
\pgfsetlinewidth{1.003750pt}%
\definecolor{currentstroke}{rgb}{0.000000,0.000000,0.000000}%
\pgfsetstrokecolor{currentstroke}%
\pgfsetdash{}{0pt}%
\pgfpathmoveto{\pgfqpoint{5.582605in}{2.920611in}}%
\pgfpathlineto{\pgfqpoint{5.732509in}{2.920611in}}%
\pgfpathlineto{\pgfqpoint{5.732509in}{2.947550in}}%
\pgfpathlineto{\pgfqpoint{5.582605in}{2.947550in}}%
\pgfpathlineto{\pgfqpoint{5.582605in}{2.920611in}}%
\pgfpathclose%
\pgfusepath{stroke,fill}%
\end{pgfscope}%
\begin{pgfscope}%
\pgfpathrectangle{\pgfqpoint{5.095420in}{2.920611in}}{\pgfqpoint{0.824468in}{0.462000in}}%
\pgfusepath{clip}%
\pgfsetbuttcap%
\pgfsetmiterjoin%
\definecolor{currentfill}{rgb}{0.121569,0.466667,0.705882}%
\pgfsetfillcolor{currentfill}%
\pgfsetfillopacity{0.500000}%
\pgfsetlinewidth{1.003750pt}%
\definecolor{currentstroke}{rgb}{0.000000,0.000000,0.000000}%
\pgfsetstrokecolor{currentstroke}%
\pgfsetdash{}{0pt}%
\pgfpathmoveto{\pgfqpoint{5.732509in}{2.920611in}}%
\pgfpathlineto{\pgfqpoint{5.882412in}{2.920611in}}%
\pgfpathlineto{\pgfqpoint{5.882412in}{2.938570in}}%
\pgfpathlineto{\pgfqpoint{5.732509in}{2.938570in}}%
\pgfpathlineto{\pgfqpoint{5.732509in}{2.920611in}}%
\pgfpathclose%
\pgfusepath{stroke,fill}%
\end{pgfscope}%
\begin{pgfscope}%
\pgfsetrectcap%
\pgfsetmiterjoin%
\pgfsetlinewidth{0.803000pt}%
\definecolor{currentstroke}{rgb}{0.000000,0.000000,0.000000}%
\pgfsetstrokecolor{currentstroke}%
\pgfsetdash{}{0pt}%
\pgfpathmoveto{\pgfqpoint{5.095420in}{2.920611in}}%
\pgfpathlineto{\pgfqpoint{5.095420in}{3.382611in}}%
\pgfusepath{stroke}%
\end{pgfscope}%
\begin{pgfscope}%
\pgfsetrectcap%
\pgfsetmiterjoin%
\pgfsetlinewidth{0.803000pt}%
\definecolor{currentstroke}{rgb}{0.000000,0.000000,0.000000}%
\pgfsetstrokecolor{currentstroke}%
\pgfsetdash{}{0pt}%
\pgfpathmoveto{\pgfqpoint{5.919888in}{2.920611in}}%
\pgfpathlineto{\pgfqpoint{5.919888in}{3.382611in}}%
\pgfusepath{stroke}%
\end{pgfscope}%
\begin{pgfscope}%
\pgfsetrectcap%
\pgfsetmiterjoin%
\pgfsetlinewidth{0.803000pt}%
\definecolor{currentstroke}{rgb}{0.000000,0.000000,0.000000}%
\pgfsetstrokecolor{currentstroke}%
\pgfsetdash{}{0pt}%
\pgfpathmoveto{\pgfqpoint{5.095420in}{2.920611in}}%
\pgfpathlineto{\pgfqpoint{5.919888in}{2.920611in}}%
\pgfusepath{stroke}%
\end{pgfscope}%
\begin{pgfscope}%
\pgfsetrectcap%
\pgfsetmiterjoin%
\pgfsetlinewidth{0.803000pt}%
\definecolor{currentstroke}{rgb}{0.000000,0.000000,0.000000}%
\pgfsetstrokecolor{currentstroke}%
\pgfsetdash{}{0pt}%
\pgfpathmoveto{\pgfqpoint{5.095420in}{3.382611in}}%
\pgfpathlineto{\pgfqpoint{5.919888in}{3.382611in}}%
\pgfusepath{stroke}%
\end{pgfscope}%
\begin{pgfscope}%
\definecolor{textcolor}{rgb}{0.000000,0.000000,0.000000}%
\pgfsetstrokecolor{textcolor}%
\pgfsetfillcolor{textcolor}%
\pgftext[x=5.507654in,y=3.465944in,,base]{\color{textcolor}\rmfamily\fontsize{11.000000}{13.200000}\selectfont Sogecap}%
\end{pgfscope}%
\begin{pgfscope}%
\pgfsetbuttcap%
\pgfsetmiterjoin%
\definecolor{currentfill}{rgb}{1.000000,1.000000,1.000000}%
\pgfsetfillcolor{currentfill}%
\pgfsetlinewidth{0.000000pt}%
\definecolor{currentstroke}{rgb}{0.000000,0.000000,0.000000}%
\pgfsetstrokecolor{currentstroke}%
\pgfsetstrokeopacity{0.000000}%
\pgfsetdash{}{0pt}%
\pgfpathmoveto{\pgfqpoint{6.084781in}{2.920611in}}%
\pgfpathlineto{\pgfqpoint{6.909249in}{2.920611in}}%
\pgfpathlineto{\pgfqpoint{6.909249in}{3.382611in}}%
\pgfpathlineto{\pgfqpoint{6.084781in}{3.382611in}}%
\pgfpathlineto{\pgfqpoint{6.084781in}{2.920611in}}%
\pgfpathclose%
\pgfusepath{fill}%
\end{pgfscope}%
\begin{pgfscope}%
\pgfpathrectangle{\pgfqpoint{6.084781in}{2.920611in}}{\pgfqpoint{0.824468in}{0.462000in}}%
\pgfusepath{clip}%
\pgfsetbuttcap%
\pgfsetmiterjoin%
\definecolor{currentfill}{rgb}{0.121569,0.466667,0.705882}%
\pgfsetfillcolor{currentfill}%
\pgfsetfillopacity{0.500000}%
\pgfsetlinewidth{1.003750pt}%
\definecolor{currentstroke}{rgb}{0.000000,0.000000,0.000000}%
\pgfsetstrokecolor{currentstroke}%
\pgfsetdash{}{0pt}%
\pgfpathmoveto{\pgfqpoint{6.122257in}{2.920611in}}%
\pgfpathlineto{\pgfqpoint{6.272160in}{2.920611in}}%
\pgfpathlineto{\pgfqpoint{6.272160in}{3.360611in}}%
\pgfpathlineto{\pgfqpoint{6.122257in}{3.360611in}}%
\pgfpathlineto{\pgfqpoint{6.122257in}{2.920611in}}%
\pgfpathclose%
\pgfusepath{stroke,fill}%
\end{pgfscope}%
\begin{pgfscope}%
\pgfpathrectangle{\pgfqpoint{6.084781in}{2.920611in}}{\pgfqpoint{0.824468in}{0.462000in}}%
\pgfusepath{clip}%
\pgfsetbuttcap%
\pgfsetmiterjoin%
\definecolor{currentfill}{rgb}{0.121569,0.466667,0.705882}%
\pgfsetfillcolor{currentfill}%
\pgfsetfillopacity{0.500000}%
\pgfsetlinewidth{1.003750pt}%
\definecolor{currentstroke}{rgb}{0.000000,0.000000,0.000000}%
\pgfsetstrokecolor{currentstroke}%
\pgfsetdash{}{0pt}%
\pgfpathmoveto{\pgfqpoint{6.272160in}{2.920611in}}%
\pgfpathlineto{\pgfqpoint{6.422064in}{2.920611in}}%
\pgfpathlineto{\pgfqpoint{6.422064in}{3.013143in}}%
\pgfpathlineto{\pgfqpoint{6.272160in}{3.013143in}}%
\pgfpathlineto{\pgfqpoint{6.272160in}{2.920611in}}%
\pgfpathclose%
\pgfusepath{stroke,fill}%
\end{pgfscope}%
\begin{pgfscope}%
\pgfpathrectangle{\pgfqpoint{6.084781in}{2.920611in}}{\pgfqpoint{0.824468in}{0.462000in}}%
\pgfusepath{clip}%
\pgfsetbuttcap%
\pgfsetmiterjoin%
\definecolor{currentfill}{rgb}{0.121569,0.466667,0.705882}%
\pgfsetfillcolor{currentfill}%
\pgfsetfillopacity{0.500000}%
\pgfsetlinewidth{1.003750pt}%
\definecolor{currentstroke}{rgb}{0.000000,0.000000,0.000000}%
\pgfsetstrokecolor{currentstroke}%
\pgfsetdash{}{0pt}%
\pgfpathmoveto{\pgfqpoint{6.422064in}{2.920611in}}%
\pgfpathlineto{\pgfqpoint{6.571967in}{2.920611in}}%
\pgfpathlineto{\pgfqpoint{6.571967in}{2.962156in}}%
\pgfpathlineto{\pgfqpoint{6.422064in}{2.962156in}}%
\pgfpathlineto{\pgfqpoint{6.422064in}{2.920611in}}%
\pgfpathclose%
\pgfusepath{stroke,fill}%
\end{pgfscope}%
\begin{pgfscope}%
\pgfpathrectangle{\pgfqpoint{6.084781in}{2.920611in}}{\pgfqpoint{0.824468in}{0.462000in}}%
\pgfusepath{clip}%
\pgfsetbuttcap%
\pgfsetmiterjoin%
\definecolor{currentfill}{rgb}{0.121569,0.466667,0.705882}%
\pgfsetfillcolor{currentfill}%
\pgfsetfillopacity{0.500000}%
\pgfsetlinewidth{1.003750pt}%
\definecolor{currentstroke}{rgb}{0.000000,0.000000,0.000000}%
\pgfsetstrokecolor{currentstroke}%
\pgfsetdash{}{0pt}%
\pgfpathmoveto{\pgfqpoint{6.571967in}{2.920611in}}%
\pgfpathlineto{\pgfqpoint{6.721870in}{2.920611in}}%
\pgfpathlineto{\pgfqpoint{6.721870in}{2.928165in}}%
\pgfpathlineto{\pgfqpoint{6.571967in}{2.928165in}}%
\pgfpathlineto{\pgfqpoint{6.571967in}{2.920611in}}%
\pgfpathclose%
\pgfusepath{stroke,fill}%
\end{pgfscope}%
\begin{pgfscope}%
\pgfpathrectangle{\pgfqpoint{6.084781in}{2.920611in}}{\pgfqpoint{0.824468in}{0.462000in}}%
\pgfusepath{clip}%
\pgfsetbuttcap%
\pgfsetmiterjoin%
\definecolor{currentfill}{rgb}{0.121569,0.466667,0.705882}%
\pgfsetfillcolor{currentfill}%
\pgfsetfillopacity{0.500000}%
\pgfsetlinewidth{1.003750pt}%
\definecolor{currentstroke}{rgb}{0.000000,0.000000,0.000000}%
\pgfsetstrokecolor{currentstroke}%
\pgfsetdash{}{0pt}%
\pgfpathmoveto{\pgfqpoint{6.721870in}{2.920611in}}%
\pgfpathlineto{\pgfqpoint{6.871774in}{2.920611in}}%
\pgfpathlineto{\pgfqpoint{6.871774in}{2.928165in}}%
\pgfpathlineto{\pgfqpoint{6.721870in}{2.928165in}}%
\pgfpathlineto{\pgfqpoint{6.721870in}{2.920611in}}%
\pgfpathclose%
\pgfusepath{stroke,fill}%
\end{pgfscope}%
\begin{pgfscope}%
\pgfsetrectcap%
\pgfsetmiterjoin%
\pgfsetlinewidth{0.803000pt}%
\definecolor{currentstroke}{rgb}{0.000000,0.000000,0.000000}%
\pgfsetstrokecolor{currentstroke}%
\pgfsetdash{}{0pt}%
\pgfpathmoveto{\pgfqpoint{6.084781in}{2.920611in}}%
\pgfpathlineto{\pgfqpoint{6.084781in}{3.382611in}}%
\pgfusepath{stroke}%
\end{pgfscope}%
\begin{pgfscope}%
\pgfsetrectcap%
\pgfsetmiterjoin%
\pgfsetlinewidth{0.803000pt}%
\definecolor{currentstroke}{rgb}{0.000000,0.000000,0.000000}%
\pgfsetstrokecolor{currentstroke}%
\pgfsetdash{}{0pt}%
\pgfpathmoveto{\pgfqpoint{6.909249in}{2.920611in}}%
\pgfpathlineto{\pgfqpoint{6.909249in}{3.382611in}}%
\pgfusepath{stroke}%
\end{pgfscope}%
\begin{pgfscope}%
\pgfsetrectcap%
\pgfsetmiterjoin%
\pgfsetlinewidth{0.803000pt}%
\definecolor{currentstroke}{rgb}{0.000000,0.000000,0.000000}%
\pgfsetstrokecolor{currentstroke}%
\pgfsetdash{}{0pt}%
\pgfpathmoveto{\pgfqpoint{6.084781in}{2.920611in}}%
\pgfpathlineto{\pgfqpoint{6.909249in}{2.920611in}}%
\pgfusepath{stroke}%
\end{pgfscope}%
\begin{pgfscope}%
\pgfsetrectcap%
\pgfsetmiterjoin%
\pgfsetlinewidth{0.803000pt}%
\definecolor{currentstroke}{rgb}{0.000000,0.000000,0.000000}%
\pgfsetstrokecolor{currentstroke}%
\pgfsetdash{}{0pt}%
\pgfpathmoveto{\pgfqpoint{6.084781in}{3.382611in}}%
\pgfpathlineto{\pgfqpoint{6.909249in}{3.382611in}}%
\pgfusepath{stroke}%
\end{pgfscope}%
\begin{pgfscope}%
\definecolor{textcolor}{rgb}{0.000000,0.000000,0.000000}%
\pgfsetstrokecolor{textcolor}%
\pgfsetfillcolor{textcolor}%
\pgftext[x=6.497015in,y=3.465944in,,base]{\color{textcolor}\rmfamily\fontsize{11.000000}{13.200000}\selectfont Harmon...}%
\end{pgfscope}%
\begin{pgfscope}%
\pgfsetbuttcap%
\pgfsetmiterjoin%
\definecolor{currentfill}{rgb}{1.000000,1.000000,1.000000}%
\pgfsetfillcolor{currentfill}%
\pgfsetlinewidth{0.000000pt}%
\definecolor{currentstroke}{rgb}{0.000000,0.000000,0.000000}%
\pgfsetstrokecolor{currentstroke}%
\pgfsetstrokeopacity{0.000000}%
\pgfsetdash{}{0pt}%
\pgfpathmoveto{\pgfqpoint{7.074143in}{2.920611in}}%
\pgfpathlineto{\pgfqpoint{7.898611in}{2.920611in}}%
\pgfpathlineto{\pgfqpoint{7.898611in}{3.382611in}}%
\pgfpathlineto{\pgfqpoint{7.074143in}{3.382611in}}%
\pgfpathlineto{\pgfqpoint{7.074143in}{2.920611in}}%
\pgfpathclose%
\pgfusepath{fill}%
\end{pgfscope}%
\begin{pgfscope}%
\pgfpathrectangle{\pgfqpoint{7.074143in}{2.920611in}}{\pgfqpoint{0.824468in}{0.462000in}}%
\pgfusepath{clip}%
\pgfsetbuttcap%
\pgfsetmiterjoin%
\definecolor{currentfill}{rgb}{0.121569,0.466667,0.705882}%
\pgfsetfillcolor{currentfill}%
\pgfsetfillopacity{0.500000}%
\pgfsetlinewidth{1.003750pt}%
\definecolor{currentstroke}{rgb}{0.000000,0.000000,0.000000}%
\pgfsetstrokecolor{currentstroke}%
\pgfsetdash{}{0pt}%
\pgfpathmoveto{\pgfqpoint{7.111619in}{2.920611in}}%
\pgfpathlineto{\pgfqpoint{7.261522in}{2.920611in}}%
\pgfpathlineto{\pgfqpoint{7.261522in}{3.360611in}}%
\pgfpathlineto{\pgfqpoint{7.111619in}{3.360611in}}%
\pgfpathlineto{\pgfqpoint{7.111619in}{2.920611in}}%
\pgfpathclose%
\pgfusepath{stroke,fill}%
\end{pgfscope}%
\begin{pgfscope}%
\pgfpathrectangle{\pgfqpoint{7.074143in}{2.920611in}}{\pgfqpoint{0.824468in}{0.462000in}}%
\pgfusepath{clip}%
\pgfsetbuttcap%
\pgfsetmiterjoin%
\definecolor{currentfill}{rgb}{0.121569,0.466667,0.705882}%
\pgfsetfillcolor{currentfill}%
\pgfsetfillopacity{0.500000}%
\pgfsetlinewidth{1.003750pt}%
\definecolor{currentstroke}{rgb}{0.000000,0.000000,0.000000}%
\pgfsetstrokecolor{currentstroke}%
\pgfsetdash{}{0pt}%
\pgfpathmoveto{\pgfqpoint{7.261522in}{2.920611in}}%
\pgfpathlineto{\pgfqpoint{7.411425in}{2.920611in}}%
\pgfpathlineto{\pgfqpoint{7.411425in}{3.044714in}}%
\pgfpathlineto{\pgfqpoint{7.261522in}{3.044714in}}%
\pgfpathlineto{\pgfqpoint{7.261522in}{2.920611in}}%
\pgfpathclose%
\pgfusepath{stroke,fill}%
\end{pgfscope}%
\begin{pgfscope}%
\pgfpathrectangle{\pgfqpoint{7.074143in}{2.920611in}}{\pgfqpoint{0.824468in}{0.462000in}}%
\pgfusepath{clip}%
\pgfsetbuttcap%
\pgfsetmiterjoin%
\definecolor{currentfill}{rgb}{0.121569,0.466667,0.705882}%
\pgfsetfillcolor{currentfill}%
\pgfsetfillopacity{0.500000}%
\pgfsetlinewidth{1.003750pt}%
\definecolor{currentstroke}{rgb}{0.000000,0.000000,0.000000}%
\pgfsetstrokecolor{currentstroke}%
\pgfsetdash{}{0pt}%
\pgfpathmoveto{\pgfqpoint{7.411425in}{2.920611in}}%
\pgfpathlineto{\pgfqpoint{7.561329in}{2.920611in}}%
\pgfpathlineto{\pgfqpoint{7.561329in}{2.977021in}}%
\pgfpathlineto{\pgfqpoint{7.411425in}{2.977021in}}%
\pgfpathlineto{\pgfqpoint{7.411425in}{2.920611in}}%
\pgfpathclose%
\pgfusepath{stroke,fill}%
\end{pgfscope}%
\begin{pgfscope}%
\pgfpathrectangle{\pgfqpoint{7.074143in}{2.920611in}}{\pgfqpoint{0.824468in}{0.462000in}}%
\pgfusepath{clip}%
\pgfsetbuttcap%
\pgfsetmiterjoin%
\definecolor{currentfill}{rgb}{0.121569,0.466667,0.705882}%
\pgfsetfillcolor{currentfill}%
\pgfsetfillopacity{0.500000}%
\pgfsetlinewidth{1.003750pt}%
\definecolor{currentstroke}{rgb}{0.000000,0.000000,0.000000}%
\pgfsetstrokecolor{currentstroke}%
\pgfsetdash{}{0pt}%
\pgfpathmoveto{\pgfqpoint{7.561329in}{2.920611in}}%
\pgfpathlineto{\pgfqpoint{7.711232in}{2.920611in}}%
\pgfpathlineto{\pgfqpoint{7.711232in}{2.982662in}}%
\pgfpathlineto{\pgfqpoint{7.561329in}{2.982662in}}%
\pgfpathlineto{\pgfqpoint{7.561329in}{2.920611in}}%
\pgfpathclose%
\pgfusepath{stroke,fill}%
\end{pgfscope}%
\begin{pgfscope}%
\pgfpathrectangle{\pgfqpoint{7.074143in}{2.920611in}}{\pgfqpoint{0.824468in}{0.462000in}}%
\pgfusepath{clip}%
\pgfsetbuttcap%
\pgfsetmiterjoin%
\definecolor{currentfill}{rgb}{0.121569,0.466667,0.705882}%
\pgfsetfillcolor{currentfill}%
\pgfsetfillopacity{0.500000}%
\pgfsetlinewidth{1.003750pt}%
\definecolor{currentstroke}{rgb}{0.000000,0.000000,0.000000}%
\pgfsetstrokecolor{currentstroke}%
\pgfsetdash{}{0pt}%
\pgfpathmoveto{\pgfqpoint{7.711232in}{2.920611in}}%
\pgfpathlineto{\pgfqpoint{7.861135in}{2.920611in}}%
\pgfpathlineto{\pgfqpoint{7.861135in}{2.960098in}}%
\pgfpathlineto{\pgfqpoint{7.711232in}{2.960098in}}%
\pgfpathlineto{\pgfqpoint{7.711232in}{2.920611in}}%
\pgfpathclose%
\pgfusepath{stroke,fill}%
\end{pgfscope}%
\begin{pgfscope}%
\pgfsetrectcap%
\pgfsetmiterjoin%
\pgfsetlinewidth{0.803000pt}%
\definecolor{currentstroke}{rgb}{0.000000,0.000000,0.000000}%
\pgfsetstrokecolor{currentstroke}%
\pgfsetdash{}{0pt}%
\pgfpathmoveto{\pgfqpoint{7.074143in}{2.920611in}}%
\pgfpathlineto{\pgfqpoint{7.074143in}{3.382611in}}%
\pgfusepath{stroke}%
\end{pgfscope}%
\begin{pgfscope}%
\pgfsetrectcap%
\pgfsetmiterjoin%
\pgfsetlinewidth{0.803000pt}%
\definecolor{currentstroke}{rgb}{0.000000,0.000000,0.000000}%
\pgfsetstrokecolor{currentstroke}%
\pgfsetdash{}{0pt}%
\pgfpathmoveto{\pgfqpoint{7.898611in}{2.920611in}}%
\pgfpathlineto{\pgfqpoint{7.898611in}{3.382611in}}%
\pgfusepath{stroke}%
\end{pgfscope}%
\begin{pgfscope}%
\pgfsetrectcap%
\pgfsetmiterjoin%
\pgfsetlinewidth{0.803000pt}%
\definecolor{currentstroke}{rgb}{0.000000,0.000000,0.000000}%
\pgfsetstrokecolor{currentstroke}%
\pgfsetdash{}{0pt}%
\pgfpathmoveto{\pgfqpoint{7.074143in}{2.920611in}}%
\pgfpathlineto{\pgfqpoint{7.898611in}{2.920611in}}%
\pgfusepath{stroke}%
\end{pgfscope}%
\begin{pgfscope}%
\pgfsetrectcap%
\pgfsetmiterjoin%
\pgfsetlinewidth{0.803000pt}%
\definecolor{currentstroke}{rgb}{0.000000,0.000000,0.000000}%
\pgfsetstrokecolor{currentstroke}%
\pgfsetdash{}{0pt}%
\pgfpathmoveto{\pgfqpoint{7.074143in}{3.382611in}}%
\pgfpathlineto{\pgfqpoint{7.898611in}{3.382611in}}%
\pgfusepath{stroke}%
\end{pgfscope}%
\begin{pgfscope}%
\definecolor{textcolor}{rgb}{0.000000,0.000000,0.000000}%
\pgfsetstrokecolor{textcolor}%
\pgfsetfillcolor{textcolor}%
\pgftext[x=7.486377in,y=3.465944in,,base]{\color{textcolor}\rmfamily\fontsize{11.000000}{13.200000}\selectfont Mutuel...}%
\end{pgfscope}%
\begin{pgfscope}%
\pgfsetbuttcap%
\pgfsetmiterjoin%
\definecolor{currentfill}{rgb}{1.000000,1.000000,1.000000}%
\pgfsetfillcolor{currentfill}%
\pgfsetlinewidth{0.000000pt}%
\definecolor{currentstroke}{rgb}{0.000000,0.000000,0.000000}%
\pgfsetstrokecolor{currentstroke}%
\pgfsetstrokeopacity{0.000000}%
\pgfsetdash{}{0pt}%
\pgfpathmoveto{\pgfqpoint{0.148611in}{2.227611in}}%
\pgfpathlineto{\pgfqpoint{0.973079in}{2.227611in}}%
\pgfpathlineto{\pgfqpoint{0.973079in}{2.689611in}}%
\pgfpathlineto{\pgfqpoint{0.148611in}{2.689611in}}%
\pgfpathlineto{\pgfqpoint{0.148611in}{2.227611in}}%
\pgfpathclose%
\pgfusepath{fill}%
\end{pgfscope}%
\begin{pgfscope}%
\pgfpathrectangle{\pgfqpoint{0.148611in}{2.227611in}}{\pgfqpoint{0.824468in}{0.462000in}}%
\pgfusepath{clip}%
\pgfsetbuttcap%
\pgfsetmiterjoin%
\definecolor{currentfill}{rgb}{0.121569,0.466667,0.705882}%
\pgfsetfillcolor{currentfill}%
\pgfsetfillopacity{0.500000}%
\pgfsetlinewidth{1.003750pt}%
\definecolor{currentstroke}{rgb}{0.000000,0.000000,0.000000}%
\pgfsetstrokecolor{currentstroke}%
\pgfsetdash{}{0pt}%
\pgfpathmoveto{\pgfqpoint{0.186087in}{2.227611in}}%
\pgfpathlineto{\pgfqpoint{0.335990in}{2.227611in}}%
\pgfpathlineto{\pgfqpoint{0.335990in}{2.667611in}}%
\pgfpathlineto{\pgfqpoint{0.186087in}{2.667611in}}%
\pgfpathlineto{\pgfqpoint{0.186087in}{2.227611in}}%
\pgfpathclose%
\pgfusepath{stroke,fill}%
\end{pgfscope}%
\begin{pgfscope}%
\pgfpathrectangle{\pgfqpoint{0.148611in}{2.227611in}}{\pgfqpoint{0.824468in}{0.462000in}}%
\pgfusepath{clip}%
\pgfsetbuttcap%
\pgfsetmiterjoin%
\definecolor{currentfill}{rgb}{0.121569,0.466667,0.705882}%
\pgfsetfillcolor{currentfill}%
\pgfsetfillopacity{0.500000}%
\pgfsetlinewidth{1.003750pt}%
\definecolor{currentstroke}{rgb}{0.000000,0.000000,0.000000}%
\pgfsetstrokecolor{currentstroke}%
\pgfsetdash{}{0pt}%
\pgfpathmoveto{\pgfqpoint{0.335990in}{2.227611in}}%
\pgfpathlineto{\pgfqpoint{0.485894in}{2.227611in}}%
\pgfpathlineto{\pgfqpoint{0.485894in}{2.487658in}}%
\pgfpathlineto{\pgfqpoint{0.335990in}{2.487658in}}%
\pgfpathlineto{\pgfqpoint{0.335990in}{2.227611in}}%
\pgfpathclose%
\pgfusepath{stroke,fill}%
\end{pgfscope}%
\begin{pgfscope}%
\pgfpathrectangle{\pgfqpoint{0.148611in}{2.227611in}}{\pgfqpoint{0.824468in}{0.462000in}}%
\pgfusepath{clip}%
\pgfsetbuttcap%
\pgfsetmiterjoin%
\definecolor{currentfill}{rgb}{0.121569,0.466667,0.705882}%
\pgfsetfillcolor{currentfill}%
\pgfsetfillopacity{0.500000}%
\pgfsetlinewidth{1.003750pt}%
\definecolor{currentstroke}{rgb}{0.000000,0.000000,0.000000}%
\pgfsetstrokecolor{currentstroke}%
\pgfsetdash{}{0pt}%
\pgfpathmoveto{\pgfqpoint{0.485894in}{2.227611in}}%
\pgfpathlineto{\pgfqpoint{0.635797in}{2.227611in}}%
\pgfpathlineto{\pgfqpoint{0.635797in}{2.311866in}}%
\pgfpathlineto{\pgfqpoint{0.485894in}{2.311866in}}%
\pgfpathlineto{\pgfqpoint{0.485894in}{2.227611in}}%
\pgfpathclose%
\pgfusepath{stroke,fill}%
\end{pgfscope}%
\begin{pgfscope}%
\pgfpathrectangle{\pgfqpoint{0.148611in}{2.227611in}}{\pgfqpoint{0.824468in}{0.462000in}}%
\pgfusepath{clip}%
\pgfsetbuttcap%
\pgfsetmiterjoin%
\definecolor{currentfill}{rgb}{0.121569,0.466667,0.705882}%
\pgfsetfillcolor{currentfill}%
\pgfsetfillopacity{0.500000}%
\pgfsetlinewidth{1.003750pt}%
\definecolor{currentstroke}{rgb}{0.000000,0.000000,0.000000}%
\pgfsetstrokecolor{currentstroke}%
\pgfsetdash{}{0pt}%
\pgfpathmoveto{\pgfqpoint{0.635797in}{2.227611in}}%
\pgfpathlineto{\pgfqpoint{0.785700in}{2.227611in}}%
\pgfpathlineto{\pgfqpoint{0.785700in}{2.272339in}}%
\pgfpathlineto{\pgfqpoint{0.635797in}{2.272339in}}%
\pgfpathlineto{\pgfqpoint{0.635797in}{2.227611in}}%
\pgfpathclose%
\pgfusepath{stroke,fill}%
\end{pgfscope}%
\begin{pgfscope}%
\pgfpathrectangle{\pgfqpoint{0.148611in}{2.227611in}}{\pgfqpoint{0.824468in}{0.462000in}}%
\pgfusepath{clip}%
\pgfsetbuttcap%
\pgfsetmiterjoin%
\definecolor{currentfill}{rgb}{0.121569,0.466667,0.705882}%
\pgfsetfillcolor{currentfill}%
\pgfsetfillopacity{0.500000}%
\pgfsetlinewidth{1.003750pt}%
\definecolor{currentstroke}{rgb}{0.000000,0.000000,0.000000}%
\pgfsetstrokecolor{currentstroke}%
\pgfsetdash{}{0pt}%
\pgfpathmoveto{\pgfqpoint{0.785700in}{2.227611in}}%
\pgfpathlineto{\pgfqpoint{0.935603in}{2.227611in}}%
\pgfpathlineto{\pgfqpoint{0.935603in}{2.260897in}}%
\pgfpathlineto{\pgfqpoint{0.785700in}{2.260897in}}%
\pgfpathlineto{\pgfqpoint{0.785700in}{2.227611in}}%
\pgfpathclose%
\pgfusepath{stroke,fill}%
\end{pgfscope}%
\begin{pgfscope}%
\pgfsetrectcap%
\pgfsetmiterjoin%
\pgfsetlinewidth{0.803000pt}%
\definecolor{currentstroke}{rgb}{0.000000,0.000000,0.000000}%
\pgfsetstrokecolor{currentstroke}%
\pgfsetdash{}{0pt}%
\pgfpathmoveto{\pgfqpoint{0.148611in}{2.227611in}}%
\pgfpathlineto{\pgfqpoint{0.148611in}{2.689611in}}%
\pgfusepath{stroke}%
\end{pgfscope}%
\begin{pgfscope}%
\pgfsetrectcap%
\pgfsetmiterjoin%
\pgfsetlinewidth{0.803000pt}%
\definecolor{currentstroke}{rgb}{0.000000,0.000000,0.000000}%
\pgfsetstrokecolor{currentstroke}%
\pgfsetdash{}{0pt}%
\pgfpathmoveto{\pgfqpoint{0.973079in}{2.227611in}}%
\pgfpathlineto{\pgfqpoint{0.973079in}{2.689611in}}%
\pgfusepath{stroke}%
\end{pgfscope}%
\begin{pgfscope}%
\pgfsetrectcap%
\pgfsetmiterjoin%
\pgfsetlinewidth{0.803000pt}%
\definecolor{currentstroke}{rgb}{0.000000,0.000000,0.000000}%
\pgfsetstrokecolor{currentstroke}%
\pgfsetdash{}{0pt}%
\pgfpathmoveto{\pgfqpoint{0.148611in}{2.227611in}}%
\pgfpathlineto{\pgfqpoint{0.973079in}{2.227611in}}%
\pgfusepath{stroke}%
\end{pgfscope}%
\begin{pgfscope}%
\pgfsetrectcap%
\pgfsetmiterjoin%
\pgfsetlinewidth{0.803000pt}%
\definecolor{currentstroke}{rgb}{0.000000,0.000000,0.000000}%
\pgfsetstrokecolor{currentstroke}%
\pgfsetdash{}{0pt}%
\pgfpathmoveto{\pgfqpoint{0.148611in}{2.689611in}}%
\pgfpathlineto{\pgfqpoint{0.973079in}{2.689611in}}%
\pgfusepath{stroke}%
\end{pgfscope}%
\begin{pgfscope}%
\definecolor{textcolor}{rgb}{0.000000,0.000000,0.000000}%
\pgfsetstrokecolor{textcolor}%
\pgfsetfillcolor{textcolor}%
\pgftext[x=0.560845in,y=2.772944in,,base]{\color{textcolor}\rmfamily\fontsize{11.000000}{13.200000}\selectfont MACIF}%
\end{pgfscope}%
\begin{pgfscope}%
\pgfsetbuttcap%
\pgfsetmiterjoin%
\definecolor{currentfill}{rgb}{1.000000,1.000000,1.000000}%
\pgfsetfillcolor{currentfill}%
\pgfsetlinewidth{0.000000pt}%
\definecolor{currentstroke}{rgb}{0.000000,0.000000,0.000000}%
\pgfsetstrokecolor{currentstroke}%
\pgfsetstrokeopacity{0.000000}%
\pgfsetdash{}{0pt}%
\pgfpathmoveto{\pgfqpoint{1.137973in}{2.227611in}}%
\pgfpathlineto{\pgfqpoint{1.962441in}{2.227611in}}%
\pgfpathlineto{\pgfqpoint{1.962441in}{2.689611in}}%
\pgfpathlineto{\pgfqpoint{1.137973in}{2.689611in}}%
\pgfpathlineto{\pgfqpoint{1.137973in}{2.227611in}}%
\pgfpathclose%
\pgfusepath{fill}%
\end{pgfscope}%
\begin{pgfscope}%
\pgfpathrectangle{\pgfqpoint{1.137973in}{2.227611in}}{\pgfqpoint{0.824468in}{0.462000in}}%
\pgfusepath{clip}%
\pgfsetbuttcap%
\pgfsetmiterjoin%
\definecolor{currentfill}{rgb}{0.121569,0.466667,0.705882}%
\pgfsetfillcolor{currentfill}%
\pgfsetfillopacity{0.500000}%
\pgfsetlinewidth{1.003750pt}%
\definecolor{currentstroke}{rgb}{0.000000,0.000000,0.000000}%
\pgfsetstrokecolor{currentstroke}%
\pgfsetdash{}{0pt}%
\pgfpathmoveto{\pgfqpoint{1.175449in}{2.227611in}}%
\pgfpathlineto{\pgfqpoint{1.325352in}{2.227611in}}%
\pgfpathlineto{\pgfqpoint{1.325352in}{2.641286in}}%
\pgfpathlineto{\pgfqpoint{1.175449in}{2.641286in}}%
\pgfpathlineto{\pgfqpoint{1.175449in}{2.227611in}}%
\pgfpathclose%
\pgfusepath{stroke,fill}%
\end{pgfscope}%
\begin{pgfscope}%
\pgfpathrectangle{\pgfqpoint{1.137973in}{2.227611in}}{\pgfqpoint{0.824468in}{0.462000in}}%
\pgfusepath{clip}%
\pgfsetbuttcap%
\pgfsetmiterjoin%
\definecolor{currentfill}{rgb}{0.121569,0.466667,0.705882}%
\pgfsetfillcolor{currentfill}%
\pgfsetfillopacity{0.500000}%
\pgfsetlinewidth{1.003750pt}%
\definecolor{currentstroke}{rgb}{0.000000,0.000000,0.000000}%
\pgfsetstrokecolor{currentstroke}%
\pgfsetdash{}{0pt}%
\pgfpathmoveto{\pgfqpoint{1.325352in}{2.227611in}}%
\pgfpathlineto{\pgfqpoint{1.475255in}{2.227611in}}%
\pgfpathlineto{\pgfqpoint{1.475255in}{2.667611in}}%
\pgfpathlineto{\pgfqpoint{1.325352in}{2.667611in}}%
\pgfpathlineto{\pgfqpoint{1.325352in}{2.227611in}}%
\pgfpathclose%
\pgfusepath{stroke,fill}%
\end{pgfscope}%
\begin{pgfscope}%
\pgfpathrectangle{\pgfqpoint{1.137973in}{2.227611in}}{\pgfqpoint{0.824468in}{0.462000in}}%
\pgfusepath{clip}%
\pgfsetbuttcap%
\pgfsetmiterjoin%
\definecolor{currentfill}{rgb}{0.121569,0.466667,0.705882}%
\pgfsetfillcolor{currentfill}%
\pgfsetfillopacity{0.500000}%
\pgfsetlinewidth{1.003750pt}%
\definecolor{currentstroke}{rgb}{0.000000,0.000000,0.000000}%
\pgfsetstrokecolor{currentstroke}%
\pgfsetdash{}{0pt}%
\pgfpathmoveto{\pgfqpoint{1.475255in}{2.227611in}}%
\pgfpathlineto{\pgfqpoint{1.625158in}{2.227611in}}%
\pgfpathlineto{\pgfqpoint{1.625158in}{2.351714in}}%
\pgfpathlineto{\pgfqpoint{1.475255in}{2.351714in}}%
\pgfpathlineto{\pgfqpoint{1.475255in}{2.227611in}}%
\pgfpathclose%
\pgfusepath{stroke,fill}%
\end{pgfscope}%
\begin{pgfscope}%
\pgfpathrectangle{\pgfqpoint{1.137973in}{2.227611in}}{\pgfqpoint{0.824468in}{0.462000in}}%
\pgfusepath{clip}%
\pgfsetbuttcap%
\pgfsetmiterjoin%
\definecolor{currentfill}{rgb}{0.121569,0.466667,0.705882}%
\pgfsetfillcolor{currentfill}%
\pgfsetfillopacity{0.500000}%
\pgfsetlinewidth{1.003750pt}%
\definecolor{currentstroke}{rgb}{0.000000,0.000000,0.000000}%
\pgfsetstrokecolor{currentstroke}%
\pgfsetdash{}{0pt}%
\pgfpathmoveto{\pgfqpoint{1.625158in}{2.227611in}}%
\pgfpathlineto{\pgfqpoint{1.775062in}{2.227611in}}%
\pgfpathlineto{\pgfqpoint{1.775062in}{2.299064in}}%
\pgfpathlineto{\pgfqpoint{1.625158in}{2.299064in}}%
\pgfpathlineto{\pgfqpoint{1.625158in}{2.227611in}}%
\pgfpathclose%
\pgfusepath{stroke,fill}%
\end{pgfscope}%
\begin{pgfscope}%
\pgfpathrectangle{\pgfqpoint{1.137973in}{2.227611in}}{\pgfqpoint{0.824468in}{0.462000in}}%
\pgfusepath{clip}%
\pgfsetbuttcap%
\pgfsetmiterjoin%
\definecolor{currentfill}{rgb}{0.121569,0.466667,0.705882}%
\pgfsetfillcolor{currentfill}%
\pgfsetfillopacity{0.500000}%
\pgfsetlinewidth{1.003750pt}%
\definecolor{currentstroke}{rgb}{0.000000,0.000000,0.000000}%
\pgfsetstrokecolor{currentstroke}%
\pgfsetdash{}{0pt}%
\pgfpathmoveto{\pgfqpoint{1.775062in}{2.227611in}}%
\pgfpathlineto{\pgfqpoint{1.924965in}{2.227611in}}%
\pgfpathlineto{\pgfqpoint{1.924965in}{2.265218in}}%
\pgfpathlineto{\pgfqpoint{1.775062in}{2.265218in}}%
\pgfpathlineto{\pgfqpoint{1.775062in}{2.227611in}}%
\pgfpathclose%
\pgfusepath{stroke,fill}%
\end{pgfscope}%
\begin{pgfscope}%
\pgfsetrectcap%
\pgfsetmiterjoin%
\pgfsetlinewidth{0.803000pt}%
\definecolor{currentstroke}{rgb}{0.000000,0.000000,0.000000}%
\pgfsetstrokecolor{currentstroke}%
\pgfsetdash{}{0pt}%
\pgfpathmoveto{\pgfqpoint{1.137973in}{2.227611in}}%
\pgfpathlineto{\pgfqpoint{1.137973in}{2.689611in}}%
\pgfusepath{stroke}%
\end{pgfscope}%
\begin{pgfscope}%
\pgfsetrectcap%
\pgfsetmiterjoin%
\pgfsetlinewidth{0.803000pt}%
\definecolor{currentstroke}{rgb}{0.000000,0.000000,0.000000}%
\pgfsetstrokecolor{currentstroke}%
\pgfsetdash{}{0pt}%
\pgfpathmoveto{\pgfqpoint{1.962441in}{2.227611in}}%
\pgfpathlineto{\pgfqpoint{1.962441in}{2.689611in}}%
\pgfusepath{stroke}%
\end{pgfscope}%
\begin{pgfscope}%
\pgfsetrectcap%
\pgfsetmiterjoin%
\pgfsetlinewidth{0.803000pt}%
\definecolor{currentstroke}{rgb}{0.000000,0.000000,0.000000}%
\pgfsetstrokecolor{currentstroke}%
\pgfsetdash{}{0pt}%
\pgfpathmoveto{\pgfqpoint{1.137973in}{2.227611in}}%
\pgfpathlineto{\pgfqpoint{1.962441in}{2.227611in}}%
\pgfusepath{stroke}%
\end{pgfscope}%
\begin{pgfscope}%
\pgfsetrectcap%
\pgfsetmiterjoin%
\pgfsetlinewidth{0.803000pt}%
\definecolor{currentstroke}{rgb}{0.000000,0.000000,0.000000}%
\pgfsetstrokecolor{currentstroke}%
\pgfsetdash{}{0pt}%
\pgfpathmoveto{\pgfqpoint{1.137973in}{2.689611in}}%
\pgfpathlineto{\pgfqpoint{1.962441in}{2.689611in}}%
\pgfusepath{stroke}%
\end{pgfscope}%
\begin{pgfscope}%
\definecolor{textcolor}{rgb}{0.000000,0.000000,0.000000}%
\pgfsetstrokecolor{textcolor}%
\pgfsetfillcolor{textcolor}%
\pgftext[x=1.550207in,y=2.772944in,,base]{\color{textcolor}\rmfamily\fontsize{11.000000}{13.200000}\selectfont Eurofil}%
\end{pgfscope}%
\begin{pgfscope}%
\pgfsetbuttcap%
\pgfsetmiterjoin%
\definecolor{currentfill}{rgb}{1.000000,1.000000,1.000000}%
\pgfsetfillcolor{currentfill}%
\pgfsetlinewidth{0.000000pt}%
\definecolor{currentstroke}{rgb}{0.000000,0.000000,0.000000}%
\pgfsetstrokecolor{currentstroke}%
\pgfsetstrokeopacity{0.000000}%
\pgfsetdash{}{0pt}%
\pgfpathmoveto{\pgfqpoint{2.127335in}{2.227611in}}%
\pgfpathlineto{\pgfqpoint{2.951803in}{2.227611in}}%
\pgfpathlineto{\pgfqpoint{2.951803in}{2.689611in}}%
\pgfpathlineto{\pgfqpoint{2.127335in}{2.689611in}}%
\pgfpathlineto{\pgfqpoint{2.127335in}{2.227611in}}%
\pgfpathclose%
\pgfusepath{fill}%
\end{pgfscope}%
\begin{pgfscope}%
\pgfpathrectangle{\pgfqpoint{2.127335in}{2.227611in}}{\pgfqpoint{0.824468in}{0.462000in}}%
\pgfusepath{clip}%
\pgfsetbuttcap%
\pgfsetmiterjoin%
\definecolor{currentfill}{rgb}{0.121569,0.466667,0.705882}%
\pgfsetfillcolor{currentfill}%
\pgfsetfillopacity{0.500000}%
\pgfsetlinewidth{1.003750pt}%
\definecolor{currentstroke}{rgb}{0.000000,0.000000,0.000000}%
\pgfsetstrokecolor{currentstroke}%
\pgfsetdash{}{0pt}%
\pgfpathmoveto{\pgfqpoint{2.164810in}{2.227611in}}%
\pgfpathlineto{\pgfqpoint{2.314714in}{2.227611in}}%
\pgfpathlineto{\pgfqpoint{2.314714in}{2.667611in}}%
\pgfpathlineto{\pgfqpoint{2.164810in}{2.667611in}}%
\pgfpathlineto{\pgfqpoint{2.164810in}{2.227611in}}%
\pgfpathclose%
\pgfusepath{stroke,fill}%
\end{pgfscope}%
\begin{pgfscope}%
\pgfpathrectangle{\pgfqpoint{2.127335in}{2.227611in}}{\pgfqpoint{0.824468in}{0.462000in}}%
\pgfusepath{clip}%
\pgfsetbuttcap%
\pgfsetmiterjoin%
\definecolor{currentfill}{rgb}{0.121569,0.466667,0.705882}%
\pgfsetfillcolor{currentfill}%
\pgfsetfillopacity{0.500000}%
\pgfsetlinewidth{1.003750pt}%
\definecolor{currentstroke}{rgb}{0.000000,0.000000,0.000000}%
\pgfsetstrokecolor{currentstroke}%
\pgfsetdash{}{0pt}%
\pgfpathmoveto{\pgfqpoint{2.314714in}{2.227611in}}%
\pgfpathlineto{\pgfqpoint{2.464617in}{2.227611in}}%
\pgfpathlineto{\pgfqpoint{2.464617in}{2.419163in}}%
\pgfpathlineto{\pgfqpoint{2.314714in}{2.419163in}}%
\pgfpathlineto{\pgfqpoint{2.314714in}{2.227611in}}%
\pgfpathclose%
\pgfusepath{stroke,fill}%
\end{pgfscope}%
\begin{pgfscope}%
\pgfpathrectangle{\pgfqpoint{2.127335in}{2.227611in}}{\pgfqpoint{0.824468in}{0.462000in}}%
\pgfusepath{clip}%
\pgfsetbuttcap%
\pgfsetmiterjoin%
\definecolor{currentfill}{rgb}{0.121569,0.466667,0.705882}%
\pgfsetfillcolor{currentfill}%
\pgfsetfillopacity{0.500000}%
\pgfsetlinewidth{1.003750pt}%
\definecolor{currentstroke}{rgb}{0.000000,0.000000,0.000000}%
\pgfsetstrokecolor{currentstroke}%
\pgfsetdash{}{0pt}%
\pgfpathmoveto{\pgfqpoint{2.464617in}{2.227611in}}%
\pgfpathlineto{\pgfqpoint{2.614520in}{2.227611in}}%
\pgfpathlineto{\pgfqpoint{2.614520in}{2.307266in}}%
\pgfpathlineto{\pgfqpoint{2.464617in}{2.307266in}}%
\pgfpathlineto{\pgfqpoint{2.464617in}{2.227611in}}%
\pgfpathclose%
\pgfusepath{stroke,fill}%
\end{pgfscope}%
\begin{pgfscope}%
\pgfpathrectangle{\pgfqpoint{2.127335in}{2.227611in}}{\pgfqpoint{0.824468in}{0.462000in}}%
\pgfusepath{clip}%
\pgfsetbuttcap%
\pgfsetmiterjoin%
\definecolor{currentfill}{rgb}{0.121569,0.466667,0.705882}%
\pgfsetfillcolor{currentfill}%
\pgfsetfillopacity{0.500000}%
\pgfsetlinewidth{1.003750pt}%
\definecolor{currentstroke}{rgb}{0.000000,0.000000,0.000000}%
\pgfsetstrokecolor{currentstroke}%
\pgfsetdash{}{0pt}%
\pgfpathmoveto{\pgfqpoint{2.614520in}{2.227611in}}%
\pgfpathlineto{\pgfqpoint{2.764423in}{2.227611in}}%
\pgfpathlineto{\pgfqpoint{2.764423in}{2.246577in}}%
\pgfpathlineto{\pgfqpoint{2.614520in}{2.246577in}}%
\pgfpathlineto{\pgfqpoint{2.614520in}{2.227611in}}%
\pgfpathclose%
\pgfusepath{stroke,fill}%
\end{pgfscope}%
\begin{pgfscope}%
\pgfpathrectangle{\pgfqpoint{2.127335in}{2.227611in}}{\pgfqpoint{0.824468in}{0.462000in}}%
\pgfusepath{clip}%
\pgfsetbuttcap%
\pgfsetmiterjoin%
\definecolor{currentfill}{rgb}{0.121569,0.466667,0.705882}%
\pgfsetfillcolor{currentfill}%
\pgfsetfillopacity{0.500000}%
\pgfsetlinewidth{1.003750pt}%
\definecolor{currentstroke}{rgb}{0.000000,0.000000,0.000000}%
\pgfsetstrokecolor{currentstroke}%
\pgfsetdash{}{0pt}%
\pgfpathmoveto{\pgfqpoint{2.764423in}{2.227611in}}%
\pgfpathlineto{\pgfqpoint{2.914327in}{2.227611in}}%
\pgfpathlineto{\pgfqpoint{2.914327in}{2.261749in}}%
\pgfpathlineto{\pgfqpoint{2.764423in}{2.261749in}}%
\pgfpathlineto{\pgfqpoint{2.764423in}{2.227611in}}%
\pgfpathclose%
\pgfusepath{stroke,fill}%
\end{pgfscope}%
\begin{pgfscope}%
\pgfsetrectcap%
\pgfsetmiterjoin%
\pgfsetlinewidth{0.803000pt}%
\definecolor{currentstroke}{rgb}{0.000000,0.000000,0.000000}%
\pgfsetstrokecolor{currentstroke}%
\pgfsetdash{}{0pt}%
\pgfpathmoveto{\pgfqpoint{2.127335in}{2.227611in}}%
\pgfpathlineto{\pgfqpoint{2.127335in}{2.689611in}}%
\pgfusepath{stroke}%
\end{pgfscope}%
\begin{pgfscope}%
\pgfsetrectcap%
\pgfsetmiterjoin%
\pgfsetlinewidth{0.803000pt}%
\definecolor{currentstroke}{rgb}{0.000000,0.000000,0.000000}%
\pgfsetstrokecolor{currentstroke}%
\pgfsetdash{}{0pt}%
\pgfpathmoveto{\pgfqpoint{2.951803in}{2.227611in}}%
\pgfpathlineto{\pgfqpoint{2.951803in}{2.689611in}}%
\pgfusepath{stroke}%
\end{pgfscope}%
\begin{pgfscope}%
\pgfsetrectcap%
\pgfsetmiterjoin%
\pgfsetlinewidth{0.803000pt}%
\definecolor{currentstroke}{rgb}{0.000000,0.000000,0.000000}%
\pgfsetstrokecolor{currentstroke}%
\pgfsetdash{}{0pt}%
\pgfpathmoveto{\pgfqpoint{2.127335in}{2.227611in}}%
\pgfpathlineto{\pgfqpoint{2.951803in}{2.227611in}}%
\pgfusepath{stroke}%
\end{pgfscope}%
\begin{pgfscope}%
\pgfsetrectcap%
\pgfsetmiterjoin%
\pgfsetlinewidth{0.803000pt}%
\definecolor{currentstroke}{rgb}{0.000000,0.000000,0.000000}%
\pgfsetstrokecolor{currentstroke}%
\pgfsetdash{}{0pt}%
\pgfpathmoveto{\pgfqpoint{2.127335in}{2.689611in}}%
\pgfpathlineto{\pgfqpoint{2.951803in}{2.689611in}}%
\pgfusepath{stroke}%
\end{pgfscope}%
\begin{pgfscope}%
\definecolor{textcolor}{rgb}{0.000000,0.000000,0.000000}%
\pgfsetstrokecolor{textcolor}%
\pgfsetfillcolor{textcolor}%
\pgftext[x=2.539569in,y=2.772944in,,base]{\color{textcolor}\rmfamily\fontsize{11.000000}{13.200000}\selectfont Active...}%
\end{pgfscope}%
\begin{pgfscope}%
\pgfsetbuttcap%
\pgfsetmiterjoin%
\definecolor{currentfill}{rgb}{1.000000,1.000000,1.000000}%
\pgfsetfillcolor{currentfill}%
\pgfsetlinewidth{0.000000pt}%
\definecolor{currentstroke}{rgb}{0.000000,0.000000,0.000000}%
\pgfsetstrokecolor{currentstroke}%
\pgfsetstrokeopacity{0.000000}%
\pgfsetdash{}{0pt}%
\pgfpathmoveto{\pgfqpoint{3.116696in}{2.227611in}}%
\pgfpathlineto{\pgfqpoint{3.941164in}{2.227611in}}%
\pgfpathlineto{\pgfqpoint{3.941164in}{2.689611in}}%
\pgfpathlineto{\pgfqpoint{3.116696in}{2.689611in}}%
\pgfpathlineto{\pgfqpoint{3.116696in}{2.227611in}}%
\pgfpathclose%
\pgfusepath{fill}%
\end{pgfscope}%
\begin{pgfscope}%
\pgfpathrectangle{\pgfqpoint{3.116696in}{2.227611in}}{\pgfqpoint{0.824468in}{0.462000in}}%
\pgfusepath{clip}%
\pgfsetbuttcap%
\pgfsetmiterjoin%
\definecolor{currentfill}{rgb}{0.121569,0.466667,0.705882}%
\pgfsetfillcolor{currentfill}%
\pgfsetfillopacity{0.500000}%
\pgfsetlinewidth{1.003750pt}%
\definecolor{currentstroke}{rgb}{0.000000,0.000000,0.000000}%
\pgfsetstrokecolor{currentstroke}%
\pgfsetdash{}{0pt}%
\pgfpathmoveto{\pgfqpoint{3.154172in}{2.227611in}}%
\pgfpathlineto{\pgfqpoint{3.304075in}{2.227611in}}%
\pgfpathlineto{\pgfqpoint{3.304075in}{2.667611in}}%
\pgfpathlineto{\pgfqpoint{3.154172in}{2.667611in}}%
\pgfpathlineto{\pgfqpoint{3.154172in}{2.227611in}}%
\pgfpathclose%
\pgfusepath{stroke,fill}%
\end{pgfscope}%
\begin{pgfscope}%
\pgfpathrectangle{\pgfqpoint{3.116696in}{2.227611in}}{\pgfqpoint{0.824468in}{0.462000in}}%
\pgfusepath{clip}%
\pgfsetbuttcap%
\pgfsetmiterjoin%
\definecolor{currentfill}{rgb}{0.121569,0.466667,0.705882}%
\pgfsetfillcolor{currentfill}%
\pgfsetfillopacity{0.500000}%
\pgfsetlinewidth{1.003750pt}%
\definecolor{currentstroke}{rgb}{0.000000,0.000000,0.000000}%
\pgfsetstrokecolor{currentstroke}%
\pgfsetdash{}{0pt}%
\pgfpathmoveto{\pgfqpoint{3.304075in}{2.227611in}}%
\pgfpathlineto{\pgfqpoint{3.453979in}{2.227611in}}%
\pgfpathlineto{\pgfqpoint{3.453979in}{2.400944in}}%
\pgfpathlineto{\pgfqpoint{3.304075in}{2.400944in}}%
\pgfpathlineto{\pgfqpoint{3.304075in}{2.227611in}}%
\pgfpathclose%
\pgfusepath{stroke,fill}%
\end{pgfscope}%
\begin{pgfscope}%
\pgfpathrectangle{\pgfqpoint{3.116696in}{2.227611in}}{\pgfqpoint{0.824468in}{0.462000in}}%
\pgfusepath{clip}%
\pgfsetbuttcap%
\pgfsetmiterjoin%
\definecolor{currentfill}{rgb}{0.121569,0.466667,0.705882}%
\pgfsetfillcolor{currentfill}%
\pgfsetfillopacity{0.500000}%
\pgfsetlinewidth{1.003750pt}%
\definecolor{currentstroke}{rgb}{0.000000,0.000000,0.000000}%
\pgfsetstrokecolor{currentstroke}%
\pgfsetdash{}{0pt}%
\pgfpathmoveto{\pgfqpoint{3.453979in}{2.227611in}}%
\pgfpathlineto{\pgfqpoint{3.603882in}{2.227611in}}%
\pgfpathlineto{\pgfqpoint{3.603882in}{2.331854in}}%
\pgfpathlineto{\pgfqpoint{3.453979in}{2.331854in}}%
\pgfpathlineto{\pgfqpoint{3.453979in}{2.227611in}}%
\pgfpathclose%
\pgfusepath{stroke,fill}%
\end{pgfscope}%
\begin{pgfscope}%
\pgfpathrectangle{\pgfqpoint{3.116696in}{2.227611in}}{\pgfqpoint{0.824468in}{0.462000in}}%
\pgfusepath{clip}%
\pgfsetbuttcap%
\pgfsetmiterjoin%
\definecolor{currentfill}{rgb}{0.121569,0.466667,0.705882}%
\pgfsetfillcolor{currentfill}%
\pgfsetfillopacity{0.500000}%
\pgfsetlinewidth{1.003750pt}%
\definecolor{currentstroke}{rgb}{0.000000,0.000000,0.000000}%
\pgfsetstrokecolor{currentstroke}%
\pgfsetdash{}{0pt}%
\pgfpathmoveto{\pgfqpoint{3.603882in}{2.227611in}}%
\pgfpathlineto{\pgfqpoint{3.753785in}{2.227611in}}%
\pgfpathlineto{\pgfqpoint{3.753785in}{2.254278in}}%
\pgfpathlineto{\pgfqpoint{3.603882in}{2.254278in}}%
\pgfpathlineto{\pgfqpoint{3.603882in}{2.227611in}}%
\pgfpathclose%
\pgfusepath{stroke,fill}%
\end{pgfscope}%
\begin{pgfscope}%
\pgfpathrectangle{\pgfqpoint{3.116696in}{2.227611in}}{\pgfqpoint{0.824468in}{0.462000in}}%
\pgfusepath{clip}%
\pgfsetbuttcap%
\pgfsetmiterjoin%
\definecolor{currentfill}{rgb}{0.121569,0.466667,0.705882}%
\pgfsetfillcolor{currentfill}%
\pgfsetfillopacity{0.500000}%
\pgfsetlinewidth{1.003750pt}%
\definecolor{currentstroke}{rgb}{0.000000,0.000000,0.000000}%
\pgfsetstrokecolor{currentstroke}%
\pgfsetdash{}{0pt}%
\pgfpathmoveto{\pgfqpoint{3.753785in}{2.227611in}}%
\pgfpathlineto{\pgfqpoint{3.903688in}{2.227611in}}%
\pgfpathlineto{\pgfqpoint{3.903688in}{2.245793in}}%
\pgfpathlineto{\pgfqpoint{3.753785in}{2.245793in}}%
\pgfpathlineto{\pgfqpoint{3.753785in}{2.227611in}}%
\pgfpathclose%
\pgfusepath{stroke,fill}%
\end{pgfscope}%
\begin{pgfscope}%
\pgfsetrectcap%
\pgfsetmiterjoin%
\pgfsetlinewidth{0.803000pt}%
\definecolor{currentstroke}{rgb}{0.000000,0.000000,0.000000}%
\pgfsetstrokecolor{currentstroke}%
\pgfsetdash{}{0pt}%
\pgfpathmoveto{\pgfqpoint{3.116696in}{2.227611in}}%
\pgfpathlineto{\pgfqpoint{3.116696in}{2.689611in}}%
\pgfusepath{stroke}%
\end{pgfscope}%
\begin{pgfscope}%
\pgfsetrectcap%
\pgfsetmiterjoin%
\pgfsetlinewidth{0.803000pt}%
\definecolor{currentstroke}{rgb}{0.000000,0.000000,0.000000}%
\pgfsetstrokecolor{currentstroke}%
\pgfsetdash{}{0pt}%
\pgfpathmoveto{\pgfqpoint{3.941164in}{2.227611in}}%
\pgfpathlineto{\pgfqpoint{3.941164in}{2.689611in}}%
\pgfusepath{stroke}%
\end{pgfscope}%
\begin{pgfscope}%
\pgfsetrectcap%
\pgfsetmiterjoin%
\pgfsetlinewidth{0.803000pt}%
\definecolor{currentstroke}{rgb}{0.000000,0.000000,0.000000}%
\pgfsetstrokecolor{currentstroke}%
\pgfsetdash{}{0pt}%
\pgfpathmoveto{\pgfqpoint{3.116696in}{2.227611in}}%
\pgfpathlineto{\pgfqpoint{3.941164in}{2.227611in}}%
\pgfusepath{stroke}%
\end{pgfscope}%
\begin{pgfscope}%
\pgfsetrectcap%
\pgfsetmiterjoin%
\pgfsetlinewidth{0.803000pt}%
\definecolor{currentstroke}{rgb}{0.000000,0.000000,0.000000}%
\pgfsetstrokecolor{currentstroke}%
\pgfsetdash{}{0pt}%
\pgfpathmoveto{\pgfqpoint{3.116696in}{2.689611in}}%
\pgfpathlineto{\pgfqpoint{3.941164in}{2.689611in}}%
\pgfusepath{stroke}%
\end{pgfscope}%
\begin{pgfscope}%
\definecolor{textcolor}{rgb}{0.000000,0.000000,0.000000}%
\pgfsetstrokecolor{textcolor}%
\pgfsetfillcolor{textcolor}%
\pgftext[x=3.528930in,y=2.772944in,,base]{\color{textcolor}\rmfamily\fontsize{11.000000}{13.200000}\selectfont AXA}%
\end{pgfscope}%
\begin{pgfscope}%
\pgfsetbuttcap%
\pgfsetmiterjoin%
\definecolor{currentfill}{rgb}{1.000000,1.000000,1.000000}%
\pgfsetfillcolor{currentfill}%
\pgfsetlinewidth{0.000000pt}%
\definecolor{currentstroke}{rgb}{0.000000,0.000000,0.000000}%
\pgfsetstrokecolor{currentstroke}%
\pgfsetstrokeopacity{0.000000}%
\pgfsetdash{}{0pt}%
\pgfpathmoveto{\pgfqpoint{4.106058in}{2.227611in}}%
\pgfpathlineto{\pgfqpoint{4.930526in}{2.227611in}}%
\pgfpathlineto{\pgfqpoint{4.930526in}{2.689611in}}%
\pgfpathlineto{\pgfqpoint{4.106058in}{2.689611in}}%
\pgfpathlineto{\pgfqpoint{4.106058in}{2.227611in}}%
\pgfpathclose%
\pgfusepath{fill}%
\end{pgfscope}%
\begin{pgfscope}%
\pgfpathrectangle{\pgfqpoint{4.106058in}{2.227611in}}{\pgfqpoint{0.824468in}{0.462000in}}%
\pgfusepath{clip}%
\pgfsetbuttcap%
\pgfsetmiterjoin%
\definecolor{currentfill}{rgb}{0.121569,0.466667,0.705882}%
\pgfsetfillcolor{currentfill}%
\pgfsetfillopacity{0.500000}%
\pgfsetlinewidth{1.003750pt}%
\definecolor{currentstroke}{rgb}{0.000000,0.000000,0.000000}%
\pgfsetstrokecolor{currentstroke}%
\pgfsetdash{}{0pt}%
\pgfpathmoveto{\pgfqpoint{4.143534in}{2.227611in}}%
\pgfpathlineto{\pgfqpoint{4.293437in}{2.227611in}}%
\pgfpathlineto{\pgfqpoint{4.293437in}{2.667611in}}%
\pgfpathlineto{\pgfqpoint{4.143534in}{2.667611in}}%
\pgfpathlineto{\pgfqpoint{4.143534in}{2.227611in}}%
\pgfpathclose%
\pgfusepath{stroke,fill}%
\end{pgfscope}%
\begin{pgfscope}%
\pgfpathrectangle{\pgfqpoint{4.106058in}{2.227611in}}{\pgfqpoint{0.824468in}{0.462000in}}%
\pgfusepath{clip}%
\pgfsetbuttcap%
\pgfsetmiterjoin%
\definecolor{currentfill}{rgb}{0.121569,0.466667,0.705882}%
\pgfsetfillcolor{currentfill}%
\pgfsetfillopacity{0.500000}%
\pgfsetlinewidth{1.003750pt}%
\definecolor{currentstroke}{rgb}{0.000000,0.000000,0.000000}%
\pgfsetstrokecolor{currentstroke}%
\pgfsetdash{}{0pt}%
\pgfpathmoveto{\pgfqpoint{4.293437in}{2.227611in}}%
\pgfpathlineto{\pgfqpoint{4.443340in}{2.227611in}}%
\pgfpathlineto{\pgfqpoint{4.443340in}{2.380944in}}%
\pgfpathlineto{\pgfqpoint{4.293437in}{2.380944in}}%
\pgfpathlineto{\pgfqpoint{4.293437in}{2.227611in}}%
\pgfpathclose%
\pgfusepath{stroke,fill}%
\end{pgfscope}%
\begin{pgfscope}%
\pgfpathrectangle{\pgfqpoint{4.106058in}{2.227611in}}{\pgfqpoint{0.824468in}{0.462000in}}%
\pgfusepath{clip}%
\pgfsetbuttcap%
\pgfsetmiterjoin%
\definecolor{currentfill}{rgb}{0.121569,0.466667,0.705882}%
\pgfsetfillcolor{currentfill}%
\pgfsetfillopacity{0.500000}%
\pgfsetlinewidth{1.003750pt}%
\definecolor{currentstroke}{rgb}{0.000000,0.000000,0.000000}%
\pgfsetstrokecolor{currentstroke}%
\pgfsetdash{}{0pt}%
\pgfpathmoveto{\pgfqpoint{4.443340in}{2.227611in}}%
\pgfpathlineto{\pgfqpoint{4.593244in}{2.227611in}}%
\pgfpathlineto{\pgfqpoint{4.593244in}{2.260944in}}%
\pgfpathlineto{\pgfqpoint{4.443340in}{2.260944in}}%
\pgfpathlineto{\pgfqpoint{4.443340in}{2.227611in}}%
\pgfpathclose%
\pgfusepath{stroke,fill}%
\end{pgfscope}%
\begin{pgfscope}%
\pgfpathrectangle{\pgfqpoint{4.106058in}{2.227611in}}{\pgfqpoint{0.824468in}{0.462000in}}%
\pgfusepath{clip}%
\pgfsetbuttcap%
\pgfsetmiterjoin%
\definecolor{currentfill}{rgb}{0.121569,0.466667,0.705882}%
\pgfsetfillcolor{currentfill}%
\pgfsetfillopacity{0.500000}%
\pgfsetlinewidth{1.003750pt}%
\definecolor{currentstroke}{rgb}{0.000000,0.000000,0.000000}%
\pgfsetstrokecolor{currentstroke}%
\pgfsetdash{}{0pt}%
\pgfpathmoveto{\pgfqpoint{4.593244in}{2.227611in}}%
\pgfpathlineto{\pgfqpoint{4.743147in}{2.227611in}}%
\pgfpathlineto{\pgfqpoint{4.743147in}{2.227611in}}%
\pgfpathlineto{\pgfqpoint{4.593244in}{2.227611in}}%
\pgfpathlineto{\pgfqpoint{4.593244in}{2.227611in}}%
\pgfpathclose%
\pgfusepath{stroke,fill}%
\end{pgfscope}%
\begin{pgfscope}%
\pgfpathrectangle{\pgfqpoint{4.106058in}{2.227611in}}{\pgfqpoint{0.824468in}{0.462000in}}%
\pgfusepath{clip}%
\pgfsetbuttcap%
\pgfsetmiterjoin%
\definecolor{currentfill}{rgb}{0.121569,0.466667,0.705882}%
\pgfsetfillcolor{currentfill}%
\pgfsetfillopacity{0.500000}%
\pgfsetlinewidth{1.003750pt}%
\definecolor{currentstroke}{rgb}{0.000000,0.000000,0.000000}%
\pgfsetstrokecolor{currentstroke}%
\pgfsetdash{}{0pt}%
\pgfpathmoveto{\pgfqpoint{4.743147in}{2.227611in}}%
\pgfpathlineto{\pgfqpoint{4.893050in}{2.227611in}}%
\pgfpathlineto{\pgfqpoint{4.893050in}{2.240944in}}%
\pgfpathlineto{\pgfqpoint{4.743147in}{2.240944in}}%
\pgfpathlineto{\pgfqpoint{4.743147in}{2.227611in}}%
\pgfpathclose%
\pgfusepath{stroke,fill}%
\end{pgfscope}%
\begin{pgfscope}%
\pgfsetrectcap%
\pgfsetmiterjoin%
\pgfsetlinewidth{0.803000pt}%
\definecolor{currentstroke}{rgb}{0.000000,0.000000,0.000000}%
\pgfsetstrokecolor{currentstroke}%
\pgfsetdash{}{0pt}%
\pgfpathmoveto{\pgfqpoint{4.106058in}{2.227611in}}%
\pgfpathlineto{\pgfqpoint{4.106058in}{2.689611in}}%
\pgfusepath{stroke}%
\end{pgfscope}%
\begin{pgfscope}%
\pgfsetrectcap%
\pgfsetmiterjoin%
\pgfsetlinewidth{0.803000pt}%
\definecolor{currentstroke}{rgb}{0.000000,0.000000,0.000000}%
\pgfsetstrokecolor{currentstroke}%
\pgfsetdash{}{0pt}%
\pgfpathmoveto{\pgfqpoint{4.930526in}{2.227611in}}%
\pgfpathlineto{\pgfqpoint{4.930526in}{2.689611in}}%
\pgfusepath{stroke}%
\end{pgfscope}%
\begin{pgfscope}%
\pgfsetrectcap%
\pgfsetmiterjoin%
\pgfsetlinewidth{0.803000pt}%
\definecolor{currentstroke}{rgb}{0.000000,0.000000,0.000000}%
\pgfsetstrokecolor{currentstroke}%
\pgfsetdash{}{0pt}%
\pgfpathmoveto{\pgfqpoint{4.106058in}{2.227611in}}%
\pgfpathlineto{\pgfqpoint{4.930526in}{2.227611in}}%
\pgfusepath{stroke}%
\end{pgfscope}%
\begin{pgfscope}%
\pgfsetrectcap%
\pgfsetmiterjoin%
\pgfsetlinewidth{0.803000pt}%
\definecolor{currentstroke}{rgb}{0.000000,0.000000,0.000000}%
\pgfsetstrokecolor{currentstroke}%
\pgfsetdash{}{0pt}%
\pgfpathmoveto{\pgfqpoint{4.106058in}{2.689611in}}%
\pgfpathlineto{\pgfqpoint{4.930526in}{2.689611in}}%
\pgfusepath{stroke}%
\end{pgfscope}%
\begin{pgfscope}%
\definecolor{textcolor}{rgb}{0.000000,0.000000,0.000000}%
\pgfsetstrokecolor{textcolor}%
\pgfsetfillcolor{textcolor}%
\pgftext[x=4.518292in,y=2.772944in,,base]{\color{textcolor}\rmfamily\fontsize{11.000000}{13.200000}\selectfont Sogessur}%
\end{pgfscope}%
\begin{pgfscope}%
\pgfsetbuttcap%
\pgfsetmiterjoin%
\definecolor{currentfill}{rgb}{1.000000,1.000000,1.000000}%
\pgfsetfillcolor{currentfill}%
\pgfsetlinewidth{0.000000pt}%
\definecolor{currentstroke}{rgb}{0.000000,0.000000,0.000000}%
\pgfsetstrokecolor{currentstroke}%
\pgfsetstrokeopacity{0.000000}%
\pgfsetdash{}{0pt}%
\pgfpathmoveto{\pgfqpoint{5.095420in}{2.227611in}}%
\pgfpathlineto{\pgfqpoint{5.919888in}{2.227611in}}%
\pgfpathlineto{\pgfqpoint{5.919888in}{2.689611in}}%
\pgfpathlineto{\pgfqpoint{5.095420in}{2.689611in}}%
\pgfpathlineto{\pgfqpoint{5.095420in}{2.227611in}}%
\pgfpathclose%
\pgfusepath{fill}%
\end{pgfscope}%
\begin{pgfscope}%
\pgfpathrectangle{\pgfqpoint{5.095420in}{2.227611in}}{\pgfqpoint{0.824468in}{0.462000in}}%
\pgfusepath{clip}%
\pgfsetbuttcap%
\pgfsetmiterjoin%
\definecolor{currentfill}{rgb}{0.121569,0.466667,0.705882}%
\pgfsetfillcolor{currentfill}%
\pgfsetfillopacity{0.500000}%
\pgfsetlinewidth{1.003750pt}%
\definecolor{currentstroke}{rgb}{0.000000,0.000000,0.000000}%
\pgfsetstrokecolor{currentstroke}%
\pgfsetdash{}{0pt}%
\pgfpathmoveto{\pgfqpoint{5.132895in}{2.227611in}}%
\pgfpathlineto{\pgfqpoint{5.282799in}{2.227611in}}%
\pgfpathlineto{\pgfqpoint{5.282799in}{2.667611in}}%
\pgfpathlineto{\pgfqpoint{5.132895in}{2.667611in}}%
\pgfpathlineto{\pgfqpoint{5.132895in}{2.227611in}}%
\pgfpathclose%
\pgfusepath{stroke,fill}%
\end{pgfscope}%
\begin{pgfscope}%
\pgfpathrectangle{\pgfqpoint{5.095420in}{2.227611in}}{\pgfqpoint{0.824468in}{0.462000in}}%
\pgfusepath{clip}%
\pgfsetbuttcap%
\pgfsetmiterjoin%
\definecolor{currentfill}{rgb}{0.121569,0.466667,0.705882}%
\pgfsetfillcolor{currentfill}%
\pgfsetfillopacity{0.500000}%
\pgfsetlinewidth{1.003750pt}%
\definecolor{currentstroke}{rgb}{0.000000,0.000000,0.000000}%
\pgfsetstrokecolor{currentstroke}%
\pgfsetdash{}{0pt}%
\pgfpathmoveto{\pgfqpoint{5.282799in}{2.227611in}}%
\pgfpathlineto{\pgfqpoint{5.432702in}{2.227611in}}%
\pgfpathlineto{\pgfqpoint{5.432702in}{2.354603in}}%
\pgfpathlineto{\pgfqpoint{5.282799in}{2.354603in}}%
\pgfpathlineto{\pgfqpoint{5.282799in}{2.227611in}}%
\pgfpathclose%
\pgfusepath{stroke,fill}%
\end{pgfscope}%
\begin{pgfscope}%
\pgfpathrectangle{\pgfqpoint{5.095420in}{2.227611in}}{\pgfqpoint{0.824468in}{0.462000in}}%
\pgfusepath{clip}%
\pgfsetbuttcap%
\pgfsetmiterjoin%
\definecolor{currentfill}{rgb}{0.121569,0.466667,0.705882}%
\pgfsetfillcolor{currentfill}%
\pgfsetfillopacity{0.500000}%
\pgfsetlinewidth{1.003750pt}%
\definecolor{currentstroke}{rgb}{0.000000,0.000000,0.000000}%
\pgfsetstrokecolor{currentstroke}%
\pgfsetdash{}{0pt}%
\pgfpathmoveto{\pgfqpoint{5.432702in}{2.227611in}}%
\pgfpathlineto{\pgfqpoint{5.582605in}{2.227611in}}%
\pgfpathlineto{\pgfqpoint{5.582605in}{2.281270in}}%
\pgfpathlineto{\pgfqpoint{5.432702in}{2.281270in}}%
\pgfpathlineto{\pgfqpoint{5.432702in}{2.227611in}}%
\pgfpathclose%
\pgfusepath{stroke,fill}%
\end{pgfscope}%
\begin{pgfscope}%
\pgfpathrectangle{\pgfqpoint{5.095420in}{2.227611in}}{\pgfqpoint{0.824468in}{0.462000in}}%
\pgfusepath{clip}%
\pgfsetbuttcap%
\pgfsetmiterjoin%
\definecolor{currentfill}{rgb}{0.121569,0.466667,0.705882}%
\pgfsetfillcolor{currentfill}%
\pgfsetfillopacity{0.500000}%
\pgfsetlinewidth{1.003750pt}%
\definecolor{currentstroke}{rgb}{0.000000,0.000000,0.000000}%
\pgfsetstrokecolor{currentstroke}%
\pgfsetdash{}{0pt}%
\pgfpathmoveto{\pgfqpoint{5.582605in}{2.227611in}}%
\pgfpathlineto{\pgfqpoint{5.732509in}{2.227611in}}%
\pgfpathlineto{\pgfqpoint{5.732509in}{2.231188in}}%
\pgfpathlineto{\pgfqpoint{5.582605in}{2.231188in}}%
\pgfpathlineto{\pgfqpoint{5.582605in}{2.227611in}}%
\pgfpathclose%
\pgfusepath{stroke,fill}%
\end{pgfscope}%
\begin{pgfscope}%
\pgfpathrectangle{\pgfqpoint{5.095420in}{2.227611in}}{\pgfqpoint{0.824468in}{0.462000in}}%
\pgfusepath{clip}%
\pgfsetbuttcap%
\pgfsetmiterjoin%
\definecolor{currentfill}{rgb}{0.121569,0.466667,0.705882}%
\pgfsetfillcolor{currentfill}%
\pgfsetfillopacity{0.500000}%
\pgfsetlinewidth{1.003750pt}%
\definecolor{currentstroke}{rgb}{0.000000,0.000000,0.000000}%
\pgfsetstrokecolor{currentstroke}%
\pgfsetdash{}{0pt}%
\pgfpathmoveto{\pgfqpoint{5.732509in}{2.227611in}}%
\pgfpathlineto{\pgfqpoint{5.882412in}{2.227611in}}%
\pgfpathlineto{\pgfqpoint{5.882412in}{2.229400in}}%
\pgfpathlineto{\pgfqpoint{5.732509in}{2.229400in}}%
\pgfpathlineto{\pgfqpoint{5.732509in}{2.227611in}}%
\pgfpathclose%
\pgfusepath{stroke,fill}%
\end{pgfscope}%
\begin{pgfscope}%
\pgfsetrectcap%
\pgfsetmiterjoin%
\pgfsetlinewidth{0.803000pt}%
\definecolor{currentstroke}{rgb}{0.000000,0.000000,0.000000}%
\pgfsetstrokecolor{currentstroke}%
\pgfsetdash{}{0pt}%
\pgfpathmoveto{\pgfqpoint{5.095420in}{2.227611in}}%
\pgfpathlineto{\pgfqpoint{5.095420in}{2.689611in}}%
\pgfusepath{stroke}%
\end{pgfscope}%
\begin{pgfscope}%
\pgfsetrectcap%
\pgfsetmiterjoin%
\pgfsetlinewidth{0.803000pt}%
\definecolor{currentstroke}{rgb}{0.000000,0.000000,0.000000}%
\pgfsetstrokecolor{currentstroke}%
\pgfsetdash{}{0pt}%
\pgfpathmoveto{\pgfqpoint{5.919888in}{2.227611in}}%
\pgfpathlineto{\pgfqpoint{5.919888in}{2.689611in}}%
\pgfusepath{stroke}%
\end{pgfscope}%
\begin{pgfscope}%
\pgfsetrectcap%
\pgfsetmiterjoin%
\pgfsetlinewidth{0.803000pt}%
\definecolor{currentstroke}{rgb}{0.000000,0.000000,0.000000}%
\pgfsetstrokecolor{currentstroke}%
\pgfsetdash{}{0pt}%
\pgfpathmoveto{\pgfqpoint{5.095420in}{2.227611in}}%
\pgfpathlineto{\pgfqpoint{5.919888in}{2.227611in}}%
\pgfusepath{stroke}%
\end{pgfscope}%
\begin{pgfscope}%
\pgfsetrectcap%
\pgfsetmiterjoin%
\pgfsetlinewidth{0.803000pt}%
\definecolor{currentstroke}{rgb}{0.000000,0.000000,0.000000}%
\pgfsetstrokecolor{currentstroke}%
\pgfsetdash{}{0pt}%
\pgfpathmoveto{\pgfqpoint{5.095420in}{2.689611in}}%
\pgfpathlineto{\pgfqpoint{5.919888in}{2.689611in}}%
\pgfusepath{stroke}%
\end{pgfscope}%
\begin{pgfscope}%
\definecolor{textcolor}{rgb}{0.000000,0.000000,0.000000}%
\pgfsetstrokecolor{textcolor}%
\pgfsetfillcolor{textcolor}%
\pgftext[x=5.507654in,y=2.772944in,,base]{\color{textcolor}\rmfamily\fontsize{11.000000}{13.200000}\selectfont Ag2r L...}%
\end{pgfscope}%
\begin{pgfscope}%
\pgfsetbuttcap%
\pgfsetmiterjoin%
\definecolor{currentfill}{rgb}{1.000000,1.000000,1.000000}%
\pgfsetfillcolor{currentfill}%
\pgfsetlinewidth{0.000000pt}%
\definecolor{currentstroke}{rgb}{0.000000,0.000000,0.000000}%
\pgfsetstrokecolor{currentstroke}%
\pgfsetstrokeopacity{0.000000}%
\pgfsetdash{}{0pt}%
\pgfpathmoveto{\pgfqpoint{6.084781in}{2.227611in}}%
\pgfpathlineto{\pgfqpoint{6.909249in}{2.227611in}}%
\pgfpathlineto{\pgfqpoint{6.909249in}{2.689611in}}%
\pgfpathlineto{\pgfqpoint{6.084781in}{2.689611in}}%
\pgfpathlineto{\pgfqpoint{6.084781in}{2.227611in}}%
\pgfpathclose%
\pgfusepath{fill}%
\end{pgfscope}%
\begin{pgfscope}%
\pgfpathrectangle{\pgfqpoint{6.084781in}{2.227611in}}{\pgfqpoint{0.824468in}{0.462000in}}%
\pgfusepath{clip}%
\pgfsetbuttcap%
\pgfsetmiterjoin%
\definecolor{currentfill}{rgb}{0.121569,0.466667,0.705882}%
\pgfsetfillcolor{currentfill}%
\pgfsetfillopacity{0.500000}%
\pgfsetlinewidth{1.003750pt}%
\definecolor{currentstroke}{rgb}{0.000000,0.000000,0.000000}%
\pgfsetstrokecolor{currentstroke}%
\pgfsetdash{}{0pt}%
\pgfpathmoveto{\pgfqpoint{6.122257in}{2.227611in}}%
\pgfpathlineto{\pgfqpoint{6.272160in}{2.227611in}}%
\pgfpathlineto{\pgfqpoint{6.272160in}{2.667611in}}%
\pgfpathlineto{\pgfqpoint{6.122257in}{2.667611in}}%
\pgfpathlineto{\pgfqpoint{6.122257in}{2.227611in}}%
\pgfpathclose%
\pgfusepath{stroke,fill}%
\end{pgfscope}%
\begin{pgfscope}%
\pgfpathrectangle{\pgfqpoint{6.084781in}{2.227611in}}{\pgfqpoint{0.824468in}{0.462000in}}%
\pgfusepath{clip}%
\pgfsetbuttcap%
\pgfsetmiterjoin%
\definecolor{currentfill}{rgb}{0.121569,0.466667,0.705882}%
\pgfsetfillcolor{currentfill}%
\pgfsetfillopacity{0.500000}%
\pgfsetlinewidth{1.003750pt}%
\definecolor{currentstroke}{rgb}{0.000000,0.000000,0.000000}%
\pgfsetstrokecolor{currentstroke}%
\pgfsetdash{}{0pt}%
\pgfpathmoveto{\pgfqpoint{6.272160in}{2.227611in}}%
\pgfpathlineto{\pgfqpoint{6.422064in}{2.227611in}}%
\pgfpathlineto{\pgfqpoint{6.422064in}{2.317407in}}%
\pgfpathlineto{\pgfqpoint{6.272160in}{2.317407in}}%
\pgfpathlineto{\pgfqpoint{6.272160in}{2.227611in}}%
\pgfpathclose%
\pgfusepath{stroke,fill}%
\end{pgfscope}%
\begin{pgfscope}%
\pgfpathrectangle{\pgfqpoint{6.084781in}{2.227611in}}{\pgfqpoint{0.824468in}{0.462000in}}%
\pgfusepath{clip}%
\pgfsetbuttcap%
\pgfsetmiterjoin%
\definecolor{currentfill}{rgb}{0.121569,0.466667,0.705882}%
\pgfsetfillcolor{currentfill}%
\pgfsetfillopacity{0.500000}%
\pgfsetlinewidth{1.003750pt}%
\definecolor{currentstroke}{rgb}{0.000000,0.000000,0.000000}%
\pgfsetstrokecolor{currentstroke}%
\pgfsetdash{}{0pt}%
\pgfpathmoveto{\pgfqpoint{6.422064in}{2.227611in}}%
\pgfpathlineto{\pgfqpoint{6.571967in}{2.227611in}}%
\pgfpathlineto{\pgfqpoint{6.571967in}{2.287475in}}%
\pgfpathlineto{\pgfqpoint{6.422064in}{2.287475in}}%
\pgfpathlineto{\pgfqpoint{6.422064in}{2.227611in}}%
\pgfpathclose%
\pgfusepath{stroke,fill}%
\end{pgfscope}%
\begin{pgfscope}%
\pgfpathrectangle{\pgfqpoint{6.084781in}{2.227611in}}{\pgfqpoint{0.824468in}{0.462000in}}%
\pgfusepath{clip}%
\pgfsetbuttcap%
\pgfsetmiterjoin%
\definecolor{currentfill}{rgb}{0.121569,0.466667,0.705882}%
\pgfsetfillcolor{currentfill}%
\pgfsetfillopacity{0.500000}%
\pgfsetlinewidth{1.003750pt}%
\definecolor{currentstroke}{rgb}{0.000000,0.000000,0.000000}%
\pgfsetstrokecolor{currentstroke}%
\pgfsetdash{}{0pt}%
\pgfpathmoveto{\pgfqpoint{6.571967in}{2.227611in}}%
\pgfpathlineto{\pgfqpoint{6.721870in}{2.227611in}}%
\pgfpathlineto{\pgfqpoint{6.721870in}{2.242577in}}%
\pgfpathlineto{\pgfqpoint{6.571967in}{2.242577in}}%
\pgfpathlineto{\pgfqpoint{6.571967in}{2.227611in}}%
\pgfpathclose%
\pgfusepath{stroke,fill}%
\end{pgfscope}%
\begin{pgfscope}%
\pgfpathrectangle{\pgfqpoint{6.084781in}{2.227611in}}{\pgfqpoint{0.824468in}{0.462000in}}%
\pgfusepath{clip}%
\pgfsetbuttcap%
\pgfsetmiterjoin%
\definecolor{currentfill}{rgb}{0.121569,0.466667,0.705882}%
\pgfsetfillcolor{currentfill}%
\pgfsetfillopacity{0.500000}%
\pgfsetlinewidth{1.003750pt}%
\definecolor{currentstroke}{rgb}{0.000000,0.000000,0.000000}%
\pgfsetstrokecolor{currentstroke}%
\pgfsetdash{}{0pt}%
\pgfpathmoveto{\pgfqpoint{6.721870in}{2.227611in}}%
\pgfpathlineto{\pgfqpoint{6.871774in}{2.227611in}}%
\pgfpathlineto{\pgfqpoint{6.871774in}{2.248563in}}%
\pgfpathlineto{\pgfqpoint{6.721870in}{2.248563in}}%
\pgfpathlineto{\pgfqpoint{6.721870in}{2.227611in}}%
\pgfpathclose%
\pgfusepath{stroke,fill}%
\end{pgfscope}%
\begin{pgfscope}%
\pgfsetrectcap%
\pgfsetmiterjoin%
\pgfsetlinewidth{0.803000pt}%
\definecolor{currentstroke}{rgb}{0.000000,0.000000,0.000000}%
\pgfsetstrokecolor{currentstroke}%
\pgfsetdash{}{0pt}%
\pgfpathmoveto{\pgfqpoint{6.084781in}{2.227611in}}%
\pgfpathlineto{\pgfqpoint{6.084781in}{2.689611in}}%
\pgfusepath{stroke}%
\end{pgfscope}%
\begin{pgfscope}%
\pgfsetrectcap%
\pgfsetmiterjoin%
\pgfsetlinewidth{0.803000pt}%
\definecolor{currentstroke}{rgb}{0.000000,0.000000,0.000000}%
\pgfsetstrokecolor{currentstroke}%
\pgfsetdash{}{0pt}%
\pgfpathmoveto{\pgfqpoint{6.909249in}{2.227611in}}%
\pgfpathlineto{\pgfqpoint{6.909249in}{2.689611in}}%
\pgfusepath{stroke}%
\end{pgfscope}%
\begin{pgfscope}%
\pgfsetrectcap%
\pgfsetmiterjoin%
\pgfsetlinewidth{0.803000pt}%
\definecolor{currentstroke}{rgb}{0.000000,0.000000,0.000000}%
\pgfsetstrokecolor{currentstroke}%
\pgfsetdash{}{0pt}%
\pgfpathmoveto{\pgfqpoint{6.084781in}{2.227611in}}%
\pgfpathlineto{\pgfqpoint{6.909249in}{2.227611in}}%
\pgfusepath{stroke}%
\end{pgfscope}%
\begin{pgfscope}%
\pgfsetrectcap%
\pgfsetmiterjoin%
\pgfsetlinewidth{0.803000pt}%
\definecolor{currentstroke}{rgb}{0.000000,0.000000,0.000000}%
\pgfsetstrokecolor{currentstroke}%
\pgfsetdash{}{0pt}%
\pgfpathmoveto{\pgfqpoint{6.084781in}{2.689611in}}%
\pgfpathlineto{\pgfqpoint{6.909249in}{2.689611in}}%
\pgfusepath{stroke}%
\end{pgfscope}%
\begin{pgfscope}%
\definecolor{textcolor}{rgb}{0.000000,0.000000,0.000000}%
\pgfsetstrokecolor{textcolor}%
\pgfsetfillcolor{textcolor}%
\pgftext[x=6.497015in,y=2.772944in,,base]{\color{textcolor}\rmfamily\fontsize{11.000000}{13.200000}\selectfont Mgen}%
\end{pgfscope}%
\begin{pgfscope}%
\pgfsetbuttcap%
\pgfsetmiterjoin%
\definecolor{currentfill}{rgb}{1.000000,1.000000,1.000000}%
\pgfsetfillcolor{currentfill}%
\pgfsetlinewidth{0.000000pt}%
\definecolor{currentstroke}{rgb}{0.000000,0.000000,0.000000}%
\pgfsetstrokecolor{currentstroke}%
\pgfsetstrokeopacity{0.000000}%
\pgfsetdash{}{0pt}%
\pgfpathmoveto{\pgfqpoint{7.074143in}{2.227611in}}%
\pgfpathlineto{\pgfqpoint{7.898611in}{2.227611in}}%
\pgfpathlineto{\pgfqpoint{7.898611in}{2.689611in}}%
\pgfpathlineto{\pgfqpoint{7.074143in}{2.689611in}}%
\pgfpathlineto{\pgfqpoint{7.074143in}{2.227611in}}%
\pgfpathclose%
\pgfusepath{fill}%
\end{pgfscope}%
\begin{pgfscope}%
\pgfpathrectangle{\pgfqpoint{7.074143in}{2.227611in}}{\pgfqpoint{0.824468in}{0.462000in}}%
\pgfusepath{clip}%
\pgfsetbuttcap%
\pgfsetmiterjoin%
\definecolor{currentfill}{rgb}{0.121569,0.466667,0.705882}%
\pgfsetfillcolor{currentfill}%
\pgfsetfillopacity{0.500000}%
\pgfsetlinewidth{1.003750pt}%
\definecolor{currentstroke}{rgb}{0.000000,0.000000,0.000000}%
\pgfsetstrokecolor{currentstroke}%
\pgfsetdash{}{0pt}%
\pgfpathmoveto{\pgfqpoint{7.111619in}{2.227611in}}%
\pgfpathlineto{\pgfqpoint{7.261522in}{2.227611in}}%
\pgfpathlineto{\pgfqpoint{7.261522in}{2.233517in}}%
\pgfpathlineto{\pgfqpoint{7.111619in}{2.233517in}}%
\pgfpathlineto{\pgfqpoint{7.111619in}{2.227611in}}%
\pgfpathclose%
\pgfusepath{stroke,fill}%
\end{pgfscope}%
\begin{pgfscope}%
\pgfpathrectangle{\pgfqpoint{7.074143in}{2.227611in}}{\pgfqpoint{0.824468in}{0.462000in}}%
\pgfusepath{clip}%
\pgfsetbuttcap%
\pgfsetmiterjoin%
\definecolor{currentfill}{rgb}{0.121569,0.466667,0.705882}%
\pgfsetfillcolor{currentfill}%
\pgfsetfillopacity{0.500000}%
\pgfsetlinewidth{1.003750pt}%
\definecolor{currentstroke}{rgb}{0.000000,0.000000,0.000000}%
\pgfsetstrokecolor{currentstroke}%
\pgfsetdash{}{0pt}%
\pgfpathmoveto{\pgfqpoint{7.261522in}{2.227611in}}%
\pgfpathlineto{\pgfqpoint{7.411425in}{2.227611in}}%
\pgfpathlineto{\pgfqpoint{7.411425in}{2.257141in}}%
\pgfpathlineto{\pgfqpoint{7.261522in}{2.257141in}}%
\pgfpathlineto{\pgfqpoint{7.261522in}{2.227611in}}%
\pgfpathclose%
\pgfusepath{stroke,fill}%
\end{pgfscope}%
\begin{pgfscope}%
\pgfpathrectangle{\pgfqpoint{7.074143in}{2.227611in}}{\pgfqpoint{0.824468in}{0.462000in}}%
\pgfusepath{clip}%
\pgfsetbuttcap%
\pgfsetmiterjoin%
\definecolor{currentfill}{rgb}{0.121569,0.466667,0.705882}%
\pgfsetfillcolor{currentfill}%
\pgfsetfillopacity{0.500000}%
\pgfsetlinewidth{1.003750pt}%
\definecolor{currentstroke}{rgb}{0.000000,0.000000,0.000000}%
\pgfsetstrokecolor{currentstroke}%
\pgfsetdash{}{0pt}%
\pgfpathmoveto{\pgfqpoint{7.411425in}{2.227611in}}%
\pgfpathlineto{\pgfqpoint{7.561329in}{2.227611in}}%
\pgfpathlineto{\pgfqpoint{7.561329in}{2.274859in}}%
\pgfpathlineto{\pgfqpoint{7.411425in}{2.274859in}}%
\pgfpathlineto{\pgfqpoint{7.411425in}{2.227611in}}%
\pgfpathclose%
\pgfusepath{stroke,fill}%
\end{pgfscope}%
\begin{pgfscope}%
\pgfpathrectangle{\pgfqpoint{7.074143in}{2.227611in}}{\pgfqpoint{0.824468in}{0.462000in}}%
\pgfusepath{clip}%
\pgfsetbuttcap%
\pgfsetmiterjoin%
\definecolor{currentfill}{rgb}{0.121569,0.466667,0.705882}%
\pgfsetfillcolor{currentfill}%
\pgfsetfillopacity{0.500000}%
\pgfsetlinewidth{1.003750pt}%
\definecolor{currentstroke}{rgb}{0.000000,0.000000,0.000000}%
\pgfsetstrokecolor{currentstroke}%
\pgfsetdash{}{0pt}%
\pgfpathmoveto{\pgfqpoint{7.561329in}{2.227611in}}%
\pgfpathlineto{\pgfqpoint{7.711232in}{2.227611in}}%
\pgfpathlineto{\pgfqpoint{7.711232in}{2.428416in}}%
\pgfpathlineto{\pgfqpoint{7.561329in}{2.428416in}}%
\pgfpathlineto{\pgfqpoint{7.561329in}{2.227611in}}%
\pgfpathclose%
\pgfusepath{stroke,fill}%
\end{pgfscope}%
\begin{pgfscope}%
\pgfpathrectangle{\pgfqpoint{7.074143in}{2.227611in}}{\pgfqpoint{0.824468in}{0.462000in}}%
\pgfusepath{clip}%
\pgfsetbuttcap%
\pgfsetmiterjoin%
\definecolor{currentfill}{rgb}{0.121569,0.466667,0.705882}%
\pgfsetfillcolor{currentfill}%
\pgfsetfillopacity{0.500000}%
\pgfsetlinewidth{1.003750pt}%
\definecolor{currentstroke}{rgb}{0.000000,0.000000,0.000000}%
\pgfsetstrokecolor{currentstroke}%
\pgfsetdash{}{0pt}%
\pgfpathmoveto{\pgfqpoint{7.711232in}{2.227611in}}%
\pgfpathlineto{\pgfqpoint{7.861135in}{2.227611in}}%
\pgfpathlineto{\pgfqpoint{7.861135in}{2.667611in}}%
\pgfpathlineto{\pgfqpoint{7.711232in}{2.667611in}}%
\pgfpathlineto{\pgfqpoint{7.711232in}{2.227611in}}%
\pgfpathclose%
\pgfusepath{stroke,fill}%
\end{pgfscope}%
\begin{pgfscope}%
\pgfsetrectcap%
\pgfsetmiterjoin%
\pgfsetlinewidth{0.803000pt}%
\definecolor{currentstroke}{rgb}{0.000000,0.000000,0.000000}%
\pgfsetstrokecolor{currentstroke}%
\pgfsetdash{}{0pt}%
\pgfpathmoveto{\pgfqpoint{7.074143in}{2.227611in}}%
\pgfpathlineto{\pgfqpoint{7.074143in}{2.689611in}}%
\pgfusepath{stroke}%
\end{pgfscope}%
\begin{pgfscope}%
\pgfsetrectcap%
\pgfsetmiterjoin%
\pgfsetlinewidth{0.803000pt}%
\definecolor{currentstroke}{rgb}{0.000000,0.000000,0.000000}%
\pgfsetstrokecolor{currentstroke}%
\pgfsetdash{}{0pt}%
\pgfpathmoveto{\pgfqpoint{7.898611in}{2.227611in}}%
\pgfpathlineto{\pgfqpoint{7.898611in}{2.689611in}}%
\pgfusepath{stroke}%
\end{pgfscope}%
\begin{pgfscope}%
\pgfsetrectcap%
\pgfsetmiterjoin%
\pgfsetlinewidth{0.803000pt}%
\definecolor{currentstroke}{rgb}{0.000000,0.000000,0.000000}%
\pgfsetstrokecolor{currentstroke}%
\pgfsetdash{}{0pt}%
\pgfpathmoveto{\pgfqpoint{7.074143in}{2.227611in}}%
\pgfpathlineto{\pgfqpoint{7.898611in}{2.227611in}}%
\pgfusepath{stroke}%
\end{pgfscope}%
\begin{pgfscope}%
\pgfsetrectcap%
\pgfsetmiterjoin%
\pgfsetlinewidth{0.803000pt}%
\definecolor{currentstroke}{rgb}{0.000000,0.000000,0.000000}%
\pgfsetstrokecolor{currentstroke}%
\pgfsetdash{}{0pt}%
\pgfpathmoveto{\pgfqpoint{7.074143in}{2.689611in}}%
\pgfpathlineto{\pgfqpoint{7.898611in}{2.689611in}}%
\pgfusepath{stroke}%
\end{pgfscope}%
\begin{pgfscope}%
\definecolor{textcolor}{rgb}{0.000000,0.000000,0.000000}%
\pgfsetstrokecolor{textcolor}%
\pgfsetfillcolor{textcolor}%
\pgftext[x=7.486377in,y=2.772944in,,base]{\color{textcolor}\rmfamily\fontsize{11.000000}{13.200000}\selectfont Zen'Up}%
\end{pgfscope}%
\begin{pgfscope}%
\pgfsetbuttcap%
\pgfsetmiterjoin%
\definecolor{currentfill}{rgb}{1.000000,1.000000,1.000000}%
\pgfsetfillcolor{currentfill}%
\pgfsetlinewidth{0.000000pt}%
\definecolor{currentstroke}{rgb}{0.000000,0.000000,0.000000}%
\pgfsetstrokecolor{currentstroke}%
\pgfsetstrokeopacity{0.000000}%
\pgfsetdash{}{0pt}%
\pgfpathmoveto{\pgfqpoint{0.148611in}{1.534611in}}%
\pgfpathlineto{\pgfqpoint{0.973079in}{1.534611in}}%
\pgfpathlineto{\pgfqpoint{0.973079in}{1.996611in}}%
\pgfpathlineto{\pgfqpoint{0.148611in}{1.996611in}}%
\pgfpathlineto{\pgfqpoint{0.148611in}{1.534611in}}%
\pgfpathclose%
\pgfusepath{fill}%
\end{pgfscope}%
\begin{pgfscope}%
\pgfpathrectangle{\pgfqpoint{0.148611in}{1.534611in}}{\pgfqpoint{0.824468in}{0.462000in}}%
\pgfusepath{clip}%
\pgfsetbuttcap%
\pgfsetmiterjoin%
\definecolor{currentfill}{rgb}{0.121569,0.466667,0.705882}%
\pgfsetfillcolor{currentfill}%
\pgfsetfillopacity{0.500000}%
\pgfsetlinewidth{1.003750pt}%
\definecolor{currentstroke}{rgb}{0.000000,0.000000,0.000000}%
\pgfsetstrokecolor{currentstroke}%
\pgfsetdash{}{0pt}%
\pgfpathmoveto{\pgfqpoint{0.186087in}{1.534611in}}%
\pgfpathlineto{\pgfqpoint{0.335990in}{1.534611in}}%
\pgfpathlineto{\pgfqpoint{0.335990in}{1.630703in}}%
\pgfpathlineto{\pgfqpoint{0.186087in}{1.630703in}}%
\pgfpathlineto{\pgfqpoint{0.186087in}{1.534611in}}%
\pgfpathclose%
\pgfusepath{stroke,fill}%
\end{pgfscope}%
\begin{pgfscope}%
\pgfpathrectangle{\pgfqpoint{0.148611in}{1.534611in}}{\pgfqpoint{0.824468in}{0.462000in}}%
\pgfusepath{clip}%
\pgfsetbuttcap%
\pgfsetmiterjoin%
\definecolor{currentfill}{rgb}{0.121569,0.466667,0.705882}%
\pgfsetfillcolor{currentfill}%
\pgfsetfillopacity{0.500000}%
\pgfsetlinewidth{1.003750pt}%
\definecolor{currentstroke}{rgb}{0.000000,0.000000,0.000000}%
\pgfsetstrokecolor{currentstroke}%
\pgfsetdash{}{0pt}%
\pgfpathmoveto{\pgfqpoint{0.335990in}{1.534611in}}%
\pgfpathlineto{\pgfqpoint{0.485894in}{1.534611in}}%
\pgfpathlineto{\pgfqpoint{0.485894in}{1.618059in}}%
\pgfpathlineto{\pgfqpoint{0.335990in}{1.618059in}}%
\pgfpathlineto{\pgfqpoint{0.335990in}{1.534611in}}%
\pgfpathclose%
\pgfusepath{stroke,fill}%
\end{pgfscope}%
\begin{pgfscope}%
\pgfpathrectangle{\pgfqpoint{0.148611in}{1.534611in}}{\pgfqpoint{0.824468in}{0.462000in}}%
\pgfusepath{clip}%
\pgfsetbuttcap%
\pgfsetmiterjoin%
\definecolor{currentfill}{rgb}{0.121569,0.466667,0.705882}%
\pgfsetfillcolor{currentfill}%
\pgfsetfillopacity{0.500000}%
\pgfsetlinewidth{1.003750pt}%
\definecolor{currentstroke}{rgb}{0.000000,0.000000,0.000000}%
\pgfsetstrokecolor{currentstroke}%
\pgfsetdash{}{0pt}%
\pgfpathmoveto{\pgfqpoint{0.485894in}{1.534611in}}%
\pgfpathlineto{\pgfqpoint{0.635797in}{1.534611in}}%
\pgfpathlineto{\pgfqpoint{0.635797in}{1.792542in}}%
\pgfpathlineto{\pgfqpoint{0.485894in}{1.792542in}}%
\pgfpathlineto{\pgfqpoint{0.485894in}{1.534611in}}%
\pgfpathclose%
\pgfusepath{stroke,fill}%
\end{pgfscope}%
\begin{pgfscope}%
\pgfpathrectangle{\pgfqpoint{0.148611in}{1.534611in}}{\pgfqpoint{0.824468in}{0.462000in}}%
\pgfusepath{clip}%
\pgfsetbuttcap%
\pgfsetmiterjoin%
\definecolor{currentfill}{rgb}{0.121569,0.466667,0.705882}%
\pgfsetfillcolor{currentfill}%
\pgfsetfillopacity{0.500000}%
\pgfsetlinewidth{1.003750pt}%
\definecolor{currentstroke}{rgb}{0.000000,0.000000,0.000000}%
\pgfsetstrokecolor{currentstroke}%
\pgfsetdash{}{0pt}%
\pgfpathmoveto{\pgfqpoint{0.635797in}{1.534611in}}%
\pgfpathlineto{\pgfqpoint{0.785700in}{1.534611in}}%
\pgfpathlineto{\pgfqpoint{0.785700in}{1.974611in}}%
\pgfpathlineto{\pgfqpoint{0.635797in}{1.974611in}}%
\pgfpathlineto{\pgfqpoint{0.635797in}{1.534611in}}%
\pgfpathclose%
\pgfusepath{stroke,fill}%
\end{pgfscope}%
\begin{pgfscope}%
\pgfpathrectangle{\pgfqpoint{0.148611in}{1.534611in}}{\pgfqpoint{0.824468in}{0.462000in}}%
\pgfusepath{clip}%
\pgfsetbuttcap%
\pgfsetmiterjoin%
\definecolor{currentfill}{rgb}{0.121569,0.466667,0.705882}%
\pgfsetfillcolor{currentfill}%
\pgfsetfillopacity{0.500000}%
\pgfsetlinewidth{1.003750pt}%
\definecolor{currentstroke}{rgb}{0.000000,0.000000,0.000000}%
\pgfsetstrokecolor{currentstroke}%
\pgfsetdash{}{0pt}%
\pgfpathmoveto{\pgfqpoint{0.785700in}{1.534611in}}%
\pgfpathlineto{\pgfqpoint{0.935603in}{1.534611in}}%
\pgfpathlineto{\pgfqpoint{0.935603in}{1.784956in}}%
\pgfpathlineto{\pgfqpoint{0.785700in}{1.784956in}}%
\pgfpathlineto{\pgfqpoint{0.785700in}{1.534611in}}%
\pgfpathclose%
\pgfusepath{stroke,fill}%
\end{pgfscope}%
\begin{pgfscope}%
\pgfsetrectcap%
\pgfsetmiterjoin%
\pgfsetlinewidth{0.803000pt}%
\definecolor{currentstroke}{rgb}{0.000000,0.000000,0.000000}%
\pgfsetstrokecolor{currentstroke}%
\pgfsetdash{}{0pt}%
\pgfpathmoveto{\pgfqpoint{0.148611in}{1.534611in}}%
\pgfpathlineto{\pgfqpoint{0.148611in}{1.996611in}}%
\pgfusepath{stroke}%
\end{pgfscope}%
\begin{pgfscope}%
\pgfsetrectcap%
\pgfsetmiterjoin%
\pgfsetlinewidth{0.803000pt}%
\definecolor{currentstroke}{rgb}{0.000000,0.000000,0.000000}%
\pgfsetstrokecolor{currentstroke}%
\pgfsetdash{}{0pt}%
\pgfpathmoveto{\pgfqpoint{0.973079in}{1.534611in}}%
\pgfpathlineto{\pgfqpoint{0.973079in}{1.996611in}}%
\pgfusepath{stroke}%
\end{pgfscope}%
\begin{pgfscope}%
\pgfsetrectcap%
\pgfsetmiterjoin%
\pgfsetlinewidth{0.803000pt}%
\definecolor{currentstroke}{rgb}{0.000000,0.000000,0.000000}%
\pgfsetstrokecolor{currentstroke}%
\pgfsetdash{}{0pt}%
\pgfpathmoveto{\pgfqpoint{0.148611in}{1.534611in}}%
\pgfpathlineto{\pgfqpoint{0.973079in}{1.534611in}}%
\pgfusepath{stroke}%
\end{pgfscope}%
\begin{pgfscope}%
\pgfsetrectcap%
\pgfsetmiterjoin%
\pgfsetlinewidth{0.803000pt}%
\definecolor{currentstroke}{rgb}{0.000000,0.000000,0.000000}%
\pgfsetstrokecolor{currentstroke}%
\pgfsetdash{}{0pt}%
\pgfpathmoveto{\pgfqpoint{0.148611in}{1.996611in}}%
\pgfpathlineto{\pgfqpoint{0.973079in}{1.996611in}}%
\pgfusepath{stroke}%
\end{pgfscope}%
\begin{pgfscope}%
\definecolor{textcolor}{rgb}{0.000000,0.000000,0.000000}%
\pgfsetstrokecolor{textcolor}%
\pgfsetfillcolor{textcolor}%
\pgftext[x=0.560845in,y=2.079944in,,base]{\color{textcolor}\rmfamily\fontsize{11.000000}{13.200000}\selectfont MGP}%
\end{pgfscope}%
\begin{pgfscope}%
\pgfsetbuttcap%
\pgfsetmiterjoin%
\definecolor{currentfill}{rgb}{1.000000,1.000000,1.000000}%
\pgfsetfillcolor{currentfill}%
\pgfsetlinewidth{0.000000pt}%
\definecolor{currentstroke}{rgb}{0.000000,0.000000,0.000000}%
\pgfsetstrokecolor{currentstroke}%
\pgfsetstrokeopacity{0.000000}%
\pgfsetdash{}{0pt}%
\pgfpathmoveto{\pgfqpoint{1.137973in}{1.534611in}}%
\pgfpathlineto{\pgfqpoint{1.962441in}{1.534611in}}%
\pgfpathlineto{\pgfqpoint{1.962441in}{1.996611in}}%
\pgfpathlineto{\pgfqpoint{1.137973in}{1.996611in}}%
\pgfpathlineto{\pgfqpoint{1.137973in}{1.534611in}}%
\pgfpathclose%
\pgfusepath{fill}%
\end{pgfscope}%
\begin{pgfscope}%
\pgfpathrectangle{\pgfqpoint{1.137973in}{1.534611in}}{\pgfqpoint{0.824468in}{0.462000in}}%
\pgfusepath{clip}%
\pgfsetbuttcap%
\pgfsetmiterjoin%
\definecolor{currentfill}{rgb}{0.121569,0.466667,0.705882}%
\pgfsetfillcolor{currentfill}%
\pgfsetfillopacity{0.500000}%
\pgfsetlinewidth{1.003750pt}%
\definecolor{currentstroke}{rgb}{0.000000,0.000000,0.000000}%
\pgfsetstrokecolor{currentstroke}%
\pgfsetdash{}{0pt}%
\pgfpathmoveto{\pgfqpoint{1.175449in}{1.534611in}}%
\pgfpathlineto{\pgfqpoint{1.325352in}{1.534611in}}%
\pgfpathlineto{\pgfqpoint{1.325352in}{1.974611in}}%
\pgfpathlineto{\pgfqpoint{1.175449in}{1.974611in}}%
\pgfpathlineto{\pgfqpoint{1.175449in}{1.534611in}}%
\pgfpathclose%
\pgfusepath{stroke,fill}%
\end{pgfscope}%
\begin{pgfscope}%
\pgfpathrectangle{\pgfqpoint{1.137973in}{1.534611in}}{\pgfqpoint{0.824468in}{0.462000in}}%
\pgfusepath{clip}%
\pgfsetbuttcap%
\pgfsetmiterjoin%
\definecolor{currentfill}{rgb}{0.121569,0.466667,0.705882}%
\pgfsetfillcolor{currentfill}%
\pgfsetfillopacity{0.500000}%
\pgfsetlinewidth{1.003750pt}%
\definecolor{currentstroke}{rgb}{0.000000,0.000000,0.000000}%
\pgfsetstrokecolor{currentstroke}%
\pgfsetdash{}{0pt}%
\pgfpathmoveto{\pgfqpoint{1.325352in}{1.534611in}}%
\pgfpathlineto{\pgfqpoint{1.475255in}{1.534611in}}%
\pgfpathlineto{\pgfqpoint{1.475255in}{1.669056in}}%
\pgfpathlineto{\pgfqpoint{1.325352in}{1.669056in}}%
\pgfpathlineto{\pgfqpoint{1.325352in}{1.534611in}}%
\pgfpathclose%
\pgfusepath{stroke,fill}%
\end{pgfscope}%
\begin{pgfscope}%
\pgfpathrectangle{\pgfqpoint{1.137973in}{1.534611in}}{\pgfqpoint{0.824468in}{0.462000in}}%
\pgfusepath{clip}%
\pgfsetbuttcap%
\pgfsetmiterjoin%
\definecolor{currentfill}{rgb}{0.121569,0.466667,0.705882}%
\pgfsetfillcolor{currentfill}%
\pgfsetfillopacity{0.500000}%
\pgfsetlinewidth{1.003750pt}%
\definecolor{currentstroke}{rgb}{0.000000,0.000000,0.000000}%
\pgfsetstrokecolor{currentstroke}%
\pgfsetdash{}{0pt}%
\pgfpathmoveto{\pgfqpoint{1.475255in}{1.534611in}}%
\pgfpathlineto{\pgfqpoint{1.625158in}{1.534611in}}%
\pgfpathlineto{\pgfqpoint{1.625158in}{1.656833in}}%
\pgfpathlineto{\pgfqpoint{1.475255in}{1.656833in}}%
\pgfpathlineto{\pgfqpoint{1.475255in}{1.534611in}}%
\pgfpathclose%
\pgfusepath{stroke,fill}%
\end{pgfscope}%
\begin{pgfscope}%
\pgfpathrectangle{\pgfqpoint{1.137973in}{1.534611in}}{\pgfqpoint{0.824468in}{0.462000in}}%
\pgfusepath{clip}%
\pgfsetbuttcap%
\pgfsetmiterjoin%
\definecolor{currentfill}{rgb}{0.121569,0.466667,0.705882}%
\pgfsetfillcolor{currentfill}%
\pgfsetfillopacity{0.500000}%
\pgfsetlinewidth{1.003750pt}%
\definecolor{currentstroke}{rgb}{0.000000,0.000000,0.000000}%
\pgfsetstrokecolor{currentstroke}%
\pgfsetdash{}{0pt}%
\pgfpathmoveto{\pgfqpoint{1.625158in}{1.534611in}}%
\pgfpathlineto{\pgfqpoint{1.775062in}{1.534611in}}%
\pgfpathlineto{\pgfqpoint{1.775062in}{1.595722in}}%
\pgfpathlineto{\pgfqpoint{1.625158in}{1.595722in}}%
\pgfpathlineto{\pgfqpoint{1.625158in}{1.534611in}}%
\pgfpathclose%
\pgfusepath{stroke,fill}%
\end{pgfscope}%
\begin{pgfscope}%
\pgfpathrectangle{\pgfqpoint{1.137973in}{1.534611in}}{\pgfqpoint{0.824468in}{0.462000in}}%
\pgfusepath{clip}%
\pgfsetbuttcap%
\pgfsetmiterjoin%
\definecolor{currentfill}{rgb}{0.121569,0.466667,0.705882}%
\pgfsetfillcolor{currentfill}%
\pgfsetfillopacity{0.500000}%
\pgfsetlinewidth{1.003750pt}%
\definecolor{currentstroke}{rgb}{0.000000,0.000000,0.000000}%
\pgfsetstrokecolor{currentstroke}%
\pgfsetdash{}{0pt}%
\pgfpathmoveto{\pgfqpoint{1.775062in}{1.534611in}}%
\pgfpathlineto{\pgfqpoint{1.924965in}{1.534611in}}%
\pgfpathlineto{\pgfqpoint{1.924965in}{1.546833in}}%
\pgfpathlineto{\pgfqpoint{1.775062in}{1.546833in}}%
\pgfpathlineto{\pgfqpoint{1.775062in}{1.534611in}}%
\pgfpathclose%
\pgfusepath{stroke,fill}%
\end{pgfscope}%
\begin{pgfscope}%
\pgfsetrectcap%
\pgfsetmiterjoin%
\pgfsetlinewidth{0.803000pt}%
\definecolor{currentstroke}{rgb}{0.000000,0.000000,0.000000}%
\pgfsetstrokecolor{currentstroke}%
\pgfsetdash{}{0pt}%
\pgfpathmoveto{\pgfqpoint{1.137973in}{1.534611in}}%
\pgfpathlineto{\pgfqpoint{1.137973in}{1.996611in}}%
\pgfusepath{stroke}%
\end{pgfscope}%
\begin{pgfscope}%
\pgfsetrectcap%
\pgfsetmiterjoin%
\pgfsetlinewidth{0.803000pt}%
\definecolor{currentstroke}{rgb}{0.000000,0.000000,0.000000}%
\pgfsetstrokecolor{currentstroke}%
\pgfsetdash{}{0pt}%
\pgfpathmoveto{\pgfqpoint{1.962441in}{1.534611in}}%
\pgfpathlineto{\pgfqpoint{1.962441in}{1.996611in}}%
\pgfusepath{stroke}%
\end{pgfscope}%
\begin{pgfscope}%
\pgfsetrectcap%
\pgfsetmiterjoin%
\pgfsetlinewidth{0.803000pt}%
\definecolor{currentstroke}{rgb}{0.000000,0.000000,0.000000}%
\pgfsetstrokecolor{currentstroke}%
\pgfsetdash{}{0pt}%
\pgfpathmoveto{\pgfqpoint{1.137973in}{1.534611in}}%
\pgfpathlineto{\pgfqpoint{1.962441in}{1.534611in}}%
\pgfusepath{stroke}%
\end{pgfscope}%
\begin{pgfscope}%
\pgfsetrectcap%
\pgfsetmiterjoin%
\pgfsetlinewidth{0.803000pt}%
\definecolor{currentstroke}{rgb}{0.000000,0.000000,0.000000}%
\pgfsetstrokecolor{currentstroke}%
\pgfsetdash{}{0pt}%
\pgfpathmoveto{\pgfqpoint{1.137973in}{1.996611in}}%
\pgfpathlineto{\pgfqpoint{1.962441in}{1.996611in}}%
\pgfusepath{stroke}%
\end{pgfscope}%
\begin{pgfscope}%
\definecolor{textcolor}{rgb}{0.000000,0.000000,0.000000}%
\pgfsetstrokecolor{textcolor}%
\pgfsetfillcolor{textcolor}%
\pgftext[x=1.550207in,y=2.079944in,,base]{\color{textcolor}\rmfamily\fontsize{11.000000}{13.200000}\selectfont Intériale}%
\end{pgfscope}%
\begin{pgfscope}%
\pgfsetbuttcap%
\pgfsetmiterjoin%
\definecolor{currentfill}{rgb}{1.000000,1.000000,1.000000}%
\pgfsetfillcolor{currentfill}%
\pgfsetlinewidth{0.000000pt}%
\definecolor{currentstroke}{rgb}{0.000000,0.000000,0.000000}%
\pgfsetstrokecolor{currentstroke}%
\pgfsetstrokeopacity{0.000000}%
\pgfsetdash{}{0pt}%
\pgfpathmoveto{\pgfqpoint{2.127335in}{1.534611in}}%
\pgfpathlineto{\pgfqpoint{2.951803in}{1.534611in}}%
\pgfpathlineto{\pgfqpoint{2.951803in}{1.996611in}}%
\pgfpathlineto{\pgfqpoint{2.127335in}{1.996611in}}%
\pgfpathlineto{\pgfqpoint{2.127335in}{1.534611in}}%
\pgfpathclose%
\pgfusepath{fill}%
\end{pgfscope}%
\begin{pgfscope}%
\pgfpathrectangle{\pgfqpoint{2.127335in}{1.534611in}}{\pgfqpoint{0.824468in}{0.462000in}}%
\pgfusepath{clip}%
\pgfsetbuttcap%
\pgfsetmiterjoin%
\definecolor{currentfill}{rgb}{0.121569,0.466667,0.705882}%
\pgfsetfillcolor{currentfill}%
\pgfsetfillopacity{0.500000}%
\pgfsetlinewidth{1.003750pt}%
\definecolor{currentstroke}{rgb}{0.000000,0.000000,0.000000}%
\pgfsetstrokecolor{currentstroke}%
\pgfsetdash{}{0pt}%
\pgfpathmoveto{\pgfqpoint{2.164810in}{1.534611in}}%
\pgfpathlineto{\pgfqpoint{2.314714in}{1.534611in}}%
\pgfpathlineto{\pgfqpoint{2.314714in}{1.974611in}}%
\pgfpathlineto{\pgfqpoint{2.164810in}{1.974611in}}%
\pgfpathlineto{\pgfqpoint{2.164810in}{1.534611in}}%
\pgfpathclose%
\pgfusepath{stroke,fill}%
\end{pgfscope}%
\begin{pgfscope}%
\pgfpathrectangle{\pgfqpoint{2.127335in}{1.534611in}}{\pgfqpoint{0.824468in}{0.462000in}}%
\pgfusepath{clip}%
\pgfsetbuttcap%
\pgfsetmiterjoin%
\definecolor{currentfill}{rgb}{0.121569,0.466667,0.705882}%
\pgfsetfillcolor{currentfill}%
\pgfsetfillopacity{0.500000}%
\pgfsetlinewidth{1.003750pt}%
\definecolor{currentstroke}{rgb}{0.000000,0.000000,0.000000}%
\pgfsetstrokecolor{currentstroke}%
\pgfsetdash{}{0pt}%
\pgfpathmoveto{\pgfqpoint{2.314714in}{1.534611in}}%
\pgfpathlineto{\pgfqpoint{2.464617in}{1.534611in}}%
\pgfpathlineto{\pgfqpoint{2.464617in}{1.637821in}}%
\pgfpathlineto{\pgfqpoint{2.314714in}{1.637821in}}%
\pgfpathlineto{\pgfqpoint{2.314714in}{1.534611in}}%
\pgfpathclose%
\pgfusepath{stroke,fill}%
\end{pgfscope}%
\begin{pgfscope}%
\pgfpathrectangle{\pgfqpoint{2.127335in}{1.534611in}}{\pgfqpoint{0.824468in}{0.462000in}}%
\pgfusepath{clip}%
\pgfsetbuttcap%
\pgfsetmiterjoin%
\definecolor{currentfill}{rgb}{0.121569,0.466667,0.705882}%
\pgfsetfillcolor{currentfill}%
\pgfsetfillopacity{0.500000}%
\pgfsetlinewidth{1.003750pt}%
\definecolor{currentstroke}{rgb}{0.000000,0.000000,0.000000}%
\pgfsetstrokecolor{currentstroke}%
\pgfsetdash{}{0pt}%
\pgfpathmoveto{\pgfqpoint{2.464617in}{1.534611in}}%
\pgfpathlineto{\pgfqpoint{2.614520in}{1.534611in}}%
\pgfpathlineto{\pgfqpoint{2.614520in}{1.659549in}}%
\pgfpathlineto{\pgfqpoint{2.464617in}{1.659549in}}%
\pgfpathlineto{\pgfqpoint{2.464617in}{1.534611in}}%
\pgfpathclose%
\pgfusepath{stroke,fill}%
\end{pgfscope}%
\begin{pgfscope}%
\pgfpathrectangle{\pgfqpoint{2.127335in}{1.534611in}}{\pgfqpoint{0.824468in}{0.462000in}}%
\pgfusepath{clip}%
\pgfsetbuttcap%
\pgfsetmiterjoin%
\definecolor{currentfill}{rgb}{0.121569,0.466667,0.705882}%
\pgfsetfillcolor{currentfill}%
\pgfsetfillopacity{0.500000}%
\pgfsetlinewidth{1.003750pt}%
\definecolor{currentstroke}{rgb}{0.000000,0.000000,0.000000}%
\pgfsetstrokecolor{currentstroke}%
\pgfsetdash{}{0pt}%
\pgfpathmoveto{\pgfqpoint{2.614520in}{1.534611in}}%
\pgfpathlineto{\pgfqpoint{2.764423in}{1.534611in}}%
\pgfpathlineto{\pgfqpoint{2.764423in}{1.806216in}}%
\pgfpathlineto{\pgfqpoint{2.614520in}{1.806216in}}%
\pgfpathlineto{\pgfqpoint{2.614520in}{1.534611in}}%
\pgfpathclose%
\pgfusepath{stroke,fill}%
\end{pgfscope}%
\begin{pgfscope}%
\pgfpathrectangle{\pgfqpoint{2.127335in}{1.534611in}}{\pgfqpoint{0.824468in}{0.462000in}}%
\pgfusepath{clip}%
\pgfsetbuttcap%
\pgfsetmiterjoin%
\definecolor{currentfill}{rgb}{0.121569,0.466667,0.705882}%
\pgfsetfillcolor{currentfill}%
\pgfsetfillopacity{0.500000}%
\pgfsetlinewidth{1.003750pt}%
\definecolor{currentstroke}{rgb}{0.000000,0.000000,0.000000}%
\pgfsetstrokecolor{currentstroke}%
\pgfsetdash{}{0pt}%
\pgfpathmoveto{\pgfqpoint{2.764423in}{1.534611in}}%
\pgfpathlineto{\pgfqpoint{2.914327in}{1.534611in}}%
\pgfpathlineto{\pgfqpoint{2.914327in}{1.746463in}}%
\pgfpathlineto{\pgfqpoint{2.764423in}{1.746463in}}%
\pgfpathlineto{\pgfqpoint{2.764423in}{1.534611in}}%
\pgfpathclose%
\pgfusepath{stroke,fill}%
\end{pgfscope}%
\begin{pgfscope}%
\pgfsetrectcap%
\pgfsetmiterjoin%
\pgfsetlinewidth{0.803000pt}%
\definecolor{currentstroke}{rgb}{0.000000,0.000000,0.000000}%
\pgfsetstrokecolor{currentstroke}%
\pgfsetdash{}{0pt}%
\pgfpathmoveto{\pgfqpoint{2.127335in}{1.534611in}}%
\pgfpathlineto{\pgfqpoint{2.127335in}{1.996611in}}%
\pgfusepath{stroke}%
\end{pgfscope}%
\begin{pgfscope}%
\pgfsetrectcap%
\pgfsetmiterjoin%
\pgfsetlinewidth{0.803000pt}%
\definecolor{currentstroke}{rgb}{0.000000,0.000000,0.000000}%
\pgfsetstrokecolor{currentstroke}%
\pgfsetdash{}{0pt}%
\pgfpathmoveto{\pgfqpoint{2.951803in}{1.534611in}}%
\pgfpathlineto{\pgfqpoint{2.951803in}{1.996611in}}%
\pgfusepath{stroke}%
\end{pgfscope}%
\begin{pgfscope}%
\pgfsetrectcap%
\pgfsetmiterjoin%
\pgfsetlinewidth{0.803000pt}%
\definecolor{currentstroke}{rgb}{0.000000,0.000000,0.000000}%
\pgfsetstrokecolor{currentstroke}%
\pgfsetdash{}{0pt}%
\pgfpathmoveto{\pgfqpoint{2.127335in}{1.534611in}}%
\pgfpathlineto{\pgfqpoint{2.951803in}{1.534611in}}%
\pgfusepath{stroke}%
\end{pgfscope}%
\begin{pgfscope}%
\pgfsetrectcap%
\pgfsetmiterjoin%
\pgfsetlinewidth{0.803000pt}%
\definecolor{currentstroke}{rgb}{0.000000,0.000000,0.000000}%
\pgfsetstrokecolor{currentstroke}%
\pgfsetdash{}{0pt}%
\pgfpathmoveto{\pgfqpoint{2.127335in}{1.996611in}}%
\pgfpathlineto{\pgfqpoint{2.951803in}{1.996611in}}%
\pgfusepath{stroke}%
\end{pgfscope}%
\begin{pgfscope}%
\definecolor{textcolor}{rgb}{0.000000,0.000000,0.000000}%
\pgfsetstrokecolor{textcolor}%
\pgfsetfillcolor{textcolor}%
\pgftext[x=2.539569in,y=2.079944in,,base]{\color{textcolor}\rmfamily\fontsize{11.000000}{13.200000}\selectfont Généra...}%
\end{pgfscope}%
\begin{pgfscope}%
\pgfsetbuttcap%
\pgfsetmiterjoin%
\definecolor{currentfill}{rgb}{1.000000,1.000000,1.000000}%
\pgfsetfillcolor{currentfill}%
\pgfsetlinewidth{0.000000pt}%
\definecolor{currentstroke}{rgb}{0.000000,0.000000,0.000000}%
\pgfsetstrokecolor{currentstroke}%
\pgfsetstrokeopacity{0.000000}%
\pgfsetdash{}{0pt}%
\pgfpathmoveto{\pgfqpoint{3.116696in}{1.534611in}}%
\pgfpathlineto{\pgfqpoint{3.941164in}{1.534611in}}%
\pgfpathlineto{\pgfqpoint{3.941164in}{1.996611in}}%
\pgfpathlineto{\pgfqpoint{3.116696in}{1.996611in}}%
\pgfpathlineto{\pgfqpoint{3.116696in}{1.534611in}}%
\pgfpathclose%
\pgfusepath{fill}%
\end{pgfscope}%
\begin{pgfscope}%
\pgfpathrectangle{\pgfqpoint{3.116696in}{1.534611in}}{\pgfqpoint{0.824468in}{0.462000in}}%
\pgfusepath{clip}%
\pgfsetbuttcap%
\pgfsetmiterjoin%
\definecolor{currentfill}{rgb}{0.121569,0.466667,0.705882}%
\pgfsetfillcolor{currentfill}%
\pgfsetfillopacity{0.500000}%
\pgfsetlinewidth{1.003750pt}%
\definecolor{currentstroke}{rgb}{0.000000,0.000000,0.000000}%
\pgfsetstrokecolor{currentstroke}%
\pgfsetdash{}{0pt}%
\pgfpathmoveto{\pgfqpoint{3.154172in}{1.534611in}}%
\pgfpathlineto{\pgfqpoint{3.304075in}{1.534611in}}%
\pgfpathlineto{\pgfqpoint{3.304075in}{1.974611in}}%
\pgfpathlineto{\pgfqpoint{3.154172in}{1.974611in}}%
\pgfpathlineto{\pgfqpoint{3.154172in}{1.534611in}}%
\pgfpathclose%
\pgfusepath{stroke,fill}%
\end{pgfscope}%
\begin{pgfscope}%
\pgfpathrectangle{\pgfqpoint{3.116696in}{1.534611in}}{\pgfqpoint{0.824468in}{0.462000in}}%
\pgfusepath{clip}%
\pgfsetbuttcap%
\pgfsetmiterjoin%
\definecolor{currentfill}{rgb}{0.121569,0.466667,0.705882}%
\pgfsetfillcolor{currentfill}%
\pgfsetfillopacity{0.500000}%
\pgfsetlinewidth{1.003750pt}%
\definecolor{currentstroke}{rgb}{0.000000,0.000000,0.000000}%
\pgfsetstrokecolor{currentstroke}%
\pgfsetdash{}{0pt}%
\pgfpathmoveto{\pgfqpoint{3.304075in}{1.534611in}}%
\pgfpathlineto{\pgfqpoint{3.453979in}{1.534611in}}%
\pgfpathlineto{\pgfqpoint{3.453979in}{1.593278in}}%
\pgfpathlineto{\pgfqpoint{3.304075in}{1.593278in}}%
\pgfpathlineto{\pgfqpoint{3.304075in}{1.534611in}}%
\pgfpathclose%
\pgfusepath{stroke,fill}%
\end{pgfscope}%
\begin{pgfscope}%
\pgfpathrectangle{\pgfqpoint{3.116696in}{1.534611in}}{\pgfqpoint{0.824468in}{0.462000in}}%
\pgfusepath{clip}%
\pgfsetbuttcap%
\pgfsetmiterjoin%
\definecolor{currentfill}{rgb}{0.121569,0.466667,0.705882}%
\pgfsetfillcolor{currentfill}%
\pgfsetfillopacity{0.500000}%
\pgfsetlinewidth{1.003750pt}%
\definecolor{currentstroke}{rgb}{0.000000,0.000000,0.000000}%
\pgfsetstrokecolor{currentstroke}%
\pgfsetdash{}{0pt}%
\pgfpathmoveto{\pgfqpoint{3.453979in}{1.534611in}}%
\pgfpathlineto{\pgfqpoint{3.603882in}{1.534611in}}%
\pgfpathlineto{\pgfqpoint{3.603882in}{1.584897in}}%
\pgfpathlineto{\pgfqpoint{3.453979in}{1.584897in}}%
\pgfpathlineto{\pgfqpoint{3.453979in}{1.534611in}}%
\pgfpathclose%
\pgfusepath{stroke,fill}%
\end{pgfscope}%
\begin{pgfscope}%
\pgfpathrectangle{\pgfqpoint{3.116696in}{1.534611in}}{\pgfqpoint{0.824468in}{0.462000in}}%
\pgfusepath{clip}%
\pgfsetbuttcap%
\pgfsetmiterjoin%
\definecolor{currentfill}{rgb}{0.121569,0.466667,0.705882}%
\pgfsetfillcolor{currentfill}%
\pgfsetfillopacity{0.500000}%
\pgfsetlinewidth{1.003750pt}%
\definecolor{currentstroke}{rgb}{0.000000,0.000000,0.000000}%
\pgfsetstrokecolor{currentstroke}%
\pgfsetdash{}{0pt}%
\pgfpathmoveto{\pgfqpoint{3.603882in}{1.534611in}}%
\pgfpathlineto{\pgfqpoint{3.753785in}{1.534611in}}%
\pgfpathlineto{\pgfqpoint{3.753785in}{1.540897in}}%
\pgfpathlineto{\pgfqpoint{3.603882in}{1.540897in}}%
\pgfpathlineto{\pgfqpoint{3.603882in}{1.534611in}}%
\pgfpathclose%
\pgfusepath{stroke,fill}%
\end{pgfscope}%
\begin{pgfscope}%
\pgfpathrectangle{\pgfqpoint{3.116696in}{1.534611in}}{\pgfqpoint{0.824468in}{0.462000in}}%
\pgfusepath{clip}%
\pgfsetbuttcap%
\pgfsetmiterjoin%
\definecolor{currentfill}{rgb}{0.121569,0.466667,0.705882}%
\pgfsetfillcolor{currentfill}%
\pgfsetfillopacity{0.500000}%
\pgfsetlinewidth{1.003750pt}%
\definecolor{currentstroke}{rgb}{0.000000,0.000000,0.000000}%
\pgfsetstrokecolor{currentstroke}%
\pgfsetdash{}{0pt}%
\pgfpathmoveto{\pgfqpoint{3.753785in}{1.534611in}}%
\pgfpathlineto{\pgfqpoint{3.903688in}{1.534611in}}%
\pgfpathlineto{\pgfqpoint{3.903688in}{1.534611in}}%
\pgfpathlineto{\pgfqpoint{3.753785in}{1.534611in}}%
\pgfpathlineto{\pgfqpoint{3.753785in}{1.534611in}}%
\pgfpathclose%
\pgfusepath{stroke,fill}%
\end{pgfscope}%
\begin{pgfscope}%
\pgfsetrectcap%
\pgfsetmiterjoin%
\pgfsetlinewidth{0.803000pt}%
\definecolor{currentstroke}{rgb}{0.000000,0.000000,0.000000}%
\pgfsetstrokecolor{currentstroke}%
\pgfsetdash{}{0pt}%
\pgfpathmoveto{\pgfqpoint{3.116696in}{1.534611in}}%
\pgfpathlineto{\pgfqpoint{3.116696in}{1.996611in}}%
\pgfusepath{stroke}%
\end{pgfscope}%
\begin{pgfscope}%
\pgfsetrectcap%
\pgfsetmiterjoin%
\pgfsetlinewidth{0.803000pt}%
\definecolor{currentstroke}{rgb}{0.000000,0.000000,0.000000}%
\pgfsetstrokecolor{currentstroke}%
\pgfsetdash{}{0pt}%
\pgfpathmoveto{\pgfqpoint{3.941164in}{1.534611in}}%
\pgfpathlineto{\pgfqpoint{3.941164in}{1.996611in}}%
\pgfusepath{stroke}%
\end{pgfscope}%
\begin{pgfscope}%
\pgfsetrectcap%
\pgfsetmiterjoin%
\pgfsetlinewidth{0.803000pt}%
\definecolor{currentstroke}{rgb}{0.000000,0.000000,0.000000}%
\pgfsetstrokecolor{currentstroke}%
\pgfsetdash{}{0pt}%
\pgfpathmoveto{\pgfqpoint{3.116696in}{1.534611in}}%
\pgfpathlineto{\pgfqpoint{3.941164in}{1.534611in}}%
\pgfusepath{stroke}%
\end{pgfscope}%
\begin{pgfscope}%
\pgfsetrectcap%
\pgfsetmiterjoin%
\pgfsetlinewidth{0.803000pt}%
\definecolor{currentstroke}{rgb}{0.000000,0.000000,0.000000}%
\pgfsetstrokecolor{currentstroke}%
\pgfsetdash{}{0pt}%
\pgfpathmoveto{\pgfqpoint{3.116696in}{1.996611in}}%
\pgfpathlineto{\pgfqpoint{3.941164in}{1.996611in}}%
\pgfusepath{stroke}%
\end{pgfscope}%
\begin{pgfscope}%
\definecolor{textcolor}{rgb}{0.000000,0.000000,0.000000}%
\pgfsetstrokecolor{textcolor}%
\pgfsetfillcolor{textcolor}%
\pgftext[x=3.528930in,y=2.079944in,,base]{\color{textcolor}\rmfamily\fontsize{11.000000}{13.200000}\selectfont Cardif}%
\end{pgfscope}%
\begin{pgfscope}%
\pgfsetbuttcap%
\pgfsetmiterjoin%
\definecolor{currentfill}{rgb}{1.000000,1.000000,1.000000}%
\pgfsetfillcolor{currentfill}%
\pgfsetlinewidth{0.000000pt}%
\definecolor{currentstroke}{rgb}{0.000000,0.000000,0.000000}%
\pgfsetstrokecolor{currentstroke}%
\pgfsetstrokeopacity{0.000000}%
\pgfsetdash{}{0pt}%
\pgfpathmoveto{\pgfqpoint{4.106058in}{1.534611in}}%
\pgfpathlineto{\pgfqpoint{4.930526in}{1.534611in}}%
\pgfpathlineto{\pgfqpoint{4.930526in}{1.996611in}}%
\pgfpathlineto{\pgfqpoint{4.106058in}{1.996611in}}%
\pgfpathlineto{\pgfqpoint{4.106058in}{1.534611in}}%
\pgfpathclose%
\pgfusepath{fill}%
\end{pgfscope}%
\begin{pgfscope}%
\pgfpathrectangle{\pgfqpoint{4.106058in}{1.534611in}}{\pgfqpoint{0.824468in}{0.462000in}}%
\pgfusepath{clip}%
\pgfsetbuttcap%
\pgfsetmiterjoin%
\definecolor{currentfill}{rgb}{0.121569,0.466667,0.705882}%
\pgfsetfillcolor{currentfill}%
\pgfsetfillopacity{0.500000}%
\pgfsetlinewidth{1.003750pt}%
\definecolor{currentstroke}{rgb}{0.000000,0.000000,0.000000}%
\pgfsetstrokecolor{currentstroke}%
\pgfsetdash{}{0pt}%
\pgfpathmoveto{\pgfqpoint{4.143534in}{1.534611in}}%
\pgfpathlineto{\pgfqpoint{4.293437in}{1.534611in}}%
\pgfpathlineto{\pgfqpoint{4.293437in}{1.706961in}}%
\pgfpathlineto{\pgfqpoint{4.143534in}{1.706961in}}%
\pgfpathlineto{\pgfqpoint{4.143534in}{1.534611in}}%
\pgfpathclose%
\pgfusepath{stroke,fill}%
\end{pgfscope}%
\begin{pgfscope}%
\pgfpathrectangle{\pgfqpoint{4.106058in}{1.534611in}}{\pgfqpoint{0.824468in}{0.462000in}}%
\pgfusepath{clip}%
\pgfsetbuttcap%
\pgfsetmiterjoin%
\definecolor{currentfill}{rgb}{0.121569,0.466667,0.705882}%
\pgfsetfillcolor{currentfill}%
\pgfsetfillopacity{0.500000}%
\pgfsetlinewidth{1.003750pt}%
\definecolor{currentstroke}{rgb}{0.000000,0.000000,0.000000}%
\pgfsetstrokecolor{currentstroke}%
\pgfsetdash{}{0pt}%
\pgfpathmoveto{\pgfqpoint{4.293437in}{1.534611in}}%
\pgfpathlineto{\pgfqpoint{4.443340in}{1.534611in}}%
\pgfpathlineto{\pgfqpoint{4.443340in}{1.650187in}}%
\pgfpathlineto{\pgfqpoint{4.293437in}{1.650187in}}%
\pgfpathlineto{\pgfqpoint{4.293437in}{1.534611in}}%
\pgfpathclose%
\pgfusepath{stroke,fill}%
\end{pgfscope}%
\begin{pgfscope}%
\pgfpathrectangle{\pgfqpoint{4.106058in}{1.534611in}}{\pgfqpoint{0.824468in}{0.462000in}}%
\pgfusepath{clip}%
\pgfsetbuttcap%
\pgfsetmiterjoin%
\definecolor{currentfill}{rgb}{0.121569,0.466667,0.705882}%
\pgfsetfillcolor{currentfill}%
\pgfsetfillopacity{0.500000}%
\pgfsetlinewidth{1.003750pt}%
\definecolor{currentstroke}{rgb}{0.000000,0.000000,0.000000}%
\pgfsetstrokecolor{currentstroke}%
\pgfsetdash{}{0pt}%
\pgfpathmoveto{\pgfqpoint{4.443340in}{1.534611in}}%
\pgfpathlineto{\pgfqpoint{4.593244in}{1.534611in}}%
\pgfpathlineto{\pgfqpoint{4.593244in}{1.818482in}}%
\pgfpathlineto{\pgfqpoint{4.443340in}{1.818482in}}%
\pgfpathlineto{\pgfqpoint{4.443340in}{1.534611in}}%
\pgfpathclose%
\pgfusepath{stroke,fill}%
\end{pgfscope}%
\begin{pgfscope}%
\pgfpathrectangle{\pgfqpoint{4.106058in}{1.534611in}}{\pgfqpoint{0.824468in}{0.462000in}}%
\pgfusepath{clip}%
\pgfsetbuttcap%
\pgfsetmiterjoin%
\definecolor{currentfill}{rgb}{0.121569,0.466667,0.705882}%
\pgfsetfillcolor{currentfill}%
\pgfsetfillopacity{0.500000}%
\pgfsetlinewidth{1.003750pt}%
\definecolor{currentstroke}{rgb}{0.000000,0.000000,0.000000}%
\pgfsetstrokecolor{currentstroke}%
\pgfsetdash{}{0pt}%
\pgfpathmoveto{\pgfqpoint{4.593244in}{1.534611in}}%
\pgfpathlineto{\pgfqpoint{4.743147in}{1.534611in}}%
\pgfpathlineto{\pgfqpoint{4.743147in}{1.974611in}}%
\pgfpathlineto{\pgfqpoint{4.593244in}{1.974611in}}%
\pgfpathlineto{\pgfqpoint{4.593244in}{1.534611in}}%
\pgfpathclose%
\pgfusepath{stroke,fill}%
\end{pgfscope}%
\begin{pgfscope}%
\pgfpathrectangle{\pgfqpoint{4.106058in}{1.534611in}}{\pgfqpoint{0.824468in}{0.462000in}}%
\pgfusepath{clip}%
\pgfsetbuttcap%
\pgfsetmiterjoin%
\definecolor{currentfill}{rgb}{0.121569,0.466667,0.705882}%
\pgfsetfillcolor{currentfill}%
\pgfsetfillopacity{0.500000}%
\pgfsetlinewidth{1.003750pt}%
\definecolor{currentstroke}{rgb}{0.000000,0.000000,0.000000}%
\pgfsetstrokecolor{currentstroke}%
\pgfsetdash{}{0pt}%
\pgfpathmoveto{\pgfqpoint{4.743147in}{1.534611in}}%
\pgfpathlineto{\pgfqpoint{4.893050in}{1.534611in}}%
\pgfpathlineto{\pgfqpoint{4.893050in}{1.857007in}}%
\pgfpathlineto{\pgfqpoint{4.743147in}{1.857007in}}%
\pgfpathlineto{\pgfqpoint{4.743147in}{1.534611in}}%
\pgfpathclose%
\pgfusepath{stroke,fill}%
\end{pgfscope}%
\begin{pgfscope}%
\pgfsetrectcap%
\pgfsetmiterjoin%
\pgfsetlinewidth{0.803000pt}%
\definecolor{currentstroke}{rgb}{0.000000,0.000000,0.000000}%
\pgfsetstrokecolor{currentstroke}%
\pgfsetdash{}{0pt}%
\pgfpathmoveto{\pgfqpoint{4.106058in}{1.534611in}}%
\pgfpathlineto{\pgfqpoint{4.106058in}{1.996611in}}%
\pgfusepath{stroke}%
\end{pgfscope}%
\begin{pgfscope}%
\pgfsetrectcap%
\pgfsetmiterjoin%
\pgfsetlinewidth{0.803000pt}%
\definecolor{currentstroke}{rgb}{0.000000,0.000000,0.000000}%
\pgfsetstrokecolor{currentstroke}%
\pgfsetdash{}{0pt}%
\pgfpathmoveto{\pgfqpoint{4.930526in}{1.534611in}}%
\pgfpathlineto{\pgfqpoint{4.930526in}{1.996611in}}%
\pgfusepath{stroke}%
\end{pgfscope}%
\begin{pgfscope}%
\pgfsetrectcap%
\pgfsetmiterjoin%
\pgfsetlinewidth{0.803000pt}%
\definecolor{currentstroke}{rgb}{0.000000,0.000000,0.000000}%
\pgfsetstrokecolor{currentstroke}%
\pgfsetdash{}{0pt}%
\pgfpathmoveto{\pgfqpoint{4.106058in}{1.534611in}}%
\pgfpathlineto{\pgfqpoint{4.930526in}{1.534611in}}%
\pgfusepath{stroke}%
\end{pgfscope}%
\begin{pgfscope}%
\pgfsetrectcap%
\pgfsetmiterjoin%
\pgfsetlinewidth{0.803000pt}%
\definecolor{currentstroke}{rgb}{0.000000,0.000000,0.000000}%
\pgfsetstrokecolor{currentstroke}%
\pgfsetdash{}{0pt}%
\pgfpathmoveto{\pgfqpoint{4.106058in}{1.996611in}}%
\pgfpathlineto{\pgfqpoint{4.930526in}{1.996611in}}%
\pgfusepath{stroke}%
\end{pgfscope}%
\begin{pgfscope}%
\definecolor{textcolor}{rgb}{0.000000,0.000000,0.000000}%
\pgfsetstrokecolor{textcolor}%
\pgfsetfillcolor{textcolor}%
\pgftext[x=4.518292in,y=2.079944in,,base]{\color{textcolor}\rmfamily\fontsize{11.000000}{13.200000}\selectfont Santiane}%
\end{pgfscope}%
\begin{pgfscope}%
\pgfsetbuttcap%
\pgfsetmiterjoin%
\definecolor{currentfill}{rgb}{1.000000,1.000000,1.000000}%
\pgfsetfillcolor{currentfill}%
\pgfsetlinewidth{0.000000pt}%
\definecolor{currentstroke}{rgb}{0.000000,0.000000,0.000000}%
\pgfsetstrokecolor{currentstroke}%
\pgfsetstrokeopacity{0.000000}%
\pgfsetdash{}{0pt}%
\pgfpathmoveto{\pgfqpoint{5.095420in}{1.534611in}}%
\pgfpathlineto{\pgfqpoint{5.919888in}{1.534611in}}%
\pgfpathlineto{\pgfqpoint{5.919888in}{1.996611in}}%
\pgfpathlineto{\pgfqpoint{5.095420in}{1.996611in}}%
\pgfpathlineto{\pgfqpoint{5.095420in}{1.534611in}}%
\pgfpathclose%
\pgfusepath{fill}%
\end{pgfscope}%
\begin{pgfscope}%
\pgfpathrectangle{\pgfqpoint{5.095420in}{1.534611in}}{\pgfqpoint{0.824468in}{0.462000in}}%
\pgfusepath{clip}%
\pgfsetbuttcap%
\pgfsetmiterjoin%
\definecolor{currentfill}{rgb}{0.121569,0.466667,0.705882}%
\pgfsetfillcolor{currentfill}%
\pgfsetfillopacity{0.500000}%
\pgfsetlinewidth{1.003750pt}%
\definecolor{currentstroke}{rgb}{0.000000,0.000000,0.000000}%
\pgfsetstrokecolor{currentstroke}%
\pgfsetdash{}{0pt}%
\pgfpathmoveto{\pgfqpoint{5.132895in}{1.534611in}}%
\pgfpathlineto{\pgfqpoint{5.282799in}{1.534611in}}%
\pgfpathlineto{\pgfqpoint{5.282799in}{1.974611in}}%
\pgfpathlineto{\pgfqpoint{5.132895in}{1.974611in}}%
\pgfpathlineto{\pgfqpoint{5.132895in}{1.534611in}}%
\pgfpathclose%
\pgfusepath{stroke,fill}%
\end{pgfscope}%
\begin{pgfscope}%
\pgfpathrectangle{\pgfqpoint{5.095420in}{1.534611in}}{\pgfqpoint{0.824468in}{0.462000in}}%
\pgfusepath{clip}%
\pgfsetbuttcap%
\pgfsetmiterjoin%
\definecolor{currentfill}{rgb}{0.121569,0.466667,0.705882}%
\pgfsetfillcolor{currentfill}%
\pgfsetfillopacity{0.500000}%
\pgfsetlinewidth{1.003750pt}%
\definecolor{currentstroke}{rgb}{0.000000,0.000000,0.000000}%
\pgfsetstrokecolor{currentstroke}%
\pgfsetdash{}{0pt}%
\pgfpathmoveto{\pgfqpoint{5.282799in}{1.534611in}}%
\pgfpathlineto{\pgfqpoint{5.432702in}{1.534611in}}%
\pgfpathlineto{\pgfqpoint{5.432702in}{1.661048in}}%
\pgfpathlineto{\pgfqpoint{5.282799in}{1.661048in}}%
\pgfpathlineto{\pgfqpoint{5.282799in}{1.534611in}}%
\pgfpathclose%
\pgfusepath{stroke,fill}%
\end{pgfscope}%
\begin{pgfscope}%
\pgfpathrectangle{\pgfqpoint{5.095420in}{1.534611in}}{\pgfqpoint{0.824468in}{0.462000in}}%
\pgfusepath{clip}%
\pgfsetbuttcap%
\pgfsetmiterjoin%
\definecolor{currentfill}{rgb}{0.121569,0.466667,0.705882}%
\pgfsetfillcolor{currentfill}%
\pgfsetfillopacity{0.500000}%
\pgfsetlinewidth{1.003750pt}%
\definecolor{currentstroke}{rgb}{0.000000,0.000000,0.000000}%
\pgfsetstrokecolor{currentstroke}%
\pgfsetdash{}{0pt}%
\pgfpathmoveto{\pgfqpoint{5.432702in}{1.534611in}}%
\pgfpathlineto{\pgfqpoint{5.582605in}{1.534611in}}%
\pgfpathlineto{\pgfqpoint{5.582605in}{1.585186in}}%
\pgfpathlineto{\pgfqpoint{5.432702in}{1.585186in}}%
\pgfpathlineto{\pgfqpoint{5.432702in}{1.534611in}}%
\pgfpathclose%
\pgfusepath{stroke,fill}%
\end{pgfscope}%
\begin{pgfscope}%
\pgfpathrectangle{\pgfqpoint{5.095420in}{1.534611in}}{\pgfqpoint{0.824468in}{0.462000in}}%
\pgfusepath{clip}%
\pgfsetbuttcap%
\pgfsetmiterjoin%
\definecolor{currentfill}{rgb}{0.121569,0.466667,0.705882}%
\pgfsetfillcolor{currentfill}%
\pgfsetfillopacity{0.500000}%
\pgfsetlinewidth{1.003750pt}%
\definecolor{currentstroke}{rgb}{0.000000,0.000000,0.000000}%
\pgfsetstrokecolor{currentstroke}%
\pgfsetdash{}{0pt}%
\pgfpathmoveto{\pgfqpoint{5.582605in}{1.534611in}}%
\pgfpathlineto{\pgfqpoint{5.732509in}{1.534611in}}%
\pgfpathlineto{\pgfqpoint{5.732509in}{1.544726in}}%
\pgfpathlineto{\pgfqpoint{5.582605in}{1.544726in}}%
\pgfpathlineto{\pgfqpoint{5.582605in}{1.534611in}}%
\pgfpathclose%
\pgfusepath{stroke,fill}%
\end{pgfscope}%
\begin{pgfscope}%
\pgfpathrectangle{\pgfqpoint{5.095420in}{1.534611in}}{\pgfqpoint{0.824468in}{0.462000in}}%
\pgfusepath{clip}%
\pgfsetbuttcap%
\pgfsetmiterjoin%
\definecolor{currentfill}{rgb}{0.121569,0.466667,0.705882}%
\pgfsetfillcolor{currentfill}%
\pgfsetfillopacity{0.500000}%
\pgfsetlinewidth{1.003750pt}%
\definecolor{currentstroke}{rgb}{0.000000,0.000000,0.000000}%
\pgfsetstrokecolor{currentstroke}%
\pgfsetdash{}{0pt}%
\pgfpathmoveto{\pgfqpoint{5.732509in}{1.534611in}}%
\pgfpathlineto{\pgfqpoint{5.882412in}{1.534611in}}%
\pgfpathlineto{\pgfqpoint{5.882412in}{1.575071in}}%
\pgfpathlineto{\pgfqpoint{5.732509in}{1.575071in}}%
\pgfpathlineto{\pgfqpoint{5.732509in}{1.534611in}}%
\pgfpathclose%
\pgfusepath{stroke,fill}%
\end{pgfscope}%
\begin{pgfscope}%
\pgfsetrectcap%
\pgfsetmiterjoin%
\pgfsetlinewidth{0.803000pt}%
\definecolor{currentstroke}{rgb}{0.000000,0.000000,0.000000}%
\pgfsetstrokecolor{currentstroke}%
\pgfsetdash{}{0pt}%
\pgfpathmoveto{\pgfqpoint{5.095420in}{1.534611in}}%
\pgfpathlineto{\pgfqpoint{5.095420in}{1.996611in}}%
\pgfusepath{stroke}%
\end{pgfscope}%
\begin{pgfscope}%
\pgfsetrectcap%
\pgfsetmiterjoin%
\pgfsetlinewidth{0.803000pt}%
\definecolor{currentstroke}{rgb}{0.000000,0.000000,0.000000}%
\pgfsetstrokecolor{currentstroke}%
\pgfsetdash{}{0pt}%
\pgfpathmoveto{\pgfqpoint{5.919888in}{1.534611in}}%
\pgfpathlineto{\pgfqpoint{5.919888in}{1.996611in}}%
\pgfusepath{stroke}%
\end{pgfscope}%
\begin{pgfscope}%
\pgfsetrectcap%
\pgfsetmiterjoin%
\pgfsetlinewidth{0.803000pt}%
\definecolor{currentstroke}{rgb}{0.000000,0.000000,0.000000}%
\pgfsetstrokecolor{currentstroke}%
\pgfsetdash{}{0pt}%
\pgfpathmoveto{\pgfqpoint{5.095420in}{1.534611in}}%
\pgfpathlineto{\pgfqpoint{5.919888in}{1.534611in}}%
\pgfusepath{stroke}%
\end{pgfscope}%
\begin{pgfscope}%
\pgfsetrectcap%
\pgfsetmiterjoin%
\pgfsetlinewidth{0.803000pt}%
\definecolor{currentstroke}{rgb}{0.000000,0.000000,0.000000}%
\pgfsetstrokecolor{currentstroke}%
\pgfsetdash{}{0pt}%
\pgfpathmoveto{\pgfqpoint{5.095420in}{1.996611in}}%
\pgfpathlineto{\pgfqpoint{5.919888in}{1.996611in}}%
\pgfusepath{stroke}%
\end{pgfscope}%
\begin{pgfscope}%
\definecolor{textcolor}{rgb}{0.000000,0.000000,0.000000}%
\pgfsetstrokecolor{textcolor}%
\pgfsetfillcolor{textcolor}%
\pgftext[x=5.507654in,y=2.079944in,,base]{\color{textcolor}\rmfamily\fontsize{11.000000}{13.200000}\selectfont Eca As...}%
\end{pgfscope}%
\begin{pgfscope}%
\pgfsetbuttcap%
\pgfsetmiterjoin%
\definecolor{currentfill}{rgb}{1.000000,1.000000,1.000000}%
\pgfsetfillcolor{currentfill}%
\pgfsetlinewidth{0.000000pt}%
\definecolor{currentstroke}{rgb}{0.000000,0.000000,0.000000}%
\pgfsetstrokecolor{currentstroke}%
\pgfsetstrokeopacity{0.000000}%
\pgfsetdash{}{0pt}%
\pgfpathmoveto{\pgfqpoint{6.084781in}{1.534611in}}%
\pgfpathlineto{\pgfqpoint{6.909249in}{1.534611in}}%
\pgfpathlineto{\pgfqpoint{6.909249in}{1.996611in}}%
\pgfpathlineto{\pgfqpoint{6.084781in}{1.996611in}}%
\pgfpathlineto{\pgfqpoint{6.084781in}{1.534611in}}%
\pgfpathclose%
\pgfusepath{fill}%
\end{pgfscope}%
\begin{pgfscope}%
\pgfpathrectangle{\pgfqpoint{6.084781in}{1.534611in}}{\pgfqpoint{0.824468in}{0.462000in}}%
\pgfusepath{clip}%
\pgfsetbuttcap%
\pgfsetmiterjoin%
\definecolor{currentfill}{rgb}{0.121569,0.466667,0.705882}%
\pgfsetfillcolor{currentfill}%
\pgfsetfillopacity{0.500000}%
\pgfsetlinewidth{1.003750pt}%
\definecolor{currentstroke}{rgb}{0.000000,0.000000,0.000000}%
\pgfsetstrokecolor{currentstroke}%
\pgfsetdash{}{0pt}%
\pgfpathmoveto{\pgfqpoint{6.122257in}{1.534611in}}%
\pgfpathlineto{\pgfqpoint{6.272160in}{1.534611in}}%
\pgfpathlineto{\pgfqpoint{6.272160in}{1.974611in}}%
\pgfpathlineto{\pgfqpoint{6.122257in}{1.974611in}}%
\pgfpathlineto{\pgfqpoint{6.122257in}{1.534611in}}%
\pgfpathclose%
\pgfusepath{stroke,fill}%
\end{pgfscope}%
\begin{pgfscope}%
\pgfpathrectangle{\pgfqpoint{6.084781in}{1.534611in}}{\pgfqpoint{0.824468in}{0.462000in}}%
\pgfusepath{clip}%
\pgfsetbuttcap%
\pgfsetmiterjoin%
\definecolor{currentfill}{rgb}{0.121569,0.466667,0.705882}%
\pgfsetfillcolor{currentfill}%
\pgfsetfillopacity{0.500000}%
\pgfsetlinewidth{1.003750pt}%
\definecolor{currentstroke}{rgb}{0.000000,0.000000,0.000000}%
\pgfsetstrokecolor{currentstroke}%
\pgfsetdash{}{0pt}%
\pgfpathmoveto{\pgfqpoint{6.272160in}{1.534611in}}%
\pgfpathlineto{\pgfqpoint{6.422064in}{1.534611in}}%
\pgfpathlineto{\pgfqpoint{6.422064in}{1.742611in}}%
\pgfpathlineto{\pgfqpoint{6.272160in}{1.742611in}}%
\pgfpathlineto{\pgfqpoint{6.272160in}{1.534611in}}%
\pgfpathclose%
\pgfusepath{stroke,fill}%
\end{pgfscope}%
\begin{pgfscope}%
\pgfpathrectangle{\pgfqpoint{6.084781in}{1.534611in}}{\pgfqpoint{0.824468in}{0.462000in}}%
\pgfusepath{clip}%
\pgfsetbuttcap%
\pgfsetmiterjoin%
\definecolor{currentfill}{rgb}{0.121569,0.466667,0.705882}%
\pgfsetfillcolor{currentfill}%
\pgfsetfillopacity{0.500000}%
\pgfsetlinewidth{1.003750pt}%
\definecolor{currentstroke}{rgb}{0.000000,0.000000,0.000000}%
\pgfsetstrokecolor{currentstroke}%
\pgfsetdash{}{0pt}%
\pgfpathmoveto{\pgfqpoint{6.422064in}{1.534611in}}%
\pgfpathlineto{\pgfqpoint{6.571967in}{1.534611in}}%
\pgfpathlineto{\pgfqpoint{6.571967in}{1.574611in}}%
\pgfpathlineto{\pgfqpoint{6.422064in}{1.574611in}}%
\pgfpathlineto{\pgfqpoint{6.422064in}{1.534611in}}%
\pgfpathclose%
\pgfusepath{stroke,fill}%
\end{pgfscope}%
\begin{pgfscope}%
\pgfpathrectangle{\pgfqpoint{6.084781in}{1.534611in}}{\pgfqpoint{0.824468in}{0.462000in}}%
\pgfusepath{clip}%
\pgfsetbuttcap%
\pgfsetmiterjoin%
\definecolor{currentfill}{rgb}{0.121569,0.466667,0.705882}%
\pgfsetfillcolor{currentfill}%
\pgfsetfillopacity{0.500000}%
\pgfsetlinewidth{1.003750pt}%
\definecolor{currentstroke}{rgb}{0.000000,0.000000,0.000000}%
\pgfsetstrokecolor{currentstroke}%
\pgfsetdash{}{0pt}%
\pgfpathmoveto{\pgfqpoint{6.571967in}{1.534611in}}%
\pgfpathlineto{\pgfqpoint{6.721870in}{1.534611in}}%
\pgfpathlineto{\pgfqpoint{6.721870in}{1.582611in}}%
\pgfpathlineto{\pgfqpoint{6.571967in}{1.582611in}}%
\pgfpathlineto{\pgfqpoint{6.571967in}{1.534611in}}%
\pgfpathclose%
\pgfusepath{stroke,fill}%
\end{pgfscope}%
\begin{pgfscope}%
\pgfpathrectangle{\pgfqpoint{6.084781in}{1.534611in}}{\pgfqpoint{0.824468in}{0.462000in}}%
\pgfusepath{clip}%
\pgfsetbuttcap%
\pgfsetmiterjoin%
\definecolor{currentfill}{rgb}{0.121569,0.466667,0.705882}%
\pgfsetfillcolor{currentfill}%
\pgfsetfillopacity{0.500000}%
\pgfsetlinewidth{1.003750pt}%
\definecolor{currentstroke}{rgb}{0.000000,0.000000,0.000000}%
\pgfsetstrokecolor{currentstroke}%
\pgfsetdash{}{0pt}%
\pgfpathmoveto{\pgfqpoint{6.721870in}{1.534611in}}%
\pgfpathlineto{\pgfqpoint{6.871774in}{1.534611in}}%
\pgfpathlineto{\pgfqpoint{6.871774in}{1.574611in}}%
\pgfpathlineto{\pgfqpoint{6.721870in}{1.574611in}}%
\pgfpathlineto{\pgfqpoint{6.721870in}{1.534611in}}%
\pgfpathclose%
\pgfusepath{stroke,fill}%
\end{pgfscope}%
\begin{pgfscope}%
\pgfsetrectcap%
\pgfsetmiterjoin%
\pgfsetlinewidth{0.803000pt}%
\definecolor{currentstroke}{rgb}{0.000000,0.000000,0.000000}%
\pgfsetstrokecolor{currentstroke}%
\pgfsetdash{}{0pt}%
\pgfpathmoveto{\pgfqpoint{6.084781in}{1.534611in}}%
\pgfpathlineto{\pgfqpoint{6.084781in}{1.996611in}}%
\pgfusepath{stroke}%
\end{pgfscope}%
\begin{pgfscope}%
\pgfsetrectcap%
\pgfsetmiterjoin%
\pgfsetlinewidth{0.803000pt}%
\definecolor{currentstroke}{rgb}{0.000000,0.000000,0.000000}%
\pgfsetstrokecolor{currentstroke}%
\pgfsetdash{}{0pt}%
\pgfpathmoveto{\pgfqpoint{6.909249in}{1.534611in}}%
\pgfpathlineto{\pgfqpoint{6.909249in}{1.996611in}}%
\pgfusepath{stroke}%
\end{pgfscope}%
\begin{pgfscope}%
\pgfsetrectcap%
\pgfsetmiterjoin%
\pgfsetlinewidth{0.803000pt}%
\definecolor{currentstroke}{rgb}{0.000000,0.000000,0.000000}%
\pgfsetstrokecolor{currentstroke}%
\pgfsetdash{}{0pt}%
\pgfpathmoveto{\pgfqpoint{6.084781in}{1.534611in}}%
\pgfpathlineto{\pgfqpoint{6.909249in}{1.534611in}}%
\pgfusepath{stroke}%
\end{pgfscope}%
\begin{pgfscope}%
\pgfsetrectcap%
\pgfsetmiterjoin%
\pgfsetlinewidth{0.803000pt}%
\definecolor{currentstroke}{rgb}{0.000000,0.000000,0.000000}%
\pgfsetstrokecolor{currentstroke}%
\pgfsetdash{}{0pt}%
\pgfpathmoveto{\pgfqpoint{6.084781in}{1.996611in}}%
\pgfpathlineto{\pgfqpoint{6.909249in}{1.996611in}}%
\pgfusepath{stroke}%
\end{pgfscope}%
\begin{pgfscope}%
\definecolor{textcolor}{rgb}{0.000000,0.000000,0.000000}%
\pgfsetstrokecolor{textcolor}%
\pgfsetfillcolor{textcolor}%
\pgftext[x=6.497015in,y=2.079944in,,base]{\color{textcolor}\rmfamily\fontsize{11.000000}{13.200000}\selectfont Groupama}%
\end{pgfscope}%
\begin{pgfscope}%
\pgfsetbuttcap%
\pgfsetmiterjoin%
\definecolor{currentfill}{rgb}{1.000000,1.000000,1.000000}%
\pgfsetfillcolor{currentfill}%
\pgfsetlinewidth{0.000000pt}%
\definecolor{currentstroke}{rgb}{0.000000,0.000000,0.000000}%
\pgfsetstrokecolor{currentstroke}%
\pgfsetstrokeopacity{0.000000}%
\pgfsetdash{}{0pt}%
\pgfpathmoveto{\pgfqpoint{7.074143in}{1.534611in}}%
\pgfpathlineto{\pgfqpoint{7.898611in}{1.534611in}}%
\pgfpathlineto{\pgfqpoint{7.898611in}{1.996611in}}%
\pgfpathlineto{\pgfqpoint{7.074143in}{1.996611in}}%
\pgfpathlineto{\pgfqpoint{7.074143in}{1.534611in}}%
\pgfpathclose%
\pgfusepath{fill}%
\end{pgfscope}%
\begin{pgfscope}%
\pgfpathrectangle{\pgfqpoint{7.074143in}{1.534611in}}{\pgfqpoint{0.824468in}{0.462000in}}%
\pgfusepath{clip}%
\pgfsetbuttcap%
\pgfsetmiterjoin%
\definecolor{currentfill}{rgb}{0.121569,0.466667,0.705882}%
\pgfsetfillcolor{currentfill}%
\pgfsetfillopacity{0.500000}%
\pgfsetlinewidth{1.003750pt}%
\definecolor{currentstroke}{rgb}{0.000000,0.000000,0.000000}%
\pgfsetstrokecolor{currentstroke}%
\pgfsetdash{}{0pt}%
\pgfpathmoveto{\pgfqpoint{7.111619in}{1.534611in}}%
\pgfpathlineto{\pgfqpoint{7.261522in}{1.534611in}}%
\pgfpathlineto{\pgfqpoint{7.261522in}{1.974611in}}%
\pgfpathlineto{\pgfqpoint{7.111619in}{1.974611in}}%
\pgfpathlineto{\pgfqpoint{7.111619in}{1.534611in}}%
\pgfpathclose%
\pgfusepath{stroke,fill}%
\end{pgfscope}%
\begin{pgfscope}%
\pgfpathrectangle{\pgfqpoint{7.074143in}{1.534611in}}{\pgfqpoint{0.824468in}{0.462000in}}%
\pgfusepath{clip}%
\pgfsetbuttcap%
\pgfsetmiterjoin%
\definecolor{currentfill}{rgb}{0.121569,0.466667,0.705882}%
\pgfsetfillcolor{currentfill}%
\pgfsetfillopacity{0.500000}%
\pgfsetlinewidth{1.003750pt}%
\definecolor{currentstroke}{rgb}{0.000000,0.000000,0.000000}%
\pgfsetstrokecolor{currentstroke}%
\pgfsetdash{}{0pt}%
\pgfpathmoveto{\pgfqpoint{7.261522in}{1.534611in}}%
\pgfpathlineto{\pgfqpoint{7.411425in}{1.534611in}}%
\pgfpathlineto{\pgfqpoint{7.411425in}{1.735313in}}%
\pgfpathlineto{\pgfqpoint{7.261522in}{1.735313in}}%
\pgfpathlineto{\pgfqpoint{7.261522in}{1.534611in}}%
\pgfpathclose%
\pgfusepath{stroke,fill}%
\end{pgfscope}%
\begin{pgfscope}%
\pgfpathrectangle{\pgfqpoint{7.074143in}{1.534611in}}{\pgfqpoint{0.824468in}{0.462000in}}%
\pgfusepath{clip}%
\pgfsetbuttcap%
\pgfsetmiterjoin%
\definecolor{currentfill}{rgb}{0.121569,0.466667,0.705882}%
\pgfsetfillcolor{currentfill}%
\pgfsetfillopacity{0.500000}%
\pgfsetlinewidth{1.003750pt}%
\definecolor{currentstroke}{rgb}{0.000000,0.000000,0.000000}%
\pgfsetstrokecolor{currentstroke}%
\pgfsetdash{}{0pt}%
\pgfpathmoveto{\pgfqpoint{7.411425in}{1.534611in}}%
\pgfpathlineto{\pgfqpoint{7.561329in}{1.534611in}}%
\pgfpathlineto{\pgfqpoint{7.561329in}{1.650401in}}%
\pgfpathlineto{\pgfqpoint{7.411425in}{1.650401in}}%
\pgfpathlineto{\pgfqpoint{7.411425in}{1.534611in}}%
\pgfpathclose%
\pgfusepath{stroke,fill}%
\end{pgfscope}%
\begin{pgfscope}%
\pgfpathrectangle{\pgfqpoint{7.074143in}{1.534611in}}{\pgfqpoint{0.824468in}{0.462000in}}%
\pgfusepath{clip}%
\pgfsetbuttcap%
\pgfsetmiterjoin%
\definecolor{currentfill}{rgb}{0.121569,0.466667,0.705882}%
\pgfsetfillcolor{currentfill}%
\pgfsetfillopacity{0.500000}%
\pgfsetlinewidth{1.003750pt}%
\definecolor{currentstroke}{rgb}{0.000000,0.000000,0.000000}%
\pgfsetstrokecolor{currentstroke}%
\pgfsetdash{}{0pt}%
\pgfpathmoveto{\pgfqpoint{7.561329in}{1.534611in}}%
\pgfpathlineto{\pgfqpoint{7.711232in}{1.534611in}}%
\pgfpathlineto{\pgfqpoint{7.711232in}{1.573208in}}%
\pgfpathlineto{\pgfqpoint{7.561329in}{1.573208in}}%
\pgfpathlineto{\pgfqpoint{7.561329in}{1.534611in}}%
\pgfpathclose%
\pgfusepath{stroke,fill}%
\end{pgfscope}%
\begin{pgfscope}%
\pgfpathrectangle{\pgfqpoint{7.074143in}{1.534611in}}{\pgfqpoint{0.824468in}{0.462000in}}%
\pgfusepath{clip}%
\pgfsetbuttcap%
\pgfsetmiterjoin%
\definecolor{currentfill}{rgb}{0.121569,0.466667,0.705882}%
\pgfsetfillcolor{currentfill}%
\pgfsetfillopacity{0.500000}%
\pgfsetlinewidth{1.003750pt}%
\definecolor{currentstroke}{rgb}{0.000000,0.000000,0.000000}%
\pgfsetstrokecolor{currentstroke}%
\pgfsetdash{}{0pt}%
\pgfpathmoveto{\pgfqpoint{7.711232in}{1.534611in}}%
\pgfpathlineto{\pgfqpoint{7.861135in}{1.534611in}}%
\pgfpathlineto{\pgfqpoint{7.861135in}{1.588646in}}%
\pgfpathlineto{\pgfqpoint{7.711232in}{1.588646in}}%
\pgfpathlineto{\pgfqpoint{7.711232in}{1.534611in}}%
\pgfpathclose%
\pgfusepath{stroke,fill}%
\end{pgfscope}%
\begin{pgfscope}%
\pgfsetrectcap%
\pgfsetmiterjoin%
\pgfsetlinewidth{0.803000pt}%
\definecolor{currentstroke}{rgb}{0.000000,0.000000,0.000000}%
\pgfsetstrokecolor{currentstroke}%
\pgfsetdash{}{0pt}%
\pgfpathmoveto{\pgfqpoint{7.074143in}{1.534611in}}%
\pgfpathlineto{\pgfqpoint{7.074143in}{1.996611in}}%
\pgfusepath{stroke}%
\end{pgfscope}%
\begin{pgfscope}%
\pgfsetrectcap%
\pgfsetmiterjoin%
\pgfsetlinewidth{0.803000pt}%
\definecolor{currentstroke}{rgb}{0.000000,0.000000,0.000000}%
\pgfsetstrokecolor{currentstroke}%
\pgfsetdash{}{0pt}%
\pgfpathmoveto{\pgfqpoint{7.898611in}{1.534611in}}%
\pgfpathlineto{\pgfqpoint{7.898611in}{1.996611in}}%
\pgfusepath{stroke}%
\end{pgfscope}%
\begin{pgfscope}%
\pgfsetrectcap%
\pgfsetmiterjoin%
\pgfsetlinewidth{0.803000pt}%
\definecolor{currentstroke}{rgb}{0.000000,0.000000,0.000000}%
\pgfsetstrokecolor{currentstroke}%
\pgfsetdash{}{0pt}%
\pgfpathmoveto{\pgfqpoint{7.074143in}{1.534611in}}%
\pgfpathlineto{\pgfqpoint{7.898611in}{1.534611in}}%
\pgfusepath{stroke}%
\end{pgfscope}%
\begin{pgfscope}%
\pgfsetrectcap%
\pgfsetmiterjoin%
\pgfsetlinewidth{0.803000pt}%
\definecolor{currentstroke}{rgb}{0.000000,0.000000,0.000000}%
\pgfsetstrokecolor{currentstroke}%
\pgfsetdash{}{0pt}%
\pgfpathmoveto{\pgfqpoint{7.074143in}{1.996611in}}%
\pgfpathlineto{\pgfqpoint{7.898611in}{1.996611in}}%
\pgfusepath{stroke}%
\end{pgfscope}%
\begin{pgfscope}%
\definecolor{textcolor}{rgb}{0.000000,0.000000,0.000000}%
\pgfsetstrokecolor{textcolor}%
\pgfsetfillcolor{textcolor}%
\pgftext[x=7.486377in,y=2.079944in,,base]{\color{textcolor}\rmfamily\fontsize{11.000000}{13.200000}\selectfont Assur ...}%
\end{pgfscope}%
\begin{pgfscope}%
\pgfsetbuttcap%
\pgfsetmiterjoin%
\definecolor{currentfill}{rgb}{1.000000,1.000000,1.000000}%
\pgfsetfillcolor{currentfill}%
\pgfsetlinewidth{0.000000pt}%
\definecolor{currentstroke}{rgb}{0.000000,0.000000,0.000000}%
\pgfsetstrokecolor{currentstroke}%
\pgfsetstrokeopacity{0.000000}%
\pgfsetdash{}{0pt}%
\pgfpathmoveto{\pgfqpoint{0.148611in}{0.841611in}}%
\pgfpathlineto{\pgfqpoint{0.973079in}{0.841611in}}%
\pgfpathlineto{\pgfqpoint{0.973079in}{1.303611in}}%
\pgfpathlineto{\pgfqpoint{0.148611in}{1.303611in}}%
\pgfpathlineto{\pgfqpoint{0.148611in}{0.841611in}}%
\pgfpathclose%
\pgfusepath{fill}%
\end{pgfscope}%
\begin{pgfscope}%
\pgfpathrectangle{\pgfqpoint{0.148611in}{0.841611in}}{\pgfqpoint{0.824468in}{0.462000in}}%
\pgfusepath{clip}%
\pgfsetbuttcap%
\pgfsetmiterjoin%
\definecolor{currentfill}{rgb}{0.121569,0.466667,0.705882}%
\pgfsetfillcolor{currentfill}%
\pgfsetfillopacity{0.500000}%
\pgfsetlinewidth{1.003750pt}%
\definecolor{currentstroke}{rgb}{0.000000,0.000000,0.000000}%
\pgfsetstrokecolor{currentstroke}%
\pgfsetdash{}{0pt}%
\pgfpathmoveto{\pgfqpoint{0.186087in}{0.841611in}}%
\pgfpathlineto{\pgfqpoint{0.335990in}{0.841611in}}%
\pgfpathlineto{\pgfqpoint{0.335990in}{1.281611in}}%
\pgfpathlineto{\pgfqpoint{0.186087in}{1.281611in}}%
\pgfpathlineto{\pgfqpoint{0.186087in}{0.841611in}}%
\pgfpathclose%
\pgfusepath{stroke,fill}%
\end{pgfscope}%
\begin{pgfscope}%
\pgfpathrectangle{\pgfqpoint{0.148611in}{0.841611in}}{\pgfqpoint{0.824468in}{0.462000in}}%
\pgfusepath{clip}%
\pgfsetbuttcap%
\pgfsetmiterjoin%
\definecolor{currentfill}{rgb}{0.121569,0.466667,0.705882}%
\pgfsetfillcolor{currentfill}%
\pgfsetfillopacity{0.500000}%
\pgfsetlinewidth{1.003750pt}%
\definecolor{currentstroke}{rgb}{0.000000,0.000000,0.000000}%
\pgfsetstrokecolor{currentstroke}%
\pgfsetdash{}{0pt}%
\pgfpathmoveto{\pgfqpoint{0.335990in}{0.841611in}}%
\pgfpathlineto{\pgfqpoint{0.485894in}{0.841611in}}%
\pgfpathlineto{\pgfqpoint{0.485894in}{0.841611in}}%
\pgfpathlineto{\pgfqpoint{0.335990in}{0.841611in}}%
\pgfpathlineto{\pgfqpoint{0.335990in}{0.841611in}}%
\pgfpathclose%
\pgfusepath{stroke,fill}%
\end{pgfscope}%
\begin{pgfscope}%
\pgfpathrectangle{\pgfqpoint{0.148611in}{0.841611in}}{\pgfqpoint{0.824468in}{0.462000in}}%
\pgfusepath{clip}%
\pgfsetbuttcap%
\pgfsetmiterjoin%
\definecolor{currentfill}{rgb}{0.121569,0.466667,0.705882}%
\pgfsetfillcolor{currentfill}%
\pgfsetfillopacity{0.500000}%
\pgfsetlinewidth{1.003750pt}%
\definecolor{currentstroke}{rgb}{0.000000,0.000000,0.000000}%
\pgfsetstrokecolor{currentstroke}%
\pgfsetdash{}{0pt}%
\pgfpathmoveto{\pgfqpoint{0.485894in}{0.841611in}}%
\pgfpathlineto{\pgfqpoint{0.635797in}{0.841611in}}%
\pgfpathlineto{\pgfqpoint{0.635797in}{0.841611in}}%
\pgfpathlineto{\pgfqpoint{0.485894in}{0.841611in}}%
\pgfpathlineto{\pgfqpoint{0.485894in}{0.841611in}}%
\pgfpathclose%
\pgfusepath{stroke,fill}%
\end{pgfscope}%
\begin{pgfscope}%
\pgfpathrectangle{\pgfqpoint{0.148611in}{0.841611in}}{\pgfqpoint{0.824468in}{0.462000in}}%
\pgfusepath{clip}%
\pgfsetbuttcap%
\pgfsetmiterjoin%
\definecolor{currentfill}{rgb}{0.121569,0.466667,0.705882}%
\pgfsetfillcolor{currentfill}%
\pgfsetfillopacity{0.500000}%
\pgfsetlinewidth{1.003750pt}%
\definecolor{currentstroke}{rgb}{0.000000,0.000000,0.000000}%
\pgfsetstrokecolor{currentstroke}%
\pgfsetdash{}{0pt}%
\pgfpathmoveto{\pgfqpoint{0.635797in}{0.841611in}}%
\pgfpathlineto{\pgfqpoint{0.785700in}{0.841611in}}%
\pgfpathlineto{\pgfqpoint{0.785700in}{0.841611in}}%
\pgfpathlineto{\pgfqpoint{0.635797in}{0.841611in}}%
\pgfpathlineto{\pgfqpoint{0.635797in}{0.841611in}}%
\pgfpathclose%
\pgfusepath{stroke,fill}%
\end{pgfscope}%
\begin{pgfscope}%
\pgfpathrectangle{\pgfqpoint{0.148611in}{0.841611in}}{\pgfqpoint{0.824468in}{0.462000in}}%
\pgfusepath{clip}%
\pgfsetbuttcap%
\pgfsetmiterjoin%
\definecolor{currentfill}{rgb}{0.121569,0.466667,0.705882}%
\pgfsetfillcolor{currentfill}%
\pgfsetfillopacity{0.500000}%
\pgfsetlinewidth{1.003750pt}%
\definecolor{currentstroke}{rgb}{0.000000,0.000000,0.000000}%
\pgfsetstrokecolor{currentstroke}%
\pgfsetdash{}{0pt}%
\pgfpathmoveto{\pgfqpoint{0.785700in}{0.841611in}}%
\pgfpathlineto{\pgfqpoint{0.935603in}{0.841611in}}%
\pgfpathlineto{\pgfqpoint{0.935603in}{0.841611in}}%
\pgfpathlineto{\pgfqpoint{0.785700in}{0.841611in}}%
\pgfpathlineto{\pgfqpoint{0.785700in}{0.841611in}}%
\pgfpathclose%
\pgfusepath{stroke,fill}%
\end{pgfscope}%
\begin{pgfscope}%
\pgfsetrectcap%
\pgfsetmiterjoin%
\pgfsetlinewidth{0.803000pt}%
\definecolor{currentstroke}{rgb}{0.000000,0.000000,0.000000}%
\pgfsetstrokecolor{currentstroke}%
\pgfsetdash{}{0pt}%
\pgfpathmoveto{\pgfqpoint{0.148611in}{0.841611in}}%
\pgfpathlineto{\pgfqpoint{0.148611in}{1.303611in}}%
\pgfusepath{stroke}%
\end{pgfscope}%
\begin{pgfscope}%
\pgfsetrectcap%
\pgfsetmiterjoin%
\pgfsetlinewidth{0.803000pt}%
\definecolor{currentstroke}{rgb}{0.000000,0.000000,0.000000}%
\pgfsetstrokecolor{currentstroke}%
\pgfsetdash{}{0pt}%
\pgfpathmoveto{\pgfqpoint{0.973079in}{0.841611in}}%
\pgfpathlineto{\pgfqpoint{0.973079in}{1.303611in}}%
\pgfusepath{stroke}%
\end{pgfscope}%
\begin{pgfscope}%
\pgfsetrectcap%
\pgfsetmiterjoin%
\pgfsetlinewidth{0.803000pt}%
\definecolor{currentstroke}{rgb}{0.000000,0.000000,0.000000}%
\pgfsetstrokecolor{currentstroke}%
\pgfsetdash{}{0pt}%
\pgfpathmoveto{\pgfqpoint{0.148611in}{0.841611in}}%
\pgfpathlineto{\pgfqpoint{0.973079in}{0.841611in}}%
\pgfusepath{stroke}%
\end{pgfscope}%
\begin{pgfscope}%
\pgfsetrectcap%
\pgfsetmiterjoin%
\pgfsetlinewidth{0.803000pt}%
\definecolor{currentstroke}{rgb}{0.000000,0.000000,0.000000}%
\pgfsetstrokecolor{currentstroke}%
\pgfsetdash{}{0pt}%
\pgfpathmoveto{\pgfqpoint{0.148611in}{1.303611in}}%
\pgfpathlineto{\pgfqpoint{0.973079in}{1.303611in}}%
\pgfusepath{stroke}%
\end{pgfscope}%
\begin{pgfscope}%
\definecolor{textcolor}{rgb}{0.000000,0.000000,0.000000}%
\pgfsetstrokecolor{textcolor}%
\pgfsetfillcolor{textcolor}%
\pgftext[x=0.560845in,y=1.386944in,,base]{\color{textcolor}\rmfamily\fontsize{11.000000}{13.200000}\selectfont MMA}%
\end{pgfscope}%
\begin{pgfscope}%
\pgfsetbuttcap%
\pgfsetmiterjoin%
\definecolor{currentfill}{rgb}{1.000000,1.000000,1.000000}%
\pgfsetfillcolor{currentfill}%
\pgfsetlinewidth{0.000000pt}%
\definecolor{currentstroke}{rgb}{0.000000,0.000000,0.000000}%
\pgfsetstrokecolor{currentstroke}%
\pgfsetstrokeopacity{0.000000}%
\pgfsetdash{}{0pt}%
\pgfpathmoveto{\pgfqpoint{1.137973in}{0.841611in}}%
\pgfpathlineto{\pgfqpoint{1.962441in}{0.841611in}}%
\pgfpathlineto{\pgfqpoint{1.962441in}{1.303611in}}%
\pgfpathlineto{\pgfqpoint{1.137973in}{1.303611in}}%
\pgfpathlineto{\pgfqpoint{1.137973in}{0.841611in}}%
\pgfpathclose%
\pgfusepath{fill}%
\end{pgfscope}%
\begin{pgfscope}%
\pgfpathrectangle{\pgfqpoint{1.137973in}{0.841611in}}{\pgfqpoint{0.824468in}{0.462000in}}%
\pgfusepath{clip}%
\pgfsetbuttcap%
\pgfsetmiterjoin%
\definecolor{currentfill}{rgb}{0.121569,0.466667,0.705882}%
\pgfsetfillcolor{currentfill}%
\pgfsetfillopacity{0.500000}%
\pgfsetlinewidth{1.003750pt}%
\definecolor{currentstroke}{rgb}{0.000000,0.000000,0.000000}%
\pgfsetstrokecolor{currentstroke}%
\pgfsetdash{}{0pt}%
\pgfpathmoveto{\pgfqpoint{1.175449in}{0.841611in}}%
\pgfpathlineto{\pgfqpoint{1.325352in}{0.841611in}}%
\pgfpathlineto{\pgfqpoint{1.325352in}{1.281611in}}%
\pgfpathlineto{\pgfqpoint{1.175449in}{1.281611in}}%
\pgfpathlineto{\pgfqpoint{1.175449in}{0.841611in}}%
\pgfpathclose%
\pgfusepath{stroke,fill}%
\end{pgfscope}%
\begin{pgfscope}%
\pgfpathrectangle{\pgfqpoint{1.137973in}{0.841611in}}{\pgfqpoint{0.824468in}{0.462000in}}%
\pgfusepath{clip}%
\pgfsetbuttcap%
\pgfsetmiterjoin%
\definecolor{currentfill}{rgb}{0.121569,0.466667,0.705882}%
\pgfsetfillcolor{currentfill}%
\pgfsetfillopacity{0.500000}%
\pgfsetlinewidth{1.003750pt}%
\definecolor{currentstroke}{rgb}{0.000000,0.000000,0.000000}%
\pgfsetstrokecolor{currentstroke}%
\pgfsetdash{}{0pt}%
\pgfpathmoveto{\pgfqpoint{1.325352in}{0.841611in}}%
\pgfpathlineto{\pgfqpoint{1.475255in}{0.841611in}}%
\pgfpathlineto{\pgfqpoint{1.475255in}{1.047861in}}%
\pgfpathlineto{\pgfqpoint{1.325352in}{1.047861in}}%
\pgfpathlineto{\pgfqpoint{1.325352in}{0.841611in}}%
\pgfpathclose%
\pgfusepath{stroke,fill}%
\end{pgfscope}%
\begin{pgfscope}%
\pgfpathrectangle{\pgfqpoint{1.137973in}{0.841611in}}{\pgfqpoint{0.824468in}{0.462000in}}%
\pgfusepath{clip}%
\pgfsetbuttcap%
\pgfsetmiterjoin%
\definecolor{currentfill}{rgb}{0.121569,0.466667,0.705882}%
\pgfsetfillcolor{currentfill}%
\pgfsetfillopacity{0.500000}%
\pgfsetlinewidth{1.003750pt}%
\definecolor{currentstroke}{rgb}{0.000000,0.000000,0.000000}%
\pgfsetstrokecolor{currentstroke}%
\pgfsetdash{}{0pt}%
\pgfpathmoveto{\pgfqpoint{1.475255in}{0.841611in}}%
\pgfpathlineto{\pgfqpoint{1.625158in}{0.841611in}}%
\pgfpathlineto{\pgfqpoint{1.625158in}{0.855361in}}%
\pgfpathlineto{\pgfqpoint{1.475255in}{0.855361in}}%
\pgfpathlineto{\pgfqpoint{1.475255in}{0.841611in}}%
\pgfpathclose%
\pgfusepath{stroke,fill}%
\end{pgfscope}%
\begin{pgfscope}%
\pgfpathrectangle{\pgfqpoint{1.137973in}{0.841611in}}{\pgfqpoint{0.824468in}{0.462000in}}%
\pgfusepath{clip}%
\pgfsetbuttcap%
\pgfsetmiterjoin%
\definecolor{currentfill}{rgb}{0.121569,0.466667,0.705882}%
\pgfsetfillcolor{currentfill}%
\pgfsetfillopacity{0.500000}%
\pgfsetlinewidth{1.003750pt}%
\definecolor{currentstroke}{rgb}{0.000000,0.000000,0.000000}%
\pgfsetstrokecolor{currentstroke}%
\pgfsetdash{}{0pt}%
\pgfpathmoveto{\pgfqpoint{1.625158in}{0.841611in}}%
\pgfpathlineto{\pgfqpoint{1.775062in}{0.841611in}}%
\pgfpathlineto{\pgfqpoint{1.775062in}{0.841611in}}%
\pgfpathlineto{\pgfqpoint{1.625158in}{0.841611in}}%
\pgfpathlineto{\pgfqpoint{1.625158in}{0.841611in}}%
\pgfpathclose%
\pgfusepath{stroke,fill}%
\end{pgfscope}%
\begin{pgfscope}%
\pgfpathrectangle{\pgfqpoint{1.137973in}{0.841611in}}{\pgfqpoint{0.824468in}{0.462000in}}%
\pgfusepath{clip}%
\pgfsetbuttcap%
\pgfsetmiterjoin%
\definecolor{currentfill}{rgb}{0.121569,0.466667,0.705882}%
\pgfsetfillcolor{currentfill}%
\pgfsetfillopacity{0.500000}%
\pgfsetlinewidth{1.003750pt}%
\definecolor{currentstroke}{rgb}{0.000000,0.000000,0.000000}%
\pgfsetstrokecolor{currentstroke}%
\pgfsetdash{}{0pt}%
\pgfpathmoveto{\pgfqpoint{1.775062in}{0.841611in}}%
\pgfpathlineto{\pgfqpoint{1.924965in}{0.841611in}}%
\pgfpathlineto{\pgfqpoint{1.924965in}{0.910361in}}%
\pgfpathlineto{\pgfqpoint{1.775062in}{0.910361in}}%
\pgfpathlineto{\pgfqpoint{1.775062in}{0.841611in}}%
\pgfpathclose%
\pgfusepath{stroke,fill}%
\end{pgfscope}%
\begin{pgfscope}%
\pgfsetrectcap%
\pgfsetmiterjoin%
\pgfsetlinewidth{0.803000pt}%
\definecolor{currentstroke}{rgb}{0.000000,0.000000,0.000000}%
\pgfsetstrokecolor{currentstroke}%
\pgfsetdash{}{0pt}%
\pgfpathmoveto{\pgfqpoint{1.137973in}{0.841611in}}%
\pgfpathlineto{\pgfqpoint{1.137973in}{1.303611in}}%
\pgfusepath{stroke}%
\end{pgfscope}%
\begin{pgfscope}%
\pgfsetrectcap%
\pgfsetmiterjoin%
\pgfsetlinewidth{0.803000pt}%
\definecolor{currentstroke}{rgb}{0.000000,0.000000,0.000000}%
\pgfsetstrokecolor{currentstroke}%
\pgfsetdash{}{0pt}%
\pgfpathmoveto{\pgfqpoint{1.962441in}{0.841611in}}%
\pgfpathlineto{\pgfqpoint{1.962441in}{1.303611in}}%
\pgfusepath{stroke}%
\end{pgfscope}%
\begin{pgfscope}%
\pgfsetrectcap%
\pgfsetmiterjoin%
\pgfsetlinewidth{0.803000pt}%
\definecolor{currentstroke}{rgb}{0.000000,0.000000,0.000000}%
\pgfsetstrokecolor{currentstroke}%
\pgfsetdash{}{0pt}%
\pgfpathmoveto{\pgfqpoint{1.137973in}{0.841611in}}%
\pgfpathlineto{\pgfqpoint{1.962441in}{0.841611in}}%
\pgfusepath{stroke}%
\end{pgfscope}%
\begin{pgfscope}%
\pgfsetrectcap%
\pgfsetmiterjoin%
\pgfsetlinewidth{0.803000pt}%
\definecolor{currentstroke}{rgb}{0.000000,0.000000,0.000000}%
\pgfsetstrokecolor{currentstroke}%
\pgfsetdash{}{0pt}%
\pgfpathmoveto{\pgfqpoint{1.137973in}{1.303611in}}%
\pgfpathlineto{\pgfqpoint{1.962441in}{1.303611in}}%
\pgfusepath{stroke}%
\end{pgfscope}%
\begin{pgfscope}%
\definecolor{textcolor}{rgb}{0.000000,0.000000,0.000000}%
\pgfsetstrokecolor{textcolor}%
\pgfsetfillcolor{textcolor}%
\pgftext[x=1.550207in,y=1.386944in,,base]{\color{textcolor}\rmfamily\fontsize{11.000000}{13.200000}\selectfont MetLife}%
\end{pgfscope}%
\begin{pgfscope}%
\pgfsetbuttcap%
\pgfsetmiterjoin%
\definecolor{currentfill}{rgb}{1.000000,1.000000,1.000000}%
\pgfsetfillcolor{currentfill}%
\pgfsetlinewidth{0.000000pt}%
\definecolor{currentstroke}{rgb}{0.000000,0.000000,0.000000}%
\pgfsetstrokecolor{currentstroke}%
\pgfsetstrokeopacity{0.000000}%
\pgfsetdash{}{0pt}%
\pgfpathmoveto{\pgfqpoint{2.127335in}{0.841611in}}%
\pgfpathlineto{\pgfqpoint{2.951803in}{0.841611in}}%
\pgfpathlineto{\pgfqpoint{2.951803in}{1.303611in}}%
\pgfpathlineto{\pgfqpoint{2.127335in}{1.303611in}}%
\pgfpathlineto{\pgfqpoint{2.127335in}{0.841611in}}%
\pgfpathclose%
\pgfusepath{fill}%
\end{pgfscope}%
\begin{pgfscope}%
\pgfpathrectangle{\pgfqpoint{2.127335in}{0.841611in}}{\pgfqpoint{0.824468in}{0.462000in}}%
\pgfusepath{clip}%
\pgfsetbuttcap%
\pgfsetmiterjoin%
\definecolor{currentfill}{rgb}{0.121569,0.466667,0.705882}%
\pgfsetfillcolor{currentfill}%
\pgfsetfillopacity{0.500000}%
\pgfsetlinewidth{1.003750pt}%
\definecolor{currentstroke}{rgb}{0.000000,0.000000,0.000000}%
\pgfsetstrokecolor{currentstroke}%
\pgfsetdash{}{0pt}%
\pgfpathmoveto{\pgfqpoint{2.164810in}{0.841611in}}%
\pgfpathlineto{\pgfqpoint{2.314714in}{0.841611in}}%
\pgfpathlineto{\pgfqpoint{2.314714in}{1.281611in}}%
\pgfpathlineto{\pgfqpoint{2.164810in}{1.281611in}}%
\pgfpathlineto{\pgfqpoint{2.164810in}{0.841611in}}%
\pgfpathclose%
\pgfusepath{stroke,fill}%
\end{pgfscope}%
\begin{pgfscope}%
\pgfpathrectangle{\pgfqpoint{2.127335in}{0.841611in}}{\pgfqpoint{0.824468in}{0.462000in}}%
\pgfusepath{clip}%
\pgfsetbuttcap%
\pgfsetmiterjoin%
\definecolor{currentfill}{rgb}{0.121569,0.466667,0.705882}%
\pgfsetfillcolor{currentfill}%
\pgfsetfillopacity{0.500000}%
\pgfsetlinewidth{1.003750pt}%
\definecolor{currentstroke}{rgb}{0.000000,0.000000,0.000000}%
\pgfsetstrokecolor{currentstroke}%
\pgfsetdash{}{0pt}%
\pgfpathmoveto{\pgfqpoint{2.314714in}{0.841611in}}%
\pgfpathlineto{\pgfqpoint{2.464617in}{0.841611in}}%
\pgfpathlineto{\pgfqpoint{2.464617in}{1.024050in}}%
\pgfpathlineto{\pgfqpoint{2.314714in}{1.024050in}}%
\pgfpathlineto{\pgfqpoint{2.314714in}{0.841611in}}%
\pgfpathclose%
\pgfusepath{stroke,fill}%
\end{pgfscope}%
\begin{pgfscope}%
\pgfpathrectangle{\pgfqpoint{2.127335in}{0.841611in}}{\pgfqpoint{0.824468in}{0.462000in}}%
\pgfusepath{clip}%
\pgfsetbuttcap%
\pgfsetmiterjoin%
\definecolor{currentfill}{rgb}{0.121569,0.466667,0.705882}%
\pgfsetfillcolor{currentfill}%
\pgfsetfillopacity{0.500000}%
\pgfsetlinewidth{1.003750pt}%
\definecolor{currentstroke}{rgb}{0.000000,0.000000,0.000000}%
\pgfsetstrokecolor{currentstroke}%
\pgfsetdash{}{0pt}%
\pgfpathmoveto{\pgfqpoint{2.464617in}{0.841611in}}%
\pgfpathlineto{\pgfqpoint{2.614520in}{0.841611in}}%
\pgfpathlineto{\pgfqpoint{2.614520in}{0.938196in}}%
\pgfpathlineto{\pgfqpoint{2.464617in}{0.938196in}}%
\pgfpathlineto{\pgfqpoint{2.464617in}{0.841611in}}%
\pgfpathclose%
\pgfusepath{stroke,fill}%
\end{pgfscope}%
\begin{pgfscope}%
\pgfpathrectangle{\pgfqpoint{2.127335in}{0.841611in}}{\pgfqpoint{0.824468in}{0.462000in}}%
\pgfusepath{clip}%
\pgfsetbuttcap%
\pgfsetmiterjoin%
\definecolor{currentfill}{rgb}{0.121569,0.466667,0.705882}%
\pgfsetfillcolor{currentfill}%
\pgfsetfillopacity{0.500000}%
\pgfsetlinewidth{1.003750pt}%
\definecolor{currentstroke}{rgb}{0.000000,0.000000,0.000000}%
\pgfsetstrokecolor{currentstroke}%
\pgfsetdash{}{0pt}%
\pgfpathmoveto{\pgfqpoint{2.614520in}{0.841611in}}%
\pgfpathlineto{\pgfqpoint{2.764423in}{0.841611in}}%
\pgfpathlineto{\pgfqpoint{2.764423in}{0.868440in}}%
\pgfpathlineto{\pgfqpoint{2.614520in}{0.868440in}}%
\pgfpathlineto{\pgfqpoint{2.614520in}{0.841611in}}%
\pgfpathclose%
\pgfusepath{stroke,fill}%
\end{pgfscope}%
\begin{pgfscope}%
\pgfpathrectangle{\pgfqpoint{2.127335in}{0.841611in}}{\pgfqpoint{0.824468in}{0.462000in}}%
\pgfusepath{clip}%
\pgfsetbuttcap%
\pgfsetmiterjoin%
\definecolor{currentfill}{rgb}{0.121569,0.466667,0.705882}%
\pgfsetfillcolor{currentfill}%
\pgfsetfillopacity{0.500000}%
\pgfsetlinewidth{1.003750pt}%
\definecolor{currentstroke}{rgb}{0.000000,0.000000,0.000000}%
\pgfsetstrokecolor{currentstroke}%
\pgfsetdash{}{0pt}%
\pgfpathmoveto{\pgfqpoint{2.764423in}{0.841611in}}%
\pgfpathlineto{\pgfqpoint{2.914327in}{0.841611in}}%
\pgfpathlineto{\pgfqpoint{2.914327in}{0.879172in}}%
\pgfpathlineto{\pgfqpoint{2.764423in}{0.879172in}}%
\pgfpathlineto{\pgfqpoint{2.764423in}{0.841611in}}%
\pgfpathclose%
\pgfusepath{stroke,fill}%
\end{pgfscope}%
\begin{pgfscope}%
\pgfsetrectcap%
\pgfsetmiterjoin%
\pgfsetlinewidth{0.803000pt}%
\definecolor{currentstroke}{rgb}{0.000000,0.000000,0.000000}%
\pgfsetstrokecolor{currentstroke}%
\pgfsetdash{}{0pt}%
\pgfpathmoveto{\pgfqpoint{2.127335in}{0.841611in}}%
\pgfpathlineto{\pgfqpoint{2.127335in}{1.303611in}}%
\pgfusepath{stroke}%
\end{pgfscope}%
\begin{pgfscope}%
\pgfsetrectcap%
\pgfsetmiterjoin%
\pgfsetlinewidth{0.803000pt}%
\definecolor{currentstroke}{rgb}{0.000000,0.000000,0.000000}%
\pgfsetstrokecolor{currentstroke}%
\pgfsetdash{}{0pt}%
\pgfpathmoveto{\pgfqpoint{2.951803in}{0.841611in}}%
\pgfpathlineto{\pgfqpoint{2.951803in}{1.303611in}}%
\pgfusepath{stroke}%
\end{pgfscope}%
\begin{pgfscope}%
\pgfsetrectcap%
\pgfsetmiterjoin%
\pgfsetlinewidth{0.803000pt}%
\definecolor{currentstroke}{rgb}{0.000000,0.000000,0.000000}%
\pgfsetstrokecolor{currentstroke}%
\pgfsetdash{}{0pt}%
\pgfpathmoveto{\pgfqpoint{2.127335in}{0.841611in}}%
\pgfpathlineto{\pgfqpoint{2.951803in}{0.841611in}}%
\pgfusepath{stroke}%
\end{pgfscope}%
\begin{pgfscope}%
\pgfsetrectcap%
\pgfsetmiterjoin%
\pgfsetlinewidth{0.803000pt}%
\definecolor{currentstroke}{rgb}{0.000000,0.000000,0.000000}%
\pgfsetstrokecolor{currentstroke}%
\pgfsetdash{}{0pt}%
\pgfpathmoveto{\pgfqpoint{2.127335in}{1.303611in}}%
\pgfpathlineto{\pgfqpoint{2.951803in}{1.303611in}}%
\pgfusepath{stroke}%
\end{pgfscope}%
\begin{pgfscope}%
\definecolor{textcolor}{rgb}{0.000000,0.000000,0.000000}%
\pgfsetstrokecolor{textcolor}%
\pgfsetfillcolor{textcolor}%
\pgftext[x=2.539569in,y=1.386944in,,base]{\color{textcolor}\rmfamily\fontsize{11.000000}{13.200000}\selectfont Crédit...}%
\end{pgfscope}%
\begin{pgfscope}%
\pgfsetbuttcap%
\pgfsetmiterjoin%
\definecolor{currentfill}{rgb}{1.000000,1.000000,1.000000}%
\pgfsetfillcolor{currentfill}%
\pgfsetlinewidth{0.000000pt}%
\definecolor{currentstroke}{rgb}{0.000000,0.000000,0.000000}%
\pgfsetstrokecolor{currentstroke}%
\pgfsetstrokeopacity{0.000000}%
\pgfsetdash{}{0pt}%
\pgfpathmoveto{\pgfqpoint{3.116696in}{0.841611in}}%
\pgfpathlineto{\pgfqpoint{3.941164in}{0.841611in}}%
\pgfpathlineto{\pgfqpoint{3.941164in}{1.303611in}}%
\pgfpathlineto{\pgfqpoint{3.116696in}{1.303611in}}%
\pgfpathlineto{\pgfqpoint{3.116696in}{0.841611in}}%
\pgfpathclose%
\pgfusepath{fill}%
\end{pgfscope}%
\begin{pgfscope}%
\pgfpathrectangle{\pgfqpoint{3.116696in}{0.841611in}}{\pgfqpoint{0.824468in}{0.462000in}}%
\pgfusepath{clip}%
\pgfsetbuttcap%
\pgfsetmiterjoin%
\definecolor{currentfill}{rgb}{0.121569,0.466667,0.705882}%
\pgfsetfillcolor{currentfill}%
\pgfsetfillopacity{0.500000}%
\pgfsetlinewidth{1.003750pt}%
\definecolor{currentstroke}{rgb}{0.000000,0.000000,0.000000}%
\pgfsetstrokecolor{currentstroke}%
\pgfsetdash{}{0pt}%
\pgfpathmoveto{\pgfqpoint{3.154172in}{0.841611in}}%
\pgfpathlineto{\pgfqpoint{3.304075in}{0.841611in}}%
\pgfpathlineto{\pgfqpoint{3.304075in}{1.281611in}}%
\pgfpathlineto{\pgfqpoint{3.154172in}{1.281611in}}%
\pgfpathlineto{\pgfqpoint{3.154172in}{0.841611in}}%
\pgfpathclose%
\pgfusepath{stroke,fill}%
\end{pgfscope}%
\begin{pgfscope}%
\pgfpathrectangle{\pgfqpoint{3.116696in}{0.841611in}}{\pgfqpoint{0.824468in}{0.462000in}}%
\pgfusepath{clip}%
\pgfsetbuttcap%
\pgfsetmiterjoin%
\definecolor{currentfill}{rgb}{0.121569,0.466667,0.705882}%
\pgfsetfillcolor{currentfill}%
\pgfsetfillopacity{0.500000}%
\pgfsetlinewidth{1.003750pt}%
\definecolor{currentstroke}{rgb}{0.000000,0.000000,0.000000}%
\pgfsetstrokecolor{currentstroke}%
\pgfsetdash{}{0pt}%
\pgfpathmoveto{\pgfqpoint{3.304075in}{0.841611in}}%
\pgfpathlineto{\pgfqpoint{3.453979in}{0.841611in}}%
\pgfpathlineto{\pgfqpoint{3.453979in}{0.951611in}}%
\pgfpathlineto{\pgfqpoint{3.304075in}{0.951611in}}%
\pgfpathlineto{\pgfqpoint{3.304075in}{0.841611in}}%
\pgfpathclose%
\pgfusepath{stroke,fill}%
\end{pgfscope}%
\begin{pgfscope}%
\pgfpathrectangle{\pgfqpoint{3.116696in}{0.841611in}}{\pgfqpoint{0.824468in}{0.462000in}}%
\pgfusepath{clip}%
\pgfsetbuttcap%
\pgfsetmiterjoin%
\definecolor{currentfill}{rgb}{0.121569,0.466667,0.705882}%
\pgfsetfillcolor{currentfill}%
\pgfsetfillopacity{0.500000}%
\pgfsetlinewidth{1.003750pt}%
\definecolor{currentstroke}{rgb}{0.000000,0.000000,0.000000}%
\pgfsetstrokecolor{currentstroke}%
\pgfsetdash{}{0pt}%
\pgfpathmoveto{\pgfqpoint{3.453979in}{0.841611in}}%
\pgfpathlineto{\pgfqpoint{3.603882in}{0.841611in}}%
\pgfpathlineto{\pgfqpoint{3.603882in}{0.841611in}}%
\pgfpathlineto{\pgfqpoint{3.453979in}{0.841611in}}%
\pgfpathlineto{\pgfqpoint{3.453979in}{0.841611in}}%
\pgfpathclose%
\pgfusepath{stroke,fill}%
\end{pgfscope}%
\begin{pgfscope}%
\pgfpathrectangle{\pgfqpoint{3.116696in}{0.841611in}}{\pgfqpoint{0.824468in}{0.462000in}}%
\pgfusepath{clip}%
\pgfsetbuttcap%
\pgfsetmiterjoin%
\definecolor{currentfill}{rgb}{0.121569,0.466667,0.705882}%
\pgfsetfillcolor{currentfill}%
\pgfsetfillopacity{0.500000}%
\pgfsetlinewidth{1.003750pt}%
\definecolor{currentstroke}{rgb}{0.000000,0.000000,0.000000}%
\pgfsetstrokecolor{currentstroke}%
\pgfsetdash{}{0pt}%
\pgfpathmoveto{\pgfqpoint{3.603882in}{0.841611in}}%
\pgfpathlineto{\pgfqpoint{3.753785in}{0.841611in}}%
\pgfpathlineto{\pgfqpoint{3.753785in}{0.896611in}}%
\pgfpathlineto{\pgfqpoint{3.603882in}{0.896611in}}%
\pgfpathlineto{\pgfqpoint{3.603882in}{0.841611in}}%
\pgfpathclose%
\pgfusepath{stroke,fill}%
\end{pgfscope}%
\begin{pgfscope}%
\pgfpathrectangle{\pgfqpoint{3.116696in}{0.841611in}}{\pgfqpoint{0.824468in}{0.462000in}}%
\pgfusepath{clip}%
\pgfsetbuttcap%
\pgfsetmiterjoin%
\definecolor{currentfill}{rgb}{0.121569,0.466667,0.705882}%
\pgfsetfillcolor{currentfill}%
\pgfsetfillopacity{0.500000}%
\pgfsetlinewidth{1.003750pt}%
\definecolor{currentstroke}{rgb}{0.000000,0.000000,0.000000}%
\pgfsetstrokecolor{currentstroke}%
\pgfsetdash{}{0pt}%
\pgfpathmoveto{\pgfqpoint{3.753785in}{0.841611in}}%
\pgfpathlineto{\pgfqpoint{3.903688in}{0.841611in}}%
\pgfpathlineto{\pgfqpoint{3.903688in}{0.951611in}}%
\pgfpathlineto{\pgfqpoint{3.753785in}{0.951611in}}%
\pgfpathlineto{\pgfqpoint{3.753785in}{0.841611in}}%
\pgfpathclose%
\pgfusepath{stroke,fill}%
\end{pgfscope}%
\begin{pgfscope}%
\pgfsetrectcap%
\pgfsetmiterjoin%
\pgfsetlinewidth{0.803000pt}%
\definecolor{currentstroke}{rgb}{0.000000,0.000000,0.000000}%
\pgfsetstrokecolor{currentstroke}%
\pgfsetdash{}{0pt}%
\pgfpathmoveto{\pgfqpoint{3.116696in}{0.841611in}}%
\pgfpathlineto{\pgfqpoint{3.116696in}{1.303611in}}%
\pgfusepath{stroke}%
\end{pgfscope}%
\begin{pgfscope}%
\pgfsetrectcap%
\pgfsetmiterjoin%
\pgfsetlinewidth{0.803000pt}%
\definecolor{currentstroke}{rgb}{0.000000,0.000000,0.000000}%
\pgfsetstrokecolor{currentstroke}%
\pgfsetdash{}{0pt}%
\pgfpathmoveto{\pgfqpoint{3.941164in}{0.841611in}}%
\pgfpathlineto{\pgfqpoint{3.941164in}{1.303611in}}%
\pgfusepath{stroke}%
\end{pgfscope}%
\begin{pgfscope}%
\pgfsetrectcap%
\pgfsetmiterjoin%
\pgfsetlinewidth{0.803000pt}%
\definecolor{currentstroke}{rgb}{0.000000,0.000000,0.000000}%
\pgfsetstrokecolor{currentstroke}%
\pgfsetdash{}{0pt}%
\pgfpathmoveto{\pgfqpoint{3.116696in}{0.841611in}}%
\pgfpathlineto{\pgfqpoint{3.941164in}{0.841611in}}%
\pgfusepath{stroke}%
\end{pgfscope}%
\begin{pgfscope}%
\pgfsetrectcap%
\pgfsetmiterjoin%
\pgfsetlinewidth{0.803000pt}%
\definecolor{currentstroke}{rgb}{0.000000,0.000000,0.000000}%
\pgfsetstrokecolor{currentstroke}%
\pgfsetdash{}{0pt}%
\pgfpathmoveto{\pgfqpoint{3.116696in}{1.303611in}}%
\pgfpathlineto{\pgfqpoint{3.941164in}{1.303611in}}%
\pgfusepath{stroke}%
\end{pgfscope}%
\begin{pgfscope}%
\definecolor{textcolor}{rgb}{0.000000,0.000000,0.000000}%
\pgfsetstrokecolor{textcolor}%
\pgfsetfillcolor{textcolor}%
\pgftext[x=3.528930in,y=1.386944in,,base]{\color{textcolor}\rmfamily\fontsize{11.000000}{13.200000}\selectfont Afi Esca}%
\end{pgfscope}%
\begin{pgfscope}%
\pgfsetbuttcap%
\pgfsetmiterjoin%
\definecolor{currentfill}{rgb}{1.000000,1.000000,1.000000}%
\pgfsetfillcolor{currentfill}%
\pgfsetlinewidth{0.000000pt}%
\definecolor{currentstroke}{rgb}{0.000000,0.000000,0.000000}%
\pgfsetstrokecolor{currentstroke}%
\pgfsetstrokeopacity{0.000000}%
\pgfsetdash{}{0pt}%
\pgfpathmoveto{\pgfqpoint{4.106058in}{0.841611in}}%
\pgfpathlineto{\pgfqpoint{4.930526in}{0.841611in}}%
\pgfpathlineto{\pgfqpoint{4.930526in}{1.303611in}}%
\pgfpathlineto{\pgfqpoint{4.106058in}{1.303611in}}%
\pgfpathlineto{\pgfqpoint{4.106058in}{0.841611in}}%
\pgfpathclose%
\pgfusepath{fill}%
\end{pgfscope}%
\begin{pgfscope}%
\pgfpathrectangle{\pgfqpoint{4.106058in}{0.841611in}}{\pgfqpoint{0.824468in}{0.462000in}}%
\pgfusepath{clip}%
\pgfsetbuttcap%
\pgfsetmiterjoin%
\definecolor{currentfill}{rgb}{0.121569,0.466667,0.705882}%
\pgfsetfillcolor{currentfill}%
\pgfsetfillopacity{0.500000}%
\pgfsetlinewidth{1.003750pt}%
\definecolor{currentstroke}{rgb}{0.000000,0.000000,0.000000}%
\pgfsetstrokecolor{currentstroke}%
\pgfsetdash{}{0pt}%
\pgfpathmoveto{\pgfqpoint{4.143534in}{0.841611in}}%
\pgfpathlineto{\pgfqpoint{4.293437in}{0.841611in}}%
\pgfpathlineto{\pgfqpoint{4.293437in}{1.281611in}}%
\pgfpathlineto{\pgfqpoint{4.143534in}{1.281611in}}%
\pgfpathlineto{\pgfqpoint{4.143534in}{0.841611in}}%
\pgfpathclose%
\pgfusepath{stroke,fill}%
\end{pgfscope}%
\begin{pgfscope}%
\pgfpathrectangle{\pgfqpoint{4.106058in}{0.841611in}}{\pgfqpoint{0.824468in}{0.462000in}}%
\pgfusepath{clip}%
\pgfsetbuttcap%
\pgfsetmiterjoin%
\definecolor{currentfill}{rgb}{0.121569,0.466667,0.705882}%
\pgfsetfillcolor{currentfill}%
\pgfsetfillopacity{0.500000}%
\pgfsetlinewidth{1.003750pt}%
\definecolor{currentstroke}{rgb}{0.000000,0.000000,0.000000}%
\pgfsetstrokecolor{currentstroke}%
\pgfsetdash{}{0pt}%
\pgfpathmoveto{\pgfqpoint{4.293437in}{0.841611in}}%
\pgfpathlineto{\pgfqpoint{4.443340in}{0.841611in}}%
\pgfpathlineto{\pgfqpoint{4.443340in}{0.892380in}}%
\pgfpathlineto{\pgfqpoint{4.293437in}{0.892380in}}%
\pgfpathlineto{\pgfqpoint{4.293437in}{0.841611in}}%
\pgfpathclose%
\pgfusepath{stroke,fill}%
\end{pgfscope}%
\begin{pgfscope}%
\pgfpathrectangle{\pgfqpoint{4.106058in}{0.841611in}}{\pgfqpoint{0.824468in}{0.462000in}}%
\pgfusepath{clip}%
\pgfsetbuttcap%
\pgfsetmiterjoin%
\definecolor{currentfill}{rgb}{0.121569,0.466667,0.705882}%
\pgfsetfillcolor{currentfill}%
\pgfsetfillopacity{0.500000}%
\pgfsetlinewidth{1.003750pt}%
\definecolor{currentstroke}{rgb}{0.000000,0.000000,0.000000}%
\pgfsetstrokecolor{currentstroke}%
\pgfsetdash{}{0pt}%
\pgfpathmoveto{\pgfqpoint{4.443340in}{0.841611in}}%
\pgfpathlineto{\pgfqpoint{4.593244in}{0.841611in}}%
\pgfpathlineto{\pgfqpoint{4.593244in}{0.858534in}}%
\pgfpathlineto{\pgfqpoint{4.443340in}{0.858534in}}%
\pgfpathlineto{\pgfqpoint{4.443340in}{0.841611in}}%
\pgfpathclose%
\pgfusepath{stroke,fill}%
\end{pgfscope}%
\begin{pgfscope}%
\pgfpathrectangle{\pgfqpoint{4.106058in}{0.841611in}}{\pgfqpoint{0.824468in}{0.462000in}}%
\pgfusepath{clip}%
\pgfsetbuttcap%
\pgfsetmiterjoin%
\definecolor{currentfill}{rgb}{0.121569,0.466667,0.705882}%
\pgfsetfillcolor{currentfill}%
\pgfsetfillopacity{0.500000}%
\pgfsetlinewidth{1.003750pt}%
\definecolor{currentstroke}{rgb}{0.000000,0.000000,0.000000}%
\pgfsetstrokecolor{currentstroke}%
\pgfsetdash{}{0pt}%
\pgfpathmoveto{\pgfqpoint{4.593244in}{0.841611in}}%
\pgfpathlineto{\pgfqpoint{4.743147in}{0.841611in}}%
\pgfpathlineto{\pgfqpoint{4.743147in}{0.858534in}}%
\pgfpathlineto{\pgfqpoint{4.593244in}{0.858534in}}%
\pgfpathlineto{\pgfqpoint{4.593244in}{0.841611in}}%
\pgfpathclose%
\pgfusepath{stroke,fill}%
\end{pgfscope}%
\begin{pgfscope}%
\pgfpathrectangle{\pgfqpoint{4.106058in}{0.841611in}}{\pgfqpoint{0.824468in}{0.462000in}}%
\pgfusepath{clip}%
\pgfsetbuttcap%
\pgfsetmiterjoin%
\definecolor{currentfill}{rgb}{0.121569,0.466667,0.705882}%
\pgfsetfillcolor{currentfill}%
\pgfsetfillopacity{0.500000}%
\pgfsetlinewidth{1.003750pt}%
\definecolor{currentstroke}{rgb}{0.000000,0.000000,0.000000}%
\pgfsetstrokecolor{currentstroke}%
\pgfsetdash{}{0pt}%
\pgfpathmoveto{\pgfqpoint{4.743147in}{0.841611in}}%
\pgfpathlineto{\pgfqpoint{4.893050in}{0.841611in}}%
\pgfpathlineto{\pgfqpoint{4.893050in}{0.875457in}}%
\pgfpathlineto{\pgfqpoint{4.743147in}{0.875457in}}%
\pgfpathlineto{\pgfqpoint{4.743147in}{0.841611in}}%
\pgfpathclose%
\pgfusepath{stroke,fill}%
\end{pgfscope}%
\begin{pgfscope}%
\pgfsetrectcap%
\pgfsetmiterjoin%
\pgfsetlinewidth{0.803000pt}%
\definecolor{currentstroke}{rgb}{0.000000,0.000000,0.000000}%
\pgfsetstrokecolor{currentstroke}%
\pgfsetdash{}{0pt}%
\pgfpathmoveto{\pgfqpoint{4.106058in}{0.841611in}}%
\pgfpathlineto{\pgfqpoint{4.106058in}{1.303611in}}%
\pgfusepath{stroke}%
\end{pgfscope}%
\begin{pgfscope}%
\pgfsetrectcap%
\pgfsetmiterjoin%
\pgfsetlinewidth{0.803000pt}%
\definecolor{currentstroke}{rgb}{0.000000,0.000000,0.000000}%
\pgfsetstrokecolor{currentstroke}%
\pgfsetdash{}{0pt}%
\pgfpathmoveto{\pgfqpoint{4.930526in}{0.841611in}}%
\pgfpathlineto{\pgfqpoint{4.930526in}{1.303611in}}%
\pgfusepath{stroke}%
\end{pgfscope}%
\begin{pgfscope}%
\pgfsetrectcap%
\pgfsetmiterjoin%
\pgfsetlinewidth{0.803000pt}%
\definecolor{currentstroke}{rgb}{0.000000,0.000000,0.000000}%
\pgfsetstrokecolor{currentstroke}%
\pgfsetdash{}{0pt}%
\pgfpathmoveto{\pgfqpoint{4.106058in}{0.841611in}}%
\pgfpathlineto{\pgfqpoint{4.930526in}{0.841611in}}%
\pgfusepath{stroke}%
\end{pgfscope}%
\begin{pgfscope}%
\pgfsetrectcap%
\pgfsetmiterjoin%
\pgfsetlinewidth{0.803000pt}%
\definecolor{currentstroke}{rgb}{0.000000,0.000000,0.000000}%
\pgfsetstrokecolor{currentstroke}%
\pgfsetdash{}{0pt}%
\pgfpathmoveto{\pgfqpoint{4.106058in}{1.303611in}}%
\pgfpathlineto{\pgfqpoint{4.930526in}{1.303611in}}%
\pgfusepath{stroke}%
\end{pgfscope}%
\begin{pgfscope}%
\definecolor{textcolor}{rgb}{0.000000,0.000000,0.000000}%
\pgfsetstrokecolor{textcolor}%
\pgfsetfillcolor{textcolor}%
\pgftext[x=4.518292in,y=1.386944in,,base]{\color{textcolor}\rmfamily\fontsize{11.000000}{13.200000}\selectfont Gan}%
\end{pgfscope}%
\begin{pgfscope}%
\pgfsetbuttcap%
\pgfsetmiterjoin%
\definecolor{currentfill}{rgb}{1.000000,1.000000,1.000000}%
\pgfsetfillcolor{currentfill}%
\pgfsetlinewidth{0.000000pt}%
\definecolor{currentstroke}{rgb}{0.000000,0.000000,0.000000}%
\pgfsetstrokecolor{currentstroke}%
\pgfsetstrokeopacity{0.000000}%
\pgfsetdash{}{0pt}%
\pgfpathmoveto{\pgfqpoint{5.095420in}{0.841611in}}%
\pgfpathlineto{\pgfqpoint{5.919888in}{0.841611in}}%
\pgfpathlineto{\pgfqpoint{5.919888in}{1.303611in}}%
\pgfpathlineto{\pgfqpoint{5.095420in}{1.303611in}}%
\pgfpathlineto{\pgfqpoint{5.095420in}{0.841611in}}%
\pgfpathclose%
\pgfusepath{fill}%
\end{pgfscope}%
\begin{pgfscope}%
\pgfpathrectangle{\pgfqpoint{5.095420in}{0.841611in}}{\pgfqpoint{0.824468in}{0.462000in}}%
\pgfusepath{clip}%
\pgfsetbuttcap%
\pgfsetmiterjoin%
\definecolor{currentfill}{rgb}{0.121569,0.466667,0.705882}%
\pgfsetfillcolor{currentfill}%
\pgfsetfillopacity{0.500000}%
\pgfsetlinewidth{1.003750pt}%
\definecolor{currentstroke}{rgb}{0.000000,0.000000,0.000000}%
\pgfsetstrokecolor{currentstroke}%
\pgfsetdash{}{0pt}%
\pgfpathmoveto{\pgfqpoint{5.132895in}{0.841611in}}%
\pgfpathlineto{\pgfqpoint{5.282799in}{0.841611in}}%
\pgfpathlineto{\pgfqpoint{5.282799in}{1.171611in}}%
\pgfpathlineto{\pgfqpoint{5.132895in}{1.171611in}}%
\pgfpathlineto{\pgfqpoint{5.132895in}{0.841611in}}%
\pgfpathclose%
\pgfusepath{stroke,fill}%
\end{pgfscope}%
\begin{pgfscope}%
\pgfpathrectangle{\pgfqpoint{5.095420in}{0.841611in}}{\pgfqpoint{0.824468in}{0.462000in}}%
\pgfusepath{clip}%
\pgfsetbuttcap%
\pgfsetmiterjoin%
\definecolor{currentfill}{rgb}{0.121569,0.466667,0.705882}%
\pgfsetfillcolor{currentfill}%
\pgfsetfillopacity{0.500000}%
\pgfsetlinewidth{1.003750pt}%
\definecolor{currentstroke}{rgb}{0.000000,0.000000,0.000000}%
\pgfsetstrokecolor{currentstroke}%
\pgfsetdash{}{0pt}%
\pgfpathmoveto{\pgfqpoint{5.282799in}{0.841611in}}%
\pgfpathlineto{\pgfqpoint{5.432702in}{0.841611in}}%
\pgfpathlineto{\pgfqpoint{5.432702in}{1.281611in}}%
\pgfpathlineto{\pgfqpoint{5.282799in}{1.281611in}}%
\pgfpathlineto{\pgfqpoint{5.282799in}{0.841611in}}%
\pgfpathclose%
\pgfusepath{stroke,fill}%
\end{pgfscope}%
\begin{pgfscope}%
\pgfpathrectangle{\pgfqpoint{5.095420in}{0.841611in}}{\pgfqpoint{0.824468in}{0.462000in}}%
\pgfusepath{clip}%
\pgfsetbuttcap%
\pgfsetmiterjoin%
\definecolor{currentfill}{rgb}{0.121569,0.466667,0.705882}%
\pgfsetfillcolor{currentfill}%
\pgfsetfillopacity{0.500000}%
\pgfsetlinewidth{1.003750pt}%
\definecolor{currentstroke}{rgb}{0.000000,0.000000,0.000000}%
\pgfsetstrokecolor{currentstroke}%
\pgfsetdash{}{0pt}%
\pgfpathmoveto{\pgfqpoint{5.432702in}{0.841611in}}%
\pgfpathlineto{\pgfqpoint{5.582605in}{0.841611in}}%
\pgfpathlineto{\pgfqpoint{5.582605in}{0.841611in}}%
\pgfpathlineto{\pgfqpoint{5.432702in}{0.841611in}}%
\pgfpathlineto{\pgfqpoint{5.432702in}{0.841611in}}%
\pgfpathclose%
\pgfusepath{stroke,fill}%
\end{pgfscope}%
\begin{pgfscope}%
\pgfpathrectangle{\pgfqpoint{5.095420in}{0.841611in}}{\pgfqpoint{0.824468in}{0.462000in}}%
\pgfusepath{clip}%
\pgfsetbuttcap%
\pgfsetmiterjoin%
\definecolor{currentfill}{rgb}{0.121569,0.466667,0.705882}%
\pgfsetfillcolor{currentfill}%
\pgfsetfillopacity{0.500000}%
\pgfsetlinewidth{1.003750pt}%
\definecolor{currentstroke}{rgb}{0.000000,0.000000,0.000000}%
\pgfsetstrokecolor{currentstroke}%
\pgfsetdash{}{0pt}%
\pgfpathmoveto{\pgfqpoint{5.582605in}{0.841611in}}%
\pgfpathlineto{\pgfqpoint{5.732509in}{0.841611in}}%
\pgfpathlineto{\pgfqpoint{5.732509in}{0.896611in}}%
\pgfpathlineto{\pgfqpoint{5.582605in}{0.896611in}}%
\pgfpathlineto{\pgfqpoint{5.582605in}{0.841611in}}%
\pgfpathclose%
\pgfusepath{stroke,fill}%
\end{pgfscope}%
\begin{pgfscope}%
\pgfpathrectangle{\pgfqpoint{5.095420in}{0.841611in}}{\pgfqpoint{0.824468in}{0.462000in}}%
\pgfusepath{clip}%
\pgfsetbuttcap%
\pgfsetmiterjoin%
\definecolor{currentfill}{rgb}{0.121569,0.466667,0.705882}%
\pgfsetfillcolor{currentfill}%
\pgfsetfillopacity{0.500000}%
\pgfsetlinewidth{1.003750pt}%
\definecolor{currentstroke}{rgb}{0.000000,0.000000,0.000000}%
\pgfsetstrokecolor{currentstroke}%
\pgfsetdash{}{0pt}%
\pgfpathmoveto{\pgfqpoint{5.732509in}{0.841611in}}%
\pgfpathlineto{\pgfqpoint{5.882412in}{0.841611in}}%
\pgfpathlineto{\pgfqpoint{5.882412in}{1.061611in}}%
\pgfpathlineto{\pgfqpoint{5.732509in}{1.061611in}}%
\pgfpathlineto{\pgfqpoint{5.732509in}{0.841611in}}%
\pgfpathclose%
\pgfusepath{stroke,fill}%
\end{pgfscope}%
\begin{pgfscope}%
\pgfsetrectcap%
\pgfsetmiterjoin%
\pgfsetlinewidth{0.803000pt}%
\definecolor{currentstroke}{rgb}{0.000000,0.000000,0.000000}%
\pgfsetstrokecolor{currentstroke}%
\pgfsetdash{}{0pt}%
\pgfpathmoveto{\pgfqpoint{5.095420in}{0.841611in}}%
\pgfpathlineto{\pgfqpoint{5.095420in}{1.303611in}}%
\pgfusepath{stroke}%
\end{pgfscope}%
\begin{pgfscope}%
\pgfsetrectcap%
\pgfsetmiterjoin%
\pgfsetlinewidth{0.803000pt}%
\definecolor{currentstroke}{rgb}{0.000000,0.000000,0.000000}%
\pgfsetstrokecolor{currentstroke}%
\pgfsetdash{}{0pt}%
\pgfpathmoveto{\pgfqpoint{5.919888in}{0.841611in}}%
\pgfpathlineto{\pgfqpoint{5.919888in}{1.303611in}}%
\pgfusepath{stroke}%
\end{pgfscope}%
\begin{pgfscope}%
\pgfsetrectcap%
\pgfsetmiterjoin%
\pgfsetlinewidth{0.803000pt}%
\definecolor{currentstroke}{rgb}{0.000000,0.000000,0.000000}%
\pgfsetstrokecolor{currentstroke}%
\pgfsetdash{}{0pt}%
\pgfpathmoveto{\pgfqpoint{5.095420in}{0.841611in}}%
\pgfpathlineto{\pgfqpoint{5.919888in}{0.841611in}}%
\pgfusepath{stroke}%
\end{pgfscope}%
\begin{pgfscope}%
\pgfsetrectcap%
\pgfsetmiterjoin%
\pgfsetlinewidth{0.803000pt}%
\definecolor{currentstroke}{rgb}{0.000000,0.000000,0.000000}%
\pgfsetstrokecolor{currentstroke}%
\pgfsetdash{}{0pt}%
\pgfpathmoveto{\pgfqpoint{5.095420in}{1.303611in}}%
\pgfpathlineto{\pgfqpoint{5.919888in}{1.303611in}}%
\pgfusepath{stroke}%
\end{pgfscope}%
\begin{pgfscope}%
\definecolor{textcolor}{rgb}{0.000000,0.000000,0.000000}%
\pgfsetstrokecolor{textcolor}%
\pgfsetfillcolor{textcolor}%
\pgftext[x=5.507654in,y=1.386944in,,base]{\color{textcolor}\rmfamily\fontsize{11.000000}{13.200000}\selectfont Magnolia}%
\end{pgfscope}%
\begin{pgfscope}%
\pgfsetbuttcap%
\pgfsetmiterjoin%
\definecolor{currentfill}{rgb}{1.000000,1.000000,1.000000}%
\pgfsetfillcolor{currentfill}%
\pgfsetlinewidth{0.000000pt}%
\definecolor{currentstroke}{rgb}{0.000000,0.000000,0.000000}%
\pgfsetstrokecolor{currentstroke}%
\pgfsetstrokeopacity{0.000000}%
\pgfsetdash{}{0pt}%
\pgfpathmoveto{\pgfqpoint{6.084781in}{0.841611in}}%
\pgfpathlineto{\pgfqpoint{6.909249in}{0.841611in}}%
\pgfpathlineto{\pgfqpoint{6.909249in}{1.303611in}}%
\pgfpathlineto{\pgfqpoint{6.084781in}{1.303611in}}%
\pgfpathlineto{\pgfqpoint{6.084781in}{0.841611in}}%
\pgfpathclose%
\pgfusepath{fill}%
\end{pgfscope}%
\begin{pgfscope}%
\pgfpathrectangle{\pgfqpoint{6.084781in}{0.841611in}}{\pgfqpoint{0.824468in}{0.462000in}}%
\pgfusepath{clip}%
\pgfsetbuttcap%
\pgfsetmiterjoin%
\definecolor{currentfill}{rgb}{0.121569,0.466667,0.705882}%
\pgfsetfillcolor{currentfill}%
\pgfsetfillopacity{0.500000}%
\pgfsetlinewidth{1.003750pt}%
\definecolor{currentstroke}{rgb}{0.000000,0.000000,0.000000}%
\pgfsetstrokecolor{currentstroke}%
\pgfsetdash{}{0pt}%
\pgfpathmoveto{\pgfqpoint{6.122257in}{0.841611in}}%
\pgfpathlineto{\pgfqpoint{6.272160in}{0.841611in}}%
\pgfpathlineto{\pgfqpoint{6.272160in}{1.281611in}}%
\pgfpathlineto{\pgfqpoint{6.122257in}{1.281611in}}%
\pgfpathlineto{\pgfqpoint{6.122257in}{0.841611in}}%
\pgfpathclose%
\pgfusepath{stroke,fill}%
\end{pgfscope}%
\begin{pgfscope}%
\pgfpathrectangle{\pgfqpoint{6.084781in}{0.841611in}}{\pgfqpoint{0.824468in}{0.462000in}}%
\pgfusepath{clip}%
\pgfsetbuttcap%
\pgfsetmiterjoin%
\definecolor{currentfill}{rgb}{0.121569,0.466667,0.705882}%
\pgfsetfillcolor{currentfill}%
\pgfsetfillopacity{0.500000}%
\pgfsetlinewidth{1.003750pt}%
\definecolor{currentstroke}{rgb}{0.000000,0.000000,0.000000}%
\pgfsetstrokecolor{currentstroke}%
\pgfsetdash{}{0pt}%
\pgfpathmoveto{\pgfqpoint{6.272160in}{0.841611in}}%
\pgfpathlineto{\pgfqpoint{6.422064in}{0.841611in}}%
\pgfpathlineto{\pgfqpoint{6.422064in}{1.093040in}}%
\pgfpathlineto{\pgfqpoint{6.272160in}{1.093040in}}%
\pgfpathlineto{\pgfqpoint{6.272160in}{0.841611in}}%
\pgfpathclose%
\pgfusepath{stroke,fill}%
\end{pgfscope}%
\begin{pgfscope}%
\pgfpathrectangle{\pgfqpoint{6.084781in}{0.841611in}}{\pgfqpoint{0.824468in}{0.462000in}}%
\pgfusepath{clip}%
\pgfsetbuttcap%
\pgfsetmiterjoin%
\definecolor{currentfill}{rgb}{0.121569,0.466667,0.705882}%
\pgfsetfillcolor{currentfill}%
\pgfsetfillopacity{0.500000}%
\pgfsetlinewidth{1.003750pt}%
\definecolor{currentstroke}{rgb}{0.000000,0.000000,0.000000}%
\pgfsetstrokecolor{currentstroke}%
\pgfsetdash{}{0pt}%
\pgfpathmoveto{\pgfqpoint{6.422064in}{0.841611in}}%
\pgfpathlineto{\pgfqpoint{6.571967in}{0.841611in}}%
\pgfpathlineto{\pgfqpoint{6.571967in}{0.967325in}}%
\pgfpathlineto{\pgfqpoint{6.422064in}{0.967325in}}%
\pgfpathlineto{\pgfqpoint{6.422064in}{0.841611in}}%
\pgfpathclose%
\pgfusepath{stroke,fill}%
\end{pgfscope}%
\begin{pgfscope}%
\pgfpathrectangle{\pgfqpoint{6.084781in}{0.841611in}}{\pgfqpoint{0.824468in}{0.462000in}}%
\pgfusepath{clip}%
\pgfsetbuttcap%
\pgfsetmiterjoin%
\definecolor{currentfill}{rgb}{0.121569,0.466667,0.705882}%
\pgfsetfillcolor{currentfill}%
\pgfsetfillopacity{0.500000}%
\pgfsetlinewidth{1.003750pt}%
\definecolor{currentstroke}{rgb}{0.000000,0.000000,0.000000}%
\pgfsetstrokecolor{currentstroke}%
\pgfsetdash{}{0pt}%
\pgfpathmoveto{\pgfqpoint{6.571967in}{0.841611in}}%
\pgfpathlineto{\pgfqpoint{6.721870in}{0.841611in}}%
\pgfpathlineto{\pgfqpoint{6.721870in}{0.873040in}}%
\pgfpathlineto{\pgfqpoint{6.571967in}{0.873040in}}%
\pgfpathlineto{\pgfqpoint{6.571967in}{0.841611in}}%
\pgfpathclose%
\pgfusepath{stroke,fill}%
\end{pgfscope}%
\begin{pgfscope}%
\pgfpathrectangle{\pgfqpoint{6.084781in}{0.841611in}}{\pgfqpoint{0.824468in}{0.462000in}}%
\pgfusepath{clip}%
\pgfsetbuttcap%
\pgfsetmiterjoin%
\definecolor{currentfill}{rgb}{0.121569,0.466667,0.705882}%
\pgfsetfillcolor{currentfill}%
\pgfsetfillopacity{0.500000}%
\pgfsetlinewidth{1.003750pt}%
\definecolor{currentstroke}{rgb}{0.000000,0.000000,0.000000}%
\pgfsetstrokecolor{currentstroke}%
\pgfsetdash{}{0pt}%
\pgfpathmoveto{\pgfqpoint{6.721870in}{0.841611in}}%
\pgfpathlineto{\pgfqpoint{6.871774in}{0.841611in}}%
\pgfpathlineto{\pgfqpoint{6.871774in}{0.841611in}}%
\pgfpathlineto{\pgfqpoint{6.721870in}{0.841611in}}%
\pgfpathlineto{\pgfqpoint{6.721870in}{0.841611in}}%
\pgfpathclose%
\pgfusepath{stroke,fill}%
\end{pgfscope}%
\begin{pgfscope}%
\pgfsetrectcap%
\pgfsetmiterjoin%
\pgfsetlinewidth{0.803000pt}%
\definecolor{currentstroke}{rgb}{0.000000,0.000000,0.000000}%
\pgfsetstrokecolor{currentstroke}%
\pgfsetdash{}{0pt}%
\pgfpathmoveto{\pgfqpoint{6.084781in}{0.841611in}}%
\pgfpathlineto{\pgfqpoint{6.084781in}{1.303611in}}%
\pgfusepath{stroke}%
\end{pgfscope}%
\begin{pgfscope}%
\pgfsetrectcap%
\pgfsetmiterjoin%
\pgfsetlinewidth{0.803000pt}%
\definecolor{currentstroke}{rgb}{0.000000,0.000000,0.000000}%
\pgfsetstrokecolor{currentstroke}%
\pgfsetdash{}{0pt}%
\pgfpathmoveto{\pgfqpoint{6.909249in}{0.841611in}}%
\pgfpathlineto{\pgfqpoint{6.909249in}{1.303611in}}%
\pgfusepath{stroke}%
\end{pgfscope}%
\begin{pgfscope}%
\pgfsetrectcap%
\pgfsetmiterjoin%
\pgfsetlinewidth{0.803000pt}%
\definecolor{currentstroke}{rgb}{0.000000,0.000000,0.000000}%
\pgfsetstrokecolor{currentstroke}%
\pgfsetdash{}{0pt}%
\pgfpathmoveto{\pgfqpoint{6.084781in}{0.841611in}}%
\pgfpathlineto{\pgfqpoint{6.909249in}{0.841611in}}%
\pgfusepath{stroke}%
\end{pgfscope}%
\begin{pgfscope}%
\pgfsetrectcap%
\pgfsetmiterjoin%
\pgfsetlinewidth{0.803000pt}%
\definecolor{currentstroke}{rgb}{0.000000,0.000000,0.000000}%
\pgfsetstrokecolor{currentstroke}%
\pgfsetdash{}{0pt}%
\pgfpathmoveto{\pgfqpoint{6.084781in}{1.303611in}}%
\pgfpathlineto{\pgfqpoint{6.909249in}{1.303611in}}%
\pgfusepath{stroke}%
\end{pgfscope}%
\begin{pgfscope}%
\definecolor{textcolor}{rgb}{0.000000,0.000000,0.000000}%
\pgfsetstrokecolor{textcolor}%
\pgfsetfillcolor{textcolor}%
\pgftext[x=6.497015in,y=1.386944in,,base]{\color{textcolor}\rmfamily\fontsize{11.000000}{13.200000}\selectfont Suravenir}%
\end{pgfscope}%
\begin{pgfscope}%
\pgfsetbuttcap%
\pgfsetmiterjoin%
\definecolor{currentfill}{rgb}{1.000000,1.000000,1.000000}%
\pgfsetfillcolor{currentfill}%
\pgfsetlinewidth{0.000000pt}%
\definecolor{currentstroke}{rgb}{0.000000,0.000000,0.000000}%
\pgfsetstrokecolor{currentstroke}%
\pgfsetstrokeopacity{0.000000}%
\pgfsetdash{}{0pt}%
\pgfpathmoveto{\pgfqpoint{7.074143in}{0.841611in}}%
\pgfpathlineto{\pgfqpoint{7.898611in}{0.841611in}}%
\pgfpathlineto{\pgfqpoint{7.898611in}{1.303611in}}%
\pgfpathlineto{\pgfqpoint{7.074143in}{1.303611in}}%
\pgfpathlineto{\pgfqpoint{7.074143in}{0.841611in}}%
\pgfpathclose%
\pgfusepath{fill}%
\end{pgfscope}%
\begin{pgfscope}%
\pgfpathrectangle{\pgfqpoint{7.074143in}{0.841611in}}{\pgfqpoint{0.824468in}{0.462000in}}%
\pgfusepath{clip}%
\pgfsetbuttcap%
\pgfsetmiterjoin%
\definecolor{currentfill}{rgb}{0.121569,0.466667,0.705882}%
\pgfsetfillcolor{currentfill}%
\pgfsetfillopacity{0.500000}%
\pgfsetlinewidth{1.003750pt}%
\definecolor{currentstroke}{rgb}{0.000000,0.000000,0.000000}%
\pgfsetstrokecolor{currentstroke}%
\pgfsetdash{}{0pt}%
\pgfpathmoveto{\pgfqpoint{7.111619in}{0.841611in}}%
\pgfpathlineto{\pgfqpoint{7.261522in}{0.841611in}}%
\pgfpathlineto{\pgfqpoint{7.261522in}{1.281611in}}%
\pgfpathlineto{\pgfqpoint{7.111619in}{1.281611in}}%
\pgfpathlineto{\pgfqpoint{7.111619in}{0.841611in}}%
\pgfpathclose%
\pgfusepath{stroke,fill}%
\end{pgfscope}%
\begin{pgfscope}%
\pgfpathrectangle{\pgfqpoint{7.074143in}{0.841611in}}{\pgfqpoint{0.824468in}{0.462000in}}%
\pgfusepath{clip}%
\pgfsetbuttcap%
\pgfsetmiterjoin%
\definecolor{currentfill}{rgb}{0.121569,0.466667,0.705882}%
\pgfsetfillcolor{currentfill}%
\pgfsetfillopacity{0.500000}%
\pgfsetlinewidth{1.003750pt}%
\definecolor{currentstroke}{rgb}{0.000000,0.000000,0.000000}%
\pgfsetstrokecolor{currentstroke}%
\pgfsetdash{}{0pt}%
\pgfpathmoveto{\pgfqpoint{7.261522in}{0.841611in}}%
\pgfpathlineto{\pgfqpoint{7.411425in}{0.841611in}}%
\pgfpathlineto{\pgfqpoint{7.411425in}{1.051135in}}%
\pgfpathlineto{\pgfqpoint{7.261522in}{1.051135in}}%
\pgfpathlineto{\pgfqpoint{7.261522in}{0.841611in}}%
\pgfpathclose%
\pgfusepath{stroke,fill}%
\end{pgfscope}%
\begin{pgfscope}%
\pgfpathrectangle{\pgfqpoint{7.074143in}{0.841611in}}{\pgfqpoint{0.824468in}{0.462000in}}%
\pgfusepath{clip}%
\pgfsetbuttcap%
\pgfsetmiterjoin%
\definecolor{currentfill}{rgb}{0.121569,0.466667,0.705882}%
\pgfsetfillcolor{currentfill}%
\pgfsetfillopacity{0.500000}%
\pgfsetlinewidth{1.003750pt}%
\definecolor{currentstroke}{rgb}{0.000000,0.000000,0.000000}%
\pgfsetstrokecolor{currentstroke}%
\pgfsetdash{}{0pt}%
\pgfpathmoveto{\pgfqpoint{7.411425in}{0.841611in}}%
\pgfpathlineto{\pgfqpoint{7.561329in}{0.841611in}}%
\pgfpathlineto{\pgfqpoint{7.561329in}{0.925421in}}%
\pgfpathlineto{\pgfqpoint{7.411425in}{0.925421in}}%
\pgfpathlineto{\pgfqpoint{7.411425in}{0.841611in}}%
\pgfpathclose%
\pgfusepath{stroke,fill}%
\end{pgfscope}%
\begin{pgfscope}%
\pgfpathrectangle{\pgfqpoint{7.074143in}{0.841611in}}{\pgfqpoint{0.824468in}{0.462000in}}%
\pgfusepath{clip}%
\pgfsetbuttcap%
\pgfsetmiterjoin%
\definecolor{currentfill}{rgb}{0.121569,0.466667,0.705882}%
\pgfsetfillcolor{currentfill}%
\pgfsetfillopacity{0.500000}%
\pgfsetlinewidth{1.003750pt}%
\definecolor{currentstroke}{rgb}{0.000000,0.000000,0.000000}%
\pgfsetstrokecolor{currentstroke}%
\pgfsetdash{}{0pt}%
\pgfpathmoveto{\pgfqpoint{7.561329in}{0.841611in}}%
\pgfpathlineto{\pgfqpoint{7.711232in}{0.841611in}}%
\pgfpathlineto{\pgfqpoint{7.711232in}{0.862563in}}%
\pgfpathlineto{\pgfqpoint{7.561329in}{0.862563in}}%
\pgfpathlineto{\pgfqpoint{7.561329in}{0.841611in}}%
\pgfpathclose%
\pgfusepath{stroke,fill}%
\end{pgfscope}%
\begin{pgfscope}%
\pgfpathrectangle{\pgfqpoint{7.074143in}{0.841611in}}{\pgfqpoint{0.824468in}{0.462000in}}%
\pgfusepath{clip}%
\pgfsetbuttcap%
\pgfsetmiterjoin%
\definecolor{currentfill}{rgb}{0.121569,0.466667,0.705882}%
\pgfsetfillcolor{currentfill}%
\pgfsetfillopacity{0.500000}%
\pgfsetlinewidth{1.003750pt}%
\definecolor{currentstroke}{rgb}{0.000000,0.000000,0.000000}%
\pgfsetstrokecolor{currentstroke}%
\pgfsetdash{}{0pt}%
\pgfpathmoveto{\pgfqpoint{7.711232in}{0.841611in}}%
\pgfpathlineto{\pgfqpoint{7.861135in}{0.841611in}}%
\pgfpathlineto{\pgfqpoint{7.861135in}{1.155897in}}%
\pgfpathlineto{\pgfqpoint{7.711232in}{1.155897in}}%
\pgfpathlineto{\pgfqpoint{7.711232in}{0.841611in}}%
\pgfpathclose%
\pgfusepath{stroke,fill}%
\end{pgfscope}%
\begin{pgfscope}%
\pgfsetrectcap%
\pgfsetmiterjoin%
\pgfsetlinewidth{0.803000pt}%
\definecolor{currentstroke}{rgb}{0.000000,0.000000,0.000000}%
\pgfsetstrokecolor{currentstroke}%
\pgfsetdash{}{0pt}%
\pgfpathmoveto{\pgfqpoint{7.074143in}{0.841611in}}%
\pgfpathlineto{\pgfqpoint{7.074143in}{1.303611in}}%
\pgfusepath{stroke}%
\end{pgfscope}%
\begin{pgfscope}%
\pgfsetrectcap%
\pgfsetmiterjoin%
\pgfsetlinewidth{0.803000pt}%
\definecolor{currentstroke}{rgb}{0.000000,0.000000,0.000000}%
\pgfsetstrokecolor{currentstroke}%
\pgfsetdash{}{0pt}%
\pgfpathmoveto{\pgfqpoint{7.898611in}{0.841611in}}%
\pgfpathlineto{\pgfqpoint{7.898611in}{1.303611in}}%
\pgfusepath{stroke}%
\end{pgfscope}%
\begin{pgfscope}%
\pgfsetrectcap%
\pgfsetmiterjoin%
\pgfsetlinewidth{0.803000pt}%
\definecolor{currentstroke}{rgb}{0.000000,0.000000,0.000000}%
\pgfsetstrokecolor{currentstroke}%
\pgfsetdash{}{0pt}%
\pgfpathmoveto{\pgfqpoint{7.074143in}{0.841611in}}%
\pgfpathlineto{\pgfqpoint{7.898611in}{0.841611in}}%
\pgfusepath{stroke}%
\end{pgfscope}%
\begin{pgfscope}%
\pgfsetrectcap%
\pgfsetmiterjoin%
\pgfsetlinewidth{0.803000pt}%
\definecolor{currentstroke}{rgb}{0.000000,0.000000,0.000000}%
\pgfsetstrokecolor{currentstroke}%
\pgfsetdash{}{0pt}%
\pgfpathmoveto{\pgfqpoint{7.074143in}{1.303611in}}%
\pgfpathlineto{\pgfqpoint{7.898611in}{1.303611in}}%
\pgfusepath{stroke}%
\end{pgfscope}%
\begin{pgfscope}%
\definecolor{textcolor}{rgb}{0.000000,0.000000,0.000000}%
\pgfsetstrokecolor{textcolor}%
\pgfsetfillcolor{textcolor}%
\pgftext[x=7.486377in,y=1.386944in,,base]{\color{textcolor}\rmfamily\fontsize{11.000000}{13.200000}\selectfont Assur ...}%
\end{pgfscope}%
\begin{pgfscope}%
\pgfsetbuttcap%
\pgfsetmiterjoin%
\definecolor{currentfill}{rgb}{1.000000,1.000000,1.000000}%
\pgfsetfillcolor{currentfill}%
\pgfsetlinewidth{0.000000pt}%
\definecolor{currentstroke}{rgb}{0.000000,0.000000,0.000000}%
\pgfsetstrokecolor{currentstroke}%
\pgfsetstrokeopacity{0.000000}%
\pgfsetdash{}{0pt}%
\pgfpathmoveto{\pgfqpoint{0.148611in}{0.148611in}}%
\pgfpathlineto{\pgfqpoint{0.973079in}{0.148611in}}%
\pgfpathlineto{\pgfqpoint{0.973079in}{0.610611in}}%
\pgfpathlineto{\pgfqpoint{0.148611in}{0.610611in}}%
\pgfpathlineto{\pgfqpoint{0.148611in}{0.148611in}}%
\pgfpathclose%
\pgfusepath{fill}%
\end{pgfscope}%
\begin{pgfscope}%
\pgfpathrectangle{\pgfqpoint{0.148611in}{0.148611in}}{\pgfqpoint{0.824468in}{0.462000in}}%
\pgfusepath{clip}%
\pgfsetbuttcap%
\pgfsetmiterjoin%
\definecolor{currentfill}{rgb}{0.121569,0.466667,0.705882}%
\pgfsetfillcolor{currentfill}%
\pgfsetfillopacity{0.500000}%
\pgfsetlinewidth{1.003750pt}%
\definecolor{currentstroke}{rgb}{0.000000,0.000000,0.000000}%
\pgfsetstrokecolor{currentstroke}%
\pgfsetdash{}{0pt}%
\pgfpathmoveto{\pgfqpoint{0.186087in}{0.148611in}}%
\pgfpathlineto{\pgfqpoint{0.335990in}{0.148611in}}%
\pgfpathlineto{\pgfqpoint{0.335990in}{0.588611in}}%
\pgfpathlineto{\pgfqpoint{0.186087in}{0.588611in}}%
\pgfpathlineto{\pgfqpoint{0.186087in}{0.148611in}}%
\pgfpathclose%
\pgfusepath{stroke,fill}%
\end{pgfscope}%
\begin{pgfscope}%
\pgfpathrectangle{\pgfqpoint{0.148611in}{0.148611in}}{\pgfqpoint{0.824468in}{0.462000in}}%
\pgfusepath{clip}%
\pgfsetbuttcap%
\pgfsetmiterjoin%
\definecolor{currentfill}{rgb}{0.121569,0.466667,0.705882}%
\pgfsetfillcolor{currentfill}%
\pgfsetfillopacity{0.500000}%
\pgfsetlinewidth{1.003750pt}%
\definecolor{currentstroke}{rgb}{0.000000,0.000000,0.000000}%
\pgfsetstrokecolor{currentstroke}%
\pgfsetdash{}{0pt}%
\pgfpathmoveto{\pgfqpoint{0.335990in}{0.148611in}}%
\pgfpathlineto{\pgfqpoint{0.485894in}{0.148611in}}%
\pgfpathlineto{\pgfqpoint{0.485894in}{0.412611in}}%
\pgfpathlineto{\pgfqpoint{0.335990in}{0.412611in}}%
\pgfpathlineto{\pgfqpoint{0.335990in}{0.148611in}}%
\pgfpathclose%
\pgfusepath{stroke,fill}%
\end{pgfscope}%
\begin{pgfscope}%
\pgfpathrectangle{\pgfqpoint{0.148611in}{0.148611in}}{\pgfqpoint{0.824468in}{0.462000in}}%
\pgfusepath{clip}%
\pgfsetbuttcap%
\pgfsetmiterjoin%
\definecolor{currentfill}{rgb}{0.121569,0.466667,0.705882}%
\pgfsetfillcolor{currentfill}%
\pgfsetfillopacity{0.500000}%
\pgfsetlinewidth{1.003750pt}%
\definecolor{currentstroke}{rgb}{0.000000,0.000000,0.000000}%
\pgfsetstrokecolor{currentstroke}%
\pgfsetdash{}{0pt}%
\pgfpathmoveto{\pgfqpoint{0.485894in}{0.148611in}}%
\pgfpathlineto{\pgfqpoint{0.635797in}{0.148611in}}%
\pgfpathlineto{\pgfqpoint{0.635797in}{0.177944in}}%
\pgfpathlineto{\pgfqpoint{0.485894in}{0.177944in}}%
\pgfpathlineto{\pgfqpoint{0.485894in}{0.148611in}}%
\pgfpathclose%
\pgfusepath{stroke,fill}%
\end{pgfscope}%
\begin{pgfscope}%
\pgfpathrectangle{\pgfqpoint{0.148611in}{0.148611in}}{\pgfqpoint{0.824468in}{0.462000in}}%
\pgfusepath{clip}%
\pgfsetbuttcap%
\pgfsetmiterjoin%
\definecolor{currentfill}{rgb}{0.121569,0.466667,0.705882}%
\pgfsetfillcolor{currentfill}%
\pgfsetfillopacity{0.500000}%
\pgfsetlinewidth{1.003750pt}%
\definecolor{currentstroke}{rgb}{0.000000,0.000000,0.000000}%
\pgfsetstrokecolor{currentstroke}%
\pgfsetdash{}{0pt}%
\pgfpathmoveto{\pgfqpoint{0.635797in}{0.148611in}}%
\pgfpathlineto{\pgfqpoint{0.785700in}{0.148611in}}%
\pgfpathlineto{\pgfqpoint{0.785700in}{0.148611in}}%
\pgfpathlineto{\pgfqpoint{0.635797in}{0.148611in}}%
\pgfpathlineto{\pgfqpoint{0.635797in}{0.148611in}}%
\pgfpathclose%
\pgfusepath{stroke,fill}%
\end{pgfscope}%
\begin{pgfscope}%
\pgfpathrectangle{\pgfqpoint{0.148611in}{0.148611in}}{\pgfqpoint{0.824468in}{0.462000in}}%
\pgfusepath{clip}%
\pgfsetbuttcap%
\pgfsetmiterjoin%
\definecolor{currentfill}{rgb}{0.121569,0.466667,0.705882}%
\pgfsetfillcolor{currentfill}%
\pgfsetfillopacity{0.500000}%
\pgfsetlinewidth{1.003750pt}%
\definecolor{currentstroke}{rgb}{0.000000,0.000000,0.000000}%
\pgfsetstrokecolor{currentstroke}%
\pgfsetdash{}{0pt}%
\pgfpathmoveto{\pgfqpoint{0.785700in}{0.148611in}}%
\pgfpathlineto{\pgfqpoint{0.935603in}{0.148611in}}%
\pgfpathlineto{\pgfqpoint{0.935603in}{0.148611in}}%
\pgfpathlineto{\pgfqpoint{0.785700in}{0.148611in}}%
\pgfpathlineto{\pgfqpoint{0.785700in}{0.148611in}}%
\pgfpathclose%
\pgfusepath{stroke,fill}%
\end{pgfscope}%
\begin{pgfscope}%
\pgfsetrectcap%
\pgfsetmiterjoin%
\pgfsetlinewidth{0.803000pt}%
\definecolor{currentstroke}{rgb}{0.000000,0.000000,0.000000}%
\pgfsetstrokecolor{currentstroke}%
\pgfsetdash{}{0pt}%
\pgfpathmoveto{\pgfqpoint{0.148611in}{0.148611in}}%
\pgfpathlineto{\pgfqpoint{0.148611in}{0.610611in}}%
\pgfusepath{stroke}%
\end{pgfscope}%
\begin{pgfscope}%
\pgfsetrectcap%
\pgfsetmiterjoin%
\pgfsetlinewidth{0.803000pt}%
\definecolor{currentstroke}{rgb}{0.000000,0.000000,0.000000}%
\pgfsetstrokecolor{currentstroke}%
\pgfsetdash{}{0pt}%
\pgfpathmoveto{\pgfqpoint{0.973079in}{0.148611in}}%
\pgfpathlineto{\pgfqpoint{0.973079in}{0.610611in}}%
\pgfusepath{stroke}%
\end{pgfscope}%
\begin{pgfscope}%
\pgfsetrectcap%
\pgfsetmiterjoin%
\pgfsetlinewidth{0.803000pt}%
\definecolor{currentstroke}{rgb}{0.000000,0.000000,0.000000}%
\pgfsetstrokecolor{currentstroke}%
\pgfsetdash{}{0pt}%
\pgfpathmoveto{\pgfqpoint{0.148611in}{0.148611in}}%
\pgfpathlineto{\pgfqpoint{0.973079in}{0.148611in}}%
\pgfusepath{stroke}%
\end{pgfscope}%
\begin{pgfscope}%
\pgfsetrectcap%
\pgfsetmiterjoin%
\pgfsetlinewidth{0.803000pt}%
\definecolor{currentstroke}{rgb}{0.000000,0.000000,0.000000}%
\pgfsetstrokecolor{currentstroke}%
\pgfsetdash{}{0pt}%
\pgfpathmoveto{\pgfqpoint{0.148611in}{0.610611in}}%
\pgfpathlineto{\pgfqpoint{0.973079in}{0.610611in}}%
\pgfusepath{stroke}%
\end{pgfscope}%
\begin{pgfscope}%
\definecolor{textcolor}{rgb}{0.000000,0.000000,0.000000}%
\pgfsetstrokecolor{textcolor}%
\pgfsetfillcolor{textcolor}%
\pgftext[x=0.560845in,y=0.693944in,,base]{\color{textcolor}\rmfamily\fontsize{11.000000}{13.200000}\selectfont AssurO...}%
\end{pgfscope}%
\begin{pgfscope}%
\pgfsetbuttcap%
\pgfsetmiterjoin%
\definecolor{currentfill}{rgb}{1.000000,1.000000,1.000000}%
\pgfsetfillcolor{currentfill}%
\pgfsetlinewidth{0.000000pt}%
\definecolor{currentstroke}{rgb}{0.000000,0.000000,0.000000}%
\pgfsetstrokecolor{currentstroke}%
\pgfsetstrokeopacity{0.000000}%
\pgfsetdash{}{0pt}%
\pgfpathmoveto{\pgfqpoint{1.137973in}{0.148611in}}%
\pgfpathlineto{\pgfqpoint{1.962441in}{0.148611in}}%
\pgfpathlineto{\pgfqpoint{1.962441in}{0.610611in}}%
\pgfpathlineto{\pgfqpoint{1.137973in}{0.610611in}}%
\pgfpathlineto{\pgfqpoint{1.137973in}{0.148611in}}%
\pgfpathclose%
\pgfusepath{fill}%
\end{pgfscope}%
\begin{pgfscope}%
\pgfpathrectangle{\pgfqpoint{1.137973in}{0.148611in}}{\pgfqpoint{0.824468in}{0.462000in}}%
\pgfusepath{clip}%
\pgfsetbuttcap%
\pgfsetmiterjoin%
\definecolor{currentfill}{rgb}{0.121569,0.466667,0.705882}%
\pgfsetfillcolor{currentfill}%
\pgfsetfillopacity{0.500000}%
\pgfsetlinewidth{1.003750pt}%
\definecolor{currentstroke}{rgb}{0.000000,0.000000,0.000000}%
\pgfsetstrokecolor{currentstroke}%
\pgfsetdash{}{0pt}%
\pgfpathmoveto{\pgfqpoint{1.175449in}{0.148611in}}%
\pgfpathlineto{\pgfqpoint{1.325352in}{0.148611in}}%
\pgfpathlineto{\pgfqpoint{1.325352in}{0.588611in}}%
\pgfpathlineto{\pgfqpoint{1.175449in}{0.588611in}}%
\pgfpathlineto{\pgfqpoint{1.175449in}{0.148611in}}%
\pgfpathclose%
\pgfusepath{stroke,fill}%
\end{pgfscope}%
\begin{pgfscope}%
\pgfpathrectangle{\pgfqpoint{1.137973in}{0.148611in}}{\pgfqpoint{0.824468in}{0.462000in}}%
\pgfusepath{clip}%
\pgfsetbuttcap%
\pgfsetmiterjoin%
\definecolor{currentfill}{rgb}{0.121569,0.466667,0.705882}%
\pgfsetfillcolor{currentfill}%
\pgfsetfillopacity{0.500000}%
\pgfsetlinewidth{1.003750pt}%
\definecolor{currentstroke}{rgb}{0.000000,0.000000,0.000000}%
\pgfsetstrokecolor{currentstroke}%
\pgfsetdash{}{0pt}%
\pgfpathmoveto{\pgfqpoint{1.325352in}{0.148611in}}%
\pgfpathlineto{\pgfqpoint{1.475255in}{0.148611in}}%
\pgfpathlineto{\pgfqpoint{1.475255in}{0.295278in}}%
\pgfpathlineto{\pgfqpoint{1.325352in}{0.295278in}}%
\pgfpathlineto{\pgfqpoint{1.325352in}{0.148611in}}%
\pgfpathclose%
\pgfusepath{stroke,fill}%
\end{pgfscope}%
\begin{pgfscope}%
\pgfpathrectangle{\pgfqpoint{1.137973in}{0.148611in}}{\pgfqpoint{0.824468in}{0.462000in}}%
\pgfusepath{clip}%
\pgfsetbuttcap%
\pgfsetmiterjoin%
\definecolor{currentfill}{rgb}{0.121569,0.466667,0.705882}%
\pgfsetfillcolor{currentfill}%
\pgfsetfillopacity{0.500000}%
\pgfsetlinewidth{1.003750pt}%
\definecolor{currentstroke}{rgb}{0.000000,0.000000,0.000000}%
\pgfsetstrokecolor{currentstroke}%
\pgfsetdash{}{0pt}%
\pgfpathmoveto{\pgfqpoint{1.475255in}{0.148611in}}%
\pgfpathlineto{\pgfqpoint{1.625158in}{0.148611in}}%
\pgfpathlineto{\pgfqpoint{1.625158in}{0.177944in}}%
\pgfpathlineto{\pgfqpoint{1.475255in}{0.177944in}}%
\pgfpathlineto{\pgfqpoint{1.475255in}{0.148611in}}%
\pgfpathclose%
\pgfusepath{stroke,fill}%
\end{pgfscope}%
\begin{pgfscope}%
\pgfpathrectangle{\pgfqpoint{1.137973in}{0.148611in}}{\pgfqpoint{0.824468in}{0.462000in}}%
\pgfusepath{clip}%
\pgfsetbuttcap%
\pgfsetmiterjoin%
\definecolor{currentfill}{rgb}{0.121569,0.466667,0.705882}%
\pgfsetfillcolor{currentfill}%
\pgfsetfillopacity{0.500000}%
\pgfsetlinewidth{1.003750pt}%
\definecolor{currentstroke}{rgb}{0.000000,0.000000,0.000000}%
\pgfsetstrokecolor{currentstroke}%
\pgfsetdash{}{0pt}%
\pgfpathmoveto{\pgfqpoint{1.625158in}{0.148611in}}%
\pgfpathlineto{\pgfqpoint{1.775062in}{0.148611in}}%
\pgfpathlineto{\pgfqpoint{1.775062in}{0.324611in}}%
\pgfpathlineto{\pgfqpoint{1.625158in}{0.324611in}}%
\pgfpathlineto{\pgfqpoint{1.625158in}{0.148611in}}%
\pgfpathclose%
\pgfusepath{stroke,fill}%
\end{pgfscope}%
\begin{pgfscope}%
\pgfpathrectangle{\pgfqpoint{1.137973in}{0.148611in}}{\pgfqpoint{0.824468in}{0.462000in}}%
\pgfusepath{clip}%
\pgfsetbuttcap%
\pgfsetmiterjoin%
\definecolor{currentfill}{rgb}{0.121569,0.466667,0.705882}%
\pgfsetfillcolor{currentfill}%
\pgfsetfillopacity{0.500000}%
\pgfsetlinewidth{1.003750pt}%
\definecolor{currentstroke}{rgb}{0.000000,0.000000,0.000000}%
\pgfsetstrokecolor{currentstroke}%
\pgfsetdash{}{0pt}%
\pgfpathmoveto{\pgfqpoint{1.775062in}{0.148611in}}%
\pgfpathlineto{\pgfqpoint{1.924965in}{0.148611in}}%
\pgfpathlineto{\pgfqpoint{1.924965in}{0.295278in}}%
\pgfpathlineto{\pgfqpoint{1.775062in}{0.295278in}}%
\pgfpathlineto{\pgfqpoint{1.775062in}{0.148611in}}%
\pgfpathclose%
\pgfusepath{stroke,fill}%
\end{pgfscope}%
\begin{pgfscope}%
\pgfsetrectcap%
\pgfsetmiterjoin%
\pgfsetlinewidth{0.803000pt}%
\definecolor{currentstroke}{rgb}{0.000000,0.000000,0.000000}%
\pgfsetstrokecolor{currentstroke}%
\pgfsetdash{}{0pt}%
\pgfpathmoveto{\pgfqpoint{1.137973in}{0.148611in}}%
\pgfpathlineto{\pgfqpoint{1.137973in}{0.610611in}}%
\pgfusepath{stroke}%
\end{pgfscope}%
\begin{pgfscope}%
\pgfsetrectcap%
\pgfsetmiterjoin%
\pgfsetlinewidth{0.803000pt}%
\definecolor{currentstroke}{rgb}{0.000000,0.000000,0.000000}%
\pgfsetstrokecolor{currentstroke}%
\pgfsetdash{}{0pt}%
\pgfpathmoveto{\pgfqpoint{1.962441in}{0.148611in}}%
\pgfpathlineto{\pgfqpoint{1.962441in}{0.610611in}}%
\pgfusepath{stroke}%
\end{pgfscope}%
\begin{pgfscope}%
\pgfsetrectcap%
\pgfsetmiterjoin%
\pgfsetlinewidth{0.803000pt}%
\definecolor{currentstroke}{rgb}{0.000000,0.000000,0.000000}%
\pgfsetstrokecolor{currentstroke}%
\pgfsetdash{}{0pt}%
\pgfpathmoveto{\pgfqpoint{1.137973in}{0.148611in}}%
\pgfpathlineto{\pgfqpoint{1.962441in}{0.148611in}}%
\pgfusepath{stroke}%
\end{pgfscope}%
\begin{pgfscope}%
\pgfsetrectcap%
\pgfsetmiterjoin%
\pgfsetlinewidth{0.803000pt}%
\definecolor{currentstroke}{rgb}{0.000000,0.000000,0.000000}%
\pgfsetstrokecolor{currentstroke}%
\pgfsetdash{}{0pt}%
\pgfpathmoveto{\pgfqpoint{1.137973in}{0.610611in}}%
\pgfpathlineto{\pgfqpoint{1.962441in}{0.610611in}}%
\pgfusepath{stroke}%
\end{pgfscope}%
\begin{pgfscope}%
\definecolor{textcolor}{rgb}{0.000000,0.000000,0.000000}%
\pgfsetstrokecolor{textcolor}%
\pgfsetfillcolor{textcolor}%
\pgftext[x=1.550207in,y=0.693944in,,base]{\color{textcolor}\rmfamily\fontsize{11.000000}{13.200000}\selectfont Carac}%
\end{pgfscope}%
\begin{pgfscope}%
\pgfsetbuttcap%
\pgfsetmiterjoin%
\definecolor{currentfill}{rgb}{1.000000,1.000000,1.000000}%
\pgfsetfillcolor{currentfill}%
\pgfsetlinewidth{0.000000pt}%
\definecolor{currentstroke}{rgb}{0.000000,0.000000,0.000000}%
\pgfsetstrokecolor{currentstroke}%
\pgfsetstrokeopacity{0.000000}%
\pgfsetdash{}{0pt}%
\pgfpathmoveto{\pgfqpoint{2.127335in}{0.148611in}}%
\pgfpathlineto{\pgfqpoint{2.951803in}{0.148611in}}%
\pgfpathlineto{\pgfqpoint{2.951803in}{0.610611in}}%
\pgfpathlineto{\pgfqpoint{2.127335in}{0.610611in}}%
\pgfpathlineto{\pgfqpoint{2.127335in}{0.148611in}}%
\pgfpathclose%
\pgfusepath{fill}%
\end{pgfscope}%
\begin{pgfscope}%
\pgfpathrectangle{\pgfqpoint{2.127335in}{0.148611in}}{\pgfqpoint{0.824468in}{0.462000in}}%
\pgfusepath{clip}%
\pgfsetbuttcap%
\pgfsetmiterjoin%
\definecolor{currentfill}{rgb}{0.121569,0.466667,0.705882}%
\pgfsetfillcolor{currentfill}%
\pgfsetfillopacity{0.500000}%
\pgfsetlinewidth{1.003750pt}%
\definecolor{currentstroke}{rgb}{0.000000,0.000000,0.000000}%
\pgfsetstrokecolor{currentstroke}%
\pgfsetdash{}{0pt}%
\pgfpathmoveto{\pgfqpoint{2.164810in}{0.148611in}}%
\pgfpathlineto{\pgfqpoint{2.314714in}{0.148611in}}%
\pgfpathlineto{\pgfqpoint{2.314714in}{0.211468in}}%
\pgfpathlineto{\pgfqpoint{2.164810in}{0.211468in}}%
\pgfpathlineto{\pgfqpoint{2.164810in}{0.148611in}}%
\pgfpathclose%
\pgfusepath{stroke,fill}%
\end{pgfscope}%
\begin{pgfscope}%
\pgfpathrectangle{\pgfqpoint{2.127335in}{0.148611in}}{\pgfqpoint{0.824468in}{0.462000in}}%
\pgfusepath{clip}%
\pgfsetbuttcap%
\pgfsetmiterjoin%
\definecolor{currentfill}{rgb}{0.121569,0.466667,0.705882}%
\pgfsetfillcolor{currentfill}%
\pgfsetfillopacity{0.500000}%
\pgfsetlinewidth{1.003750pt}%
\definecolor{currentstroke}{rgb}{0.000000,0.000000,0.000000}%
\pgfsetstrokecolor{currentstroke}%
\pgfsetdash{}{0pt}%
\pgfpathmoveto{\pgfqpoint{2.314714in}{0.148611in}}%
\pgfpathlineto{\pgfqpoint{2.464617in}{0.148611in}}%
\pgfpathlineto{\pgfqpoint{2.464617in}{0.148611in}}%
\pgfpathlineto{\pgfqpoint{2.314714in}{0.148611in}}%
\pgfpathlineto{\pgfqpoint{2.314714in}{0.148611in}}%
\pgfpathclose%
\pgfusepath{stroke,fill}%
\end{pgfscope}%
\begin{pgfscope}%
\pgfpathrectangle{\pgfqpoint{2.127335in}{0.148611in}}{\pgfqpoint{0.824468in}{0.462000in}}%
\pgfusepath{clip}%
\pgfsetbuttcap%
\pgfsetmiterjoin%
\definecolor{currentfill}{rgb}{0.121569,0.466667,0.705882}%
\pgfsetfillcolor{currentfill}%
\pgfsetfillopacity{0.500000}%
\pgfsetlinewidth{1.003750pt}%
\definecolor{currentstroke}{rgb}{0.000000,0.000000,0.000000}%
\pgfsetstrokecolor{currentstroke}%
\pgfsetdash{}{0pt}%
\pgfpathmoveto{\pgfqpoint{2.464617in}{0.148611in}}%
\pgfpathlineto{\pgfqpoint{2.614520in}{0.148611in}}%
\pgfpathlineto{\pgfqpoint{2.614520in}{0.148611in}}%
\pgfpathlineto{\pgfqpoint{2.464617in}{0.148611in}}%
\pgfpathlineto{\pgfqpoint{2.464617in}{0.148611in}}%
\pgfpathclose%
\pgfusepath{stroke,fill}%
\end{pgfscope}%
\begin{pgfscope}%
\pgfpathrectangle{\pgfqpoint{2.127335in}{0.148611in}}{\pgfqpoint{0.824468in}{0.462000in}}%
\pgfusepath{clip}%
\pgfsetbuttcap%
\pgfsetmiterjoin%
\definecolor{currentfill}{rgb}{0.121569,0.466667,0.705882}%
\pgfsetfillcolor{currentfill}%
\pgfsetfillopacity{0.500000}%
\pgfsetlinewidth{1.003750pt}%
\definecolor{currentstroke}{rgb}{0.000000,0.000000,0.000000}%
\pgfsetstrokecolor{currentstroke}%
\pgfsetdash{}{0pt}%
\pgfpathmoveto{\pgfqpoint{2.614520in}{0.148611in}}%
\pgfpathlineto{\pgfqpoint{2.764423in}{0.148611in}}%
\pgfpathlineto{\pgfqpoint{2.764423in}{0.588611in}}%
\pgfpathlineto{\pgfqpoint{2.614520in}{0.588611in}}%
\pgfpathlineto{\pgfqpoint{2.614520in}{0.148611in}}%
\pgfpathclose%
\pgfusepath{stroke,fill}%
\end{pgfscope}%
\begin{pgfscope}%
\pgfpathrectangle{\pgfqpoint{2.127335in}{0.148611in}}{\pgfqpoint{0.824468in}{0.462000in}}%
\pgfusepath{clip}%
\pgfsetbuttcap%
\pgfsetmiterjoin%
\definecolor{currentfill}{rgb}{0.121569,0.466667,0.705882}%
\pgfsetfillcolor{currentfill}%
\pgfsetfillopacity{0.500000}%
\pgfsetlinewidth{1.003750pt}%
\definecolor{currentstroke}{rgb}{0.000000,0.000000,0.000000}%
\pgfsetstrokecolor{currentstroke}%
\pgfsetdash{}{0pt}%
\pgfpathmoveto{\pgfqpoint{2.764423in}{0.148611in}}%
\pgfpathlineto{\pgfqpoint{2.914327in}{0.148611in}}%
\pgfpathlineto{\pgfqpoint{2.914327in}{0.274325in}}%
\pgfpathlineto{\pgfqpoint{2.764423in}{0.274325in}}%
\pgfpathlineto{\pgfqpoint{2.764423in}{0.148611in}}%
\pgfpathclose%
\pgfusepath{stroke,fill}%
\end{pgfscope}%
\begin{pgfscope}%
\pgfsetrectcap%
\pgfsetmiterjoin%
\pgfsetlinewidth{0.803000pt}%
\definecolor{currentstroke}{rgb}{0.000000,0.000000,0.000000}%
\pgfsetstrokecolor{currentstroke}%
\pgfsetdash{}{0pt}%
\pgfpathmoveto{\pgfqpoint{2.127335in}{0.148611in}}%
\pgfpathlineto{\pgfqpoint{2.127335in}{0.610611in}}%
\pgfusepath{stroke}%
\end{pgfscope}%
\begin{pgfscope}%
\pgfsetrectcap%
\pgfsetmiterjoin%
\pgfsetlinewidth{0.803000pt}%
\definecolor{currentstroke}{rgb}{0.000000,0.000000,0.000000}%
\pgfsetstrokecolor{currentstroke}%
\pgfsetdash{}{0pt}%
\pgfpathmoveto{\pgfqpoint{2.951803in}{0.148611in}}%
\pgfpathlineto{\pgfqpoint{2.951803in}{0.610611in}}%
\pgfusepath{stroke}%
\end{pgfscope}%
\begin{pgfscope}%
\pgfsetrectcap%
\pgfsetmiterjoin%
\pgfsetlinewidth{0.803000pt}%
\definecolor{currentstroke}{rgb}{0.000000,0.000000,0.000000}%
\pgfsetstrokecolor{currentstroke}%
\pgfsetdash{}{0pt}%
\pgfpathmoveto{\pgfqpoint{2.127335in}{0.148611in}}%
\pgfpathlineto{\pgfqpoint{2.951803in}{0.148611in}}%
\pgfusepath{stroke}%
\end{pgfscope}%
\begin{pgfscope}%
\pgfsetrectcap%
\pgfsetmiterjoin%
\pgfsetlinewidth{0.803000pt}%
\definecolor{currentstroke}{rgb}{0.000000,0.000000,0.000000}%
\pgfsetstrokecolor{currentstroke}%
\pgfsetdash{}{0pt}%
\pgfpathmoveto{\pgfqpoint{2.127335in}{0.610611in}}%
\pgfpathlineto{\pgfqpoint{2.951803in}{0.610611in}}%
\pgfusepath{stroke}%
\end{pgfscope}%
\begin{pgfscope}%
\definecolor{textcolor}{rgb}{0.000000,0.000000,0.000000}%
\pgfsetstrokecolor{textcolor}%
\pgfsetfillcolor{textcolor}%
\pgftext[x=2.539569in,y=0.693944in,,base]{\color{textcolor}\rmfamily\fontsize{11.000000}{13.200000}\selectfont Mapa}%
\end{pgfscope}%
\begin{pgfscope}%
\pgfsetbuttcap%
\pgfsetmiterjoin%
\definecolor{currentfill}{rgb}{1.000000,1.000000,1.000000}%
\pgfsetfillcolor{currentfill}%
\pgfsetlinewidth{0.000000pt}%
\definecolor{currentstroke}{rgb}{0.000000,0.000000,0.000000}%
\pgfsetstrokecolor{currentstroke}%
\pgfsetstrokeopacity{0.000000}%
\pgfsetdash{}{0pt}%
\pgfpathmoveto{\pgfqpoint{3.116696in}{0.148611in}}%
\pgfpathlineto{\pgfqpoint{3.941164in}{0.148611in}}%
\pgfpathlineto{\pgfqpoint{3.941164in}{0.610611in}}%
\pgfpathlineto{\pgfqpoint{3.116696in}{0.610611in}}%
\pgfpathlineto{\pgfqpoint{3.116696in}{0.148611in}}%
\pgfpathclose%
\pgfusepath{fill}%
\end{pgfscope}%
\begin{pgfscope}%
\pgfpathrectangle{\pgfqpoint{3.116696in}{0.148611in}}{\pgfqpoint{0.824468in}{0.462000in}}%
\pgfusepath{clip}%
\pgfsetbuttcap%
\pgfsetmiterjoin%
\definecolor{currentfill}{rgb}{0.121569,0.466667,0.705882}%
\pgfsetfillcolor{currentfill}%
\pgfsetfillopacity{0.500000}%
\pgfsetlinewidth{1.003750pt}%
\definecolor{currentstroke}{rgb}{0.000000,0.000000,0.000000}%
\pgfsetstrokecolor{currentstroke}%
\pgfsetdash{}{0pt}%
\pgfpathmoveto{\pgfqpoint{3.154172in}{0.148611in}}%
\pgfpathlineto{\pgfqpoint{3.304075in}{0.148611in}}%
\pgfpathlineto{\pgfqpoint{3.304075in}{0.588611in}}%
\pgfpathlineto{\pgfqpoint{3.154172in}{0.588611in}}%
\pgfpathlineto{\pgfqpoint{3.154172in}{0.148611in}}%
\pgfpathclose%
\pgfusepath{stroke,fill}%
\end{pgfscope}%
\begin{pgfscope}%
\pgfpathrectangle{\pgfqpoint{3.116696in}{0.148611in}}{\pgfqpoint{0.824468in}{0.462000in}}%
\pgfusepath{clip}%
\pgfsetbuttcap%
\pgfsetmiterjoin%
\definecolor{currentfill}{rgb}{0.121569,0.466667,0.705882}%
\pgfsetfillcolor{currentfill}%
\pgfsetfillopacity{0.500000}%
\pgfsetlinewidth{1.003750pt}%
\definecolor{currentstroke}{rgb}{0.000000,0.000000,0.000000}%
\pgfsetstrokecolor{currentstroke}%
\pgfsetdash{}{0pt}%
\pgfpathmoveto{\pgfqpoint{3.304075in}{0.148611in}}%
\pgfpathlineto{\pgfqpoint{3.453979in}{0.148611in}}%
\pgfpathlineto{\pgfqpoint{3.453979in}{0.242897in}}%
\pgfpathlineto{\pgfqpoint{3.304075in}{0.242897in}}%
\pgfpathlineto{\pgfqpoint{3.304075in}{0.148611in}}%
\pgfpathclose%
\pgfusepath{stroke,fill}%
\end{pgfscope}%
\begin{pgfscope}%
\pgfpathrectangle{\pgfqpoint{3.116696in}{0.148611in}}{\pgfqpoint{0.824468in}{0.462000in}}%
\pgfusepath{clip}%
\pgfsetbuttcap%
\pgfsetmiterjoin%
\definecolor{currentfill}{rgb}{0.121569,0.466667,0.705882}%
\pgfsetfillcolor{currentfill}%
\pgfsetfillopacity{0.500000}%
\pgfsetlinewidth{1.003750pt}%
\definecolor{currentstroke}{rgb}{0.000000,0.000000,0.000000}%
\pgfsetstrokecolor{currentstroke}%
\pgfsetdash{}{0pt}%
\pgfpathmoveto{\pgfqpoint{3.453979in}{0.148611in}}%
\pgfpathlineto{\pgfqpoint{3.603882in}{0.148611in}}%
\pgfpathlineto{\pgfqpoint{3.603882in}{0.211468in}}%
\pgfpathlineto{\pgfqpoint{3.453979in}{0.211468in}}%
\pgfpathlineto{\pgfqpoint{3.453979in}{0.148611in}}%
\pgfpathclose%
\pgfusepath{stroke,fill}%
\end{pgfscope}%
\begin{pgfscope}%
\pgfpathrectangle{\pgfqpoint{3.116696in}{0.148611in}}{\pgfqpoint{0.824468in}{0.462000in}}%
\pgfusepath{clip}%
\pgfsetbuttcap%
\pgfsetmiterjoin%
\definecolor{currentfill}{rgb}{0.121569,0.466667,0.705882}%
\pgfsetfillcolor{currentfill}%
\pgfsetfillopacity{0.500000}%
\pgfsetlinewidth{1.003750pt}%
\definecolor{currentstroke}{rgb}{0.000000,0.000000,0.000000}%
\pgfsetstrokecolor{currentstroke}%
\pgfsetdash{}{0pt}%
\pgfpathmoveto{\pgfqpoint{3.603882in}{0.148611in}}%
\pgfpathlineto{\pgfqpoint{3.753785in}{0.148611in}}%
\pgfpathlineto{\pgfqpoint{3.753785in}{0.148611in}}%
\pgfpathlineto{\pgfqpoint{3.603882in}{0.148611in}}%
\pgfpathlineto{\pgfqpoint{3.603882in}{0.148611in}}%
\pgfpathclose%
\pgfusepath{stroke,fill}%
\end{pgfscope}%
\begin{pgfscope}%
\pgfpathrectangle{\pgfqpoint{3.116696in}{0.148611in}}{\pgfqpoint{0.824468in}{0.462000in}}%
\pgfusepath{clip}%
\pgfsetbuttcap%
\pgfsetmiterjoin%
\definecolor{currentfill}{rgb}{0.121569,0.466667,0.705882}%
\pgfsetfillcolor{currentfill}%
\pgfsetfillopacity{0.500000}%
\pgfsetlinewidth{1.003750pt}%
\definecolor{currentstroke}{rgb}{0.000000,0.000000,0.000000}%
\pgfsetstrokecolor{currentstroke}%
\pgfsetdash{}{0pt}%
\pgfpathmoveto{\pgfqpoint{3.753785in}{0.148611in}}%
\pgfpathlineto{\pgfqpoint{3.903688in}{0.148611in}}%
\pgfpathlineto{\pgfqpoint{3.903688in}{0.148611in}}%
\pgfpathlineto{\pgfqpoint{3.753785in}{0.148611in}}%
\pgfpathlineto{\pgfqpoint{3.753785in}{0.148611in}}%
\pgfpathclose%
\pgfusepath{stroke,fill}%
\end{pgfscope}%
\begin{pgfscope}%
\pgfsetrectcap%
\pgfsetmiterjoin%
\pgfsetlinewidth{0.803000pt}%
\definecolor{currentstroke}{rgb}{0.000000,0.000000,0.000000}%
\pgfsetstrokecolor{currentstroke}%
\pgfsetdash{}{0pt}%
\pgfpathmoveto{\pgfqpoint{3.116696in}{0.148611in}}%
\pgfpathlineto{\pgfqpoint{3.116696in}{0.610611in}}%
\pgfusepath{stroke}%
\end{pgfscope}%
\begin{pgfscope}%
\pgfsetrectcap%
\pgfsetmiterjoin%
\pgfsetlinewidth{0.803000pt}%
\definecolor{currentstroke}{rgb}{0.000000,0.000000,0.000000}%
\pgfsetstrokecolor{currentstroke}%
\pgfsetdash{}{0pt}%
\pgfpathmoveto{\pgfqpoint{3.941164in}{0.148611in}}%
\pgfpathlineto{\pgfqpoint{3.941164in}{0.610611in}}%
\pgfusepath{stroke}%
\end{pgfscope}%
\begin{pgfscope}%
\pgfsetrectcap%
\pgfsetmiterjoin%
\pgfsetlinewidth{0.803000pt}%
\definecolor{currentstroke}{rgb}{0.000000,0.000000,0.000000}%
\pgfsetstrokecolor{currentstroke}%
\pgfsetdash{}{0pt}%
\pgfpathmoveto{\pgfqpoint{3.116696in}{0.148611in}}%
\pgfpathlineto{\pgfqpoint{3.941164in}{0.148611in}}%
\pgfusepath{stroke}%
\end{pgfscope}%
\begin{pgfscope}%
\pgfsetrectcap%
\pgfsetmiterjoin%
\pgfsetlinewidth{0.803000pt}%
\definecolor{currentstroke}{rgb}{0.000000,0.000000,0.000000}%
\pgfsetstrokecolor{currentstroke}%
\pgfsetdash{}{0pt}%
\pgfpathmoveto{\pgfqpoint{3.116696in}{0.610611in}}%
\pgfpathlineto{\pgfqpoint{3.941164in}{0.610611in}}%
\pgfusepath{stroke}%
\end{pgfscope}%
\begin{pgfscope}%
\definecolor{textcolor}{rgb}{0.000000,0.000000,0.000000}%
\pgfsetstrokecolor{textcolor}%
\pgfsetfillcolor{textcolor}%
\pgftext[x=3.528930in,y=0.693944in,,base]{\color{textcolor}\rmfamily\fontsize{11.000000}{13.200000}\selectfont Malako...}%
\end{pgfscope}%
\begin{pgfscope}%
\pgfsetbuttcap%
\pgfsetmiterjoin%
\definecolor{currentfill}{rgb}{1.000000,1.000000,1.000000}%
\pgfsetfillcolor{currentfill}%
\pgfsetlinewidth{0.000000pt}%
\definecolor{currentstroke}{rgb}{0.000000,0.000000,0.000000}%
\pgfsetstrokecolor{currentstroke}%
\pgfsetstrokeopacity{0.000000}%
\pgfsetdash{}{0pt}%
\pgfpathmoveto{\pgfqpoint{4.106058in}{0.148611in}}%
\pgfpathlineto{\pgfqpoint{4.930526in}{0.148611in}}%
\pgfpathlineto{\pgfqpoint{4.930526in}{0.610611in}}%
\pgfpathlineto{\pgfqpoint{4.106058in}{0.610611in}}%
\pgfpathlineto{\pgfqpoint{4.106058in}{0.148611in}}%
\pgfpathclose%
\pgfusepath{fill}%
\end{pgfscope}%
\begin{pgfscope}%
\pgfpathrectangle{\pgfqpoint{4.106058in}{0.148611in}}{\pgfqpoint{0.824468in}{0.462000in}}%
\pgfusepath{clip}%
\pgfsetbuttcap%
\pgfsetmiterjoin%
\definecolor{currentfill}{rgb}{0.121569,0.466667,0.705882}%
\pgfsetfillcolor{currentfill}%
\pgfsetfillopacity{0.500000}%
\pgfsetlinewidth{1.003750pt}%
\definecolor{currentstroke}{rgb}{0.000000,0.000000,0.000000}%
\pgfsetstrokecolor{currentstroke}%
\pgfsetdash{}{0pt}%
\pgfpathmoveto{\pgfqpoint{4.143534in}{0.148611in}}%
\pgfpathlineto{\pgfqpoint{4.293437in}{0.148611in}}%
\pgfpathlineto{\pgfqpoint{4.293437in}{0.588611in}}%
\pgfpathlineto{\pgfqpoint{4.143534in}{0.588611in}}%
\pgfpathlineto{\pgfqpoint{4.143534in}{0.148611in}}%
\pgfpathclose%
\pgfusepath{stroke,fill}%
\end{pgfscope}%
\begin{pgfscope}%
\pgfpathrectangle{\pgfqpoint{4.106058in}{0.148611in}}{\pgfqpoint{0.824468in}{0.462000in}}%
\pgfusepath{clip}%
\pgfsetbuttcap%
\pgfsetmiterjoin%
\definecolor{currentfill}{rgb}{0.121569,0.466667,0.705882}%
\pgfsetfillcolor{currentfill}%
\pgfsetfillopacity{0.500000}%
\pgfsetlinewidth{1.003750pt}%
\definecolor{currentstroke}{rgb}{0.000000,0.000000,0.000000}%
\pgfsetstrokecolor{currentstroke}%
\pgfsetdash{}{0pt}%
\pgfpathmoveto{\pgfqpoint{4.293437in}{0.148611in}}%
\pgfpathlineto{\pgfqpoint{4.443340in}{0.148611in}}%
\pgfpathlineto{\pgfqpoint{4.443340in}{0.468611in}}%
\pgfpathlineto{\pgfqpoint{4.293437in}{0.468611in}}%
\pgfpathlineto{\pgfqpoint{4.293437in}{0.148611in}}%
\pgfpathclose%
\pgfusepath{stroke,fill}%
\end{pgfscope}%
\begin{pgfscope}%
\pgfpathrectangle{\pgfqpoint{4.106058in}{0.148611in}}{\pgfqpoint{0.824468in}{0.462000in}}%
\pgfusepath{clip}%
\pgfsetbuttcap%
\pgfsetmiterjoin%
\definecolor{currentfill}{rgb}{0.121569,0.466667,0.705882}%
\pgfsetfillcolor{currentfill}%
\pgfsetfillopacity{0.500000}%
\pgfsetlinewidth{1.003750pt}%
\definecolor{currentstroke}{rgb}{0.000000,0.000000,0.000000}%
\pgfsetstrokecolor{currentstroke}%
\pgfsetdash{}{0pt}%
\pgfpathmoveto{\pgfqpoint{4.443340in}{0.148611in}}%
\pgfpathlineto{\pgfqpoint{4.593244in}{0.148611in}}%
\pgfpathlineto{\pgfqpoint{4.593244in}{0.228611in}}%
\pgfpathlineto{\pgfqpoint{4.443340in}{0.228611in}}%
\pgfpathlineto{\pgfqpoint{4.443340in}{0.148611in}}%
\pgfpathclose%
\pgfusepath{stroke,fill}%
\end{pgfscope}%
\begin{pgfscope}%
\pgfpathrectangle{\pgfqpoint{4.106058in}{0.148611in}}{\pgfqpoint{0.824468in}{0.462000in}}%
\pgfusepath{clip}%
\pgfsetbuttcap%
\pgfsetmiterjoin%
\definecolor{currentfill}{rgb}{0.121569,0.466667,0.705882}%
\pgfsetfillcolor{currentfill}%
\pgfsetfillopacity{0.500000}%
\pgfsetlinewidth{1.003750pt}%
\definecolor{currentstroke}{rgb}{0.000000,0.000000,0.000000}%
\pgfsetstrokecolor{currentstroke}%
\pgfsetdash{}{0pt}%
\pgfpathmoveto{\pgfqpoint{4.593244in}{0.148611in}}%
\pgfpathlineto{\pgfqpoint{4.743147in}{0.148611in}}%
\pgfpathlineto{\pgfqpoint{4.743147in}{0.168611in}}%
\pgfpathlineto{\pgfqpoint{4.593244in}{0.168611in}}%
\pgfpathlineto{\pgfqpoint{4.593244in}{0.148611in}}%
\pgfpathclose%
\pgfusepath{stroke,fill}%
\end{pgfscope}%
\begin{pgfscope}%
\pgfpathrectangle{\pgfqpoint{4.106058in}{0.148611in}}{\pgfqpoint{0.824468in}{0.462000in}}%
\pgfusepath{clip}%
\pgfsetbuttcap%
\pgfsetmiterjoin%
\definecolor{currentfill}{rgb}{0.121569,0.466667,0.705882}%
\pgfsetfillcolor{currentfill}%
\pgfsetfillopacity{0.500000}%
\pgfsetlinewidth{1.003750pt}%
\definecolor{currentstroke}{rgb}{0.000000,0.000000,0.000000}%
\pgfsetstrokecolor{currentstroke}%
\pgfsetdash{}{0pt}%
\pgfpathmoveto{\pgfqpoint{4.743147in}{0.148611in}}%
\pgfpathlineto{\pgfqpoint{4.893050in}{0.148611in}}%
\pgfpathlineto{\pgfqpoint{4.893050in}{0.148611in}}%
\pgfpathlineto{\pgfqpoint{4.743147in}{0.148611in}}%
\pgfpathlineto{\pgfqpoint{4.743147in}{0.148611in}}%
\pgfpathclose%
\pgfusepath{stroke,fill}%
\end{pgfscope}%
\begin{pgfscope}%
\pgfsetrectcap%
\pgfsetmiterjoin%
\pgfsetlinewidth{0.803000pt}%
\definecolor{currentstroke}{rgb}{0.000000,0.000000,0.000000}%
\pgfsetstrokecolor{currentstroke}%
\pgfsetdash{}{0pt}%
\pgfpathmoveto{\pgfqpoint{4.106058in}{0.148611in}}%
\pgfpathlineto{\pgfqpoint{4.106058in}{0.610611in}}%
\pgfusepath{stroke}%
\end{pgfscope}%
\begin{pgfscope}%
\pgfsetrectcap%
\pgfsetmiterjoin%
\pgfsetlinewidth{0.803000pt}%
\definecolor{currentstroke}{rgb}{0.000000,0.000000,0.000000}%
\pgfsetstrokecolor{currentstroke}%
\pgfsetdash{}{0pt}%
\pgfpathmoveto{\pgfqpoint{4.930526in}{0.148611in}}%
\pgfpathlineto{\pgfqpoint{4.930526in}{0.610611in}}%
\pgfusepath{stroke}%
\end{pgfscope}%
\begin{pgfscope}%
\pgfsetrectcap%
\pgfsetmiterjoin%
\pgfsetlinewidth{0.803000pt}%
\definecolor{currentstroke}{rgb}{0.000000,0.000000,0.000000}%
\pgfsetstrokecolor{currentstroke}%
\pgfsetdash{}{0pt}%
\pgfpathmoveto{\pgfqpoint{4.106058in}{0.148611in}}%
\pgfpathlineto{\pgfqpoint{4.930526in}{0.148611in}}%
\pgfusepath{stroke}%
\end{pgfscope}%
\begin{pgfscope}%
\pgfsetrectcap%
\pgfsetmiterjoin%
\pgfsetlinewidth{0.803000pt}%
\definecolor{currentstroke}{rgb}{0.000000,0.000000,0.000000}%
\pgfsetstrokecolor{currentstroke}%
\pgfsetdash{}{0pt}%
\pgfpathmoveto{\pgfqpoint{4.106058in}{0.610611in}}%
\pgfpathlineto{\pgfqpoint{4.930526in}{0.610611in}}%
\pgfusepath{stroke}%
\end{pgfscope}%
\begin{pgfscope}%
\definecolor{textcolor}{rgb}{0.000000,0.000000,0.000000}%
\pgfsetstrokecolor{textcolor}%
\pgfsetfillcolor{textcolor}%
\pgftext[x=4.518292in,y=0.693944in,,base]{\color{textcolor}\rmfamily\fontsize{11.000000}{13.200000}\selectfont Euro-A...}%
\end{pgfscope}%
\begin{pgfscope}%
\pgfsetbuttcap%
\pgfsetmiterjoin%
\definecolor{currentfill}{rgb}{1.000000,1.000000,1.000000}%
\pgfsetfillcolor{currentfill}%
\pgfsetlinewidth{0.000000pt}%
\definecolor{currentstroke}{rgb}{0.000000,0.000000,0.000000}%
\pgfsetstrokecolor{currentstroke}%
\pgfsetstrokeopacity{0.000000}%
\pgfsetdash{}{0pt}%
\pgfpathmoveto{\pgfqpoint{5.095420in}{0.148611in}}%
\pgfpathlineto{\pgfqpoint{5.919888in}{0.148611in}}%
\pgfpathlineto{\pgfqpoint{5.919888in}{0.610611in}}%
\pgfpathlineto{\pgfqpoint{5.095420in}{0.610611in}}%
\pgfpathlineto{\pgfqpoint{5.095420in}{0.148611in}}%
\pgfpathclose%
\pgfusepath{fill}%
\end{pgfscope}%
\begin{pgfscope}%
\pgfpathrectangle{\pgfqpoint{5.095420in}{0.148611in}}{\pgfqpoint{0.824468in}{0.462000in}}%
\pgfusepath{clip}%
\pgfsetbuttcap%
\pgfsetmiterjoin%
\definecolor{currentfill}{rgb}{0.121569,0.466667,0.705882}%
\pgfsetfillcolor{currentfill}%
\pgfsetfillopacity{0.500000}%
\pgfsetlinewidth{1.003750pt}%
\definecolor{currentstroke}{rgb}{0.000000,0.000000,0.000000}%
\pgfsetstrokecolor{currentstroke}%
\pgfsetdash{}{0pt}%
\pgfpathmoveto{\pgfqpoint{5.132895in}{0.148611in}}%
\pgfpathlineto{\pgfqpoint{5.282799in}{0.148611in}}%
\pgfpathlineto{\pgfqpoint{5.282799in}{0.250150in}}%
\pgfpathlineto{\pgfqpoint{5.132895in}{0.250150in}}%
\pgfpathlineto{\pgfqpoint{5.132895in}{0.148611in}}%
\pgfpathclose%
\pgfusepath{stroke,fill}%
\end{pgfscope}%
\begin{pgfscope}%
\pgfpathrectangle{\pgfqpoint{5.095420in}{0.148611in}}{\pgfqpoint{0.824468in}{0.462000in}}%
\pgfusepath{clip}%
\pgfsetbuttcap%
\pgfsetmiterjoin%
\definecolor{currentfill}{rgb}{0.121569,0.466667,0.705882}%
\pgfsetfillcolor{currentfill}%
\pgfsetfillopacity{0.500000}%
\pgfsetlinewidth{1.003750pt}%
\definecolor{currentstroke}{rgb}{0.000000,0.000000,0.000000}%
\pgfsetstrokecolor{currentstroke}%
\pgfsetdash{}{0pt}%
\pgfpathmoveto{\pgfqpoint{5.282799in}{0.148611in}}%
\pgfpathlineto{\pgfqpoint{5.432702in}{0.148611in}}%
\pgfpathlineto{\pgfqpoint{5.432702in}{0.385534in}}%
\pgfpathlineto{\pgfqpoint{5.282799in}{0.385534in}}%
\pgfpathlineto{\pgfqpoint{5.282799in}{0.148611in}}%
\pgfpathclose%
\pgfusepath{stroke,fill}%
\end{pgfscope}%
\begin{pgfscope}%
\pgfpathrectangle{\pgfqpoint{5.095420in}{0.148611in}}{\pgfqpoint{0.824468in}{0.462000in}}%
\pgfusepath{clip}%
\pgfsetbuttcap%
\pgfsetmiterjoin%
\definecolor{currentfill}{rgb}{0.121569,0.466667,0.705882}%
\pgfsetfillcolor{currentfill}%
\pgfsetfillopacity{0.500000}%
\pgfsetlinewidth{1.003750pt}%
\definecolor{currentstroke}{rgb}{0.000000,0.000000,0.000000}%
\pgfsetstrokecolor{currentstroke}%
\pgfsetdash{}{0pt}%
\pgfpathmoveto{\pgfqpoint{5.432702in}{0.148611in}}%
\pgfpathlineto{\pgfqpoint{5.582605in}{0.148611in}}%
\pgfpathlineto{\pgfqpoint{5.582605in}{0.148611in}}%
\pgfpathlineto{\pgfqpoint{5.432702in}{0.148611in}}%
\pgfpathlineto{\pgfqpoint{5.432702in}{0.148611in}}%
\pgfpathclose%
\pgfusepath{stroke,fill}%
\end{pgfscope}%
\begin{pgfscope}%
\pgfpathrectangle{\pgfqpoint{5.095420in}{0.148611in}}{\pgfqpoint{0.824468in}{0.462000in}}%
\pgfusepath{clip}%
\pgfsetbuttcap%
\pgfsetmiterjoin%
\definecolor{currentfill}{rgb}{0.121569,0.466667,0.705882}%
\pgfsetfillcolor{currentfill}%
\pgfsetfillopacity{0.500000}%
\pgfsetlinewidth{1.003750pt}%
\definecolor{currentstroke}{rgb}{0.000000,0.000000,0.000000}%
\pgfsetstrokecolor{currentstroke}%
\pgfsetdash{}{0pt}%
\pgfpathmoveto{\pgfqpoint{5.582605in}{0.148611in}}%
\pgfpathlineto{\pgfqpoint{5.732509in}{0.148611in}}%
\pgfpathlineto{\pgfqpoint{5.732509in}{0.317842in}}%
\pgfpathlineto{\pgfqpoint{5.582605in}{0.317842in}}%
\pgfpathlineto{\pgfqpoint{5.582605in}{0.148611in}}%
\pgfpathclose%
\pgfusepath{stroke,fill}%
\end{pgfscope}%
\begin{pgfscope}%
\pgfpathrectangle{\pgfqpoint{5.095420in}{0.148611in}}{\pgfqpoint{0.824468in}{0.462000in}}%
\pgfusepath{clip}%
\pgfsetbuttcap%
\pgfsetmiterjoin%
\definecolor{currentfill}{rgb}{0.121569,0.466667,0.705882}%
\pgfsetfillcolor{currentfill}%
\pgfsetfillopacity{0.500000}%
\pgfsetlinewidth{1.003750pt}%
\definecolor{currentstroke}{rgb}{0.000000,0.000000,0.000000}%
\pgfsetstrokecolor{currentstroke}%
\pgfsetdash{}{0pt}%
\pgfpathmoveto{\pgfqpoint{5.732509in}{0.148611in}}%
\pgfpathlineto{\pgfqpoint{5.882412in}{0.148611in}}%
\pgfpathlineto{\pgfqpoint{5.882412in}{0.588611in}}%
\pgfpathlineto{\pgfqpoint{5.732509in}{0.588611in}}%
\pgfpathlineto{\pgfqpoint{5.732509in}{0.148611in}}%
\pgfpathclose%
\pgfusepath{stroke,fill}%
\end{pgfscope}%
\begin{pgfscope}%
\pgfsetrectcap%
\pgfsetmiterjoin%
\pgfsetlinewidth{0.803000pt}%
\definecolor{currentstroke}{rgb}{0.000000,0.000000,0.000000}%
\pgfsetstrokecolor{currentstroke}%
\pgfsetdash{}{0pt}%
\pgfpathmoveto{\pgfqpoint{5.095420in}{0.148611in}}%
\pgfpathlineto{\pgfqpoint{5.095420in}{0.610611in}}%
\pgfusepath{stroke}%
\end{pgfscope}%
\begin{pgfscope}%
\pgfsetrectcap%
\pgfsetmiterjoin%
\pgfsetlinewidth{0.803000pt}%
\definecolor{currentstroke}{rgb}{0.000000,0.000000,0.000000}%
\pgfsetstrokecolor{currentstroke}%
\pgfsetdash{}{0pt}%
\pgfpathmoveto{\pgfqpoint{5.919888in}{0.148611in}}%
\pgfpathlineto{\pgfqpoint{5.919888in}{0.610611in}}%
\pgfusepath{stroke}%
\end{pgfscope}%
\begin{pgfscope}%
\pgfsetrectcap%
\pgfsetmiterjoin%
\pgfsetlinewidth{0.803000pt}%
\definecolor{currentstroke}{rgb}{0.000000,0.000000,0.000000}%
\pgfsetstrokecolor{currentstroke}%
\pgfsetdash{}{0pt}%
\pgfpathmoveto{\pgfqpoint{5.095420in}{0.148611in}}%
\pgfpathlineto{\pgfqpoint{5.919888in}{0.148611in}}%
\pgfusepath{stroke}%
\end{pgfscope}%
\begin{pgfscope}%
\pgfsetrectcap%
\pgfsetmiterjoin%
\pgfsetlinewidth{0.803000pt}%
\definecolor{currentstroke}{rgb}{0.000000,0.000000,0.000000}%
\pgfsetstrokecolor{currentstroke}%
\pgfsetdash{}{0pt}%
\pgfpathmoveto{\pgfqpoint{5.095420in}{0.610611in}}%
\pgfpathlineto{\pgfqpoint{5.919888in}{0.610611in}}%
\pgfusepath{stroke}%
\end{pgfscope}%
\begin{pgfscope}%
\definecolor{textcolor}{rgb}{0.000000,0.000000,0.000000}%
\pgfsetstrokecolor{textcolor}%
\pgfsetfillcolor{textcolor}%
\pgftext[x=5.507654in,y=0.693944in,,base]{\color{textcolor}\rmfamily\fontsize{11.000000}{13.200000}\selectfont Peyrac...}%
\end{pgfscope}%
\begin{pgfscope}%
\pgfsetbuttcap%
\pgfsetmiterjoin%
\definecolor{currentfill}{rgb}{1.000000,1.000000,1.000000}%
\pgfsetfillcolor{currentfill}%
\pgfsetlinewidth{0.000000pt}%
\definecolor{currentstroke}{rgb}{0.000000,0.000000,0.000000}%
\pgfsetstrokecolor{currentstroke}%
\pgfsetstrokeopacity{0.000000}%
\pgfsetdash{}{0pt}%
\pgfpathmoveto{\pgfqpoint{6.084781in}{0.148611in}}%
\pgfpathlineto{\pgfqpoint{6.909249in}{0.148611in}}%
\pgfpathlineto{\pgfqpoint{6.909249in}{0.610611in}}%
\pgfpathlineto{\pgfqpoint{6.084781in}{0.610611in}}%
\pgfpathlineto{\pgfqpoint{6.084781in}{0.148611in}}%
\pgfpathclose%
\pgfusepath{fill}%
\end{pgfscope}%
\begin{pgfscope}%
\pgfpathrectangle{\pgfqpoint{6.084781in}{0.148611in}}{\pgfqpoint{0.824468in}{0.462000in}}%
\pgfusepath{clip}%
\pgfsetbuttcap%
\pgfsetmiterjoin%
\definecolor{currentfill}{rgb}{0.121569,0.466667,0.705882}%
\pgfsetfillcolor{currentfill}%
\pgfsetfillopacity{0.500000}%
\pgfsetlinewidth{1.003750pt}%
\definecolor{currentstroke}{rgb}{0.000000,0.000000,0.000000}%
\pgfsetstrokecolor{currentstroke}%
\pgfsetdash{}{0pt}%
\pgfpathmoveto{\pgfqpoint{6.122257in}{0.148611in}}%
\pgfpathlineto{\pgfqpoint{6.272160in}{0.148611in}}%
\pgfpathlineto{\pgfqpoint{6.272160in}{0.588611in}}%
\pgfpathlineto{\pgfqpoint{6.122257in}{0.588611in}}%
\pgfpathlineto{\pgfqpoint{6.122257in}{0.148611in}}%
\pgfpathclose%
\pgfusepath{stroke,fill}%
\end{pgfscope}%
\begin{pgfscope}%
\pgfpathrectangle{\pgfqpoint{6.084781in}{0.148611in}}{\pgfqpoint{0.824468in}{0.462000in}}%
\pgfusepath{clip}%
\pgfsetbuttcap%
\pgfsetmiterjoin%
\definecolor{currentfill}{rgb}{0.121569,0.466667,0.705882}%
\pgfsetfillcolor{currentfill}%
\pgfsetfillopacity{0.500000}%
\pgfsetlinewidth{1.003750pt}%
\definecolor{currentstroke}{rgb}{0.000000,0.000000,0.000000}%
\pgfsetstrokecolor{currentstroke}%
\pgfsetdash{}{0pt}%
\pgfpathmoveto{\pgfqpoint{6.272160in}{0.148611in}}%
\pgfpathlineto{\pgfqpoint{6.422064in}{0.148611in}}%
\pgfpathlineto{\pgfqpoint{6.422064in}{0.148611in}}%
\pgfpathlineto{\pgfqpoint{6.272160in}{0.148611in}}%
\pgfpathlineto{\pgfqpoint{6.272160in}{0.148611in}}%
\pgfpathclose%
\pgfusepath{stroke,fill}%
\end{pgfscope}%
\begin{pgfscope}%
\pgfpathrectangle{\pgfqpoint{6.084781in}{0.148611in}}{\pgfqpoint{0.824468in}{0.462000in}}%
\pgfusepath{clip}%
\pgfsetbuttcap%
\pgfsetmiterjoin%
\definecolor{currentfill}{rgb}{0.121569,0.466667,0.705882}%
\pgfsetfillcolor{currentfill}%
\pgfsetfillopacity{0.500000}%
\pgfsetlinewidth{1.003750pt}%
\definecolor{currentstroke}{rgb}{0.000000,0.000000,0.000000}%
\pgfsetstrokecolor{currentstroke}%
\pgfsetdash{}{0pt}%
\pgfpathmoveto{\pgfqpoint{6.422064in}{0.148611in}}%
\pgfpathlineto{\pgfqpoint{6.571967in}{0.148611in}}%
\pgfpathlineto{\pgfqpoint{6.571967in}{0.148611in}}%
\pgfpathlineto{\pgfqpoint{6.422064in}{0.148611in}}%
\pgfpathlineto{\pgfqpoint{6.422064in}{0.148611in}}%
\pgfpathclose%
\pgfusepath{stroke,fill}%
\end{pgfscope}%
\begin{pgfscope}%
\pgfpathrectangle{\pgfqpoint{6.084781in}{0.148611in}}{\pgfqpoint{0.824468in}{0.462000in}}%
\pgfusepath{clip}%
\pgfsetbuttcap%
\pgfsetmiterjoin%
\definecolor{currentfill}{rgb}{0.121569,0.466667,0.705882}%
\pgfsetfillcolor{currentfill}%
\pgfsetfillopacity{0.500000}%
\pgfsetlinewidth{1.003750pt}%
\definecolor{currentstroke}{rgb}{0.000000,0.000000,0.000000}%
\pgfsetstrokecolor{currentstroke}%
\pgfsetdash{}{0pt}%
\pgfpathmoveto{\pgfqpoint{6.571967in}{0.148611in}}%
\pgfpathlineto{\pgfqpoint{6.721870in}{0.148611in}}%
\pgfpathlineto{\pgfqpoint{6.721870in}{0.148611in}}%
\pgfpathlineto{\pgfqpoint{6.571967in}{0.148611in}}%
\pgfpathlineto{\pgfqpoint{6.571967in}{0.148611in}}%
\pgfpathclose%
\pgfusepath{stroke,fill}%
\end{pgfscope}%
\begin{pgfscope}%
\pgfpathrectangle{\pgfqpoint{6.084781in}{0.148611in}}{\pgfqpoint{0.824468in}{0.462000in}}%
\pgfusepath{clip}%
\pgfsetbuttcap%
\pgfsetmiterjoin%
\definecolor{currentfill}{rgb}{0.121569,0.466667,0.705882}%
\pgfsetfillcolor{currentfill}%
\pgfsetfillopacity{0.500000}%
\pgfsetlinewidth{1.003750pt}%
\definecolor{currentstroke}{rgb}{0.000000,0.000000,0.000000}%
\pgfsetstrokecolor{currentstroke}%
\pgfsetdash{}{0pt}%
\pgfpathmoveto{\pgfqpoint{6.721870in}{0.148611in}}%
\pgfpathlineto{\pgfqpoint{6.871774in}{0.148611in}}%
\pgfpathlineto{\pgfqpoint{6.871774in}{0.236611in}}%
\pgfpathlineto{\pgfqpoint{6.721870in}{0.236611in}}%
\pgfpathlineto{\pgfqpoint{6.721870in}{0.148611in}}%
\pgfpathclose%
\pgfusepath{stroke,fill}%
\end{pgfscope}%
\begin{pgfscope}%
\pgfsetrectcap%
\pgfsetmiterjoin%
\pgfsetlinewidth{0.803000pt}%
\definecolor{currentstroke}{rgb}{0.000000,0.000000,0.000000}%
\pgfsetstrokecolor{currentstroke}%
\pgfsetdash{}{0pt}%
\pgfpathmoveto{\pgfqpoint{6.084781in}{0.148611in}}%
\pgfpathlineto{\pgfqpoint{6.084781in}{0.610611in}}%
\pgfusepath{stroke}%
\end{pgfscope}%
\begin{pgfscope}%
\pgfsetrectcap%
\pgfsetmiterjoin%
\pgfsetlinewidth{0.803000pt}%
\definecolor{currentstroke}{rgb}{0.000000,0.000000,0.000000}%
\pgfsetstrokecolor{currentstroke}%
\pgfsetdash{}{0pt}%
\pgfpathmoveto{\pgfqpoint{6.909249in}{0.148611in}}%
\pgfpathlineto{\pgfqpoint{6.909249in}{0.610611in}}%
\pgfusepath{stroke}%
\end{pgfscope}%
\begin{pgfscope}%
\pgfsetrectcap%
\pgfsetmiterjoin%
\pgfsetlinewidth{0.803000pt}%
\definecolor{currentstroke}{rgb}{0.000000,0.000000,0.000000}%
\pgfsetstrokecolor{currentstroke}%
\pgfsetdash{}{0pt}%
\pgfpathmoveto{\pgfqpoint{6.084781in}{0.148611in}}%
\pgfpathlineto{\pgfqpoint{6.909249in}{0.148611in}}%
\pgfusepath{stroke}%
\end{pgfscope}%
\begin{pgfscope}%
\pgfsetrectcap%
\pgfsetmiterjoin%
\pgfsetlinewidth{0.803000pt}%
\definecolor{currentstroke}{rgb}{0.000000,0.000000,0.000000}%
\pgfsetstrokecolor{currentstroke}%
\pgfsetdash{}{0pt}%
\pgfpathmoveto{\pgfqpoint{6.084781in}{0.610611in}}%
\pgfpathlineto{\pgfqpoint{6.909249in}{0.610611in}}%
\pgfusepath{stroke}%
\end{pgfscope}%
\begin{pgfscope}%
\definecolor{textcolor}{rgb}{0.000000,0.000000,0.000000}%
\pgfsetstrokecolor{textcolor}%
\pgfsetfillcolor{textcolor}%
\pgftext[x=6.497015in,y=0.693944in,,base]{\color{textcolor}\rmfamily\fontsize{11.000000}{13.200000}\selectfont Sma}%
\end{pgfscope}%
\begin{pgfscope}%
\pgfsetbuttcap%
\pgfsetmiterjoin%
\definecolor{currentfill}{rgb}{1.000000,1.000000,1.000000}%
\pgfsetfillcolor{currentfill}%
\pgfsetlinewidth{0.000000pt}%
\definecolor{currentstroke}{rgb}{0.000000,0.000000,0.000000}%
\pgfsetstrokecolor{currentstroke}%
\pgfsetstrokeopacity{0.000000}%
\pgfsetdash{}{0pt}%
\pgfpathmoveto{\pgfqpoint{7.074143in}{0.148611in}}%
\pgfpathlineto{\pgfqpoint{7.898611in}{0.148611in}}%
\pgfpathlineto{\pgfqpoint{7.898611in}{0.610611in}}%
\pgfpathlineto{\pgfqpoint{7.074143in}{0.610611in}}%
\pgfpathlineto{\pgfqpoint{7.074143in}{0.148611in}}%
\pgfpathclose%
\pgfusepath{fill}%
\end{pgfscope}%
\begin{pgfscope}%
\pgfpathrectangle{\pgfqpoint{7.074143in}{0.148611in}}{\pgfqpoint{0.824468in}{0.462000in}}%
\pgfusepath{clip}%
\pgfsetbuttcap%
\pgfsetmiterjoin%
\definecolor{currentfill}{rgb}{0.121569,0.466667,0.705882}%
\pgfsetfillcolor{currentfill}%
\pgfsetfillopacity{0.500000}%
\pgfsetlinewidth{1.003750pt}%
\definecolor{currentstroke}{rgb}{0.000000,0.000000,0.000000}%
\pgfsetstrokecolor{currentstroke}%
\pgfsetdash{}{0pt}%
\pgfpathmoveto{\pgfqpoint{7.111619in}{0.148611in}}%
\pgfpathlineto{\pgfqpoint{7.261522in}{0.148611in}}%
\pgfpathlineto{\pgfqpoint{7.261522in}{0.588611in}}%
\pgfpathlineto{\pgfqpoint{7.111619in}{0.588611in}}%
\pgfpathlineto{\pgfqpoint{7.111619in}{0.148611in}}%
\pgfpathclose%
\pgfusepath{stroke,fill}%
\end{pgfscope}%
\begin{pgfscope}%
\pgfpathrectangle{\pgfqpoint{7.074143in}{0.148611in}}{\pgfqpoint{0.824468in}{0.462000in}}%
\pgfusepath{clip}%
\pgfsetbuttcap%
\pgfsetmiterjoin%
\definecolor{currentfill}{rgb}{0.121569,0.466667,0.705882}%
\pgfsetfillcolor{currentfill}%
\pgfsetfillopacity{0.500000}%
\pgfsetlinewidth{1.003750pt}%
\definecolor{currentstroke}{rgb}{0.000000,0.000000,0.000000}%
\pgfsetstrokecolor{currentstroke}%
\pgfsetdash{}{0pt}%
\pgfpathmoveto{\pgfqpoint{7.261522in}{0.148611in}}%
\pgfpathlineto{\pgfqpoint{7.411425in}{0.148611in}}%
\pgfpathlineto{\pgfqpoint{7.411425in}{0.148611in}}%
\pgfpathlineto{\pgfqpoint{7.261522in}{0.148611in}}%
\pgfpathlineto{\pgfqpoint{7.261522in}{0.148611in}}%
\pgfpathclose%
\pgfusepath{stroke,fill}%
\end{pgfscope}%
\begin{pgfscope}%
\pgfpathrectangle{\pgfqpoint{7.074143in}{0.148611in}}{\pgfqpoint{0.824468in}{0.462000in}}%
\pgfusepath{clip}%
\pgfsetbuttcap%
\pgfsetmiterjoin%
\definecolor{currentfill}{rgb}{0.121569,0.466667,0.705882}%
\pgfsetfillcolor{currentfill}%
\pgfsetfillopacity{0.500000}%
\pgfsetlinewidth{1.003750pt}%
\definecolor{currentstroke}{rgb}{0.000000,0.000000,0.000000}%
\pgfsetstrokecolor{currentstroke}%
\pgfsetdash{}{0pt}%
\pgfpathmoveto{\pgfqpoint{7.411425in}{0.148611in}}%
\pgfpathlineto{\pgfqpoint{7.561329in}{0.148611in}}%
\pgfpathlineto{\pgfqpoint{7.561329in}{0.148611in}}%
\pgfpathlineto{\pgfqpoint{7.411425in}{0.148611in}}%
\pgfpathlineto{\pgfqpoint{7.411425in}{0.148611in}}%
\pgfpathclose%
\pgfusepath{stroke,fill}%
\end{pgfscope}%
\begin{pgfscope}%
\pgfpathrectangle{\pgfqpoint{7.074143in}{0.148611in}}{\pgfqpoint{0.824468in}{0.462000in}}%
\pgfusepath{clip}%
\pgfsetbuttcap%
\pgfsetmiterjoin%
\definecolor{currentfill}{rgb}{0.121569,0.466667,0.705882}%
\pgfsetfillcolor{currentfill}%
\pgfsetfillopacity{0.500000}%
\pgfsetlinewidth{1.003750pt}%
\definecolor{currentstroke}{rgb}{0.000000,0.000000,0.000000}%
\pgfsetstrokecolor{currentstroke}%
\pgfsetdash{}{0pt}%
\pgfpathmoveto{\pgfqpoint{7.561329in}{0.148611in}}%
\pgfpathlineto{\pgfqpoint{7.711232in}{0.148611in}}%
\pgfpathlineto{\pgfqpoint{7.711232in}{0.148611in}}%
\pgfpathlineto{\pgfqpoint{7.561329in}{0.148611in}}%
\pgfpathlineto{\pgfqpoint{7.561329in}{0.148611in}}%
\pgfpathclose%
\pgfusepath{stroke,fill}%
\end{pgfscope}%
\begin{pgfscope}%
\pgfpathrectangle{\pgfqpoint{7.074143in}{0.148611in}}{\pgfqpoint{0.824468in}{0.462000in}}%
\pgfusepath{clip}%
\pgfsetbuttcap%
\pgfsetmiterjoin%
\definecolor{currentfill}{rgb}{0.121569,0.466667,0.705882}%
\pgfsetfillcolor{currentfill}%
\pgfsetfillopacity{0.500000}%
\pgfsetlinewidth{1.003750pt}%
\definecolor{currentstroke}{rgb}{0.000000,0.000000,0.000000}%
\pgfsetstrokecolor{currentstroke}%
\pgfsetdash{}{0pt}%
\pgfpathmoveto{\pgfqpoint{7.711232in}{0.148611in}}%
\pgfpathlineto{\pgfqpoint{7.861135in}{0.148611in}}%
\pgfpathlineto{\pgfqpoint{7.861135in}{0.148611in}}%
\pgfpathlineto{\pgfqpoint{7.711232in}{0.148611in}}%
\pgfpathlineto{\pgfqpoint{7.711232in}{0.148611in}}%
\pgfpathclose%
\pgfusepath{stroke,fill}%
\end{pgfscope}%
\begin{pgfscope}%
\pgfsetrectcap%
\pgfsetmiterjoin%
\pgfsetlinewidth{0.803000pt}%
\definecolor{currentstroke}{rgb}{0.000000,0.000000,0.000000}%
\pgfsetstrokecolor{currentstroke}%
\pgfsetdash{}{0pt}%
\pgfpathmoveto{\pgfqpoint{7.074143in}{0.148611in}}%
\pgfpathlineto{\pgfqpoint{7.074143in}{0.610611in}}%
\pgfusepath{stroke}%
\end{pgfscope}%
\begin{pgfscope}%
\pgfsetrectcap%
\pgfsetmiterjoin%
\pgfsetlinewidth{0.803000pt}%
\definecolor{currentstroke}{rgb}{0.000000,0.000000,0.000000}%
\pgfsetstrokecolor{currentstroke}%
\pgfsetdash{}{0pt}%
\pgfpathmoveto{\pgfqpoint{7.898611in}{0.148611in}}%
\pgfpathlineto{\pgfqpoint{7.898611in}{0.610611in}}%
\pgfusepath{stroke}%
\end{pgfscope}%
\begin{pgfscope}%
\pgfsetrectcap%
\pgfsetmiterjoin%
\pgfsetlinewidth{0.803000pt}%
\definecolor{currentstroke}{rgb}{0.000000,0.000000,0.000000}%
\pgfsetstrokecolor{currentstroke}%
\pgfsetdash{}{0pt}%
\pgfpathmoveto{\pgfqpoint{7.074143in}{0.148611in}}%
\pgfpathlineto{\pgfqpoint{7.898611in}{0.148611in}}%
\pgfusepath{stroke}%
\end{pgfscope}%
\begin{pgfscope}%
\pgfsetrectcap%
\pgfsetmiterjoin%
\pgfsetlinewidth{0.803000pt}%
\definecolor{currentstroke}{rgb}{0.000000,0.000000,0.000000}%
\pgfsetstrokecolor{currentstroke}%
\pgfsetdash{}{0pt}%
\pgfpathmoveto{\pgfqpoint{7.074143in}{0.610611in}}%
\pgfpathlineto{\pgfqpoint{7.898611in}{0.610611in}}%
\pgfusepath{stroke}%
\end{pgfscope}%
\begin{pgfscope}%
\definecolor{textcolor}{rgb}{0.000000,0.000000,0.000000}%
\pgfsetstrokecolor{textcolor}%
\pgfsetfillcolor{textcolor}%
\pgftext[x=7.486377in,y=0.693944in,,base]{\color{textcolor}\rmfamily\fontsize{11.000000}{13.200000}\selectfont Hiscox}%
\end{pgfscope}%
\end{pgfpicture}%
\makeatother%
\endgroup%

    \caption{Stars distribution per assureur}
    \label{fig:distrib_split_noscale}
\end{figure}


\begin{figure}[H]
    \advance\leftskip-3cm
    %% Creator: Matplotlib, PGF backend
%%
%% To include the figure in your LaTeX document, write
%%   \input{<filename>.pgf}
%%
%% Make sure the required packages are loaded in your preamble
%%   \usepackage{pgf}
%%
%% Also ensure that all the required font packages are loaded; for instance,
%% the lmodern package is sometimes necessary when using math font.
%%   \usepackage{lmodern}
%%
%% Figures using additional raster images can only be included by \input if
%% they are in the same directory as the main LaTeX file. For loading figures
%% from other directories you can use the `import` package
%%   \usepackage{import}
%%
%% and then include the figures with
%%   \import{<path to file>}{<filename>.pgf}
%%
%% Matplotlib used the following preamble
%%
\begingroup%
\makeatletter%
\begin{pgfpicture}%
\pgfpathrectangle{\pgfpointorigin}{\pgfqpoint{7.998611in}{5.057558in}}%
\pgfusepath{use as bounding box, clip}%
\begin{pgfscope}%
\pgfsetbuttcap%
\pgfsetmiterjoin%
\definecolor{currentfill}{rgb}{1.000000,1.000000,1.000000}%
\pgfsetfillcolor{currentfill}%
\pgfsetlinewidth{0.000000pt}%
\definecolor{currentstroke}{rgb}{1.000000,1.000000,1.000000}%
\pgfsetstrokecolor{currentstroke}%
\pgfsetdash{}{0pt}%
\pgfpathmoveto{\pgfqpoint{0.000000in}{0.000000in}}%
\pgfpathlineto{\pgfqpoint{7.998611in}{0.000000in}}%
\pgfpathlineto{\pgfqpoint{7.998611in}{5.057558in}}%
\pgfpathlineto{\pgfqpoint{0.000000in}{5.057558in}}%
\pgfpathlineto{\pgfqpoint{0.000000in}{0.000000in}}%
\pgfpathclose%
\pgfusepath{fill}%
\end{pgfscope}%
\begin{pgfscope}%
\pgfsetbuttcap%
\pgfsetmiterjoin%
\definecolor{currentfill}{rgb}{1.000000,1.000000,1.000000}%
\pgfsetfillcolor{currentfill}%
\pgfsetlinewidth{0.000000pt}%
\definecolor{currentstroke}{rgb}{0.000000,0.000000,0.000000}%
\pgfsetstrokecolor{currentstroke}%
\pgfsetstrokeopacity{0.000000}%
\pgfsetdash{}{0pt}%
\pgfpathmoveto{\pgfqpoint{0.148611in}{4.525453in}}%
\pgfpathlineto{\pgfqpoint{0.973079in}{4.525453in}}%
\pgfpathlineto{\pgfqpoint{0.973079in}{4.768611in}}%
\pgfpathlineto{\pgfqpoint{0.148611in}{4.768611in}}%
\pgfpathlineto{\pgfqpoint{0.148611in}{4.525453in}}%
\pgfpathclose%
\pgfusepath{fill}%
\end{pgfscope}%
\begin{pgfscope}%
\pgfpathrectangle{\pgfqpoint{0.148611in}{4.525453in}}{\pgfqpoint{0.824468in}{0.243158in}}%
\pgfusepath{clip}%
\pgfsetbuttcap%
\pgfsetmiterjoin%
\definecolor{currentfill}{rgb}{0.121569,0.466667,0.705882}%
\pgfsetfillcolor{currentfill}%
\pgfsetfillopacity{0.500000}%
\pgfsetlinewidth{1.003750pt}%
\definecolor{currentstroke}{rgb}{0.000000,0.000000,0.000000}%
\pgfsetstrokecolor{currentstroke}%
\pgfsetdash{}{0pt}%
\pgfpathmoveto{\pgfqpoint{0.186087in}{4.525453in}}%
\pgfpathlineto{\pgfqpoint{0.335990in}{4.525453in}}%
\pgfpathlineto{\pgfqpoint{0.335990in}{4.642047in}}%
\pgfpathlineto{\pgfqpoint{0.186087in}{4.642047in}}%
\pgfpathlineto{\pgfqpoint{0.186087in}{4.525453in}}%
\pgfpathclose%
\pgfusepath{stroke,fill}%
\end{pgfscope}%
\begin{pgfscope}%
\pgfpathrectangle{\pgfqpoint{0.148611in}{4.525453in}}{\pgfqpoint{0.824468in}{0.243158in}}%
\pgfusepath{clip}%
\pgfsetbuttcap%
\pgfsetmiterjoin%
\definecolor{currentfill}{rgb}{0.121569,0.466667,0.705882}%
\pgfsetfillcolor{currentfill}%
\pgfsetfillopacity{0.500000}%
\pgfsetlinewidth{1.003750pt}%
\definecolor{currentstroke}{rgb}{0.000000,0.000000,0.000000}%
\pgfsetstrokecolor{currentstroke}%
\pgfsetdash{}{0pt}%
\pgfpathmoveto{\pgfqpoint{0.335990in}{4.525453in}}%
\pgfpathlineto{\pgfqpoint{0.485894in}{4.525453in}}%
\pgfpathlineto{\pgfqpoint{0.485894in}{4.618704in}}%
\pgfpathlineto{\pgfqpoint{0.335990in}{4.618704in}}%
\pgfpathlineto{\pgfqpoint{0.335990in}{4.525453in}}%
\pgfpathclose%
\pgfusepath{stroke,fill}%
\end{pgfscope}%
\begin{pgfscope}%
\pgfpathrectangle{\pgfqpoint{0.148611in}{4.525453in}}{\pgfqpoint{0.824468in}{0.243158in}}%
\pgfusepath{clip}%
\pgfsetbuttcap%
\pgfsetmiterjoin%
\definecolor{currentfill}{rgb}{0.121569,0.466667,0.705882}%
\pgfsetfillcolor{currentfill}%
\pgfsetfillopacity{0.500000}%
\pgfsetlinewidth{1.003750pt}%
\definecolor{currentstroke}{rgb}{0.000000,0.000000,0.000000}%
\pgfsetstrokecolor{currentstroke}%
\pgfsetdash{}{0pt}%
\pgfpathmoveto{\pgfqpoint{0.485894in}{4.525453in}}%
\pgfpathlineto{\pgfqpoint{0.635797in}{4.525453in}}%
\pgfpathlineto{\pgfqpoint{0.635797in}{4.660527in}}%
\pgfpathlineto{\pgfqpoint{0.485894in}{4.660527in}}%
\pgfpathlineto{\pgfqpoint{0.485894in}{4.525453in}}%
\pgfpathclose%
\pgfusepath{stroke,fill}%
\end{pgfscope}%
\begin{pgfscope}%
\pgfpathrectangle{\pgfqpoint{0.148611in}{4.525453in}}{\pgfqpoint{0.824468in}{0.243158in}}%
\pgfusepath{clip}%
\pgfsetbuttcap%
\pgfsetmiterjoin%
\definecolor{currentfill}{rgb}{0.121569,0.466667,0.705882}%
\pgfsetfillcolor{currentfill}%
\pgfsetfillopacity{0.500000}%
\pgfsetlinewidth{1.003750pt}%
\definecolor{currentstroke}{rgb}{0.000000,0.000000,0.000000}%
\pgfsetstrokecolor{currentstroke}%
\pgfsetdash{}{0pt}%
\pgfpathmoveto{\pgfqpoint{0.635797in}{4.525453in}}%
\pgfpathlineto{\pgfqpoint{0.785700in}{4.525453in}}%
\pgfpathlineto{\pgfqpoint{0.785700in}{4.717183in}}%
\pgfpathlineto{\pgfqpoint{0.635797in}{4.717183in}}%
\pgfpathlineto{\pgfqpoint{0.635797in}{4.525453in}}%
\pgfpathclose%
\pgfusepath{stroke,fill}%
\end{pgfscope}%
\begin{pgfscope}%
\pgfpathrectangle{\pgfqpoint{0.148611in}{4.525453in}}{\pgfqpoint{0.824468in}{0.243158in}}%
\pgfusepath{clip}%
\pgfsetbuttcap%
\pgfsetmiterjoin%
\definecolor{currentfill}{rgb}{0.121569,0.466667,0.705882}%
\pgfsetfillcolor{currentfill}%
\pgfsetfillopacity{0.500000}%
\pgfsetlinewidth{1.003750pt}%
\definecolor{currentstroke}{rgb}{0.000000,0.000000,0.000000}%
\pgfsetstrokecolor{currentstroke}%
\pgfsetdash{}{0pt}%
\pgfpathmoveto{\pgfqpoint{0.785700in}{4.525453in}}%
\pgfpathlineto{\pgfqpoint{0.935603in}{4.525453in}}%
\pgfpathlineto{\pgfqpoint{0.935603in}{4.705633in}}%
\pgfpathlineto{\pgfqpoint{0.785700in}{4.705633in}}%
\pgfpathlineto{\pgfqpoint{0.785700in}{4.525453in}}%
\pgfpathclose%
\pgfusepath{stroke,fill}%
\end{pgfscope}%
\begin{pgfscope}%
\pgfsetrectcap%
\pgfsetmiterjoin%
\pgfsetlinewidth{0.803000pt}%
\definecolor{currentstroke}{rgb}{0.000000,0.000000,0.000000}%
\pgfsetstrokecolor{currentstroke}%
\pgfsetdash{}{0pt}%
\pgfpathmoveto{\pgfqpoint{0.148611in}{4.525453in}}%
\pgfpathlineto{\pgfqpoint{0.148611in}{4.768611in}}%
\pgfusepath{stroke}%
\end{pgfscope}%
\begin{pgfscope}%
\pgfsetrectcap%
\pgfsetmiterjoin%
\pgfsetlinewidth{0.803000pt}%
\definecolor{currentstroke}{rgb}{0.000000,0.000000,0.000000}%
\pgfsetstrokecolor{currentstroke}%
\pgfsetdash{}{0pt}%
\pgfpathmoveto{\pgfqpoint{0.973079in}{4.525453in}}%
\pgfpathlineto{\pgfqpoint{0.973079in}{4.768611in}}%
\pgfusepath{stroke}%
\end{pgfscope}%
\begin{pgfscope}%
\pgfsetrectcap%
\pgfsetmiterjoin%
\pgfsetlinewidth{0.803000pt}%
\definecolor{currentstroke}{rgb}{0.000000,0.000000,0.000000}%
\pgfsetstrokecolor{currentstroke}%
\pgfsetdash{}{0pt}%
\pgfpathmoveto{\pgfqpoint{0.148611in}{4.525453in}}%
\pgfpathlineto{\pgfqpoint{0.973079in}{4.525453in}}%
\pgfusepath{stroke}%
\end{pgfscope}%
\begin{pgfscope}%
\pgfsetrectcap%
\pgfsetmiterjoin%
\pgfsetlinewidth{0.803000pt}%
\definecolor{currentstroke}{rgb}{0.000000,0.000000,0.000000}%
\pgfsetstrokecolor{currentstroke}%
\pgfsetdash{}{0pt}%
\pgfpathmoveto{\pgfqpoint{0.148611in}{4.768611in}}%
\pgfpathlineto{\pgfqpoint{0.973079in}{4.768611in}}%
\pgfusepath{stroke}%
\end{pgfscope}%
\begin{pgfscope}%
\definecolor{textcolor}{rgb}{0.000000,0.000000,0.000000}%
\pgfsetstrokecolor{textcolor}%
\pgfsetfillcolor{textcolor}%
\pgftext[x=0.560845in,y=4.851944in,,base]{\color{textcolor}\rmfamily\fontsize{11.000000}{13.200000}\selectfont Direct...}%
\end{pgfscope}%
\begin{pgfscope}%
\pgfsetbuttcap%
\pgfsetmiterjoin%
\definecolor{currentfill}{rgb}{1.000000,1.000000,1.000000}%
\pgfsetfillcolor{currentfill}%
\pgfsetlinewidth{0.000000pt}%
\definecolor{currentstroke}{rgb}{0.000000,0.000000,0.000000}%
\pgfsetstrokecolor{currentstroke}%
\pgfsetstrokeopacity{0.000000}%
\pgfsetdash{}{0pt}%
\pgfpathmoveto{\pgfqpoint{1.137973in}{4.525453in}}%
\pgfpathlineto{\pgfqpoint{1.962441in}{4.525453in}}%
\pgfpathlineto{\pgfqpoint{1.962441in}{4.768611in}}%
\pgfpathlineto{\pgfqpoint{1.137973in}{4.768611in}}%
\pgfpathlineto{\pgfqpoint{1.137973in}{4.525453in}}%
\pgfpathclose%
\pgfusepath{fill}%
\end{pgfscope}%
\begin{pgfscope}%
\pgfpathrectangle{\pgfqpoint{1.137973in}{4.525453in}}{\pgfqpoint{0.824468in}{0.243158in}}%
\pgfusepath{clip}%
\pgfsetbuttcap%
\pgfsetmiterjoin%
\definecolor{currentfill}{rgb}{0.121569,0.466667,0.705882}%
\pgfsetfillcolor{currentfill}%
\pgfsetfillopacity{0.500000}%
\pgfsetlinewidth{1.003750pt}%
\definecolor{currentstroke}{rgb}{0.000000,0.000000,0.000000}%
\pgfsetstrokecolor{currentstroke}%
\pgfsetdash{}{0pt}%
\pgfpathmoveto{\pgfqpoint{1.175449in}{4.525453in}}%
\pgfpathlineto{\pgfqpoint{1.325352in}{4.525453in}}%
\pgfpathlineto{\pgfqpoint{1.325352in}{4.572018in}}%
\pgfpathlineto{\pgfqpoint{1.175449in}{4.572018in}}%
\pgfpathlineto{\pgfqpoint{1.175449in}{4.525453in}}%
\pgfpathclose%
\pgfusepath{stroke,fill}%
\end{pgfscope}%
\begin{pgfscope}%
\pgfpathrectangle{\pgfqpoint{1.137973in}{4.525453in}}{\pgfqpoint{0.824468in}{0.243158in}}%
\pgfusepath{clip}%
\pgfsetbuttcap%
\pgfsetmiterjoin%
\definecolor{currentfill}{rgb}{0.121569,0.466667,0.705882}%
\pgfsetfillcolor{currentfill}%
\pgfsetfillopacity{0.500000}%
\pgfsetlinewidth{1.003750pt}%
\definecolor{currentstroke}{rgb}{0.000000,0.000000,0.000000}%
\pgfsetstrokecolor{currentstroke}%
\pgfsetdash{}{0pt}%
\pgfpathmoveto{\pgfqpoint{1.325352in}{4.525453in}}%
\pgfpathlineto{\pgfqpoint{1.475255in}{4.525453in}}%
\pgfpathlineto{\pgfqpoint{1.475255in}{4.565817in}}%
\pgfpathlineto{\pgfqpoint{1.325352in}{4.565817in}}%
\pgfpathlineto{\pgfqpoint{1.325352in}{4.525453in}}%
\pgfpathclose%
\pgfusepath{stroke,fill}%
\end{pgfscope}%
\begin{pgfscope}%
\pgfpathrectangle{\pgfqpoint{1.137973in}{4.525453in}}{\pgfqpoint{0.824468in}{0.243158in}}%
\pgfusepath{clip}%
\pgfsetbuttcap%
\pgfsetmiterjoin%
\definecolor{currentfill}{rgb}{0.121569,0.466667,0.705882}%
\pgfsetfillcolor{currentfill}%
\pgfsetfillopacity{0.500000}%
\pgfsetlinewidth{1.003750pt}%
\definecolor{currentstroke}{rgb}{0.000000,0.000000,0.000000}%
\pgfsetstrokecolor{currentstroke}%
\pgfsetdash{}{0pt}%
\pgfpathmoveto{\pgfqpoint{1.475255in}{4.525453in}}%
\pgfpathlineto{\pgfqpoint{1.625158in}{4.525453in}}%
\pgfpathlineto{\pgfqpoint{1.625158in}{4.589890in}}%
\pgfpathlineto{\pgfqpoint{1.475255in}{4.589890in}}%
\pgfpathlineto{\pgfqpoint{1.475255in}{4.525453in}}%
\pgfpathclose%
\pgfusepath{stroke,fill}%
\end{pgfscope}%
\begin{pgfscope}%
\pgfpathrectangle{\pgfqpoint{1.137973in}{4.525453in}}{\pgfqpoint{0.824468in}{0.243158in}}%
\pgfusepath{clip}%
\pgfsetbuttcap%
\pgfsetmiterjoin%
\definecolor{currentfill}{rgb}{0.121569,0.466667,0.705882}%
\pgfsetfillcolor{currentfill}%
\pgfsetfillopacity{0.500000}%
\pgfsetlinewidth{1.003750pt}%
\definecolor{currentstroke}{rgb}{0.000000,0.000000,0.000000}%
\pgfsetstrokecolor{currentstroke}%
\pgfsetdash{}{0pt}%
\pgfpathmoveto{\pgfqpoint{1.625158in}{4.525453in}}%
\pgfpathlineto{\pgfqpoint{1.775062in}{4.525453in}}%
\pgfpathlineto{\pgfqpoint{1.775062in}{4.699311in}}%
\pgfpathlineto{\pgfqpoint{1.625158in}{4.699311in}}%
\pgfpathlineto{\pgfqpoint{1.625158in}{4.525453in}}%
\pgfpathclose%
\pgfusepath{stroke,fill}%
\end{pgfscope}%
\begin{pgfscope}%
\pgfpathrectangle{\pgfqpoint{1.137973in}{4.525453in}}{\pgfqpoint{0.824468in}{0.243158in}}%
\pgfusepath{clip}%
\pgfsetbuttcap%
\pgfsetmiterjoin%
\definecolor{currentfill}{rgb}{0.121569,0.466667,0.705882}%
\pgfsetfillcolor{currentfill}%
\pgfsetfillopacity{0.500000}%
\pgfsetlinewidth{1.003750pt}%
\definecolor{currentstroke}{rgb}{0.000000,0.000000,0.000000}%
\pgfsetstrokecolor{currentstroke}%
\pgfsetdash{}{0pt}%
\pgfpathmoveto{\pgfqpoint{1.775062in}{4.525453in}}%
\pgfpathlineto{\pgfqpoint{1.924965in}{4.525453in}}%
\pgfpathlineto{\pgfqpoint{1.924965in}{4.721560in}}%
\pgfpathlineto{\pgfqpoint{1.775062in}{4.721560in}}%
\pgfpathlineto{\pgfqpoint{1.775062in}{4.525453in}}%
\pgfpathclose%
\pgfusepath{stroke,fill}%
\end{pgfscope}%
\begin{pgfscope}%
\pgfsetrectcap%
\pgfsetmiterjoin%
\pgfsetlinewidth{0.803000pt}%
\definecolor{currentstroke}{rgb}{0.000000,0.000000,0.000000}%
\pgfsetstrokecolor{currentstroke}%
\pgfsetdash{}{0pt}%
\pgfpathmoveto{\pgfqpoint{1.137973in}{4.525453in}}%
\pgfpathlineto{\pgfqpoint{1.137973in}{4.768611in}}%
\pgfusepath{stroke}%
\end{pgfscope}%
\begin{pgfscope}%
\pgfsetrectcap%
\pgfsetmiterjoin%
\pgfsetlinewidth{0.803000pt}%
\definecolor{currentstroke}{rgb}{0.000000,0.000000,0.000000}%
\pgfsetstrokecolor{currentstroke}%
\pgfsetdash{}{0pt}%
\pgfpathmoveto{\pgfqpoint{1.962441in}{4.525453in}}%
\pgfpathlineto{\pgfqpoint{1.962441in}{4.768611in}}%
\pgfusepath{stroke}%
\end{pgfscope}%
\begin{pgfscope}%
\pgfsetrectcap%
\pgfsetmiterjoin%
\pgfsetlinewidth{0.803000pt}%
\definecolor{currentstroke}{rgb}{0.000000,0.000000,0.000000}%
\pgfsetstrokecolor{currentstroke}%
\pgfsetdash{}{0pt}%
\pgfpathmoveto{\pgfqpoint{1.137973in}{4.525453in}}%
\pgfpathlineto{\pgfqpoint{1.962441in}{4.525453in}}%
\pgfusepath{stroke}%
\end{pgfscope}%
\begin{pgfscope}%
\pgfsetrectcap%
\pgfsetmiterjoin%
\pgfsetlinewidth{0.803000pt}%
\definecolor{currentstroke}{rgb}{0.000000,0.000000,0.000000}%
\pgfsetstrokecolor{currentstroke}%
\pgfsetdash{}{0pt}%
\pgfpathmoveto{\pgfqpoint{1.137973in}{4.768611in}}%
\pgfpathlineto{\pgfqpoint{1.962441in}{4.768611in}}%
\pgfusepath{stroke}%
\end{pgfscope}%
\begin{pgfscope}%
\definecolor{textcolor}{rgb}{0.000000,0.000000,0.000000}%
\pgfsetstrokecolor{textcolor}%
\pgfsetfillcolor{textcolor}%
\pgftext[x=1.550207in,y=4.851944in,,base]{\color{textcolor}\rmfamily\fontsize{11.000000}{13.200000}\selectfont L'oliv...}%
\end{pgfscope}%
\begin{pgfscope}%
\pgfsetbuttcap%
\pgfsetmiterjoin%
\definecolor{currentfill}{rgb}{1.000000,1.000000,1.000000}%
\pgfsetfillcolor{currentfill}%
\pgfsetlinewidth{0.000000pt}%
\definecolor{currentstroke}{rgb}{0.000000,0.000000,0.000000}%
\pgfsetstrokecolor{currentstroke}%
\pgfsetstrokeopacity{0.000000}%
\pgfsetdash{}{0pt}%
\pgfpathmoveto{\pgfqpoint{2.127335in}{4.525453in}}%
\pgfpathlineto{\pgfqpoint{2.951803in}{4.525453in}}%
\pgfpathlineto{\pgfqpoint{2.951803in}{4.768611in}}%
\pgfpathlineto{\pgfqpoint{2.127335in}{4.768611in}}%
\pgfpathlineto{\pgfqpoint{2.127335in}{4.525453in}}%
\pgfpathclose%
\pgfusepath{fill}%
\end{pgfscope}%
\begin{pgfscope}%
\pgfpathrectangle{\pgfqpoint{2.127335in}{4.525453in}}{\pgfqpoint{0.824468in}{0.243158in}}%
\pgfusepath{clip}%
\pgfsetbuttcap%
\pgfsetmiterjoin%
\definecolor{currentfill}{rgb}{0.121569,0.466667,0.705882}%
\pgfsetfillcolor{currentfill}%
\pgfsetfillopacity{0.500000}%
\pgfsetlinewidth{1.003750pt}%
\definecolor{currentstroke}{rgb}{0.000000,0.000000,0.000000}%
\pgfsetstrokecolor{currentstroke}%
\pgfsetdash{}{0pt}%
\pgfpathmoveto{\pgfqpoint{2.164810in}{4.525453in}}%
\pgfpathlineto{\pgfqpoint{2.314714in}{4.525453in}}%
\pgfpathlineto{\pgfqpoint{2.314714in}{4.552079in}}%
\pgfpathlineto{\pgfqpoint{2.164810in}{4.552079in}}%
\pgfpathlineto{\pgfqpoint{2.164810in}{4.525453in}}%
\pgfpathclose%
\pgfusepath{stroke,fill}%
\end{pgfscope}%
\begin{pgfscope}%
\pgfpathrectangle{\pgfqpoint{2.127335in}{4.525453in}}{\pgfqpoint{0.824468in}{0.243158in}}%
\pgfusepath{clip}%
\pgfsetbuttcap%
\pgfsetmiterjoin%
\definecolor{currentfill}{rgb}{0.121569,0.466667,0.705882}%
\pgfsetfillcolor{currentfill}%
\pgfsetfillopacity{0.500000}%
\pgfsetlinewidth{1.003750pt}%
\definecolor{currentstroke}{rgb}{0.000000,0.000000,0.000000}%
\pgfsetstrokecolor{currentstroke}%
\pgfsetdash{}{0pt}%
\pgfpathmoveto{\pgfqpoint{2.314714in}{4.525453in}}%
\pgfpathlineto{\pgfqpoint{2.464617in}{4.525453in}}%
\pgfpathlineto{\pgfqpoint{2.464617in}{4.542110in}}%
\pgfpathlineto{\pgfqpoint{2.314714in}{4.542110in}}%
\pgfpathlineto{\pgfqpoint{2.314714in}{4.525453in}}%
\pgfpathclose%
\pgfusepath{stroke,fill}%
\end{pgfscope}%
\begin{pgfscope}%
\pgfpathrectangle{\pgfqpoint{2.127335in}{4.525453in}}{\pgfqpoint{0.824468in}{0.243158in}}%
\pgfusepath{clip}%
\pgfsetbuttcap%
\pgfsetmiterjoin%
\definecolor{currentfill}{rgb}{0.121569,0.466667,0.705882}%
\pgfsetfillcolor{currentfill}%
\pgfsetfillopacity{0.500000}%
\pgfsetlinewidth{1.003750pt}%
\definecolor{currentstroke}{rgb}{0.000000,0.000000,0.000000}%
\pgfsetstrokecolor{currentstroke}%
\pgfsetdash{}{0pt}%
\pgfpathmoveto{\pgfqpoint{2.464617in}{4.525453in}}%
\pgfpathlineto{\pgfqpoint{2.614520in}{4.525453in}}%
\pgfpathlineto{\pgfqpoint{2.614520in}{4.531289in}}%
\pgfpathlineto{\pgfqpoint{2.464617in}{4.531289in}}%
\pgfpathlineto{\pgfqpoint{2.464617in}{4.525453in}}%
\pgfpathclose%
\pgfusepath{stroke,fill}%
\end{pgfscope}%
\begin{pgfscope}%
\pgfpathrectangle{\pgfqpoint{2.127335in}{4.525453in}}{\pgfqpoint{0.824468in}{0.243158in}}%
\pgfusepath{clip}%
\pgfsetbuttcap%
\pgfsetmiterjoin%
\definecolor{currentfill}{rgb}{0.121569,0.466667,0.705882}%
\pgfsetfillcolor{currentfill}%
\pgfsetfillopacity{0.500000}%
\pgfsetlinewidth{1.003750pt}%
\definecolor{currentstroke}{rgb}{0.000000,0.000000,0.000000}%
\pgfsetstrokecolor{currentstroke}%
\pgfsetdash{}{0pt}%
\pgfpathmoveto{\pgfqpoint{2.614520in}{4.525453in}}%
\pgfpathlineto{\pgfqpoint{2.764423in}{4.525453in}}%
\pgfpathlineto{\pgfqpoint{2.764423in}{4.529708in}}%
\pgfpathlineto{\pgfqpoint{2.614520in}{4.529708in}}%
\pgfpathlineto{\pgfqpoint{2.614520in}{4.525453in}}%
\pgfpathclose%
\pgfusepath{stroke,fill}%
\end{pgfscope}%
\begin{pgfscope}%
\pgfpathrectangle{\pgfqpoint{2.127335in}{4.525453in}}{\pgfqpoint{0.824468in}{0.243158in}}%
\pgfusepath{clip}%
\pgfsetbuttcap%
\pgfsetmiterjoin%
\definecolor{currentfill}{rgb}{0.121569,0.466667,0.705882}%
\pgfsetfillcolor{currentfill}%
\pgfsetfillopacity{0.500000}%
\pgfsetlinewidth{1.003750pt}%
\definecolor{currentstroke}{rgb}{0.000000,0.000000,0.000000}%
\pgfsetstrokecolor{currentstroke}%
\pgfsetdash{}{0pt}%
\pgfpathmoveto{\pgfqpoint{2.764423in}{4.525453in}}%
\pgfpathlineto{\pgfqpoint{2.914327in}{4.525453in}}%
\pgfpathlineto{\pgfqpoint{2.914327in}{4.529708in}}%
\pgfpathlineto{\pgfqpoint{2.764423in}{4.529708in}}%
\pgfpathlineto{\pgfqpoint{2.764423in}{4.525453in}}%
\pgfpathclose%
\pgfusepath{stroke,fill}%
\end{pgfscope}%
\begin{pgfscope}%
\pgfsetrectcap%
\pgfsetmiterjoin%
\pgfsetlinewidth{0.803000pt}%
\definecolor{currentstroke}{rgb}{0.000000,0.000000,0.000000}%
\pgfsetstrokecolor{currentstroke}%
\pgfsetdash{}{0pt}%
\pgfpathmoveto{\pgfqpoint{2.127335in}{4.525453in}}%
\pgfpathlineto{\pgfqpoint{2.127335in}{4.768611in}}%
\pgfusepath{stroke}%
\end{pgfscope}%
\begin{pgfscope}%
\pgfsetrectcap%
\pgfsetmiterjoin%
\pgfsetlinewidth{0.803000pt}%
\definecolor{currentstroke}{rgb}{0.000000,0.000000,0.000000}%
\pgfsetstrokecolor{currentstroke}%
\pgfsetdash{}{0pt}%
\pgfpathmoveto{\pgfqpoint{2.951803in}{4.525453in}}%
\pgfpathlineto{\pgfqpoint{2.951803in}{4.768611in}}%
\pgfusepath{stroke}%
\end{pgfscope}%
\begin{pgfscope}%
\pgfsetrectcap%
\pgfsetmiterjoin%
\pgfsetlinewidth{0.803000pt}%
\definecolor{currentstroke}{rgb}{0.000000,0.000000,0.000000}%
\pgfsetstrokecolor{currentstroke}%
\pgfsetdash{}{0pt}%
\pgfpathmoveto{\pgfqpoint{2.127335in}{4.525453in}}%
\pgfpathlineto{\pgfqpoint{2.951803in}{4.525453in}}%
\pgfusepath{stroke}%
\end{pgfscope}%
\begin{pgfscope}%
\pgfsetrectcap%
\pgfsetmiterjoin%
\pgfsetlinewidth{0.803000pt}%
\definecolor{currentstroke}{rgb}{0.000000,0.000000,0.000000}%
\pgfsetstrokecolor{currentstroke}%
\pgfsetdash{}{0pt}%
\pgfpathmoveto{\pgfqpoint{2.127335in}{4.768611in}}%
\pgfpathlineto{\pgfqpoint{2.951803in}{4.768611in}}%
\pgfusepath{stroke}%
\end{pgfscope}%
\begin{pgfscope}%
\definecolor{textcolor}{rgb}{0.000000,0.000000,0.000000}%
\pgfsetstrokecolor{textcolor}%
\pgfsetfillcolor{textcolor}%
\pgftext[x=2.539569in,y=4.851944in,,base]{\color{textcolor}\rmfamily\fontsize{11.000000}{13.200000}\selectfont Matmut}%
\end{pgfscope}%
\begin{pgfscope}%
\pgfsetbuttcap%
\pgfsetmiterjoin%
\definecolor{currentfill}{rgb}{1.000000,1.000000,1.000000}%
\pgfsetfillcolor{currentfill}%
\pgfsetlinewidth{0.000000pt}%
\definecolor{currentstroke}{rgb}{0.000000,0.000000,0.000000}%
\pgfsetstrokecolor{currentstroke}%
\pgfsetstrokeopacity{0.000000}%
\pgfsetdash{}{0pt}%
\pgfpathmoveto{\pgfqpoint{3.116696in}{4.525453in}}%
\pgfpathlineto{\pgfqpoint{3.941164in}{4.525453in}}%
\pgfpathlineto{\pgfqpoint{3.941164in}{4.768611in}}%
\pgfpathlineto{\pgfqpoint{3.116696in}{4.768611in}}%
\pgfpathlineto{\pgfqpoint{3.116696in}{4.525453in}}%
\pgfpathclose%
\pgfusepath{fill}%
\end{pgfscope}%
\begin{pgfscope}%
\pgfpathrectangle{\pgfqpoint{3.116696in}{4.525453in}}{\pgfqpoint{0.824468in}{0.243158in}}%
\pgfusepath{clip}%
\pgfsetbuttcap%
\pgfsetmiterjoin%
\definecolor{currentfill}{rgb}{0.121569,0.466667,0.705882}%
\pgfsetfillcolor{currentfill}%
\pgfsetfillopacity{0.500000}%
\pgfsetlinewidth{1.003750pt}%
\definecolor{currentstroke}{rgb}{0.000000,0.000000,0.000000}%
\pgfsetstrokecolor{currentstroke}%
\pgfsetdash{}{0pt}%
\pgfpathmoveto{\pgfqpoint{3.154172in}{4.525453in}}%
\pgfpathlineto{\pgfqpoint{3.304075in}{4.525453in}}%
\pgfpathlineto{\pgfqpoint{3.304075in}{4.557915in}}%
\pgfpathlineto{\pgfqpoint{3.154172in}{4.557915in}}%
\pgfpathlineto{\pgfqpoint{3.154172in}{4.525453in}}%
\pgfpathclose%
\pgfusepath{stroke,fill}%
\end{pgfscope}%
\begin{pgfscope}%
\pgfpathrectangle{\pgfqpoint{3.116696in}{4.525453in}}{\pgfqpoint{0.824468in}{0.243158in}}%
\pgfusepath{clip}%
\pgfsetbuttcap%
\pgfsetmiterjoin%
\definecolor{currentfill}{rgb}{0.121569,0.466667,0.705882}%
\pgfsetfillcolor{currentfill}%
\pgfsetfillopacity{0.500000}%
\pgfsetlinewidth{1.003750pt}%
\definecolor{currentstroke}{rgb}{0.000000,0.000000,0.000000}%
\pgfsetstrokecolor{currentstroke}%
\pgfsetdash{}{0pt}%
\pgfpathmoveto{\pgfqpoint{3.304075in}{4.525453in}}%
\pgfpathlineto{\pgfqpoint{3.453979in}{4.525453in}}%
\pgfpathlineto{\pgfqpoint{3.453979in}{4.538341in}}%
\pgfpathlineto{\pgfqpoint{3.304075in}{4.538341in}}%
\pgfpathlineto{\pgfqpoint{3.304075in}{4.525453in}}%
\pgfpathclose%
\pgfusepath{stroke,fill}%
\end{pgfscope}%
\begin{pgfscope}%
\pgfpathrectangle{\pgfqpoint{3.116696in}{4.525453in}}{\pgfqpoint{0.824468in}{0.243158in}}%
\pgfusepath{clip}%
\pgfsetbuttcap%
\pgfsetmiterjoin%
\definecolor{currentfill}{rgb}{0.121569,0.466667,0.705882}%
\pgfsetfillcolor{currentfill}%
\pgfsetfillopacity{0.500000}%
\pgfsetlinewidth{1.003750pt}%
\definecolor{currentstroke}{rgb}{0.000000,0.000000,0.000000}%
\pgfsetstrokecolor{currentstroke}%
\pgfsetdash{}{0pt}%
\pgfpathmoveto{\pgfqpoint{3.453979in}{4.525453in}}%
\pgfpathlineto{\pgfqpoint{3.603882in}{4.525453in}}%
\pgfpathlineto{\pgfqpoint{3.603882in}{4.545635in}}%
\pgfpathlineto{\pgfqpoint{3.453979in}{4.545635in}}%
\pgfpathlineto{\pgfqpoint{3.453979in}{4.525453in}}%
\pgfpathclose%
\pgfusepath{stroke,fill}%
\end{pgfscope}%
\begin{pgfscope}%
\pgfpathrectangle{\pgfqpoint{3.116696in}{4.525453in}}{\pgfqpoint{0.824468in}{0.243158in}}%
\pgfusepath{clip}%
\pgfsetbuttcap%
\pgfsetmiterjoin%
\definecolor{currentfill}{rgb}{0.121569,0.466667,0.705882}%
\pgfsetfillcolor{currentfill}%
\pgfsetfillopacity{0.500000}%
\pgfsetlinewidth{1.003750pt}%
\definecolor{currentstroke}{rgb}{0.000000,0.000000,0.000000}%
\pgfsetstrokecolor{currentstroke}%
\pgfsetdash{}{0pt}%
\pgfpathmoveto{\pgfqpoint{3.603882in}{4.525453in}}%
\pgfpathlineto{\pgfqpoint{3.753785in}{4.525453in}}%
\pgfpathlineto{\pgfqpoint{3.753785in}{4.547337in}}%
\pgfpathlineto{\pgfqpoint{3.603882in}{4.547337in}}%
\pgfpathlineto{\pgfqpoint{3.603882in}{4.525453in}}%
\pgfpathclose%
\pgfusepath{stroke,fill}%
\end{pgfscope}%
\begin{pgfscope}%
\pgfpathrectangle{\pgfqpoint{3.116696in}{4.525453in}}{\pgfqpoint{0.824468in}{0.243158in}}%
\pgfusepath{clip}%
\pgfsetbuttcap%
\pgfsetmiterjoin%
\definecolor{currentfill}{rgb}{0.121569,0.466667,0.705882}%
\pgfsetfillcolor{currentfill}%
\pgfsetfillopacity{0.500000}%
\pgfsetlinewidth{1.003750pt}%
\definecolor{currentstroke}{rgb}{0.000000,0.000000,0.000000}%
\pgfsetstrokecolor{currentstroke}%
\pgfsetdash{}{0pt}%
\pgfpathmoveto{\pgfqpoint{3.753785in}{4.525453in}}%
\pgfpathlineto{\pgfqpoint{3.903688in}{4.525453in}}%
\pgfpathlineto{\pgfqpoint{3.903688in}{4.542717in}}%
\pgfpathlineto{\pgfqpoint{3.753785in}{4.542717in}}%
\pgfpathlineto{\pgfqpoint{3.753785in}{4.525453in}}%
\pgfpathclose%
\pgfusepath{stroke,fill}%
\end{pgfscope}%
\begin{pgfscope}%
\pgfsetrectcap%
\pgfsetmiterjoin%
\pgfsetlinewidth{0.803000pt}%
\definecolor{currentstroke}{rgb}{0.000000,0.000000,0.000000}%
\pgfsetstrokecolor{currentstroke}%
\pgfsetdash{}{0pt}%
\pgfpathmoveto{\pgfqpoint{3.116696in}{4.525453in}}%
\pgfpathlineto{\pgfqpoint{3.116696in}{4.768611in}}%
\pgfusepath{stroke}%
\end{pgfscope}%
\begin{pgfscope}%
\pgfsetrectcap%
\pgfsetmiterjoin%
\pgfsetlinewidth{0.803000pt}%
\definecolor{currentstroke}{rgb}{0.000000,0.000000,0.000000}%
\pgfsetstrokecolor{currentstroke}%
\pgfsetdash{}{0pt}%
\pgfpathmoveto{\pgfqpoint{3.941164in}{4.525453in}}%
\pgfpathlineto{\pgfqpoint{3.941164in}{4.768611in}}%
\pgfusepath{stroke}%
\end{pgfscope}%
\begin{pgfscope}%
\pgfsetrectcap%
\pgfsetmiterjoin%
\pgfsetlinewidth{0.803000pt}%
\definecolor{currentstroke}{rgb}{0.000000,0.000000,0.000000}%
\pgfsetstrokecolor{currentstroke}%
\pgfsetdash{}{0pt}%
\pgfpathmoveto{\pgfqpoint{3.116696in}{4.525453in}}%
\pgfpathlineto{\pgfqpoint{3.941164in}{4.525453in}}%
\pgfusepath{stroke}%
\end{pgfscope}%
\begin{pgfscope}%
\pgfsetrectcap%
\pgfsetmiterjoin%
\pgfsetlinewidth{0.803000pt}%
\definecolor{currentstroke}{rgb}{0.000000,0.000000,0.000000}%
\pgfsetstrokecolor{currentstroke}%
\pgfsetdash{}{0pt}%
\pgfpathmoveto{\pgfqpoint{3.116696in}{4.768611in}}%
\pgfpathlineto{\pgfqpoint{3.941164in}{4.768611in}}%
\pgfusepath{stroke}%
\end{pgfscope}%
\begin{pgfscope}%
\definecolor{textcolor}{rgb}{0.000000,0.000000,0.000000}%
\pgfsetstrokecolor{textcolor}%
\pgfsetfillcolor{textcolor}%
\pgftext[x=3.528930in,y=4.851944in,,base]{\color{textcolor}\rmfamily\fontsize{11.000000}{13.200000}\selectfont Néolia...}%
\end{pgfscope}%
\begin{pgfscope}%
\pgfsetbuttcap%
\pgfsetmiterjoin%
\definecolor{currentfill}{rgb}{1.000000,1.000000,1.000000}%
\pgfsetfillcolor{currentfill}%
\pgfsetlinewidth{0.000000pt}%
\definecolor{currentstroke}{rgb}{0.000000,0.000000,0.000000}%
\pgfsetstrokecolor{currentstroke}%
\pgfsetstrokeopacity{0.000000}%
\pgfsetdash{}{0pt}%
\pgfpathmoveto{\pgfqpoint{4.106058in}{4.525453in}}%
\pgfpathlineto{\pgfqpoint{4.930526in}{4.525453in}}%
\pgfpathlineto{\pgfqpoint{4.930526in}{4.768611in}}%
\pgfpathlineto{\pgfqpoint{4.106058in}{4.768611in}}%
\pgfpathlineto{\pgfqpoint{4.106058in}{4.525453in}}%
\pgfpathclose%
\pgfusepath{fill}%
\end{pgfscope}%
\begin{pgfscope}%
\pgfpathrectangle{\pgfqpoint{4.106058in}{4.525453in}}{\pgfqpoint{0.824468in}{0.243158in}}%
\pgfusepath{clip}%
\pgfsetbuttcap%
\pgfsetmiterjoin%
\definecolor{currentfill}{rgb}{0.121569,0.466667,0.705882}%
\pgfsetfillcolor{currentfill}%
\pgfsetfillopacity{0.500000}%
\pgfsetlinewidth{1.003750pt}%
\definecolor{currentstroke}{rgb}{0.000000,0.000000,0.000000}%
\pgfsetstrokecolor{currentstroke}%
\pgfsetdash{}{0pt}%
\pgfpathmoveto{\pgfqpoint{4.143534in}{4.525453in}}%
\pgfpathlineto{\pgfqpoint{4.293437in}{4.525453in}}%
\pgfpathlineto{\pgfqpoint{4.293437in}{4.539313in}}%
\pgfpathlineto{\pgfqpoint{4.143534in}{4.539313in}}%
\pgfpathlineto{\pgfqpoint{4.143534in}{4.525453in}}%
\pgfpathclose%
\pgfusepath{stroke,fill}%
\end{pgfscope}%
\begin{pgfscope}%
\pgfpathrectangle{\pgfqpoint{4.106058in}{4.525453in}}{\pgfqpoint{0.824468in}{0.243158in}}%
\pgfusepath{clip}%
\pgfsetbuttcap%
\pgfsetmiterjoin%
\definecolor{currentfill}{rgb}{0.121569,0.466667,0.705882}%
\pgfsetfillcolor{currentfill}%
\pgfsetfillopacity{0.500000}%
\pgfsetlinewidth{1.003750pt}%
\definecolor{currentstroke}{rgb}{0.000000,0.000000,0.000000}%
\pgfsetstrokecolor{currentstroke}%
\pgfsetdash{}{0pt}%
\pgfpathmoveto{\pgfqpoint{4.293437in}{4.525453in}}%
\pgfpathlineto{\pgfqpoint{4.443340in}{4.525453in}}%
\pgfpathlineto{\pgfqpoint{4.443340in}{4.531046in}}%
\pgfpathlineto{\pgfqpoint{4.293437in}{4.531046in}}%
\pgfpathlineto{\pgfqpoint{4.293437in}{4.525453in}}%
\pgfpathclose%
\pgfusepath{stroke,fill}%
\end{pgfscope}%
\begin{pgfscope}%
\pgfpathrectangle{\pgfqpoint{4.106058in}{4.525453in}}{\pgfqpoint{0.824468in}{0.243158in}}%
\pgfusepath{clip}%
\pgfsetbuttcap%
\pgfsetmiterjoin%
\definecolor{currentfill}{rgb}{0.121569,0.466667,0.705882}%
\pgfsetfillcolor{currentfill}%
\pgfsetfillopacity{0.500000}%
\pgfsetlinewidth{1.003750pt}%
\definecolor{currentstroke}{rgb}{0.000000,0.000000,0.000000}%
\pgfsetstrokecolor{currentstroke}%
\pgfsetdash{}{0pt}%
\pgfpathmoveto{\pgfqpoint{4.443340in}{4.525453in}}%
\pgfpathlineto{\pgfqpoint{4.593244in}{4.525453in}}%
\pgfpathlineto{\pgfqpoint{4.593244in}{4.531046in}}%
\pgfpathlineto{\pgfqpoint{4.443340in}{4.531046in}}%
\pgfpathlineto{\pgfqpoint{4.443340in}{4.525453in}}%
\pgfpathclose%
\pgfusepath{stroke,fill}%
\end{pgfscope}%
\begin{pgfscope}%
\pgfpathrectangle{\pgfqpoint{4.106058in}{4.525453in}}{\pgfqpoint{0.824468in}{0.243158in}}%
\pgfusepath{clip}%
\pgfsetbuttcap%
\pgfsetmiterjoin%
\definecolor{currentfill}{rgb}{0.121569,0.466667,0.705882}%
\pgfsetfillcolor{currentfill}%
\pgfsetfillopacity{0.500000}%
\pgfsetlinewidth{1.003750pt}%
\definecolor{currentstroke}{rgb}{0.000000,0.000000,0.000000}%
\pgfsetstrokecolor{currentstroke}%
\pgfsetdash{}{0pt}%
\pgfpathmoveto{\pgfqpoint{4.593244in}{4.525453in}}%
\pgfpathlineto{\pgfqpoint{4.743147in}{4.525453in}}%
\pgfpathlineto{\pgfqpoint{4.743147in}{4.530803in}}%
\pgfpathlineto{\pgfqpoint{4.593244in}{4.530803in}}%
\pgfpathlineto{\pgfqpoint{4.593244in}{4.525453in}}%
\pgfpathclose%
\pgfusepath{stroke,fill}%
\end{pgfscope}%
\begin{pgfscope}%
\pgfpathrectangle{\pgfqpoint{4.106058in}{4.525453in}}{\pgfqpoint{0.824468in}{0.243158in}}%
\pgfusepath{clip}%
\pgfsetbuttcap%
\pgfsetmiterjoin%
\definecolor{currentfill}{rgb}{0.121569,0.466667,0.705882}%
\pgfsetfillcolor{currentfill}%
\pgfsetfillopacity{0.500000}%
\pgfsetlinewidth{1.003750pt}%
\definecolor{currentstroke}{rgb}{0.000000,0.000000,0.000000}%
\pgfsetstrokecolor{currentstroke}%
\pgfsetdash{}{0pt}%
\pgfpathmoveto{\pgfqpoint{4.743147in}{4.525453in}}%
\pgfpathlineto{\pgfqpoint{4.893050in}{4.525453in}}%
\pgfpathlineto{\pgfqpoint{4.893050in}{4.529465in}}%
\pgfpathlineto{\pgfqpoint{4.743147in}{4.529465in}}%
\pgfpathlineto{\pgfqpoint{4.743147in}{4.525453in}}%
\pgfpathclose%
\pgfusepath{stroke,fill}%
\end{pgfscope}%
\begin{pgfscope}%
\pgfsetrectcap%
\pgfsetmiterjoin%
\pgfsetlinewidth{0.803000pt}%
\definecolor{currentstroke}{rgb}{0.000000,0.000000,0.000000}%
\pgfsetstrokecolor{currentstroke}%
\pgfsetdash{}{0pt}%
\pgfpathmoveto{\pgfqpoint{4.106058in}{4.525453in}}%
\pgfpathlineto{\pgfqpoint{4.106058in}{4.768611in}}%
\pgfusepath{stroke}%
\end{pgfscope}%
\begin{pgfscope}%
\pgfsetrectcap%
\pgfsetmiterjoin%
\pgfsetlinewidth{0.803000pt}%
\definecolor{currentstroke}{rgb}{0.000000,0.000000,0.000000}%
\pgfsetstrokecolor{currentstroke}%
\pgfsetdash{}{0pt}%
\pgfpathmoveto{\pgfqpoint{4.930526in}{4.525453in}}%
\pgfpathlineto{\pgfqpoint{4.930526in}{4.768611in}}%
\pgfusepath{stroke}%
\end{pgfscope}%
\begin{pgfscope}%
\pgfsetrectcap%
\pgfsetmiterjoin%
\pgfsetlinewidth{0.803000pt}%
\definecolor{currentstroke}{rgb}{0.000000,0.000000,0.000000}%
\pgfsetstrokecolor{currentstroke}%
\pgfsetdash{}{0pt}%
\pgfpathmoveto{\pgfqpoint{4.106058in}{4.525453in}}%
\pgfpathlineto{\pgfqpoint{4.930526in}{4.525453in}}%
\pgfusepath{stroke}%
\end{pgfscope}%
\begin{pgfscope}%
\pgfsetrectcap%
\pgfsetmiterjoin%
\pgfsetlinewidth{0.803000pt}%
\definecolor{currentstroke}{rgb}{0.000000,0.000000,0.000000}%
\pgfsetstrokecolor{currentstroke}%
\pgfsetdash{}{0pt}%
\pgfpathmoveto{\pgfqpoint{4.106058in}{4.768611in}}%
\pgfpathlineto{\pgfqpoint{4.930526in}{4.768611in}}%
\pgfusepath{stroke}%
\end{pgfscope}%
\begin{pgfscope}%
\definecolor{textcolor}{rgb}{0.000000,0.000000,0.000000}%
\pgfsetstrokecolor{textcolor}%
\pgfsetfillcolor{textcolor}%
\pgftext[x=4.518292in,y=4.851944in,,base]{\color{textcolor}\rmfamily\fontsize{11.000000}{13.200000}\selectfont APRIL}%
\end{pgfscope}%
\begin{pgfscope}%
\pgfsetbuttcap%
\pgfsetmiterjoin%
\definecolor{currentfill}{rgb}{1.000000,1.000000,1.000000}%
\pgfsetfillcolor{currentfill}%
\pgfsetlinewidth{0.000000pt}%
\definecolor{currentstroke}{rgb}{0.000000,0.000000,0.000000}%
\pgfsetstrokecolor{currentstroke}%
\pgfsetstrokeopacity{0.000000}%
\pgfsetdash{}{0pt}%
\pgfpathmoveto{\pgfqpoint{5.095420in}{4.525453in}}%
\pgfpathlineto{\pgfqpoint{5.919888in}{4.525453in}}%
\pgfpathlineto{\pgfqpoint{5.919888in}{4.768611in}}%
\pgfpathlineto{\pgfqpoint{5.095420in}{4.768611in}}%
\pgfpathlineto{\pgfqpoint{5.095420in}{4.525453in}}%
\pgfpathclose%
\pgfusepath{fill}%
\end{pgfscope}%
\begin{pgfscope}%
\pgfpathrectangle{\pgfqpoint{5.095420in}{4.525453in}}{\pgfqpoint{0.824468in}{0.243158in}}%
\pgfusepath{clip}%
\pgfsetbuttcap%
\pgfsetmiterjoin%
\definecolor{currentfill}{rgb}{0.121569,0.466667,0.705882}%
\pgfsetfillcolor{currentfill}%
\pgfsetfillopacity{0.500000}%
\pgfsetlinewidth{1.003750pt}%
\definecolor{currentstroke}{rgb}{0.000000,0.000000,0.000000}%
\pgfsetstrokecolor{currentstroke}%
\pgfsetdash{}{0pt}%
\pgfpathmoveto{\pgfqpoint{5.132895in}{4.525453in}}%
\pgfpathlineto{\pgfqpoint{5.282799in}{4.525453in}}%
\pgfpathlineto{\pgfqpoint{5.282799in}{4.532505in}}%
\pgfpathlineto{\pgfqpoint{5.132895in}{4.532505in}}%
\pgfpathlineto{\pgfqpoint{5.132895in}{4.525453in}}%
\pgfpathclose%
\pgfusepath{stroke,fill}%
\end{pgfscope}%
\begin{pgfscope}%
\pgfpathrectangle{\pgfqpoint{5.095420in}{4.525453in}}{\pgfqpoint{0.824468in}{0.243158in}}%
\pgfusepath{clip}%
\pgfsetbuttcap%
\pgfsetmiterjoin%
\definecolor{currentfill}{rgb}{0.121569,0.466667,0.705882}%
\pgfsetfillcolor{currentfill}%
\pgfsetfillopacity{0.500000}%
\pgfsetlinewidth{1.003750pt}%
\definecolor{currentstroke}{rgb}{0.000000,0.000000,0.000000}%
\pgfsetstrokecolor{currentstroke}%
\pgfsetdash{}{0pt}%
\pgfpathmoveto{\pgfqpoint{5.282799in}{4.525453in}}%
\pgfpathlineto{\pgfqpoint{5.432702in}{4.525453in}}%
\pgfpathlineto{\pgfqpoint{5.432702in}{4.527034in}}%
\pgfpathlineto{\pgfqpoint{5.282799in}{4.527034in}}%
\pgfpathlineto{\pgfqpoint{5.282799in}{4.525453in}}%
\pgfpathclose%
\pgfusepath{stroke,fill}%
\end{pgfscope}%
\begin{pgfscope}%
\pgfpathrectangle{\pgfqpoint{5.095420in}{4.525453in}}{\pgfqpoint{0.824468in}{0.243158in}}%
\pgfusepath{clip}%
\pgfsetbuttcap%
\pgfsetmiterjoin%
\definecolor{currentfill}{rgb}{0.121569,0.466667,0.705882}%
\pgfsetfillcolor{currentfill}%
\pgfsetfillopacity{0.500000}%
\pgfsetlinewidth{1.003750pt}%
\definecolor{currentstroke}{rgb}{0.000000,0.000000,0.000000}%
\pgfsetstrokecolor{currentstroke}%
\pgfsetdash{}{0pt}%
\pgfpathmoveto{\pgfqpoint{5.432702in}{4.525453in}}%
\pgfpathlineto{\pgfqpoint{5.582605in}{4.525453in}}%
\pgfpathlineto{\pgfqpoint{5.582605in}{4.526304in}}%
\pgfpathlineto{\pgfqpoint{5.432702in}{4.526304in}}%
\pgfpathlineto{\pgfqpoint{5.432702in}{4.525453in}}%
\pgfpathclose%
\pgfusepath{stroke,fill}%
\end{pgfscope}%
\begin{pgfscope}%
\pgfpathrectangle{\pgfqpoint{5.095420in}{4.525453in}}{\pgfqpoint{0.824468in}{0.243158in}}%
\pgfusepath{clip}%
\pgfsetbuttcap%
\pgfsetmiterjoin%
\definecolor{currentfill}{rgb}{0.121569,0.466667,0.705882}%
\pgfsetfillcolor{currentfill}%
\pgfsetfillopacity{0.500000}%
\pgfsetlinewidth{1.003750pt}%
\definecolor{currentstroke}{rgb}{0.000000,0.000000,0.000000}%
\pgfsetstrokecolor{currentstroke}%
\pgfsetdash{}{0pt}%
\pgfpathmoveto{\pgfqpoint{5.582605in}{4.525453in}}%
\pgfpathlineto{\pgfqpoint{5.732509in}{4.525453in}}%
\pgfpathlineto{\pgfqpoint{5.732509in}{4.526791in}}%
\pgfpathlineto{\pgfqpoint{5.582605in}{4.526791in}}%
\pgfpathlineto{\pgfqpoint{5.582605in}{4.525453in}}%
\pgfpathclose%
\pgfusepath{stroke,fill}%
\end{pgfscope}%
\begin{pgfscope}%
\pgfpathrectangle{\pgfqpoint{5.095420in}{4.525453in}}{\pgfqpoint{0.824468in}{0.243158in}}%
\pgfusepath{clip}%
\pgfsetbuttcap%
\pgfsetmiterjoin%
\definecolor{currentfill}{rgb}{0.121569,0.466667,0.705882}%
\pgfsetfillcolor{currentfill}%
\pgfsetfillopacity{0.500000}%
\pgfsetlinewidth{1.003750pt}%
\definecolor{currentstroke}{rgb}{0.000000,0.000000,0.000000}%
\pgfsetstrokecolor{currentstroke}%
\pgfsetdash{}{0pt}%
\pgfpathmoveto{\pgfqpoint{5.732509in}{4.525453in}}%
\pgfpathlineto{\pgfqpoint{5.882412in}{4.525453in}}%
\pgfpathlineto{\pgfqpoint{5.882412in}{4.526304in}}%
\pgfpathlineto{\pgfqpoint{5.732509in}{4.526304in}}%
\pgfpathlineto{\pgfqpoint{5.732509in}{4.525453in}}%
\pgfpathclose%
\pgfusepath{stroke,fill}%
\end{pgfscope}%
\begin{pgfscope}%
\pgfsetrectcap%
\pgfsetmiterjoin%
\pgfsetlinewidth{0.803000pt}%
\definecolor{currentstroke}{rgb}{0.000000,0.000000,0.000000}%
\pgfsetstrokecolor{currentstroke}%
\pgfsetdash{}{0pt}%
\pgfpathmoveto{\pgfqpoint{5.095420in}{4.525453in}}%
\pgfpathlineto{\pgfqpoint{5.095420in}{4.768611in}}%
\pgfusepath{stroke}%
\end{pgfscope}%
\begin{pgfscope}%
\pgfsetrectcap%
\pgfsetmiterjoin%
\pgfsetlinewidth{0.803000pt}%
\definecolor{currentstroke}{rgb}{0.000000,0.000000,0.000000}%
\pgfsetstrokecolor{currentstroke}%
\pgfsetdash{}{0pt}%
\pgfpathmoveto{\pgfqpoint{5.919888in}{4.525453in}}%
\pgfpathlineto{\pgfqpoint{5.919888in}{4.768611in}}%
\pgfusepath{stroke}%
\end{pgfscope}%
\begin{pgfscope}%
\pgfsetrectcap%
\pgfsetmiterjoin%
\pgfsetlinewidth{0.803000pt}%
\definecolor{currentstroke}{rgb}{0.000000,0.000000,0.000000}%
\pgfsetstrokecolor{currentstroke}%
\pgfsetdash{}{0pt}%
\pgfpathmoveto{\pgfqpoint{5.095420in}{4.525453in}}%
\pgfpathlineto{\pgfqpoint{5.919888in}{4.525453in}}%
\pgfusepath{stroke}%
\end{pgfscope}%
\begin{pgfscope}%
\pgfsetrectcap%
\pgfsetmiterjoin%
\pgfsetlinewidth{0.803000pt}%
\definecolor{currentstroke}{rgb}{0.000000,0.000000,0.000000}%
\pgfsetstrokecolor{currentstroke}%
\pgfsetdash{}{0pt}%
\pgfpathmoveto{\pgfqpoint{5.095420in}{4.768611in}}%
\pgfpathlineto{\pgfqpoint{5.919888in}{4.768611in}}%
\pgfusepath{stroke}%
\end{pgfscope}%
\begin{pgfscope}%
\definecolor{textcolor}{rgb}{0.000000,0.000000,0.000000}%
\pgfsetstrokecolor{textcolor}%
\pgfsetfillcolor{textcolor}%
\pgftext[x=5.507654in,y=4.851944in,,base]{\color{textcolor}\rmfamily\fontsize{11.000000}{13.200000}\selectfont SantéVet}%
\end{pgfscope}%
\begin{pgfscope}%
\pgfsetbuttcap%
\pgfsetmiterjoin%
\definecolor{currentfill}{rgb}{1.000000,1.000000,1.000000}%
\pgfsetfillcolor{currentfill}%
\pgfsetlinewidth{0.000000pt}%
\definecolor{currentstroke}{rgb}{0.000000,0.000000,0.000000}%
\pgfsetstrokecolor{currentstroke}%
\pgfsetstrokeopacity{0.000000}%
\pgfsetdash{}{0pt}%
\pgfpathmoveto{\pgfqpoint{6.084781in}{4.525453in}}%
\pgfpathlineto{\pgfqpoint{6.909249in}{4.525453in}}%
\pgfpathlineto{\pgfqpoint{6.909249in}{4.768611in}}%
\pgfpathlineto{\pgfqpoint{6.084781in}{4.768611in}}%
\pgfpathlineto{\pgfqpoint{6.084781in}{4.525453in}}%
\pgfpathclose%
\pgfusepath{fill}%
\end{pgfscope}%
\begin{pgfscope}%
\pgfpathrectangle{\pgfqpoint{6.084781in}{4.525453in}}{\pgfqpoint{0.824468in}{0.243158in}}%
\pgfusepath{clip}%
\pgfsetbuttcap%
\pgfsetmiterjoin%
\definecolor{currentfill}{rgb}{0.121569,0.466667,0.705882}%
\pgfsetfillcolor{currentfill}%
\pgfsetfillopacity{0.500000}%
\pgfsetlinewidth{1.003750pt}%
\definecolor{currentstroke}{rgb}{0.000000,0.000000,0.000000}%
\pgfsetstrokecolor{currentstroke}%
\pgfsetdash{}{0pt}%
\pgfpathmoveto{\pgfqpoint{6.122257in}{4.525453in}}%
\pgfpathlineto{\pgfqpoint{6.272160in}{4.525453in}}%
\pgfpathlineto{\pgfqpoint{6.272160in}{4.550620in}}%
\pgfpathlineto{\pgfqpoint{6.122257in}{4.550620in}}%
\pgfpathlineto{\pgfqpoint{6.122257in}{4.525453in}}%
\pgfpathclose%
\pgfusepath{stroke,fill}%
\end{pgfscope}%
\begin{pgfscope}%
\pgfpathrectangle{\pgfqpoint{6.084781in}{4.525453in}}{\pgfqpoint{0.824468in}{0.243158in}}%
\pgfusepath{clip}%
\pgfsetbuttcap%
\pgfsetmiterjoin%
\definecolor{currentfill}{rgb}{0.121569,0.466667,0.705882}%
\pgfsetfillcolor{currentfill}%
\pgfsetfillopacity{0.500000}%
\pgfsetlinewidth{1.003750pt}%
\definecolor{currentstroke}{rgb}{0.000000,0.000000,0.000000}%
\pgfsetstrokecolor{currentstroke}%
\pgfsetdash{}{0pt}%
\pgfpathmoveto{\pgfqpoint{6.272160in}{4.525453in}}%
\pgfpathlineto{\pgfqpoint{6.422064in}{4.525453in}}%
\pgfpathlineto{\pgfqpoint{6.422064in}{4.531046in}}%
\pgfpathlineto{\pgfqpoint{6.272160in}{4.531046in}}%
\pgfpathlineto{\pgfqpoint{6.272160in}{4.525453in}}%
\pgfpathclose%
\pgfusepath{stroke,fill}%
\end{pgfscope}%
\begin{pgfscope}%
\pgfpathrectangle{\pgfqpoint{6.084781in}{4.525453in}}{\pgfqpoint{0.824468in}{0.243158in}}%
\pgfusepath{clip}%
\pgfsetbuttcap%
\pgfsetmiterjoin%
\definecolor{currentfill}{rgb}{0.121569,0.466667,0.705882}%
\pgfsetfillcolor{currentfill}%
\pgfsetfillopacity{0.500000}%
\pgfsetlinewidth{1.003750pt}%
\definecolor{currentstroke}{rgb}{0.000000,0.000000,0.000000}%
\pgfsetstrokecolor{currentstroke}%
\pgfsetdash{}{0pt}%
\pgfpathmoveto{\pgfqpoint{6.422064in}{4.525453in}}%
\pgfpathlineto{\pgfqpoint{6.571967in}{4.525453in}}%
\pgfpathlineto{\pgfqpoint{6.571967in}{4.526669in}}%
\pgfpathlineto{\pgfqpoint{6.422064in}{4.526669in}}%
\pgfpathlineto{\pgfqpoint{6.422064in}{4.525453in}}%
\pgfpathclose%
\pgfusepath{stroke,fill}%
\end{pgfscope}%
\begin{pgfscope}%
\pgfpathrectangle{\pgfqpoint{6.084781in}{4.525453in}}{\pgfqpoint{0.824468in}{0.243158in}}%
\pgfusepath{clip}%
\pgfsetbuttcap%
\pgfsetmiterjoin%
\definecolor{currentfill}{rgb}{0.121569,0.466667,0.705882}%
\pgfsetfillcolor{currentfill}%
\pgfsetfillopacity{0.500000}%
\pgfsetlinewidth{1.003750pt}%
\definecolor{currentstroke}{rgb}{0.000000,0.000000,0.000000}%
\pgfsetstrokecolor{currentstroke}%
\pgfsetdash{}{0pt}%
\pgfpathmoveto{\pgfqpoint{6.571967in}{4.525453in}}%
\pgfpathlineto{\pgfqpoint{6.721870in}{4.525453in}}%
\pgfpathlineto{\pgfqpoint{6.721870in}{4.525818in}}%
\pgfpathlineto{\pgfqpoint{6.571967in}{4.525818in}}%
\pgfpathlineto{\pgfqpoint{6.571967in}{4.525453in}}%
\pgfpathclose%
\pgfusepath{stroke,fill}%
\end{pgfscope}%
\begin{pgfscope}%
\pgfpathrectangle{\pgfqpoint{6.084781in}{4.525453in}}{\pgfqpoint{0.824468in}{0.243158in}}%
\pgfusepath{clip}%
\pgfsetbuttcap%
\pgfsetmiterjoin%
\definecolor{currentfill}{rgb}{0.121569,0.466667,0.705882}%
\pgfsetfillcolor{currentfill}%
\pgfsetfillopacity{0.500000}%
\pgfsetlinewidth{1.003750pt}%
\definecolor{currentstroke}{rgb}{0.000000,0.000000,0.000000}%
\pgfsetstrokecolor{currentstroke}%
\pgfsetdash{}{0pt}%
\pgfpathmoveto{\pgfqpoint{6.721870in}{4.525453in}}%
\pgfpathlineto{\pgfqpoint{6.871774in}{4.525453in}}%
\pgfpathlineto{\pgfqpoint{6.871774in}{4.525575in}}%
\pgfpathlineto{\pgfqpoint{6.721870in}{4.525575in}}%
\pgfpathlineto{\pgfqpoint{6.721870in}{4.525453in}}%
\pgfpathclose%
\pgfusepath{stroke,fill}%
\end{pgfscope}%
\begin{pgfscope}%
\pgfsetrectcap%
\pgfsetmiterjoin%
\pgfsetlinewidth{0.803000pt}%
\definecolor{currentstroke}{rgb}{0.000000,0.000000,0.000000}%
\pgfsetstrokecolor{currentstroke}%
\pgfsetdash{}{0pt}%
\pgfpathmoveto{\pgfqpoint{6.084781in}{4.525453in}}%
\pgfpathlineto{\pgfqpoint{6.084781in}{4.768611in}}%
\pgfusepath{stroke}%
\end{pgfscope}%
\begin{pgfscope}%
\pgfsetrectcap%
\pgfsetmiterjoin%
\pgfsetlinewidth{0.803000pt}%
\definecolor{currentstroke}{rgb}{0.000000,0.000000,0.000000}%
\pgfsetstrokecolor{currentstroke}%
\pgfsetdash{}{0pt}%
\pgfpathmoveto{\pgfqpoint{6.909249in}{4.525453in}}%
\pgfpathlineto{\pgfqpoint{6.909249in}{4.768611in}}%
\pgfusepath{stroke}%
\end{pgfscope}%
\begin{pgfscope}%
\pgfsetrectcap%
\pgfsetmiterjoin%
\pgfsetlinewidth{0.803000pt}%
\definecolor{currentstroke}{rgb}{0.000000,0.000000,0.000000}%
\pgfsetstrokecolor{currentstroke}%
\pgfsetdash{}{0pt}%
\pgfpathmoveto{\pgfqpoint{6.084781in}{4.525453in}}%
\pgfpathlineto{\pgfqpoint{6.909249in}{4.525453in}}%
\pgfusepath{stroke}%
\end{pgfscope}%
\begin{pgfscope}%
\pgfsetrectcap%
\pgfsetmiterjoin%
\pgfsetlinewidth{0.803000pt}%
\definecolor{currentstroke}{rgb}{0.000000,0.000000,0.000000}%
\pgfsetstrokecolor{currentstroke}%
\pgfsetdash{}{0pt}%
\pgfpathmoveto{\pgfqpoint{6.084781in}{4.768611in}}%
\pgfpathlineto{\pgfqpoint{6.909249in}{4.768611in}}%
\pgfusepath{stroke}%
\end{pgfscope}%
\begin{pgfscope}%
\definecolor{textcolor}{rgb}{0.000000,0.000000,0.000000}%
\pgfsetstrokecolor{textcolor}%
\pgfsetfillcolor{textcolor}%
\pgftext[x=6.497015in,y=4.851944in,,base]{\color{textcolor}\rmfamily\fontsize{11.000000}{13.200000}\selectfont Mercer}%
\end{pgfscope}%
\begin{pgfscope}%
\pgfsetbuttcap%
\pgfsetmiterjoin%
\definecolor{currentfill}{rgb}{1.000000,1.000000,1.000000}%
\pgfsetfillcolor{currentfill}%
\pgfsetlinewidth{0.000000pt}%
\definecolor{currentstroke}{rgb}{0.000000,0.000000,0.000000}%
\pgfsetstrokecolor{currentstroke}%
\pgfsetstrokeopacity{0.000000}%
\pgfsetdash{}{0pt}%
\pgfpathmoveto{\pgfqpoint{7.074143in}{4.525453in}}%
\pgfpathlineto{\pgfqpoint{7.898611in}{4.525453in}}%
\pgfpathlineto{\pgfqpoint{7.898611in}{4.768611in}}%
\pgfpathlineto{\pgfqpoint{7.074143in}{4.768611in}}%
\pgfpathlineto{\pgfqpoint{7.074143in}{4.525453in}}%
\pgfpathclose%
\pgfusepath{fill}%
\end{pgfscope}%
\begin{pgfscope}%
\pgfpathrectangle{\pgfqpoint{7.074143in}{4.525453in}}{\pgfqpoint{0.824468in}{0.243158in}}%
\pgfusepath{clip}%
\pgfsetbuttcap%
\pgfsetmiterjoin%
\definecolor{currentfill}{rgb}{0.121569,0.466667,0.705882}%
\pgfsetfillcolor{currentfill}%
\pgfsetfillopacity{0.500000}%
\pgfsetlinewidth{1.003750pt}%
\definecolor{currentstroke}{rgb}{0.000000,0.000000,0.000000}%
\pgfsetstrokecolor{currentstroke}%
\pgfsetdash{}{0pt}%
\pgfpathmoveto{\pgfqpoint{7.111619in}{4.525453in}}%
\pgfpathlineto{\pgfqpoint{7.261522in}{4.525453in}}%
\pgfpathlineto{\pgfqpoint{7.261522in}{4.536517in}}%
\pgfpathlineto{\pgfqpoint{7.111619in}{4.536517in}}%
\pgfpathlineto{\pgfqpoint{7.111619in}{4.525453in}}%
\pgfpathclose%
\pgfusepath{stroke,fill}%
\end{pgfscope}%
\begin{pgfscope}%
\pgfpathrectangle{\pgfqpoint{7.074143in}{4.525453in}}{\pgfqpoint{0.824468in}{0.243158in}}%
\pgfusepath{clip}%
\pgfsetbuttcap%
\pgfsetmiterjoin%
\definecolor{currentfill}{rgb}{0.121569,0.466667,0.705882}%
\pgfsetfillcolor{currentfill}%
\pgfsetfillopacity{0.500000}%
\pgfsetlinewidth{1.003750pt}%
\definecolor{currentstroke}{rgb}{0.000000,0.000000,0.000000}%
\pgfsetstrokecolor{currentstroke}%
\pgfsetdash{}{0pt}%
\pgfpathmoveto{\pgfqpoint{7.261522in}{4.525453in}}%
\pgfpathlineto{\pgfqpoint{7.411425in}{4.525453in}}%
\pgfpathlineto{\pgfqpoint{7.411425in}{4.527155in}}%
\pgfpathlineto{\pgfqpoint{7.261522in}{4.527155in}}%
\pgfpathlineto{\pgfqpoint{7.261522in}{4.525453in}}%
\pgfpathclose%
\pgfusepath{stroke,fill}%
\end{pgfscope}%
\begin{pgfscope}%
\pgfpathrectangle{\pgfqpoint{7.074143in}{4.525453in}}{\pgfqpoint{0.824468in}{0.243158in}}%
\pgfusepath{clip}%
\pgfsetbuttcap%
\pgfsetmiterjoin%
\definecolor{currentfill}{rgb}{0.121569,0.466667,0.705882}%
\pgfsetfillcolor{currentfill}%
\pgfsetfillopacity{0.500000}%
\pgfsetlinewidth{1.003750pt}%
\definecolor{currentstroke}{rgb}{0.000000,0.000000,0.000000}%
\pgfsetstrokecolor{currentstroke}%
\pgfsetdash{}{0pt}%
\pgfpathmoveto{\pgfqpoint{7.411425in}{4.525453in}}%
\pgfpathlineto{\pgfqpoint{7.561329in}{4.525453in}}%
\pgfpathlineto{\pgfqpoint{7.561329in}{4.527642in}}%
\pgfpathlineto{\pgfqpoint{7.411425in}{4.527642in}}%
\pgfpathlineto{\pgfqpoint{7.411425in}{4.525453in}}%
\pgfpathclose%
\pgfusepath{stroke,fill}%
\end{pgfscope}%
\begin{pgfscope}%
\pgfpathrectangle{\pgfqpoint{7.074143in}{4.525453in}}{\pgfqpoint{0.824468in}{0.243158in}}%
\pgfusepath{clip}%
\pgfsetbuttcap%
\pgfsetmiterjoin%
\definecolor{currentfill}{rgb}{0.121569,0.466667,0.705882}%
\pgfsetfillcolor{currentfill}%
\pgfsetfillopacity{0.500000}%
\pgfsetlinewidth{1.003750pt}%
\definecolor{currentstroke}{rgb}{0.000000,0.000000,0.000000}%
\pgfsetstrokecolor{currentstroke}%
\pgfsetdash{}{0pt}%
\pgfpathmoveto{\pgfqpoint{7.561329in}{4.525453in}}%
\pgfpathlineto{\pgfqpoint{7.711232in}{4.525453in}}%
\pgfpathlineto{\pgfqpoint{7.711232in}{4.525575in}}%
\pgfpathlineto{\pgfqpoint{7.561329in}{4.525575in}}%
\pgfpathlineto{\pgfqpoint{7.561329in}{4.525453in}}%
\pgfpathclose%
\pgfusepath{stroke,fill}%
\end{pgfscope}%
\begin{pgfscope}%
\pgfpathrectangle{\pgfqpoint{7.074143in}{4.525453in}}{\pgfqpoint{0.824468in}{0.243158in}}%
\pgfusepath{clip}%
\pgfsetbuttcap%
\pgfsetmiterjoin%
\definecolor{currentfill}{rgb}{0.121569,0.466667,0.705882}%
\pgfsetfillcolor{currentfill}%
\pgfsetfillopacity{0.500000}%
\pgfsetlinewidth{1.003750pt}%
\definecolor{currentstroke}{rgb}{0.000000,0.000000,0.000000}%
\pgfsetstrokecolor{currentstroke}%
\pgfsetdash{}{0pt}%
\pgfpathmoveto{\pgfqpoint{7.711232in}{4.525453in}}%
\pgfpathlineto{\pgfqpoint{7.861135in}{4.525453in}}%
\pgfpathlineto{\pgfqpoint{7.861135in}{4.525940in}}%
\pgfpathlineto{\pgfqpoint{7.711232in}{4.525940in}}%
\pgfpathlineto{\pgfqpoint{7.711232in}{4.525453in}}%
\pgfpathclose%
\pgfusepath{stroke,fill}%
\end{pgfscope}%
\begin{pgfscope}%
\pgfsetrectcap%
\pgfsetmiterjoin%
\pgfsetlinewidth{0.803000pt}%
\definecolor{currentstroke}{rgb}{0.000000,0.000000,0.000000}%
\pgfsetstrokecolor{currentstroke}%
\pgfsetdash{}{0pt}%
\pgfpathmoveto{\pgfqpoint{7.074143in}{4.525453in}}%
\pgfpathlineto{\pgfqpoint{7.074143in}{4.768611in}}%
\pgfusepath{stroke}%
\end{pgfscope}%
\begin{pgfscope}%
\pgfsetrectcap%
\pgfsetmiterjoin%
\pgfsetlinewidth{0.803000pt}%
\definecolor{currentstroke}{rgb}{0.000000,0.000000,0.000000}%
\pgfsetstrokecolor{currentstroke}%
\pgfsetdash{}{0pt}%
\pgfpathmoveto{\pgfqpoint{7.898611in}{4.525453in}}%
\pgfpathlineto{\pgfqpoint{7.898611in}{4.768611in}}%
\pgfusepath{stroke}%
\end{pgfscope}%
\begin{pgfscope}%
\pgfsetrectcap%
\pgfsetmiterjoin%
\pgfsetlinewidth{0.803000pt}%
\definecolor{currentstroke}{rgb}{0.000000,0.000000,0.000000}%
\pgfsetstrokecolor{currentstroke}%
\pgfsetdash{}{0pt}%
\pgfpathmoveto{\pgfqpoint{7.074143in}{4.525453in}}%
\pgfpathlineto{\pgfqpoint{7.898611in}{4.525453in}}%
\pgfusepath{stroke}%
\end{pgfscope}%
\begin{pgfscope}%
\pgfsetrectcap%
\pgfsetmiterjoin%
\pgfsetlinewidth{0.803000pt}%
\definecolor{currentstroke}{rgb}{0.000000,0.000000,0.000000}%
\pgfsetstrokecolor{currentstroke}%
\pgfsetdash{}{0pt}%
\pgfpathmoveto{\pgfqpoint{7.074143in}{4.768611in}}%
\pgfpathlineto{\pgfqpoint{7.898611in}{4.768611in}}%
\pgfusepath{stroke}%
\end{pgfscope}%
\begin{pgfscope}%
\definecolor{textcolor}{rgb}{0.000000,0.000000,0.000000}%
\pgfsetstrokecolor{textcolor}%
\pgfsetfillcolor{textcolor}%
\pgftext[x=7.486377in,y=4.851944in,,base]{\color{textcolor}\rmfamily\fontsize{11.000000}{13.200000}\selectfont Generali}%
\end{pgfscope}%
\begin{pgfscope}%
\pgfsetbuttcap%
\pgfsetmiterjoin%
\definecolor{currentfill}{rgb}{1.000000,1.000000,1.000000}%
\pgfsetfillcolor{currentfill}%
\pgfsetlinewidth{0.000000pt}%
\definecolor{currentstroke}{rgb}{0.000000,0.000000,0.000000}%
\pgfsetstrokecolor{currentstroke}%
\pgfsetstrokeopacity{0.000000}%
\pgfsetdash{}{0pt}%
\pgfpathmoveto{\pgfqpoint{0.148611in}{3.795980in}}%
\pgfpathlineto{\pgfqpoint{0.973079in}{3.795980in}}%
\pgfpathlineto{\pgfqpoint{0.973079in}{4.039137in}}%
\pgfpathlineto{\pgfqpoint{0.148611in}{4.039137in}}%
\pgfpathlineto{\pgfqpoint{0.148611in}{3.795980in}}%
\pgfpathclose%
\pgfusepath{fill}%
\end{pgfscope}%
\begin{pgfscope}%
\pgfpathrectangle{\pgfqpoint{0.148611in}{3.795980in}}{\pgfqpoint{0.824468in}{0.243158in}}%
\pgfusepath{clip}%
\pgfsetbuttcap%
\pgfsetmiterjoin%
\definecolor{currentfill}{rgb}{0.121569,0.466667,0.705882}%
\pgfsetfillcolor{currentfill}%
\pgfsetfillopacity{0.500000}%
\pgfsetlinewidth{1.003750pt}%
\definecolor{currentstroke}{rgb}{0.000000,0.000000,0.000000}%
\pgfsetstrokecolor{currentstroke}%
\pgfsetdash{}{0pt}%
\pgfpathmoveto{\pgfqpoint{0.186087in}{3.795980in}}%
\pgfpathlineto{\pgfqpoint{0.335990in}{3.795980in}}%
\pgfpathlineto{\pgfqpoint{0.335990in}{3.842544in}}%
\pgfpathlineto{\pgfqpoint{0.186087in}{3.842544in}}%
\pgfpathlineto{\pgfqpoint{0.186087in}{3.795980in}}%
\pgfpathclose%
\pgfusepath{stroke,fill}%
\end{pgfscope}%
\begin{pgfscope}%
\pgfpathrectangle{\pgfqpoint{0.148611in}{3.795980in}}{\pgfqpoint{0.824468in}{0.243158in}}%
\pgfusepath{clip}%
\pgfsetbuttcap%
\pgfsetmiterjoin%
\definecolor{currentfill}{rgb}{0.121569,0.466667,0.705882}%
\pgfsetfillcolor{currentfill}%
\pgfsetfillopacity{0.500000}%
\pgfsetlinewidth{1.003750pt}%
\definecolor{currentstroke}{rgb}{0.000000,0.000000,0.000000}%
\pgfsetstrokecolor{currentstroke}%
\pgfsetdash{}{0pt}%
\pgfpathmoveto{\pgfqpoint{0.335990in}{3.795980in}}%
\pgfpathlineto{\pgfqpoint{0.485894in}{3.795980in}}%
\pgfpathlineto{\pgfqpoint{0.485894in}{3.809596in}}%
\pgfpathlineto{\pgfqpoint{0.335990in}{3.809596in}}%
\pgfpathlineto{\pgfqpoint{0.335990in}{3.795980in}}%
\pgfpathclose%
\pgfusepath{stroke,fill}%
\end{pgfscope}%
\begin{pgfscope}%
\pgfpathrectangle{\pgfqpoint{0.148611in}{3.795980in}}{\pgfqpoint{0.824468in}{0.243158in}}%
\pgfusepath{clip}%
\pgfsetbuttcap%
\pgfsetmiterjoin%
\definecolor{currentfill}{rgb}{0.121569,0.466667,0.705882}%
\pgfsetfillcolor{currentfill}%
\pgfsetfillopacity{0.500000}%
\pgfsetlinewidth{1.003750pt}%
\definecolor{currentstroke}{rgb}{0.000000,0.000000,0.000000}%
\pgfsetstrokecolor{currentstroke}%
\pgfsetdash{}{0pt}%
\pgfpathmoveto{\pgfqpoint{0.485894in}{3.795980in}}%
\pgfpathlineto{\pgfqpoint{0.635797in}{3.795980in}}%
\pgfpathlineto{\pgfqpoint{0.635797in}{3.802180in}}%
\pgfpathlineto{\pgfqpoint{0.485894in}{3.802180in}}%
\pgfpathlineto{\pgfqpoint{0.485894in}{3.795980in}}%
\pgfpathclose%
\pgfusepath{stroke,fill}%
\end{pgfscope}%
\begin{pgfscope}%
\pgfpathrectangle{\pgfqpoint{0.148611in}{3.795980in}}{\pgfqpoint{0.824468in}{0.243158in}}%
\pgfusepath{clip}%
\pgfsetbuttcap%
\pgfsetmiterjoin%
\definecolor{currentfill}{rgb}{0.121569,0.466667,0.705882}%
\pgfsetfillcolor{currentfill}%
\pgfsetfillopacity{0.500000}%
\pgfsetlinewidth{1.003750pt}%
\definecolor{currentstroke}{rgb}{0.000000,0.000000,0.000000}%
\pgfsetstrokecolor{currentstroke}%
\pgfsetdash{}{0pt}%
\pgfpathmoveto{\pgfqpoint{0.635797in}{3.795980in}}%
\pgfpathlineto{\pgfqpoint{0.785700in}{3.795980in}}%
\pgfpathlineto{\pgfqpoint{0.785700in}{3.798168in}}%
\pgfpathlineto{\pgfqpoint{0.635797in}{3.798168in}}%
\pgfpathlineto{\pgfqpoint{0.635797in}{3.795980in}}%
\pgfpathclose%
\pgfusepath{stroke,fill}%
\end{pgfscope}%
\begin{pgfscope}%
\pgfpathrectangle{\pgfqpoint{0.148611in}{3.795980in}}{\pgfqpoint{0.824468in}{0.243158in}}%
\pgfusepath{clip}%
\pgfsetbuttcap%
\pgfsetmiterjoin%
\definecolor{currentfill}{rgb}{0.121569,0.466667,0.705882}%
\pgfsetfillcolor{currentfill}%
\pgfsetfillopacity{0.500000}%
\pgfsetlinewidth{1.003750pt}%
\definecolor{currentstroke}{rgb}{0.000000,0.000000,0.000000}%
\pgfsetstrokecolor{currentstroke}%
\pgfsetdash{}{0pt}%
\pgfpathmoveto{\pgfqpoint{0.785700in}{3.795980in}}%
\pgfpathlineto{\pgfqpoint{0.935603in}{3.795980in}}%
\pgfpathlineto{\pgfqpoint{0.935603in}{3.797195in}}%
\pgfpathlineto{\pgfqpoint{0.785700in}{3.797195in}}%
\pgfpathlineto{\pgfqpoint{0.785700in}{3.795980in}}%
\pgfpathclose%
\pgfusepath{stroke,fill}%
\end{pgfscope}%
\begin{pgfscope}%
\pgfsetrectcap%
\pgfsetmiterjoin%
\pgfsetlinewidth{0.803000pt}%
\definecolor{currentstroke}{rgb}{0.000000,0.000000,0.000000}%
\pgfsetstrokecolor{currentstroke}%
\pgfsetdash{}{0pt}%
\pgfpathmoveto{\pgfqpoint{0.148611in}{3.795980in}}%
\pgfpathlineto{\pgfqpoint{0.148611in}{4.039137in}}%
\pgfusepath{stroke}%
\end{pgfscope}%
\begin{pgfscope}%
\pgfsetrectcap%
\pgfsetmiterjoin%
\pgfsetlinewidth{0.803000pt}%
\definecolor{currentstroke}{rgb}{0.000000,0.000000,0.000000}%
\pgfsetstrokecolor{currentstroke}%
\pgfsetdash{}{0pt}%
\pgfpathmoveto{\pgfqpoint{0.973079in}{3.795980in}}%
\pgfpathlineto{\pgfqpoint{0.973079in}{4.039137in}}%
\pgfusepath{stroke}%
\end{pgfscope}%
\begin{pgfscope}%
\pgfsetrectcap%
\pgfsetmiterjoin%
\pgfsetlinewidth{0.803000pt}%
\definecolor{currentstroke}{rgb}{0.000000,0.000000,0.000000}%
\pgfsetstrokecolor{currentstroke}%
\pgfsetdash{}{0pt}%
\pgfpathmoveto{\pgfqpoint{0.148611in}{3.795980in}}%
\pgfpathlineto{\pgfqpoint{0.973079in}{3.795980in}}%
\pgfusepath{stroke}%
\end{pgfscope}%
\begin{pgfscope}%
\pgfsetrectcap%
\pgfsetmiterjoin%
\pgfsetlinewidth{0.803000pt}%
\definecolor{currentstroke}{rgb}{0.000000,0.000000,0.000000}%
\pgfsetstrokecolor{currentstroke}%
\pgfsetdash{}{0pt}%
\pgfpathmoveto{\pgfqpoint{0.148611in}{4.039137in}}%
\pgfpathlineto{\pgfqpoint{0.973079in}{4.039137in}}%
\pgfusepath{stroke}%
\end{pgfscope}%
\begin{pgfscope}%
\definecolor{textcolor}{rgb}{0.000000,0.000000,0.000000}%
\pgfsetstrokecolor{textcolor}%
\pgfsetfillcolor{textcolor}%
\pgftext[x=0.560845in,y=4.122471in,,base]{\color{textcolor}\rmfamily\fontsize{11.000000}{13.200000}\selectfont Allianz}%
\end{pgfscope}%
\begin{pgfscope}%
\pgfsetbuttcap%
\pgfsetmiterjoin%
\definecolor{currentfill}{rgb}{1.000000,1.000000,1.000000}%
\pgfsetfillcolor{currentfill}%
\pgfsetlinewidth{0.000000pt}%
\definecolor{currentstroke}{rgb}{0.000000,0.000000,0.000000}%
\pgfsetstrokecolor{currentstroke}%
\pgfsetstrokeopacity{0.000000}%
\pgfsetdash{}{0pt}%
\pgfpathmoveto{\pgfqpoint{1.137973in}{3.795980in}}%
\pgfpathlineto{\pgfqpoint{1.962441in}{3.795980in}}%
\pgfpathlineto{\pgfqpoint{1.962441in}{4.039137in}}%
\pgfpathlineto{\pgfqpoint{1.137973in}{4.039137in}}%
\pgfpathlineto{\pgfqpoint{1.137973in}{3.795980in}}%
\pgfpathclose%
\pgfusepath{fill}%
\end{pgfscope}%
\begin{pgfscope}%
\pgfpathrectangle{\pgfqpoint{1.137973in}{3.795980in}}{\pgfqpoint{0.824468in}{0.243158in}}%
\pgfusepath{clip}%
\pgfsetbuttcap%
\pgfsetmiterjoin%
\definecolor{currentfill}{rgb}{0.121569,0.466667,0.705882}%
\pgfsetfillcolor{currentfill}%
\pgfsetfillopacity{0.500000}%
\pgfsetlinewidth{1.003750pt}%
\definecolor{currentstroke}{rgb}{0.000000,0.000000,0.000000}%
\pgfsetstrokecolor{currentstroke}%
\pgfsetdash{}{0pt}%
\pgfpathmoveto{\pgfqpoint{1.175449in}{3.795980in}}%
\pgfpathlineto{\pgfqpoint{1.325352in}{3.795980in}}%
\pgfpathlineto{\pgfqpoint{1.325352in}{3.802788in}}%
\pgfpathlineto{\pgfqpoint{1.175449in}{3.802788in}}%
\pgfpathlineto{\pgfqpoint{1.175449in}{3.795980in}}%
\pgfpathclose%
\pgfusepath{stroke,fill}%
\end{pgfscope}%
\begin{pgfscope}%
\pgfpathrectangle{\pgfqpoint{1.137973in}{3.795980in}}{\pgfqpoint{0.824468in}{0.243158in}}%
\pgfusepath{clip}%
\pgfsetbuttcap%
\pgfsetmiterjoin%
\definecolor{currentfill}{rgb}{0.121569,0.466667,0.705882}%
\pgfsetfillcolor{currentfill}%
\pgfsetfillopacity{0.500000}%
\pgfsetlinewidth{1.003750pt}%
\definecolor{currentstroke}{rgb}{0.000000,0.000000,0.000000}%
\pgfsetstrokecolor{currentstroke}%
\pgfsetdash{}{0pt}%
\pgfpathmoveto{\pgfqpoint{1.325352in}{3.795980in}}%
\pgfpathlineto{\pgfqpoint{1.475255in}{3.795980in}}%
\pgfpathlineto{\pgfqpoint{1.475255in}{3.804490in}}%
\pgfpathlineto{\pgfqpoint{1.325352in}{3.804490in}}%
\pgfpathlineto{\pgfqpoint{1.325352in}{3.795980in}}%
\pgfpathclose%
\pgfusepath{stroke,fill}%
\end{pgfscope}%
\begin{pgfscope}%
\pgfpathrectangle{\pgfqpoint{1.137973in}{3.795980in}}{\pgfqpoint{0.824468in}{0.243158in}}%
\pgfusepath{clip}%
\pgfsetbuttcap%
\pgfsetmiterjoin%
\definecolor{currentfill}{rgb}{0.121569,0.466667,0.705882}%
\pgfsetfillcolor{currentfill}%
\pgfsetfillopacity{0.500000}%
\pgfsetlinewidth{1.003750pt}%
\definecolor{currentstroke}{rgb}{0.000000,0.000000,0.000000}%
\pgfsetstrokecolor{currentstroke}%
\pgfsetdash{}{0pt}%
\pgfpathmoveto{\pgfqpoint{1.475255in}{3.795980in}}%
\pgfpathlineto{\pgfqpoint{1.625158in}{3.795980in}}%
\pgfpathlineto{\pgfqpoint{1.625158in}{3.814460in}}%
\pgfpathlineto{\pgfqpoint{1.475255in}{3.814460in}}%
\pgfpathlineto{\pgfqpoint{1.475255in}{3.795980in}}%
\pgfpathclose%
\pgfusepath{stroke,fill}%
\end{pgfscope}%
\begin{pgfscope}%
\pgfpathrectangle{\pgfqpoint{1.137973in}{3.795980in}}{\pgfqpoint{0.824468in}{0.243158in}}%
\pgfusepath{clip}%
\pgfsetbuttcap%
\pgfsetmiterjoin%
\definecolor{currentfill}{rgb}{0.121569,0.466667,0.705882}%
\pgfsetfillcolor{currentfill}%
\pgfsetfillopacity{0.500000}%
\pgfsetlinewidth{1.003750pt}%
\definecolor{currentstroke}{rgb}{0.000000,0.000000,0.000000}%
\pgfsetstrokecolor{currentstroke}%
\pgfsetdash{}{0pt}%
\pgfpathmoveto{\pgfqpoint{1.625158in}{3.795980in}}%
\pgfpathlineto{\pgfqpoint{1.775062in}{3.795980in}}%
\pgfpathlineto{\pgfqpoint{1.775062in}{3.837316in}}%
\pgfpathlineto{\pgfqpoint{1.625158in}{3.837316in}}%
\pgfpathlineto{\pgfqpoint{1.625158in}{3.795980in}}%
\pgfpathclose%
\pgfusepath{stroke,fill}%
\end{pgfscope}%
\begin{pgfscope}%
\pgfpathrectangle{\pgfqpoint{1.137973in}{3.795980in}}{\pgfqpoint{0.824468in}{0.243158in}}%
\pgfusepath{clip}%
\pgfsetbuttcap%
\pgfsetmiterjoin%
\definecolor{currentfill}{rgb}{0.121569,0.466667,0.705882}%
\pgfsetfillcolor{currentfill}%
\pgfsetfillopacity{0.500000}%
\pgfsetlinewidth{1.003750pt}%
\definecolor{currentstroke}{rgb}{0.000000,0.000000,0.000000}%
\pgfsetstrokecolor{currentstroke}%
\pgfsetdash{}{0pt}%
\pgfpathmoveto{\pgfqpoint{1.775062in}{3.795980in}}%
\pgfpathlineto{\pgfqpoint{1.924965in}{3.795980in}}%
\pgfpathlineto{\pgfqpoint{1.924965in}{3.845219in}}%
\pgfpathlineto{\pgfqpoint{1.775062in}{3.845219in}}%
\pgfpathlineto{\pgfqpoint{1.775062in}{3.795980in}}%
\pgfpathclose%
\pgfusepath{stroke,fill}%
\end{pgfscope}%
\begin{pgfscope}%
\pgfsetrectcap%
\pgfsetmiterjoin%
\pgfsetlinewidth{0.803000pt}%
\definecolor{currentstroke}{rgb}{0.000000,0.000000,0.000000}%
\pgfsetstrokecolor{currentstroke}%
\pgfsetdash{}{0pt}%
\pgfpathmoveto{\pgfqpoint{1.137973in}{3.795980in}}%
\pgfpathlineto{\pgfqpoint{1.137973in}{4.039137in}}%
\pgfusepath{stroke}%
\end{pgfscope}%
\begin{pgfscope}%
\pgfsetrectcap%
\pgfsetmiterjoin%
\pgfsetlinewidth{0.803000pt}%
\definecolor{currentstroke}{rgb}{0.000000,0.000000,0.000000}%
\pgfsetstrokecolor{currentstroke}%
\pgfsetdash{}{0pt}%
\pgfpathmoveto{\pgfqpoint{1.962441in}{3.795980in}}%
\pgfpathlineto{\pgfqpoint{1.962441in}{4.039137in}}%
\pgfusepath{stroke}%
\end{pgfscope}%
\begin{pgfscope}%
\pgfsetrectcap%
\pgfsetmiterjoin%
\pgfsetlinewidth{0.803000pt}%
\definecolor{currentstroke}{rgb}{0.000000,0.000000,0.000000}%
\pgfsetstrokecolor{currentstroke}%
\pgfsetdash{}{0pt}%
\pgfpathmoveto{\pgfqpoint{1.137973in}{3.795980in}}%
\pgfpathlineto{\pgfqpoint{1.962441in}{3.795980in}}%
\pgfusepath{stroke}%
\end{pgfscope}%
\begin{pgfscope}%
\pgfsetrectcap%
\pgfsetmiterjoin%
\pgfsetlinewidth{0.803000pt}%
\definecolor{currentstroke}{rgb}{0.000000,0.000000,0.000000}%
\pgfsetstrokecolor{currentstroke}%
\pgfsetdash{}{0pt}%
\pgfpathmoveto{\pgfqpoint{1.137973in}{4.039137in}}%
\pgfpathlineto{\pgfqpoint{1.962441in}{4.039137in}}%
\pgfusepath{stroke}%
\end{pgfscope}%
\begin{pgfscope}%
\definecolor{textcolor}{rgb}{0.000000,0.000000,0.000000}%
\pgfsetstrokecolor{textcolor}%
\pgfsetfillcolor{textcolor}%
\pgftext[x=1.550207in,y=4.122471in,,base]{\color{textcolor}\rmfamily\fontsize{11.000000}{13.200000}\selectfont APRIL ...}%
\end{pgfscope}%
\begin{pgfscope}%
\pgfsetbuttcap%
\pgfsetmiterjoin%
\definecolor{currentfill}{rgb}{1.000000,1.000000,1.000000}%
\pgfsetfillcolor{currentfill}%
\pgfsetlinewidth{0.000000pt}%
\definecolor{currentstroke}{rgb}{0.000000,0.000000,0.000000}%
\pgfsetstrokecolor{currentstroke}%
\pgfsetstrokeopacity{0.000000}%
\pgfsetdash{}{0pt}%
\pgfpathmoveto{\pgfqpoint{2.127335in}{3.795980in}}%
\pgfpathlineto{\pgfqpoint{2.951803in}{3.795980in}}%
\pgfpathlineto{\pgfqpoint{2.951803in}{4.039137in}}%
\pgfpathlineto{\pgfqpoint{2.127335in}{4.039137in}}%
\pgfpathlineto{\pgfqpoint{2.127335in}{3.795980in}}%
\pgfpathclose%
\pgfusepath{fill}%
\end{pgfscope}%
\begin{pgfscope}%
\pgfpathrectangle{\pgfqpoint{2.127335in}{3.795980in}}{\pgfqpoint{0.824468in}{0.243158in}}%
\pgfusepath{clip}%
\pgfsetbuttcap%
\pgfsetmiterjoin%
\definecolor{currentfill}{rgb}{0.121569,0.466667,0.705882}%
\pgfsetfillcolor{currentfill}%
\pgfsetfillopacity{0.500000}%
\pgfsetlinewidth{1.003750pt}%
\definecolor{currentstroke}{rgb}{0.000000,0.000000,0.000000}%
\pgfsetstrokecolor{currentstroke}%
\pgfsetdash{}{0pt}%
\pgfpathmoveto{\pgfqpoint{2.164810in}{3.795980in}}%
\pgfpathlineto{\pgfqpoint{2.314714in}{3.795980in}}%
\pgfpathlineto{\pgfqpoint{2.314714in}{3.814095in}}%
\pgfpathlineto{\pgfqpoint{2.164810in}{3.814095in}}%
\pgfpathlineto{\pgfqpoint{2.164810in}{3.795980in}}%
\pgfpathclose%
\pgfusepath{stroke,fill}%
\end{pgfscope}%
\begin{pgfscope}%
\pgfpathrectangle{\pgfqpoint{2.127335in}{3.795980in}}{\pgfqpoint{0.824468in}{0.243158in}}%
\pgfusepath{clip}%
\pgfsetbuttcap%
\pgfsetmiterjoin%
\definecolor{currentfill}{rgb}{0.121569,0.466667,0.705882}%
\pgfsetfillcolor{currentfill}%
\pgfsetfillopacity{0.500000}%
\pgfsetlinewidth{1.003750pt}%
\definecolor{currentstroke}{rgb}{0.000000,0.000000,0.000000}%
\pgfsetstrokecolor{currentstroke}%
\pgfsetdash{}{0pt}%
\pgfpathmoveto{\pgfqpoint{2.314714in}{3.795980in}}%
\pgfpathlineto{\pgfqpoint{2.464617in}{3.795980in}}%
\pgfpathlineto{\pgfqpoint{2.464617in}{3.801329in}}%
\pgfpathlineto{\pgfqpoint{2.314714in}{3.801329in}}%
\pgfpathlineto{\pgfqpoint{2.314714in}{3.795980in}}%
\pgfpathclose%
\pgfusepath{stroke,fill}%
\end{pgfscope}%
\begin{pgfscope}%
\pgfpathrectangle{\pgfqpoint{2.127335in}{3.795980in}}{\pgfqpoint{0.824468in}{0.243158in}}%
\pgfusepath{clip}%
\pgfsetbuttcap%
\pgfsetmiterjoin%
\definecolor{currentfill}{rgb}{0.121569,0.466667,0.705882}%
\pgfsetfillcolor{currentfill}%
\pgfsetfillopacity{0.500000}%
\pgfsetlinewidth{1.003750pt}%
\definecolor{currentstroke}{rgb}{0.000000,0.000000,0.000000}%
\pgfsetstrokecolor{currentstroke}%
\pgfsetdash{}{0pt}%
\pgfpathmoveto{\pgfqpoint{2.464617in}{3.795980in}}%
\pgfpathlineto{\pgfqpoint{2.614520in}{3.795980in}}%
\pgfpathlineto{\pgfqpoint{2.614520in}{3.796831in}}%
\pgfpathlineto{\pgfqpoint{2.464617in}{3.796831in}}%
\pgfpathlineto{\pgfqpoint{2.464617in}{3.795980in}}%
\pgfpathclose%
\pgfusepath{stroke,fill}%
\end{pgfscope}%
\begin{pgfscope}%
\pgfpathrectangle{\pgfqpoint{2.127335in}{3.795980in}}{\pgfqpoint{0.824468in}{0.243158in}}%
\pgfusepath{clip}%
\pgfsetbuttcap%
\pgfsetmiterjoin%
\definecolor{currentfill}{rgb}{0.121569,0.466667,0.705882}%
\pgfsetfillcolor{currentfill}%
\pgfsetfillopacity{0.500000}%
\pgfsetlinewidth{1.003750pt}%
\definecolor{currentstroke}{rgb}{0.000000,0.000000,0.000000}%
\pgfsetstrokecolor{currentstroke}%
\pgfsetdash{}{0pt}%
\pgfpathmoveto{\pgfqpoint{2.614520in}{3.795980in}}%
\pgfpathlineto{\pgfqpoint{2.764423in}{3.795980in}}%
\pgfpathlineto{\pgfqpoint{2.764423in}{3.796952in}}%
\pgfpathlineto{\pgfqpoint{2.614520in}{3.796952in}}%
\pgfpathlineto{\pgfqpoint{2.614520in}{3.795980in}}%
\pgfpathclose%
\pgfusepath{stroke,fill}%
\end{pgfscope}%
\begin{pgfscope}%
\pgfpathrectangle{\pgfqpoint{2.127335in}{3.795980in}}{\pgfqpoint{0.824468in}{0.243158in}}%
\pgfusepath{clip}%
\pgfsetbuttcap%
\pgfsetmiterjoin%
\definecolor{currentfill}{rgb}{0.121569,0.466667,0.705882}%
\pgfsetfillcolor{currentfill}%
\pgfsetfillopacity{0.500000}%
\pgfsetlinewidth{1.003750pt}%
\definecolor{currentstroke}{rgb}{0.000000,0.000000,0.000000}%
\pgfsetstrokecolor{currentstroke}%
\pgfsetdash{}{0pt}%
\pgfpathmoveto{\pgfqpoint{2.764423in}{3.795980in}}%
\pgfpathlineto{\pgfqpoint{2.914327in}{3.795980in}}%
\pgfpathlineto{\pgfqpoint{2.914327in}{3.796709in}}%
\pgfpathlineto{\pgfqpoint{2.764423in}{3.796709in}}%
\pgfpathlineto{\pgfqpoint{2.764423in}{3.795980in}}%
\pgfpathclose%
\pgfusepath{stroke,fill}%
\end{pgfscope}%
\begin{pgfscope}%
\pgfsetrectcap%
\pgfsetmiterjoin%
\pgfsetlinewidth{0.803000pt}%
\definecolor{currentstroke}{rgb}{0.000000,0.000000,0.000000}%
\pgfsetstrokecolor{currentstroke}%
\pgfsetdash{}{0pt}%
\pgfpathmoveto{\pgfqpoint{2.127335in}{3.795980in}}%
\pgfpathlineto{\pgfqpoint{2.127335in}{4.039137in}}%
\pgfusepath{stroke}%
\end{pgfscope}%
\begin{pgfscope}%
\pgfsetrectcap%
\pgfsetmiterjoin%
\pgfsetlinewidth{0.803000pt}%
\definecolor{currentstroke}{rgb}{0.000000,0.000000,0.000000}%
\pgfsetstrokecolor{currentstroke}%
\pgfsetdash{}{0pt}%
\pgfpathmoveto{\pgfqpoint{2.951803in}{3.795980in}}%
\pgfpathlineto{\pgfqpoint{2.951803in}{4.039137in}}%
\pgfusepath{stroke}%
\end{pgfscope}%
\begin{pgfscope}%
\pgfsetrectcap%
\pgfsetmiterjoin%
\pgfsetlinewidth{0.803000pt}%
\definecolor{currentstroke}{rgb}{0.000000,0.000000,0.000000}%
\pgfsetstrokecolor{currentstroke}%
\pgfsetdash{}{0pt}%
\pgfpathmoveto{\pgfqpoint{2.127335in}{3.795980in}}%
\pgfpathlineto{\pgfqpoint{2.951803in}{3.795980in}}%
\pgfusepath{stroke}%
\end{pgfscope}%
\begin{pgfscope}%
\pgfsetrectcap%
\pgfsetmiterjoin%
\pgfsetlinewidth{0.803000pt}%
\definecolor{currentstroke}{rgb}{0.000000,0.000000,0.000000}%
\pgfsetstrokecolor{currentstroke}%
\pgfsetdash{}{0pt}%
\pgfpathmoveto{\pgfqpoint{2.127335in}{4.039137in}}%
\pgfpathlineto{\pgfqpoint{2.951803in}{4.039137in}}%
\pgfusepath{stroke}%
\end{pgfscope}%
\begin{pgfscope}%
\definecolor{textcolor}{rgb}{0.000000,0.000000,0.000000}%
\pgfsetstrokecolor{textcolor}%
\pgfsetfillcolor{textcolor}%
\pgftext[x=2.539569in,y=4.122471in,,base]{\color{textcolor}\rmfamily\fontsize{11.000000}{13.200000}\selectfont Cegema...}%
\end{pgfscope}%
\begin{pgfscope}%
\pgfsetbuttcap%
\pgfsetmiterjoin%
\definecolor{currentfill}{rgb}{1.000000,1.000000,1.000000}%
\pgfsetfillcolor{currentfill}%
\pgfsetlinewidth{0.000000pt}%
\definecolor{currentstroke}{rgb}{0.000000,0.000000,0.000000}%
\pgfsetstrokecolor{currentstroke}%
\pgfsetstrokeopacity{0.000000}%
\pgfsetdash{}{0pt}%
\pgfpathmoveto{\pgfqpoint{3.116696in}{3.795980in}}%
\pgfpathlineto{\pgfqpoint{3.941164in}{3.795980in}}%
\pgfpathlineto{\pgfqpoint{3.941164in}{4.039137in}}%
\pgfpathlineto{\pgfqpoint{3.116696in}{4.039137in}}%
\pgfpathlineto{\pgfqpoint{3.116696in}{3.795980in}}%
\pgfpathclose%
\pgfusepath{fill}%
\end{pgfscope}%
\begin{pgfscope}%
\pgfpathrectangle{\pgfqpoint{3.116696in}{3.795980in}}{\pgfqpoint{0.824468in}{0.243158in}}%
\pgfusepath{clip}%
\pgfsetbuttcap%
\pgfsetmiterjoin%
\definecolor{currentfill}{rgb}{0.121569,0.466667,0.705882}%
\pgfsetfillcolor{currentfill}%
\pgfsetfillopacity{0.500000}%
\pgfsetlinewidth{1.003750pt}%
\definecolor{currentstroke}{rgb}{0.000000,0.000000,0.000000}%
\pgfsetstrokecolor{currentstroke}%
\pgfsetdash{}{0pt}%
\pgfpathmoveto{\pgfqpoint{3.154172in}{3.795980in}}%
\pgfpathlineto{\pgfqpoint{3.304075in}{3.795980in}}%
\pgfpathlineto{\pgfqpoint{3.304075in}{3.797803in}}%
\pgfpathlineto{\pgfqpoint{3.154172in}{3.797803in}}%
\pgfpathlineto{\pgfqpoint{3.154172in}{3.795980in}}%
\pgfpathclose%
\pgfusepath{stroke,fill}%
\end{pgfscope}%
\begin{pgfscope}%
\pgfpathrectangle{\pgfqpoint{3.116696in}{3.795980in}}{\pgfqpoint{0.824468in}{0.243158in}}%
\pgfusepath{clip}%
\pgfsetbuttcap%
\pgfsetmiterjoin%
\definecolor{currentfill}{rgb}{0.121569,0.466667,0.705882}%
\pgfsetfillcolor{currentfill}%
\pgfsetfillopacity{0.500000}%
\pgfsetlinewidth{1.003750pt}%
\definecolor{currentstroke}{rgb}{0.000000,0.000000,0.000000}%
\pgfsetstrokecolor{currentstroke}%
\pgfsetdash{}{0pt}%
\pgfpathmoveto{\pgfqpoint{3.304075in}{3.795980in}}%
\pgfpathlineto{\pgfqpoint{3.453979in}{3.795980in}}%
\pgfpathlineto{\pgfqpoint{3.453979in}{3.796101in}}%
\pgfpathlineto{\pgfqpoint{3.304075in}{3.796101in}}%
\pgfpathlineto{\pgfqpoint{3.304075in}{3.795980in}}%
\pgfpathclose%
\pgfusepath{stroke,fill}%
\end{pgfscope}%
\begin{pgfscope}%
\pgfpathrectangle{\pgfqpoint{3.116696in}{3.795980in}}{\pgfqpoint{0.824468in}{0.243158in}}%
\pgfusepath{clip}%
\pgfsetbuttcap%
\pgfsetmiterjoin%
\definecolor{currentfill}{rgb}{0.121569,0.466667,0.705882}%
\pgfsetfillcolor{currentfill}%
\pgfsetfillopacity{0.500000}%
\pgfsetlinewidth{1.003750pt}%
\definecolor{currentstroke}{rgb}{0.000000,0.000000,0.000000}%
\pgfsetstrokecolor{currentstroke}%
\pgfsetdash{}{0pt}%
\pgfpathmoveto{\pgfqpoint{3.453979in}{3.795980in}}%
\pgfpathlineto{\pgfqpoint{3.603882in}{3.795980in}}%
\pgfpathlineto{\pgfqpoint{3.603882in}{3.796223in}}%
\pgfpathlineto{\pgfqpoint{3.453979in}{3.796223in}}%
\pgfpathlineto{\pgfqpoint{3.453979in}{3.795980in}}%
\pgfpathclose%
\pgfusepath{stroke,fill}%
\end{pgfscope}%
\begin{pgfscope}%
\pgfpathrectangle{\pgfqpoint{3.116696in}{3.795980in}}{\pgfqpoint{0.824468in}{0.243158in}}%
\pgfusepath{clip}%
\pgfsetbuttcap%
\pgfsetmiterjoin%
\definecolor{currentfill}{rgb}{0.121569,0.466667,0.705882}%
\pgfsetfillcolor{currentfill}%
\pgfsetfillopacity{0.500000}%
\pgfsetlinewidth{1.003750pt}%
\definecolor{currentstroke}{rgb}{0.000000,0.000000,0.000000}%
\pgfsetstrokecolor{currentstroke}%
\pgfsetdash{}{0pt}%
\pgfpathmoveto{\pgfqpoint{3.603882in}{3.795980in}}%
\pgfpathlineto{\pgfqpoint{3.753785in}{3.795980in}}%
\pgfpathlineto{\pgfqpoint{3.753785in}{3.795980in}}%
\pgfpathlineto{\pgfqpoint{3.603882in}{3.795980in}}%
\pgfpathlineto{\pgfqpoint{3.603882in}{3.795980in}}%
\pgfpathclose%
\pgfusepath{stroke,fill}%
\end{pgfscope}%
\begin{pgfscope}%
\pgfpathrectangle{\pgfqpoint{3.116696in}{3.795980in}}{\pgfqpoint{0.824468in}{0.243158in}}%
\pgfusepath{clip}%
\pgfsetbuttcap%
\pgfsetmiterjoin%
\definecolor{currentfill}{rgb}{0.121569,0.466667,0.705882}%
\pgfsetfillcolor{currentfill}%
\pgfsetfillopacity{0.500000}%
\pgfsetlinewidth{1.003750pt}%
\definecolor{currentstroke}{rgb}{0.000000,0.000000,0.000000}%
\pgfsetstrokecolor{currentstroke}%
\pgfsetdash{}{0pt}%
\pgfpathmoveto{\pgfqpoint{3.753785in}{3.795980in}}%
\pgfpathlineto{\pgfqpoint{3.903688in}{3.795980in}}%
\pgfpathlineto{\pgfqpoint{3.903688in}{3.795980in}}%
\pgfpathlineto{\pgfqpoint{3.753785in}{3.795980in}}%
\pgfpathlineto{\pgfqpoint{3.753785in}{3.795980in}}%
\pgfpathclose%
\pgfusepath{stroke,fill}%
\end{pgfscope}%
\begin{pgfscope}%
\pgfsetrectcap%
\pgfsetmiterjoin%
\pgfsetlinewidth{0.803000pt}%
\definecolor{currentstroke}{rgb}{0.000000,0.000000,0.000000}%
\pgfsetstrokecolor{currentstroke}%
\pgfsetdash{}{0pt}%
\pgfpathmoveto{\pgfqpoint{3.116696in}{3.795980in}}%
\pgfpathlineto{\pgfqpoint{3.116696in}{4.039137in}}%
\pgfusepath{stroke}%
\end{pgfscope}%
\begin{pgfscope}%
\pgfsetrectcap%
\pgfsetmiterjoin%
\pgfsetlinewidth{0.803000pt}%
\definecolor{currentstroke}{rgb}{0.000000,0.000000,0.000000}%
\pgfsetstrokecolor{currentstroke}%
\pgfsetdash{}{0pt}%
\pgfpathmoveto{\pgfqpoint{3.941164in}{3.795980in}}%
\pgfpathlineto{\pgfqpoint{3.941164in}{4.039137in}}%
\pgfusepath{stroke}%
\end{pgfscope}%
\begin{pgfscope}%
\pgfsetrectcap%
\pgfsetmiterjoin%
\pgfsetlinewidth{0.803000pt}%
\definecolor{currentstroke}{rgb}{0.000000,0.000000,0.000000}%
\pgfsetstrokecolor{currentstroke}%
\pgfsetdash{}{0pt}%
\pgfpathmoveto{\pgfqpoint{3.116696in}{3.795980in}}%
\pgfpathlineto{\pgfqpoint{3.941164in}{3.795980in}}%
\pgfusepath{stroke}%
\end{pgfscope}%
\begin{pgfscope}%
\pgfsetrectcap%
\pgfsetmiterjoin%
\pgfsetlinewidth{0.803000pt}%
\definecolor{currentstroke}{rgb}{0.000000,0.000000,0.000000}%
\pgfsetstrokecolor{currentstroke}%
\pgfsetdash{}{0pt}%
\pgfpathmoveto{\pgfqpoint{3.116696in}{4.039137in}}%
\pgfpathlineto{\pgfqpoint{3.941164in}{4.039137in}}%
\pgfusepath{stroke}%
\end{pgfscope}%
\begin{pgfscope}%
\definecolor{textcolor}{rgb}{0.000000,0.000000,0.000000}%
\pgfsetstrokecolor{textcolor}%
\pgfsetfillcolor{textcolor}%
\pgftext[x=3.528930in,y=4.122471in,,base]{\color{textcolor}\rmfamily\fontsize{11.000000}{13.200000}\selectfont LCL}%
\end{pgfscope}%
\begin{pgfscope}%
\pgfsetbuttcap%
\pgfsetmiterjoin%
\definecolor{currentfill}{rgb}{1.000000,1.000000,1.000000}%
\pgfsetfillcolor{currentfill}%
\pgfsetlinewidth{0.000000pt}%
\definecolor{currentstroke}{rgb}{0.000000,0.000000,0.000000}%
\pgfsetstrokecolor{currentstroke}%
\pgfsetstrokeopacity{0.000000}%
\pgfsetdash{}{0pt}%
\pgfpathmoveto{\pgfqpoint{4.106058in}{3.795980in}}%
\pgfpathlineto{\pgfqpoint{4.930526in}{3.795980in}}%
\pgfpathlineto{\pgfqpoint{4.930526in}{4.039137in}}%
\pgfpathlineto{\pgfqpoint{4.106058in}{4.039137in}}%
\pgfpathlineto{\pgfqpoint{4.106058in}{3.795980in}}%
\pgfpathclose%
\pgfusepath{fill}%
\end{pgfscope}%
\begin{pgfscope}%
\pgfpathrectangle{\pgfqpoint{4.106058in}{3.795980in}}{\pgfqpoint{0.824468in}{0.243158in}}%
\pgfusepath{clip}%
\pgfsetbuttcap%
\pgfsetmiterjoin%
\definecolor{currentfill}{rgb}{0.121569,0.466667,0.705882}%
\pgfsetfillcolor{currentfill}%
\pgfsetfillopacity{0.500000}%
\pgfsetlinewidth{1.003750pt}%
\definecolor{currentstroke}{rgb}{0.000000,0.000000,0.000000}%
\pgfsetstrokecolor{currentstroke}%
\pgfsetdash{}{0pt}%
\pgfpathmoveto{\pgfqpoint{4.143534in}{3.795980in}}%
\pgfpathlineto{\pgfqpoint{4.293437in}{3.795980in}}%
\pgfpathlineto{\pgfqpoint{4.293437in}{3.804976in}}%
\pgfpathlineto{\pgfqpoint{4.143534in}{3.804976in}}%
\pgfpathlineto{\pgfqpoint{4.143534in}{3.795980in}}%
\pgfpathclose%
\pgfusepath{stroke,fill}%
\end{pgfscope}%
\begin{pgfscope}%
\pgfpathrectangle{\pgfqpoint{4.106058in}{3.795980in}}{\pgfqpoint{0.824468in}{0.243158in}}%
\pgfusepath{clip}%
\pgfsetbuttcap%
\pgfsetmiterjoin%
\definecolor{currentfill}{rgb}{0.121569,0.466667,0.705882}%
\pgfsetfillcolor{currentfill}%
\pgfsetfillopacity{0.500000}%
\pgfsetlinewidth{1.003750pt}%
\definecolor{currentstroke}{rgb}{0.000000,0.000000,0.000000}%
\pgfsetstrokecolor{currentstroke}%
\pgfsetdash{}{0pt}%
\pgfpathmoveto{\pgfqpoint{4.293437in}{3.795980in}}%
\pgfpathlineto{\pgfqpoint{4.443340in}{3.795980in}}%
\pgfpathlineto{\pgfqpoint{4.443340in}{3.801572in}}%
\pgfpathlineto{\pgfqpoint{4.293437in}{3.801572in}}%
\pgfpathlineto{\pgfqpoint{4.293437in}{3.795980in}}%
\pgfpathclose%
\pgfusepath{stroke,fill}%
\end{pgfscope}%
\begin{pgfscope}%
\pgfpathrectangle{\pgfqpoint{4.106058in}{3.795980in}}{\pgfqpoint{0.824468in}{0.243158in}}%
\pgfusepath{clip}%
\pgfsetbuttcap%
\pgfsetmiterjoin%
\definecolor{currentfill}{rgb}{0.121569,0.466667,0.705882}%
\pgfsetfillcolor{currentfill}%
\pgfsetfillopacity{0.500000}%
\pgfsetlinewidth{1.003750pt}%
\definecolor{currentstroke}{rgb}{0.000000,0.000000,0.000000}%
\pgfsetstrokecolor{currentstroke}%
\pgfsetdash{}{0pt}%
\pgfpathmoveto{\pgfqpoint{4.443340in}{3.795980in}}%
\pgfpathlineto{\pgfqpoint{4.593244in}{3.795980in}}%
\pgfpathlineto{\pgfqpoint{4.593244in}{3.797438in}}%
\pgfpathlineto{\pgfqpoint{4.443340in}{3.797438in}}%
\pgfpathlineto{\pgfqpoint{4.443340in}{3.795980in}}%
\pgfpathclose%
\pgfusepath{stroke,fill}%
\end{pgfscope}%
\begin{pgfscope}%
\pgfpathrectangle{\pgfqpoint{4.106058in}{3.795980in}}{\pgfqpoint{0.824468in}{0.243158in}}%
\pgfusepath{clip}%
\pgfsetbuttcap%
\pgfsetmiterjoin%
\definecolor{currentfill}{rgb}{0.121569,0.466667,0.705882}%
\pgfsetfillcolor{currentfill}%
\pgfsetfillopacity{0.500000}%
\pgfsetlinewidth{1.003750pt}%
\definecolor{currentstroke}{rgb}{0.000000,0.000000,0.000000}%
\pgfsetstrokecolor{currentstroke}%
\pgfsetdash{}{0pt}%
\pgfpathmoveto{\pgfqpoint{4.593244in}{3.795980in}}%
\pgfpathlineto{\pgfqpoint{4.743147in}{3.795980in}}%
\pgfpathlineto{\pgfqpoint{4.743147in}{3.796952in}}%
\pgfpathlineto{\pgfqpoint{4.593244in}{3.796952in}}%
\pgfpathlineto{\pgfqpoint{4.593244in}{3.795980in}}%
\pgfpathclose%
\pgfusepath{stroke,fill}%
\end{pgfscope}%
\begin{pgfscope}%
\pgfpathrectangle{\pgfqpoint{4.106058in}{3.795980in}}{\pgfqpoint{0.824468in}{0.243158in}}%
\pgfusepath{clip}%
\pgfsetbuttcap%
\pgfsetmiterjoin%
\definecolor{currentfill}{rgb}{0.121569,0.466667,0.705882}%
\pgfsetfillcolor{currentfill}%
\pgfsetfillopacity{0.500000}%
\pgfsetlinewidth{1.003750pt}%
\definecolor{currentstroke}{rgb}{0.000000,0.000000,0.000000}%
\pgfsetstrokecolor{currentstroke}%
\pgfsetdash{}{0pt}%
\pgfpathmoveto{\pgfqpoint{4.743147in}{3.795980in}}%
\pgfpathlineto{\pgfqpoint{4.893050in}{3.795980in}}%
\pgfpathlineto{\pgfqpoint{4.893050in}{3.796466in}}%
\pgfpathlineto{\pgfqpoint{4.743147in}{3.796466in}}%
\pgfpathlineto{\pgfqpoint{4.743147in}{3.795980in}}%
\pgfpathclose%
\pgfusepath{stroke,fill}%
\end{pgfscope}%
\begin{pgfscope}%
\pgfsetrectcap%
\pgfsetmiterjoin%
\pgfsetlinewidth{0.803000pt}%
\definecolor{currentstroke}{rgb}{0.000000,0.000000,0.000000}%
\pgfsetstrokecolor{currentstroke}%
\pgfsetdash{}{0pt}%
\pgfpathmoveto{\pgfqpoint{4.106058in}{3.795980in}}%
\pgfpathlineto{\pgfqpoint{4.106058in}{4.039137in}}%
\pgfusepath{stroke}%
\end{pgfscope}%
\begin{pgfscope}%
\pgfsetrectcap%
\pgfsetmiterjoin%
\pgfsetlinewidth{0.803000pt}%
\definecolor{currentstroke}{rgb}{0.000000,0.000000,0.000000}%
\pgfsetstrokecolor{currentstroke}%
\pgfsetdash{}{0pt}%
\pgfpathmoveto{\pgfqpoint{4.930526in}{3.795980in}}%
\pgfpathlineto{\pgfqpoint{4.930526in}{4.039137in}}%
\pgfusepath{stroke}%
\end{pgfscope}%
\begin{pgfscope}%
\pgfsetrectcap%
\pgfsetmiterjoin%
\pgfsetlinewidth{0.803000pt}%
\definecolor{currentstroke}{rgb}{0.000000,0.000000,0.000000}%
\pgfsetstrokecolor{currentstroke}%
\pgfsetdash{}{0pt}%
\pgfpathmoveto{\pgfqpoint{4.106058in}{3.795980in}}%
\pgfpathlineto{\pgfqpoint{4.930526in}{3.795980in}}%
\pgfusepath{stroke}%
\end{pgfscope}%
\begin{pgfscope}%
\pgfsetrectcap%
\pgfsetmiterjoin%
\pgfsetlinewidth{0.803000pt}%
\definecolor{currentstroke}{rgb}{0.000000,0.000000,0.000000}%
\pgfsetstrokecolor{currentstroke}%
\pgfsetdash{}{0pt}%
\pgfpathmoveto{\pgfqpoint{4.106058in}{4.039137in}}%
\pgfpathlineto{\pgfqpoint{4.930526in}{4.039137in}}%
\pgfusepath{stroke}%
\end{pgfscope}%
\begin{pgfscope}%
\definecolor{textcolor}{rgb}{0.000000,0.000000,0.000000}%
\pgfsetstrokecolor{textcolor}%
\pgfsetfillcolor{textcolor}%
\pgftext[x=4.518292in,y=4.122471in,,base]{\color{textcolor}\rmfamily\fontsize{11.000000}{13.200000}\selectfont Afer}%
\end{pgfscope}%
\begin{pgfscope}%
\pgfsetbuttcap%
\pgfsetmiterjoin%
\definecolor{currentfill}{rgb}{1.000000,1.000000,1.000000}%
\pgfsetfillcolor{currentfill}%
\pgfsetlinewidth{0.000000pt}%
\definecolor{currentstroke}{rgb}{0.000000,0.000000,0.000000}%
\pgfsetstrokecolor{currentstroke}%
\pgfsetstrokeopacity{0.000000}%
\pgfsetdash{}{0pt}%
\pgfpathmoveto{\pgfqpoint{5.095420in}{3.795980in}}%
\pgfpathlineto{\pgfqpoint{5.919888in}{3.795980in}}%
\pgfpathlineto{\pgfqpoint{5.919888in}{4.039137in}}%
\pgfpathlineto{\pgfqpoint{5.095420in}{4.039137in}}%
\pgfpathlineto{\pgfqpoint{5.095420in}{3.795980in}}%
\pgfpathclose%
\pgfusepath{fill}%
\end{pgfscope}%
\begin{pgfscope}%
\pgfpathrectangle{\pgfqpoint{5.095420in}{3.795980in}}{\pgfqpoint{0.824468in}{0.243158in}}%
\pgfusepath{clip}%
\pgfsetbuttcap%
\pgfsetmiterjoin%
\definecolor{currentfill}{rgb}{0.121569,0.466667,0.705882}%
\pgfsetfillcolor{currentfill}%
\pgfsetfillopacity{0.500000}%
\pgfsetlinewidth{1.003750pt}%
\definecolor{currentstroke}{rgb}{0.000000,0.000000,0.000000}%
\pgfsetstrokecolor{currentstroke}%
\pgfsetdash{}{0pt}%
\pgfpathmoveto{\pgfqpoint{5.132895in}{3.795980in}}%
\pgfpathlineto{\pgfqpoint{5.282799in}{3.795980in}}%
\pgfpathlineto{\pgfqpoint{5.282799in}{3.831116in}}%
\pgfpathlineto{\pgfqpoint{5.132895in}{3.831116in}}%
\pgfpathlineto{\pgfqpoint{5.132895in}{3.795980in}}%
\pgfpathclose%
\pgfusepath{stroke,fill}%
\end{pgfscope}%
\begin{pgfscope}%
\pgfpathrectangle{\pgfqpoint{5.095420in}{3.795980in}}{\pgfqpoint{0.824468in}{0.243158in}}%
\pgfusepath{clip}%
\pgfsetbuttcap%
\pgfsetmiterjoin%
\definecolor{currentfill}{rgb}{0.121569,0.466667,0.705882}%
\pgfsetfillcolor{currentfill}%
\pgfsetfillopacity{0.500000}%
\pgfsetlinewidth{1.003750pt}%
\definecolor{currentstroke}{rgb}{0.000000,0.000000,0.000000}%
\pgfsetstrokecolor{currentstroke}%
\pgfsetdash{}{0pt}%
\pgfpathmoveto{\pgfqpoint{5.282799in}{3.795980in}}%
\pgfpathlineto{\pgfqpoint{5.432702in}{3.795980in}}%
\pgfpathlineto{\pgfqpoint{5.432702in}{3.810812in}}%
\pgfpathlineto{\pgfqpoint{5.282799in}{3.810812in}}%
\pgfpathlineto{\pgfqpoint{5.282799in}{3.795980in}}%
\pgfpathclose%
\pgfusepath{stroke,fill}%
\end{pgfscope}%
\begin{pgfscope}%
\pgfpathrectangle{\pgfqpoint{5.095420in}{3.795980in}}{\pgfqpoint{0.824468in}{0.243158in}}%
\pgfusepath{clip}%
\pgfsetbuttcap%
\pgfsetmiterjoin%
\definecolor{currentfill}{rgb}{0.121569,0.466667,0.705882}%
\pgfsetfillcolor{currentfill}%
\pgfsetfillopacity{0.500000}%
\pgfsetlinewidth{1.003750pt}%
\definecolor{currentstroke}{rgb}{0.000000,0.000000,0.000000}%
\pgfsetstrokecolor{currentstroke}%
\pgfsetdash{}{0pt}%
\pgfpathmoveto{\pgfqpoint{5.432702in}{3.795980in}}%
\pgfpathlineto{\pgfqpoint{5.582605in}{3.795980in}}%
\pgfpathlineto{\pgfqpoint{5.582605in}{3.802788in}}%
\pgfpathlineto{\pgfqpoint{5.432702in}{3.802788in}}%
\pgfpathlineto{\pgfqpoint{5.432702in}{3.795980in}}%
\pgfpathclose%
\pgfusepath{stroke,fill}%
\end{pgfscope}%
\begin{pgfscope}%
\pgfpathrectangle{\pgfqpoint{5.095420in}{3.795980in}}{\pgfqpoint{0.824468in}{0.243158in}}%
\pgfusepath{clip}%
\pgfsetbuttcap%
\pgfsetmiterjoin%
\definecolor{currentfill}{rgb}{0.121569,0.466667,0.705882}%
\pgfsetfillcolor{currentfill}%
\pgfsetfillopacity{0.500000}%
\pgfsetlinewidth{1.003750pt}%
\definecolor{currentstroke}{rgb}{0.000000,0.000000,0.000000}%
\pgfsetstrokecolor{currentstroke}%
\pgfsetdash{}{0pt}%
\pgfpathmoveto{\pgfqpoint{5.582605in}{3.795980in}}%
\pgfpathlineto{\pgfqpoint{5.732509in}{3.795980in}}%
\pgfpathlineto{\pgfqpoint{5.732509in}{3.803639in}}%
\pgfpathlineto{\pgfqpoint{5.582605in}{3.803639in}}%
\pgfpathlineto{\pgfqpoint{5.582605in}{3.795980in}}%
\pgfpathclose%
\pgfusepath{stroke,fill}%
\end{pgfscope}%
\begin{pgfscope}%
\pgfpathrectangle{\pgfqpoint{5.095420in}{3.795980in}}{\pgfqpoint{0.824468in}{0.243158in}}%
\pgfusepath{clip}%
\pgfsetbuttcap%
\pgfsetmiterjoin%
\definecolor{currentfill}{rgb}{0.121569,0.466667,0.705882}%
\pgfsetfillcolor{currentfill}%
\pgfsetfillopacity{0.500000}%
\pgfsetlinewidth{1.003750pt}%
\definecolor{currentstroke}{rgb}{0.000000,0.000000,0.000000}%
\pgfsetstrokecolor{currentstroke}%
\pgfsetdash{}{0pt}%
\pgfpathmoveto{\pgfqpoint{5.732509in}{3.795980in}}%
\pgfpathlineto{\pgfqpoint{5.882412in}{3.795980in}}%
\pgfpathlineto{\pgfqpoint{5.882412in}{3.798776in}}%
\pgfpathlineto{\pgfqpoint{5.732509in}{3.798776in}}%
\pgfpathlineto{\pgfqpoint{5.732509in}{3.795980in}}%
\pgfpathclose%
\pgfusepath{stroke,fill}%
\end{pgfscope}%
\begin{pgfscope}%
\pgfsetrectcap%
\pgfsetmiterjoin%
\pgfsetlinewidth{0.803000pt}%
\definecolor{currentstroke}{rgb}{0.000000,0.000000,0.000000}%
\pgfsetstrokecolor{currentstroke}%
\pgfsetdash{}{0pt}%
\pgfpathmoveto{\pgfqpoint{5.095420in}{3.795980in}}%
\pgfpathlineto{\pgfqpoint{5.095420in}{4.039137in}}%
\pgfusepath{stroke}%
\end{pgfscope}%
\begin{pgfscope}%
\pgfsetrectcap%
\pgfsetmiterjoin%
\pgfsetlinewidth{0.803000pt}%
\definecolor{currentstroke}{rgb}{0.000000,0.000000,0.000000}%
\pgfsetstrokecolor{currentstroke}%
\pgfsetdash{}{0pt}%
\pgfpathmoveto{\pgfqpoint{5.919888in}{3.795980in}}%
\pgfpathlineto{\pgfqpoint{5.919888in}{4.039137in}}%
\pgfusepath{stroke}%
\end{pgfscope}%
\begin{pgfscope}%
\pgfsetrectcap%
\pgfsetmiterjoin%
\pgfsetlinewidth{0.803000pt}%
\definecolor{currentstroke}{rgb}{0.000000,0.000000,0.000000}%
\pgfsetstrokecolor{currentstroke}%
\pgfsetdash{}{0pt}%
\pgfpathmoveto{\pgfqpoint{5.095420in}{3.795980in}}%
\pgfpathlineto{\pgfqpoint{5.919888in}{3.795980in}}%
\pgfusepath{stroke}%
\end{pgfscope}%
\begin{pgfscope}%
\pgfsetrectcap%
\pgfsetmiterjoin%
\pgfsetlinewidth{0.803000pt}%
\definecolor{currentstroke}{rgb}{0.000000,0.000000,0.000000}%
\pgfsetstrokecolor{currentstroke}%
\pgfsetdash{}{0pt}%
\pgfpathmoveto{\pgfqpoint{5.095420in}{4.039137in}}%
\pgfpathlineto{\pgfqpoint{5.919888in}{4.039137in}}%
\pgfusepath{stroke}%
\end{pgfscope}%
\begin{pgfscope}%
\definecolor{textcolor}{rgb}{0.000000,0.000000,0.000000}%
\pgfsetstrokecolor{textcolor}%
\pgfsetfillcolor{textcolor}%
\pgftext[x=5.507654in,y=4.122471in,,base]{\color{textcolor}\rmfamily\fontsize{11.000000}{13.200000}\selectfont Pacifica}%
\end{pgfscope}%
\begin{pgfscope}%
\pgfsetbuttcap%
\pgfsetmiterjoin%
\definecolor{currentfill}{rgb}{1.000000,1.000000,1.000000}%
\pgfsetfillcolor{currentfill}%
\pgfsetlinewidth{0.000000pt}%
\definecolor{currentstroke}{rgb}{0.000000,0.000000,0.000000}%
\pgfsetstrokecolor{currentstroke}%
\pgfsetstrokeopacity{0.000000}%
\pgfsetdash{}{0pt}%
\pgfpathmoveto{\pgfqpoint{6.084781in}{3.795980in}}%
\pgfpathlineto{\pgfqpoint{6.909249in}{3.795980in}}%
\pgfpathlineto{\pgfqpoint{6.909249in}{4.039137in}}%
\pgfpathlineto{\pgfqpoint{6.084781in}{4.039137in}}%
\pgfpathlineto{\pgfqpoint{6.084781in}{3.795980in}}%
\pgfpathclose%
\pgfusepath{fill}%
\end{pgfscope}%
\begin{pgfscope}%
\pgfpathrectangle{\pgfqpoint{6.084781in}{3.795980in}}{\pgfqpoint{0.824468in}{0.243158in}}%
\pgfusepath{clip}%
\pgfsetbuttcap%
\pgfsetmiterjoin%
\definecolor{currentfill}{rgb}{0.121569,0.466667,0.705882}%
\pgfsetfillcolor{currentfill}%
\pgfsetfillopacity{0.500000}%
\pgfsetlinewidth{1.003750pt}%
\definecolor{currentstroke}{rgb}{0.000000,0.000000,0.000000}%
\pgfsetstrokecolor{currentstroke}%
\pgfsetdash{}{0pt}%
\pgfpathmoveto{\pgfqpoint{6.122257in}{3.795980in}}%
\pgfpathlineto{\pgfqpoint{6.272160in}{3.795980in}}%
\pgfpathlineto{\pgfqpoint{6.272160in}{3.808137in}}%
\pgfpathlineto{\pgfqpoint{6.122257in}{3.808137in}}%
\pgfpathlineto{\pgfqpoint{6.122257in}{3.795980in}}%
\pgfpathclose%
\pgfusepath{stroke,fill}%
\end{pgfscope}%
\begin{pgfscope}%
\pgfpathrectangle{\pgfqpoint{6.084781in}{3.795980in}}{\pgfqpoint{0.824468in}{0.243158in}}%
\pgfusepath{clip}%
\pgfsetbuttcap%
\pgfsetmiterjoin%
\definecolor{currentfill}{rgb}{0.121569,0.466667,0.705882}%
\pgfsetfillcolor{currentfill}%
\pgfsetfillopacity{0.500000}%
\pgfsetlinewidth{1.003750pt}%
\definecolor{currentstroke}{rgb}{0.000000,0.000000,0.000000}%
\pgfsetstrokecolor{currentstroke}%
\pgfsetdash{}{0pt}%
\pgfpathmoveto{\pgfqpoint{6.272160in}{3.795980in}}%
\pgfpathlineto{\pgfqpoint{6.422064in}{3.795980in}}%
\pgfpathlineto{\pgfqpoint{6.422064in}{3.798411in}}%
\pgfpathlineto{\pgfqpoint{6.272160in}{3.798411in}}%
\pgfpathlineto{\pgfqpoint{6.272160in}{3.795980in}}%
\pgfpathclose%
\pgfusepath{stroke,fill}%
\end{pgfscope}%
\begin{pgfscope}%
\pgfpathrectangle{\pgfqpoint{6.084781in}{3.795980in}}{\pgfqpoint{0.824468in}{0.243158in}}%
\pgfusepath{clip}%
\pgfsetbuttcap%
\pgfsetmiterjoin%
\definecolor{currentfill}{rgb}{0.121569,0.466667,0.705882}%
\pgfsetfillcolor{currentfill}%
\pgfsetfillopacity{0.500000}%
\pgfsetlinewidth{1.003750pt}%
\definecolor{currentstroke}{rgb}{0.000000,0.000000,0.000000}%
\pgfsetstrokecolor{currentstroke}%
\pgfsetdash{}{0pt}%
\pgfpathmoveto{\pgfqpoint{6.422064in}{3.795980in}}%
\pgfpathlineto{\pgfqpoint{6.571967in}{3.795980in}}%
\pgfpathlineto{\pgfqpoint{6.571967in}{3.798168in}}%
\pgfpathlineto{\pgfqpoint{6.422064in}{3.798168in}}%
\pgfpathlineto{\pgfqpoint{6.422064in}{3.795980in}}%
\pgfpathclose%
\pgfusepath{stroke,fill}%
\end{pgfscope}%
\begin{pgfscope}%
\pgfpathrectangle{\pgfqpoint{6.084781in}{3.795980in}}{\pgfqpoint{0.824468in}{0.243158in}}%
\pgfusepath{clip}%
\pgfsetbuttcap%
\pgfsetmiterjoin%
\definecolor{currentfill}{rgb}{0.121569,0.466667,0.705882}%
\pgfsetfillcolor{currentfill}%
\pgfsetfillopacity{0.500000}%
\pgfsetlinewidth{1.003750pt}%
\definecolor{currentstroke}{rgb}{0.000000,0.000000,0.000000}%
\pgfsetstrokecolor{currentstroke}%
\pgfsetdash{}{0pt}%
\pgfpathmoveto{\pgfqpoint{6.571967in}{3.795980in}}%
\pgfpathlineto{\pgfqpoint{6.721870in}{3.795980in}}%
\pgfpathlineto{\pgfqpoint{6.721870in}{3.795980in}}%
\pgfpathlineto{\pgfqpoint{6.571967in}{3.795980in}}%
\pgfpathlineto{\pgfqpoint{6.571967in}{3.795980in}}%
\pgfpathclose%
\pgfusepath{stroke,fill}%
\end{pgfscope}%
\begin{pgfscope}%
\pgfpathrectangle{\pgfqpoint{6.084781in}{3.795980in}}{\pgfqpoint{0.824468in}{0.243158in}}%
\pgfusepath{clip}%
\pgfsetbuttcap%
\pgfsetmiterjoin%
\definecolor{currentfill}{rgb}{0.121569,0.466667,0.705882}%
\pgfsetfillcolor{currentfill}%
\pgfsetfillopacity{0.500000}%
\pgfsetlinewidth{1.003750pt}%
\definecolor{currentstroke}{rgb}{0.000000,0.000000,0.000000}%
\pgfsetstrokecolor{currentstroke}%
\pgfsetdash{}{0pt}%
\pgfpathmoveto{\pgfqpoint{6.721870in}{3.795980in}}%
\pgfpathlineto{\pgfqpoint{6.871774in}{3.795980in}}%
\pgfpathlineto{\pgfqpoint{6.871774in}{3.795980in}}%
\pgfpathlineto{\pgfqpoint{6.721870in}{3.795980in}}%
\pgfpathlineto{\pgfqpoint{6.721870in}{3.795980in}}%
\pgfpathclose%
\pgfusepath{stroke,fill}%
\end{pgfscope}%
\begin{pgfscope}%
\pgfsetrectcap%
\pgfsetmiterjoin%
\pgfsetlinewidth{0.803000pt}%
\definecolor{currentstroke}{rgb}{0.000000,0.000000,0.000000}%
\pgfsetstrokecolor{currentstroke}%
\pgfsetdash{}{0pt}%
\pgfpathmoveto{\pgfqpoint{6.084781in}{3.795980in}}%
\pgfpathlineto{\pgfqpoint{6.084781in}{4.039137in}}%
\pgfusepath{stroke}%
\end{pgfscope}%
\begin{pgfscope}%
\pgfsetrectcap%
\pgfsetmiterjoin%
\pgfsetlinewidth{0.803000pt}%
\definecolor{currentstroke}{rgb}{0.000000,0.000000,0.000000}%
\pgfsetstrokecolor{currentstroke}%
\pgfsetdash{}{0pt}%
\pgfpathmoveto{\pgfqpoint{6.909249in}{3.795980in}}%
\pgfpathlineto{\pgfqpoint{6.909249in}{4.039137in}}%
\pgfusepath{stroke}%
\end{pgfscope}%
\begin{pgfscope}%
\pgfsetrectcap%
\pgfsetmiterjoin%
\pgfsetlinewidth{0.803000pt}%
\definecolor{currentstroke}{rgb}{0.000000,0.000000,0.000000}%
\pgfsetstrokecolor{currentstroke}%
\pgfsetdash{}{0pt}%
\pgfpathmoveto{\pgfqpoint{6.084781in}{3.795980in}}%
\pgfpathlineto{\pgfqpoint{6.909249in}{3.795980in}}%
\pgfusepath{stroke}%
\end{pgfscope}%
\begin{pgfscope}%
\pgfsetrectcap%
\pgfsetmiterjoin%
\pgfsetlinewidth{0.803000pt}%
\definecolor{currentstroke}{rgb}{0.000000,0.000000,0.000000}%
\pgfsetstrokecolor{currentstroke}%
\pgfsetdash{}{0pt}%
\pgfpathmoveto{\pgfqpoint{6.084781in}{4.039137in}}%
\pgfpathlineto{\pgfqpoint{6.909249in}{4.039137in}}%
\pgfusepath{stroke}%
\end{pgfscope}%
\begin{pgfscope}%
\definecolor{textcolor}{rgb}{0.000000,0.000000,0.000000}%
\pgfsetstrokecolor{textcolor}%
\pgfsetfillcolor{textcolor}%
\pgftext[x=6.497015in,y=4.122471in,,base]{\color{textcolor}\rmfamily\fontsize{11.000000}{13.200000}\selectfont SwissLife}%
\end{pgfscope}%
\begin{pgfscope}%
\pgfsetbuttcap%
\pgfsetmiterjoin%
\definecolor{currentfill}{rgb}{1.000000,1.000000,1.000000}%
\pgfsetfillcolor{currentfill}%
\pgfsetlinewidth{0.000000pt}%
\definecolor{currentstroke}{rgb}{0.000000,0.000000,0.000000}%
\pgfsetstrokecolor{currentstroke}%
\pgfsetstrokeopacity{0.000000}%
\pgfsetdash{}{0pt}%
\pgfpathmoveto{\pgfqpoint{7.074143in}{3.795980in}}%
\pgfpathlineto{\pgfqpoint{7.898611in}{3.795980in}}%
\pgfpathlineto{\pgfqpoint{7.898611in}{4.039137in}}%
\pgfpathlineto{\pgfqpoint{7.074143in}{4.039137in}}%
\pgfpathlineto{\pgfqpoint{7.074143in}{3.795980in}}%
\pgfpathclose%
\pgfusepath{fill}%
\end{pgfscope}%
\begin{pgfscope}%
\pgfpathrectangle{\pgfqpoint{7.074143in}{3.795980in}}{\pgfqpoint{0.824468in}{0.243158in}}%
\pgfusepath{clip}%
\pgfsetbuttcap%
\pgfsetmiterjoin%
\definecolor{currentfill}{rgb}{0.121569,0.466667,0.705882}%
\pgfsetfillcolor{currentfill}%
\pgfsetfillopacity{0.500000}%
\pgfsetlinewidth{1.003750pt}%
\definecolor{currentstroke}{rgb}{0.000000,0.000000,0.000000}%
\pgfsetstrokecolor{currentstroke}%
\pgfsetdash{}{0pt}%
\pgfpathmoveto{\pgfqpoint{7.111619in}{3.795980in}}%
\pgfpathlineto{\pgfqpoint{7.261522in}{3.795980in}}%
\pgfpathlineto{\pgfqpoint{7.261522in}{3.831237in}}%
\pgfpathlineto{\pgfqpoint{7.111619in}{3.831237in}}%
\pgfpathlineto{\pgfqpoint{7.111619in}{3.795980in}}%
\pgfpathclose%
\pgfusepath{stroke,fill}%
\end{pgfscope}%
\begin{pgfscope}%
\pgfpathrectangle{\pgfqpoint{7.074143in}{3.795980in}}{\pgfqpoint{0.824468in}{0.243158in}}%
\pgfusepath{clip}%
\pgfsetbuttcap%
\pgfsetmiterjoin%
\definecolor{currentfill}{rgb}{0.121569,0.466667,0.705882}%
\pgfsetfillcolor{currentfill}%
\pgfsetfillopacity{0.500000}%
\pgfsetlinewidth{1.003750pt}%
\definecolor{currentstroke}{rgb}{0.000000,0.000000,0.000000}%
\pgfsetstrokecolor{currentstroke}%
\pgfsetdash{}{0pt}%
\pgfpathmoveto{\pgfqpoint{7.261522in}{3.795980in}}%
\pgfpathlineto{\pgfqpoint{7.411425in}{3.795980in}}%
\pgfpathlineto{\pgfqpoint{7.411425in}{3.818958in}}%
\pgfpathlineto{\pgfqpoint{7.261522in}{3.818958in}}%
\pgfpathlineto{\pgfqpoint{7.261522in}{3.795980in}}%
\pgfpathclose%
\pgfusepath{stroke,fill}%
\end{pgfscope}%
\begin{pgfscope}%
\pgfpathrectangle{\pgfqpoint{7.074143in}{3.795980in}}{\pgfqpoint{0.824468in}{0.243158in}}%
\pgfusepath{clip}%
\pgfsetbuttcap%
\pgfsetmiterjoin%
\definecolor{currentfill}{rgb}{0.121569,0.466667,0.705882}%
\pgfsetfillcolor{currentfill}%
\pgfsetfillopacity{0.500000}%
\pgfsetlinewidth{1.003750pt}%
\definecolor{currentstroke}{rgb}{0.000000,0.000000,0.000000}%
\pgfsetstrokecolor{currentstroke}%
\pgfsetdash{}{0pt}%
\pgfpathmoveto{\pgfqpoint{7.411425in}{3.795980in}}%
\pgfpathlineto{\pgfqpoint{7.561329in}{3.795980in}}%
\pgfpathlineto{\pgfqpoint{7.561329in}{3.806435in}}%
\pgfpathlineto{\pgfqpoint{7.411425in}{3.806435in}}%
\pgfpathlineto{\pgfqpoint{7.411425in}{3.795980in}}%
\pgfpathclose%
\pgfusepath{stroke,fill}%
\end{pgfscope}%
\begin{pgfscope}%
\pgfpathrectangle{\pgfqpoint{7.074143in}{3.795980in}}{\pgfqpoint{0.824468in}{0.243158in}}%
\pgfusepath{clip}%
\pgfsetbuttcap%
\pgfsetmiterjoin%
\definecolor{currentfill}{rgb}{0.121569,0.466667,0.705882}%
\pgfsetfillcolor{currentfill}%
\pgfsetfillopacity{0.500000}%
\pgfsetlinewidth{1.003750pt}%
\definecolor{currentstroke}{rgb}{0.000000,0.000000,0.000000}%
\pgfsetstrokecolor{currentstroke}%
\pgfsetdash{}{0pt}%
\pgfpathmoveto{\pgfqpoint{7.561329in}{3.795980in}}%
\pgfpathlineto{\pgfqpoint{7.711232in}{3.795980in}}%
\pgfpathlineto{\pgfqpoint{7.711232in}{3.799748in}}%
\pgfpathlineto{\pgfqpoint{7.561329in}{3.799748in}}%
\pgfpathlineto{\pgfqpoint{7.561329in}{3.795980in}}%
\pgfpathclose%
\pgfusepath{stroke,fill}%
\end{pgfscope}%
\begin{pgfscope}%
\pgfpathrectangle{\pgfqpoint{7.074143in}{3.795980in}}{\pgfqpoint{0.824468in}{0.243158in}}%
\pgfusepath{clip}%
\pgfsetbuttcap%
\pgfsetmiterjoin%
\definecolor{currentfill}{rgb}{0.121569,0.466667,0.705882}%
\pgfsetfillcolor{currentfill}%
\pgfsetfillopacity{0.500000}%
\pgfsetlinewidth{1.003750pt}%
\definecolor{currentstroke}{rgb}{0.000000,0.000000,0.000000}%
\pgfsetstrokecolor{currentstroke}%
\pgfsetdash{}{0pt}%
\pgfpathmoveto{\pgfqpoint{7.711232in}{3.795980in}}%
\pgfpathlineto{\pgfqpoint{7.861135in}{3.795980in}}%
\pgfpathlineto{\pgfqpoint{7.861135in}{3.797925in}}%
\pgfpathlineto{\pgfqpoint{7.711232in}{3.797925in}}%
\pgfpathlineto{\pgfqpoint{7.711232in}{3.795980in}}%
\pgfpathclose%
\pgfusepath{stroke,fill}%
\end{pgfscope}%
\begin{pgfscope}%
\pgfsetrectcap%
\pgfsetmiterjoin%
\pgfsetlinewidth{0.803000pt}%
\definecolor{currentstroke}{rgb}{0.000000,0.000000,0.000000}%
\pgfsetstrokecolor{currentstroke}%
\pgfsetdash{}{0pt}%
\pgfpathmoveto{\pgfqpoint{7.074143in}{3.795980in}}%
\pgfpathlineto{\pgfqpoint{7.074143in}{4.039137in}}%
\pgfusepath{stroke}%
\end{pgfscope}%
\begin{pgfscope}%
\pgfsetrectcap%
\pgfsetmiterjoin%
\pgfsetlinewidth{0.803000pt}%
\definecolor{currentstroke}{rgb}{0.000000,0.000000,0.000000}%
\pgfsetstrokecolor{currentstroke}%
\pgfsetdash{}{0pt}%
\pgfpathmoveto{\pgfqpoint{7.898611in}{3.795980in}}%
\pgfpathlineto{\pgfqpoint{7.898611in}{4.039137in}}%
\pgfusepath{stroke}%
\end{pgfscope}%
\begin{pgfscope}%
\pgfsetrectcap%
\pgfsetmiterjoin%
\pgfsetlinewidth{0.803000pt}%
\definecolor{currentstroke}{rgb}{0.000000,0.000000,0.000000}%
\pgfsetstrokecolor{currentstroke}%
\pgfsetdash{}{0pt}%
\pgfpathmoveto{\pgfqpoint{7.074143in}{3.795980in}}%
\pgfpathlineto{\pgfqpoint{7.898611in}{3.795980in}}%
\pgfusepath{stroke}%
\end{pgfscope}%
\begin{pgfscope}%
\pgfsetrectcap%
\pgfsetmiterjoin%
\pgfsetlinewidth{0.803000pt}%
\definecolor{currentstroke}{rgb}{0.000000,0.000000,0.000000}%
\pgfsetstrokecolor{currentstroke}%
\pgfsetdash{}{0pt}%
\pgfpathmoveto{\pgfqpoint{7.074143in}{4.039137in}}%
\pgfpathlineto{\pgfqpoint{7.898611in}{4.039137in}}%
\pgfusepath{stroke}%
\end{pgfscope}%
\begin{pgfscope}%
\definecolor{textcolor}{rgb}{0.000000,0.000000,0.000000}%
\pgfsetstrokecolor{textcolor}%
\pgfsetfillcolor{textcolor}%
\pgftext[x=7.486377in,y=4.122471in,,base]{\color{textcolor}\rmfamily\fontsize{11.000000}{13.200000}\selectfont MAAF}%
\end{pgfscope}%
\begin{pgfscope}%
\pgfsetbuttcap%
\pgfsetmiterjoin%
\definecolor{currentfill}{rgb}{1.000000,1.000000,1.000000}%
\pgfsetfillcolor{currentfill}%
\pgfsetlinewidth{0.000000pt}%
\definecolor{currentstroke}{rgb}{0.000000,0.000000,0.000000}%
\pgfsetstrokecolor{currentstroke}%
\pgfsetstrokeopacity{0.000000}%
\pgfsetdash{}{0pt}%
\pgfpathmoveto{\pgfqpoint{0.148611in}{3.066506in}}%
\pgfpathlineto{\pgfqpoint{0.973079in}{3.066506in}}%
\pgfpathlineto{\pgfqpoint{0.973079in}{3.309664in}}%
\pgfpathlineto{\pgfqpoint{0.148611in}{3.309664in}}%
\pgfpathlineto{\pgfqpoint{0.148611in}{3.066506in}}%
\pgfpathclose%
\pgfusepath{fill}%
\end{pgfscope}%
\begin{pgfscope}%
\pgfpathrectangle{\pgfqpoint{0.148611in}{3.066506in}}{\pgfqpoint{0.824468in}{0.243158in}}%
\pgfusepath{clip}%
\pgfsetbuttcap%
\pgfsetmiterjoin%
\definecolor{currentfill}{rgb}{0.121569,0.466667,0.705882}%
\pgfsetfillcolor{currentfill}%
\pgfsetfillopacity{0.500000}%
\pgfsetlinewidth{1.003750pt}%
\definecolor{currentstroke}{rgb}{0.000000,0.000000,0.000000}%
\pgfsetstrokecolor{currentstroke}%
\pgfsetdash{}{0pt}%
\pgfpathmoveto{\pgfqpoint{0.186087in}{3.066506in}}%
\pgfpathlineto{\pgfqpoint{0.335990in}{3.066506in}}%
\pgfpathlineto{\pgfqpoint{0.335990in}{3.068208in}}%
\pgfpathlineto{\pgfqpoint{0.186087in}{3.068208in}}%
\pgfpathlineto{\pgfqpoint{0.186087in}{3.066506in}}%
\pgfpathclose%
\pgfusepath{stroke,fill}%
\end{pgfscope}%
\begin{pgfscope}%
\pgfpathrectangle{\pgfqpoint{0.148611in}{3.066506in}}{\pgfqpoint{0.824468in}{0.243158in}}%
\pgfusepath{clip}%
\pgfsetbuttcap%
\pgfsetmiterjoin%
\definecolor{currentfill}{rgb}{0.121569,0.466667,0.705882}%
\pgfsetfillcolor{currentfill}%
\pgfsetfillopacity{0.500000}%
\pgfsetlinewidth{1.003750pt}%
\definecolor{currentstroke}{rgb}{0.000000,0.000000,0.000000}%
\pgfsetstrokecolor{currentstroke}%
\pgfsetdash{}{0pt}%
\pgfpathmoveto{\pgfqpoint{0.335990in}{3.066506in}}%
\pgfpathlineto{\pgfqpoint{0.485894in}{3.066506in}}%
\pgfpathlineto{\pgfqpoint{0.485894in}{3.067114in}}%
\pgfpathlineto{\pgfqpoint{0.335990in}{3.067114in}}%
\pgfpathlineto{\pgfqpoint{0.335990in}{3.066506in}}%
\pgfpathclose%
\pgfusepath{stroke,fill}%
\end{pgfscope}%
\begin{pgfscope}%
\pgfpathrectangle{\pgfqpoint{0.148611in}{3.066506in}}{\pgfqpoint{0.824468in}{0.243158in}}%
\pgfusepath{clip}%
\pgfsetbuttcap%
\pgfsetmiterjoin%
\definecolor{currentfill}{rgb}{0.121569,0.466667,0.705882}%
\pgfsetfillcolor{currentfill}%
\pgfsetfillopacity{0.500000}%
\pgfsetlinewidth{1.003750pt}%
\definecolor{currentstroke}{rgb}{0.000000,0.000000,0.000000}%
\pgfsetstrokecolor{currentstroke}%
\pgfsetdash{}{0pt}%
\pgfpathmoveto{\pgfqpoint{0.485894in}{3.066506in}}%
\pgfpathlineto{\pgfqpoint{0.635797in}{3.066506in}}%
\pgfpathlineto{\pgfqpoint{0.635797in}{3.066506in}}%
\pgfpathlineto{\pgfqpoint{0.485894in}{3.066506in}}%
\pgfpathlineto{\pgfqpoint{0.485894in}{3.066506in}}%
\pgfpathclose%
\pgfusepath{stroke,fill}%
\end{pgfscope}%
\begin{pgfscope}%
\pgfpathrectangle{\pgfqpoint{0.148611in}{3.066506in}}{\pgfqpoint{0.824468in}{0.243158in}}%
\pgfusepath{clip}%
\pgfsetbuttcap%
\pgfsetmiterjoin%
\definecolor{currentfill}{rgb}{0.121569,0.466667,0.705882}%
\pgfsetfillcolor{currentfill}%
\pgfsetfillopacity{0.500000}%
\pgfsetlinewidth{1.003750pt}%
\definecolor{currentstroke}{rgb}{0.000000,0.000000,0.000000}%
\pgfsetstrokecolor{currentstroke}%
\pgfsetdash{}{0pt}%
\pgfpathmoveto{\pgfqpoint{0.635797in}{3.066506in}}%
\pgfpathlineto{\pgfqpoint{0.785700in}{3.066506in}}%
\pgfpathlineto{\pgfqpoint{0.785700in}{3.066871in}}%
\pgfpathlineto{\pgfqpoint{0.635797in}{3.066871in}}%
\pgfpathlineto{\pgfqpoint{0.635797in}{3.066506in}}%
\pgfpathclose%
\pgfusepath{stroke,fill}%
\end{pgfscope}%
\begin{pgfscope}%
\pgfpathrectangle{\pgfqpoint{0.148611in}{3.066506in}}{\pgfqpoint{0.824468in}{0.243158in}}%
\pgfusepath{clip}%
\pgfsetbuttcap%
\pgfsetmiterjoin%
\definecolor{currentfill}{rgb}{0.121569,0.466667,0.705882}%
\pgfsetfillcolor{currentfill}%
\pgfsetfillopacity{0.500000}%
\pgfsetlinewidth{1.003750pt}%
\definecolor{currentstroke}{rgb}{0.000000,0.000000,0.000000}%
\pgfsetstrokecolor{currentstroke}%
\pgfsetdash{}{0pt}%
\pgfpathmoveto{\pgfqpoint{0.785700in}{3.066506in}}%
\pgfpathlineto{\pgfqpoint{0.935603in}{3.066506in}}%
\pgfpathlineto{\pgfqpoint{0.935603in}{3.066871in}}%
\pgfpathlineto{\pgfqpoint{0.785700in}{3.066871in}}%
\pgfpathlineto{\pgfqpoint{0.785700in}{3.066506in}}%
\pgfpathclose%
\pgfusepath{stroke,fill}%
\end{pgfscope}%
\begin{pgfscope}%
\pgfsetrectcap%
\pgfsetmiterjoin%
\pgfsetlinewidth{0.803000pt}%
\definecolor{currentstroke}{rgb}{0.000000,0.000000,0.000000}%
\pgfsetstrokecolor{currentstroke}%
\pgfsetdash{}{0pt}%
\pgfpathmoveto{\pgfqpoint{0.148611in}{3.066506in}}%
\pgfpathlineto{\pgfqpoint{0.148611in}{3.309664in}}%
\pgfusepath{stroke}%
\end{pgfscope}%
\begin{pgfscope}%
\pgfsetrectcap%
\pgfsetmiterjoin%
\pgfsetlinewidth{0.803000pt}%
\definecolor{currentstroke}{rgb}{0.000000,0.000000,0.000000}%
\pgfsetstrokecolor{currentstroke}%
\pgfsetdash{}{0pt}%
\pgfpathmoveto{\pgfqpoint{0.973079in}{3.066506in}}%
\pgfpathlineto{\pgfqpoint{0.973079in}{3.309664in}}%
\pgfusepath{stroke}%
\end{pgfscope}%
\begin{pgfscope}%
\pgfsetrectcap%
\pgfsetmiterjoin%
\pgfsetlinewidth{0.803000pt}%
\definecolor{currentstroke}{rgb}{0.000000,0.000000,0.000000}%
\pgfsetstrokecolor{currentstroke}%
\pgfsetdash{}{0pt}%
\pgfpathmoveto{\pgfqpoint{0.148611in}{3.066506in}}%
\pgfpathlineto{\pgfqpoint{0.973079in}{3.066506in}}%
\pgfusepath{stroke}%
\end{pgfscope}%
\begin{pgfscope}%
\pgfsetrectcap%
\pgfsetmiterjoin%
\pgfsetlinewidth{0.803000pt}%
\definecolor{currentstroke}{rgb}{0.000000,0.000000,0.000000}%
\pgfsetstrokecolor{currentstroke}%
\pgfsetdash{}{0pt}%
\pgfpathmoveto{\pgfqpoint{0.148611in}{3.309664in}}%
\pgfpathlineto{\pgfqpoint{0.973079in}{3.309664in}}%
\pgfusepath{stroke}%
\end{pgfscope}%
\begin{pgfscope}%
\definecolor{textcolor}{rgb}{0.000000,0.000000,0.000000}%
\pgfsetstrokecolor{textcolor}%
\pgfsetfillcolor{textcolor}%
\pgftext[x=0.560845in,y=3.392997in,,base]{\color{textcolor}\rmfamily\fontsize{11.000000}{13.200000}\selectfont Solly ...}%
\end{pgfscope}%
\begin{pgfscope}%
\pgfsetbuttcap%
\pgfsetmiterjoin%
\definecolor{currentfill}{rgb}{1.000000,1.000000,1.000000}%
\pgfsetfillcolor{currentfill}%
\pgfsetlinewidth{0.000000pt}%
\definecolor{currentstroke}{rgb}{0.000000,0.000000,0.000000}%
\pgfsetstrokecolor{currentstroke}%
\pgfsetstrokeopacity{0.000000}%
\pgfsetdash{}{0pt}%
\pgfpathmoveto{\pgfqpoint{1.137973in}{3.066506in}}%
\pgfpathlineto{\pgfqpoint{1.962441in}{3.066506in}}%
\pgfpathlineto{\pgfqpoint{1.962441in}{3.309664in}}%
\pgfpathlineto{\pgfqpoint{1.137973in}{3.309664in}}%
\pgfpathlineto{\pgfqpoint{1.137973in}{3.066506in}}%
\pgfpathclose%
\pgfusepath{fill}%
\end{pgfscope}%
\begin{pgfscope}%
\pgfpathrectangle{\pgfqpoint{1.137973in}{3.066506in}}{\pgfqpoint{0.824468in}{0.243158in}}%
\pgfusepath{clip}%
\pgfsetbuttcap%
\pgfsetmiterjoin%
\definecolor{currentfill}{rgb}{0.121569,0.466667,0.705882}%
\pgfsetfillcolor{currentfill}%
\pgfsetfillopacity{0.500000}%
\pgfsetlinewidth{1.003750pt}%
\definecolor{currentstroke}{rgb}{0.000000,0.000000,0.000000}%
\pgfsetstrokecolor{currentstroke}%
\pgfsetdash{}{0pt}%
\pgfpathmoveto{\pgfqpoint{1.175449in}{3.066506in}}%
\pgfpathlineto{\pgfqpoint{1.325352in}{3.066506in}}%
\pgfpathlineto{\pgfqpoint{1.325352in}{3.092645in}}%
\pgfpathlineto{\pgfqpoint{1.175449in}{3.092645in}}%
\pgfpathlineto{\pgfqpoint{1.175449in}{3.066506in}}%
\pgfpathclose%
\pgfusepath{stroke,fill}%
\end{pgfscope}%
\begin{pgfscope}%
\pgfpathrectangle{\pgfqpoint{1.137973in}{3.066506in}}{\pgfqpoint{0.824468in}{0.243158in}}%
\pgfusepath{clip}%
\pgfsetbuttcap%
\pgfsetmiterjoin%
\definecolor{currentfill}{rgb}{0.121569,0.466667,0.705882}%
\pgfsetfillcolor{currentfill}%
\pgfsetfillopacity{0.500000}%
\pgfsetlinewidth{1.003750pt}%
\definecolor{currentstroke}{rgb}{0.000000,0.000000,0.000000}%
\pgfsetstrokecolor{currentstroke}%
\pgfsetdash{}{0pt}%
\pgfpathmoveto{\pgfqpoint{1.325352in}{3.066506in}}%
\pgfpathlineto{\pgfqpoint{1.475255in}{3.066506in}}%
\pgfpathlineto{\pgfqpoint{1.475255in}{3.093375in}}%
\pgfpathlineto{\pgfqpoint{1.325352in}{3.093375in}}%
\pgfpathlineto{\pgfqpoint{1.325352in}{3.066506in}}%
\pgfpathclose%
\pgfusepath{stroke,fill}%
\end{pgfscope}%
\begin{pgfscope}%
\pgfpathrectangle{\pgfqpoint{1.137973in}{3.066506in}}{\pgfqpoint{0.824468in}{0.243158in}}%
\pgfusepath{clip}%
\pgfsetbuttcap%
\pgfsetmiterjoin%
\definecolor{currentfill}{rgb}{0.121569,0.466667,0.705882}%
\pgfsetfillcolor{currentfill}%
\pgfsetfillopacity{0.500000}%
\pgfsetlinewidth{1.003750pt}%
\definecolor{currentstroke}{rgb}{0.000000,0.000000,0.000000}%
\pgfsetstrokecolor{currentstroke}%
\pgfsetdash{}{0pt}%
\pgfpathmoveto{\pgfqpoint{1.475255in}{3.066506in}}%
\pgfpathlineto{\pgfqpoint{1.625158in}{3.066506in}}%
\pgfpathlineto{\pgfqpoint{1.625158in}{3.088390in}}%
\pgfpathlineto{\pgfqpoint{1.475255in}{3.088390in}}%
\pgfpathlineto{\pgfqpoint{1.475255in}{3.066506in}}%
\pgfpathclose%
\pgfusepath{stroke,fill}%
\end{pgfscope}%
\begin{pgfscope}%
\pgfpathrectangle{\pgfqpoint{1.137973in}{3.066506in}}{\pgfqpoint{0.824468in}{0.243158in}}%
\pgfusepath{clip}%
\pgfsetbuttcap%
\pgfsetmiterjoin%
\definecolor{currentfill}{rgb}{0.121569,0.466667,0.705882}%
\pgfsetfillcolor{currentfill}%
\pgfsetfillopacity{0.500000}%
\pgfsetlinewidth{1.003750pt}%
\definecolor{currentstroke}{rgb}{0.000000,0.000000,0.000000}%
\pgfsetstrokecolor{currentstroke}%
\pgfsetdash{}{0pt}%
\pgfpathmoveto{\pgfqpoint{1.625158in}{3.066506in}}%
\pgfpathlineto{\pgfqpoint{1.775062in}{3.066506in}}%
\pgfpathlineto{\pgfqpoint{1.775062in}{3.094104in}}%
\pgfpathlineto{\pgfqpoint{1.625158in}{3.094104in}}%
\pgfpathlineto{\pgfqpoint{1.625158in}{3.066506in}}%
\pgfpathclose%
\pgfusepath{stroke,fill}%
\end{pgfscope}%
\begin{pgfscope}%
\pgfpathrectangle{\pgfqpoint{1.137973in}{3.066506in}}{\pgfqpoint{0.824468in}{0.243158in}}%
\pgfusepath{clip}%
\pgfsetbuttcap%
\pgfsetmiterjoin%
\definecolor{currentfill}{rgb}{0.121569,0.466667,0.705882}%
\pgfsetfillcolor{currentfill}%
\pgfsetfillopacity{0.500000}%
\pgfsetlinewidth{1.003750pt}%
\definecolor{currentstroke}{rgb}{0.000000,0.000000,0.000000}%
\pgfsetstrokecolor{currentstroke}%
\pgfsetdash{}{0pt}%
\pgfpathmoveto{\pgfqpoint{1.775062in}{3.066506in}}%
\pgfpathlineto{\pgfqpoint{1.924965in}{3.066506in}}%
\pgfpathlineto{\pgfqpoint{1.924965in}{3.085351in}}%
\pgfpathlineto{\pgfqpoint{1.775062in}{3.085351in}}%
\pgfpathlineto{\pgfqpoint{1.775062in}{3.066506in}}%
\pgfpathclose%
\pgfusepath{stroke,fill}%
\end{pgfscope}%
\begin{pgfscope}%
\pgfsetrectcap%
\pgfsetmiterjoin%
\pgfsetlinewidth{0.803000pt}%
\definecolor{currentstroke}{rgb}{0.000000,0.000000,0.000000}%
\pgfsetstrokecolor{currentstroke}%
\pgfsetdash{}{0pt}%
\pgfpathmoveto{\pgfqpoint{1.137973in}{3.066506in}}%
\pgfpathlineto{\pgfqpoint{1.137973in}{3.309664in}}%
\pgfusepath{stroke}%
\end{pgfscope}%
\begin{pgfscope}%
\pgfsetrectcap%
\pgfsetmiterjoin%
\pgfsetlinewidth{0.803000pt}%
\definecolor{currentstroke}{rgb}{0.000000,0.000000,0.000000}%
\pgfsetstrokecolor{currentstroke}%
\pgfsetdash{}{0pt}%
\pgfpathmoveto{\pgfqpoint{1.962441in}{3.066506in}}%
\pgfpathlineto{\pgfqpoint{1.962441in}{3.309664in}}%
\pgfusepath{stroke}%
\end{pgfscope}%
\begin{pgfscope}%
\pgfsetrectcap%
\pgfsetmiterjoin%
\pgfsetlinewidth{0.803000pt}%
\definecolor{currentstroke}{rgb}{0.000000,0.000000,0.000000}%
\pgfsetstrokecolor{currentstroke}%
\pgfsetdash{}{0pt}%
\pgfpathmoveto{\pgfqpoint{1.137973in}{3.066506in}}%
\pgfpathlineto{\pgfqpoint{1.962441in}{3.066506in}}%
\pgfusepath{stroke}%
\end{pgfscope}%
\begin{pgfscope}%
\pgfsetrectcap%
\pgfsetmiterjoin%
\pgfsetlinewidth{0.803000pt}%
\definecolor{currentstroke}{rgb}{0.000000,0.000000,0.000000}%
\pgfsetstrokecolor{currentstroke}%
\pgfsetdash{}{0pt}%
\pgfpathmoveto{\pgfqpoint{1.137973in}{3.309664in}}%
\pgfpathlineto{\pgfqpoint{1.962441in}{3.309664in}}%
\pgfusepath{stroke}%
\end{pgfscope}%
\begin{pgfscope}%
\definecolor{textcolor}{rgb}{0.000000,0.000000,0.000000}%
\pgfsetstrokecolor{textcolor}%
\pgfsetfillcolor{textcolor}%
\pgftext[x=1.550207in,y=3.392997in,,base]{\color{textcolor}\rmfamily\fontsize{11.000000}{13.200000}\selectfont GMF}%
\end{pgfscope}%
\begin{pgfscope}%
\pgfsetbuttcap%
\pgfsetmiterjoin%
\definecolor{currentfill}{rgb}{1.000000,1.000000,1.000000}%
\pgfsetfillcolor{currentfill}%
\pgfsetlinewidth{0.000000pt}%
\definecolor{currentstroke}{rgb}{0.000000,0.000000,0.000000}%
\pgfsetstrokecolor{currentstroke}%
\pgfsetstrokeopacity{0.000000}%
\pgfsetdash{}{0pt}%
\pgfpathmoveto{\pgfqpoint{2.127335in}{3.066506in}}%
\pgfpathlineto{\pgfqpoint{2.951803in}{3.066506in}}%
\pgfpathlineto{\pgfqpoint{2.951803in}{3.309664in}}%
\pgfpathlineto{\pgfqpoint{2.127335in}{3.309664in}}%
\pgfpathlineto{\pgfqpoint{2.127335in}{3.066506in}}%
\pgfpathclose%
\pgfusepath{fill}%
\end{pgfscope}%
\begin{pgfscope}%
\pgfpathrectangle{\pgfqpoint{2.127335in}{3.066506in}}{\pgfqpoint{0.824468in}{0.243158in}}%
\pgfusepath{clip}%
\pgfsetbuttcap%
\pgfsetmiterjoin%
\definecolor{currentfill}{rgb}{0.121569,0.466667,0.705882}%
\pgfsetfillcolor{currentfill}%
\pgfsetfillopacity{0.500000}%
\pgfsetlinewidth{1.003750pt}%
\definecolor{currentstroke}{rgb}{0.000000,0.000000,0.000000}%
\pgfsetstrokecolor{currentstroke}%
\pgfsetdash{}{0pt}%
\pgfpathmoveto{\pgfqpoint{2.164810in}{3.066506in}}%
\pgfpathlineto{\pgfqpoint{2.314714in}{3.066506in}}%
\pgfpathlineto{\pgfqpoint{2.314714in}{3.075503in}}%
\pgfpathlineto{\pgfqpoint{2.164810in}{3.075503in}}%
\pgfpathlineto{\pgfqpoint{2.164810in}{3.066506in}}%
\pgfpathclose%
\pgfusepath{stroke,fill}%
\end{pgfscope}%
\begin{pgfscope}%
\pgfpathrectangle{\pgfqpoint{2.127335in}{3.066506in}}{\pgfqpoint{0.824468in}{0.243158in}}%
\pgfusepath{clip}%
\pgfsetbuttcap%
\pgfsetmiterjoin%
\definecolor{currentfill}{rgb}{0.121569,0.466667,0.705882}%
\pgfsetfillcolor{currentfill}%
\pgfsetfillopacity{0.500000}%
\pgfsetlinewidth{1.003750pt}%
\definecolor{currentstroke}{rgb}{0.000000,0.000000,0.000000}%
\pgfsetstrokecolor{currentstroke}%
\pgfsetdash{}{0pt}%
\pgfpathmoveto{\pgfqpoint{2.314714in}{3.066506in}}%
\pgfpathlineto{\pgfqpoint{2.464617in}{3.066506in}}%
\pgfpathlineto{\pgfqpoint{2.464617in}{3.076962in}}%
\pgfpathlineto{\pgfqpoint{2.314714in}{3.076962in}}%
\pgfpathlineto{\pgfqpoint{2.314714in}{3.066506in}}%
\pgfpathclose%
\pgfusepath{stroke,fill}%
\end{pgfscope}%
\begin{pgfscope}%
\pgfpathrectangle{\pgfqpoint{2.127335in}{3.066506in}}{\pgfqpoint{0.824468in}{0.243158in}}%
\pgfusepath{clip}%
\pgfsetbuttcap%
\pgfsetmiterjoin%
\definecolor{currentfill}{rgb}{0.121569,0.466667,0.705882}%
\pgfsetfillcolor{currentfill}%
\pgfsetfillopacity{0.500000}%
\pgfsetlinewidth{1.003750pt}%
\definecolor{currentstroke}{rgb}{0.000000,0.000000,0.000000}%
\pgfsetstrokecolor{currentstroke}%
\pgfsetdash{}{0pt}%
\pgfpathmoveto{\pgfqpoint{2.464617in}{3.066506in}}%
\pgfpathlineto{\pgfqpoint{2.614520in}{3.066506in}}%
\pgfpathlineto{\pgfqpoint{2.614520in}{3.076232in}}%
\pgfpathlineto{\pgfqpoint{2.464617in}{3.076232in}}%
\pgfpathlineto{\pgfqpoint{2.464617in}{3.066506in}}%
\pgfpathclose%
\pgfusepath{stroke,fill}%
\end{pgfscope}%
\begin{pgfscope}%
\pgfpathrectangle{\pgfqpoint{2.127335in}{3.066506in}}{\pgfqpoint{0.824468in}{0.243158in}}%
\pgfusepath{clip}%
\pgfsetbuttcap%
\pgfsetmiterjoin%
\definecolor{currentfill}{rgb}{0.121569,0.466667,0.705882}%
\pgfsetfillcolor{currentfill}%
\pgfsetfillopacity{0.500000}%
\pgfsetlinewidth{1.003750pt}%
\definecolor{currentstroke}{rgb}{0.000000,0.000000,0.000000}%
\pgfsetstrokecolor{currentstroke}%
\pgfsetdash{}{0pt}%
\pgfpathmoveto{\pgfqpoint{2.614520in}{3.066506in}}%
\pgfpathlineto{\pgfqpoint{2.764423in}{3.066506in}}%
\pgfpathlineto{\pgfqpoint{2.764423in}{3.089120in}}%
\pgfpathlineto{\pgfqpoint{2.614520in}{3.089120in}}%
\pgfpathlineto{\pgfqpoint{2.614520in}{3.066506in}}%
\pgfpathclose%
\pgfusepath{stroke,fill}%
\end{pgfscope}%
\begin{pgfscope}%
\pgfpathrectangle{\pgfqpoint{2.127335in}{3.066506in}}{\pgfqpoint{0.824468in}{0.243158in}}%
\pgfusepath{clip}%
\pgfsetbuttcap%
\pgfsetmiterjoin%
\definecolor{currentfill}{rgb}{0.121569,0.466667,0.705882}%
\pgfsetfillcolor{currentfill}%
\pgfsetfillopacity{0.500000}%
\pgfsetlinewidth{1.003750pt}%
\definecolor{currentstroke}{rgb}{0.000000,0.000000,0.000000}%
\pgfsetstrokecolor{currentstroke}%
\pgfsetdash{}{0pt}%
\pgfpathmoveto{\pgfqpoint{2.764423in}{3.066506in}}%
\pgfpathlineto{\pgfqpoint{2.914327in}{3.066506in}}%
\pgfpathlineto{\pgfqpoint{2.914327in}{3.098360in}}%
\pgfpathlineto{\pgfqpoint{2.764423in}{3.098360in}}%
\pgfpathlineto{\pgfqpoint{2.764423in}{3.066506in}}%
\pgfpathclose%
\pgfusepath{stroke,fill}%
\end{pgfscope}%
\begin{pgfscope}%
\pgfsetrectcap%
\pgfsetmiterjoin%
\pgfsetlinewidth{0.803000pt}%
\definecolor{currentstroke}{rgb}{0.000000,0.000000,0.000000}%
\pgfsetstrokecolor{currentstroke}%
\pgfsetdash{}{0pt}%
\pgfpathmoveto{\pgfqpoint{2.127335in}{3.066506in}}%
\pgfpathlineto{\pgfqpoint{2.127335in}{3.309664in}}%
\pgfusepath{stroke}%
\end{pgfscope}%
\begin{pgfscope}%
\pgfsetrectcap%
\pgfsetmiterjoin%
\pgfsetlinewidth{0.803000pt}%
\definecolor{currentstroke}{rgb}{0.000000,0.000000,0.000000}%
\pgfsetstrokecolor{currentstroke}%
\pgfsetdash{}{0pt}%
\pgfpathmoveto{\pgfqpoint{2.951803in}{3.066506in}}%
\pgfpathlineto{\pgfqpoint{2.951803in}{3.309664in}}%
\pgfusepath{stroke}%
\end{pgfscope}%
\begin{pgfscope}%
\pgfsetrectcap%
\pgfsetmiterjoin%
\pgfsetlinewidth{0.803000pt}%
\definecolor{currentstroke}{rgb}{0.000000,0.000000,0.000000}%
\pgfsetstrokecolor{currentstroke}%
\pgfsetdash{}{0pt}%
\pgfpathmoveto{\pgfqpoint{2.127335in}{3.066506in}}%
\pgfpathlineto{\pgfqpoint{2.951803in}{3.066506in}}%
\pgfusepath{stroke}%
\end{pgfscope}%
\begin{pgfscope}%
\pgfsetrectcap%
\pgfsetmiterjoin%
\pgfsetlinewidth{0.803000pt}%
\definecolor{currentstroke}{rgb}{0.000000,0.000000,0.000000}%
\pgfsetstrokecolor{currentstroke}%
\pgfsetdash{}{0pt}%
\pgfpathmoveto{\pgfqpoint{2.127335in}{3.309664in}}%
\pgfpathlineto{\pgfqpoint{2.951803in}{3.309664in}}%
\pgfusepath{stroke}%
\end{pgfscope}%
\begin{pgfscope}%
\definecolor{textcolor}{rgb}{0.000000,0.000000,0.000000}%
\pgfsetstrokecolor{textcolor}%
\pgfsetfillcolor{textcolor}%
\pgftext[x=2.539569in,y=3.392997in,,base]{\color{textcolor}\rmfamily\fontsize{11.000000}{13.200000}\selectfont AMV}%
\end{pgfscope}%
\begin{pgfscope}%
\pgfsetbuttcap%
\pgfsetmiterjoin%
\definecolor{currentfill}{rgb}{1.000000,1.000000,1.000000}%
\pgfsetfillcolor{currentfill}%
\pgfsetlinewidth{0.000000pt}%
\definecolor{currentstroke}{rgb}{0.000000,0.000000,0.000000}%
\pgfsetstrokecolor{currentstroke}%
\pgfsetstrokeopacity{0.000000}%
\pgfsetdash{}{0pt}%
\pgfpathmoveto{\pgfqpoint{3.116696in}{3.066506in}}%
\pgfpathlineto{\pgfqpoint{3.941164in}{3.066506in}}%
\pgfpathlineto{\pgfqpoint{3.941164in}{3.309664in}}%
\pgfpathlineto{\pgfqpoint{3.116696in}{3.309664in}}%
\pgfpathlineto{\pgfqpoint{3.116696in}{3.066506in}}%
\pgfpathclose%
\pgfusepath{fill}%
\end{pgfscope}%
\begin{pgfscope}%
\pgfpathrectangle{\pgfqpoint{3.116696in}{3.066506in}}{\pgfqpoint{0.824468in}{0.243158in}}%
\pgfusepath{clip}%
\pgfsetbuttcap%
\pgfsetmiterjoin%
\definecolor{currentfill}{rgb}{0.121569,0.466667,0.705882}%
\pgfsetfillcolor{currentfill}%
\pgfsetfillopacity{0.500000}%
\pgfsetlinewidth{1.003750pt}%
\definecolor{currentstroke}{rgb}{0.000000,0.000000,0.000000}%
\pgfsetstrokecolor{currentstroke}%
\pgfsetdash{}{0pt}%
\pgfpathmoveto{\pgfqpoint{3.154172in}{3.066506in}}%
\pgfpathlineto{\pgfqpoint{3.304075in}{3.066506in}}%
\pgfpathlineto{\pgfqpoint{3.304075in}{3.080001in}}%
\pgfpathlineto{\pgfqpoint{3.154172in}{3.080001in}}%
\pgfpathlineto{\pgfqpoint{3.154172in}{3.066506in}}%
\pgfpathclose%
\pgfusepath{stroke,fill}%
\end{pgfscope}%
\begin{pgfscope}%
\pgfpathrectangle{\pgfqpoint{3.116696in}{3.066506in}}{\pgfqpoint{0.824468in}{0.243158in}}%
\pgfusepath{clip}%
\pgfsetbuttcap%
\pgfsetmiterjoin%
\definecolor{currentfill}{rgb}{0.121569,0.466667,0.705882}%
\pgfsetfillcolor{currentfill}%
\pgfsetfillopacity{0.500000}%
\pgfsetlinewidth{1.003750pt}%
\definecolor{currentstroke}{rgb}{0.000000,0.000000,0.000000}%
\pgfsetstrokecolor{currentstroke}%
\pgfsetdash{}{0pt}%
\pgfpathmoveto{\pgfqpoint{3.304075in}{3.066506in}}%
\pgfpathlineto{\pgfqpoint{3.453979in}{3.066506in}}%
\pgfpathlineto{\pgfqpoint{3.453979in}{3.068573in}}%
\pgfpathlineto{\pgfqpoint{3.304075in}{3.068573in}}%
\pgfpathlineto{\pgfqpoint{3.304075in}{3.066506in}}%
\pgfpathclose%
\pgfusepath{stroke,fill}%
\end{pgfscope}%
\begin{pgfscope}%
\pgfpathrectangle{\pgfqpoint{3.116696in}{3.066506in}}{\pgfqpoint{0.824468in}{0.243158in}}%
\pgfusepath{clip}%
\pgfsetbuttcap%
\pgfsetmiterjoin%
\definecolor{currentfill}{rgb}{0.121569,0.466667,0.705882}%
\pgfsetfillcolor{currentfill}%
\pgfsetfillopacity{0.500000}%
\pgfsetlinewidth{1.003750pt}%
\definecolor{currentstroke}{rgb}{0.000000,0.000000,0.000000}%
\pgfsetstrokecolor{currentstroke}%
\pgfsetdash{}{0pt}%
\pgfpathmoveto{\pgfqpoint{3.453979in}{3.066506in}}%
\pgfpathlineto{\pgfqpoint{3.603882in}{3.066506in}}%
\pgfpathlineto{\pgfqpoint{3.603882in}{3.068330in}}%
\pgfpathlineto{\pgfqpoint{3.453979in}{3.068330in}}%
\pgfpathlineto{\pgfqpoint{3.453979in}{3.066506in}}%
\pgfpathclose%
\pgfusepath{stroke,fill}%
\end{pgfscope}%
\begin{pgfscope}%
\pgfpathrectangle{\pgfqpoint{3.116696in}{3.066506in}}{\pgfqpoint{0.824468in}{0.243158in}}%
\pgfusepath{clip}%
\pgfsetbuttcap%
\pgfsetmiterjoin%
\definecolor{currentfill}{rgb}{0.121569,0.466667,0.705882}%
\pgfsetfillcolor{currentfill}%
\pgfsetfillopacity{0.500000}%
\pgfsetlinewidth{1.003750pt}%
\definecolor{currentstroke}{rgb}{0.000000,0.000000,0.000000}%
\pgfsetstrokecolor{currentstroke}%
\pgfsetdash{}{0pt}%
\pgfpathmoveto{\pgfqpoint{3.603882in}{3.066506in}}%
\pgfpathlineto{\pgfqpoint{3.753785in}{3.066506in}}%
\pgfpathlineto{\pgfqpoint{3.753785in}{3.066506in}}%
\pgfpathlineto{\pgfqpoint{3.603882in}{3.066506in}}%
\pgfpathlineto{\pgfqpoint{3.603882in}{3.066506in}}%
\pgfpathclose%
\pgfusepath{stroke,fill}%
\end{pgfscope}%
\begin{pgfscope}%
\pgfpathrectangle{\pgfqpoint{3.116696in}{3.066506in}}{\pgfqpoint{0.824468in}{0.243158in}}%
\pgfusepath{clip}%
\pgfsetbuttcap%
\pgfsetmiterjoin%
\definecolor{currentfill}{rgb}{0.121569,0.466667,0.705882}%
\pgfsetfillcolor{currentfill}%
\pgfsetfillopacity{0.500000}%
\pgfsetlinewidth{1.003750pt}%
\definecolor{currentstroke}{rgb}{0.000000,0.000000,0.000000}%
\pgfsetstrokecolor{currentstroke}%
\pgfsetdash{}{0pt}%
\pgfpathmoveto{\pgfqpoint{3.753785in}{3.066506in}}%
\pgfpathlineto{\pgfqpoint{3.903688in}{3.066506in}}%
\pgfpathlineto{\pgfqpoint{3.903688in}{3.066992in}}%
\pgfpathlineto{\pgfqpoint{3.753785in}{3.066992in}}%
\pgfpathlineto{\pgfqpoint{3.753785in}{3.066506in}}%
\pgfpathclose%
\pgfusepath{stroke,fill}%
\end{pgfscope}%
\begin{pgfscope}%
\pgfsetrectcap%
\pgfsetmiterjoin%
\pgfsetlinewidth{0.803000pt}%
\definecolor{currentstroke}{rgb}{0.000000,0.000000,0.000000}%
\pgfsetstrokecolor{currentstroke}%
\pgfsetdash{}{0pt}%
\pgfpathmoveto{\pgfqpoint{3.116696in}{3.066506in}}%
\pgfpathlineto{\pgfqpoint{3.116696in}{3.309664in}}%
\pgfusepath{stroke}%
\end{pgfscope}%
\begin{pgfscope}%
\pgfsetrectcap%
\pgfsetmiterjoin%
\pgfsetlinewidth{0.803000pt}%
\definecolor{currentstroke}{rgb}{0.000000,0.000000,0.000000}%
\pgfsetstrokecolor{currentstroke}%
\pgfsetdash{}{0pt}%
\pgfpathmoveto{\pgfqpoint{3.941164in}{3.066506in}}%
\pgfpathlineto{\pgfqpoint{3.941164in}{3.309664in}}%
\pgfusepath{stroke}%
\end{pgfscope}%
\begin{pgfscope}%
\pgfsetrectcap%
\pgfsetmiterjoin%
\pgfsetlinewidth{0.803000pt}%
\definecolor{currentstroke}{rgb}{0.000000,0.000000,0.000000}%
\pgfsetstrokecolor{currentstroke}%
\pgfsetdash{}{0pt}%
\pgfpathmoveto{\pgfqpoint{3.116696in}{3.066506in}}%
\pgfpathlineto{\pgfqpoint{3.941164in}{3.066506in}}%
\pgfusepath{stroke}%
\end{pgfscope}%
\begin{pgfscope}%
\pgfsetrectcap%
\pgfsetmiterjoin%
\pgfsetlinewidth{0.803000pt}%
\definecolor{currentstroke}{rgb}{0.000000,0.000000,0.000000}%
\pgfsetstrokecolor{currentstroke}%
\pgfsetdash{}{0pt}%
\pgfpathmoveto{\pgfqpoint{3.116696in}{3.309664in}}%
\pgfpathlineto{\pgfqpoint{3.941164in}{3.309664in}}%
\pgfusepath{stroke}%
\end{pgfscope}%
\begin{pgfscope}%
\definecolor{textcolor}{rgb}{0.000000,0.000000,0.000000}%
\pgfsetstrokecolor{textcolor}%
\pgfsetfillcolor{textcolor}%
\pgftext[x=3.528930in,y=3.392997in,,base]{\color{textcolor}\rmfamily\fontsize{11.000000}{13.200000}\selectfont CNP As...}%
\end{pgfscope}%
\begin{pgfscope}%
\pgfsetbuttcap%
\pgfsetmiterjoin%
\definecolor{currentfill}{rgb}{1.000000,1.000000,1.000000}%
\pgfsetfillcolor{currentfill}%
\pgfsetlinewidth{0.000000pt}%
\definecolor{currentstroke}{rgb}{0.000000,0.000000,0.000000}%
\pgfsetstrokecolor{currentstroke}%
\pgfsetstrokeopacity{0.000000}%
\pgfsetdash{}{0pt}%
\pgfpathmoveto{\pgfqpoint{4.106058in}{3.066506in}}%
\pgfpathlineto{\pgfqpoint{4.930526in}{3.066506in}}%
\pgfpathlineto{\pgfqpoint{4.930526in}{3.309664in}}%
\pgfpathlineto{\pgfqpoint{4.106058in}{3.309664in}}%
\pgfpathlineto{\pgfqpoint{4.106058in}{3.066506in}}%
\pgfpathclose%
\pgfusepath{fill}%
\end{pgfscope}%
\begin{pgfscope}%
\pgfpathrectangle{\pgfqpoint{4.106058in}{3.066506in}}{\pgfqpoint{0.824468in}{0.243158in}}%
\pgfusepath{clip}%
\pgfsetbuttcap%
\pgfsetmiterjoin%
\definecolor{currentfill}{rgb}{0.121569,0.466667,0.705882}%
\pgfsetfillcolor{currentfill}%
\pgfsetfillopacity{0.500000}%
\pgfsetlinewidth{1.003750pt}%
\definecolor{currentstroke}{rgb}{0.000000,0.000000,0.000000}%
\pgfsetstrokecolor{currentstroke}%
\pgfsetdash{}{0pt}%
\pgfpathmoveto{\pgfqpoint{4.143534in}{3.066506in}}%
\pgfpathlineto{\pgfqpoint{4.293437in}{3.066506in}}%
\pgfpathlineto{\pgfqpoint{4.293437in}{3.104682in}}%
\pgfpathlineto{\pgfqpoint{4.143534in}{3.104682in}}%
\pgfpathlineto{\pgfqpoint{4.143534in}{3.066506in}}%
\pgfpathclose%
\pgfusepath{stroke,fill}%
\end{pgfscope}%
\begin{pgfscope}%
\pgfpathrectangle{\pgfqpoint{4.106058in}{3.066506in}}{\pgfqpoint{0.824468in}{0.243158in}}%
\pgfusepath{clip}%
\pgfsetbuttcap%
\pgfsetmiterjoin%
\definecolor{currentfill}{rgb}{0.121569,0.466667,0.705882}%
\pgfsetfillcolor{currentfill}%
\pgfsetfillopacity{0.500000}%
\pgfsetlinewidth{1.003750pt}%
\definecolor{currentstroke}{rgb}{0.000000,0.000000,0.000000}%
\pgfsetstrokecolor{currentstroke}%
\pgfsetdash{}{0pt}%
\pgfpathmoveto{\pgfqpoint{4.293437in}{3.066506in}}%
\pgfpathlineto{\pgfqpoint{4.443340in}{3.066506in}}%
\pgfpathlineto{\pgfqpoint{4.443340in}{3.084135in}}%
\pgfpathlineto{\pgfqpoint{4.293437in}{3.084135in}}%
\pgfpathlineto{\pgfqpoint{4.293437in}{3.066506in}}%
\pgfpathclose%
\pgfusepath{stroke,fill}%
\end{pgfscope}%
\begin{pgfscope}%
\pgfpathrectangle{\pgfqpoint{4.106058in}{3.066506in}}{\pgfqpoint{0.824468in}{0.243158in}}%
\pgfusepath{clip}%
\pgfsetbuttcap%
\pgfsetmiterjoin%
\definecolor{currentfill}{rgb}{0.121569,0.466667,0.705882}%
\pgfsetfillcolor{currentfill}%
\pgfsetfillopacity{0.500000}%
\pgfsetlinewidth{1.003750pt}%
\definecolor{currentstroke}{rgb}{0.000000,0.000000,0.000000}%
\pgfsetstrokecolor{currentstroke}%
\pgfsetdash{}{0pt}%
\pgfpathmoveto{\pgfqpoint{4.443340in}{3.066506in}}%
\pgfpathlineto{\pgfqpoint{4.593244in}{3.066506in}}%
\pgfpathlineto{\pgfqpoint{4.593244in}{3.074895in}}%
\pgfpathlineto{\pgfqpoint{4.443340in}{3.074895in}}%
\pgfpathlineto{\pgfqpoint{4.443340in}{3.066506in}}%
\pgfpathclose%
\pgfusepath{stroke,fill}%
\end{pgfscope}%
\begin{pgfscope}%
\pgfpathrectangle{\pgfqpoint{4.106058in}{3.066506in}}{\pgfqpoint{0.824468in}{0.243158in}}%
\pgfusepath{clip}%
\pgfsetbuttcap%
\pgfsetmiterjoin%
\definecolor{currentfill}{rgb}{0.121569,0.466667,0.705882}%
\pgfsetfillcolor{currentfill}%
\pgfsetfillopacity{0.500000}%
\pgfsetlinewidth{1.003750pt}%
\definecolor{currentstroke}{rgb}{0.000000,0.000000,0.000000}%
\pgfsetstrokecolor{currentstroke}%
\pgfsetdash{}{0pt}%
\pgfpathmoveto{\pgfqpoint{4.593244in}{3.066506in}}%
\pgfpathlineto{\pgfqpoint{4.743147in}{3.066506in}}%
\pgfpathlineto{\pgfqpoint{4.743147in}{3.071491in}}%
\pgfpathlineto{\pgfqpoint{4.593244in}{3.071491in}}%
\pgfpathlineto{\pgfqpoint{4.593244in}{3.066506in}}%
\pgfpathclose%
\pgfusepath{stroke,fill}%
\end{pgfscope}%
\begin{pgfscope}%
\pgfpathrectangle{\pgfqpoint{4.106058in}{3.066506in}}{\pgfqpoint{0.824468in}{0.243158in}}%
\pgfusepath{clip}%
\pgfsetbuttcap%
\pgfsetmiterjoin%
\definecolor{currentfill}{rgb}{0.121569,0.466667,0.705882}%
\pgfsetfillcolor{currentfill}%
\pgfsetfillopacity{0.500000}%
\pgfsetlinewidth{1.003750pt}%
\definecolor{currentstroke}{rgb}{0.000000,0.000000,0.000000}%
\pgfsetstrokecolor{currentstroke}%
\pgfsetdash{}{0pt}%
\pgfpathmoveto{\pgfqpoint{4.743147in}{3.066506in}}%
\pgfpathlineto{\pgfqpoint{4.893050in}{3.066506in}}%
\pgfpathlineto{\pgfqpoint{4.893050in}{3.069181in}}%
\pgfpathlineto{\pgfqpoint{4.743147in}{3.069181in}}%
\pgfpathlineto{\pgfqpoint{4.743147in}{3.066506in}}%
\pgfpathclose%
\pgfusepath{stroke,fill}%
\end{pgfscope}%
\begin{pgfscope}%
\pgfsetrectcap%
\pgfsetmiterjoin%
\pgfsetlinewidth{0.803000pt}%
\definecolor{currentstroke}{rgb}{0.000000,0.000000,0.000000}%
\pgfsetstrokecolor{currentstroke}%
\pgfsetdash{}{0pt}%
\pgfpathmoveto{\pgfqpoint{4.106058in}{3.066506in}}%
\pgfpathlineto{\pgfqpoint{4.106058in}{3.309664in}}%
\pgfusepath{stroke}%
\end{pgfscope}%
\begin{pgfscope}%
\pgfsetrectcap%
\pgfsetmiterjoin%
\pgfsetlinewidth{0.803000pt}%
\definecolor{currentstroke}{rgb}{0.000000,0.000000,0.000000}%
\pgfsetstrokecolor{currentstroke}%
\pgfsetdash{}{0pt}%
\pgfpathmoveto{\pgfqpoint{4.930526in}{3.066506in}}%
\pgfpathlineto{\pgfqpoint{4.930526in}{3.309664in}}%
\pgfusepath{stroke}%
\end{pgfscope}%
\begin{pgfscope}%
\pgfsetrectcap%
\pgfsetmiterjoin%
\pgfsetlinewidth{0.803000pt}%
\definecolor{currentstroke}{rgb}{0.000000,0.000000,0.000000}%
\pgfsetstrokecolor{currentstroke}%
\pgfsetdash{}{0pt}%
\pgfpathmoveto{\pgfqpoint{4.106058in}{3.066506in}}%
\pgfpathlineto{\pgfqpoint{4.930526in}{3.066506in}}%
\pgfusepath{stroke}%
\end{pgfscope}%
\begin{pgfscope}%
\pgfsetrectcap%
\pgfsetmiterjoin%
\pgfsetlinewidth{0.803000pt}%
\definecolor{currentstroke}{rgb}{0.000000,0.000000,0.000000}%
\pgfsetstrokecolor{currentstroke}%
\pgfsetdash{}{0pt}%
\pgfpathmoveto{\pgfqpoint{4.106058in}{3.309664in}}%
\pgfpathlineto{\pgfqpoint{4.930526in}{3.309664in}}%
\pgfusepath{stroke}%
\end{pgfscope}%
\begin{pgfscope}%
\definecolor{textcolor}{rgb}{0.000000,0.000000,0.000000}%
\pgfsetstrokecolor{textcolor}%
\pgfsetfillcolor{textcolor}%
\pgftext[x=4.518292in,y=3.392997in,,base]{\color{textcolor}\rmfamily\fontsize{11.000000}{13.200000}\selectfont MAIF}%
\end{pgfscope}%
\begin{pgfscope}%
\pgfsetbuttcap%
\pgfsetmiterjoin%
\definecolor{currentfill}{rgb}{1.000000,1.000000,1.000000}%
\pgfsetfillcolor{currentfill}%
\pgfsetlinewidth{0.000000pt}%
\definecolor{currentstroke}{rgb}{0.000000,0.000000,0.000000}%
\pgfsetstrokecolor{currentstroke}%
\pgfsetstrokeopacity{0.000000}%
\pgfsetdash{}{0pt}%
\pgfpathmoveto{\pgfqpoint{5.095420in}{3.066506in}}%
\pgfpathlineto{\pgfqpoint{5.919888in}{3.066506in}}%
\pgfpathlineto{\pgfqpoint{5.919888in}{3.309664in}}%
\pgfpathlineto{\pgfqpoint{5.095420in}{3.309664in}}%
\pgfpathlineto{\pgfqpoint{5.095420in}{3.066506in}}%
\pgfpathclose%
\pgfusepath{fill}%
\end{pgfscope}%
\begin{pgfscope}%
\pgfpathrectangle{\pgfqpoint{5.095420in}{3.066506in}}{\pgfqpoint{0.824468in}{0.243158in}}%
\pgfusepath{clip}%
\pgfsetbuttcap%
\pgfsetmiterjoin%
\definecolor{currentfill}{rgb}{0.121569,0.466667,0.705882}%
\pgfsetfillcolor{currentfill}%
\pgfsetfillopacity{0.500000}%
\pgfsetlinewidth{1.003750pt}%
\definecolor{currentstroke}{rgb}{0.000000,0.000000,0.000000}%
\pgfsetstrokecolor{currentstroke}%
\pgfsetdash{}{0pt}%
\pgfpathmoveto{\pgfqpoint{5.132895in}{3.066506in}}%
\pgfpathlineto{\pgfqpoint{5.282799in}{3.066506in}}%
\pgfpathlineto{\pgfqpoint{5.282799in}{3.072463in}}%
\pgfpathlineto{\pgfqpoint{5.132895in}{3.072463in}}%
\pgfpathlineto{\pgfqpoint{5.132895in}{3.066506in}}%
\pgfpathclose%
\pgfusepath{stroke,fill}%
\end{pgfscope}%
\begin{pgfscope}%
\pgfpathrectangle{\pgfqpoint{5.095420in}{3.066506in}}{\pgfqpoint{0.824468in}{0.243158in}}%
\pgfusepath{clip}%
\pgfsetbuttcap%
\pgfsetmiterjoin%
\definecolor{currentfill}{rgb}{0.121569,0.466667,0.705882}%
\pgfsetfillcolor{currentfill}%
\pgfsetfillopacity{0.500000}%
\pgfsetlinewidth{1.003750pt}%
\definecolor{currentstroke}{rgb}{0.000000,0.000000,0.000000}%
\pgfsetstrokecolor{currentstroke}%
\pgfsetdash{}{0pt}%
\pgfpathmoveto{\pgfqpoint{5.282799in}{3.066506in}}%
\pgfpathlineto{\pgfqpoint{5.432702in}{3.066506in}}%
\pgfpathlineto{\pgfqpoint{5.432702in}{3.067965in}}%
\pgfpathlineto{\pgfqpoint{5.282799in}{3.067965in}}%
\pgfpathlineto{\pgfqpoint{5.282799in}{3.066506in}}%
\pgfpathclose%
\pgfusepath{stroke,fill}%
\end{pgfscope}%
\begin{pgfscope}%
\pgfpathrectangle{\pgfqpoint{5.095420in}{3.066506in}}{\pgfqpoint{0.824468in}{0.243158in}}%
\pgfusepath{clip}%
\pgfsetbuttcap%
\pgfsetmiterjoin%
\definecolor{currentfill}{rgb}{0.121569,0.466667,0.705882}%
\pgfsetfillcolor{currentfill}%
\pgfsetfillopacity{0.500000}%
\pgfsetlinewidth{1.003750pt}%
\definecolor{currentstroke}{rgb}{0.000000,0.000000,0.000000}%
\pgfsetstrokecolor{currentstroke}%
\pgfsetdash{}{0pt}%
\pgfpathmoveto{\pgfqpoint{5.432702in}{3.066506in}}%
\pgfpathlineto{\pgfqpoint{5.582605in}{3.066506in}}%
\pgfpathlineto{\pgfqpoint{5.582605in}{3.067235in}}%
\pgfpathlineto{\pgfqpoint{5.432702in}{3.067235in}}%
\pgfpathlineto{\pgfqpoint{5.432702in}{3.066506in}}%
\pgfpathclose%
\pgfusepath{stroke,fill}%
\end{pgfscope}%
\begin{pgfscope}%
\pgfpathrectangle{\pgfqpoint{5.095420in}{3.066506in}}{\pgfqpoint{0.824468in}{0.243158in}}%
\pgfusepath{clip}%
\pgfsetbuttcap%
\pgfsetmiterjoin%
\definecolor{currentfill}{rgb}{0.121569,0.466667,0.705882}%
\pgfsetfillcolor{currentfill}%
\pgfsetfillopacity{0.500000}%
\pgfsetlinewidth{1.003750pt}%
\definecolor{currentstroke}{rgb}{0.000000,0.000000,0.000000}%
\pgfsetstrokecolor{currentstroke}%
\pgfsetdash{}{0pt}%
\pgfpathmoveto{\pgfqpoint{5.582605in}{3.066506in}}%
\pgfpathlineto{\pgfqpoint{5.732509in}{3.066506in}}%
\pgfpathlineto{\pgfqpoint{5.732509in}{3.066871in}}%
\pgfpathlineto{\pgfqpoint{5.582605in}{3.066871in}}%
\pgfpathlineto{\pgfqpoint{5.582605in}{3.066506in}}%
\pgfpathclose%
\pgfusepath{stroke,fill}%
\end{pgfscope}%
\begin{pgfscope}%
\pgfpathrectangle{\pgfqpoint{5.095420in}{3.066506in}}{\pgfqpoint{0.824468in}{0.243158in}}%
\pgfusepath{clip}%
\pgfsetbuttcap%
\pgfsetmiterjoin%
\definecolor{currentfill}{rgb}{0.121569,0.466667,0.705882}%
\pgfsetfillcolor{currentfill}%
\pgfsetfillopacity{0.500000}%
\pgfsetlinewidth{1.003750pt}%
\definecolor{currentstroke}{rgb}{0.000000,0.000000,0.000000}%
\pgfsetstrokecolor{currentstroke}%
\pgfsetdash{}{0pt}%
\pgfpathmoveto{\pgfqpoint{5.732509in}{3.066506in}}%
\pgfpathlineto{\pgfqpoint{5.882412in}{3.066506in}}%
\pgfpathlineto{\pgfqpoint{5.882412in}{3.066749in}}%
\pgfpathlineto{\pgfqpoint{5.732509in}{3.066749in}}%
\pgfpathlineto{\pgfqpoint{5.732509in}{3.066506in}}%
\pgfpathclose%
\pgfusepath{stroke,fill}%
\end{pgfscope}%
\begin{pgfscope}%
\pgfsetrectcap%
\pgfsetmiterjoin%
\pgfsetlinewidth{0.803000pt}%
\definecolor{currentstroke}{rgb}{0.000000,0.000000,0.000000}%
\pgfsetstrokecolor{currentstroke}%
\pgfsetdash{}{0pt}%
\pgfpathmoveto{\pgfqpoint{5.095420in}{3.066506in}}%
\pgfpathlineto{\pgfqpoint{5.095420in}{3.309664in}}%
\pgfusepath{stroke}%
\end{pgfscope}%
\begin{pgfscope}%
\pgfsetrectcap%
\pgfsetmiterjoin%
\pgfsetlinewidth{0.803000pt}%
\definecolor{currentstroke}{rgb}{0.000000,0.000000,0.000000}%
\pgfsetstrokecolor{currentstroke}%
\pgfsetdash{}{0pt}%
\pgfpathmoveto{\pgfqpoint{5.919888in}{3.066506in}}%
\pgfpathlineto{\pgfqpoint{5.919888in}{3.309664in}}%
\pgfusepath{stroke}%
\end{pgfscope}%
\begin{pgfscope}%
\pgfsetrectcap%
\pgfsetmiterjoin%
\pgfsetlinewidth{0.803000pt}%
\definecolor{currentstroke}{rgb}{0.000000,0.000000,0.000000}%
\pgfsetstrokecolor{currentstroke}%
\pgfsetdash{}{0pt}%
\pgfpathmoveto{\pgfqpoint{5.095420in}{3.066506in}}%
\pgfpathlineto{\pgfqpoint{5.919888in}{3.066506in}}%
\pgfusepath{stroke}%
\end{pgfscope}%
\begin{pgfscope}%
\pgfsetrectcap%
\pgfsetmiterjoin%
\pgfsetlinewidth{0.803000pt}%
\definecolor{currentstroke}{rgb}{0.000000,0.000000,0.000000}%
\pgfsetstrokecolor{currentstroke}%
\pgfsetdash{}{0pt}%
\pgfpathmoveto{\pgfqpoint{5.095420in}{3.309664in}}%
\pgfpathlineto{\pgfqpoint{5.919888in}{3.309664in}}%
\pgfusepath{stroke}%
\end{pgfscope}%
\begin{pgfscope}%
\definecolor{textcolor}{rgb}{0.000000,0.000000,0.000000}%
\pgfsetstrokecolor{textcolor}%
\pgfsetfillcolor{textcolor}%
\pgftext[x=5.507654in,y=3.392997in,,base]{\color{textcolor}\rmfamily\fontsize{11.000000}{13.200000}\selectfont Sogecap}%
\end{pgfscope}%
\begin{pgfscope}%
\pgfsetbuttcap%
\pgfsetmiterjoin%
\definecolor{currentfill}{rgb}{1.000000,1.000000,1.000000}%
\pgfsetfillcolor{currentfill}%
\pgfsetlinewidth{0.000000pt}%
\definecolor{currentstroke}{rgb}{0.000000,0.000000,0.000000}%
\pgfsetstrokecolor{currentstroke}%
\pgfsetstrokeopacity{0.000000}%
\pgfsetdash{}{0pt}%
\pgfpathmoveto{\pgfqpoint{6.084781in}{3.066506in}}%
\pgfpathlineto{\pgfqpoint{6.909249in}{3.066506in}}%
\pgfpathlineto{\pgfqpoint{6.909249in}{3.309664in}}%
\pgfpathlineto{\pgfqpoint{6.084781in}{3.309664in}}%
\pgfpathlineto{\pgfqpoint{6.084781in}{3.066506in}}%
\pgfpathclose%
\pgfusepath{fill}%
\end{pgfscope}%
\begin{pgfscope}%
\pgfpathrectangle{\pgfqpoint{6.084781in}{3.066506in}}{\pgfqpoint{0.824468in}{0.243158in}}%
\pgfusepath{clip}%
\pgfsetbuttcap%
\pgfsetmiterjoin%
\definecolor{currentfill}{rgb}{0.121569,0.466667,0.705882}%
\pgfsetfillcolor{currentfill}%
\pgfsetfillopacity{0.500000}%
\pgfsetlinewidth{1.003750pt}%
\definecolor{currentstroke}{rgb}{0.000000,0.000000,0.000000}%
\pgfsetstrokecolor{currentstroke}%
\pgfsetdash{}{0pt}%
\pgfpathmoveto{\pgfqpoint{6.122257in}{3.066506in}}%
\pgfpathlineto{\pgfqpoint{6.272160in}{3.066506in}}%
\pgfpathlineto{\pgfqpoint{6.272160in}{3.094834in}}%
\pgfpathlineto{\pgfqpoint{6.122257in}{3.094834in}}%
\pgfpathlineto{\pgfqpoint{6.122257in}{3.066506in}}%
\pgfpathclose%
\pgfusepath{stroke,fill}%
\end{pgfscope}%
\begin{pgfscope}%
\pgfpathrectangle{\pgfqpoint{6.084781in}{3.066506in}}{\pgfqpoint{0.824468in}{0.243158in}}%
\pgfusepath{clip}%
\pgfsetbuttcap%
\pgfsetmiterjoin%
\definecolor{currentfill}{rgb}{0.121569,0.466667,0.705882}%
\pgfsetfillcolor{currentfill}%
\pgfsetfillopacity{0.500000}%
\pgfsetlinewidth{1.003750pt}%
\definecolor{currentstroke}{rgb}{0.000000,0.000000,0.000000}%
\pgfsetstrokecolor{currentstroke}%
\pgfsetdash{}{0pt}%
\pgfpathmoveto{\pgfqpoint{6.272160in}{3.066506in}}%
\pgfpathlineto{\pgfqpoint{6.422064in}{3.066506in}}%
\pgfpathlineto{\pgfqpoint{6.422064in}{3.072463in}}%
\pgfpathlineto{\pgfqpoint{6.272160in}{3.072463in}}%
\pgfpathlineto{\pgfqpoint{6.272160in}{3.066506in}}%
\pgfpathclose%
\pgfusepath{stroke,fill}%
\end{pgfscope}%
\begin{pgfscope}%
\pgfpathrectangle{\pgfqpoint{6.084781in}{3.066506in}}{\pgfqpoint{0.824468in}{0.243158in}}%
\pgfusepath{clip}%
\pgfsetbuttcap%
\pgfsetmiterjoin%
\definecolor{currentfill}{rgb}{0.121569,0.466667,0.705882}%
\pgfsetfillcolor{currentfill}%
\pgfsetfillopacity{0.500000}%
\pgfsetlinewidth{1.003750pt}%
\definecolor{currentstroke}{rgb}{0.000000,0.000000,0.000000}%
\pgfsetstrokecolor{currentstroke}%
\pgfsetdash{}{0pt}%
\pgfpathmoveto{\pgfqpoint{6.422064in}{3.066506in}}%
\pgfpathlineto{\pgfqpoint{6.571967in}{3.066506in}}%
\pgfpathlineto{\pgfqpoint{6.571967in}{3.069181in}}%
\pgfpathlineto{\pgfqpoint{6.422064in}{3.069181in}}%
\pgfpathlineto{\pgfqpoint{6.422064in}{3.066506in}}%
\pgfpathclose%
\pgfusepath{stroke,fill}%
\end{pgfscope}%
\begin{pgfscope}%
\pgfpathrectangle{\pgfqpoint{6.084781in}{3.066506in}}{\pgfqpoint{0.824468in}{0.243158in}}%
\pgfusepath{clip}%
\pgfsetbuttcap%
\pgfsetmiterjoin%
\definecolor{currentfill}{rgb}{0.121569,0.466667,0.705882}%
\pgfsetfillcolor{currentfill}%
\pgfsetfillopacity{0.500000}%
\pgfsetlinewidth{1.003750pt}%
\definecolor{currentstroke}{rgb}{0.000000,0.000000,0.000000}%
\pgfsetstrokecolor{currentstroke}%
\pgfsetdash{}{0pt}%
\pgfpathmoveto{\pgfqpoint{6.571967in}{3.066506in}}%
\pgfpathlineto{\pgfqpoint{6.721870in}{3.066506in}}%
\pgfpathlineto{\pgfqpoint{6.721870in}{3.066992in}}%
\pgfpathlineto{\pgfqpoint{6.571967in}{3.066992in}}%
\pgfpathlineto{\pgfqpoint{6.571967in}{3.066506in}}%
\pgfpathclose%
\pgfusepath{stroke,fill}%
\end{pgfscope}%
\begin{pgfscope}%
\pgfpathrectangle{\pgfqpoint{6.084781in}{3.066506in}}{\pgfqpoint{0.824468in}{0.243158in}}%
\pgfusepath{clip}%
\pgfsetbuttcap%
\pgfsetmiterjoin%
\definecolor{currentfill}{rgb}{0.121569,0.466667,0.705882}%
\pgfsetfillcolor{currentfill}%
\pgfsetfillopacity{0.500000}%
\pgfsetlinewidth{1.003750pt}%
\definecolor{currentstroke}{rgb}{0.000000,0.000000,0.000000}%
\pgfsetstrokecolor{currentstroke}%
\pgfsetdash{}{0pt}%
\pgfpathmoveto{\pgfqpoint{6.721870in}{3.066506in}}%
\pgfpathlineto{\pgfqpoint{6.871774in}{3.066506in}}%
\pgfpathlineto{\pgfqpoint{6.871774in}{3.066992in}}%
\pgfpathlineto{\pgfqpoint{6.721870in}{3.066992in}}%
\pgfpathlineto{\pgfqpoint{6.721870in}{3.066506in}}%
\pgfpathclose%
\pgfusepath{stroke,fill}%
\end{pgfscope}%
\begin{pgfscope}%
\pgfsetrectcap%
\pgfsetmiterjoin%
\pgfsetlinewidth{0.803000pt}%
\definecolor{currentstroke}{rgb}{0.000000,0.000000,0.000000}%
\pgfsetstrokecolor{currentstroke}%
\pgfsetdash{}{0pt}%
\pgfpathmoveto{\pgfqpoint{6.084781in}{3.066506in}}%
\pgfpathlineto{\pgfqpoint{6.084781in}{3.309664in}}%
\pgfusepath{stroke}%
\end{pgfscope}%
\begin{pgfscope}%
\pgfsetrectcap%
\pgfsetmiterjoin%
\pgfsetlinewidth{0.803000pt}%
\definecolor{currentstroke}{rgb}{0.000000,0.000000,0.000000}%
\pgfsetstrokecolor{currentstroke}%
\pgfsetdash{}{0pt}%
\pgfpathmoveto{\pgfqpoint{6.909249in}{3.066506in}}%
\pgfpathlineto{\pgfqpoint{6.909249in}{3.309664in}}%
\pgfusepath{stroke}%
\end{pgfscope}%
\begin{pgfscope}%
\pgfsetrectcap%
\pgfsetmiterjoin%
\pgfsetlinewidth{0.803000pt}%
\definecolor{currentstroke}{rgb}{0.000000,0.000000,0.000000}%
\pgfsetstrokecolor{currentstroke}%
\pgfsetdash{}{0pt}%
\pgfpathmoveto{\pgfqpoint{6.084781in}{3.066506in}}%
\pgfpathlineto{\pgfqpoint{6.909249in}{3.066506in}}%
\pgfusepath{stroke}%
\end{pgfscope}%
\begin{pgfscope}%
\pgfsetrectcap%
\pgfsetmiterjoin%
\pgfsetlinewidth{0.803000pt}%
\definecolor{currentstroke}{rgb}{0.000000,0.000000,0.000000}%
\pgfsetstrokecolor{currentstroke}%
\pgfsetdash{}{0pt}%
\pgfpathmoveto{\pgfqpoint{6.084781in}{3.309664in}}%
\pgfpathlineto{\pgfqpoint{6.909249in}{3.309664in}}%
\pgfusepath{stroke}%
\end{pgfscope}%
\begin{pgfscope}%
\definecolor{textcolor}{rgb}{0.000000,0.000000,0.000000}%
\pgfsetstrokecolor{textcolor}%
\pgfsetfillcolor{textcolor}%
\pgftext[x=6.497015in,y=3.392997in,,base]{\color{textcolor}\rmfamily\fontsize{11.000000}{13.200000}\selectfont Harmon...}%
\end{pgfscope}%
\begin{pgfscope}%
\pgfsetbuttcap%
\pgfsetmiterjoin%
\definecolor{currentfill}{rgb}{1.000000,1.000000,1.000000}%
\pgfsetfillcolor{currentfill}%
\pgfsetlinewidth{0.000000pt}%
\definecolor{currentstroke}{rgb}{0.000000,0.000000,0.000000}%
\pgfsetstrokecolor{currentstroke}%
\pgfsetstrokeopacity{0.000000}%
\pgfsetdash{}{0pt}%
\pgfpathmoveto{\pgfqpoint{7.074143in}{3.066506in}}%
\pgfpathlineto{\pgfqpoint{7.898611in}{3.066506in}}%
\pgfpathlineto{\pgfqpoint{7.898611in}{3.309664in}}%
\pgfpathlineto{\pgfqpoint{7.074143in}{3.309664in}}%
\pgfpathlineto{\pgfqpoint{7.074143in}{3.066506in}}%
\pgfpathclose%
\pgfusepath{fill}%
\end{pgfscope}%
\begin{pgfscope}%
\pgfpathrectangle{\pgfqpoint{7.074143in}{3.066506in}}{\pgfqpoint{0.824468in}{0.243158in}}%
\pgfusepath{clip}%
\pgfsetbuttcap%
\pgfsetmiterjoin%
\definecolor{currentfill}{rgb}{0.121569,0.466667,0.705882}%
\pgfsetfillcolor{currentfill}%
\pgfsetfillopacity{0.500000}%
\pgfsetlinewidth{1.003750pt}%
\definecolor{currentstroke}{rgb}{0.000000,0.000000,0.000000}%
\pgfsetstrokecolor{currentstroke}%
\pgfsetdash{}{0pt}%
\pgfpathmoveto{\pgfqpoint{7.111619in}{3.066506in}}%
\pgfpathlineto{\pgfqpoint{7.261522in}{3.066506in}}%
\pgfpathlineto{\pgfqpoint{7.261522in}{3.075989in}}%
\pgfpathlineto{\pgfqpoint{7.111619in}{3.075989in}}%
\pgfpathlineto{\pgfqpoint{7.111619in}{3.066506in}}%
\pgfpathclose%
\pgfusepath{stroke,fill}%
\end{pgfscope}%
\begin{pgfscope}%
\pgfpathrectangle{\pgfqpoint{7.074143in}{3.066506in}}{\pgfqpoint{0.824468in}{0.243158in}}%
\pgfusepath{clip}%
\pgfsetbuttcap%
\pgfsetmiterjoin%
\definecolor{currentfill}{rgb}{0.121569,0.466667,0.705882}%
\pgfsetfillcolor{currentfill}%
\pgfsetfillopacity{0.500000}%
\pgfsetlinewidth{1.003750pt}%
\definecolor{currentstroke}{rgb}{0.000000,0.000000,0.000000}%
\pgfsetstrokecolor{currentstroke}%
\pgfsetdash{}{0pt}%
\pgfpathmoveto{\pgfqpoint{7.261522in}{3.066506in}}%
\pgfpathlineto{\pgfqpoint{7.411425in}{3.066506in}}%
\pgfpathlineto{\pgfqpoint{7.411425in}{3.069181in}}%
\pgfpathlineto{\pgfqpoint{7.261522in}{3.069181in}}%
\pgfpathlineto{\pgfqpoint{7.261522in}{3.066506in}}%
\pgfpathclose%
\pgfusepath{stroke,fill}%
\end{pgfscope}%
\begin{pgfscope}%
\pgfpathrectangle{\pgfqpoint{7.074143in}{3.066506in}}{\pgfqpoint{0.824468in}{0.243158in}}%
\pgfusepath{clip}%
\pgfsetbuttcap%
\pgfsetmiterjoin%
\definecolor{currentfill}{rgb}{0.121569,0.466667,0.705882}%
\pgfsetfillcolor{currentfill}%
\pgfsetfillopacity{0.500000}%
\pgfsetlinewidth{1.003750pt}%
\definecolor{currentstroke}{rgb}{0.000000,0.000000,0.000000}%
\pgfsetstrokecolor{currentstroke}%
\pgfsetdash{}{0pt}%
\pgfpathmoveto{\pgfqpoint{7.411425in}{3.066506in}}%
\pgfpathlineto{\pgfqpoint{7.561329in}{3.066506in}}%
\pgfpathlineto{\pgfqpoint{7.561329in}{3.067722in}}%
\pgfpathlineto{\pgfqpoint{7.411425in}{3.067722in}}%
\pgfpathlineto{\pgfqpoint{7.411425in}{3.066506in}}%
\pgfpathclose%
\pgfusepath{stroke,fill}%
\end{pgfscope}%
\begin{pgfscope}%
\pgfpathrectangle{\pgfqpoint{7.074143in}{3.066506in}}{\pgfqpoint{0.824468in}{0.243158in}}%
\pgfusepath{clip}%
\pgfsetbuttcap%
\pgfsetmiterjoin%
\definecolor{currentfill}{rgb}{0.121569,0.466667,0.705882}%
\pgfsetfillcolor{currentfill}%
\pgfsetfillopacity{0.500000}%
\pgfsetlinewidth{1.003750pt}%
\definecolor{currentstroke}{rgb}{0.000000,0.000000,0.000000}%
\pgfsetstrokecolor{currentstroke}%
\pgfsetdash{}{0pt}%
\pgfpathmoveto{\pgfqpoint{7.561329in}{3.066506in}}%
\pgfpathlineto{\pgfqpoint{7.711232in}{3.066506in}}%
\pgfpathlineto{\pgfqpoint{7.711232in}{3.067843in}}%
\pgfpathlineto{\pgfqpoint{7.561329in}{3.067843in}}%
\pgfpathlineto{\pgfqpoint{7.561329in}{3.066506in}}%
\pgfpathclose%
\pgfusepath{stroke,fill}%
\end{pgfscope}%
\begin{pgfscope}%
\pgfpathrectangle{\pgfqpoint{7.074143in}{3.066506in}}{\pgfqpoint{0.824468in}{0.243158in}}%
\pgfusepath{clip}%
\pgfsetbuttcap%
\pgfsetmiterjoin%
\definecolor{currentfill}{rgb}{0.121569,0.466667,0.705882}%
\pgfsetfillcolor{currentfill}%
\pgfsetfillopacity{0.500000}%
\pgfsetlinewidth{1.003750pt}%
\definecolor{currentstroke}{rgb}{0.000000,0.000000,0.000000}%
\pgfsetstrokecolor{currentstroke}%
\pgfsetdash{}{0pt}%
\pgfpathmoveto{\pgfqpoint{7.711232in}{3.066506in}}%
\pgfpathlineto{\pgfqpoint{7.861135in}{3.066506in}}%
\pgfpathlineto{\pgfqpoint{7.861135in}{3.067357in}}%
\pgfpathlineto{\pgfqpoint{7.711232in}{3.067357in}}%
\pgfpathlineto{\pgfqpoint{7.711232in}{3.066506in}}%
\pgfpathclose%
\pgfusepath{stroke,fill}%
\end{pgfscope}%
\begin{pgfscope}%
\pgfsetrectcap%
\pgfsetmiterjoin%
\pgfsetlinewidth{0.803000pt}%
\definecolor{currentstroke}{rgb}{0.000000,0.000000,0.000000}%
\pgfsetstrokecolor{currentstroke}%
\pgfsetdash{}{0pt}%
\pgfpathmoveto{\pgfqpoint{7.074143in}{3.066506in}}%
\pgfpathlineto{\pgfqpoint{7.074143in}{3.309664in}}%
\pgfusepath{stroke}%
\end{pgfscope}%
\begin{pgfscope}%
\pgfsetrectcap%
\pgfsetmiterjoin%
\pgfsetlinewidth{0.803000pt}%
\definecolor{currentstroke}{rgb}{0.000000,0.000000,0.000000}%
\pgfsetstrokecolor{currentstroke}%
\pgfsetdash{}{0pt}%
\pgfpathmoveto{\pgfqpoint{7.898611in}{3.066506in}}%
\pgfpathlineto{\pgfqpoint{7.898611in}{3.309664in}}%
\pgfusepath{stroke}%
\end{pgfscope}%
\begin{pgfscope}%
\pgfsetrectcap%
\pgfsetmiterjoin%
\pgfsetlinewidth{0.803000pt}%
\definecolor{currentstroke}{rgb}{0.000000,0.000000,0.000000}%
\pgfsetstrokecolor{currentstroke}%
\pgfsetdash{}{0pt}%
\pgfpathmoveto{\pgfqpoint{7.074143in}{3.066506in}}%
\pgfpathlineto{\pgfqpoint{7.898611in}{3.066506in}}%
\pgfusepath{stroke}%
\end{pgfscope}%
\begin{pgfscope}%
\pgfsetrectcap%
\pgfsetmiterjoin%
\pgfsetlinewidth{0.803000pt}%
\definecolor{currentstroke}{rgb}{0.000000,0.000000,0.000000}%
\pgfsetstrokecolor{currentstroke}%
\pgfsetdash{}{0pt}%
\pgfpathmoveto{\pgfqpoint{7.074143in}{3.309664in}}%
\pgfpathlineto{\pgfqpoint{7.898611in}{3.309664in}}%
\pgfusepath{stroke}%
\end{pgfscope}%
\begin{pgfscope}%
\definecolor{textcolor}{rgb}{0.000000,0.000000,0.000000}%
\pgfsetstrokecolor{textcolor}%
\pgfsetfillcolor{textcolor}%
\pgftext[x=7.486377in,y=3.392997in,,base]{\color{textcolor}\rmfamily\fontsize{11.000000}{13.200000}\selectfont Mutuel...}%
\end{pgfscope}%
\begin{pgfscope}%
\pgfsetbuttcap%
\pgfsetmiterjoin%
\definecolor{currentfill}{rgb}{1.000000,1.000000,1.000000}%
\pgfsetfillcolor{currentfill}%
\pgfsetlinewidth{0.000000pt}%
\definecolor{currentstroke}{rgb}{0.000000,0.000000,0.000000}%
\pgfsetstrokecolor{currentstroke}%
\pgfsetstrokeopacity{0.000000}%
\pgfsetdash{}{0pt}%
\pgfpathmoveto{\pgfqpoint{0.148611in}{2.337032in}}%
\pgfpathlineto{\pgfqpoint{0.973079in}{2.337032in}}%
\pgfpathlineto{\pgfqpoint{0.973079in}{2.580190in}}%
\pgfpathlineto{\pgfqpoint{0.148611in}{2.580190in}}%
\pgfpathlineto{\pgfqpoint{0.148611in}{2.337032in}}%
\pgfpathclose%
\pgfusepath{fill}%
\end{pgfscope}%
\begin{pgfscope}%
\pgfpathrectangle{\pgfqpoint{0.148611in}{2.337032in}}{\pgfqpoint{0.824468in}{0.243158in}}%
\pgfusepath{clip}%
\pgfsetbuttcap%
\pgfsetmiterjoin%
\definecolor{currentfill}{rgb}{0.121569,0.466667,0.705882}%
\pgfsetfillcolor{currentfill}%
\pgfsetfillopacity{0.500000}%
\pgfsetlinewidth{1.003750pt}%
\definecolor{currentstroke}{rgb}{0.000000,0.000000,0.000000}%
\pgfsetstrokecolor{currentstroke}%
\pgfsetdash{}{0pt}%
\pgfpathmoveto{\pgfqpoint{0.186087in}{2.337032in}}%
\pgfpathlineto{\pgfqpoint{0.335990in}{2.337032in}}%
\pgfpathlineto{\pgfqpoint{0.335990in}{2.388460in}}%
\pgfpathlineto{\pgfqpoint{0.186087in}{2.388460in}}%
\pgfpathlineto{\pgfqpoint{0.186087in}{2.337032in}}%
\pgfpathclose%
\pgfusepath{stroke,fill}%
\end{pgfscope}%
\begin{pgfscope}%
\pgfpathrectangle{\pgfqpoint{0.148611in}{2.337032in}}{\pgfqpoint{0.824468in}{0.243158in}}%
\pgfusepath{clip}%
\pgfsetbuttcap%
\pgfsetmiterjoin%
\definecolor{currentfill}{rgb}{0.121569,0.466667,0.705882}%
\pgfsetfillcolor{currentfill}%
\pgfsetfillopacity{0.500000}%
\pgfsetlinewidth{1.003750pt}%
\definecolor{currentstroke}{rgb}{0.000000,0.000000,0.000000}%
\pgfsetstrokecolor{currentstroke}%
\pgfsetdash{}{0pt}%
\pgfpathmoveto{\pgfqpoint{0.335990in}{2.337032in}}%
\pgfpathlineto{\pgfqpoint{0.485894in}{2.337032in}}%
\pgfpathlineto{\pgfqpoint{0.485894in}{2.367427in}}%
\pgfpathlineto{\pgfqpoint{0.335990in}{2.367427in}}%
\pgfpathlineto{\pgfqpoint{0.335990in}{2.337032in}}%
\pgfpathclose%
\pgfusepath{stroke,fill}%
\end{pgfscope}%
\begin{pgfscope}%
\pgfpathrectangle{\pgfqpoint{0.148611in}{2.337032in}}{\pgfqpoint{0.824468in}{0.243158in}}%
\pgfusepath{clip}%
\pgfsetbuttcap%
\pgfsetmiterjoin%
\definecolor{currentfill}{rgb}{0.121569,0.466667,0.705882}%
\pgfsetfillcolor{currentfill}%
\pgfsetfillopacity{0.500000}%
\pgfsetlinewidth{1.003750pt}%
\definecolor{currentstroke}{rgb}{0.000000,0.000000,0.000000}%
\pgfsetstrokecolor{currentstroke}%
\pgfsetdash{}{0pt}%
\pgfpathmoveto{\pgfqpoint{0.485894in}{2.337032in}}%
\pgfpathlineto{\pgfqpoint{0.635797in}{2.337032in}}%
\pgfpathlineto{\pgfqpoint{0.635797in}{2.346880in}}%
\pgfpathlineto{\pgfqpoint{0.485894in}{2.346880in}}%
\pgfpathlineto{\pgfqpoint{0.485894in}{2.337032in}}%
\pgfpathclose%
\pgfusepath{stroke,fill}%
\end{pgfscope}%
\begin{pgfscope}%
\pgfpathrectangle{\pgfqpoint{0.148611in}{2.337032in}}{\pgfqpoint{0.824468in}{0.243158in}}%
\pgfusepath{clip}%
\pgfsetbuttcap%
\pgfsetmiterjoin%
\definecolor{currentfill}{rgb}{0.121569,0.466667,0.705882}%
\pgfsetfillcolor{currentfill}%
\pgfsetfillopacity{0.500000}%
\pgfsetlinewidth{1.003750pt}%
\definecolor{currentstroke}{rgb}{0.000000,0.000000,0.000000}%
\pgfsetstrokecolor{currentstroke}%
\pgfsetdash{}{0pt}%
\pgfpathmoveto{\pgfqpoint{0.635797in}{2.337032in}}%
\pgfpathlineto{\pgfqpoint{0.785700in}{2.337032in}}%
\pgfpathlineto{\pgfqpoint{0.785700in}{2.342260in}}%
\pgfpathlineto{\pgfqpoint{0.635797in}{2.342260in}}%
\pgfpathlineto{\pgfqpoint{0.635797in}{2.337032in}}%
\pgfpathclose%
\pgfusepath{stroke,fill}%
\end{pgfscope}%
\begin{pgfscope}%
\pgfpathrectangle{\pgfqpoint{0.148611in}{2.337032in}}{\pgfqpoint{0.824468in}{0.243158in}}%
\pgfusepath{clip}%
\pgfsetbuttcap%
\pgfsetmiterjoin%
\definecolor{currentfill}{rgb}{0.121569,0.466667,0.705882}%
\pgfsetfillcolor{currentfill}%
\pgfsetfillopacity{0.500000}%
\pgfsetlinewidth{1.003750pt}%
\definecolor{currentstroke}{rgb}{0.000000,0.000000,0.000000}%
\pgfsetstrokecolor{currentstroke}%
\pgfsetdash{}{0pt}%
\pgfpathmoveto{\pgfqpoint{0.785700in}{2.337032in}}%
\pgfpathlineto{\pgfqpoint{0.935603in}{2.337032in}}%
\pgfpathlineto{\pgfqpoint{0.935603in}{2.340923in}}%
\pgfpathlineto{\pgfqpoint{0.785700in}{2.340923in}}%
\pgfpathlineto{\pgfqpoint{0.785700in}{2.337032in}}%
\pgfpathclose%
\pgfusepath{stroke,fill}%
\end{pgfscope}%
\begin{pgfscope}%
\pgfsetrectcap%
\pgfsetmiterjoin%
\pgfsetlinewidth{0.803000pt}%
\definecolor{currentstroke}{rgb}{0.000000,0.000000,0.000000}%
\pgfsetstrokecolor{currentstroke}%
\pgfsetdash{}{0pt}%
\pgfpathmoveto{\pgfqpoint{0.148611in}{2.337032in}}%
\pgfpathlineto{\pgfqpoint{0.148611in}{2.580190in}}%
\pgfusepath{stroke}%
\end{pgfscope}%
\begin{pgfscope}%
\pgfsetrectcap%
\pgfsetmiterjoin%
\pgfsetlinewidth{0.803000pt}%
\definecolor{currentstroke}{rgb}{0.000000,0.000000,0.000000}%
\pgfsetstrokecolor{currentstroke}%
\pgfsetdash{}{0pt}%
\pgfpathmoveto{\pgfqpoint{0.973079in}{2.337032in}}%
\pgfpathlineto{\pgfqpoint{0.973079in}{2.580190in}}%
\pgfusepath{stroke}%
\end{pgfscope}%
\begin{pgfscope}%
\pgfsetrectcap%
\pgfsetmiterjoin%
\pgfsetlinewidth{0.803000pt}%
\definecolor{currentstroke}{rgb}{0.000000,0.000000,0.000000}%
\pgfsetstrokecolor{currentstroke}%
\pgfsetdash{}{0pt}%
\pgfpathmoveto{\pgfqpoint{0.148611in}{2.337032in}}%
\pgfpathlineto{\pgfqpoint{0.973079in}{2.337032in}}%
\pgfusepath{stroke}%
\end{pgfscope}%
\begin{pgfscope}%
\pgfsetrectcap%
\pgfsetmiterjoin%
\pgfsetlinewidth{0.803000pt}%
\definecolor{currentstroke}{rgb}{0.000000,0.000000,0.000000}%
\pgfsetstrokecolor{currentstroke}%
\pgfsetdash{}{0pt}%
\pgfpathmoveto{\pgfqpoint{0.148611in}{2.580190in}}%
\pgfpathlineto{\pgfqpoint{0.973079in}{2.580190in}}%
\pgfusepath{stroke}%
\end{pgfscope}%
\begin{pgfscope}%
\definecolor{textcolor}{rgb}{0.000000,0.000000,0.000000}%
\pgfsetstrokecolor{textcolor}%
\pgfsetfillcolor{textcolor}%
\pgftext[x=0.560845in,y=2.663523in,,base]{\color{textcolor}\rmfamily\fontsize{11.000000}{13.200000}\selectfont MACIF}%
\end{pgfscope}%
\begin{pgfscope}%
\pgfsetbuttcap%
\pgfsetmiterjoin%
\definecolor{currentfill}{rgb}{1.000000,1.000000,1.000000}%
\pgfsetfillcolor{currentfill}%
\pgfsetlinewidth{0.000000pt}%
\definecolor{currentstroke}{rgb}{0.000000,0.000000,0.000000}%
\pgfsetstrokecolor{currentstroke}%
\pgfsetstrokeopacity{0.000000}%
\pgfsetdash{}{0pt}%
\pgfpathmoveto{\pgfqpoint{1.137973in}{2.337032in}}%
\pgfpathlineto{\pgfqpoint{1.962441in}{2.337032in}}%
\pgfpathlineto{\pgfqpoint{1.962441in}{2.580190in}}%
\pgfpathlineto{\pgfqpoint{1.137973in}{2.580190in}}%
\pgfpathlineto{\pgfqpoint{1.137973in}{2.337032in}}%
\pgfpathclose%
\pgfusepath{fill}%
\end{pgfscope}%
\begin{pgfscope}%
\pgfpathrectangle{\pgfqpoint{1.137973in}{2.337032in}}{\pgfqpoint{0.824468in}{0.243158in}}%
\pgfusepath{clip}%
\pgfsetbuttcap%
\pgfsetmiterjoin%
\definecolor{currentfill}{rgb}{0.121569,0.466667,0.705882}%
\pgfsetfillcolor{currentfill}%
\pgfsetfillopacity{0.500000}%
\pgfsetlinewidth{1.003750pt}%
\definecolor{currentstroke}{rgb}{0.000000,0.000000,0.000000}%
\pgfsetstrokecolor{currentstroke}%
\pgfsetdash{}{0pt}%
\pgfpathmoveto{\pgfqpoint{1.175449in}{2.337032in}}%
\pgfpathlineto{\pgfqpoint{1.325352in}{2.337032in}}%
\pgfpathlineto{\pgfqpoint{1.325352in}{2.350406in}}%
\pgfpathlineto{\pgfqpoint{1.175449in}{2.350406in}}%
\pgfpathlineto{\pgfqpoint{1.175449in}{2.337032in}}%
\pgfpathclose%
\pgfusepath{stroke,fill}%
\end{pgfscope}%
\begin{pgfscope}%
\pgfpathrectangle{\pgfqpoint{1.137973in}{2.337032in}}{\pgfqpoint{0.824468in}{0.243158in}}%
\pgfusepath{clip}%
\pgfsetbuttcap%
\pgfsetmiterjoin%
\definecolor{currentfill}{rgb}{0.121569,0.466667,0.705882}%
\pgfsetfillcolor{currentfill}%
\pgfsetfillopacity{0.500000}%
\pgfsetlinewidth{1.003750pt}%
\definecolor{currentstroke}{rgb}{0.000000,0.000000,0.000000}%
\pgfsetstrokecolor{currentstroke}%
\pgfsetdash{}{0pt}%
\pgfpathmoveto{\pgfqpoint{1.325352in}{2.337032in}}%
\pgfpathlineto{\pgfqpoint{1.475255in}{2.337032in}}%
\pgfpathlineto{\pgfqpoint{1.475255in}{2.351257in}}%
\pgfpathlineto{\pgfqpoint{1.325352in}{2.351257in}}%
\pgfpathlineto{\pgfqpoint{1.325352in}{2.337032in}}%
\pgfpathclose%
\pgfusepath{stroke,fill}%
\end{pgfscope}%
\begin{pgfscope}%
\pgfpathrectangle{\pgfqpoint{1.137973in}{2.337032in}}{\pgfqpoint{0.824468in}{0.243158in}}%
\pgfusepath{clip}%
\pgfsetbuttcap%
\pgfsetmiterjoin%
\definecolor{currentfill}{rgb}{0.121569,0.466667,0.705882}%
\pgfsetfillcolor{currentfill}%
\pgfsetfillopacity{0.500000}%
\pgfsetlinewidth{1.003750pt}%
\definecolor{currentstroke}{rgb}{0.000000,0.000000,0.000000}%
\pgfsetstrokecolor{currentstroke}%
\pgfsetdash{}{0pt}%
\pgfpathmoveto{\pgfqpoint{1.475255in}{2.337032in}}%
\pgfpathlineto{\pgfqpoint{1.625158in}{2.337032in}}%
\pgfpathlineto{\pgfqpoint{1.625158in}{2.341044in}}%
\pgfpathlineto{\pgfqpoint{1.475255in}{2.341044in}}%
\pgfpathlineto{\pgfqpoint{1.475255in}{2.337032in}}%
\pgfpathclose%
\pgfusepath{stroke,fill}%
\end{pgfscope}%
\begin{pgfscope}%
\pgfpathrectangle{\pgfqpoint{1.137973in}{2.337032in}}{\pgfqpoint{0.824468in}{0.243158in}}%
\pgfusepath{clip}%
\pgfsetbuttcap%
\pgfsetmiterjoin%
\definecolor{currentfill}{rgb}{0.121569,0.466667,0.705882}%
\pgfsetfillcolor{currentfill}%
\pgfsetfillopacity{0.500000}%
\pgfsetlinewidth{1.003750pt}%
\definecolor{currentstroke}{rgb}{0.000000,0.000000,0.000000}%
\pgfsetstrokecolor{currentstroke}%
\pgfsetdash{}{0pt}%
\pgfpathmoveto{\pgfqpoint{1.625158in}{2.337032in}}%
\pgfpathlineto{\pgfqpoint{1.775062in}{2.337032in}}%
\pgfpathlineto{\pgfqpoint{1.775062in}{2.339342in}}%
\pgfpathlineto{\pgfqpoint{1.625158in}{2.339342in}}%
\pgfpathlineto{\pgfqpoint{1.625158in}{2.337032in}}%
\pgfpathclose%
\pgfusepath{stroke,fill}%
\end{pgfscope}%
\begin{pgfscope}%
\pgfpathrectangle{\pgfqpoint{1.137973in}{2.337032in}}{\pgfqpoint{0.824468in}{0.243158in}}%
\pgfusepath{clip}%
\pgfsetbuttcap%
\pgfsetmiterjoin%
\definecolor{currentfill}{rgb}{0.121569,0.466667,0.705882}%
\pgfsetfillcolor{currentfill}%
\pgfsetfillopacity{0.500000}%
\pgfsetlinewidth{1.003750pt}%
\definecolor{currentstroke}{rgb}{0.000000,0.000000,0.000000}%
\pgfsetstrokecolor{currentstroke}%
\pgfsetdash{}{0pt}%
\pgfpathmoveto{\pgfqpoint{1.775062in}{2.337032in}}%
\pgfpathlineto{\pgfqpoint{1.924965in}{2.337032in}}%
\pgfpathlineto{\pgfqpoint{1.924965in}{2.338248in}}%
\pgfpathlineto{\pgfqpoint{1.775062in}{2.338248in}}%
\pgfpathlineto{\pgfqpoint{1.775062in}{2.337032in}}%
\pgfpathclose%
\pgfusepath{stroke,fill}%
\end{pgfscope}%
\begin{pgfscope}%
\pgfsetrectcap%
\pgfsetmiterjoin%
\pgfsetlinewidth{0.803000pt}%
\definecolor{currentstroke}{rgb}{0.000000,0.000000,0.000000}%
\pgfsetstrokecolor{currentstroke}%
\pgfsetdash{}{0pt}%
\pgfpathmoveto{\pgfqpoint{1.137973in}{2.337032in}}%
\pgfpathlineto{\pgfqpoint{1.137973in}{2.580190in}}%
\pgfusepath{stroke}%
\end{pgfscope}%
\begin{pgfscope}%
\pgfsetrectcap%
\pgfsetmiterjoin%
\pgfsetlinewidth{0.803000pt}%
\definecolor{currentstroke}{rgb}{0.000000,0.000000,0.000000}%
\pgfsetstrokecolor{currentstroke}%
\pgfsetdash{}{0pt}%
\pgfpathmoveto{\pgfqpoint{1.962441in}{2.337032in}}%
\pgfpathlineto{\pgfqpoint{1.962441in}{2.580190in}}%
\pgfusepath{stroke}%
\end{pgfscope}%
\begin{pgfscope}%
\pgfsetrectcap%
\pgfsetmiterjoin%
\pgfsetlinewidth{0.803000pt}%
\definecolor{currentstroke}{rgb}{0.000000,0.000000,0.000000}%
\pgfsetstrokecolor{currentstroke}%
\pgfsetdash{}{0pt}%
\pgfpathmoveto{\pgfqpoint{1.137973in}{2.337032in}}%
\pgfpathlineto{\pgfqpoint{1.962441in}{2.337032in}}%
\pgfusepath{stroke}%
\end{pgfscope}%
\begin{pgfscope}%
\pgfsetrectcap%
\pgfsetmiterjoin%
\pgfsetlinewidth{0.803000pt}%
\definecolor{currentstroke}{rgb}{0.000000,0.000000,0.000000}%
\pgfsetstrokecolor{currentstroke}%
\pgfsetdash{}{0pt}%
\pgfpathmoveto{\pgfqpoint{1.137973in}{2.580190in}}%
\pgfpathlineto{\pgfqpoint{1.962441in}{2.580190in}}%
\pgfusepath{stroke}%
\end{pgfscope}%
\begin{pgfscope}%
\definecolor{textcolor}{rgb}{0.000000,0.000000,0.000000}%
\pgfsetstrokecolor{textcolor}%
\pgfsetfillcolor{textcolor}%
\pgftext[x=1.550207in,y=2.663523in,,base]{\color{textcolor}\rmfamily\fontsize{11.000000}{13.200000}\selectfont Eurofil}%
\end{pgfscope}%
\begin{pgfscope}%
\pgfsetbuttcap%
\pgfsetmiterjoin%
\definecolor{currentfill}{rgb}{1.000000,1.000000,1.000000}%
\pgfsetfillcolor{currentfill}%
\pgfsetlinewidth{0.000000pt}%
\definecolor{currentstroke}{rgb}{0.000000,0.000000,0.000000}%
\pgfsetstrokecolor{currentstroke}%
\pgfsetstrokeopacity{0.000000}%
\pgfsetdash{}{0pt}%
\pgfpathmoveto{\pgfqpoint{2.127335in}{2.337032in}}%
\pgfpathlineto{\pgfqpoint{2.951803in}{2.337032in}}%
\pgfpathlineto{\pgfqpoint{2.951803in}{2.580190in}}%
\pgfpathlineto{\pgfqpoint{2.127335in}{2.580190in}}%
\pgfpathlineto{\pgfqpoint{2.127335in}{2.337032in}}%
\pgfpathclose%
\pgfusepath{fill}%
\end{pgfscope}%
\begin{pgfscope}%
\pgfpathrectangle{\pgfqpoint{2.127335in}{2.337032in}}{\pgfqpoint{0.824468in}{0.243158in}}%
\pgfusepath{clip}%
\pgfsetbuttcap%
\pgfsetmiterjoin%
\definecolor{currentfill}{rgb}{0.121569,0.466667,0.705882}%
\pgfsetfillcolor{currentfill}%
\pgfsetfillopacity{0.500000}%
\pgfsetlinewidth{1.003750pt}%
\definecolor{currentstroke}{rgb}{0.000000,0.000000,0.000000}%
\pgfsetstrokecolor{currentstroke}%
\pgfsetdash{}{0pt}%
\pgfpathmoveto{\pgfqpoint{2.164810in}{2.337032in}}%
\pgfpathlineto{\pgfqpoint{2.314714in}{2.337032in}}%
\pgfpathlineto{\pgfqpoint{2.314714in}{2.365238in}}%
\pgfpathlineto{\pgfqpoint{2.164810in}{2.365238in}}%
\pgfpathlineto{\pgfqpoint{2.164810in}{2.337032in}}%
\pgfpathclose%
\pgfusepath{stroke,fill}%
\end{pgfscope}%
\begin{pgfscope}%
\pgfpathrectangle{\pgfqpoint{2.127335in}{2.337032in}}{\pgfqpoint{0.824468in}{0.243158in}}%
\pgfusepath{clip}%
\pgfsetbuttcap%
\pgfsetmiterjoin%
\definecolor{currentfill}{rgb}{0.121569,0.466667,0.705882}%
\pgfsetfillcolor{currentfill}%
\pgfsetfillopacity{0.500000}%
\pgfsetlinewidth{1.003750pt}%
\definecolor{currentstroke}{rgb}{0.000000,0.000000,0.000000}%
\pgfsetstrokecolor{currentstroke}%
\pgfsetdash{}{0pt}%
\pgfpathmoveto{\pgfqpoint{2.314714in}{2.337032in}}%
\pgfpathlineto{\pgfqpoint{2.464617in}{2.337032in}}%
\pgfpathlineto{\pgfqpoint{2.464617in}{2.349312in}}%
\pgfpathlineto{\pgfqpoint{2.314714in}{2.349312in}}%
\pgfpathlineto{\pgfqpoint{2.314714in}{2.337032in}}%
\pgfpathclose%
\pgfusepath{stroke,fill}%
\end{pgfscope}%
\begin{pgfscope}%
\pgfpathrectangle{\pgfqpoint{2.127335in}{2.337032in}}{\pgfqpoint{0.824468in}{0.243158in}}%
\pgfusepath{clip}%
\pgfsetbuttcap%
\pgfsetmiterjoin%
\definecolor{currentfill}{rgb}{0.121569,0.466667,0.705882}%
\pgfsetfillcolor{currentfill}%
\pgfsetfillopacity{0.500000}%
\pgfsetlinewidth{1.003750pt}%
\definecolor{currentstroke}{rgb}{0.000000,0.000000,0.000000}%
\pgfsetstrokecolor{currentstroke}%
\pgfsetdash{}{0pt}%
\pgfpathmoveto{\pgfqpoint{2.464617in}{2.337032in}}%
\pgfpathlineto{\pgfqpoint{2.614520in}{2.337032in}}%
\pgfpathlineto{\pgfqpoint{2.614520in}{2.342138in}}%
\pgfpathlineto{\pgfqpoint{2.464617in}{2.342138in}}%
\pgfpathlineto{\pgfqpoint{2.464617in}{2.337032in}}%
\pgfpathclose%
\pgfusepath{stroke,fill}%
\end{pgfscope}%
\begin{pgfscope}%
\pgfpathrectangle{\pgfqpoint{2.127335in}{2.337032in}}{\pgfqpoint{0.824468in}{0.243158in}}%
\pgfusepath{clip}%
\pgfsetbuttcap%
\pgfsetmiterjoin%
\definecolor{currentfill}{rgb}{0.121569,0.466667,0.705882}%
\pgfsetfillcolor{currentfill}%
\pgfsetfillopacity{0.500000}%
\pgfsetlinewidth{1.003750pt}%
\definecolor{currentstroke}{rgb}{0.000000,0.000000,0.000000}%
\pgfsetstrokecolor{currentstroke}%
\pgfsetdash{}{0pt}%
\pgfpathmoveto{\pgfqpoint{2.614520in}{2.337032in}}%
\pgfpathlineto{\pgfqpoint{2.764423in}{2.337032in}}%
\pgfpathlineto{\pgfqpoint{2.764423in}{2.338248in}}%
\pgfpathlineto{\pgfqpoint{2.614520in}{2.338248in}}%
\pgfpathlineto{\pgfqpoint{2.614520in}{2.337032in}}%
\pgfpathclose%
\pgfusepath{stroke,fill}%
\end{pgfscope}%
\begin{pgfscope}%
\pgfpathrectangle{\pgfqpoint{2.127335in}{2.337032in}}{\pgfqpoint{0.824468in}{0.243158in}}%
\pgfusepath{clip}%
\pgfsetbuttcap%
\pgfsetmiterjoin%
\definecolor{currentfill}{rgb}{0.121569,0.466667,0.705882}%
\pgfsetfillcolor{currentfill}%
\pgfsetfillopacity{0.500000}%
\pgfsetlinewidth{1.003750pt}%
\definecolor{currentstroke}{rgb}{0.000000,0.000000,0.000000}%
\pgfsetstrokecolor{currentstroke}%
\pgfsetdash{}{0pt}%
\pgfpathmoveto{\pgfqpoint{2.764423in}{2.337032in}}%
\pgfpathlineto{\pgfqpoint{2.914327in}{2.337032in}}%
\pgfpathlineto{\pgfqpoint{2.914327in}{2.339221in}}%
\pgfpathlineto{\pgfqpoint{2.764423in}{2.339221in}}%
\pgfpathlineto{\pgfqpoint{2.764423in}{2.337032in}}%
\pgfpathclose%
\pgfusepath{stroke,fill}%
\end{pgfscope}%
\begin{pgfscope}%
\pgfsetrectcap%
\pgfsetmiterjoin%
\pgfsetlinewidth{0.803000pt}%
\definecolor{currentstroke}{rgb}{0.000000,0.000000,0.000000}%
\pgfsetstrokecolor{currentstroke}%
\pgfsetdash{}{0pt}%
\pgfpathmoveto{\pgfqpoint{2.127335in}{2.337032in}}%
\pgfpathlineto{\pgfqpoint{2.127335in}{2.580190in}}%
\pgfusepath{stroke}%
\end{pgfscope}%
\begin{pgfscope}%
\pgfsetrectcap%
\pgfsetmiterjoin%
\pgfsetlinewidth{0.803000pt}%
\definecolor{currentstroke}{rgb}{0.000000,0.000000,0.000000}%
\pgfsetstrokecolor{currentstroke}%
\pgfsetdash{}{0pt}%
\pgfpathmoveto{\pgfqpoint{2.951803in}{2.337032in}}%
\pgfpathlineto{\pgfqpoint{2.951803in}{2.580190in}}%
\pgfusepath{stroke}%
\end{pgfscope}%
\begin{pgfscope}%
\pgfsetrectcap%
\pgfsetmiterjoin%
\pgfsetlinewidth{0.803000pt}%
\definecolor{currentstroke}{rgb}{0.000000,0.000000,0.000000}%
\pgfsetstrokecolor{currentstroke}%
\pgfsetdash{}{0pt}%
\pgfpathmoveto{\pgfqpoint{2.127335in}{2.337032in}}%
\pgfpathlineto{\pgfqpoint{2.951803in}{2.337032in}}%
\pgfusepath{stroke}%
\end{pgfscope}%
\begin{pgfscope}%
\pgfsetrectcap%
\pgfsetmiterjoin%
\pgfsetlinewidth{0.803000pt}%
\definecolor{currentstroke}{rgb}{0.000000,0.000000,0.000000}%
\pgfsetstrokecolor{currentstroke}%
\pgfsetdash{}{0pt}%
\pgfpathmoveto{\pgfqpoint{2.127335in}{2.580190in}}%
\pgfpathlineto{\pgfqpoint{2.951803in}{2.580190in}}%
\pgfusepath{stroke}%
\end{pgfscope}%
\begin{pgfscope}%
\definecolor{textcolor}{rgb}{0.000000,0.000000,0.000000}%
\pgfsetstrokecolor{textcolor}%
\pgfsetfillcolor{textcolor}%
\pgftext[x=2.539569in,y=2.663523in,,base]{\color{textcolor}\rmfamily\fontsize{11.000000}{13.200000}\selectfont Active...}%
\end{pgfscope}%
\begin{pgfscope}%
\pgfsetbuttcap%
\pgfsetmiterjoin%
\definecolor{currentfill}{rgb}{1.000000,1.000000,1.000000}%
\pgfsetfillcolor{currentfill}%
\pgfsetlinewidth{0.000000pt}%
\definecolor{currentstroke}{rgb}{0.000000,0.000000,0.000000}%
\pgfsetstrokecolor{currentstroke}%
\pgfsetstrokeopacity{0.000000}%
\pgfsetdash{}{0pt}%
\pgfpathmoveto{\pgfqpoint{3.116696in}{2.337032in}}%
\pgfpathlineto{\pgfqpoint{3.941164in}{2.337032in}}%
\pgfpathlineto{\pgfqpoint{3.941164in}{2.580190in}}%
\pgfpathlineto{\pgfqpoint{3.116696in}{2.580190in}}%
\pgfpathlineto{\pgfqpoint{3.116696in}{2.337032in}}%
\pgfpathclose%
\pgfusepath{fill}%
\end{pgfscope}%
\begin{pgfscope}%
\pgfpathrectangle{\pgfqpoint{3.116696in}{2.337032in}}{\pgfqpoint{0.824468in}{0.243158in}}%
\pgfusepath{clip}%
\pgfsetbuttcap%
\pgfsetmiterjoin%
\definecolor{currentfill}{rgb}{0.121569,0.466667,0.705882}%
\pgfsetfillcolor{currentfill}%
\pgfsetfillopacity{0.500000}%
\pgfsetlinewidth{1.003750pt}%
\definecolor{currentstroke}{rgb}{0.000000,0.000000,0.000000}%
\pgfsetstrokecolor{currentstroke}%
\pgfsetdash{}{0pt}%
\pgfpathmoveto{\pgfqpoint{3.154172in}{2.337032in}}%
\pgfpathlineto{\pgfqpoint{3.304075in}{2.337032in}}%
\pgfpathlineto{\pgfqpoint{3.304075in}{2.381165in}}%
\pgfpathlineto{\pgfqpoint{3.154172in}{2.381165in}}%
\pgfpathlineto{\pgfqpoint{3.154172in}{2.337032in}}%
\pgfpathclose%
\pgfusepath{stroke,fill}%
\end{pgfscope}%
\begin{pgfscope}%
\pgfpathrectangle{\pgfqpoint{3.116696in}{2.337032in}}{\pgfqpoint{0.824468in}{0.243158in}}%
\pgfusepath{clip}%
\pgfsetbuttcap%
\pgfsetmiterjoin%
\definecolor{currentfill}{rgb}{0.121569,0.466667,0.705882}%
\pgfsetfillcolor{currentfill}%
\pgfsetfillopacity{0.500000}%
\pgfsetlinewidth{1.003750pt}%
\definecolor{currentstroke}{rgb}{0.000000,0.000000,0.000000}%
\pgfsetstrokecolor{currentstroke}%
\pgfsetdash{}{0pt}%
\pgfpathmoveto{\pgfqpoint{3.304075in}{2.337032in}}%
\pgfpathlineto{\pgfqpoint{3.453979in}{2.337032in}}%
\pgfpathlineto{\pgfqpoint{3.453979in}{2.354418in}}%
\pgfpathlineto{\pgfqpoint{3.304075in}{2.354418in}}%
\pgfpathlineto{\pgfqpoint{3.304075in}{2.337032in}}%
\pgfpathclose%
\pgfusepath{stroke,fill}%
\end{pgfscope}%
\begin{pgfscope}%
\pgfpathrectangle{\pgfqpoint{3.116696in}{2.337032in}}{\pgfqpoint{0.824468in}{0.243158in}}%
\pgfusepath{clip}%
\pgfsetbuttcap%
\pgfsetmiterjoin%
\definecolor{currentfill}{rgb}{0.121569,0.466667,0.705882}%
\pgfsetfillcolor{currentfill}%
\pgfsetfillopacity{0.500000}%
\pgfsetlinewidth{1.003750pt}%
\definecolor{currentstroke}{rgb}{0.000000,0.000000,0.000000}%
\pgfsetstrokecolor{currentstroke}%
\pgfsetdash{}{0pt}%
\pgfpathmoveto{\pgfqpoint{3.453979in}{2.337032in}}%
\pgfpathlineto{\pgfqpoint{3.603882in}{2.337032in}}%
\pgfpathlineto{\pgfqpoint{3.603882in}{2.347488in}}%
\pgfpathlineto{\pgfqpoint{3.453979in}{2.347488in}}%
\pgfpathlineto{\pgfqpoint{3.453979in}{2.337032in}}%
\pgfpathclose%
\pgfusepath{stroke,fill}%
\end{pgfscope}%
\begin{pgfscope}%
\pgfpathrectangle{\pgfqpoint{3.116696in}{2.337032in}}{\pgfqpoint{0.824468in}{0.243158in}}%
\pgfusepath{clip}%
\pgfsetbuttcap%
\pgfsetmiterjoin%
\definecolor{currentfill}{rgb}{0.121569,0.466667,0.705882}%
\pgfsetfillcolor{currentfill}%
\pgfsetfillopacity{0.500000}%
\pgfsetlinewidth{1.003750pt}%
\definecolor{currentstroke}{rgb}{0.000000,0.000000,0.000000}%
\pgfsetstrokecolor{currentstroke}%
\pgfsetdash{}{0pt}%
\pgfpathmoveto{\pgfqpoint{3.603882in}{2.337032in}}%
\pgfpathlineto{\pgfqpoint{3.753785in}{2.337032in}}%
\pgfpathlineto{\pgfqpoint{3.753785in}{2.339707in}}%
\pgfpathlineto{\pgfqpoint{3.603882in}{2.339707in}}%
\pgfpathlineto{\pgfqpoint{3.603882in}{2.337032in}}%
\pgfpathclose%
\pgfusepath{stroke,fill}%
\end{pgfscope}%
\begin{pgfscope}%
\pgfpathrectangle{\pgfqpoint{3.116696in}{2.337032in}}{\pgfqpoint{0.824468in}{0.243158in}}%
\pgfusepath{clip}%
\pgfsetbuttcap%
\pgfsetmiterjoin%
\definecolor{currentfill}{rgb}{0.121569,0.466667,0.705882}%
\pgfsetfillcolor{currentfill}%
\pgfsetfillopacity{0.500000}%
\pgfsetlinewidth{1.003750pt}%
\definecolor{currentstroke}{rgb}{0.000000,0.000000,0.000000}%
\pgfsetstrokecolor{currentstroke}%
\pgfsetdash{}{0pt}%
\pgfpathmoveto{\pgfqpoint{3.753785in}{2.337032in}}%
\pgfpathlineto{\pgfqpoint{3.903688in}{2.337032in}}%
\pgfpathlineto{\pgfqpoint{3.903688in}{2.338856in}}%
\pgfpathlineto{\pgfqpoint{3.753785in}{2.338856in}}%
\pgfpathlineto{\pgfqpoint{3.753785in}{2.337032in}}%
\pgfpathclose%
\pgfusepath{stroke,fill}%
\end{pgfscope}%
\begin{pgfscope}%
\pgfsetrectcap%
\pgfsetmiterjoin%
\pgfsetlinewidth{0.803000pt}%
\definecolor{currentstroke}{rgb}{0.000000,0.000000,0.000000}%
\pgfsetstrokecolor{currentstroke}%
\pgfsetdash{}{0pt}%
\pgfpathmoveto{\pgfqpoint{3.116696in}{2.337032in}}%
\pgfpathlineto{\pgfqpoint{3.116696in}{2.580190in}}%
\pgfusepath{stroke}%
\end{pgfscope}%
\begin{pgfscope}%
\pgfsetrectcap%
\pgfsetmiterjoin%
\pgfsetlinewidth{0.803000pt}%
\definecolor{currentstroke}{rgb}{0.000000,0.000000,0.000000}%
\pgfsetstrokecolor{currentstroke}%
\pgfsetdash{}{0pt}%
\pgfpathmoveto{\pgfqpoint{3.941164in}{2.337032in}}%
\pgfpathlineto{\pgfqpoint{3.941164in}{2.580190in}}%
\pgfusepath{stroke}%
\end{pgfscope}%
\begin{pgfscope}%
\pgfsetrectcap%
\pgfsetmiterjoin%
\pgfsetlinewidth{0.803000pt}%
\definecolor{currentstroke}{rgb}{0.000000,0.000000,0.000000}%
\pgfsetstrokecolor{currentstroke}%
\pgfsetdash{}{0pt}%
\pgfpathmoveto{\pgfqpoint{3.116696in}{2.337032in}}%
\pgfpathlineto{\pgfqpoint{3.941164in}{2.337032in}}%
\pgfusepath{stroke}%
\end{pgfscope}%
\begin{pgfscope}%
\pgfsetrectcap%
\pgfsetmiterjoin%
\pgfsetlinewidth{0.803000pt}%
\definecolor{currentstroke}{rgb}{0.000000,0.000000,0.000000}%
\pgfsetstrokecolor{currentstroke}%
\pgfsetdash{}{0pt}%
\pgfpathmoveto{\pgfqpoint{3.116696in}{2.580190in}}%
\pgfpathlineto{\pgfqpoint{3.941164in}{2.580190in}}%
\pgfusepath{stroke}%
\end{pgfscope}%
\begin{pgfscope}%
\definecolor{textcolor}{rgb}{0.000000,0.000000,0.000000}%
\pgfsetstrokecolor{textcolor}%
\pgfsetfillcolor{textcolor}%
\pgftext[x=3.528930in,y=2.663523in,,base]{\color{textcolor}\rmfamily\fontsize{11.000000}{13.200000}\selectfont AXA}%
\end{pgfscope}%
\begin{pgfscope}%
\pgfsetbuttcap%
\pgfsetmiterjoin%
\definecolor{currentfill}{rgb}{1.000000,1.000000,1.000000}%
\pgfsetfillcolor{currentfill}%
\pgfsetlinewidth{0.000000pt}%
\definecolor{currentstroke}{rgb}{0.000000,0.000000,0.000000}%
\pgfsetstrokecolor{currentstroke}%
\pgfsetstrokeopacity{0.000000}%
\pgfsetdash{}{0pt}%
\pgfpathmoveto{\pgfqpoint{4.106058in}{2.337032in}}%
\pgfpathlineto{\pgfqpoint{4.930526in}{2.337032in}}%
\pgfpathlineto{\pgfqpoint{4.930526in}{2.580190in}}%
\pgfpathlineto{\pgfqpoint{4.106058in}{2.580190in}}%
\pgfpathlineto{\pgfqpoint{4.106058in}{2.337032in}}%
\pgfpathclose%
\pgfusepath{fill}%
\end{pgfscope}%
\begin{pgfscope}%
\pgfpathrectangle{\pgfqpoint{4.106058in}{2.337032in}}{\pgfqpoint{0.824468in}{0.243158in}}%
\pgfusepath{clip}%
\pgfsetbuttcap%
\pgfsetmiterjoin%
\definecolor{currentfill}{rgb}{0.121569,0.466667,0.705882}%
\pgfsetfillcolor{currentfill}%
\pgfsetfillopacity{0.500000}%
\pgfsetlinewidth{1.003750pt}%
\definecolor{currentstroke}{rgb}{0.000000,0.000000,0.000000}%
\pgfsetstrokecolor{currentstroke}%
\pgfsetdash{}{0pt}%
\pgfpathmoveto{\pgfqpoint{4.143534in}{2.337032in}}%
\pgfpathlineto{\pgfqpoint{4.293437in}{2.337032in}}%
\pgfpathlineto{\pgfqpoint{4.293437in}{2.345056in}}%
\pgfpathlineto{\pgfqpoint{4.143534in}{2.345056in}}%
\pgfpathlineto{\pgfqpoint{4.143534in}{2.337032in}}%
\pgfpathclose%
\pgfusepath{stroke,fill}%
\end{pgfscope}%
\begin{pgfscope}%
\pgfpathrectangle{\pgfqpoint{4.106058in}{2.337032in}}{\pgfqpoint{0.824468in}{0.243158in}}%
\pgfusepath{clip}%
\pgfsetbuttcap%
\pgfsetmiterjoin%
\definecolor{currentfill}{rgb}{0.121569,0.466667,0.705882}%
\pgfsetfillcolor{currentfill}%
\pgfsetfillopacity{0.500000}%
\pgfsetlinewidth{1.003750pt}%
\definecolor{currentstroke}{rgb}{0.000000,0.000000,0.000000}%
\pgfsetstrokecolor{currentstroke}%
\pgfsetdash{}{0pt}%
\pgfpathmoveto{\pgfqpoint{4.293437in}{2.337032in}}%
\pgfpathlineto{\pgfqpoint{4.443340in}{2.337032in}}%
\pgfpathlineto{\pgfqpoint{4.443340in}{2.339828in}}%
\pgfpathlineto{\pgfqpoint{4.293437in}{2.339828in}}%
\pgfpathlineto{\pgfqpoint{4.293437in}{2.337032in}}%
\pgfpathclose%
\pgfusepath{stroke,fill}%
\end{pgfscope}%
\begin{pgfscope}%
\pgfpathrectangle{\pgfqpoint{4.106058in}{2.337032in}}{\pgfqpoint{0.824468in}{0.243158in}}%
\pgfusepath{clip}%
\pgfsetbuttcap%
\pgfsetmiterjoin%
\definecolor{currentfill}{rgb}{0.121569,0.466667,0.705882}%
\pgfsetfillcolor{currentfill}%
\pgfsetfillopacity{0.500000}%
\pgfsetlinewidth{1.003750pt}%
\definecolor{currentstroke}{rgb}{0.000000,0.000000,0.000000}%
\pgfsetstrokecolor{currentstroke}%
\pgfsetdash{}{0pt}%
\pgfpathmoveto{\pgfqpoint{4.443340in}{2.337032in}}%
\pgfpathlineto{\pgfqpoint{4.593244in}{2.337032in}}%
\pgfpathlineto{\pgfqpoint{4.593244in}{2.337640in}}%
\pgfpathlineto{\pgfqpoint{4.443340in}{2.337640in}}%
\pgfpathlineto{\pgfqpoint{4.443340in}{2.337032in}}%
\pgfpathclose%
\pgfusepath{stroke,fill}%
\end{pgfscope}%
\begin{pgfscope}%
\pgfpathrectangle{\pgfqpoint{4.106058in}{2.337032in}}{\pgfqpoint{0.824468in}{0.243158in}}%
\pgfusepath{clip}%
\pgfsetbuttcap%
\pgfsetmiterjoin%
\definecolor{currentfill}{rgb}{0.121569,0.466667,0.705882}%
\pgfsetfillcolor{currentfill}%
\pgfsetfillopacity{0.500000}%
\pgfsetlinewidth{1.003750pt}%
\definecolor{currentstroke}{rgb}{0.000000,0.000000,0.000000}%
\pgfsetstrokecolor{currentstroke}%
\pgfsetdash{}{0pt}%
\pgfpathmoveto{\pgfqpoint{4.593244in}{2.337032in}}%
\pgfpathlineto{\pgfqpoint{4.743147in}{2.337032in}}%
\pgfpathlineto{\pgfqpoint{4.743147in}{2.337032in}}%
\pgfpathlineto{\pgfqpoint{4.593244in}{2.337032in}}%
\pgfpathlineto{\pgfqpoint{4.593244in}{2.337032in}}%
\pgfpathclose%
\pgfusepath{stroke,fill}%
\end{pgfscope}%
\begin{pgfscope}%
\pgfpathrectangle{\pgfqpoint{4.106058in}{2.337032in}}{\pgfqpoint{0.824468in}{0.243158in}}%
\pgfusepath{clip}%
\pgfsetbuttcap%
\pgfsetmiterjoin%
\definecolor{currentfill}{rgb}{0.121569,0.466667,0.705882}%
\pgfsetfillcolor{currentfill}%
\pgfsetfillopacity{0.500000}%
\pgfsetlinewidth{1.003750pt}%
\definecolor{currentstroke}{rgb}{0.000000,0.000000,0.000000}%
\pgfsetstrokecolor{currentstroke}%
\pgfsetdash{}{0pt}%
\pgfpathmoveto{\pgfqpoint{4.743147in}{2.337032in}}%
\pgfpathlineto{\pgfqpoint{4.893050in}{2.337032in}}%
\pgfpathlineto{\pgfqpoint{4.893050in}{2.337275in}}%
\pgfpathlineto{\pgfqpoint{4.743147in}{2.337275in}}%
\pgfpathlineto{\pgfqpoint{4.743147in}{2.337032in}}%
\pgfpathclose%
\pgfusepath{stroke,fill}%
\end{pgfscope}%
\begin{pgfscope}%
\pgfsetrectcap%
\pgfsetmiterjoin%
\pgfsetlinewidth{0.803000pt}%
\definecolor{currentstroke}{rgb}{0.000000,0.000000,0.000000}%
\pgfsetstrokecolor{currentstroke}%
\pgfsetdash{}{0pt}%
\pgfpathmoveto{\pgfqpoint{4.106058in}{2.337032in}}%
\pgfpathlineto{\pgfqpoint{4.106058in}{2.580190in}}%
\pgfusepath{stroke}%
\end{pgfscope}%
\begin{pgfscope}%
\pgfsetrectcap%
\pgfsetmiterjoin%
\pgfsetlinewidth{0.803000pt}%
\definecolor{currentstroke}{rgb}{0.000000,0.000000,0.000000}%
\pgfsetstrokecolor{currentstroke}%
\pgfsetdash{}{0pt}%
\pgfpathmoveto{\pgfqpoint{4.930526in}{2.337032in}}%
\pgfpathlineto{\pgfqpoint{4.930526in}{2.580190in}}%
\pgfusepath{stroke}%
\end{pgfscope}%
\begin{pgfscope}%
\pgfsetrectcap%
\pgfsetmiterjoin%
\pgfsetlinewidth{0.803000pt}%
\definecolor{currentstroke}{rgb}{0.000000,0.000000,0.000000}%
\pgfsetstrokecolor{currentstroke}%
\pgfsetdash{}{0pt}%
\pgfpathmoveto{\pgfqpoint{4.106058in}{2.337032in}}%
\pgfpathlineto{\pgfqpoint{4.930526in}{2.337032in}}%
\pgfusepath{stroke}%
\end{pgfscope}%
\begin{pgfscope}%
\pgfsetrectcap%
\pgfsetmiterjoin%
\pgfsetlinewidth{0.803000pt}%
\definecolor{currentstroke}{rgb}{0.000000,0.000000,0.000000}%
\pgfsetstrokecolor{currentstroke}%
\pgfsetdash{}{0pt}%
\pgfpathmoveto{\pgfqpoint{4.106058in}{2.580190in}}%
\pgfpathlineto{\pgfqpoint{4.930526in}{2.580190in}}%
\pgfusepath{stroke}%
\end{pgfscope}%
\begin{pgfscope}%
\definecolor{textcolor}{rgb}{0.000000,0.000000,0.000000}%
\pgfsetstrokecolor{textcolor}%
\pgfsetfillcolor{textcolor}%
\pgftext[x=4.518292in,y=2.663523in,,base]{\color{textcolor}\rmfamily\fontsize{11.000000}{13.200000}\selectfont Sogessur}%
\end{pgfscope}%
\begin{pgfscope}%
\pgfsetbuttcap%
\pgfsetmiterjoin%
\definecolor{currentfill}{rgb}{1.000000,1.000000,1.000000}%
\pgfsetfillcolor{currentfill}%
\pgfsetlinewidth{0.000000pt}%
\definecolor{currentstroke}{rgb}{0.000000,0.000000,0.000000}%
\pgfsetstrokecolor{currentstroke}%
\pgfsetstrokeopacity{0.000000}%
\pgfsetdash{}{0pt}%
\pgfpathmoveto{\pgfqpoint{5.095420in}{2.337032in}}%
\pgfpathlineto{\pgfqpoint{5.919888in}{2.337032in}}%
\pgfpathlineto{\pgfqpoint{5.919888in}{2.580190in}}%
\pgfpathlineto{\pgfqpoint{5.095420in}{2.580190in}}%
\pgfpathlineto{\pgfqpoint{5.095420in}{2.337032in}}%
\pgfpathclose%
\pgfusepath{fill}%
\end{pgfscope}%
\begin{pgfscope}%
\pgfpathrectangle{\pgfqpoint{5.095420in}{2.337032in}}{\pgfqpoint{0.824468in}{0.243158in}}%
\pgfusepath{clip}%
\pgfsetbuttcap%
\pgfsetmiterjoin%
\definecolor{currentfill}{rgb}{0.121569,0.466667,0.705882}%
\pgfsetfillcolor{currentfill}%
\pgfsetfillopacity{0.500000}%
\pgfsetlinewidth{1.003750pt}%
\definecolor{currentstroke}{rgb}{0.000000,0.000000,0.000000}%
\pgfsetstrokecolor{currentstroke}%
\pgfsetdash{}{0pt}%
\pgfpathmoveto{\pgfqpoint{5.132895in}{2.337032in}}%
\pgfpathlineto{\pgfqpoint{5.282799in}{2.337032in}}%
\pgfpathlineto{\pgfqpoint{5.282799in}{2.366941in}}%
\pgfpathlineto{\pgfqpoint{5.132895in}{2.366941in}}%
\pgfpathlineto{\pgfqpoint{5.132895in}{2.337032in}}%
\pgfpathclose%
\pgfusepath{stroke,fill}%
\end{pgfscope}%
\begin{pgfscope}%
\pgfpathrectangle{\pgfqpoint{5.095420in}{2.337032in}}{\pgfqpoint{0.824468in}{0.243158in}}%
\pgfusepath{clip}%
\pgfsetbuttcap%
\pgfsetmiterjoin%
\definecolor{currentfill}{rgb}{0.121569,0.466667,0.705882}%
\pgfsetfillcolor{currentfill}%
\pgfsetfillopacity{0.500000}%
\pgfsetlinewidth{1.003750pt}%
\definecolor{currentstroke}{rgb}{0.000000,0.000000,0.000000}%
\pgfsetstrokecolor{currentstroke}%
\pgfsetdash{}{0pt}%
\pgfpathmoveto{\pgfqpoint{5.282799in}{2.337032in}}%
\pgfpathlineto{\pgfqpoint{5.432702in}{2.337032in}}%
\pgfpathlineto{\pgfqpoint{5.432702in}{2.345664in}}%
\pgfpathlineto{\pgfqpoint{5.282799in}{2.345664in}}%
\pgfpathlineto{\pgfqpoint{5.282799in}{2.337032in}}%
\pgfpathclose%
\pgfusepath{stroke,fill}%
\end{pgfscope}%
\begin{pgfscope}%
\pgfpathrectangle{\pgfqpoint{5.095420in}{2.337032in}}{\pgfqpoint{0.824468in}{0.243158in}}%
\pgfusepath{clip}%
\pgfsetbuttcap%
\pgfsetmiterjoin%
\definecolor{currentfill}{rgb}{0.121569,0.466667,0.705882}%
\pgfsetfillcolor{currentfill}%
\pgfsetfillopacity{0.500000}%
\pgfsetlinewidth{1.003750pt}%
\definecolor{currentstroke}{rgb}{0.000000,0.000000,0.000000}%
\pgfsetstrokecolor{currentstroke}%
\pgfsetdash{}{0pt}%
\pgfpathmoveto{\pgfqpoint{5.432702in}{2.337032in}}%
\pgfpathlineto{\pgfqpoint{5.582605in}{2.337032in}}%
\pgfpathlineto{\pgfqpoint{5.582605in}{2.340680in}}%
\pgfpathlineto{\pgfqpoint{5.432702in}{2.340680in}}%
\pgfpathlineto{\pgfqpoint{5.432702in}{2.337032in}}%
\pgfpathclose%
\pgfusepath{stroke,fill}%
\end{pgfscope}%
\begin{pgfscope}%
\pgfpathrectangle{\pgfqpoint{5.095420in}{2.337032in}}{\pgfqpoint{0.824468in}{0.243158in}}%
\pgfusepath{clip}%
\pgfsetbuttcap%
\pgfsetmiterjoin%
\definecolor{currentfill}{rgb}{0.121569,0.466667,0.705882}%
\pgfsetfillcolor{currentfill}%
\pgfsetfillopacity{0.500000}%
\pgfsetlinewidth{1.003750pt}%
\definecolor{currentstroke}{rgb}{0.000000,0.000000,0.000000}%
\pgfsetstrokecolor{currentstroke}%
\pgfsetdash{}{0pt}%
\pgfpathmoveto{\pgfqpoint{5.582605in}{2.337032in}}%
\pgfpathlineto{\pgfqpoint{5.732509in}{2.337032in}}%
\pgfpathlineto{\pgfqpoint{5.732509in}{2.337275in}}%
\pgfpathlineto{\pgfqpoint{5.582605in}{2.337275in}}%
\pgfpathlineto{\pgfqpoint{5.582605in}{2.337032in}}%
\pgfpathclose%
\pgfusepath{stroke,fill}%
\end{pgfscope}%
\begin{pgfscope}%
\pgfpathrectangle{\pgfqpoint{5.095420in}{2.337032in}}{\pgfqpoint{0.824468in}{0.243158in}}%
\pgfusepath{clip}%
\pgfsetbuttcap%
\pgfsetmiterjoin%
\definecolor{currentfill}{rgb}{0.121569,0.466667,0.705882}%
\pgfsetfillcolor{currentfill}%
\pgfsetfillopacity{0.500000}%
\pgfsetlinewidth{1.003750pt}%
\definecolor{currentstroke}{rgb}{0.000000,0.000000,0.000000}%
\pgfsetstrokecolor{currentstroke}%
\pgfsetdash{}{0pt}%
\pgfpathmoveto{\pgfqpoint{5.732509in}{2.337032in}}%
\pgfpathlineto{\pgfqpoint{5.882412in}{2.337032in}}%
\pgfpathlineto{\pgfqpoint{5.882412in}{2.337154in}}%
\pgfpathlineto{\pgfqpoint{5.732509in}{2.337154in}}%
\pgfpathlineto{\pgfqpoint{5.732509in}{2.337032in}}%
\pgfpathclose%
\pgfusepath{stroke,fill}%
\end{pgfscope}%
\begin{pgfscope}%
\pgfsetrectcap%
\pgfsetmiterjoin%
\pgfsetlinewidth{0.803000pt}%
\definecolor{currentstroke}{rgb}{0.000000,0.000000,0.000000}%
\pgfsetstrokecolor{currentstroke}%
\pgfsetdash{}{0pt}%
\pgfpathmoveto{\pgfqpoint{5.095420in}{2.337032in}}%
\pgfpathlineto{\pgfqpoint{5.095420in}{2.580190in}}%
\pgfusepath{stroke}%
\end{pgfscope}%
\begin{pgfscope}%
\pgfsetrectcap%
\pgfsetmiterjoin%
\pgfsetlinewidth{0.803000pt}%
\definecolor{currentstroke}{rgb}{0.000000,0.000000,0.000000}%
\pgfsetstrokecolor{currentstroke}%
\pgfsetdash{}{0pt}%
\pgfpathmoveto{\pgfqpoint{5.919888in}{2.337032in}}%
\pgfpathlineto{\pgfqpoint{5.919888in}{2.580190in}}%
\pgfusepath{stroke}%
\end{pgfscope}%
\begin{pgfscope}%
\pgfsetrectcap%
\pgfsetmiterjoin%
\pgfsetlinewidth{0.803000pt}%
\definecolor{currentstroke}{rgb}{0.000000,0.000000,0.000000}%
\pgfsetstrokecolor{currentstroke}%
\pgfsetdash{}{0pt}%
\pgfpathmoveto{\pgfqpoint{5.095420in}{2.337032in}}%
\pgfpathlineto{\pgfqpoint{5.919888in}{2.337032in}}%
\pgfusepath{stroke}%
\end{pgfscope}%
\begin{pgfscope}%
\pgfsetrectcap%
\pgfsetmiterjoin%
\pgfsetlinewidth{0.803000pt}%
\definecolor{currentstroke}{rgb}{0.000000,0.000000,0.000000}%
\pgfsetstrokecolor{currentstroke}%
\pgfsetdash{}{0pt}%
\pgfpathmoveto{\pgfqpoint{5.095420in}{2.580190in}}%
\pgfpathlineto{\pgfqpoint{5.919888in}{2.580190in}}%
\pgfusepath{stroke}%
\end{pgfscope}%
\begin{pgfscope}%
\definecolor{textcolor}{rgb}{0.000000,0.000000,0.000000}%
\pgfsetstrokecolor{textcolor}%
\pgfsetfillcolor{textcolor}%
\pgftext[x=5.507654in,y=2.663523in,,base]{\color{textcolor}\rmfamily\fontsize{11.000000}{13.200000}\selectfont Ag2r L...}%
\end{pgfscope}%
\begin{pgfscope}%
\pgfsetbuttcap%
\pgfsetmiterjoin%
\definecolor{currentfill}{rgb}{1.000000,1.000000,1.000000}%
\pgfsetfillcolor{currentfill}%
\pgfsetlinewidth{0.000000pt}%
\definecolor{currentstroke}{rgb}{0.000000,0.000000,0.000000}%
\pgfsetstrokecolor{currentstroke}%
\pgfsetstrokeopacity{0.000000}%
\pgfsetdash{}{0pt}%
\pgfpathmoveto{\pgfqpoint{6.084781in}{2.337032in}}%
\pgfpathlineto{\pgfqpoint{6.909249in}{2.337032in}}%
\pgfpathlineto{\pgfqpoint{6.909249in}{2.580190in}}%
\pgfpathlineto{\pgfqpoint{6.084781in}{2.580190in}}%
\pgfpathlineto{\pgfqpoint{6.084781in}{2.337032in}}%
\pgfpathclose%
\pgfusepath{fill}%
\end{pgfscope}%
\begin{pgfscope}%
\pgfpathrectangle{\pgfqpoint{6.084781in}{2.337032in}}{\pgfqpoint{0.824468in}{0.243158in}}%
\pgfusepath{clip}%
\pgfsetbuttcap%
\pgfsetmiterjoin%
\definecolor{currentfill}{rgb}{0.121569,0.466667,0.705882}%
\pgfsetfillcolor{currentfill}%
\pgfsetfillopacity{0.500000}%
\pgfsetlinewidth{1.003750pt}%
\definecolor{currentstroke}{rgb}{0.000000,0.000000,0.000000}%
\pgfsetstrokecolor{currentstroke}%
\pgfsetdash{}{0pt}%
\pgfpathmoveto{\pgfqpoint{6.122257in}{2.337032in}}%
\pgfpathlineto{\pgfqpoint{6.272160in}{2.337032in}}%
\pgfpathlineto{\pgfqpoint{6.272160in}{2.354904in}}%
\pgfpathlineto{\pgfqpoint{6.122257in}{2.354904in}}%
\pgfpathlineto{\pgfqpoint{6.122257in}{2.337032in}}%
\pgfpathclose%
\pgfusepath{stroke,fill}%
\end{pgfscope}%
\begin{pgfscope}%
\pgfpathrectangle{\pgfqpoint{6.084781in}{2.337032in}}{\pgfqpoint{0.824468in}{0.243158in}}%
\pgfusepath{clip}%
\pgfsetbuttcap%
\pgfsetmiterjoin%
\definecolor{currentfill}{rgb}{0.121569,0.466667,0.705882}%
\pgfsetfillcolor{currentfill}%
\pgfsetfillopacity{0.500000}%
\pgfsetlinewidth{1.003750pt}%
\definecolor{currentstroke}{rgb}{0.000000,0.000000,0.000000}%
\pgfsetstrokecolor{currentstroke}%
\pgfsetdash{}{0pt}%
\pgfpathmoveto{\pgfqpoint{6.272160in}{2.337032in}}%
\pgfpathlineto{\pgfqpoint{6.422064in}{2.337032in}}%
\pgfpathlineto{\pgfqpoint{6.422064in}{2.340680in}}%
\pgfpathlineto{\pgfqpoint{6.272160in}{2.340680in}}%
\pgfpathlineto{\pgfqpoint{6.272160in}{2.337032in}}%
\pgfpathclose%
\pgfusepath{stroke,fill}%
\end{pgfscope}%
\begin{pgfscope}%
\pgfpathrectangle{\pgfqpoint{6.084781in}{2.337032in}}{\pgfqpoint{0.824468in}{0.243158in}}%
\pgfusepath{clip}%
\pgfsetbuttcap%
\pgfsetmiterjoin%
\definecolor{currentfill}{rgb}{0.121569,0.466667,0.705882}%
\pgfsetfillcolor{currentfill}%
\pgfsetfillopacity{0.500000}%
\pgfsetlinewidth{1.003750pt}%
\definecolor{currentstroke}{rgb}{0.000000,0.000000,0.000000}%
\pgfsetstrokecolor{currentstroke}%
\pgfsetdash{}{0pt}%
\pgfpathmoveto{\pgfqpoint{6.422064in}{2.337032in}}%
\pgfpathlineto{\pgfqpoint{6.571967in}{2.337032in}}%
\pgfpathlineto{\pgfqpoint{6.571967in}{2.339464in}}%
\pgfpathlineto{\pgfqpoint{6.422064in}{2.339464in}}%
\pgfpathlineto{\pgfqpoint{6.422064in}{2.337032in}}%
\pgfpathclose%
\pgfusepath{stroke,fill}%
\end{pgfscope}%
\begin{pgfscope}%
\pgfpathrectangle{\pgfqpoint{6.084781in}{2.337032in}}{\pgfqpoint{0.824468in}{0.243158in}}%
\pgfusepath{clip}%
\pgfsetbuttcap%
\pgfsetmiterjoin%
\definecolor{currentfill}{rgb}{0.121569,0.466667,0.705882}%
\pgfsetfillcolor{currentfill}%
\pgfsetfillopacity{0.500000}%
\pgfsetlinewidth{1.003750pt}%
\definecolor{currentstroke}{rgb}{0.000000,0.000000,0.000000}%
\pgfsetstrokecolor{currentstroke}%
\pgfsetdash{}{0pt}%
\pgfpathmoveto{\pgfqpoint{6.571967in}{2.337032in}}%
\pgfpathlineto{\pgfqpoint{6.721870in}{2.337032in}}%
\pgfpathlineto{\pgfqpoint{6.721870in}{2.337640in}}%
\pgfpathlineto{\pgfqpoint{6.571967in}{2.337640in}}%
\pgfpathlineto{\pgfqpoint{6.571967in}{2.337032in}}%
\pgfpathclose%
\pgfusepath{stroke,fill}%
\end{pgfscope}%
\begin{pgfscope}%
\pgfpathrectangle{\pgfqpoint{6.084781in}{2.337032in}}{\pgfqpoint{0.824468in}{0.243158in}}%
\pgfusepath{clip}%
\pgfsetbuttcap%
\pgfsetmiterjoin%
\definecolor{currentfill}{rgb}{0.121569,0.466667,0.705882}%
\pgfsetfillcolor{currentfill}%
\pgfsetfillopacity{0.500000}%
\pgfsetlinewidth{1.003750pt}%
\definecolor{currentstroke}{rgb}{0.000000,0.000000,0.000000}%
\pgfsetstrokecolor{currentstroke}%
\pgfsetdash{}{0pt}%
\pgfpathmoveto{\pgfqpoint{6.721870in}{2.337032in}}%
\pgfpathlineto{\pgfqpoint{6.871774in}{2.337032in}}%
\pgfpathlineto{\pgfqpoint{6.871774in}{2.337883in}}%
\pgfpathlineto{\pgfqpoint{6.721870in}{2.337883in}}%
\pgfpathlineto{\pgfqpoint{6.721870in}{2.337032in}}%
\pgfpathclose%
\pgfusepath{stroke,fill}%
\end{pgfscope}%
\begin{pgfscope}%
\pgfsetrectcap%
\pgfsetmiterjoin%
\pgfsetlinewidth{0.803000pt}%
\definecolor{currentstroke}{rgb}{0.000000,0.000000,0.000000}%
\pgfsetstrokecolor{currentstroke}%
\pgfsetdash{}{0pt}%
\pgfpathmoveto{\pgfqpoint{6.084781in}{2.337032in}}%
\pgfpathlineto{\pgfqpoint{6.084781in}{2.580190in}}%
\pgfusepath{stroke}%
\end{pgfscope}%
\begin{pgfscope}%
\pgfsetrectcap%
\pgfsetmiterjoin%
\pgfsetlinewidth{0.803000pt}%
\definecolor{currentstroke}{rgb}{0.000000,0.000000,0.000000}%
\pgfsetstrokecolor{currentstroke}%
\pgfsetdash{}{0pt}%
\pgfpathmoveto{\pgfqpoint{6.909249in}{2.337032in}}%
\pgfpathlineto{\pgfqpoint{6.909249in}{2.580190in}}%
\pgfusepath{stroke}%
\end{pgfscope}%
\begin{pgfscope}%
\pgfsetrectcap%
\pgfsetmiterjoin%
\pgfsetlinewidth{0.803000pt}%
\definecolor{currentstroke}{rgb}{0.000000,0.000000,0.000000}%
\pgfsetstrokecolor{currentstroke}%
\pgfsetdash{}{0pt}%
\pgfpathmoveto{\pgfqpoint{6.084781in}{2.337032in}}%
\pgfpathlineto{\pgfqpoint{6.909249in}{2.337032in}}%
\pgfusepath{stroke}%
\end{pgfscope}%
\begin{pgfscope}%
\pgfsetrectcap%
\pgfsetmiterjoin%
\pgfsetlinewidth{0.803000pt}%
\definecolor{currentstroke}{rgb}{0.000000,0.000000,0.000000}%
\pgfsetstrokecolor{currentstroke}%
\pgfsetdash{}{0pt}%
\pgfpathmoveto{\pgfqpoint{6.084781in}{2.580190in}}%
\pgfpathlineto{\pgfqpoint{6.909249in}{2.580190in}}%
\pgfusepath{stroke}%
\end{pgfscope}%
\begin{pgfscope}%
\definecolor{textcolor}{rgb}{0.000000,0.000000,0.000000}%
\pgfsetstrokecolor{textcolor}%
\pgfsetfillcolor{textcolor}%
\pgftext[x=6.497015in,y=2.663523in,,base]{\color{textcolor}\rmfamily\fontsize{11.000000}{13.200000}\selectfont Mgen}%
\end{pgfscope}%
\begin{pgfscope}%
\pgfsetbuttcap%
\pgfsetmiterjoin%
\definecolor{currentfill}{rgb}{1.000000,1.000000,1.000000}%
\pgfsetfillcolor{currentfill}%
\pgfsetlinewidth{0.000000pt}%
\definecolor{currentstroke}{rgb}{0.000000,0.000000,0.000000}%
\pgfsetstrokecolor{currentstroke}%
\pgfsetstrokeopacity{0.000000}%
\pgfsetdash{}{0pt}%
\pgfpathmoveto{\pgfqpoint{7.074143in}{2.337032in}}%
\pgfpathlineto{\pgfqpoint{7.898611in}{2.337032in}}%
\pgfpathlineto{\pgfqpoint{7.898611in}{2.580190in}}%
\pgfpathlineto{\pgfqpoint{7.074143in}{2.580190in}}%
\pgfpathlineto{\pgfqpoint{7.074143in}{2.337032in}}%
\pgfpathclose%
\pgfusepath{fill}%
\end{pgfscope}%
\begin{pgfscope}%
\pgfpathrectangle{\pgfqpoint{7.074143in}{2.337032in}}{\pgfqpoint{0.824468in}{0.243158in}}%
\pgfusepath{clip}%
\pgfsetbuttcap%
\pgfsetmiterjoin%
\definecolor{currentfill}{rgb}{0.121569,0.466667,0.705882}%
\pgfsetfillcolor{currentfill}%
\pgfsetfillopacity{0.500000}%
\pgfsetlinewidth{1.003750pt}%
\definecolor{currentstroke}{rgb}{0.000000,0.000000,0.000000}%
\pgfsetstrokecolor{currentstroke}%
\pgfsetdash{}{0pt}%
\pgfpathmoveto{\pgfqpoint{7.111619in}{2.337032in}}%
\pgfpathlineto{\pgfqpoint{7.261522in}{2.337032in}}%
\pgfpathlineto{\pgfqpoint{7.261522in}{2.337275in}}%
\pgfpathlineto{\pgfqpoint{7.111619in}{2.337275in}}%
\pgfpathlineto{\pgfqpoint{7.111619in}{2.337032in}}%
\pgfpathclose%
\pgfusepath{stroke,fill}%
\end{pgfscope}%
\begin{pgfscope}%
\pgfpathrectangle{\pgfqpoint{7.074143in}{2.337032in}}{\pgfqpoint{0.824468in}{0.243158in}}%
\pgfusepath{clip}%
\pgfsetbuttcap%
\pgfsetmiterjoin%
\definecolor{currentfill}{rgb}{0.121569,0.466667,0.705882}%
\pgfsetfillcolor{currentfill}%
\pgfsetfillopacity{0.500000}%
\pgfsetlinewidth{1.003750pt}%
\definecolor{currentstroke}{rgb}{0.000000,0.000000,0.000000}%
\pgfsetstrokecolor{currentstroke}%
\pgfsetdash{}{0pt}%
\pgfpathmoveto{\pgfqpoint{7.261522in}{2.337032in}}%
\pgfpathlineto{\pgfqpoint{7.411425in}{2.337032in}}%
\pgfpathlineto{\pgfqpoint{7.411425in}{2.338248in}}%
\pgfpathlineto{\pgfqpoint{7.261522in}{2.338248in}}%
\pgfpathlineto{\pgfqpoint{7.261522in}{2.337032in}}%
\pgfpathclose%
\pgfusepath{stroke,fill}%
\end{pgfscope}%
\begin{pgfscope}%
\pgfpathrectangle{\pgfqpoint{7.074143in}{2.337032in}}{\pgfqpoint{0.824468in}{0.243158in}}%
\pgfusepath{clip}%
\pgfsetbuttcap%
\pgfsetmiterjoin%
\definecolor{currentfill}{rgb}{0.121569,0.466667,0.705882}%
\pgfsetfillcolor{currentfill}%
\pgfsetfillopacity{0.500000}%
\pgfsetlinewidth{1.003750pt}%
\definecolor{currentstroke}{rgb}{0.000000,0.000000,0.000000}%
\pgfsetstrokecolor{currentstroke}%
\pgfsetdash{}{0pt}%
\pgfpathmoveto{\pgfqpoint{7.411425in}{2.337032in}}%
\pgfpathlineto{\pgfqpoint{7.561329in}{2.337032in}}%
\pgfpathlineto{\pgfqpoint{7.561329in}{2.338977in}}%
\pgfpathlineto{\pgfqpoint{7.411425in}{2.338977in}}%
\pgfpathlineto{\pgfqpoint{7.411425in}{2.337032in}}%
\pgfpathclose%
\pgfusepath{stroke,fill}%
\end{pgfscope}%
\begin{pgfscope}%
\pgfpathrectangle{\pgfqpoint{7.074143in}{2.337032in}}{\pgfqpoint{0.824468in}{0.243158in}}%
\pgfusepath{clip}%
\pgfsetbuttcap%
\pgfsetmiterjoin%
\definecolor{currentfill}{rgb}{0.121569,0.466667,0.705882}%
\pgfsetfillcolor{currentfill}%
\pgfsetfillopacity{0.500000}%
\pgfsetlinewidth{1.003750pt}%
\definecolor{currentstroke}{rgb}{0.000000,0.000000,0.000000}%
\pgfsetstrokecolor{currentstroke}%
\pgfsetdash{}{0pt}%
\pgfpathmoveto{\pgfqpoint{7.561329in}{2.337032in}}%
\pgfpathlineto{\pgfqpoint{7.711232in}{2.337032in}}%
\pgfpathlineto{\pgfqpoint{7.711232in}{2.345300in}}%
\pgfpathlineto{\pgfqpoint{7.561329in}{2.345300in}}%
\pgfpathlineto{\pgfqpoint{7.561329in}{2.337032in}}%
\pgfpathclose%
\pgfusepath{stroke,fill}%
\end{pgfscope}%
\begin{pgfscope}%
\pgfpathrectangle{\pgfqpoint{7.074143in}{2.337032in}}{\pgfqpoint{0.824468in}{0.243158in}}%
\pgfusepath{clip}%
\pgfsetbuttcap%
\pgfsetmiterjoin%
\definecolor{currentfill}{rgb}{0.121569,0.466667,0.705882}%
\pgfsetfillcolor{currentfill}%
\pgfsetfillopacity{0.500000}%
\pgfsetlinewidth{1.003750pt}%
\definecolor{currentstroke}{rgb}{0.000000,0.000000,0.000000}%
\pgfsetstrokecolor{currentstroke}%
\pgfsetdash{}{0pt}%
\pgfpathmoveto{\pgfqpoint{7.711232in}{2.337032in}}%
\pgfpathlineto{\pgfqpoint{7.861135in}{2.337032in}}%
\pgfpathlineto{\pgfqpoint{7.861135in}{2.355147in}}%
\pgfpathlineto{\pgfqpoint{7.711232in}{2.355147in}}%
\pgfpathlineto{\pgfqpoint{7.711232in}{2.337032in}}%
\pgfpathclose%
\pgfusepath{stroke,fill}%
\end{pgfscope}%
\begin{pgfscope}%
\pgfsetrectcap%
\pgfsetmiterjoin%
\pgfsetlinewidth{0.803000pt}%
\definecolor{currentstroke}{rgb}{0.000000,0.000000,0.000000}%
\pgfsetstrokecolor{currentstroke}%
\pgfsetdash{}{0pt}%
\pgfpathmoveto{\pgfqpoint{7.074143in}{2.337032in}}%
\pgfpathlineto{\pgfqpoint{7.074143in}{2.580190in}}%
\pgfusepath{stroke}%
\end{pgfscope}%
\begin{pgfscope}%
\pgfsetrectcap%
\pgfsetmiterjoin%
\pgfsetlinewidth{0.803000pt}%
\definecolor{currentstroke}{rgb}{0.000000,0.000000,0.000000}%
\pgfsetstrokecolor{currentstroke}%
\pgfsetdash{}{0pt}%
\pgfpathmoveto{\pgfqpoint{7.898611in}{2.337032in}}%
\pgfpathlineto{\pgfqpoint{7.898611in}{2.580190in}}%
\pgfusepath{stroke}%
\end{pgfscope}%
\begin{pgfscope}%
\pgfsetrectcap%
\pgfsetmiterjoin%
\pgfsetlinewidth{0.803000pt}%
\definecolor{currentstroke}{rgb}{0.000000,0.000000,0.000000}%
\pgfsetstrokecolor{currentstroke}%
\pgfsetdash{}{0pt}%
\pgfpathmoveto{\pgfqpoint{7.074143in}{2.337032in}}%
\pgfpathlineto{\pgfqpoint{7.898611in}{2.337032in}}%
\pgfusepath{stroke}%
\end{pgfscope}%
\begin{pgfscope}%
\pgfsetrectcap%
\pgfsetmiterjoin%
\pgfsetlinewidth{0.803000pt}%
\definecolor{currentstroke}{rgb}{0.000000,0.000000,0.000000}%
\pgfsetstrokecolor{currentstroke}%
\pgfsetdash{}{0pt}%
\pgfpathmoveto{\pgfqpoint{7.074143in}{2.580190in}}%
\pgfpathlineto{\pgfqpoint{7.898611in}{2.580190in}}%
\pgfusepath{stroke}%
\end{pgfscope}%
\begin{pgfscope}%
\definecolor{textcolor}{rgb}{0.000000,0.000000,0.000000}%
\pgfsetstrokecolor{textcolor}%
\pgfsetfillcolor{textcolor}%
\pgftext[x=7.486377in,y=2.663523in,,base]{\color{textcolor}\rmfamily\fontsize{11.000000}{13.200000}\selectfont Zen'Up}%
\end{pgfscope}%
\begin{pgfscope}%
\pgfsetbuttcap%
\pgfsetmiterjoin%
\definecolor{currentfill}{rgb}{1.000000,1.000000,1.000000}%
\pgfsetfillcolor{currentfill}%
\pgfsetlinewidth{0.000000pt}%
\definecolor{currentstroke}{rgb}{0.000000,0.000000,0.000000}%
\pgfsetstrokecolor{currentstroke}%
\pgfsetstrokeopacity{0.000000}%
\pgfsetdash{}{0pt}%
\pgfpathmoveto{\pgfqpoint{0.148611in}{1.607558in}}%
\pgfpathlineto{\pgfqpoint{0.973079in}{1.607558in}}%
\pgfpathlineto{\pgfqpoint{0.973079in}{1.850716in}}%
\pgfpathlineto{\pgfqpoint{0.148611in}{1.850716in}}%
\pgfpathlineto{\pgfqpoint{0.148611in}{1.607558in}}%
\pgfpathclose%
\pgfusepath{fill}%
\end{pgfscope}%
\begin{pgfscope}%
\pgfpathrectangle{\pgfqpoint{0.148611in}{1.607558in}}{\pgfqpoint{0.824468in}{0.243158in}}%
\pgfusepath{clip}%
\pgfsetbuttcap%
\pgfsetmiterjoin%
\definecolor{currentfill}{rgb}{0.121569,0.466667,0.705882}%
\pgfsetfillcolor{currentfill}%
\pgfsetfillopacity{0.500000}%
\pgfsetlinewidth{1.003750pt}%
\definecolor{currentstroke}{rgb}{0.000000,0.000000,0.000000}%
\pgfsetstrokecolor{currentstroke}%
\pgfsetdash{}{0pt}%
\pgfpathmoveto{\pgfqpoint{0.186087in}{1.607558in}}%
\pgfpathlineto{\pgfqpoint{0.335990in}{1.607558in}}%
\pgfpathlineto{\pgfqpoint{0.335990in}{1.612178in}}%
\pgfpathlineto{\pgfqpoint{0.186087in}{1.612178in}}%
\pgfpathlineto{\pgfqpoint{0.186087in}{1.607558in}}%
\pgfpathclose%
\pgfusepath{stroke,fill}%
\end{pgfscope}%
\begin{pgfscope}%
\pgfpathrectangle{\pgfqpoint{0.148611in}{1.607558in}}{\pgfqpoint{0.824468in}{0.243158in}}%
\pgfusepath{clip}%
\pgfsetbuttcap%
\pgfsetmiterjoin%
\definecolor{currentfill}{rgb}{0.121569,0.466667,0.705882}%
\pgfsetfillcolor{currentfill}%
\pgfsetfillopacity{0.500000}%
\pgfsetlinewidth{1.003750pt}%
\definecolor{currentstroke}{rgb}{0.000000,0.000000,0.000000}%
\pgfsetstrokecolor{currentstroke}%
\pgfsetdash{}{0pt}%
\pgfpathmoveto{\pgfqpoint{0.335990in}{1.607558in}}%
\pgfpathlineto{\pgfqpoint{0.485894in}{1.607558in}}%
\pgfpathlineto{\pgfqpoint{0.485894in}{1.611571in}}%
\pgfpathlineto{\pgfqpoint{0.335990in}{1.611571in}}%
\pgfpathlineto{\pgfqpoint{0.335990in}{1.607558in}}%
\pgfpathclose%
\pgfusepath{stroke,fill}%
\end{pgfscope}%
\begin{pgfscope}%
\pgfpathrectangle{\pgfqpoint{0.148611in}{1.607558in}}{\pgfqpoint{0.824468in}{0.243158in}}%
\pgfusepath{clip}%
\pgfsetbuttcap%
\pgfsetmiterjoin%
\definecolor{currentfill}{rgb}{0.121569,0.466667,0.705882}%
\pgfsetfillcolor{currentfill}%
\pgfsetfillopacity{0.500000}%
\pgfsetlinewidth{1.003750pt}%
\definecolor{currentstroke}{rgb}{0.000000,0.000000,0.000000}%
\pgfsetstrokecolor{currentstroke}%
\pgfsetdash{}{0pt}%
\pgfpathmoveto{\pgfqpoint{0.485894in}{1.607558in}}%
\pgfpathlineto{\pgfqpoint{0.635797in}{1.607558in}}%
\pgfpathlineto{\pgfqpoint{0.635797in}{1.619960in}}%
\pgfpathlineto{\pgfqpoint{0.485894in}{1.619960in}}%
\pgfpathlineto{\pgfqpoint{0.485894in}{1.607558in}}%
\pgfpathclose%
\pgfusepath{stroke,fill}%
\end{pgfscope}%
\begin{pgfscope}%
\pgfpathrectangle{\pgfqpoint{0.148611in}{1.607558in}}{\pgfqpoint{0.824468in}{0.243158in}}%
\pgfusepath{clip}%
\pgfsetbuttcap%
\pgfsetmiterjoin%
\definecolor{currentfill}{rgb}{0.121569,0.466667,0.705882}%
\pgfsetfillcolor{currentfill}%
\pgfsetfillopacity{0.500000}%
\pgfsetlinewidth{1.003750pt}%
\definecolor{currentstroke}{rgb}{0.000000,0.000000,0.000000}%
\pgfsetstrokecolor{currentstroke}%
\pgfsetdash{}{0pt}%
\pgfpathmoveto{\pgfqpoint{0.635797in}{1.607558in}}%
\pgfpathlineto{\pgfqpoint{0.785700in}{1.607558in}}%
\pgfpathlineto{\pgfqpoint{0.785700in}{1.628713in}}%
\pgfpathlineto{\pgfqpoint{0.635797in}{1.628713in}}%
\pgfpathlineto{\pgfqpoint{0.635797in}{1.607558in}}%
\pgfpathclose%
\pgfusepath{stroke,fill}%
\end{pgfscope}%
\begin{pgfscope}%
\pgfpathrectangle{\pgfqpoint{0.148611in}{1.607558in}}{\pgfqpoint{0.824468in}{0.243158in}}%
\pgfusepath{clip}%
\pgfsetbuttcap%
\pgfsetmiterjoin%
\definecolor{currentfill}{rgb}{0.121569,0.466667,0.705882}%
\pgfsetfillcolor{currentfill}%
\pgfsetfillopacity{0.500000}%
\pgfsetlinewidth{1.003750pt}%
\definecolor{currentstroke}{rgb}{0.000000,0.000000,0.000000}%
\pgfsetstrokecolor{currentstroke}%
\pgfsetdash{}{0pt}%
\pgfpathmoveto{\pgfqpoint{0.785700in}{1.607558in}}%
\pgfpathlineto{\pgfqpoint{0.935603in}{1.607558in}}%
\pgfpathlineto{\pgfqpoint{0.935603in}{1.619595in}}%
\pgfpathlineto{\pgfqpoint{0.785700in}{1.619595in}}%
\pgfpathlineto{\pgfqpoint{0.785700in}{1.607558in}}%
\pgfpathclose%
\pgfusepath{stroke,fill}%
\end{pgfscope}%
\begin{pgfscope}%
\pgfsetrectcap%
\pgfsetmiterjoin%
\pgfsetlinewidth{0.803000pt}%
\definecolor{currentstroke}{rgb}{0.000000,0.000000,0.000000}%
\pgfsetstrokecolor{currentstroke}%
\pgfsetdash{}{0pt}%
\pgfpathmoveto{\pgfqpoint{0.148611in}{1.607558in}}%
\pgfpathlineto{\pgfqpoint{0.148611in}{1.850716in}}%
\pgfusepath{stroke}%
\end{pgfscope}%
\begin{pgfscope}%
\pgfsetrectcap%
\pgfsetmiterjoin%
\pgfsetlinewidth{0.803000pt}%
\definecolor{currentstroke}{rgb}{0.000000,0.000000,0.000000}%
\pgfsetstrokecolor{currentstroke}%
\pgfsetdash{}{0pt}%
\pgfpathmoveto{\pgfqpoint{0.973079in}{1.607558in}}%
\pgfpathlineto{\pgfqpoint{0.973079in}{1.850716in}}%
\pgfusepath{stroke}%
\end{pgfscope}%
\begin{pgfscope}%
\pgfsetrectcap%
\pgfsetmiterjoin%
\pgfsetlinewidth{0.803000pt}%
\definecolor{currentstroke}{rgb}{0.000000,0.000000,0.000000}%
\pgfsetstrokecolor{currentstroke}%
\pgfsetdash{}{0pt}%
\pgfpathmoveto{\pgfqpoint{0.148611in}{1.607558in}}%
\pgfpathlineto{\pgfqpoint{0.973079in}{1.607558in}}%
\pgfusepath{stroke}%
\end{pgfscope}%
\begin{pgfscope}%
\pgfsetrectcap%
\pgfsetmiterjoin%
\pgfsetlinewidth{0.803000pt}%
\definecolor{currentstroke}{rgb}{0.000000,0.000000,0.000000}%
\pgfsetstrokecolor{currentstroke}%
\pgfsetdash{}{0pt}%
\pgfpathmoveto{\pgfqpoint{0.148611in}{1.850716in}}%
\pgfpathlineto{\pgfqpoint{0.973079in}{1.850716in}}%
\pgfusepath{stroke}%
\end{pgfscope}%
\begin{pgfscope}%
\definecolor{textcolor}{rgb}{0.000000,0.000000,0.000000}%
\pgfsetstrokecolor{textcolor}%
\pgfsetfillcolor{textcolor}%
\pgftext[x=0.560845in,y=1.934050in,,base]{\color{textcolor}\rmfamily\fontsize{11.000000}{13.200000}\selectfont MGP}%
\end{pgfscope}%
\begin{pgfscope}%
\pgfsetbuttcap%
\pgfsetmiterjoin%
\definecolor{currentfill}{rgb}{1.000000,1.000000,1.000000}%
\pgfsetfillcolor{currentfill}%
\pgfsetlinewidth{0.000000pt}%
\definecolor{currentstroke}{rgb}{0.000000,0.000000,0.000000}%
\pgfsetstrokecolor{currentstroke}%
\pgfsetstrokeopacity{0.000000}%
\pgfsetdash{}{0pt}%
\pgfpathmoveto{\pgfqpoint{1.137973in}{1.607558in}}%
\pgfpathlineto{\pgfqpoint{1.962441in}{1.607558in}}%
\pgfpathlineto{\pgfqpoint{1.962441in}{1.850716in}}%
\pgfpathlineto{\pgfqpoint{1.137973in}{1.850716in}}%
\pgfpathlineto{\pgfqpoint{1.137973in}{1.607558in}}%
\pgfpathclose%
\pgfusepath{fill}%
\end{pgfscope}%
\begin{pgfscope}%
\pgfpathrectangle{\pgfqpoint{1.137973in}{1.607558in}}{\pgfqpoint{0.824468in}{0.243158in}}%
\pgfusepath{clip}%
\pgfsetbuttcap%
\pgfsetmiterjoin%
\definecolor{currentfill}{rgb}{0.121569,0.466667,0.705882}%
\pgfsetfillcolor{currentfill}%
\pgfsetfillopacity{0.500000}%
\pgfsetlinewidth{1.003750pt}%
\definecolor{currentstroke}{rgb}{0.000000,0.000000,0.000000}%
\pgfsetstrokecolor{currentstroke}%
\pgfsetdash{}{0pt}%
\pgfpathmoveto{\pgfqpoint{1.175449in}{1.607558in}}%
\pgfpathlineto{\pgfqpoint{1.325352in}{1.607558in}}%
\pgfpathlineto{\pgfqpoint{1.325352in}{1.611935in}}%
\pgfpathlineto{\pgfqpoint{1.175449in}{1.611935in}}%
\pgfpathlineto{\pgfqpoint{1.175449in}{1.607558in}}%
\pgfpathclose%
\pgfusepath{stroke,fill}%
\end{pgfscope}%
\begin{pgfscope}%
\pgfpathrectangle{\pgfqpoint{1.137973in}{1.607558in}}{\pgfqpoint{0.824468in}{0.243158in}}%
\pgfusepath{clip}%
\pgfsetbuttcap%
\pgfsetmiterjoin%
\definecolor{currentfill}{rgb}{0.121569,0.466667,0.705882}%
\pgfsetfillcolor{currentfill}%
\pgfsetfillopacity{0.500000}%
\pgfsetlinewidth{1.003750pt}%
\definecolor{currentstroke}{rgb}{0.000000,0.000000,0.000000}%
\pgfsetstrokecolor{currentstroke}%
\pgfsetdash{}{0pt}%
\pgfpathmoveto{\pgfqpoint{1.325352in}{1.607558in}}%
\pgfpathlineto{\pgfqpoint{1.475255in}{1.607558in}}%
\pgfpathlineto{\pgfqpoint{1.475255in}{1.608896in}}%
\pgfpathlineto{\pgfqpoint{1.325352in}{1.608896in}}%
\pgfpathlineto{\pgfqpoint{1.325352in}{1.607558in}}%
\pgfpathclose%
\pgfusepath{stroke,fill}%
\end{pgfscope}%
\begin{pgfscope}%
\pgfpathrectangle{\pgfqpoint{1.137973in}{1.607558in}}{\pgfqpoint{0.824468in}{0.243158in}}%
\pgfusepath{clip}%
\pgfsetbuttcap%
\pgfsetmiterjoin%
\definecolor{currentfill}{rgb}{0.121569,0.466667,0.705882}%
\pgfsetfillcolor{currentfill}%
\pgfsetfillopacity{0.500000}%
\pgfsetlinewidth{1.003750pt}%
\definecolor{currentstroke}{rgb}{0.000000,0.000000,0.000000}%
\pgfsetstrokecolor{currentstroke}%
\pgfsetdash{}{0pt}%
\pgfpathmoveto{\pgfqpoint{1.475255in}{1.607558in}}%
\pgfpathlineto{\pgfqpoint{1.625158in}{1.607558in}}%
\pgfpathlineto{\pgfqpoint{1.625158in}{1.608774in}}%
\pgfpathlineto{\pgfqpoint{1.475255in}{1.608774in}}%
\pgfpathlineto{\pgfqpoint{1.475255in}{1.607558in}}%
\pgfpathclose%
\pgfusepath{stroke,fill}%
\end{pgfscope}%
\begin{pgfscope}%
\pgfpathrectangle{\pgfqpoint{1.137973in}{1.607558in}}{\pgfqpoint{0.824468in}{0.243158in}}%
\pgfusepath{clip}%
\pgfsetbuttcap%
\pgfsetmiterjoin%
\definecolor{currentfill}{rgb}{0.121569,0.466667,0.705882}%
\pgfsetfillcolor{currentfill}%
\pgfsetfillopacity{0.500000}%
\pgfsetlinewidth{1.003750pt}%
\definecolor{currentstroke}{rgb}{0.000000,0.000000,0.000000}%
\pgfsetstrokecolor{currentstroke}%
\pgfsetdash{}{0pt}%
\pgfpathmoveto{\pgfqpoint{1.625158in}{1.607558in}}%
\pgfpathlineto{\pgfqpoint{1.775062in}{1.607558in}}%
\pgfpathlineto{\pgfqpoint{1.775062in}{1.608166in}}%
\pgfpathlineto{\pgfqpoint{1.625158in}{1.608166in}}%
\pgfpathlineto{\pgfqpoint{1.625158in}{1.607558in}}%
\pgfpathclose%
\pgfusepath{stroke,fill}%
\end{pgfscope}%
\begin{pgfscope}%
\pgfpathrectangle{\pgfqpoint{1.137973in}{1.607558in}}{\pgfqpoint{0.824468in}{0.243158in}}%
\pgfusepath{clip}%
\pgfsetbuttcap%
\pgfsetmiterjoin%
\definecolor{currentfill}{rgb}{0.121569,0.466667,0.705882}%
\pgfsetfillcolor{currentfill}%
\pgfsetfillopacity{0.500000}%
\pgfsetlinewidth{1.003750pt}%
\definecolor{currentstroke}{rgb}{0.000000,0.000000,0.000000}%
\pgfsetstrokecolor{currentstroke}%
\pgfsetdash{}{0pt}%
\pgfpathmoveto{\pgfqpoint{1.775062in}{1.607558in}}%
\pgfpathlineto{\pgfqpoint{1.924965in}{1.607558in}}%
\pgfpathlineto{\pgfqpoint{1.924965in}{1.607680in}}%
\pgfpathlineto{\pgfqpoint{1.775062in}{1.607680in}}%
\pgfpathlineto{\pgfqpoint{1.775062in}{1.607558in}}%
\pgfpathclose%
\pgfusepath{stroke,fill}%
\end{pgfscope}%
\begin{pgfscope}%
\pgfsetrectcap%
\pgfsetmiterjoin%
\pgfsetlinewidth{0.803000pt}%
\definecolor{currentstroke}{rgb}{0.000000,0.000000,0.000000}%
\pgfsetstrokecolor{currentstroke}%
\pgfsetdash{}{0pt}%
\pgfpathmoveto{\pgfqpoint{1.137973in}{1.607558in}}%
\pgfpathlineto{\pgfqpoint{1.137973in}{1.850716in}}%
\pgfusepath{stroke}%
\end{pgfscope}%
\begin{pgfscope}%
\pgfsetrectcap%
\pgfsetmiterjoin%
\pgfsetlinewidth{0.803000pt}%
\definecolor{currentstroke}{rgb}{0.000000,0.000000,0.000000}%
\pgfsetstrokecolor{currentstroke}%
\pgfsetdash{}{0pt}%
\pgfpathmoveto{\pgfqpoint{1.962441in}{1.607558in}}%
\pgfpathlineto{\pgfqpoint{1.962441in}{1.850716in}}%
\pgfusepath{stroke}%
\end{pgfscope}%
\begin{pgfscope}%
\pgfsetrectcap%
\pgfsetmiterjoin%
\pgfsetlinewidth{0.803000pt}%
\definecolor{currentstroke}{rgb}{0.000000,0.000000,0.000000}%
\pgfsetstrokecolor{currentstroke}%
\pgfsetdash{}{0pt}%
\pgfpathmoveto{\pgfqpoint{1.137973in}{1.607558in}}%
\pgfpathlineto{\pgfqpoint{1.962441in}{1.607558in}}%
\pgfusepath{stroke}%
\end{pgfscope}%
\begin{pgfscope}%
\pgfsetrectcap%
\pgfsetmiterjoin%
\pgfsetlinewidth{0.803000pt}%
\definecolor{currentstroke}{rgb}{0.000000,0.000000,0.000000}%
\pgfsetstrokecolor{currentstroke}%
\pgfsetdash{}{0pt}%
\pgfpathmoveto{\pgfqpoint{1.137973in}{1.850716in}}%
\pgfpathlineto{\pgfqpoint{1.962441in}{1.850716in}}%
\pgfusepath{stroke}%
\end{pgfscope}%
\begin{pgfscope}%
\definecolor{textcolor}{rgb}{0.000000,0.000000,0.000000}%
\pgfsetstrokecolor{textcolor}%
\pgfsetfillcolor{textcolor}%
\pgftext[x=1.550207in,y=1.934050in,,base]{\color{textcolor}\rmfamily\fontsize{11.000000}{13.200000}\selectfont Intériale}%
\end{pgfscope}%
\begin{pgfscope}%
\pgfsetbuttcap%
\pgfsetmiterjoin%
\definecolor{currentfill}{rgb}{1.000000,1.000000,1.000000}%
\pgfsetfillcolor{currentfill}%
\pgfsetlinewidth{0.000000pt}%
\definecolor{currentstroke}{rgb}{0.000000,0.000000,0.000000}%
\pgfsetstrokecolor{currentstroke}%
\pgfsetstrokeopacity{0.000000}%
\pgfsetdash{}{0pt}%
\pgfpathmoveto{\pgfqpoint{2.127335in}{1.607558in}}%
\pgfpathlineto{\pgfqpoint{2.951803in}{1.607558in}}%
\pgfpathlineto{\pgfqpoint{2.951803in}{1.850716in}}%
\pgfpathlineto{\pgfqpoint{2.127335in}{1.850716in}}%
\pgfpathlineto{\pgfqpoint{2.127335in}{1.607558in}}%
\pgfpathclose%
\pgfusepath{fill}%
\end{pgfscope}%
\begin{pgfscope}%
\pgfpathrectangle{\pgfqpoint{2.127335in}{1.607558in}}{\pgfqpoint{0.824468in}{0.243158in}}%
\pgfusepath{clip}%
\pgfsetbuttcap%
\pgfsetmiterjoin%
\definecolor{currentfill}{rgb}{0.121569,0.466667,0.705882}%
\pgfsetfillcolor{currentfill}%
\pgfsetfillopacity{0.500000}%
\pgfsetlinewidth{1.003750pt}%
\definecolor{currentstroke}{rgb}{0.000000,0.000000,0.000000}%
\pgfsetstrokecolor{currentstroke}%
\pgfsetdash{}{0pt}%
\pgfpathmoveto{\pgfqpoint{2.164810in}{1.607558in}}%
\pgfpathlineto{\pgfqpoint{2.314714in}{1.607558in}}%
\pgfpathlineto{\pgfqpoint{2.314714in}{1.617406in}}%
\pgfpathlineto{\pgfqpoint{2.164810in}{1.617406in}}%
\pgfpathlineto{\pgfqpoint{2.164810in}{1.607558in}}%
\pgfpathclose%
\pgfusepath{stroke,fill}%
\end{pgfscope}%
\begin{pgfscope}%
\pgfpathrectangle{\pgfqpoint{2.127335in}{1.607558in}}{\pgfqpoint{0.824468in}{0.243158in}}%
\pgfusepath{clip}%
\pgfsetbuttcap%
\pgfsetmiterjoin%
\definecolor{currentfill}{rgb}{0.121569,0.466667,0.705882}%
\pgfsetfillcolor{currentfill}%
\pgfsetfillopacity{0.500000}%
\pgfsetlinewidth{1.003750pt}%
\definecolor{currentstroke}{rgb}{0.000000,0.000000,0.000000}%
\pgfsetstrokecolor{currentstroke}%
\pgfsetdash{}{0pt}%
\pgfpathmoveto{\pgfqpoint{2.314714in}{1.607558in}}%
\pgfpathlineto{\pgfqpoint{2.464617in}{1.607558in}}%
\pgfpathlineto{\pgfqpoint{2.464617in}{1.609868in}}%
\pgfpathlineto{\pgfqpoint{2.314714in}{1.609868in}}%
\pgfpathlineto{\pgfqpoint{2.314714in}{1.607558in}}%
\pgfpathclose%
\pgfusepath{stroke,fill}%
\end{pgfscope}%
\begin{pgfscope}%
\pgfpathrectangle{\pgfqpoint{2.127335in}{1.607558in}}{\pgfqpoint{0.824468in}{0.243158in}}%
\pgfusepath{clip}%
\pgfsetbuttcap%
\pgfsetmiterjoin%
\definecolor{currentfill}{rgb}{0.121569,0.466667,0.705882}%
\pgfsetfillcolor{currentfill}%
\pgfsetfillopacity{0.500000}%
\pgfsetlinewidth{1.003750pt}%
\definecolor{currentstroke}{rgb}{0.000000,0.000000,0.000000}%
\pgfsetstrokecolor{currentstroke}%
\pgfsetdash{}{0pt}%
\pgfpathmoveto{\pgfqpoint{2.464617in}{1.607558in}}%
\pgfpathlineto{\pgfqpoint{2.614520in}{1.607558in}}%
\pgfpathlineto{\pgfqpoint{2.614520in}{1.610355in}}%
\pgfpathlineto{\pgfqpoint{2.464617in}{1.610355in}}%
\pgfpathlineto{\pgfqpoint{2.464617in}{1.607558in}}%
\pgfpathclose%
\pgfusepath{stroke,fill}%
\end{pgfscope}%
\begin{pgfscope}%
\pgfpathrectangle{\pgfqpoint{2.127335in}{1.607558in}}{\pgfqpoint{0.824468in}{0.243158in}}%
\pgfusepath{clip}%
\pgfsetbuttcap%
\pgfsetmiterjoin%
\definecolor{currentfill}{rgb}{0.121569,0.466667,0.705882}%
\pgfsetfillcolor{currentfill}%
\pgfsetfillopacity{0.500000}%
\pgfsetlinewidth{1.003750pt}%
\definecolor{currentstroke}{rgb}{0.000000,0.000000,0.000000}%
\pgfsetstrokecolor{currentstroke}%
\pgfsetdash{}{0pt}%
\pgfpathmoveto{\pgfqpoint{2.614520in}{1.607558in}}%
\pgfpathlineto{\pgfqpoint{2.764423in}{1.607558in}}%
\pgfpathlineto{\pgfqpoint{2.764423in}{1.613637in}}%
\pgfpathlineto{\pgfqpoint{2.614520in}{1.613637in}}%
\pgfpathlineto{\pgfqpoint{2.614520in}{1.607558in}}%
\pgfpathclose%
\pgfusepath{stroke,fill}%
\end{pgfscope}%
\begin{pgfscope}%
\pgfpathrectangle{\pgfqpoint{2.127335in}{1.607558in}}{\pgfqpoint{0.824468in}{0.243158in}}%
\pgfusepath{clip}%
\pgfsetbuttcap%
\pgfsetmiterjoin%
\definecolor{currentfill}{rgb}{0.121569,0.466667,0.705882}%
\pgfsetfillcolor{currentfill}%
\pgfsetfillopacity{0.500000}%
\pgfsetlinewidth{1.003750pt}%
\definecolor{currentstroke}{rgb}{0.000000,0.000000,0.000000}%
\pgfsetstrokecolor{currentstroke}%
\pgfsetdash{}{0pt}%
\pgfpathmoveto{\pgfqpoint{2.764423in}{1.607558in}}%
\pgfpathlineto{\pgfqpoint{2.914327in}{1.607558in}}%
\pgfpathlineto{\pgfqpoint{2.914327in}{1.612300in}}%
\pgfpathlineto{\pgfqpoint{2.764423in}{1.612300in}}%
\pgfpathlineto{\pgfqpoint{2.764423in}{1.607558in}}%
\pgfpathclose%
\pgfusepath{stroke,fill}%
\end{pgfscope}%
\begin{pgfscope}%
\pgfsetrectcap%
\pgfsetmiterjoin%
\pgfsetlinewidth{0.803000pt}%
\definecolor{currentstroke}{rgb}{0.000000,0.000000,0.000000}%
\pgfsetstrokecolor{currentstroke}%
\pgfsetdash{}{0pt}%
\pgfpathmoveto{\pgfqpoint{2.127335in}{1.607558in}}%
\pgfpathlineto{\pgfqpoint{2.127335in}{1.850716in}}%
\pgfusepath{stroke}%
\end{pgfscope}%
\begin{pgfscope}%
\pgfsetrectcap%
\pgfsetmiterjoin%
\pgfsetlinewidth{0.803000pt}%
\definecolor{currentstroke}{rgb}{0.000000,0.000000,0.000000}%
\pgfsetstrokecolor{currentstroke}%
\pgfsetdash{}{0pt}%
\pgfpathmoveto{\pgfqpoint{2.951803in}{1.607558in}}%
\pgfpathlineto{\pgfqpoint{2.951803in}{1.850716in}}%
\pgfusepath{stroke}%
\end{pgfscope}%
\begin{pgfscope}%
\pgfsetrectcap%
\pgfsetmiterjoin%
\pgfsetlinewidth{0.803000pt}%
\definecolor{currentstroke}{rgb}{0.000000,0.000000,0.000000}%
\pgfsetstrokecolor{currentstroke}%
\pgfsetdash{}{0pt}%
\pgfpathmoveto{\pgfqpoint{2.127335in}{1.607558in}}%
\pgfpathlineto{\pgfqpoint{2.951803in}{1.607558in}}%
\pgfusepath{stroke}%
\end{pgfscope}%
\begin{pgfscope}%
\pgfsetrectcap%
\pgfsetmiterjoin%
\pgfsetlinewidth{0.803000pt}%
\definecolor{currentstroke}{rgb}{0.000000,0.000000,0.000000}%
\pgfsetstrokecolor{currentstroke}%
\pgfsetdash{}{0pt}%
\pgfpathmoveto{\pgfqpoint{2.127335in}{1.850716in}}%
\pgfpathlineto{\pgfqpoint{2.951803in}{1.850716in}}%
\pgfusepath{stroke}%
\end{pgfscope}%
\begin{pgfscope}%
\definecolor{textcolor}{rgb}{0.000000,0.000000,0.000000}%
\pgfsetstrokecolor{textcolor}%
\pgfsetfillcolor{textcolor}%
\pgftext[x=2.539569in,y=1.934050in,,base]{\color{textcolor}\rmfamily\fontsize{11.000000}{13.200000}\selectfont Généra...}%
\end{pgfscope}%
\begin{pgfscope}%
\pgfsetbuttcap%
\pgfsetmiterjoin%
\definecolor{currentfill}{rgb}{1.000000,1.000000,1.000000}%
\pgfsetfillcolor{currentfill}%
\pgfsetlinewidth{0.000000pt}%
\definecolor{currentstroke}{rgb}{0.000000,0.000000,0.000000}%
\pgfsetstrokecolor{currentstroke}%
\pgfsetstrokeopacity{0.000000}%
\pgfsetdash{}{0pt}%
\pgfpathmoveto{\pgfqpoint{3.116696in}{1.607558in}}%
\pgfpathlineto{\pgfqpoint{3.941164in}{1.607558in}}%
\pgfpathlineto{\pgfqpoint{3.941164in}{1.850716in}}%
\pgfpathlineto{\pgfqpoint{3.116696in}{1.850716in}}%
\pgfpathlineto{\pgfqpoint{3.116696in}{1.607558in}}%
\pgfpathclose%
\pgfusepath{fill}%
\end{pgfscope}%
\begin{pgfscope}%
\pgfpathrectangle{\pgfqpoint{3.116696in}{1.607558in}}{\pgfqpoint{0.824468in}{0.243158in}}%
\pgfusepath{clip}%
\pgfsetbuttcap%
\pgfsetmiterjoin%
\definecolor{currentfill}{rgb}{0.121569,0.466667,0.705882}%
\pgfsetfillcolor{currentfill}%
\pgfsetfillopacity{0.500000}%
\pgfsetlinewidth{1.003750pt}%
\definecolor{currentstroke}{rgb}{0.000000,0.000000,0.000000}%
\pgfsetstrokecolor{currentstroke}%
\pgfsetdash{}{0pt}%
\pgfpathmoveto{\pgfqpoint{3.154172in}{1.607558in}}%
\pgfpathlineto{\pgfqpoint{3.304075in}{1.607558in}}%
\pgfpathlineto{\pgfqpoint{3.304075in}{1.633090in}}%
\pgfpathlineto{\pgfqpoint{3.154172in}{1.633090in}}%
\pgfpathlineto{\pgfqpoint{3.154172in}{1.607558in}}%
\pgfpathclose%
\pgfusepath{stroke,fill}%
\end{pgfscope}%
\begin{pgfscope}%
\pgfpathrectangle{\pgfqpoint{3.116696in}{1.607558in}}{\pgfqpoint{0.824468in}{0.243158in}}%
\pgfusepath{clip}%
\pgfsetbuttcap%
\pgfsetmiterjoin%
\definecolor{currentfill}{rgb}{0.121569,0.466667,0.705882}%
\pgfsetfillcolor{currentfill}%
\pgfsetfillopacity{0.500000}%
\pgfsetlinewidth{1.003750pt}%
\definecolor{currentstroke}{rgb}{0.000000,0.000000,0.000000}%
\pgfsetstrokecolor{currentstroke}%
\pgfsetdash{}{0pt}%
\pgfpathmoveto{\pgfqpoint{3.304075in}{1.607558in}}%
\pgfpathlineto{\pgfqpoint{3.453979in}{1.607558in}}%
\pgfpathlineto{\pgfqpoint{3.453979in}{1.610963in}}%
\pgfpathlineto{\pgfqpoint{3.304075in}{1.610963in}}%
\pgfpathlineto{\pgfqpoint{3.304075in}{1.607558in}}%
\pgfpathclose%
\pgfusepath{stroke,fill}%
\end{pgfscope}%
\begin{pgfscope}%
\pgfpathrectangle{\pgfqpoint{3.116696in}{1.607558in}}{\pgfqpoint{0.824468in}{0.243158in}}%
\pgfusepath{clip}%
\pgfsetbuttcap%
\pgfsetmiterjoin%
\definecolor{currentfill}{rgb}{0.121569,0.466667,0.705882}%
\pgfsetfillcolor{currentfill}%
\pgfsetfillopacity{0.500000}%
\pgfsetlinewidth{1.003750pt}%
\definecolor{currentstroke}{rgb}{0.000000,0.000000,0.000000}%
\pgfsetstrokecolor{currentstroke}%
\pgfsetdash{}{0pt}%
\pgfpathmoveto{\pgfqpoint{3.453979in}{1.607558in}}%
\pgfpathlineto{\pgfqpoint{3.603882in}{1.607558in}}%
\pgfpathlineto{\pgfqpoint{3.603882in}{1.610476in}}%
\pgfpathlineto{\pgfqpoint{3.453979in}{1.610476in}}%
\pgfpathlineto{\pgfqpoint{3.453979in}{1.607558in}}%
\pgfpathclose%
\pgfusepath{stroke,fill}%
\end{pgfscope}%
\begin{pgfscope}%
\pgfpathrectangle{\pgfqpoint{3.116696in}{1.607558in}}{\pgfqpoint{0.824468in}{0.243158in}}%
\pgfusepath{clip}%
\pgfsetbuttcap%
\pgfsetmiterjoin%
\definecolor{currentfill}{rgb}{0.121569,0.466667,0.705882}%
\pgfsetfillcolor{currentfill}%
\pgfsetfillopacity{0.500000}%
\pgfsetlinewidth{1.003750pt}%
\definecolor{currentstroke}{rgb}{0.000000,0.000000,0.000000}%
\pgfsetstrokecolor{currentstroke}%
\pgfsetdash{}{0pt}%
\pgfpathmoveto{\pgfqpoint{3.603882in}{1.607558in}}%
\pgfpathlineto{\pgfqpoint{3.753785in}{1.607558in}}%
\pgfpathlineto{\pgfqpoint{3.753785in}{1.607923in}}%
\pgfpathlineto{\pgfqpoint{3.603882in}{1.607923in}}%
\pgfpathlineto{\pgfqpoint{3.603882in}{1.607558in}}%
\pgfpathclose%
\pgfusepath{stroke,fill}%
\end{pgfscope}%
\begin{pgfscope}%
\pgfpathrectangle{\pgfqpoint{3.116696in}{1.607558in}}{\pgfqpoint{0.824468in}{0.243158in}}%
\pgfusepath{clip}%
\pgfsetbuttcap%
\pgfsetmiterjoin%
\definecolor{currentfill}{rgb}{0.121569,0.466667,0.705882}%
\pgfsetfillcolor{currentfill}%
\pgfsetfillopacity{0.500000}%
\pgfsetlinewidth{1.003750pt}%
\definecolor{currentstroke}{rgb}{0.000000,0.000000,0.000000}%
\pgfsetstrokecolor{currentstroke}%
\pgfsetdash{}{0pt}%
\pgfpathmoveto{\pgfqpoint{3.753785in}{1.607558in}}%
\pgfpathlineto{\pgfqpoint{3.903688in}{1.607558in}}%
\pgfpathlineto{\pgfqpoint{3.903688in}{1.607558in}}%
\pgfpathlineto{\pgfqpoint{3.753785in}{1.607558in}}%
\pgfpathlineto{\pgfqpoint{3.753785in}{1.607558in}}%
\pgfpathclose%
\pgfusepath{stroke,fill}%
\end{pgfscope}%
\begin{pgfscope}%
\pgfsetrectcap%
\pgfsetmiterjoin%
\pgfsetlinewidth{0.803000pt}%
\definecolor{currentstroke}{rgb}{0.000000,0.000000,0.000000}%
\pgfsetstrokecolor{currentstroke}%
\pgfsetdash{}{0pt}%
\pgfpathmoveto{\pgfqpoint{3.116696in}{1.607558in}}%
\pgfpathlineto{\pgfqpoint{3.116696in}{1.850716in}}%
\pgfusepath{stroke}%
\end{pgfscope}%
\begin{pgfscope}%
\pgfsetrectcap%
\pgfsetmiterjoin%
\pgfsetlinewidth{0.803000pt}%
\definecolor{currentstroke}{rgb}{0.000000,0.000000,0.000000}%
\pgfsetstrokecolor{currentstroke}%
\pgfsetdash{}{0pt}%
\pgfpathmoveto{\pgfqpoint{3.941164in}{1.607558in}}%
\pgfpathlineto{\pgfqpoint{3.941164in}{1.850716in}}%
\pgfusepath{stroke}%
\end{pgfscope}%
\begin{pgfscope}%
\pgfsetrectcap%
\pgfsetmiterjoin%
\pgfsetlinewidth{0.803000pt}%
\definecolor{currentstroke}{rgb}{0.000000,0.000000,0.000000}%
\pgfsetstrokecolor{currentstroke}%
\pgfsetdash{}{0pt}%
\pgfpathmoveto{\pgfqpoint{3.116696in}{1.607558in}}%
\pgfpathlineto{\pgfqpoint{3.941164in}{1.607558in}}%
\pgfusepath{stroke}%
\end{pgfscope}%
\begin{pgfscope}%
\pgfsetrectcap%
\pgfsetmiterjoin%
\pgfsetlinewidth{0.803000pt}%
\definecolor{currentstroke}{rgb}{0.000000,0.000000,0.000000}%
\pgfsetstrokecolor{currentstroke}%
\pgfsetdash{}{0pt}%
\pgfpathmoveto{\pgfqpoint{3.116696in}{1.850716in}}%
\pgfpathlineto{\pgfqpoint{3.941164in}{1.850716in}}%
\pgfusepath{stroke}%
\end{pgfscope}%
\begin{pgfscope}%
\definecolor{textcolor}{rgb}{0.000000,0.000000,0.000000}%
\pgfsetstrokecolor{textcolor}%
\pgfsetfillcolor{textcolor}%
\pgftext[x=3.528930in,y=1.934050in,,base]{\color{textcolor}\rmfamily\fontsize{11.000000}{13.200000}\selectfont Cardif}%
\end{pgfscope}%
\begin{pgfscope}%
\pgfsetbuttcap%
\pgfsetmiterjoin%
\definecolor{currentfill}{rgb}{1.000000,1.000000,1.000000}%
\pgfsetfillcolor{currentfill}%
\pgfsetlinewidth{0.000000pt}%
\definecolor{currentstroke}{rgb}{0.000000,0.000000,0.000000}%
\pgfsetstrokecolor{currentstroke}%
\pgfsetstrokeopacity{0.000000}%
\pgfsetdash{}{0pt}%
\pgfpathmoveto{\pgfqpoint{4.106058in}{1.607558in}}%
\pgfpathlineto{\pgfqpoint{4.930526in}{1.607558in}}%
\pgfpathlineto{\pgfqpoint{4.930526in}{1.850716in}}%
\pgfpathlineto{\pgfqpoint{4.106058in}{1.850716in}}%
\pgfpathlineto{\pgfqpoint{4.106058in}{1.607558in}}%
\pgfpathclose%
\pgfusepath{fill}%
\end{pgfscope}%
\begin{pgfscope}%
\pgfpathrectangle{\pgfqpoint{4.106058in}{1.607558in}}{\pgfqpoint{0.824468in}{0.243158in}}%
\pgfusepath{clip}%
\pgfsetbuttcap%
\pgfsetmiterjoin%
\definecolor{currentfill}{rgb}{0.121569,0.466667,0.705882}%
\pgfsetfillcolor{currentfill}%
\pgfsetfillopacity{0.500000}%
\pgfsetlinewidth{1.003750pt}%
\definecolor{currentstroke}{rgb}{0.000000,0.000000,0.000000}%
\pgfsetstrokecolor{currentstroke}%
\pgfsetdash{}{0pt}%
\pgfpathmoveto{\pgfqpoint{4.143534in}{1.607558in}}%
\pgfpathlineto{\pgfqpoint{4.293437in}{1.607558in}}%
\pgfpathlineto{\pgfqpoint{4.293437in}{1.617893in}}%
\pgfpathlineto{\pgfqpoint{4.143534in}{1.617893in}}%
\pgfpathlineto{\pgfqpoint{4.143534in}{1.607558in}}%
\pgfpathclose%
\pgfusepath{stroke,fill}%
\end{pgfscope}%
\begin{pgfscope}%
\pgfpathrectangle{\pgfqpoint{4.106058in}{1.607558in}}{\pgfqpoint{0.824468in}{0.243158in}}%
\pgfusepath{clip}%
\pgfsetbuttcap%
\pgfsetmiterjoin%
\definecolor{currentfill}{rgb}{0.121569,0.466667,0.705882}%
\pgfsetfillcolor{currentfill}%
\pgfsetfillopacity{0.500000}%
\pgfsetlinewidth{1.003750pt}%
\definecolor{currentstroke}{rgb}{0.000000,0.000000,0.000000}%
\pgfsetstrokecolor{currentstroke}%
\pgfsetdash{}{0pt}%
\pgfpathmoveto{\pgfqpoint{4.293437in}{1.607558in}}%
\pgfpathlineto{\pgfqpoint{4.443340in}{1.607558in}}%
\pgfpathlineto{\pgfqpoint{4.443340in}{1.614488in}}%
\pgfpathlineto{\pgfqpoint{4.293437in}{1.614488in}}%
\pgfpathlineto{\pgfqpoint{4.293437in}{1.607558in}}%
\pgfpathclose%
\pgfusepath{stroke,fill}%
\end{pgfscope}%
\begin{pgfscope}%
\pgfpathrectangle{\pgfqpoint{4.106058in}{1.607558in}}{\pgfqpoint{0.824468in}{0.243158in}}%
\pgfusepath{clip}%
\pgfsetbuttcap%
\pgfsetmiterjoin%
\definecolor{currentfill}{rgb}{0.121569,0.466667,0.705882}%
\pgfsetfillcolor{currentfill}%
\pgfsetfillopacity{0.500000}%
\pgfsetlinewidth{1.003750pt}%
\definecolor{currentstroke}{rgb}{0.000000,0.000000,0.000000}%
\pgfsetstrokecolor{currentstroke}%
\pgfsetdash{}{0pt}%
\pgfpathmoveto{\pgfqpoint{4.443340in}{1.607558in}}%
\pgfpathlineto{\pgfqpoint{4.593244in}{1.607558in}}%
\pgfpathlineto{\pgfqpoint{4.593244in}{1.624580in}}%
\pgfpathlineto{\pgfqpoint{4.443340in}{1.624580in}}%
\pgfpathlineto{\pgfqpoint{4.443340in}{1.607558in}}%
\pgfpathclose%
\pgfusepath{stroke,fill}%
\end{pgfscope}%
\begin{pgfscope}%
\pgfpathrectangle{\pgfqpoint{4.106058in}{1.607558in}}{\pgfqpoint{0.824468in}{0.243158in}}%
\pgfusepath{clip}%
\pgfsetbuttcap%
\pgfsetmiterjoin%
\definecolor{currentfill}{rgb}{0.121569,0.466667,0.705882}%
\pgfsetfillcolor{currentfill}%
\pgfsetfillopacity{0.500000}%
\pgfsetlinewidth{1.003750pt}%
\definecolor{currentstroke}{rgb}{0.000000,0.000000,0.000000}%
\pgfsetstrokecolor{currentstroke}%
\pgfsetdash{}{0pt}%
\pgfpathmoveto{\pgfqpoint{4.593244in}{1.607558in}}%
\pgfpathlineto{\pgfqpoint{4.743147in}{1.607558in}}%
\pgfpathlineto{\pgfqpoint{4.743147in}{1.633941in}}%
\pgfpathlineto{\pgfqpoint{4.593244in}{1.633941in}}%
\pgfpathlineto{\pgfqpoint{4.593244in}{1.607558in}}%
\pgfpathclose%
\pgfusepath{stroke,fill}%
\end{pgfscope}%
\begin{pgfscope}%
\pgfpathrectangle{\pgfqpoint{4.106058in}{1.607558in}}{\pgfqpoint{0.824468in}{0.243158in}}%
\pgfusepath{clip}%
\pgfsetbuttcap%
\pgfsetmiterjoin%
\definecolor{currentfill}{rgb}{0.121569,0.466667,0.705882}%
\pgfsetfillcolor{currentfill}%
\pgfsetfillopacity{0.500000}%
\pgfsetlinewidth{1.003750pt}%
\definecolor{currentstroke}{rgb}{0.000000,0.000000,0.000000}%
\pgfsetstrokecolor{currentstroke}%
\pgfsetdash{}{0pt}%
\pgfpathmoveto{\pgfqpoint{4.743147in}{1.607558in}}%
\pgfpathlineto{\pgfqpoint{4.893050in}{1.607558in}}%
\pgfpathlineto{\pgfqpoint{4.893050in}{1.626890in}}%
\pgfpathlineto{\pgfqpoint{4.743147in}{1.626890in}}%
\pgfpathlineto{\pgfqpoint{4.743147in}{1.607558in}}%
\pgfpathclose%
\pgfusepath{stroke,fill}%
\end{pgfscope}%
\begin{pgfscope}%
\pgfsetrectcap%
\pgfsetmiterjoin%
\pgfsetlinewidth{0.803000pt}%
\definecolor{currentstroke}{rgb}{0.000000,0.000000,0.000000}%
\pgfsetstrokecolor{currentstroke}%
\pgfsetdash{}{0pt}%
\pgfpathmoveto{\pgfqpoint{4.106058in}{1.607558in}}%
\pgfpathlineto{\pgfqpoint{4.106058in}{1.850716in}}%
\pgfusepath{stroke}%
\end{pgfscope}%
\begin{pgfscope}%
\pgfsetrectcap%
\pgfsetmiterjoin%
\pgfsetlinewidth{0.803000pt}%
\definecolor{currentstroke}{rgb}{0.000000,0.000000,0.000000}%
\pgfsetstrokecolor{currentstroke}%
\pgfsetdash{}{0pt}%
\pgfpathmoveto{\pgfqpoint{4.930526in}{1.607558in}}%
\pgfpathlineto{\pgfqpoint{4.930526in}{1.850716in}}%
\pgfusepath{stroke}%
\end{pgfscope}%
\begin{pgfscope}%
\pgfsetrectcap%
\pgfsetmiterjoin%
\pgfsetlinewidth{0.803000pt}%
\definecolor{currentstroke}{rgb}{0.000000,0.000000,0.000000}%
\pgfsetstrokecolor{currentstroke}%
\pgfsetdash{}{0pt}%
\pgfpathmoveto{\pgfqpoint{4.106058in}{1.607558in}}%
\pgfpathlineto{\pgfqpoint{4.930526in}{1.607558in}}%
\pgfusepath{stroke}%
\end{pgfscope}%
\begin{pgfscope}%
\pgfsetrectcap%
\pgfsetmiterjoin%
\pgfsetlinewidth{0.803000pt}%
\definecolor{currentstroke}{rgb}{0.000000,0.000000,0.000000}%
\pgfsetstrokecolor{currentstroke}%
\pgfsetdash{}{0pt}%
\pgfpathmoveto{\pgfqpoint{4.106058in}{1.850716in}}%
\pgfpathlineto{\pgfqpoint{4.930526in}{1.850716in}}%
\pgfusepath{stroke}%
\end{pgfscope}%
\begin{pgfscope}%
\definecolor{textcolor}{rgb}{0.000000,0.000000,0.000000}%
\pgfsetstrokecolor{textcolor}%
\pgfsetfillcolor{textcolor}%
\pgftext[x=4.518292in,y=1.934050in,,base]{\color{textcolor}\rmfamily\fontsize{11.000000}{13.200000}\selectfont Santiane}%
\end{pgfscope}%
\begin{pgfscope}%
\pgfsetbuttcap%
\pgfsetmiterjoin%
\definecolor{currentfill}{rgb}{1.000000,1.000000,1.000000}%
\pgfsetfillcolor{currentfill}%
\pgfsetlinewidth{0.000000pt}%
\definecolor{currentstroke}{rgb}{0.000000,0.000000,0.000000}%
\pgfsetstrokecolor{currentstroke}%
\pgfsetstrokeopacity{0.000000}%
\pgfsetdash{}{0pt}%
\pgfpathmoveto{\pgfqpoint{5.095420in}{1.607558in}}%
\pgfpathlineto{\pgfqpoint{5.919888in}{1.607558in}}%
\pgfpathlineto{\pgfqpoint{5.919888in}{1.850716in}}%
\pgfpathlineto{\pgfqpoint{5.095420in}{1.850716in}}%
\pgfpathlineto{\pgfqpoint{5.095420in}{1.607558in}}%
\pgfpathclose%
\pgfusepath{fill}%
\end{pgfscope}%
\begin{pgfscope}%
\pgfpathrectangle{\pgfqpoint{5.095420in}{1.607558in}}{\pgfqpoint{0.824468in}{0.243158in}}%
\pgfusepath{clip}%
\pgfsetbuttcap%
\pgfsetmiterjoin%
\definecolor{currentfill}{rgb}{0.121569,0.466667,0.705882}%
\pgfsetfillcolor{currentfill}%
\pgfsetfillopacity{0.500000}%
\pgfsetlinewidth{1.003750pt}%
\definecolor{currentstroke}{rgb}{0.000000,0.000000,0.000000}%
\pgfsetstrokecolor{currentstroke}%
\pgfsetdash{}{0pt}%
\pgfpathmoveto{\pgfqpoint{5.132895in}{1.607558in}}%
\pgfpathlineto{\pgfqpoint{5.282799in}{1.607558in}}%
\pgfpathlineto{\pgfqpoint{5.282799in}{1.618136in}}%
\pgfpathlineto{\pgfqpoint{5.132895in}{1.618136in}}%
\pgfpathlineto{\pgfqpoint{5.132895in}{1.607558in}}%
\pgfpathclose%
\pgfusepath{stroke,fill}%
\end{pgfscope}%
\begin{pgfscope}%
\pgfpathrectangle{\pgfqpoint{5.095420in}{1.607558in}}{\pgfqpoint{0.824468in}{0.243158in}}%
\pgfusepath{clip}%
\pgfsetbuttcap%
\pgfsetmiterjoin%
\definecolor{currentfill}{rgb}{0.121569,0.466667,0.705882}%
\pgfsetfillcolor{currentfill}%
\pgfsetfillopacity{0.500000}%
\pgfsetlinewidth{1.003750pt}%
\definecolor{currentstroke}{rgb}{0.000000,0.000000,0.000000}%
\pgfsetstrokecolor{currentstroke}%
\pgfsetdash{}{0pt}%
\pgfpathmoveto{\pgfqpoint{5.282799in}{1.607558in}}%
\pgfpathlineto{\pgfqpoint{5.432702in}{1.607558in}}%
\pgfpathlineto{\pgfqpoint{5.432702in}{1.610598in}}%
\pgfpathlineto{\pgfqpoint{5.282799in}{1.610598in}}%
\pgfpathlineto{\pgfqpoint{5.282799in}{1.607558in}}%
\pgfpathclose%
\pgfusepath{stroke,fill}%
\end{pgfscope}%
\begin{pgfscope}%
\pgfpathrectangle{\pgfqpoint{5.095420in}{1.607558in}}{\pgfqpoint{0.824468in}{0.243158in}}%
\pgfusepath{clip}%
\pgfsetbuttcap%
\pgfsetmiterjoin%
\definecolor{currentfill}{rgb}{0.121569,0.466667,0.705882}%
\pgfsetfillcolor{currentfill}%
\pgfsetfillopacity{0.500000}%
\pgfsetlinewidth{1.003750pt}%
\definecolor{currentstroke}{rgb}{0.000000,0.000000,0.000000}%
\pgfsetstrokecolor{currentstroke}%
\pgfsetdash{}{0pt}%
\pgfpathmoveto{\pgfqpoint{5.432702in}{1.607558in}}%
\pgfpathlineto{\pgfqpoint{5.582605in}{1.607558in}}%
\pgfpathlineto{\pgfqpoint{5.582605in}{1.608774in}}%
\pgfpathlineto{\pgfqpoint{5.432702in}{1.608774in}}%
\pgfpathlineto{\pgfqpoint{5.432702in}{1.607558in}}%
\pgfpathclose%
\pgfusepath{stroke,fill}%
\end{pgfscope}%
\begin{pgfscope}%
\pgfpathrectangle{\pgfqpoint{5.095420in}{1.607558in}}{\pgfqpoint{0.824468in}{0.243158in}}%
\pgfusepath{clip}%
\pgfsetbuttcap%
\pgfsetmiterjoin%
\definecolor{currentfill}{rgb}{0.121569,0.466667,0.705882}%
\pgfsetfillcolor{currentfill}%
\pgfsetfillopacity{0.500000}%
\pgfsetlinewidth{1.003750pt}%
\definecolor{currentstroke}{rgb}{0.000000,0.000000,0.000000}%
\pgfsetstrokecolor{currentstroke}%
\pgfsetdash{}{0pt}%
\pgfpathmoveto{\pgfqpoint{5.582605in}{1.607558in}}%
\pgfpathlineto{\pgfqpoint{5.732509in}{1.607558in}}%
\pgfpathlineto{\pgfqpoint{5.732509in}{1.607802in}}%
\pgfpathlineto{\pgfqpoint{5.582605in}{1.607802in}}%
\pgfpathlineto{\pgfqpoint{5.582605in}{1.607558in}}%
\pgfpathclose%
\pgfusepath{stroke,fill}%
\end{pgfscope}%
\begin{pgfscope}%
\pgfpathrectangle{\pgfqpoint{5.095420in}{1.607558in}}{\pgfqpoint{0.824468in}{0.243158in}}%
\pgfusepath{clip}%
\pgfsetbuttcap%
\pgfsetmiterjoin%
\definecolor{currentfill}{rgb}{0.121569,0.466667,0.705882}%
\pgfsetfillcolor{currentfill}%
\pgfsetfillopacity{0.500000}%
\pgfsetlinewidth{1.003750pt}%
\definecolor{currentstroke}{rgb}{0.000000,0.000000,0.000000}%
\pgfsetstrokecolor{currentstroke}%
\pgfsetdash{}{0pt}%
\pgfpathmoveto{\pgfqpoint{5.732509in}{1.607558in}}%
\pgfpathlineto{\pgfqpoint{5.882412in}{1.607558in}}%
\pgfpathlineto{\pgfqpoint{5.882412in}{1.608531in}}%
\pgfpathlineto{\pgfqpoint{5.732509in}{1.608531in}}%
\pgfpathlineto{\pgfqpoint{5.732509in}{1.607558in}}%
\pgfpathclose%
\pgfusepath{stroke,fill}%
\end{pgfscope}%
\begin{pgfscope}%
\pgfsetrectcap%
\pgfsetmiterjoin%
\pgfsetlinewidth{0.803000pt}%
\definecolor{currentstroke}{rgb}{0.000000,0.000000,0.000000}%
\pgfsetstrokecolor{currentstroke}%
\pgfsetdash{}{0pt}%
\pgfpathmoveto{\pgfqpoint{5.095420in}{1.607558in}}%
\pgfpathlineto{\pgfqpoint{5.095420in}{1.850716in}}%
\pgfusepath{stroke}%
\end{pgfscope}%
\begin{pgfscope}%
\pgfsetrectcap%
\pgfsetmiterjoin%
\pgfsetlinewidth{0.803000pt}%
\definecolor{currentstroke}{rgb}{0.000000,0.000000,0.000000}%
\pgfsetstrokecolor{currentstroke}%
\pgfsetdash{}{0pt}%
\pgfpathmoveto{\pgfqpoint{5.919888in}{1.607558in}}%
\pgfpathlineto{\pgfqpoint{5.919888in}{1.850716in}}%
\pgfusepath{stroke}%
\end{pgfscope}%
\begin{pgfscope}%
\pgfsetrectcap%
\pgfsetmiterjoin%
\pgfsetlinewidth{0.803000pt}%
\definecolor{currentstroke}{rgb}{0.000000,0.000000,0.000000}%
\pgfsetstrokecolor{currentstroke}%
\pgfsetdash{}{0pt}%
\pgfpathmoveto{\pgfqpoint{5.095420in}{1.607558in}}%
\pgfpathlineto{\pgfqpoint{5.919888in}{1.607558in}}%
\pgfusepath{stroke}%
\end{pgfscope}%
\begin{pgfscope}%
\pgfsetrectcap%
\pgfsetmiterjoin%
\pgfsetlinewidth{0.803000pt}%
\definecolor{currentstroke}{rgb}{0.000000,0.000000,0.000000}%
\pgfsetstrokecolor{currentstroke}%
\pgfsetdash{}{0pt}%
\pgfpathmoveto{\pgfqpoint{5.095420in}{1.850716in}}%
\pgfpathlineto{\pgfqpoint{5.919888in}{1.850716in}}%
\pgfusepath{stroke}%
\end{pgfscope}%
\begin{pgfscope}%
\definecolor{textcolor}{rgb}{0.000000,0.000000,0.000000}%
\pgfsetstrokecolor{textcolor}%
\pgfsetfillcolor{textcolor}%
\pgftext[x=5.507654in,y=1.934050in,,base]{\color{textcolor}\rmfamily\fontsize{11.000000}{13.200000}\selectfont Eca As...}%
\end{pgfscope}%
\begin{pgfscope}%
\pgfsetbuttcap%
\pgfsetmiterjoin%
\definecolor{currentfill}{rgb}{1.000000,1.000000,1.000000}%
\pgfsetfillcolor{currentfill}%
\pgfsetlinewidth{0.000000pt}%
\definecolor{currentstroke}{rgb}{0.000000,0.000000,0.000000}%
\pgfsetstrokecolor{currentstroke}%
\pgfsetstrokeopacity{0.000000}%
\pgfsetdash{}{0pt}%
\pgfpathmoveto{\pgfqpoint{6.084781in}{1.607558in}}%
\pgfpathlineto{\pgfqpoint{6.909249in}{1.607558in}}%
\pgfpathlineto{\pgfqpoint{6.909249in}{1.850716in}}%
\pgfpathlineto{\pgfqpoint{6.084781in}{1.850716in}}%
\pgfpathlineto{\pgfqpoint{6.084781in}{1.607558in}}%
\pgfpathclose%
\pgfusepath{fill}%
\end{pgfscope}%
\begin{pgfscope}%
\pgfpathrectangle{\pgfqpoint{6.084781in}{1.607558in}}{\pgfqpoint{0.824468in}{0.243158in}}%
\pgfusepath{clip}%
\pgfsetbuttcap%
\pgfsetmiterjoin%
\definecolor{currentfill}{rgb}{0.121569,0.466667,0.705882}%
\pgfsetfillcolor{currentfill}%
\pgfsetfillopacity{0.500000}%
\pgfsetlinewidth{1.003750pt}%
\definecolor{currentstroke}{rgb}{0.000000,0.000000,0.000000}%
\pgfsetstrokecolor{currentstroke}%
\pgfsetdash{}{0pt}%
\pgfpathmoveto{\pgfqpoint{6.122257in}{1.607558in}}%
\pgfpathlineto{\pgfqpoint{6.272160in}{1.607558in}}%
\pgfpathlineto{\pgfqpoint{6.272160in}{1.614245in}}%
\pgfpathlineto{\pgfqpoint{6.122257in}{1.614245in}}%
\pgfpathlineto{\pgfqpoint{6.122257in}{1.607558in}}%
\pgfpathclose%
\pgfusepath{stroke,fill}%
\end{pgfscope}%
\begin{pgfscope}%
\pgfpathrectangle{\pgfqpoint{6.084781in}{1.607558in}}{\pgfqpoint{0.824468in}{0.243158in}}%
\pgfusepath{clip}%
\pgfsetbuttcap%
\pgfsetmiterjoin%
\definecolor{currentfill}{rgb}{0.121569,0.466667,0.705882}%
\pgfsetfillcolor{currentfill}%
\pgfsetfillopacity{0.500000}%
\pgfsetlinewidth{1.003750pt}%
\definecolor{currentstroke}{rgb}{0.000000,0.000000,0.000000}%
\pgfsetstrokecolor{currentstroke}%
\pgfsetdash{}{0pt}%
\pgfpathmoveto{\pgfqpoint{6.272160in}{1.607558in}}%
\pgfpathlineto{\pgfqpoint{6.422064in}{1.607558in}}%
\pgfpathlineto{\pgfqpoint{6.422064in}{1.610720in}}%
\pgfpathlineto{\pgfqpoint{6.272160in}{1.610720in}}%
\pgfpathlineto{\pgfqpoint{6.272160in}{1.607558in}}%
\pgfpathclose%
\pgfusepath{stroke,fill}%
\end{pgfscope}%
\begin{pgfscope}%
\pgfpathrectangle{\pgfqpoint{6.084781in}{1.607558in}}{\pgfqpoint{0.824468in}{0.243158in}}%
\pgfusepath{clip}%
\pgfsetbuttcap%
\pgfsetmiterjoin%
\definecolor{currentfill}{rgb}{0.121569,0.466667,0.705882}%
\pgfsetfillcolor{currentfill}%
\pgfsetfillopacity{0.500000}%
\pgfsetlinewidth{1.003750pt}%
\definecolor{currentstroke}{rgb}{0.000000,0.000000,0.000000}%
\pgfsetstrokecolor{currentstroke}%
\pgfsetdash{}{0pt}%
\pgfpathmoveto{\pgfqpoint{6.422064in}{1.607558in}}%
\pgfpathlineto{\pgfqpoint{6.571967in}{1.607558in}}%
\pgfpathlineto{\pgfqpoint{6.571967in}{1.608166in}}%
\pgfpathlineto{\pgfqpoint{6.422064in}{1.608166in}}%
\pgfpathlineto{\pgfqpoint{6.422064in}{1.607558in}}%
\pgfpathclose%
\pgfusepath{stroke,fill}%
\end{pgfscope}%
\begin{pgfscope}%
\pgfpathrectangle{\pgfqpoint{6.084781in}{1.607558in}}{\pgfqpoint{0.824468in}{0.243158in}}%
\pgfusepath{clip}%
\pgfsetbuttcap%
\pgfsetmiterjoin%
\definecolor{currentfill}{rgb}{0.121569,0.466667,0.705882}%
\pgfsetfillcolor{currentfill}%
\pgfsetfillopacity{0.500000}%
\pgfsetlinewidth{1.003750pt}%
\definecolor{currentstroke}{rgb}{0.000000,0.000000,0.000000}%
\pgfsetstrokecolor{currentstroke}%
\pgfsetdash{}{0pt}%
\pgfpathmoveto{\pgfqpoint{6.571967in}{1.607558in}}%
\pgfpathlineto{\pgfqpoint{6.721870in}{1.607558in}}%
\pgfpathlineto{\pgfqpoint{6.721870in}{1.608288in}}%
\pgfpathlineto{\pgfqpoint{6.571967in}{1.608288in}}%
\pgfpathlineto{\pgfqpoint{6.571967in}{1.607558in}}%
\pgfpathclose%
\pgfusepath{stroke,fill}%
\end{pgfscope}%
\begin{pgfscope}%
\pgfpathrectangle{\pgfqpoint{6.084781in}{1.607558in}}{\pgfqpoint{0.824468in}{0.243158in}}%
\pgfusepath{clip}%
\pgfsetbuttcap%
\pgfsetmiterjoin%
\definecolor{currentfill}{rgb}{0.121569,0.466667,0.705882}%
\pgfsetfillcolor{currentfill}%
\pgfsetfillopacity{0.500000}%
\pgfsetlinewidth{1.003750pt}%
\definecolor{currentstroke}{rgb}{0.000000,0.000000,0.000000}%
\pgfsetstrokecolor{currentstroke}%
\pgfsetdash{}{0pt}%
\pgfpathmoveto{\pgfqpoint{6.721870in}{1.607558in}}%
\pgfpathlineto{\pgfqpoint{6.871774in}{1.607558in}}%
\pgfpathlineto{\pgfqpoint{6.871774in}{1.608166in}}%
\pgfpathlineto{\pgfqpoint{6.721870in}{1.608166in}}%
\pgfpathlineto{\pgfqpoint{6.721870in}{1.607558in}}%
\pgfpathclose%
\pgfusepath{stroke,fill}%
\end{pgfscope}%
\begin{pgfscope}%
\pgfsetrectcap%
\pgfsetmiterjoin%
\pgfsetlinewidth{0.803000pt}%
\definecolor{currentstroke}{rgb}{0.000000,0.000000,0.000000}%
\pgfsetstrokecolor{currentstroke}%
\pgfsetdash{}{0pt}%
\pgfpathmoveto{\pgfqpoint{6.084781in}{1.607558in}}%
\pgfpathlineto{\pgfqpoint{6.084781in}{1.850716in}}%
\pgfusepath{stroke}%
\end{pgfscope}%
\begin{pgfscope}%
\pgfsetrectcap%
\pgfsetmiterjoin%
\pgfsetlinewidth{0.803000pt}%
\definecolor{currentstroke}{rgb}{0.000000,0.000000,0.000000}%
\pgfsetstrokecolor{currentstroke}%
\pgfsetdash{}{0pt}%
\pgfpathmoveto{\pgfqpoint{6.909249in}{1.607558in}}%
\pgfpathlineto{\pgfqpoint{6.909249in}{1.850716in}}%
\pgfusepath{stroke}%
\end{pgfscope}%
\begin{pgfscope}%
\pgfsetrectcap%
\pgfsetmiterjoin%
\pgfsetlinewidth{0.803000pt}%
\definecolor{currentstroke}{rgb}{0.000000,0.000000,0.000000}%
\pgfsetstrokecolor{currentstroke}%
\pgfsetdash{}{0pt}%
\pgfpathmoveto{\pgfqpoint{6.084781in}{1.607558in}}%
\pgfpathlineto{\pgfqpoint{6.909249in}{1.607558in}}%
\pgfusepath{stroke}%
\end{pgfscope}%
\begin{pgfscope}%
\pgfsetrectcap%
\pgfsetmiterjoin%
\pgfsetlinewidth{0.803000pt}%
\definecolor{currentstroke}{rgb}{0.000000,0.000000,0.000000}%
\pgfsetstrokecolor{currentstroke}%
\pgfsetdash{}{0pt}%
\pgfpathmoveto{\pgfqpoint{6.084781in}{1.850716in}}%
\pgfpathlineto{\pgfqpoint{6.909249in}{1.850716in}}%
\pgfusepath{stroke}%
\end{pgfscope}%
\begin{pgfscope}%
\definecolor{textcolor}{rgb}{0.000000,0.000000,0.000000}%
\pgfsetstrokecolor{textcolor}%
\pgfsetfillcolor{textcolor}%
\pgftext[x=6.497015in,y=1.934050in,,base]{\color{textcolor}\rmfamily\fontsize{11.000000}{13.200000}\selectfont Groupama}%
\end{pgfscope}%
\begin{pgfscope}%
\pgfsetbuttcap%
\pgfsetmiterjoin%
\definecolor{currentfill}{rgb}{1.000000,1.000000,1.000000}%
\pgfsetfillcolor{currentfill}%
\pgfsetlinewidth{0.000000pt}%
\definecolor{currentstroke}{rgb}{0.000000,0.000000,0.000000}%
\pgfsetstrokecolor{currentstroke}%
\pgfsetstrokeopacity{0.000000}%
\pgfsetdash{}{0pt}%
\pgfpathmoveto{\pgfqpoint{7.074143in}{1.607558in}}%
\pgfpathlineto{\pgfqpoint{7.898611in}{1.607558in}}%
\pgfpathlineto{\pgfqpoint{7.898611in}{1.850716in}}%
\pgfpathlineto{\pgfqpoint{7.074143in}{1.850716in}}%
\pgfpathlineto{\pgfqpoint{7.074143in}{1.607558in}}%
\pgfpathclose%
\pgfusepath{fill}%
\end{pgfscope}%
\begin{pgfscope}%
\pgfpathrectangle{\pgfqpoint{7.074143in}{1.607558in}}{\pgfqpoint{0.824468in}{0.243158in}}%
\pgfusepath{clip}%
\pgfsetbuttcap%
\pgfsetmiterjoin%
\definecolor{currentfill}{rgb}{0.121569,0.466667,0.705882}%
\pgfsetfillcolor{currentfill}%
\pgfsetfillopacity{0.500000}%
\pgfsetlinewidth{1.003750pt}%
\definecolor{currentstroke}{rgb}{0.000000,0.000000,0.000000}%
\pgfsetstrokecolor{currentstroke}%
\pgfsetdash{}{0pt}%
\pgfpathmoveto{\pgfqpoint{7.111619in}{1.607558in}}%
\pgfpathlineto{\pgfqpoint{7.261522in}{1.607558in}}%
\pgfpathlineto{\pgfqpoint{7.261522in}{1.614488in}}%
\pgfpathlineto{\pgfqpoint{7.111619in}{1.614488in}}%
\pgfpathlineto{\pgfqpoint{7.111619in}{1.607558in}}%
\pgfpathclose%
\pgfusepath{stroke,fill}%
\end{pgfscope}%
\begin{pgfscope}%
\pgfpathrectangle{\pgfqpoint{7.074143in}{1.607558in}}{\pgfqpoint{0.824468in}{0.243158in}}%
\pgfusepath{clip}%
\pgfsetbuttcap%
\pgfsetmiterjoin%
\definecolor{currentfill}{rgb}{0.121569,0.466667,0.705882}%
\pgfsetfillcolor{currentfill}%
\pgfsetfillopacity{0.500000}%
\pgfsetlinewidth{1.003750pt}%
\definecolor{currentstroke}{rgb}{0.000000,0.000000,0.000000}%
\pgfsetstrokecolor{currentstroke}%
\pgfsetdash{}{0pt}%
\pgfpathmoveto{\pgfqpoint{7.261522in}{1.607558in}}%
\pgfpathlineto{\pgfqpoint{7.411425in}{1.607558in}}%
\pgfpathlineto{\pgfqpoint{7.411425in}{1.610720in}}%
\pgfpathlineto{\pgfqpoint{7.261522in}{1.610720in}}%
\pgfpathlineto{\pgfqpoint{7.261522in}{1.607558in}}%
\pgfpathclose%
\pgfusepath{stroke,fill}%
\end{pgfscope}%
\begin{pgfscope}%
\pgfpathrectangle{\pgfqpoint{7.074143in}{1.607558in}}{\pgfqpoint{0.824468in}{0.243158in}}%
\pgfusepath{clip}%
\pgfsetbuttcap%
\pgfsetmiterjoin%
\definecolor{currentfill}{rgb}{0.121569,0.466667,0.705882}%
\pgfsetfillcolor{currentfill}%
\pgfsetfillopacity{0.500000}%
\pgfsetlinewidth{1.003750pt}%
\definecolor{currentstroke}{rgb}{0.000000,0.000000,0.000000}%
\pgfsetstrokecolor{currentstroke}%
\pgfsetdash{}{0pt}%
\pgfpathmoveto{\pgfqpoint{7.411425in}{1.607558in}}%
\pgfpathlineto{\pgfqpoint{7.561329in}{1.607558in}}%
\pgfpathlineto{\pgfqpoint{7.561329in}{1.609382in}}%
\pgfpathlineto{\pgfqpoint{7.411425in}{1.609382in}}%
\pgfpathlineto{\pgfqpoint{7.411425in}{1.607558in}}%
\pgfpathclose%
\pgfusepath{stroke,fill}%
\end{pgfscope}%
\begin{pgfscope}%
\pgfpathrectangle{\pgfqpoint{7.074143in}{1.607558in}}{\pgfqpoint{0.824468in}{0.243158in}}%
\pgfusepath{clip}%
\pgfsetbuttcap%
\pgfsetmiterjoin%
\definecolor{currentfill}{rgb}{0.121569,0.466667,0.705882}%
\pgfsetfillcolor{currentfill}%
\pgfsetfillopacity{0.500000}%
\pgfsetlinewidth{1.003750pt}%
\definecolor{currentstroke}{rgb}{0.000000,0.000000,0.000000}%
\pgfsetstrokecolor{currentstroke}%
\pgfsetdash{}{0pt}%
\pgfpathmoveto{\pgfqpoint{7.561329in}{1.607558in}}%
\pgfpathlineto{\pgfqpoint{7.711232in}{1.607558in}}%
\pgfpathlineto{\pgfqpoint{7.711232in}{1.608166in}}%
\pgfpathlineto{\pgfqpoint{7.561329in}{1.608166in}}%
\pgfpathlineto{\pgfqpoint{7.561329in}{1.607558in}}%
\pgfpathclose%
\pgfusepath{stroke,fill}%
\end{pgfscope}%
\begin{pgfscope}%
\pgfpathrectangle{\pgfqpoint{7.074143in}{1.607558in}}{\pgfqpoint{0.824468in}{0.243158in}}%
\pgfusepath{clip}%
\pgfsetbuttcap%
\pgfsetmiterjoin%
\definecolor{currentfill}{rgb}{0.121569,0.466667,0.705882}%
\pgfsetfillcolor{currentfill}%
\pgfsetfillopacity{0.500000}%
\pgfsetlinewidth{1.003750pt}%
\definecolor{currentstroke}{rgb}{0.000000,0.000000,0.000000}%
\pgfsetstrokecolor{currentstroke}%
\pgfsetdash{}{0pt}%
\pgfpathmoveto{\pgfqpoint{7.711232in}{1.607558in}}%
\pgfpathlineto{\pgfqpoint{7.861135in}{1.607558in}}%
\pgfpathlineto{\pgfqpoint{7.861135in}{1.608410in}}%
\pgfpathlineto{\pgfqpoint{7.711232in}{1.608410in}}%
\pgfpathlineto{\pgfqpoint{7.711232in}{1.607558in}}%
\pgfpathclose%
\pgfusepath{stroke,fill}%
\end{pgfscope}%
\begin{pgfscope}%
\pgfsetrectcap%
\pgfsetmiterjoin%
\pgfsetlinewidth{0.803000pt}%
\definecolor{currentstroke}{rgb}{0.000000,0.000000,0.000000}%
\pgfsetstrokecolor{currentstroke}%
\pgfsetdash{}{0pt}%
\pgfpathmoveto{\pgfqpoint{7.074143in}{1.607558in}}%
\pgfpathlineto{\pgfqpoint{7.074143in}{1.850716in}}%
\pgfusepath{stroke}%
\end{pgfscope}%
\begin{pgfscope}%
\pgfsetrectcap%
\pgfsetmiterjoin%
\pgfsetlinewidth{0.803000pt}%
\definecolor{currentstroke}{rgb}{0.000000,0.000000,0.000000}%
\pgfsetstrokecolor{currentstroke}%
\pgfsetdash{}{0pt}%
\pgfpathmoveto{\pgfqpoint{7.898611in}{1.607558in}}%
\pgfpathlineto{\pgfqpoint{7.898611in}{1.850716in}}%
\pgfusepath{stroke}%
\end{pgfscope}%
\begin{pgfscope}%
\pgfsetrectcap%
\pgfsetmiterjoin%
\pgfsetlinewidth{0.803000pt}%
\definecolor{currentstroke}{rgb}{0.000000,0.000000,0.000000}%
\pgfsetstrokecolor{currentstroke}%
\pgfsetdash{}{0pt}%
\pgfpathmoveto{\pgfqpoint{7.074143in}{1.607558in}}%
\pgfpathlineto{\pgfqpoint{7.898611in}{1.607558in}}%
\pgfusepath{stroke}%
\end{pgfscope}%
\begin{pgfscope}%
\pgfsetrectcap%
\pgfsetmiterjoin%
\pgfsetlinewidth{0.803000pt}%
\definecolor{currentstroke}{rgb}{0.000000,0.000000,0.000000}%
\pgfsetstrokecolor{currentstroke}%
\pgfsetdash{}{0pt}%
\pgfpathmoveto{\pgfqpoint{7.074143in}{1.850716in}}%
\pgfpathlineto{\pgfqpoint{7.898611in}{1.850716in}}%
\pgfusepath{stroke}%
\end{pgfscope}%
\begin{pgfscope}%
\definecolor{textcolor}{rgb}{0.000000,0.000000,0.000000}%
\pgfsetstrokecolor{textcolor}%
\pgfsetfillcolor{textcolor}%
\pgftext[x=7.486377in,y=1.934050in,,base]{\color{textcolor}\rmfamily\fontsize{11.000000}{13.200000}\selectfont Assur ...}%
\end{pgfscope}%
\begin{pgfscope}%
\pgfsetbuttcap%
\pgfsetmiterjoin%
\definecolor{currentfill}{rgb}{1.000000,1.000000,1.000000}%
\pgfsetfillcolor{currentfill}%
\pgfsetlinewidth{0.000000pt}%
\definecolor{currentstroke}{rgb}{0.000000,0.000000,0.000000}%
\pgfsetstrokecolor{currentstroke}%
\pgfsetstrokeopacity{0.000000}%
\pgfsetdash{}{0pt}%
\pgfpathmoveto{\pgfqpoint{0.148611in}{0.878085in}}%
\pgfpathlineto{\pgfqpoint{0.973079in}{0.878085in}}%
\pgfpathlineto{\pgfqpoint{0.973079in}{1.121243in}}%
\pgfpathlineto{\pgfqpoint{0.148611in}{1.121243in}}%
\pgfpathlineto{\pgfqpoint{0.148611in}{0.878085in}}%
\pgfpathclose%
\pgfusepath{fill}%
\end{pgfscope}%
\begin{pgfscope}%
\pgfpathrectangle{\pgfqpoint{0.148611in}{0.878085in}}{\pgfqpoint{0.824468in}{0.243158in}}%
\pgfusepath{clip}%
\pgfsetbuttcap%
\pgfsetmiterjoin%
\definecolor{currentfill}{rgb}{0.121569,0.466667,0.705882}%
\pgfsetfillcolor{currentfill}%
\pgfsetfillopacity{0.500000}%
\pgfsetlinewidth{1.003750pt}%
\definecolor{currentstroke}{rgb}{0.000000,0.000000,0.000000}%
\pgfsetstrokecolor{currentstroke}%
\pgfsetdash{}{0pt}%
\pgfpathmoveto{\pgfqpoint{0.186087in}{0.878085in}}%
\pgfpathlineto{\pgfqpoint{0.335990in}{0.878085in}}%
\pgfpathlineto{\pgfqpoint{0.335990in}{0.878571in}}%
\pgfpathlineto{\pgfqpoint{0.186087in}{0.878571in}}%
\pgfpathlineto{\pgfqpoint{0.186087in}{0.878085in}}%
\pgfpathclose%
\pgfusepath{stroke,fill}%
\end{pgfscope}%
\begin{pgfscope}%
\pgfpathrectangle{\pgfqpoint{0.148611in}{0.878085in}}{\pgfqpoint{0.824468in}{0.243158in}}%
\pgfusepath{clip}%
\pgfsetbuttcap%
\pgfsetmiterjoin%
\definecolor{currentfill}{rgb}{0.121569,0.466667,0.705882}%
\pgfsetfillcolor{currentfill}%
\pgfsetfillopacity{0.500000}%
\pgfsetlinewidth{1.003750pt}%
\definecolor{currentstroke}{rgb}{0.000000,0.000000,0.000000}%
\pgfsetstrokecolor{currentstroke}%
\pgfsetdash{}{0pt}%
\pgfpathmoveto{\pgfqpoint{0.335990in}{0.878085in}}%
\pgfpathlineto{\pgfqpoint{0.485894in}{0.878085in}}%
\pgfpathlineto{\pgfqpoint{0.485894in}{0.878085in}}%
\pgfpathlineto{\pgfqpoint{0.335990in}{0.878085in}}%
\pgfpathlineto{\pgfqpoint{0.335990in}{0.878085in}}%
\pgfpathclose%
\pgfusepath{stroke,fill}%
\end{pgfscope}%
\begin{pgfscope}%
\pgfpathrectangle{\pgfqpoint{0.148611in}{0.878085in}}{\pgfqpoint{0.824468in}{0.243158in}}%
\pgfusepath{clip}%
\pgfsetbuttcap%
\pgfsetmiterjoin%
\definecolor{currentfill}{rgb}{0.121569,0.466667,0.705882}%
\pgfsetfillcolor{currentfill}%
\pgfsetfillopacity{0.500000}%
\pgfsetlinewidth{1.003750pt}%
\definecolor{currentstroke}{rgb}{0.000000,0.000000,0.000000}%
\pgfsetstrokecolor{currentstroke}%
\pgfsetdash{}{0pt}%
\pgfpathmoveto{\pgfqpoint{0.485894in}{0.878085in}}%
\pgfpathlineto{\pgfqpoint{0.635797in}{0.878085in}}%
\pgfpathlineto{\pgfqpoint{0.635797in}{0.878085in}}%
\pgfpathlineto{\pgfqpoint{0.485894in}{0.878085in}}%
\pgfpathlineto{\pgfqpoint{0.485894in}{0.878085in}}%
\pgfpathclose%
\pgfusepath{stroke,fill}%
\end{pgfscope}%
\begin{pgfscope}%
\pgfpathrectangle{\pgfqpoint{0.148611in}{0.878085in}}{\pgfqpoint{0.824468in}{0.243158in}}%
\pgfusepath{clip}%
\pgfsetbuttcap%
\pgfsetmiterjoin%
\definecolor{currentfill}{rgb}{0.121569,0.466667,0.705882}%
\pgfsetfillcolor{currentfill}%
\pgfsetfillopacity{0.500000}%
\pgfsetlinewidth{1.003750pt}%
\definecolor{currentstroke}{rgb}{0.000000,0.000000,0.000000}%
\pgfsetstrokecolor{currentstroke}%
\pgfsetdash{}{0pt}%
\pgfpathmoveto{\pgfqpoint{0.635797in}{0.878085in}}%
\pgfpathlineto{\pgfqpoint{0.785700in}{0.878085in}}%
\pgfpathlineto{\pgfqpoint{0.785700in}{0.878085in}}%
\pgfpathlineto{\pgfqpoint{0.635797in}{0.878085in}}%
\pgfpathlineto{\pgfqpoint{0.635797in}{0.878085in}}%
\pgfpathclose%
\pgfusepath{stroke,fill}%
\end{pgfscope}%
\begin{pgfscope}%
\pgfpathrectangle{\pgfqpoint{0.148611in}{0.878085in}}{\pgfqpoint{0.824468in}{0.243158in}}%
\pgfusepath{clip}%
\pgfsetbuttcap%
\pgfsetmiterjoin%
\definecolor{currentfill}{rgb}{0.121569,0.466667,0.705882}%
\pgfsetfillcolor{currentfill}%
\pgfsetfillopacity{0.500000}%
\pgfsetlinewidth{1.003750pt}%
\definecolor{currentstroke}{rgb}{0.000000,0.000000,0.000000}%
\pgfsetstrokecolor{currentstroke}%
\pgfsetdash{}{0pt}%
\pgfpathmoveto{\pgfqpoint{0.785700in}{0.878085in}}%
\pgfpathlineto{\pgfqpoint{0.935603in}{0.878085in}}%
\pgfpathlineto{\pgfqpoint{0.935603in}{0.878085in}}%
\pgfpathlineto{\pgfqpoint{0.785700in}{0.878085in}}%
\pgfpathlineto{\pgfqpoint{0.785700in}{0.878085in}}%
\pgfpathclose%
\pgfusepath{stroke,fill}%
\end{pgfscope}%
\begin{pgfscope}%
\pgfsetrectcap%
\pgfsetmiterjoin%
\pgfsetlinewidth{0.803000pt}%
\definecolor{currentstroke}{rgb}{0.000000,0.000000,0.000000}%
\pgfsetstrokecolor{currentstroke}%
\pgfsetdash{}{0pt}%
\pgfpathmoveto{\pgfqpoint{0.148611in}{0.878085in}}%
\pgfpathlineto{\pgfqpoint{0.148611in}{1.121243in}}%
\pgfusepath{stroke}%
\end{pgfscope}%
\begin{pgfscope}%
\pgfsetrectcap%
\pgfsetmiterjoin%
\pgfsetlinewidth{0.803000pt}%
\definecolor{currentstroke}{rgb}{0.000000,0.000000,0.000000}%
\pgfsetstrokecolor{currentstroke}%
\pgfsetdash{}{0pt}%
\pgfpathmoveto{\pgfqpoint{0.973079in}{0.878085in}}%
\pgfpathlineto{\pgfqpoint{0.973079in}{1.121243in}}%
\pgfusepath{stroke}%
\end{pgfscope}%
\begin{pgfscope}%
\pgfsetrectcap%
\pgfsetmiterjoin%
\pgfsetlinewidth{0.803000pt}%
\definecolor{currentstroke}{rgb}{0.000000,0.000000,0.000000}%
\pgfsetstrokecolor{currentstroke}%
\pgfsetdash{}{0pt}%
\pgfpathmoveto{\pgfqpoint{0.148611in}{0.878085in}}%
\pgfpathlineto{\pgfqpoint{0.973079in}{0.878085in}}%
\pgfusepath{stroke}%
\end{pgfscope}%
\begin{pgfscope}%
\pgfsetrectcap%
\pgfsetmiterjoin%
\pgfsetlinewidth{0.803000pt}%
\definecolor{currentstroke}{rgb}{0.000000,0.000000,0.000000}%
\pgfsetstrokecolor{currentstroke}%
\pgfsetdash{}{0pt}%
\pgfpathmoveto{\pgfqpoint{0.148611in}{1.121243in}}%
\pgfpathlineto{\pgfqpoint{0.973079in}{1.121243in}}%
\pgfusepath{stroke}%
\end{pgfscope}%
\begin{pgfscope}%
\definecolor{textcolor}{rgb}{0.000000,0.000000,0.000000}%
\pgfsetstrokecolor{textcolor}%
\pgfsetfillcolor{textcolor}%
\pgftext[x=0.560845in,y=1.204576in,,base]{\color{textcolor}\rmfamily\fontsize{11.000000}{13.200000}\selectfont MMA}%
\end{pgfscope}%
\begin{pgfscope}%
\pgfsetbuttcap%
\pgfsetmiterjoin%
\definecolor{currentfill}{rgb}{1.000000,1.000000,1.000000}%
\pgfsetfillcolor{currentfill}%
\pgfsetlinewidth{0.000000pt}%
\definecolor{currentstroke}{rgb}{0.000000,0.000000,0.000000}%
\pgfsetstrokecolor{currentstroke}%
\pgfsetstrokeopacity{0.000000}%
\pgfsetdash{}{0pt}%
\pgfpathmoveto{\pgfqpoint{1.137973in}{0.878085in}}%
\pgfpathlineto{\pgfqpoint{1.962441in}{0.878085in}}%
\pgfpathlineto{\pgfqpoint{1.962441in}{1.121243in}}%
\pgfpathlineto{\pgfqpoint{1.137973in}{1.121243in}}%
\pgfpathlineto{\pgfqpoint{1.137973in}{0.878085in}}%
\pgfpathclose%
\pgfusepath{fill}%
\end{pgfscope}%
\begin{pgfscope}%
\pgfpathrectangle{\pgfqpoint{1.137973in}{0.878085in}}{\pgfqpoint{0.824468in}{0.243158in}}%
\pgfusepath{clip}%
\pgfsetbuttcap%
\pgfsetmiterjoin%
\definecolor{currentfill}{rgb}{0.121569,0.466667,0.705882}%
\pgfsetfillcolor{currentfill}%
\pgfsetfillopacity{0.500000}%
\pgfsetlinewidth{1.003750pt}%
\definecolor{currentstroke}{rgb}{0.000000,0.000000,0.000000}%
\pgfsetstrokecolor{currentstroke}%
\pgfsetdash{}{0pt}%
\pgfpathmoveto{\pgfqpoint{1.175449in}{0.878085in}}%
\pgfpathlineto{\pgfqpoint{1.325352in}{0.878085in}}%
\pgfpathlineto{\pgfqpoint{1.325352in}{0.881975in}}%
\pgfpathlineto{\pgfqpoint{1.175449in}{0.881975in}}%
\pgfpathlineto{\pgfqpoint{1.175449in}{0.878085in}}%
\pgfpathclose%
\pgfusepath{stroke,fill}%
\end{pgfscope}%
\begin{pgfscope}%
\pgfpathrectangle{\pgfqpoint{1.137973in}{0.878085in}}{\pgfqpoint{0.824468in}{0.243158in}}%
\pgfusepath{clip}%
\pgfsetbuttcap%
\pgfsetmiterjoin%
\definecolor{currentfill}{rgb}{0.121569,0.466667,0.705882}%
\pgfsetfillcolor{currentfill}%
\pgfsetfillopacity{0.500000}%
\pgfsetlinewidth{1.003750pt}%
\definecolor{currentstroke}{rgb}{0.000000,0.000000,0.000000}%
\pgfsetstrokecolor{currentstroke}%
\pgfsetdash{}{0pt}%
\pgfpathmoveto{\pgfqpoint{1.325352in}{0.878085in}}%
\pgfpathlineto{\pgfqpoint{1.475255in}{0.878085in}}%
\pgfpathlineto{\pgfqpoint{1.475255in}{0.879908in}}%
\pgfpathlineto{\pgfqpoint{1.325352in}{0.879908in}}%
\pgfpathlineto{\pgfqpoint{1.325352in}{0.878085in}}%
\pgfpathclose%
\pgfusepath{stroke,fill}%
\end{pgfscope}%
\begin{pgfscope}%
\pgfpathrectangle{\pgfqpoint{1.137973in}{0.878085in}}{\pgfqpoint{0.824468in}{0.243158in}}%
\pgfusepath{clip}%
\pgfsetbuttcap%
\pgfsetmiterjoin%
\definecolor{currentfill}{rgb}{0.121569,0.466667,0.705882}%
\pgfsetfillcolor{currentfill}%
\pgfsetfillopacity{0.500000}%
\pgfsetlinewidth{1.003750pt}%
\definecolor{currentstroke}{rgb}{0.000000,0.000000,0.000000}%
\pgfsetstrokecolor{currentstroke}%
\pgfsetdash{}{0pt}%
\pgfpathmoveto{\pgfqpoint{1.475255in}{0.878085in}}%
\pgfpathlineto{\pgfqpoint{1.625158in}{0.878085in}}%
\pgfpathlineto{\pgfqpoint{1.625158in}{0.878206in}}%
\pgfpathlineto{\pgfqpoint{1.475255in}{0.878206in}}%
\pgfpathlineto{\pgfqpoint{1.475255in}{0.878085in}}%
\pgfpathclose%
\pgfusepath{stroke,fill}%
\end{pgfscope}%
\begin{pgfscope}%
\pgfpathrectangle{\pgfqpoint{1.137973in}{0.878085in}}{\pgfqpoint{0.824468in}{0.243158in}}%
\pgfusepath{clip}%
\pgfsetbuttcap%
\pgfsetmiterjoin%
\definecolor{currentfill}{rgb}{0.121569,0.466667,0.705882}%
\pgfsetfillcolor{currentfill}%
\pgfsetfillopacity{0.500000}%
\pgfsetlinewidth{1.003750pt}%
\definecolor{currentstroke}{rgb}{0.000000,0.000000,0.000000}%
\pgfsetstrokecolor{currentstroke}%
\pgfsetdash{}{0pt}%
\pgfpathmoveto{\pgfqpoint{1.625158in}{0.878085in}}%
\pgfpathlineto{\pgfqpoint{1.775062in}{0.878085in}}%
\pgfpathlineto{\pgfqpoint{1.775062in}{0.878085in}}%
\pgfpathlineto{\pgfqpoint{1.625158in}{0.878085in}}%
\pgfpathlineto{\pgfqpoint{1.625158in}{0.878085in}}%
\pgfpathclose%
\pgfusepath{stroke,fill}%
\end{pgfscope}%
\begin{pgfscope}%
\pgfpathrectangle{\pgfqpoint{1.137973in}{0.878085in}}{\pgfqpoint{0.824468in}{0.243158in}}%
\pgfusepath{clip}%
\pgfsetbuttcap%
\pgfsetmiterjoin%
\definecolor{currentfill}{rgb}{0.121569,0.466667,0.705882}%
\pgfsetfillcolor{currentfill}%
\pgfsetfillopacity{0.500000}%
\pgfsetlinewidth{1.003750pt}%
\definecolor{currentstroke}{rgb}{0.000000,0.000000,0.000000}%
\pgfsetstrokecolor{currentstroke}%
\pgfsetdash{}{0pt}%
\pgfpathmoveto{\pgfqpoint{1.775062in}{0.878085in}}%
\pgfpathlineto{\pgfqpoint{1.924965in}{0.878085in}}%
\pgfpathlineto{\pgfqpoint{1.924965in}{0.878693in}}%
\pgfpathlineto{\pgfqpoint{1.775062in}{0.878693in}}%
\pgfpathlineto{\pgfqpoint{1.775062in}{0.878085in}}%
\pgfpathclose%
\pgfusepath{stroke,fill}%
\end{pgfscope}%
\begin{pgfscope}%
\pgfsetrectcap%
\pgfsetmiterjoin%
\pgfsetlinewidth{0.803000pt}%
\definecolor{currentstroke}{rgb}{0.000000,0.000000,0.000000}%
\pgfsetstrokecolor{currentstroke}%
\pgfsetdash{}{0pt}%
\pgfpathmoveto{\pgfqpoint{1.137973in}{0.878085in}}%
\pgfpathlineto{\pgfqpoint{1.137973in}{1.121243in}}%
\pgfusepath{stroke}%
\end{pgfscope}%
\begin{pgfscope}%
\pgfsetrectcap%
\pgfsetmiterjoin%
\pgfsetlinewidth{0.803000pt}%
\definecolor{currentstroke}{rgb}{0.000000,0.000000,0.000000}%
\pgfsetstrokecolor{currentstroke}%
\pgfsetdash{}{0pt}%
\pgfpathmoveto{\pgfqpoint{1.962441in}{0.878085in}}%
\pgfpathlineto{\pgfqpoint{1.962441in}{1.121243in}}%
\pgfusepath{stroke}%
\end{pgfscope}%
\begin{pgfscope}%
\pgfsetrectcap%
\pgfsetmiterjoin%
\pgfsetlinewidth{0.803000pt}%
\definecolor{currentstroke}{rgb}{0.000000,0.000000,0.000000}%
\pgfsetstrokecolor{currentstroke}%
\pgfsetdash{}{0pt}%
\pgfpathmoveto{\pgfqpoint{1.137973in}{0.878085in}}%
\pgfpathlineto{\pgfqpoint{1.962441in}{0.878085in}}%
\pgfusepath{stroke}%
\end{pgfscope}%
\begin{pgfscope}%
\pgfsetrectcap%
\pgfsetmiterjoin%
\pgfsetlinewidth{0.803000pt}%
\definecolor{currentstroke}{rgb}{0.000000,0.000000,0.000000}%
\pgfsetstrokecolor{currentstroke}%
\pgfsetdash{}{0pt}%
\pgfpathmoveto{\pgfqpoint{1.137973in}{1.121243in}}%
\pgfpathlineto{\pgfqpoint{1.962441in}{1.121243in}}%
\pgfusepath{stroke}%
\end{pgfscope}%
\begin{pgfscope}%
\definecolor{textcolor}{rgb}{0.000000,0.000000,0.000000}%
\pgfsetstrokecolor{textcolor}%
\pgfsetfillcolor{textcolor}%
\pgftext[x=1.550207in,y=1.204576in,,base]{\color{textcolor}\rmfamily\fontsize{11.000000}{13.200000}\selectfont MetLife}%
\end{pgfscope}%
\begin{pgfscope}%
\pgfsetbuttcap%
\pgfsetmiterjoin%
\definecolor{currentfill}{rgb}{1.000000,1.000000,1.000000}%
\pgfsetfillcolor{currentfill}%
\pgfsetlinewidth{0.000000pt}%
\definecolor{currentstroke}{rgb}{0.000000,0.000000,0.000000}%
\pgfsetstrokecolor{currentstroke}%
\pgfsetstrokeopacity{0.000000}%
\pgfsetdash{}{0pt}%
\pgfpathmoveto{\pgfqpoint{2.127335in}{0.878085in}}%
\pgfpathlineto{\pgfqpoint{2.951803in}{0.878085in}}%
\pgfpathlineto{\pgfqpoint{2.951803in}{1.121243in}}%
\pgfpathlineto{\pgfqpoint{2.127335in}{1.121243in}}%
\pgfpathlineto{\pgfqpoint{2.127335in}{0.878085in}}%
\pgfpathclose%
\pgfusepath{fill}%
\end{pgfscope}%
\begin{pgfscope}%
\pgfpathrectangle{\pgfqpoint{2.127335in}{0.878085in}}{\pgfqpoint{0.824468in}{0.243158in}}%
\pgfusepath{clip}%
\pgfsetbuttcap%
\pgfsetmiterjoin%
\definecolor{currentfill}{rgb}{0.121569,0.466667,0.705882}%
\pgfsetfillcolor{currentfill}%
\pgfsetfillopacity{0.500000}%
\pgfsetlinewidth{1.003750pt}%
\definecolor{currentstroke}{rgb}{0.000000,0.000000,0.000000}%
\pgfsetstrokecolor{currentstroke}%
\pgfsetdash{}{0pt}%
\pgfpathmoveto{\pgfqpoint{2.164810in}{0.878085in}}%
\pgfpathlineto{\pgfqpoint{2.314714in}{0.878085in}}%
\pgfpathlineto{\pgfqpoint{2.314714in}{0.888054in}}%
\pgfpathlineto{\pgfqpoint{2.164810in}{0.888054in}}%
\pgfpathlineto{\pgfqpoint{2.164810in}{0.878085in}}%
\pgfpathclose%
\pgfusepath{stroke,fill}%
\end{pgfscope}%
\begin{pgfscope}%
\pgfpathrectangle{\pgfqpoint{2.127335in}{0.878085in}}{\pgfqpoint{0.824468in}{0.243158in}}%
\pgfusepath{clip}%
\pgfsetbuttcap%
\pgfsetmiterjoin%
\definecolor{currentfill}{rgb}{0.121569,0.466667,0.705882}%
\pgfsetfillcolor{currentfill}%
\pgfsetfillopacity{0.500000}%
\pgfsetlinewidth{1.003750pt}%
\definecolor{currentstroke}{rgb}{0.000000,0.000000,0.000000}%
\pgfsetstrokecolor{currentstroke}%
\pgfsetdash{}{0pt}%
\pgfpathmoveto{\pgfqpoint{2.314714in}{0.878085in}}%
\pgfpathlineto{\pgfqpoint{2.464617in}{0.878085in}}%
\pgfpathlineto{\pgfqpoint{2.464617in}{0.882218in}}%
\pgfpathlineto{\pgfqpoint{2.314714in}{0.882218in}}%
\pgfpathlineto{\pgfqpoint{2.314714in}{0.878085in}}%
\pgfpathclose%
\pgfusepath{stroke,fill}%
\end{pgfscope}%
\begin{pgfscope}%
\pgfpathrectangle{\pgfqpoint{2.127335in}{0.878085in}}{\pgfqpoint{0.824468in}{0.243158in}}%
\pgfusepath{clip}%
\pgfsetbuttcap%
\pgfsetmiterjoin%
\definecolor{currentfill}{rgb}{0.121569,0.466667,0.705882}%
\pgfsetfillcolor{currentfill}%
\pgfsetfillopacity{0.500000}%
\pgfsetlinewidth{1.003750pt}%
\definecolor{currentstroke}{rgb}{0.000000,0.000000,0.000000}%
\pgfsetstrokecolor{currentstroke}%
\pgfsetdash{}{0pt}%
\pgfpathmoveto{\pgfqpoint{2.464617in}{0.878085in}}%
\pgfpathlineto{\pgfqpoint{2.614520in}{0.878085in}}%
\pgfpathlineto{\pgfqpoint{2.614520in}{0.880273in}}%
\pgfpathlineto{\pgfqpoint{2.464617in}{0.880273in}}%
\pgfpathlineto{\pgfqpoint{2.464617in}{0.878085in}}%
\pgfpathclose%
\pgfusepath{stroke,fill}%
\end{pgfscope}%
\begin{pgfscope}%
\pgfpathrectangle{\pgfqpoint{2.127335in}{0.878085in}}{\pgfqpoint{0.824468in}{0.243158in}}%
\pgfusepath{clip}%
\pgfsetbuttcap%
\pgfsetmiterjoin%
\definecolor{currentfill}{rgb}{0.121569,0.466667,0.705882}%
\pgfsetfillcolor{currentfill}%
\pgfsetfillopacity{0.500000}%
\pgfsetlinewidth{1.003750pt}%
\definecolor{currentstroke}{rgb}{0.000000,0.000000,0.000000}%
\pgfsetstrokecolor{currentstroke}%
\pgfsetdash{}{0pt}%
\pgfpathmoveto{\pgfqpoint{2.614520in}{0.878085in}}%
\pgfpathlineto{\pgfqpoint{2.764423in}{0.878085in}}%
\pgfpathlineto{\pgfqpoint{2.764423in}{0.878693in}}%
\pgfpathlineto{\pgfqpoint{2.614520in}{0.878693in}}%
\pgfpathlineto{\pgfqpoint{2.614520in}{0.878085in}}%
\pgfpathclose%
\pgfusepath{stroke,fill}%
\end{pgfscope}%
\begin{pgfscope}%
\pgfpathrectangle{\pgfqpoint{2.127335in}{0.878085in}}{\pgfqpoint{0.824468in}{0.243158in}}%
\pgfusepath{clip}%
\pgfsetbuttcap%
\pgfsetmiterjoin%
\definecolor{currentfill}{rgb}{0.121569,0.466667,0.705882}%
\pgfsetfillcolor{currentfill}%
\pgfsetfillopacity{0.500000}%
\pgfsetlinewidth{1.003750pt}%
\definecolor{currentstroke}{rgb}{0.000000,0.000000,0.000000}%
\pgfsetstrokecolor{currentstroke}%
\pgfsetdash{}{0pt}%
\pgfpathmoveto{\pgfqpoint{2.764423in}{0.878085in}}%
\pgfpathlineto{\pgfqpoint{2.914327in}{0.878085in}}%
\pgfpathlineto{\pgfqpoint{2.914327in}{0.878936in}}%
\pgfpathlineto{\pgfqpoint{2.764423in}{0.878936in}}%
\pgfpathlineto{\pgfqpoint{2.764423in}{0.878085in}}%
\pgfpathclose%
\pgfusepath{stroke,fill}%
\end{pgfscope}%
\begin{pgfscope}%
\pgfsetrectcap%
\pgfsetmiterjoin%
\pgfsetlinewidth{0.803000pt}%
\definecolor{currentstroke}{rgb}{0.000000,0.000000,0.000000}%
\pgfsetstrokecolor{currentstroke}%
\pgfsetdash{}{0pt}%
\pgfpathmoveto{\pgfqpoint{2.127335in}{0.878085in}}%
\pgfpathlineto{\pgfqpoint{2.127335in}{1.121243in}}%
\pgfusepath{stroke}%
\end{pgfscope}%
\begin{pgfscope}%
\pgfsetrectcap%
\pgfsetmiterjoin%
\pgfsetlinewidth{0.803000pt}%
\definecolor{currentstroke}{rgb}{0.000000,0.000000,0.000000}%
\pgfsetstrokecolor{currentstroke}%
\pgfsetdash{}{0pt}%
\pgfpathmoveto{\pgfqpoint{2.951803in}{0.878085in}}%
\pgfpathlineto{\pgfqpoint{2.951803in}{1.121243in}}%
\pgfusepath{stroke}%
\end{pgfscope}%
\begin{pgfscope}%
\pgfsetrectcap%
\pgfsetmiterjoin%
\pgfsetlinewidth{0.803000pt}%
\definecolor{currentstroke}{rgb}{0.000000,0.000000,0.000000}%
\pgfsetstrokecolor{currentstroke}%
\pgfsetdash{}{0pt}%
\pgfpathmoveto{\pgfqpoint{2.127335in}{0.878085in}}%
\pgfpathlineto{\pgfqpoint{2.951803in}{0.878085in}}%
\pgfusepath{stroke}%
\end{pgfscope}%
\begin{pgfscope}%
\pgfsetrectcap%
\pgfsetmiterjoin%
\pgfsetlinewidth{0.803000pt}%
\definecolor{currentstroke}{rgb}{0.000000,0.000000,0.000000}%
\pgfsetstrokecolor{currentstroke}%
\pgfsetdash{}{0pt}%
\pgfpathmoveto{\pgfqpoint{2.127335in}{1.121243in}}%
\pgfpathlineto{\pgfqpoint{2.951803in}{1.121243in}}%
\pgfusepath{stroke}%
\end{pgfscope}%
\begin{pgfscope}%
\definecolor{textcolor}{rgb}{0.000000,0.000000,0.000000}%
\pgfsetstrokecolor{textcolor}%
\pgfsetfillcolor{textcolor}%
\pgftext[x=2.539569in,y=1.204576in,,base]{\color{textcolor}\rmfamily\fontsize{11.000000}{13.200000}\selectfont Crédit...}%
\end{pgfscope}%
\begin{pgfscope}%
\pgfsetbuttcap%
\pgfsetmiterjoin%
\definecolor{currentfill}{rgb}{1.000000,1.000000,1.000000}%
\pgfsetfillcolor{currentfill}%
\pgfsetlinewidth{0.000000pt}%
\definecolor{currentstroke}{rgb}{0.000000,0.000000,0.000000}%
\pgfsetstrokecolor{currentstroke}%
\pgfsetstrokeopacity{0.000000}%
\pgfsetdash{}{0pt}%
\pgfpathmoveto{\pgfqpoint{3.116696in}{0.878085in}}%
\pgfpathlineto{\pgfqpoint{3.941164in}{0.878085in}}%
\pgfpathlineto{\pgfqpoint{3.941164in}{1.121243in}}%
\pgfpathlineto{\pgfqpoint{3.116696in}{1.121243in}}%
\pgfpathlineto{\pgfqpoint{3.116696in}{0.878085in}}%
\pgfpathclose%
\pgfusepath{fill}%
\end{pgfscope}%
\begin{pgfscope}%
\pgfpathrectangle{\pgfqpoint{3.116696in}{0.878085in}}{\pgfqpoint{0.824468in}{0.243158in}}%
\pgfusepath{clip}%
\pgfsetbuttcap%
\pgfsetmiterjoin%
\definecolor{currentfill}{rgb}{0.121569,0.466667,0.705882}%
\pgfsetfillcolor{currentfill}%
\pgfsetfillopacity{0.500000}%
\pgfsetlinewidth{1.003750pt}%
\definecolor{currentstroke}{rgb}{0.000000,0.000000,0.000000}%
\pgfsetstrokecolor{currentstroke}%
\pgfsetdash{}{0pt}%
\pgfpathmoveto{\pgfqpoint{3.154172in}{0.878085in}}%
\pgfpathlineto{\pgfqpoint{3.304075in}{0.878085in}}%
\pgfpathlineto{\pgfqpoint{3.304075in}{0.880030in}}%
\pgfpathlineto{\pgfqpoint{3.154172in}{0.880030in}}%
\pgfpathlineto{\pgfqpoint{3.154172in}{0.878085in}}%
\pgfpathclose%
\pgfusepath{stroke,fill}%
\end{pgfscope}%
\begin{pgfscope}%
\pgfpathrectangle{\pgfqpoint{3.116696in}{0.878085in}}{\pgfqpoint{0.824468in}{0.243158in}}%
\pgfusepath{clip}%
\pgfsetbuttcap%
\pgfsetmiterjoin%
\definecolor{currentfill}{rgb}{0.121569,0.466667,0.705882}%
\pgfsetfillcolor{currentfill}%
\pgfsetfillopacity{0.500000}%
\pgfsetlinewidth{1.003750pt}%
\definecolor{currentstroke}{rgb}{0.000000,0.000000,0.000000}%
\pgfsetstrokecolor{currentstroke}%
\pgfsetdash{}{0pt}%
\pgfpathmoveto{\pgfqpoint{3.304075in}{0.878085in}}%
\pgfpathlineto{\pgfqpoint{3.453979in}{0.878085in}}%
\pgfpathlineto{\pgfqpoint{3.453979in}{0.878571in}}%
\pgfpathlineto{\pgfqpoint{3.304075in}{0.878571in}}%
\pgfpathlineto{\pgfqpoint{3.304075in}{0.878085in}}%
\pgfpathclose%
\pgfusepath{stroke,fill}%
\end{pgfscope}%
\begin{pgfscope}%
\pgfpathrectangle{\pgfqpoint{3.116696in}{0.878085in}}{\pgfqpoint{0.824468in}{0.243158in}}%
\pgfusepath{clip}%
\pgfsetbuttcap%
\pgfsetmiterjoin%
\definecolor{currentfill}{rgb}{0.121569,0.466667,0.705882}%
\pgfsetfillcolor{currentfill}%
\pgfsetfillopacity{0.500000}%
\pgfsetlinewidth{1.003750pt}%
\definecolor{currentstroke}{rgb}{0.000000,0.000000,0.000000}%
\pgfsetstrokecolor{currentstroke}%
\pgfsetdash{}{0pt}%
\pgfpathmoveto{\pgfqpoint{3.453979in}{0.878085in}}%
\pgfpathlineto{\pgfqpoint{3.603882in}{0.878085in}}%
\pgfpathlineto{\pgfqpoint{3.603882in}{0.878085in}}%
\pgfpathlineto{\pgfqpoint{3.453979in}{0.878085in}}%
\pgfpathlineto{\pgfqpoint{3.453979in}{0.878085in}}%
\pgfpathclose%
\pgfusepath{stroke,fill}%
\end{pgfscope}%
\begin{pgfscope}%
\pgfpathrectangle{\pgfqpoint{3.116696in}{0.878085in}}{\pgfqpoint{0.824468in}{0.243158in}}%
\pgfusepath{clip}%
\pgfsetbuttcap%
\pgfsetmiterjoin%
\definecolor{currentfill}{rgb}{0.121569,0.466667,0.705882}%
\pgfsetfillcolor{currentfill}%
\pgfsetfillopacity{0.500000}%
\pgfsetlinewidth{1.003750pt}%
\definecolor{currentstroke}{rgb}{0.000000,0.000000,0.000000}%
\pgfsetstrokecolor{currentstroke}%
\pgfsetdash{}{0pt}%
\pgfpathmoveto{\pgfqpoint{3.603882in}{0.878085in}}%
\pgfpathlineto{\pgfqpoint{3.753785in}{0.878085in}}%
\pgfpathlineto{\pgfqpoint{3.753785in}{0.878328in}}%
\pgfpathlineto{\pgfqpoint{3.603882in}{0.878328in}}%
\pgfpathlineto{\pgfqpoint{3.603882in}{0.878085in}}%
\pgfpathclose%
\pgfusepath{stroke,fill}%
\end{pgfscope}%
\begin{pgfscope}%
\pgfpathrectangle{\pgfqpoint{3.116696in}{0.878085in}}{\pgfqpoint{0.824468in}{0.243158in}}%
\pgfusepath{clip}%
\pgfsetbuttcap%
\pgfsetmiterjoin%
\definecolor{currentfill}{rgb}{0.121569,0.466667,0.705882}%
\pgfsetfillcolor{currentfill}%
\pgfsetfillopacity{0.500000}%
\pgfsetlinewidth{1.003750pt}%
\definecolor{currentstroke}{rgb}{0.000000,0.000000,0.000000}%
\pgfsetstrokecolor{currentstroke}%
\pgfsetdash{}{0pt}%
\pgfpathmoveto{\pgfqpoint{3.753785in}{0.878085in}}%
\pgfpathlineto{\pgfqpoint{3.903688in}{0.878085in}}%
\pgfpathlineto{\pgfqpoint{3.903688in}{0.878571in}}%
\pgfpathlineto{\pgfqpoint{3.753785in}{0.878571in}}%
\pgfpathlineto{\pgfqpoint{3.753785in}{0.878085in}}%
\pgfpathclose%
\pgfusepath{stroke,fill}%
\end{pgfscope}%
\begin{pgfscope}%
\pgfsetrectcap%
\pgfsetmiterjoin%
\pgfsetlinewidth{0.803000pt}%
\definecolor{currentstroke}{rgb}{0.000000,0.000000,0.000000}%
\pgfsetstrokecolor{currentstroke}%
\pgfsetdash{}{0pt}%
\pgfpathmoveto{\pgfqpoint{3.116696in}{0.878085in}}%
\pgfpathlineto{\pgfqpoint{3.116696in}{1.121243in}}%
\pgfusepath{stroke}%
\end{pgfscope}%
\begin{pgfscope}%
\pgfsetrectcap%
\pgfsetmiterjoin%
\pgfsetlinewidth{0.803000pt}%
\definecolor{currentstroke}{rgb}{0.000000,0.000000,0.000000}%
\pgfsetstrokecolor{currentstroke}%
\pgfsetdash{}{0pt}%
\pgfpathmoveto{\pgfqpoint{3.941164in}{0.878085in}}%
\pgfpathlineto{\pgfqpoint{3.941164in}{1.121243in}}%
\pgfusepath{stroke}%
\end{pgfscope}%
\begin{pgfscope}%
\pgfsetrectcap%
\pgfsetmiterjoin%
\pgfsetlinewidth{0.803000pt}%
\definecolor{currentstroke}{rgb}{0.000000,0.000000,0.000000}%
\pgfsetstrokecolor{currentstroke}%
\pgfsetdash{}{0pt}%
\pgfpathmoveto{\pgfqpoint{3.116696in}{0.878085in}}%
\pgfpathlineto{\pgfqpoint{3.941164in}{0.878085in}}%
\pgfusepath{stroke}%
\end{pgfscope}%
\begin{pgfscope}%
\pgfsetrectcap%
\pgfsetmiterjoin%
\pgfsetlinewidth{0.803000pt}%
\definecolor{currentstroke}{rgb}{0.000000,0.000000,0.000000}%
\pgfsetstrokecolor{currentstroke}%
\pgfsetdash{}{0pt}%
\pgfpathmoveto{\pgfqpoint{3.116696in}{1.121243in}}%
\pgfpathlineto{\pgfqpoint{3.941164in}{1.121243in}}%
\pgfusepath{stroke}%
\end{pgfscope}%
\begin{pgfscope}%
\definecolor{textcolor}{rgb}{0.000000,0.000000,0.000000}%
\pgfsetstrokecolor{textcolor}%
\pgfsetfillcolor{textcolor}%
\pgftext[x=3.528930in,y=1.204576in,,base]{\color{textcolor}\rmfamily\fontsize{11.000000}{13.200000}\selectfont Afi Esca}%
\end{pgfscope}%
\begin{pgfscope}%
\pgfsetbuttcap%
\pgfsetmiterjoin%
\definecolor{currentfill}{rgb}{1.000000,1.000000,1.000000}%
\pgfsetfillcolor{currentfill}%
\pgfsetlinewidth{0.000000pt}%
\definecolor{currentstroke}{rgb}{0.000000,0.000000,0.000000}%
\pgfsetstrokecolor{currentstroke}%
\pgfsetstrokeopacity{0.000000}%
\pgfsetdash{}{0pt}%
\pgfpathmoveto{\pgfqpoint{4.106058in}{0.878085in}}%
\pgfpathlineto{\pgfqpoint{4.930526in}{0.878085in}}%
\pgfpathlineto{\pgfqpoint{4.930526in}{1.121243in}}%
\pgfpathlineto{\pgfqpoint{4.106058in}{1.121243in}}%
\pgfpathlineto{\pgfqpoint{4.106058in}{0.878085in}}%
\pgfpathclose%
\pgfusepath{fill}%
\end{pgfscope}%
\begin{pgfscope}%
\pgfpathrectangle{\pgfqpoint{4.106058in}{0.878085in}}{\pgfqpoint{0.824468in}{0.243158in}}%
\pgfusepath{clip}%
\pgfsetbuttcap%
\pgfsetmiterjoin%
\definecolor{currentfill}{rgb}{0.121569,0.466667,0.705882}%
\pgfsetfillcolor{currentfill}%
\pgfsetfillopacity{0.500000}%
\pgfsetlinewidth{1.003750pt}%
\definecolor{currentstroke}{rgb}{0.000000,0.000000,0.000000}%
\pgfsetstrokecolor{currentstroke}%
\pgfsetdash{}{0pt}%
\pgfpathmoveto{\pgfqpoint{4.143534in}{0.878085in}}%
\pgfpathlineto{\pgfqpoint{4.293437in}{0.878085in}}%
\pgfpathlineto{\pgfqpoint{4.293437in}{0.881246in}}%
\pgfpathlineto{\pgfqpoint{4.143534in}{0.881246in}}%
\pgfpathlineto{\pgfqpoint{4.143534in}{0.878085in}}%
\pgfpathclose%
\pgfusepath{stroke,fill}%
\end{pgfscope}%
\begin{pgfscope}%
\pgfpathrectangle{\pgfqpoint{4.106058in}{0.878085in}}{\pgfqpoint{0.824468in}{0.243158in}}%
\pgfusepath{clip}%
\pgfsetbuttcap%
\pgfsetmiterjoin%
\definecolor{currentfill}{rgb}{0.121569,0.466667,0.705882}%
\pgfsetfillcolor{currentfill}%
\pgfsetfillopacity{0.500000}%
\pgfsetlinewidth{1.003750pt}%
\definecolor{currentstroke}{rgb}{0.000000,0.000000,0.000000}%
\pgfsetstrokecolor{currentstroke}%
\pgfsetdash{}{0pt}%
\pgfpathmoveto{\pgfqpoint{4.293437in}{0.878085in}}%
\pgfpathlineto{\pgfqpoint{4.443340in}{0.878085in}}%
\pgfpathlineto{\pgfqpoint{4.443340in}{0.878450in}}%
\pgfpathlineto{\pgfqpoint{4.293437in}{0.878450in}}%
\pgfpathlineto{\pgfqpoint{4.293437in}{0.878085in}}%
\pgfpathclose%
\pgfusepath{stroke,fill}%
\end{pgfscope}%
\begin{pgfscope}%
\pgfpathrectangle{\pgfqpoint{4.106058in}{0.878085in}}{\pgfqpoint{0.824468in}{0.243158in}}%
\pgfusepath{clip}%
\pgfsetbuttcap%
\pgfsetmiterjoin%
\definecolor{currentfill}{rgb}{0.121569,0.466667,0.705882}%
\pgfsetfillcolor{currentfill}%
\pgfsetfillopacity{0.500000}%
\pgfsetlinewidth{1.003750pt}%
\definecolor{currentstroke}{rgb}{0.000000,0.000000,0.000000}%
\pgfsetstrokecolor{currentstroke}%
\pgfsetdash{}{0pt}%
\pgfpathmoveto{\pgfqpoint{4.443340in}{0.878085in}}%
\pgfpathlineto{\pgfqpoint{4.593244in}{0.878085in}}%
\pgfpathlineto{\pgfqpoint{4.593244in}{0.878206in}}%
\pgfpathlineto{\pgfqpoint{4.443340in}{0.878206in}}%
\pgfpathlineto{\pgfqpoint{4.443340in}{0.878085in}}%
\pgfpathclose%
\pgfusepath{stroke,fill}%
\end{pgfscope}%
\begin{pgfscope}%
\pgfpathrectangle{\pgfqpoint{4.106058in}{0.878085in}}{\pgfqpoint{0.824468in}{0.243158in}}%
\pgfusepath{clip}%
\pgfsetbuttcap%
\pgfsetmiterjoin%
\definecolor{currentfill}{rgb}{0.121569,0.466667,0.705882}%
\pgfsetfillcolor{currentfill}%
\pgfsetfillopacity{0.500000}%
\pgfsetlinewidth{1.003750pt}%
\definecolor{currentstroke}{rgb}{0.000000,0.000000,0.000000}%
\pgfsetstrokecolor{currentstroke}%
\pgfsetdash{}{0pt}%
\pgfpathmoveto{\pgfqpoint{4.593244in}{0.878085in}}%
\pgfpathlineto{\pgfqpoint{4.743147in}{0.878085in}}%
\pgfpathlineto{\pgfqpoint{4.743147in}{0.878206in}}%
\pgfpathlineto{\pgfqpoint{4.593244in}{0.878206in}}%
\pgfpathlineto{\pgfqpoint{4.593244in}{0.878085in}}%
\pgfpathclose%
\pgfusepath{stroke,fill}%
\end{pgfscope}%
\begin{pgfscope}%
\pgfpathrectangle{\pgfqpoint{4.106058in}{0.878085in}}{\pgfqpoint{0.824468in}{0.243158in}}%
\pgfusepath{clip}%
\pgfsetbuttcap%
\pgfsetmiterjoin%
\definecolor{currentfill}{rgb}{0.121569,0.466667,0.705882}%
\pgfsetfillcolor{currentfill}%
\pgfsetfillopacity{0.500000}%
\pgfsetlinewidth{1.003750pt}%
\definecolor{currentstroke}{rgb}{0.000000,0.000000,0.000000}%
\pgfsetstrokecolor{currentstroke}%
\pgfsetdash{}{0pt}%
\pgfpathmoveto{\pgfqpoint{4.743147in}{0.878085in}}%
\pgfpathlineto{\pgfqpoint{4.893050in}{0.878085in}}%
\pgfpathlineto{\pgfqpoint{4.893050in}{0.878328in}}%
\pgfpathlineto{\pgfqpoint{4.743147in}{0.878328in}}%
\pgfpathlineto{\pgfqpoint{4.743147in}{0.878085in}}%
\pgfpathclose%
\pgfusepath{stroke,fill}%
\end{pgfscope}%
\begin{pgfscope}%
\pgfsetrectcap%
\pgfsetmiterjoin%
\pgfsetlinewidth{0.803000pt}%
\definecolor{currentstroke}{rgb}{0.000000,0.000000,0.000000}%
\pgfsetstrokecolor{currentstroke}%
\pgfsetdash{}{0pt}%
\pgfpathmoveto{\pgfqpoint{4.106058in}{0.878085in}}%
\pgfpathlineto{\pgfqpoint{4.106058in}{1.121243in}}%
\pgfusepath{stroke}%
\end{pgfscope}%
\begin{pgfscope}%
\pgfsetrectcap%
\pgfsetmiterjoin%
\pgfsetlinewidth{0.803000pt}%
\definecolor{currentstroke}{rgb}{0.000000,0.000000,0.000000}%
\pgfsetstrokecolor{currentstroke}%
\pgfsetdash{}{0pt}%
\pgfpathmoveto{\pgfqpoint{4.930526in}{0.878085in}}%
\pgfpathlineto{\pgfqpoint{4.930526in}{1.121243in}}%
\pgfusepath{stroke}%
\end{pgfscope}%
\begin{pgfscope}%
\pgfsetrectcap%
\pgfsetmiterjoin%
\pgfsetlinewidth{0.803000pt}%
\definecolor{currentstroke}{rgb}{0.000000,0.000000,0.000000}%
\pgfsetstrokecolor{currentstroke}%
\pgfsetdash{}{0pt}%
\pgfpathmoveto{\pgfqpoint{4.106058in}{0.878085in}}%
\pgfpathlineto{\pgfqpoint{4.930526in}{0.878085in}}%
\pgfusepath{stroke}%
\end{pgfscope}%
\begin{pgfscope}%
\pgfsetrectcap%
\pgfsetmiterjoin%
\pgfsetlinewidth{0.803000pt}%
\definecolor{currentstroke}{rgb}{0.000000,0.000000,0.000000}%
\pgfsetstrokecolor{currentstroke}%
\pgfsetdash{}{0pt}%
\pgfpathmoveto{\pgfqpoint{4.106058in}{1.121243in}}%
\pgfpathlineto{\pgfqpoint{4.930526in}{1.121243in}}%
\pgfusepath{stroke}%
\end{pgfscope}%
\begin{pgfscope}%
\definecolor{textcolor}{rgb}{0.000000,0.000000,0.000000}%
\pgfsetstrokecolor{textcolor}%
\pgfsetfillcolor{textcolor}%
\pgftext[x=4.518292in,y=1.204576in,,base]{\color{textcolor}\rmfamily\fontsize{11.000000}{13.200000}\selectfont Gan}%
\end{pgfscope}%
\begin{pgfscope}%
\pgfsetbuttcap%
\pgfsetmiterjoin%
\definecolor{currentfill}{rgb}{1.000000,1.000000,1.000000}%
\pgfsetfillcolor{currentfill}%
\pgfsetlinewidth{0.000000pt}%
\definecolor{currentstroke}{rgb}{0.000000,0.000000,0.000000}%
\pgfsetstrokecolor{currentstroke}%
\pgfsetstrokeopacity{0.000000}%
\pgfsetdash{}{0pt}%
\pgfpathmoveto{\pgfqpoint{5.095420in}{0.878085in}}%
\pgfpathlineto{\pgfqpoint{5.919888in}{0.878085in}}%
\pgfpathlineto{\pgfqpoint{5.919888in}{1.121243in}}%
\pgfpathlineto{\pgfqpoint{5.095420in}{1.121243in}}%
\pgfpathlineto{\pgfqpoint{5.095420in}{0.878085in}}%
\pgfpathclose%
\pgfusepath{fill}%
\end{pgfscope}%
\begin{pgfscope}%
\pgfpathrectangle{\pgfqpoint{5.095420in}{0.878085in}}{\pgfqpoint{0.824468in}{0.243158in}}%
\pgfusepath{clip}%
\pgfsetbuttcap%
\pgfsetmiterjoin%
\definecolor{currentfill}{rgb}{0.121569,0.466667,0.705882}%
\pgfsetfillcolor{currentfill}%
\pgfsetfillopacity{0.500000}%
\pgfsetlinewidth{1.003750pt}%
\definecolor{currentstroke}{rgb}{0.000000,0.000000,0.000000}%
\pgfsetstrokecolor{currentstroke}%
\pgfsetdash{}{0pt}%
\pgfpathmoveto{\pgfqpoint{5.132895in}{0.878085in}}%
\pgfpathlineto{\pgfqpoint{5.282799in}{0.878085in}}%
\pgfpathlineto{\pgfqpoint{5.282799in}{0.878814in}}%
\pgfpathlineto{\pgfqpoint{5.132895in}{0.878814in}}%
\pgfpathlineto{\pgfqpoint{5.132895in}{0.878085in}}%
\pgfpathclose%
\pgfusepath{stroke,fill}%
\end{pgfscope}%
\begin{pgfscope}%
\pgfpathrectangle{\pgfqpoint{5.095420in}{0.878085in}}{\pgfqpoint{0.824468in}{0.243158in}}%
\pgfusepath{clip}%
\pgfsetbuttcap%
\pgfsetmiterjoin%
\definecolor{currentfill}{rgb}{0.121569,0.466667,0.705882}%
\pgfsetfillcolor{currentfill}%
\pgfsetfillopacity{0.500000}%
\pgfsetlinewidth{1.003750pt}%
\definecolor{currentstroke}{rgb}{0.000000,0.000000,0.000000}%
\pgfsetstrokecolor{currentstroke}%
\pgfsetdash{}{0pt}%
\pgfpathmoveto{\pgfqpoint{5.282799in}{0.878085in}}%
\pgfpathlineto{\pgfqpoint{5.432702in}{0.878085in}}%
\pgfpathlineto{\pgfqpoint{5.432702in}{0.879057in}}%
\pgfpathlineto{\pgfqpoint{5.282799in}{0.879057in}}%
\pgfpathlineto{\pgfqpoint{5.282799in}{0.878085in}}%
\pgfpathclose%
\pgfusepath{stroke,fill}%
\end{pgfscope}%
\begin{pgfscope}%
\pgfpathrectangle{\pgfqpoint{5.095420in}{0.878085in}}{\pgfqpoint{0.824468in}{0.243158in}}%
\pgfusepath{clip}%
\pgfsetbuttcap%
\pgfsetmiterjoin%
\definecolor{currentfill}{rgb}{0.121569,0.466667,0.705882}%
\pgfsetfillcolor{currentfill}%
\pgfsetfillopacity{0.500000}%
\pgfsetlinewidth{1.003750pt}%
\definecolor{currentstroke}{rgb}{0.000000,0.000000,0.000000}%
\pgfsetstrokecolor{currentstroke}%
\pgfsetdash{}{0pt}%
\pgfpathmoveto{\pgfqpoint{5.432702in}{0.878085in}}%
\pgfpathlineto{\pgfqpoint{5.582605in}{0.878085in}}%
\pgfpathlineto{\pgfqpoint{5.582605in}{0.878085in}}%
\pgfpathlineto{\pgfqpoint{5.432702in}{0.878085in}}%
\pgfpathlineto{\pgfqpoint{5.432702in}{0.878085in}}%
\pgfpathclose%
\pgfusepath{stroke,fill}%
\end{pgfscope}%
\begin{pgfscope}%
\pgfpathrectangle{\pgfqpoint{5.095420in}{0.878085in}}{\pgfqpoint{0.824468in}{0.243158in}}%
\pgfusepath{clip}%
\pgfsetbuttcap%
\pgfsetmiterjoin%
\definecolor{currentfill}{rgb}{0.121569,0.466667,0.705882}%
\pgfsetfillcolor{currentfill}%
\pgfsetfillopacity{0.500000}%
\pgfsetlinewidth{1.003750pt}%
\definecolor{currentstroke}{rgb}{0.000000,0.000000,0.000000}%
\pgfsetstrokecolor{currentstroke}%
\pgfsetdash{}{0pt}%
\pgfpathmoveto{\pgfqpoint{5.582605in}{0.878085in}}%
\pgfpathlineto{\pgfqpoint{5.732509in}{0.878085in}}%
\pgfpathlineto{\pgfqpoint{5.732509in}{0.878206in}}%
\pgfpathlineto{\pgfqpoint{5.582605in}{0.878206in}}%
\pgfpathlineto{\pgfqpoint{5.582605in}{0.878085in}}%
\pgfpathclose%
\pgfusepath{stroke,fill}%
\end{pgfscope}%
\begin{pgfscope}%
\pgfpathrectangle{\pgfqpoint{5.095420in}{0.878085in}}{\pgfqpoint{0.824468in}{0.243158in}}%
\pgfusepath{clip}%
\pgfsetbuttcap%
\pgfsetmiterjoin%
\definecolor{currentfill}{rgb}{0.121569,0.466667,0.705882}%
\pgfsetfillcolor{currentfill}%
\pgfsetfillopacity{0.500000}%
\pgfsetlinewidth{1.003750pt}%
\definecolor{currentstroke}{rgb}{0.000000,0.000000,0.000000}%
\pgfsetstrokecolor{currentstroke}%
\pgfsetdash{}{0pt}%
\pgfpathmoveto{\pgfqpoint{5.732509in}{0.878085in}}%
\pgfpathlineto{\pgfqpoint{5.882412in}{0.878085in}}%
\pgfpathlineto{\pgfqpoint{5.882412in}{0.878571in}}%
\pgfpathlineto{\pgfqpoint{5.732509in}{0.878571in}}%
\pgfpathlineto{\pgfqpoint{5.732509in}{0.878085in}}%
\pgfpathclose%
\pgfusepath{stroke,fill}%
\end{pgfscope}%
\begin{pgfscope}%
\pgfsetrectcap%
\pgfsetmiterjoin%
\pgfsetlinewidth{0.803000pt}%
\definecolor{currentstroke}{rgb}{0.000000,0.000000,0.000000}%
\pgfsetstrokecolor{currentstroke}%
\pgfsetdash{}{0pt}%
\pgfpathmoveto{\pgfqpoint{5.095420in}{0.878085in}}%
\pgfpathlineto{\pgfqpoint{5.095420in}{1.121243in}}%
\pgfusepath{stroke}%
\end{pgfscope}%
\begin{pgfscope}%
\pgfsetrectcap%
\pgfsetmiterjoin%
\pgfsetlinewidth{0.803000pt}%
\definecolor{currentstroke}{rgb}{0.000000,0.000000,0.000000}%
\pgfsetstrokecolor{currentstroke}%
\pgfsetdash{}{0pt}%
\pgfpathmoveto{\pgfqpoint{5.919888in}{0.878085in}}%
\pgfpathlineto{\pgfqpoint{5.919888in}{1.121243in}}%
\pgfusepath{stroke}%
\end{pgfscope}%
\begin{pgfscope}%
\pgfsetrectcap%
\pgfsetmiterjoin%
\pgfsetlinewidth{0.803000pt}%
\definecolor{currentstroke}{rgb}{0.000000,0.000000,0.000000}%
\pgfsetstrokecolor{currentstroke}%
\pgfsetdash{}{0pt}%
\pgfpathmoveto{\pgfqpoint{5.095420in}{0.878085in}}%
\pgfpathlineto{\pgfqpoint{5.919888in}{0.878085in}}%
\pgfusepath{stroke}%
\end{pgfscope}%
\begin{pgfscope}%
\pgfsetrectcap%
\pgfsetmiterjoin%
\pgfsetlinewidth{0.803000pt}%
\definecolor{currentstroke}{rgb}{0.000000,0.000000,0.000000}%
\pgfsetstrokecolor{currentstroke}%
\pgfsetdash{}{0pt}%
\pgfpathmoveto{\pgfqpoint{5.095420in}{1.121243in}}%
\pgfpathlineto{\pgfqpoint{5.919888in}{1.121243in}}%
\pgfusepath{stroke}%
\end{pgfscope}%
\begin{pgfscope}%
\definecolor{textcolor}{rgb}{0.000000,0.000000,0.000000}%
\pgfsetstrokecolor{textcolor}%
\pgfsetfillcolor{textcolor}%
\pgftext[x=5.507654in,y=1.204576in,,base]{\color{textcolor}\rmfamily\fontsize{11.000000}{13.200000}\selectfont Magnolia}%
\end{pgfscope}%
\begin{pgfscope}%
\pgfsetbuttcap%
\pgfsetmiterjoin%
\definecolor{currentfill}{rgb}{1.000000,1.000000,1.000000}%
\pgfsetfillcolor{currentfill}%
\pgfsetlinewidth{0.000000pt}%
\definecolor{currentstroke}{rgb}{0.000000,0.000000,0.000000}%
\pgfsetstrokecolor{currentstroke}%
\pgfsetstrokeopacity{0.000000}%
\pgfsetdash{}{0pt}%
\pgfpathmoveto{\pgfqpoint{6.084781in}{0.878085in}}%
\pgfpathlineto{\pgfqpoint{6.909249in}{0.878085in}}%
\pgfpathlineto{\pgfqpoint{6.909249in}{1.121243in}}%
\pgfpathlineto{\pgfqpoint{6.084781in}{1.121243in}}%
\pgfpathlineto{\pgfqpoint{6.084781in}{0.878085in}}%
\pgfpathclose%
\pgfusepath{fill}%
\end{pgfscope}%
\begin{pgfscope}%
\pgfpathrectangle{\pgfqpoint{6.084781in}{0.878085in}}{\pgfqpoint{0.824468in}{0.243158in}}%
\pgfusepath{clip}%
\pgfsetbuttcap%
\pgfsetmiterjoin%
\definecolor{currentfill}{rgb}{0.121569,0.466667,0.705882}%
\pgfsetfillcolor{currentfill}%
\pgfsetfillopacity{0.500000}%
\pgfsetlinewidth{1.003750pt}%
\definecolor{currentstroke}{rgb}{0.000000,0.000000,0.000000}%
\pgfsetstrokecolor{currentstroke}%
\pgfsetdash{}{0pt}%
\pgfpathmoveto{\pgfqpoint{6.122257in}{0.878085in}}%
\pgfpathlineto{\pgfqpoint{6.272160in}{0.878085in}}%
\pgfpathlineto{\pgfqpoint{6.272160in}{0.879787in}}%
\pgfpathlineto{\pgfqpoint{6.122257in}{0.879787in}}%
\pgfpathlineto{\pgfqpoint{6.122257in}{0.878085in}}%
\pgfpathclose%
\pgfusepath{stroke,fill}%
\end{pgfscope}%
\begin{pgfscope}%
\pgfpathrectangle{\pgfqpoint{6.084781in}{0.878085in}}{\pgfqpoint{0.824468in}{0.243158in}}%
\pgfusepath{clip}%
\pgfsetbuttcap%
\pgfsetmiterjoin%
\definecolor{currentfill}{rgb}{0.121569,0.466667,0.705882}%
\pgfsetfillcolor{currentfill}%
\pgfsetfillopacity{0.500000}%
\pgfsetlinewidth{1.003750pt}%
\definecolor{currentstroke}{rgb}{0.000000,0.000000,0.000000}%
\pgfsetstrokecolor{currentstroke}%
\pgfsetdash{}{0pt}%
\pgfpathmoveto{\pgfqpoint{6.272160in}{0.878085in}}%
\pgfpathlineto{\pgfqpoint{6.422064in}{0.878085in}}%
\pgfpathlineto{\pgfqpoint{6.422064in}{0.879057in}}%
\pgfpathlineto{\pgfqpoint{6.272160in}{0.879057in}}%
\pgfpathlineto{\pgfqpoint{6.272160in}{0.878085in}}%
\pgfpathclose%
\pgfusepath{stroke,fill}%
\end{pgfscope}%
\begin{pgfscope}%
\pgfpathrectangle{\pgfqpoint{6.084781in}{0.878085in}}{\pgfqpoint{0.824468in}{0.243158in}}%
\pgfusepath{clip}%
\pgfsetbuttcap%
\pgfsetmiterjoin%
\definecolor{currentfill}{rgb}{0.121569,0.466667,0.705882}%
\pgfsetfillcolor{currentfill}%
\pgfsetfillopacity{0.500000}%
\pgfsetlinewidth{1.003750pt}%
\definecolor{currentstroke}{rgb}{0.000000,0.000000,0.000000}%
\pgfsetstrokecolor{currentstroke}%
\pgfsetdash{}{0pt}%
\pgfpathmoveto{\pgfqpoint{6.422064in}{0.878085in}}%
\pgfpathlineto{\pgfqpoint{6.571967in}{0.878085in}}%
\pgfpathlineto{\pgfqpoint{6.571967in}{0.878571in}}%
\pgfpathlineto{\pgfqpoint{6.422064in}{0.878571in}}%
\pgfpathlineto{\pgfqpoint{6.422064in}{0.878085in}}%
\pgfpathclose%
\pgfusepath{stroke,fill}%
\end{pgfscope}%
\begin{pgfscope}%
\pgfpathrectangle{\pgfqpoint{6.084781in}{0.878085in}}{\pgfqpoint{0.824468in}{0.243158in}}%
\pgfusepath{clip}%
\pgfsetbuttcap%
\pgfsetmiterjoin%
\definecolor{currentfill}{rgb}{0.121569,0.466667,0.705882}%
\pgfsetfillcolor{currentfill}%
\pgfsetfillopacity{0.500000}%
\pgfsetlinewidth{1.003750pt}%
\definecolor{currentstroke}{rgb}{0.000000,0.000000,0.000000}%
\pgfsetstrokecolor{currentstroke}%
\pgfsetdash{}{0pt}%
\pgfpathmoveto{\pgfqpoint{6.571967in}{0.878085in}}%
\pgfpathlineto{\pgfqpoint{6.721870in}{0.878085in}}%
\pgfpathlineto{\pgfqpoint{6.721870in}{0.878206in}}%
\pgfpathlineto{\pgfqpoint{6.571967in}{0.878206in}}%
\pgfpathlineto{\pgfqpoint{6.571967in}{0.878085in}}%
\pgfpathclose%
\pgfusepath{stroke,fill}%
\end{pgfscope}%
\begin{pgfscope}%
\pgfpathrectangle{\pgfqpoint{6.084781in}{0.878085in}}{\pgfqpoint{0.824468in}{0.243158in}}%
\pgfusepath{clip}%
\pgfsetbuttcap%
\pgfsetmiterjoin%
\definecolor{currentfill}{rgb}{0.121569,0.466667,0.705882}%
\pgfsetfillcolor{currentfill}%
\pgfsetfillopacity{0.500000}%
\pgfsetlinewidth{1.003750pt}%
\definecolor{currentstroke}{rgb}{0.000000,0.000000,0.000000}%
\pgfsetstrokecolor{currentstroke}%
\pgfsetdash{}{0pt}%
\pgfpathmoveto{\pgfqpoint{6.721870in}{0.878085in}}%
\pgfpathlineto{\pgfqpoint{6.871774in}{0.878085in}}%
\pgfpathlineto{\pgfqpoint{6.871774in}{0.878085in}}%
\pgfpathlineto{\pgfqpoint{6.721870in}{0.878085in}}%
\pgfpathlineto{\pgfqpoint{6.721870in}{0.878085in}}%
\pgfpathclose%
\pgfusepath{stroke,fill}%
\end{pgfscope}%
\begin{pgfscope}%
\pgfsetrectcap%
\pgfsetmiterjoin%
\pgfsetlinewidth{0.803000pt}%
\definecolor{currentstroke}{rgb}{0.000000,0.000000,0.000000}%
\pgfsetstrokecolor{currentstroke}%
\pgfsetdash{}{0pt}%
\pgfpathmoveto{\pgfqpoint{6.084781in}{0.878085in}}%
\pgfpathlineto{\pgfqpoint{6.084781in}{1.121243in}}%
\pgfusepath{stroke}%
\end{pgfscope}%
\begin{pgfscope}%
\pgfsetrectcap%
\pgfsetmiterjoin%
\pgfsetlinewidth{0.803000pt}%
\definecolor{currentstroke}{rgb}{0.000000,0.000000,0.000000}%
\pgfsetstrokecolor{currentstroke}%
\pgfsetdash{}{0pt}%
\pgfpathmoveto{\pgfqpoint{6.909249in}{0.878085in}}%
\pgfpathlineto{\pgfqpoint{6.909249in}{1.121243in}}%
\pgfusepath{stroke}%
\end{pgfscope}%
\begin{pgfscope}%
\pgfsetrectcap%
\pgfsetmiterjoin%
\pgfsetlinewidth{0.803000pt}%
\definecolor{currentstroke}{rgb}{0.000000,0.000000,0.000000}%
\pgfsetstrokecolor{currentstroke}%
\pgfsetdash{}{0pt}%
\pgfpathmoveto{\pgfqpoint{6.084781in}{0.878085in}}%
\pgfpathlineto{\pgfqpoint{6.909249in}{0.878085in}}%
\pgfusepath{stroke}%
\end{pgfscope}%
\begin{pgfscope}%
\pgfsetrectcap%
\pgfsetmiterjoin%
\pgfsetlinewidth{0.803000pt}%
\definecolor{currentstroke}{rgb}{0.000000,0.000000,0.000000}%
\pgfsetstrokecolor{currentstroke}%
\pgfsetdash{}{0pt}%
\pgfpathmoveto{\pgfqpoint{6.084781in}{1.121243in}}%
\pgfpathlineto{\pgfqpoint{6.909249in}{1.121243in}}%
\pgfusepath{stroke}%
\end{pgfscope}%
\begin{pgfscope}%
\definecolor{textcolor}{rgb}{0.000000,0.000000,0.000000}%
\pgfsetstrokecolor{textcolor}%
\pgfsetfillcolor{textcolor}%
\pgftext[x=6.497015in,y=1.204576in,,base]{\color{textcolor}\rmfamily\fontsize{11.000000}{13.200000}\selectfont Suravenir}%
\end{pgfscope}%
\begin{pgfscope}%
\pgfsetbuttcap%
\pgfsetmiterjoin%
\definecolor{currentfill}{rgb}{1.000000,1.000000,1.000000}%
\pgfsetfillcolor{currentfill}%
\pgfsetlinewidth{0.000000pt}%
\definecolor{currentstroke}{rgb}{0.000000,0.000000,0.000000}%
\pgfsetstrokecolor{currentstroke}%
\pgfsetstrokeopacity{0.000000}%
\pgfsetdash{}{0pt}%
\pgfpathmoveto{\pgfqpoint{7.074143in}{0.878085in}}%
\pgfpathlineto{\pgfqpoint{7.898611in}{0.878085in}}%
\pgfpathlineto{\pgfqpoint{7.898611in}{1.121243in}}%
\pgfpathlineto{\pgfqpoint{7.074143in}{1.121243in}}%
\pgfpathlineto{\pgfqpoint{7.074143in}{0.878085in}}%
\pgfpathclose%
\pgfusepath{fill}%
\end{pgfscope}%
\begin{pgfscope}%
\pgfpathrectangle{\pgfqpoint{7.074143in}{0.878085in}}{\pgfqpoint{0.824468in}{0.243158in}}%
\pgfusepath{clip}%
\pgfsetbuttcap%
\pgfsetmiterjoin%
\definecolor{currentfill}{rgb}{0.121569,0.466667,0.705882}%
\pgfsetfillcolor{currentfill}%
\pgfsetfillopacity{0.500000}%
\pgfsetlinewidth{1.003750pt}%
\definecolor{currentstroke}{rgb}{0.000000,0.000000,0.000000}%
\pgfsetstrokecolor{currentstroke}%
\pgfsetdash{}{0pt}%
\pgfpathmoveto{\pgfqpoint{7.111619in}{0.878085in}}%
\pgfpathlineto{\pgfqpoint{7.261522in}{0.878085in}}%
\pgfpathlineto{\pgfqpoint{7.261522in}{0.880638in}}%
\pgfpathlineto{\pgfqpoint{7.111619in}{0.880638in}}%
\pgfpathlineto{\pgfqpoint{7.111619in}{0.878085in}}%
\pgfpathclose%
\pgfusepath{stroke,fill}%
\end{pgfscope}%
\begin{pgfscope}%
\pgfpathrectangle{\pgfqpoint{7.074143in}{0.878085in}}{\pgfqpoint{0.824468in}{0.243158in}}%
\pgfusepath{clip}%
\pgfsetbuttcap%
\pgfsetmiterjoin%
\definecolor{currentfill}{rgb}{0.121569,0.466667,0.705882}%
\pgfsetfillcolor{currentfill}%
\pgfsetfillopacity{0.500000}%
\pgfsetlinewidth{1.003750pt}%
\definecolor{currentstroke}{rgb}{0.000000,0.000000,0.000000}%
\pgfsetstrokecolor{currentstroke}%
\pgfsetdash{}{0pt}%
\pgfpathmoveto{\pgfqpoint{7.261522in}{0.878085in}}%
\pgfpathlineto{\pgfqpoint{7.411425in}{0.878085in}}%
\pgfpathlineto{\pgfqpoint{7.411425in}{0.879301in}}%
\pgfpathlineto{\pgfqpoint{7.261522in}{0.879301in}}%
\pgfpathlineto{\pgfqpoint{7.261522in}{0.878085in}}%
\pgfpathclose%
\pgfusepath{stroke,fill}%
\end{pgfscope}%
\begin{pgfscope}%
\pgfpathrectangle{\pgfqpoint{7.074143in}{0.878085in}}{\pgfqpoint{0.824468in}{0.243158in}}%
\pgfusepath{clip}%
\pgfsetbuttcap%
\pgfsetmiterjoin%
\definecolor{currentfill}{rgb}{0.121569,0.466667,0.705882}%
\pgfsetfillcolor{currentfill}%
\pgfsetfillopacity{0.500000}%
\pgfsetlinewidth{1.003750pt}%
\definecolor{currentstroke}{rgb}{0.000000,0.000000,0.000000}%
\pgfsetstrokecolor{currentstroke}%
\pgfsetdash{}{0pt}%
\pgfpathmoveto{\pgfqpoint{7.411425in}{0.878085in}}%
\pgfpathlineto{\pgfqpoint{7.561329in}{0.878085in}}%
\pgfpathlineto{\pgfqpoint{7.561329in}{0.878571in}}%
\pgfpathlineto{\pgfqpoint{7.411425in}{0.878571in}}%
\pgfpathlineto{\pgfqpoint{7.411425in}{0.878085in}}%
\pgfpathclose%
\pgfusepath{stroke,fill}%
\end{pgfscope}%
\begin{pgfscope}%
\pgfpathrectangle{\pgfqpoint{7.074143in}{0.878085in}}{\pgfqpoint{0.824468in}{0.243158in}}%
\pgfusepath{clip}%
\pgfsetbuttcap%
\pgfsetmiterjoin%
\definecolor{currentfill}{rgb}{0.121569,0.466667,0.705882}%
\pgfsetfillcolor{currentfill}%
\pgfsetfillopacity{0.500000}%
\pgfsetlinewidth{1.003750pt}%
\definecolor{currentstroke}{rgb}{0.000000,0.000000,0.000000}%
\pgfsetstrokecolor{currentstroke}%
\pgfsetdash{}{0pt}%
\pgfpathmoveto{\pgfqpoint{7.561329in}{0.878085in}}%
\pgfpathlineto{\pgfqpoint{7.711232in}{0.878085in}}%
\pgfpathlineto{\pgfqpoint{7.711232in}{0.878206in}}%
\pgfpathlineto{\pgfqpoint{7.561329in}{0.878206in}}%
\pgfpathlineto{\pgfqpoint{7.561329in}{0.878085in}}%
\pgfpathclose%
\pgfusepath{stroke,fill}%
\end{pgfscope}%
\begin{pgfscope}%
\pgfpathrectangle{\pgfqpoint{7.074143in}{0.878085in}}{\pgfqpoint{0.824468in}{0.243158in}}%
\pgfusepath{clip}%
\pgfsetbuttcap%
\pgfsetmiterjoin%
\definecolor{currentfill}{rgb}{0.121569,0.466667,0.705882}%
\pgfsetfillcolor{currentfill}%
\pgfsetfillopacity{0.500000}%
\pgfsetlinewidth{1.003750pt}%
\definecolor{currentstroke}{rgb}{0.000000,0.000000,0.000000}%
\pgfsetstrokecolor{currentstroke}%
\pgfsetdash{}{0pt}%
\pgfpathmoveto{\pgfqpoint{7.711232in}{0.878085in}}%
\pgfpathlineto{\pgfqpoint{7.861135in}{0.878085in}}%
\pgfpathlineto{\pgfqpoint{7.861135in}{0.879908in}}%
\pgfpathlineto{\pgfqpoint{7.711232in}{0.879908in}}%
\pgfpathlineto{\pgfqpoint{7.711232in}{0.878085in}}%
\pgfpathclose%
\pgfusepath{stroke,fill}%
\end{pgfscope}%
\begin{pgfscope}%
\pgfsetrectcap%
\pgfsetmiterjoin%
\pgfsetlinewidth{0.803000pt}%
\definecolor{currentstroke}{rgb}{0.000000,0.000000,0.000000}%
\pgfsetstrokecolor{currentstroke}%
\pgfsetdash{}{0pt}%
\pgfpathmoveto{\pgfqpoint{7.074143in}{0.878085in}}%
\pgfpathlineto{\pgfqpoint{7.074143in}{1.121243in}}%
\pgfusepath{stroke}%
\end{pgfscope}%
\begin{pgfscope}%
\pgfsetrectcap%
\pgfsetmiterjoin%
\pgfsetlinewidth{0.803000pt}%
\definecolor{currentstroke}{rgb}{0.000000,0.000000,0.000000}%
\pgfsetstrokecolor{currentstroke}%
\pgfsetdash{}{0pt}%
\pgfpathmoveto{\pgfqpoint{7.898611in}{0.878085in}}%
\pgfpathlineto{\pgfqpoint{7.898611in}{1.121243in}}%
\pgfusepath{stroke}%
\end{pgfscope}%
\begin{pgfscope}%
\pgfsetrectcap%
\pgfsetmiterjoin%
\pgfsetlinewidth{0.803000pt}%
\definecolor{currentstroke}{rgb}{0.000000,0.000000,0.000000}%
\pgfsetstrokecolor{currentstroke}%
\pgfsetdash{}{0pt}%
\pgfpathmoveto{\pgfqpoint{7.074143in}{0.878085in}}%
\pgfpathlineto{\pgfqpoint{7.898611in}{0.878085in}}%
\pgfusepath{stroke}%
\end{pgfscope}%
\begin{pgfscope}%
\pgfsetrectcap%
\pgfsetmiterjoin%
\pgfsetlinewidth{0.803000pt}%
\definecolor{currentstroke}{rgb}{0.000000,0.000000,0.000000}%
\pgfsetstrokecolor{currentstroke}%
\pgfsetdash{}{0pt}%
\pgfpathmoveto{\pgfqpoint{7.074143in}{1.121243in}}%
\pgfpathlineto{\pgfqpoint{7.898611in}{1.121243in}}%
\pgfusepath{stroke}%
\end{pgfscope}%
\begin{pgfscope}%
\definecolor{textcolor}{rgb}{0.000000,0.000000,0.000000}%
\pgfsetstrokecolor{textcolor}%
\pgfsetfillcolor{textcolor}%
\pgftext[x=7.486377in,y=1.204576in,,base]{\color{textcolor}\rmfamily\fontsize{11.000000}{13.200000}\selectfont Assur ...}%
\end{pgfscope}%
\begin{pgfscope}%
\pgfsetbuttcap%
\pgfsetmiterjoin%
\definecolor{currentfill}{rgb}{1.000000,1.000000,1.000000}%
\pgfsetfillcolor{currentfill}%
\pgfsetlinewidth{0.000000pt}%
\definecolor{currentstroke}{rgb}{0.000000,0.000000,0.000000}%
\pgfsetstrokecolor{currentstroke}%
\pgfsetstrokeopacity{0.000000}%
\pgfsetdash{}{0pt}%
\pgfpathmoveto{\pgfqpoint{0.148611in}{0.148611in}}%
\pgfpathlineto{\pgfqpoint{0.973079in}{0.148611in}}%
\pgfpathlineto{\pgfqpoint{0.973079in}{0.391769in}}%
\pgfpathlineto{\pgfqpoint{0.148611in}{0.391769in}}%
\pgfpathlineto{\pgfqpoint{0.148611in}{0.148611in}}%
\pgfpathclose%
\pgfusepath{fill}%
\end{pgfscope}%
\begin{pgfscope}%
\pgfpathrectangle{\pgfqpoint{0.148611in}{0.148611in}}{\pgfqpoint{0.824468in}{0.243158in}}%
\pgfusepath{clip}%
\pgfsetbuttcap%
\pgfsetmiterjoin%
\definecolor{currentfill}{rgb}{0.121569,0.466667,0.705882}%
\pgfsetfillcolor{currentfill}%
\pgfsetfillopacity{0.500000}%
\pgfsetlinewidth{1.003750pt}%
\definecolor{currentstroke}{rgb}{0.000000,0.000000,0.000000}%
\pgfsetstrokecolor{currentstroke}%
\pgfsetdash{}{0pt}%
\pgfpathmoveto{\pgfqpoint{0.186087in}{0.148611in}}%
\pgfpathlineto{\pgfqpoint{0.335990in}{0.148611in}}%
\pgfpathlineto{\pgfqpoint{0.335990in}{0.150435in}}%
\pgfpathlineto{\pgfqpoint{0.186087in}{0.150435in}}%
\pgfpathlineto{\pgfqpoint{0.186087in}{0.148611in}}%
\pgfpathclose%
\pgfusepath{stroke,fill}%
\end{pgfscope}%
\begin{pgfscope}%
\pgfpathrectangle{\pgfqpoint{0.148611in}{0.148611in}}{\pgfqpoint{0.824468in}{0.243158in}}%
\pgfusepath{clip}%
\pgfsetbuttcap%
\pgfsetmiterjoin%
\definecolor{currentfill}{rgb}{0.121569,0.466667,0.705882}%
\pgfsetfillcolor{currentfill}%
\pgfsetfillopacity{0.500000}%
\pgfsetlinewidth{1.003750pt}%
\definecolor{currentstroke}{rgb}{0.000000,0.000000,0.000000}%
\pgfsetstrokecolor{currentstroke}%
\pgfsetdash{}{0pt}%
\pgfpathmoveto{\pgfqpoint{0.335990in}{0.148611in}}%
\pgfpathlineto{\pgfqpoint{0.485894in}{0.148611in}}%
\pgfpathlineto{\pgfqpoint{0.485894in}{0.149705in}}%
\pgfpathlineto{\pgfqpoint{0.335990in}{0.149705in}}%
\pgfpathlineto{\pgfqpoint{0.335990in}{0.148611in}}%
\pgfpathclose%
\pgfusepath{stroke,fill}%
\end{pgfscope}%
\begin{pgfscope}%
\pgfpathrectangle{\pgfqpoint{0.148611in}{0.148611in}}{\pgfqpoint{0.824468in}{0.243158in}}%
\pgfusepath{clip}%
\pgfsetbuttcap%
\pgfsetmiterjoin%
\definecolor{currentfill}{rgb}{0.121569,0.466667,0.705882}%
\pgfsetfillcolor{currentfill}%
\pgfsetfillopacity{0.500000}%
\pgfsetlinewidth{1.003750pt}%
\definecolor{currentstroke}{rgb}{0.000000,0.000000,0.000000}%
\pgfsetstrokecolor{currentstroke}%
\pgfsetdash{}{0pt}%
\pgfpathmoveto{\pgfqpoint{0.485894in}{0.148611in}}%
\pgfpathlineto{\pgfqpoint{0.635797in}{0.148611in}}%
\pgfpathlineto{\pgfqpoint{0.635797in}{0.148733in}}%
\pgfpathlineto{\pgfqpoint{0.485894in}{0.148733in}}%
\pgfpathlineto{\pgfqpoint{0.485894in}{0.148611in}}%
\pgfpathclose%
\pgfusepath{stroke,fill}%
\end{pgfscope}%
\begin{pgfscope}%
\pgfpathrectangle{\pgfqpoint{0.148611in}{0.148611in}}{\pgfqpoint{0.824468in}{0.243158in}}%
\pgfusepath{clip}%
\pgfsetbuttcap%
\pgfsetmiterjoin%
\definecolor{currentfill}{rgb}{0.121569,0.466667,0.705882}%
\pgfsetfillcolor{currentfill}%
\pgfsetfillopacity{0.500000}%
\pgfsetlinewidth{1.003750pt}%
\definecolor{currentstroke}{rgb}{0.000000,0.000000,0.000000}%
\pgfsetstrokecolor{currentstroke}%
\pgfsetdash{}{0pt}%
\pgfpathmoveto{\pgfqpoint{0.635797in}{0.148611in}}%
\pgfpathlineto{\pgfqpoint{0.785700in}{0.148611in}}%
\pgfpathlineto{\pgfqpoint{0.785700in}{0.148611in}}%
\pgfpathlineto{\pgfqpoint{0.635797in}{0.148611in}}%
\pgfpathlineto{\pgfqpoint{0.635797in}{0.148611in}}%
\pgfpathclose%
\pgfusepath{stroke,fill}%
\end{pgfscope}%
\begin{pgfscope}%
\pgfpathrectangle{\pgfqpoint{0.148611in}{0.148611in}}{\pgfqpoint{0.824468in}{0.243158in}}%
\pgfusepath{clip}%
\pgfsetbuttcap%
\pgfsetmiterjoin%
\definecolor{currentfill}{rgb}{0.121569,0.466667,0.705882}%
\pgfsetfillcolor{currentfill}%
\pgfsetfillopacity{0.500000}%
\pgfsetlinewidth{1.003750pt}%
\definecolor{currentstroke}{rgb}{0.000000,0.000000,0.000000}%
\pgfsetstrokecolor{currentstroke}%
\pgfsetdash{}{0pt}%
\pgfpathmoveto{\pgfqpoint{0.785700in}{0.148611in}}%
\pgfpathlineto{\pgfqpoint{0.935603in}{0.148611in}}%
\pgfpathlineto{\pgfqpoint{0.935603in}{0.148611in}}%
\pgfpathlineto{\pgfqpoint{0.785700in}{0.148611in}}%
\pgfpathlineto{\pgfqpoint{0.785700in}{0.148611in}}%
\pgfpathclose%
\pgfusepath{stroke,fill}%
\end{pgfscope}%
\begin{pgfscope}%
\pgfsetrectcap%
\pgfsetmiterjoin%
\pgfsetlinewidth{0.803000pt}%
\definecolor{currentstroke}{rgb}{0.000000,0.000000,0.000000}%
\pgfsetstrokecolor{currentstroke}%
\pgfsetdash{}{0pt}%
\pgfpathmoveto{\pgfqpoint{0.148611in}{0.148611in}}%
\pgfpathlineto{\pgfqpoint{0.148611in}{0.391769in}}%
\pgfusepath{stroke}%
\end{pgfscope}%
\begin{pgfscope}%
\pgfsetrectcap%
\pgfsetmiterjoin%
\pgfsetlinewidth{0.803000pt}%
\definecolor{currentstroke}{rgb}{0.000000,0.000000,0.000000}%
\pgfsetstrokecolor{currentstroke}%
\pgfsetdash{}{0pt}%
\pgfpathmoveto{\pgfqpoint{0.973079in}{0.148611in}}%
\pgfpathlineto{\pgfqpoint{0.973079in}{0.391769in}}%
\pgfusepath{stroke}%
\end{pgfscope}%
\begin{pgfscope}%
\pgfsetrectcap%
\pgfsetmiterjoin%
\pgfsetlinewidth{0.803000pt}%
\definecolor{currentstroke}{rgb}{0.000000,0.000000,0.000000}%
\pgfsetstrokecolor{currentstroke}%
\pgfsetdash{}{0pt}%
\pgfpathmoveto{\pgfqpoint{0.148611in}{0.148611in}}%
\pgfpathlineto{\pgfqpoint{0.973079in}{0.148611in}}%
\pgfusepath{stroke}%
\end{pgfscope}%
\begin{pgfscope}%
\pgfsetrectcap%
\pgfsetmiterjoin%
\pgfsetlinewidth{0.803000pt}%
\definecolor{currentstroke}{rgb}{0.000000,0.000000,0.000000}%
\pgfsetstrokecolor{currentstroke}%
\pgfsetdash{}{0pt}%
\pgfpathmoveto{\pgfqpoint{0.148611in}{0.391769in}}%
\pgfpathlineto{\pgfqpoint{0.973079in}{0.391769in}}%
\pgfusepath{stroke}%
\end{pgfscope}%
\begin{pgfscope}%
\definecolor{textcolor}{rgb}{0.000000,0.000000,0.000000}%
\pgfsetstrokecolor{textcolor}%
\pgfsetfillcolor{textcolor}%
\pgftext[x=0.560845in,y=0.475102in,,base]{\color{textcolor}\rmfamily\fontsize{11.000000}{13.200000}\selectfont AssurO...}%
\end{pgfscope}%
\begin{pgfscope}%
\pgfsetbuttcap%
\pgfsetmiterjoin%
\definecolor{currentfill}{rgb}{1.000000,1.000000,1.000000}%
\pgfsetfillcolor{currentfill}%
\pgfsetlinewidth{0.000000pt}%
\definecolor{currentstroke}{rgb}{0.000000,0.000000,0.000000}%
\pgfsetstrokecolor{currentstroke}%
\pgfsetstrokeopacity{0.000000}%
\pgfsetdash{}{0pt}%
\pgfpathmoveto{\pgfqpoint{1.137973in}{0.148611in}}%
\pgfpathlineto{\pgfqpoint{1.962441in}{0.148611in}}%
\pgfpathlineto{\pgfqpoint{1.962441in}{0.391769in}}%
\pgfpathlineto{\pgfqpoint{1.137973in}{0.391769in}}%
\pgfpathlineto{\pgfqpoint{1.137973in}{0.148611in}}%
\pgfpathclose%
\pgfusepath{fill}%
\end{pgfscope}%
\begin{pgfscope}%
\pgfpathrectangle{\pgfqpoint{1.137973in}{0.148611in}}{\pgfqpoint{0.824468in}{0.243158in}}%
\pgfusepath{clip}%
\pgfsetbuttcap%
\pgfsetmiterjoin%
\definecolor{currentfill}{rgb}{0.121569,0.466667,0.705882}%
\pgfsetfillcolor{currentfill}%
\pgfsetfillopacity{0.500000}%
\pgfsetlinewidth{1.003750pt}%
\definecolor{currentstroke}{rgb}{0.000000,0.000000,0.000000}%
\pgfsetstrokecolor{currentstroke}%
\pgfsetdash{}{0pt}%
\pgfpathmoveto{\pgfqpoint{1.175449in}{0.148611in}}%
\pgfpathlineto{\pgfqpoint{1.325352in}{0.148611in}}%
\pgfpathlineto{\pgfqpoint{1.325352in}{0.150435in}}%
\pgfpathlineto{\pgfqpoint{1.175449in}{0.150435in}}%
\pgfpathlineto{\pgfqpoint{1.175449in}{0.148611in}}%
\pgfpathclose%
\pgfusepath{stroke,fill}%
\end{pgfscope}%
\begin{pgfscope}%
\pgfpathrectangle{\pgfqpoint{1.137973in}{0.148611in}}{\pgfqpoint{0.824468in}{0.243158in}}%
\pgfusepath{clip}%
\pgfsetbuttcap%
\pgfsetmiterjoin%
\definecolor{currentfill}{rgb}{0.121569,0.466667,0.705882}%
\pgfsetfillcolor{currentfill}%
\pgfsetfillopacity{0.500000}%
\pgfsetlinewidth{1.003750pt}%
\definecolor{currentstroke}{rgb}{0.000000,0.000000,0.000000}%
\pgfsetstrokecolor{currentstroke}%
\pgfsetdash{}{0pt}%
\pgfpathmoveto{\pgfqpoint{1.325352in}{0.148611in}}%
\pgfpathlineto{\pgfqpoint{1.475255in}{0.148611in}}%
\pgfpathlineto{\pgfqpoint{1.475255in}{0.149219in}}%
\pgfpathlineto{\pgfqpoint{1.325352in}{0.149219in}}%
\pgfpathlineto{\pgfqpoint{1.325352in}{0.148611in}}%
\pgfpathclose%
\pgfusepath{stroke,fill}%
\end{pgfscope}%
\begin{pgfscope}%
\pgfpathrectangle{\pgfqpoint{1.137973in}{0.148611in}}{\pgfqpoint{0.824468in}{0.243158in}}%
\pgfusepath{clip}%
\pgfsetbuttcap%
\pgfsetmiterjoin%
\definecolor{currentfill}{rgb}{0.121569,0.466667,0.705882}%
\pgfsetfillcolor{currentfill}%
\pgfsetfillopacity{0.500000}%
\pgfsetlinewidth{1.003750pt}%
\definecolor{currentstroke}{rgb}{0.000000,0.000000,0.000000}%
\pgfsetstrokecolor{currentstroke}%
\pgfsetdash{}{0pt}%
\pgfpathmoveto{\pgfqpoint{1.475255in}{0.148611in}}%
\pgfpathlineto{\pgfqpoint{1.625158in}{0.148611in}}%
\pgfpathlineto{\pgfqpoint{1.625158in}{0.148733in}}%
\pgfpathlineto{\pgfqpoint{1.475255in}{0.148733in}}%
\pgfpathlineto{\pgfqpoint{1.475255in}{0.148611in}}%
\pgfpathclose%
\pgfusepath{stroke,fill}%
\end{pgfscope}%
\begin{pgfscope}%
\pgfpathrectangle{\pgfqpoint{1.137973in}{0.148611in}}{\pgfqpoint{0.824468in}{0.243158in}}%
\pgfusepath{clip}%
\pgfsetbuttcap%
\pgfsetmiterjoin%
\definecolor{currentfill}{rgb}{0.121569,0.466667,0.705882}%
\pgfsetfillcolor{currentfill}%
\pgfsetfillopacity{0.500000}%
\pgfsetlinewidth{1.003750pt}%
\definecolor{currentstroke}{rgb}{0.000000,0.000000,0.000000}%
\pgfsetstrokecolor{currentstroke}%
\pgfsetdash{}{0pt}%
\pgfpathmoveto{\pgfqpoint{1.625158in}{0.148611in}}%
\pgfpathlineto{\pgfqpoint{1.775062in}{0.148611in}}%
\pgfpathlineto{\pgfqpoint{1.775062in}{0.149341in}}%
\pgfpathlineto{\pgfqpoint{1.625158in}{0.149341in}}%
\pgfpathlineto{\pgfqpoint{1.625158in}{0.148611in}}%
\pgfpathclose%
\pgfusepath{stroke,fill}%
\end{pgfscope}%
\begin{pgfscope}%
\pgfpathrectangle{\pgfqpoint{1.137973in}{0.148611in}}{\pgfqpoint{0.824468in}{0.243158in}}%
\pgfusepath{clip}%
\pgfsetbuttcap%
\pgfsetmiterjoin%
\definecolor{currentfill}{rgb}{0.121569,0.466667,0.705882}%
\pgfsetfillcolor{currentfill}%
\pgfsetfillopacity{0.500000}%
\pgfsetlinewidth{1.003750pt}%
\definecolor{currentstroke}{rgb}{0.000000,0.000000,0.000000}%
\pgfsetstrokecolor{currentstroke}%
\pgfsetdash{}{0pt}%
\pgfpathmoveto{\pgfqpoint{1.775062in}{0.148611in}}%
\pgfpathlineto{\pgfqpoint{1.924965in}{0.148611in}}%
\pgfpathlineto{\pgfqpoint{1.924965in}{0.149219in}}%
\pgfpathlineto{\pgfqpoint{1.775062in}{0.149219in}}%
\pgfpathlineto{\pgfqpoint{1.775062in}{0.148611in}}%
\pgfpathclose%
\pgfusepath{stroke,fill}%
\end{pgfscope}%
\begin{pgfscope}%
\pgfsetrectcap%
\pgfsetmiterjoin%
\pgfsetlinewidth{0.803000pt}%
\definecolor{currentstroke}{rgb}{0.000000,0.000000,0.000000}%
\pgfsetstrokecolor{currentstroke}%
\pgfsetdash{}{0pt}%
\pgfpathmoveto{\pgfqpoint{1.137973in}{0.148611in}}%
\pgfpathlineto{\pgfqpoint{1.137973in}{0.391769in}}%
\pgfusepath{stroke}%
\end{pgfscope}%
\begin{pgfscope}%
\pgfsetrectcap%
\pgfsetmiterjoin%
\pgfsetlinewidth{0.803000pt}%
\definecolor{currentstroke}{rgb}{0.000000,0.000000,0.000000}%
\pgfsetstrokecolor{currentstroke}%
\pgfsetdash{}{0pt}%
\pgfpathmoveto{\pgfqpoint{1.962441in}{0.148611in}}%
\pgfpathlineto{\pgfqpoint{1.962441in}{0.391769in}}%
\pgfusepath{stroke}%
\end{pgfscope}%
\begin{pgfscope}%
\pgfsetrectcap%
\pgfsetmiterjoin%
\pgfsetlinewidth{0.803000pt}%
\definecolor{currentstroke}{rgb}{0.000000,0.000000,0.000000}%
\pgfsetstrokecolor{currentstroke}%
\pgfsetdash{}{0pt}%
\pgfpathmoveto{\pgfqpoint{1.137973in}{0.148611in}}%
\pgfpathlineto{\pgfqpoint{1.962441in}{0.148611in}}%
\pgfusepath{stroke}%
\end{pgfscope}%
\begin{pgfscope}%
\pgfsetrectcap%
\pgfsetmiterjoin%
\pgfsetlinewidth{0.803000pt}%
\definecolor{currentstroke}{rgb}{0.000000,0.000000,0.000000}%
\pgfsetstrokecolor{currentstroke}%
\pgfsetdash{}{0pt}%
\pgfpathmoveto{\pgfqpoint{1.137973in}{0.391769in}}%
\pgfpathlineto{\pgfqpoint{1.962441in}{0.391769in}}%
\pgfusepath{stroke}%
\end{pgfscope}%
\begin{pgfscope}%
\definecolor{textcolor}{rgb}{0.000000,0.000000,0.000000}%
\pgfsetstrokecolor{textcolor}%
\pgfsetfillcolor{textcolor}%
\pgftext[x=1.550207in,y=0.475102in,,base]{\color{textcolor}\rmfamily\fontsize{11.000000}{13.200000}\selectfont Carac}%
\end{pgfscope}%
\begin{pgfscope}%
\pgfsetbuttcap%
\pgfsetmiterjoin%
\definecolor{currentfill}{rgb}{1.000000,1.000000,1.000000}%
\pgfsetfillcolor{currentfill}%
\pgfsetlinewidth{0.000000pt}%
\definecolor{currentstroke}{rgb}{0.000000,0.000000,0.000000}%
\pgfsetstrokecolor{currentstroke}%
\pgfsetstrokeopacity{0.000000}%
\pgfsetdash{}{0pt}%
\pgfpathmoveto{\pgfqpoint{2.127335in}{0.148611in}}%
\pgfpathlineto{\pgfqpoint{2.951803in}{0.148611in}}%
\pgfpathlineto{\pgfqpoint{2.951803in}{0.391769in}}%
\pgfpathlineto{\pgfqpoint{2.127335in}{0.391769in}}%
\pgfpathlineto{\pgfqpoint{2.127335in}{0.148611in}}%
\pgfpathclose%
\pgfusepath{fill}%
\end{pgfscope}%
\begin{pgfscope}%
\pgfpathrectangle{\pgfqpoint{2.127335in}{0.148611in}}{\pgfqpoint{0.824468in}{0.243158in}}%
\pgfusepath{clip}%
\pgfsetbuttcap%
\pgfsetmiterjoin%
\definecolor{currentfill}{rgb}{0.121569,0.466667,0.705882}%
\pgfsetfillcolor{currentfill}%
\pgfsetfillopacity{0.500000}%
\pgfsetlinewidth{1.003750pt}%
\definecolor{currentstroke}{rgb}{0.000000,0.000000,0.000000}%
\pgfsetstrokecolor{currentstroke}%
\pgfsetdash{}{0pt}%
\pgfpathmoveto{\pgfqpoint{2.164810in}{0.148611in}}%
\pgfpathlineto{\pgfqpoint{2.314714in}{0.148611in}}%
\pgfpathlineto{\pgfqpoint{2.314714in}{0.148733in}}%
\pgfpathlineto{\pgfqpoint{2.164810in}{0.148733in}}%
\pgfpathlineto{\pgfqpoint{2.164810in}{0.148611in}}%
\pgfpathclose%
\pgfusepath{stroke,fill}%
\end{pgfscope}%
\begin{pgfscope}%
\pgfpathrectangle{\pgfqpoint{2.127335in}{0.148611in}}{\pgfqpoint{0.824468in}{0.243158in}}%
\pgfusepath{clip}%
\pgfsetbuttcap%
\pgfsetmiterjoin%
\definecolor{currentfill}{rgb}{0.121569,0.466667,0.705882}%
\pgfsetfillcolor{currentfill}%
\pgfsetfillopacity{0.500000}%
\pgfsetlinewidth{1.003750pt}%
\definecolor{currentstroke}{rgb}{0.000000,0.000000,0.000000}%
\pgfsetstrokecolor{currentstroke}%
\pgfsetdash{}{0pt}%
\pgfpathmoveto{\pgfqpoint{2.314714in}{0.148611in}}%
\pgfpathlineto{\pgfqpoint{2.464617in}{0.148611in}}%
\pgfpathlineto{\pgfqpoint{2.464617in}{0.148611in}}%
\pgfpathlineto{\pgfqpoint{2.314714in}{0.148611in}}%
\pgfpathlineto{\pgfqpoint{2.314714in}{0.148611in}}%
\pgfpathclose%
\pgfusepath{stroke,fill}%
\end{pgfscope}%
\begin{pgfscope}%
\pgfpathrectangle{\pgfqpoint{2.127335in}{0.148611in}}{\pgfqpoint{0.824468in}{0.243158in}}%
\pgfusepath{clip}%
\pgfsetbuttcap%
\pgfsetmiterjoin%
\definecolor{currentfill}{rgb}{0.121569,0.466667,0.705882}%
\pgfsetfillcolor{currentfill}%
\pgfsetfillopacity{0.500000}%
\pgfsetlinewidth{1.003750pt}%
\definecolor{currentstroke}{rgb}{0.000000,0.000000,0.000000}%
\pgfsetstrokecolor{currentstroke}%
\pgfsetdash{}{0pt}%
\pgfpathmoveto{\pgfqpoint{2.464617in}{0.148611in}}%
\pgfpathlineto{\pgfqpoint{2.614520in}{0.148611in}}%
\pgfpathlineto{\pgfqpoint{2.614520in}{0.148611in}}%
\pgfpathlineto{\pgfqpoint{2.464617in}{0.148611in}}%
\pgfpathlineto{\pgfqpoint{2.464617in}{0.148611in}}%
\pgfpathclose%
\pgfusepath{stroke,fill}%
\end{pgfscope}%
\begin{pgfscope}%
\pgfpathrectangle{\pgfqpoint{2.127335in}{0.148611in}}{\pgfqpoint{0.824468in}{0.243158in}}%
\pgfusepath{clip}%
\pgfsetbuttcap%
\pgfsetmiterjoin%
\definecolor{currentfill}{rgb}{0.121569,0.466667,0.705882}%
\pgfsetfillcolor{currentfill}%
\pgfsetfillopacity{0.500000}%
\pgfsetlinewidth{1.003750pt}%
\definecolor{currentstroke}{rgb}{0.000000,0.000000,0.000000}%
\pgfsetstrokecolor{currentstroke}%
\pgfsetdash{}{0pt}%
\pgfpathmoveto{\pgfqpoint{2.614520in}{0.148611in}}%
\pgfpathlineto{\pgfqpoint{2.764423in}{0.148611in}}%
\pgfpathlineto{\pgfqpoint{2.764423in}{0.149462in}}%
\pgfpathlineto{\pgfqpoint{2.614520in}{0.149462in}}%
\pgfpathlineto{\pgfqpoint{2.614520in}{0.148611in}}%
\pgfpathclose%
\pgfusepath{stroke,fill}%
\end{pgfscope}%
\begin{pgfscope}%
\pgfpathrectangle{\pgfqpoint{2.127335in}{0.148611in}}{\pgfqpoint{0.824468in}{0.243158in}}%
\pgfusepath{clip}%
\pgfsetbuttcap%
\pgfsetmiterjoin%
\definecolor{currentfill}{rgb}{0.121569,0.466667,0.705882}%
\pgfsetfillcolor{currentfill}%
\pgfsetfillopacity{0.500000}%
\pgfsetlinewidth{1.003750pt}%
\definecolor{currentstroke}{rgb}{0.000000,0.000000,0.000000}%
\pgfsetstrokecolor{currentstroke}%
\pgfsetdash{}{0pt}%
\pgfpathmoveto{\pgfqpoint{2.764423in}{0.148611in}}%
\pgfpathlineto{\pgfqpoint{2.914327in}{0.148611in}}%
\pgfpathlineto{\pgfqpoint{2.914327in}{0.148854in}}%
\pgfpathlineto{\pgfqpoint{2.764423in}{0.148854in}}%
\pgfpathlineto{\pgfqpoint{2.764423in}{0.148611in}}%
\pgfpathclose%
\pgfusepath{stroke,fill}%
\end{pgfscope}%
\begin{pgfscope}%
\pgfsetrectcap%
\pgfsetmiterjoin%
\pgfsetlinewidth{0.803000pt}%
\definecolor{currentstroke}{rgb}{0.000000,0.000000,0.000000}%
\pgfsetstrokecolor{currentstroke}%
\pgfsetdash{}{0pt}%
\pgfpathmoveto{\pgfqpoint{2.127335in}{0.148611in}}%
\pgfpathlineto{\pgfqpoint{2.127335in}{0.391769in}}%
\pgfusepath{stroke}%
\end{pgfscope}%
\begin{pgfscope}%
\pgfsetrectcap%
\pgfsetmiterjoin%
\pgfsetlinewidth{0.803000pt}%
\definecolor{currentstroke}{rgb}{0.000000,0.000000,0.000000}%
\pgfsetstrokecolor{currentstroke}%
\pgfsetdash{}{0pt}%
\pgfpathmoveto{\pgfqpoint{2.951803in}{0.148611in}}%
\pgfpathlineto{\pgfqpoint{2.951803in}{0.391769in}}%
\pgfusepath{stroke}%
\end{pgfscope}%
\begin{pgfscope}%
\pgfsetrectcap%
\pgfsetmiterjoin%
\pgfsetlinewidth{0.803000pt}%
\definecolor{currentstroke}{rgb}{0.000000,0.000000,0.000000}%
\pgfsetstrokecolor{currentstroke}%
\pgfsetdash{}{0pt}%
\pgfpathmoveto{\pgfqpoint{2.127335in}{0.148611in}}%
\pgfpathlineto{\pgfqpoint{2.951803in}{0.148611in}}%
\pgfusepath{stroke}%
\end{pgfscope}%
\begin{pgfscope}%
\pgfsetrectcap%
\pgfsetmiterjoin%
\pgfsetlinewidth{0.803000pt}%
\definecolor{currentstroke}{rgb}{0.000000,0.000000,0.000000}%
\pgfsetstrokecolor{currentstroke}%
\pgfsetdash{}{0pt}%
\pgfpathmoveto{\pgfqpoint{2.127335in}{0.391769in}}%
\pgfpathlineto{\pgfqpoint{2.951803in}{0.391769in}}%
\pgfusepath{stroke}%
\end{pgfscope}%
\begin{pgfscope}%
\definecolor{textcolor}{rgb}{0.000000,0.000000,0.000000}%
\pgfsetstrokecolor{textcolor}%
\pgfsetfillcolor{textcolor}%
\pgftext[x=2.539569in,y=0.475102in,,base]{\color{textcolor}\rmfamily\fontsize{11.000000}{13.200000}\selectfont Mapa}%
\end{pgfscope}%
\begin{pgfscope}%
\pgfsetbuttcap%
\pgfsetmiterjoin%
\definecolor{currentfill}{rgb}{1.000000,1.000000,1.000000}%
\pgfsetfillcolor{currentfill}%
\pgfsetlinewidth{0.000000pt}%
\definecolor{currentstroke}{rgb}{0.000000,0.000000,0.000000}%
\pgfsetstrokecolor{currentstroke}%
\pgfsetstrokeopacity{0.000000}%
\pgfsetdash{}{0pt}%
\pgfpathmoveto{\pgfqpoint{3.116696in}{0.148611in}}%
\pgfpathlineto{\pgfqpoint{3.941164in}{0.148611in}}%
\pgfpathlineto{\pgfqpoint{3.941164in}{0.391769in}}%
\pgfpathlineto{\pgfqpoint{3.116696in}{0.391769in}}%
\pgfpathlineto{\pgfqpoint{3.116696in}{0.148611in}}%
\pgfpathclose%
\pgfusepath{fill}%
\end{pgfscope}%
\begin{pgfscope}%
\pgfpathrectangle{\pgfqpoint{3.116696in}{0.148611in}}{\pgfqpoint{0.824468in}{0.243158in}}%
\pgfusepath{clip}%
\pgfsetbuttcap%
\pgfsetmiterjoin%
\definecolor{currentfill}{rgb}{0.121569,0.466667,0.705882}%
\pgfsetfillcolor{currentfill}%
\pgfsetfillopacity{0.500000}%
\pgfsetlinewidth{1.003750pt}%
\definecolor{currentstroke}{rgb}{0.000000,0.000000,0.000000}%
\pgfsetstrokecolor{currentstroke}%
\pgfsetdash{}{0pt}%
\pgfpathmoveto{\pgfqpoint{3.154172in}{0.148611in}}%
\pgfpathlineto{\pgfqpoint{3.304075in}{0.148611in}}%
\pgfpathlineto{\pgfqpoint{3.304075in}{0.152015in}}%
\pgfpathlineto{\pgfqpoint{3.154172in}{0.152015in}}%
\pgfpathlineto{\pgfqpoint{3.154172in}{0.148611in}}%
\pgfpathclose%
\pgfusepath{stroke,fill}%
\end{pgfscope}%
\begin{pgfscope}%
\pgfpathrectangle{\pgfqpoint{3.116696in}{0.148611in}}{\pgfqpoint{0.824468in}{0.243158in}}%
\pgfusepath{clip}%
\pgfsetbuttcap%
\pgfsetmiterjoin%
\definecolor{currentfill}{rgb}{0.121569,0.466667,0.705882}%
\pgfsetfillcolor{currentfill}%
\pgfsetfillopacity{0.500000}%
\pgfsetlinewidth{1.003750pt}%
\definecolor{currentstroke}{rgb}{0.000000,0.000000,0.000000}%
\pgfsetstrokecolor{currentstroke}%
\pgfsetdash{}{0pt}%
\pgfpathmoveto{\pgfqpoint{3.304075in}{0.148611in}}%
\pgfpathlineto{\pgfqpoint{3.453979in}{0.148611in}}%
\pgfpathlineto{\pgfqpoint{3.453979in}{0.149341in}}%
\pgfpathlineto{\pgfqpoint{3.304075in}{0.149341in}}%
\pgfpathlineto{\pgfqpoint{3.304075in}{0.148611in}}%
\pgfpathclose%
\pgfusepath{stroke,fill}%
\end{pgfscope}%
\begin{pgfscope}%
\pgfpathrectangle{\pgfqpoint{3.116696in}{0.148611in}}{\pgfqpoint{0.824468in}{0.243158in}}%
\pgfusepath{clip}%
\pgfsetbuttcap%
\pgfsetmiterjoin%
\definecolor{currentfill}{rgb}{0.121569,0.466667,0.705882}%
\pgfsetfillcolor{currentfill}%
\pgfsetfillopacity{0.500000}%
\pgfsetlinewidth{1.003750pt}%
\definecolor{currentstroke}{rgb}{0.000000,0.000000,0.000000}%
\pgfsetstrokecolor{currentstroke}%
\pgfsetdash{}{0pt}%
\pgfpathmoveto{\pgfqpoint{3.453979in}{0.148611in}}%
\pgfpathlineto{\pgfqpoint{3.603882in}{0.148611in}}%
\pgfpathlineto{\pgfqpoint{3.603882in}{0.149097in}}%
\pgfpathlineto{\pgfqpoint{3.453979in}{0.149097in}}%
\pgfpathlineto{\pgfqpoint{3.453979in}{0.148611in}}%
\pgfpathclose%
\pgfusepath{stroke,fill}%
\end{pgfscope}%
\begin{pgfscope}%
\pgfpathrectangle{\pgfqpoint{3.116696in}{0.148611in}}{\pgfqpoint{0.824468in}{0.243158in}}%
\pgfusepath{clip}%
\pgfsetbuttcap%
\pgfsetmiterjoin%
\definecolor{currentfill}{rgb}{0.121569,0.466667,0.705882}%
\pgfsetfillcolor{currentfill}%
\pgfsetfillopacity{0.500000}%
\pgfsetlinewidth{1.003750pt}%
\definecolor{currentstroke}{rgb}{0.000000,0.000000,0.000000}%
\pgfsetstrokecolor{currentstroke}%
\pgfsetdash{}{0pt}%
\pgfpathmoveto{\pgfqpoint{3.603882in}{0.148611in}}%
\pgfpathlineto{\pgfqpoint{3.753785in}{0.148611in}}%
\pgfpathlineto{\pgfqpoint{3.753785in}{0.148611in}}%
\pgfpathlineto{\pgfqpoint{3.603882in}{0.148611in}}%
\pgfpathlineto{\pgfqpoint{3.603882in}{0.148611in}}%
\pgfpathclose%
\pgfusepath{stroke,fill}%
\end{pgfscope}%
\begin{pgfscope}%
\pgfpathrectangle{\pgfqpoint{3.116696in}{0.148611in}}{\pgfqpoint{0.824468in}{0.243158in}}%
\pgfusepath{clip}%
\pgfsetbuttcap%
\pgfsetmiterjoin%
\definecolor{currentfill}{rgb}{0.121569,0.466667,0.705882}%
\pgfsetfillcolor{currentfill}%
\pgfsetfillopacity{0.500000}%
\pgfsetlinewidth{1.003750pt}%
\definecolor{currentstroke}{rgb}{0.000000,0.000000,0.000000}%
\pgfsetstrokecolor{currentstroke}%
\pgfsetdash{}{0pt}%
\pgfpathmoveto{\pgfqpoint{3.753785in}{0.148611in}}%
\pgfpathlineto{\pgfqpoint{3.903688in}{0.148611in}}%
\pgfpathlineto{\pgfqpoint{3.903688in}{0.148611in}}%
\pgfpathlineto{\pgfqpoint{3.753785in}{0.148611in}}%
\pgfpathlineto{\pgfqpoint{3.753785in}{0.148611in}}%
\pgfpathclose%
\pgfusepath{stroke,fill}%
\end{pgfscope}%
\begin{pgfscope}%
\pgfsetrectcap%
\pgfsetmiterjoin%
\pgfsetlinewidth{0.803000pt}%
\definecolor{currentstroke}{rgb}{0.000000,0.000000,0.000000}%
\pgfsetstrokecolor{currentstroke}%
\pgfsetdash{}{0pt}%
\pgfpathmoveto{\pgfqpoint{3.116696in}{0.148611in}}%
\pgfpathlineto{\pgfqpoint{3.116696in}{0.391769in}}%
\pgfusepath{stroke}%
\end{pgfscope}%
\begin{pgfscope}%
\pgfsetrectcap%
\pgfsetmiterjoin%
\pgfsetlinewidth{0.803000pt}%
\definecolor{currentstroke}{rgb}{0.000000,0.000000,0.000000}%
\pgfsetstrokecolor{currentstroke}%
\pgfsetdash{}{0pt}%
\pgfpathmoveto{\pgfqpoint{3.941164in}{0.148611in}}%
\pgfpathlineto{\pgfqpoint{3.941164in}{0.391769in}}%
\pgfusepath{stroke}%
\end{pgfscope}%
\begin{pgfscope}%
\pgfsetrectcap%
\pgfsetmiterjoin%
\pgfsetlinewidth{0.803000pt}%
\definecolor{currentstroke}{rgb}{0.000000,0.000000,0.000000}%
\pgfsetstrokecolor{currentstroke}%
\pgfsetdash{}{0pt}%
\pgfpathmoveto{\pgfqpoint{3.116696in}{0.148611in}}%
\pgfpathlineto{\pgfqpoint{3.941164in}{0.148611in}}%
\pgfusepath{stroke}%
\end{pgfscope}%
\begin{pgfscope}%
\pgfsetrectcap%
\pgfsetmiterjoin%
\pgfsetlinewidth{0.803000pt}%
\definecolor{currentstroke}{rgb}{0.000000,0.000000,0.000000}%
\pgfsetstrokecolor{currentstroke}%
\pgfsetdash{}{0pt}%
\pgfpathmoveto{\pgfqpoint{3.116696in}{0.391769in}}%
\pgfpathlineto{\pgfqpoint{3.941164in}{0.391769in}}%
\pgfusepath{stroke}%
\end{pgfscope}%
\begin{pgfscope}%
\definecolor{textcolor}{rgb}{0.000000,0.000000,0.000000}%
\pgfsetstrokecolor{textcolor}%
\pgfsetfillcolor{textcolor}%
\pgftext[x=3.528930in,y=0.475102in,,base]{\color{textcolor}\rmfamily\fontsize{11.000000}{13.200000}\selectfont Malako...}%
\end{pgfscope}%
\begin{pgfscope}%
\pgfsetbuttcap%
\pgfsetmiterjoin%
\definecolor{currentfill}{rgb}{1.000000,1.000000,1.000000}%
\pgfsetfillcolor{currentfill}%
\pgfsetlinewidth{0.000000pt}%
\definecolor{currentstroke}{rgb}{0.000000,0.000000,0.000000}%
\pgfsetstrokecolor{currentstroke}%
\pgfsetstrokeopacity{0.000000}%
\pgfsetdash{}{0pt}%
\pgfpathmoveto{\pgfqpoint{4.106058in}{0.148611in}}%
\pgfpathlineto{\pgfqpoint{4.930526in}{0.148611in}}%
\pgfpathlineto{\pgfqpoint{4.930526in}{0.391769in}}%
\pgfpathlineto{\pgfqpoint{4.106058in}{0.391769in}}%
\pgfpathlineto{\pgfqpoint{4.106058in}{0.148611in}}%
\pgfpathclose%
\pgfusepath{fill}%
\end{pgfscope}%
\begin{pgfscope}%
\pgfpathrectangle{\pgfqpoint{4.106058in}{0.148611in}}{\pgfqpoint{0.824468in}{0.243158in}}%
\pgfusepath{clip}%
\pgfsetbuttcap%
\pgfsetmiterjoin%
\definecolor{currentfill}{rgb}{0.121569,0.466667,0.705882}%
\pgfsetfillcolor{currentfill}%
\pgfsetfillopacity{0.500000}%
\pgfsetlinewidth{1.003750pt}%
\definecolor{currentstroke}{rgb}{0.000000,0.000000,0.000000}%
\pgfsetstrokecolor{currentstroke}%
\pgfsetdash{}{0pt}%
\pgfpathmoveto{\pgfqpoint{4.143534in}{0.148611in}}%
\pgfpathlineto{\pgfqpoint{4.293437in}{0.148611in}}%
\pgfpathlineto{\pgfqpoint{4.293437in}{0.151286in}}%
\pgfpathlineto{\pgfqpoint{4.143534in}{0.151286in}}%
\pgfpathlineto{\pgfqpoint{4.143534in}{0.148611in}}%
\pgfpathclose%
\pgfusepath{stroke,fill}%
\end{pgfscope}%
\begin{pgfscope}%
\pgfpathrectangle{\pgfqpoint{4.106058in}{0.148611in}}{\pgfqpoint{0.824468in}{0.243158in}}%
\pgfusepath{clip}%
\pgfsetbuttcap%
\pgfsetmiterjoin%
\definecolor{currentfill}{rgb}{0.121569,0.466667,0.705882}%
\pgfsetfillcolor{currentfill}%
\pgfsetfillopacity{0.500000}%
\pgfsetlinewidth{1.003750pt}%
\definecolor{currentstroke}{rgb}{0.000000,0.000000,0.000000}%
\pgfsetstrokecolor{currentstroke}%
\pgfsetdash{}{0pt}%
\pgfpathmoveto{\pgfqpoint{4.293437in}{0.148611in}}%
\pgfpathlineto{\pgfqpoint{4.443340in}{0.148611in}}%
\pgfpathlineto{\pgfqpoint{4.443340in}{0.150556in}}%
\pgfpathlineto{\pgfqpoint{4.293437in}{0.150556in}}%
\pgfpathlineto{\pgfqpoint{4.293437in}{0.148611in}}%
\pgfpathclose%
\pgfusepath{stroke,fill}%
\end{pgfscope}%
\begin{pgfscope}%
\pgfpathrectangle{\pgfqpoint{4.106058in}{0.148611in}}{\pgfqpoint{0.824468in}{0.243158in}}%
\pgfusepath{clip}%
\pgfsetbuttcap%
\pgfsetmiterjoin%
\definecolor{currentfill}{rgb}{0.121569,0.466667,0.705882}%
\pgfsetfillcolor{currentfill}%
\pgfsetfillopacity{0.500000}%
\pgfsetlinewidth{1.003750pt}%
\definecolor{currentstroke}{rgb}{0.000000,0.000000,0.000000}%
\pgfsetstrokecolor{currentstroke}%
\pgfsetdash{}{0pt}%
\pgfpathmoveto{\pgfqpoint{4.443340in}{0.148611in}}%
\pgfpathlineto{\pgfqpoint{4.593244in}{0.148611in}}%
\pgfpathlineto{\pgfqpoint{4.593244in}{0.149097in}}%
\pgfpathlineto{\pgfqpoint{4.443340in}{0.149097in}}%
\pgfpathlineto{\pgfqpoint{4.443340in}{0.148611in}}%
\pgfpathclose%
\pgfusepath{stroke,fill}%
\end{pgfscope}%
\begin{pgfscope}%
\pgfpathrectangle{\pgfqpoint{4.106058in}{0.148611in}}{\pgfqpoint{0.824468in}{0.243158in}}%
\pgfusepath{clip}%
\pgfsetbuttcap%
\pgfsetmiterjoin%
\definecolor{currentfill}{rgb}{0.121569,0.466667,0.705882}%
\pgfsetfillcolor{currentfill}%
\pgfsetfillopacity{0.500000}%
\pgfsetlinewidth{1.003750pt}%
\definecolor{currentstroke}{rgb}{0.000000,0.000000,0.000000}%
\pgfsetstrokecolor{currentstroke}%
\pgfsetdash{}{0pt}%
\pgfpathmoveto{\pgfqpoint{4.593244in}{0.148611in}}%
\pgfpathlineto{\pgfqpoint{4.743147in}{0.148611in}}%
\pgfpathlineto{\pgfqpoint{4.743147in}{0.148733in}}%
\pgfpathlineto{\pgfqpoint{4.593244in}{0.148733in}}%
\pgfpathlineto{\pgfqpoint{4.593244in}{0.148611in}}%
\pgfpathclose%
\pgfusepath{stroke,fill}%
\end{pgfscope}%
\begin{pgfscope}%
\pgfpathrectangle{\pgfqpoint{4.106058in}{0.148611in}}{\pgfqpoint{0.824468in}{0.243158in}}%
\pgfusepath{clip}%
\pgfsetbuttcap%
\pgfsetmiterjoin%
\definecolor{currentfill}{rgb}{0.121569,0.466667,0.705882}%
\pgfsetfillcolor{currentfill}%
\pgfsetfillopacity{0.500000}%
\pgfsetlinewidth{1.003750pt}%
\definecolor{currentstroke}{rgb}{0.000000,0.000000,0.000000}%
\pgfsetstrokecolor{currentstroke}%
\pgfsetdash{}{0pt}%
\pgfpathmoveto{\pgfqpoint{4.743147in}{0.148611in}}%
\pgfpathlineto{\pgfqpoint{4.893050in}{0.148611in}}%
\pgfpathlineto{\pgfqpoint{4.893050in}{0.148611in}}%
\pgfpathlineto{\pgfqpoint{4.743147in}{0.148611in}}%
\pgfpathlineto{\pgfqpoint{4.743147in}{0.148611in}}%
\pgfpathclose%
\pgfusepath{stroke,fill}%
\end{pgfscope}%
\begin{pgfscope}%
\pgfsetrectcap%
\pgfsetmiterjoin%
\pgfsetlinewidth{0.803000pt}%
\definecolor{currentstroke}{rgb}{0.000000,0.000000,0.000000}%
\pgfsetstrokecolor{currentstroke}%
\pgfsetdash{}{0pt}%
\pgfpathmoveto{\pgfqpoint{4.106058in}{0.148611in}}%
\pgfpathlineto{\pgfqpoint{4.106058in}{0.391769in}}%
\pgfusepath{stroke}%
\end{pgfscope}%
\begin{pgfscope}%
\pgfsetrectcap%
\pgfsetmiterjoin%
\pgfsetlinewidth{0.803000pt}%
\definecolor{currentstroke}{rgb}{0.000000,0.000000,0.000000}%
\pgfsetstrokecolor{currentstroke}%
\pgfsetdash{}{0pt}%
\pgfpathmoveto{\pgfqpoint{4.930526in}{0.148611in}}%
\pgfpathlineto{\pgfqpoint{4.930526in}{0.391769in}}%
\pgfusepath{stroke}%
\end{pgfscope}%
\begin{pgfscope}%
\pgfsetrectcap%
\pgfsetmiterjoin%
\pgfsetlinewidth{0.803000pt}%
\definecolor{currentstroke}{rgb}{0.000000,0.000000,0.000000}%
\pgfsetstrokecolor{currentstroke}%
\pgfsetdash{}{0pt}%
\pgfpathmoveto{\pgfqpoint{4.106058in}{0.148611in}}%
\pgfpathlineto{\pgfqpoint{4.930526in}{0.148611in}}%
\pgfusepath{stroke}%
\end{pgfscope}%
\begin{pgfscope}%
\pgfsetrectcap%
\pgfsetmiterjoin%
\pgfsetlinewidth{0.803000pt}%
\definecolor{currentstroke}{rgb}{0.000000,0.000000,0.000000}%
\pgfsetstrokecolor{currentstroke}%
\pgfsetdash{}{0pt}%
\pgfpathmoveto{\pgfqpoint{4.106058in}{0.391769in}}%
\pgfpathlineto{\pgfqpoint{4.930526in}{0.391769in}}%
\pgfusepath{stroke}%
\end{pgfscope}%
\begin{pgfscope}%
\definecolor{textcolor}{rgb}{0.000000,0.000000,0.000000}%
\pgfsetstrokecolor{textcolor}%
\pgfsetfillcolor{textcolor}%
\pgftext[x=4.518292in,y=0.475102in,,base]{\color{textcolor}\rmfamily\fontsize{11.000000}{13.200000}\selectfont Euro-A...}%
\end{pgfscope}%
\begin{pgfscope}%
\pgfsetbuttcap%
\pgfsetmiterjoin%
\definecolor{currentfill}{rgb}{1.000000,1.000000,1.000000}%
\pgfsetfillcolor{currentfill}%
\pgfsetlinewidth{0.000000pt}%
\definecolor{currentstroke}{rgb}{0.000000,0.000000,0.000000}%
\pgfsetstrokecolor{currentstroke}%
\pgfsetstrokeopacity{0.000000}%
\pgfsetdash{}{0pt}%
\pgfpathmoveto{\pgfqpoint{5.095420in}{0.148611in}}%
\pgfpathlineto{\pgfqpoint{5.919888in}{0.148611in}}%
\pgfpathlineto{\pgfqpoint{5.919888in}{0.391769in}}%
\pgfpathlineto{\pgfqpoint{5.095420in}{0.391769in}}%
\pgfpathlineto{\pgfqpoint{5.095420in}{0.148611in}}%
\pgfpathclose%
\pgfusepath{fill}%
\end{pgfscope}%
\begin{pgfscope}%
\pgfpathrectangle{\pgfqpoint{5.095420in}{0.148611in}}{\pgfqpoint{0.824468in}{0.243158in}}%
\pgfusepath{clip}%
\pgfsetbuttcap%
\pgfsetmiterjoin%
\definecolor{currentfill}{rgb}{0.121569,0.466667,0.705882}%
\pgfsetfillcolor{currentfill}%
\pgfsetfillopacity{0.500000}%
\pgfsetlinewidth{1.003750pt}%
\definecolor{currentstroke}{rgb}{0.000000,0.000000,0.000000}%
\pgfsetstrokecolor{currentstroke}%
\pgfsetdash{}{0pt}%
\pgfpathmoveto{\pgfqpoint{5.132895in}{0.148611in}}%
\pgfpathlineto{\pgfqpoint{5.282799in}{0.148611in}}%
\pgfpathlineto{\pgfqpoint{5.282799in}{0.148976in}}%
\pgfpathlineto{\pgfqpoint{5.132895in}{0.148976in}}%
\pgfpathlineto{\pgfqpoint{5.132895in}{0.148611in}}%
\pgfpathclose%
\pgfusepath{stroke,fill}%
\end{pgfscope}%
\begin{pgfscope}%
\pgfpathrectangle{\pgfqpoint{5.095420in}{0.148611in}}{\pgfqpoint{0.824468in}{0.243158in}}%
\pgfusepath{clip}%
\pgfsetbuttcap%
\pgfsetmiterjoin%
\definecolor{currentfill}{rgb}{0.121569,0.466667,0.705882}%
\pgfsetfillcolor{currentfill}%
\pgfsetfillopacity{0.500000}%
\pgfsetlinewidth{1.003750pt}%
\definecolor{currentstroke}{rgb}{0.000000,0.000000,0.000000}%
\pgfsetstrokecolor{currentstroke}%
\pgfsetdash{}{0pt}%
\pgfpathmoveto{\pgfqpoint{5.282799in}{0.148611in}}%
\pgfpathlineto{\pgfqpoint{5.432702in}{0.148611in}}%
\pgfpathlineto{\pgfqpoint{5.432702in}{0.149462in}}%
\pgfpathlineto{\pgfqpoint{5.282799in}{0.149462in}}%
\pgfpathlineto{\pgfqpoint{5.282799in}{0.148611in}}%
\pgfpathclose%
\pgfusepath{stroke,fill}%
\end{pgfscope}%
\begin{pgfscope}%
\pgfpathrectangle{\pgfqpoint{5.095420in}{0.148611in}}{\pgfqpoint{0.824468in}{0.243158in}}%
\pgfusepath{clip}%
\pgfsetbuttcap%
\pgfsetmiterjoin%
\definecolor{currentfill}{rgb}{0.121569,0.466667,0.705882}%
\pgfsetfillcolor{currentfill}%
\pgfsetfillopacity{0.500000}%
\pgfsetlinewidth{1.003750pt}%
\definecolor{currentstroke}{rgb}{0.000000,0.000000,0.000000}%
\pgfsetstrokecolor{currentstroke}%
\pgfsetdash{}{0pt}%
\pgfpathmoveto{\pgfqpoint{5.432702in}{0.148611in}}%
\pgfpathlineto{\pgfqpoint{5.582605in}{0.148611in}}%
\pgfpathlineto{\pgfqpoint{5.582605in}{0.148611in}}%
\pgfpathlineto{\pgfqpoint{5.432702in}{0.148611in}}%
\pgfpathlineto{\pgfqpoint{5.432702in}{0.148611in}}%
\pgfpathclose%
\pgfusepath{stroke,fill}%
\end{pgfscope}%
\begin{pgfscope}%
\pgfpathrectangle{\pgfqpoint{5.095420in}{0.148611in}}{\pgfqpoint{0.824468in}{0.243158in}}%
\pgfusepath{clip}%
\pgfsetbuttcap%
\pgfsetmiterjoin%
\definecolor{currentfill}{rgb}{0.121569,0.466667,0.705882}%
\pgfsetfillcolor{currentfill}%
\pgfsetfillopacity{0.500000}%
\pgfsetlinewidth{1.003750pt}%
\definecolor{currentstroke}{rgb}{0.000000,0.000000,0.000000}%
\pgfsetstrokecolor{currentstroke}%
\pgfsetdash{}{0pt}%
\pgfpathmoveto{\pgfqpoint{5.582605in}{0.148611in}}%
\pgfpathlineto{\pgfqpoint{5.732509in}{0.148611in}}%
\pgfpathlineto{\pgfqpoint{5.732509in}{0.149219in}}%
\pgfpathlineto{\pgfqpoint{5.582605in}{0.149219in}}%
\pgfpathlineto{\pgfqpoint{5.582605in}{0.148611in}}%
\pgfpathclose%
\pgfusepath{stroke,fill}%
\end{pgfscope}%
\begin{pgfscope}%
\pgfpathrectangle{\pgfqpoint{5.095420in}{0.148611in}}{\pgfqpoint{0.824468in}{0.243158in}}%
\pgfusepath{clip}%
\pgfsetbuttcap%
\pgfsetmiterjoin%
\definecolor{currentfill}{rgb}{0.121569,0.466667,0.705882}%
\pgfsetfillcolor{currentfill}%
\pgfsetfillopacity{0.500000}%
\pgfsetlinewidth{1.003750pt}%
\definecolor{currentstroke}{rgb}{0.000000,0.000000,0.000000}%
\pgfsetstrokecolor{currentstroke}%
\pgfsetdash{}{0pt}%
\pgfpathmoveto{\pgfqpoint{5.732509in}{0.148611in}}%
\pgfpathlineto{\pgfqpoint{5.882412in}{0.148611in}}%
\pgfpathlineto{\pgfqpoint{5.882412in}{0.150192in}}%
\pgfpathlineto{\pgfqpoint{5.732509in}{0.150192in}}%
\pgfpathlineto{\pgfqpoint{5.732509in}{0.148611in}}%
\pgfpathclose%
\pgfusepath{stroke,fill}%
\end{pgfscope}%
\begin{pgfscope}%
\pgfsetrectcap%
\pgfsetmiterjoin%
\pgfsetlinewidth{0.803000pt}%
\definecolor{currentstroke}{rgb}{0.000000,0.000000,0.000000}%
\pgfsetstrokecolor{currentstroke}%
\pgfsetdash{}{0pt}%
\pgfpathmoveto{\pgfqpoint{5.095420in}{0.148611in}}%
\pgfpathlineto{\pgfqpoint{5.095420in}{0.391769in}}%
\pgfusepath{stroke}%
\end{pgfscope}%
\begin{pgfscope}%
\pgfsetrectcap%
\pgfsetmiterjoin%
\pgfsetlinewidth{0.803000pt}%
\definecolor{currentstroke}{rgb}{0.000000,0.000000,0.000000}%
\pgfsetstrokecolor{currentstroke}%
\pgfsetdash{}{0pt}%
\pgfpathmoveto{\pgfqpoint{5.919888in}{0.148611in}}%
\pgfpathlineto{\pgfqpoint{5.919888in}{0.391769in}}%
\pgfusepath{stroke}%
\end{pgfscope}%
\begin{pgfscope}%
\pgfsetrectcap%
\pgfsetmiterjoin%
\pgfsetlinewidth{0.803000pt}%
\definecolor{currentstroke}{rgb}{0.000000,0.000000,0.000000}%
\pgfsetstrokecolor{currentstroke}%
\pgfsetdash{}{0pt}%
\pgfpathmoveto{\pgfqpoint{5.095420in}{0.148611in}}%
\pgfpathlineto{\pgfqpoint{5.919888in}{0.148611in}}%
\pgfusepath{stroke}%
\end{pgfscope}%
\begin{pgfscope}%
\pgfsetrectcap%
\pgfsetmiterjoin%
\pgfsetlinewidth{0.803000pt}%
\definecolor{currentstroke}{rgb}{0.000000,0.000000,0.000000}%
\pgfsetstrokecolor{currentstroke}%
\pgfsetdash{}{0pt}%
\pgfpathmoveto{\pgfqpoint{5.095420in}{0.391769in}}%
\pgfpathlineto{\pgfqpoint{5.919888in}{0.391769in}}%
\pgfusepath{stroke}%
\end{pgfscope}%
\begin{pgfscope}%
\definecolor{textcolor}{rgb}{0.000000,0.000000,0.000000}%
\pgfsetstrokecolor{textcolor}%
\pgfsetfillcolor{textcolor}%
\pgftext[x=5.507654in,y=0.475102in,,base]{\color{textcolor}\rmfamily\fontsize{11.000000}{13.200000}\selectfont Peyrac...}%
\end{pgfscope}%
\begin{pgfscope}%
\pgfsetbuttcap%
\pgfsetmiterjoin%
\definecolor{currentfill}{rgb}{1.000000,1.000000,1.000000}%
\pgfsetfillcolor{currentfill}%
\pgfsetlinewidth{0.000000pt}%
\definecolor{currentstroke}{rgb}{0.000000,0.000000,0.000000}%
\pgfsetstrokecolor{currentstroke}%
\pgfsetstrokeopacity{0.000000}%
\pgfsetdash{}{0pt}%
\pgfpathmoveto{\pgfqpoint{6.084781in}{0.148611in}}%
\pgfpathlineto{\pgfqpoint{6.909249in}{0.148611in}}%
\pgfpathlineto{\pgfqpoint{6.909249in}{0.391769in}}%
\pgfpathlineto{\pgfqpoint{6.084781in}{0.391769in}}%
\pgfpathlineto{\pgfqpoint{6.084781in}{0.148611in}}%
\pgfpathclose%
\pgfusepath{fill}%
\end{pgfscope}%
\begin{pgfscope}%
\pgfpathrectangle{\pgfqpoint{6.084781in}{0.148611in}}{\pgfqpoint{0.824468in}{0.243158in}}%
\pgfusepath{clip}%
\pgfsetbuttcap%
\pgfsetmiterjoin%
\definecolor{currentfill}{rgb}{0.121569,0.466667,0.705882}%
\pgfsetfillcolor{currentfill}%
\pgfsetfillopacity{0.500000}%
\pgfsetlinewidth{1.003750pt}%
\definecolor{currentstroke}{rgb}{0.000000,0.000000,0.000000}%
\pgfsetstrokecolor{currentstroke}%
\pgfsetdash{}{0pt}%
\pgfpathmoveto{\pgfqpoint{6.122257in}{0.148611in}}%
\pgfpathlineto{\pgfqpoint{6.272160in}{0.148611in}}%
\pgfpathlineto{\pgfqpoint{6.272160in}{0.149219in}}%
\pgfpathlineto{\pgfqpoint{6.122257in}{0.149219in}}%
\pgfpathlineto{\pgfqpoint{6.122257in}{0.148611in}}%
\pgfpathclose%
\pgfusepath{stroke,fill}%
\end{pgfscope}%
\begin{pgfscope}%
\pgfpathrectangle{\pgfqpoint{6.084781in}{0.148611in}}{\pgfqpoint{0.824468in}{0.243158in}}%
\pgfusepath{clip}%
\pgfsetbuttcap%
\pgfsetmiterjoin%
\definecolor{currentfill}{rgb}{0.121569,0.466667,0.705882}%
\pgfsetfillcolor{currentfill}%
\pgfsetfillopacity{0.500000}%
\pgfsetlinewidth{1.003750pt}%
\definecolor{currentstroke}{rgb}{0.000000,0.000000,0.000000}%
\pgfsetstrokecolor{currentstroke}%
\pgfsetdash{}{0pt}%
\pgfpathmoveto{\pgfqpoint{6.272160in}{0.148611in}}%
\pgfpathlineto{\pgfqpoint{6.422064in}{0.148611in}}%
\pgfpathlineto{\pgfqpoint{6.422064in}{0.148611in}}%
\pgfpathlineto{\pgfqpoint{6.272160in}{0.148611in}}%
\pgfpathlineto{\pgfqpoint{6.272160in}{0.148611in}}%
\pgfpathclose%
\pgfusepath{stroke,fill}%
\end{pgfscope}%
\begin{pgfscope}%
\pgfpathrectangle{\pgfqpoint{6.084781in}{0.148611in}}{\pgfqpoint{0.824468in}{0.243158in}}%
\pgfusepath{clip}%
\pgfsetbuttcap%
\pgfsetmiterjoin%
\definecolor{currentfill}{rgb}{0.121569,0.466667,0.705882}%
\pgfsetfillcolor{currentfill}%
\pgfsetfillopacity{0.500000}%
\pgfsetlinewidth{1.003750pt}%
\definecolor{currentstroke}{rgb}{0.000000,0.000000,0.000000}%
\pgfsetstrokecolor{currentstroke}%
\pgfsetdash{}{0pt}%
\pgfpathmoveto{\pgfqpoint{6.422064in}{0.148611in}}%
\pgfpathlineto{\pgfqpoint{6.571967in}{0.148611in}}%
\pgfpathlineto{\pgfqpoint{6.571967in}{0.148611in}}%
\pgfpathlineto{\pgfqpoint{6.422064in}{0.148611in}}%
\pgfpathlineto{\pgfqpoint{6.422064in}{0.148611in}}%
\pgfpathclose%
\pgfusepath{stroke,fill}%
\end{pgfscope}%
\begin{pgfscope}%
\pgfpathrectangle{\pgfqpoint{6.084781in}{0.148611in}}{\pgfqpoint{0.824468in}{0.243158in}}%
\pgfusepath{clip}%
\pgfsetbuttcap%
\pgfsetmiterjoin%
\definecolor{currentfill}{rgb}{0.121569,0.466667,0.705882}%
\pgfsetfillcolor{currentfill}%
\pgfsetfillopacity{0.500000}%
\pgfsetlinewidth{1.003750pt}%
\definecolor{currentstroke}{rgb}{0.000000,0.000000,0.000000}%
\pgfsetstrokecolor{currentstroke}%
\pgfsetdash{}{0pt}%
\pgfpathmoveto{\pgfqpoint{6.571967in}{0.148611in}}%
\pgfpathlineto{\pgfqpoint{6.721870in}{0.148611in}}%
\pgfpathlineto{\pgfqpoint{6.721870in}{0.148611in}}%
\pgfpathlineto{\pgfqpoint{6.571967in}{0.148611in}}%
\pgfpathlineto{\pgfqpoint{6.571967in}{0.148611in}}%
\pgfpathclose%
\pgfusepath{stroke,fill}%
\end{pgfscope}%
\begin{pgfscope}%
\pgfpathrectangle{\pgfqpoint{6.084781in}{0.148611in}}{\pgfqpoint{0.824468in}{0.243158in}}%
\pgfusepath{clip}%
\pgfsetbuttcap%
\pgfsetmiterjoin%
\definecolor{currentfill}{rgb}{0.121569,0.466667,0.705882}%
\pgfsetfillcolor{currentfill}%
\pgfsetfillopacity{0.500000}%
\pgfsetlinewidth{1.003750pt}%
\definecolor{currentstroke}{rgb}{0.000000,0.000000,0.000000}%
\pgfsetstrokecolor{currentstroke}%
\pgfsetdash{}{0pt}%
\pgfpathmoveto{\pgfqpoint{6.721870in}{0.148611in}}%
\pgfpathlineto{\pgfqpoint{6.871774in}{0.148611in}}%
\pgfpathlineto{\pgfqpoint{6.871774in}{0.148733in}}%
\pgfpathlineto{\pgfqpoint{6.721870in}{0.148733in}}%
\pgfpathlineto{\pgfqpoint{6.721870in}{0.148611in}}%
\pgfpathclose%
\pgfusepath{stroke,fill}%
\end{pgfscope}%
\begin{pgfscope}%
\pgfsetrectcap%
\pgfsetmiterjoin%
\pgfsetlinewidth{0.803000pt}%
\definecolor{currentstroke}{rgb}{0.000000,0.000000,0.000000}%
\pgfsetstrokecolor{currentstroke}%
\pgfsetdash{}{0pt}%
\pgfpathmoveto{\pgfqpoint{6.084781in}{0.148611in}}%
\pgfpathlineto{\pgfqpoint{6.084781in}{0.391769in}}%
\pgfusepath{stroke}%
\end{pgfscope}%
\begin{pgfscope}%
\pgfsetrectcap%
\pgfsetmiterjoin%
\pgfsetlinewidth{0.803000pt}%
\definecolor{currentstroke}{rgb}{0.000000,0.000000,0.000000}%
\pgfsetstrokecolor{currentstroke}%
\pgfsetdash{}{0pt}%
\pgfpathmoveto{\pgfqpoint{6.909249in}{0.148611in}}%
\pgfpathlineto{\pgfqpoint{6.909249in}{0.391769in}}%
\pgfusepath{stroke}%
\end{pgfscope}%
\begin{pgfscope}%
\pgfsetrectcap%
\pgfsetmiterjoin%
\pgfsetlinewidth{0.803000pt}%
\definecolor{currentstroke}{rgb}{0.000000,0.000000,0.000000}%
\pgfsetstrokecolor{currentstroke}%
\pgfsetdash{}{0pt}%
\pgfpathmoveto{\pgfqpoint{6.084781in}{0.148611in}}%
\pgfpathlineto{\pgfqpoint{6.909249in}{0.148611in}}%
\pgfusepath{stroke}%
\end{pgfscope}%
\begin{pgfscope}%
\pgfsetrectcap%
\pgfsetmiterjoin%
\pgfsetlinewidth{0.803000pt}%
\definecolor{currentstroke}{rgb}{0.000000,0.000000,0.000000}%
\pgfsetstrokecolor{currentstroke}%
\pgfsetdash{}{0pt}%
\pgfpathmoveto{\pgfqpoint{6.084781in}{0.391769in}}%
\pgfpathlineto{\pgfqpoint{6.909249in}{0.391769in}}%
\pgfusepath{stroke}%
\end{pgfscope}%
\begin{pgfscope}%
\definecolor{textcolor}{rgb}{0.000000,0.000000,0.000000}%
\pgfsetstrokecolor{textcolor}%
\pgfsetfillcolor{textcolor}%
\pgftext[x=6.497015in,y=0.475102in,,base]{\color{textcolor}\rmfamily\fontsize{11.000000}{13.200000}\selectfont Sma}%
\end{pgfscope}%
\begin{pgfscope}%
\pgfsetbuttcap%
\pgfsetmiterjoin%
\definecolor{currentfill}{rgb}{1.000000,1.000000,1.000000}%
\pgfsetfillcolor{currentfill}%
\pgfsetlinewidth{0.000000pt}%
\definecolor{currentstroke}{rgb}{0.000000,0.000000,0.000000}%
\pgfsetstrokecolor{currentstroke}%
\pgfsetstrokeopacity{0.000000}%
\pgfsetdash{}{0pt}%
\pgfpathmoveto{\pgfqpoint{7.074143in}{0.148611in}}%
\pgfpathlineto{\pgfqpoint{7.898611in}{0.148611in}}%
\pgfpathlineto{\pgfqpoint{7.898611in}{0.391769in}}%
\pgfpathlineto{\pgfqpoint{7.074143in}{0.391769in}}%
\pgfpathlineto{\pgfqpoint{7.074143in}{0.148611in}}%
\pgfpathclose%
\pgfusepath{fill}%
\end{pgfscope}%
\begin{pgfscope}%
\pgfpathrectangle{\pgfqpoint{7.074143in}{0.148611in}}{\pgfqpoint{0.824468in}{0.243158in}}%
\pgfusepath{clip}%
\pgfsetbuttcap%
\pgfsetmiterjoin%
\definecolor{currentfill}{rgb}{0.121569,0.466667,0.705882}%
\pgfsetfillcolor{currentfill}%
\pgfsetfillopacity{0.500000}%
\pgfsetlinewidth{1.003750pt}%
\definecolor{currentstroke}{rgb}{0.000000,0.000000,0.000000}%
\pgfsetstrokecolor{currentstroke}%
\pgfsetdash{}{0pt}%
\pgfpathmoveto{\pgfqpoint{7.111619in}{0.148611in}}%
\pgfpathlineto{\pgfqpoint{7.261522in}{0.148611in}}%
\pgfpathlineto{\pgfqpoint{7.261522in}{0.148733in}}%
\pgfpathlineto{\pgfqpoint{7.111619in}{0.148733in}}%
\pgfpathlineto{\pgfqpoint{7.111619in}{0.148611in}}%
\pgfpathclose%
\pgfusepath{stroke,fill}%
\end{pgfscope}%
\begin{pgfscope}%
\pgfpathrectangle{\pgfqpoint{7.074143in}{0.148611in}}{\pgfqpoint{0.824468in}{0.243158in}}%
\pgfusepath{clip}%
\pgfsetbuttcap%
\pgfsetmiterjoin%
\definecolor{currentfill}{rgb}{0.121569,0.466667,0.705882}%
\pgfsetfillcolor{currentfill}%
\pgfsetfillopacity{0.500000}%
\pgfsetlinewidth{1.003750pt}%
\definecolor{currentstroke}{rgb}{0.000000,0.000000,0.000000}%
\pgfsetstrokecolor{currentstroke}%
\pgfsetdash{}{0pt}%
\pgfpathmoveto{\pgfqpoint{7.261522in}{0.148611in}}%
\pgfpathlineto{\pgfqpoint{7.411425in}{0.148611in}}%
\pgfpathlineto{\pgfqpoint{7.411425in}{0.148611in}}%
\pgfpathlineto{\pgfqpoint{7.261522in}{0.148611in}}%
\pgfpathlineto{\pgfqpoint{7.261522in}{0.148611in}}%
\pgfpathclose%
\pgfusepath{stroke,fill}%
\end{pgfscope}%
\begin{pgfscope}%
\pgfpathrectangle{\pgfqpoint{7.074143in}{0.148611in}}{\pgfqpoint{0.824468in}{0.243158in}}%
\pgfusepath{clip}%
\pgfsetbuttcap%
\pgfsetmiterjoin%
\definecolor{currentfill}{rgb}{0.121569,0.466667,0.705882}%
\pgfsetfillcolor{currentfill}%
\pgfsetfillopacity{0.500000}%
\pgfsetlinewidth{1.003750pt}%
\definecolor{currentstroke}{rgb}{0.000000,0.000000,0.000000}%
\pgfsetstrokecolor{currentstroke}%
\pgfsetdash{}{0pt}%
\pgfpathmoveto{\pgfqpoint{7.411425in}{0.148611in}}%
\pgfpathlineto{\pgfqpoint{7.561329in}{0.148611in}}%
\pgfpathlineto{\pgfqpoint{7.561329in}{0.148611in}}%
\pgfpathlineto{\pgfqpoint{7.411425in}{0.148611in}}%
\pgfpathlineto{\pgfqpoint{7.411425in}{0.148611in}}%
\pgfpathclose%
\pgfusepath{stroke,fill}%
\end{pgfscope}%
\begin{pgfscope}%
\pgfpathrectangle{\pgfqpoint{7.074143in}{0.148611in}}{\pgfqpoint{0.824468in}{0.243158in}}%
\pgfusepath{clip}%
\pgfsetbuttcap%
\pgfsetmiterjoin%
\definecolor{currentfill}{rgb}{0.121569,0.466667,0.705882}%
\pgfsetfillcolor{currentfill}%
\pgfsetfillopacity{0.500000}%
\pgfsetlinewidth{1.003750pt}%
\definecolor{currentstroke}{rgb}{0.000000,0.000000,0.000000}%
\pgfsetstrokecolor{currentstroke}%
\pgfsetdash{}{0pt}%
\pgfpathmoveto{\pgfqpoint{7.561329in}{0.148611in}}%
\pgfpathlineto{\pgfqpoint{7.711232in}{0.148611in}}%
\pgfpathlineto{\pgfqpoint{7.711232in}{0.148611in}}%
\pgfpathlineto{\pgfqpoint{7.561329in}{0.148611in}}%
\pgfpathlineto{\pgfqpoint{7.561329in}{0.148611in}}%
\pgfpathclose%
\pgfusepath{stroke,fill}%
\end{pgfscope}%
\begin{pgfscope}%
\pgfpathrectangle{\pgfqpoint{7.074143in}{0.148611in}}{\pgfqpoint{0.824468in}{0.243158in}}%
\pgfusepath{clip}%
\pgfsetbuttcap%
\pgfsetmiterjoin%
\definecolor{currentfill}{rgb}{0.121569,0.466667,0.705882}%
\pgfsetfillcolor{currentfill}%
\pgfsetfillopacity{0.500000}%
\pgfsetlinewidth{1.003750pt}%
\definecolor{currentstroke}{rgb}{0.000000,0.000000,0.000000}%
\pgfsetstrokecolor{currentstroke}%
\pgfsetdash{}{0pt}%
\pgfpathmoveto{\pgfqpoint{7.711232in}{0.148611in}}%
\pgfpathlineto{\pgfqpoint{7.861135in}{0.148611in}}%
\pgfpathlineto{\pgfqpoint{7.861135in}{0.148611in}}%
\pgfpathlineto{\pgfqpoint{7.711232in}{0.148611in}}%
\pgfpathlineto{\pgfqpoint{7.711232in}{0.148611in}}%
\pgfpathclose%
\pgfusepath{stroke,fill}%
\end{pgfscope}%
\begin{pgfscope}%
\pgfsetrectcap%
\pgfsetmiterjoin%
\pgfsetlinewidth{0.803000pt}%
\definecolor{currentstroke}{rgb}{0.000000,0.000000,0.000000}%
\pgfsetstrokecolor{currentstroke}%
\pgfsetdash{}{0pt}%
\pgfpathmoveto{\pgfqpoint{7.074143in}{0.148611in}}%
\pgfpathlineto{\pgfqpoint{7.074143in}{0.391769in}}%
\pgfusepath{stroke}%
\end{pgfscope}%
\begin{pgfscope}%
\pgfsetrectcap%
\pgfsetmiterjoin%
\pgfsetlinewidth{0.803000pt}%
\definecolor{currentstroke}{rgb}{0.000000,0.000000,0.000000}%
\pgfsetstrokecolor{currentstroke}%
\pgfsetdash{}{0pt}%
\pgfpathmoveto{\pgfqpoint{7.898611in}{0.148611in}}%
\pgfpathlineto{\pgfqpoint{7.898611in}{0.391769in}}%
\pgfusepath{stroke}%
\end{pgfscope}%
\begin{pgfscope}%
\pgfsetrectcap%
\pgfsetmiterjoin%
\pgfsetlinewidth{0.803000pt}%
\definecolor{currentstroke}{rgb}{0.000000,0.000000,0.000000}%
\pgfsetstrokecolor{currentstroke}%
\pgfsetdash{}{0pt}%
\pgfpathmoveto{\pgfqpoint{7.074143in}{0.148611in}}%
\pgfpathlineto{\pgfqpoint{7.898611in}{0.148611in}}%
\pgfusepath{stroke}%
\end{pgfscope}%
\begin{pgfscope}%
\pgfsetrectcap%
\pgfsetmiterjoin%
\pgfsetlinewidth{0.803000pt}%
\definecolor{currentstroke}{rgb}{0.000000,0.000000,0.000000}%
\pgfsetstrokecolor{currentstroke}%
\pgfsetdash{}{0pt}%
\pgfpathmoveto{\pgfqpoint{7.074143in}{0.391769in}}%
\pgfpathlineto{\pgfqpoint{7.898611in}{0.391769in}}%
\pgfusepath{stroke}%
\end{pgfscope}%
\begin{pgfscope}%
\definecolor{textcolor}{rgb}{0.000000,0.000000,0.000000}%
\pgfsetstrokecolor{textcolor}%
\pgfsetfillcolor{textcolor}%
\pgftext[x=7.486377in,y=0.475102in,,base]{\color{textcolor}\rmfamily\fontsize{11.000000}{13.200000}\selectfont Hiscox}%
\end{pgfscope}%
\end{pgfpicture}%
\makeatother%
\endgroup%

    \caption{Stars distribution per assureur (y-axis scaled)}
    \label{fig:distrib_split_scale}
\end{figure}

\restoregeometry

\newgeometry{bottom=0cm}
We also looked at the mean note per assureur, we used a gradient color to show the number of rating per assureur, the gradient intensity is defined by the number of ratings in \cref{fig:mean_note_per_assureur} and by the rank of the assureur (ordered by number of rating) in \cref{fig:mean_note_per_assureur_linear}

\begin{figure}[H]
    \advance\leftskip-3cm
    %% Creator: Matplotlib, PGF backend
%%
%% To include the figure in your LaTeX document, write
%%   \input{<filename>.pgf}
%%
%% Make sure the required packages are loaded in your preamble
%%   \usepackage{pgf}
%%
%% Also ensure that all the required font packages are loaded; for instance,
%% the lmodern package is sometimes necessary when using math font.
%%   \usepackage{lmodern}
%%
%% Figures using additional raster images can only be included by \input if
%% they are in the same directory as the main LaTeX file. For loading figures
%% from other directories you can use the `import` package
%%   \usepackage{import}
%%
%% and then include the figures with
%%   \import{<path to file>}{<filename>.pgf}
%%
%% Matplotlib used the following preamble
%%
\begingroup%
\makeatletter%
\begin{pgfpicture}%
\pgfpathrectangle{\pgfpointorigin}{\pgfqpoint{7.962191in}{3.967519in}}%
\pgfusepath{use as bounding box, clip}%
\begin{pgfscope}%
\pgfsetbuttcap%
\pgfsetmiterjoin%
\definecolor{currentfill}{rgb}{1.000000,1.000000,1.000000}%
\pgfsetfillcolor{currentfill}%
\pgfsetlinewidth{0.000000pt}%
\definecolor{currentstroke}{rgb}{1.000000,1.000000,1.000000}%
\pgfsetstrokecolor{currentstroke}%
\pgfsetdash{}{0pt}%
\pgfpathmoveto{\pgfqpoint{0.000000in}{0.000000in}}%
\pgfpathlineto{\pgfqpoint{7.962191in}{0.000000in}}%
\pgfpathlineto{\pgfqpoint{7.962191in}{3.967519in}}%
\pgfpathlineto{\pgfqpoint{0.000000in}{3.967519in}}%
\pgfpathlineto{\pgfqpoint{0.000000in}{0.000000in}}%
\pgfpathclose%
\pgfusepath{fill}%
\end{pgfscope}%
\begin{pgfscope}%
\pgfsetbuttcap%
\pgfsetmiterjoin%
\definecolor{currentfill}{rgb}{1.000000,1.000000,1.000000}%
\pgfsetfillcolor{currentfill}%
\pgfsetlinewidth{0.000000pt}%
\definecolor{currentstroke}{rgb}{0.000000,0.000000,0.000000}%
\pgfsetstrokecolor{currentstroke}%
\pgfsetstrokeopacity{0.000000}%
\pgfsetdash{}{0pt}%
\pgfpathmoveto{\pgfqpoint{0.499691in}{1.172519in}}%
\pgfpathlineto{\pgfqpoint{7.862191in}{1.172519in}}%
\pgfpathlineto{\pgfqpoint{7.862191in}{3.867519in}}%
\pgfpathlineto{\pgfqpoint{0.499691in}{3.867519in}}%
\pgfpathlineto{\pgfqpoint{0.499691in}{1.172519in}}%
\pgfpathclose%
\pgfusepath{fill}%
\end{pgfscope}%
\begin{pgfscope}%
\pgfpathrectangle{\pgfqpoint{0.499691in}{1.172519in}}{\pgfqpoint{7.362500in}{2.695000in}}%
\pgfusepath{clip}%
\pgfsetbuttcap%
\pgfsetmiterjoin%
\definecolor{currentfill}{rgb}{0.548178,0.714944,0.584403}%
\pgfsetfillcolor{currentfill}%
\pgfsetlinewidth{0.000000pt}%
\definecolor{currentstroke}{rgb}{0.000000,0.000000,0.000000}%
\pgfsetstrokecolor{currentstroke}%
\pgfsetstrokeopacity{0.000000}%
\pgfsetdash{}{0pt}%
\pgfpathmoveto{\pgfqpoint{0.512838in}{1.172519in}}%
\pgfpathlineto{\pgfqpoint{0.618017in}{1.172519in}}%
\pgfpathlineto{\pgfqpoint{0.618017in}{3.308273in}}%
\pgfpathlineto{\pgfqpoint{0.512838in}{3.308273in}}%
\pgfpathlineto{\pgfqpoint{0.512838in}{1.172519in}}%
\pgfpathclose%
\pgfusepath{fill}%
\end{pgfscope}%
\begin{pgfscope}%
\pgfpathrectangle{\pgfqpoint{0.499691in}{1.172519in}}{\pgfqpoint{7.362500in}{2.695000in}}%
\pgfusepath{clip}%
\pgfsetbuttcap%
\pgfsetmiterjoin%
\definecolor{currentfill}{rgb}{0.614104,0.751679,0.595029}%
\pgfsetfillcolor{currentfill}%
\pgfsetlinewidth{0.000000pt}%
\definecolor{currentstroke}{rgb}{0.000000,0.000000,0.000000}%
\pgfsetstrokecolor{currentstroke}%
\pgfsetstrokeopacity{0.000000}%
\pgfsetdash{}{0pt}%
\pgfpathmoveto{\pgfqpoint{0.644312in}{1.172519in}}%
\pgfpathlineto{\pgfqpoint{0.749490in}{1.172519in}}%
\pgfpathlineto{\pgfqpoint{0.749490in}{2.572783in}}%
\pgfpathlineto{\pgfqpoint{0.644312in}{2.572783in}}%
\pgfpathlineto{\pgfqpoint{0.644312in}{1.172519in}}%
\pgfpathclose%
\pgfusepath{fill}%
\end{pgfscope}%
\begin{pgfscope}%
\pgfpathrectangle{\pgfqpoint{0.499691in}{1.172519in}}{\pgfqpoint{7.362500in}{2.695000in}}%
\pgfusepath{clip}%
\pgfsetbuttcap%
\pgfsetmiterjoin%
\definecolor{currentfill}{rgb}{0.487377,0.679908,0.570839}%
\pgfsetfillcolor{currentfill}%
\pgfsetlinewidth{0.000000pt}%
\definecolor{currentstroke}{rgb}{0.000000,0.000000,0.000000}%
\pgfsetstrokecolor{currentstroke}%
\pgfsetstrokeopacity{0.000000}%
\pgfsetdash{}{0pt}%
\pgfpathmoveto{\pgfqpoint{0.775785in}{1.172519in}}%
\pgfpathlineto{\pgfqpoint{0.880963in}{1.172519in}}%
\pgfpathlineto{\pgfqpoint{0.880963in}{3.455430in}}%
\pgfpathlineto{\pgfqpoint{0.775785in}{3.455430in}}%
\pgfpathlineto{\pgfqpoint{0.775785in}{1.172519in}}%
\pgfpathclose%
\pgfusepath{fill}%
\end{pgfscope}%
\begin{pgfscope}%
\pgfpathrectangle{\pgfqpoint{0.499691in}{1.172519in}}{\pgfqpoint{7.362500in}{2.695000in}}%
\pgfusepath{clip}%
\pgfsetbuttcap%
\pgfsetmiterjoin%
\definecolor{currentfill}{rgb}{0.556377,0.719609,0.586195}%
\pgfsetfillcolor{currentfill}%
\pgfsetlinewidth{0.000000pt}%
\definecolor{currentstroke}{rgb}{0.000000,0.000000,0.000000}%
\pgfsetstrokecolor{currentstroke}%
\pgfsetstrokeopacity{0.000000}%
\pgfsetdash{}{0pt}%
\pgfpathmoveto{\pgfqpoint{0.907258in}{1.172519in}}%
\pgfpathlineto{\pgfqpoint{1.012437in}{1.172519in}}%
\pgfpathlineto{\pgfqpoint{1.012437in}{2.156618in}}%
\pgfpathlineto{\pgfqpoint{0.907258in}{2.156618in}}%
\pgfpathlineto{\pgfqpoint{0.907258in}{1.172519in}}%
\pgfpathclose%
\pgfusepath{fill}%
\end{pgfscope}%
\begin{pgfscope}%
\pgfpathrectangle{\pgfqpoint{0.499691in}{1.172519in}}{\pgfqpoint{7.362500in}{2.695000in}}%
\pgfusepath{clip}%
\pgfsetbuttcap%
\pgfsetmiterjoin%
\definecolor{currentfill}{rgb}{0.596607,0.742633,0.594077}%
\pgfsetfillcolor{currentfill}%
\pgfsetlinewidth{0.000000pt}%
\definecolor{currentstroke}{rgb}{0.000000,0.000000,0.000000}%
\pgfsetstrokecolor{currentstroke}%
\pgfsetstrokeopacity{0.000000}%
\pgfsetdash{}{0pt}%
\pgfpathmoveto{\pgfqpoint{1.038731in}{1.172519in}}%
\pgfpathlineto{\pgfqpoint{1.143910in}{1.172519in}}%
\pgfpathlineto{\pgfqpoint{1.143910in}{2.163009in}}%
\pgfpathlineto{\pgfqpoint{1.038731in}{2.163009in}}%
\pgfpathlineto{\pgfqpoint{1.038731in}{1.172519in}}%
\pgfpathclose%
\pgfusepath{fill}%
\end{pgfscope}%
\begin{pgfscope}%
\pgfpathrectangle{\pgfqpoint{0.499691in}{1.172519in}}{\pgfqpoint{7.362500in}{2.695000in}}%
\pgfusepath{clip}%
\pgfsetbuttcap%
\pgfsetmiterjoin%
\definecolor{currentfill}{rgb}{0.635089,0.762521,0.595670}%
\pgfsetfillcolor{currentfill}%
\pgfsetlinewidth{0.000000pt}%
\definecolor{currentstroke}{rgb}{0.000000,0.000000,0.000000}%
\pgfsetstrokecolor{currentstroke}%
\pgfsetstrokeopacity{0.000000}%
\pgfsetdash{}{0pt}%
\pgfpathmoveto{\pgfqpoint{1.170205in}{1.172519in}}%
\pgfpathlineto{\pgfqpoint{1.275383in}{1.172519in}}%
\pgfpathlineto{\pgfqpoint{1.275383in}{2.192935in}}%
\pgfpathlineto{\pgfqpoint{1.170205in}{2.192935in}}%
\pgfpathlineto{\pgfqpoint{1.170205in}{1.172519in}}%
\pgfpathclose%
\pgfusepath{fill}%
\end{pgfscope}%
\begin{pgfscope}%
\pgfpathrectangle{\pgfqpoint{0.499691in}{1.172519in}}{\pgfqpoint{7.362500in}{2.695000in}}%
\pgfusepath{clip}%
\pgfsetbuttcap%
\pgfsetmiterjoin%
\definecolor{currentfill}{rgb}{0.652651,0.771509,0.595697}%
\pgfsetfillcolor{currentfill}%
\pgfsetlinewidth{0.000000pt}%
\definecolor{currentstroke}{rgb}{0.000000,0.000000,0.000000}%
\pgfsetstrokecolor{currentstroke}%
\pgfsetstrokeopacity{0.000000}%
\pgfsetdash{}{0pt}%
\pgfpathmoveto{\pgfqpoint{1.301678in}{1.172519in}}%
\pgfpathlineto{\pgfqpoint{1.406856in}{1.172519in}}%
\pgfpathlineto{\pgfqpoint{1.406856in}{2.329526in}}%
\pgfpathlineto{\pgfqpoint{1.301678in}{2.329526in}}%
\pgfpathlineto{\pgfqpoint{1.301678in}{1.172519in}}%
\pgfpathclose%
\pgfusepath{fill}%
\end{pgfscope}%
\begin{pgfscope}%
\pgfpathrectangle{\pgfqpoint{0.499691in}{1.172519in}}{\pgfqpoint{7.362500in}{2.695000in}}%
\pgfusepath{clip}%
\pgfsetbuttcap%
\pgfsetmiterjoin%
\definecolor{currentfill}{rgb}{0.603611,0.746252,0.594501}%
\pgfsetfillcolor{currentfill}%
\pgfsetlinewidth{0.000000pt}%
\definecolor{currentstroke}{rgb}{0.000000,0.000000,0.000000}%
\pgfsetstrokecolor{currentstroke}%
\pgfsetstrokeopacity{0.000000}%
\pgfsetdash{}{0pt}%
\pgfpathmoveto{\pgfqpoint{1.433151in}{1.172519in}}%
\pgfpathlineto{\pgfqpoint{1.538330in}{1.172519in}}%
\pgfpathlineto{\pgfqpoint{1.538330in}{1.984076in}}%
\pgfpathlineto{\pgfqpoint{1.433151in}{1.984076in}}%
\pgfpathlineto{\pgfqpoint{1.433151in}{1.172519in}}%
\pgfpathclose%
\pgfusepath{fill}%
\end{pgfscope}%
\begin{pgfscope}%
\pgfpathrectangle{\pgfqpoint{0.499691in}{1.172519in}}{\pgfqpoint{7.362500in}{2.695000in}}%
\pgfusepath{clip}%
\pgfsetbuttcap%
\pgfsetmiterjoin%
\definecolor{currentfill}{rgb}{0.568623,0.726620,0.588802}%
\pgfsetfillcolor{currentfill}%
\pgfsetlinewidth{0.000000pt}%
\definecolor{currentstroke}{rgb}{0.000000,0.000000,0.000000}%
\pgfsetstrokecolor{currentstroke}%
\pgfsetstrokeopacity{0.000000}%
\pgfsetdash{}{0pt}%
\pgfpathmoveto{\pgfqpoint{1.564624in}{1.172519in}}%
\pgfpathlineto{\pgfqpoint{1.669803in}{1.172519in}}%
\pgfpathlineto{\pgfqpoint{1.669803in}{2.061439in}}%
\pgfpathlineto{\pgfqpoint{1.564624in}{2.061439in}}%
\pgfpathlineto{\pgfqpoint{1.564624in}{1.172519in}}%
\pgfpathclose%
\pgfusepath{fill}%
\end{pgfscope}%
\begin{pgfscope}%
\pgfpathrectangle{\pgfqpoint{0.499691in}{1.172519in}}{\pgfqpoint{7.362500in}{2.695000in}}%
\pgfusepath{clip}%
\pgfsetbuttcap%
\pgfsetmiterjoin%
\definecolor{currentfill}{rgb}{0.649123,0.769720,0.595746}%
\pgfsetfillcolor{currentfill}%
\pgfsetlinewidth{0.000000pt}%
\definecolor{currentstroke}{rgb}{0.000000,0.000000,0.000000}%
\pgfsetstrokecolor{currentstroke}%
\pgfsetstrokeopacity{0.000000}%
\pgfsetdash{}{0pt}%
\pgfpathmoveto{\pgfqpoint{1.696097in}{1.172519in}}%
\pgfpathlineto{\pgfqpoint{1.801276in}{1.172519in}}%
\pgfpathlineto{\pgfqpoint{1.801276in}{2.669822in}}%
\pgfpathlineto{\pgfqpoint{1.696097in}{2.669822in}}%
\pgfpathlineto{\pgfqpoint{1.696097in}{1.172519in}}%
\pgfpathclose%
\pgfusepath{fill}%
\end{pgfscope}%
\begin{pgfscope}%
\pgfpathrectangle{\pgfqpoint{0.499691in}{1.172519in}}{\pgfqpoint{7.362500in}{2.695000in}}%
\pgfusepath{clip}%
\pgfsetbuttcap%
\pgfsetmiterjoin%
\definecolor{currentfill}{rgb}{0.642098,0.766125,0.595747}%
\pgfsetfillcolor{currentfill}%
\pgfsetlinewidth{0.000000pt}%
\definecolor{currentstroke}{rgb}{0.000000,0.000000,0.000000}%
\pgfsetstrokecolor{currentstroke}%
\pgfsetstrokeopacity{0.000000}%
\pgfsetdash{}{0pt}%
\pgfpathmoveto{\pgfqpoint{1.827571in}{1.172519in}}%
\pgfpathlineto{\pgfqpoint{1.932749in}{1.172519in}}%
\pgfpathlineto{\pgfqpoint{1.932749in}{2.271675in}}%
\pgfpathlineto{\pgfqpoint{1.827571in}{2.271675in}}%
\pgfpathlineto{\pgfqpoint{1.827571in}{1.172519in}}%
\pgfpathclose%
\pgfusepath{fill}%
\end{pgfscope}%
\begin{pgfscope}%
\pgfpathrectangle{\pgfqpoint{0.499691in}{1.172519in}}{\pgfqpoint{7.362500in}{2.695000in}}%
\pgfusepath{clip}%
\pgfsetbuttcap%
\pgfsetmiterjoin%
\definecolor{currentfill}{rgb}{0.652651,0.771509,0.595697}%
\pgfsetfillcolor{currentfill}%
\pgfsetlinewidth{0.000000pt}%
\definecolor{currentstroke}{rgb}{0.000000,0.000000,0.000000}%
\pgfsetstrokecolor{currentstroke}%
\pgfsetstrokeopacity{0.000000}%
\pgfsetdash{}{0pt}%
\pgfpathmoveto{\pgfqpoint{1.959044in}{1.172519in}}%
\pgfpathlineto{\pgfqpoint{2.064222in}{1.172519in}}%
\pgfpathlineto{\pgfqpoint{2.064222in}{2.005564in}}%
\pgfpathlineto{\pgfqpoint{1.959044in}{2.005564in}}%
\pgfpathlineto{\pgfqpoint{1.959044in}{1.172519in}}%
\pgfpathclose%
\pgfusepath{fill}%
\end{pgfscope}%
\begin{pgfscope}%
\pgfpathrectangle{\pgfqpoint{0.499691in}{1.172519in}}{\pgfqpoint{7.362500in}{2.695000in}}%
\pgfusepath{clip}%
\pgfsetbuttcap%
\pgfsetmiterjoin%
\definecolor{currentfill}{rgb}{0.635089,0.762521,0.595670}%
\pgfsetfillcolor{currentfill}%
\pgfsetlinewidth{0.000000pt}%
\definecolor{currentstroke}{rgb}{0.000000,0.000000,0.000000}%
\pgfsetstrokecolor{currentstroke}%
\pgfsetstrokeopacity{0.000000}%
\pgfsetdash{}{0pt}%
\pgfpathmoveto{\pgfqpoint{2.090517in}{1.172519in}}%
\pgfpathlineto{\pgfqpoint{2.195696in}{1.172519in}}%
\pgfpathlineto{\pgfqpoint{2.195696in}{1.998952in}}%
\pgfpathlineto{\pgfqpoint{2.090517in}{1.998952in}}%
\pgfpathlineto{\pgfqpoint{2.090517in}{1.172519in}}%
\pgfpathclose%
\pgfusepath{fill}%
\end{pgfscope}%
\begin{pgfscope}%
\pgfpathrectangle{\pgfqpoint{0.499691in}{1.172519in}}{\pgfqpoint{7.362500in}{2.695000in}}%
\pgfusepath{clip}%
\pgfsetbuttcap%
\pgfsetmiterjoin%
\definecolor{currentfill}{rgb}{0.652651,0.771509,0.595697}%
\pgfsetfillcolor{currentfill}%
\pgfsetlinewidth{0.000000pt}%
\definecolor{currentstroke}{rgb}{0.000000,0.000000,0.000000}%
\pgfsetstrokecolor{currentstroke}%
\pgfsetstrokeopacity{0.000000}%
\pgfsetdash{}{0pt}%
\pgfpathmoveto{\pgfqpoint{2.221990in}{1.172519in}}%
\pgfpathlineto{\pgfqpoint{2.327169in}{1.172519in}}%
\pgfpathlineto{\pgfqpoint{2.327169in}{2.564543in}}%
\pgfpathlineto{\pgfqpoint{2.221990in}{2.564543in}}%
\pgfpathlineto{\pgfqpoint{2.221990in}{1.172519in}}%
\pgfpathclose%
\pgfusepath{fill}%
\end{pgfscope}%
\begin{pgfscope}%
\pgfpathrectangle{\pgfqpoint{0.499691in}{1.172519in}}{\pgfqpoint{7.362500in}{2.695000in}}%
\pgfusepath{clip}%
\pgfsetbuttcap%
\pgfsetmiterjoin%
\definecolor{currentfill}{rgb}{0.617599,0.753489,0.595178}%
\pgfsetfillcolor{currentfill}%
\pgfsetlinewidth{0.000000pt}%
\definecolor{currentstroke}{rgb}{0.000000,0.000000,0.000000}%
\pgfsetstrokecolor{currentstroke}%
\pgfsetstrokeopacity{0.000000}%
\pgfsetdash{}{0pt}%
\pgfpathmoveto{\pgfqpoint{2.353463in}{1.172519in}}%
\pgfpathlineto{\pgfqpoint{2.458642in}{1.172519in}}%
\pgfpathlineto{\pgfqpoint{2.458642in}{1.936580in}}%
\pgfpathlineto{\pgfqpoint{2.353463in}{1.936580in}}%
\pgfpathlineto{\pgfqpoint{2.353463in}{1.172519in}}%
\pgfpathclose%
\pgfusepath{fill}%
\end{pgfscope}%
\begin{pgfscope}%
\pgfpathrectangle{\pgfqpoint{0.499691in}{1.172519in}}{\pgfqpoint{7.362500in}{2.695000in}}%
\pgfusepath{clip}%
\pgfsetbuttcap%
\pgfsetmiterjoin%
\definecolor{currentfill}{rgb}{0.624593,0.757104,0.595421}%
\pgfsetfillcolor{currentfill}%
\pgfsetlinewidth{0.000000pt}%
\definecolor{currentstroke}{rgb}{0.000000,0.000000,0.000000}%
\pgfsetstrokecolor{currentstroke}%
\pgfsetstrokeopacity{0.000000}%
\pgfsetdash{}{0pt}%
\pgfpathmoveto{\pgfqpoint{2.484937in}{1.172519in}}%
\pgfpathlineto{\pgfqpoint{2.590115in}{1.172519in}}%
\pgfpathlineto{\pgfqpoint{2.590115in}{2.037571in}}%
\pgfpathlineto{\pgfqpoint{2.484937in}{2.037571in}}%
\pgfpathlineto{\pgfqpoint{2.484937in}{1.172519in}}%
\pgfpathclose%
\pgfusepath{fill}%
\end{pgfscope}%
\begin{pgfscope}%
\pgfpathrectangle{\pgfqpoint{0.499691in}{1.172519in}}{\pgfqpoint{7.362500in}{2.695000in}}%
\pgfusepath{clip}%
\pgfsetbuttcap%
\pgfsetmiterjoin%
\definecolor{currentfill}{rgb}{0.635089,0.762521,0.595670}%
\pgfsetfillcolor{currentfill}%
\pgfsetlinewidth{0.000000pt}%
\definecolor{currentstroke}{rgb}{0.000000,0.000000,0.000000}%
\pgfsetstrokecolor{currentstroke}%
\pgfsetstrokeopacity{0.000000}%
\pgfsetdash{}{0pt}%
\pgfpathmoveto{\pgfqpoint{2.616410in}{1.172519in}}%
\pgfpathlineto{\pgfqpoint{2.721588in}{1.172519in}}%
\pgfpathlineto{\pgfqpoint{2.721588in}{2.198768in}}%
\pgfpathlineto{\pgfqpoint{2.616410in}{2.198768in}}%
\pgfpathlineto{\pgfqpoint{2.616410in}{1.172519in}}%
\pgfpathclose%
\pgfusepath{fill}%
\end{pgfscope}%
\begin{pgfscope}%
\pgfpathrectangle{\pgfqpoint{0.499691in}{1.172519in}}{\pgfqpoint{7.362500in}{2.695000in}}%
\pgfusepath{clip}%
\pgfsetbuttcap%
\pgfsetmiterjoin%
\definecolor{currentfill}{rgb}{0.207614,0.220467,0.411508}%
\pgfsetfillcolor{currentfill}%
\pgfsetlinewidth{0.000000pt}%
\definecolor{currentstroke}{rgb}{0.000000,0.000000,0.000000}%
\pgfsetstrokecolor{currentstroke}%
\pgfsetstrokeopacity{0.000000}%
\pgfsetdash{}{0pt}%
\pgfpathmoveto{\pgfqpoint{2.747883in}{1.172519in}}%
\pgfpathlineto{\pgfqpoint{2.853062in}{1.172519in}}%
\pgfpathlineto{\pgfqpoint{2.853062in}{3.090136in}}%
\pgfpathlineto{\pgfqpoint{2.747883in}{3.090136in}}%
\pgfpathlineto{\pgfqpoint{2.747883in}{1.172519in}}%
\pgfpathclose%
\pgfusepath{fill}%
\end{pgfscope}%
\begin{pgfscope}%
\pgfpathrectangle{\pgfqpoint{0.499691in}{1.172519in}}{\pgfqpoint{7.362500in}{2.695000in}}%
\pgfusepath{clip}%
\pgfsetbuttcap%
\pgfsetmiterjoin%
\definecolor{currentfill}{rgb}{0.638590,0.764325,0.595722}%
\pgfsetfillcolor{currentfill}%
\pgfsetlinewidth{0.000000pt}%
\definecolor{currentstroke}{rgb}{0.000000,0.000000,0.000000}%
\pgfsetstrokecolor{currentstroke}%
\pgfsetstrokeopacity{0.000000}%
\pgfsetdash{}{0pt}%
\pgfpathmoveto{\pgfqpoint{2.879356in}{1.172519in}}%
\pgfpathlineto{\pgfqpoint{2.984535in}{1.172519in}}%
\pgfpathlineto{\pgfqpoint{2.984535in}{2.114778in}}%
\pgfpathlineto{\pgfqpoint{2.879356in}{2.114778in}}%
\pgfpathlineto{\pgfqpoint{2.879356in}{1.172519in}}%
\pgfpathclose%
\pgfusepath{fill}%
\end{pgfscope}%
\begin{pgfscope}%
\pgfpathrectangle{\pgfqpoint{0.499691in}{1.172519in}}{\pgfqpoint{7.362500in}{2.695000in}}%
\pgfusepath{clip}%
\pgfsetbuttcap%
\pgfsetmiterjoin%
\definecolor{currentfill}{rgb}{0.652651,0.771509,0.595697}%
\pgfsetfillcolor{currentfill}%
\pgfsetlinewidth{0.000000pt}%
\definecolor{currentstroke}{rgb}{0.000000,0.000000,0.000000}%
\pgfsetstrokecolor{currentstroke}%
\pgfsetstrokeopacity{0.000000}%
\pgfsetdash{}{0pt}%
\pgfpathmoveto{\pgfqpoint{3.010830in}{1.172519in}}%
\pgfpathlineto{\pgfqpoint{3.116008in}{1.172519in}}%
\pgfpathlineto{\pgfqpoint{3.116008in}{2.114269in}}%
\pgfpathlineto{\pgfqpoint{3.010830in}{2.114269in}}%
\pgfpathlineto{\pgfqpoint{3.010830in}{1.172519in}}%
\pgfpathclose%
\pgfusepath{fill}%
\end{pgfscope}%
\begin{pgfscope}%
\pgfpathrectangle{\pgfqpoint{0.499691in}{1.172519in}}{\pgfqpoint{7.362500in}{2.695000in}}%
\pgfusepath{clip}%
\pgfsetbuttcap%
\pgfsetmiterjoin%
\definecolor{currentfill}{rgb}{0.614104,0.751679,0.595029}%
\pgfsetfillcolor{currentfill}%
\pgfsetlinewidth{0.000000pt}%
\definecolor{currentstroke}{rgb}{0.000000,0.000000,0.000000}%
\pgfsetstrokecolor{currentstroke}%
\pgfsetstrokeopacity{0.000000}%
\pgfsetdash{}{0pt}%
\pgfpathmoveto{\pgfqpoint{3.142303in}{1.172519in}}%
\pgfpathlineto{\pgfqpoint{3.247481in}{1.172519in}}%
\pgfpathlineto{\pgfqpoint{3.247481in}{2.311510in}}%
\pgfpathlineto{\pgfqpoint{3.142303in}{2.311510in}}%
\pgfpathlineto{\pgfqpoint{3.142303in}{1.172519in}}%
\pgfpathclose%
\pgfusepath{fill}%
\end{pgfscope}%
\begin{pgfscope}%
\pgfpathrectangle{\pgfqpoint{0.499691in}{1.172519in}}{\pgfqpoint{7.362500in}{2.695000in}}%
\pgfusepath{clip}%
\pgfsetbuttcap%
\pgfsetmiterjoin%
\definecolor{currentfill}{rgb}{0.491283,0.682262,0.571685}%
\pgfsetfillcolor{currentfill}%
\pgfsetlinewidth{0.000000pt}%
\definecolor{currentstroke}{rgb}{0.000000,0.000000,0.000000}%
\pgfsetstrokecolor{currentstroke}%
\pgfsetstrokeopacity{0.000000}%
\pgfsetdash{}{0pt}%
\pgfpathmoveto{\pgfqpoint{3.273776in}{1.172519in}}%
\pgfpathlineto{\pgfqpoint{3.378955in}{1.172519in}}%
\pgfpathlineto{\pgfqpoint{3.378955in}{2.841948in}}%
\pgfpathlineto{\pgfqpoint{3.273776in}{2.841948in}}%
\pgfpathlineto{\pgfqpoint{3.273776in}{1.172519in}}%
\pgfpathclose%
\pgfusepath{fill}%
\end{pgfscope}%
\begin{pgfscope}%
\pgfpathrectangle{\pgfqpoint{0.499691in}{1.172519in}}{\pgfqpoint{7.362500in}{2.695000in}}%
\pgfusepath{clip}%
\pgfsetbuttcap%
\pgfsetmiterjoin%
\definecolor{currentfill}{rgb}{0.652651,0.771509,0.595697}%
\pgfsetfillcolor{currentfill}%
\pgfsetlinewidth{0.000000pt}%
\definecolor{currentstroke}{rgb}{0.000000,0.000000,0.000000}%
\pgfsetstrokecolor{currentstroke}%
\pgfsetstrokeopacity{0.000000}%
\pgfsetdash{}{0pt}%
\pgfpathmoveto{\pgfqpoint{3.405249in}{1.172519in}}%
\pgfpathlineto{\pgfqpoint{3.510428in}{1.172519in}}%
\pgfpathlineto{\pgfqpoint{3.510428in}{2.031509in}}%
\pgfpathlineto{\pgfqpoint{3.405249in}{2.031509in}}%
\pgfpathlineto{\pgfqpoint{3.405249in}{1.172519in}}%
\pgfpathclose%
\pgfusepath{fill}%
\end{pgfscope}%
\begin{pgfscope}%
\pgfpathrectangle{\pgfqpoint{0.499691in}{1.172519in}}{\pgfqpoint{7.362500in}{2.695000in}}%
\pgfusepath{clip}%
\pgfsetbuttcap%
\pgfsetmiterjoin%
\definecolor{currentfill}{rgb}{0.638590,0.764325,0.595722}%
\pgfsetfillcolor{currentfill}%
\pgfsetlinewidth{0.000000pt}%
\definecolor{currentstroke}{rgb}{0.000000,0.000000,0.000000}%
\pgfsetstrokecolor{currentstroke}%
\pgfsetstrokeopacity{0.000000}%
\pgfsetdash{}{0pt}%
\pgfpathmoveto{\pgfqpoint{3.536722in}{1.172519in}}%
\pgfpathlineto{\pgfqpoint{3.641901in}{1.172519in}}%
\pgfpathlineto{\pgfqpoint{3.641901in}{2.062872in}}%
\pgfpathlineto{\pgfqpoint{3.536722in}{2.062872in}}%
\pgfpathlineto{\pgfqpoint{3.536722in}{1.172519in}}%
\pgfpathclose%
\pgfusepath{fill}%
\end{pgfscope}%
\begin{pgfscope}%
\pgfpathrectangle{\pgfqpoint{0.499691in}{1.172519in}}{\pgfqpoint{7.362500in}{2.695000in}}%
\pgfusepath{clip}%
\pgfsetbuttcap%
\pgfsetmiterjoin%
\definecolor{currentfill}{rgb}{0.642098,0.766125,0.595747}%
\pgfsetfillcolor{currentfill}%
\pgfsetlinewidth{0.000000pt}%
\definecolor{currentstroke}{rgb}{0.000000,0.000000,0.000000}%
\pgfsetstrokecolor{currentstroke}%
\pgfsetstrokeopacity{0.000000}%
\pgfsetdash{}{0pt}%
\pgfpathmoveto{\pgfqpoint{3.668196in}{1.172519in}}%
\pgfpathlineto{\pgfqpoint{3.773374in}{1.172519in}}%
\pgfpathlineto{\pgfqpoint{3.773374in}{2.192355in}}%
\pgfpathlineto{\pgfqpoint{3.668196in}{2.192355in}}%
\pgfpathlineto{\pgfqpoint{3.668196in}{1.172519in}}%
\pgfpathclose%
\pgfusepath{fill}%
\end{pgfscope}%
\begin{pgfscope}%
\pgfpathrectangle{\pgfqpoint{0.499691in}{1.172519in}}{\pgfqpoint{7.362500in}{2.695000in}}%
\pgfusepath{clip}%
\pgfsetbuttcap%
\pgfsetmiterjoin%
\definecolor{currentfill}{rgb}{0.624593,0.757104,0.595421}%
\pgfsetfillcolor{currentfill}%
\pgfsetlinewidth{0.000000pt}%
\definecolor{currentstroke}{rgb}{0.000000,0.000000,0.000000}%
\pgfsetstrokecolor{currentstroke}%
\pgfsetstrokeopacity{0.000000}%
\pgfsetdash{}{0pt}%
\pgfpathmoveto{\pgfqpoint{3.799669in}{1.172519in}}%
\pgfpathlineto{\pgfqpoint{3.904847in}{1.172519in}}%
\pgfpathlineto{\pgfqpoint{3.904847in}{2.763403in}}%
\pgfpathlineto{\pgfqpoint{3.799669in}{2.763403in}}%
\pgfpathlineto{\pgfqpoint{3.799669in}{1.172519in}}%
\pgfpathclose%
\pgfusepath{fill}%
\end{pgfscope}%
\begin{pgfscope}%
\pgfpathrectangle{\pgfqpoint{0.499691in}{1.172519in}}{\pgfqpoint{7.362500in}{2.695000in}}%
\pgfusepath{clip}%
\pgfsetbuttcap%
\pgfsetmiterjoin%
\definecolor{currentfill}{rgb}{0.610607,0.749870,0.594868}%
\pgfsetfillcolor{currentfill}%
\pgfsetlinewidth{0.000000pt}%
\definecolor{currentstroke}{rgb}{0.000000,0.000000,0.000000}%
\pgfsetstrokecolor{currentstroke}%
\pgfsetstrokeopacity{0.000000}%
\pgfsetdash{}{0pt}%
\pgfpathmoveto{\pgfqpoint{3.931142in}{1.172519in}}%
\pgfpathlineto{\pgfqpoint{4.036321in}{1.172519in}}%
\pgfpathlineto{\pgfqpoint{4.036321in}{1.975378in}}%
\pgfpathlineto{\pgfqpoint{3.931142in}{1.975378in}}%
\pgfpathlineto{\pgfqpoint{3.931142in}{1.172519in}}%
\pgfpathclose%
\pgfusepath{fill}%
\end{pgfscope}%
\begin{pgfscope}%
\pgfpathrectangle{\pgfqpoint{0.499691in}{1.172519in}}{\pgfqpoint{7.362500in}{2.695000in}}%
\pgfusepath{clip}%
\pgfsetbuttcap%
\pgfsetmiterjoin%
\definecolor{currentfill}{rgb}{0.656222,0.773271,0.595544}%
\pgfsetfillcolor{currentfill}%
\pgfsetlinewidth{0.000000pt}%
\definecolor{currentstroke}{rgb}{0.000000,0.000000,0.000000}%
\pgfsetstrokecolor{currentstroke}%
\pgfsetstrokeopacity{0.000000}%
\pgfsetdash{}{0pt}%
\pgfpathmoveto{\pgfqpoint{4.062615in}{1.172519in}}%
\pgfpathlineto{\pgfqpoint{4.167794in}{1.172519in}}%
\pgfpathlineto{\pgfqpoint{4.167794in}{1.751022in}}%
\pgfpathlineto{\pgfqpoint{4.062615in}{1.751022in}}%
\pgfpathlineto{\pgfqpoint{4.062615in}{1.172519in}}%
\pgfpathclose%
\pgfusepath{fill}%
\end{pgfscope}%
\begin{pgfscope}%
\pgfpathrectangle{\pgfqpoint{0.499691in}{1.172519in}}{\pgfqpoint{7.362500in}{2.695000in}}%
\pgfusepath{clip}%
\pgfsetbuttcap%
\pgfsetmiterjoin%
\definecolor{currentfill}{rgb}{0.649123,0.769720,0.595746}%
\pgfsetfillcolor{currentfill}%
\pgfsetlinewidth{0.000000pt}%
\definecolor{currentstroke}{rgb}{0.000000,0.000000,0.000000}%
\pgfsetstrokecolor{currentstroke}%
\pgfsetstrokeopacity{0.000000}%
\pgfsetdash{}{0pt}%
\pgfpathmoveto{\pgfqpoint{4.194088in}{1.172519in}}%
\pgfpathlineto{\pgfqpoint{4.299267in}{1.172519in}}%
\pgfpathlineto{\pgfqpoint{4.299267in}{2.210152in}}%
\pgfpathlineto{\pgfqpoint{4.194088in}{2.210152in}}%
\pgfpathlineto{\pgfqpoint{4.194088in}{1.172519in}}%
\pgfpathclose%
\pgfusepath{fill}%
\end{pgfscope}%
\begin{pgfscope}%
\pgfpathrectangle{\pgfqpoint{0.499691in}{1.172519in}}{\pgfqpoint{7.362500in}{2.695000in}}%
\pgfusepath{clip}%
\pgfsetbuttcap%
\pgfsetmiterjoin%
\definecolor{currentfill}{rgb}{0.165098,0.366532,0.479263}%
\pgfsetfillcolor{currentfill}%
\pgfsetlinewidth{0.000000pt}%
\definecolor{currentstroke}{rgb}{0.000000,0.000000,0.000000}%
\pgfsetstrokecolor{currentstroke}%
\pgfsetstrokeopacity{0.000000}%
\pgfsetdash{}{0pt}%
\pgfpathmoveto{\pgfqpoint{4.325562in}{1.172519in}}%
\pgfpathlineto{\pgfqpoint{4.430740in}{1.172519in}}%
\pgfpathlineto{\pgfqpoint{4.430740in}{3.388047in}}%
\pgfpathlineto{\pgfqpoint{4.325562in}{3.388047in}}%
\pgfpathlineto{\pgfqpoint{4.325562in}{1.172519in}}%
\pgfpathclose%
\pgfusepath{fill}%
\end{pgfscope}%
\begin{pgfscope}%
\pgfpathrectangle{\pgfqpoint{0.499691in}{1.172519in}}{\pgfqpoint{7.362500in}{2.695000in}}%
\pgfusepath{clip}%
\pgfsetbuttcap%
\pgfsetmiterjoin%
\definecolor{currentfill}{rgb}{0.656222,0.773271,0.595544}%
\pgfsetfillcolor{currentfill}%
\pgfsetlinewidth{0.000000pt}%
\definecolor{currentstroke}{rgb}{0.000000,0.000000,0.000000}%
\pgfsetstrokecolor{currentstroke}%
\pgfsetstrokeopacity{0.000000}%
\pgfsetdash{}{0pt}%
\pgfpathmoveto{\pgfqpoint{4.457035in}{1.172519in}}%
\pgfpathlineto{\pgfqpoint{4.562213in}{1.172519in}}%
\pgfpathlineto{\pgfqpoint{4.562213in}{1.911718in}}%
\pgfpathlineto{\pgfqpoint{4.457035in}{1.911718in}}%
\pgfpathlineto{\pgfqpoint{4.457035in}{1.172519in}}%
\pgfpathclose%
\pgfusepath{fill}%
\end{pgfscope}%
\begin{pgfscope}%
\pgfpathrectangle{\pgfqpoint{0.499691in}{1.172519in}}{\pgfqpoint{7.362500in}{2.695000in}}%
\pgfusepath{clip}%
\pgfsetbuttcap%
\pgfsetmiterjoin%
\definecolor{currentfill}{rgb}{0.560465,0.721944,0.587075}%
\pgfsetfillcolor{currentfill}%
\pgfsetlinewidth{0.000000pt}%
\definecolor{currentstroke}{rgb}{0.000000,0.000000,0.000000}%
\pgfsetstrokecolor{currentstroke}%
\pgfsetstrokeopacity{0.000000}%
\pgfsetdash{}{0pt}%
\pgfpathmoveto{\pgfqpoint{4.588508in}{1.172519in}}%
\pgfpathlineto{\pgfqpoint{4.693687in}{1.172519in}}%
\pgfpathlineto{\pgfqpoint{4.693687in}{2.240671in}}%
\pgfpathlineto{\pgfqpoint{4.588508in}{2.240671in}}%
\pgfpathlineto{\pgfqpoint{4.588508in}{1.172519in}}%
\pgfpathclose%
\pgfusepath{fill}%
\end{pgfscope}%
\begin{pgfscope}%
\pgfpathrectangle{\pgfqpoint{0.499691in}{1.172519in}}{\pgfqpoint{7.362500in}{2.695000in}}%
\pgfusepath{clip}%
\pgfsetbuttcap%
\pgfsetmiterjoin%
\definecolor{currentfill}{rgb}{0.523411,0.700989,0.578806}%
\pgfsetfillcolor{currentfill}%
\pgfsetlinewidth{0.000000pt}%
\definecolor{currentstroke}{rgb}{0.000000,0.000000,0.000000}%
\pgfsetstrokecolor{currentstroke}%
\pgfsetstrokeopacity{0.000000}%
\pgfsetdash{}{0pt}%
\pgfpathmoveto{\pgfqpoint{4.719981in}{1.172519in}}%
\pgfpathlineto{\pgfqpoint{4.825160in}{1.172519in}}%
\pgfpathlineto{\pgfqpoint{4.825160in}{2.217872in}}%
\pgfpathlineto{\pgfqpoint{4.719981in}{2.217872in}}%
\pgfpathlineto{\pgfqpoint{4.719981in}{1.172519in}}%
\pgfpathclose%
\pgfusepath{fill}%
\end{pgfscope}%
\begin{pgfscope}%
\pgfpathrectangle{\pgfqpoint{0.499691in}{1.172519in}}{\pgfqpoint{7.362500in}{2.695000in}}%
\pgfusepath{clip}%
\pgfsetbuttcap%
\pgfsetmiterjoin%
\definecolor{currentfill}{rgb}{0.564547,0.724281,0.587944}%
\pgfsetfillcolor{currentfill}%
\pgfsetlinewidth{0.000000pt}%
\definecolor{currentstroke}{rgb}{0.000000,0.000000,0.000000}%
\pgfsetstrokecolor{currentstroke}%
\pgfsetstrokeopacity{0.000000}%
\pgfsetdash{}{0pt}%
\pgfpathmoveto{\pgfqpoint{4.851455in}{1.172519in}}%
\pgfpathlineto{\pgfqpoint{4.956633in}{1.172519in}}%
\pgfpathlineto{\pgfqpoint{4.956633in}{2.234577in}}%
\pgfpathlineto{\pgfqpoint{4.851455in}{2.234577in}}%
\pgfpathlineto{\pgfqpoint{4.851455in}{1.172519in}}%
\pgfpathclose%
\pgfusepath{fill}%
\end{pgfscope}%
\begin{pgfscope}%
\pgfpathrectangle{\pgfqpoint{0.499691in}{1.172519in}}{\pgfqpoint{7.362500in}{2.695000in}}%
\pgfusepath{clip}%
\pgfsetbuttcap%
\pgfsetmiterjoin%
\definecolor{currentfill}{rgb}{0.588929,0.738349,0.592927}%
\pgfsetfillcolor{currentfill}%
\pgfsetlinewidth{0.000000pt}%
\definecolor{currentstroke}{rgb}{0.000000,0.000000,0.000000}%
\pgfsetstrokecolor{currentstroke}%
\pgfsetstrokeopacity{0.000000}%
\pgfsetdash{}{0pt}%
\pgfpathmoveto{\pgfqpoint{4.982928in}{1.172519in}}%
\pgfpathlineto{\pgfqpoint{5.088106in}{1.172519in}}%
\pgfpathlineto{\pgfqpoint{5.088106in}{3.249165in}}%
\pgfpathlineto{\pgfqpoint{4.982928in}{3.249165in}}%
\pgfpathlineto{\pgfqpoint{4.982928in}{1.172519in}}%
\pgfpathclose%
\pgfusepath{fill}%
\end{pgfscope}%
\begin{pgfscope}%
\pgfpathrectangle{\pgfqpoint{0.499691in}{1.172519in}}{\pgfqpoint{7.362500in}{2.695000in}}%
\pgfusepath{clip}%
\pgfsetbuttcap%
\pgfsetmiterjoin%
\definecolor{currentfill}{rgb}{0.656222,0.773271,0.595544}%
\pgfsetfillcolor{currentfill}%
\pgfsetlinewidth{0.000000pt}%
\definecolor{currentstroke}{rgb}{0.000000,0.000000,0.000000}%
\pgfsetstrokecolor{currentstroke}%
\pgfsetstrokeopacity{0.000000}%
\pgfsetdash{}{0pt}%
\pgfpathmoveto{\pgfqpoint{5.114401in}{1.172519in}}%
\pgfpathlineto{\pgfqpoint{5.219580in}{1.172519in}}%
\pgfpathlineto{\pgfqpoint{5.219580in}{1.751022in}}%
\pgfpathlineto{\pgfqpoint{5.114401in}{1.751022in}}%
\pgfpathlineto{\pgfqpoint{5.114401in}{1.172519in}}%
\pgfpathclose%
\pgfusepath{fill}%
\end{pgfscope}%
\begin{pgfscope}%
\pgfpathrectangle{\pgfqpoint{0.499691in}{1.172519in}}{\pgfqpoint{7.362500in}{2.695000in}}%
\pgfusepath{clip}%
\pgfsetbuttcap%
\pgfsetmiterjoin%
\definecolor{currentfill}{rgb}{0.656222,0.773271,0.595544}%
\pgfsetfillcolor{currentfill}%
\pgfsetlinewidth{0.000000pt}%
\definecolor{currentstroke}{rgb}{0.000000,0.000000,0.000000}%
\pgfsetstrokecolor{currentstroke}%
\pgfsetstrokeopacity{0.000000}%
\pgfsetdash{}{0pt}%
\pgfpathmoveto{\pgfqpoint{5.245874in}{1.172519in}}%
\pgfpathlineto{\pgfqpoint{5.351053in}{1.172519in}}%
\pgfpathlineto{\pgfqpoint{5.351053in}{2.573106in}}%
\pgfpathlineto{\pgfqpoint{5.245874in}{2.573106in}}%
\pgfpathlineto{\pgfqpoint{5.245874in}{1.172519in}}%
\pgfpathclose%
\pgfusepath{fill}%
\end{pgfscope}%
\begin{pgfscope}%
\pgfpathrectangle{\pgfqpoint{0.499691in}{1.172519in}}{\pgfqpoint{7.362500in}{2.695000in}}%
\pgfusepath{clip}%
\pgfsetbuttcap%
\pgfsetmiterjoin%
\definecolor{currentfill}{rgb}{0.652651,0.771509,0.595697}%
\pgfsetfillcolor{currentfill}%
\pgfsetlinewidth{0.000000pt}%
\definecolor{currentstroke}{rgb}{0.000000,0.000000,0.000000}%
\pgfsetstrokecolor{currentstroke}%
\pgfsetstrokeopacity{0.000000}%
\pgfsetdash{}{0pt}%
\pgfpathmoveto{\pgfqpoint{5.377347in}{1.172519in}}%
\pgfpathlineto{\pgfqpoint{5.482526in}{1.172519in}}%
\pgfpathlineto{\pgfqpoint{5.482526in}{1.964155in}}%
\pgfpathlineto{\pgfqpoint{5.377347in}{1.964155in}}%
\pgfpathlineto{\pgfqpoint{5.377347in}{1.172519in}}%
\pgfpathclose%
\pgfusepath{fill}%
\end{pgfscope}%
\begin{pgfscope}%
\pgfpathrectangle{\pgfqpoint{0.499691in}{1.172519in}}{\pgfqpoint{7.362500in}{2.695000in}}%
\pgfusepath{clip}%
\pgfsetbuttcap%
\pgfsetmiterjoin%
\definecolor{currentfill}{rgb}{0.656222,0.773271,0.595544}%
\pgfsetfillcolor{currentfill}%
\pgfsetlinewidth{0.000000pt}%
\definecolor{currentstroke}{rgb}{0.000000,0.000000,0.000000}%
\pgfsetstrokecolor{currentstroke}%
\pgfsetstrokeopacity{0.000000}%
\pgfsetdash{}{0pt}%
\pgfpathmoveto{\pgfqpoint{5.508821in}{1.172519in}}%
\pgfpathlineto{\pgfqpoint{5.613999in}{1.172519in}}%
\pgfpathlineto{\pgfqpoint{5.613999in}{3.428682in}}%
\pgfpathlineto{\pgfqpoint{5.508821in}{3.428682in}}%
\pgfpathlineto{\pgfqpoint{5.508821in}{1.172519in}}%
\pgfpathclose%
\pgfusepath{fill}%
\end{pgfscope}%
\begin{pgfscope}%
\pgfpathrectangle{\pgfqpoint{0.499691in}{1.172519in}}{\pgfqpoint{7.362500in}{2.695000in}}%
\pgfusepath{clip}%
\pgfsetbuttcap%
\pgfsetmiterjoin%
\definecolor{currentfill}{rgb}{0.584872,0.735999,0.592127}%
\pgfsetfillcolor{currentfill}%
\pgfsetlinewidth{0.000000pt}%
\definecolor{currentstroke}{rgb}{0.000000,0.000000,0.000000}%
\pgfsetstrokecolor{currentstroke}%
\pgfsetstrokeopacity{0.000000}%
\pgfsetdash{}{0pt}%
\pgfpathmoveto{\pgfqpoint{5.640294in}{1.172519in}}%
\pgfpathlineto{\pgfqpoint{5.745472in}{1.172519in}}%
\pgfpathlineto{\pgfqpoint{5.745472in}{2.334408in}}%
\pgfpathlineto{\pgfqpoint{5.640294in}{2.334408in}}%
\pgfpathlineto{\pgfqpoint{5.640294in}{1.172519in}}%
\pgfpathclose%
\pgfusepath{fill}%
\end{pgfscope}%
\begin{pgfscope}%
\pgfpathrectangle{\pgfqpoint{0.499691in}{1.172519in}}{\pgfqpoint{7.362500in}{2.695000in}}%
\pgfusepath{clip}%
\pgfsetbuttcap%
\pgfsetmiterjoin%
\definecolor{currentfill}{rgb}{0.617599,0.753489,0.595178}%
\pgfsetfillcolor{currentfill}%
\pgfsetlinewidth{0.000000pt}%
\definecolor{currentstroke}{rgb}{0.000000,0.000000,0.000000}%
\pgfsetstrokecolor{currentstroke}%
\pgfsetstrokeopacity{0.000000}%
\pgfsetdash{}{0pt}%
\pgfpathmoveto{\pgfqpoint{5.771767in}{1.172519in}}%
\pgfpathlineto{\pgfqpoint{5.876946in}{1.172519in}}%
\pgfpathlineto{\pgfqpoint{5.876946in}{1.922190in}}%
\pgfpathlineto{\pgfqpoint{5.771767in}{1.922190in}}%
\pgfpathlineto{\pgfqpoint{5.771767in}{1.172519in}}%
\pgfpathclose%
\pgfusepath{fill}%
\end{pgfscope}%
\begin{pgfscope}%
\pgfpathrectangle{\pgfqpoint{0.499691in}{1.172519in}}{\pgfqpoint{7.362500in}{2.695000in}}%
\pgfusepath{clip}%
\pgfsetbuttcap%
\pgfsetmiterjoin%
\definecolor{currentfill}{rgb}{0.649123,0.769720,0.595746}%
\pgfsetfillcolor{currentfill}%
\pgfsetlinewidth{0.000000pt}%
\definecolor{currentstroke}{rgb}{0.000000,0.000000,0.000000}%
\pgfsetstrokecolor{currentstroke}%
\pgfsetstrokeopacity{0.000000}%
\pgfsetdash{}{0pt}%
\pgfpathmoveto{\pgfqpoint{5.903240in}{1.172519in}}%
\pgfpathlineto{\pgfqpoint{6.008419in}{1.172519in}}%
\pgfpathlineto{\pgfqpoint{6.008419in}{2.154883in}}%
\pgfpathlineto{\pgfqpoint{5.903240in}{2.154883in}}%
\pgfpathlineto{\pgfqpoint{5.903240in}{1.172519in}}%
\pgfpathclose%
\pgfusepath{fill}%
\end{pgfscope}%
\begin{pgfscope}%
\pgfpathrectangle{\pgfqpoint{0.499691in}{1.172519in}}{\pgfqpoint{7.362500in}{2.695000in}}%
\pgfusepath{clip}%
\pgfsetbuttcap%
\pgfsetmiterjoin%
\definecolor{currentfill}{rgb}{0.624593,0.757104,0.595421}%
\pgfsetfillcolor{currentfill}%
\pgfsetlinewidth{0.000000pt}%
\definecolor{currentstroke}{rgb}{0.000000,0.000000,0.000000}%
\pgfsetstrokecolor{currentstroke}%
\pgfsetstrokeopacity{0.000000}%
\pgfsetdash{}{0pt}%
\pgfpathmoveto{\pgfqpoint{6.034713in}{1.172519in}}%
\pgfpathlineto{\pgfqpoint{6.139892in}{1.172519in}}%
\pgfpathlineto{\pgfqpoint{6.139892in}{2.063802in}}%
\pgfpathlineto{\pgfqpoint{6.034713in}{2.063802in}}%
\pgfpathlineto{\pgfqpoint{6.034713in}{1.172519in}}%
\pgfpathclose%
\pgfusepath{fill}%
\end{pgfscope}%
\begin{pgfscope}%
\pgfpathrectangle{\pgfqpoint{0.499691in}{1.172519in}}{\pgfqpoint{7.362500in}{2.695000in}}%
\pgfusepath{clip}%
\pgfsetbuttcap%
\pgfsetmiterjoin%
\definecolor{currentfill}{rgb}{0.638590,0.764325,0.595722}%
\pgfsetfillcolor{currentfill}%
\pgfsetlinewidth{0.000000pt}%
\definecolor{currentstroke}{rgb}{0.000000,0.000000,0.000000}%
\pgfsetstrokecolor{currentstroke}%
\pgfsetstrokeopacity{0.000000}%
\pgfsetdash{}{0pt}%
\pgfpathmoveto{\pgfqpoint{6.166187in}{1.172519in}}%
\pgfpathlineto{\pgfqpoint{6.271365in}{1.172519in}}%
\pgfpathlineto{\pgfqpoint{6.271365in}{2.216537in}}%
\pgfpathlineto{\pgfqpoint{6.166187in}{2.216537in}}%
\pgfpathlineto{\pgfqpoint{6.166187in}{1.172519in}}%
\pgfpathclose%
\pgfusepath{fill}%
\end{pgfscope}%
\begin{pgfscope}%
\pgfpathrectangle{\pgfqpoint{0.499691in}{1.172519in}}{\pgfqpoint{7.362500in}{2.695000in}}%
\pgfusepath{clip}%
\pgfsetbuttcap%
\pgfsetmiterjoin%
\definecolor{currentfill}{rgb}{0.515241,0.696325,0.576964}%
\pgfsetfillcolor{currentfill}%
\pgfsetlinewidth{0.000000pt}%
\definecolor{currentstroke}{rgb}{0.000000,0.000000,0.000000}%
\pgfsetstrokecolor{currentstroke}%
\pgfsetstrokeopacity{0.000000}%
\pgfsetdash{}{0pt}%
\pgfpathmoveto{\pgfqpoint{6.297660in}{1.172519in}}%
\pgfpathlineto{\pgfqpoint{6.402838in}{1.172519in}}%
\pgfpathlineto{\pgfqpoint{6.402838in}{2.789775in}}%
\pgfpathlineto{\pgfqpoint{6.297660in}{2.789775in}}%
\pgfpathlineto{\pgfqpoint{6.297660in}{1.172519in}}%
\pgfpathclose%
\pgfusepath{fill}%
\end{pgfscope}%
\begin{pgfscope}%
\pgfpathrectangle{\pgfqpoint{0.499691in}{1.172519in}}{\pgfqpoint{7.362500in}{2.695000in}}%
\pgfusepath{clip}%
\pgfsetbuttcap%
\pgfsetmiterjoin%
\definecolor{currentfill}{rgb}{0.568623,0.726620,0.588802}%
\pgfsetfillcolor{currentfill}%
\pgfsetlinewidth{0.000000pt}%
\definecolor{currentstroke}{rgb}{0.000000,0.000000,0.000000}%
\pgfsetstrokecolor{currentstroke}%
\pgfsetstrokeopacity{0.000000}%
\pgfsetdash{}{0pt}%
\pgfpathmoveto{\pgfqpoint{6.429133in}{1.172519in}}%
\pgfpathlineto{\pgfqpoint{6.534312in}{1.172519in}}%
\pgfpathlineto{\pgfqpoint{6.534312in}{2.289773in}}%
\pgfpathlineto{\pgfqpoint{6.429133in}{2.289773in}}%
\pgfpathlineto{\pgfqpoint{6.429133in}{1.172519in}}%
\pgfpathclose%
\pgfusepath{fill}%
\end{pgfscope}%
\begin{pgfscope}%
\pgfpathrectangle{\pgfqpoint{0.499691in}{1.172519in}}{\pgfqpoint{7.362500in}{2.695000in}}%
\pgfusepath{clip}%
\pgfsetbuttcap%
\pgfsetmiterjoin%
\definecolor{currentfill}{rgb}{0.652651,0.771509,0.595697}%
\pgfsetfillcolor{currentfill}%
\pgfsetlinewidth{0.000000pt}%
\definecolor{currentstroke}{rgb}{0.000000,0.000000,0.000000}%
\pgfsetstrokecolor{currentstroke}%
\pgfsetstrokeopacity{0.000000}%
\pgfsetdash{}{0pt}%
\pgfpathmoveto{\pgfqpoint{6.560606in}{1.172519in}}%
\pgfpathlineto{\pgfqpoint{6.665785in}{1.172519in}}%
\pgfpathlineto{\pgfqpoint{6.665785in}{3.279924in}}%
\pgfpathlineto{\pgfqpoint{6.560606in}{3.279924in}}%
\pgfpathlineto{\pgfqpoint{6.560606in}{1.172519in}}%
\pgfpathclose%
\pgfusepath{fill}%
\end{pgfscope}%
\begin{pgfscope}%
\pgfpathrectangle{\pgfqpoint{0.499691in}{1.172519in}}{\pgfqpoint{7.362500in}{2.695000in}}%
\pgfusepath{clip}%
\pgfsetbuttcap%
\pgfsetmiterjoin%
\definecolor{currentfill}{rgb}{0.552281,0.717276,0.585304}%
\pgfsetfillcolor{currentfill}%
\pgfsetlinewidth{0.000000pt}%
\definecolor{currentstroke}{rgb}{0.000000,0.000000,0.000000}%
\pgfsetstrokecolor{currentstroke}%
\pgfsetstrokeopacity{0.000000}%
\pgfsetdash{}{0pt}%
\pgfpathmoveto{\pgfqpoint{6.692080in}{1.172519in}}%
\pgfpathlineto{\pgfqpoint{6.797258in}{1.172519in}}%
\pgfpathlineto{\pgfqpoint{6.797258in}{3.178818in}}%
\pgfpathlineto{\pgfqpoint{6.692080in}{3.178818in}}%
\pgfpathlineto{\pgfqpoint{6.692080in}{1.172519in}}%
\pgfpathclose%
\pgfusepath{fill}%
\end{pgfscope}%
\begin{pgfscope}%
\pgfpathrectangle{\pgfqpoint{0.499691in}{1.172519in}}{\pgfqpoint{7.362500in}{2.695000in}}%
\pgfusepath{clip}%
\pgfsetbuttcap%
\pgfsetmiterjoin%
\definecolor{currentfill}{rgb}{0.642098,0.766125,0.595747}%
\pgfsetfillcolor{currentfill}%
\pgfsetlinewidth{0.000000pt}%
\definecolor{currentstroke}{rgb}{0.000000,0.000000,0.000000}%
\pgfsetstrokecolor{currentstroke}%
\pgfsetstrokeopacity{0.000000}%
\pgfsetdash{}{0pt}%
\pgfpathmoveto{\pgfqpoint{6.823553in}{1.172519in}}%
\pgfpathlineto{\pgfqpoint{6.928731in}{1.172519in}}%
\pgfpathlineto{\pgfqpoint{6.928731in}{2.281317in}}%
\pgfpathlineto{\pgfqpoint{6.823553in}{2.281317in}}%
\pgfpathlineto{\pgfqpoint{6.823553in}{1.172519in}}%
\pgfpathclose%
\pgfusepath{fill}%
\end{pgfscope}%
\begin{pgfscope}%
\pgfpathrectangle{\pgfqpoint{0.499691in}{1.172519in}}{\pgfqpoint{7.362500in}{2.695000in}}%
\pgfusepath{clip}%
\pgfsetbuttcap%
\pgfsetmiterjoin%
\definecolor{currentfill}{rgb}{0.656222,0.773271,0.595544}%
\pgfsetfillcolor{currentfill}%
\pgfsetlinewidth{0.000000pt}%
\definecolor{currentstroke}{rgb}{0.000000,0.000000,0.000000}%
\pgfsetstrokecolor{currentstroke}%
\pgfsetstrokeopacity{0.000000}%
\pgfsetdash{}{0pt}%
\pgfpathmoveto{\pgfqpoint{6.955026in}{1.172519in}}%
\pgfpathlineto{\pgfqpoint{7.060205in}{1.172519in}}%
\pgfpathlineto{\pgfqpoint{7.060205in}{2.136691in}}%
\pgfpathlineto{\pgfqpoint{6.955026in}{2.136691in}}%
\pgfpathlineto{\pgfqpoint{6.955026in}{1.172519in}}%
\pgfpathclose%
\pgfusepath{fill}%
\end{pgfscope}%
\begin{pgfscope}%
\pgfpathrectangle{\pgfqpoint{0.499691in}{1.172519in}}{\pgfqpoint{7.362500in}{2.695000in}}%
\pgfusepath{clip}%
\pgfsetbuttcap%
\pgfsetmiterjoin%
\definecolor{currentfill}{rgb}{0.645609,0.767924,0.595756}%
\pgfsetfillcolor{currentfill}%
\pgfsetlinewidth{0.000000pt}%
\definecolor{currentstroke}{rgb}{0.000000,0.000000,0.000000}%
\pgfsetstrokecolor{currentstroke}%
\pgfsetstrokeopacity{0.000000}%
\pgfsetdash{}{0pt}%
\pgfpathmoveto{\pgfqpoint{7.086499in}{1.172519in}}%
\pgfpathlineto{\pgfqpoint{7.191678in}{1.172519in}}%
\pgfpathlineto{\pgfqpoint{7.191678in}{2.080448in}}%
\pgfpathlineto{\pgfqpoint{7.086499in}{2.080448in}}%
\pgfpathlineto{\pgfqpoint{7.086499in}{1.172519in}}%
\pgfpathclose%
\pgfusepath{fill}%
\end{pgfscope}%
\begin{pgfscope}%
\pgfpathrectangle{\pgfqpoint{0.499691in}{1.172519in}}{\pgfqpoint{7.362500in}{2.695000in}}%
\pgfusepath{clip}%
\pgfsetbuttcap%
\pgfsetmiterjoin%
\definecolor{currentfill}{rgb}{0.642098,0.766125,0.595747}%
\pgfsetfillcolor{currentfill}%
\pgfsetlinewidth{0.000000pt}%
\definecolor{currentstroke}{rgb}{0.000000,0.000000,0.000000}%
\pgfsetstrokecolor{currentstroke}%
\pgfsetstrokeopacity{0.000000}%
\pgfsetdash{}{0pt}%
\pgfpathmoveto{\pgfqpoint{7.217972in}{1.172519in}}%
\pgfpathlineto{\pgfqpoint{7.323151in}{1.172519in}}%
\pgfpathlineto{\pgfqpoint{7.323151in}{1.998091in}}%
\pgfpathlineto{\pgfqpoint{7.217972in}{1.998091in}}%
\pgfpathlineto{\pgfqpoint{7.217972in}{1.172519in}}%
\pgfpathclose%
\pgfusepath{fill}%
\end{pgfscope}%
\begin{pgfscope}%
\pgfpathrectangle{\pgfqpoint{0.499691in}{1.172519in}}{\pgfqpoint{7.362500in}{2.695000in}}%
\pgfusepath{clip}%
\pgfsetbuttcap%
\pgfsetmiterjoin%
\definecolor{currentfill}{rgb}{0.652651,0.771509,0.595697}%
\pgfsetfillcolor{currentfill}%
\pgfsetlinewidth{0.000000pt}%
\definecolor{currentstroke}{rgb}{0.000000,0.000000,0.000000}%
\pgfsetstrokecolor{currentstroke}%
\pgfsetstrokeopacity{0.000000}%
\pgfsetdash{}{0pt}%
\pgfpathmoveto{\pgfqpoint{7.349446in}{1.172519in}}%
\pgfpathlineto{\pgfqpoint{7.454624in}{1.172519in}}%
\pgfpathlineto{\pgfqpoint{7.454624in}{2.352666in}}%
\pgfpathlineto{\pgfqpoint{7.349446in}{2.352666in}}%
\pgfpathlineto{\pgfqpoint{7.349446in}{1.172519in}}%
\pgfpathclose%
\pgfusepath{fill}%
\end{pgfscope}%
\begin{pgfscope}%
\pgfpathrectangle{\pgfqpoint{0.499691in}{1.172519in}}{\pgfqpoint{7.362500in}{2.695000in}}%
\pgfusepath{clip}%
\pgfsetbuttcap%
\pgfsetmiterjoin%
\definecolor{currentfill}{rgb}{0.652651,0.771509,0.595697}%
\pgfsetfillcolor{currentfill}%
\pgfsetlinewidth{0.000000pt}%
\definecolor{currentstroke}{rgb}{0.000000,0.000000,0.000000}%
\pgfsetstrokecolor{currentstroke}%
\pgfsetstrokeopacity{0.000000}%
\pgfsetdash{}{0pt}%
\pgfpathmoveto{\pgfqpoint{7.480919in}{1.172519in}}%
\pgfpathlineto{\pgfqpoint{7.586097in}{1.172519in}}%
\pgfpathlineto{\pgfqpoint{7.586097in}{2.158117in}}%
\pgfpathlineto{\pgfqpoint{7.480919in}{2.158117in}}%
\pgfpathlineto{\pgfqpoint{7.480919in}{1.172519in}}%
\pgfpathclose%
\pgfusepath{fill}%
\end{pgfscope}%
\begin{pgfscope}%
\pgfpathrectangle{\pgfqpoint{0.499691in}{1.172519in}}{\pgfqpoint{7.362500in}{2.695000in}}%
\pgfusepath{clip}%
\pgfsetbuttcap%
\pgfsetmiterjoin%
\definecolor{currentfill}{rgb}{0.638590,0.764325,0.595722}%
\pgfsetfillcolor{currentfill}%
\pgfsetlinewidth{0.000000pt}%
\definecolor{currentstroke}{rgb}{0.000000,0.000000,0.000000}%
\pgfsetstrokecolor{currentstroke}%
\pgfsetstrokeopacity{0.000000}%
\pgfsetdash{}{0pt}%
\pgfpathmoveto{\pgfqpoint{7.612392in}{1.172519in}}%
\pgfpathlineto{\pgfqpoint{7.717571in}{1.172519in}}%
\pgfpathlineto{\pgfqpoint{7.717571in}{1.985777in}}%
\pgfpathlineto{\pgfqpoint{7.612392in}{1.985777in}}%
\pgfpathlineto{\pgfqpoint{7.612392in}{1.172519in}}%
\pgfpathclose%
\pgfusepath{fill}%
\end{pgfscope}%
\begin{pgfscope}%
\pgfpathrectangle{\pgfqpoint{0.499691in}{1.172519in}}{\pgfqpoint{7.362500in}{2.695000in}}%
\pgfusepath{clip}%
\pgfsetbuttcap%
\pgfsetmiterjoin%
\definecolor{currentfill}{rgb}{0.621095,0.755296,0.595308}%
\pgfsetfillcolor{currentfill}%
\pgfsetlinewidth{0.000000pt}%
\definecolor{currentstroke}{rgb}{0.000000,0.000000,0.000000}%
\pgfsetstrokecolor{currentstroke}%
\pgfsetstrokeopacity{0.000000}%
\pgfsetdash{}{0pt}%
\pgfpathmoveto{\pgfqpoint{7.743865in}{1.172519in}}%
\pgfpathlineto{\pgfqpoint{7.849044in}{1.172519in}}%
\pgfpathlineto{\pgfqpoint{7.849044in}{3.739185in}}%
\pgfpathlineto{\pgfqpoint{7.743865in}{3.739185in}}%
\pgfpathlineto{\pgfqpoint{7.743865in}{1.172519in}}%
\pgfpathclose%
\pgfusepath{fill}%
\end{pgfscope}%
\begin{pgfscope}%
\pgfsetbuttcap%
\pgfsetroundjoin%
\definecolor{currentfill}{rgb}{0.000000,0.000000,0.000000}%
\pgfsetfillcolor{currentfill}%
\pgfsetlinewidth{0.803000pt}%
\definecolor{currentstroke}{rgb}{0.000000,0.000000,0.000000}%
\pgfsetstrokecolor{currentstroke}%
\pgfsetdash{}{0pt}%
\pgfsys@defobject{currentmarker}{\pgfqpoint{0.000000in}{-0.048611in}}{\pgfqpoint{0.000000in}{0.000000in}}{%
\pgfpathmoveto{\pgfqpoint{0.000000in}{0.000000in}}%
\pgfpathlineto{\pgfqpoint{0.000000in}{-0.048611in}}%
\pgfusepath{stroke,fill}%
}%
\begin{pgfscope}%
\pgfsys@transformshift{0.565428in}{1.172519in}%
\pgfsys@useobject{currentmarker}{}%
\end{pgfscope}%
\end{pgfscope}%
\begin{pgfscope}%
\definecolor{textcolor}{rgb}{0.000000,0.000000,0.000000}%
\pgfsetstrokecolor{textcolor}%
\pgfsetfillcolor{textcolor}%
\pgftext[x=0.586261in, y=0.835021in, left, base,rotate=90.000000]{\color{textcolor}\rmfamily\fontsize{6.000000}{7.200000}\selectfont AMV}%
\end{pgfscope}%
\begin{pgfscope}%
\pgfsetbuttcap%
\pgfsetroundjoin%
\definecolor{currentfill}{rgb}{0.000000,0.000000,0.000000}%
\pgfsetfillcolor{currentfill}%
\pgfsetlinewidth{0.803000pt}%
\definecolor{currentstroke}{rgb}{0.000000,0.000000,0.000000}%
\pgfsetstrokecolor{currentstroke}%
\pgfsetdash{}{0pt}%
\pgfsys@defobject{currentmarker}{\pgfqpoint{0.000000in}{-0.048611in}}{\pgfqpoint{0.000000in}{0.000000in}}{%
\pgfpathmoveto{\pgfqpoint{0.000000in}{0.000000in}}%
\pgfpathlineto{\pgfqpoint{0.000000in}{-0.048611in}}%
\pgfusepath{stroke,fill}%
}%
\begin{pgfscope}%
\pgfsys@transformshift{0.696901in}{1.172519in}%
\pgfsys@useobject{currentmarker}{}%
\end{pgfscope}%
\end{pgfscope}%
\begin{pgfscope}%
\definecolor{textcolor}{rgb}{0.000000,0.000000,0.000000}%
\pgfsetstrokecolor{textcolor}%
\pgfsetfillcolor{textcolor}%
\pgftext[x=0.717734in, y=0.759097in, left, base,rotate=90.000000]{\color{textcolor}\rmfamily\fontsize{6.000000}{7.200000}\selectfont APRIL}%
\end{pgfscope}%
\begin{pgfscope}%
\pgfsetbuttcap%
\pgfsetroundjoin%
\definecolor{currentfill}{rgb}{0.000000,0.000000,0.000000}%
\pgfsetfillcolor{currentfill}%
\pgfsetlinewidth{0.803000pt}%
\definecolor{currentstroke}{rgb}{0.000000,0.000000,0.000000}%
\pgfsetstrokecolor{currentstroke}%
\pgfsetdash{}{0pt}%
\pgfsys@defobject{currentmarker}{\pgfqpoint{0.000000in}{-0.048611in}}{\pgfqpoint{0.000000in}{0.000000in}}{%
\pgfpathmoveto{\pgfqpoint{0.000000in}{0.000000in}}%
\pgfpathlineto{\pgfqpoint{0.000000in}{-0.048611in}}%
\pgfusepath{stroke,fill}%
}%
\begin{pgfscope}%
\pgfsys@transformshift{0.828374in}{1.172519in}%
\pgfsys@useobject{currentmarker}{}%
\end{pgfscope}%
\end{pgfscope}%
\begin{pgfscope}%
\definecolor{textcolor}{rgb}{0.000000,0.000000,0.000000}%
\pgfsetstrokecolor{textcolor}%
\pgfsetfillcolor{textcolor}%
\pgftext[x=0.849207in, y=0.491508in, left, base,rotate=90.000000]{\color{textcolor}\rmfamily\fontsize{6.000000}{7.200000}\selectfont APRIL Moto}%
\end{pgfscope}%
\begin{pgfscope}%
\pgfsetbuttcap%
\pgfsetroundjoin%
\definecolor{currentfill}{rgb}{0.000000,0.000000,0.000000}%
\pgfsetfillcolor{currentfill}%
\pgfsetlinewidth{0.803000pt}%
\definecolor{currentstroke}{rgb}{0.000000,0.000000,0.000000}%
\pgfsetstrokecolor{currentstroke}%
\pgfsetdash{}{0pt}%
\pgfsys@defobject{currentmarker}{\pgfqpoint{0.000000in}{-0.048611in}}{\pgfqpoint{0.000000in}{0.000000in}}{%
\pgfpathmoveto{\pgfqpoint{0.000000in}{0.000000in}}%
\pgfpathlineto{\pgfqpoint{0.000000in}{-0.048611in}}%
\pgfusepath{stroke,fill}%
}%
\begin{pgfscope}%
\pgfsys@transformshift{0.959847in}{1.172519in}%
\pgfsys@useobject{currentmarker}{}%
\end{pgfscope}%
\end{pgfscope}%
\begin{pgfscope}%
\definecolor{textcolor}{rgb}{0.000000,0.000000,0.000000}%
\pgfsetstrokecolor{textcolor}%
\pgfsetfillcolor{textcolor}%
\pgftext[x=0.980681in, y=0.851225in, left, base,rotate=90.000000]{\color{textcolor}\rmfamily\fontsize{6.000000}{7.200000}\selectfont AXA}%
\end{pgfscope}%
\begin{pgfscope}%
\pgfsetbuttcap%
\pgfsetroundjoin%
\definecolor{currentfill}{rgb}{0.000000,0.000000,0.000000}%
\pgfsetfillcolor{currentfill}%
\pgfsetlinewidth{0.803000pt}%
\definecolor{currentstroke}{rgb}{0.000000,0.000000,0.000000}%
\pgfsetstrokecolor{currentstroke}%
\pgfsetdash{}{0pt}%
\pgfsys@defobject{currentmarker}{\pgfqpoint{0.000000in}{-0.048611in}}{\pgfqpoint{0.000000in}{0.000000in}}{%
\pgfpathmoveto{\pgfqpoint{0.000000in}{0.000000in}}%
\pgfpathlineto{\pgfqpoint{0.000000in}{-0.048611in}}%
\pgfusepath{stroke,fill}%
}%
\begin{pgfscope}%
\pgfsys@transformshift{1.091321in}{1.172519in}%
\pgfsys@useobject{currentmarker}{}%
\end{pgfscope}%
\end{pgfscope}%
\begin{pgfscope}%
\definecolor{textcolor}{rgb}{0.000000,0.000000,0.000000}%
\pgfsetstrokecolor{textcolor}%
\pgfsetfillcolor{textcolor}%
\pgftext[x=1.112154in, y=0.263038in, left, base,rotate=90.000000]{\color{textcolor}\rmfamily\fontsize{6.000000}{7.200000}\selectfont Active Assurances}%
\end{pgfscope}%
\begin{pgfscope}%
\pgfsetbuttcap%
\pgfsetroundjoin%
\definecolor{currentfill}{rgb}{0.000000,0.000000,0.000000}%
\pgfsetfillcolor{currentfill}%
\pgfsetlinewidth{0.803000pt}%
\definecolor{currentstroke}{rgb}{0.000000,0.000000,0.000000}%
\pgfsetstrokecolor{currentstroke}%
\pgfsetdash{}{0pt}%
\pgfsys@defobject{currentmarker}{\pgfqpoint{0.000000in}{-0.048611in}}{\pgfqpoint{0.000000in}{0.000000in}}{%
\pgfpathmoveto{\pgfqpoint{0.000000in}{0.000000in}}%
\pgfpathlineto{\pgfqpoint{0.000000in}{-0.048611in}}%
\pgfusepath{stroke,fill}%
}%
\begin{pgfscope}%
\pgfsys@transformshift{1.222794in}{1.172519in}%
\pgfsys@useobject{currentmarker}{}%
\end{pgfscope}%
\end{pgfscope}%
\begin{pgfscope}%
\definecolor{textcolor}{rgb}{0.000000,0.000000,0.000000}%
\pgfsetstrokecolor{textcolor}%
\pgfsetfillcolor{textcolor}%
\pgftext[x=1.243627in, y=0.882937in, left, base,rotate=90.000000]{\color{textcolor}\rmfamily\fontsize{6.000000}{7.200000}\selectfont Afer}%
\end{pgfscope}%
\begin{pgfscope}%
\pgfsetbuttcap%
\pgfsetroundjoin%
\definecolor{currentfill}{rgb}{0.000000,0.000000,0.000000}%
\pgfsetfillcolor{currentfill}%
\pgfsetlinewidth{0.803000pt}%
\definecolor{currentstroke}{rgb}{0.000000,0.000000,0.000000}%
\pgfsetstrokecolor{currentstroke}%
\pgfsetdash{}{0pt}%
\pgfsys@defobject{currentmarker}{\pgfqpoint{0.000000in}{-0.048611in}}{\pgfqpoint{0.000000in}{0.000000in}}{%
\pgfpathmoveto{\pgfqpoint{0.000000in}{0.000000in}}%
\pgfpathlineto{\pgfqpoint{0.000000in}{-0.048611in}}%
\pgfusepath{stroke,fill}%
}%
\begin{pgfscope}%
\pgfsys@transformshift{1.354267in}{1.172519in}%
\pgfsys@useobject{currentmarker}{}%
\end{pgfscope}%
\end{pgfscope}%
\begin{pgfscope}%
\definecolor{textcolor}{rgb}{0.000000,0.000000,0.000000}%
\pgfsetstrokecolor{textcolor}%
\pgfsetfillcolor{textcolor}%
\pgftext[x=1.375100in, y=0.701922in, left, base,rotate=90.000000]{\color{textcolor}\rmfamily\fontsize{6.000000}{7.200000}\selectfont Afi Esca}%
\end{pgfscope}%
\begin{pgfscope}%
\pgfsetbuttcap%
\pgfsetroundjoin%
\definecolor{currentfill}{rgb}{0.000000,0.000000,0.000000}%
\pgfsetfillcolor{currentfill}%
\pgfsetlinewidth{0.803000pt}%
\definecolor{currentstroke}{rgb}{0.000000,0.000000,0.000000}%
\pgfsetstrokecolor{currentstroke}%
\pgfsetdash{}{0pt}%
\pgfsys@defobject{currentmarker}{\pgfqpoint{0.000000in}{-0.048611in}}{\pgfqpoint{0.000000in}{0.000000in}}{%
\pgfpathmoveto{\pgfqpoint{0.000000in}{0.000000in}}%
\pgfpathlineto{\pgfqpoint{0.000000in}{-0.048611in}}%
\pgfusepath{stroke,fill}%
}%
\begin{pgfscope}%
\pgfsys@transformshift{1.485740in}{1.172519in}%
\pgfsys@useobject{currentmarker}{}%
\end{pgfscope}%
\end{pgfscope}%
\begin{pgfscope}%
\definecolor{textcolor}{rgb}{0.000000,0.000000,0.000000}%
\pgfsetstrokecolor{textcolor}%
\pgfsetfillcolor{textcolor}%
\pgftext[x=1.506574in, y=0.265893in, left, base,rotate=90.000000]{\color{textcolor}\rmfamily\fontsize{6.000000}{7.200000}\selectfont Ag2r La Mondiale}%
\end{pgfscope}%
\begin{pgfscope}%
\pgfsetbuttcap%
\pgfsetroundjoin%
\definecolor{currentfill}{rgb}{0.000000,0.000000,0.000000}%
\pgfsetfillcolor{currentfill}%
\pgfsetlinewidth{0.803000pt}%
\definecolor{currentstroke}{rgb}{0.000000,0.000000,0.000000}%
\pgfsetstrokecolor{currentstroke}%
\pgfsetdash{}{0pt}%
\pgfsys@defobject{currentmarker}{\pgfqpoint{0.000000in}{-0.048611in}}{\pgfqpoint{0.000000in}{0.000000in}}{%
\pgfpathmoveto{\pgfqpoint{0.000000in}{0.000000in}}%
\pgfpathlineto{\pgfqpoint{0.000000in}{-0.048611in}}%
\pgfusepath{stroke,fill}%
}%
\begin{pgfscope}%
\pgfsys@transformshift{1.617213in}{1.172519in}%
\pgfsys@useobject{currentmarker}{}%
\end{pgfscope}%
\end{pgfscope}%
\begin{pgfscope}%
\definecolor{textcolor}{rgb}{0.000000,0.000000,0.000000}%
\pgfsetstrokecolor{textcolor}%
\pgfsetfillcolor{textcolor}%
\pgftext[x=1.638047in, y=0.759868in, left, base,rotate=90.000000]{\color{textcolor}\rmfamily\fontsize{6.000000}{7.200000}\selectfont Allianz}%
\end{pgfscope}%
\begin{pgfscope}%
\pgfsetbuttcap%
\pgfsetroundjoin%
\definecolor{currentfill}{rgb}{0.000000,0.000000,0.000000}%
\pgfsetfillcolor{currentfill}%
\pgfsetlinewidth{0.803000pt}%
\definecolor{currentstroke}{rgb}{0.000000,0.000000,0.000000}%
\pgfsetstrokecolor{currentstroke}%
\pgfsetdash{}{0pt}%
\pgfsys@defobject{currentmarker}{\pgfqpoint{0.000000in}{-0.048611in}}{\pgfqpoint{0.000000in}{0.000000in}}{%
\pgfpathmoveto{\pgfqpoint{0.000000in}{0.000000in}}%
\pgfpathlineto{\pgfqpoint{0.000000in}{-0.048611in}}%
\pgfusepath{stroke,fill}%
}%
\begin{pgfscope}%
\pgfsys@transformshift{1.748687in}{1.172519in}%
\pgfsys@useobject{currentmarker}{}%
\end{pgfscope}%
\end{pgfscope}%
\begin{pgfscope}%
\definecolor{textcolor}{rgb}{0.000000,0.000000,0.000000}%
\pgfsetstrokecolor{textcolor}%
\pgfsetfillcolor{textcolor}%
\pgftext[x=1.769520in, y=0.370445in, left, base,rotate=90.000000]{\color{textcolor}\rmfamily\fontsize{6.000000}{7.200000}\selectfont Assur Bon Plan}%
\end{pgfscope}%
\begin{pgfscope}%
\pgfsetbuttcap%
\pgfsetroundjoin%
\definecolor{currentfill}{rgb}{0.000000,0.000000,0.000000}%
\pgfsetfillcolor{currentfill}%
\pgfsetlinewidth{0.803000pt}%
\definecolor{currentstroke}{rgb}{0.000000,0.000000,0.000000}%
\pgfsetstrokecolor{currentstroke}%
\pgfsetdash{}{0pt}%
\pgfsys@defobject{currentmarker}{\pgfqpoint{0.000000in}{-0.048611in}}{\pgfqpoint{0.000000in}{0.000000in}}{%
\pgfpathmoveto{\pgfqpoint{0.000000in}{0.000000in}}%
\pgfpathlineto{\pgfqpoint{0.000000in}{-0.048611in}}%
\pgfusepath{stroke,fill}%
}%
\begin{pgfscope}%
\pgfsys@transformshift{1.880160in}{1.172519in}%
\pgfsys@useobject{currentmarker}{}%
\end{pgfscope}%
\end{pgfscope}%
\begin{pgfscope}%
\definecolor{textcolor}{rgb}{0.000000,0.000000,0.000000}%
\pgfsetstrokecolor{textcolor}%
\pgfsetfillcolor{textcolor}%
\pgftext[x=1.900993in, y=0.505782in, left, base,rotate=90.000000]{\color{textcolor}\rmfamily\fontsize{6.000000}{7.200000}\selectfont Assur O'Poil}%
\end{pgfscope}%
\begin{pgfscope}%
\pgfsetbuttcap%
\pgfsetroundjoin%
\definecolor{currentfill}{rgb}{0.000000,0.000000,0.000000}%
\pgfsetfillcolor{currentfill}%
\pgfsetlinewidth{0.803000pt}%
\definecolor{currentstroke}{rgb}{0.000000,0.000000,0.000000}%
\pgfsetstrokecolor{currentstroke}%
\pgfsetdash{}{0pt}%
\pgfsys@defobject{currentmarker}{\pgfqpoint{0.000000in}{-0.048611in}}{\pgfqpoint{0.000000in}{0.000000in}}{%
\pgfpathmoveto{\pgfqpoint{0.000000in}{0.000000in}}%
\pgfpathlineto{\pgfqpoint{0.000000in}{-0.048611in}}%
\pgfusepath{stroke,fill}%
}%
\begin{pgfscope}%
\pgfsys@transformshift{2.011633in}{1.172519in}%
\pgfsys@useobject{currentmarker}{}%
\end{pgfscope}%
\end{pgfscope}%
\begin{pgfscope}%
\definecolor{textcolor}{rgb}{0.000000,0.000000,0.000000}%
\pgfsetstrokecolor{textcolor}%
\pgfsetfillcolor{textcolor}%
\pgftext[x=2.032466in, y=0.528081in, left, base,rotate=90.000000]{\color{textcolor}\rmfamily\fontsize{6.000000}{7.200000}\selectfont AssurOnline}%
\end{pgfscope}%
\begin{pgfscope}%
\pgfsetbuttcap%
\pgfsetroundjoin%
\definecolor{currentfill}{rgb}{0.000000,0.000000,0.000000}%
\pgfsetfillcolor{currentfill}%
\pgfsetlinewidth{0.803000pt}%
\definecolor{currentstroke}{rgb}{0.000000,0.000000,0.000000}%
\pgfsetstrokecolor{currentstroke}%
\pgfsetdash{}{0pt}%
\pgfsys@defobject{currentmarker}{\pgfqpoint{0.000000in}{-0.048611in}}{\pgfqpoint{0.000000in}{0.000000in}}{%
\pgfpathmoveto{\pgfqpoint{0.000000in}{0.000000in}}%
\pgfpathlineto{\pgfqpoint{0.000000in}{-0.048611in}}%
\pgfusepath{stroke,fill}%
}%
\begin{pgfscope}%
\pgfsys@transformshift{2.143106in}{1.172519in}%
\pgfsys@useobject{currentmarker}{}%
\end{pgfscope}%
\end{pgfscope}%
\begin{pgfscope}%
\definecolor{textcolor}{rgb}{0.000000,0.000000,0.000000}%
\pgfsetstrokecolor{textcolor}%
\pgfsetfillcolor{textcolor}%
\pgftext[x=2.163940in, y=0.333716in, left, base,rotate=90.000000]{\color{textcolor}\rmfamily\fontsize{6.000000}{7.200000}\selectfont CNP Assurances}%
\end{pgfscope}%
\begin{pgfscope}%
\pgfsetbuttcap%
\pgfsetroundjoin%
\definecolor{currentfill}{rgb}{0.000000,0.000000,0.000000}%
\pgfsetfillcolor{currentfill}%
\pgfsetlinewidth{0.803000pt}%
\definecolor{currentstroke}{rgb}{0.000000,0.000000,0.000000}%
\pgfsetstrokecolor{currentstroke}%
\pgfsetdash{}{0pt}%
\pgfsys@defobject{currentmarker}{\pgfqpoint{0.000000in}{-0.048611in}}{\pgfqpoint{0.000000in}{0.000000in}}{%
\pgfpathmoveto{\pgfqpoint{0.000000in}{0.000000in}}%
\pgfpathlineto{\pgfqpoint{0.000000in}{-0.048611in}}%
\pgfusepath{stroke,fill}%
}%
\begin{pgfscope}%
\pgfsys@transformshift{2.274580in}{1.172519in}%
\pgfsys@useobject{currentmarker}{}%
\end{pgfscope}%
\end{pgfscope}%
\begin{pgfscope}%
\definecolor{textcolor}{rgb}{0.000000,0.000000,0.000000}%
\pgfsetstrokecolor{textcolor}%
\pgfsetfillcolor{textcolor}%
\pgftext[x=2.295413in, y=0.815269in, left, base,rotate=90.000000]{\color{textcolor}\rmfamily\fontsize{6.000000}{7.200000}\selectfont Carac}%
\end{pgfscope}%
\begin{pgfscope}%
\pgfsetbuttcap%
\pgfsetroundjoin%
\definecolor{currentfill}{rgb}{0.000000,0.000000,0.000000}%
\pgfsetfillcolor{currentfill}%
\pgfsetlinewidth{0.803000pt}%
\definecolor{currentstroke}{rgb}{0.000000,0.000000,0.000000}%
\pgfsetstrokecolor{currentstroke}%
\pgfsetdash{}{0pt}%
\pgfsys@defobject{currentmarker}{\pgfqpoint{0.000000in}{-0.048611in}}{\pgfqpoint{0.000000in}{0.000000in}}{%
\pgfpathmoveto{\pgfqpoint{0.000000in}{0.000000in}}%
\pgfpathlineto{\pgfqpoint{0.000000in}{-0.048611in}}%
\pgfusepath{stroke,fill}%
}%
\begin{pgfscope}%
\pgfsys@transformshift{2.406053in}{1.172519in}%
\pgfsys@useobject{currentmarker}{}%
\end{pgfscope}%
\end{pgfscope}%
\begin{pgfscope}%
\definecolor{textcolor}{rgb}{0.000000,0.000000,0.000000}%
\pgfsetstrokecolor{textcolor}%
\pgfsetfillcolor{textcolor}%
\pgftext[x=2.426886in, y=0.794050in, left, base,rotate=90.000000]{\color{textcolor}\rmfamily\fontsize{6.000000}{7.200000}\selectfont Cardif}%
\end{pgfscope}%
\begin{pgfscope}%
\pgfsetbuttcap%
\pgfsetroundjoin%
\definecolor{currentfill}{rgb}{0.000000,0.000000,0.000000}%
\pgfsetfillcolor{currentfill}%
\pgfsetlinewidth{0.803000pt}%
\definecolor{currentstroke}{rgb}{0.000000,0.000000,0.000000}%
\pgfsetstrokecolor{currentstroke}%
\pgfsetdash{}{0pt}%
\pgfsys@defobject{currentmarker}{\pgfqpoint{0.000000in}{-0.048611in}}{\pgfqpoint{0.000000in}{0.000000in}}{%
\pgfpathmoveto{\pgfqpoint{0.000000in}{0.000000in}}%
\pgfpathlineto{\pgfqpoint{0.000000in}{-0.048611in}}%
\pgfusepath{stroke,fill}%
}%
\begin{pgfscope}%
\pgfsys@transformshift{2.537526in}{1.172519in}%
\pgfsys@useobject{currentmarker}{}%
\end{pgfscope}%
\end{pgfscope}%
\begin{pgfscope}%
\definecolor{textcolor}{rgb}{0.000000,0.000000,0.000000}%
\pgfsetstrokecolor{textcolor}%
\pgfsetfillcolor{textcolor}%
\pgftext[x=2.558359in, y=0.200385in, left, base,rotate=90.000000]{\color{textcolor}\rmfamily\fontsize{6.000000}{7.200000}\selectfont Cegema Assurances}%
\end{pgfscope}%
\begin{pgfscope}%
\pgfsetbuttcap%
\pgfsetroundjoin%
\definecolor{currentfill}{rgb}{0.000000,0.000000,0.000000}%
\pgfsetfillcolor{currentfill}%
\pgfsetlinewidth{0.803000pt}%
\definecolor{currentstroke}{rgb}{0.000000,0.000000,0.000000}%
\pgfsetstrokecolor{currentstroke}%
\pgfsetdash{}{0pt}%
\pgfsys@defobject{currentmarker}{\pgfqpoint{0.000000in}{-0.048611in}}{\pgfqpoint{0.000000in}{0.000000in}}{%
\pgfpathmoveto{\pgfqpoint{0.000000in}{0.000000in}}%
\pgfpathlineto{\pgfqpoint{0.000000in}{-0.048611in}}%
\pgfusepath{stroke,fill}%
}%
\begin{pgfscope}%
\pgfsys@transformshift{2.668999in}{1.172519in}%
\pgfsys@useobject{currentmarker}{}%
\end{pgfscope}%
\end{pgfscope}%
\begin{pgfscope}%
\definecolor{textcolor}{rgb}{0.000000,0.000000,0.000000}%
\pgfsetstrokecolor{textcolor}%
\pgfsetfillcolor{textcolor}%
\pgftext[x=2.689832in, y=0.438112in, left, base,rotate=90.000000]{\color{textcolor}\rmfamily\fontsize{6.000000}{7.200000}\selectfont Crédit Mutuel}%
\end{pgfscope}%
\begin{pgfscope}%
\pgfsetbuttcap%
\pgfsetroundjoin%
\definecolor{currentfill}{rgb}{0.000000,0.000000,0.000000}%
\pgfsetfillcolor{currentfill}%
\pgfsetlinewidth{0.803000pt}%
\definecolor{currentstroke}{rgb}{0.000000,0.000000,0.000000}%
\pgfsetstrokecolor{currentstroke}%
\pgfsetdash{}{0pt}%
\pgfsys@defobject{currentmarker}{\pgfqpoint{0.000000in}{-0.048611in}}{\pgfqpoint{0.000000in}{0.000000in}}{%
\pgfpathmoveto{\pgfqpoint{0.000000in}{0.000000in}}%
\pgfpathlineto{\pgfqpoint{0.000000in}{-0.048611in}}%
\pgfusepath{stroke,fill}%
}%
\begin{pgfscope}%
\pgfsys@transformshift{2.800472in}{1.172519in}%
\pgfsys@useobject{currentmarker}{}%
\end{pgfscope}%
\end{pgfscope}%
\begin{pgfscope}%
\definecolor{textcolor}{rgb}{0.000000,0.000000,0.000000}%
\pgfsetstrokecolor{textcolor}%
\pgfsetfillcolor{textcolor}%
\pgftext[x=2.821306in, y=0.312883in, left, base,rotate=90.000000]{\color{textcolor}\rmfamily\fontsize{6.000000}{7.200000}\selectfont Direct Assurance}%
\end{pgfscope}%
\begin{pgfscope}%
\pgfsetbuttcap%
\pgfsetroundjoin%
\definecolor{currentfill}{rgb}{0.000000,0.000000,0.000000}%
\pgfsetfillcolor{currentfill}%
\pgfsetlinewidth{0.803000pt}%
\definecolor{currentstroke}{rgb}{0.000000,0.000000,0.000000}%
\pgfsetstrokecolor{currentstroke}%
\pgfsetdash{}{0pt}%
\pgfsys@defobject{currentmarker}{\pgfqpoint{0.000000in}{-0.048611in}}{\pgfqpoint{0.000000in}{0.000000in}}{%
\pgfpathmoveto{\pgfqpoint{0.000000in}{0.000000in}}%
\pgfpathlineto{\pgfqpoint{0.000000in}{-0.048611in}}%
\pgfusepath{stroke,fill}%
}%
\begin{pgfscope}%
\pgfsys@transformshift{2.931946in}{1.172519in}%
\pgfsys@useobject{currentmarker}{}%
\end{pgfscope}%
\end{pgfscope}%
\begin{pgfscope}%
\definecolor{textcolor}{rgb}{0.000000,0.000000,0.000000}%
\pgfsetstrokecolor{textcolor}%
\pgfsetfillcolor{textcolor}%
\pgftext[x=2.952779in, y=0.384488in, left, base,rotate=90.000000]{\color{textcolor}\rmfamily\fontsize{6.000000}{7.200000}\selectfont Eca Assurances}%
\end{pgfscope}%
\begin{pgfscope}%
\pgfsetbuttcap%
\pgfsetroundjoin%
\definecolor{currentfill}{rgb}{0.000000,0.000000,0.000000}%
\pgfsetfillcolor{currentfill}%
\pgfsetlinewidth{0.803000pt}%
\definecolor{currentstroke}{rgb}{0.000000,0.000000,0.000000}%
\pgfsetstrokecolor{currentstroke}%
\pgfsetdash{}{0pt}%
\pgfsys@defobject{currentmarker}{\pgfqpoint{0.000000in}{-0.048611in}}{\pgfqpoint{0.000000in}{0.000000in}}{%
\pgfpathmoveto{\pgfqpoint{0.000000in}{0.000000in}}%
\pgfpathlineto{\pgfqpoint{0.000000in}{-0.048611in}}%
\pgfusepath{stroke,fill}%
}%
\begin{pgfscope}%
\pgfsys@transformshift{3.063419in}{1.172519in}%
\pgfsys@useobject{currentmarker}{}%
\end{pgfscope}%
\end{pgfscope}%
\begin{pgfscope}%
\definecolor{textcolor}{rgb}{0.000000,0.000000,0.000000}%
\pgfsetstrokecolor{textcolor}%
\pgfsetfillcolor{textcolor}%
\pgftext[x=3.084252in, y=0.374225in, left, base,rotate=90.000000]{\color{textcolor}\rmfamily\fontsize{6.000000}{7.200000}\selectfont Euro-Assurance}%
\end{pgfscope}%
\begin{pgfscope}%
\pgfsetbuttcap%
\pgfsetroundjoin%
\definecolor{currentfill}{rgb}{0.000000,0.000000,0.000000}%
\pgfsetfillcolor{currentfill}%
\pgfsetlinewidth{0.803000pt}%
\definecolor{currentstroke}{rgb}{0.000000,0.000000,0.000000}%
\pgfsetstrokecolor{currentstroke}%
\pgfsetdash{}{0pt}%
\pgfsys@defobject{currentmarker}{\pgfqpoint{0.000000in}{-0.048611in}}{\pgfqpoint{0.000000in}{0.000000in}}{%
\pgfpathmoveto{\pgfqpoint{0.000000in}{0.000000in}}%
\pgfpathlineto{\pgfqpoint{0.000000in}{-0.048611in}}%
\pgfusepath{stroke,fill}%
}%
\begin{pgfscope}%
\pgfsys@transformshift{3.194892in}{1.172519in}%
\pgfsys@useobject{currentmarker}{}%
\end{pgfscope}%
\end{pgfscope}%
\begin{pgfscope}%
\definecolor{textcolor}{rgb}{0.000000,0.000000,0.000000}%
\pgfsetstrokecolor{textcolor}%
\pgfsetfillcolor{textcolor}%
\pgftext[x=3.215725in, y=0.771751in, left, base,rotate=90.000000]{\color{textcolor}\rmfamily\fontsize{6.000000}{7.200000}\selectfont Eurofil}%
\end{pgfscope}%
\begin{pgfscope}%
\pgfsetbuttcap%
\pgfsetroundjoin%
\definecolor{currentfill}{rgb}{0.000000,0.000000,0.000000}%
\pgfsetfillcolor{currentfill}%
\pgfsetlinewidth{0.803000pt}%
\definecolor{currentstroke}{rgb}{0.000000,0.000000,0.000000}%
\pgfsetstrokecolor{currentstroke}%
\pgfsetdash{}{0pt}%
\pgfsys@defobject{currentmarker}{\pgfqpoint{0.000000in}{-0.048611in}}{\pgfqpoint{0.000000in}{0.000000in}}{%
\pgfpathmoveto{\pgfqpoint{0.000000in}{0.000000in}}%
\pgfpathlineto{\pgfqpoint{0.000000in}{-0.048611in}}%
\pgfusepath{stroke,fill}%
}%
\begin{pgfscope}%
\pgfsys@transformshift{3.326365in}{1.172519in}%
\pgfsys@useobject{currentmarker}{}%
\end{pgfscope}%
\end{pgfscope}%
\begin{pgfscope}%
\definecolor{textcolor}{rgb}{0.000000,0.000000,0.000000}%
\pgfsetstrokecolor{textcolor}%
\pgfsetfillcolor{textcolor}%
\pgftext[x=3.347199in, y=0.840422in, left, base,rotate=90.000000]{\color{textcolor}\rmfamily\fontsize{6.000000}{7.200000}\selectfont GMF}%
\end{pgfscope}%
\begin{pgfscope}%
\pgfsetbuttcap%
\pgfsetroundjoin%
\definecolor{currentfill}{rgb}{0.000000,0.000000,0.000000}%
\pgfsetfillcolor{currentfill}%
\pgfsetlinewidth{0.803000pt}%
\definecolor{currentstroke}{rgb}{0.000000,0.000000,0.000000}%
\pgfsetstrokecolor{currentstroke}%
\pgfsetdash{}{0pt}%
\pgfsys@defobject{currentmarker}{\pgfqpoint{0.000000in}{-0.048611in}}{\pgfqpoint{0.000000in}{0.000000in}}{%
\pgfpathmoveto{\pgfqpoint{0.000000in}{0.000000in}}%
\pgfpathlineto{\pgfqpoint{0.000000in}{-0.048611in}}%
\pgfusepath{stroke,fill}%
}%
\begin{pgfscope}%
\pgfsys@transformshift{3.457838in}{1.172519in}%
\pgfsys@useobject{currentmarker}{}%
\end{pgfscope}%
\end{pgfscope}%
\begin{pgfscope}%
\definecolor{textcolor}{rgb}{0.000000,0.000000,0.000000}%
\pgfsetstrokecolor{textcolor}%
\pgfsetfillcolor{textcolor}%
\pgftext[x=3.478672in, y=0.889573in, left, base,rotate=90.000000]{\color{textcolor}\rmfamily\fontsize{6.000000}{7.200000}\selectfont Gan}%
\end{pgfscope}%
\begin{pgfscope}%
\pgfsetbuttcap%
\pgfsetroundjoin%
\definecolor{currentfill}{rgb}{0.000000,0.000000,0.000000}%
\pgfsetfillcolor{currentfill}%
\pgfsetlinewidth{0.803000pt}%
\definecolor{currentstroke}{rgb}{0.000000,0.000000,0.000000}%
\pgfsetstrokecolor{currentstroke}%
\pgfsetdash{}{0pt}%
\pgfsys@defobject{currentmarker}{\pgfqpoint{0.000000in}{-0.048611in}}{\pgfqpoint{0.000000in}{0.000000in}}{%
\pgfpathmoveto{\pgfqpoint{0.000000in}{0.000000in}}%
\pgfpathlineto{\pgfqpoint{0.000000in}{-0.048611in}}%
\pgfusepath{stroke,fill}%
}%
\begin{pgfscope}%
\pgfsys@transformshift{3.589312in}{1.172519in}%
\pgfsys@useobject{currentmarker}{}%
\end{pgfscope}%
\end{pgfscope}%
\begin{pgfscope}%
\definecolor{textcolor}{rgb}{0.000000,0.000000,0.000000}%
\pgfsetstrokecolor{textcolor}%
\pgfsetfillcolor{textcolor}%
\pgftext[x=3.610145in, y=0.699761in, left, base,rotate=90.000000]{\color{textcolor}\rmfamily\fontsize{6.000000}{7.200000}\selectfont Generali}%
\end{pgfscope}%
\begin{pgfscope}%
\pgfsetbuttcap%
\pgfsetroundjoin%
\definecolor{currentfill}{rgb}{0.000000,0.000000,0.000000}%
\pgfsetfillcolor{currentfill}%
\pgfsetlinewidth{0.803000pt}%
\definecolor{currentstroke}{rgb}{0.000000,0.000000,0.000000}%
\pgfsetstrokecolor{currentstroke}%
\pgfsetdash{}{0pt}%
\pgfsys@defobject{currentmarker}{\pgfqpoint{0.000000in}{-0.048611in}}{\pgfqpoint{0.000000in}{0.000000in}}{%
\pgfpathmoveto{\pgfqpoint{0.000000in}{0.000000in}}%
\pgfpathlineto{\pgfqpoint{0.000000in}{-0.048611in}}%
\pgfusepath{stroke,fill}%
}%
\begin{pgfscope}%
\pgfsys@transformshift{3.720785in}{1.172519in}%
\pgfsys@useobject{currentmarker}{}%
\end{pgfscope}%
\end{pgfscope}%
\begin{pgfscope}%
\definecolor{textcolor}{rgb}{0.000000,0.000000,0.000000}%
\pgfsetstrokecolor{textcolor}%
\pgfsetfillcolor{textcolor}%
\pgftext[x=3.741618in, y=0.607941in, left, base,rotate=90.000000]{\color{textcolor}\rmfamily\fontsize{6.000000}{7.200000}\selectfont Groupama}%
\end{pgfscope}%
\begin{pgfscope}%
\pgfsetbuttcap%
\pgfsetroundjoin%
\definecolor{currentfill}{rgb}{0.000000,0.000000,0.000000}%
\pgfsetfillcolor{currentfill}%
\pgfsetlinewidth{0.803000pt}%
\definecolor{currentstroke}{rgb}{0.000000,0.000000,0.000000}%
\pgfsetstrokecolor{currentstroke}%
\pgfsetdash{}{0pt}%
\pgfsys@defobject{currentmarker}{\pgfqpoint{0.000000in}{-0.048611in}}{\pgfqpoint{0.000000in}{0.000000in}}{%
\pgfpathmoveto{\pgfqpoint{0.000000in}{0.000000in}}%
\pgfpathlineto{\pgfqpoint{0.000000in}{-0.048611in}}%
\pgfusepath{stroke,fill}%
}%
\begin{pgfscope}%
\pgfsys@transformshift{3.852258in}{1.172519in}%
\pgfsys@useobject{currentmarker}{}%
\end{pgfscope}%
\end{pgfscope}%
\begin{pgfscope}%
\definecolor{textcolor}{rgb}{0.000000,0.000000,0.000000}%
\pgfsetstrokecolor{textcolor}%
\pgfsetfillcolor{textcolor}%
\pgftext[x=3.873091in, y=0.581707in, left, base,rotate=90.000000]{\color{textcolor}\rmfamily\fontsize{6.000000}{7.200000}\selectfont Génération}%
\end{pgfscope}%
\begin{pgfscope}%
\pgfsetbuttcap%
\pgfsetroundjoin%
\definecolor{currentfill}{rgb}{0.000000,0.000000,0.000000}%
\pgfsetfillcolor{currentfill}%
\pgfsetlinewidth{0.803000pt}%
\definecolor{currentstroke}{rgb}{0.000000,0.000000,0.000000}%
\pgfsetstrokecolor{currentstroke}%
\pgfsetdash{}{0pt}%
\pgfsys@defobject{currentmarker}{\pgfqpoint{0.000000in}{-0.048611in}}{\pgfqpoint{0.000000in}{0.000000in}}{%
\pgfpathmoveto{\pgfqpoint{0.000000in}{0.000000in}}%
\pgfpathlineto{\pgfqpoint{0.000000in}{-0.048611in}}%
\pgfusepath{stroke,fill}%
}%
\begin{pgfscope}%
\pgfsys@transformshift{3.983731in}{1.172519in}%
\pgfsys@useobject{currentmarker}{}%
\end{pgfscope}%
\end{pgfscope}%
\begin{pgfscope}%
\definecolor{textcolor}{rgb}{0.000000,0.000000,0.000000}%
\pgfsetstrokecolor{textcolor}%
\pgfsetfillcolor{textcolor}%
\pgftext[x=4.004565in, y=0.216048in, left, base,rotate=90.000000]{\color{textcolor}\rmfamily\fontsize{6.000000}{7.200000}\selectfont Harmonie Mutuelle}%
\end{pgfscope}%
\begin{pgfscope}%
\pgfsetbuttcap%
\pgfsetroundjoin%
\definecolor{currentfill}{rgb}{0.000000,0.000000,0.000000}%
\pgfsetfillcolor{currentfill}%
\pgfsetlinewidth{0.803000pt}%
\definecolor{currentstroke}{rgb}{0.000000,0.000000,0.000000}%
\pgfsetstrokecolor{currentstroke}%
\pgfsetdash{}{0pt}%
\pgfsys@defobject{currentmarker}{\pgfqpoint{0.000000in}{-0.048611in}}{\pgfqpoint{0.000000in}{0.000000in}}{%
\pgfpathmoveto{\pgfqpoint{0.000000in}{0.000000in}}%
\pgfpathlineto{\pgfqpoint{0.000000in}{-0.048611in}}%
\pgfusepath{stroke,fill}%
}%
\begin{pgfscope}%
\pgfsys@transformshift{4.115205in}{1.172519in}%
\pgfsys@useobject{currentmarker}{}%
\end{pgfscope}%
\end{pgfscope}%
\begin{pgfscope}%
\definecolor{textcolor}{rgb}{0.000000,0.000000,0.000000}%
\pgfsetstrokecolor{textcolor}%
\pgfsetfillcolor{textcolor}%
\pgftext[x=4.136038in, y=0.783248in, left, base,rotate=90.000000]{\color{textcolor}\rmfamily\fontsize{6.000000}{7.200000}\selectfont Hiscox}%
\end{pgfscope}%
\begin{pgfscope}%
\pgfsetbuttcap%
\pgfsetroundjoin%
\definecolor{currentfill}{rgb}{0.000000,0.000000,0.000000}%
\pgfsetfillcolor{currentfill}%
\pgfsetlinewidth{0.803000pt}%
\definecolor{currentstroke}{rgb}{0.000000,0.000000,0.000000}%
\pgfsetstrokecolor{currentstroke}%
\pgfsetdash{}{0pt}%
\pgfsys@defobject{currentmarker}{\pgfqpoint{0.000000in}{-0.048611in}}{\pgfqpoint{0.000000in}{0.000000in}}{%
\pgfpathmoveto{\pgfqpoint{0.000000in}{0.000000in}}%
\pgfpathlineto{\pgfqpoint{0.000000in}{-0.048611in}}%
\pgfusepath{stroke,fill}%
}%
\begin{pgfscope}%
\pgfsys@transformshift{4.246678in}{1.172519in}%
\pgfsys@useobject{currentmarker}{}%
\end{pgfscope}%
\end{pgfscope}%
\begin{pgfscope}%
\definecolor{textcolor}{rgb}{0.000000,0.000000,0.000000}%
\pgfsetstrokecolor{textcolor}%
\pgfsetfillcolor{textcolor}%
\pgftext[x=4.267511in, y=0.703928in, left, base,rotate=90.000000]{\color{textcolor}\rmfamily\fontsize{6.000000}{7.200000}\selectfont Intériale}%
\end{pgfscope}%
\begin{pgfscope}%
\pgfsetbuttcap%
\pgfsetroundjoin%
\definecolor{currentfill}{rgb}{0.000000,0.000000,0.000000}%
\pgfsetfillcolor{currentfill}%
\pgfsetlinewidth{0.803000pt}%
\definecolor{currentstroke}{rgb}{0.000000,0.000000,0.000000}%
\pgfsetstrokecolor{currentstroke}%
\pgfsetdash{}{0pt}%
\pgfsys@defobject{currentmarker}{\pgfqpoint{0.000000in}{-0.048611in}}{\pgfqpoint{0.000000in}{0.000000in}}{%
\pgfpathmoveto{\pgfqpoint{0.000000in}{0.000000in}}%
\pgfpathlineto{\pgfqpoint{0.000000in}{-0.048611in}}%
\pgfusepath{stroke,fill}%
}%
\begin{pgfscope}%
\pgfsys@transformshift{4.378151in}{1.172519in}%
\pgfsys@useobject{currentmarker}{}%
\end{pgfscope}%
\end{pgfscope}%
\begin{pgfscope}%
\definecolor{textcolor}{rgb}{0.000000,0.000000,0.000000}%
\pgfsetstrokecolor{textcolor}%
\pgfsetfillcolor{textcolor}%
\pgftext[x=4.398984in, y=0.219520in, left, base,rotate=90.000000]{\color{textcolor}\rmfamily\fontsize{6.000000}{7.200000}\selectfont L'olivier Assurance}%
\end{pgfscope}%
\begin{pgfscope}%
\pgfsetbuttcap%
\pgfsetroundjoin%
\definecolor{currentfill}{rgb}{0.000000,0.000000,0.000000}%
\pgfsetfillcolor{currentfill}%
\pgfsetlinewidth{0.803000pt}%
\definecolor{currentstroke}{rgb}{0.000000,0.000000,0.000000}%
\pgfsetstrokecolor{currentstroke}%
\pgfsetdash{}{0pt}%
\pgfsys@defobject{currentmarker}{\pgfqpoint{0.000000in}{-0.048611in}}{\pgfqpoint{0.000000in}{0.000000in}}{%
\pgfpathmoveto{\pgfqpoint{0.000000in}{0.000000in}}%
\pgfpathlineto{\pgfqpoint{0.000000in}{-0.048611in}}%
\pgfusepath{stroke,fill}%
}%
\begin{pgfscope}%
\pgfsys@transformshift{4.509624in}{1.172519in}%
\pgfsys@useobject{currentmarker}{}%
\end{pgfscope}%
\end{pgfscope}%
\begin{pgfscope}%
\definecolor{textcolor}{rgb}{0.000000,0.000000,0.000000}%
\pgfsetstrokecolor{textcolor}%
\pgfsetfillcolor{textcolor}%
\pgftext[x=4.530457in, y=0.877150in, left, base,rotate=90.000000]{\color{textcolor}\rmfamily\fontsize{6.000000}{7.200000}\selectfont LCL}%
\end{pgfscope}%
\begin{pgfscope}%
\pgfsetbuttcap%
\pgfsetroundjoin%
\definecolor{currentfill}{rgb}{0.000000,0.000000,0.000000}%
\pgfsetfillcolor{currentfill}%
\pgfsetlinewidth{0.803000pt}%
\definecolor{currentstroke}{rgb}{0.000000,0.000000,0.000000}%
\pgfsetstrokecolor{currentstroke}%
\pgfsetdash{}{0pt}%
\pgfsys@defobject{currentmarker}{\pgfqpoint{0.000000in}{-0.048611in}}{\pgfqpoint{0.000000in}{0.000000in}}{%
\pgfpathmoveto{\pgfqpoint{0.000000in}{0.000000in}}%
\pgfpathlineto{\pgfqpoint{0.000000in}{-0.048611in}}%
\pgfusepath{stroke,fill}%
}%
\begin{pgfscope}%
\pgfsys@transformshift{4.641097in}{1.172519in}%
\pgfsys@useobject{currentmarker}{}%
\end{pgfscope}%
\end{pgfscope}%
\begin{pgfscope}%
\definecolor{textcolor}{rgb}{0.000000,0.000000,0.000000}%
\pgfsetstrokecolor{textcolor}%
\pgfsetfillcolor{textcolor}%
\pgftext[x=4.661931in, y=0.769513in, left, base,rotate=90.000000]{\color{textcolor}\rmfamily\fontsize{6.000000}{7.200000}\selectfont MAAF}%
\end{pgfscope}%
\begin{pgfscope}%
\pgfsetbuttcap%
\pgfsetroundjoin%
\definecolor{currentfill}{rgb}{0.000000,0.000000,0.000000}%
\pgfsetfillcolor{currentfill}%
\pgfsetlinewidth{0.803000pt}%
\definecolor{currentstroke}{rgb}{0.000000,0.000000,0.000000}%
\pgfsetstrokecolor{currentstroke}%
\pgfsetdash{}{0pt}%
\pgfsys@defobject{currentmarker}{\pgfqpoint{0.000000in}{-0.048611in}}{\pgfqpoint{0.000000in}{0.000000in}}{%
\pgfpathmoveto{\pgfqpoint{0.000000in}{0.000000in}}%
\pgfpathlineto{\pgfqpoint{0.000000in}{-0.048611in}}%
\pgfusepath{stroke,fill}%
}%
\begin{pgfscope}%
\pgfsys@transformshift{4.772571in}{1.172519in}%
\pgfsys@useobject{currentmarker}{}%
\end{pgfscope}%
\end{pgfscope}%
\begin{pgfscope}%
\definecolor{textcolor}{rgb}{0.000000,0.000000,0.000000}%
\pgfsetstrokecolor{textcolor}%
\pgfsetfillcolor{textcolor}%
\pgftext[x=4.793404in, y=0.737492in, left, base,rotate=90.000000]{\color{textcolor}\rmfamily\fontsize{6.000000}{7.200000}\selectfont MACIF}%
\end{pgfscope}%
\begin{pgfscope}%
\pgfsetbuttcap%
\pgfsetroundjoin%
\definecolor{currentfill}{rgb}{0.000000,0.000000,0.000000}%
\pgfsetfillcolor{currentfill}%
\pgfsetlinewidth{0.803000pt}%
\definecolor{currentstroke}{rgb}{0.000000,0.000000,0.000000}%
\pgfsetstrokecolor{currentstroke}%
\pgfsetdash{}{0pt}%
\pgfsys@defobject{currentmarker}{\pgfqpoint{0.000000in}{-0.048611in}}{\pgfqpoint{0.000000in}{0.000000in}}{%
\pgfpathmoveto{\pgfqpoint{0.000000in}{0.000000in}}%
\pgfpathlineto{\pgfqpoint{0.000000in}{-0.048611in}}%
\pgfusepath{stroke,fill}%
}%
\begin{pgfscope}%
\pgfsys@transformshift{4.904044in}{1.172519in}%
\pgfsys@useobject{currentmarker}{}%
\end{pgfscope}%
\end{pgfscope}%
\begin{pgfscope}%
\definecolor{textcolor}{rgb}{0.000000,0.000000,0.000000}%
\pgfsetstrokecolor{textcolor}%
\pgfsetfillcolor{textcolor}%
\pgftext[x=4.924877in, y=0.807321in, left, base,rotate=90.000000]{\color{textcolor}\rmfamily\fontsize{6.000000}{7.200000}\selectfont MAIF}%
\end{pgfscope}%
\begin{pgfscope}%
\pgfsetbuttcap%
\pgfsetroundjoin%
\definecolor{currentfill}{rgb}{0.000000,0.000000,0.000000}%
\pgfsetfillcolor{currentfill}%
\pgfsetlinewidth{0.803000pt}%
\definecolor{currentstroke}{rgb}{0.000000,0.000000,0.000000}%
\pgfsetstrokecolor{currentstroke}%
\pgfsetdash{}{0pt}%
\pgfsys@defobject{currentmarker}{\pgfqpoint{0.000000in}{-0.048611in}}{\pgfqpoint{0.000000in}{0.000000in}}{%
\pgfpathmoveto{\pgfqpoint{0.000000in}{0.000000in}}%
\pgfpathlineto{\pgfqpoint{0.000000in}{-0.048611in}}%
\pgfusepath{stroke,fill}%
}%
\begin{pgfscope}%
\pgfsys@transformshift{5.035517in}{1.172519in}%
\pgfsys@useobject{currentmarker}{}%
\end{pgfscope}%
\end{pgfscope}%
\begin{pgfscope}%
\definecolor{textcolor}{rgb}{0.000000,0.000000,0.000000}%
\pgfsetstrokecolor{textcolor}%
\pgfsetfillcolor{textcolor}%
\pgftext[x=5.056350in, y=0.837722in, left, base,rotate=90.000000]{\color{textcolor}\rmfamily\fontsize{6.000000}{7.200000}\selectfont MGP}%
\end{pgfscope}%
\begin{pgfscope}%
\pgfsetbuttcap%
\pgfsetroundjoin%
\definecolor{currentfill}{rgb}{0.000000,0.000000,0.000000}%
\pgfsetfillcolor{currentfill}%
\pgfsetlinewidth{0.803000pt}%
\definecolor{currentstroke}{rgb}{0.000000,0.000000,0.000000}%
\pgfsetstrokecolor{currentstroke}%
\pgfsetdash{}{0pt}%
\pgfsys@defobject{currentmarker}{\pgfqpoint{0.000000in}{-0.048611in}}{\pgfqpoint{0.000000in}{0.000000in}}{%
\pgfpathmoveto{\pgfqpoint{0.000000in}{0.000000in}}%
\pgfpathlineto{\pgfqpoint{0.000000in}{-0.048611in}}%
\pgfusepath{stroke,fill}%
}%
\begin{pgfscope}%
\pgfsys@transformshift{5.166990in}{1.172519in}%
\pgfsys@useobject{currentmarker}{}%
\end{pgfscope}%
\end{pgfscope}%
\begin{pgfscope}%
\definecolor{textcolor}{rgb}{0.000000,0.000000,0.000000}%
\pgfsetstrokecolor{textcolor}%
\pgfsetfillcolor{textcolor}%
\pgftext[x=5.187824in, y=0.818818in, left, base,rotate=90.000000]{\color{textcolor}\rmfamily\fontsize{6.000000}{7.200000}\selectfont MMA}%
\end{pgfscope}%
\begin{pgfscope}%
\pgfsetbuttcap%
\pgfsetroundjoin%
\definecolor{currentfill}{rgb}{0.000000,0.000000,0.000000}%
\pgfsetfillcolor{currentfill}%
\pgfsetlinewidth{0.803000pt}%
\definecolor{currentstroke}{rgb}{0.000000,0.000000,0.000000}%
\pgfsetstrokecolor{currentstroke}%
\pgfsetdash{}{0pt}%
\pgfsys@defobject{currentmarker}{\pgfqpoint{0.000000in}{-0.048611in}}{\pgfqpoint{0.000000in}{0.000000in}}{%
\pgfpathmoveto{\pgfqpoint{0.000000in}{0.000000in}}%
\pgfpathlineto{\pgfqpoint{0.000000in}{-0.048611in}}%
\pgfusepath{stroke,fill}%
}%
\begin{pgfscope}%
\pgfsys@transformshift{5.298463in}{1.172519in}%
\pgfsys@useobject{currentmarker}{}%
\end{pgfscope}%
\end{pgfscope}%
\begin{pgfscope}%
\definecolor{textcolor}{rgb}{0.000000,0.000000,0.000000}%
\pgfsetstrokecolor{textcolor}%
\pgfsetfillcolor{textcolor}%
\pgftext[x=5.319297in, y=0.665734in, left, base,rotate=90.000000]{\color{textcolor}\rmfamily\fontsize{6.000000}{7.200000}\selectfont Magnolia}%
\end{pgfscope}%
\begin{pgfscope}%
\pgfsetbuttcap%
\pgfsetroundjoin%
\definecolor{currentfill}{rgb}{0.000000,0.000000,0.000000}%
\pgfsetfillcolor{currentfill}%
\pgfsetlinewidth{0.803000pt}%
\definecolor{currentstroke}{rgb}{0.000000,0.000000,0.000000}%
\pgfsetstrokecolor{currentstroke}%
\pgfsetdash{}{0pt}%
\pgfsys@defobject{currentmarker}{\pgfqpoint{0.000000in}{-0.048611in}}{\pgfqpoint{0.000000in}{0.000000in}}{%
\pgfpathmoveto{\pgfqpoint{0.000000in}{0.000000in}}%
\pgfpathlineto{\pgfqpoint{0.000000in}{-0.048611in}}%
\pgfusepath{stroke,fill}%
}%
\begin{pgfscope}%
\pgfsys@transformshift{5.429937in}{1.172519in}%
\pgfsys@useobject{currentmarker}{}%
\end{pgfscope}%
\end{pgfscope}%
\begin{pgfscope}%
\definecolor{textcolor}{rgb}{0.000000,0.000000,0.000000}%
\pgfsetstrokecolor{textcolor}%
\pgfsetfillcolor{textcolor}%
\pgftext[x=5.450770in, y=0.263733in, left, base,rotate=90.000000]{\color{textcolor}\rmfamily\fontsize{6.000000}{7.200000}\selectfont Malakoff Humanis}%
\end{pgfscope}%
\begin{pgfscope}%
\pgfsetbuttcap%
\pgfsetroundjoin%
\definecolor{currentfill}{rgb}{0.000000,0.000000,0.000000}%
\pgfsetfillcolor{currentfill}%
\pgfsetlinewidth{0.803000pt}%
\definecolor{currentstroke}{rgb}{0.000000,0.000000,0.000000}%
\pgfsetstrokecolor{currentstroke}%
\pgfsetdash{}{0pt}%
\pgfsys@defobject{currentmarker}{\pgfqpoint{0.000000in}{-0.048611in}}{\pgfqpoint{0.000000in}{0.000000in}}{%
\pgfpathmoveto{\pgfqpoint{0.000000in}{0.000000in}}%
\pgfpathlineto{\pgfqpoint{0.000000in}{-0.048611in}}%
\pgfusepath{stroke,fill}%
}%
\begin{pgfscope}%
\pgfsys@transformshift{5.561410in}{1.172519in}%
\pgfsys@useobject{currentmarker}{}%
\end{pgfscope}%
\end{pgfscope}%
\begin{pgfscope}%
\definecolor{textcolor}{rgb}{0.000000,0.000000,0.000000}%
\pgfsetstrokecolor{textcolor}%
\pgfsetfillcolor{textcolor}%
\pgftext[x=5.582243in, y=0.826226in, left, base,rotate=90.000000]{\color{textcolor}\rmfamily\fontsize{6.000000}{7.200000}\selectfont Mapa}%
\end{pgfscope}%
\begin{pgfscope}%
\pgfsetbuttcap%
\pgfsetroundjoin%
\definecolor{currentfill}{rgb}{0.000000,0.000000,0.000000}%
\pgfsetfillcolor{currentfill}%
\pgfsetlinewidth{0.803000pt}%
\definecolor{currentstroke}{rgb}{0.000000,0.000000,0.000000}%
\pgfsetstrokecolor{currentstroke}%
\pgfsetdash{}{0pt}%
\pgfsys@defobject{currentmarker}{\pgfqpoint{0.000000in}{-0.048611in}}{\pgfqpoint{0.000000in}{0.000000in}}{%
\pgfpathmoveto{\pgfqpoint{0.000000in}{0.000000in}}%
\pgfpathlineto{\pgfqpoint{0.000000in}{-0.048611in}}%
\pgfusepath{stroke,fill}%
}%
\begin{pgfscope}%
\pgfsys@transformshift{5.692883in}{1.172519in}%
\pgfsys@useobject{currentmarker}{}%
\end{pgfscope}%
\end{pgfscope}%
\begin{pgfscope}%
\definecolor{textcolor}{rgb}{0.000000,0.000000,0.000000}%
\pgfsetstrokecolor{textcolor}%
\pgfsetfillcolor{textcolor}%
\pgftext[x=5.713716in, y=0.716273in, left, base,rotate=90.000000]{\color{textcolor}\rmfamily\fontsize{6.000000}{7.200000}\selectfont Matmut}%
\end{pgfscope}%
\begin{pgfscope}%
\pgfsetbuttcap%
\pgfsetroundjoin%
\definecolor{currentfill}{rgb}{0.000000,0.000000,0.000000}%
\pgfsetfillcolor{currentfill}%
\pgfsetlinewidth{0.803000pt}%
\definecolor{currentstroke}{rgb}{0.000000,0.000000,0.000000}%
\pgfsetstrokecolor{currentstroke}%
\pgfsetdash{}{0pt}%
\pgfsys@defobject{currentmarker}{\pgfqpoint{0.000000in}{-0.048611in}}{\pgfqpoint{0.000000in}{0.000000in}}{%
\pgfpathmoveto{\pgfqpoint{0.000000in}{0.000000in}}%
\pgfpathlineto{\pgfqpoint{0.000000in}{-0.048611in}}%
\pgfusepath{stroke,fill}%
}%
\begin{pgfscope}%
\pgfsys@transformshift{5.824356in}{1.172519in}%
\pgfsys@useobject{currentmarker}{}%
\end{pgfscope}%
\end{pgfscope}%
\begin{pgfscope}%
\definecolor{textcolor}{rgb}{0.000000,0.000000,0.000000}%
\pgfsetstrokecolor{textcolor}%
\pgfsetfillcolor{textcolor}%
\pgftext[x=5.845190in, y=0.767584in, left, base,rotate=90.000000]{\color{textcolor}\rmfamily\fontsize{6.000000}{7.200000}\selectfont Mercer}%
\end{pgfscope}%
\begin{pgfscope}%
\pgfsetbuttcap%
\pgfsetroundjoin%
\definecolor{currentfill}{rgb}{0.000000,0.000000,0.000000}%
\pgfsetfillcolor{currentfill}%
\pgfsetlinewidth{0.803000pt}%
\definecolor{currentstroke}{rgb}{0.000000,0.000000,0.000000}%
\pgfsetstrokecolor{currentstroke}%
\pgfsetdash{}{0pt}%
\pgfsys@defobject{currentmarker}{\pgfqpoint{0.000000in}{-0.048611in}}{\pgfqpoint{0.000000in}{0.000000in}}{%
\pgfpathmoveto{\pgfqpoint{0.000000in}{0.000000in}}%
\pgfpathlineto{\pgfqpoint{0.000000in}{-0.048611in}}%
\pgfusepath{stroke,fill}%
}%
\begin{pgfscope}%
\pgfsys@transformshift{5.955830in}{1.172519in}%
\pgfsys@useobject{currentmarker}{}%
\end{pgfscope}%
\end{pgfscope}%
\begin{pgfscope}%
\definecolor{textcolor}{rgb}{0.000000,0.000000,0.000000}%
\pgfsetstrokecolor{textcolor}%
\pgfsetfillcolor{textcolor}%
\pgftext[x=5.976663in, y=0.729081in, left, base,rotate=90.000000]{\color{textcolor}\rmfamily\fontsize{6.000000}{7.200000}\selectfont MetLife}%
\end{pgfscope}%
\begin{pgfscope}%
\pgfsetbuttcap%
\pgfsetroundjoin%
\definecolor{currentfill}{rgb}{0.000000,0.000000,0.000000}%
\pgfsetfillcolor{currentfill}%
\pgfsetlinewidth{0.803000pt}%
\definecolor{currentstroke}{rgb}{0.000000,0.000000,0.000000}%
\pgfsetstrokecolor{currentstroke}%
\pgfsetdash{}{0pt}%
\pgfsys@defobject{currentmarker}{\pgfqpoint{0.000000in}{-0.048611in}}{\pgfqpoint{0.000000in}{0.000000in}}{%
\pgfpathmoveto{\pgfqpoint{0.000000in}{0.000000in}}%
\pgfpathlineto{\pgfqpoint{0.000000in}{-0.048611in}}%
\pgfusepath{stroke,fill}%
}%
\begin{pgfscope}%
\pgfsys@transformshift{6.087303in}{1.172519in}%
\pgfsys@useobject{currentmarker}{}%
\end{pgfscope}%
\end{pgfscope}%
\begin{pgfscope}%
\definecolor{textcolor}{rgb}{0.000000,0.000000,0.000000}%
\pgfsetstrokecolor{textcolor}%
\pgfsetfillcolor{textcolor}%
\pgftext[x=6.108136in, y=0.831626in, left, base,rotate=90.000000]{\color{textcolor}\rmfamily\fontsize{6.000000}{7.200000}\selectfont Mgen}%
\end{pgfscope}%
\begin{pgfscope}%
\pgfsetbuttcap%
\pgfsetroundjoin%
\definecolor{currentfill}{rgb}{0.000000,0.000000,0.000000}%
\pgfsetfillcolor{currentfill}%
\pgfsetlinewidth{0.803000pt}%
\definecolor{currentstroke}{rgb}{0.000000,0.000000,0.000000}%
\pgfsetstrokecolor{currentstroke}%
\pgfsetdash{}{0pt}%
\pgfsys@defobject{currentmarker}{\pgfqpoint{0.000000in}{-0.048611in}}{\pgfqpoint{0.000000in}{0.000000in}}{%
\pgfpathmoveto{\pgfqpoint{0.000000in}{0.000000in}}%
\pgfpathlineto{\pgfqpoint{0.000000in}{-0.048611in}}%
\pgfusepath{stroke,fill}%
}%
\begin{pgfscope}%
\pgfsys@transformshift{6.218776in}{1.172519in}%
\pgfsys@useobject{currentmarker}{}%
\end{pgfscope}%
\end{pgfscope}%
\begin{pgfscope}%
\definecolor{textcolor}{rgb}{0.000000,0.000000,0.000000}%
\pgfsetstrokecolor{textcolor}%
\pgfsetfillcolor{textcolor}%
\pgftext[x=6.239609in, y=0.100000in, left, base,rotate=90.000000]{\color{textcolor}\rmfamily\fontsize{6.000000}{7.200000}\selectfont Mutuelle des Motards}%
\end{pgfscope}%
\begin{pgfscope}%
\pgfsetbuttcap%
\pgfsetroundjoin%
\definecolor{currentfill}{rgb}{0.000000,0.000000,0.000000}%
\pgfsetfillcolor{currentfill}%
\pgfsetlinewidth{0.803000pt}%
\definecolor{currentstroke}{rgb}{0.000000,0.000000,0.000000}%
\pgfsetstrokecolor{currentstroke}%
\pgfsetdash{}{0pt}%
\pgfsys@defobject{currentmarker}{\pgfqpoint{0.000000in}{-0.048611in}}{\pgfqpoint{0.000000in}{0.000000in}}{%
\pgfpathmoveto{\pgfqpoint{0.000000in}{0.000000in}}%
\pgfpathlineto{\pgfqpoint{0.000000in}{-0.048611in}}%
\pgfusepath{stroke,fill}%
}%
\begin{pgfscope}%
\pgfsys@transformshift{6.350249in}{1.172519in}%
\pgfsys@useobject{currentmarker}{}%
\end{pgfscope}%
\end{pgfscope}%
\begin{pgfscope}%
\definecolor{textcolor}{rgb}{0.000000,0.000000,0.000000}%
\pgfsetstrokecolor{textcolor}%
\pgfsetfillcolor{textcolor}%
\pgftext[x=6.371082in, y=0.411493in, left, base,rotate=90.000000]{\color{textcolor}\rmfamily\fontsize{6.000000}{7.200000}\selectfont Néoliane Santé}%
\end{pgfscope}%
\begin{pgfscope}%
\pgfsetbuttcap%
\pgfsetroundjoin%
\definecolor{currentfill}{rgb}{0.000000,0.000000,0.000000}%
\pgfsetfillcolor{currentfill}%
\pgfsetlinewidth{0.803000pt}%
\definecolor{currentstroke}{rgb}{0.000000,0.000000,0.000000}%
\pgfsetstrokecolor{currentstroke}%
\pgfsetdash{}{0pt}%
\pgfsys@defobject{currentmarker}{\pgfqpoint{0.000000in}{-0.048611in}}{\pgfqpoint{0.000000in}{0.000000in}}{%
\pgfpathmoveto{\pgfqpoint{0.000000in}{0.000000in}}%
\pgfpathlineto{\pgfqpoint{0.000000in}{-0.048611in}}%
\pgfusepath{stroke,fill}%
}%
\begin{pgfscope}%
\pgfsys@transformshift{6.481722in}{1.172519in}%
\pgfsys@useobject{currentmarker}{}%
\end{pgfscope}%
\end{pgfscope}%
\begin{pgfscope}%
\definecolor{textcolor}{rgb}{0.000000,0.000000,0.000000}%
\pgfsetstrokecolor{textcolor}%
\pgfsetfillcolor{textcolor}%
\pgftext[x=6.502556in, y=0.728928in, left, base,rotate=90.000000]{\color{textcolor}\rmfamily\fontsize{6.000000}{7.200000}\selectfont Pacifica}%
\end{pgfscope}%
\begin{pgfscope}%
\pgfsetbuttcap%
\pgfsetroundjoin%
\definecolor{currentfill}{rgb}{0.000000,0.000000,0.000000}%
\pgfsetfillcolor{currentfill}%
\pgfsetlinewidth{0.803000pt}%
\definecolor{currentstroke}{rgb}{0.000000,0.000000,0.000000}%
\pgfsetstrokecolor{currentstroke}%
\pgfsetdash{}{0pt}%
\pgfsys@defobject{currentmarker}{\pgfqpoint{0.000000in}{-0.048611in}}{\pgfqpoint{0.000000in}{0.000000in}}{%
\pgfpathmoveto{\pgfqpoint{0.000000in}{0.000000in}}%
\pgfpathlineto{\pgfqpoint{0.000000in}{-0.048611in}}%
\pgfusepath{stroke,fill}%
}%
\begin{pgfscope}%
\pgfsys@transformshift{6.613196in}{1.172519in}%
\pgfsys@useobject{currentmarker}{}%
\end{pgfscope}%
\end{pgfscope}%
\begin{pgfscope}%
\definecolor{textcolor}{rgb}{0.000000,0.000000,0.000000}%
\pgfsetstrokecolor{textcolor}%
\pgfsetfillcolor{textcolor}%
\pgftext[x=6.634029in, y=0.247915in, left, base,rotate=90.000000]{\color{textcolor}\rmfamily\fontsize{6.000000}{7.200000}\selectfont Peyrac Assurances}%
\end{pgfscope}%
\begin{pgfscope}%
\pgfsetbuttcap%
\pgfsetroundjoin%
\definecolor{currentfill}{rgb}{0.000000,0.000000,0.000000}%
\pgfsetfillcolor{currentfill}%
\pgfsetlinewidth{0.803000pt}%
\definecolor{currentstroke}{rgb}{0.000000,0.000000,0.000000}%
\pgfsetstrokecolor{currentstroke}%
\pgfsetdash{}{0pt}%
\pgfsys@defobject{currentmarker}{\pgfqpoint{0.000000in}{-0.048611in}}{\pgfqpoint{0.000000in}{0.000000in}}{%
\pgfpathmoveto{\pgfqpoint{0.000000in}{0.000000in}}%
\pgfpathlineto{\pgfqpoint{0.000000in}{-0.048611in}}%
\pgfusepath{stroke,fill}%
}%
\begin{pgfscope}%
\pgfsys@transformshift{6.744669in}{1.172519in}%
\pgfsys@useobject{currentmarker}{}%
\end{pgfscope}%
\end{pgfscope}%
\begin{pgfscope}%
\definecolor{textcolor}{rgb}{0.000000,0.000000,0.000000}%
\pgfsetstrokecolor{textcolor}%
\pgfsetfillcolor{textcolor}%
\pgftext[x=6.765502in, y=0.692199in, left, base,rotate=90.000000]{\color{textcolor}\rmfamily\fontsize{6.000000}{7.200000}\selectfont Santiane}%
\end{pgfscope}%
\begin{pgfscope}%
\pgfsetbuttcap%
\pgfsetroundjoin%
\definecolor{currentfill}{rgb}{0.000000,0.000000,0.000000}%
\pgfsetfillcolor{currentfill}%
\pgfsetlinewidth{0.803000pt}%
\definecolor{currentstroke}{rgb}{0.000000,0.000000,0.000000}%
\pgfsetstrokecolor{currentstroke}%
\pgfsetdash{}{0pt}%
\pgfsys@defobject{currentmarker}{\pgfqpoint{0.000000in}{-0.048611in}}{\pgfqpoint{0.000000in}{0.000000in}}{%
\pgfpathmoveto{\pgfqpoint{0.000000in}{0.000000in}}%
\pgfpathlineto{\pgfqpoint{0.000000in}{-0.048611in}}%
\pgfusepath{stroke,fill}%
}%
\begin{pgfscope}%
\pgfsys@transformshift{6.876142in}{1.172519in}%
\pgfsys@useobject{currentmarker}{}%
\end{pgfscope}%
\end{pgfscope}%
\begin{pgfscope}%
\definecolor{textcolor}{rgb}{0.000000,0.000000,0.000000}%
\pgfsetstrokecolor{textcolor}%
\pgfsetfillcolor{textcolor}%
\pgftext[x=6.896975in, y=0.676536in, left, base,rotate=90.000000]{\color{textcolor}\rmfamily\fontsize{6.000000}{7.200000}\selectfont SantéVet}%
\end{pgfscope}%
\begin{pgfscope}%
\pgfsetbuttcap%
\pgfsetroundjoin%
\definecolor{currentfill}{rgb}{0.000000,0.000000,0.000000}%
\pgfsetfillcolor{currentfill}%
\pgfsetlinewidth{0.803000pt}%
\definecolor{currentstroke}{rgb}{0.000000,0.000000,0.000000}%
\pgfsetstrokecolor{currentstroke}%
\pgfsetdash{}{0pt}%
\pgfsys@defobject{currentmarker}{\pgfqpoint{0.000000in}{-0.048611in}}{\pgfqpoint{0.000000in}{0.000000in}}{%
\pgfpathmoveto{\pgfqpoint{0.000000in}{0.000000in}}%
\pgfpathlineto{\pgfqpoint{0.000000in}{-0.048611in}}%
\pgfusepath{stroke,fill}%
}%
\begin{pgfscope}%
\pgfsys@transformshift{7.007615in}{1.172519in}%
\pgfsys@useobject{currentmarker}{}%
\end{pgfscope}%
\end{pgfscope}%
\begin{pgfscope}%
\definecolor{textcolor}{rgb}{0.000000,0.000000,0.000000}%
\pgfsetstrokecolor{textcolor}%
\pgfsetfillcolor{textcolor}%
\pgftext[x=7.028449in, y=0.884712in, left, base,rotate=90.000000]{\color{textcolor}\rmfamily\fontsize{6.000000}{7.200000}\selectfont Sma}%
\end{pgfscope}%
\begin{pgfscope}%
\pgfsetbuttcap%
\pgfsetroundjoin%
\definecolor{currentfill}{rgb}{0.000000,0.000000,0.000000}%
\pgfsetfillcolor{currentfill}%
\pgfsetlinewidth{0.803000pt}%
\definecolor{currentstroke}{rgb}{0.000000,0.000000,0.000000}%
\pgfsetstrokecolor{currentstroke}%
\pgfsetdash{}{0pt}%
\pgfsys@defobject{currentmarker}{\pgfqpoint{0.000000in}{-0.048611in}}{\pgfqpoint{0.000000in}{0.000000in}}{%
\pgfpathmoveto{\pgfqpoint{0.000000in}{0.000000in}}%
\pgfpathlineto{\pgfqpoint{0.000000in}{-0.048611in}}%
\pgfusepath{stroke,fill}%
}%
\begin{pgfscope}%
\pgfsys@transformshift{7.139088in}{1.172519in}%
\pgfsys@useobject{currentmarker}{}%
\end{pgfscope}%
\end{pgfscope}%
\begin{pgfscope}%
\definecolor{textcolor}{rgb}{0.000000,0.000000,0.000000}%
\pgfsetstrokecolor{textcolor}%
\pgfsetfillcolor{textcolor}%
\pgftext[x=7.159922in, y=0.718819in, left, base,rotate=90.000000]{\color{textcolor}\rmfamily\fontsize{6.000000}{7.200000}\selectfont Sogecap}%
\end{pgfscope}%
\begin{pgfscope}%
\pgfsetbuttcap%
\pgfsetroundjoin%
\definecolor{currentfill}{rgb}{0.000000,0.000000,0.000000}%
\pgfsetfillcolor{currentfill}%
\pgfsetlinewidth{0.803000pt}%
\definecolor{currentstroke}{rgb}{0.000000,0.000000,0.000000}%
\pgfsetstrokecolor{currentstroke}%
\pgfsetdash{}{0pt}%
\pgfsys@defobject{currentmarker}{\pgfqpoint{0.000000in}{-0.048611in}}{\pgfqpoint{0.000000in}{0.000000in}}{%
\pgfpathmoveto{\pgfqpoint{0.000000in}{0.000000in}}%
\pgfpathlineto{\pgfqpoint{0.000000in}{-0.048611in}}%
\pgfusepath{stroke,fill}%
}%
\begin{pgfscope}%
\pgfsys@transformshift{7.270562in}{1.172519in}%
\pgfsys@useobject{currentmarker}{}%
\end{pgfscope}%
\end{pgfscope}%
\begin{pgfscope}%
\definecolor{textcolor}{rgb}{0.000000,0.000000,0.000000}%
\pgfsetstrokecolor{textcolor}%
\pgfsetfillcolor{textcolor}%
\pgftext[x=7.291395in, y=0.693820in, left, base,rotate=90.000000]{\color{textcolor}\rmfamily\fontsize{6.000000}{7.200000}\selectfont Sogessur}%
\end{pgfscope}%
\begin{pgfscope}%
\pgfsetbuttcap%
\pgfsetroundjoin%
\definecolor{currentfill}{rgb}{0.000000,0.000000,0.000000}%
\pgfsetfillcolor{currentfill}%
\pgfsetlinewidth{0.803000pt}%
\definecolor{currentstroke}{rgb}{0.000000,0.000000,0.000000}%
\pgfsetstrokecolor{currentstroke}%
\pgfsetdash{}{0pt}%
\pgfsys@defobject{currentmarker}{\pgfqpoint{0.000000in}{-0.048611in}}{\pgfqpoint{0.000000in}{0.000000in}}{%
\pgfpathmoveto{\pgfqpoint{0.000000in}{0.000000in}}%
\pgfpathlineto{\pgfqpoint{0.000000in}{-0.048611in}}%
\pgfusepath{stroke,fill}%
}%
\begin{pgfscope}%
\pgfsys@transformshift{7.402035in}{1.172519in}%
\pgfsys@useobject{currentmarker}{}%
\end{pgfscope}%
\end{pgfscope}%
\begin{pgfscope}%
\definecolor{textcolor}{rgb}{0.000000,0.000000,0.000000}%
\pgfsetstrokecolor{textcolor}%
\pgfsetfillcolor{textcolor}%
\pgftext[x=7.422868in, y=0.609793in, left, base,rotate=90.000000]{\color{textcolor}\rmfamily\fontsize{6.000000}{7.200000}\selectfont Solly Azar}%
\end{pgfscope}%
\begin{pgfscope}%
\pgfsetbuttcap%
\pgfsetroundjoin%
\definecolor{currentfill}{rgb}{0.000000,0.000000,0.000000}%
\pgfsetfillcolor{currentfill}%
\pgfsetlinewidth{0.803000pt}%
\definecolor{currentstroke}{rgb}{0.000000,0.000000,0.000000}%
\pgfsetstrokecolor{currentstroke}%
\pgfsetdash{}{0pt}%
\pgfsys@defobject{currentmarker}{\pgfqpoint{0.000000in}{-0.048611in}}{\pgfqpoint{0.000000in}{0.000000in}}{%
\pgfpathmoveto{\pgfqpoint{0.000000in}{0.000000in}}%
\pgfpathlineto{\pgfqpoint{0.000000in}{-0.048611in}}%
\pgfusepath{stroke,fill}%
}%
\begin{pgfscope}%
\pgfsys@transformshift{7.533508in}{1.172519in}%
\pgfsys@useobject{currentmarker}{}%
\end{pgfscope}%
\end{pgfscope}%
\begin{pgfscope}%
\definecolor{textcolor}{rgb}{0.000000,0.000000,0.000000}%
\pgfsetstrokecolor{textcolor}%
\pgfsetfillcolor{textcolor}%
\pgftext[x=7.554341in, y=0.652076in, left, base,rotate=90.000000]{\color{textcolor}\rmfamily\fontsize{6.000000}{7.200000}\selectfont Suravenir}%
\end{pgfscope}%
\begin{pgfscope}%
\pgfsetbuttcap%
\pgfsetroundjoin%
\definecolor{currentfill}{rgb}{0.000000,0.000000,0.000000}%
\pgfsetfillcolor{currentfill}%
\pgfsetlinewidth{0.803000pt}%
\definecolor{currentstroke}{rgb}{0.000000,0.000000,0.000000}%
\pgfsetstrokecolor{currentstroke}%
\pgfsetdash{}{0pt}%
\pgfsys@defobject{currentmarker}{\pgfqpoint{0.000000in}{-0.048611in}}{\pgfqpoint{0.000000in}{0.000000in}}{%
\pgfpathmoveto{\pgfqpoint{0.000000in}{0.000000in}}%
\pgfpathlineto{\pgfqpoint{0.000000in}{-0.048611in}}%
\pgfusepath{stroke,fill}%
}%
\begin{pgfscope}%
\pgfsys@transformshift{7.664981in}{1.172519in}%
\pgfsys@useobject{currentmarker}{}%
\end{pgfscope}%
\end{pgfscope}%
\begin{pgfscope}%
\definecolor{textcolor}{rgb}{0.000000,0.000000,0.000000}%
\pgfsetstrokecolor{textcolor}%
\pgfsetfillcolor{textcolor}%
\pgftext[x=7.685815in, y=0.666119in, left, base,rotate=90.000000]{\color{textcolor}\rmfamily\fontsize{6.000000}{7.200000}\selectfont SwissLife}%
\end{pgfscope}%
\begin{pgfscope}%
\pgfsetbuttcap%
\pgfsetroundjoin%
\definecolor{currentfill}{rgb}{0.000000,0.000000,0.000000}%
\pgfsetfillcolor{currentfill}%
\pgfsetlinewidth{0.803000pt}%
\definecolor{currentstroke}{rgb}{0.000000,0.000000,0.000000}%
\pgfsetstrokecolor{currentstroke}%
\pgfsetdash{}{0pt}%
\pgfsys@defobject{currentmarker}{\pgfqpoint{0.000000in}{-0.048611in}}{\pgfqpoint{0.000000in}{0.000000in}}{%
\pgfpathmoveto{\pgfqpoint{0.000000in}{0.000000in}}%
\pgfpathlineto{\pgfqpoint{0.000000in}{-0.048611in}}%
\pgfusepath{stroke,fill}%
}%
\begin{pgfscope}%
\pgfsys@transformshift{7.796455in}{1.172519in}%
\pgfsys@useobject{currentmarker}{}%
\end{pgfscope}%
\end{pgfscope}%
\begin{pgfscope}%
\definecolor{textcolor}{rgb}{0.000000,0.000000,0.000000}%
\pgfsetstrokecolor{textcolor}%
\pgfsetfillcolor{textcolor}%
\pgftext[x=7.817288in, y=0.751380in, left, base,rotate=90.000000]{\color{textcolor}\rmfamily\fontsize{6.000000}{7.200000}\selectfont Zen'Up}%
\end{pgfscope}%
\begin{pgfscope}%
\pgfsetbuttcap%
\pgfsetroundjoin%
\definecolor{currentfill}{rgb}{0.000000,0.000000,0.000000}%
\pgfsetfillcolor{currentfill}%
\pgfsetlinewidth{0.803000pt}%
\definecolor{currentstroke}{rgb}{0.000000,0.000000,0.000000}%
\pgfsetstrokecolor{currentstroke}%
\pgfsetdash{}{0pt}%
\pgfsys@defobject{currentmarker}{\pgfqpoint{-0.048611in}{0.000000in}}{\pgfqpoint{-0.000000in}{0.000000in}}{%
\pgfpathmoveto{\pgfqpoint{-0.000000in}{0.000000in}}%
\pgfpathlineto{\pgfqpoint{-0.048611in}{0.000000in}}%
\pgfusepath{stroke,fill}%
}%
\begin{pgfscope}%
\pgfsys@transformshift{0.499691in}{1.172519in}%
\pgfsys@useobject{currentmarker}{}%
\end{pgfscope}%
\end{pgfscope}%
\begin{pgfscope}%
\definecolor{textcolor}{rgb}{0.000000,0.000000,0.000000}%
\pgfsetstrokecolor{textcolor}%
\pgfsetfillcolor{textcolor}%
\pgftext[x=0.375463in, y=1.151299in, left, base,rotate=90.000000]{\color{textcolor}\rmfamily\fontsize{10.000000}{12.000000}\selectfont \(\displaystyle {0}\)}%
\end{pgfscope}%
\begin{pgfscope}%
\pgfsetbuttcap%
\pgfsetroundjoin%
\definecolor{currentfill}{rgb}{0.000000,0.000000,0.000000}%
\pgfsetfillcolor{currentfill}%
\pgfsetlinewidth{0.803000pt}%
\definecolor{currentstroke}{rgb}{0.000000,0.000000,0.000000}%
\pgfsetstrokecolor{currentstroke}%
\pgfsetdash{}{0pt}%
\pgfsys@defobject{currentmarker}{\pgfqpoint{-0.048611in}{0.000000in}}{\pgfqpoint{-0.000000in}{0.000000in}}{%
\pgfpathmoveto{\pgfqpoint{-0.000000in}{0.000000in}}%
\pgfpathlineto{\pgfqpoint{-0.048611in}{0.000000in}}%
\pgfusepath{stroke,fill}%
}%
\begin{pgfscope}%
\pgfsys@transformshift{0.499691in}{1.751022in}%
\pgfsys@useobject{currentmarker}{}%
\end{pgfscope}%
\end{pgfscope}%
\begin{pgfscope}%
\definecolor{textcolor}{rgb}{0.000000,0.000000,0.000000}%
\pgfsetstrokecolor{textcolor}%
\pgfsetfillcolor{textcolor}%
\pgftext[x=0.375463in, y=1.729803in, left, base,rotate=90.000000]{\color{textcolor}\rmfamily\fontsize{10.000000}{12.000000}\selectfont \(\displaystyle {1}\)}%
\end{pgfscope}%
\begin{pgfscope}%
\pgfsetbuttcap%
\pgfsetroundjoin%
\definecolor{currentfill}{rgb}{0.000000,0.000000,0.000000}%
\pgfsetfillcolor{currentfill}%
\pgfsetlinewidth{0.803000pt}%
\definecolor{currentstroke}{rgb}{0.000000,0.000000,0.000000}%
\pgfsetstrokecolor{currentstroke}%
\pgfsetdash{}{0pt}%
\pgfsys@defobject{currentmarker}{\pgfqpoint{-0.048611in}{0.000000in}}{\pgfqpoint{-0.000000in}{0.000000in}}{%
\pgfpathmoveto{\pgfqpoint{-0.000000in}{0.000000in}}%
\pgfpathlineto{\pgfqpoint{-0.048611in}{0.000000in}}%
\pgfusepath{stroke,fill}%
}%
\begin{pgfscope}%
\pgfsys@transformshift{0.499691in}{2.329526in}%
\pgfsys@useobject{currentmarker}{}%
\end{pgfscope}%
\end{pgfscope}%
\begin{pgfscope}%
\definecolor{textcolor}{rgb}{0.000000,0.000000,0.000000}%
\pgfsetstrokecolor{textcolor}%
\pgfsetfillcolor{textcolor}%
\pgftext[x=0.375463in, y=2.308306in, left, base,rotate=90.000000]{\color{textcolor}\rmfamily\fontsize{10.000000}{12.000000}\selectfont \(\displaystyle {2}\)}%
\end{pgfscope}%
\begin{pgfscope}%
\pgfsetbuttcap%
\pgfsetroundjoin%
\definecolor{currentfill}{rgb}{0.000000,0.000000,0.000000}%
\pgfsetfillcolor{currentfill}%
\pgfsetlinewidth{0.803000pt}%
\definecolor{currentstroke}{rgb}{0.000000,0.000000,0.000000}%
\pgfsetstrokecolor{currentstroke}%
\pgfsetdash{}{0pt}%
\pgfsys@defobject{currentmarker}{\pgfqpoint{-0.048611in}{0.000000in}}{\pgfqpoint{-0.000000in}{0.000000in}}{%
\pgfpathmoveto{\pgfqpoint{-0.000000in}{0.000000in}}%
\pgfpathlineto{\pgfqpoint{-0.048611in}{0.000000in}}%
\pgfusepath{stroke,fill}%
}%
\begin{pgfscope}%
\pgfsys@transformshift{0.499691in}{2.908029in}%
\pgfsys@useobject{currentmarker}{}%
\end{pgfscope}%
\end{pgfscope}%
\begin{pgfscope}%
\definecolor{textcolor}{rgb}{0.000000,0.000000,0.000000}%
\pgfsetstrokecolor{textcolor}%
\pgfsetfillcolor{textcolor}%
\pgftext[x=0.375463in, y=2.886810in, left, base,rotate=90.000000]{\color{textcolor}\rmfamily\fontsize{10.000000}{12.000000}\selectfont \(\displaystyle {3}\)}%
\end{pgfscope}%
\begin{pgfscope}%
\pgfsetbuttcap%
\pgfsetroundjoin%
\definecolor{currentfill}{rgb}{0.000000,0.000000,0.000000}%
\pgfsetfillcolor{currentfill}%
\pgfsetlinewidth{0.803000pt}%
\definecolor{currentstroke}{rgb}{0.000000,0.000000,0.000000}%
\pgfsetstrokecolor{currentstroke}%
\pgfsetdash{}{0pt}%
\pgfsys@defobject{currentmarker}{\pgfqpoint{-0.048611in}{0.000000in}}{\pgfqpoint{-0.000000in}{0.000000in}}{%
\pgfpathmoveto{\pgfqpoint{-0.000000in}{0.000000in}}%
\pgfpathlineto{\pgfqpoint{-0.048611in}{0.000000in}}%
\pgfusepath{stroke,fill}%
}%
\begin{pgfscope}%
\pgfsys@transformshift{0.499691in}{3.486533in}%
\pgfsys@useobject{currentmarker}{}%
\end{pgfscope}%
\end{pgfscope}%
\begin{pgfscope}%
\definecolor{textcolor}{rgb}{0.000000,0.000000,0.000000}%
\pgfsetstrokecolor{textcolor}%
\pgfsetfillcolor{textcolor}%
\pgftext[x=0.375463in, y=3.465313in, left, base,rotate=90.000000]{\color{textcolor}\rmfamily\fontsize{10.000000}{12.000000}\selectfont \(\displaystyle {4}\)}%
\end{pgfscope}%
\begin{pgfscope}%
\definecolor{textcolor}{rgb}{0.000000,0.000000,0.000000}%
\pgfsetstrokecolor{textcolor}%
\pgfsetfillcolor{textcolor}%
\pgftext[x=0.223457in,y=2.520019in,,bottom,rotate=90.000000]{\color{textcolor}\rmfamily\fontsize{10.000000}{12.000000}\selectfont Mean note}%
\end{pgfscope}%
\begin{pgfscope}%
\pgfpathrectangle{\pgfqpoint{0.499691in}{1.172519in}}{\pgfqpoint{7.362500in}{2.695000in}}%
\pgfusepath{clip}%
\pgfsetrectcap%
\pgfsetroundjoin%
\pgfsetlinewidth{2.710125pt}%
\definecolor{currentstroke}{rgb}{0.260000,0.260000,0.260000}%
\pgfsetstrokecolor{currentstroke}%
\pgfsetdash{}{0pt}%
\pgfusepath{stroke}%
\end{pgfscope}%
\begin{pgfscope}%
\pgfpathrectangle{\pgfqpoint{0.499691in}{1.172519in}}{\pgfqpoint{7.362500in}{2.695000in}}%
\pgfusepath{clip}%
\pgfsetrectcap%
\pgfsetroundjoin%
\pgfsetlinewidth{2.710125pt}%
\definecolor{currentstroke}{rgb}{0.260000,0.260000,0.260000}%
\pgfsetstrokecolor{currentstroke}%
\pgfsetdash{}{0pt}%
\pgfusepath{stroke}%
\end{pgfscope}%
\begin{pgfscope}%
\pgfpathrectangle{\pgfqpoint{0.499691in}{1.172519in}}{\pgfqpoint{7.362500in}{2.695000in}}%
\pgfusepath{clip}%
\pgfsetrectcap%
\pgfsetroundjoin%
\pgfsetlinewidth{2.710125pt}%
\definecolor{currentstroke}{rgb}{0.260000,0.260000,0.260000}%
\pgfsetstrokecolor{currentstroke}%
\pgfsetdash{}{0pt}%
\pgfusepath{stroke}%
\end{pgfscope}%
\begin{pgfscope}%
\pgfpathrectangle{\pgfqpoint{0.499691in}{1.172519in}}{\pgfqpoint{7.362500in}{2.695000in}}%
\pgfusepath{clip}%
\pgfsetrectcap%
\pgfsetroundjoin%
\pgfsetlinewidth{2.710125pt}%
\definecolor{currentstroke}{rgb}{0.260000,0.260000,0.260000}%
\pgfsetstrokecolor{currentstroke}%
\pgfsetdash{}{0pt}%
\pgfusepath{stroke}%
\end{pgfscope}%
\begin{pgfscope}%
\pgfpathrectangle{\pgfqpoint{0.499691in}{1.172519in}}{\pgfqpoint{7.362500in}{2.695000in}}%
\pgfusepath{clip}%
\pgfsetrectcap%
\pgfsetroundjoin%
\pgfsetlinewidth{2.710125pt}%
\definecolor{currentstroke}{rgb}{0.260000,0.260000,0.260000}%
\pgfsetstrokecolor{currentstroke}%
\pgfsetdash{}{0pt}%
\pgfusepath{stroke}%
\end{pgfscope}%
\begin{pgfscope}%
\pgfpathrectangle{\pgfqpoint{0.499691in}{1.172519in}}{\pgfqpoint{7.362500in}{2.695000in}}%
\pgfusepath{clip}%
\pgfsetrectcap%
\pgfsetroundjoin%
\pgfsetlinewidth{2.710125pt}%
\definecolor{currentstroke}{rgb}{0.260000,0.260000,0.260000}%
\pgfsetstrokecolor{currentstroke}%
\pgfsetdash{}{0pt}%
\pgfusepath{stroke}%
\end{pgfscope}%
\begin{pgfscope}%
\pgfpathrectangle{\pgfqpoint{0.499691in}{1.172519in}}{\pgfqpoint{7.362500in}{2.695000in}}%
\pgfusepath{clip}%
\pgfsetrectcap%
\pgfsetroundjoin%
\pgfsetlinewidth{2.710125pt}%
\definecolor{currentstroke}{rgb}{0.260000,0.260000,0.260000}%
\pgfsetstrokecolor{currentstroke}%
\pgfsetdash{}{0pt}%
\pgfusepath{stroke}%
\end{pgfscope}%
\begin{pgfscope}%
\pgfpathrectangle{\pgfqpoint{0.499691in}{1.172519in}}{\pgfqpoint{7.362500in}{2.695000in}}%
\pgfusepath{clip}%
\pgfsetrectcap%
\pgfsetroundjoin%
\pgfsetlinewidth{2.710125pt}%
\definecolor{currentstroke}{rgb}{0.260000,0.260000,0.260000}%
\pgfsetstrokecolor{currentstroke}%
\pgfsetdash{}{0pt}%
\pgfusepath{stroke}%
\end{pgfscope}%
\begin{pgfscope}%
\pgfpathrectangle{\pgfqpoint{0.499691in}{1.172519in}}{\pgfqpoint{7.362500in}{2.695000in}}%
\pgfusepath{clip}%
\pgfsetrectcap%
\pgfsetroundjoin%
\pgfsetlinewidth{2.710125pt}%
\definecolor{currentstroke}{rgb}{0.260000,0.260000,0.260000}%
\pgfsetstrokecolor{currentstroke}%
\pgfsetdash{}{0pt}%
\pgfusepath{stroke}%
\end{pgfscope}%
\begin{pgfscope}%
\pgfpathrectangle{\pgfqpoint{0.499691in}{1.172519in}}{\pgfqpoint{7.362500in}{2.695000in}}%
\pgfusepath{clip}%
\pgfsetrectcap%
\pgfsetroundjoin%
\pgfsetlinewidth{2.710125pt}%
\definecolor{currentstroke}{rgb}{0.260000,0.260000,0.260000}%
\pgfsetstrokecolor{currentstroke}%
\pgfsetdash{}{0pt}%
\pgfusepath{stroke}%
\end{pgfscope}%
\begin{pgfscope}%
\pgfpathrectangle{\pgfqpoint{0.499691in}{1.172519in}}{\pgfqpoint{7.362500in}{2.695000in}}%
\pgfusepath{clip}%
\pgfsetrectcap%
\pgfsetroundjoin%
\pgfsetlinewidth{2.710125pt}%
\definecolor{currentstroke}{rgb}{0.260000,0.260000,0.260000}%
\pgfsetstrokecolor{currentstroke}%
\pgfsetdash{}{0pt}%
\pgfusepath{stroke}%
\end{pgfscope}%
\begin{pgfscope}%
\pgfpathrectangle{\pgfqpoint{0.499691in}{1.172519in}}{\pgfqpoint{7.362500in}{2.695000in}}%
\pgfusepath{clip}%
\pgfsetrectcap%
\pgfsetroundjoin%
\pgfsetlinewidth{2.710125pt}%
\definecolor{currentstroke}{rgb}{0.260000,0.260000,0.260000}%
\pgfsetstrokecolor{currentstroke}%
\pgfsetdash{}{0pt}%
\pgfusepath{stroke}%
\end{pgfscope}%
\begin{pgfscope}%
\pgfpathrectangle{\pgfqpoint{0.499691in}{1.172519in}}{\pgfqpoint{7.362500in}{2.695000in}}%
\pgfusepath{clip}%
\pgfsetrectcap%
\pgfsetroundjoin%
\pgfsetlinewidth{2.710125pt}%
\definecolor{currentstroke}{rgb}{0.260000,0.260000,0.260000}%
\pgfsetstrokecolor{currentstroke}%
\pgfsetdash{}{0pt}%
\pgfusepath{stroke}%
\end{pgfscope}%
\begin{pgfscope}%
\pgfpathrectangle{\pgfqpoint{0.499691in}{1.172519in}}{\pgfqpoint{7.362500in}{2.695000in}}%
\pgfusepath{clip}%
\pgfsetrectcap%
\pgfsetroundjoin%
\pgfsetlinewidth{2.710125pt}%
\definecolor{currentstroke}{rgb}{0.260000,0.260000,0.260000}%
\pgfsetstrokecolor{currentstroke}%
\pgfsetdash{}{0pt}%
\pgfusepath{stroke}%
\end{pgfscope}%
\begin{pgfscope}%
\pgfpathrectangle{\pgfqpoint{0.499691in}{1.172519in}}{\pgfqpoint{7.362500in}{2.695000in}}%
\pgfusepath{clip}%
\pgfsetrectcap%
\pgfsetroundjoin%
\pgfsetlinewidth{2.710125pt}%
\definecolor{currentstroke}{rgb}{0.260000,0.260000,0.260000}%
\pgfsetstrokecolor{currentstroke}%
\pgfsetdash{}{0pt}%
\pgfusepath{stroke}%
\end{pgfscope}%
\begin{pgfscope}%
\pgfpathrectangle{\pgfqpoint{0.499691in}{1.172519in}}{\pgfqpoint{7.362500in}{2.695000in}}%
\pgfusepath{clip}%
\pgfsetrectcap%
\pgfsetroundjoin%
\pgfsetlinewidth{2.710125pt}%
\definecolor{currentstroke}{rgb}{0.260000,0.260000,0.260000}%
\pgfsetstrokecolor{currentstroke}%
\pgfsetdash{}{0pt}%
\pgfusepath{stroke}%
\end{pgfscope}%
\begin{pgfscope}%
\pgfpathrectangle{\pgfqpoint{0.499691in}{1.172519in}}{\pgfqpoint{7.362500in}{2.695000in}}%
\pgfusepath{clip}%
\pgfsetrectcap%
\pgfsetroundjoin%
\pgfsetlinewidth{2.710125pt}%
\definecolor{currentstroke}{rgb}{0.260000,0.260000,0.260000}%
\pgfsetstrokecolor{currentstroke}%
\pgfsetdash{}{0pt}%
\pgfusepath{stroke}%
\end{pgfscope}%
\begin{pgfscope}%
\pgfpathrectangle{\pgfqpoint{0.499691in}{1.172519in}}{\pgfqpoint{7.362500in}{2.695000in}}%
\pgfusepath{clip}%
\pgfsetrectcap%
\pgfsetroundjoin%
\pgfsetlinewidth{2.710125pt}%
\definecolor{currentstroke}{rgb}{0.260000,0.260000,0.260000}%
\pgfsetstrokecolor{currentstroke}%
\pgfsetdash{}{0pt}%
\pgfusepath{stroke}%
\end{pgfscope}%
\begin{pgfscope}%
\pgfpathrectangle{\pgfqpoint{0.499691in}{1.172519in}}{\pgfqpoint{7.362500in}{2.695000in}}%
\pgfusepath{clip}%
\pgfsetrectcap%
\pgfsetroundjoin%
\pgfsetlinewidth{2.710125pt}%
\definecolor{currentstroke}{rgb}{0.260000,0.260000,0.260000}%
\pgfsetstrokecolor{currentstroke}%
\pgfsetdash{}{0pt}%
\pgfusepath{stroke}%
\end{pgfscope}%
\begin{pgfscope}%
\pgfpathrectangle{\pgfqpoint{0.499691in}{1.172519in}}{\pgfqpoint{7.362500in}{2.695000in}}%
\pgfusepath{clip}%
\pgfsetrectcap%
\pgfsetroundjoin%
\pgfsetlinewidth{2.710125pt}%
\definecolor{currentstroke}{rgb}{0.260000,0.260000,0.260000}%
\pgfsetstrokecolor{currentstroke}%
\pgfsetdash{}{0pt}%
\pgfusepath{stroke}%
\end{pgfscope}%
\begin{pgfscope}%
\pgfpathrectangle{\pgfqpoint{0.499691in}{1.172519in}}{\pgfqpoint{7.362500in}{2.695000in}}%
\pgfusepath{clip}%
\pgfsetrectcap%
\pgfsetroundjoin%
\pgfsetlinewidth{2.710125pt}%
\definecolor{currentstroke}{rgb}{0.260000,0.260000,0.260000}%
\pgfsetstrokecolor{currentstroke}%
\pgfsetdash{}{0pt}%
\pgfusepath{stroke}%
\end{pgfscope}%
\begin{pgfscope}%
\pgfpathrectangle{\pgfqpoint{0.499691in}{1.172519in}}{\pgfqpoint{7.362500in}{2.695000in}}%
\pgfusepath{clip}%
\pgfsetrectcap%
\pgfsetroundjoin%
\pgfsetlinewidth{2.710125pt}%
\definecolor{currentstroke}{rgb}{0.260000,0.260000,0.260000}%
\pgfsetstrokecolor{currentstroke}%
\pgfsetdash{}{0pt}%
\pgfusepath{stroke}%
\end{pgfscope}%
\begin{pgfscope}%
\pgfpathrectangle{\pgfqpoint{0.499691in}{1.172519in}}{\pgfqpoint{7.362500in}{2.695000in}}%
\pgfusepath{clip}%
\pgfsetrectcap%
\pgfsetroundjoin%
\pgfsetlinewidth{2.710125pt}%
\definecolor{currentstroke}{rgb}{0.260000,0.260000,0.260000}%
\pgfsetstrokecolor{currentstroke}%
\pgfsetdash{}{0pt}%
\pgfusepath{stroke}%
\end{pgfscope}%
\begin{pgfscope}%
\pgfpathrectangle{\pgfqpoint{0.499691in}{1.172519in}}{\pgfqpoint{7.362500in}{2.695000in}}%
\pgfusepath{clip}%
\pgfsetrectcap%
\pgfsetroundjoin%
\pgfsetlinewidth{2.710125pt}%
\definecolor{currentstroke}{rgb}{0.260000,0.260000,0.260000}%
\pgfsetstrokecolor{currentstroke}%
\pgfsetdash{}{0pt}%
\pgfusepath{stroke}%
\end{pgfscope}%
\begin{pgfscope}%
\pgfpathrectangle{\pgfqpoint{0.499691in}{1.172519in}}{\pgfqpoint{7.362500in}{2.695000in}}%
\pgfusepath{clip}%
\pgfsetrectcap%
\pgfsetroundjoin%
\pgfsetlinewidth{2.710125pt}%
\definecolor{currentstroke}{rgb}{0.260000,0.260000,0.260000}%
\pgfsetstrokecolor{currentstroke}%
\pgfsetdash{}{0pt}%
\pgfusepath{stroke}%
\end{pgfscope}%
\begin{pgfscope}%
\pgfpathrectangle{\pgfqpoint{0.499691in}{1.172519in}}{\pgfqpoint{7.362500in}{2.695000in}}%
\pgfusepath{clip}%
\pgfsetrectcap%
\pgfsetroundjoin%
\pgfsetlinewidth{2.710125pt}%
\definecolor{currentstroke}{rgb}{0.260000,0.260000,0.260000}%
\pgfsetstrokecolor{currentstroke}%
\pgfsetdash{}{0pt}%
\pgfusepath{stroke}%
\end{pgfscope}%
\begin{pgfscope}%
\pgfpathrectangle{\pgfqpoint{0.499691in}{1.172519in}}{\pgfqpoint{7.362500in}{2.695000in}}%
\pgfusepath{clip}%
\pgfsetrectcap%
\pgfsetroundjoin%
\pgfsetlinewidth{2.710125pt}%
\definecolor{currentstroke}{rgb}{0.260000,0.260000,0.260000}%
\pgfsetstrokecolor{currentstroke}%
\pgfsetdash{}{0pt}%
\pgfusepath{stroke}%
\end{pgfscope}%
\begin{pgfscope}%
\pgfpathrectangle{\pgfqpoint{0.499691in}{1.172519in}}{\pgfqpoint{7.362500in}{2.695000in}}%
\pgfusepath{clip}%
\pgfsetrectcap%
\pgfsetroundjoin%
\pgfsetlinewidth{2.710125pt}%
\definecolor{currentstroke}{rgb}{0.260000,0.260000,0.260000}%
\pgfsetstrokecolor{currentstroke}%
\pgfsetdash{}{0pt}%
\pgfusepath{stroke}%
\end{pgfscope}%
\begin{pgfscope}%
\pgfpathrectangle{\pgfqpoint{0.499691in}{1.172519in}}{\pgfqpoint{7.362500in}{2.695000in}}%
\pgfusepath{clip}%
\pgfsetrectcap%
\pgfsetroundjoin%
\pgfsetlinewidth{2.710125pt}%
\definecolor{currentstroke}{rgb}{0.260000,0.260000,0.260000}%
\pgfsetstrokecolor{currentstroke}%
\pgfsetdash{}{0pt}%
\pgfusepath{stroke}%
\end{pgfscope}%
\begin{pgfscope}%
\pgfpathrectangle{\pgfqpoint{0.499691in}{1.172519in}}{\pgfqpoint{7.362500in}{2.695000in}}%
\pgfusepath{clip}%
\pgfsetrectcap%
\pgfsetroundjoin%
\pgfsetlinewidth{2.710125pt}%
\definecolor{currentstroke}{rgb}{0.260000,0.260000,0.260000}%
\pgfsetstrokecolor{currentstroke}%
\pgfsetdash{}{0pt}%
\pgfusepath{stroke}%
\end{pgfscope}%
\begin{pgfscope}%
\pgfpathrectangle{\pgfqpoint{0.499691in}{1.172519in}}{\pgfqpoint{7.362500in}{2.695000in}}%
\pgfusepath{clip}%
\pgfsetrectcap%
\pgfsetroundjoin%
\pgfsetlinewidth{2.710125pt}%
\definecolor{currentstroke}{rgb}{0.260000,0.260000,0.260000}%
\pgfsetstrokecolor{currentstroke}%
\pgfsetdash{}{0pt}%
\pgfusepath{stroke}%
\end{pgfscope}%
\begin{pgfscope}%
\pgfpathrectangle{\pgfqpoint{0.499691in}{1.172519in}}{\pgfqpoint{7.362500in}{2.695000in}}%
\pgfusepath{clip}%
\pgfsetrectcap%
\pgfsetroundjoin%
\pgfsetlinewidth{2.710125pt}%
\definecolor{currentstroke}{rgb}{0.260000,0.260000,0.260000}%
\pgfsetstrokecolor{currentstroke}%
\pgfsetdash{}{0pt}%
\pgfusepath{stroke}%
\end{pgfscope}%
\begin{pgfscope}%
\pgfpathrectangle{\pgfqpoint{0.499691in}{1.172519in}}{\pgfqpoint{7.362500in}{2.695000in}}%
\pgfusepath{clip}%
\pgfsetrectcap%
\pgfsetroundjoin%
\pgfsetlinewidth{2.710125pt}%
\definecolor{currentstroke}{rgb}{0.260000,0.260000,0.260000}%
\pgfsetstrokecolor{currentstroke}%
\pgfsetdash{}{0pt}%
\pgfusepath{stroke}%
\end{pgfscope}%
\begin{pgfscope}%
\pgfpathrectangle{\pgfqpoint{0.499691in}{1.172519in}}{\pgfqpoint{7.362500in}{2.695000in}}%
\pgfusepath{clip}%
\pgfsetrectcap%
\pgfsetroundjoin%
\pgfsetlinewidth{2.710125pt}%
\definecolor{currentstroke}{rgb}{0.260000,0.260000,0.260000}%
\pgfsetstrokecolor{currentstroke}%
\pgfsetdash{}{0pt}%
\pgfusepath{stroke}%
\end{pgfscope}%
\begin{pgfscope}%
\pgfpathrectangle{\pgfqpoint{0.499691in}{1.172519in}}{\pgfqpoint{7.362500in}{2.695000in}}%
\pgfusepath{clip}%
\pgfsetrectcap%
\pgfsetroundjoin%
\pgfsetlinewidth{2.710125pt}%
\definecolor{currentstroke}{rgb}{0.260000,0.260000,0.260000}%
\pgfsetstrokecolor{currentstroke}%
\pgfsetdash{}{0pt}%
\pgfusepath{stroke}%
\end{pgfscope}%
\begin{pgfscope}%
\pgfpathrectangle{\pgfqpoint{0.499691in}{1.172519in}}{\pgfqpoint{7.362500in}{2.695000in}}%
\pgfusepath{clip}%
\pgfsetrectcap%
\pgfsetroundjoin%
\pgfsetlinewidth{2.710125pt}%
\definecolor{currentstroke}{rgb}{0.260000,0.260000,0.260000}%
\pgfsetstrokecolor{currentstroke}%
\pgfsetdash{}{0pt}%
\pgfusepath{stroke}%
\end{pgfscope}%
\begin{pgfscope}%
\pgfpathrectangle{\pgfqpoint{0.499691in}{1.172519in}}{\pgfqpoint{7.362500in}{2.695000in}}%
\pgfusepath{clip}%
\pgfsetrectcap%
\pgfsetroundjoin%
\pgfsetlinewidth{2.710125pt}%
\definecolor{currentstroke}{rgb}{0.260000,0.260000,0.260000}%
\pgfsetstrokecolor{currentstroke}%
\pgfsetdash{}{0pt}%
\pgfusepath{stroke}%
\end{pgfscope}%
\begin{pgfscope}%
\pgfpathrectangle{\pgfqpoint{0.499691in}{1.172519in}}{\pgfqpoint{7.362500in}{2.695000in}}%
\pgfusepath{clip}%
\pgfsetrectcap%
\pgfsetroundjoin%
\pgfsetlinewidth{2.710125pt}%
\definecolor{currentstroke}{rgb}{0.260000,0.260000,0.260000}%
\pgfsetstrokecolor{currentstroke}%
\pgfsetdash{}{0pt}%
\pgfusepath{stroke}%
\end{pgfscope}%
\begin{pgfscope}%
\pgfpathrectangle{\pgfqpoint{0.499691in}{1.172519in}}{\pgfqpoint{7.362500in}{2.695000in}}%
\pgfusepath{clip}%
\pgfsetrectcap%
\pgfsetroundjoin%
\pgfsetlinewidth{2.710125pt}%
\definecolor{currentstroke}{rgb}{0.260000,0.260000,0.260000}%
\pgfsetstrokecolor{currentstroke}%
\pgfsetdash{}{0pt}%
\pgfusepath{stroke}%
\end{pgfscope}%
\begin{pgfscope}%
\pgfpathrectangle{\pgfqpoint{0.499691in}{1.172519in}}{\pgfqpoint{7.362500in}{2.695000in}}%
\pgfusepath{clip}%
\pgfsetrectcap%
\pgfsetroundjoin%
\pgfsetlinewidth{2.710125pt}%
\definecolor{currentstroke}{rgb}{0.260000,0.260000,0.260000}%
\pgfsetstrokecolor{currentstroke}%
\pgfsetdash{}{0pt}%
\pgfusepath{stroke}%
\end{pgfscope}%
\begin{pgfscope}%
\pgfpathrectangle{\pgfqpoint{0.499691in}{1.172519in}}{\pgfqpoint{7.362500in}{2.695000in}}%
\pgfusepath{clip}%
\pgfsetrectcap%
\pgfsetroundjoin%
\pgfsetlinewidth{2.710125pt}%
\definecolor{currentstroke}{rgb}{0.260000,0.260000,0.260000}%
\pgfsetstrokecolor{currentstroke}%
\pgfsetdash{}{0pt}%
\pgfusepath{stroke}%
\end{pgfscope}%
\begin{pgfscope}%
\pgfpathrectangle{\pgfqpoint{0.499691in}{1.172519in}}{\pgfqpoint{7.362500in}{2.695000in}}%
\pgfusepath{clip}%
\pgfsetrectcap%
\pgfsetroundjoin%
\pgfsetlinewidth{2.710125pt}%
\definecolor{currentstroke}{rgb}{0.260000,0.260000,0.260000}%
\pgfsetstrokecolor{currentstroke}%
\pgfsetdash{}{0pt}%
\pgfusepath{stroke}%
\end{pgfscope}%
\begin{pgfscope}%
\pgfpathrectangle{\pgfqpoint{0.499691in}{1.172519in}}{\pgfqpoint{7.362500in}{2.695000in}}%
\pgfusepath{clip}%
\pgfsetrectcap%
\pgfsetroundjoin%
\pgfsetlinewidth{2.710125pt}%
\definecolor{currentstroke}{rgb}{0.260000,0.260000,0.260000}%
\pgfsetstrokecolor{currentstroke}%
\pgfsetdash{}{0pt}%
\pgfusepath{stroke}%
\end{pgfscope}%
\begin{pgfscope}%
\pgfpathrectangle{\pgfqpoint{0.499691in}{1.172519in}}{\pgfqpoint{7.362500in}{2.695000in}}%
\pgfusepath{clip}%
\pgfsetrectcap%
\pgfsetroundjoin%
\pgfsetlinewidth{2.710125pt}%
\definecolor{currentstroke}{rgb}{0.260000,0.260000,0.260000}%
\pgfsetstrokecolor{currentstroke}%
\pgfsetdash{}{0pt}%
\pgfusepath{stroke}%
\end{pgfscope}%
\begin{pgfscope}%
\pgfpathrectangle{\pgfqpoint{0.499691in}{1.172519in}}{\pgfqpoint{7.362500in}{2.695000in}}%
\pgfusepath{clip}%
\pgfsetrectcap%
\pgfsetroundjoin%
\pgfsetlinewidth{2.710125pt}%
\definecolor{currentstroke}{rgb}{0.260000,0.260000,0.260000}%
\pgfsetstrokecolor{currentstroke}%
\pgfsetdash{}{0pt}%
\pgfusepath{stroke}%
\end{pgfscope}%
\begin{pgfscope}%
\pgfpathrectangle{\pgfqpoint{0.499691in}{1.172519in}}{\pgfqpoint{7.362500in}{2.695000in}}%
\pgfusepath{clip}%
\pgfsetrectcap%
\pgfsetroundjoin%
\pgfsetlinewidth{2.710125pt}%
\definecolor{currentstroke}{rgb}{0.260000,0.260000,0.260000}%
\pgfsetstrokecolor{currentstroke}%
\pgfsetdash{}{0pt}%
\pgfusepath{stroke}%
\end{pgfscope}%
\begin{pgfscope}%
\pgfpathrectangle{\pgfqpoint{0.499691in}{1.172519in}}{\pgfqpoint{7.362500in}{2.695000in}}%
\pgfusepath{clip}%
\pgfsetrectcap%
\pgfsetroundjoin%
\pgfsetlinewidth{2.710125pt}%
\definecolor{currentstroke}{rgb}{0.260000,0.260000,0.260000}%
\pgfsetstrokecolor{currentstroke}%
\pgfsetdash{}{0pt}%
\pgfusepath{stroke}%
\end{pgfscope}%
\begin{pgfscope}%
\pgfpathrectangle{\pgfqpoint{0.499691in}{1.172519in}}{\pgfqpoint{7.362500in}{2.695000in}}%
\pgfusepath{clip}%
\pgfsetrectcap%
\pgfsetroundjoin%
\pgfsetlinewidth{2.710125pt}%
\definecolor{currentstroke}{rgb}{0.260000,0.260000,0.260000}%
\pgfsetstrokecolor{currentstroke}%
\pgfsetdash{}{0pt}%
\pgfusepath{stroke}%
\end{pgfscope}%
\begin{pgfscope}%
\pgfpathrectangle{\pgfqpoint{0.499691in}{1.172519in}}{\pgfqpoint{7.362500in}{2.695000in}}%
\pgfusepath{clip}%
\pgfsetrectcap%
\pgfsetroundjoin%
\pgfsetlinewidth{2.710125pt}%
\definecolor{currentstroke}{rgb}{0.260000,0.260000,0.260000}%
\pgfsetstrokecolor{currentstroke}%
\pgfsetdash{}{0pt}%
\pgfusepath{stroke}%
\end{pgfscope}%
\begin{pgfscope}%
\pgfpathrectangle{\pgfqpoint{0.499691in}{1.172519in}}{\pgfqpoint{7.362500in}{2.695000in}}%
\pgfusepath{clip}%
\pgfsetrectcap%
\pgfsetroundjoin%
\pgfsetlinewidth{2.710125pt}%
\definecolor{currentstroke}{rgb}{0.260000,0.260000,0.260000}%
\pgfsetstrokecolor{currentstroke}%
\pgfsetdash{}{0pt}%
\pgfusepath{stroke}%
\end{pgfscope}%
\begin{pgfscope}%
\pgfpathrectangle{\pgfqpoint{0.499691in}{1.172519in}}{\pgfqpoint{7.362500in}{2.695000in}}%
\pgfusepath{clip}%
\pgfsetrectcap%
\pgfsetroundjoin%
\pgfsetlinewidth{2.710125pt}%
\definecolor{currentstroke}{rgb}{0.260000,0.260000,0.260000}%
\pgfsetstrokecolor{currentstroke}%
\pgfsetdash{}{0pt}%
\pgfusepath{stroke}%
\end{pgfscope}%
\begin{pgfscope}%
\pgfpathrectangle{\pgfqpoint{0.499691in}{1.172519in}}{\pgfqpoint{7.362500in}{2.695000in}}%
\pgfusepath{clip}%
\pgfsetrectcap%
\pgfsetroundjoin%
\pgfsetlinewidth{2.710125pt}%
\definecolor{currentstroke}{rgb}{0.260000,0.260000,0.260000}%
\pgfsetstrokecolor{currentstroke}%
\pgfsetdash{}{0pt}%
\pgfusepath{stroke}%
\end{pgfscope}%
\begin{pgfscope}%
\pgfpathrectangle{\pgfqpoint{0.499691in}{1.172519in}}{\pgfqpoint{7.362500in}{2.695000in}}%
\pgfusepath{clip}%
\pgfsetrectcap%
\pgfsetroundjoin%
\pgfsetlinewidth{2.710125pt}%
\definecolor{currentstroke}{rgb}{0.260000,0.260000,0.260000}%
\pgfsetstrokecolor{currentstroke}%
\pgfsetdash{}{0pt}%
\pgfusepath{stroke}%
\end{pgfscope}%
\begin{pgfscope}%
\pgfpathrectangle{\pgfqpoint{0.499691in}{1.172519in}}{\pgfqpoint{7.362500in}{2.695000in}}%
\pgfusepath{clip}%
\pgfsetrectcap%
\pgfsetroundjoin%
\pgfsetlinewidth{2.710125pt}%
\definecolor{currentstroke}{rgb}{0.260000,0.260000,0.260000}%
\pgfsetstrokecolor{currentstroke}%
\pgfsetdash{}{0pt}%
\pgfusepath{stroke}%
\end{pgfscope}%
\begin{pgfscope}%
\pgfpathrectangle{\pgfqpoint{0.499691in}{1.172519in}}{\pgfqpoint{7.362500in}{2.695000in}}%
\pgfusepath{clip}%
\pgfsetrectcap%
\pgfsetroundjoin%
\pgfsetlinewidth{2.710125pt}%
\definecolor{currentstroke}{rgb}{0.260000,0.260000,0.260000}%
\pgfsetstrokecolor{currentstroke}%
\pgfsetdash{}{0pt}%
\pgfusepath{stroke}%
\end{pgfscope}%
\begin{pgfscope}%
\pgfpathrectangle{\pgfqpoint{0.499691in}{1.172519in}}{\pgfqpoint{7.362500in}{2.695000in}}%
\pgfusepath{clip}%
\pgfsetrectcap%
\pgfsetroundjoin%
\pgfsetlinewidth{2.710125pt}%
\definecolor{currentstroke}{rgb}{0.260000,0.260000,0.260000}%
\pgfsetstrokecolor{currentstroke}%
\pgfsetdash{}{0pt}%
\pgfusepath{stroke}%
\end{pgfscope}%
\begin{pgfscope}%
\pgfsetrectcap%
\pgfsetmiterjoin%
\pgfsetlinewidth{0.803000pt}%
\definecolor{currentstroke}{rgb}{0.000000,0.000000,0.000000}%
\pgfsetstrokecolor{currentstroke}%
\pgfsetdash{}{0pt}%
\pgfpathmoveto{\pgfqpoint{0.499691in}{1.172519in}}%
\pgfpathlineto{\pgfqpoint{0.499691in}{3.867519in}}%
\pgfusepath{stroke}%
\end{pgfscope}%
\begin{pgfscope}%
\pgfsetrectcap%
\pgfsetmiterjoin%
\pgfsetlinewidth{0.803000pt}%
\definecolor{currentstroke}{rgb}{0.000000,0.000000,0.000000}%
\pgfsetstrokecolor{currentstroke}%
\pgfsetdash{}{0pt}%
\pgfpathmoveto{\pgfqpoint{7.862191in}{1.172519in}}%
\pgfpathlineto{\pgfqpoint{7.862191in}{3.867519in}}%
\pgfusepath{stroke}%
\end{pgfscope}%
\begin{pgfscope}%
\pgfsetrectcap%
\pgfsetmiterjoin%
\pgfsetlinewidth{0.803000pt}%
\definecolor{currentstroke}{rgb}{0.000000,0.000000,0.000000}%
\pgfsetstrokecolor{currentstroke}%
\pgfsetdash{}{0pt}%
\pgfpathmoveto{\pgfqpoint{0.499691in}{1.172519in}}%
\pgfpathlineto{\pgfqpoint{7.862191in}{1.172519in}}%
\pgfusepath{stroke}%
\end{pgfscope}%
\begin{pgfscope}%
\pgfsetrectcap%
\pgfsetmiterjoin%
\pgfsetlinewidth{0.803000pt}%
\definecolor{currentstroke}{rgb}{0.000000,0.000000,0.000000}%
\pgfsetstrokecolor{currentstroke}%
\pgfsetdash{}{0pt}%
\pgfpathmoveto{\pgfqpoint{0.499691in}{3.867519in}}%
\pgfpathlineto{\pgfqpoint{7.862191in}{3.867519in}}%
\pgfusepath{stroke}%
\end{pgfscope}%
\end{pgfpicture}%
\makeatother%
\endgroup%

    \caption{Mean note per assureur (colored by number of ratings)}
    \label{fig:mean_note_per_assureur}
\end{figure}
\begin{figure}[H]
    \advance\leftskip-3cm
    %% Creator: Matplotlib, PGF backend
%%
%% To include the figure in your LaTeX document, write
%%   \input{<filename>.pgf}
%%
%% Make sure the required packages are loaded in your preamble
%%   \usepackage{pgf}
%%
%% Also ensure that all the required font packages are loaded; for instance,
%% the lmodern package is sometimes necessary when using math font.
%%   \usepackage{lmodern}
%%
%% Figures using additional raster images can only be included by \input if
%% they are in the same directory as the main LaTeX file. For loading figures
%% from other directories you can use the `import` package
%%   \usepackage{import}
%%
%% and then include the figures with
%%   \import{<path to file>}{<filename>.pgf}
%%
%% Matplotlib used the following preamble
%%
\begingroup%
\makeatletter%
\begin{pgfpicture}%
\pgfpathrectangle{\pgfpointorigin}{\pgfqpoint{7.962191in}{3.967519in}}%
\pgfusepath{use as bounding box, clip}%
\begin{pgfscope}%
\pgfsetbuttcap%
\pgfsetmiterjoin%
\definecolor{currentfill}{rgb}{1.000000,1.000000,1.000000}%
\pgfsetfillcolor{currentfill}%
\pgfsetlinewidth{0.000000pt}%
\definecolor{currentstroke}{rgb}{1.000000,1.000000,1.000000}%
\pgfsetstrokecolor{currentstroke}%
\pgfsetdash{}{0pt}%
\pgfpathmoveto{\pgfqpoint{0.000000in}{0.000000in}}%
\pgfpathlineto{\pgfqpoint{7.962191in}{0.000000in}}%
\pgfpathlineto{\pgfqpoint{7.962191in}{3.967519in}}%
\pgfpathlineto{\pgfqpoint{0.000000in}{3.967519in}}%
\pgfpathlineto{\pgfqpoint{0.000000in}{0.000000in}}%
\pgfpathclose%
\pgfusepath{fill}%
\end{pgfscope}%
\begin{pgfscope}%
\pgfsetbuttcap%
\pgfsetmiterjoin%
\definecolor{currentfill}{rgb}{1.000000,1.000000,1.000000}%
\pgfsetfillcolor{currentfill}%
\pgfsetlinewidth{0.000000pt}%
\definecolor{currentstroke}{rgb}{0.000000,0.000000,0.000000}%
\pgfsetstrokecolor{currentstroke}%
\pgfsetstrokeopacity{0.000000}%
\pgfsetdash{}{0pt}%
\pgfpathmoveto{\pgfqpoint{0.499691in}{1.172519in}}%
\pgfpathlineto{\pgfqpoint{7.862191in}{1.172519in}}%
\pgfpathlineto{\pgfqpoint{7.862191in}{3.867519in}}%
\pgfpathlineto{\pgfqpoint{0.499691in}{3.867519in}}%
\pgfpathlineto{\pgfqpoint{0.499691in}{1.172519in}}%
\pgfpathclose%
\pgfusepath{fill}%
\end{pgfscope}%
\begin{pgfscope}%
\pgfpathrectangle{\pgfqpoint{0.499691in}{1.172519in}}{\pgfqpoint{7.362500in}{2.695000in}}%
\pgfusepath{clip}%
\pgfsetbuttcap%
\pgfsetmiterjoin%
\definecolor{currentfill}{rgb}{0.192291,0.294409,0.451057}%
\pgfsetfillcolor{currentfill}%
\pgfsetlinewidth{0.000000pt}%
\definecolor{currentstroke}{rgb}{0.000000,0.000000,0.000000}%
\pgfsetstrokecolor{currentstroke}%
\pgfsetstrokeopacity{0.000000}%
\pgfsetdash{}{0pt}%
\pgfpathmoveto{\pgfqpoint{0.512838in}{1.172519in}}%
\pgfpathlineto{\pgfqpoint{0.618017in}{1.172519in}}%
\pgfpathlineto{\pgfqpoint{0.618017in}{3.308273in}}%
\pgfpathlineto{\pgfqpoint{0.512838in}{3.308273in}}%
\pgfpathlineto{\pgfqpoint{0.512838in}{1.172519in}}%
\pgfpathclose%
\pgfusepath{fill}%
\end{pgfscope}%
\begin{pgfscope}%
\pgfpathrectangle{\pgfqpoint{0.499691in}{1.172519in}}{\pgfqpoint{7.362500in}{2.695000in}}%
\pgfusepath{clip}%
\pgfsetbuttcap%
\pgfsetmiterjoin%
\definecolor{currentfill}{rgb}{0.180817,0.420502,0.492726}%
\pgfsetfillcolor{currentfill}%
\pgfsetlinewidth{0.000000pt}%
\definecolor{currentstroke}{rgb}{0.000000,0.000000,0.000000}%
\pgfsetstrokecolor{currentstroke}%
\pgfsetstrokeopacity{0.000000}%
\pgfsetdash{}{0pt}%
\pgfpathmoveto{\pgfqpoint{0.644312in}{1.172519in}}%
\pgfpathlineto{\pgfqpoint{0.749490in}{1.172519in}}%
\pgfpathlineto{\pgfqpoint{0.749490in}{2.572783in}}%
\pgfpathlineto{\pgfqpoint{0.644312in}{2.572783in}}%
\pgfpathlineto{\pgfqpoint{0.644312in}{1.172519in}}%
\pgfpathclose%
\pgfusepath{fill}%
\end{pgfscope}%
\begin{pgfscope}%
\pgfpathrectangle{\pgfqpoint{0.499691in}{1.172519in}}{\pgfqpoint{7.362500in}{2.695000in}}%
\pgfusepath{clip}%
\pgfsetbuttcap%
\pgfsetmiterjoin%
\definecolor{currentfill}{rgb}{0.203716,0.253116,0.427350}%
\pgfsetfillcolor{currentfill}%
\pgfsetlinewidth{0.000000pt}%
\definecolor{currentstroke}{rgb}{0.000000,0.000000,0.000000}%
\pgfsetstrokecolor{currentstroke}%
\pgfsetstrokeopacity{0.000000}%
\pgfsetdash{}{0pt}%
\pgfpathmoveto{\pgfqpoint{0.775785in}{1.172519in}}%
\pgfpathlineto{\pgfqpoint{0.880963in}{1.172519in}}%
\pgfpathlineto{\pgfqpoint{0.880963in}{3.455430in}}%
\pgfpathlineto{\pgfqpoint{0.775785in}{3.455430in}}%
\pgfpathlineto{\pgfqpoint{0.775785in}{1.172519in}}%
\pgfpathclose%
\pgfusepath{fill}%
\end{pgfscope}%
\begin{pgfscope}%
\pgfpathrectangle{\pgfqpoint{0.499691in}{1.172519in}}{\pgfqpoint{7.362500in}{2.695000in}}%
\pgfusepath{clip}%
\pgfsetbuttcap%
\pgfsetmiterjoin%
\definecolor{currentfill}{rgb}{0.183729,0.315078,0.461321}%
\pgfsetfillcolor{currentfill}%
\pgfsetlinewidth{0.000000pt}%
\definecolor{currentstroke}{rgb}{0.000000,0.000000,0.000000}%
\pgfsetstrokecolor{currentstroke}%
\pgfsetstrokeopacity{0.000000}%
\pgfsetdash{}{0pt}%
\pgfpathmoveto{\pgfqpoint{0.907258in}{1.172519in}}%
\pgfpathlineto{\pgfqpoint{1.012437in}{1.172519in}}%
\pgfpathlineto{\pgfqpoint{1.012437in}{2.156618in}}%
\pgfpathlineto{\pgfqpoint{0.907258in}{2.156618in}}%
\pgfpathlineto{\pgfqpoint{0.907258in}{1.172519in}}%
\pgfpathclose%
\pgfusepath{fill}%
\end{pgfscope}%
\begin{pgfscope}%
\pgfpathrectangle{\pgfqpoint{0.499691in}{1.172519in}}{\pgfqpoint{7.362500in}{2.695000in}}%
\pgfusepath{clip}%
\pgfsetbuttcap%
\pgfsetmiterjoin%
\definecolor{currentfill}{rgb}{0.164073,0.382217,0.483403}%
\pgfsetfillcolor{currentfill}%
\pgfsetlinewidth{0.000000pt}%
\definecolor{currentstroke}{rgb}{0.000000,0.000000,0.000000}%
\pgfsetstrokecolor{currentstroke}%
\pgfsetstrokeopacity{0.000000}%
\pgfsetdash{}{0pt}%
\pgfpathmoveto{\pgfqpoint{1.038731in}{1.172519in}}%
\pgfpathlineto{\pgfqpoint{1.143910in}{1.172519in}}%
\pgfpathlineto{\pgfqpoint{1.143910in}{2.163009in}}%
\pgfpathlineto{\pgfqpoint{1.038731in}{2.163009in}}%
\pgfpathlineto{\pgfqpoint{1.038731in}{1.172519in}}%
\pgfpathclose%
\pgfusepath{fill}%
\end{pgfscope}%
\begin{pgfscope}%
\pgfpathrectangle{\pgfqpoint{0.499691in}{1.172519in}}{\pgfqpoint{7.362500in}{2.695000in}}%
\pgfusepath{clip}%
\pgfsetbuttcap%
\pgfsetmiterjoin%
\definecolor{currentfill}{rgb}{0.274508,0.512094,0.516671}%
\pgfsetfillcolor{currentfill}%
\pgfsetlinewidth{0.000000pt}%
\definecolor{currentstroke}{rgb}{0.000000,0.000000,0.000000}%
\pgfsetstrokecolor{currentstroke}%
\pgfsetstrokeopacity{0.000000}%
\pgfsetdash{}{0pt}%
\pgfpathmoveto{\pgfqpoint{1.170205in}{1.172519in}}%
\pgfpathlineto{\pgfqpoint{1.275383in}{1.172519in}}%
\pgfpathlineto{\pgfqpoint{1.275383in}{2.192935in}}%
\pgfpathlineto{\pgfqpoint{1.170205in}{2.192935in}}%
\pgfpathlineto{\pgfqpoint{1.170205in}{1.172519in}}%
\pgfpathclose%
\pgfusepath{fill}%
\end{pgfscope}%
\begin{pgfscope}%
\pgfpathrectangle{\pgfqpoint{0.499691in}{1.172519in}}{\pgfqpoint{7.362500in}{2.695000in}}%
\pgfusepath{clip}%
\pgfsetbuttcap%
\pgfsetmiterjoin%
\definecolor{currentfill}{rgb}{0.531667,0.705639,0.580692}%
\pgfsetfillcolor{currentfill}%
\pgfsetlinewidth{0.000000pt}%
\definecolor{currentstroke}{rgb}{0.000000,0.000000,0.000000}%
\pgfsetstrokecolor{currentstroke}%
\pgfsetstrokeopacity{0.000000}%
\pgfsetdash{}{0pt}%
\pgfpathmoveto{\pgfqpoint{1.301678in}{1.172519in}}%
\pgfpathlineto{\pgfqpoint{1.406856in}{1.172519in}}%
\pgfpathlineto{\pgfqpoint{1.406856in}{2.329526in}}%
\pgfpathlineto{\pgfqpoint{1.301678in}{2.329526in}}%
\pgfpathlineto{\pgfqpoint{1.301678in}{1.172519in}}%
\pgfpathclose%
\pgfusepath{fill}%
\end{pgfscope}%
\begin{pgfscope}%
\pgfpathrectangle{\pgfqpoint{0.499691in}{1.172519in}}{\pgfqpoint{7.362500in}{2.695000in}}%
\pgfusepath{clip}%
\pgfsetbuttcap%
\pgfsetmiterjoin%
\definecolor{currentfill}{rgb}{0.165465,0.392113,0.485883}%
\pgfsetfillcolor{currentfill}%
\pgfsetlinewidth{0.000000pt}%
\definecolor{currentstroke}{rgb}{0.000000,0.000000,0.000000}%
\pgfsetstrokecolor{currentstroke}%
\pgfsetstrokeopacity{0.000000}%
\pgfsetdash{}{0pt}%
\pgfpathmoveto{\pgfqpoint{1.433151in}{1.172519in}}%
\pgfpathlineto{\pgfqpoint{1.538330in}{1.172519in}}%
\pgfpathlineto{\pgfqpoint{1.538330in}{1.984076in}}%
\pgfpathlineto{\pgfqpoint{1.433151in}{1.984076in}}%
\pgfpathlineto{\pgfqpoint{1.433151in}{1.172519in}}%
\pgfpathclose%
\pgfusepath{fill}%
\end{pgfscope}%
\begin{pgfscope}%
\pgfpathrectangle{\pgfqpoint{0.499691in}{1.172519in}}{\pgfqpoint{7.362500in}{2.695000in}}%
\pgfusepath{clip}%
\pgfsetbuttcap%
\pgfsetmiterjoin%
\definecolor{currentfill}{rgb}{0.171813,0.343039,0.472200}%
\pgfsetfillcolor{currentfill}%
\pgfsetlinewidth{0.000000pt}%
\definecolor{currentstroke}{rgb}{0.000000,0.000000,0.000000}%
\pgfsetstrokecolor{currentstroke}%
\pgfsetstrokeopacity{0.000000}%
\pgfsetdash{}{0pt}%
\pgfpathmoveto{\pgfqpoint{1.564624in}{1.172519in}}%
\pgfpathlineto{\pgfqpoint{1.669803in}{1.172519in}}%
\pgfpathlineto{\pgfqpoint{1.669803in}{2.061439in}}%
\pgfpathlineto{\pgfqpoint{1.564624in}{2.061439in}}%
\pgfpathlineto{\pgfqpoint{1.564624in}{1.172519in}}%
\pgfpathclose%
\pgfusepath{fill}%
\end{pgfscope}%
\begin{pgfscope}%
\pgfpathrectangle{\pgfqpoint{0.499691in}{1.172519in}}{\pgfqpoint{7.362500in}{2.695000in}}%
\pgfusepath{clip}%
\pgfsetbuttcap%
\pgfsetmiterjoin%
\definecolor{currentfill}{rgb}{0.405274,0.625213,0.552953}%
\pgfsetfillcolor{currentfill}%
\pgfsetlinewidth{0.000000pt}%
\definecolor{currentstroke}{rgb}{0.000000,0.000000,0.000000}%
\pgfsetstrokecolor{currentstroke}%
\pgfsetstrokeopacity{0.000000}%
\pgfsetdash{}{0pt}%
\pgfpathmoveto{\pgfqpoint{1.696097in}{1.172519in}}%
\pgfpathlineto{\pgfqpoint{1.801276in}{1.172519in}}%
\pgfpathlineto{\pgfqpoint{1.801276in}{2.669822in}}%
\pgfpathlineto{\pgfqpoint{1.696097in}{2.669822in}}%
\pgfpathlineto{\pgfqpoint{1.696097in}{1.172519in}}%
\pgfpathclose%
\pgfusepath{fill}%
\end{pgfscope}%
\begin{pgfscope}%
\pgfpathrectangle{\pgfqpoint{0.499691in}{1.172519in}}{\pgfqpoint{7.362500in}{2.695000in}}%
\pgfusepath{clip}%
\pgfsetbuttcap%
\pgfsetmiterjoin%
\definecolor{currentfill}{rgb}{0.321598,0.556089,0.531070}%
\pgfsetfillcolor{currentfill}%
\pgfsetlinewidth{0.000000pt}%
\definecolor{currentstroke}{rgb}{0.000000,0.000000,0.000000}%
\pgfsetstrokecolor{currentstroke}%
\pgfsetstrokeopacity{0.000000}%
\pgfsetdash{}{0pt}%
\pgfpathmoveto{\pgfqpoint{1.827571in}{1.172519in}}%
\pgfpathlineto{\pgfqpoint{1.932749in}{1.172519in}}%
\pgfpathlineto{\pgfqpoint{1.932749in}{2.271675in}}%
\pgfpathlineto{\pgfqpoint{1.827571in}{2.271675in}}%
\pgfpathlineto{\pgfqpoint{1.827571in}{1.172519in}}%
\pgfpathclose%
\pgfusepath{fill}%
\end{pgfscope}%
\begin{pgfscope}%
\pgfpathrectangle{\pgfqpoint{0.499691in}{1.172519in}}{\pgfqpoint{7.362500in}{2.695000in}}%
\pgfusepath{clip}%
\pgfsetbuttcap%
\pgfsetmiterjoin%
\definecolor{currentfill}{rgb}{0.552281,0.717276,0.585304}%
\pgfsetfillcolor{currentfill}%
\pgfsetlinewidth{0.000000pt}%
\definecolor{currentstroke}{rgb}{0.000000,0.000000,0.000000}%
\pgfsetstrokecolor{currentstroke}%
\pgfsetstrokeopacity{0.000000}%
\pgfsetdash{}{0pt}%
\pgfpathmoveto{\pgfqpoint{1.959044in}{1.172519in}}%
\pgfpathlineto{\pgfqpoint{2.064222in}{1.172519in}}%
\pgfpathlineto{\pgfqpoint{2.064222in}{2.005564in}}%
\pgfpathlineto{\pgfqpoint{1.959044in}{2.005564in}}%
\pgfpathlineto{\pgfqpoint{1.959044in}{1.172519in}}%
\pgfpathclose%
\pgfusepath{fill}%
\end{pgfscope}%
\begin{pgfscope}%
\pgfpathrectangle{\pgfqpoint{0.499691in}{1.172519in}}{\pgfqpoint{7.362500in}{2.695000in}}%
\pgfusepath{clip}%
\pgfsetbuttcap%
\pgfsetmiterjoin%
\definecolor{currentfill}{rgb}{0.251824,0.491212,0.511055}%
\pgfsetfillcolor{currentfill}%
\pgfsetlinewidth{0.000000pt}%
\definecolor{currentstroke}{rgb}{0.000000,0.000000,0.000000}%
\pgfsetstrokecolor{currentstroke}%
\pgfsetstrokeopacity{0.000000}%
\pgfsetdash{}{0pt}%
\pgfpathmoveto{\pgfqpoint{2.090517in}{1.172519in}}%
\pgfpathlineto{\pgfqpoint{2.195696in}{1.172519in}}%
\pgfpathlineto{\pgfqpoint{2.195696in}{1.998952in}}%
\pgfpathlineto{\pgfqpoint{2.090517in}{1.998952in}}%
\pgfpathlineto{\pgfqpoint{2.090517in}{1.172519in}}%
\pgfpathclose%
\pgfusepath{fill}%
\end{pgfscope}%
\begin{pgfscope}%
\pgfpathrectangle{\pgfqpoint{0.499691in}{1.172519in}}{\pgfqpoint{7.362500in}{2.695000in}}%
\pgfusepath{clip}%
\pgfsetbuttcap%
\pgfsetmiterjoin%
\definecolor{currentfill}{rgb}{0.472019,0.670464,0.567533}%
\pgfsetfillcolor{currentfill}%
\pgfsetlinewidth{0.000000pt}%
\definecolor{currentstroke}{rgb}{0.000000,0.000000,0.000000}%
\pgfsetstrokecolor{currentstroke}%
\pgfsetstrokeopacity{0.000000}%
\pgfsetdash{}{0pt}%
\pgfpathmoveto{\pgfqpoint{2.221990in}{1.172519in}}%
\pgfpathlineto{\pgfqpoint{2.327169in}{1.172519in}}%
\pgfpathlineto{\pgfqpoint{2.327169in}{2.564543in}}%
\pgfpathlineto{\pgfqpoint{2.221990in}{2.564543in}}%
\pgfpathlineto{\pgfqpoint{2.221990in}{1.172519in}}%
\pgfpathclose%
\pgfusepath{fill}%
\end{pgfscope}%
\begin{pgfscope}%
\pgfpathrectangle{\pgfqpoint{0.499691in}{1.172519in}}{\pgfqpoint{7.362500in}{2.695000in}}%
\pgfusepath{clip}%
\pgfsetbuttcap%
\pgfsetmiterjoin%
\definecolor{currentfill}{rgb}{0.199045,0.441193,0.497870}%
\pgfsetfillcolor{currentfill}%
\pgfsetlinewidth{0.000000pt}%
\definecolor{currentstroke}{rgb}{0.000000,0.000000,0.000000}%
\pgfsetstrokecolor{currentstroke}%
\pgfsetstrokeopacity{0.000000}%
\pgfsetdash{}{0pt}%
\pgfpathmoveto{\pgfqpoint{2.353463in}{1.172519in}}%
\pgfpathlineto{\pgfqpoint{2.458642in}{1.172519in}}%
\pgfpathlineto{\pgfqpoint{2.458642in}{1.936580in}}%
\pgfpathlineto{\pgfqpoint{2.353463in}{1.936580in}}%
\pgfpathlineto{\pgfqpoint{2.353463in}{1.172519in}}%
\pgfpathclose%
\pgfusepath{fill}%
\end{pgfscope}%
\begin{pgfscope}%
\pgfpathrectangle{\pgfqpoint{0.499691in}{1.172519in}}{\pgfqpoint{7.362500in}{2.695000in}}%
\pgfusepath{clip}%
\pgfsetbuttcap%
\pgfsetmiterjoin%
\definecolor{currentfill}{rgb}{0.218070,0.459928,0.502705}%
\pgfsetfillcolor{currentfill}%
\pgfsetlinewidth{0.000000pt}%
\definecolor{currentstroke}{rgb}{0.000000,0.000000,0.000000}%
\pgfsetstrokecolor{currentstroke}%
\pgfsetstrokeopacity{0.000000}%
\pgfsetdash{}{0pt}%
\pgfpathmoveto{\pgfqpoint{2.484937in}{1.172519in}}%
\pgfpathlineto{\pgfqpoint{2.590115in}{1.172519in}}%
\pgfpathlineto{\pgfqpoint{2.590115in}{2.037571in}}%
\pgfpathlineto{\pgfqpoint{2.484937in}{2.037571in}}%
\pgfpathlineto{\pgfqpoint{2.484937in}{1.172519in}}%
\pgfpathclose%
\pgfusepath{fill}%
\end{pgfscope}%
\begin{pgfscope}%
\pgfpathrectangle{\pgfqpoint{0.499691in}{1.172519in}}{\pgfqpoint{7.362500in}{2.695000in}}%
\pgfusepath{clip}%
\pgfsetbuttcap%
\pgfsetmiterjoin%
\definecolor{currentfill}{rgb}{0.263192,0.501649,0.513869}%
\pgfsetfillcolor{currentfill}%
\pgfsetlinewidth{0.000000pt}%
\definecolor{currentstroke}{rgb}{0.000000,0.000000,0.000000}%
\pgfsetstrokecolor{currentstroke}%
\pgfsetstrokeopacity{0.000000}%
\pgfsetdash{}{0pt}%
\pgfpathmoveto{\pgfqpoint{2.616410in}{1.172519in}}%
\pgfpathlineto{\pgfqpoint{2.721588in}{1.172519in}}%
\pgfpathlineto{\pgfqpoint{2.721588in}{2.198768in}}%
\pgfpathlineto{\pgfqpoint{2.616410in}{2.198768in}}%
\pgfpathlineto{\pgfqpoint{2.616410in}{1.172519in}}%
\pgfpathclose%
\pgfusepath{fill}%
\end{pgfscope}%
\begin{pgfscope}%
\pgfpathrectangle{\pgfqpoint{0.499691in}{1.172519in}}{\pgfqpoint{7.362500in}{2.695000in}}%
\pgfusepath{clip}%
\pgfsetbuttcap%
\pgfsetmiterjoin%
\definecolor{currentfill}{rgb}{0.206675,0.230070,0.415765}%
\pgfsetfillcolor{currentfill}%
\pgfsetlinewidth{0.000000pt}%
\definecolor{currentstroke}{rgb}{0.000000,0.000000,0.000000}%
\pgfsetstrokecolor{currentstroke}%
\pgfsetstrokeopacity{0.000000}%
\pgfsetdash{}{0pt}%
\pgfpathmoveto{\pgfqpoint{2.747883in}{1.172519in}}%
\pgfpathlineto{\pgfqpoint{2.853062in}{1.172519in}}%
\pgfpathlineto{\pgfqpoint{2.853062in}{3.090136in}}%
\pgfpathlineto{\pgfqpoint{2.747883in}{3.090136in}}%
\pgfpathlineto{\pgfqpoint{2.747883in}{1.172519in}}%
\pgfpathclose%
\pgfusepath{fill}%
\end{pgfscope}%
\begin{pgfscope}%
\pgfpathrectangle{\pgfqpoint{0.499691in}{1.172519in}}{\pgfqpoint{7.362500in}{2.695000in}}%
\pgfusepath{clip}%
\pgfsetbuttcap%
\pgfsetmiterjoin%
\definecolor{currentfill}{rgb}{0.296193,0.532476,0.523173}%
\pgfsetfillcolor{currentfill}%
\pgfsetlinewidth{0.000000pt}%
\definecolor{currentstroke}{rgb}{0.000000,0.000000,0.000000}%
\pgfsetstrokecolor{currentstroke}%
\pgfsetstrokeopacity{0.000000}%
\pgfsetdash{}{0pt}%
\pgfpathmoveto{\pgfqpoint{2.879356in}{1.172519in}}%
\pgfpathlineto{\pgfqpoint{2.984535in}{1.172519in}}%
\pgfpathlineto{\pgfqpoint{2.984535in}{2.114778in}}%
\pgfpathlineto{\pgfqpoint{2.879356in}{2.114778in}}%
\pgfpathlineto{\pgfqpoint{2.879356in}{1.172519in}}%
\pgfpathclose%
\pgfusepath{fill}%
\end{pgfscope}%
\begin{pgfscope}%
\pgfpathrectangle{\pgfqpoint{0.499691in}{1.172519in}}{\pgfqpoint{7.362500in}{2.695000in}}%
\pgfusepath{clip}%
\pgfsetbuttcap%
\pgfsetmiterjoin%
\definecolor{currentfill}{rgb}{0.421819,0.637162,0.556681}%
\pgfsetfillcolor{currentfill}%
\pgfsetlinewidth{0.000000pt}%
\definecolor{currentstroke}{rgb}{0.000000,0.000000,0.000000}%
\pgfsetstrokecolor{currentstroke}%
\pgfsetstrokeopacity{0.000000}%
\pgfsetdash{}{0pt}%
\pgfpathmoveto{\pgfqpoint{3.010830in}{1.172519in}}%
\pgfpathlineto{\pgfqpoint{3.116008in}{1.172519in}}%
\pgfpathlineto{\pgfqpoint{3.116008in}{2.114269in}}%
\pgfpathlineto{\pgfqpoint{3.010830in}{2.114269in}}%
\pgfpathlineto{\pgfqpoint{3.010830in}{1.172519in}}%
\pgfpathclose%
\pgfusepath{fill}%
\end{pgfscope}%
\begin{pgfscope}%
\pgfpathrectangle{\pgfqpoint{0.499691in}{1.172519in}}{\pgfqpoint{7.362500in}{2.695000in}}%
\pgfusepath{clip}%
\pgfsetbuttcap%
\pgfsetmiterjoin%
\definecolor{currentfill}{rgb}{0.173626,0.410260,0.490252}%
\pgfsetfillcolor{currentfill}%
\pgfsetlinewidth{0.000000pt}%
\definecolor{currentstroke}{rgb}{0.000000,0.000000,0.000000}%
\pgfsetstrokecolor{currentstroke}%
\pgfsetstrokeopacity{0.000000}%
\pgfsetdash{}{0pt}%
\pgfpathmoveto{\pgfqpoint{3.142303in}{1.172519in}}%
\pgfpathlineto{\pgfqpoint{3.247481in}{1.172519in}}%
\pgfpathlineto{\pgfqpoint{3.247481in}{2.311510in}}%
\pgfpathlineto{\pgfqpoint{3.142303in}{2.311510in}}%
\pgfpathlineto{\pgfqpoint{3.142303in}{1.172519in}}%
\pgfpathclose%
\pgfusepath{fill}%
\end{pgfscope}%
\begin{pgfscope}%
\pgfpathrectangle{\pgfqpoint{0.499691in}{1.172519in}}{\pgfqpoint{7.362500in}{2.695000in}}%
\pgfusepath{clip}%
\pgfsetbuttcap%
\pgfsetmiterjoin%
\definecolor{currentfill}{rgb}{0.201564,0.264259,0.433512}%
\pgfsetfillcolor{currentfill}%
\pgfsetlinewidth{0.000000pt}%
\definecolor{currentstroke}{rgb}{0.000000,0.000000,0.000000}%
\pgfsetstrokecolor{currentstroke}%
\pgfsetstrokeopacity{0.000000}%
\pgfsetdash{}{0pt}%
\pgfpathmoveto{\pgfqpoint{3.273776in}{1.172519in}}%
\pgfpathlineto{\pgfqpoint{3.378955in}{1.172519in}}%
\pgfpathlineto{\pgfqpoint{3.378955in}{2.841948in}}%
\pgfpathlineto{\pgfqpoint{3.273776in}{2.841948in}}%
\pgfpathlineto{\pgfqpoint{3.273776in}{1.172519in}}%
\pgfpathclose%
\pgfusepath{fill}%
\end{pgfscope}%
\begin{pgfscope}%
\pgfpathrectangle{\pgfqpoint{0.499691in}{1.172519in}}{\pgfqpoint{7.362500in}{2.695000in}}%
\pgfusepath{clip}%
\pgfsetbuttcap%
\pgfsetmiterjoin%
\definecolor{currentfill}{rgb}{0.453441,0.658609,0.563554}%
\pgfsetfillcolor{currentfill}%
\pgfsetlinewidth{0.000000pt}%
\definecolor{currentstroke}{rgb}{0.000000,0.000000,0.000000}%
\pgfsetstrokecolor{currentstroke}%
\pgfsetstrokeopacity{0.000000}%
\pgfsetdash{}{0pt}%
\pgfpathmoveto{\pgfqpoint{3.405249in}{1.172519in}}%
\pgfpathlineto{\pgfqpoint{3.510428in}{1.172519in}}%
\pgfpathlineto{\pgfqpoint{3.510428in}{2.031509in}}%
\pgfpathlineto{\pgfqpoint{3.405249in}{2.031509in}}%
\pgfpathlineto{\pgfqpoint{3.405249in}{1.172519in}}%
\pgfpathclose%
\pgfusepath{fill}%
\end{pgfscope}%
\begin{pgfscope}%
\pgfpathrectangle{\pgfqpoint{0.499691in}{1.172519in}}{\pgfqpoint{7.362500in}{2.695000in}}%
\pgfusepath{clip}%
\pgfsetbuttcap%
\pgfsetmiterjoin%
\definecolor{currentfill}{rgb}{0.308751,0.544266,0.527144}%
\pgfsetfillcolor{currentfill}%
\pgfsetlinewidth{0.000000pt}%
\definecolor{currentstroke}{rgb}{0.000000,0.000000,0.000000}%
\pgfsetstrokecolor{currentstroke}%
\pgfsetstrokeopacity{0.000000}%
\pgfsetdash{}{0pt}%
\pgfpathmoveto{\pgfqpoint{3.536722in}{1.172519in}}%
\pgfpathlineto{\pgfqpoint{3.641901in}{1.172519in}}%
\pgfpathlineto{\pgfqpoint{3.641901in}{2.062872in}}%
\pgfpathlineto{\pgfqpoint{3.536722in}{2.062872in}}%
\pgfpathlineto{\pgfqpoint{3.536722in}{1.172519in}}%
\pgfpathclose%
\pgfusepath{fill}%
\end{pgfscope}%
\begin{pgfscope}%
\pgfpathrectangle{\pgfqpoint{0.499691in}{1.172519in}}{\pgfqpoint{7.362500in}{2.695000in}}%
\pgfusepath{clip}%
\pgfsetbuttcap%
\pgfsetmiterjoin%
\definecolor{currentfill}{rgb}{0.334797,0.567947,0.534943}%
\pgfsetfillcolor{currentfill}%
\pgfsetlinewidth{0.000000pt}%
\definecolor{currentstroke}{rgb}{0.000000,0.000000,0.000000}%
\pgfsetstrokecolor{currentstroke}%
\pgfsetstrokeopacity{0.000000}%
\pgfsetdash{}{0pt}%
\pgfpathmoveto{\pgfqpoint{3.668196in}{1.172519in}}%
\pgfpathlineto{\pgfqpoint{3.773374in}{1.172519in}}%
\pgfpathlineto{\pgfqpoint{3.773374in}{2.192355in}}%
\pgfpathlineto{\pgfqpoint{3.668196in}{2.192355in}}%
\pgfpathlineto{\pgfqpoint{3.668196in}{1.172519in}}%
\pgfpathclose%
\pgfusepath{fill}%
\end{pgfscope}%
\begin{pgfscope}%
\pgfpathrectangle{\pgfqpoint{0.499691in}{1.172519in}}{\pgfqpoint{7.362500in}{2.695000in}}%
\pgfusepath{clip}%
\pgfsetbuttcap%
\pgfsetmiterjoin%
\definecolor{currentfill}{rgb}{0.229175,0.470352,0.505459}%
\pgfsetfillcolor{currentfill}%
\pgfsetlinewidth{0.000000pt}%
\definecolor{currentstroke}{rgb}{0.000000,0.000000,0.000000}%
\pgfsetstrokecolor{currentstroke}%
\pgfsetstrokeopacity{0.000000}%
\pgfsetdash{}{0pt}%
\pgfpathmoveto{\pgfqpoint{3.799669in}{1.172519in}}%
\pgfpathlineto{\pgfqpoint{3.904847in}{1.172519in}}%
\pgfpathlineto{\pgfqpoint{3.904847in}{2.763403in}}%
\pgfpathlineto{\pgfqpoint{3.799669in}{2.763403in}}%
\pgfpathlineto{\pgfqpoint{3.799669in}{1.172519in}}%
\pgfpathclose%
\pgfusepath{fill}%
\end{pgfscope}%
\begin{pgfscope}%
\pgfpathrectangle{\pgfqpoint{0.499691in}{1.172519in}}{\pgfqpoint{7.362500in}{2.695000in}}%
\pgfusepath{clip}%
\pgfsetbuttcap%
\pgfsetmiterjoin%
\definecolor{currentfill}{rgb}{0.168248,0.400127,0.487820}%
\pgfsetfillcolor{currentfill}%
\pgfsetlinewidth{0.000000pt}%
\definecolor{currentstroke}{rgb}{0.000000,0.000000,0.000000}%
\pgfsetstrokecolor{currentstroke}%
\pgfsetstrokeopacity{0.000000}%
\pgfsetdash{}{0pt}%
\pgfpathmoveto{\pgfqpoint{3.931142in}{1.172519in}}%
\pgfpathlineto{\pgfqpoint{4.036321in}{1.172519in}}%
\pgfpathlineto{\pgfqpoint{4.036321in}{1.975378in}}%
\pgfpathlineto{\pgfqpoint{3.931142in}{1.975378in}}%
\pgfpathlineto{\pgfqpoint{3.931142in}{1.172519in}}%
\pgfpathclose%
\pgfusepath{fill}%
\end{pgfscope}%
\begin{pgfscope}%
\pgfpathrectangle{\pgfqpoint{0.499691in}{1.172519in}}{\pgfqpoint{7.362500in}{2.695000in}}%
\pgfusepath{clip}%
\pgfsetbuttcap%
\pgfsetmiterjoin%
\definecolor{currentfill}{rgb}{0.206675,0.230070,0.415765}%
\pgfsetfillcolor{currentfill}%
\pgfsetlinewidth{0.000000pt}%
\definecolor{currentstroke}{rgb}{0.000000,0.000000,0.000000}%
\pgfsetstrokecolor{currentstroke}%
\pgfsetstrokeopacity{0.000000}%
\pgfsetdash{}{0pt}%
\pgfpathmoveto{\pgfqpoint{4.062615in}{1.172519in}}%
\pgfpathlineto{\pgfqpoint{4.167794in}{1.172519in}}%
\pgfpathlineto{\pgfqpoint{4.167794in}{1.751022in}}%
\pgfpathlineto{\pgfqpoint{4.062615in}{1.751022in}}%
\pgfpathlineto{\pgfqpoint{4.062615in}{1.172519in}}%
\pgfpathclose%
\pgfusepath{fill}%
\end{pgfscope}%
\begin{pgfscope}%
\pgfpathrectangle{\pgfqpoint{0.499691in}{1.172519in}}{\pgfqpoint{7.362500in}{2.695000in}}%
\pgfusepath{clip}%
\pgfsetbuttcap%
\pgfsetmiterjoin%
\definecolor{currentfill}{rgb}{0.374243,0.601306,0.545528}%
\pgfsetfillcolor{currentfill}%
\pgfsetlinewidth{0.000000pt}%
\definecolor{currentstroke}{rgb}{0.000000,0.000000,0.000000}%
\pgfsetstrokecolor{currentstroke}%
\pgfsetstrokeopacity{0.000000}%
\pgfsetdash{}{0pt}%
\pgfpathmoveto{\pgfqpoint{4.194088in}{1.172519in}}%
\pgfpathlineto{\pgfqpoint{4.299267in}{1.172519in}}%
\pgfpathlineto{\pgfqpoint{4.299267in}{2.210152in}}%
\pgfpathlineto{\pgfqpoint{4.194088in}{2.210152in}}%
\pgfpathlineto{\pgfqpoint{4.194088in}{1.172519in}}%
\pgfpathclose%
\pgfusepath{fill}%
\end{pgfscope}%
\begin{pgfscope}%
\pgfpathrectangle{\pgfqpoint{0.499691in}{1.172519in}}{\pgfqpoint{7.362500in}{2.695000in}}%
\pgfusepath{clip}%
\pgfsetbuttcap%
\pgfsetmiterjoin%
\definecolor{currentfill}{rgb}{0.205383,0.241736,0.421423}%
\pgfsetfillcolor{currentfill}%
\pgfsetlinewidth{0.000000pt}%
\definecolor{currentstroke}{rgb}{0.000000,0.000000,0.000000}%
\pgfsetstrokecolor{currentstroke}%
\pgfsetstrokeopacity{0.000000}%
\pgfsetdash{}{0pt}%
\pgfpathmoveto{\pgfqpoint{4.325562in}{1.172519in}}%
\pgfpathlineto{\pgfqpoint{4.430740in}{1.172519in}}%
\pgfpathlineto{\pgfqpoint{4.430740in}{3.388047in}}%
\pgfpathlineto{\pgfqpoint{4.325562in}{3.388047in}}%
\pgfpathlineto{\pgfqpoint{4.325562in}{1.172519in}}%
\pgfpathclose%
\pgfusepath{fill}%
\end{pgfscope}%
\begin{pgfscope}%
\pgfpathrectangle{\pgfqpoint{0.499691in}{1.172519in}}{\pgfqpoint{7.362500in}{2.695000in}}%
\pgfusepath{clip}%
\pgfsetbuttcap%
\pgfsetmiterjoin%
\definecolor{currentfill}{rgb}{0.588929,0.738349,0.592927}%
\pgfsetfillcolor{currentfill}%
\pgfsetlinewidth{0.000000pt}%
\definecolor{currentstroke}{rgb}{0.000000,0.000000,0.000000}%
\pgfsetstrokecolor{currentstroke}%
\pgfsetstrokeopacity{0.000000}%
\pgfsetdash{}{0pt}%
\pgfpathmoveto{\pgfqpoint{4.457035in}{1.172519in}}%
\pgfpathlineto{\pgfqpoint{4.562213in}{1.172519in}}%
\pgfpathlineto{\pgfqpoint{4.562213in}{1.911718in}}%
\pgfpathlineto{\pgfqpoint{4.457035in}{1.911718in}}%
\pgfpathlineto{\pgfqpoint{4.457035in}{1.172519in}}%
\pgfpathclose%
\pgfusepath{fill}%
\end{pgfscope}%
\begin{pgfscope}%
\pgfpathrectangle{\pgfqpoint{0.499691in}{1.172519in}}{\pgfqpoint{7.362500in}{2.695000in}}%
\pgfusepath{clip}%
\pgfsetbuttcap%
\pgfsetmiterjoin%
\definecolor{currentfill}{rgb}{0.179277,0.325175,0.465628}%
\pgfsetfillcolor{currentfill}%
\pgfsetlinewidth{0.000000pt}%
\definecolor{currentstroke}{rgb}{0.000000,0.000000,0.000000}%
\pgfsetstrokecolor{currentstroke}%
\pgfsetstrokeopacity{0.000000}%
\pgfsetdash{}{0pt}%
\pgfpathmoveto{\pgfqpoint{4.588508in}{1.172519in}}%
\pgfpathlineto{\pgfqpoint{4.693687in}{1.172519in}}%
\pgfpathlineto{\pgfqpoint{4.693687in}{2.240671in}}%
\pgfpathlineto{\pgfqpoint{4.588508in}{2.240671in}}%
\pgfpathlineto{\pgfqpoint{4.588508in}{1.172519in}}%
\pgfpathclose%
\pgfusepath{fill}%
\end{pgfscope}%
\begin{pgfscope}%
\pgfpathrectangle{\pgfqpoint{0.499691in}{1.172519in}}{\pgfqpoint{7.362500in}{2.695000in}}%
\pgfusepath{clip}%
\pgfsetbuttcap%
\pgfsetmiterjoin%
\definecolor{currentfill}{rgb}{0.196092,0.283810,0.445045}%
\pgfsetfillcolor{currentfill}%
\pgfsetlinewidth{0.000000pt}%
\definecolor{currentstroke}{rgb}{0.000000,0.000000,0.000000}%
\pgfsetstrokecolor{currentstroke}%
\pgfsetstrokeopacity{0.000000}%
\pgfsetdash{}{0pt}%
\pgfpathmoveto{\pgfqpoint{4.719981in}{1.172519in}}%
\pgfpathlineto{\pgfqpoint{4.825160in}{1.172519in}}%
\pgfpathlineto{\pgfqpoint{4.825160in}{2.217872in}}%
\pgfpathlineto{\pgfqpoint{4.719981in}{2.217872in}}%
\pgfpathlineto{\pgfqpoint{4.719981in}{1.172519in}}%
\pgfpathclose%
\pgfusepath{fill}%
\end{pgfscope}%
\begin{pgfscope}%
\pgfpathrectangle{\pgfqpoint{0.499691in}{1.172519in}}{\pgfqpoint{7.362500in}{2.695000in}}%
\pgfusepath{clip}%
\pgfsetbuttcap%
\pgfsetmiterjoin%
\definecolor{currentfill}{rgb}{0.174980,0.335140,0.469449}%
\pgfsetfillcolor{currentfill}%
\pgfsetlinewidth{0.000000pt}%
\definecolor{currentstroke}{rgb}{0.000000,0.000000,0.000000}%
\pgfsetstrokecolor{currentstroke}%
\pgfsetstrokeopacity{0.000000}%
\pgfsetdash{}{0pt}%
\pgfpathmoveto{\pgfqpoint{4.851455in}{1.172519in}}%
\pgfpathlineto{\pgfqpoint{4.956633in}{1.172519in}}%
\pgfpathlineto{\pgfqpoint{4.956633in}{2.234577in}}%
\pgfpathlineto{\pgfqpoint{4.851455in}{2.234577in}}%
\pgfpathlineto{\pgfqpoint{4.851455in}{1.172519in}}%
\pgfpathclose%
\pgfusepath{fill}%
\end{pgfscope}%
\begin{pgfscope}%
\pgfpathrectangle{\pgfqpoint{0.499691in}{1.172519in}}{\pgfqpoint{7.362500in}{2.695000in}}%
\pgfusepath{clip}%
\pgfsetbuttcap%
\pgfsetmiterjoin%
\definecolor{currentfill}{rgb}{0.164337,0.372401,0.480845}%
\pgfsetfillcolor{currentfill}%
\pgfsetlinewidth{0.000000pt}%
\definecolor{currentstroke}{rgb}{0.000000,0.000000,0.000000}%
\pgfsetstrokecolor{currentstroke}%
\pgfsetstrokeopacity{0.000000}%
\pgfsetdash{}{0pt}%
\pgfpathmoveto{\pgfqpoint{4.982928in}{1.172519in}}%
\pgfpathlineto{\pgfqpoint{5.088106in}{1.172519in}}%
\pgfpathlineto{\pgfqpoint{5.088106in}{3.249165in}}%
\pgfpathlineto{\pgfqpoint{4.982928in}{3.249165in}}%
\pgfpathlineto{\pgfqpoint{4.982928in}{1.172519in}}%
\pgfpathclose%
\pgfusepath{fill}%
\end{pgfscope}%
\begin{pgfscope}%
\pgfpathrectangle{\pgfqpoint{0.499691in}{1.172519in}}{\pgfqpoint{7.362500in}{2.695000in}}%
\pgfusepath{clip}%
\pgfsetbuttcap%
\pgfsetmiterjoin%
\definecolor{currentfill}{rgb}{0.642098,0.766125,0.595747}%
\pgfsetfillcolor{currentfill}%
\pgfsetlinewidth{0.000000pt}%
\definecolor{currentstroke}{rgb}{0.000000,0.000000,0.000000}%
\pgfsetstrokecolor{currentstroke}%
\pgfsetstrokeopacity{0.000000}%
\pgfsetdash{}{0pt}%
\pgfpathmoveto{\pgfqpoint{5.114401in}{1.172519in}}%
\pgfpathlineto{\pgfqpoint{5.219580in}{1.172519in}}%
\pgfpathlineto{\pgfqpoint{5.219580in}{1.751022in}}%
\pgfpathlineto{\pgfqpoint{5.114401in}{1.751022in}}%
\pgfpathlineto{\pgfqpoint{5.114401in}{1.172519in}}%
\pgfpathclose%
\pgfusepath{fill}%
\end{pgfscope}%
\begin{pgfscope}%
\pgfpathrectangle{\pgfqpoint{0.499691in}{1.172519in}}{\pgfqpoint{7.362500in}{2.695000in}}%
\pgfusepath{clip}%
\pgfsetbuttcap%
\pgfsetmiterjoin%
\definecolor{currentfill}{rgb}{0.568623,0.726620,0.588802}%
\pgfsetfillcolor{currentfill}%
\pgfsetlinewidth{0.000000pt}%
\definecolor{currentstroke}{rgb}{0.000000,0.000000,0.000000}%
\pgfsetstrokecolor{currentstroke}%
\pgfsetstrokeopacity{0.000000}%
\pgfsetdash{}{0pt}%
\pgfpathmoveto{\pgfqpoint{5.245874in}{1.172519in}}%
\pgfpathlineto{\pgfqpoint{5.351053in}{1.172519in}}%
\pgfpathlineto{\pgfqpoint{5.351053in}{2.573106in}}%
\pgfpathlineto{\pgfqpoint{5.245874in}{2.573106in}}%
\pgfpathlineto{\pgfqpoint{5.245874in}{1.172519in}}%
\pgfpathclose%
\pgfusepath{fill}%
\end{pgfscope}%
\begin{pgfscope}%
\pgfpathrectangle{\pgfqpoint{0.499691in}{1.172519in}}{\pgfqpoint{7.362500in}{2.695000in}}%
\pgfusepath{clip}%
\pgfsetbuttcap%
\pgfsetmiterjoin%
\definecolor{currentfill}{rgb}{0.439094,0.649090,0.560463}%
\pgfsetfillcolor{currentfill}%
\pgfsetlinewidth{0.000000pt}%
\definecolor{currentstroke}{rgb}{0.000000,0.000000,0.000000}%
\pgfsetstrokecolor{currentstroke}%
\pgfsetstrokeopacity{0.000000}%
\pgfsetdash{}{0pt}%
\pgfpathmoveto{\pgfqpoint{5.377347in}{1.172519in}}%
\pgfpathlineto{\pgfqpoint{5.482526in}{1.172519in}}%
\pgfpathlineto{\pgfqpoint{5.482526in}{1.964155in}}%
\pgfpathlineto{\pgfqpoint{5.377347in}{1.964155in}}%
\pgfpathlineto{\pgfqpoint{5.377347in}{1.172519in}}%
\pgfpathclose%
\pgfusepath{fill}%
\end{pgfscope}%
\begin{pgfscope}%
\pgfpathrectangle{\pgfqpoint{0.499691in}{1.172519in}}{\pgfqpoint{7.362500in}{2.695000in}}%
\pgfusepath{clip}%
\pgfsetbuttcap%
\pgfsetmiterjoin%
\definecolor{currentfill}{rgb}{0.607110,0.748061,0.594692}%
\pgfsetfillcolor{currentfill}%
\pgfsetlinewidth{0.000000pt}%
\definecolor{currentstroke}{rgb}{0.000000,0.000000,0.000000}%
\pgfsetstrokecolor{currentstroke}%
\pgfsetstrokeopacity{0.000000}%
\pgfsetdash{}{0pt}%
\pgfpathmoveto{\pgfqpoint{5.508821in}{1.172519in}}%
\pgfpathlineto{\pgfqpoint{5.613999in}{1.172519in}}%
\pgfpathlineto{\pgfqpoint{5.613999in}{3.428682in}}%
\pgfpathlineto{\pgfqpoint{5.508821in}{3.428682in}}%
\pgfpathlineto{\pgfqpoint{5.508821in}{1.172519in}}%
\pgfpathclose%
\pgfusepath{fill}%
\end{pgfscope}%
\begin{pgfscope}%
\pgfpathrectangle{\pgfqpoint{0.499691in}{1.172519in}}{\pgfqpoint{7.362500in}{2.695000in}}%
\pgfusepath{clip}%
\pgfsetbuttcap%
\pgfsetmiterjoin%
\definecolor{currentfill}{rgb}{0.165839,0.362623,0.478178}%
\pgfsetfillcolor{currentfill}%
\pgfsetlinewidth{0.000000pt}%
\definecolor{currentstroke}{rgb}{0.000000,0.000000,0.000000}%
\pgfsetstrokecolor{currentstroke}%
\pgfsetstrokeopacity{0.000000}%
\pgfsetdash{}{0pt}%
\pgfpathmoveto{\pgfqpoint{5.640294in}{1.172519in}}%
\pgfpathlineto{\pgfqpoint{5.745472in}{1.172519in}}%
\pgfpathlineto{\pgfqpoint{5.745472in}{2.334408in}}%
\pgfpathlineto{\pgfqpoint{5.640294in}{2.334408in}}%
\pgfpathlineto{\pgfqpoint{5.640294in}{1.172519in}}%
\pgfpathclose%
\pgfusepath{fill}%
\end{pgfscope}%
\begin{pgfscope}%
\pgfpathrectangle{\pgfqpoint{0.499691in}{1.172519in}}{\pgfqpoint{7.362500in}{2.695000in}}%
\pgfusepath{clip}%
\pgfsetbuttcap%
\pgfsetmiterjoin%
\definecolor{currentfill}{rgb}{0.189413,0.430822,0.495265}%
\pgfsetfillcolor{currentfill}%
\pgfsetlinewidth{0.000000pt}%
\definecolor{currentstroke}{rgb}{0.000000,0.000000,0.000000}%
\pgfsetstrokecolor{currentstroke}%
\pgfsetstrokeopacity{0.000000}%
\pgfsetdash{}{0pt}%
\pgfpathmoveto{\pgfqpoint{5.771767in}{1.172519in}}%
\pgfpathlineto{\pgfqpoint{5.876946in}{1.172519in}}%
\pgfpathlineto{\pgfqpoint{5.876946in}{1.922190in}}%
\pgfpathlineto{\pgfqpoint{5.771767in}{1.922190in}}%
\pgfpathlineto{\pgfqpoint{5.771767in}{1.172519in}}%
\pgfpathclose%
\pgfusepath{fill}%
\end{pgfscope}%
\begin{pgfscope}%
\pgfpathrectangle{\pgfqpoint{0.499691in}{1.172519in}}{\pgfqpoint{7.362500in}{2.695000in}}%
\pgfusepath{clip}%
\pgfsetbuttcap%
\pgfsetmiterjoin%
\definecolor{currentfill}{rgb}{0.389428,0.613257,0.549244}%
\pgfsetfillcolor{currentfill}%
\pgfsetlinewidth{0.000000pt}%
\definecolor{currentstroke}{rgb}{0.000000,0.000000,0.000000}%
\pgfsetstrokecolor{currentstroke}%
\pgfsetstrokeopacity{0.000000}%
\pgfsetdash{}{0pt}%
\pgfpathmoveto{\pgfqpoint{5.903240in}{1.172519in}}%
\pgfpathlineto{\pgfqpoint{6.008419in}{1.172519in}}%
\pgfpathlineto{\pgfqpoint{6.008419in}{2.154883in}}%
\pgfpathlineto{\pgfqpoint{5.903240in}{2.154883in}}%
\pgfpathlineto{\pgfqpoint{5.903240in}{1.172519in}}%
\pgfpathclose%
\pgfusepath{fill}%
\end{pgfscope}%
\begin{pgfscope}%
\pgfpathrectangle{\pgfqpoint{0.499691in}{1.172519in}}{\pgfqpoint{7.362500in}{2.695000in}}%
\pgfusepath{clip}%
\pgfsetbuttcap%
\pgfsetmiterjoin%
\definecolor{currentfill}{rgb}{0.240460,0.480780,0.508247}%
\pgfsetfillcolor{currentfill}%
\pgfsetlinewidth{0.000000pt}%
\definecolor{currentstroke}{rgb}{0.000000,0.000000,0.000000}%
\pgfsetstrokecolor{currentstroke}%
\pgfsetstrokeopacity{0.000000}%
\pgfsetdash{}{0pt}%
\pgfpathmoveto{\pgfqpoint{6.034713in}{1.172519in}}%
\pgfpathlineto{\pgfqpoint{6.139892in}{1.172519in}}%
\pgfpathlineto{\pgfqpoint{6.139892in}{2.063802in}}%
\pgfpathlineto{\pgfqpoint{6.034713in}{2.063802in}}%
\pgfpathlineto{\pgfqpoint{6.034713in}{1.172519in}}%
\pgfpathclose%
\pgfusepath{fill}%
\end{pgfscope}%
\begin{pgfscope}%
\pgfpathrectangle{\pgfqpoint{0.499691in}{1.172519in}}{\pgfqpoint{7.362500in}{2.695000in}}%
\pgfusepath{clip}%
\pgfsetbuttcap%
\pgfsetmiterjoin%
\definecolor{currentfill}{rgb}{0.308751,0.544266,0.527144}%
\pgfsetfillcolor{currentfill}%
\pgfsetlinewidth{0.000000pt}%
\definecolor{currentstroke}{rgb}{0.000000,0.000000,0.000000}%
\pgfsetstrokecolor{currentstroke}%
\pgfsetstrokeopacity{0.000000}%
\pgfsetdash{}{0pt}%
\pgfpathmoveto{\pgfqpoint{6.166187in}{1.172519in}}%
\pgfpathlineto{\pgfqpoint{6.271365in}{1.172519in}}%
\pgfpathlineto{\pgfqpoint{6.271365in}{2.216537in}}%
\pgfpathlineto{\pgfqpoint{6.166187in}{2.216537in}}%
\pgfpathlineto{\pgfqpoint{6.166187in}{1.172519in}}%
\pgfpathclose%
\pgfusepath{fill}%
\end{pgfscope}%
\begin{pgfscope}%
\pgfpathrectangle{\pgfqpoint{0.499691in}{1.172519in}}{\pgfqpoint{7.362500in}{2.695000in}}%
\pgfusepath{clip}%
\pgfsetbuttcap%
\pgfsetmiterjoin%
\definecolor{currentfill}{rgb}{0.198801,0.275197,0.439878}%
\pgfsetfillcolor{currentfill}%
\pgfsetlinewidth{0.000000pt}%
\definecolor{currentstroke}{rgb}{0.000000,0.000000,0.000000}%
\pgfsetstrokecolor{currentstroke}%
\pgfsetstrokeopacity{0.000000}%
\pgfsetdash{}{0pt}%
\pgfpathmoveto{\pgfqpoint{6.297660in}{1.172519in}}%
\pgfpathlineto{\pgfqpoint{6.402838in}{1.172519in}}%
\pgfpathlineto{\pgfqpoint{6.402838in}{2.789775in}}%
\pgfpathlineto{\pgfqpoint{6.297660in}{2.789775in}}%
\pgfpathlineto{\pgfqpoint{6.297660in}{1.172519in}}%
\pgfpathclose%
\pgfusepath{fill}%
\end{pgfscope}%
\begin{pgfscope}%
\pgfpathrectangle{\pgfqpoint{0.499691in}{1.172519in}}{\pgfqpoint{7.362500in}{2.695000in}}%
\pgfusepath{clip}%
\pgfsetbuttcap%
\pgfsetmiterjoin%
\definecolor{currentfill}{rgb}{0.168406,0.352848,0.475324}%
\pgfsetfillcolor{currentfill}%
\pgfsetlinewidth{0.000000pt}%
\definecolor{currentstroke}{rgb}{0.000000,0.000000,0.000000}%
\pgfsetstrokecolor{currentstroke}%
\pgfsetstrokeopacity{0.000000}%
\pgfsetdash{}{0pt}%
\pgfpathmoveto{\pgfqpoint{6.429133in}{1.172519in}}%
\pgfpathlineto{\pgfqpoint{6.534312in}{1.172519in}}%
\pgfpathlineto{\pgfqpoint{6.534312in}{2.289773in}}%
\pgfpathlineto{\pgfqpoint{6.429133in}{2.289773in}}%
\pgfpathlineto{\pgfqpoint{6.429133in}{1.172519in}}%
\pgfpathclose%
\pgfusepath{fill}%
\end{pgfscope}%
\begin{pgfscope}%
\pgfpathrectangle{\pgfqpoint{0.499691in}{1.172519in}}{\pgfqpoint{7.362500in}{2.695000in}}%
\pgfusepath{clip}%
\pgfsetbuttcap%
\pgfsetmiterjoin%
\definecolor{currentfill}{rgb}{0.491283,0.682262,0.571685}%
\pgfsetfillcolor{currentfill}%
\pgfsetlinewidth{0.000000pt}%
\definecolor{currentstroke}{rgb}{0.000000,0.000000,0.000000}%
\pgfsetstrokecolor{currentstroke}%
\pgfsetstrokeopacity{0.000000}%
\pgfsetdash{}{0pt}%
\pgfpathmoveto{\pgfqpoint{6.560606in}{1.172519in}}%
\pgfpathlineto{\pgfqpoint{6.665785in}{1.172519in}}%
\pgfpathlineto{\pgfqpoint{6.665785in}{3.279924in}}%
\pgfpathlineto{\pgfqpoint{6.560606in}{3.279924in}}%
\pgfpathlineto{\pgfqpoint{6.560606in}{1.172519in}}%
\pgfpathclose%
\pgfusepath{fill}%
\end{pgfscope}%
\begin{pgfscope}%
\pgfpathrectangle{\pgfqpoint{0.499691in}{1.172519in}}{\pgfqpoint{7.362500in}{2.695000in}}%
\pgfusepath{clip}%
\pgfsetbuttcap%
\pgfsetmiterjoin%
\definecolor{currentfill}{rgb}{0.188125,0.304829,0.456476}%
\pgfsetfillcolor{currentfill}%
\pgfsetlinewidth{0.000000pt}%
\definecolor{currentstroke}{rgb}{0.000000,0.000000,0.000000}%
\pgfsetstrokecolor{currentstroke}%
\pgfsetstrokeopacity{0.000000}%
\pgfsetdash{}{0pt}%
\pgfpathmoveto{\pgfqpoint{6.692080in}{1.172519in}}%
\pgfpathlineto{\pgfqpoint{6.797258in}{1.172519in}}%
\pgfpathlineto{\pgfqpoint{6.797258in}{3.178818in}}%
\pgfpathlineto{\pgfqpoint{6.692080in}{3.178818in}}%
\pgfpathlineto{\pgfqpoint{6.692080in}{1.172519in}}%
\pgfpathclose%
\pgfusepath{fill}%
\end{pgfscope}%
\begin{pgfscope}%
\pgfpathrectangle{\pgfqpoint{0.499691in}{1.172519in}}{\pgfqpoint{7.362500in}{2.695000in}}%
\pgfusepath{clip}%
\pgfsetbuttcap%
\pgfsetmiterjoin%
\definecolor{currentfill}{rgb}{0.348423,0.579837,0.538765}%
\pgfsetfillcolor{currentfill}%
\pgfsetlinewidth{0.000000pt}%
\definecolor{currentstroke}{rgb}{0.000000,0.000000,0.000000}%
\pgfsetstrokecolor{currentstroke}%
\pgfsetstrokeopacity{0.000000}%
\pgfsetdash{}{0pt}%
\pgfpathmoveto{\pgfqpoint{6.823553in}{1.172519in}}%
\pgfpathlineto{\pgfqpoint{6.928731in}{1.172519in}}%
\pgfpathlineto{\pgfqpoint{6.928731in}{2.281317in}}%
\pgfpathlineto{\pgfqpoint{6.823553in}{2.281317in}}%
\pgfpathlineto{\pgfqpoint{6.823553in}{1.172519in}}%
\pgfpathclose%
\pgfusepath{fill}%
\end{pgfscope}%
\begin{pgfscope}%
\pgfpathrectangle{\pgfqpoint{0.499691in}{1.172519in}}{\pgfqpoint{7.362500in}{2.695000in}}%
\pgfusepath{clip}%
\pgfsetbuttcap%
\pgfsetmiterjoin%
\definecolor{currentfill}{rgb}{0.624593,0.757104,0.595421}%
\pgfsetfillcolor{currentfill}%
\pgfsetlinewidth{0.000000pt}%
\definecolor{currentstroke}{rgb}{0.000000,0.000000,0.000000}%
\pgfsetstrokecolor{currentstroke}%
\pgfsetstrokeopacity{0.000000}%
\pgfsetdash{}{0pt}%
\pgfpathmoveto{\pgfqpoint{6.955026in}{1.172519in}}%
\pgfpathlineto{\pgfqpoint{7.060205in}{1.172519in}}%
\pgfpathlineto{\pgfqpoint{7.060205in}{2.136691in}}%
\pgfpathlineto{\pgfqpoint{6.955026in}{2.136691in}}%
\pgfpathlineto{\pgfqpoint{6.955026in}{1.172519in}}%
\pgfpathclose%
\pgfusepath{fill}%
\end{pgfscope}%
\begin{pgfscope}%
\pgfpathrectangle{\pgfqpoint{0.499691in}{1.172519in}}{\pgfqpoint{7.362500in}{2.695000in}}%
\pgfusepath{clip}%
\pgfsetbuttcap%
\pgfsetmiterjoin%
\definecolor{currentfill}{rgb}{0.359677,0.589370,0.541787}%
\pgfsetfillcolor{currentfill}%
\pgfsetlinewidth{0.000000pt}%
\definecolor{currentstroke}{rgb}{0.000000,0.000000,0.000000}%
\pgfsetstrokecolor{currentstroke}%
\pgfsetstrokeopacity{0.000000}%
\pgfsetdash{}{0pt}%
\pgfpathmoveto{\pgfqpoint{7.086499in}{1.172519in}}%
\pgfpathlineto{\pgfqpoint{7.191678in}{1.172519in}}%
\pgfpathlineto{\pgfqpoint{7.191678in}{2.080448in}}%
\pgfpathlineto{\pgfqpoint{7.086499in}{2.080448in}}%
\pgfpathlineto{\pgfqpoint{7.086499in}{1.172519in}}%
\pgfpathclose%
\pgfusepath{fill}%
\end{pgfscope}%
\begin{pgfscope}%
\pgfpathrectangle{\pgfqpoint{0.499691in}{1.172519in}}{\pgfqpoint{7.362500in}{2.695000in}}%
\pgfusepath{clip}%
\pgfsetbuttcap%
\pgfsetmiterjoin%
\definecolor{currentfill}{rgb}{0.348423,0.579837,0.538765}%
\pgfsetfillcolor{currentfill}%
\pgfsetlinewidth{0.000000pt}%
\definecolor{currentstroke}{rgb}{0.000000,0.000000,0.000000}%
\pgfsetstrokecolor{currentstroke}%
\pgfsetstrokeopacity{0.000000}%
\pgfsetdash{}{0pt}%
\pgfpathmoveto{\pgfqpoint{7.217972in}{1.172519in}}%
\pgfpathlineto{\pgfqpoint{7.323151in}{1.172519in}}%
\pgfpathlineto{\pgfqpoint{7.323151in}{1.998091in}}%
\pgfpathlineto{\pgfqpoint{7.217972in}{1.998091in}}%
\pgfpathlineto{\pgfqpoint{7.217972in}{1.172519in}}%
\pgfpathclose%
\pgfusepath{fill}%
\end{pgfscope}%
\begin{pgfscope}%
\pgfpathrectangle{\pgfqpoint{0.499691in}{1.172519in}}{\pgfqpoint{7.362500in}{2.695000in}}%
\pgfusepath{clip}%
\pgfsetbuttcap%
\pgfsetmiterjoin%
\definecolor{currentfill}{rgb}{0.552281,0.717276,0.585304}%
\pgfsetfillcolor{currentfill}%
\pgfsetlinewidth{0.000000pt}%
\definecolor{currentstroke}{rgb}{0.000000,0.000000,0.000000}%
\pgfsetstrokecolor{currentstroke}%
\pgfsetstrokeopacity{0.000000}%
\pgfsetdash{}{0pt}%
\pgfpathmoveto{\pgfqpoint{7.349446in}{1.172519in}}%
\pgfpathlineto{\pgfqpoint{7.454624in}{1.172519in}}%
\pgfpathlineto{\pgfqpoint{7.454624in}{2.352666in}}%
\pgfpathlineto{\pgfqpoint{7.349446in}{2.352666in}}%
\pgfpathlineto{\pgfqpoint{7.349446in}{1.172519in}}%
\pgfpathclose%
\pgfusepath{fill}%
\end{pgfscope}%
\begin{pgfscope}%
\pgfpathrectangle{\pgfqpoint{0.499691in}{1.172519in}}{\pgfqpoint{7.362500in}{2.695000in}}%
\pgfusepath{clip}%
\pgfsetbuttcap%
\pgfsetmiterjoin%
\definecolor{currentfill}{rgb}{0.511189,0.693989,0.576059}%
\pgfsetfillcolor{currentfill}%
\pgfsetlinewidth{0.000000pt}%
\definecolor{currentstroke}{rgb}{0.000000,0.000000,0.000000}%
\pgfsetstrokecolor{currentstroke}%
\pgfsetstrokeopacity{0.000000}%
\pgfsetdash{}{0pt}%
\pgfpathmoveto{\pgfqpoint{7.480919in}{1.172519in}}%
\pgfpathlineto{\pgfqpoint{7.586097in}{1.172519in}}%
\pgfpathlineto{\pgfqpoint{7.586097in}{2.158117in}}%
\pgfpathlineto{\pgfqpoint{7.480919in}{2.158117in}}%
\pgfpathlineto{\pgfqpoint{7.480919in}{1.172519in}}%
\pgfpathclose%
\pgfusepath{fill}%
\end{pgfscope}%
\begin{pgfscope}%
\pgfpathrectangle{\pgfqpoint{0.499691in}{1.172519in}}{\pgfqpoint{7.362500in}{2.695000in}}%
\pgfusepath{clip}%
\pgfsetbuttcap%
\pgfsetmiterjoin%
\definecolor{currentfill}{rgb}{0.283759,0.520718,0.519153}%
\pgfsetfillcolor{currentfill}%
\pgfsetlinewidth{0.000000pt}%
\definecolor{currentstroke}{rgb}{0.000000,0.000000,0.000000}%
\pgfsetstrokecolor{currentstroke}%
\pgfsetstrokeopacity{0.000000}%
\pgfsetdash{}{0pt}%
\pgfpathmoveto{\pgfqpoint{7.612392in}{1.172519in}}%
\pgfpathlineto{\pgfqpoint{7.717571in}{1.172519in}}%
\pgfpathlineto{\pgfqpoint{7.717571in}{1.985777in}}%
\pgfpathlineto{\pgfqpoint{7.612392in}{1.985777in}}%
\pgfpathlineto{\pgfqpoint{7.612392in}{1.172519in}}%
\pgfpathclose%
\pgfusepath{fill}%
\end{pgfscope}%
\begin{pgfscope}%
\pgfpathrectangle{\pgfqpoint{0.499691in}{1.172519in}}{\pgfqpoint{7.362500in}{2.695000in}}%
\pgfusepath{clip}%
\pgfsetbuttcap%
\pgfsetmiterjoin%
\definecolor{currentfill}{rgb}{0.209419,0.451595,0.500534}%
\pgfsetfillcolor{currentfill}%
\pgfsetlinewidth{0.000000pt}%
\definecolor{currentstroke}{rgb}{0.000000,0.000000,0.000000}%
\pgfsetstrokecolor{currentstroke}%
\pgfsetstrokeopacity{0.000000}%
\pgfsetdash{}{0pt}%
\pgfpathmoveto{\pgfqpoint{7.743865in}{1.172519in}}%
\pgfpathlineto{\pgfqpoint{7.849044in}{1.172519in}}%
\pgfpathlineto{\pgfqpoint{7.849044in}{3.739185in}}%
\pgfpathlineto{\pgfqpoint{7.743865in}{3.739185in}}%
\pgfpathlineto{\pgfqpoint{7.743865in}{1.172519in}}%
\pgfpathclose%
\pgfusepath{fill}%
\end{pgfscope}%
\begin{pgfscope}%
\pgfsetbuttcap%
\pgfsetroundjoin%
\definecolor{currentfill}{rgb}{0.000000,0.000000,0.000000}%
\pgfsetfillcolor{currentfill}%
\pgfsetlinewidth{0.803000pt}%
\definecolor{currentstroke}{rgb}{0.000000,0.000000,0.000000}%
\pgfsetstrokecolor{currentstroke}%
\pgfsetdash{}{0pt}%
\pgfsys@defobject{currentmarker}{\pgfqpoint{0.000000in}{-0.048611in}}{\pgfqpoint{0.000000in}{0.000000in}}{%
\pgfpathmoveto{\pgfqpoint{0.000000in}{0.000000in}}%
\pgfpathlineto{\pgfqpoint{0.000000in}{-0.048611in}}%
\pgfusepath{stroke,fill}%
}%
\begin{pgfscope}%
\pgfsys@transformshift{0.565428in}{1.172519in}%
\pgfsys@useobject{currentmarker}{}%
\end{pgfscope}%
\end{pgfscope}%
\begin{pgfscope}%
\definecolor{textcolor}{rgb}{0.000000,0.000000,0.000000}%
\pgfsetstrokecolor{textcolor}%
\pgfsetfillcolor{textcolor}%
\pgftext[x=0.586261in, y=0.835021in, left, base,rotate=90.000000]{\color{textcolor}\rmfamily\fontsize{6.000000}{7.200000}\selectfont AMV}%
\end{pgfscope}%
\begin{pgfscope}%
\pgfsetbuttcap%
\pgfsetroundjoin%
\definecolor{currentfill}{rgb}{0.000000,0.000000,0.000000}%
\pgfsetfillcolor{currentfill}%
\pgfsetlinewidth{0.803000pt}%
\definecolor{currentstroke}{rgb}{0.000000,0.000000,0.000000}%
\pgfsetstrokecolor{currentstroke}%
\pgfsetdash{}{0pt}%
\pgfsys@defobject{currentmarker}{\pgfqpoint{0.000000in}{-0.048611in}}{\pgfqpoint{0.000000in}{0.000000in}}{%
\pgfpathmoveto{\pgfqpoint{0.000000in}{0.000000in}}%
\pgfpathlineto{\pgfqpoint{0.000000in}{-0.048611in}}%
\pgfusepath{stroke,fill}%
}%
\begin{pgfscope}%
\pgfsys@transformshift{0.696901in}{1.172519in}%
\pgfsys@useobject{currentmarker}{}%
\end{pgfscope}%
\end{pgfscope}%
\begin{pgfscope}%
\definecolor{textcolor}{rgb}{0.000000,0.000000,0.000000}%
\pgfsetstrokecolor{textcolor}%
\pgfsetfillcolor{textcolor}%
\pgftext[x=0.717734in, y=0.759097in, left, base,rotate=90.000000]{\color{textcolor}\rmfamily\fontsize{6.000000}{7.200000}\selectfont APRIL}%
\end{pgfscope}%
\begin{pgfscope}%
\pgfsetbuttcap%
\pgfsetroundjoin%
\definecolor{currentfill}{rgb}{0.000000,0.000000,0.000000}%
\pgfsetfillcolor{currentfill}%
\pgfsetlinewidth{0.803000pt}%
\definecolor{currentstroke}{rgb}{0.000000,0.000000,0.000000}%
\pgfsetstrokecolor{currentstroke}%
\pgfsetdash{}{0pt}%
\pgfsys@defobject{currentmarker}{\pgfqpoint{0.000000in}{-0.048611in}}{\pgfqpoint{0.000000in}{0.000000in}}{%
\pgfpathmoveto{\pgfqpoint{0.000000in}{0.000000in}}%
\pgfpathlineto{\pgfqpoint{0.000000in}{-0.048611in}}%
\pgfusepath{stroke,fill}%
}%
\begin{pgfscope}%
\pgfsys@transformshift{0.828374in}{1.172519in}%
\pgfsys@useobject{currentmarker}{}%
\end{pgfscope}%
\end{pgfscope}%
\begin{pgfscope}%
\definecolor{textcolor}{rgb}{0.000000,0.000000,0.000000}%
\pgfsetstrokecolor{textcolor}%
\pgfsetfillcolor{textcolor}%
\pgftext[x=0.849207in, y=0.491508in, left, base,rotate=90.000000]{\color{textcolor}\rmfamily\fontsize{6.000000}{7.200000}\selectfont APRIL Moto}%
\end{pgfscope}%
\begin{pgfscope}%
\pgfsetbuttcap%
\pgfsetroundjoin%
\definecolor{currentfill}{rgb}{0.000000,0.000000,0.000000}%
\pgfsetfillcolor{currentfill}%
\pgfsetlinewidth{0.803000pt}%
\definecolor{currentstroke}{rgb}{0.000000,0.000000,0.000000}%
\pgfsetstrokecolor{currentstroke}%
\pgfsetdash{}{0pt}%
\pgfsys@defobject{currentmarker}{\pgfqpoint{0.000000in}{-0.048611in}}{\pgfqpoint{0.000000in}{0.000000in}}{%
\pgfpathmoveto{\pgfqpoint{0.000000in}{0.000000in}}%
\pgfpathlineto{\pgfqpoint{0.000000in}{-0.048611in}}%
\pgfusepath{stroke,fill}%
}%
\begin{pgfscope}%
\pgfsys@transformshift{0.959847in}{1.172519in}%
\pgfsys@useobject{currentmarker}{}%
\end{pgfscope}%
\end{pgfscope}%
\begin{pgfscope}%
\definecolor{textcolor}{rgb}{0.000000,0.000000,0.000000}%
\pgfsetstrokecolor{textcolor}%
\pgfsetfillcolor{textcolor}%
\pgftext[x=0.980681in, y=0.851225in, left, base,rotate=90.000000]{\color{textcolor}\rmfamily\fontsize{6.000000}{7.200000}\selectfont AXA}%
\end{pgfscope}%
\begin{pgfscope}%
\pgfsetbuttcap%
\pgfsetroundjoin%
\definecolor{currentfill}{rgb}{0.000000,0.000000,0.000000}%
\pgfsetfillcolor{currentfill}%
\pgfsetlinewidth{0.803000pt}%
\definecolor{currentstroke}{rgb}{0.000000,0.000000,0.000000}%
\pgfsetstrokecolor{currentstroke}%
\pgfsetdash{}{0pt}%
\pgfsys@defobject{currentmarker}{\pgfqpoint{0.000000in}{-0.048611in}}{\pgfqpoint{0.000000in}{0.000000in}}{%
\pgfpathmoveto{\pgfqpoint{0.000000in}{0.000000in}}%
\pgfpathlineto{\pgfqpoint{0.000000in}{-0.048611in}}%
\pgfusepath{stroke,fill}%
}%
\begin{pgfscope}%
\pgfsys@transformshift{1.091321in}{1.172519in}%
\pgfsys@useobject{currentmarker}{}%
\end{pgfscope}%
\end{pgfscope}%
\begin{pgfscope}%
\definecolor{textcolor}{rgb}{0.000000,0.000000,0.000000}%
\pgfsetstrokecolor{textcolor}%
\pgfsetfillcolor{textcolor}%
\pgftext[x=1.112154in, y=0.263038in, left, base,rotate=90.000000]{\color{textcolor}\rmfamily\fontsize{6.000000}{7.200000}\selectfont Active Assurances}%
\end{pgfscope}%
\begin{pgfscope}%
\pgfsetbuttcap%
\pgfsetroundjoin%
\definecolor{currentfill}{rgb}{0.000000,0.000000,0.000000}%
\pgfsetfillcolor{currentfill}%
\pgfsetlinewidth{0.803000pt}%
\definecolor{currentstroke}{rgb}{0.000000,0.000000,0.000000}%
\pgfsetstrokecolor{currentstroke}%
\pgfsetdash{}{0pt}%
\pgfsys@defobject{currentmarker}{\pgfqpoint{0.000000in}{-0.048611in}}{\pgfqpoint{0.000000in}{0.000000in}}{%
\pgfpathmoveto{\pgfqpoint{0.000000in}{0.000000in}}%
\pgfpathlineto{\pgfqpoint{0.000000in}{-0.048611in}}%
\pgfusepath{stroke,fill}%
}%
\begin{pgfscope}%
\pgfsys@transformshift{1.222794in}{1.172519in}%
\pgfsys@useobject{currentmarker}{}%
\end{pgfscope}%
\end{pgfscope}%
\begin{pgfscope}%
\definecolor{textcolor}{rgb}{0.000000,0.000000,0.000000}%
\pgfsetstrokecolor{textcolor}%
\pgfsetfillcolor{textcolor}%
\pgftext[x=1.243627in, y=0.882937in, left, base,rotate=90.000000]{\color{textcolor}\rmfamily\fontsize{6.000000}{7.200000}\selectfont Afer}%
\end{pgfscope}%
\begin{pgfscope}%
\pgfsetbuttcap%
\pgfsetroundjoin%
\definecolor{currentfill}{rgb}{0.000000,0.000000,0.000000}%
\pgfsetfillcolor{currentfill}%
\pgfsetlinewidth{0.803000pt}%
\definecolor{currentstroke}{rgb}{0.000000,0.000000,0.000000}%
\pgfsetstrokecolor{currentstroke}%
\pgfsetdash{}{0pt}%
\pgfsys@defobject{currentmarker}{\pgfqpoint{0.000000in}{-0.048611in}}{\pgfqpoint{0.000000in}{0.000000in}}{%
\pgfpathmoveto{\pgfqpoint{0.000000in}{0.000000in}}%
\pgfpathlineto{\pgfqpoint{0.000000in}{-0.048611in}}%
\pgfusepath{stroke,fill}%
}%
\begin{pgfscope}%
\pgfsys@transformshift{1.354267in}{1.172519in}%
\pgfsys@useobject{currentmarker}{}%
\end{pgfscope}%
\end{pgfscope}%
\begin{pgfscope}%
\definecolor{textcolor}{rgb}{0.000000,0.000000,0.000000}%
\pgfsetstrokecolor{textcolor}%
\pgfsetfillcolor{textcolor}%
\pgftext[x=1.375100in, y=0.701922in, left, base,rotate=90.000000]{\color{textcolor}\rmfamily\fontsize{6.000000}{7.200000}\selectfont Afi Esca}%
\end{pgfscope}%
\begin{pgfscope}%
\pgfsetbuttcap%
\pgfsetroundjoin%
\definecolor{currentfill}{rgb}{0.000000,0.000000,0.000000}%
\pgfsetfillcolor{currentfill}%
\pgfsetlinewidth{0.803000pt}%
\definecolor{currentstroke}{rgb}{0.000000,0.000000,0.000000}%
\pgfsetstrokecolor{currentstroke}%
\pgfsetdash{}{0pt}%
\pgfsys@defobject{currentmarker}{\pgfqpoint{0.000000in}{-0.048611in}}{\pgfqpoint{0.000000in}{0.000000in}}{%
\pgfpathmoveto{\pgfqpoint{0.000000in}{0.000000in}}%
\pgfpathlineto{\pgfqpoint{0.000000in}{-0.048611in}}%
\pgfusepath{stroke,fill}%
}%
\begin{pgfscope}%
\pgfsys@transformshift{1.485740in}{1.172519in}%
\pgfsys@useobject{currentmarker}{}%
\end{pgfscope}%
\end{pgfscope}%
\begin{pgfscope}%
\definecolor{textcolor}{rgb}{0.000000,0.000000,0.000000}%
\pgfsetstrokecolor{textcolor}%
\pgfsetfillcolor{textcolor}%
\pgftext[x=1.506574in, y=0.265893in, left, base,rotate=90.000000]{\color{textcolor}\rmfamily\fontsize{6.000000}{7.200000}\selectfont Ag2r La Mondiale}%
\end{pgfscope}%
\begin{pgfscope}%
\pgfsetbuttcap%
\pgfsetroundjoin%
\definecolor{currentfill}{rgb}{0.000000,0.000000,0.000000}%
\pgfsetfillcolor{currentfill}%
\pgfsetlinewidth{0.803000pt}%
\definecolor{currentstroke}{rgb}{0.000000,0.000000,0.000000}%
\pgfsetstrokecolor{currentstroke}%
\pgfsetdash{}{0pt}%
\pgfsys@defobject{currentmarker}{\pgfqpoint{0.000000in}{-0.048611in}}{\pgfqpoint{0.000000in}{0.000000in}}{%
\pgfpathmoveto{\pgfqpoint{0.000000in}{0.000000in}}%
\pgfpathlineto{\pgfqpoint{0.000000in}{-0.048611in}}%
\pgfusepath{stroke,fill}%
}%
\begin{pgfscope}%
\pgfsys@transformshift{1.617213in}{1.172519in}%
\pgfsys@useobject{currentmarker}{}%
\end{pgfscope}%
\end{pgfscope}%
\begin{pgfscope}%
\definecolor{textcolor}{rgb}{0.000000,0.000000,0.000000}%
\pgfsetstrokecolor{textcolor}%
\pgfsetfillcolor{textcolor}%
\pgftext[x=1.638047in, y=0.759868in, left, base,rotate=90.000000]{\color{textcolor}\rmfamily\fontsize{6.000000}{7.200000}\selectfont Allianz}%
\end{pgfscope}%
\begin{pgfscope}%
\pgfsetbuttcap%
\pgfsetroundjoin%
\definecolor{currentfill}{rgb}{0.000000,0.000000,0.000000}%
\pgfsetfillcolor{currentfill}%
\pgfsetlinewidth{0.803000pt}%
\definecolor{currentstroke}{rgb}{0.000000,0.000000,0.000000}%
\pgfsetstrokecolor{currentstroke}%
\pgfsetdash{}{0pt}%
\pgfsys@defobject{currentmarker}{\pgfqpoint{0.000000in}{-0.048611in}}{\pgfqpoint{0.000000in}{0.000000in}}{%
\pgfpathmoveto{\pgfqpoint{0.000000in}{0.000000in}}%
\pgfpathlineto{\pgfqpoint{0.000000in}{-0.048611in}}%
\pgfusepath{stroke,fill}%
}%
\begin{pgfscope}%
\pgfsys@transformshift{1.748687in}{1.172519in}%
\pgfsys@useobject{currentmarker}{}%
\end{pgfscope}%
\end{pgfscope}%
\begin{pgfscope}%
\definecolor{textcolor}{rgb}{0.000000,0.000000,0.000000}%
\pgfsetstrokecolor{textcolor}%
\pgfsetfillcolor{textcolor}%
\pgftext[x=1.769520in, y=0.370445in, left, base,rotate=90.000000]{\color{textcolor}\rmfamily\fontsize{6.000000}{7.200000}\selectfont Assur Bon Plan}%
\end{pgfscope}%
\begin{pgfscope}%
\pgfsetbuttcap%
\pgfsetroundjoin%
\definecolor{currentfill}{rgb}{0.000000,0.000000,0.000000}%
\pgfsetfillcolor{currentfill}%
\pgfsetlinewidth{0.803000pt}%
\definecolor{currentstroke}{rgb}{0.000000,0.000000,0.000000}%
\pgfsetstrokecolor{currentstroke}%
\pgfsetdash{}{0pt}%
\pgfsys@defobject{currentmarker}{\pgfqpoint{0.000000in}{-0.048611in}}{\pgfqpoint{0.000000in}{0.000000in}}{%
\pgfpathmoveto{\pgfqpoint{0.000000in}{0.000000in}}%
\pgfpathlineto{\pgfqpoint{0.000000in}{-0.048611in}}%
\pgfusepath{stroke,fill}%
}%
\begin{pgfscope}%
\pgfsys@transformshift{1.880160in}{1.172519in}%
\pgfsys@useobject{currentmarker}{}%
\end{pgfscope}%
\end{pgfscope}%
\begin{pgfscope}%
\definecolor{textcolor}{rgb}{0.000000,0.000000,0.000000}%
\pgfsetstrokecolor{textcolor}%
\pgfsetfillcolor{textcolor}%
\pgftext[x=1.900993in, y=0.505782in, left, base,rotate=90.000000]{\color{textcolor}\rmfamily\fontsize{6.000000}{7.200000}\selectfont Assur O'Poil}%
\end{pgfscope}%
\begin{pgfscope}%
\pgfsetbuttcap%
\pgfsetroundjoin%
\definecolor{currentfill}{rgb}{0.000000,0.000000,0.000000}%
\pgfsetfillcolor{currentfill}%
\pgfsetlinewidth{0.803000pt}%
\definecolor{currentstroke}{rgb}{0.000000,0.000000,0.000000}%
\pgfsetstrokecolor{currentstroke}%
\pgfsetdash{}{0pt}%
\pgfsys@defobject{currentmarker}{\pgfqpoint{0.000000in}{-0.048611in}}{\pgfqpoint{0.000000in}{0.000000in}}{%
\pgfpathmoveto{\pgfqpoint{0.000000in}{0.000000in}}%
\pgfpathlineto{\pgfqpoint{0.000000in}{-0.048611in}}%
\pgfusepath{stroke,fill}%
}%
\begin{pgfscope}%
\pgfsys@transformshift{2.011633in}{1.172519in}%
\pgfsys@useobject{currentmarker}{}%
\end{pgfscope}%
\end{pgfscope}%
\begin{pgfscope}%
\definecolor{textcolor}{rgb}{0.000000,0.000000,0.000000}%
\pgfsetstrokecolor{textcolor}%
\pgfsetfillcolor{textcolor}%
\pgftext[x=2.032466in, y=0.528081in, left, base,rotate=90.000000]{\color{textcolor}\rmfamily\fontsize{6.000000}{7.200000}\selectfont AssurOnline}%
\end{pgfscope}%
\begin{pgfscope}%
\pgfsetbuttcap%
\pgfsetroundjoin%
\definecolor{currentfill}{rgb}{0.000000,0.000000,0.000000}%
\pgfsetfillcolor{currentfill}%
\pgfsetlinewidth{0.803000pt}%
\definecolor{currentstroke}{rgb}{0.000000,0.000000,0.000000}%
\pgfsetstrokecolor{currentstroke}%
\pgfsetdash{}{0pt}%
\pgfsys@defobject{currentmarker}{\pgfqpoint{0.000000in}{-0.048611in}}{\pgfqpoint{0.000000in}{0.000000in}}{%
\pgfpathmoveto{\pgfqpoint{0.000000in}{0.000000in}}%
\pgfpathlineto{\pgfqpoint{0.000000in}{-0.048611in}}%
\pgfusepath{stroke,fill}%
}%
\begin{pgfscope}%
\pgfsys@transformshift{2.143106in}{1.172519in}%
\pgfsys@useobject{currentmarker}{}%
\end{pgfscope}%
\end{pgfscope}%
\begin{pgfscope}%
\definecolor{textcolor}{rgb}{0.000000,0.000000,0.000000}%
\pgfsetstrokecolor{textcolor}%
\pgfsetfillcolor{textcolor}%
\pgftext[x=2.163940in, y=0.333716in, left, base,rotate=90.000000]{\color{textcolor}\rmfamily\fontsize{6.000000}{7.200000}\selectfont CNP Assurances}%
\end{pgfscope}%
\begin{pgfscope}%
\pgfsetbuttcap%
\pgfsetroundjoin%
\definecolor{currentfill}{rgb}{0.000000,0.000000,0.000000}%
\pgfsetfillcolor{currentfill}%
\pgfsetlinewidth{0.803000pt}%
\definecolor{currentstroke}{rgb}{0.000000,0.000000,0.000000}%
\pgfsetstrokecolor{currentstroke}%
\pgfsetdash{}{0pt}%
\pgfsys@defobject{currentmarker}{\pgfqpoint{0.000000in}{-0.048611in}}{\pgfqpoint{0.000000in}{0.000000in}}{%
\pgfpathmoveto{\pgfqpoint{0.000000in}{0.000000in}}%
\pgfpathlineto{\pgfqpoint{0.000000in}{-0.048611in}}%
\pgfusepath{stroke,fill}%
}%
\begin{pgfscope}%
\pgfsys@transformshift{2.274580in}{1.172519in}%
\pgfsys@useobject{currentmarker}{}%
\end{pgfscope}%
\end{pgfscope}%
\begin{pgfscope}%
\definecolor{textcolor}{rgb}{0.000000,0.000000,0.000000}%
\pgfsetstrokecolor{textcolor}%
\pgfsetfillcolor{textcolor}%
\pgftext[x=2.295413in, y=0.815269in, left, base,rotate=90.000000]{\color{textcolor}\rmfamily\fontsize{6.000000}{7.200000}\selectfont Carac}%
\end{pgfscope}%
\begin{pgfscope}%
\pgfsetbuttcap%
\pgfsetroundjoin%
\definecolor{currentfill}{rgb}{0.000000,0.000000,0.000000}%
\pgfsetfillcolor{currentfill}%
\pgfsetlinewidth{0.803000pt}%
\definecolor{currentstroke}{rgb}{0.000000,0.000000,0.000000}%
\pgfsetstrokecolor{currentstroke}%
\pgfsetdash{}{0pt}%
\pgfsys@defobject{currentmarker}{\pgfqpoint{0.000000in}{-0.048611in}}{\pgfqpoint{0.000000in}{0.000000in}}{%
\pgfpathmoveto{\pgfqpoint{0.000000in}{0.000000in}}%
\pgfpathlineto{\pgfqpoint{0.000000in}{-0.048611in}}%
\pgfusepath{stroke,fill}%
}%
\begin{pgfscope}%
\pgfsys@transformshift{2.406053in}{1.172519in}%
\pgfsys@useobject{currentmarker}{}%
\end{pgfscope}%
\end{pgfscope}%
\begin{pgfscope}%
\definecolor{textcolor}{rgb}{0.000000,0.000000,0.000000}%
\pgfsetstrokecolor{textcolor}%
\pgfsetfillcolor{textcolor}%
\pgftext[x=2.426886in, y=0.794050in, left, base,rotate=90.000000]{\color{textcolor}\rmfamily\fontsize{6.000000}{7.200000}\selectfont Cardif}%
\end{pgfscope}%
\begin{pgfscope}%
\pgfsetbuttcap%
\pgfsetroundjoin%
\definecolor{currentfill}{rgb}{0.000000,0.000000,0.000000}%
\pgfsetfillcolor{currentfill}%
\pgfsetlinewidth{0.803000pt}%
\definecolor{currentstroke}{rgb}{0.000000,0.000000,0.000000}%
\pgfsetstrokecolor{currentstroke}%
\pgfsetdash{}{0pt}%
\pgfsys@defobject{currentmarker}{\pgfqpoint{0.000000in}{-0.048611in}}{\pgfqpoint{0.000000in}{0.000000in}}{%
\pgfpathmoveto{\pgfqpoint{0.000000in}{0.000000in}}%
\pgfpathlineto{\pgfqpoint{0.000000in}{-0.048611in}}%
\pgfusepath{stroke,fill}%
}%
\begin{pgfscope}%
\pgfsys@transformshift{2.537526in}{1.172519in}%
\pgfsys@useobject{currentmarker}{}%
\end{pgfscope}%
\end{pgfscope}%
\begin{pgfscope}%
\definecolor{textcolor}{rgb}{0.000000,0.000000,0.000000}%
\pgfsetstrokecolor{textcolor}%
\pgfsetfillcolor{textcolor}%
\pgftext[x=2.558359in, y=0.200385in, left, base,rotate=90.000000]{\color{textcolor}\rmfamily\fontsize{6.000000}{7.200000}\selectfont Cegema Assurances}%
\end{pgfscope}%
\begin{pgfscope}%
\pgfsetbuttcap%
\pgfsetroundjoin%
\definecolor{currentfill}{rgb}{0.000000,0.000000,0.000000}%
\pgfsetfillcolor{currentfill}%
\pgfsetlinewidth{0.803000pt}%
\definecolor{currentstroke}{rgb}{0.000000,0.000000,0.000000}%
\pgfsetstrokecolor{currentstroke}%
\pgfsetdash{}{0pt}%
\pgfsys@defobject{currentmarker}{\pgfqpoint{0.000000in}{-0.048611in}}{\pgfqpoint{0.000000in}{0.000000in}}{%
\pgfpathmoveto{\pgfqpoint{0.000000in}{0.000000in}}%
\pgfpathlineto{\pgfqpoint{0.000000in}{-0.048611in}}%
\pgfusepath{stroke,fill}%
}%
\begin{pgfscope}%
\pgfsys@transformshift{2.668999in}{1.172519in}%
\pgfsys@useobject{currentmarker}{}%
\end{pgfscope}%
\end{pgfscope}%
\begin{pgfscope}%
\definecolor{textcolor}{rgb}{0.000000,0.000000,0.000000}%
\pgfsetstrokecolor{textcolor}%
\pgfsetfillcolor{textcolor}%
\pgftext[x=2.689832in, y=0.438112in, left, base,rotate=90.000000]{\color{textcolor}\rmfamily\fontsize{6.000000}{7.200000}\selectfont Crédit Mutuel}%
\end{pgfscope}%
\begin{pgfscope}%
\pgfsetbuttcap%
\pgfsetroundjoin%
\definecolor{currentfill}{rgb}{0.000000,0.000000,0.000000}%
\pgfsetfillcolor{currentfill}%
\pgfsetlinewidth{0.803000pt}%
\definecolor{currentstroke}{rgb}{0.000000,0.000000,0.000000}%
\pgfsetstrokecolor{currentstroke}%
\pgfsetdash{}{0pt}%
\pgfsys@defobject{currentmarker}{\pgfqpoint{0.000000in}{-0.048611in}}{\pgfqpoint{0.000000in}{0.000000in}}{%
\pgfpathmoveto{\pgfqpoint{0.000000in}{0.000000in}}%
\pgfpathlineto{\pgfqpoint{0.000000in}{-0.048611in}}%
\pgfusepath{stroke,fill}%
}%
\begin{pgfscope}%
\pgfsys@transformshift{2.800472in}{1.172519in}%
\pgfsys@useobject{currentmarker}{}%
\end{pgfscope}%
\end{pgfscope}%
\begin{pgfscope}%
\definecolor{textcolor}{rgb}{0.000000,0.000000,0.000000}%
\pgfsetstrokecolor{textcolor}%
\pgfsetfillcolor{textcolor}%
\pgftext[x=2.821306in, y=0.312883in, left, base,rotate=90.000000]{\color{textcolor}\rmfamily\fontsize{6.000000}{7.200000}\selectfont Direct Assurance}%
\end{pgfscope}%
\begin{pgfscope}%
\pgfsetbuttcap%
\pgfsetroundjoin%
\definecolor{currentfill}{rgb}{0.000000,0.000000,0.000000}%
\pgfsetfillcolor{currentfill}%
\pgfsetlinewidth{0.803000pt}%
\definecolor{currentstroke}{rgb}{0.000000,0.000000,0.000000}%
\pgfsetstrokecolor{currentstroke}%
\pgfsetdash{}{0pt}%
\pgfsys@defobject{currentmarker}{\pgfqpoint{0.000000in}{-0.048611in}}{\pgfqpoint{0.000000in}{0.000000in}}{%
\pgfpathmoveto{\pgfqpoint{0.000000in}{0.000000in}}%
\pgfpathlineto{\pgfqpoint{0.000000in}{-0.048611in}}%
\pgfusepath{stroke,fill}%
}%
\begin{pgfscope}%
\pgfsys@transformshift{2.931946in}{1.172519in}%
\pgfsys@useobject{currentmarker}{}%
\end{pgfscope}%
\end{pgfscope}%
\begin{pgfscope}%
\definecolor{textcolor}{rgb}{0.000000,0.000000,0.000000}%
\pgfsetstrokecolor{textcolor}%
\pgfsetfillcolor{textcolor}%
\pgftext[x=2.952779in, y=0.384488in, left, base,rotate=90.000000]{\color{textcolor}\rmfamily\fontsize{6.000000}{7.200000}\selectfont Eca Assurances}%
\end{pgfscope}%
\begin{pgfscope}%
\pgfsetbuttcap%
\pgfsetroundjoin%
\definecolor{currentfill}{rgb}{0.000000,0.000000,0.000000}%
\pgfsetfillcolor{currentfill}%
\pgfsetlinewidth{0.803000pt}%
\definecolor{currentstroke}{rgb}{0.000000,0.000000,0.000000}%
\pgfsetstrokecolor{currentstroke}%
\pgfsetdash{}{0pt}%
\pgfsys@defobject{currentmarker}{\pgfqpoint{0.000000in}{-0.048611in}}{\pgfqpoint{0.000000in}{0.000000in}}{%
\pgfpathmoveto{\pgfqpoint{0.000000in}{0.000000in}}%
\pgfpathlineto{\pgfqpoint{0.000000in}{-0.048611in}}%
\pgfusepath{stroke,fill}%
}%
\begin{pgfscope}%
\pgfsys@transformshift{3.063419in}{1.172519in}%
\pgfsys@useobject{currentmarker}{}%
\end{pgfscope}%
\end{pgfscope}%
\begin{pgfscope}%
\definecolor{textcolor}{rgb}{0.000000,0.000000,0.000000}%
\pgfsetstrokecolor{textcolor}%
\pgfsetfillcolor{textcolor}%
\pgftext[x=3.084252in, y=0.374225in, left, base,rotate=90.000000]{\color{textcolor}\rmfamily\fontsize{6.000000}{7.200000}\selectfont Euro-Assurance}%
\end{pgfscope}%
\begin{pgfscope}%
\pgfsetbuttcap%
\pgfsetroundjoin%
\definecolor{currentfill}{rgb}{0.000000,0.000000,0.000000}%
\pgfsetfillcolor{currentfill}%
\pgfsetlinewidth{0.803000pt}%
\definecolor{currentstroke}{rgb}{0.000000,0.000000,0.000000}%
\pgfsetstrokecolor{currentstroke}%
\pgfsetdash{}{0pt}%
\pgfsys@defobject{currentmarker}{\pgfqpoint{0.000000in}{-0.048611in}}{\pgfqpoint{0.000000in}{0.000000in}}{%
\pgfpathmoveto{\pgfqpoint{0.000000in}{0.000000in}}%
\pgfpathlineto{\pgfqpoint{0.000000in}{-0.048611in}}%
\pgfusepath{stroke,fill}%
}%
\begin{pgfscope}%
\pgfsys@transformshift{3.194892in}{1.172519in}%
\pgfsys@useobject{currentmarker}{}%
\end{pgfscope}%
\end{pgfscope}%
\begin{pgfscope}%
\definecolor{textcolor}{rgb}{0.000000,0.000000,0.000000}%
\pgfsetstrokecolor{textcolor}%
\pgfsetfillcolor{textcolor}%
\pgftext[x=3.215725in, y=0.771751in, left, base,rotate=90.000000]{\color{textcolor}\rmfamily\fontsize{6.000000}{7.200000}\selectfont Eurofil}%
\end{pgfscope}%
\begin{pgfscope}%
\pgfsetbuttcap%
\pgfsetroundjoin%
\definecolor{currentfill}{rgb}{0.000000,0.000000,0.000000}%
\pgfsetfillcolor{currentfill}%
\pgfsetlinewidth{0.803000pt}%
\definecolor{currentstroke}{rgb}{0.000000,0.000000,0.000000}%
\pgfsetstrokecolor{currentstroke}%
\pgfsetdash{}{0pt}%
\pgfsys@defobject{currentmarker}{\pgfqpoint{0.000000in}{-0.048611in}}{\pgfqpoint{0.000000in}{0.000000in}}{%
\pgfpathmoveto{\pgfqpoint{0.000000in}{0.000000in}}%
\pgfpathlineto{\pgfqpoint{0.000000in}{-0.048611in}}%
\pgfusepath{stroke,fill}%
}%
\begin{pgfscope}%
\pgfsys@transformshift{3.326365in}{1.172519in}%
\pgfsys@useobject{currentmarker}{}%
\end{pgfscope}%
\end{pgfscope}%
\begin{pgfscope}%
\definecolor{textcolor}{rgb}{0.000000,0.000000,0.000000}%
\pgfsetstrokecolor{textcolor}%
\pgfsetfillcolor{textcolor}%
\pgftext[x=3.347199in, y=0.840422in, left, base,rotate=90.000000]{\color{textcolor}\rmfamily\fontsize{6.000000}{7.200000}\selectfont GMF}%
\end{pgfscope}%
\begin{pgfscope}%
\pgfsetbuttcap%
\pgfsetroundjoin%
\definecolor{currentfill}{rgb}{0.000000,0.000000,0.000000}%
\pgfsetfillcolor{currentfill}%
\pgfsetlinewidth{0.803000pt}%
\definecolor{currentstroke}{rgb}{0.000000,0.000000,0.000000}%
\pgfsetstrokecolor{currentstroke}%
\pgfsetdash{}{0pt}%
\pgfsys@defobject{currentmarker}{\pgfqpoint{0.000000in}{-0.048611in}}{\pgfqpoint{0.000000in}{0.000000in}}{%
\pgfpathmoveto{\pgfqpoint{0.000000in}{0.000000in}}%
\pgfpathlineto{\pgfqpoint{0.000000in}{-0.048611in}}%
\pgfusepath{stroke,fill}%
}%
\begin{pgfscope}%
\pgfsys@transformshift{3.457838in}{1.172519in}%
\pgfsys@useobject{currentmarker}{}%
\end{pgfscope}%
\end{pgfscope}%
\begin{pgfscope}%
\definecolor{textcolor}{rgb}{0.000000,0.000000,0.000000}%
\pgfsetstrokecolor{textcolor}%
\pgfsetfillcolor{textcolor}%
\pgftext[x=3.478672in, y=0.889573in, left, base,rotate=90.000000]{\color{textcolor}\rmfamily\fontsize{6.000000}{7.200000}\selectfont Gan}%
\end{pgfscope}%
\begin{pgfscope}%
\pgfsetbuttcap%
\pgfsetroundjoin%
\definecolor{currentfill}{rgb}{0.000000,0.000000,0.000000}%
\pgfsetfillcolor{currentfill}%
\pgfsetlinewidth{0.803000pt}%
\definecolor{currentstroke}{rgb}{0.000000,0.000000,0.000000}%
\pgfsetstrokecolor{currentstroke}%
\pgfsetdash{}{0pt}%
\pgfsys@defobject{currentmarker}{\pgfqpoint{0.000000in}{-0.048611in}}{\pgfqpoint{0.000000in}{0.000000in}}{%
\pgfpathmoveto{\pgfqpoint{0.000000in}{0.000000in}}%
\pgfpathlineto{\pgfqpoint{0.000000in}{-0.048611in}}%
\pgfusepath{stroke,fill}%
}%
\begin{pgfscope}%
\pgfsys@transformshift{3.589312in}{1.172519in}%
\pgfsys@useobject{currentmarker}{}%
\end{pgfscope}%
\end{pgfscope}%
\begin{pgfscope}%
\definecolor{textcolor}{rgb}{0.000000,0.000000,0.000000}%
\pgfsetstrokecolor{textcolor}%
\pgfsetfillcolor{textcolor}%
\pgftext[x=3.610145in, y=0.699761in, left, base,rotate=90.000000]{\color{textcolor}\rmfamily\fontsize{6.000000}{7.200000}\selectfont Generali}%
\end{pgfscope}%
\begin{pgfscope}%
\pgfsetbuttcap%
\pgfsetroundjoin%
\definecolor{currentfill}{rgb}{0.000000,0.000000,0.000000}%
\pgfsetfillcolor{currentfill}%
\pgfsetlinewidth{0.803000pt}%
\definecolor{currentstroke}{rgb}{0.000000,0.000000,0.000000}%
\pgfsetstrokecolor{currentstroke}%
\pgfsetdash{}{0pt}%
\pgfsys@defobject{currentmarker}{\pgfqpoint{0.000000in}{-0.048611in}}{\pgfqpoint{0.000000in}{0.000000in}}{%
\pgfpathmoveto{\pgfqpoint{0.000000in}{0.000000in}}%
\pgfpathlineto{\pgfqpoint{0.000000in}{-0.048611in}}%
\pgfusepath{stroke,fill}%
}%
\begin{pgfscope}%
\pgfsys@transformshift{3.720785in}{1.172519in}%
\pgfsys@useobject{currentmarker}{}%
\end{pgfscope}%
\end{pgfscope}%
\begin{pgfscope}%
\definecolor{textcolor}{rgb}{0.000000,0.000000,0.000000}%
\pgfsetstrokecolor{textcolor}%
\pgfsetfillcolor{textcolor}%
\pgftext[x=3.741618in, y=0.607941in, left, base,rotate=90.000000]{\color{textcolor}\rmfamily\fontsize{6.000000}{7.200000}\selectfont Groupama}%
\end{pgfscope}%
\begin{pgfscope}%
\pgfsetbuttcap%
\pgfsetroundjoin%
\definecolor{currentfill}{rgb}{0.000000,0.000000,0.000000}%
\pgfsetfillcolor{currentfill}%
\pgfsetlinewidth{0.803000pt}%
\definecolor{currentstroke}{rgb}{0.000000,0.000000,0.000000}%
\pgfsetstrokecolor{currentstroke}%
\pgfsetdash{}{0pt}%
\pgfsys@defobject{currentmarker}{\pgfqpoint{0.000000in}{-0.048611in}}{\pgfqpoint{0.000000in}{0.000000in}}{%
\pgfpathmoveto{\pgfqpoint{0.000000in}{0.000000in}}%
\pgfpathlineto{\pgfqpoint{0.000000in}{-0.048611in}}%
\pgfusepath{stroke,fill}%
}%
\begin{pgfscope}%
\pgfsys@transformshift{3.852258in}{1.172519in}%
\pgfsys@useobject{currentmarker}{}%
\end{pgfscope}%
\end{pgfscope}%
\begin{pgfscope}%
\definecolor{textcolor}{rgb}{0.000000,0.000000,0.000000}%
\pgfsetstrokecolor{textcolor}%
\pgfsetfillcolor{textcolor}%
\pgftext[x=3.873091in, y=0.581707in, left, base,rotate=90.000000]{\color{textcolor}\rmfamily\fontsize{6.000000}{7.200000}\selectfont Génération}%
\end{pgfscope}%
\begin{pgfscope}%
\pgfsetbuttcap%
\pgfsetroundjoin%
\definecolor{currentfill}{rgb}{0.000000,0.000000,0.000000}%
\pgfsetfillcolor{currentfill}%
\pgfsetlinewidth{0.803000pt}%
\definecolor{currentstroke}{rgb}{0.000000,0.000000,0.000000}%
\pgfsetstrokecolor{currentstroke}%
\pgfsetdash{}{0pt}%
\pgfsys@defobject{currentmarker}{\pgfqpoint{0.000000in}{-0.048611in}}{\pgfqpoint{0.000000in}{0.000000in}}{%
\pgfpathmoveto{\pgfqpoint{0.000000in}{0.000000in}}%
\pgfpathlineto{\pgfqpoint{0.000000in}{-0.048611in}}%
\pgfusepath{stroke,fill}%
}%
\begin{pgfscope}%
\pgfsys@transformshift{3.983731in}{1.172519in}%
\pgfsys@useobject{currentmarker}{}%
\end{pgfscope}%
\end{pgfscope}%
\begin{pgfscope}%
\definecolor{textcolor}{rgb}{0.000000,0.000000,0.000000}%
\pgfsetstrokecolor{textcolor}%
\pgfsetfillcolor{textcolor}%
\pgftext[x=4.004565in, y=0.216048in, left, base,rotate=90.000000]{\color{textcolor}\rmfamily\fontsize{6.000000}{7.200000}\selectfont Harmonie Mutuelle}%
\end{pgfscope}%
\begin{pgfscope}%
\pgfsetbuttcap%
\pgfsetroundjoin%
\definecolor{currentfill}{rgb}{0.000000,0.000000,0.000000}%
\pgfsetfillcolor{currentfill}%
\pgfsetlinewidth{0.803000pt}%
\definecolor{currentstroke}{rgb}{0.000000,0.000000,0.000000}%
\pgfsetstrokecolor{currentstroke}%
\pgfsetdash{}{0pt}%
\pgfsys@defobject{currentmarker}{\pgfqpoint{0.000000in}{-0.048611in}}{\pgfqpoint{0.000000in}{0.000000in}}{%
\pgfpathmoveto{\pgfqpoint{0.000000in}{0.000000in}}%
\pgfpathlineto{\pgfqpoint{0.000000in}{-0.048611in}}%
\pgfusepath{stroke,fill}%
}%
\begin{pgfscope}%
\pgfsys@transformshift{4.115205in}{1.172519in}%
\pgfsys@useobject{currentmarker}{}%
\end{pgfscope}%
\end{pgfscope}%
\begin{pgfscope}%
\definecolor{textcolor}{rgb}{0.000000,0.000000,0.000000}%
\pgfsetstrokecolor{textcolor}%
\pgfsetfillcolor{textcolor}%
\pgftext[x=4.136038in, y=0.783248in, left, base,rotate=90.000000]{\color{textcolor}\rmfamily\fontsize{6.000000}{7.200000}\selectfont Hiscox}%
\end{pgfscope}%
\begin{pgfscope}%
\pgfsetbuttcap%
\pgfsetroundjoin%
\definecolor{currentfill}{rgb}{0.000000,0.000000,0.000000}%
\pgfsetfillcolor{currentfill}%
\pgfsetlinewidth{0.803000pt}%
\definecolor{currentstroke}{rgb}{0.000000,0.000000,0.000000}%
\pgfsetstrokecolor{currentstroke}%
\pgfsetdash{}{0pt}%
\pgfsys@defobject{currentmarker}{\pgfqpoint{0.000000in}{-0.048611in}}{\pgfqpoint{0.000000in}{0.000000in}}{%
\pgfpathmoveto{\pgfqpoint{0.000000in}{0.000000in}}%
\pgfpathlineto{\pgfqpoint{0.000000in}{-0.048611in}}%
\pgfusepath{stroke,fill}%
}%
\begin{pgfscope}%
\pgfsys@transformshift{4.246678in}{1.172519in}%
\pgfsys@useobject{currentmarker}{}%
\end{pgfscope}%
\end{pgfscope}%
\begin{pgfscope}%
\definecolor{textcolor}{rgb}{0.000000,0.000000,0.000000}%
\pgfsetstrokecolor{textcolor}%
\pgfsetfillcolor{textcolor}%
\pgftext[x=4.267511in, y=0.703928in, left, base,rotate=90.000000]{\color{textcolor}\rmfamily\fontsize{6.000000}{7.200000}\selectfont Intériale}%
\end{pgfscope}%
\begin{pgfscope}%
\pgfsetbuttcap%
\pgfsetroundjoin%
\definecolor{currentfill}{rgb}{0.000000,0.000000,0.000000}%
\pgfsetfillcolor{currentfill}%
\pgfsetlinewidth{0.803000pt}%
\definecolor{currentstroke}{rgb}{0.000000,0.000000,0.000000}%
\pgfsetstrokecolor{currentstroke}%
\pgfsetdash{}{0pt}%
\pgfsys@defobject{currentmarker}{\pgfqpoint{0.000000in}{-0.048611in}}{\pgfqpoint{0.000000in}{0.000000in}}{%
\pgfpathmoveto{\pgfqpoint{0.000000in}{0.000000in}}%
\pgfpathlineto{\pgfqpoint{0.000000in}{-0.048611in}}%
\pgfusepath{stroke,fill}%
}%
\begin{pgfscope}%
\pgfsys@transformshift{4.378151in}{1.172519in}%
\pgfsys@useobject{currentmarker}{}%
\end{pgfscope}%
\end{pgfscope}%
\begin{pgfscope}%
\definecolor{textcolor}{rgb}{0.000000,0.000000,0.000000}%
\pgfsetstrokecolor{textcolor}%
\pgfsetfillcolor{textcolor}%
\pgftext[x=4.398984in, y=0.219520in, left, base,rotate=90.000000]{\color{textcolor}\rmfamily\fontsize{6.000000}{7.200000}\selectfont L'olivier Assurance}%
\end{pgfscope}%
\begin{pgfscope}%
\pgfsetbuttcap%
\pgfsetroundjoin%
\definecolor{currentfill}{rgb}{0.000000,0.000000,0.000000}%
\pgfsetfillcolor{currentfill}%
\pgfsetlinewidth{0.803000pt}%
\definecolor{currentstroke}{rgb}{0.000000,0.000000,0.000000}%
\pgfsetstrokecolor{currentstroke}%
\pgfsetdash{}{0pt}%
\pgfsys@defobject{currentmarker}{\pgfqpoint{0.000000in}{-0.048611in}}{\pgfqpoint{0.000000in}{0.000000in}}{%
\pgfpathmoveto{\pgfqpoint{0.000000in}{0.000000in}}%
\pgfpathlineto{\pgfqpoint{0.000000in}{-0.048611in}}%
\pgfusepath{stroke,fill}%
}%
\begin{pgfscope}%
\pgfsys@transformshift{4.509624in}{1.172519in}%
\pgfsys@useobject{currentmarker}{}%
\end{pgfscope}%
\end{pgfscope}%
\begin{pgfscope}%
\definecolor{textcolor}{rgb}{0.000000,0.000000,0.000000}%
\pgfsetstrokecolor{textcolor}%
\pgfsetfillcolor{textcolor}%
\pgftext[x=4.530457in, y=0.877150in, left, base,rotate=90.000000]{\color{textcolor}\rmfamily\fontsize{6.000000}{7.200000}\selectfont LCL}%
\end{pgfscope}%
\begin{pgfscope}%
\pgfsetbuttcap%
\pgfsetroundjoin%
\definecolor{currentfill}{rgb}{0.000000,0.000000,0.000000}%
\pgfsetfillcolor{currentfill}%
\pgfsetlinewidth{0.803000pt}%
\definecolor{currentstroke}{rgb}{0.000000,0.000000,0.000000}%
\pgfsetstrokecolor{currentstroke}%
\pgfsetdash{}{0pt}%
\pgfsys@defobject{currentmarker}{\pgfqpoint{0.000000in}{-0.048611in}}{\pgfqpoint{0.000000in}{0.000000in}}{%
\pgfpathmoveto{\pgfqpoint{0.000000in}{0.000000in}}%
\pgfpathlineto{\pgfqpoint{0.000000in}{-0.048611in}}%
\pgfusepath{stroke,fill}%
}%
\begin{pgfscope}%
\pgfsys@transformshift{4.641097in}{1.172519in}%
\pgfsys@useobject{currentmarker}{}%
\end{pgfscope}%
\end{pgfscope}%
\begin{pgfscope}%
\definecolor{textcolor}{rgb}{0.000000,0.000000,0.000000}%
\pgfsetstrokecolor{textcolor}%
\pgfsetfillcolor{textcolor}%
\pgftext[x=4.661931in, y=0.769513in, left, base,rotate=90.000000]{\color{textcolor}\rmfamily\fontsize{6.000000}{7.200000}\selectfont MAAF}%
\end{pgfscope}%
\begin{pgfscope}%
\pgfsetbuttcap%
\pgfsetroundjoin%
\definecolor{currentfill}{rgb}{0.000000,0.000000,0.000000}%
\pgfsetfillcolor{currentfill}%
\pgfsetlinewidth{0.803000pt}%
\definecolor{currentstroke}{rgb}{0.000000,0.000000,0.000000}%
\pgfsetstrokecolor{currentstroke}%
\pgfsetdash{}{0pt}%
\pgfsys@defobject{currentmarker}{\pgfqpoint{0.000000in}{-0.048611in}}{\pgfqpoint{0.000000in}{0.000000in}}{%
\pgfpathmoveto{\pgfqpoint{0.000000in}{0.000000in}}%
\pgfpathlineto{\pgfqpoint{0.000000in}{-0.048611in}}%
\pgfusepath{stroke,fill}%
}%
\begin{pgfscope}%
\pgfsys@transformshift{4.772571in}{1.172519in}%
\pgfsys@useobject{currentmarker}{}%
\end{pgfscope}%
\end{pgfscope}%
\begin{pgfscope}%
\definecolor{textcolor}{rgb}{0.000000,0.000000,0.000000}%
\pgfsetstrokecolor{textcolor}%
\pgfsetfillcolor{textcolor}%
\pgftext[x=4.793404in, y=0.737492in, left, base,rotate=90.000000]{\color{textcolor}\rmfamily\fontsize{6.000000}{7.200000}\selectfont MACIF}%
\end{pgfscope}%
\begin{pgfscope}%
\pgfsetbuttcap%
\pgfsetroundjoin%
\definecolor{currentfill}{rgb}{0.000000,0.000000,0.000000}%
\pgfsetfillcolor{currentfill}%
\pgfsetlinewidth{0.803000pt}%
\definecolor{currentstroke}{rgb}{0.000000,0.000000,0.000000}%
\pgfsetstrokecolor{currentstroke}%
\pgfsetdash{}{0pt}%
\pgfsys@defobject{currentmarker}{\pgfqpoint{0.000000in}{-0.048611in}}{\pgfqpoint{0.000000in}{0.000000in}}{%
\pgfpathmoveto{\pgfqpoint{0.000000in}{0.000000in}}%
\pgfpathlineto{\pgfqpoint{0.000000in}{-0.048611in}}%
\pgfusepath{stroke,fill}%
}%
\begin{pgfscope}%
\pgfsys@transformshift{4.904044in}{1.172519in}%
\pgfsys@useobject{currentmarker}{}%
\end{pgfscope}%
\end{pgfscope}%
\begin{pgfscope}%
\definecolor{textcolor}{rgb}{0.000000,0.000000,0.000000}%
\pgfsetstrokecolor{textcolor}%
\pgfsetfillcolor{textcolor}%
\pgftext[x=4.924877in, y=0.807321in, left, base,rotate=90.000000]{\color{textcolor}\rmfamily\fontsize{6.000000}{7.200000}\selectfont MAIF}%
\end{pgfscope}%
\begin{pgfscope}%
\pgfsetbuttcap%
\pgfsetroundjoin%
\definecolor{currentfill}{rgb}{0.000000,0.000000,0.000000}%
\pgfsetfillcolor{currentfill}%
\pgfsetlinewidth{0.803000pt}%
\definecolor{currentstroke}{rgb}{0.000000,0.000000,0.000000}%
\pgfsetstrokecolor{currentstroke}%
\pgfsetdash{}{0pt}%
\pgfsys@defobject{currentmarker}{\pgfqpoint{0.000000in}{-0.048611in}}{\pgfqpoint{0.000000in}{0.000000in}}{%
\pgfpathmoveto{\pgfqpoint{0.000000in}{0.000000in}}%
\pgfpathlineto{\pgfqpoint{0.000000in}{-0.048611in}}%
\pgfusepath{stroke,fill}%
}%
\begin{pgfscope}%
\pgfsys@transformshift{5.035517in}{1.172519in}%
\pgfsys@useobject{currentmarker}{}%
\end{pgfscope}%
\end{pgfscope}%
\begin{pgfscope}%
\definecolor{textcolor}{rgb}{0.000000,0.000000,0.000000}%
\pgfsetstrokecolor{textcolor}%
\pgfsetfillcolor{textcolor}%
\pgftext[x=5.056350in, y=0.837722in, left, base,rotate=90.000000]{\color{textcolor}\rmfamily\fontsize{6.000000}{7.200000}\selectfont MGP}%
\end{pgfscope}%
\begin{pgfscope}%
\pgfsetbuttcap%
\pgfsetroundjoin%
\definecolor{currentfill}{rgb}{0.000000,0.000000,0.000000}%
\pgfsetfillcolor{currentfill}%
\pgfsetlinewidth{0.803000pt}%
\definecolor{currentstroke}{rgb}{0.000000,0.000000,0.000000}%
\pgfsetstrokecolor{currentstroke}%
\pgfsetdash{}{0pt}%
\pgfsys@defobject{currentmarker}{\pgfqpoint{0.000000in}{-0.048611in}}{\pgfqpoint{0.000000in}{0.000000in}}{%
\pgfpathmoveto{\pgfqpoint{0.000000in}{0.000000in}}%
\pgfpathlineto{\pgfqpoint{0.000000in}{-0.048611in}}%
\pgfusepath{stroke,fill}%
}%
\begin{pgfscope}%
\pgfsys@transformshift{5.166990in}{1.172519in}%
\pgfsys@useobject{currentmarker}{}%
\end{pgfscope}%
\end{pgfscope}%
\begin{pgfscope}%
\definecolor{textcolor}{rgb}{0.000000,0.000000,0.000000}%
\pgfsetstrokecolor{textcolor}%
\pgfsetfillcolor{textcolor}%
\pgftext[x=5.187824in, y=0.818818in, left, base,rotate=90.000000]{\color{textcolor}\rmfamily\fontsize{6.000000}{7.200000}\selectfont MMA}%
\end{pgfscope}%
\begin{pgfscope}%
\pgfsetbuttcap%
\pgfsetroundjoin%
\definecolor{currentfill}{rgb}{0.000000,0.000000,0.000000}%
\pgfsetfillcolor{currentfill}%
\pgfsetlinewidth{0.803000pt}%
\definecolor{currentstroke}{rgb}{0.000000,0.000000,0.000000}%
\pgfsetstrokecolor{currentstroke}%
\pgfsetdash{}{0pt}%
\pgfsys@defobject{currentmarker}{\pgfqpoint{0.000000in}{-0.048611in}}{\pgfqpoint{0.000000in}{0.000000in}}{%
\pgfpathmoveto{\pgfqpoint{0.000000in}{0.000000in}}%
\pgfpathlineto{\pgfqpoint{0.000000in}{-0.048611in}}%
\pgfusepath{stroke,fill}%
}%
\begin{pgfscope}%
\pgfsys@transformshift{5.298463in}{1.172519in}%
\pgfsys@useobject{currentmarker}{}%
\end{pgfscope}%
\end{pgfscope}%
\begin{pgfscope}%
\definecolor{textcolor}{rgb}{0.000000,0.000000,0.000000}%
\pgfsetstrokecolor{textcolor}%
\pgfsetfillcolor{textcolor}%
\pgftext[x=5.319297in, y=0.665734in, left, base,rotate=90.000000]{\color{textcolor}\rmfamily\fontsize{6.000000}{7.200000}\selectfont Magnolia}%
\end{pgfscope}%
\begin{pgfscope}%
\pgfsetbuttcap%
\pgfsetroundjoin%
\definecolor{currentfill}{rgb}{0.000000,0.000000,0.000000}%
\pgfsetfillcolor{currentfill}%
\pgfsetlinewidth{0.803000pt}%
\definecolor{currentstroke}{rgb}{0.000000,0.000000,0.000000}%
\pgfsetstrokecolor{currentstroke}%
\pgfsetdash{}{0pt}%
\pgfsys@defobject{currentmarker}{\pgfqpoint{0.000000in}{-0.048611in}}{\pgfqpoint{0.000000in}{0.000000in}}{%
\pgfpathmoveto{\pgfqpoint{0.000000in}{0.000000in}}%
\pgfpathlineto{\pgfqpoint{0.000000in}{-0.048611in}}%
\pgfusepath{stroke,fill}%
}%
\begin{pgfscope}%
\pgfsys@transformshift{5.429937in}{1.172519in}%
\pgfsys@useobject{currentmarker}{}%
\end{pgfscope}%
\end{pgfscope}%
\begin{pgfscope}%
\definecolor{textcolor}{rgb}{0.000000,0.000000,0.000000}%
\pgfsetstrokecolor{textcolor}%
\pgfsetfillcolor{textcolor}%
\pgftext[x=5.450770in, y=0.263733in, left, base,rotate=90.000000]{\color{textcolor}\rmfamily\fontsize{6.000000}{7.200000}\selectfont Malakoff Humanis}%
\end{pgfscope}%
\begin{pgfscope}%
\pgfsetbuttcap%
\pgfsetroundjoin%
\definecolor{currentfill}{rgb}{0.000000,0.000000,0.000000}%
\pgfsetfillcolor{currentfill}%
\pgfsetlinewidth{0.803000pt}%
\definecolor{currentstroke}{rgb}{0.000000,0.000000,0.000000}%
\pgfsetstrokecolor{currentstroke}%
\pgfsetdash{}{0pt}%
\pgfsys@defobject{currentmarker}{\pgfqpoint{0.000000in}{-0.048611in}}{\pgfqpoint{0.000000in}{0.000000in}}{%
\pgfpathmoveto{\pgfqpoint{0.000000in}{0.000000in}}%
\pgfpathlineto{\pgfqpoint{0.000000in}{-0.048611in}}%
\pgfusepath{stroke,fill}%
}%
\begin{pgfscope}%
\pgfsys@transformshift{5.561410in}{1.172519in}%
\pgfsys@useobject{currentmarker}{}%
\end{pgfscope}%
\end{pgfscope}%
\begin{pgfscope}%
\definecolor{textcolor}{rgb}{0.000000,0.000000,0.000000}%
\pgfsetstrokecolor{textcolor}%
\pgfsetfillcolor{textcolor}%
\pgftext[x=5.582243in, y=0.826226in, left, base,rotate=90.000000]{\color{textcolor}\rmfamily\fontsize{6.000000}{7.200000}\selectfont Mapa}%
\end{pgfscope}%
\begin{pgfscope}%
\pgfsetbuttcap%
\pgfsetroundjoin%
\definecolor{currentfill}{rgb}{0.000000,0.000000,0.000000}%
\pgfsetfillcolor{currentfill}%
\pgfsetlinewidth{0.803000pt}%
\definecolor{currentstroke}{rgb}{0.000000,0.000000,0.000000}%
\pgfsetstrokecolor{currentstroke}%
\pgfsetdash{}{0pt}%
\pgfsys@defobject{currentmarker}{\pgfqpoint{0.000000in}{-0.048611in}}{\pgfqpoint{0.000000in}{0.000000in}}{%
\pgfpathmoveto{\pgfqpoint{0.000000in}{0.000000in}}%
\pgfpathlineto{\pgfqpoint{0.000000in}{-0.048611in}}%
\pgfusepath{stroke,fill}%
}%
\begin{pgfscope}%
\pgfsys@transformshift{5.692883in}{1.172519in}%
\pgfsys@useobject{currentmarker}{}%
\end{pgfscope}%
\end{pgfscope}%
\begin{pgfscope}%
\definecolor{textcolor}{rgb}{0.000000,0.000000,0.000000}%
\pgfsetstrokecolor{textcolor}%
\pgfsetfillcolor{textcolor}%
\pgftext[x=5.713716in, y=0.716273in, left, base,rotate=90.000000]{\color{textcolor}\rmfamily\fontsize{6.000000}{7.200000}\selectfont Matmut}%
\end{pgfscope}%
\begin{pgfscope}%
\pgfsetbuttcap%
\pgfsetroundjoin%
\definecolor{currentfill}{rgb}{0.000000,0.000000,0.000000}%
\pgfsetfillcolor{currentfill}%
\pgfsetlinewidth{0.803000pt}%
\definecolor{currentstroke}{rgb}{0.000000,0.000000,0.000000}%
\pgfsetstrokecolor{currentstroke}%
\pgfsetdash{}{0pt}%
\pgfsys@defobject{currentmarker}{\pgfqpoint{0.000000in}{-0.048611in}}{\pgfqpoint{0.000000in}{0.000000in}}{%
\pgfpathmoveto{\pgfqpoint{0.000000in}{0.000000in}}%
\pgfpathlineto{\pgfqpoint{0.000000in}{-0.048611in}}%
\pgfusepath{stroke,fill}%
}%
\begin{pgfscope}%
\pgfsys@transformshift{5.824356in}{1.172519in}%
\pgfsys@useobject{currentmarker}{}%
\end{pgfscope}%
\end{pgfscope}%
\begin{pgfscope}%
\definecolor{textcolor}{rgb}{0.000000,0.000000,0.000000}%
\pgfsetstrokecolor{textcolor}%
\pgfsetfillcolor{textcolor}%
\pgftext[x=5.845190in, y=0.767584in, left, base,rotate=90.000000]{\color{textcolor}\rmfamily\fontsize{6.000000}{7.200000}\selectfont Mercer}%
\end{pgfscope}%
\begin{pgfscope}%
\pgfsetbuttcap%
\pgfsetroundjoin%
\definecolor{currentfill}{rgb}{0.000000,0.000000,0.000000}%
\pgfsetfillcolor{currentfill}%
\pgfsetlinewidth{0.803000pt}%
\definecolor{currentstroke}{rgb}{0.000000,0.000000,0.000000}%
\pgfsetstrokecolor{currentstroke}%
\pgfsetdash{}{0pt}%
\pgfsys@defobject{currentmarker}{\pgfqpoint{0.000000in}{-0.048611in}}{\pgfqpoint{0.000000in}{0.000000in}}{%
\pgfpathmoveto{\pgfqpoint{0.000000in}{0.000000in}}%
\pgfpathlineto{\pgfqpoint{0.000000in}{-0.048611in}}%
\pgfusepath{stroke,fill}%
}%
\begin{pgfscope}%
\pgfsys@transformshift{5.955830in}{1.172519in}%
\pgfsys@useobject{currentmarker}{}%
\end{pgfscope}%
\end{pgfscope}%
\begin{pgfscope}%
\definecolor{textcolor}{rgb}{0.000000,0.000000,0.000000}%
\pgfsetstrokecolor{textcolor}%
\pgfsetfillcolor{textcolor}%
\pgftext[x=5.976663in, y=0.729081in, left, base,rotate=90.000000]{\color{textcolor}\rmfamily\fontsize{6.000000}{7.200000}\selectfont MetLife}%
\end{pgfscope}%
\begin{pgfscope}%
\pgfsetbuttcap%
\pgfsetroundjoin%
\definecolor{currentfill}{rgb}{0.000000,0.000000,0.000000}%
\pgfsetfillcolor{currentfill}%
\pgfsetlinewidth{0.803000pt}%
\definecolor{currentstroke}{rgb}{0.000000,0.000000,0.000000}%
\pgfsetstrokecolor{currentstroke}%
\pgfsetdash{}{0pt}%
\pgfsys@defobject{currentmarker}{\pgfqpoint{0.000000in}{-0.048611in}}{\pgfqpoint{0.000000in}{0.000000in}}{%
\pgfpathmoveto{\pgfqpoint{0.000000in}{0.000000in}}%
\pgfpathlineto{\pgfqpoint{0.000000in}{-0.048611in}}%
\pgfusepath{stroke,fill}%
}%
\begin{pgfscope}%
\pgfsys@transformshift{6.087303in}{1.172519in}%
\pgfsys@useobject{currentmarker}{}%
\end{pgfscope}%
\end{pgfscope}%
\begin{pgfscope}%
\definecolor{textcolor}{rgb}{0.000000,0.000000,0.000000}%
\pgfsetstrokecolor{textcolor}%
\pgfsetfillcolor{textcolor}%
\pgftext[x=6.108136in, y=0.831626in, left, base,rotate=90.000000]{\color{textcolor}\rmfamily\fontsize{6.000000}{7.200000}\selectfont Mgen}%
\end{pgfscope}%
\begin{pgfscope}%
\pgfsetbuttcap%
\pgfsetroundjoin%
\definecolor{currentfill}{rgb}{0.000000,0.000000,0.000000}%
\pgfsetfillcolor{currentfill}%
\pgfsetlinewidth{0.803000pt}%
\definecolor{currentstroke}{rgb}{0.000000,0.000000,0.000000}%
\pgfsetstrokecolor{currentstroke}%
\pgfsetdash{}{0pt}%
\pgfsys@defobject{currentmarker}{\pgfqpoint{0.000000in}{-0.048611in}}{\pgfqpoint{0.000000in}{0.000000in}}{%
\pgfpathmoveto{\pgfqpoint{0.000000in}{0.000000in}}%
\pgfpathlineto{\pgfqpoint{0.000000in}{-0.048611in}}%
\pgfusepath{stroke,fill}%
}%
\begin{pgfscope}%
\pgfsys@transformshift{6.218776in}{1.172519in}%
\pgfsys@useobject{currentmarker}{}%
\end{pgfscope}%
\end{pgfscope}%
\begin{pgfscope}%
\definecolor{textcolor}{rgb}{0.000000,0.000000,0.000000}%
\pgfsetstrokecolor{textcolor}%
\pgfsetfillcolor{textcolor}%
\pgftext[x=6.239609in, y=0.100000in, left, base,rotate=90.000000]{\color{textcolor}\rmfamily\fontsize{6.000000}{7.200000}\selectfont Mutuelle des Motards}%
\end{pgfscope}%
\begin{pgfscope}%
\pgfsetbuttcap%
\pgfsetroundjoin%
\definecolor{currentfill}{rgb}{0.000000,0.000000,0.000000}%
\pgfsetfillcolor{currentfill}%
\pgfsetlinewidth{0.803000pt}%
\definecolor{currentstroke}{rgb}{0.000000,0.000000,0.000000}%
\pgfsetstrokecolor{currentstroke}%
\pgfsetdash{}{0pt}%
\pgfsys@defobject{currentmarker}{\pgfqpoint{0.000000in}{-0.048611in}}{\pgfqpoint{0.000000in}{0.000000in}}{%
\pgfpathmoveto{\pgfqpoint{0.000000in}{0.000000in}}%
\pgfpathlineto{\pgfqpoint{0.000000in}{-0.048611in}}%
\pgfusepath{stroke,fill}%
}%
\begin{pgfscope}%
\pgfsys@transformshift{6.350249in}{1.172519in}%
\pgfsys@useobject{currentmarker}{}%
\end{pgfscope}%
\end{pgfscope}%
\begin{pgfscope}%
\definecolor{textcolor}{rgb}{0.000000,0.000000,0.000000}%
\pgfsetstrokecolor{textcolor}%
\pgfsetfillcolor{textcolor}%
\pgftext[x=6.371082in, y=0.411493in, left, base,rotate=90.000000]{\color{textcolor}\rmfamily\fontsize{6.000000}{7.200000}\selectfont Néoliane Santé}%
\end{pgfscope}%
\begin{pgfscope}%
\pgfsetbuttcap%
\pgfsetroundjoin%
\definecolor{currentfill}{rgb}{0.000000,0.000000,0.000000}%
\pgfsetfillcolor{currentfill}%
\pgfsetlinewidth{0.803000pt}%
\definecolor{currentstroke}{rgb}{0.000000,0.000000,0.000000}%
\pgfsetstrokecolor{currentstroke}%
\pgfsetdash{}{0pt}%
\pgfsys@defobject{currentmarker}{\pgfqpoint{0.000000in}{-0.048611in}}{\pgfqpoint{0.000000in}{0.000000in}}{%
\pgfpathmoveto{\pgfqpoint{0.000000in}{0.000000in}}%
\pgfpathlineto{\pgfqpoint{0.000000in}{-0.048611in}}%
\pgfusepath{stroke,fill}%
}%
\begin{pgfscope}%
\pgfsys@transformshift{6.481722in}{1.172519in}%
\pgfsys@useobject{currentmarker}{}%
\end{pgfscope}%
\end{pgfscope}%
\begin{pgfscope}%
\definecolor{textcolor}{rgb}{0.000000,0.000000,0.000000}%
\pgfsetstrokecolor{textcolor}%
\pgfsetfillcolor{textcolor}%
\pgftext[x=6.502556in, y=0.728928in, left, base,rotate=90.000000]{\color{textcolor}\rmfamily\fontsize{6.000000}{7.200000}\selectfont Pacifica}%
\end{pgfscope}%
\begin{pgfscope}%
\pgfsetbuttcap%
\pgfsetroundjoin%
\definecolor{currentfill}{rgb}{0.000000,0.000000,0.000000}%
\pgfsetfillcolor{currentfill}%
\pgfsetlinewidth{0.803000pt}%
\definecolor{currentstroke}{rgb}{0.000000,0.000000,0.000000}%
\pgfsetstrokecolor{currentstroke}%
\pgfsetdash{}{0pt}%
\pgfsys@defobject{currentmarker}{\pgfqpoint{0.000000in}{-0.048611in}}{\pgfqpoint{0.000000in}{0.000000in}}{%
\pgfpathmoveto{\pgfqpoint{0.000000in}{0.000000in}}%
\pgfpathlineto{\pgfqpoint{0.000000in}{-0.048611in}}%
\pgfusepath{stroke,fill}%
}%
\begin{pgfscope}%
\pgfsys@transformshift{6.613196in}{1.172519in}%
\pgfsys@useobject{currentmarker}{}%
\end{pgfscope}%
\end{pgfscope}%
\begin{pgfscope}%
\definecolor{textcolor}{rgb}{0.000000,0.000000,0.000000}%
\pgfsetstrokecolor{textcolor}%
\pgfsetfillcolor{textcolor}%
\pgftext[x=6.634029in, y=0.247915in, left, base,rotate=90.000000]{\color{textcolor}\rmfamily\fontsize{6.000000}{7.200000}\selectfont Peyrac Assurances}%
\end{pgfscope}%
\begin{pgfscope}%
\pgfsetbuttcap%
\pgfsetroundjoin%
\definecolor{currentfill}{rgb}{0.000000,0.000000,0.000000}%
\pgfsetfillcolor{currentfill}%
\pgfsetlinewidth{0.803000pt}%
\definecolor{currentstroke}{rgb}{0.000000,0.000000,0.000000}%
\pgfsetstrokecolor{currentstroke}%
\pgfsetdash{}{0pt}%
\pgfsys@defobject{currentmarker}{\pgfqpoint{0.000000in}{-0.048611in}}{\pgfqpoint{0.000000in}{0.000000in}}{%
\pgfpathmoveto{\pgfqpoint{0.000000in}{0.000000in}}%
\pgfpathlineto{\pgfqpoint{0.000000in}{-0.048611in}}%
\pgfusepath{stroke,fill}%
}%
\begin{pgfscope}%
\pgfsys@transformshift{6.744669in}{1.172519in}%
\pgfsys@useobject{currentmarker}{}%
\end{pgfscope}%
\end{pgfscope}%
\begin{pgfscope}%
\definecolor{textcolor}{rgb}{0.000000,0.000000,0.000000}%
\pgfsetstrokecolor{textcolor}%
\pgfsetfillcolor{textcolor}%
\pgftext[x=6.765502in, y=0.692199in, left, base,rotate=90.000000]{\color{textcolor}\rmfamily\fontsize{6.000000}{7.200000}\selectfont Santiane}%
\end{pgfscope}%
\begin{pgfscope}%
\pgfsetbuttcap%
\pgfsetroundjoin%
\definecolor{currentfill}{rgb}{0.000000,0.000000,0.000000}%
\pgfsetfillcolor{currentfill}%
\pgfsetlinewidth{0.803000pt}%
\definecolor{currentstroke}{rgb}{0.000000,0.000000,0.000000}%
\pgfsetstrokecolor{currentstroke}%
\pgfsetdash{}{0pt}%
\pgfsys@defobject{currentmarker}{\pgfqpoint{0.000000in}{-0.048611in}}{\pgfqpoint{0.000000in}{0.000000in}}{%
\pgfpathmoveto{\pgfqpoint{0.000000in}{0.000000in}}%
\pgfpathlineto{\pgfqpoint{0.000000in}{-0.048611in}}%
\pgfusepath{stroke,fill}%
}%
\begin{pgfscope}%
\pgfsys@transformshift{6.876142in}{1.172519in}%
\pgfsys@useobject{currentmarker}{}%
\end{pgfscope}%
\end{pgfscope}%
\begin{pgfscope}%
\definecolor{textcolor}{rgb}{0.000000,0.000000,0.000000}%
\pgfsetstrokecolor{textcolor}%
\pgfsetfillcolor{textcolor}%
\pgftext[x=6.896975in, y=0.676536in, left, base,rotate=90.000000]{\color{textcolor}\rmfamily\fontsize{6.000000}{7.200000}\selectfont SantéVet}%
\end{pgfscope}%
\begin{pgfscope}%
\pgfsetbuttcap%
\pgfsetroundjoin%
\definecolor{currentfill}{rgb}{0.000000,0.000000,0.000000}%
\pgfsetfillcolor{currentfill}%
\pgfsetlinewidth{0.803000pt}%
\definecolor{currentstroke}{rgb}{0.000000,0.000000,0.000000}%
\pgfsetstrokecolor{currentstroke}%
\pgfsetdash{}{0pt}%
\pgfsys@defobject{currentmarker}{\pgfqpoint{0.000000in}{-0.048611in}}{\pgfqpoint{0.000000in}{0.000000in}}{%
\pgfpathmoveto{\pgfqpoint{0.000000in}{0.000000in}}%
\pgfpathlineto{\pgfqpoint{0.000000in}{-0.048611in}}%
\pgfusepath{stroke,fill}%
}%
\begin{pgfscope}%
\pgfsys@transformshift{7.007615in}{1.172519in}%
\pgfsys@useobject{currentmarker}{}%
\end{pgfscope}%
\end{pgfscope}%
\begin{pgfscope}%
\definecolor{textcolor}{rgb}{0.000000,0.000000,0.000000}%
\pgfsetstrokecolor{textcolor}%
\pgfsetfillcolor{textcolor}%
\pgftext[x=7.028449in, y=0.884712in, left, base,rotate=90.000000]{\color{textcolor}\rmfamily\fontsize{6.000000}{7.200000}\selectfont Sma}%
\end{pgfscope}%
\begin{pgfscope}%
\pgfsetbuttcap%
\pgfsetroundjoin%
\definecolor{currentfill}{rgb}{0.000000,0.000000,0.000000}%
\pgfsetfillcolor{currentfill}%
\pgfsetlinewidth{0.803000pt}%
\definecolor{currentstroke}{rgb}{0.000000,0.000000,0.000000}%
\pgfsetstrokecolor{currentstroke}%
\pgfsetdash{}{0pt}%
\pgfsys@defobject{currentmarker}{\pgfqpoint{0.000000in}{-0.048611in}}{\pgfqpoint{0.000000in}{0.000000in}}{%
\pgfpathmoveto{\pgfqpoint{0.000000in}{0.000000in}}%
\pgfpathlineto{\pgfqpoint{0.000000in}{-0.048611in}}%
\pgfusepath{stroke,fill}%
}%
\begin{pgfscope}%
\pgfsys@transformshift{7.139088in}{1.172519in}%
\pgfsys@useobject{currentmarker}{}%
\end{pgfscope}%
\end{pgfscope}%
\begin{pgfscope}%
\definecolor{textcolor}{rgb}{0.000000,0.000000,0.000000}%
\pgfsetstrokecolor{textcolor}%
\pgfsetfillcolor{textcolor}%
\pgftext[x=7.159922in, y=0.718819in, left, base,rotate=90.000000]{\color{textcolor}\rmfamily\fontsize{6.000000}{7.200000}\selectfont Sogecap}%
\end{pgfscope}%
\begin{pgfscope}%
\pgfsetbuttcap%
\pgfsetroundjoin%
\definecolor{currentfill}{rgb}{0.000000,0.000000,0.000000}%
\pgfsetfillcolor{currentfill}%
\pgfsetlinewidth{0.803000pt}%
\definecolor{currentstroke}{rgb}{0.000000,0.000000,0.000000}%
\pgfsetstrokecolor{currentstroke}%
\pgfsetdash{}{0pt}%
\pgfsys@defobject{currentmarker}{\pgfqpoint{0.000000in}{-0.048611in}}{\pgfqpoint{0.000000in}{0.000000in}}{%
\pgfpathmoveto{\pgfqpoint{0.000000in}{0.000000in}}%
\pgfpathlineto{\pgfqpoint{0.000000in}{-0.048611in}}%
\pgfusepath{stroke,fill}%
}%
\begin{pgfscope}%
\pgfsys@transformshift{7.270562in}{1.172519in}%
\pgfsys@useobject{currentmarker}{}%
\end{pgfscope}%
\end{pgfscope}%
\begin{pgfscope}%
\definecolor{textcolor}{rgb}{0.000000,0.000000,0.000000}%
\pgfsetstrokecolor{textcolor}%
\pgfsetfillcolor{textcolor}%
\pgftext[x=7.291395in, y=0.693820in, left, base,rotate=90.000000]{\color{textcolor}\rmfamily\fontsize{6.000000}{7.200000}\selectfont Sogessur}%
\end{pgfscope}%
\begin{pgfscope}%
\pgfsetbuttcap%
\pgfsetroundjoin%
\definecolor{currentfill}{rgb}{0.000000,0.000000,0.000000}%
\pgfsetfillcolor{currentfill}%
\pgfsetlinewidth{0.803000pt}%
\definecolor{currentstroke}{rgb}{0.000000,0.000000,0.000000}%
\pgfsetstrokecolor{currentstroke}%
\pgfsetdash{}{0pt}%
\pgfsys@defobject{currentmarker}{\pgfqpoint{0.000000in}{-0.048611in}}{\pgfqpoint{0.000000in}{0.000000in}}{%
\pgfpathmoveto{\pgfqpoint{0.000000in}{0.000000in}}%
\pgfpathlineto{\pgfqpoint{0.000000in}{-0.048611in}}%
\pgfusepath{stroke,fill}%
}%
\begin{pgfscope}%
\pgfsys@transformshift{7.402035in}{1.172519in}%
\pgfsys@useobject{currentmarker}{}%
\end{pgfscope}%
\end{pgfscope}%
\begin{pgfscope}%
\definecolor{textcolor}{rgb}{0.000000,0.000000,0.000000}%
\pgfsetstrokecolor{textcolor}%
\pgfsetfillcolor{textcolor}%
\pgftext[x=7.422868in, y=0.609793in, left, base,rotate=90.000000]{\color{textcolor}\rmfamily\fontsize{6.000000}{7.200000}\selectfont Solly Azar}%
\end{pgfscope}%
\begin{pgfscope}%
\pgfsetbuttcap%
\pgfsetroundjoin%
\definecolor{currentfill}{rgb}{0.000000,0.000000,0.000000}%
\pgfsetfillcolor{currentfill}%
\pgfsetlinewidth{0.803000pt}%
\definecolor{currentstroke}{rgb}{0.000000,0.000000,0.000000}%
\pgfsetstrokecolor{currentstroke}%
\pgfsetdash{}{0pt}%
\pgfsys@defobject{currentmarker}{\pgfqpoint{0.000000in}{-0.048611in}}{\pgfqpoint{0.000000in}{0.000000in}}{%
\pgfpathmoveto{\pgfqpoint{0.000000in}{0.000000in}}%
\pgfpathlineto{\pgfqpoint{0.000000in}{-0.048611in}}%
\pgfusepath{stroke,fill}%
}%
\begin{pgfscope}%
\pgfsys@transformshift{7.533508in}{1.172519in}%
\pgfsys@useobject{currentmarker}{}%
\end{pgfscope}%
\end{pgfscope}%
\begin{pgfscope}%
\definecolor{textcolor}{rgb}{0.000000,0.000000,0.000000}%
\pgfsetstrokecolor{textcolor}%
\pgfsetfillcolor{textcolor}%
\pgftext[x=7.554341in, y=0.652076in, left, base,rotate=90.000000]{\color{textcolor}\rmfamily\fontsize{6.000000}{7.200000}\selectfont Suravenir}%
\end{pgfscope}%
\begin{pgfscope}%
\pgfsetbuttcap%
\pgfsetroundjoin%
\definecolor{currentfill}{rgb}{0.000000,0.000000,0.000000}%
\pgfsetfillcolor{currentfill}%
\pgfsetlinewidth{0.803000pt}%
\definecolor{currentstroke}{rgb}{0.000000,0.000000,0.000000}%
\pgfsetstrokecolor{currentstroke}%
\pgfsetdash{}{0pt}%
\pgfsys@defobject{currentmarker}{\pgfqpoint{0.000000in}{-0.048611in}}{\pgfqpoint{0.000000in}{0.000000in}}{%
\pgfpathmoveto{\pgfqpoint{0.000000in}{0.000000in}}%
\pgfpathlineto{\pgfqpoint{0.000000in}{-0.048611in}}%
\pgfusepath{stroke,fill}%
}%
\begin{pgfscope}%
\pgfsys@transformshift{7.664981in}{1.172519in}%
\pgfsys@useobject{currentmarker}{}%
\end{pgfscope}%
\end{pgfscope}%
\begin{pgfscope}%
\definecolor{textcolor}{rgb}{0.000000,0.000000,0.000000}%
\pgfsetstrokecolor{textcolor}%
\pgfsetfillcolor{textcolor}%
\pgftext[x=7.685815in, y=0.666119in, left, base,rotate=90.000000]{\color{textcolor}\rmfamily\fontsize{6.000000}{7.200000}\selectfont SwissLife}%
\end{pgfscope}%
\begin{pgfscope}%
\pgfsetbuttcap%
\pgfsetroundjoin%
\definecolor{currentfill}{rgb}{0.000000,0.000000,0.000000}%
\pgfsetfillcolor{currentfill}%
\pgfsetlinewidth{0.803000pt}%
\definecolor{currentstroke}{rgb}{0.000000,0.000000,0.000000}%
\pgfsetstrokecolor{currentstroke}%
\pgfsetdash{}{0pt}%
\pgfsys@defobject{currentmarker}{\pgfqpoint{0.000000in}{-0.048611in}}{\pgfqpoint{0.000000in}{0.000000in}}{%
\pgfpathmoveto{\pgfqpoint{0.000000in}{0.000000in}}%
\pgfpathlineto{\pgfqpoint{0.000000in}{-0.048611in}}%
\pgfusepath{stroke,fill}%
}%
\begin{pgfscope}%
\pgfsys@transformshift{7.796455in}{1.172519in}%
\pgfsys@useobject{currentmarker}{}%
\end{pgfscope}%
\end{pgfscope}%
\begin{pgfscope}%
\definecolor{textcolor}{rgb}{0.000000,0.000000,0.000000}%
\pgfsetstrokecolor{textcolor}%
\pgfsetfillcolor{textcolor}%
\pgftext[x=7.817288in, y=0.751380in, left, base,rotate=90.000000]{\color{textcolor}\rmfamily\fontsize{6.000000}{7.200000}\selectfont Zen'Up}%
\end{pgfscope}%
\begin{pgfscope}%
\pgfsetbuttcap%
\pgfsetroundjoin%
\definecolor{currentfill}{rgb}{0.000000,0.000000,0.000000}%
\pgfsetfillcolor{currentfill}%
\pgfsetlinewidth{0.803000pt}%
\definecolor{currentstroke}{rgb}{0.000000,0.000000,0.000000}%
\pgfsetstrokecolor{currentstroke}%
\pgfsetdash{}{0pt}%
\pgfsys@defobject{currentmarker}{\pgfqpoint{-0.048611in}{0.000000in}}{\pgfqpoint{-0.000000in}{0.000000in}}{%
\pgfpathmoveto{\pgfqpoint{-0.000000in}{0.000000in}}%
\pgfpathlineto{\pgfqpoint{-0.048611in}{0.000000in}}%
\pgfusepath{stroke,fill}%
}%
\begin{pgfscope}%
\pgfsys@transformshift{0.499691in}{1.172519in}%
\pgfsys@useobject{currentmarker}{}%
\end{pgfscope}%
\end{pgfscope}%
\begin{pgfscope}%
\definecolor{textcolor}{rgb}{0.000000,0.000000,0.000000}%
\pgfsetstrokecolor{textcolor}%
\pgfsetfillcolor{textcolor}%
\pgftext[x=0.375463in, y=1.151299in, left, base,rotate=90.000000]{\color{textcolor}\rmfamily\fontsize{10.000000}{12.000000}\selectfont \(\displaystyle {0}\)}%
\end{pgfscope}%
\begin{pgfscope}%
\pgfsetbuttcap%
\pgfsetroundjoin%
\definecolor{currentfill}{rgb}{0.000000,0.000000,0.000000}%
\pgfsetfillcolor{currentfill}%
\pgfsetlinewidth{0.803000pt}%
\definecolor{currentstroke}{rgb}{0.000000,0.000000,0.000000}%
\pgfsetstrokecolor{currentstroke}%
\pgfsetdash{}{0pt}%
\pgfsys@defobject{currentmarker}{\pgfqpoint{-0.048611in}{0.000000in}}{\pgfqpoint{-0.000000in}{0.000000in}}{%
\pgfpathmoveto{\pgfqpoint{-0.000000in}{0.000000in}}%
\pgfpathlineto{\pgfqpoint{-0.048611in}{0.000000in}}%
\pgfusepath{stroke,fill}%
}%
\begin{pgfscope}%
\pgfsys@transformshift{0.499691in}{1.751022in}%
\pgfsys@useobject{currentmarker}{}%
\end{pgfscope}%
\end{pgfscope}%
\begin{pgfscope}%
\definecolor{textcolor}{rgb}{0.000000,0.000000,0.000000}%
\pgfsetstrokecolor{textcolor}%
\pgfsetfillcolor{textcolor}%
\pgftext[x=0.375463in, y=1.729803in, left, base,rotate=90.000000]{\color{textcolor}\rmfamily\fontsize{10.000000}{12.000000}\selectfont \(\displaystyle {1}\)}%
\end{pgfscope}%
\begin{pgfscope}%
\pgfsetbuttcap%
\pgfsetroundjoin%
\definecolor{currentfill}{rgb}{0.000000,0.000000,0.000000}%
\pgfsetfillcolor{currentfill}%
\pgfsetlinewidth{0.803000pt}%
\definecolor{currentstroke}{rgb}{0.000000,0.000000,0.000000}%
\pgfsetstrokecolor{currentstroke}%
\pgfsetdash{}{0pt}%
\pgfsys@defobject{currentmarker}{\pgfqpoint{-0.048611in}{0.000000in}}{\pgfqpoint{-0.000000in}{0.000000in}}{%
\pgfpathmoveto{\pgfqpoint{-0.000000in}{0.000000in}}%
\pgfpathlineto{\pgfqpoint{-0.048611in}{0.000000in}}%
\pgfusepath{stroke,fill}%
}%
\begin{pgfscope}%
\pgfsys@transformshift{0.499691in}{2.329526in}%
\pgfsys@useobject{currentmarker}{}%
\end{pgfscope}%
\end{pgfscope}%
\begin{pgfscope}%
\definecolor{textcolor}{rgb}{0.000000,0.000000,0.000000}%
\pgfsetstrokecolor{textcolor}%
\pgfsetfillcolor{textcolor}%
\pgftext[x=0.375463in, y=2.308306in, left, base,rotate=90.000000]{\color{textcolor}\rmfamily\fontsize{10.000000}{12.000000}\selectfont \(\displaystyle {2}\)}%
\end{pgfscope}%
\begin{pgfscope}%
\pgfsetbuttcap%
\pgfsetroundjoin%
\definecolor{currentfill}{rgb}{0.000000,0.000000,0.000000}%
\pgfsetfillcolor{currentfill}%
\pgfsetlinewidth{0.803000pt}%
\definecolor{currentstroke}{rgb}{0.000000,0.000000,0.000000}%
\pgfsetstrokecolor{currentstroke}%
\pgfsetdash{}{0pt}%
\pgfsys@defobject{currentmarker}{\pgfqpoint{-0.048611in}{0.000000in}}{\pgfqpoint{-0.000000in}{0.000000in}}{%
\pgfpathmoveto{\pgfqpoint{-0.000000in}{0.000000in}}%
\pgfpathlineto{\pgfqpoint{-0.048611in}{0.000000in}}%
\pgfusepath{stroke,fill}%
}%
\begin{pgfscope}%
\pgfsys@transformshift{0.499691in}{2.908029in}%
\pgfsys@useobject{currentmarker}{}%
\end{pgfscope}%
\end{pgfscope}%
\begin{pgfscope}%
\definecolor{textcolor}{rgb}{0.000000,0.000000,0.000000}%
\pgfsetstrokecolor{textcolor}%
\pgfsetfillcolor{textcolor}%
\pgftext[x=0.375463in, y=2.886810in, left, base,rotate=90.000000]{\color{textcolor}\rmfamily\fontsize{10.000000}{12.000000}\selectfont \(\displaystyle {3}\)}%
\end{pgfscope}%
\begin{pgfscope}%
\pgfsetbuttcap%
\pgfsetroundjoin%
\definecolor{currentfill}{rgb}{0.000000,0.000000,0.000000}%
\pgfsetfillcolor{currentfill}%
\pgfsetlinewidth{0.803000pt}%
\definecolor{currentstroke}{rgb}{0.000000,0.000000,0.000000}%
\pgfsetstrokecolor{currentstroke}%
\pgfsetdash{}{0pt}%
\pgfsys@defobject{currentmarker}{\pgfqpoint{-0.048611in}{0.000000in}}{\pgfqpoint{-0.000000in}{0.000000in}}{%
\pgfpathmoveto{\pgfqpoint{-0.000000in}{0.000000in}}%
\pgfpathlineto{\pgfqpoint{-0.048611in}{0.000000in}}%
\pgfusepath{stroke,fill}%
}%
\begin{pgfscope}%
\pgfsys@transformshift{0.499691in}{3.486533in}%
\pgfsys@useobject{currentmarker}{}%
\end{pgfscope}%
\end{pgfscope}%
\begin{pgfscope}%
\definecolor{textcolor}{rgb}{0.000000,0.000000,0.000000}%
\pgfsetstrokecolor{textcolor}%
\pgfsetfillcolor{textcolor}%
\pgftext[x=0.375463in, y=3.465313in, left, base,rotate=90.000000]{\color{textcolor}\rmfamily\fontsize{10.000000}{12.000000}\selectfont \(\displaystyle {4}\)}%
\end{pgfscope}%
\begin{pgfscope}%
\definecolor{textcolor}{rgb}{0.000000,0.000000,0.000000}%
\pgfsetstrokecolor{textcolor}%
\pgfsetfillcolor{textcolor}%
\pgftext[x=0.223457in,y=2.520019in,,bottom,rotate=90.000000]{\color{textcolor}\rmfamily\fontsize{10.000000}{12.000000}\selectfont Mean note}%
\end{pgfscope}%
\begin{pgfscope}%
\pgfpathrectangle{\pgfqpoint{0.499691in}{1.172519in}}{\pgfqpoint{7.362500in}{2.695000in}}%
\pgfusepath{clip}%
\pgfsetrectcap%
\pgfsetroundjoin%
\pgfsetlinewidth{2.710125pt}%
\definecolor{currentstroke}{rgb}{0.260000,0.260000,0.260000}%
\pgfsetstrokecolor{currentstroke}%
\pgfsetdash{}{0pt}%
\pgfusepath{stroke}%
\end{pgfscope}%
\begin{pgfscope}%
\pgfpathrectangle{\pgfqpoint{0.499691in}{1.172519in}}{\pgfqpoint{7.362500in}{2.695000in}}%
\pgfusepath{clip}%
\pgfsetrectcap%
\pgfsetroundjoin%
\pgfsetlinewidth{2.710125pt}%
\definecolor{currentstroke}{rgb}{0.260000,0.260000,0.260000}%
\pgfsetstrokecolor{currentstroke}%
\pgfsetdash{}{0pt}%
\pgfusepath{stroke}%
\end{pgfscope}%
\begin{pgfscope}%
\pgfpathrectangle{\pgfqpoint{0.499691in}{1.172519in}}{\pgfqpoint{7.362500in}{2.695000in}}%
\pgfusepath{clip}%
\pgfsetrectcap%
\pgfsetroundjoin%
\pgfsetlinewidth{2.710125pt}%
\definecolor{currentstroke}{rgb}{0.260000,0.260000,0.260000}%
\pgfsetstrokecolor{currentstroke}%
\pgfsetdash{}{0pt}%
\pgfusepath{stroke}%
\end{pgfscope}%
\begin{pgfscope}%
\pgfpathrectangle{\pgfqpoint{0.499691in}{1.172519in}}{\pgfqpoint{7.362500in}{2.695000in}}%
\pgfusepath{clip}%
\pgfsetrectcap%
\pgfsetroundjoin%
\pgfsetlinewidth{2.710125pt}%
\definecolor{currentstroke}{rgb}{0.260000,0.260000,0.260000}%
\pgfsetstrokecolor{currentstroke}%
\pgfsetdash{}{0pt}%
\pgfusepath{stroke}%
\end{pgfscope}%
\begin{pgfscope}%
\pgfpathrectangle{\pgfqpoint{0.499691in}{1.172519in}}{\pgfqpoint{7.362500in}{2.695000in}}%
\pgfusepath{clip}%
\pgfsetrectcap%
\pgfsetroundjoin%
\pgfsetlinewidth{2.710125pt}%
\definecolor{currentstroke}{rgb}{0.260000,0.260000,0.260000}%
\pgfsetstrokecolor{currentstroke}%
\pgfsetdash{}{0pt}%
\pgfusepath{stroke}%
\end{pgfscope}%
\begin{pgfscope}%
\pgfpathrectangle{\pgfqpoint{0.499691in}{1.172519in}}{\pgfqpoint{7.362500in}{2.695000in}}%
\pgfusepath{clip}%
\pgfsetrectcap%
\pgfsetroundjoin%
\pgfsetlinewidth{2.710125pt}%
\definecolor{currentstroke}{rgb}{0.260000,0.260000,0.260000}%
\pgfsetstrokecolor{currentstroke}%
\pgfsetdash{}{0pt}%
\pgfusepath{stroke}%
\end{pgfscope}%
\begin{pgfscope}%
\pgfpathrectangle{\pgfqpoint{0.499691in}{1.172519in}}{\pgfqpoint{7.362500in}{2.695000in}}%
\pgfusepath{clip}%
\pgfsetrectcap%
\pgfsetroundjoin%
\pgfsetlinewidth{2.710125pt}%
\definecolor{currentstroke}{rgb}{0.260000,0.260000,0.260000}%
\pgfsetstrokecolor{currentstroke}%
\pgfsetdash{}{0pt}%
\pgfusepath{stroke}%
\end{pgfscope}%
\begin{pgfscope}%
\pgfpathrectangle{\pgfqpoint{0.499691in}{1.172519in}}{\pgfqpoint{7.362500in}{2.695000in}}%
\pgfusepath{clip}%
\pgfsetrectcap%
\pgfsetroundjoin%
\pgfsetlinewidth{2.710125pt}%
\definecolor{currentstroke}{rgb}{0.260000,0.260000,0.260000}%
\pgfsetstrokecolor{currentstroke}%
\pgfsetdash{}{0pt}%
\pgfusepath{stroke}%
\end{pgfscope}%
\begin{pgfscope}%
\pgfpathrectangle{\pgfqpoint{0.499691in}{1.172519in}}{\pgfqpoint{7.362500in}{2.695000in}}%
\pgfusepath{clip}%
\pgfsetrectcap%
\pgfsetroundjoin%
\pgfsetlinewidth{2.710125pt}%
\definecolor{currentstroke}{rgb}{0.260000,0.260000,0.260000}%
\pgfsetstrokecolor{currentstroke}%
\pgfsetdash{}{0pt}%
\pgfusepath{stroke}%
\end{pgfscope}%
\begin{pgfscope}%
\pgfpathrectangle{\pgfqpoint{0.499691in}{1.172519in}}{\pgfqpoint{7.362500in}{2.695000in}}%
\pgfusepath{clip}%
\pgfsetrectcap%
\pgfsetroundjoin%
\pgfsetlinewidth{2.710125pt}%
\definecolor{currentstroke}{rgb}{0.260000,0.260000,0.260000}%
\pgfsetstrokecolor{currentstroke}%
\pgfsetdash{}{0pt}%
\pgfusepath{stroke}%
\end{pgfscope}%
\begin{pgfscope}%
\pgfpathrectangle{\pgfqpoint{0.499691in}{1.172519in}}{\pgfqpoint{7.362500in}{2.695000in}}%
\pgfusepath{clip}%
\pgfsetrectcap%
\pgfsetroundjoin%
\pgfsetlinewidth{2.710125pt}%
\definecolor{currentstroke}{rgb}{0.260000,0.260000,0.260000}%
\pgfsetstrokecolor{currentstroke}%
\pgfsetdash{}{0pt}%
\pgfusepath{stroke}%
\end{pgfscope}%
\begin{pgfscope}%
\pgfpathrectangle{\pgfqpoint{0.499691in}{1.172519in}}{\pgfqpoint{7.362500in}{2.695000in}}%
\pgfusepath{clip}%
\pgfsetrectcap%
\pgfsetroundjoin%
\pgfsetlinewidth{2.710125pt}%
\definecolor{currentstroke}{rgb}{0.260000,0.260000,0.260000}%
\pgfsetstrokecolor{currentstroke}%
\pgfsetdash{}{0pt}%
\pgfusepath{stroke}%
\end{pgfscope}%
\begin{pgfscope}%
\pgfpathrectangle{\pgfqpoint{0.499691in}{1.172519in}}{\pgfqpoint{7.362500in}{2.695000in}}%
\pgfusepath{clip}%
\pgfsetrectcap%
\pgfsetroundjoin%
\pgfsetlinewidth{2.710125pt}%
\definecolor{currentstroke}{rgb}{0.260000,0.260000,0.260000}%
\pgfsetstrokecolor{currentstroke}%
\pgfsetdash{}{0pt}%
\pgfusepath{stroke}%
\end{pgfscope}%
\begin{pgfscope}%
\pgfpathrectangle{\pgfqpoint{0.499691in}{1.172519in}}{\pgfqpoint{7.362500in}{2.695000in}}%
\pgfusepath{clip}%
\pgfsetrectcap%
\pgfsetroundjoin%
\pgfsetlinewidth{2.710125pt}%
\definecolor{currentstroke}{rgb}{0.260000,0.260000,0.260000}%
\pgfsetstrokecolor{currentstroke}%
\pgfsetdash{}{0pt}%
\pgfusepath{stroke}%
\end{pgfscope}%
\begin{pgfscope}%
\pgfpathrectangle{\pgfqpoint{0.499691in}{1.172519in}}{\pgfqpoint{7.362500in}{2.695000in}}%
\pgfusepath{clip}%
\pgfsetrectcap%
\pgfsetroundjoin%
\pgfsetlinewidth{2.710125pt}%
\definecolor{currentstroke}{rgb}{0.260000,0.260000,0.260000}%
\pgfsetstrokecolor{currentstroke}%
\pgfsetdash{}{0pt}%
\pgfusepath{stroke}%
\end{pgfscope}%
\begin{pgfscope}%
\pgfpathrectangle{\pgfqpoint{0.499691in}{1.172519in}}{\pgfqpoint{7.362500in}{2.695000in}}%
\pgfusepath{clip}%
\pgfsetrectcap%
\pgfsetroundjoin%
\pgfsetlinewidth{2.710125pt}%
\definecolor{currentstroke}{rgb}{0.260000,0.260000,0.260000}%
\pgfsetstrokecolor{currentstroke}%
\pgfsetdash{}{0pt}%
\pgfusepath{stroke}%
\end{pgfscope}%
\begin{pgfscope}%
\pgfpathrectangle{\pgfqpoint{0.499691in}{1.172519in}}{\pgfqpoint{7.362500in}{2.695000in}}%
\pgfusepath{clip}%
\pgfsetrectcap%
\pgfsetroundjoin%
\pgfsetlinewidth{2.710125pt}%
\definecolor{currentstroke}{rgb}{0.260000,0.260000,0.260000}%
\pgfsetstrokecolor{currentstroke}%
\pgfsetdash{}{0pt}%
\pgfusepath{stroke}%
\end{pgfscope}%
\begin{pgfscope}%
\pgfpathrectangle{\pgfqpoint{0.499691in}{1.172519in}}{\pgfqpoint{7.362500in}{2.695000in}}%
\pgfusepath{clip}%
\pgfsetrectcap%
\pgfsetroundjoin%
\pgfsetlinewidth{2.710125pt}%
\definecolor{currentstroke}{rgb}{0.260000,0.260000,0.260000}%
\pgfsetstrokecolor{currentstroke}%
\pgfsetdash{}{0pt}%
\pgfusepath{stroke}%
\end{pgfscope}%
\begin{pgfscope}%
\pgfpathrectangle{\pgfqpoint{0.499691in}{1.172519in}}{\pgfqpoint{7.362500in}{2.695000in}}%
\pgfusepath{clip}%
\pgfsetrectcap%
\pgfsetroundjoin%
\pgfsetlinewidth{2.710125pt}%
\definecolor{currentstroke}{rgb}{0.260000,0.260000,0.260000}%
\pgfsetstrokecolor{currentstroke}%
\pgfsetdash{}{0pt}%
\pgfusepath{stroke}%
\end{pgfscope}%
\begin{pgfscope}%
\pgfpathrectangle{\pgfqpoint{0.499691in}{1.172519in}}{\pgfqpoint{7.362500in}{2.695000in}}%
\pgfusepath{clip}%
\pgfsetrectcap%
\pgfsetroundjoin%
\pgfsetlinewidth{2.710125pt}%
\definecolor{currentstroke}{rgb}{0.260000,0.260000,0.260000}%
\pgfsetstrokecolor{currentstroke}%
\pgfsetdash{}{0pt}%
\pgfusepath{stroke}%
\end{pgfscope}%
\begin{pgfscope}%
\pgfpathrectangle{\pgfqpoint{0.499691in}{1.172519in}}{\pgfqpoint{7.362500in}{2.695000in}}%
\pgfusepath{clip}%
\pgfsetrectcap%
\pgfsetroundjoin%
\pgfsetlinewidth{2.710125pt}%
\definecolor{currentstroke}{rgb}{0.260000,0.260000,0.260000}%
\pgfsetstrokecolor{currentstroke}%
\pgfsetdash{}{0pt}%
\pgfusepath{stroke}%
\end{pgfscope}%
\begin{pgfscope}%
\pgfpathrectangle{\pgfqpoint{0.499691in}{1.172519in}}{\pgfqpoint{7.362500in}{2.695000in}}%
\pgfusepath{clip}%
\pgfsetrectcap%
\pgfsetroundjoin%
\pgfsetlinewidth{2.710125pt}%
\definecolor{currentstroke}{rgb}{0.260000,0.260000,0.260000}%
\pgfsetstrokecolor{currentstroke}%
\pgfsetdash{}{0pt}%
\pgfusepath{stroke}%
\end{pgfscope}%
\begin{pgfscope}%
\pgfpathrectangle{\pgfqpoint{0.499691in}{1.172519in}}{\pgfqpoint{7.362500in}{2.695000in}}%
\pgfusepath{clip}%
\pgfsetrectcap%
\pgfsetroundjoin%
\pgfsetlinewidth{2.710125pt}%
\definecolor{currentstroke}{rgb}{0.260000,0.260000,0.260000}%
\pgfsetstrokecolor{currentstroke}%
\pgfsetdash{}{0pt}%
\pgfusepath{stroke}%
\end{pgfscope}%
\begin{pgfscope}%
\pgfpathrectangle{\pgfqpoint{0.499691in}{1.172519in}}{\pgfqpoint{7.362500in}{2.695000in}}%
\pgfusepath{clip}%
\pgfsetrectcap%
\pgfsetroundjoin%
\pgfsetlinewidth{2.710125pt}%
\definecolor{currentstroke}{rgb}{0.260000,0.260000,0.260000}%
\pgfsetstrokecolor{currentstroke}%
\pgfsetdash{}{0pt}%
\pgfusepath{stroke}%
\end{pgfscope}%
\begin{pgfscope}%
\pgfpathrectangle{\pgfqpoint{0.499691in}{1.172519in}}{\pgfqpoint{7.362500in}{2.695000in}}%
\pgfusepath{clip}%
\pgfsetrectcap%
\pgfsetroundjoin%
\pgfsetlinewidth{2.710125pt}%
\definecolor{currentstroke}{rgb}{0.260000,0.260000,0.260000}%
\pgfsetstrokecolor{currentstroke}%
\pgfsetdash{}{0pt}%
\pgfusepath{stroke}%
\end{pgfscope}%
\begin{pgfscope}%
\pgfpathrectangle{\pgfqpoint{0.499691in}{1.172519in}}{\pgfqpoint{7.362500in}{2.695000in}}%
\pgfusepath{clip}%
\pgfsetrectcap%
\pgfsetroundjoin%
\pgfsetlinewidth{2.710125pt}%
\definecolor{currentstroke}{rgb}{0.260000,0.260000,0.260000}%
\pgfsetstrokecolor{currentstroke}%
\pgfsetdash{}{0pt}%
\pgfusepath{stroke}%
\end{pgfscope}%
\begin{pgfscope}%
\pgfpathrectangle{\pgfqpoint{0.499691in}{1.172519in}}{\pgfqpoint{7.362500in}{2.695000in}}%
\pgfusepath{clip}%
\pgfsetrectcap%
\pgfsetroundjoin%
\pgfsetlinewidth{2.710125pt}%
\definecolor{currentstroke}{rgb}{0.260000,0.260000,0.260000}%
\pgfsetstrokecolor{currentstroke}%
\pgfsetdash{}{0pt}%
\pgfusepath{stroke}%
\end{pgfscope}%
\begin{pgfscope}%
\pgfpathrectangle{\pgfqpoint{0.499691in}{1.172519in}}{\pgfqpoint{7.362500in}{2.695000in}}%
\pgfusepath{clip}%
\pgfsetrectcap%
\pgfsetroundjoin%
\pgfsetlinewidth{2.710125pt}%
\definecolor{currentstroke}{rgb}{0.260000,0.260000,0.260000}%
\pgfsetstrokecolor{currentstroke}%
\pgfsetdash{}{0pt}%
\pgfusepath{stroke}%
\end{pgfscope}%
\begin{pgfscope}%
\pgfpathrectangle{\pgfqpoint{0.499691in}{1.172519in}}{\pgfqpoint{7.362500in}{2.695000in}}%
\pgfusepath{clip}%
\pgfsetrectcap%
\pgfsetroundjoin%
\pgfsetlinewidth{2.710125pt}%
\definecolor{currentstroke}{rgb}{0.260000,0.260000,0.260000}%
\pgfsetstrokecolor{currentstroke}%
\pgfsetdash{}{0pt}%
\pgfusepath{stroke}%
\end{pgfscope}%
\begin{pgfscope}%
\pgfpathrectangle{\pgfqpoint{0.499691in}{1.172519in}}{\pgfqpoint{7.362500in}{2.695000in}}%
\pgfusepath{clip}%
\pgfsetrectcap%
\pgfsetroundjoin%
\pgfsetlinewidth{2.710125pt}%
\definecolor{currentstroke}{rgb}{0.260000,0.260000,0.260000}%
\pgfsetstrokecolor{currentstroke}%
\pgfsetdash{}{0pt}%
\pgfusepath{stroke}%
\end{pgfscope}%
\begin{pgfscope}%
\pgfpathrectangle{\pgfqpoint{0.499691in}{1.172519in}}{\pgfqpoint{7.362500in}{2.695000in}}%
\pgfusepath{clip}%
\pgfsetrectcap%
\pgfsetroundjoin%
\pgfsetlinewidth{2.710125pt}%
\definecolor{currentstroke}{rgb}{0.260000,0.260000,0.260000}%
\pgfsetstrokecolor{currentstroke}%
\pgfsetdash{}{0pt}%
\pgfusepath{stroke}%
\end{pgfscope}%
\begin{pgfscope}%
\pgfpathrectangle{\pgfqpoint{0.499691in}{1.172519in}}{\pgfqpoint{7.362500in}{2.695000in}}%
\pgfusepath{clip}%
\pgfsetrectcap%
\pgfsetroundjoin%
\pgfsetlinewidth{2.710125pt}%
\definecolor{currentstroke}{rgb}{0.260000,0.260000,0.260000}%
\pgfsetstrokecolor{currentstroke}%
\pgfsetdash{}{0pt}%
\pgfusepath{stroke}%
\end{pgfscope}%
\begin{pgfscope}%
\pgfpathrectangle{\pgfqpoint{0.499691in}{1.172519in}}{\pgfqpoint{7.362500in}{2.695000in}}%
\pgfusepath{clip}%
\pgfsetrectcap%
\pgfsetroundjoin%
\pgfsetlinewidth{2.710125pt}%
\definecolor{currentstroke}{rgb}{0.260000,0.260000,0.260000}%
\pgfsetstrokecolor{currentstroke}%
\pgfsetdash{}{0pt}%
\pgfusepath{stroke}%
\end{pgfscope}%
\begin{pgfscope}%
\pgfpathrectangle{\pgfqpoint{0.499691in}{1.172519in}}{\pgfqpoint{7.362500in}{2.695000in}}%
\pgfusepath{clip}%
\pgfsetrectcap%
\pgfsetroundjoin%
\pgfsetlinewidth{2.710125pt}%
\definecolor{currentstroke}{rgb}{0.260000,0.260000,0.260000}%
\pgfsetstrokecolor{currentstroke}%
\pgfsetdash{}{0pt}%
\pgfusepath{stroke}%
\end{pgfscope}%
\begin{pgfscope}%
\pgfpathrectangle{\pgfqpoint{0.499691in}{1.172519in}}{\pgfqpoint{7.362500in}{2.695000in}}%
\pgfusepath{clip}%
\pgfsetrectcap%
\pgfsetroundjoin%
\pgfsetlinewidth{2.710125pt}%
\definecolor{currentstroke}{rgb}{0.260000,0.260000,0.260000}%
\pgfsetstrokecolor{currentstroke}%
\pgfsetdash{}{0pt}%
\pgfusepath{stroke}%
\end{pgfscope}%
\begin{pgfscope}%
\pgfpathrectangle{\pgfqpoint{0.499691in}{1.172519in}}{\pgfqpoint{7.362500in}{2.695000in}}%
\pgfusepath{clip}%
\pgfsetrectcap%
\pgfsetroundjoin%
\pgfsetlinewidth{2.710125pt}%
\definecolor{currentstroke}{rgb}{0.260000,0.260000,0.260000}%
\pgfsetstrokecolor{currentstroke}%
\pgfsetdash{}{0pt}%
\pgfusepath{stroke}%
\end{pgfscope}%
\begin{pgfscope}%
\pgfpathrectangle{\pgfqpoint{0.499691in}{1.172519in}}{\pgfqpoint{7.362500in}{2.695000in}}%
\pgfusepath{clip}%
\pgfsetrectcap%
\pgfsetroundjoin%
\pgfsetlinewidth{2.710125pt}%
\definecolor{currentstroke}{rgb}{0.260000,0.260000,0.260000}%
\pgfsetstrokecolor{currentstroke}%
\pgfsetdash{}{0pt}%
\pgfusepath{stroke}%
\end{pgfscope}%
\begin{pgfscope}%
\pgfpathrectangle{\pgfqpoint{0.499691in}{1.172519in}}{\pgfqpoint{7.362500in}{2.695000in}}%
\pgfusepath{clip}%
\pgfsetrectcap%
\pgfsetroundjoin%
\pgfsetlinewidth{2.710125pt}%
\definecolor{currentstroke}{rgb}{0.260000,0.260000,0.260000}%
\pgfsetstrokecolor{currentstroke}%
\pgfsetdash{}{0pt}%
\pgfusepath{stroke}%
\end{pgfscope}%
\begin{pgfscope}%
\pgfpathrectangle{\pgfqpoint{0.499691in}{1.172519in}}{\pgfqpoint{7.362500in}{2.695000in}}%
\pgfusepath{clip}%
\pgfsetrectcap%
\pgfsetroundjoin%
\pgfsetlinewidth{2.710125pt}%
\definecolor{currentstroke}{rgb}{0.260000,0.260000,0.260000}%
\pgfsetstrokecolor{currentstroke}%
\pgfsetdash{}{0pt}%
\pgfusepath{stroke}%
\end{pgfscope}%
\begin{pgfscope}%
\pgfpathrectangle{\pgfqpoint{0.499691in}{1.172519in}}{\pgfqpoint{7.362500in}{2.695000in}}%
\pgfusepath{clip}%
\pgfsetrectcap%
\pgfsetroundjoin%
\pgfsetlinewidth{2.710125pt}%
\definecolor{currentstroke}{rgb}{0.260000,0.260000,0.260000}%
\pgfsetstrokecolor{currentstroke}%
\pgfsetdash{}{0pt}%
\pgfusepath{stroke}%
\end{pgfscope}%
\begin{pgfscope}%
\pgfpathrectangle{\pgfqpoint{0.499691in}{1.172519in}}{\pgfqpoint{7.362500in}{2.695000in}}%
\pgfusepath{clip}%
\pgfsetrectcap%
\pgfsetroundjoin%
\pgfsetlinewidth{2.710125pt}%
\definecolor{currentstroke}{rgb}{0.260000,0.260000,0.260000}%
\pgfsetstrokecolor{currentstroke}%
\pgfsetdash{}{0pt}%
\pgfusepath{stroke}%
\end{pgfscope}%
\begin{pgfscope}%
\pgfpathrectangle{\pgfqpoint{0.499691in}{1.172519in}}{\pgfqpoint{7.362500in}{2.695000in}}%
\pgfusepath{clip}%
\pgfsetrectcap%
\pgfsetroundjoin%
\pgfsetlinewidth{2.710125pt}%
\definecolor{currentstroke}{rgb}{0.260000,0.260000,0.260000}%
\pgfsetstrokecolor{currentstroke}%
\pgfsetdash{}{0pt}%
\pgfusepath{stroke}%
\end{pgfscope}%
\begin{pgfscope}%
\pgfpathrectangle{\pgfqpoint{0.499691in}{1.172519in}}{\pgfqpoint{7.362500in}{2.695000in}}%
\pgfusepath{clip}%
\pgfsetrectcap%
\pgfsetroundjoin%
\pgfsetlinewidth{2.710125pt}%
\definecolor{currentstroke}{rgb}{0.260000,0.260000,0.260000}%
\pgfsetstrokecolor{currentstroke}%
\pgfsetdash{}{0pt}%
\pgfusepath{stroke}%
\end{pgfscope}%
\begin{pgfscope}%
\pgfpathrectangle{\pgfqpoint{0.499691in}{1.172519in}}{\pgfqpoint{7.362500in}{2.695000in}}%
\pgfusepath{clip}%
\pgfsetrectcap%
\pgfsetroundjoin%
\pgfsetlinewidth{2.710125pt}%
\definecolor{currentstroke}{rgb}{0.260000,0.260000,0.260000}%
\pgfsetstrokecolor{currentstroke}%
\pgfsetdash{}{0pt}%
\pgfusepath{stroke}%
\end{pgfscope}%
\begin{pgfscope}%
\pgfpathrectangle{\pgfqpoint{0.499691in}{1.172519in}}{\pgfqpoint{7.362500in}{2.695000in}}%
\pgfusepath{clip}%
\pgfsetrectcap%
\pgfsetroundjoin%
\pgfsetlinewidth{2.710125pt}%
\definecolor{currentstroke}{rgb}{0.260000,0.260000,0.260000}%
\pgfsetstrokecolor{currentstroke}%
\pgfsetdash{}{0pt}%
\pgfusepath{stroke}%
\end{pgfscope}%
\begin{pgfscope}%
\pgfpathrectangle{\pgfqpoint{0.499691in}{1.172519in}}{\pgfqpoint{7.362500in}{2.695000in}}%
\pgfusepath{clip}%
\pgfsetrectcap%
\pgfsetroundjoin%
\pgfsetlinewidth{2.710125pt}%
\definecolor{currentstroke}{rgb}{0.260000,0.260000,0.260000}%
\pgfsetstrokecolor{currentstroke}%
\pgfsetdash{}{0pt}%
\pgfusepath{stroke}%
\end{pgfscope}%
\begin{pgfscope}%
\pgfpathrectangle{\pgfqpoint{0.499691in}{1.172519in}}{\pgfqpoint{7.362500in}{2.695000in}}%
\pgfusepath{clip}%
\pgfsetrectcap%
\pgfsetroundjoin%
\pgfsetlinewidth{2.710125pt}%
\definecolor{currentstroke}{rgb}{0.260000,0.260000,0.260000}%
\pgfsetstrokecolor{currentstroke}%
\pgfsetdash{}{0pt}%
\pgfusepath{stroke}%
\end{pgfscope}%
\begin{pgfscope}%
\pgfpathrectangle{\pgfqpoint{0.499691in}{1.172519in}}{\pgfqpoint{7.362500in}{2.695000in}}%
\pgfusepath{clip}%
\pgfsetrectcap%
\pgfsetroundjoin%
\pgfsetlinewidth{2.710125pt}%
\definecolor{currentstroke}{rgb}{0.260000,0.260000,0.260000}%
\pgfsetstrokecolor{currentstroke}%
\pgfsetdash{}{0pt}%
\pgfusepath{stroke}%
\end{pgfscope}%
\begin{pgfscope}%
\pgfpathrectangle{\pgfqpoint{0.499691in}{1.172519in}}{\pgfqpoint{7.362500in}{2.695000in}}%
\pgfusepath{clip}%
\pgfsetrectcap%
\pgfsetroundjoin%
\pgfsetlinewidth{2.710125pt}%
\definecolor{currentstroke}{rgb}{0.260000,0.260000,0.260000}%
\pgfsetstrokecolor{currentstroke}%
\pgfsetdash{}{0pt}%
\pgfusepath{stroke}%
\end{pgfscope}%
\begin{pgfscope}%
\pgfpathrectangle{\pgfqpoint{0.499691in}{1.172519in}}{\pgfqpoint{7.362500in}{2.695000in}}%
\pgfusepath{clip}%
\pgfsetrectcap%
\pgfsetroundjoin%
\pgfsetlinewidth{2.710125pt}%
\definecolor{currentstroke}{rgb}{0.260000,0.260000,0.260000}%
\pgfsetstrokecolor{currentstroke}%
\pgfsetdash{}{0pt}%
\pgfusepath{stroke}%
\end{pgfscope}%
\begin{pgfscope}%
\pgfpathrectangle{\pgfqpoint{0.499691in}{1.172519in}}{\pgfqpoint{7.362500in}{2.695000in}}%
\pgfusepath{clip}%
\pgfsetrectcap%
\pgfsetroundjoin%
\pgfsetlinewidth{2.710125pt}%
\definecolor{currentstroke}{rgb}{0.260000,0.260000,0.260000}%
\pgfsetstrokecolor{currentstroke}%
\pgfsetdash{}{0pt}%
\pgfusepath{stroke}%
\end{pgfscope}%
\begin{pgfscope}%
\pgfpathrectangle{\pgfqpoint{0.499691in}{1.172519in}}{\pgfqpoint{7.362500in}{2.695000in}}%
\pgfusepath{clip}%
\pgfsetrectcap%
\pgfsetroundjoin%
\pgfsetlinewidth{2.710125pt}%
\definecolor{currentstroke}{rgb}{0.260000,0.260000,0.260000}%
\pgfsetstrokecolor{currentstroke}%
\pgfsetdash{}{0pt}%
\pgfusepath{stroke}%
\end{pgfscope}%
\begin{pgfscope}%
\pgfpathrectangle{\pgfqpoint{0.499691in}{1.172519in}}{\pgfqpoint{7.362500in}{2.695000in}}%
\pgfusepath{clip}%
\pgfsetrectcap%
\pgfsetroundjoin%
\pgfsetlinewidth{2.710125pt}%
\definecolor{currentstroke}{rgb}{0.260000,0.260000,0.260000}%
\pgfsetstrokecolor{currentstroke}%
\pgfsetdash{}{0pt}%
\pgfusepath{stroke}%
\end{pgfscope}%
\begin{pgfscope}%
\pgfpathrectangle{\pgfqpoint{0.499691in}{1.172519in}}{\pgfqpoint{7.362500in}{2.695000in}}%
\pgfusepath{clip}%
\pgfsetrectcap%
\pgfsetroundjoin%
\pgfsetlinewidth{2.710125pt}%
\definecolor{currentstroke}{rgb}{0.260000,0.260000,0.260000}%
\pgfsetstrokecolor{currentstroke}%
\pgfsetdash{}{0pt}%
\pgfusepath{stroke}%
\end{pgfscope}%
\begin{pgfscope}%
\pgfpathrectangle{\pgfqpoint{0.499691in}{1.172519in}}{\pgfqpoint{7.362500in}{2.695000in}}%
\pgfusepath{clip}%
\pgfsetrectcap%
\pgfsetroundjoin%
\pgfsetlinewidth{2.710125pt}%
\definecolor{currentstroke}{rgb}{0.260000,0.260000,0.260000}%
\pgfsetstrokecolor{currentstroke}%
\pgfsetdash{}{0pt}%
\pgfusepath{stroke}%
\end{pgfscope}%
\begin{pgfscope}%
\pgfpathrectangle{\pgfqpoint{0.499691in}{1.172519in}}{\pgfqpoint{7.362500in}{2.695000in}}%
\pgfusepath{clip}%
\pgfsetrectcap%
\pgfsetroundjoin%
\pgfsetlinewidth{2.710125pt}%
\definecolor{currentstroke}{rgb}{0.260000,0.260000,0.260000}%
\pgfsetstrokecolor{currentstroke}%
\pgfsetdash{}{0pt}%
\pgfusepath{stroke}%
\end{pgfscope}%
\begin{pgfscope}%
\pgfsetrectcap%
\pgfsetmiterjoin%
\pgfsetlinewidth{0.803000pt}%
\definecolor{currentstroke}{rgb}{0.000000,0.000000,0.000000}%
\pgfsetstrokecolor{currentstroke}%
\pgfsetdash{}{0pt}%
\pgfpathmoveto{\pgfqpoint{0.499691in}{1.172519in}}%
\pgfpathlineto{\pgfqpoint{0.499691in}{3.867519in}}%
\pgfusepath{stroke}%
\end{pgfscope}%
\begin{pgfscope}%
\pgfsetrectcap%
\pgfsetmiterjoin%
\pgfsetlinewidth{0.803000pt}%
\definecolor{currentstroke}{rgb}{0.000000,0.000000,0.000000}%
\pgfsetstrokecolor{currentstroke}%
\pgfsetdash{}{0pt}%
\pgfpathmoveto{\pgfqpoint{7.862191in}{1.172519in}}%
\pgfpathlineto{\pgfqpoint{7.862191in}{3.867519in}}%
\pgfusepath{stroke}%
\end{pgfscope}%
\begin{pgfscope}%
\pgfsetrectcap%
\pgfsetmiterjoin%
\pgfsetlinewidth{0.803000pt}%
\definecolor{currentstroke}{rgb}{0.000000,0.000000,0.000000}%
\pgfsetstrokecolor{currentstroke}%
\pgfsetdash{}{0pt}%
\pgfpathmoveto{\pgfqpoint{0.499691in}{1.172519in}}%
\pgfpathlineto{\pgfqpoint{7.862191in}{1.172519in}}%
\pgfusepath{stroke}%
\end{pgfscope}%
\begin{pgfscope}%
\pgfsetrectcap%
\pgfsetmiterjoin%
\pgfsetlinewidth{0.803000pt}%
\definecolor{currentstroke}{rgb}{0.000000,0.000000,0.000000}%
\pgfsetstrokecolor{currentstroke}%
\pgfsetdash{}{0pt}%
\pgfpathmoveto{\pgfqpoint{0.499691in}{3.867519in}}%
\pgfpathlineto{\pgfqpoint{7.862191in}{3.867519in}}%
\pgfusepath{stroke}%
\end{pgfscope}%
\end{pgfpicture}%
\makeatother%
\endgroup%

    \caption{Mean note per assureur (colored by rank)}
    \label{fig:mean_note_per_assureur_linear}
\end{figure}
\restoregeometry

And so naturally we looked which assureurs were the most represented in the train dataset (\cref{fig:nbnote_per_assureur}).

\begin{figure}[H]
    \advance\leftskip-2.5cm
    %% Creator: Matplotlib, PGF backend
%%
%% To include the figure in your LaTeX document, write
%%   \input{<filename>.pgf}
%%
%% Make sure the required packages are loaded in your preamble
%%   \usepackage{pgf}
%%
%% Also ensure that all the required font packages are loaded; for instance,
%% the lmodern package is sometimes necessary when using math font.
%%   \usepackage{lmodern}
%%
%% Figures using additional raster images can only be included by \input if
%% they are in the same directory as the main LaTeX file. For loading figures
%% from other directories you can use the `import` package
%%   \usepackage{import}
%%
%% and then include the figures with
%%   \import{<path to file>}{<filename>.pgf}
%%
%% Matplotlib used the following preamble
%%
\begingroup%
\makeatletter%
\begin{pgfpicture}%
\pgfpathrectangle{\pgfpointorigin}{\pgfqpoint{7.962191in}{5.053099in}}%
\pgfusepath{use as bounding box, clip}%
\begin{pgfscope}%
\pgfsetbuttcap%
\pgfsetmiterjoin%
\definecolor{currentfill}{rgb}{1.000000,1.000000,1.000000}%
\pgfsetfillcolor{currentfill}%
\pgfsetlinewidth{0.000000pt}%
\definecolor{currentstroke}{rgb}{1.000000,1.000000,1.000000}%
\pgfsetstrokecolor{currentstroke}%
\pgfsetdash{}{0pt}%
\pgfpathmoveto{\pgfqpoint{0.000000in}{0.000000in}}%
\pgfpathlineto{\pgfqpoint{7.962191in}{0.000000in}}%
\pgfpathlineto{\pgfqpoint{7.962191in}{5.053099in}}%
\pgfpathlineto{\pgfqpoint{0.000000in}{5.053099in}}%
\pgfpathlineto{\pgfqpoint{0.000000in}{0.000000in}}%
\pgfpathclose%
\pgfusepath{fill}%
\end{pgfscope}%
\begin{pgfscope}%
\pgfsetbuttcap%
\pgfsetmiterjoin%
\definecolor{currentfill}{rgb}{1.000000,1.000000,1.000000}%
\pgfsetfillcolor{currentfill}%
\pgfsetlinewidth{0.000000pt}%
\definecolor{currentstroke}{rgb}{0.000000,0.000000,0.000000}%
\pgfsetstrokecolor{currentstroke}%
\pgfsetstrokeopacity{0.000000}%
\pgfsetdash{}{0pt}%
\pgfpathmoveto{\pgfqpoint{0.499691in}{1.103099in}}%
\pgfpathlineto{\pgfqpoint{7.862191in}{1.103099in}}%
\pgfpathlineto{\pgfqpoint{7.862191in}{4.953099in}}%
\pgfpathlineto{\pgfqpoint{0.499691in}{4.953099in}}%
\pgfpathlineto{\pgfqpoint{0.499691in}{1.103099in}}%
\pgfpathclose%
\pgfusepath{fill}%
\end{pgfscope}%
\begin{pgfscope}%
\pgfpathrectangle{\pgfqpoint{0.499691in}{1.103099in}}{\pgfqpoint{7.362500in}{3.850000in}}%
\pgfusepath{clip}%
\pgfsetbuttcap%
\pgfsetmiterjoin%
\definecolor{currentfill}{rgb}{0.914216,0.537745,0.399510}%
\pgfsetfillcolor{currentfill}%
\pgfsetlinewidth{0.000000pt}%
\definecolor{currentstroke}{rgb}{0.000000,0.000000,0.000000}%
\pgfsetstrokecolor{currentstroke}%
\pgfsetstrokeopacity{0.000000}%
\pgfsetdash{}{0pt}%
\pgfpathmoveto{\pgfqpoint{0.512838in}{1.103099in}}%
\pgfpathlineto{\pgfqpoint{0.618017in}{1.103099in}}%
\pgfpathlineto{\pgfqpoint{0.618017in}{1.530959in}}%
\pgfpathlineto{\pgfqpoint{0.512838in}{1.530959in}}%
\pgfpathlineto{\pgfqpoint{0.512838in}{1.103099in}}%
\pgfpathclose%
\pgfusepath{fill}%
\end{pgfscope}%
\begin{pgfscope}%
\pgfpathrectangle{\pgfqpoint{0.499691in}{1.103099in}}{\pgfqpoint{7.362500in}{3.850000in}}%
\pgfusepath{clip}%
\pgfsetbuttcap%
\pgfsetmiterjoin%
\definecolor{currentfill}{rgb}{0.914216,0.537745,0.399510}%
\pgfsetfillcolor{currentfill}%
\pgfsetlinewidth{0.000000pt}%
\definecolor{currentstroke}{rgb}{0.000000,0.000000,0.000000}%
\pgfsetstrokecolor{currentstroke}%
\pgfsetstrokeopacity{0.000000}%
\pgfsetdash{}{0pt}%
\pgfpathmoveto{\pgfqpoint{0.644312in}{1.103099in}}%
\pgfpathlineto{\pgfqpoint{0.749490in}{1.103099in}}%
\pgfpathlineto{\pgfqpoint{0.749490in}{1.279094in}}%
\pgfpathlineto{\pgfqpoint{0.644312in}{1.279094in}}%
\pgfpathlineto{\pgfqpoint{0.644312in}{1.103099in}}%
\pgfpathclose%
\pgfusepath{fill}%
\end{pgfscope}%
\begin{pgfscope}%
\pgfpathrectangle{\pgfqpoint{0.499691in}{1.103099in}}{\pgfqpoint{7.362500in}{3.850000in}}%
\pgfusepath{clip}%
\pgfsetbuttcap%
\pgfsetmiterjoin%
\definecolor{currentfill}{rgb}{0.914216,0.537745,0.399510}%
\pgfsetfillcolor{currentfill}%
\pgfsetlinewidth{0.000000pt}%
\definecolor{currentstroke}{rgb}{0.000000,0.000000,0.000000}%
\pgfsetstrokecolor{currentstroke}%
\pgfsetstrokeopacity{0.000000}%
\pgfsetdash{}{0pt}%
\pgfpathmoveto{\pgfqpoint{0.775785in}{1.103099in}}%
\pgfpathlineto{\pgfqpoint{0.880963in}{1.103099in}}%
\pgfpathlineto{\pgfqpoint{0.880963in}{1.739293in}}%
\pgfpathlineto{\pgfqpoint{0.775785in}{1.739293in}}%
\pgfpathlineto{\pgfqpoint{0.775785in}{1.103099in}}%
\pgfpathclose%
\pgfusepath{fill}%
\end{pgfscope}%
\begin{pgfscope}%
\pgfpathrectangle{\pgfqpoint{0.499691in}{1.103099in}}{\pgfqpoint{7.362500in}{3.850000in}}%
\pgfusepath{clip}%
\pgfsetbuttcap%
\pgfsetmiterjoin%
\definecolor{currentfill}{rgb}{0.914216,0.537745,0.399510}%
\pgfsetfillcolor{currentfill}%
\pgfsetlinewidth{0.000000pt}%
\definecolor{currentstroke}{rgb}{0.000000,0.000000,0.000000}%
\pgfsetstrokecolor{currentstroke}%
\pgfsetstrokeopacity{0.000000}%
\pgfsetdash{}{0pt}%
\pgfpathmoveto{\pgfqpoint{0.907258in}{1.103099in}}%
\pgfpathlineto{\pgfqpoint{1.012437in}{1.103099in}}%
\pgfpathlineto{\pgfqpoint{1.012437in}{1.494268in}}%
\pgfpathlineto{\pgfqpoint{0.907258in}{1.494268in}}%
\pgfpathlineto{\pgfqpoint{0.907258in}{1.103099in}}%
\pgfpathclose%
\pgfusepath{fill}%
\end{pgfscope}%
\begin{pgfscope}%
\pgfpathrectangle{\pgfqpoint{0.499691in}{1.103099in}}{\pgfqpoint{7.362500in}{3.850000in}}%
\pgfusepath{clip}%
\pgfsetbuttcap%
\pgfsetmiterjoin%
\definecolor{currentfill}{rgb}{0.914216,0.537745,0.399510}%
\pgfsetfillcolor{currentfill}%
\pgfsetlinewidth{0.000000pt}%
\definecolor{currentstroke}{rgb}{0.000000,0.000000,0.000000}%
\pgfsetstrokecolor{currentstroke}%
\pgfsetstrokeopacity{0.000000}%
\pgfsetdash{}{0pt}%
\pgfpathmoveto{\pgfqpoint{1.038731in}{1.103099in}}%
\pgfpathlineto{\pgfqpoint{1.143910in}{1.103099in}}%
\pgfpathlineto{\pgfqpoint{1.143910in}{1.353721in}}%
\pgfpathlineto{\pgfqpoint{1.038731in}{1.353721in}}%
\pgfpathlineto{\pgfqpoint{1.038731in}{1.103099in}}%
\pgfpathclose%
\pgfusepath{fill}%
\end{pgfscope}%
\begin{pgfscope}%
\pgfpathrectangle{\pgfqpoint{0.499691in}{1.103099in}}{\pgfqpoint{7.362500in}{3.850000in}}%
\pgfusepath{clip}%
\pgfsetbuttcap%
\pgfsetmiterjoin%
\definecolor{currentfill}{rgb}{0.914216,0.537745,0.399510}%
\pgfsetfillcolor{currentfill}%
\pgfsetlinewidth{0.000000pt}%
\definecolor{currentstroke}{rgb}{0.000000,0.000000,0.000000}%
\pgfsetstrokecolor{currentstroke}%
\pgfsetstrokeopacity{0.000000}%
\pgfsetdash{}{0pt}%
\pgfpathmoveto{\pgfqpoint{1.170205in}{1.103099in}}%
\pgfpathlineto{\pgfqpoint{1.275383in}{1.103099in}}%
\pgfpathlineto{\pgfqpoint{1.275383in}{1.192651in}}%
\pgfpathlineto{\pgfqpoint{1.170205in}{1.192651in}}%
\pgfpathlineto{\pgfqpoint{1.170205in}{1.103099in}}%
\pgfpathclose%
\pgfusepath{fill}%
\end{pgfscope}%
\begin{pgfscope}%
\pgfpathrectangle{\pgfqpoint{0.499691in}{1.103099in}}{\pgfqpoint{7.362500in}{3.850000in}}%
\pgfusepath{clip}%
\pgfsetbuttcap%
\pgfsetmiterjoin%
\definecolor{currentfill}{rgb}{0.914216,0.537745,0.399510}%
\pgfsetfillcolor{currentfill}%
\pgfsetlinewidth{0.000000pt}%
\definecolor{currentstroke}{rgb}{0.000000,0.000000,0.000000}%
\pgfsetstrokecolor{currentstroke}%
\pgfsetstrokeopacity{0.000000}%
\pgfsetdash{}{0pt}%
\pgfpathmoveto{\pgfqpoint{1.301678in}{1.103099in}}%
\pgfpathlineto{\pgfqpoint{1.406856in}{1.103099in}}%
\pgfpathlineto{\pgfqpoint{1.406856in}{1.119268in}}%
\pgfpathlineto{\pgfqpoint{1.301678in}{1.119268in}}%
\pgfpathlineto{\pgfqpoint{1.301678in}{1.103099in}}%
\pgfpathclose%
\pgfusepath{fill}%
\end{pgfscope}%
\begin{pgfscope}%
\pgfpathrectangle{\pgfqpoint{0.499691in}{1.103099in}}{\pgfqpoint{7.362500in}{3.850000in}}%
\pgfusepath{clip}%
\pgfsetbuttcap%
\pgfsetmiterjoin%
\definecolor{currentfill}{rgb}{0.914216,0.537745,0.399510}%
\pgfsetfillcolor{currentfill}%
\pgfsetlinewidth{0.000000pt}%
\definecolor{currentstroke}{rgb}{0.000000,0.000000,0.000000}%
\pgfsetstrokecolor{currentstroke}%
\pgfsetstrokeopacity{0.000000}%
\pgfsetdash{}{0pt}%
\pgfpathmoveto{\pgfqpoint{1.433151in}{1.103099in}}%
\pgfpathlineto{\pgfqpoint{1.538330in}{1.103099in}}%
\pgfpathlineto{\pgfqpoint{1.538330in}{1.320760in}}%
\pgfpathlineto{\pgfqpoint{1.433151in}{1.320760in}}%
\pgfpathlineto{\pgfqpoint{1.433151in}{1.103099in}}%
\pgfpathclose%
\pgfusepath{fill}%
\end{pgfscope}%
\begin{pgfscope}%
\pgfpathrectangle{\pgfqpoint{0.499691in}{1.103099in}}{\pgfqpoint{7.362500in}{3.850000in}}%
\pgfusepath{clip}%
\pgfsetbuttcap%
\pgfsetmiterjoin%
\definecolor{currentfill}{rgb}{0.914216,0.537745,0.399510}%
\pgfsetfillcolor{currentfill}%
\pgfsetlinewidth{0.000000pt}%
\definecolor{currentstroke}{rgb}{0.000000,0.000000,0.000000}%
\pgfsetstrokecolor{currentstroke}%
\pgfsetstrokeopacity{0.000000}%
\pgfsetdash{}{0pt}%
\pgfpathmoveto{\pgfqpoint{1.564624in}{1.103099in}}%
\pgfpathlineto{\pgfqpoint{1.669803in}{1.103099in}}%
\pgfpathlineto{\pgfqpoint{1.669803in}{1.460064in}}%
\pgfpathlineto{\pgfqpoint{1.564624in}{1.460064in}}%
\pgfpathlineto{\pgfqpoint{1.564624in}{1.103099in}}%
\pgfpathclose%
\pgfusepath{fill}%
\end{pgfscope}%
\begin{pgfscope}%
\pgfpathrectangle{\pgfqpoint{0.499691in}{1.103099in}}{\pgfqpoint{7.362500in}{3.850000in}}%
\pgfusepath{clip}%
\pgfsetbuttcap%
\pgfsetmiterjoin%
\definecolor{currentfill}{rgb}{0.914216,0.537745,0.399510}%
\pgfsetfillcolor{currentfill}%
\pgfsetlinewidth{0.000000pt}%
\definecolor{currentstroke}{rgb}{0.000000,0.000000,0.000000}%
\pgfsetstrokecolor{currentstroke}%
\pgfsetstrokeopacity{0.000000}%
\pgfsetdash{}{0pt}%
\pgfpathmoveto{\pgfqpoint{1.696097in}{1.103099in}}%
\pgfpathlineto{\pgfqpoint{1.801276in}{1.103099in}}%
\pgfpathlineto{\pgfqpoint{1.801276in}{1.134815in}}%
\pgfpathlineto{\pgfqpoint{1.696097in}{1.134815in}}%
\pgfpathlineto{\pgfqpoint{1.696097in}{1.103099in}}%
\pgfpathclose%
\pgfusepath{fill}%
\end{pgfscope}%
\begin{pgfscope}%
\pgfpathrectangle{\pgfqpoint{0.499691in}{1.103099in}}{\pgfqpoint{7.362500in}{3.850000in}}%
\pgfusepath{clip}%
\pgfsetbuttcap%
\pgfsetmiterjoin%
\definecolor{currentfill}{rgb}{0.914216,0.537745,0.399510}%
\pgfsetfillcolor{currentfill}%
\pgfsetlinewidth{0.000000pt}%
\definecolor{currentstroke}{rgb}{0.000000,0.000000,0.000000}%
\pgfsetstrokecolor{currentstroke}%
\pgfsetstrokeopacity{0.000000}%
\pgfsetdash{}{0pt}%
\pgfpathmoveto{\pgfqpoint{1.827571in}{1.103099in}}%
\pgfpathlineto{\pgfqpoint{1.932749in}{1.103099in}}%
\pgfpathlineto{\pgfqpoint{1.932749in}{1.171507in}}%
\pgfpathlineto{\pgfqpoint{1.827571in}{1.171507in}}%
\pgfpathlineto{\pgfqpoint{1.827571in}{1.103099in}}%
\pgfpathclose%
\pgfusepath{fill}%
\end{pgfscope}%
\begin{pgfscope}%
\pgfpathrectangle{\pgfqpoint{0.499691in}{1.103099in}}{\pgfqpoint{7.362500in}{3.850000in}}%
\pgfusepath{clip}%
\pgfsetbuttcap%
\pgfsetmiterjoin%
\definecolor{currentfill}{rgb}{0.914216,0.537745,0.399510}%
\pgfsetfillcolor{currentfill}%
\pgfsetlinewidth{0.000000pt}%
\definecolor{currentstroke}{rgb}{0.000000,0.000000,0.000000}%
\pgfsetstrokecolor{currentstroke}%
\pgfsetstrokeopacity{0.000000}%
\pgfsetdash{}{0pt}%
\pgfpathmoveto{\pgfqpoint{1.959044in}{1.103099in}}%
\pgfpathlineto{\pgfqpoint{2.064222in}{1.103099in}}%
\pgfpathlineto{\pgfqpoint{2.064222in}{1.118646in}}%
\pgfpathlineto{\pgfqpoint{1.959044in}{1.118646in}}%
\pgfpathlineto{\pgfqpoint{1.959044in}{1.103099in}}%
\pgfpathclose%
\pgfusepath{fill}%
\end{pgfscope}%
\begin{pgfscope}%
\pgfpathrectangle{\pgfqpoint{0.499691in}{1.103099in}}{\pgfqpoint{7.362500in}{3.850000in}}%
\pgfusepath{clip}%
\pgfsetbuttcap%
\pgfsetmiterjoin%
\definecolor{currentfill}{rgb}{0.914216,0.537745,0.399510}%
\pgfsetfillcolor{currentfill}%
\pgfsetlinewidth{0.000000pt}%
\definecolor{currentstroke}{rgb}{0.000000,0.000000,0.000000}%
\pgfsetstrokecolor{currentstroke}%
\pgfsetstrokeopacity{0.000000}%
\pgfsetdash{}{0pt}%
\pgfpathmoveto{\pgfqpoint{2.090517in}{1.103099in}}%
\pgfpathlineto{\pgfqpoint{2.195696in}{1.103099in}}%
\pgfpathlineto{\pgfqpoint{2.195696in}{1.194517in}}%
\pgfpathlineto{\pgfqpoint{2.090517in}{1.194517in}}%
\pgfpathlineto{\pgfqpoint{2.090517in}{1.103099in}}%
\pgfpathclose%
\pgfusepath{fill}%
\end{pgfscope}%
\begin{pgfscope}%
\pgfpathrectangle{\pgfqpoint{0.499691in}{1.103099in}}{\pgfqpoint{7.362500in}{3.850000in}}%
\pgfusepath{clip}%
\pgfsetbuttcap%
\pgfsetmiterjoin%
\definecolor{currentfill}{rgb}{0.914216,0.537745,0.399510}%
\pgfsetfillcolor{currentfill}%
\pgfsetlinewidth{0.000000pt}%
\definecolor{currentstroke}{rgb}{0.000000,0.000000,0.000000}%
\pgfsetstrokecolor{currentstroke}%
\pgfsetstrokeopacity{0.000000}%
\pgfsetdash{}{0pt}%
\pgfpathmoveto{\pgfqpoint{2.221990in}{1.103099in}}%
\pgfpathlineto{\pgfqpoint{2.327169in}{1.103099in}}%
\pgfpathlineto{\pgfqpoint{2.327169in}{1.122999in}}%
\pgfpathlineto{\pgfqpoint{2.221990in}{1.122999in}}%
\pgfpathlineto{\pgfqpoint{2.221990in}{1.103099in}}%
\pgfpathclose%
\pgfusepath{fill}%
\end{pgfscope}%
\begin{pgfscope}%
\pgfpathrectangle{\pgfqpoint{0.499691in}{1.103099in}}{\pgfqpoint{7.362500in}{3.850000in}}%
\pgfusepath{clip}%
\pgfsetbuttcap%
\pgfsetmiterjoin%
\definecolor{currentfill}{rgb}{0.914216,0.537745,0.399510}%
\pgfsetfillcolor{currentfill}%
\pgfsetlinewidth{0.000000pt}%
\definecolor{currentstroke}{rgb}{0.000000,0.000000,0.000000}%
\pgfsetstrokecolor{currentstroke}%
\pgfsetstrokeopacity{0.000000}%
\pgfsetdash{}{0pt}%
\pgfpathmoveto{\pgfqpoint{2.353463in}{1.103099in}}%
\pgfpathlineto{\pgfqpoint{2.458642in}{1.103099in}}%
\pgfpathlineto{\pgfqpoint{2.458642in}{1.267900in}}%
\pgfpathlineto{\pgfqpoint{2.353463in}{1.267900in}}%
\pgfpathlineto{\pgfqpoint{2.353463in}{1.103099in}}%
\pgfpathclose%
\pgfusepath{fill}%
\end{pgfscope}%
\begin{pgfscope}%
\pgfpathrectangle{\pgfqpoint{0.499691in}{1.103099in}}{\pgfqpoint{7.362500in}{3.850000in}}%
\pgfusepath{clip}%
\pgfsetbuttcap%
\pgfsetmiterjoin%
\definecolor{currentfill}{rgb}{0.914216,0.537745,0.399510}%
\pgfsetfillcolor{currentfill}%
\pgfsetlinewidth{0.000000pt}%
\definecolor{currentstroke}{rgb}{0.000000,0.000000,0.000000}%
\pgfsetstrokecolor{currentstroke}%
\pgfsetstrokeopacity{0.000000}%
\pgfsetdash{}{0pt}%
\pgfpathmoveto{\pgfqpoint{2.484937in}{1.103099in}}%
\pgfpathlineto{\pgfqpoint{2.590115in}{1.103099in}}%
\pgfpathlineto{\pgfqpoint{2.590115in}{1.236183in}}%
\pgfpathlineto{\pgfqpoint{2.484937in}{1.236183in}}%
\pgfpathlineto{\pgfqpoint{2.484937in}{1.103099in}}%
\pgfpathclose%
\pgfusepath{fill}%
\end{pgfscope}%
\begin{pgfscope}%
\pgfpathrectangle{\pgfqpoint{0.499691in}{1.103099in}}{\pgfqpoint{7.362500in}{3.850000in}}%
\pgfusepath{clip}%
\pgfsetbuttcap%
\pgfsetmiterjoin%
\definecolor{currentfill}{rgb}{0.914216,0.537745,0.399510}%
\pgfsetfillcolor{currentfill}%
\pgfsetlinewidth{0.000000pt}%
\definecolor{currentstroke}{rgb}{0.000000,0.000000,0.000000}%
\pgfsetstrokecolor{currentstroke}%
\pgfsetstrokeopacity{0.000000}%
\pgfsetdash{}{0pt}%
\pgfpathmoveto{\pgfqpoint{2.616410in}{1.103099in}}%
\pgfpathlineto{\pgfqpoint{2.721588in}{1.103099in}}%
\pgfpathlineto{\pgfqpoint{2.721588in}{1.193895in}}%
\pgfpathlineto{\pgfqpoint{2.616410in}{1.193895in}}%
\pgfpathlineto{\pgfqpoint{2.616410in}{1.103099in}}%
\pgfpathclose%
\pgfusepath{fill}%
\end{pgfscope}%
\begin{pgfscope}%
\pgfpathrectangle{\pgfqpoint{0.499691in}{1.103099in}}{\pgfqpoint{7.362500in}{3.850000in}}%
\pgfusepath{clip}%
\pgfsetbuttcap%
\pgfsetmiterjoin%
\definecolor{currentfill}{rgb}{0.914216,0.537745,0.399510}%
\pgfsetfillcolor{currentfill}%
\pgfsetlinewidth{0.000000pt}%
\definecolor{currentstroke}{rgb}{0.000000,0.000000,0.000000}%
\pgfsetstrokecolor{currentstroke}%
\pgfsetstrokeopacity{0.000000}%
\pgfsetdash{}{0pt}%
\pgfpathmoveto{\pgfqpoint{2.747883in}{1.103099in}}%
\pgfpathlineto{\pgfqpoint{2.853062in}{1.103099in}}%
\pgfpathlineto{\pgfqpoint{2.853062in}{4.769765in}}%
\pgfpathlineto{\pgfqpoint{2.747883in}{4.769765in}}%
\pgfpathlineto{\pgfqpoint{2.747883in}{1.103099in}}%
\pgfpathclose%
\pgfusepath{fill}%
\end{pgfscope}%
\begin{pgfscope}%
\pgfpathrectangle{\pgfqpoint{0.499691in}{1.103099in}}{\pgfqpoint{7.362500in}{3.850000in}}%
\pgfusepath{clip}%
\pgfsetbuttcap%
\pgfsetmiterjoin%
\definecolor{currentfill}{rgb}{0.914216,0.537745,0.399510}%
\pgfsetfillcolor{currentfill}%
\pgfsetlinewidth{0.000000pt}%
\definecolor{currentstroke}{rgb}{0.000000,0.000000,0.000000}%
\pgfsetstrokecolor{currentstroke}%
\pgfsetstrokeopacity{0.000000}%
\pgfsetdash{}{0pt}%
\pgfpathmoveto{\pgfqpoint{2.879356in}{1.103099in}}%
\pgfpathlineto{\pgfqpoint{2.984535in}{1.103099in}}%
\pgfpathlineto{\pgfqpoint{2.984535in}{1.185188in}}%
\pgfpathlineto{\pgfqpoint{2.879356in}{1.185188in}}%
\pgfpathlineto{\pgfqpoint{2.879356in}{1.103099in}}%
\pgfpathclose%
\pgfusepath{fill}%
\end{pgfscope}%
\begin{pgfscope}%
\pgfpathrectangle{\pgfqpoint{0.499691in}{1.103099in}}{\pgfqpoint{7.362500in}{3.850000in}}%
\pgfusepath{clip}%
\pgfsetbuttcap%
\pgfsetmiterjoin%
\definecolor{currentfill}{rgb}{0.914216,0.537745,0.399510}%
\pgfsetfillcolor{currentfill}%
\pgfsetlinewidth{0.000000pt}%
\definecolor{currentstroke}{rgb}{0.000000,0.000000,0.000000}%
\pgfsetstrokecolor{currentstroke}%
\pgfsetstrokeopacity{0.000000}%
\pgfsetdash{}{0pt}%
\pgfpathmoveto{\pgfqpoint{3.010830in}{1.103099in}}%
\pgfpathlineto{\pgfqpoint{3.116008in}{1.103099in}}%
\pgfpathlineto{\pgfqpoint{3.116008in}{1.129840in}}%
\pgfpathlineto{\pgfqpoint{3.010830in}{1.129840in}}%
\pgfpathlineto{\pgfqpoint{3.010830in}{1.103099in}}%
\pgfpathclose%
\pgfusepath{fill}%
\end{pgfscope}%
\begin{pgfscope}%
\pgfpathrectangle{\pgfqpoint{0.499691in}{1.103099in}}{\pgfqpoint{7.362500in}{3.850000in}}%
\pgfusepath{clip}%
\pgfsetbuttcap%
\pgfsetmiterjoin%
\definecolor{currentfill}{rgb}{0.914216,0.537745,0.399510}%
\pgfsetfillcolor{currentfill}%
\pgfsetlinewidth{0.000000pt}%
\definecolor{currentstroke}{rgb}{0.000000,0.000000,0.000000}%
\pgfsetstrokecolor{currentstroke}%
\pgfsetstrokeopacity{0.000000}%
\pgfsetdash{}{0pt}%
\pgfpathmoveto{\pgfqpoint{3.142303in}{1.103099in}}%
\pgfpathlineto{\pgfqpoint{3.247481in}{1.103099in}}%
\pgfpathlineto{\pgfqpoint{3.247481in}{1.282825in}}%
\pgfpathlineto{\pgfqpoint{3.142303in}{1.282825in}}%
\pgfpathlineto{\pgfqpoint{3.142303in}{1.103099in}}%
\pgfpathclose%
\pgfusepath{fill}%
\end{pgfscope}%
\begin{pgfscope}%
\pgfpathrectangle{\pgfqpoint{0.499691in}{1.103099in}}{\pgfqpoint{7.362500in}{3.850000in}}%
\pgfusepath{clip}%
\pgfsetbuttcap%
\pgfsetmiterjoin%
\definecolor{currentfill}{rgb}{0.914216,0.537745,0.399510}%
\pgfsetfillcolor{currentfill}%
\pgfsetlinewidth{0.000000pt}%
\definecolor{currentstroke}{rgb}{0.000000,0.000000,0.000000}%
\pgfsetstrokecolor{currentstroke}%
\pgfsetstrokeopacity{0.000000}%
\pgfsetdash{}{0pt}%
\pgfpathmoveto{\pgfqpoint{3.273776in}{1.103099in}}%
\pgfpathlineto{\pgfqpoint{3.378955in}{1.103099in}}%
\pgfpathlineto{\pgfqpoint{3.378955in}{1.723746in}}%
\pgfpathlineto{\pgfqpoint{3.273776in}{1.723746in}}%
\pgfpathlineto{\pgfqpoint{3.273776in}{1.103099in}}%
\pgfpathclose%
\pgfusepath{fill}%
\end{pgfscope}%
\begin{pgfscope}%
\pgfpathrectangle{\pgfqpoint{0.499691in}{1.103099in}}{\pgfqpoint{7.362500in}{3.850000in}}%
\pgfusepath{clip}%
\pgfsetbuttcap%
\pgfsetmiterjoin%
\definecolor{currentfill}{rgb}{0.914216,0.537745,0.399510}%
\pgfsetfillcolor{currentfill}%
\pgfsetlinewidth{0.000000pt}%
\definecolor{currentstroke}{rgb}{0.000000,0.000000,0.000000}%
\pgfsetstrokecolor{currentstroke}%
\pgfsetstrokeopacity{0.000000}%
\pgfsetdash{}{0pt}%
\pgfpathmoveto{\pgfqpoint{3.405249in}{1.103099in}}%
\pgfpathlineto{\pgfqpoint{3.510428in}{1.103099in}}%
\pgfpathlineto{\pgfqpoint{3.510428in}{1.123621in}}%
\pgfpathlineto{\pgfqpoint{3.405249in}{1.123621in}}%
\pgfpathlineto{\pgfqpoint{3.405249in}{1.103099in}}%
\pgfpathclose%
\pgfusepath{fill}%
\end{pgfscope}%
\begin{pgfscope}%
\pgfpathrectangle{\pgfqpoint{0.499691in}{1.103099in}}{\pgfqpoint{7.362500in}{3.850000in}}%
\pgfusepath{clip}%
\pgfsetbuttcap%
\pgfsetmiterjoin%
\definecolor{currentfill}{rgb}{0.914216,0.537745,0.399510}%
\pgfsetfillcolor{currentfill}%
\pgfsetlinewidth{0.000000pt}%
\definecolor{currentstroke}{rgb}{0.000000,0.000000,0.000000}%
\pgfsetstrokecolor{currentstroke}%
\pgfsetstrokeopacity{0.000000}%
\pgfsetdash{}{0pt}%
\pgfpathmoveto{\pgfqpoint{3.536722in}{1.103099in}}%
\pgfpathlineto{\pgfqpoint{3.641901in}{1.103099in}}%
\pgfpathlineto{\pgfqpoint{3.641901in}{1.182701in}}%
\pgfpathlineto{\pgfqpoint{3.536722in}{1.182701in}}%
\pgfpathlineto{\pgfqpoint{3.536722in}{1.103099in}}%
\pgfpathclose%
\pgfusepath{fill}%
\end{pgfscope}%
\begin{pgfscope}%
\pgfpathrectangle{\pgfqpoint{0.499691in}{1.103099in}}{\pgfqpoint{7.362500in}{3.850000in}}%
\pgfusepath{clip}%
\pgfsetbuttcap%
\pgfsetmiterjoin%
\definecolor{currentfill}{rgb}{0.914216,0.537745,0.399510}%
\pgfsetfillcolor{currentfill}%
\pgfsetlinewidth{0.000000pt}%
\definecolor{currentstroke}{rgb}{0.000000,0.000000,0.000000}%
\pgfsetstrokecolor{currentstroke}%
\pgfsetstrokeopacity{0.000000}%
\pgfsetdash{}{0pt}%
\pgfpathmoveto{\pgfqpoint{3.668196in}{1.103099in}}%
\pgfpathlineto{\pgfqpoint{3.773374in}{1.103099in}}%
\pgfpathlineto{\pgfqpoint{3.773374in}{1.163422in}}%
\pgfpathlineto{\pgfqpoint{3.668196in}{1.163422in}}%
\pgfpathlineto{\pgfqpoint{3.668196in}{1.103099in}}%
\pgfpathclose%
\pgfusepath{fill}%
\end{pgfscope}%
\begin{pgfscope}%
\pgfpathrectangle{\pgfqpoint{0.499691in}{1.103099in}}{\pgfqpoint{7.362500in}{3.850000in}}%
\pgfusepath{clip}%
\pgfsetbuttcap%
\pgfsetmiterjoin%
\definecolor{currentfill}{rgb}{0.914216,0.537745,0.399510}%
\pgfsetfillcolor{currentfill}%
\pgfsetlinewidth{0.000000pt}%
\definecolor{currentstroke}{rgb}{0.000000,0.000000,0.000000}%
\pgfsetstrokecolor{currentstroke}%
\pgfsetstrokeopacity{0.000000}%
\pgfsetdash{}{0pt}%
\pgfpathmoveto{\pgfqpoint{3.799669in}{1.103099in}}%
\pgfpathlineto{\pgfqpoint{3.904847in}{1.103099in}}%
\pgfpathlineto{\pgfqpoint{3.904847in}{1.234940in}}%
\pgfpathlineto{\pgfqpoint{3.799669in}{1.234940in}}%
\pgfpathlineto{\pgfqpoint{3.799669in}{1.103099in}}%
\pgfpathclose%
\pgfusepath{fill}%
\end{pgfscope}%
\begin{pgfscope}%
\pgfpathrectangle{\pgfqpoint{0.499691in}{1.103099in}}{\pgfqpoint{7.362500in}{3.850000in}}%
\pgfusepath{clip}%
\pgfsetbuttcap%
\pgfsetmiterjoin%
\definecolor{currentfill}{rgb}{0.914216,0.537745,0.399510}%
\pgfsetfillcolor{currentfill}%
\pgfsetlinewidth{0.000000pt}%
\definecolor{currentstroke}{rgb}{0.000000,0.000000,0.000000}%
\pgfsetstrokecolor{currentstroke}%
\pgfsetstrokeopacity{0.000000}%
\pgfsetdash{}{0pt}%
\pgfpathmoveto{\pgfqpoint{3.931142in}{1.103099in}}%
\pgfpathlineto{\pgfqpoint{4.036321in}{1.103099in}}%
\pgfpathlineto{\pgfqpoint{4.036321in}{1.297129in}}%
\pgfpathlineto{\pgfqpoint{3.931142in}{1.297129in}}%
\pgfpathlineto{\pgfqpoint{3.931142in}{1.103099in}}%
\pgfpathclose%
\pgfusepath{fill}%
\end{pgfscope}%
\begin{pgfscope}%
\pgfpathrectangle{\pgfqpoint{0.499691in}{1.103099in}}{\pgfqpoint{7.362500in}{3.850000in}}%
\pgfusepath{clip}%
\pgfsetbuttcap%
\pgfsetmiterjoin%
\definecolor{currentfill}{rgb}{0.914216,0.537745,0.399510}%
\pgfsetfillcolor{currentfill}%
\pgfsetlinewidth{0.000000pt}%
\definecolor{currentstroke}{rgb}{0.000000,0.000000,0.000000}%
\pgfsetstrokecolor{currentstroke}%
\pgfsetstrokeopacity{0.000000}%
\pgfsetdash{}{0pt}%
\pgfpathmoveto{\pgfqpoint{4.062615in}{1.103099in}}%
\pgfpathlineto{\pgfqpoint{4.167794in}{1.103099in}}%
\pgfpathlineto{\pgfqpoint{4.167794in}{1.103721in}}%
\pgfpathlineto{\pgfqpoint{4.062615in}{1.103721in}}%
\pgfpathlineto{\pgfqpoint{4.062615in}{1.103099in}}%
\pgfpathclose%
\pgfusepath{fill}%
\end{pgfscope}%
\begin{pgfscope}%
\pgfpathrectangle{\pgfqpoint{0.499691in}{1.103099in}}{\pgfqpoint{7.362500in}{3.850000in}}%
\pgfusepath{clip}%
\pgfsetbuttcap%
\pgfsetmiterjoin%
\definecolor{currentfill}{rgb}{0.914216,0.537745,0.399510}%
\pgfsetfillcolor{currentfill}%
\pgfsetlinewidth{0.000000pt}%
\definecolor{currentstroke}{rgb}{0.000000,0.000000,0.000000}%
\pgfsetstrokecolor{currentstroke}%
\pgfsetstrokeopacity{0.000000}%
\pgfsetdash{}{0pt}%
\pgfpathmoveto{\pgfqpoint{4.194088in}{1.103099in}}%
\pgfpathlineto{\pgfqpoint{4.299267in}{1.103099in}}%
\pgfpathlineto{\pgfqpoint{4.299267in}{1.142278in}}%
\pgfpathlineto{\pgfqpoint{4.194088in}{1.142278in}}%
\pgfpathlineto{\pgfqpoint{4.194088in}{1.103099in}}%
\pgfpathclose%
\pgfusepath{fill}%
\end{pgfscope}%
\begin{pgfscope}%
\pgfpathrectangle{\pgfqpoint{0.499691in}{1.103099in}}{\pgfqpoint{7.362500in}{3.850000in}}%
\pgfusepath{clip}%
\pgfsetbuttcap%
\pgfsetmiterjoin%
\definecolor{currentfill}{rgb}{0.914216,0.537745,0.399510}%
\pgfsetfillcolor{currentfill}%
\pgfsetlinewidth{0.000000pt}%
\definecolor{currentstroke}{rgb}{0.000000,0.000000,0.000000}%
\pgfsetstrokecolor{currentstroke}%
\pgfsetstrokeopacity{0.000000}%
\pgfsetdash{}{0pt}%
\pgfpathmoveto{\pgfqpoint{4.325562in}{1.103099in}}%
\pgfpathlineto{\pgfqpoint{4.430740in}{1.103099in}}%
\pgfpathlineto{\pgfqpoint{4.430740in}{3.769765in}}%
\pgfpathlineto{\pgfqpoint{4.325562in}{3.769765in}}%
\pgfpathlineto{\pgfqpoint{4.325562in}{1.103099in}}%
\pgfpathclose%
\pgfusepath{fill}%
\end{pgfscope}%
\begin{pgfscope}%
\pgfpathrectangle{\pgfqpoint{0.499691in}{1.103099in}}{\pgfqpoint{7.362500in}{3.850000in}}%
\pgfusepath{clip}%
\pgfsetbuttcap%
\pgfsetmiterjoin%
\definecolor{currentfill}{rgb}{0.914216,0.537745,0.399510}%
\pgfsetfillcolor{currentfill}%
\pgfsetlinewidth{0.000000pt}%
\definecolor{currentstroke}{rgb}{0.000000,0.000000,0.000000}%
\pgfsetstrokecolor{currentstroke}%
\pgfsetstrokeopacity{0.000000}%
\pgfsetdash{}{0pt}%
\pgfpathmoveto{\pgfqpoint{4.457035in}{1.103099in}}%
\pgfpathlineto{\pgfqpoint{4.562213in}{1.103099in}}%
\pgfpathlineto{\pgfqpoint{4.562213in}{1.114293in}}%
\pgfpathlineto{\pgfqpoint{4.457035in}{1.114293in}}%
\pgfpathlineto{\pgfqpoint{4.457035in}{1.103099in}}%
\pgfpathclose%
\pgfusepath{fill}%
\end{pgfscope}%
\begin{pgfscope}%
\pgfpathrectangle{\pgfqpoint{0.499691in}{1.103099in}}{\pgfqpoint{7.362500in}{3.850000in}}%
\pgfusepath{clip}%
\pgfsetbuttcap%
\pgfsetmiterjoin%
\definecolor{currentfill}{rgb}{0.914216,0.537745,0.399510}%
\pgfsetfillcolor{currentfill}%
\pgfsetlinewidth{0.000000pt}%
\definecolor{currentstroke}{rgb}{0.000000,0.000000,0.000000}%
\pgfsetstrokecolor{currentstroke}%
\pgfsetstrokeopacity{0.000000}%
\pgfsetdash{}{0pt}%
\pgfpathmoveto{\pgfqpoint{4.588508in}{1.103099in}}%
\pgfpathlineto{\pgfqpoint{4.693687in}{1.103099in}}%
\pgfpathlineto{\pgfqpoint{4.693687in}{1.483696in}}%
\pgfpathlineto{\pgfqpoint{4.588508in}{1.483696in}}%
\pgfpathlineto{\pgfqpoint{4.588508in}{1.103099in}}%
\pgfpathclose%
\pgfusepath{fill}%
\end{pgfscope}%
\begin{pgfscope}%
\pgfpathrectangle{\pgfqpoint{0.499691in}{1.103099in}}{\pgfqpoint{7.362500in}{3.850000in}}%
\pgfusepath{clip}%
\pgfsetbuttcap%
\pgfsetmiterjoin%
\definecolor{currentfill}{rgb}{0.914216,0.537745,0.399510}%
\pgfsetfillcolor{currentfill}%
\pgfsetlinewidth{0.000000pt}%
\definecolor{currentstroke}{rgb}{0.000000,0.000000,0.000000}%
\pgfsetstrokecolor{currentstroke}%
\pgfsetstrokeopacity{0.000000}%
\pgfsetdash{}{0pt}%
\pgfpathmoveto{\pgfqpoint{4.719981in}{1.103099in}}%
\pgfpathlineto{\pgfqpoint{4.825160in}{1.103099in}}%
\pgfpathlineto{\pgfqpoint{4.825160in}{1.618646in}}%
\pgfpathlineto{\pgfqpoint{4.719981in}{1.618646in}}%
\pgfpathlineto{\pgfqpoint{4.719981in}{1.103099in}}%
\pgfpathclose%
\pgfusepath{fill}%
\end{pgfscope}%
\begin{pgfscope}%
\pgfpathrectangle{\pgfqpoint{0.499691in}{1.103099in}}{\pgfqpoint{7.362500in}{3.850000in}}%
\pgfusepath{clip}%
\pgfsetbuttcap%
\pgfsetmiterjoin%
\definecolor{currentfill}{rgb}{0.914216,0.537745,0.399510}%
\pgfsetfillcolor{currentfill}%
\pgfsetlinewidth{0.000000pt}%
\definecolor{currentstroke}{rgb}{0.000000,0.000000,0.000000}%
\pgfsetstrokecolor{currentstroke}%
\pgfsetstrokeopacity{0.000000}%
\pgfsetdash{}{0pt}%
\pgfpathmoveto{\pgfqpoint{4.851455in}{1.103099in}}%
\pgfpathlineto{\pgfqpoint{4.956633in}{1.103099in}}%
\pgfpathlineto{\pgfqpoint{4.956633in}{1.470636in}}%
\pgfpathlineto{\pgfqpoint{4.851455in}{1.470636in}}%
\pgfpathlineto{\pgfqpoint{4.851455in}{1.103099in}}%
\pgfpathclose%
\pgfusepath{fill}%
\end{pgfscope}%
\begin{pgfscope}%
\pgfpathrectangle{\pgfqpoint{0.499691in}{1.103099in}}{\pgfqpoint{7.362500in}{3.850000in}}%
\pgfusepath{clip}%
\pgfsetbuttcap%
\pgfsetmiterjoin%
\definecolor{currentfill}{rgb}{0.914216,0.537745,0.399510}%
\pgfsetfillcolor{currentfill}%
\pgfsetlinewidth{0.000000pt}%
\definecolor{currentstroke}{rgb}{0.000000,0.000000,0.000000}%
\pgfsetstrokecolor{currentstroke}%
\pgfsetstrokeopacity{0.000000}%
\pgfsetdash{}{0pt}%
\pgfpathmoveto{\pgfqpoint{4.982928in}{1.103099in}}%
\pgfpathlineto{\pgfqpoint{5.088106in}{1.103099in}}%
\pgfpathlineto{\pgfqpoint{5.088106in}{1.380462in}}%
\pgfpathlineto{\pgfqpoint{4.982928in}{1.380462in}}%
\pgfpathlineto{\pgfqpoint{4.982928in}{1.103099in}}%
\pgfpathclose%
\pgfusepath{fill}%
\end{pgfscope}%
\begin{pgfscope}%
\pgfpathrectangle{\pgfqpoint{0.499691in}{1.103099in}}{\pgfqpoint{7.362500in}{3.850000in}}%
\pgfusepath{clip}%
\pgfsetbuttcap%
\pgfsetmiterjoin%
\definecolor{currentfill}{rgb}{0.914216,0.537745,0.399510}%
\pgfsetfillcolor{currentfill}%
\pgfsetlinewidth{0.000000pt}%
\definecolor{currentstroke}{rgb}{0.000000,0.000000,0.000000}%
\pgfsetstrokecolor{currentstroke}%
\pgfsetstrokeopacity{0.000000}%
\pgfsetdash{}{0pt}%
\pgfpathmoveto{\pgfqpoint{5.114401in}{1.103099in}}%
\pgfpathlineto{\pgfqpoint{5.219580in}{1.103099in}}%
\pgfpathlineto{\pgfqpoint{5.219580in}{1.105586in}}%
\pgfpathlineto{\pgfqpoint{5.114401in}{1.105586in}}%
\pgfpathlineto{\pgfqpoint{5.114401in}{1.103099in}}%
\pgfpathclose%
\pgfusepath{fill}%
\end{pgfscope}%
\begin{pgfscope}%
\pgfpathrectangle{\pgfqpoint{0.499691in}{1.103099in}}{\pgfqpoint{7.362500in}{3.850000in}}%
\pgfusepath{clip}%
\pgfsetbuttcap%
\pgfsetmiterjoin%
\definecolor{currentfill}{rgb}{0.914216,0.537745,0.399510}%
\pgfsetfillcolor{currentfill}%
\pgfsetlinewidth{0.000000pt}%
\definecolor{currentstroke}{rgb}{0.000000,0.000000,0.000000}%
\pgfsetstrokecolor{currentstroke}%
\pgfsetstrokeopacity{0.000000}%
\pgfsetdash{}{0pt}%
\pgfpathmoveto{\pgfqpoint{5.245874in}{1.103099in}}%
\pgfpathlineto{\pgfqpoint{5.351053in}{1.103099in}}%
\pgfpathlineto{\pgfqpoint{5.351053in}{1.114915in}}%
\pgfpathlineto{\pgfqpoint{5.245874in}{1.114915in}}%
\pgfpathlineto{\pgfqpoint{5.245874in}{1.103099in}}%
\pgfpathclose%
\pgfusepath{fill}%
\end{pgfscope}%
\begin{pgfscope}%
\pgfpathrectangle{\pgfqpoint{0.499691in}{1.103099in}}{\pgfqpoint{7.362500in}{3.850000in}}%
\pgfusepath{clip}%
\pgfsetbuttcap%
\pgfsetmiterjoin%
\definecolor{currentfill}{rgb}{0.914216,0.537745,0.399510}%
\pgfsetfillcolor{currentfill}%
\pgfsetlinewidth{0.000000pt}%
\definecolor{currentstroke}{rgb}{0.000000,0.000000,0.000000}%
\pgfsetstrokecolor{currentstroke}%
\pgfsetstrokeopacity{0.000000}%
\pgfsetdash{}{0pt}%
\pgfpathmoveto{\pgfqpoint{5.377347in}{1.103099in}}%
\pgfpathlineto{\pgfqpoint{5.482526in}{1.103099in}}%
\pgfpathlineto{\pgfqpoint{5.482526in}{1.126731in}}%
\pgfpathlineto{\pgfqpoint{5.377347in}{1.126731in}}%
\pgfpathlineto{\pgfqpoint{5.377347in}{1.103099in}}%
\pgfpathclose%
\pgfusepath{fill}%
\end{pgfscope}%
\begin{pgfscope}%
\pgfpathrectangle{\pgfqpoint{0.499691in}{1.103099in}}{\pgfqpoint{7.362500in}{3.850000in}}%
\pgfusepath{clip}%
\pgfsetbuttcap%
\pgfsetmiterjoin%
\definecolor{currentfill}{rgb}{0.914216,0.537745,0.399510}%
\pgfsetfillcolor{currentfill}%
\pgfsetlinewidth{0.000000pt}%
\definecolor{currentstroke}{rgb}{0.000000,0.000000,0.000000}%
\pgfsetstrokecolor{currentstroke}%
\pgfsetstrokeopacity{0.000000}%
\pgfsetdash{}{0pt}%
\pgfpathmoveto{\pgfqpoint{5.508821in}{1.103099in}}%
\pgfpathlineto{\pgfqpoint{5.613999in}{1.103099in}}%
\pgfpathlineto{\pgfqpoint{5.613999in}{1.109318in}}%
\pgfpathlineto{\pgfqpoint{5.508821in}{1.109318in}}%
\pgfpathlineto{\pgfqpoint{5.508821in}{1.103099in}}%
\pgfpathclose%
\pgfusepath{fill}%
\end{pgfscope}%
\begin{pgfscope}%
\pgfpathrectangle{\pgfqpoint{0.499691in}{1.103099in}}{\pgfqpoint{7.362500in}{3.850000in}}%
\pgfusepath{clip}%
\pgfsetbuttcap%
\pgfsetmiterjoin%
\definecolor{currentfill}{rgb}{0.914216,0.537745,0.399510}%
\pgfsetfillcolor{currentfill}%
\pgfsetlinewidth{0.000000pt}%
\definecolor{currentstroke}{rgb}{0.000000,0.000000,0.000000}%
\pgfsetstrokecolor{currentstroke}%
\pgfsetstrokeopacity{0.000000}%
\pgfsetdash{}{0pt}%
\pgfpathmoveto{\pgfqpoint{5.640294in}{1.103099in}}%
\pgfpathlineto{\pgfqpoint{5.745472in}{1.103099in}}%
\pgfpathlineto{\pgfqpoint{5.745472in}{1.397875in}}%
\pgfpathlineto{\pgfqpoint{5.640294in}{1.397875in}}%
\pgfpathlineto{\pgfqpoint{5.640294in}{1.103099in}}%
\pgfpathclose%
\pgfusepath{fill}%
\end{pgfscope}%
\begin{pgfscope}%
\pgfpathrectangle{\pgfqpoint{0.499691in}{1.103099in}}{\pgfqpoint{7.362500in}{3.850000in}}%
\pgfusepath{clip}%
\pgfsetbuttcap%
\pgfsetmiterjoin%
\definecolor{currentfill}{rgb}{0.914216,0.537745,0.399510}%
\pgfsetfillcolor{currentfill}%
\pgfsetlinewidth{0.000000pt}%
\definecolor{currentstroke}{rgb}{0.000000,0.000000,0.000000}%
\pgfsetstrokecolor{currentstroke}%
\pgfsetstrokeopacity{0.000000}%
\pgfsetdash{}{0pt}%
\pgfpathmoveto{\pgfqpoint{5.771767in}{1.103099in}}%
\pgfpathlineto{\pgfqpoint{5.876946in}{1.103099in}}%
\pgfpathlineto{\pgfqpoint{5.876946in}{1.269144in}}%
\pgfpathlineto{\pgfqpoint{5.771767in}{1.269144in}}%
\pgfpathlineto{\pgfqpoint{5.771767in}{1.103099in}}%
\pgfpathclose%
\pgfusepath{fill}%
\end{pgfscope}%
\begin{pgfscope}%
\pgfpathrectangle{\pgfqpoint{0.499691in}{1.103099in}}{\pgfqpoint{7.362500in}{3.850000in}}%
\pgfusepath{clip}%
\pgfsetbuttcap%
\pgfsetmiterjoin%
\definecolor{currentfill}{rgb}{0.914216,0.537745,0.399510}%
\pgfsetfillcolor{currentfill}%
\pgfsetlinewidth{0.000000pt}%
\definecolor{currentstroke}{rgb}{0.000000,0.000000,0.000000}%
\pgfsetstrokecolor{currentstroke}%
\pgfsetstrokeopacity{0.000000}%
\pgfsetdash{}{0pt}%
\pgfpathmoveto{\pgfqpoint{5.903240in}{1.103099in}}%
\pgfpathlineto{\pgfqpoint{6.008419in}{1.103099in}}%
\pgfpathlineto{\pgfqpoint{6.008419in}{1.136059in}}%
\pgfpathlineto{\pgfqpoint{5.903240in}{1.136059in}}%
\pgfpathlineto{\pgfqpoint{5.903240in}{1.103099in}}%
\pgfpathclose%
\pgfusepath{fill}%
\end{pgfscope}%
\begin{pgfscope}%
\pgfpathrectangle{\pgfqpoint{0.499691in}{1.103099in}}{\pgfqpoint{7.362500in}{3.850000in}}%
\pgfusepath{clip}%
\pgfsetbuttcap%
\pgfsetmiterjoin%
\definecolor{currentfill}{rgb}{0.914216,0.537745,0.399510}%
\pgfsetfillcolor{currentfill}%
\pgfsetlinewidth{0.000000pt}%
\definecolor{currentstroke}{rgb}{0.000000,0.000000,0.000000}%
\pgfsetstrokecolor{currentstroke}%
\pgfsetstrokeopacity{0.000000}%
\pgfsetdash{}{0pt}%
\pgfpathmoveto{\pgfqpoint{6.034713in}{1.103099in}}%
\pgfpathlineto{\pgfqpoint{6.139892in}{1.103099in}}%
\pgfpathlineto{\pgfqpoint{6.139892in}{1.233074in}}%
\pgfpathlineto{\pgfqpoint{6.034713in}{1.233074in}}%
\pgfpathlineto{\pgfqpoint{6.034713in}{1.103099in}}%
\pgfpathclose%
\pgfusepath{fill}%
\end{pgfscope}%
\begin{pgfscope}%
\pgfpathrectangle{\pgfqpoint{0.499691in}{1.103099in}}{\pgfqpoint{7.362500in}{3.850000in}}%
\pgfusepath{clip}%
\pgfsetbuttcap%
\pgfsetmiterjoin%
\definecolor{currentfill}{rgb}{0.914216,0.537745,0.399510}%
\pgfsetfillcolor{currentfill}%
\pgfsetlinewidth{0.000000pt}%
\definecolor{currentstroke}{rgb}{0.000000,0.000000,0.000000}%
\pgfsetstrokecolor{currentstroke}%
\pgfsetstrokeopacity{0.000000}%
\pgfsetdash{}{0pt}%
\pgfpathmoveto{\pgfqpoint{6.166187in}{1.103099in}}%
\pgfpathlineto{\pgfqpoint{6.271365in}{1.103099in}}%
\pgfpathlineto{\pgfqpoint{6.271365in}{1.182701in}}%
\pgfpathlineto{\pgfqpoint{6.166187in}{1.182701in}}%
\pgfpathlineto{\pgfqpoint{6.166187in}{1.103099in}}%
\pgfpathclose%
\pgfusepath{fill}%
\end{pgfscope}%
\begin{pgfscope}%
\pgfpathrectangle{\pgfqpoint{0.499691in}{1.103099in}}{\pgfqpoint{7.362500in}{3.850000in}}%
\pgfusepath{clip}%
\pgfsetbuttcap%
\pgfsetmiterjoin%
\definecolor{currentfill}{rgb}{0.914216,0.537745,0.399510}%
\pgfsetfillcolor{currentfill}%
\pgfsetlinewidth{0.000000pt}%
\definecolor{currentstroke}{rgb}{0.000000,0.000000,0.000000}%
\pgfsetstrokecolor{currentstroke}%
\pgfsetstrokeopacity{0.000000}%
\pgfsetdash{}{0pt}%
\pgfpathmoveto{\pgfqpoint{6.297660in}{1.103099in}}%
\pgfpathlineto{\pgfqpoint{6.402838in}{1.103099in}}%
\pgfpathlineto{\pgfqpoint{6.402838in}{1.638547in}}%
\pgfpathlineto{\pgfqpoint{6.297660in}{1.638547in}}%
\pgfpathlineto{\pgfqpoint{6.297660in}{1.103099in}}%
\pgfpathclose%
\pgfusepath{fill}%
\end{pgfscope}%
\begin{pgfscope}%
\pgfpathrectangle{\pgfqpoint{0.499691in}{1.103099in}}{\pgfqpoint{7.362500in}{3.850000in}}%
\pgfusepath{clip}%
\pgfsetbuttcap%
\pgfsetmiterjoin%
\definecolor{currentfill}{rgb}{0.914216,0.537745,0.399510}%
\pgfsetfillcolor{currentfill}%
\pgfsetlinewidth{0.000000pt}%
\definecolor{currentstroke}{rgb}{0.000000,0.000000,0.000000}%
\pgfsetstrokecolor{currentstroke}%
\pgfsetstrokeopacity{0.000000}%
\pgfsetdash{}{0pt}%
\pgfpathmoveto{\pgfqpoint{6.429133in}{1.103099in}}%
\pgfpathlineto{\pgfqpoint{6.534312in}{1.103099in}}%
\pgfpathlineto{\pgfqpoint{6.534312in}{1.447004in}}%
\pgfpathlineto{\pgfqpoint{6.429133in}{1.447004in}}%
\pgfpathlineto{\pgfqpoint{6.429133in}{1.103099in}}%
\pgfpathclose%
\pgfusepath{fill}%
\end{pgfscope}%
\begin{pgfscope}%
\pgfpathrectangle{\pgfqpoint{0.499691in}{1.103099in}}{\pgfqpoint{7.362500in}{3.850000in}}%
\pgfusepath{clip}%
\pgfsetbuttcap%
\pgfsetmiterjoin%
\definecolor{currentfill}{rgb}{0.914216,0.537745,0.399510}%
\pgfsetfillcolor{currentfill}%
\pgfsetlinewidth{0.000000pt}%
\definecolor{currentstroke}{rgb}{0.000000,0.000000,0.000000}%
\pgfsetstrokecolor{currentstroke}%
\pgfsetstrokeopacity{0.000000}%
\pgfsetdash{}{0pt}%
\pgfpathmoveto{\pgfqpoint{6.560606in}{1.103099in}}%
\pgfpathlineto{\pgfqpoint{6.665785in}{1.103099in}}%
\pgfpathlineto{\pgfqpoint{6.665785in}{1.120512in}}%
\pgfpathlineto{\pgfqpoint{6.560606in}{1.120512in}}%
\pgfpathlineto{\pgfqpoint{6.560606in}{1.103099in}}%
\pgfpathclose%
\pgfusepath{fill}%
\end{pgfscope}%
\begin{pgfscope}%
\pgfpathrectangle{\pgfqpoint{0.499691in}{1.103099in}}{\pgfqpoint{7.362500in}{3.850000in}}%
\pgfusepath{clip}%
\pgfsetbuttcap%
\pgfsetmiterjoin%
\definecolor{currentfill}{rgb}{0.914216,0.537745,0.399510}%
\pgfsetfillcolor{currentfill}%
\pgfsetlinewidth{0.000000pt}%
\definecolor{currentstroke}{rgb}{0.000000,0.000000,0.000000}%
\pgfsetstrokecolor{currentstroke}%
\pgfsetstrokeopacity{0.000000}%
\pgfsetdash{}{0pt}%
\pgfpathmoveto{\pgfqpoint{6.692080in}{1.103099in}}%
\pgfpathlineto{\pgfqpoint{6.797258in}{1.103099in}}%
\pgfpathlineto{\pgfqpoint{6.797258in}{1.512303in}}%
\pgfpathlineto{\pgfqpoint{6.692080in}{1.512303in}}%
\pgfpathlineto{\pgfqpoint{6.692080in}{1.103099in}}%
\pgfpathclose%
\pgfusepath{fill}%
\end{pgfscope}%
\begin{pgfscope}%
\pgfpathrectangle{\pgfqpoint{0.499691in}{1.103099in}}{\pgfqpoint{7.362500in}{3.850000in}}%
\pgfusepath{clip}%
\pgfsetbuttcap%
\pgfsetmiterjoin%
\definecolor{currentfill}{rgb}{0.914216,0.537745,0.399510}%
\pgfsetfillcolor{currentfill}%
\pgfsetlinewidth{0.000000pt}%
\definecolor{currentstroke}{rgb}{0.000000,0.000000,0.000000}%
\pgfsetstrokecolor{currentstroke}%
\pgfsetstrokeopacity{0.000000}%
\pgfsetdash{}{0pt}%
\pgfpathmoveto{\pgfqpoint{6.823553in}{1.103099in}}%
\pgfpathlineto{\pgfqpoint{6.928731in}{1.103099in}}%
\pgfpathlineto{\pgfqpoint{6.928731in}{1.162800in}}%
\pgfpathlineto{\pgfqpoint{6.823553in}{1.162800in}}%
\pgfpathlineto{\pgfqpoint{6.823553in}{1.103099in}}%
\pgfpathclose%
\pgfusepath{fill}%
\end{pgfscope}%
\begin{pgfscope}%
\pgfpathrectangle{\pgfqpoint{0.499691in}{1.103099in}}{\pgfqpoint{7.362500in}{3.850000in}}%
\pgfusepath{clip}%
\pgfsetbuttcap%
\pgfsetmiterjoin%
\definecolor{currentfill}{rgb}{0.914216,0.537745,0.399510}%
\pgfsetfillcolor{currentfill}%
\pgfsetlinewidth{0.000000pt}%
\definecolor{currentstroke}{rgb}{0.000000,0.000000,0.000000}%
\pgfsetstrokecolor{currentstroke}%
\pgfsetstrokeopacity{0.000000}%
\pgfsetdash{}{0pt}%
\pgfpathmoveto{\pgfqpoint{6.955026in}{1.103099in}}%
\pgfpathlineto{\pgfqpoint{7.060205in}{1.103099in}}%
\pgfpathlineto{\pgfqpoint{7.060205in}{1.106830in}}%
\pgfpathlineto{\pgfqpoint{6.955026in}{1.106830in}}%
\pgfpathlineto{\pgfqpoint{6.955026in}{1.103099in}}%
\pgfpathclose%
\pgfusepath{fill}%
\end{pgfscope}%
\begin{pgfscope}%
\pgfpathrectangle{\pgfqpoint{0.499691in}{1.103099in}}{\pgfqpoint{7.362500in}{3.850000in}}%
\pgfusepath{clip}%
\pgfsetbuttcap%
\pgfsetmiterjoin%
\definecolor{currentfill}{rgb}{0.914216,0.537745,0.399510}%
\pgfsetfillcolor{currentfill}%
\pgfsetlinewidth{0.000000pt}%
\definecolor{currentstroke}{rgb}{0.000000,0.000000,0.000000}%
\pgfsetstrokecolor{currentstroke}%
\pgfsetstrokeopacity{0.000000}%
\pgfsetdash{}{0pt}%
\pgfpathmoveto{\pgfqpoint{7.086499in}{1.103099in}}%
\pgfpathlineto{\pgfqpoint{7.191678in}{1.103099in}}%
\pgfpathlineto{\pgfqpoint{7.191678in}{1.147875in}}%
\pgfpathlineto{\pgfqpoint{7.086499in}{1.147875in}}%
\pgfpathlineto{\pgfqpoint{7.086499in}{1.103099in}}%
\pgfpathclose%
\pgfusepath{fill}%
\end{pgfscope}%
\begin{pgfscope}%
\pgfpathrectangle{\pgfqpoint{0.499691in}{1.103099in}}{\pgfqpoint{7.362500in}{3.850000in}}%
\pgfusepath{clip}%
\pgfsetbuttcap%
\pgfsetmiterjoin%
\definecolor{currentfill}{rgb}{0.914216,0.537745,0.399510}%
\pgfsetfillcolor{currentfill}%
\pgfsetlinewidth{0.000000pt}%
\definecolor{currentstroke}{rgb}{0.000000,0.000000,0.000000}%
\pgfsetstrokecolor{currentstroke}%
\pgfsetstrokeopacity{0.000000}%
\pgfsetdash{}{0pt}%
\pgfpathmoveto{\pgfqpoint{7.217972in}{1.103099in}}%
\pgfpathlineto{\pgfqpoint{7.323151in}{1.103099in}}%
\pgfpathlineto{\pgfqpoint{7.323151in}{1.162800in}}%
\pgfpathlineto{\pgfqpoint{7.217972in}{1.162800in}}%
\pgfpathlineto{\pgfqpoint{7.217972in}{1.103099in}}%
\pgfpathclose%
\pgfusepath{fill}%
\end{pgfscope}%
\begin{pgfscope}%
\pgfpathrectangle{\pgfqpoint{0.499691in}{1.103099in}}{\pgfqpoint{7.362500in}{3.850000in}}%
\pgfusepath{clip}%
\pgfsetbuttcap%
\pgfsetmiterjoin%
\definecolor{currentfill}{rgb}{0.914216,0.537745,0.399510}%
\pgfsetfillcolor{currentfill}%
\pgfsetlinewidth{0.000000pt}%
\definecolor{currentstroke}{rgb}{0.000000,0.000000,0.000000}%
\pgfsetstrokecolor{currentstroke}%
\pgfsetstrokeopacity{0.000000}%
\pgfsetdash{}{0pt}%
\pgfpathmoveto{\pgfqpoint{7.349446in}{1.103099in}}%
\pgfpathlineto{\pgfqpoint{7.454624in}{1.103099in}}%
\pgfpathlineto{\pgfqpoint{7.454624in}{1.118646in}}%
\pgfpathlineto{\pgfqpoint{7.349446in}{1.118646in}}%
\pgfpathlineto{\pgfqpoint{7.349446in}{1.103099in}}%
\pgfpathclose%
\pgfusepath{fill}%
\end{pgfscope}%
\begin{pgfscope}%
\pgfpathrectangle{\pgfqpoint{0.499691in}{1.103099in}}{\pgfqpoint{7.362500in}{3.850000in}}%
\pgfusepath{clip}%
\pgfsetbuttcap%
\pgfsetmiterjoin%
\definecolor{currentfill}{rgb}{0.914216,0.537745,0.399510}%
\pgfsetfillcolor{currentfill}%
\pgfsetlinewidth{0.000000pt}%
\definecolor{currentstroke}{rgb}{0.000000,0.000000,0.000000}%
\pgfsetstrokecolor{currentstroke}%
\pgfsetstrokeopacity{0.000000}%
\pgfsetdash{}{0pt}%
\pgfpathmoveto{\pgfqpoint{7.480919in}{1.103099in}}%
\pgfpathlineto{\pgfqpoint{7.586097in}{1.103099in}}%
\pgfpathlineto{\pgfqpoint{7.586097in}{1.119890in}}%
\pgfpathlineto{\pgfqpoint{7.480919in}{1.119890in}}%
\pgfpathlineto{\pgfqpoint{7.480919in}{1.103099in}}%
\pgfpathclose%
\pgfusepath{fill}%
\end{pgfscope}%
\begin{pgfscope}%
\pgfpathrectangle{\pgfqpoint{0.499691in}{1.103099in}}{\pgfqpoint{7.362500in}{3.850000in}}%
\pgfusepath{clip}%
\pgfsetbuttcap%
\pgfsetmiterjoin%
\definecolor{currentfill}{rgb}{0.914216,0.537745,0.399510}%
\pgfsetfillcolor{currentfill}%
\pgfsetlinewidth{0.000000pt}%
\definecolor{currentstroke}{rgb}{0.000000,0.000000,0.000000}%
\pgfsetstrokecolor{currentstroke}%
\pgfsetstrokeopacity{0.000000}%
\pgfsetdash{}{0pt}%
\pgfpathmoveto{\pgfqpoint{7.612392in}{1.103099in}}%
\pgfpathlineto{\pgfqpoint{7.717571in}{1.103099in}}%
\pgfpathlineto{\pgfqpoint{7.717571in}{1.188920in}}%
\pgfpathlineto{\pgfqpoint{7.612392in}{1.188920in}}%
\pgfpathlineto{\pgfqpoint{7.612392in}{1.103099in}}%
\pgfpathclose%
\pgfusepath{fill}%
\end{pgfscope}%
\begin{pgfscope}%
\pgfpathrectangle{\pgfqpoint{0.499691in}{1.103099in}}{\pgfqpoint{7.362500in}{3.850000in}}%
\pgfusepath{clip}%
\pgfsetbuttcap%
\pgfsetmiterjoin%
\definecolor{currentfill}{rgb}{0.914216,0.537745,0.399510}%
\pgfsetfillcolor{currentfill}%
\pgfsetlinewidth{0.000000pt}%
\definecolor{currentstroke}{rgb}{0.000000,0.000000,0.000000}%
\pgfsetstrokecolor{currentstroke}%
\pgfsetstrokeopacity{0.000000}%
\pgfsetdash{}{0pt}%
\pgfpathmoveto{\pgfqpoint{7.743865in}{1.103099in}}%
\pgfpathlineto{\pgfqpoint{7.849044in}{1.103099in}}%
\pgfpathlineto{\pgfqpoint{7.849044in}{1.255462in}}%
\pgfpathlineto{\pgfqpoint{7.743865in}{1.255462in}}%
\pgfpathlineto{\pgfqpoint{7.743865in}{1.103099in}}%
\pgfpathclose%
\pgfusepath{fill}%
\end{pgfscope}%
\begin{pgfscope}%
\pgfsetbuttcap%
\pgfsetroundjoin%
\definecolor{currentfill}{rgb}{0.000000,0.000000,0.000000}%
\pgfsetfillcolor{currentfill}%
\pgfsetlinewidth{0.803000pt}%
\definecolor{currentstroke}{rgb}{0.000000,0.000000,0.000000}%
\pgfsetstrokecolor{currentstroke}%
\pgfsetdash{}{0pt}%
\pgfsys@defobject{currentmarker}{\pgfqpoint{0.000000in}{-0.048611in}}{\pgfqpoint{0.000000in}{0.000000in}}{%
\pgfpathmoveto{\pgfqpoint{0.000000in}{0.000000in}}%
\pgfpathlineto{\pgfqpoint{0.000000in}{-0.048611in}}%
\pgfusepath{stroke,fill}%
}%
\begin{pgfscope}%
\pgfsys@transformshift{0.565428in}{1.103099in}%
\pgfsys@useobject{currentmarker}{}%
\end{pgfscope}%
\end{pgfscope}%
\begin{pgfscope}%
\definecolor{textcolor}{rgb}{0.000000,0.000000,0.000000}%
\pgfsetstrokecolor{textcolor}%
\pgfsetfillcolor{textcolor}%
\pgftext[x=0.582789in, y=0.787123in, left, base,rotate=90.000000]{\color{textcolor}\rmfamily\fontsize{5.000000}{6.000000}\selectfont AMV}%
\end{pgfscope}%
\begin{pgfscope}%
\pgfsetbuttcap%
\pgfsetroundjoin%
\definecolor{currentfill}{rgb}{0.000000,0.000000,0.000000}%
\pgfsetfillcolor{currentfill}%
\pgfsetlinewidth{0.803000pt}%
\definecolor{currentstroke}{rgb}{0.000000,0.000000,0.000000}%
\pgfsetstrokecolor{currentstroke}%
\pgfsetdash{}{0pt}%
\pgfsys@defobject{currentmarker}{\pgfqpoint{0.000000in}{-0.048611in}}{\pgfqpoint{0.000000in}{0.000000in}}{%
\pgfpathmoveto{\pgfqpoint{0.000000in}{0.000000in}}%
\pgfpathlineto{\pgfqpoint{0.000000in}{-0.048611in}}%
\pgfusepath{stroke,fill}%
}%
\begin{pgfscope}%
\pgfsys@transformshift{0.696901in}{1.103099in}%
\pgfsys@useobject{currentmarker}{}%
\end{pgfscope}%
\end{pgfscope}%
\begin{pgfscope}%
\definecolor{textcolor}{rgb}{0.000000,0.000000,0.000000}%
\pgfsetstrokecolor{textcolor}%
\pgfsetfillcolor{textcolor}%
\pgftext[x=0.714262in, y=0.715942in, left, base,rotate=90.000000]{\color{textcolor}\rmfamily\fontsize{5.000000}{6.000000}\selectfont APRIL}%
\end{pgfscope}%
\begin{pgfscope}%
\pgfsetbuttcap%
\pgfsetroundjoin%
\definecolor{currentfill}{rgb}{0.000000,0.000000,0.000000}%
\pgfsetfillcolor{currentfill}%
\pgfsetlinewidth{0.803000pt}%
\definecolor{currentstroke}{rgb}{0.000000,0.000000,0.000000}%
\pgfsetstrokecolor{currentstroke}%
\pgfsetdash{}{0pt}%
\pgfsys@defobject{currentmarker}{\pgfqpoint{0.000000in}{-0.048611in}}{\pgfqpoint{0.000000in}{0.000000in}}{%
\pgfpathmoveto{\pgfqpoint{0.000000in}{0.000000in}}%
\pgfpathlineto{\pgfqpoint{0.000000in}{-0.048611in}}%
\pgfusepath{stroke,fill}%
}%
\begin{pgfscope}%
\pgfsys@transformshift{0.828374in}{1.103099in}%
\pgfsys@useobject{currentmarker}{}%
\end{pgfscope}%
\end{pgfscope}%
\begin{pgfscope}%
\definecolor{textcolor}{rgb}{0.000000,0.000000,0.000000}%
\pgfsetstrokecolor{textcolor}%
\pgfsetfillcolor{textcolor}%
\pgftext[x=0.845735in, y=0.468446in, left, base,rotate=90.000000]{\color{textcolor}\rmfamily\fontsize{5.000000}{6.000000}\selectfont APRIL Moto}%
\end{pgfscope}%
\begin{pgfscope}%
\pgfsetbuttcap%
\pgfsetroundjoin%
\definecolor{currentfill}{rgb}{0.000000,0.000000,0.000000}%
\pgfsetfillcolor{currentfill}%
\pgfsetlinewidth{0.803000pt}%
\definecolor{currentstroke}{rgb}{0.000000,0.000000,0.000000}%
\pgfsetstrokecolor{currentstroke}%
\pgfsetdash{}{0pt}%
\pgfsys@defobject{currentmarker}{\pgfqpoint{0.000000in}{-0.048611in}}{\pgfqpoint{0.000000in}{0.000000in}}{%
\pgfpathmoveto{\pgfqpoint{0.000000in}{0.000000in}}%
\pgfpathlineto{\pgfqpoint{0.000000in}{-0.048611in}}%
\pgfusepath{stroke,fill}%
}%
\begin{pgfscope}%
\pgfsys@transformshift{0.959847in}{1.103099in}%
\pgfsys@useobject{currentmarker}{}%
\end{pgfscope}%
\end{pgfscope}%
\begin{pgfscope}%
\definecolor{textcolor}{rgb}{0.000000,0.000000,0.000000}%
\pgfsetstrokecolor{textcolor}%
\pgfsetfillcolor{textcolor}%
\pgftext[x=0.977208in, y=0.801591in, left, base,rotate=90.000000]{\color{textcolor}\rmfamily\fontsize{5.000000}{6.000000}\selectfont AXA}%
\end{pgfscope}%
\begin{pgfscope}%
\pgfsetbuttcap%
\pgfsetroundjoin%
\definecolor{currentfill}{rgb}{0.000000,0.000000,0.000000}%
\pgfsetfillcolor{currentfill}%
\pgfsetlinewidth{0.803000pt}%
\definecolor{currentstroke}{rgb}{0.000000,0.000000,0.000000}%
\pgfsetstrokecolor{currentstroke}%
\pgfsetdash{}{0pt}%
\pgfsys@defobject{currentmarker}{\pgfqpoint{0.000000in}{-0.048611in}}{\pgfqpoint{0.000000in}{0.000000in}}{%
\pgfpathmoveto{\pgfqpoint{0.000000in}{0.000000in}}%
\pgfpathlineto{\pgfqpoint{0.000000in}{-0.048611in}}%
\pgfusepath{stroke,fill}%
}%
\begin{pgfscope}%
\pgfsys@transformshift{1.091321in}{1.103099in}%
\pgfsys@useobject{currentmarker}{}%
\end{pgfscope}%
\end{pgfscope}%
\begin{pgfscope}%
\definecolor{textcolor}{rgb}{0.000000,0.000000,0.000000}%
\pgfsetstrokecolor{textcolor}%
\pgfsetfillcolor{textcolor}%
\pgftext[x=1.108682in, y=0.250948in, left, base,rotate=90.000000]{\color{textcolor}\rmfamily\fontsize{5.000000}{6.000000}\selectfont Active Assurances}%
\end{pgfscope}%
\begin{pgfscope}%
\pgfsetbuttcap%
\pgfsetroundjoin%
\definecolor{currentfill}{rgb}{0.000000,0.000000,0.000000}%
\pgfsetfillcolor{currentfill}%
\pgfsetlinewidth{0.803000pt}%
\definecolor{currentstroke}{rgb}{0.000000,0.000000,0.000000}%
\pgfsetstrokecolor{currentstroke}%
\pgfsetdash{}{0pt}%
\pgfsys@defobject{currentmarker}{\pgfqpoint{0.000000in}{-0.048611in}}{\pgfqpoint{0.000000in}{0.000000in}}{%
\pgfpathmoveto{\pgfqpoint{0.000000in}{0.000000in}}%
\pgfpathlineto{\pgfqpoint{0.000000in}{-0.048611in}}%
\pgfusepath{stroke,fill}%
}%
\begin{pgfscope}%
\pgfsys@transformshift{1.222794in}{1.103099in}%
\pgfsys@useobject{currentmarker}{}%
\end{pgfscope}%
\end{pgfscope}%
\begin{pgfscope}%
\definecolor{textcolor}{rgb}{0.000000,0.000000,0.000000}%
\pgfsetstrokecolor{textcolor}%
\pgfsetfillcolor{textcolor}%
\pgftext[x=1.240155in, y=0.827344in, left, base,rotate=90.000000]{\color{textcolor}\rmfamily\fontsize{5.000000}{6.000000}\selectfont Afer}%
\end{pgfscope}%
\begin{pgfscope}%
\pgfsetbuttcap%
\pgfsetroundjoin%
\definecolor{currentfill}{rgb}{0.000000,0.000000,0.000000}%
\pgfsetfillcolor{currentfill}%
\pgfsetlinewidth{0.803000pt}%
\definecolor{currentstroke}{rgb}{0.000000,0.000000,0.000000}%
\pgfsetstrokecolor{currentstroke}%
\pgfsetdash{}{0pt}%
\pgfsys@defobject{currentmarker}{\pgfqpoint{0.000000in}{-0.048611in}}{\pgfqpoint{0.000000in}{0.000000in}}{%
\pgfpathmoveto{\pgfqpoint{0.000000in}{0.000000in}}%
\pgfpathlineto{\pgfqpoint{0.000000in}{-0.048611in}}%
\pgfusepath{stroke,fill}%
}%
\begin{pgfscope}%
\pgfsys@transformshift{1.354267in}{1.103099in}%
\pgfsys@useobject{currentmarker}{}%
\end{pgfscope}%
\end{pgfscope}%
\begin{pgfscope}%
\definecolor{textcolor}{rgb}{0.000000,0.000000,0.000000}%
\pgfsetstrokecolor{textcolor}%
\pgfsetfillcolor{textcolor}%
\pgftext[x=1.371628in, y=0.656335in, left, base,rotate=90.000000]{\color{textcolor}\rmfamily\fontsize{5.000000}{6.000000}\selectfont Afi Esca}%
\end{pgfscope}%
\begin{pgfscope}%
\pgfsetbuttcap%
\pgfsetroundjoin%
\definecolor{currentfill}{rgb}{0.000000,0.000000,0.000000}%
\pgfsetfillcolor{currentfill}%
\pgfsetlinewidth{0.803000pt}%
\definecolor{currentstroke}{rgb}{0.000000,0.000000,0.000000}%
\pgfsetstrokecolor{currentstroke}%
\pgfsetdash{}{0pt}%
\pgfsys@defobject{currentmarker}{\pgfqpoint{0.000000in}{-0.048611in}}{\pgfqpoint{0.000000in}{0.000000in}}{%
\pgfpathmoveto{\pgfqpoint{0.000000in}{0.000000in}}%
\pgfpathlineto{\pgfqpoint{0.000000in}{-0.048611in}}%
\pgfusepath{stroke,fill}%
}%
\begin{pgfscope}%
\pgfsys@transformshift{1.485740in}{1.103099in}%
\pgfsys@useobject{currentmarker}{}%
\end{pgfscope}%
\end{pgfscope}%
\begin{pgfscope}%
\definecolor{textcolor}{rgb}{0.000000,0.000000,0.000000}%
\pgfsetstrokecolor{textcolor}%
\pgfsetfillcolor{textcolor}%
\pgftext[x=1.503101in, y=0.255480in, left, base,rotate=90.000000]{\color{textcolor}\rmfamily\fontsize{5.000000}{6.000000}\selectfont Ag2r La Mondiale}%
\end{pgfscope}%
\begin{pgfscope}%
\pgfsetbuttcap%
\pgfsetroundjoin%
\definecolor{currentfill}{rgb}{0.000000,0.000000,0.000000}%
\pgfsetfillcolor{currentfill}%
\pgfsetlinewidth{0.803000pt}%
\definecolor{currentstroke}{rgb}{0.000000,0.000000,0.000000}%
\pgfsetstrokecolor{currentstroke}%
\pgfsetdash{}{0pt}%
\pgfsys@defobject{currentmarker}{\pgfqpoint{0.000000in}{-0.048611in}}{\pgfqpoint{0.000000in}{0.000000in}}{%
\pgfpathmoveto{\pgfqpoint{0.000000in}{0.000000in}}%
\pgfpathlineto{\pgfqpoint{0.000000in}{-0.048611in}}%
\pgfusepath{stroke,fill}%
}%
\begin{pgfscope}%
\pgfsys@transformshift{1.617213in}{1.103099in}%
\pgfsys@useobject{currentmarker}{}%
\end{pgfscope}%
\end{pgfscope}%
\begin{pgfscope}%
\definecolor{textcolor}{rgb}{0.000000,0.000000,0.000000}%
\pgfsetstrokecolor{textcolor}%
\pgfsetfillcolor{textcolor}%
\pgftext[x=1.634575in, y=0.712084in, left, base,rotate=90.000000]{\color{textcolor}\rmfamily\fontsize{5.000000}{6.000000}\selectfont Allianz}%
\end{pgfscope}%
\begin{pgfscope}%
\pgfsetbuttcap%
\pgfsetroundjoin%
\definecolor{currentfill}{rgb}{0.000000,0.000000,0.000000}%
\pgfsetfillcolor{currentfill}%
\pgfsetlinewidth{0.803000pt}%
\definecolor{currentstroke}{rgb}{0.000000,0.000000,0.000000}%
\pgfsetstrokecolor{currentstroke}%
\pgfsetdash{}{0pt}%
\pgfsys@defobject{currentmarker}{\pgfqpoint{0.000000in}{-0.048611in}}{\pgfqpoint{0.000000in}{0.000000in}}{%
\pgfpathmoveto{\pgfqpoint{0.000000in}{0.000000in}}%
\pgfpathlineto{\pgfqpoint{0.000000in}{-0.048611in}}%
\pgfusepath{stroke,fill}%
}%
\begin{pgfscope}%
\pgfsys@transformshift{1.748687in}{1.103099in}%
\pgfsys@useobject{currentmarker}{}%
\end{pgfscope}%
\end{pgfscope}%
\begin{pgfscope}%
\definecolor{textcolor}{rgb}{0.000000,0.000000,0.000000}%
\pgfsetstrokecolor{textcolor}%
\pgfsetfillcolor{textcolor}%
\pgftext[x=1.766048in, y=0.352221in, left, base,rotate=90.000000]{\color{textcolor}\rmfamily\fontsize{5.000000}{6.000000}\selectfont Assur Bon Plan}%
\end{pgfscope}%
\begin{pgfscope}%
\pgfsetbuttcap%
\pgfsetroundjoin%
\definecolor{currentfill}{rgb}{0.000000,0.000000,0.000000}%
\pgfsetfillcolor{currentfill}%
\pgfsetlinewidth{0.803000pt}%
\definecolor{currentstroke}{rgb}{0.000000,0.000000,0.000000}%
\pgfsetstrokecolor{currentstroke}%
\pgfsetdash{}{0pt}%
\pgfsys@defobject{currentmarker}{\pgfqpoint{0.000000in}{-0.048611in}}{\pgfqpoint{0.000000in}{0.000000in}}{%
\pgfpathmoveto{\pgfqpoint{0.000000in}{0.000000in}}%
\pgfpathlineto{\pgfqpoint{0.000000in}{-0.048611in}}%
\pgfusepath{stroke,fill}%
}%
\begin{pgfscope}%
\pgfsys@transformshift{1.880160in}{1.103099in}%
\pgfsys@useobject{currentmarker}{}%
\end{pgfscope}%
\end{pgfscope}%
\begin{pgfscope}%
\definecolor{textcolor}{rgb}{0.000000,0.000000,0.000000}%
\pgfsetstrokecolor{textcolor}%
\pgfsetfillcolor{textcolor}%
\pgftext[x=1.897521in, y=0.476452in, left, base,rotate=90.000000]{\color{textcolor}\rmfamily\fontsize{5.000000}{6.000000}\selectfont Assur O'Poil}%
\end{pgfscope}%
\begin{pgfscope}%
\pgfsetbuttcap%
\pgfsetroundjoin%
\definecolor{currentfill}{rgb}{0.000000,0.000000,0.000000}%
\pgfsetfillcolor{currentfill}%
\pgfsetlinewidth{0.803000pt}%
\definecolor{currentstroke}{rgb}{0.000000,0.000000,0.000000}%
\pgfsetstrokecolor{currentstroke}%
\pgfsetdash{}{0pt}%
\pgfsys@defobject{currentmarker}{\pgfqpoint{0.000000in}{-0.048611in}}{\pgfqpoint{0.000000in}{0.000000in}}{%
\pgfpathmoveto{\pgfqpoint{0.000000in}{0.000000in}}%
\pgfpathlineto{\pgfqpoint{0.000000in}{-0.048611in}}%
\pgfusepath{stroke,fill}%
}%
\begin{pgfscope}%
\pgfsys@transformshift{2.011633in}{1.103099in}%
\pgfsys@useobject{currentmarker}{}%
\end{pgfscope}%
\end{pgfscope}%
\begin{pgfscope}%
\definecolor{textcolor}{rgb}{0.000000,0.000000,0.000000}%
\pgfsetstrokecolor{textcolor}%
\pgfsetfillcolor{textcolor}%
\pgftext[x=2.028994in, y=0.497960in, left, base,rotate=90.000000]{\color{textcolor}\rmfamily\fontsize{5.000000}{6.000000}\selectfont AssurOnline}%
\end{pgfscope}%
\begin{pgfscope}%
\pgfsetbuttcap%
\pgfsetroundjoin%
\definecolor{currentfill}{rgb}{0.000000,0.000000,0.000000}%
\pgfsetfillcolor{currentfill}%
\pgfsetlinewidth{0.803000pt}%
\definecolor{currentstroke}{rgb}{0.000000,0.000000,0.000000}%
\pgfsetstrokecolor{currentstroke}%
\pgfsetdash{}{0pt}%
\pgfsys@defobject{currentmarker}{\pgfqpoint{0.000000in}{-0.048611in}}{\pgfqpoint{0.000000in}{0.000000in}}{%
\pgfpathmoveto{\pgfqpoint{0.000000in}{0.000000in}}%
\pgfpathlineto{\pgfqpoint{0.000000in}{-0.048611in}}%
\pgfusepath{stroke,fill}%
}%
\begin{pgfscope}%
\pgfsys@transformshift{2.143106in}{1.103099in}%
\pgfsys@useobject{currentmarker}{}%
\end{pgfscope}%
\end{pgfscope}%
\begin{pgfscope}%
\definecolor{textcolor}{rgb}{0.000000,0.000000,0.000000}%
\pgfsetstrokecolor{textcolor}%
\pgfsetfillcolor{textcolor}%
\pgftext[x=2.160467in, y=0.319621in, left, base,rotate=90.000000]{\color{textcolor}\rmfamily\fontsize{5.000000}{6.000000}\selectfont CNP Assurances}%
\end{pgfscope}%
\begin{pgfscope}%
\pgfsetbuttcap%
\pgfsetroundjoin%
\definecolor{currentfill}{rgb}{0.000000,0.000000,0.000000}%
\pgfsetfillcolor{currentfill}%
\pgfsetlinewidth{0.803000pt}%
\definecolor{currentstroke}{rgb}{0.000000,0.000000,0.000000}%
\pgfsetstrokecolor{currentstroke}%
\pgfsetdash{}{0pt}%
\pgfsys@defobject{currentmarker}{\pgfqpoint{0.000000in}{-0.048611in}}{\pgfqpoint{0.000000in}{0.000000in}}{%
\pgfpathmoveto{\pgfqpoint{0.000000in}{0.000000in}}%
\pgfpathlineto{\pgfqpoint{0.000000in}{-0.048611in}}%
\pgfusepath{stroke,fill}%
}%
\begin{pgfscope}%
\pgfsys@transformshift{2.274580in}{1.103099in}%
\pgfsys@useobject{currentmarker}{}%
\end{pgfscope}%
\end{pgfscope}%
\begin{pgfscope}%
\definecolor{textcolor}{rgb}{0.000000,0.000000,0.000000}%
\pgfsetstrokecolor{textcolor}%
\pgfsetfillcolor{textcolor}%
\pgftext[x=2.291941in, y=0.764747in, left, base,rotate=90.000000]{\color{textcolor}\rmfamily\fontsize{5.000000}{6.000000}\selectfont Carac}%
\end{pgfscope}%
\begin{pgfscope}%
\pgfsetbuttcap%
\pgfsetroundjoin%
\definecolor{currentfill}{rgb}{0.000000,0.000000,0.000000}%
\pgfsetfillcolor{currentfill}%
\pgfsetlinewidth{0.803000pt}%
\definecolor{currentstroke}{rgb}{0.000000,0.000000,0.000000}%
\pgfsetstrokecolor{currentstroke}%
\pgfsetdash{}{0pt}%
\pgfsys@defobject{currentmarker}{\pgfqpoint{0.000000in}{-0.048611in}}{\pgfqpoint{0.000000in}{0.000000in}}{%
\pgfpathmoveto{\pgfqpoint{0.000000in}{0.000000in}}%
\pgfpathlineto{\pgfqpoint{0.000000in}{-0.048611in}}%
\pgfusepath{stroke,fill}%
}%
\begin{pgfscope}%
\pgfsys@transformshift{2.406053in}{1.103099in}%
\pgfsys@useobject{currentmarker}{}%
\end{pgfscope}%
\end{pgfscope}%
\begin{pgfscope}%
\definecolor{textcolor}{rgb}{0.000000,0.000000,0.000000}%
\pgfsetstrokecolor{textcolor}%
\pgfsetfillcolor{textcolor}%
\pgftext[x=2.423414in, y=0.744010in, left, base,rotate=90.000000]{\color{textcolor}\rmfamily\fontsize{5.000000}{6.000000}\selectfont Cardif}%
\end{pgfscope}%
\begin{pgfscope}%
\pgfsetbuttcap%
\pgfsetroundjoin%
\definecolor{currentfill}{rgb}{0.000000,0.000000,0.000000}%
\pgfsetfillcolor{currentfill}%
\pgfsetlinewidth{0.803000pt}%
\definecolor{currentstroke}{rgb}{0.000000,0.000000,0.000000}%
\pgfsetstrokecolor{currentstroke}%
\pgfsetdash{}{0pt}%
\pgfsys@defobject{currentmarker}{\pgfqpoint{0.000000in}{-0.048611in}}{\pgfqpoint{0.000000in}{0.000000in}}{%
\pgfpathmoveto{\pgfqpoint{0.000000in}{0.000000in}}%
\pgfpathlineto{\pgfqpoint{0.000000in}{-0.048611in}}%
\pgfusepath{stroke,fill}%
}%
\begin{pgfscope}%
\pgfsys@transformshift{2.537526in}{1.103099in}%
\pgfsys@useobject{currentmarker}{}%
\end{pgfscope}%
\end{pgfscope}%
\begin{pgfscope}%
\definecolor{textcolor}{rgb}{0.000000,0.000000,0.000000}%
\pgfsetstrokecolor{textcolor}%
\pgfsetfillcolor{textcolor}%
\pgftext[x=2.554887in, y=0.194619in, left, base,rotate=90.000000]{\color{textcolor}\rmfamily\fontsize{5.000000}{6.000000}\selectfont Cegema Assurances}%
\end{pgfscope}%
\begin{pgfscope}%
\pgfsetbuttcap%
\pgfsetroundjoin%
\definecolor{currentfill}{rgb}{0.000000,0.000000,0.000000}%
\pgfsetfillcolor{currentfill}%
\pgfsetlinewidth{0.803000pt}%
\definecolor{currentstroke}{rgb}{0.000000,0.000000,0.000000}%
\pgfsetstrokecolor{currentstroke}%
\pgfsetdash{}{0pt}%
\pgfsys@defobject{currentmarker}{\pgfqpoint{0.000000in}{-0.048611in}}{\pgfqpoint{0.000000in}{0.000000in}}{%
\pgfpathmoveto{\pgfqpoint{0.000000in}{0.000000in}}%
\pgfpathlineto{\pgfqpoint{0.000000in}{-0.048611in}}%
\pgfusepath{stroke,fill}%
}%
\begin{pgfscope}%
\pgfsys@transformshift{2.668999in}{1.103099in}%
\pgfsys@useobject{currentmarker}{}%
\end{pgfscope}%
\end{pgfscope}%
\begin{pgfscope}%
\definecolor{textcolor}{rgb}{0.000000,0.000000,0.000000}%
\pgfsetstrokecolor{textcolor}%
\pgfsetfillcolor{textcolor}%
\pgftext[x=2.686360in, y=0.414048in, left, base,rotate=90.000000]{\color{textcolor}\rmfamily\fontsize{5.000000}{6.000000}\selectfont Crédit Mutuel}%
\end{pgfscope}%
\begin{pgfscope}%
\pgfsetbuttcap%
\pgfsetroundjoin%
\definecolor{currentfill}{rgb}{0.000000,0.000000,0.000000}%
\pgfsetfillcolor{currentfill}%
\pgfsetlinewidth{0.803000pt}%
\definecolor{currentstroke}{rgb}{0.000000,0.000000,0.000000}%
\pgfsetstrokecolor{currentstroke}%
\pgfsetdash{}{0pt}%
\pgfsys@defobject{currentmarker}{\pgfqpoint{0.000000in}{-0.048611in}}{\pgfqpoint{0.000000in}{0.000000in}}{%
\pgfpathmoveto{\pgfqpoint{0.000000in}{0.000000in}}%
\pgfpathlineto{\pgfqpoint{0.000000in}{-0.048611in}}%
\pgfusepath{stroke,fill}%
}%
\begin{pgfscope}%
\pgfsys@transformshift{2.800472in}{1.103099in}%
\pgfsys@useobject{currentmarker}{}%
\end{pgfscope}%
\end{pgfscope}%
\begin{pgfscope}%
\definecolor{textcolor}{rgb}{0.000000,0.000000,0.000000}%
\pgfsetstrokecolor{textcolor}%
\pgfsetfillcolor{textcolor}%
\pgftext[x=2.817833in, y=0.297052in, left, base,rotate=90.000000]{\color{textcolor}\rmfamily\fontsize{5.000000}{6.000000}\selectfont Direct Assurance}%
\end{pgfscope}%
\begin{pgfscope}%
\pgfsetbuttcap%
\pgfsetroundjoin%
\definecolor{currentfill}{rgb}{0.000000,0.000000,0.000000}%
\pgfsetfillcolor{currentfill}%
\pgfsetlinewidth{0.803000pt}%
\definecolor{currentstroke}{rgb}{0.000000,0.000000,0.000000}%
\pgfsetstrokecolor{currentstroke}%
\pgfsetdash{}{0pt}%
\pgfsys@defobject{currentmarker}{\pgfqpoint{0.000000in}{-0.048611in}}{\pgfqpoint{0.000000in}{0.000000in}}{%
\pgfpathmoveto{\pgfqpoint{0.000000in}{0.000000in}}%
\pgfpathlineto{\pgfqpoint{0.000000in}{-0.048611in}}%
\pgfusepath{stroke,fill}%
}%
\begin{pgfscope}%
\pgfsys@transformshift{2.931946in}{1.103099in}%
\pgfsys@useobject{currentmarker}{}%
\end{pgfscope}%
\end{pgfscope}%
\begin{pgfscope}%
\definecolor{textcolor}{rgb}{0.000000,0.000000,0.000000}%
\pgfsetstrokecolor{textcolor}%
\pgfsetfillcolor{textcolor}%
\pgftext[x=2.949307in, y=0.364568in, left, base,rotate=90.000000]{\color{textcolor}\rmfamily\fontsize{5.000000}{6.000000}\selectfont Eca Assurances}%
\end{pgfscope}%
\begin{pgfscope}%
\pgfsetbuttcap%
\pgfsetroundjoin%
\definecolor{currentfill}{rgb}{0.000000,0.000000,0.000000}%
\pgfsetfillcolor{currentfill}%
\pgfsetlinewidth{0.803000pt}%
\definecolor{currentstroke}{rgb}{0.000000,0.000000,0.000000}%
\pgfsetstrokecolor{currentstroke}%
\pgfsetdash{}{0pt}%
\pgfsys@defobject{currentmarker}{\pgfqpoint{0.000000in}{-0.048611in}}{\pgfqpoint{0.000000in}{0.000000in}}{%
\pgfpathmoveto{\pgfqpoint{0.000000in}{0.000000in}}%
\pgfpathlineto{\pgfqpoint{0.000000in}{-0.048611in}}%
\pgfusepath{stroke,fill}%
}%
\begin{pgfscope}%
\pgfsys@transformshift{3.063419in}{1.103099in}%
\pgfsys@useobject{currentmarker}{}%
\end{pgfscope}%
\end{pgfscope}%
\begin{pgfscope}%
\definecolor{textcolor}{rgb}{0.000000,0.000000,0.000000}%
\pgfsetstrokecolor{textcolor}%
\pgfsetfillcolor{textcolor}%
\pgftext[x=3.080780in, y=0.355405in, left, base,rotate=90.000000]{\color{textcolor}\rmfamily\fontsize{5.000000}{6.000000}\selectfont Euro-Assurance}%
\end{pgfscope}%
\begin{pgfscope}%
\pgfsetbuttcap%
\pgfsetroundjoin%
\definecolor{currentfill}{rgb}{0.000000,0.000000,0.000000}%
\pgfsetfillcolor{currentfill}%
\pgfsetlinewidth{0.803000pt}%
\definecolor{currentstroke}{rgb}{0.000000,0.000000,0.000000}%
\pgfsetstrokecolor{currentstroke}%
\pgfsetdash{}{0pt}%
\pgfsys@defobject{currentmarker}{\pgfqpoint{0.000000in}{-0.048611in}}{\pgfqpoint{0.000000in}{0.000000in}}{%
\pgfpathmoveto{\pgfqpoint{0.000000in}{0.000000in}}%
\pgfpathlineto{\pgfqpoint{0.000000in}{-0.048611in}}%
\pgfusepath{stroke,fill}%
}%
\begin{pgfscope}%
\pgfsys@transformshift{3.194892in}{1.103099in}%
\pgfsys@useobject{currentmarker}{}%
\end{pgfscope}%
\end{pgfscope}%
\begin{pgfscope}%
\definecolor{textcolor}{rgb}{0.000000,0.000000,0.000000}%
\pgfsetstrokecolor{textcolor}%
\pgfsetfillcolor{textcolor}%
\pgftext[x=3.212253in, y=0.720090in, left, base,rotate=90.000000]{\color{textcolor}\rmfamily\fontsize{5.000000}{6.000000}\selectfont Eurofil}%
\end{pgfscope}%
\begin{pgfscope}%
\pgfsetbuttcap%
\pgfsetroundjoin%
\definecolor{currentfill}{rgb}{0.000000,0.000000,0.000000}%
\pgfsetfillcolor{currentfill}%
\pgfsetlinewidth{0.803000pt}%
\definecolor{currentstroke}{rgb}{0.000000,0.000000,0.000000}%
\pgfsetstrokecolor{currentstroke}%
\pgfsetdash{}{0pt}%
\pgfsys@defobject{currentmarker}{\pgfqpoint{0.000000in}{-0.048611in}}{\pgfqpoint{0.000000in}{0.000000in}}{%
\pgfpathmoveto{\pgfqpoint{0.000000in}{0.000000in}}%
\pgfpathlineto{\pgfqpoint{0.000000in}{-0.048611in}}%
\pgfusepath{stroke,fill}%
}%
\begin{pgfscope}%
\pgfsys@transformshift{3.326365in}{1.103099in}%
\pgfsys@useobject{currentmarker}{}%
\end{pgfscope}%
\end{pgfscope}%
\begin{pgfscope}%
\definecolor{textcolor}{rgb}{0.000000,0.000000,0.000000}%
\pgfsetstrokecolor{textcolor}%
\pgfsetfillcolor{textcolor}%
\pgftext[x=3.343726in, y=0.791463in, left, base,rotate=90.000000]{\color{textcolor}\rmfamily\fontsize{5.000000}{6.000000}\selectfont GMF}%
\end{pgfscope}%
\begin{pgfscope}%
\pgfsetbuttcap%
\pgfsetroundjoin%
\definecolor{currentfill}{rgb}{0.000000,0.000000,0.000000}%
\pgfsetfillcolor{currentfill}%
\pgfsetlinewidth{0.803000pt}%
\definecolor{currentstroke}{rgb}{0.000000,0.000000,0.000000}%
\pgfsetstrokecolor{currentstroke}%
\pgfsetdash{}{0pt}%
\pgfsys@defobject{currentmarker}{\pgfqpoint{0.000000in}{-0.048611in}}{\pgfqpoint{0.000000in}{0.000000in}}{%
\pgfpathmoveto{\pgfqpoint{0.000000in}{0.000000in}}%
\pgfpathlineto{\pgfqpoint{0.000000in}{-0.048611in}}%
\pgfusepath{stroke,fill}%
}%
\begin{pgfscope}%
\pgfsys@transformshift{3.457838in}{1.103099in}%
\pgfsys@useobject{currentmarker}{}%
\end{pgfscope}%
\end{pgfscope}%
\begin{pgfscope}%
\definecolor{textcolor}{rgb}{0.000000,0.000000,0.000000}%
\pgfsetstrokecolor{textcolor}%
\pgfsetfillcolor{textcolor}%
\pgftext[x=3.475200in, y=0.834770in, left, base,rotate=90.000000]{\color{textcolor}\rmfamily\fontsize{5.000000}{6.000000}\selectfont Gan}%
\end{pgfscope}%
\begin{pgfscope}%
\pgfsetbuttcap%
\pgfsetroundjoin%
\definecolor{currentfill}{rgb}{0.000000,0.000000,0.000000}%
\pgfsetfillcolor{currentfill}%
\pgfsetlinewidth{0.803000pt}%
\definecolor{currentstroke}{rgb}{0.000000,0.000000,0.000000}%
\pgfsetstrokecolor{currentstroke}%
\pgfsetdash{}{0pt}%
\pgfsys@defobject{currentmarker}{\pgfqpoint{0.000000in}{-0.048611in}}{\pgfqpoint{0.000000in}{0.000000in}}{%
\pgfpathmoveto{\pgfqpoint{0.000000in}{0.000000in}}%
\pgfpathlineto{\pgfqpoint{0.000000in}{-0.048611in}}%
\pgfusepath{stroke,fill}%
}%
\begin{pgfscope}%
\pgfsys@transformshift{3.589312in}{1.103099in}%
\pgfsys@useobject{currentmarker}{}%
\end{pgfscope}%
\end{pgfscope}%
\begin{pgfscope}%
\definecolor{textcolor}{rgb}{0.000000,0.000000,0.000000}%
\pgfsetstrokecolor{textcolor}%
\pgfsetfillcolor{textcolor}%
\pgftext[x=3.606673in, y=0.656335in, left, base,rotate=90.000000]{\color{textcolor}\rmfamily\fontsize{5.000000}{6.000000}\selectfont Generali}%
\end{pgfscope}%
\begin{pgfscope}%
\pgfsetbuttcap%
\pgfsetroundjoin%
\definecolor{currentfill}{rgb}{0.000000,0.000000,0.000000}%
\pgfsetfillcolor{currentfill}%
\pgfsetlinewidth{0.803000pt}%
\definecolor{currentstroke}{rgb}{0.000000,0.000000,0.000000}%
\pgfsetstrokecolor{currentstroke}%
\pgfsetdash{}{0pt}%
\pgfsys@defobject{currentmarker}{\pgfqpoint{0.000000in}{-0.048611in}}{\pgfqpoint{0.000000in}{0.000000in}}{%
\pgfpathmoveto{\pgfqpoint{0.000000in}{0.000000in}}%
\pgfpathlineto{\pgfqpoint{0.000000in}{-0.048611in}}%
\pgfusepath{stroke,fill}%
}%
\begin{pgfscope}%
\pgfsys@transformshift{3.720785in}{1.103099in}%
\pgfsys@useobject{currentmarker}{}%
\end{pgfscope}%
\end{pgfscope}%
\begin{pgfscope}%
\definecolor{textcolor}{rgb}{0.000000,0.000000,0.000000}%
\pgfsetstrokecolor{textcolor}%
\pgfsetfillcolor{textcolor}%
\pgftext[x=3.738146in, y=0.574349in, left, base,rotate=90.000000]{\color{textcolor}\rmfamily\fontsize{5.000000}{6.000000}\selectfont Groupama}%
\end{pgfscope}%
\begin{pgfscope}%
\pgfsetbuttcap%
\pgfsetroundjoin%
\definecolor{currentfill}{rgb}{0.000000,0.000000,0.000000}%
\pgfsetfillcolor{currentfill}%
\pgfsetlinewidth{0.803000pt}%
\definecolor{currentstroke}{rgb}{0.000000,0.000000,0.000000}%
\pgfsetstrokecolor{currentstroke}%
\pgfsetdash{}{0pt}%
\pgfsys@defobject{currentmarker}{\pgfqpoint{0.000000in}{-0.048611in}}{\pgfqpoint{0.000000in}{0.000000in}}{%
\pgfpathmoveto{\pgfqpoint{0.000000in}{0.000000in}}%
\pgfpathlineto{\pgfqpoint{0.000000in}{-0.048611in}}%
\pgfusepath{stroke,fill}%
}%
\begin{pgfscope}%
\pgfsys@transformshift{3.852258in}{1.103099in}%
\pgfsys@useobject{currentmarker}{}%
\end{pgfscope}%
\end{pgfscope}%
\begin{pgfscope}%
\definecolor{textcolor}{rgb}{0.000000,0.000000,0.000000}%
\pgfsetstrokecolor{textcolor}%
\pgfsetfillcolor{textcolor}%
\pgftext[x=3.869619in, y=0.547344in, left, base,rotate=90.000000]{\color{textcolor}\rmfamily\fontsize{5.000000}{6.000000}\selectfont Génération}%
\end{pgfscope}%
\begin{pgfscope}%
\pgfsetbuttcap%
\pgfsetroundjoin%
\definecolor{currentfill}{rgb}{0.000000,0.000000,0.000000}%
\pgfsetfillcolor{currentfill}%
\pgfsetlinewidth{0.803000pt}%
\definecolor{currentstroke}{rgb}{0.000000,0.000000,0.000000}%
\pgfsetstrokecolor{currentstroke}%
\pgfsetdash{}{0pt}%
\pgfsys@defobject{currentmarker}{\pgfqpoint{0.000000in}{-0.048611in}}{\pgfqpoint{0.000000in}{0.000000in}}{%
\pgfpathmoveto{\pgfqpoint{0.000000in}{0.000000in}}%
\pgfpathlineto{\pgfqpoint{0.000000in}{-0.048611in}}%
\pgfusepath{stroke,fill}%
}%
\begin{pgfscope}%
\pgfsys@transformshift{3.983731in}{1.103099in}%
\pgfsys@useobject{currentmarker}{}%
\end{pgfscope}%
\end{pgfscope}%
\begin{pgfscope}%
\definecolor{textcolor}{rgb}{0.000000,0.000000,0.000000}%
\pgfsetstrokecolor{textcolor}%
\pgfsetfillcolor{textcolor}%
\pgftext[x=4.001092in, y=0.208990in, left, base,rotate=90.000000]{\color{textcolor}\rmfamily\fontsize{5.000000}{6.000000}\selectfont Harmonie Mutuelle}%
\end{pgfscope}%
\begin{pgfscope}%
\pgfsetbuttcap%
\pgfsetroundjoin%
\definecolor{currentfill}{rgb}{0.000000,0.000000,0.000000}%
\pgfsetfillcolor{currentfill}%
\pgfsetlinewidth{0.803000pt}%
\definecolor{currentstroke}{rgb}{0.000000,0.000000,0.000000}%
\pgfsetstrokecolor{currentstroke}%
\pgfsetdash{}{0pt}%
\pgfsys@defobject{currentmarker}{\pgfqpoint{0.000000in}{-0.048611in}}{\pgfqpoint{0.000000in}{0.000000in}}{%
\pgfpathmoveto{\pgfqpoint{0.000000in}{0.000000in}}%
\pgfpathlineto{\pgfqpoint{0.000000in}{-0.048611in}}%
\pgfusepath{stroke,fill}%
}%
\begin{pgfscope}%
\pgfsys@transformshift{4.115205in}{1.103099in}%
\pgfsys@useobject{currentmarker}{}%
\end{pgfscope}%
\end{pgfscope}%
\begin{pgfscope}%
\definecolor{textcolor}{rgb}{0.000000,0.000000,0.000000}%
\pgfsetstrokecolor{textcolor}%
\pgfsetfillcolor{textcolor}%
\pgftext[x=4.132566in, y=0.734750in, left, base,rotate=90.000000]{\color{textcolor}\rmfamily\fontsize{5.000000}{6.000000}\selectfont Hiscox}%
\end{pgfscope}%
\begin{pgfscope}%
\pgfsetbuttcap%
\pgfsetroundjoin%
\definecolor{currentfill}{rgb}{0.000000,0.000000,0.000000}%
\pgfsetfillcolor{currentfill}%
\pgfsetlinewidth{0.803000pt}%
\definecolor{currentstroke}{rgb}{0.000000,0.000000,0.000000}%
\pgfsetstrokecolor{currentstroke}%
\pgfsetdash{}{0pt}%
\pgfsys@defobject{currentmarker}{\pgfqpoint{0.000000in}{-0.048611in}}{\pgfqpoint{0.000000in}{0.000000in}}{%
\pgfpathmoveto{\pgfqpoint{0.000000in}{0.000000in}}%
\pgfpathlineto{\pgfqpoint{0.000000in}{-0.048611in}}%
\pgfusepath{stroke,fill}%
}%
\begin{pgfscope}%
\pgfsys@transformshift{4.246678in}{1.103099in}%
\pgfsys@useobject{currentmarker}{}%
\end{pgfscope}%
\end{pgfscope}%
\begin{pgfscope}%
\definecolor{textcolor}{rgb}{0.000000,0.000000,0.000000}%
\pgfsetstrokecolor{textcolor}%
\pgfsetfillcolor{textcolor}%
\pgftext[x=4.264039in, y=0.658554in, left, base,rotate=90.000000]{\color{textcolor}\rmfamily\fontsize{5.000000}{6.000000}\selectfont Intériale}%
\end{pgfscope}%
\begin{pgfscope}%
\pgfsetbuttcap%
\pgfsetroundjoin%
\definecolor{currentfill}{rgb}{0.000000,0.000000,0.000000}%
\pgfsetfillcolor{currentfill}%
\pgfsetlinewidth{0.803000pt}%
\definecolor{currentstroke}{rgb}{0.000000,0.000000,0.000000}%
\pgfsetstrokecolor{currentstroke}%
\pgfsetdash{}{0pt}%
\pgfsys@defobject{currentmarker}{\pgfqpoint{0.000000in}{-0.048611in}}{\pgfqpoint{0.000000in}{0.000000in}}{%
\pgfpathmoveto{\pgfqpoint{0.000000in}{0.000000in}}%
\pgfpathlineto{\pgfqpoint{0.000000in}{-0.048611in}}%
\pgfusepath{stroke,fill}%
}%
\begin{pgfscope}%
\pgfsys@transformshift{4.378151in}{1.103099in}%
\pgfsys@useobject{currentmarker}{}%
\end{pgfscope}%
\end{pgfscope}%
\begin{pgfscope}%
\definecolor{textcolor}{rgb}{0.000000,0.000000,0.000000}%
\pgfsetstrokecolor{textcolor}%
\pgfsetfillcolor{textcolor}%
\pgftext[x=4.395512in, y=0.208316in, left, base,rotate=90.000000]{\color{textcolor}\rmfamily\fontsize{5.000000}{6.000000}\selectfont L'olivier Assurance}%
\end{pgfscope}%
\begin{pgfscope}%
\pgfsetbuttcap%
\pgfsetroundjoin%
\definecolor{currentfill}{rgb}{0.000000,0.000000,0.000000}%
\pgfsetfillcolor{currentfill}%
\pgfsetlinewidth{0.803000pt}%
\definecolor{currentstroke}{rgb}{0.000000,0.000000,0.000000}%
\pgfsetstrokecolor{currentstroke}%
\pgfsetdash{}{0pt}%
\pgfsys@defobject{currentmarker}{\pgfqpoint{0.000000in}{-0.048611in}}{\pgfqpoint{0.000000in}{0.000000in}}{%
\pgfpathmoveto{\pgfqpoint{0.000000in}{0.000000in}}%
\pgfpathlineto{\pgfqpoint{0.000000in}{-0.048611in}}%
\pgfusepath{stroke,fill}%
}%
\begin{pgfscope}%
\pgfsys@transformshift{4.509624in}{1.103099in}%
\pgfsys@useobject{currentmarker}{}%
\end{pgfscope}%
\end{pgfscope}%
\begin{pgfscope}%
\definecolor{textcolor}{rgb}{0.000000,0.000000,0.000000}%
\pgfsetstrokecolor{textcolor}%
\pgfsetfillcolor{textcolor}%
\pgftext[x=4.526985in, y=0.823968in, left, base,rotate=90.000000]{\color{textcolor}\rmfamily\fontsize{5.000000}{6.000000}\selectfont LCL}%
\end{pgfscope}%
\begin{pgfscope}%
\pgfsetbuttcap%
\pgfsetroundjoin%
\definecolor{currentfill}{rgb}{0.000000,0.000000,0.000000}%
\pgfsetfillcolor{currentfill}%
\pgfsetlinewidth{0.803000pt}%
\definecolor{currentstroke}{rgb}{0.000000,0.000000,0.000000}%
\pgfsetstrokecolor{currentstroke}%
\pgfsetdash{}{0pt}%
\pgfsys@defobject{currentmarker}{\pgfqpoint{0.000000in}{-0.048611in}}{\pgfqpoint{0.000000in}{0.000000in}}{%
\pgfpathmoveto{\pgfqpoint{0.000000in}{0.000000in}}%
\pgfpathlineto{\pgfqpoint{0.000000in}{-0.048611in}}%
\pgfusepath{stroke,fill}%
}%
\begin{pgfscope}%
\pgfsys@transformshift{4.641097in}{1.103099in}%
\pgfsys@useobject{currentmarker}{}%
\end{pgfscope}%
\end{pgfscope}%
\begin{pgfscope}%
\definecolor{textcolor}{rgb}{0.000000,0.000000,0.000000}%
\pgfsetstrokecolor{textcolor}%
\pgfsetfillcolor{textcolor}%
\pgftext[x=4.658458in, y=0.727033in, left, base,rotate=90.000000]{\color{textcolor}\rmfamily\fontsize{5.000000}{6.000000}\selectfont MAAF}%
\end{pgfscope}%
\begin{pgfscope}%
\pgfsetbuttcap%
\pgfsetroundjoin%
\definecolor{currentfill}{rgb}{0.000000,0.000000,0.000000}%
\pgfsetfillcolor{currentfill}%
\pgfsetlinewidth{0.803000pt}%
\definecolor{currentstroke}{rgb}{0.000000,0.000000,0.000000}%
\pgfsetstrokecolor{currentstroke}%
\pgfsetdash{}{0pt}%
\pgfsys@defobject{currentmarker}{\pgfqpoint{0.000000in}{-0.048611in}}{\pgfqpoint{0.000000in}{0.000000in}}{%
\pgfpathmoveto{\pgfqpoint{0.000000in}{0.000000in}}%
\pgfpathlineto{\pgfqpoint{0.000000in}{-0.048611in}}%
\pgfusepath{stroke,fill}%
}%
\begin{pgfscope}%
\pgfsys@transformshift{4.772571in}{1.103099in}%
\pgfsys@useobject{currentmarker}{}%
\end{pgfscope}%
\end{pgfscope}%
\begin{pgfscope}%
\definecolor{textcolor}{rgb}{0.000000,0.000000,0.000000}%
\pgfsetstrokecolor{textcolor}%
\pgfsetfillcolor{textcolor}%
\pgftext[x=4.789932in, y=0.696652in, left, base,rotate=90.000000]{\color{textcolor}\rmfamily\fontsize{5.000000}{6.000000}\selectfont MACIF}%
\end{pgfscope}%
\begin{pgfscope}%
\pgfsetbuttcap%
\pgfsetroundjoin%
\definecolor{currentfill}{rgb}{0.000000,0.000000,0.000000}%
\pgfsetfillcolor{currentfill}%
\pgfsetlinewidth{0.803000pt}%
\definecolor{currentstroke}{rgb}{0.000000,0.000000,0.000000}%
\pgfsetstrokecolor{currentstroke}%
\pgfsetdash{}{0pt}%
\pgfsys@defobject{currentmarker}{\pgfqpoint{0.000000in}{-0.048611in}}{\pgfqpoint{0.000000in}{0.000000in}}{%
\pgfpathmoveto{\pgfqpoint{0.000000in}{0.000000in}}%
\pgfpathlineto{\pgfqpoint{0.000000in}{-0.048611in}}%
\pgfusepath{stroke,fill}%
}%
\begin{pgfscope}%
\pgfsys@transformshift{4.904044in}{1.103099in}%
\pgfsys@useobject{currentmarker}{}%
\end{pgfscope}%
\end{pgfscope}%
\begin{pgfscope}%
\definecolor{textcolor}{rgb}{0.000000,0.000000,0.000000}%
\pgfsetstrokecolor{textcolor}%
\pgfsetfillcolor{textcolor}%
\pgftext[x=4.921405in, y=0.760792in, left, base,rotate=90.000000]{\color{textcolor}\rmfamily\fontsize{5.000000}{6.000000}\selectfont MAIF}%
\end{pgfscope}%
\begin{pgfscope}%
\pgfsetbuttcap%
\pgfsetroundjoin%
\definecolor{currentfill}{rgb}{0.000000,0.000000,0.000000}%
\pgfsetfillcolor{currentfill}%
\pgfsetlinewidth{0.803000pt}%
\definecolor{currentstroke}{rgb}{0.000000,0.000000,0.000000}%
\pgfsetstrokecolor{currentstroke}%
\pgfsetdash{}{0pt}%
\pgfsys@defobject{currentmarker}{\pgfqpoint{0.000000in}{-0.048611in}}{\pgfqpoint{0.000000in}{0.000000in}}{%
\pgfpathmoveto{\pgfqpoint{0.000000in}{0.000000in}}%
\pgfpathlineto{\pgfqpoint{0.000000in}{-0.048611in}}%
\pgfusepath{stroke,fill}%
}%
\begin{pgfscope}%
\pgfsys@transformshift{5.035517in}{1.103099in}%
\pgfsys@useobject{currentmarker}{}%
\end{pgfscope}%
\end{pgfscope}%
\begin{pgfscope}%
\definecolor{textcolor}{rgb}{0.000000,0.000000,0.000000}%
\pgfsetstrokecolor{textcolor}%
\pgfsetfillcolor{textcolor}%
\pgftext[x=5.052878in, y=0.789052in, left, base,rotate=90.000000]{\color{textcolor}\rmfamily\fontsize{5.000000}{6.000000}\selectfont MGP}%
\end{pgfscope}%
\begin{pgfscope}%
\pgfsetbuttcap%
\pgfsetroundjoin%
\definecolor{currentfill}{rgb}{0.000000,0.000000,0.000000}%
\pgfsetfillcolor{currentfill}%
\pgfsetlinewidth{0.803000pt}%
\definecolor{currentstroke}{rgb}{0.000000,0.000000,0.000000}%
\pgfsetstrokecolor{currentstroke}%
\pgfsetdash{}{0pt}%
\pgfsys@defobject{currentmarker}{\pgfqpoint{0.000000in}{-0.048611in}}{\pgfqpoint{0.000000in}{0.000000in}}{%
\pgfpathmoveto{\pgfqpoint{0.000000in}{0.000000in}}%
\pgfpathlineto{\pgfqpoint{0.000000in}{-0.048611in}}%
\pgfusepath{stroke,fill}%
}%
\begin{pgfscope}%
\pgfsys@transformshift{5.166990in}{1.103099in}%
\pgfsys@useobject{currentmarker}{}%
\end{pgfscope}%
\end{pgfscope}%
\begin{pgfscope}%
\definecolor{textcolor}{rgb}{0.000000,0.000000,0.000000}%
\pgfsetstrokecolor{textcolor}%
\pgfsetfillcolor{textcolor}%
\pgftext[x=5.184351in, y=0.772655in, left, base,rotate=90.000000]{\color{textcolor}\rmfamily\fontsize{5.000000}{6.000000}\selectfont MMA}%
\end{pgfscope}%
\begin{pgfscope}%
\pgfsetbuttcap%
\pgfsetroundjoin%
\definecolor{currentfill}{rgb}{0.000000,0.000000,0.000000}%
\pgfsetfillcolor{currentfill}%
\pgfsetlinewidth{0.803000pt}%
\definecolor{currentstroke}{rgb}{0.000000,0.000000,0.000000}%
\pgfsetstrokecolor{currentstroke}%
\pgfsetdash{}{0pt}%
\pgfsys@defobject{currentmarker}{\pgfqpoint{0.000000in}{-0.048611in}}{\pgfqpoint{0.000000in}{0.000000in}}{%
\pgfpathmoveto{\pgfqpoint{0.000000in}{0.000000in}}%
\pgfpathlineto{\pgfqpoint{0.000000in}{-0.048611in}}%
\pgfusepath{stroke,fill}%
}%
\begin{pgfscope}%
\pgfsys@transformshift{5.298463in}{1.103099in}%
\pgfsys@useobject{currentmarker}{}%
\end{pgfscope}%
\end{pgfscope}%
\begin{pgfscope}%
\definecolor{textcolor}{rgb}{0.000000,0.000000,0.000000}%
\pgfsetstrokecolor{textcolor}%
\pgfsetfillcolor{textcolor}%
\pgftext[x=5.315825in, y=0.626241in, left, base,rotate=90.000000]{\color{textcolor}\rmfamily\fontsize{5.000000}{6.000000}\selectfont Magnolia}%
\end{pgfscope}%
\begin{pgfscope}%
\pgfsetbuttcap%
\pgfsetroundjoin%
\definecolor{currentfill}{rgb}{0.000000,0.000000,0.000000}%
\pgfsetfillcolor{currentfill}%
\pgfsetlinewidth{0.803000pt}%
\definecolor{currentstroke}{rgb}{0.000000,0.000000,0.000000}%
\pgfsetstrokecolor{currentstroke}%
\pgfsetdash{}{0pt}%
\pgfsys@defobject{currentmarker}{\pgfqpoint{0.000000in}{-0.048611in}}{\pgfqpoint{0.000000in}{0.000000in}}{%
\pgfpathmoveto{\pgfqpoint{0.000000in}{0.000000in}}%
\pgfpathlineto{\pgfqpoint{0.000000in}{-0.048611in}}%
\pgfusepath{stroke,fill}%
}%
\begin{pgfscope}%
\pgfsys@transformshift{5.429937in}{1.103099in}%
\pgfsys@useobject{currentmarker}{}%
\end{pgfscope}%
\end{pgfscope}%
\begin{pgfscope}%
\definecolor{textcolor}{rgb}{0.000000,0.000000,0.000000}%
\pgfsetstrokecolor{textcolor}%
\pgfsetfillcolor{textcolor}%
\pgftext[x=5.447298in, y=0.250947in, left, base,rotate=90.000000]{\color{textcolor}\rmfamily\fontsize{5.000000}{6.000000}\selectfont Malakoff Humanis}%
\end{pgfscope}%
\begin{pgfscope}%
\pgfsetbuttcap%
\pgfsetroundjoin%
\definecolor{currentfill}{rgb}{0.000000,0.000000,0.000000}%
\pgfsetfillcolor{currentfill}%
\pgfsetlinewidth{0.803000pt}%
\definecolor{currentstroke}{rgb}{0.000000,0.000000,0.000000}%
\pgfsetstrokecolor{currentstroke}%
\pgfsetdash{}{0pt}%
\pgfsys@defobject{currentmarker}{\pgfqpoint{0.000000in}{-0.048611in}}{\pgfqpoint{0.000000in}{0.000000in}}{%
\pgfpathmoveto{\pgfqpoint{0.000000in}{0.000000in}}%
\pgfpathlineto{\pgfqpoint{0.000000in}{-0.048611in}}%
\pgfusepath{stroke,fill}%
}%
\begin{pgfscope}%
\pgfsys@transformshift{5.561410in}{1.103099in}%
\pgfsys@useobject{currentmarker}{}%
\end{pgfscope}%
\end{pgfscope}%
\begin{pgfscope}%
\definecolor{textcolor}{rgb}{0.000000,0.000000,0.000000}%
\pgfsetstrokecolor{textcolor}%
\pgfsetfillcolor{textcolor}%
\pgftext[x=5.578771in, y=0.776706in, left, base,rotate=90.000000]{\color{textcolor}\rmfamily\fontsize{5.000000}{6.000000}\selectfont Mapa}%
\end{pgfscope}%
\begin{pgfscope}%
\pgfsetbuttcap%
\pgfsetroundjoin%
\definecolor{currentfill}{rgb}{0.000000,0.000000,0.000000}%
\pgfsetfillcolor{currentfill}%
\pgfsetlinewidth{0.803000pt}%
\definecolor{currentstroke}{rgb}{0.000000,0.000000,0.000000}%
\pgfsetstrokecolor{currentstroke}%
\pgfsetdash{}{0pt}%
\pgfsys@defobject{currentmarker}{\pgfqpoint{0.000000in}{-0.048611in}}{\pgfqpoint{0.000000in}{0.000000in}}{%
\pgfpathmoveto{\pgfqpoint{0.000000in}{0.000000in}}%
\pgfpathlineto{\pgfqpoint{0.000000in}{-0.048611in}}%
\pgfusepath{stroke,fill}%
}%
\begin{pgfscope}%
\pgfsys@transformshift{5.692883in}{1.103099in}%
\pgfsys@useobject{currentmarker}{}%
\end{pgfscope}%
\end{pgfscope}%
\begin{pgfscope}%
\definecolor{textcolor}{rgb}{0.000000,0.000000,0.000000}%
\pgfsetstrokecolor{textcolor}%
\pgfsetfillcolor{textcolor}%
\pgftext[x=5.710244in, y=0.674950in, left, base,rotate=90.000000]{\color{textcolor}\rmfamily\fontsize{5.000000}{6.000000}\selectfont Matmut}%
\end{pgfscope}%
\begin{pgfscope}%
\pgfsetbuttcap%
\pgfsetroundjoin%
\definecolor{currentfill}{rgb}{0.000000,0.000000,0.000000}%
\pgfsetfillcolor{currentfill}%
\pgfsetlinewidth{0.803000pt}%
\definecolor{currentstroke}{rgb}{0.000000,0.000000,0.000000}%
\pgfsetstrokecolor{currentstroke}%
\pgfsetdash{}{0pt}%
\pgfsys@defobject{currentmarker}{\pgfqpoint{0.000000in}{-0.048611in}}{\pgfqpoint{0.000000in}{0.000000in}}{%
\pgfpathmoveto{\pgfqpoint{0.000000in}{0.000000in}}%
\pgfpathlineto{\pgfqpoint{0.000000in}{-0.048611in}}%
\pgfusepath{stroke,fill}%
}%
\begin{pgfscope}%
\pgfsys@transformshift{5.824356in}{1.103099in}%
\pgfsys@useobject{currentmarker}{}%
\end{pgfscope}%
\end{pgfscope}%
\begin{pgfscope}%
\definecolor{textcolor}{rgb}{0.000000,0.000000,0.000000}%
\pgfsetstrokecolor{textcolor}%
\pgfsetfillcolor{textcolor}%
\pgftext[x=5.841717in, y=0.720765in, left, base,rotate=90.000000]{\color{textcolor}\rmfamily\fontsize{5.000000}{6.000000}\selectfont Mercer}%
\end{pgfscope}%
\begin{pgfscope}%
\pgfsetbuttcap%
\pgfsetroundjoin%
\definecolor{currentfill}{rgb}{0.000000,0.000000,0.000000}%
\pgfsetfillcolor{currentfill}%
\pgfsetlinewidth{0.803000pt}%
\definecolor{currentstroke}{rgb}{0.000000,0.000000,0.000000}%
\pgfsetstrokecolor{currentstroke}%
\pgfsetdash{}{0pt}%
\pgfsys@defobject{currentmarker}{\pgfqpoint{0.000000in}{-0.048611in}}{\pgfqpoint{0.000000in}{0.000000in}}{%
\pgfpathmoveto{\pgfqpoint{0.000000in}{0.000000in}}%
\pgfpathlineto{\pgfqpoint{0.000000in}{-0.048611in}}%
\pgfusepath{stroke,fill}%
}%
\begin{pgfscope}%
\pgfsys@transformshift{5.955830in}{1.103099in}%
\pgfsys@useobject{currentmarker}{}%
\end{pgfscope}%
\end{pgfscope}%
\begin{pgfscope}%
\definecolor{textcolor}{rgb}{0.000000,0.000000,0.000000}%
\pgfsetstrokecolor{textcolor}%
\pgfsetfillcolor{textcolor}%
\pgftext[x=5.973191in, y=0.684789in, left, base,rotate=90.000000]{\color{textcolor}\rmfamily\fontsize{5.000000}{6.000000}\selectfont MetLife}%
\end{pgfscope}%
\begin{pgfscope}%
\pgfsetbuttcap%
\pgfsetroundjoin%
\definecolor{currentfill}{rgb}{0.000000,0.000000,0.000000}%
\pgfsetfillcolor{currentfill}%
\pgfsetlinewidth{0.803000pt}%
\definecolor{currentstroke}{rgb}{0.000000,0.000000,0.000000}%
\pgfsetstrokecolor{currentstroke}%
\pgfsetdash{}{0pt}%
\pgfsys@defobject{currentmarker}{\pgfqpoint{0.000000in}{-0.048611in}}{\pgfqpoint{0.000000in}{0.000000in}}{%
\pgfpathmoveto{\pgfqpoint{0.000000in}{0.000000in}}%
\pgfpathlineto{\pgfqpoint{0.000000in}{-0.048611in}}%
\pgfusepath{stroke,fill}%
}%
\begin{pgfscope}%
\pgfsys@transformshift{6.087303in}{1.103099in}%
\pgfsys@useobject{currentmarker}{}%
\end{pgfscope}%
\end{pgfscope}%
\begin{pgfscope}%
\definecolor{textcolor}{rgb}{0.000000,0.000000,0.000000}%
\pgfsetstrokecolor{textcolor}%
\pgfsetfillcolor{textcolor}%
\pgftext[x=6.104664in, y=0.781529in, left, base,rotate=90.000000]{\color{textcolor}\rmfamily\fontsize{5.000000}{6.000000}\selectfont Mgen}%
\end{pgfscope}%
\begin{pgfscope}%
\pgfsetbuttcap%
\pgfsetroundjoin%
\definecolor{currentfill}{rgb}{0.000000,0.000000,0.000000}%
\pgfsetfillcolor{currentfill}%
\pgfsetlinewidth{0.803000pt}%
\definecolor{currentstroke}{rgb}{0.000000,0.000000,0.000000}%
\pgfsetstrokecolor{currentstroke}%
\pgfsetdash{}{0pt}%
\pgfsys@defobject{currentmarker}{\pgfqpoint{0.000000in}{-0.048611in}}{\pgfqpoint{0.000000in}{0.000000in}}{%
\pgfpathmoveto{\pgfqpoint{0.000000in}{0.000000in}}%
\pgfpathlineto{\pgfqpoint{0.000000in}{-0.048611in}}%
\pgfusepath{stroke,fill}%
}%
\begin{pgfscope}%
\pgfsys@transformshift{6.218776in}{1.103099in}%
\pgfsys@useobject{currentmarker}{}%
\end{pgfscope}%
\end{pgfscope}%
\begin{pgfscope}%
\definecolor{textcolor}{rgb}{0.000000,0.000000,0.000000}%
\pgfsetstrokecolor{textcolor}%
\pgfsetfillcolor{textcolor}%
\pgftext[x=6.236137in, y=0.100000in, left, base,rotate=90.000000]{\color{textcolor}\rmfamily\fontsize{5.000000}{6.000000}\selectfont Mutuelle des Motards}%
\end{pgfscope}%
\begin{pgfscope}%
\pgfsetbuttcap%
\pgfsetroundjoin%
\definecolor{currentfill}{rgb}{0.000000,0.000000,0.000000}%
\pgfsetfillcolor{currentfill}%
\pgfsetlinewidth{0.803000pt}%
\definecolor{currentstroke}{rgb}{0.000000,0.000000,0.000000}%
\pgfsetstrokecolor{currentstroke}%
\pgfsetdash{}{0pt}%
\pgfsys@defobject{currentmarker}{\pgfqpoint{0.000000in}{-0.048611in}}{\pgfqpoint{0.000000in}{0.000000in}}{%
\pgfpathmoveto{\pgfqpoint{0.000000in}{0.000000in}}%
\pgfpathlineto{\pgfqpoint{0.000000in}{-0.048611in}}%
\pgfusepath{stroke,fill}%
}%
\begin{pgfscope}%
\pgfsys@transformshift{6.350249in}{1.103099in}%
\pgfsys@useobject{currentmarker}{}%
\end{pgfscope}%
\end{pgfscope}%
\begin{pgfscope}%
\definecolor{textcolor}{rgb}{0.000000,0.000000,0.000000}%
\pgfsetstrokecolor{textcolor}%
\pgfsetfillcolor{textcolor}%
\pgftext[x=6.367610in, y=0.388488in, left, base,rotate=90.000000]{\color{textcolor}\rmfamily\fontsize{5.000000}{6.000000}\selectfont Néoliane Santé}%
\end{pgfscope}%
\begin{pgfscope}%
\pgfsetbuttcap%
\pgfsetroundjoin%
\definecolor{currentfill}{rgb}{0.000000,0.000000,0.000000}%
\pgfsetfillcolor{currentfill}%
\pgfsetlinewidth{0.803000pt}%
\definecolor{currentstroke}{rgb}{0.000000,0.000000,0.000000}%
\pgfsetstrokecolor{currentstroke}%
\pgfsetdash{}{0pt}%
\pgfsys@defobject{currentmarker}{\pgfqpoint{0.000000in}{-0.048611in}}{\pgfqpoint{0.000000in}{0.000000in}}{%
\pgfpathmoveto{\pgfqpoint{0.000000in}{0.000000in}}%
\pgfpathlineto{\pgfqpoint{0.000000in}{-0.048611in}}%
\pgfusepath{stroke,fill}%
}%
\begin{pgfscope}%
\pgfsys@transformshift{6.481722in}{1.103099in}%
\pgfsys@useobject{currentmarker}{}%
\end{pgfscope}%
\end{pgfscope}%
\begin{pgfscope}%
\definecolor{textcolor}{rgb}{0.000000,0.000000,0.000000}%
\pgfsetstrokecolor{textcolor}%
\pgfsetfillcolor{textcolor}%
\pgftext[x=6.499083in, y=0.680062in, left, base,rotate=90.000000]{\color{textcolor}\rmfamily\fontsize{5.000000}{6.000000}\selectfont Pacifica}%
\end{pgfscope}%
\begin{pgfscope}%
\pgfsetbuttcap%
\pgfsetroundjoin%
\definecolor{currentfill}{rgb}{0.000000,0.000000,0.000000}%
\pgfsetfillcolor{currentfill}%
\pgfsetlinewidth{0.803000pt}%
\definecolor{currentstroke}{rgb}{0.000000,0.000000,0.000000}%
\pgfsetstrokecolor{currentstroke}%
\pgfsetdash{}{0pt}%
\pgfsys@defobject{currentmarker}{\pgfqpoint{0.000000in}{-0.048611in}}{\pgfqpoint{0.000000in}{0.000000in}}{%
\pgfpathmoveto{\pgfqpoint{0.000000in}{0.000000in}}%
\pgfpathlineto{\pgfqpoint{0.000000in}{-0.048611in}}%
\pgfusepath{stroke,fill}%
}%
\begin{pgfscope}%
\pgfsys@transformshift{6.613196in}{1.103099in}%
\pgfsys@useobject{currentmarker}{}%
\end{pgfscope}%
\end{pgfscope}%
\begin{pgfscope}%
\definecolor{textcolor}{rgb}{0.000000,0.000000,0.000000}%
\pgfsetstrokecolor{textcolor}%
\pgfsetfillcolor{textcolor}%
\pgftext[x=6.630557in, y=0.237252in, left, base,rotate=90.000000]{\color{textcolor}\rmfamily\fontsize{5.000000}{6.000000}\selectfont Peyrac Assurances}%
\end{pgfscope}%
\begin{pgfscope}%
\pgfsetbuttcap%
\pgfsetroundjoin%
\definecolor{currentfill}{rgb}{0.000000,0.000000,0.000000}%
\pgfsetfillcolor{currentfill}%
\pgfsetlinewidth{0.803000pt}%
\definecolor{currentstroke}{rgb}{0.000000,0.000000,0.000000}%
\pgfsetstrokecolor{currentstroke}%
\pgfsetdash{}{0pt}%
\pgfsys@defobject{currentmarker}{\pgfqpoint{0.000000in}{-0.048611in}}{\pgfqpoint{0.000000in}{0.000000in}}{%
\pgfpathmoveto{\pgfqpoint{0.000000in}{0.000000in}}%
\pgfpathlineto{\pgfqpoint{0.000000in}{-0.048611in}}%
\pgfusepath{stroke,fill}%
}%
\begin{pgfscope}%
\pgfsys@transformshift{6.744669in}{1.103099in}%
\pgfsys@useobject{currentmarker}{}%
\end{pgfscope}%
\end{pgfscope}%
\begin{pgfscope}%
\definecolor{textcolor}{rgb}{0.000000,0.000000,0.000000}%
\pgfsetstrokecolor{textcolor}%
\pgfsetfillcolor{textcolor}%
\pgftext[x=6.762030in, y=0.649487in, left, base,rotate=90.000000]{\color{textcolor}\rmfamily\fontsize{5.000000}{6.000000}\selectfont Santiane}%
\end{pgfscope}%
\begin{pgfscope}%
\pgfsetbuttcap%
\pgfsetroundjoin%
\definecolor{currentfill}{rgb}{0.000000,0.000000,0.000000}%
\pgfsetfillcolor{currentfill}%
\pgfsetlinewidth{0.803000pt}%
\definecolor{currentstroke}{rgb}{0.000000,0.000000,0.000000}%
\pgfsetstrokecolor{currentstroke}%
\pgfsetdash{}{0pt}%
\pgfsys@defobject{currentmarker}{\pgfqpoint{0.000000in}{-0.048611in}}{\pgfqpoint{0.000000in}{0.000000in}}{%
\pgfpathmoveto{\pgfqpoint{0.000000in}{0.000000in}}%
\pgfpathlineto{\pgfqpoint{0.000000in}{-0.048611in}}%
\pgfusepath{stroke,fill}%
}%
\begin{pgfscope}%
\pgfsys@transformshift{6.876142in}{1.103099in}%
\pgfsys@useobject{currentmarker}{}%
\end{pgfscope}%
\end{pgfscope}%
\begin{pgfscope}%
\definecolor{textcolor}{rgb}{0.000000,0.000000,0.000000}%
\pgfsetstrokecolor{textcolor}%
\pgfsetfillcolor{textcolor}%
\pgftext[x=6.893503in, y=0.635887in, left, base,rotate=90.000000]{\color{textcolor}\rmfamily\fontsize{5.000000}{6.000000}\selectfont SantéVet}%
\end{pgfscope}%
\begin{pgfscope}%
\pgfsetbuttcap%
\pgfsetroundjoin%
\definecolor{currentfill}{rgb}{0.000000,0.000000,0.000000}%
\pgfsetfillcolor{currentfill}%
\pgfsetlinewidth{0.803000pt}%
\definecolor{currentstroke}{rgb}{0.000000,0.000000,0.000000}%
\pgfsetstrokecolor{currentstroke}%
\pgfsetdash{}{0pt}%
\pgfsys@defobject{currentmarker}{\pgfqpoint{0.000000in}{-0.048611in}}{\pgfqpoint{0.000000in}{0.000000in}}{%
\pgfpathmoveto{\pgfqpoint{0.000000in}{0.000000in}}%
\pgfpathlineto{\pgfqpoint{0.000000in}{-0.048611in}}%
\pgfusepath{stroke,fill}%
}%
\begin{pgfscope}%
\pgfsys@transformshift{7.007615in}{1.103099in}%
\pgfsys@useobject{currentmarker}{}%
\end{pgfscope}%
\end{pgfscope}%
\begin{pgfscope}%
\definecolor{textcolor}{rgb}{0.000000,0.000000,0.000000}%
\pgfsetstrokecolor{textcolor}%
\pgfsetfillcolor{textcolor}%
\pgftext[x=7.024976in, y=0.830333in, left, base,rotate=90.000000]{\color{textcolor}\rmfamily\fontsize{5.000000}{6.000000}\selectfont Sma}%
\end{pgfscope}%
\begin{pgfscope}%
\pgfsetbuttcap%
\pgfsetroundjoin%
\definecolor{currentfill}{rgb}{0.000000,0.000000,0.000000}%
\pgfsetfillcolor{currentfill}%
\pgfsetlinewidth{0.803000pt}%
\definecolor{currentstroke}{rgb}{0.000000,0.000000,0.000000}%
\pgfsetstrokecolor{currentstroke}%
\pgfsetdash{}{0pt}%
\pgfsys@defobject{currentmarker}{\pgfqpoint{0.000000in}{-0.048611in}}{\pgfqpoint{0.000000in}{0.000000in}}{%
\pgfpathmoveto{\pgfqpoint{0.000000in}{0.000000in}}%
\pgfpathlineto{\pgfqpoint{0.000000in}{-0.048611in}}%
\pgfusepath{stroke,fill}%
}%
\begin{pgfscope}%
\pgfsys@transformshift{7.139088in}{1.103099in}%
\pgfsys@useobject{currentmarker}{}%
\end{pgfscope}%
\end{pgfscope}%
\begin{pgfscope}%
\definecolor{textcolor}{rgb}{0.000000,0.000000,0.000000}%
\pgfsetstrokecolor{textcolor}%
\pgfsetfillcolor{textcolor}%
\pgftext[x=7.156450in, y=0.675046in, left, base,rotate=90.000000]{\color{textcolor}\rmfamily\fontsize{5.000000}{6.000000}\selectfont Sogecap}%
\end{pgfscope}%
\begin{pgfscope}%
\pgfsetbuttcap%
\pgfsetroundjoin%
\definecolor{currentfill}{rgb}{0.000000,0.000000,0.000000}%
\pgfsetfillcolor{currentfill}%
\pgfsetlinewidth{0.803000pt}%
\definecolor{currentstroke}{rgb}{0.000000,0.000000,0.000000}%
\pgfsetstrokecolor{currentstroke}%
\pgfsetdash{}{0pt}%
\pgfsys@defobject{currentmarker}{\pgfqpoint{0.000000in}{-0.048611in}}{\pgfqpoint{0.000000in}{0.000000in}}{%
\pgfpathmoveto{\pgfqpoint{0.000000in}{0.000000in}}%
\pgfpathlineto{\pgfqpoint{0.000000in}{-0.048611in}}%
\pgfusepath{stroke,fill}%
}%
\begin{pgfscope}%
\pgfsys@transformshift{7.270562in}{1.103099in}%
\pgfsys@useobject{currentmarker}{}%
\end{pgfscope}%
\end{pgfscope}%
\begin{pgfscope}%
\definecolor{textcolor}{rgb}{0.000000,0.000000,0.000000}%
\pgfsetstrokecolor{textcolor}%
\pgfsetfillcolor{textcolor}%
\pgftext[x=7.287923in, y=0.650933in, left, base,rotate=90.000000]{\color{textcolor}\rmfamily\fontsize{5.000000}{6.000000}\selectfont Sogessur}%
\end{pgfscope}%
\begin{pgfscope}%
\pgfsetbuttcap%
\pgfsetroundjoin%
\definecolor{currentfill}{rgb}{0.000000,0.000000,0.000000}%
\pgfsetfillcolor{currentfill}%
\pgfsetlinewidth{0.803000pt}%
\definecolor{currentstroke}{rgb}{0.000000,0.000000,0.000000}%
\pgfsetstrokecolor{currentstroke}%
\pgfsetdash{}{0pt}%
\pgfsys@defobject{currentmarker}{\pgfqpoint{0.000000in}{-0.048611in}}{\pgfqpoint{0.000000in}{0.000000in}}{%
\pgfpathmoveto{\pgfqpoint{0.000000in}{0.000000in}}%
\pgfpathlineto{\pgfqpoint{0.000000in}{-0.048611in}}%
\pgfusepath{stroke,fill}%
}%
\begin{pgfscope}%
\pgfsys@transformshift{7.402035in}{1.103099in}%
\pgfsys@useobject{currentmarker}{}%
\end{pgfscope}%
\end{pgfscope}%
\begin{pgfscope}%
\definecolor{textcolor}{rgb}{0.000000,0.000000,0.000000}%
\pgfsetstrokecolor{textcolor}%
\pgfsetfillcolor{textcolor}%
\pgftext[x=7.419396in, y=0.572711in, left, base,rotate=90.000000]{\color{textcolor}\rmfamily\fontsize{5.000000}{6.000000}\selectfont Solly Azar}%
\end{pgfscope}%
\begin{pgfscope}%
\pgfsetbuttcap%
\pgfsetroundjoin%
\definecolor{currentfill}{rgb}{0.000000,0.000000,0.000000}%
\pgfsetfillcolor{currentfill}%
\pgfsetlinewidth{0.803000pt}%
\definecolor{currentstroke}{rgb}{0.000000,0.000000,0.000000}%
\pgfsetstrokecolor{currentstroke}%
\pgfsetdash{}{0pt}%
\pgfsys@defobject{currentmarker}{\pgfqpoint{0.000000in}{-0.048611in}}{\pgfqpoint{0.000000in}{0.000000in}}{%
\pgfpathmoveto{\pgfqpoint{0.000000in}{0.000000in}}%
\pgfpathlineto{\pgfqpoint{0.000000in}{-0.048611in}}%
\pgfusepath{stroke,fill}%
}%
\begin{pgfscope}%
\pgfsys@transformshift{7.533508in}{1.103099in}%
\pgfsys@useobject{currentmarker}{}%
\end{pgfscope}%
\end{pgfscope}%
\begin{pgfscope}%
\definecolor{textcolor}{rgb}{0.000000,0.000000,0.000000}%
\pgfsetstrokecolor{textcolor}%
\pgfsetfillcolor{textcolor}%
\pgftext[x=7.550869in, y=0.611871in, left, base,rotate=90.000000]{\color{textcolor}\rmfamily\fontsize{5.000000}{6.000000}\selectfont Suravenir}%
\end{pgfscope}%
\begin{pgfscope}%
\pgfsetbuttcap%
\pgfsetroundjoin%
\definecolor{currentfill}{rgb}{0.000000,0.000000,0.000000}%
\pgfsetfillcolor{currentfill}%
\pgfsetlinewidth{0.803000pt}%
\definecolor{currentstroke}{rgb}{0.000000,0.000000,0.000000}%
\pgfsetstrokecolor{currentstroke}%
\pgfsetdash{}{0pt}%
\pgfsys@defobject{currentmarker}{\pgfqpoint{0.000000in}{-0.048611in}}{\pgfqpoint{0.000000in}{0.000000in}}{%
\pgfpathmoveto{\pgfqpoint{0.000000in}{0.000000in}}%
\pgfpathlineto{\pgfqpoint{0.000000in}{-0.048611in}}%
\pgfusepath{stroke,fill}%
}%
\begin{pgfscope}%
\pgfsys@transformshift{7.664981in}{1.103099in}%
\pgfsys@useobject{currentmarker}{}%
\end{pgfscope}%
\end{pgfscope}%
\begin{pgfscope}%
\definecolor{textcolor}{rgb}{0.000000,0.000000,0.000000}%
\pgfsetstrokecolor{textcolor}%
\pgfsetfillcolor{textcolor}%
\pgftext[x=7.682342in, y=0.624602in, left, base,rotate=90.000000]{\color{textcolor}\rmfamily\fontsize{5.000000}{6.000000}\selectfont SwissLife}%
\end{pgfscope}%
\begin{pgfscope}%
\pgfsetbuttcap%
\pgfsetroundjoin%
\definecolor{currentfill}{rgb}{0.000000,0.000000,0.000000}%
\pgfsetfillcolor{currentfill}%
\pgfsetlinewidth{0.803000pt}%
\definecolor{currentstroke}{rgb}{0.000000,0.000000,0.000000}%
\pgfsetstrokecolor{currentstroke}%
\pgfsetdash{}{0pt}%
\pgfsys@defobject{currentmarker}{\pgfqpoint{0.000000in}{-0.048611in}}{\pgfqpoint{0.000000in}{0.000000in}}{%
\pgfpathmoveto{\pgfqpoint{0.000000in}{0.000000in}}%
\pgfpathlineto{\pgfqpoint{0.000000in}{-0.048611in}}%
\pgfusepath{stroke,fill}%
}%
\begin{pgfscope}%
\pgfsys@transformshift{7.796455in}{1.103099in}%
\pgfsys@useobject{currentmarker}{}%
\end{pgfscope}%
\end{pgfscope}%
\begin{pgfscope}%
\definecolor{textcolor}{rgb}{0.000000,0.000000,0.000000}%
\pgfsetstrokecolor{textcolor}%
\pgfsetfillcolor{textcolor}%
\pgftext[x=7.813816in, y=0.706296in, left, base,rotate=90.000000]{\color{textcolor}\rmfamily\fontsize{5.000000}{6.000000}\selectfont Zen'Up}%
\end{pgfscope}%
\begin{pgfscope}%
\pgfsetbuttcap%
\pgfsetroundjoin%
\definecolor{currentfill}{rgb}{0.000000,0.000000,0.000000}%
\pgfsetfillcolor{currentfill}%
\pgfsetlinewidth{0.803000pt}%
\definecolor{currentstroke}{rgb}{0.000000,0.000000,0.000000}%
\pgfsetstrokecolor{currentstroke}%
\pgfsetdash{}{0pt}%
\pgfsys@defobject{currentmarker}{\pgfqpoint{-0.048611in}{0.000000in}}{\pgfqpoint{-0.000000in}{0.000000in}}{%
\pgfpathmoveto{\pgfqpoint{-0.000000in}{0.000000in}}%
\pgfpathlineto{\pgfqpoint{-0.048611in}{0.000000in}}%
\pgfusepath{stroke,fill}%
}%
\begin{pgfscope}%
\pgfsys@transformshift{0.499691in}{1.103099in}%
\pgfsys@useobject{currentmarker}{}%
\end{pgfscope}%
\end{pgfscope}%
\begin{pgfscope}%
\definecolor{textcolor}{rgb}{0.000000,0.000000,0.000000}%
\pgfsetstrokecolor{textcolor}%
\pgfsetfillcolor{textcolor}%
\pgftext[x=0.375463in, y=1.081879in, left, base,rotate=90.000000]{\color{textcolor}\rmfamily\fontsize{10.000000}{12.000000}\selectfont \(\displaystyle {0}\)}%
\end{pgfscope}%
\begin{pgfscope}%
\pgfsetbuttcap%
\pgfsetroundjoin%
\definecolor{currentfill}{rgb}{0.000000,0.000000,0.000000}%
\pgfsetfillcolor{currentfill}%
\pgfsetlinewidth{0.803000pt}%
\definecolor{currentstroke}{rgb}{0.000000,0.000000,0.000000}%
\pgfsetstrokecolor{currentstroke}%
\pgfsetdash{}{0pt}%
\pgfsys@defobject{currentmarker}{\pgfqpoint{-0.048611in}{0.000000in}}{\pgfqpoint{-0.000000in}{0.000000in}}{%
\pgfpathmoveto{\pgfqpoint{-0.000000in}{0.000000in}}%
\pgfpathlineto{\pgfqpoint{-0.048611in}{0.000000in}}%
\pgfusepath{stroke,fill}%
}%
\begin{pgfscope}%
\pgfsys@transformshift{0.499691in}{1.724989in}%
\pgfsys@useobject{currentmarker}{}%
\end{pgfscope}%
\end{pgfscope}%
\begin{pgfscope}%
\definecolor{textcolor}{rgb}{0.000000,0.000000,0.000000}%
\pgfsetstrokecolor{textcolor}%
\pgfsetfillcolor{textcolor}%
\pgftext[x=0.375463in, y=1.495436in, left, base,rotate=90.000000]{\color{textcolor}\rmfamily\fontsize{10.000000}{12.000000}\selectfont \(\displaystyle {1000}\)}%
\end{pgfscope}%
\begin{pgfscope}%
\pgfsetbuttcap%
\pgfsetroundjoin%
\definecolor{currentfill}{rgb}{0.000000,0.000000,0.000000}%
\pgfsetfillcolor{currentfill}%
\pgfsetlinewidth{0.803000pt}%
\definecolor{currentstroke}{rgb}{0.000000,0.000000,0.000000}%
\pgfsetstrokecolor{currentstroke}%
\pgfsetdash{}{0pt}%
\pgfsys@defobject{currentmarker}{\pgfqpoint{-0.048611in}{0.000000in}}{\pgfqpoint{-0.000000in}{0.000000in}}{%
\pgfpathmoveto{\pgfqpoint{-0.000000in}{0.000000in}}%
\pgfpathlineto{\pgfqpoint{-0.048611in}{0.000000in}}%
\pgfusepath{stroke,fill}%
}%
\begin{pgfscope}%
\pgfsys@transformshift{0.499691in}{2.346880in}%
\pgfsys@useobject{currentmarker}{}%
\end{pgfscope}%
\end{pgfscope}%
\begin{pgfscope}%
\definecolor{textcolor}{rgb}{0.000000,0.000000,0.000000}%
\pgfsetstrokecolor{textcolor}%
\pgfsetfillcolor{textcolor}%
\pgftext[x=0.375463in, y=2.117327in, left, base,rotate=90.000000]{\color{textcolor}\rmfamily\fontsize{10.000000}{12.000000}\selectfont \(\displaystyle {2000}\)}%
\end{pgfscope}%
\begin{pgfscope}%
\pgfsetbuttcap%
\pgfsetroundjoin%
\definecolor{currentfill}{rgb}{0.000000,0.000000,0.000000}%
\pgfsetfillcolor{currentfill}%
\pgfsetlinewidth{0.803000pt}%
\definecolor{currentstroke}{rgb}{0.000000,0.000000,0.000000}%
\pgfsetstrokecolor{currentstroke}%
\pgfsetdash{}{0pt}%
\pgfsys@defobject{currentmarker}{\pgfqpoint{-0.048611in}{0.000000in}}{\pgfqpoint{-0.000000in}{0.000000in}}{%
\pgfpathmoveto{\pgfqpoint{-0.000000in}{0.000000in}}%
\pgfpathlineto{\pgfqpoint{-0.048611in}{0.000000in}}%
\pgfusepath{stroke,fill}%
}%
\begin{pgfscope}%
\pgfsys@transformshift{0.499691in}{2.968770in}%
\pgfsys@useobject{currentmarker}{}%
\end{pgfscope}%
\end{pgfscope}%
\begin{pgfscope}%
\definecolor{textcolor}{rgb}{0.000000,0.000000,0.000000}%
\pgfsetstrokecolor{textcolor}%
\pgfsetfillcolor{textcolor}%
\pgftext[x=0.375463in, y=2.739217in, left, base,rotate=90.000000]{\color{textcolor}\rmfamily\fontsize{10.000000}{12.000000}\selectfont \(\displaystyle {3000}\)}%
\end{pgfscope}%
\begin{pgfscope}%
\pgfsetbuttcap%
\pgfsetroundjoin%
\definecolor{currentfill}{rgb}{0.000000,0.000000,0.000000}%
\pgfsetfillcolor{currentfill}%
\pgfsetlinewidth{0.803000pt}%
\definecolor{currentstroke}{rgb}{0.000000,0.000000,0.000000}%
\pgfsetstrokecolor{currentstroke}%
\pgfsetdash{}{0pt}%
\pgfsys@defobject{currentmarker}{\pgfqpoint{-0.048611in}{0.000000in}}{\pgfqpoint{-0.000000in}{0.000000in}}{%
\pgfpathmoveto{\pgfqpoint{-0.000000in}{0.000000in}}%
\pgfpathlineto{\pgfqpoint{-0.048611in}{0.000000in}}%
\pgfusepath{stroke,fill}%
}%
\begin{pgfscope}%
\pgfsys@transformshift{0.499691in}{3.590661in}%
\pgfsys@useobject{currentmarker}{}%
\end{pgfscope}%
\end{pgfscope}%
\begin{pgfscope}%
\definecolor{textcolor}{rgb}{0.000000,0.000000,0.000000}%
\pgfsetstrokecolor{textcolor}%
\pgfsetfillcolor{textcolor}%
\pgftext[x=0.375463in, y=3.361108in, left, base,rotate=90.000000]{\color{textcolor}\rmfamily\fontsize{10.000000}{12.000000}\selectfont \(\displaystyle {4000}\)}%
\end{pgfscope}%
\begin{pgfscope}%
\pgfsetbuttcap%
\pgfsetroundjoin%
\definecolor{currentfill}{rgb}{0.000000,0.000000,0.000000}%
\pgfsetfillcolor{currentfill}%
\pgfsetlinewidth{0.803000pt}%
\definecolor{currentstroke}{rgb}{0.000000,0.000000,0.000000}%
\pgfsetstrokecolor{currentstroke}%
\pgfsetdash{}{0pt}%
\pgfsys@defobject{currentmarker}{\pgfqpoint{-0.048611in}{0.000000in}}{\pgfqpoint{-0.000000in}{0.000000in}}{%
\pgfpathmoveto{\pgfqpoint{-0.000000in}{0.000000in}}%
\pgfpathlineto{\pgfqpoint{-0.048611in}{0.000000in}}%
\pgfusepath{stroke,fill}%
}%
\begin{pgfscope}%
\pgfsys@transformshift{0.499691in}{4.212551in}%
\pgfsys@useobject{currentmarker}{}%
\end{pgfscope}%
\end{pgfscope}%
\begin{pgfscope}%
\definecolor{textcolor}{rgb}{0.000000,0.000000,0.000000}%
\pgfsetstrokecolor{textcolor}%
\pgfsetfillcolor{textcolor}%
\pgftext[x=0.375463in, y=3.982998in, left, base,rotate=90.000000]{\color{textcolor}\rmfamily\fontsize{10.000000}{12.000000}\selectfont \(\displaystyle {5000}\)}%
\end{pgfscope}%
\begin{pgfscope}%
\pgfsetbuttcap%
\pgfsetroundjoin%
\definecolor{currentfill}{rgb}{0.000000,0.000000,0.000000}%
\pgfsetfillcolor{currentfill}%
\pgfsetlinewidth{0.803000pt}%
\definecolor{currentstroke}{rgb}{0.000000,0.000000,0.000000}%
\pgfsetstrokecolor{currentstroke}%
\pgfsetdash{}{0pt}%
\pgfsys@defobject{currentmarker}{\pgfqpoint{-0.048611in}{0.000000in}}{\pgfqpoint{-0.000000in}{0.000000in}}{%
\pgfpathmoveto{\pgfqpoint{-0.000000in}{0.000000in}}%
\pgfpathlineto{\pgfqpoint{-0.048611in}{0.000000in}}%
\pgfusepath{stroke,fill}%
}%
\begin{pgfscope}%
\pgfsys@transformshift{0.499691in}{4.834442in}%
\pgfsys@useobject{currentmarker}{}%
\end{pgfscope}%
\end{pgfscope}%
\begin{pgfscope}%
\definecolor{textcolor}{rgb}{0.000000,0.000000,0.000000}%
\pgfsetstrokecolor{textcolor}%
\pgfsetfillcolor{textcolor}%
\pgftext[x=0.375463in, y=4.604889in, left, base,rotate=90.000000]{\color{textcolor}\rmfamily\fontsize{10.000000}{12.000000}\selectfont \(\displaystyle {6000}\)}%
\end{pgfscope}%
\begin{pgfscope}%
\definecolor{textcolor}{rgb}{0.000000,0.000000,0.000000}%
\pgfsetstrokecolor{textcolor}%
\pgfsetfillcolor{textcolor}%
\pgftext[x=0.223457in,y=3.028099in,,bottom,rotate=90.000000]{\color{textcolor}\rmfamily\fontsize{10.000000}{12.000000}\selectfont Count}%
\end{pgfscope}%
\begin{pgfscope}%
\pgfpathrectangle{\pgfqpoint{0.499691in}{1.103099in}}{\pgfqpoint{7.362500in}{3.850000in}}%
\pgfusepath{clip}%
\pgfsetrectcap%
\pgfsetroundjoin%
\pgfsetlinewidth{2.710125pt}%
\definecolor{currentstroke}{rgb}{0.260000,0.260000,0.260000}%
\pgfsetstrokecolor{currentstroke}%
\pgfsetdash{}{0pt}%
\pgfusepath{stroke}%
\end{pgfscope}%
\begin{pgfscope}%
\pgfpathrectangle{\pgfqpoint{0.499691in}{1.103099in}}{\pgfqpoint{7.362500in}{3.850000in}}%
\pgfusepath{clip}%
\pgfsetrectcap%
\pgfsetroundjoin%
\pgfsetlinewidth{2.710125pt}%
\definecolor{currentstroke}{rgb}{0.260000,0.260000,0.260000}%
\pgfsetstrokecolor{currentstroke}%
\pgfsetdash{}{0pt}%
\pgfusepath{stroke}%
\end{pgfscope}%
\begin{pgfscope}%
\pgfpathrectangle{\pgfqpoint{0.499691in}{1.103099in}}{\pgfqpoint{7.362500in}{3.850000in}}%
\pgfusepath{clip}%
\pgfsetrectcap%
\pgfsetroundjoin%
\pgfsetlinewidth{2.710125pt}%
\definecolor{currentstroke}{rgb}{0.260000,0.260000,0.260000}%
\pgfsetstrokecolor{currentstroke}%
\pgfsetdash{}{0pt}%
\pgfusepath{stroke}%
\end{pgfscope}%
\begin{pgfscope}%
\pgfpathrectangle{\pgfqpoint{0.499691in}{1.103099in}}{\pgfqpoint{7.362500in}{3.850000in}}%
\pgfusepath{clip}%
\pgfsetrectcap%
\pgfsetroundjoin%
\pgfsetlinewidth{2.710125pt}%
\definecolor{currentstroke}{rgb}{0.260000,0.260000,0.260000}%
\pgfsetstrokecolor{currentstroke}%
\pgfsetdash{}{0pt}%
\pgfusepath{stroke}%
\end{pgfscope}%
\begin{pgfscope}%
\pgfpathrectangle{\pgfqpoint{0.499691in}{1.103099in}}{\pgfqpoint{7.362500in}{3.850000in}}%
\pgfusepath{clip}%
\pgfsetrectcap%
\pgfsetroundjoin%
\pgfsetlinewidth{2.710125pt}%
\definecolor{currentstroke}{rgb}{0.260000,0.260000,0.260000}%
\pgfsetstrokecolor{currentstroke}%
\pgfsetdash{}{0pt}%
\pgfusepath{stroke}%
\end{pgfscope}%
\begin{pgfscope}%
\pgfpathrectangle{\pgfqpoint{0.499691in}{1.103099in}}{\pgfqpoint{7.362500in}{3.850000in}}%
\pgfusepath{clip}%
\pgfsetrectcap%
\pgfsetroundjoin%
\pgfsetlinewidth{2.710125pt}%
\definecolor{currentstroke}{rgb}{0.260000,0.260000,0.260000}%
\pgfsetstrokecolor{currentstroke}%
\pgfsetdash{}{0pt}%
\pgfusepath{stroke}%
\end{pgfscope}%
\begin{pgfscope}%
\pgfpathrectangle{\pgfqpoint{0.499691in}{1.103099in}}{\pgfqpoint{7.362500in}{3.850000in}}%
\pgfusepath{clip}%
\pgfsetrectcap%
\pgfsetroundjoin%
\pgfsetlinewidth{2.710125pt}%
\definecolor{currentstroke}{rgb}{0.260000,0.260000,0.260000}%
\pgfsetstrokecolor{currentstroke}%
\pgfsetdash{}{0pt}%
\pgfusepath{stroke}%
\end{pgfscope}%
\begin{pgfscope}%
\pgfpathrectangle{\pgfqpoint{0.499691in}{1.103099in}}{\pgfqpoint{7.362500in}{3.850000in}}%
\pgfusepath{clip}%
\pgfsetrectcap%
\pgfsetroundjoin%
\pgfsetlinewidth{2.710125pt}%
\definecolor{currentstroke}{rgb}{0.260000,0.260000,0.260000}%
\pgfsetstrokecolor{currentstroke}%
\pgfsetdash{}{0pt}%
\pgfusepath{stroke}%
\end{pgfscope}%
\begin{pgfscope}%
\pgfpathrectangle{\pgfqpoint{0.499691in}{1.103099in}}{\pgfqpoint{7.362500in}{3.850000in}}%
\pgfusepath{clip}%
\pgfsetrectcap%
\pgfsetroundjoin%
\pgfsetlinewidth{2.710125pt}%
\definecolor{currentstroke}{rgb}{0.260000,0.260000,0.260000}%
\pgfsetstrokecolor{currentstroke}%
\pgfsetdash{}{0pt}%
\pgfusepath{stroke}%
\end{pgfscope}%
\begin{pgfscope}%
\pgfpathrectangle{\pgfqpoint{0.499691in}{1.103099in}}{\pgfqpoint{7.362500in}{3.850000in}}%
\pgfusepath{clip}%
\pgfsetrectcap%
\pgfsetroundjoin%
\pgfsetlinewidth{2.710125pt}%
\definecolor{currentstroke}{rgb}{0.260000,0.260000,0.260000}%
\pgfsetstrokecolor{currentstroke}%
\pgfsetdash{}{0pt}%
\pgfusepath{stroke}%
\end{pgfscope}%
\begin{pgfscope}%
\pgfpathrectangle{\pgfqpoint{0.499691in}{1.103099in}}{\pgfqpoint{7.362500in}{3.850000in}}%
\pgfusepath{clip}%
\pgfsetrectcap%
\pgfsetroundjoin%
\pgfsetlinewidth{2.710125pt}%
\definecolor{currentstroke}{rgb}{0.260000,0.260000,0.260000}%
\pgfsetstrokecolor{currentstroke}%
\pgfsetdash{}{0pt}%
\pgfusepath{stroke}%
\end{pgfscope}%
\begin{pgfscope}%
\pgfpathrectangle{\pgfqpoint{0.499691in}{1.103099in}}{\pgfqpoint{7.362500in}{3.850000in}}%
\pgfusepath{clip}%
\pgfsetrectcap%
\pgfsetroundjoin%
\pgfsetlinewidth{2.710125pt}%
\definecolor{currentstroke}{rgb}{0.260000,0.260000,0.260000}%
\pgfsetstrokecolor{currentstroke}%
\pgfsetdash{}{0pt}%
\pgfusepath{stroke}%
\end{pgfscope}%
\begin{pgfscope}%
\pgfpathrectangle{\pgfqpoint{0.499691in}{1.103099in}}{\pgfqpoint{7.362500in}{3.850000in}}%
\pgfusepath{clip}%
\pgfsetrectcap%
\pgfsetroundjoin%
\pgfsetlinewidth{2.710125pt}%
\definecolor{currentstroke}{rgb}{0.260000,0.260000,0.260000}%
\pgfsetstrokecolor{currentstroke}%
\pgfsetdash{}{0pt}%
\pgfusepath{stroke}%
\end{pgfscope}%
\begin{pgfscope}%
\pgfpathrectangle{\pgfqpoint{0.499691in}{1.103099in}}{\pgfqpoint{7.362500in}{3.850000in}}%
\pgfusepath{clip}%
\pgfsetrectcap%
\pgfsetroundjoin%
\pgfsetlinewidth{2.710125pt}%
\definecolor{currentstroke}{rgb}{0.260000,0.260000,0.260000}%
\pgfsetstrokecolor{currentstroke}%
\pgfsetdash{}{0pt}%
\pgfusepath{stroke}%
\end{pgfscope}%
\begin{pgfscope}%
\pgfpathrectangle{\pgfqpoint{0.499691in}{1.103099in}}{\pgfqpoint{7.362500in}{3.850000in}}%
\pgfusepath{clip}%
\pgfsetrectcap%
\pgfsetroundjoin%
\pgfsetlinewidth{2.710125pt}%
\definecolor{currentstroke}{rgb}{0.260000,0.260000,0.260000}%
\pgfsetstrokecolor{currentstroke}%
\pgfsetdash{}{0pt}%
\pgfusepath{stroke}%
\end{pgfscope}%
\begin{pgfscope}%
\pgfpathrectangle{\pgfqpoint{0.499691in}{1.103099in}}{\pgfqpoint{7.362500in}{3.850000in}}%
\pgfusepath{clip}%
\pgfsetrectcap%
\pgfsetroundjoin%
\pgfsetlinewidth{2.710125pt}%
\definecolor{currentstroke}{rgb}{0.260000,0.260000,0.260000}%
\pgfsetstrokecolor{currentstroke}%
\pgfsetdash{}{0pt}%
\pgfusepath{stroke}%
\end{pgfscope}%
\begin{pgfscope}%
\pgfpathrectangle{\pgfqpoint{0.499691in}{1.103099in}}{\pgfqpoint{7.362500in}{3.850000in}}%
\pgfusepath{clip}%
\pgfsetrectcap%
\pgfsetroundjoin%
\pgfsetlinewidth{2.710125pt}%
\definecolor{currentstroke}{rgb}{0.260000,0.260000,0.260000}%
\pgfsetstrokecolor{currentstroke}%
\pgfsetdash{}{0pt}%
\pgfusepath{stroke}%
\end{pgfscope}%
\begin{pgfscope}%
\pgfpathrectangle{\pgfqpoint{0.499691in}{1.103099in}}{\pgfqpoint{7.362500in}{3.850000in}}%
\pgfusepath{clip}%
\pgfsetrectcap%
\pgfsetroundjoin%
\pgfsetlinewidth{2.710125pt}%
\definecolor{currentstroke}{rgb}{0.260000,0.260000,0.260000}%
\pgfsetstrokecolor{currentstroke}%
\pgfsetdash{}{0pt}%
\pgfusepath{stroke}%
\end{pgfscope}%
\begin{pgfscope}%
\pgfpathrectangle{\pgfqpoint{0.499691in}{1.103099in}}{\pgfqpoint{7.362500in}{3.850000in}}%
\pgfusepath{clip}%
\pgfsetrectcap%
\pgfsetroundjoin%
\pgfsetlinewidth{2.710125pt}%
\definecolor{currentstroke}{rgb}{0.260000,0.260000,0.260000}%
\pgfsetstrokecolor{currentstroke}%
\pgfsetdash{}{0pt}%
\pgfusepath{stroke}%
\end{pgfscope}%
\begin{pgfscope}%
\pgfpathrectangle{\pgfqpoint{0.499691in}{1.103099in}}{\pgfqpoint{7.362500in}{3.850000in}}%
\pgfusepath{clip}%
\pgfsetrectcap%
\pgfsetroundjoin%
\pgfsetlinewidth{2.710125pt}%
\definecolor{currentstroke}{rgb}{0.260000,0.260000,0.260000}%
\pgfsetstrokecolor{currentstroke}%
\pgfsetdash{}{0pt}%
\pgfusepath{stroke}%
\end{pgfscope}%
\begin{pgfscope}%
\pgfpathrectangle{\pgfqpoint{0.499691in}{1.103099in}}{\pgfqpoint{7.362500in}{3.850000in}}%
\pgfusepath{clip}%
\pgfsetrectcap%
\pgfsetroundjoin%
\pgfsetlinewidth{2.710125pt}%
\definecolor{currentstroke}{rgb}{0.260000,0.260000,0.260000}%
\pgfsetstrokecolor{currentstroke}%
\pgfsetdash{}{0pt}%
\pgfusepath{stroke}%
\end{pgfscope}%
\begin{pgfscope}%
\pgfpathrectangle{\pgfqpoint{0.499691in}{1.103099in}}{\pgfqpoint{7.362500in}{3.850000in}}%
\pgfusepath{clip}%
\pgfsetrectcap%
\pgfsetroundjoin%
\pgfsetlinewidth{2.710125pt}%
\definecolor{currentstroke}{rgb}{0.260000,0.260000,0.260000}%
\pgfsetstrokecolor{currentstroke}%
\pgfsetdash{}{0pt}%
\pgfusepath{stroke}%
\end{pgfscope}%
\begin{pgfscope}%
\pgfpathrectangle{\pgfqpoint{0.499691in}{1.103099in}}{\pgfqpoint{7.362500in}{3.850000in}}%
\pgfusepath{clip}%
\pgfsetrectcap%
\pgfsetroundjoin%
\pgfsetlinewidth{2.710125pt}%
\definecolor{currentstroke}{rgb}{0.260000,0.260000,0.260000}%
\pgfsetstrokecolor{currentstroke}%
\pgfsetdash{}{0pt}%
\pgfusepath{stroke}%
\end{pgfscope}%
\begin{pgfscope}%
\pgfpathrectangle{\pgfqpoint{0.499691in}{1.103099in}}{\pgfqpoint{7.362500in}{3.850000in}}%
\pgfusepath{clip}%
\pgfsetrectcap%
\pgfsetroundjoin%
\pgfsetlinewidth{2.710125pt}%
\definecolor{currentstroke}{rgb}{0.260000,0.260000,0.260000}%
\pgfsetstrokecolor{currentstroke}%
\pgfsetdash{}{0pt}%
\pgfusepath{stroke}%
\end{pgfscope}%
\begin{pgfscope}%
\pgfpathrectangle{\pgfqpoint{0.499691in}{1.103099in}}{\pgfqpoint{7.362500in}{3.850000in}}%
\pgfusepath{clip}%
\pgfsetrectcap%
\pgfsetroundjoin%
\pgfsetlinewidth{2.710125pt}%
\definecolor{currentstroke}{rgb}{0.260000,0.260000,0.260000}%
\pgfsetstrokecolor{currentstroke}%
\pgfsetdash{}{0pt}%
\pgfusepath{stroke}%
\end{pgfscope}%
\begin{pgfscope}%
\pgfpathrectangle{\pgfqpoint{0.499691in}{1.103099in}}{\pgfqpoint{7.362500in}{3.850000in}}%
\pgfusepath{clip}%
\pgfsetrectcap%
\pgfsetroundjoin%
\pgfsetlinewidth{2.710125pt}%
\definecolor{currentstroke}{rgb}{0.260000,0.260000,0.260000}%
\pgfsetstrokecolor{currentstroke}%
\pgfsetdash{}{0pt}%
\pgfusepath{stroke}%
\end{pgfscope}%
\begin{pgfscope}%
\pgfpathrectangle{\pgfqpoint{0.499691in}{1.103099in}}{\pgfqpoint{7.362500in}{3.850000in}}%
\pgfusepath{clip}%
\pgfsetrectcap%
\pgfsetroundjoin%
\pgfsetlinewidth{2.710125pt}%
\definecolor{currentstroke}{rgb}{0.260000,0.260000,0.260000}%
\pgfsetstrokecolor{currentstroke}%
\pgfsetdash{}{0pt}%
\pgfusepath{stroke}%
\end{pgfscope}%
\begin{pgfscope}%
\pgfpathrectangle{\pgfqpoint{0.499691in}{1.103099in}}{\pgfqpoint{7.362500in}{3.850000in}}%
\pgfusepath{clip}%
\pgfsetrectcap%
\pgfsetroundjoin%
\pgfsetlinewidth{2.710125pt}%
\definecolor{currentstroke}{rgb}{0.260000,0.260000,0.260000}%
\pgfsetstrokecolor{currentstroke}%
\pgfsetdash{}{0pt}%
\pgfusepath{stroke}%
\end{pgfscope}%
\begin{pgfscope}%
\pgfpathrectangle{\pgfqpoint{0.499691in}{1.103099in}}{\pgfqpoint{7.362500in}{3.850000in}}%
\pgfusepath{clip}%
\pgfsetrectcap%
\pgfsetroundjoin%
\pgfsetlinewidth{2.710125pt}%
\definecolor{currentstroke}{rgb}{0.260000,0.260000,0.260000}%
\pgfsetstrokecolor{currentstroke}%
\pgfsetdash{}{0pt}%
\pgfusepath{stroke}%
\end{pgfscope}%
\begin{pgfscope}%
\pgfpathrectangle{\pgfqpoint{0.499691in}{1.103099in}}{\pgfqpoint{7.362500in}{3.850000in}}%
\pgfusepath{clip}%
\pgfsetrectcap%
\pgfsetroundjoin%
\pgfsetlinewidth{2.710125pt}%
\definecolor{currentstroke}{rgb}{0.260000,0.260000,0.260000}%
\pgfsetstrokecolor{currentstroke}%
\pgfsetdash{}{0pt}%
\pgfusepath{stroke}%
\end{pgfscope}%
\begin{pgfscope}%
\pgfpathrectangle{\pgfqpoint{0.499691in}{1.103099in}}{\pgfqpoint{7.362500in}{3.850000in}}%
\pgfusepath{clip}%
\pgfsetrectcap%
\pgfsetroundjoin%
\pgfsetlinewidth{2.710125pt}%
\definecolor{currentstroke}{rgb}{0.260000,0.260000,0.260000}%
\pgfsetstrokecolor{currentstroke}%
\pgfsetdash{}{0pt}%
\pgfusepath{stroke}%
\end{pgfscope}%
\begin{pgfscope}%
\pgfpathrectangle{\pgfqpoint{0.499691in}{1.103099in}}{\pgfqpoint{7.362500in}{3.850000in}}%
\pgfusepath{clip}%
\pgfsetrectcap%
\pgfsetroundjoin%
\pgfsetlinewidth{2.710125pt}%
\definecolor{currentstroke}{rgb}{0.260000,0.260000,0.260000}%
\pgfsetstrokecolor{currentstroke}%
\pgfsetdash{}{0pt}%
\pgfusepath{stroke}%
\end{pgfscope}%
\begin{pgfscope}%
\pgfpathrectangle{\pgfqpoint{0.499691in}{1.103099in}}{\pgfqpoint{7.362500in}{3.850000in}}%
\pgfusepath{clip}%
\pgfsetrectcap%
\pgfsetroundjoin%
\pgfsetlinewidth{2.710125pt}%
\definecolor{currentstroke}{rgb}{0.260000,0.260000,0.260000}%
\pgfsetstrokecolor{currentstroke}%
\pgfsetdash{}{0pt}%
\pgfusepath{stroke}%
\end{pgfscope}%
\begin{pgfscope}%
\pgfpathrectangle{\pgfqpoint{0.499691in}{1.103099in}}{\pgfqpoint{7.362500in}{3.850000in}}%
\pgfusepath{clip}%
\pgfsetrectcap%
\pgfsetroundjoin%
\pgfsetlinewidth{2.710125pt}%
\definecolor{currentstroke}{rgb}{0.260000,0.260000,0.260000}%
\pgfsetstrokecolor{currentstroke}%
\pgfsetdash{}{0pt}%
\pgfusepath{stroke}%
\end{pgfscope}%
\begin{pgfscope}%
\pgfpathrectangle{\pgfqpoint{0.499691in}{1.103099in}}{\pgfqpoint{7.362500in}{3.850000in}}%
\pgfusepath{clip}%
\pgfsetrectcap%
\pgfsetroundjoin%
\pgfsetlinewidth{2.710125pt}%
\definecolor{currentstroke}{rgb}{0.260000,0.260000,0.260000}%
\pgfsetstrokecolor{currentstroke}%
\pgfsetdash{}{0pt}%
\pgfusepath{stroke}%
\end{pgfscope}%
\begin{pgfscope}%
\pgfpathrectangle{\pgfqpoint{0.499691in}{1.103099in}}{\pgfqpoint{7.362500in}{3.850000in}}%
\pgfusepath{clip}%
\pgfsetrectcap%
\pgfsetroundjoin%
\pgfsetlinewidth{2.710125pt}%
\definecolor{currentstroke}{rgb}{0.260000,0.260000,0.260000}%
\pgfsetstrokecolor{currentstroke}%
\pgfsetdash{}{0pt}%
\pgfusepath{stroke}%
\end{pgfscope}%
\begin{pgfscope}%
\pgfpathrectangle{\pgfqpoint{0.499691in}{1.103099in}}{\pgfqpoint{7.362500in}{3.850000in}}%
\pgfusepath{clip}%
\pgfsetrectcap%
\pgfsetroundjoin%
\pgfsetlinewidth{2.710125pt}%
\definecolor{currentstroke}{rgb}{0.260000,0.260000,0.260000}%
\pgfsetstrokecolor{currentstroke}%
\pgfsetdash{}{0pt}%
\pgfusepath{stroke}%
\end{pgfscope}%
\begin{pgfscope}%
\pgfpathrectangle{\pgfqpoint{0.499691in}{1.103099in}}{\pgfqpoint{7.362500in}{3.850000in}}%
\pgfusepath{clip}%
\pgfsetrectcap%
\pgfsetroundjoin%
\pgfsetlinewidth{2.710125pt}%
\definecolor{currentstroke}{rgb}{0.260000,0.260000,0.260000}%
\pgfsetstrokecolor{currentstroke}%
\pgfsetdash{}{0pt}%
\pgfusepath{stroke}%
\end{pgfscope}%
\begin{pgfscope}%
\pgfpathrectangle{\pgfqpoint{0.499691in}{1.103099in}}{\pgfqpoint{7.362500in}{3.850000in}}%
\pgfusepath{clip}%
\pgfsetrectcap%
\pgfsetroundjoin%
\pgfsetlinewidth{2.710125pt}%
\definecolor{currentstroke}{rgb}{0.260000,0.260000,0.260000}%
\pgfsetstrokecolor{currentstroke}%
\pgfsetdash{}{0pt}%
\pgfusepath{stroke}%
\end{pgfscope}%
\begin{pgfscope}%
\pgfpathrectangle{\pgfqpoint{0.499691in}{1.103099in}}{\pgfqpoint{7.362500in}{3.850000in}}%
\pgfusepath{clip}%
\pgfsetrectcap%
\pgfsetroundjoin%
\pgfsetlinewidth{2.710125pt}%
\definecolor{currentstroke}{rgb}{0.260000,0.260000,0.260000}%
\pgfsetstrokecolor{currentstroke}%
\pgfsetdash{}{0pt}%
\pgfusepath{stroke}%
\end{pgfscope}%
\begin{pgfscope}%
\pgfpathrectangle{\pgfqpoint{0.499691in}{1.103099in}}{\pgfqpoint{7.362500in}{3.850000in}}%
\pgfusepath{clip}%
\pgfsetrectcap%
\pgfsetroundjoin%
\pgfsetlinewidth{2.710125pt}%
\definecolor{currentstroke}{rgb}{0.260000,0.260000,0.260000}%
\pgfsetstrokecolor{currentstroke}%
\pgfsetdash{}{0pt}%
\pgfusepath{stroke}%
\end{pgfscope}%
\begin{pgfscope}%
\pgfpathrectangle{\pgfqpoint{0.499691in}{1.103099in}}{\pgfqpoint{7.362500in}{3.850000in}}%
\pgfusepath{clip}%
\pgfsetrectcap%
\pgfsetroundjoin%
\pgfsetlinewidth{2.710125pt}%
\definecolor{currentstroke}{rgb}{0.260000,0.260000,0.260000}%
\pgfsetstrokecolor{currentstroke}%
\pgfsetdash{}{0pt}%
\pgfusepath{stroke}%
\end{pgfscope}%
\begin{pgfscope}%
\pgfpathrectangle{\pgfqpoint{0.499691in}{1.103099in}}{\pgfqpoint{7.362500in}{3.850000in}}%
\pgfusepath{clip}%
\pgfsetrectcap%
\pgfsetroundjoin%
\pgfsetlinewidth{2.710125pt}%
\definecolor{currentstroke}{rgb}{0.260000,0.260000,0.260000}%
\pgfsetstrokecolor{currentstroke}%
\pgfsetdash{}{0pt}%
\pgfusepath{stroke}%
\end{pgfscope}%
\begin{pgfscope}%
\pgfpathrectangle{\pgfqpoint{0.499691in}{1.103099in}}{\pgfqpoint{7.362500in}{3.850000in}}%
\pgfusepath{clip}%
\pgfsetrectcap%
\pgfsetroundjoin%
\pgfsetlinewidth{2.710125pt}%
\definecolor{currentstroke}{rgb}{0.260000,0.260000,0.260000}%
\pgfsetstrokecolor{currentstroke}%
\pgfsetdash{}{0pt}%
\pgfusepath{stroke}%
\end{pgfscope}%
\begin{pgfscope}%
\pgfpathrectangle{\pgfqpoint{0.499691in}{1.103099in}}{\pgfqpoint{7.362500in}{3.850000in}}%
\pgfusepath{clip}%
\pgfsetrectcap%
\pgfsetroundjoin%
\pgfsetlinewidth{2.710125pt}%
\definecolor{currentstroke}{rgb}{0.260000,0.260000,0.260000}%
\pgfsetstrokecolor{currentstroke}%
\pgfsetdash{}{0pt}%
\pgfusepath{stroke}%
\end{pgfscope}%
\begin{pgfscope}%
\pgfpathrectangle{\pgfqpoint{0.499691in}{1.103099in}}{\pgfqpoint{7.362500in}{3.850000in}}%
\pgfusepath{clip}%
\pgfsetrectcap%
\pgfsetroundjoin%
\pgfsetlinewidth{2.710125pt}%
\definecolor{currentstroke}{rgb}{0.260000,0.260000,0.260000}%
\pgfsetstrokecolor{currentstroke}%
\pgfsetdash{}{0pt}%
\pgfusepath{stroke}%
\end{pgfscope}%
\begin{pgfscope}%
\pgfpathrectangle{\pgfqpoint{0.499691in}{1.103099in}}{\pgfqpoint{7.362500in}{3.850000in}}%
\pgfusepath{clip}%
\pgfsetrectcap%
\pgfsetroundjoin%
\pgfsetlinewidth{2.710125pt}%
\definecolor{currentstroke}{rgb}{0.260000,0.260000,0.260000}%
\pgfsetstrokecolor{currentstroke}%
\pgfsetdash{}{0pt}%
\pgfusepath{stroke}%
\end{pgfscope}%
\begin{pgfscope}%
\pgfpathrectangle{\pgfqpoint{0.499691in}{1.103099in}}{\pgfqpoint{7.362500in}{3.850000in}}%
\pgfusepath{clip}%
\pgfsetrectcap%
\pgfsetroundjoin%
\pgfsetlinewidth{2.710125pt}%
\definecolor{currentstroke}{rgb}{0.260000,0.260000,0.260000}%
\pgfsetstrokecolor{currentstroke}%
\pgfsetdash{}{0pt}%
\pgfusepath{stroke}%
\end{pgfscope}%
\begin{pgfscope}%
\pgfpathrectangle{\pgfqpoint{0.499691in}{1.103099in}}{\pgfqpoint{7.362500in}{3.850000in}}%
\pgfusepath{clip}%
\pgfsetrectcap%
\pgfsetroundjoin%
\pgfsetlinewidth{2.710125pt}%
\definecolor{currentstroke}{rgb}{0.260000,0.260000,0.260000}%
\pgfsetstrokecolor{currentstroke}%
\pgfsetdash{}{0pt}%
\pgfusepath{stroke}%
\end{pgfscope}%
\begin{pgfscope}%
\pgfpathrectangle{\pgfqpoint{0.499691in}{1.103099in}}{\pgfqpoint{7.362500in}{3.850000in}}%
\pgfusepath{clip}%
\pgfsetrectcap%
\pgfsetroundjoin%
\pgfsetlinewidth{2.710125pt}%
\definecolor{currentstroke}{rgb}{0.260000,0.260000,0.260000}%
\pgfsetstrokecolor{currentstroke}%
\pgfsetdash{}{0pt}%
\pgfusepath{stroke}%
\end{pgfscope}%
\begin{pgfscope}%
\pgfpathrectangle{\pgfqpoint{0.499691in}{1.103099in}}{\pgfqpoint{7.362500in}{3.850000in}}%
\pgfusepath{clip}%
\pgfsetrectcap%
\pgfsetroundjoin%
\pgfsetlinewidth{2.710125pt}%
\definecolor{currentstroke}{rgb}{0.260000,0.260000,0.260000}%
\pgfsetstrokecolor{currentstroke}%
\pgfsetdash{}{0pt}%
\pgfusepath{stroke}%
\end{pgfscope}%
\begin{pgfscope}%
\pgfpathrectangle{\pgfqpoint{0.499691in}{1.103099in}}{\pgfqpoint{7.362500in}{3.850000in}}%
\pgfusepath{clip}%
\pgfsetrectcap%
\pgfsetroundjoin%
\pgfsetlinewidth{2.710125pt}%
\definecolor{currentstroke}{rgb}{0.260000,0.260000,0.260000}%
\pgfsetstrokecolor{currentstroke}%
\pgfsetdash{}{0pt}%
\pgfusepath{stroke}%
\end{pgfscope}%
\begin{pgfscope}%
\pgfpathrectangle{\pgfqpoint{0.499691in}{1.103099in}}{\pgfqpoint{7.362500in}{3.850000in}}%
\pgfusepath{clip}%
\pgfsetrectcap%
\pgfsetroundjoin%
\pgfsetlinewidth{2.710125pt}%
\definecolor{currentstroke}{rgb}{0.260000,0.260000,0.260000}%
\pgfsetstrokecolor{currentstroke}%
\pgfsetdash{}{0pt}%
\pgfusepath{stroke}%
\end{pgfscope}%
\begin{pgfscope}%
\pgfpathrectangle{\pgfqpoint{0.499691in}{1.103099in}}{\pgfqpoint{7.362500in}{3.850000in}}%
\pgfusepath{clip}%
\pgfsetrectcap%
\pgfsetroundjoin%
\pgfsetlinewidth{2.710125pt}%
\definecolor{currentstroke}{rgb}{0.260000,0.260000,0.260000}%
\pgfsetstrokecolor{currentstroke}%
\pgfsetdash{}{0pt}%
\pgfusepath{stroke}%
\end{pgfscope}%
\begin{pgfscope}%
\pgfpathrectangle{\pgfqpoint{0.499691in}{1.103099in}}{\pgfqpoint{7.362500in}{3.850000in}}%
\pgfusepath{clip}%
\pgfsetrectcap%
\pgfsetroundjoin%
\pgfsetlinewidth{2.710125pt}%
\definecolor{currentstroke}{rgb}{0.260000,0.260000,0.260000}%
\pgfsetstrokecolor{currentstroke}%
\pgfsetdash{}{0pt}%
\pgfusepath{stroke}%
\end{pgfscope}%
\begin{pgfscope}%
\pgfpathrectangle{\pgfqpoint{0.499691in}{1.103099in}}{\pgfqpoint{7.362500in}{3.850000in}}%
\pgfusepath{clip}%
\pgfsetrectcap%
\pgfsetroundjoin%
\pgfsetlinewidth{2.710125pt}%
\definecolor{currentstroke}{rgb}{0.260000,0.260000,0.260000}%
\pgfsetstrokecolor{currentstroke}%
\pgfsetdash{}{0pt}%
\pgfusepath{stroke}%
\end{pgfscope}%
\begin{pgfscope}%
\pgfsetrectcap%
\pgfsetmiterjoin%
\pgfsetlinewidth{0.803000pt}%
\definecolor{currentstroke}{rgb}{0.000000,0.000000,0.000000}%
\pgfsetstrokecolor{currentstroke}%
\pgfsetdash{}{0pt}%
\pgfpathmoveto{\pgfqpoint{0.499691in}{1.103099in}}%
\pgfpathlineto{\pgfqpoint{0.499691in}{4.953099in}}%
\pgfusepath{stroke}%
\end{pgfscope}%
\begin{pgfscope}%
\pgfsetrectcap%
\pgfsetmiterjoin%
\pgfsetlinewidth{0.803000pt}%
\definecolor{currentstroke}{rgb}{0.000000,0.000000,0.000000}%
\pgfsetstrokecolor{currentstroke}%
\pgfsetdash{}{0pt}%
\pgfpathmoveto{\pgfqpoint{7.862191in}{1.103099in}}%
\pgfpathlineto{\pgfqpoint{7.862191in}{4.953099in}}%
\pgfusepath{stroke}%
\end{pgfscope}%
\begin{pgfscope}%
\pgfsetrectcap%
\pgfsetmiterjoin%
\pgfsetlinewidth{0.803000pt}%
\definecolor{currentstroke}{rgb}{0.000000,0.000000,0.000000}%
\pgfsetstrokecolor{currentstroke}%
\pgfsetdash{}{0pt}%
\pgfpathmoveto{\pgfqpoint{0.499691in}{1.103099in}}%
\pgfpathlineto{\pgfqpoint{7.862191in}{1.103099in}}%
\pgfusepath{stroke}%
\end{pgfscope}%
\begin{pgfscope}%
\pgfsetrectcap%
\pgfsetmiterjoin%
\pgfsetlinewidth{0.803000pt}%
\definecolor{currentstroke}{rgb}{0.000000,0.000000,0.000000}%
\pgfsetstrokecolor{currentstroke}%
\pgfsetdash{}{0pt}%
\pgfpathmoveto{\pgfqpoint{0.499691in}{4.953099in}}%
\pgfpathlineto{\pgfqpoint{7.862191in}{4.953099in}}%
\pgfusepath{stroke}%
\end{pgfscope}%
\end{pgfpicture}%
\makeatother%
\endgroup%

    \caption{Number of note per assureur}
    \label{fig:nbnote_per_assureur}
\end{figure}

Finally we watch the number of review per date using a calendar

\begin{figure}[H]
    \advance\leftskip-3cm
    %% Creator: Matplotlib, PGF backend
%%
%% To include the figure in your LaTeX document, write
%%   \input{<filename>.pgf}
%%
%% Make sure the required packages are loaded in your preamble
%%   \usepackage{pgf}
%%
%% Also ensure that all the required font packages are loaded; for instance,
%% the lmodern package is sometimes necessary when using math font.
%%   \usepackage{lmodern}
%%
%% Figures using additional raster images can only be included by \input if
%% they are in the same directory as the main LaTeX file. For loading figures
%% from other directories you can use the `import` package
%%   \usepackage{import}
%%
%% and then include the figures with
%%   \import{<path to file>}{<filename>.pgf}
%%
%% Matplotlib used the following preamble
%%
\begingroup%
\makeatletter%
\begin{pgfpicture}%
\pgfpathrectangle{\pgfpointorigin}{\pgfqpoint{5.130943in}{10.924340in}}%
\pgfusepath{use as bounding box, clip}%
\begin{pgfscope}%
\pgfsetbuttcap%
\pgfsetmiterjoin%
\definecolor{currentfill}{rgb}{1.000000,1.000000,1.000000}%
\pgfsetfillcolor{currentfill}%
\pgfsetlinewidth{0.000000pt}%
\definecolor{currentstroke}{rgb}{1.000000,1.000000,1.000000}%
\pgfsetstrokecolor{currentstroke}%
\pgfsetdash{}{0pt}%
\pgfpathmoveto{\pgfqpoint{0.000000in}{0.000000in}}%
\pgfpathlineto{\pgfqpoint{5.130943in}{0.000000in}}%
\pgfpathlineto{\pgfqpoint{5.130943in}{10.924340in}}%
\pgfpathlineto{\pgfqpoint{0.000000in}{10.924340in}}%
\pgfpathlineto{\pgfqpoint{0.000000in}{0.000000in}}%
\pgfpathclose%
\pgfusepath{fill}%
\end{pgfscope}%
\begin{pgfscope}%
\pgfpathrectangle{\pgfqpoint{0.380943in}{9.960189in}}{\pgfqpoint{4.650000in}{0.614151in}}%
\pgfusepath{clip}%
\pgfsetbuttcap%
\pgfsetroundjoin%
\pgfsetlinewidth{0.250937pt}%
\definecolor{currentstroke}{rgb}{1.000000,1.000000,1.000000}%
\pgfsetstrokecolor{currentstroke}%
\pgfsetdash{}{0pt}%
\pgfpathmoveto{\pgfqpoint{0.380943in}{10.574340in}}%
\pgfpathlineto{\pgfqpoint{0.468679in}{10.574340in}}%
\pgfpathlineto{\pgfqpoint{0.468679in}{10.486604in}}%
\pgfpathlineto{\pgfqpoint{0.380943in}{10.486604in}}%
\pgfpathlineto{\pgfqpoint{0.380943in}{10.574340in}}%
\pgfusepath{stroke}%
\end{pgfscope}%
\begin{pgfscope}%
\pgfpathrectangle{\pgfqpoint{0.380943in}{9.960189in}}{\pgfqpoint{4.650000in}{0.614151in}}%
\pgfusepath{clip}%
\pgfsetbuttcap%
\pgfsetroundjoin%
\definecolor{currentfill}{rgb}{1.000000,1.000000,0.929412}%
\pgfsetfillcolor{currentfill}%
\pgfsetlinewidth{0.250937pt}%
\definecolor{currentstroke}{rgb}{1.000000,1.000000,1.000000}%
\pgfsetstrokecolor{currentstroke}%
\pgfsetdash{}{0pt}%
\pgfpathmoveto{\pgfqpoint{0.468679in}{10.574340in}}%
\pgfpathlineto{\pgfqpoint{0.556415in}{10.574340in}}%
\pgfpathlineto{\pgfqpoint{0.556415in}{10.486604in}}%
\pgfpathlineto{\pgfqpoint{0.468679in}{10.486604in}}%
\pgfpathlineto{\pgfqpoint{0.468679in}{10.574340in}}%
\pgfusepath{stroke,fill}%
\end{pgfscope}%
\begin{pgfscope}%
\pgfpathrectangle{\pgfqpoint{0.380943in}{9.960189in}}{\pgfqpoint{4.650000in}{0.614151in}}%
\pgfusepath{clip}%
\pgfsetbuttcap%
\pgfsetroundjoin%
\definecolor{currentfill}{rgb}{1.000000,1.000000,0.929412}%
\pgfsetfillcolor{currentfill}%
\pgfsetlinewidth{0.250937pt}%
\definecolor{currentstroke}{rgb}{1.000000,1.000000,1.000000}%
\pgfsetstrokecolor{currentstroke}%
\pgfsetdash{}{0pt}%
\pgfpathmoveto{\pgfqpoint{0.556415in}{10.574340in}}%
\pgfpathlineto{\pgfqpoint{0.644151in}{10.574340in}}%
\pgfpathlineto{\pgfqpoint{0.644151in}{10.486604in}}%
\pgfpathlineto{\pgfqpoint{0.556415in}{10.486604in}}%
\pgfpathlineto{\pgfqpoint{0.556415in}{10.574340in}}%
\pgfusepath{stroke,fill}%
\end{pgfscope}%
\begin{pgfscope}%
\pgfpathrectangle{\pgfqpoint{0.380943in}{9.960189in}}{\pgfqpoint{4.650000in}{0.614151in}}%
\pgfusepath{clip}%
\pgfsetbuttcap%
\pgfsetroundjoin%
\definecolor{currentfill}{rgb}{1.000000,1.000000,0.929412}%
\pgfsetfillcolor{currentfill}%
\pgfsetlinewidth{0.250937pt}%
\definecolor{currentstroke}{rgb}{1.000000,1.000000,1.000000}%
\pgfsetstrokecolor{currentstroke}%
\pgfsetdash{}{0pt}%
\pgfpathmoveto{\pgfqpoint{0.644151in}{10.574340in}}%
\pgfpathlineto{\pgfqpoint{0.731886in}{10.574340in}}%
\pgfpathlineto{\pgfqpoint{0.731886in}{10.486604in}}%
\pgfpathlineto{\pgfqpoint{0.644151in}{10.486604in}}%
\pgfpathlineto{\pgfqpoint{0.644151in}{10.574340in}}%
\pgfusepath{stroke,fill}%
\end{pgfscope}%
\begin{pgfscope}%
\pgfpathrectangle{\pgfqpoint{0.380943in}{9.960189in}}{\pgfqpoint{4.650000in}{0.614151in}}%
\pgfusepath{clip}%
\pgfsetbuttcap%
\pgfsetroundjoin%
\definecolor{currentfill}{rgb}{1.000000,1.000000,0.929412}%
\pgfsetfillcolor{currentfill}%
\pgfsetlinewidth{0.250937pt}%
\definecolor{currentstroke}{rgb}{1.000000,1.000000,1.000000}%
\pgfsetstrokecolor{currentstroke}%
\pgfsetdash{}{0pt}%
\pgfpathmoveto{\pgfqpoint{0.731886in}{10.574340in}}%
\pgfpathlineto{\pgfqpoint{0.819622in}{10.574340in}}%
\pgfpathlineto{\pgfqpoint{0.819622in}{10.486604in}}%
\pgfpathlineto{\pgfqpoint{0.731886in}{10.486604in}}%
\pgfpathlineto{\pgfqpoint{0.731886in}{10.574340in}}%
\pgfusepath{stroke,fill}%
\end{pgfscope}%
\begin{pgfscope}%
\pgfpathrectangle{\pgfqpoint{0.380943in}{9.960189in}}{\pgfqpoint{4.650000in}{0.614151in}}%
\pgfusepath{clip}%
\pgfsetbuttcap%
\pgfsetroundjoin%
\definecolor{currentfill}{rgb}{1.000000,1.000000,0.929412}%
\pgfsetfillcolor{currentfill}%
\pgfsetlinewidth{0.250937pt}%
\definecolor{currentstroke}{rgb}{1.000000,1.000000,1.000000}%
\pgfsetstrokecolor{currentstroke}%
\pgfsetdash{}{0pt}%
\pgfpathmoveto{\pgfqpoint{0.819622in}{10.574340in}}%
\pgfpathlineto{\pgfqpoint{0.907358in}{10.574340in}}%
\pgfpathlineto{\pgfqpoint{0.907358in}{10.486604in}}%
\pgfpathlineto{\pgfqpoint{0.819622in}{10.486604in}}%
\pgfpathlineto{\pgfqpoint{0.819622in}{10.574340in}}%
\pgfusepath{stroke,fill}%
\end{pgfscope}%
\begin{pgfscope}%
\pgfpathrectangle{\pgfqpoint{0.380943in}{9.960189in}}{\pgfqpoint{4.650000in}{0.614151in}}%
\pgfusepath{clip}%
\pgfsetbuttcap%
\pgfsetroundjoin%
\definecolor{currentfill}{rgb}{1.000000,1.000000,0.929412}%
\pgfsetfillcolor{currentfill}%
\pgfsetlinewidth{0.250937pt}%
\definecolor{currentstroke}{rgb}{1.000000,1.000000,1.000000}%
\pgfsetstrokecolor{currentstroke}%
\pgfsetdash{}{0pt}%
\pgfpathmoveto{\pgfqpoint{0.907358in}{10.574340in}}%
\pgfpathlineto{\pgfqpoint{0.995094in}{10.574340in}}%
\pgfpathlineto{\pgfqpoint{0.995094in}{10.486604in}}%
\pgfpathlineto{\pgfqpoint{0.907358in}{10.486604in}}%
\pgfpathlineto{\pgfqpoint{0.907358in}{10.574340in}}%
\pgfusepath{stroke,fill}%
\end{pgfscope}%
\begin{pgfscope}%
\pgfpathrectangle{\pgfqpoint{0.380943in}{9.960189in}}{\pgfqpoint{4.650000in}{0.614151in}}%
\pgfusepath{clip}%
\pgfsetbuttcap%
\pgfsetroundjoin%
\definecolor{currentfill}{rgb}{1.000000,1.000000,0.929412}%
\pgfsetfillcolor{currentfill}%
\pgfsetlinewidth{0.250937pt}%
\definecolor{currentstroke}{rgb}{1.000000,1.000000,1.000000}%
\pgfsetstrokecolor{currentstroke}%
\pgfsetdash{}{0pt}%
\pgfpathmoveto{\pgfqpoint{0.995094in}{10.574340in}}%
\pgfpathlineto{\pgfqpoint{1.082830in}{10.574340in}}%
\pgfpathlineto{\pgfqpoint{1.082830in}{10.486604in}}%
\pgfpathlineto{\pgfqpoint{0.995094in}{10.486604in}}%
\pgfpathlineto{\pgfqpoint{0.995094in}{10.574340in}}%
\pgfusepath{stroke,fill}%
\end{pgfscope}%
\begin{pgfscope}%
\pgfpathrectangle{\pgfqpoint{0.380943in}{9.960189in}}{\pgfqpoint{4.650000in}{0.614151in}}%
\pgfusepath{clip}%
\pgfsetbuttcap%
\pgfsetroundjoin%
\definecolor{currentfill}{rgb}{1.000000,1.000000,0.929412}%
\pgfsetfillcolor{currentfill}%
\pgfsetlinewidth{0.250937pt}%
\definecolor{currentstroke}{rgb}{1.000000,1.000000,1.000000}%
\pgfsetstrokecolor{currentstroke}%
\pgfsetdash{}{0pt}%
\pgfpathmoveto{\pgfqpoint{1.082830in}{10.574340in}}%
\pgfpathlineto{\pgfqpoint{1.170566in}{10.574340in}}%
\pgfpathlineto{\pgfqpoint{1.170566in}{10.486604in}}%
\pgfpathlineto{\pgfqpoint{1.082830in}{10.486604in}}%
\pgfpathlineto{\pgfqpoint{1.082830in}{10.574340in}}%
\pgfusepath{stroke,fill}%
\end{pgfscope}%
\begin{pgfscope}%
\pgfpathrectangle{\pgfqpoint{0.380943in}{9.960189in}}{\pgfqpoint{4.650000in}{0.614151in}}%
\pgfusepath{clip}%
\pgfsetbuttcap%
\pgfsetroundjoin%
\definecolor{currentfill}{rgb}{1.000000,1.000000,0.929412}%
\pgfsetfillcolor{currentfill}%
\pgfsetlinewidth{0.250937pt}%
\definecolor{currentstroke}{rgb}{1.000000,1.000000,1.000000}%
\pgfsetstrokecolor{currentstroke}%
\pgfsetdash{}{0pt}%
\pgfpathmoveto{\pgfqpoint{1.170566in}{10.574340in}}%
\pgfpathlineto{\pgfqpoint{1.258302in}{10.574340in}}%
\pgfpathlineto{\pgfqpoint{1.258302in}{10.486604in}}%
\pgfpathlineto{\pgfqpoint{1.170566in}{10.486604in}}%
\pgfpathlineto{\pgfqpoint{1.170566in}{10.574340in}}%
\pgfusepath{stroke,fill}%
\end{pgfscope}%
\begin{pgfscope}%
\pgfpathrectangle{\pgfqpoint{0.380943in}{9.960189in}}{\pgfqpoint{4.650000in}{0.614151in}}%
\pgfusepath{clip}%
\pgfsetbuttcap%
\pgfsetroundjoin%
\definecolor{currentfill}{rgb}{1.000000,1.000000,0.929412}%
\pgfsetfillcolor{currentfill}%
\pgfsetlinewidth{0.250937pt}%
\definecolor{currentstroke}{rgb}{1.000000,1.000000,1.000000}%
\pgfsetstrokecolor{currentstroke}%
\pgfsetdash{}{0pt}%
\pgfpathmoveto{\pgfqpoint{1.258302in}{10.574340in}}%
\pgfpathlineto{\pgfqpoint{1.346037in}{10.574340in}}%
\pgfpathlineto{\pgfqpoint{1.346037in}{10.486604in}}%
\pgfpathlineto{\pgfqpoint{1.258302in}{10.486604in}}%
\pgfpathlineto{\pgfqpoint{1.258302in}{10.574340in}}%
\pgfusepath{stroke,fill}%
\end{pgfscope}%
\begin{pgfscope}%
\pgfpathrectangle{\pgfqpoint{0.380943in}{9.960189in}}{\pgfqpoint{4.650000in}{0.614151in}}%
\pgfusepath{clip}%
\pgfsetbuttcap%
\pgfsetroundjoin%
\definecolor{currentfill}{rgb}{1.000000,1.000000,0.929412}%
\pgfsetfillcolor{currentfill}%
\pgfsetlinewidth{0.250937pt}%
\definecolor{currentstroke}{rgb}{1.000000,1.000000,1.000000}%
\pgfsetstrokecolor{currentstroke}%
\pgfsetdash{}{0pt}%
\pgfpathmoveto{\pgfqpoint{1.346037in}{10.574340in}}%
\pgfpathlineto{\pgfqpoint{1.433773in}{10.574340in}}%
\pgfpathlineto{\pgfqpoint{1.433773in}{10.486604in}}%
\pgfpathlineto{\pgfqpoint{1.346037in}{10.486604in}}%
\pgfpathlineto{\pgfqpoint{1.346037in}{10.574340in}}%
\pgfusepath{stroke,fill}%
\end{pgfscope}%
\begin{pgfscope}%
\pgfpathrectangle{\pgfqpoint{0.380943in}{9.960189in}}{\pgfqpoint{4.650000in}{0.614151in}}%
\pgfusepath{clip}%
\pgfsetbuttcap%
\pgfsetroundjoin%
\definecolor{currentfill}{rgb}{1.000000,1.000000,0.929412}%
\pgfsetfillcolor{currentfill}%
\pgfsetlinewidth{0.250937pt}%
\definecolor{currentstroke}{rgb}{1.000000,1.000000,1.000000}%
\pgfsetstrokecolor{currentstroke}%
\pgfsetdash{}{0pt}%
\pgfpathmoveto{\pgfqpoint{1.433773in}{10.574340in}}%
\pgfpathlineto{\pgfqpoint{1.521509in}{10.574340in}}%
\pgfpathlineto{\pgfqpoint{1.521509in}{10.486604in}}%
\pgfpathlineto{\pgfqpoint{1.433773in}{10.486604in}}%
\pgfpathlineto{\pgfqpoint{1.433773in}{10.574340in}}%
\pgfusepath{stroke,fill}%
\end{pgfscope}%
\begin{pgfscope}%
\pgfpathrectangle{\pgfqpoint{0.380943in}{9.960189in}}{\pgfqpoint{4.650000in}{0.614151in}}%
\pgfusepath{clip}%
\pgfsetbuttcap%
\pgfsetroundjoin%
\definecolor{currentfill}{rgb}{1.000000,1.000000,0.929412}%
\pgfsetfillcolor{currentfill}%
\pgfsetlinewidth{0.250937pt}%
\definecolor{currentstroke}{rgb}{1.000000,1.000000,1.000000}%
\pgfsetstrokecolor{currentstroke}%
\pgfsetdash{}{0pt}%
\pgfpathmoveto{\pgfqpoint{1.521509in}{10.574340in}}%
\pgfpathlineto{\pgfqpoint{1.609245in}{10.574340in}}%
\pgfpathlineto{\pgfqpoint{1.609245in}{10.486604in}}%
\pgfpathlineto{\pgfqpoint{1.521509in}{10.486604in}}%
\pgfpathlineto{\pgfqpoint{1.521509in}{10.574340in}}%
\pgfusepath{stroke,fill}%
\end{pgfscope}%
\begin{pgfscope}%
\pgfpathrectangle{\pgfqpoint{0.380943in}{9.960189in}}{\pgfqpoint{4.650000in}{0.614151in}}%
\pgfusepath{clip}%
\pgfsetbuttcap%
\pgfsetroundjoin%
\definecolor{currentfill}{rgb}{1.000000,1.000000,0.929412}%
\pgfsetfillcolor{currentfill}%
\pgfsetlinewidth{0.250937pt}%
\definecolor{currentstroke}{rgb}{1.000000,1.000000,1.000000}%
\pgfsetstrokecolor{currentstroke}%
\pgfsetdash{}{0pt}%
\pgfpathmoveto{\pgfqpoint{1.609245in}{10.574340in}}%
\pgfpathlineto{\pgfqpoint{1.696981in}{10.574340in}}%
\pgfpathlineto{\pgfqpoint{1.696981in}{10.486604in}}%
\pgfpathlineto{\pgfqpoint{1.609245in}{10.486604in}}%
\pgfpathlineto{\pgfqpoint{1.609245in}{10.574340in}}%
\pgfusepath{stroke,fill}%
\end{pgfscope}%
\begin{pgfscope}%
\pgfpathrectangle{\pgfqpoint{0.380943in}{9.960189in}}{\pgfqpoint{4.650000in}{0.614151in}}%
\pgfusepath{clip}%
\pgfsetbuttcap%
\pgfsetroundjoin%
\definecolor{currentfill}{rgb}{1.000000,1.000000,0.929412}%
\pgfsetfillcolor{currentfill}%
\pgfsetlinewidth{0.250937pt}%
\definecolor{currentstroke}{rgb}{1.000000,1.000000,1.000000}%
\pgfsetstrokecolor{currentstroke}%
\pgfsetdash{}{0pt}%
\pgfpathmoveto{\pgfqpoint{1.696981in}{10.574340in}}%
\pgfpathlineto{\pgfqpoint{1.784717in}{10.574340in}}%
\pgfpathlineto{\pgfqpoint{1.784717in}{10.486604in}}%
\pgfpathlineto{\pgfqpoint{1.696981in}{10.486604in}}%
\pgfpathlineto{\pgfqpoint{1.696981in}{10.574340in}}%
\pgfusepath{stroke,fill}%
\end{pgfscope}%
\begin{pgfscope}%
\pgfpathrectangle{\pgfqpoint{0.380943in}{9.960189in}}{\pgfqpoint{4.650000in}{0.614151in}}%
\pgfusepath{clip}%
\pgfsetbuttcap%
\pgfsetroundjoin%
\definecolor{currentfill}{rgb}{1.000000,1.000000,0.929412}%
\pgfsetfillcolor{currentfill}%
\pgfsetlinewidth{0.250937pt}%
\definecolor{currentstroke}{rgb}{1.000000,1.000000,1.000000}%
\pgfsetstrokecolor{currentstroke}%
\pgfsetdash{}{0pt}%
\pgfpathmoveto{\pgfqpoint{1.784717in}{10.574340in}}%
\pgfpathlineto{\pgfqpoint{1.872452in}{10.574340in}}%
\pgfpathlineto{\pgfqpoint{1.872452in}{10.486604in}}%
\pgfpathlineto{\pgfqpoint{1.784717in}{10.486604in}}%
\pgfpathlineto{\pgfqpoint{1.784717in}{10.574340in}}%
\pgfusepath{stroke,fill}%
\end{pgfscope}%
\begin{pgfscope}%
\pgfpathrectangle{\pgfqpoint{0.380943in}{9.960189in}}{\pgfqpoint{4.650000in}{0.614151in}}%
\pgfusepath{clip}%
\pgfsetbuttcap%
\pgfsetroundjoin%
\definecolor{currentfill}{rgb}{1.000000,1.000000,0.929412}%
\pgfsetfillcolor{currentfill}%
\pgfsetlinewidth{0.250937pt}%
\definecolor{currentstroke}{rgb}{1.000000,1.000000,1.000000}%
\pgfsetstrokecolor{currentstroke}%
\pgfsetdash{}{0pt}%
\pgfpathmoveto{\pgfqpoint{1.872452in}{10.574340in}}%
\pgfpathlineto{\pgfqpoint{1.960188in}{10.574340in}}%
\pgfpathlineto{\pgfqpoint{1.960188in}{10.486604in}}%
\pgfpathlineto{\pgfqpoint{1.872452in}{10.486604in}}%
\pgfpathlineto{\pgfqpoint{1.872452in}{10.574340in}}%
\pgfusepath{stroke,fill}%
\end{pgfscope}%
\begin{pgfscope}%
\pgfpathrectangle{\pgfqpoint{0.380943in}{9.960189in}}{\pgfqpoint{4.650000in}{0.614151in}}%
\pgfusepath{clip}%
\pgfsetbuttcap%
\pgfsetroundjoin%
\definecolor{currentfill}{rgb}{1.000000,1.000000,0.929412}%
\pgfsetfillcolor{currentfill}%
\pgfsetlinewidth{0.250937pt}%
\definecolor{currentstroke}{rgb}{1.000000,1.000000,1.000000}%
\pgfsetstrokecolor{currentstroke}%
\pgfsetdash{}{0pt}%
\pgfpathmoveto{\pgfqpoint{1.960188in}{10.574340in}}%
\pgfpathlineto{\pgfqpoint{2.047924in}{10.574340in}}%
\pgfpathlineto{\pgfqpoint{2.047924in}{10.486604in}}%
\pgfpathlineto{\pgfqpoint{1.960188in}{10.486604in}}%
\pgfpathlineto{\pgfqpoint{1.960188in}{10.574340in}}%
\pgfusepath{stroke,fill}%
\end{pgfscope}%
\begin{pgfscope}%
\pgfpathrectangle{\pgfqpoint{0.380943in}{9.960189in}}{\pgfqpoint{4.650000in}{0.614151in}}%
\pgfusepath{clip}%
\pgfsetbuttcap%
\pgfsetroundjoin%
\definecolor{currentfill}{rgb}{1.000000,1.000000,0.929412}%
\pgfsetfillcolor{currentfill}%
\pgfsetlinewidth{0.250937pt}%
\definecolor{currentstroke}{rgb}{1.000000,1.000000,1.000000}%
\pgfsetstrokecolor{currentstroke}%
\pgfsetdash{}{0pt}%
\pgfpathmoveto{\pgfqpoint{2.047924in}{10.574340in}}%
\pgfpathlineto{\pgfqpoint{2.135660in}{10.574340in}}%
\pgfpathlineto{\pgfqpoint{2.135660in}{10.486604in}}%
\pgfpathlineto{\pgfqpoint{2.047924in}{10.486604in}}%
\pgfpathlineto{\pgfqpoint{2.047924in}{10.574340in}}%
\pgfusepath{stroke,fill}%
\end{pgfscope}%
\begin{pgfscope}%
\pgfpathrectangle{\pgfqpoint{0.380943in}{9.960189in}}{\pgfqpoint{4.650000in}{0.614151in}}%
\pgfusepath{clip}%
\pgfsetbuttcap%
\pgfsetroundjoin%
\definecolor{currentfill}{rgb}{1.000000,1.000000,0.929412}%
\pgfsetfillcolor{currentfill}%
\pgfsetlinewidth{0.250937pt}%
\definecolor{currentstroke}{rgb}{1.000000,1.000000,1.000000}%
\pgfsetstrokecolor{currentstroke}%
\pgfsetdash{}{0pt}%
\pgfpathmoveto{\pgfqpoint{2.135660in}{10.574340in}}%
\pgfpathlineto{\pgfqpoint{2.223396in}{10.574340in}}%
\pgfpathlineto{\pgfqpoint{2.223396in}{10.486604in}}%
\pgfpathlineto{\pgfqpoint{2.135660in}{10.486604in}}%
\pgfpathlineto{\pgfqpoint{2.135660in}{10.574340in}}%
\pgfusepath{stroke,fill}%
\end{pgfscope}%
\begin{pgfscope}%
\pgfpathrectangle{\pgfqpoint{0.380943in}{9.960189in}}{\pgfqpoint{4.650000in}{0.614151in}}%
\pgfusepath{clip}%
\pgfsetbuttcap%
\pgfsetroundjoin%
\definecolor{currentfill}{rgb}{1.000000,1.000000,0.929412}%
\pgfsetfillcolor{currentfill}%
\pgfsetlinewidth{0.250937pt}%
\definecolor{currentstroke}{rgb}{1.000000,1.000000,1.000000}%
\pgfsetstrokecolor{currentstroke}%
\pgfsetdash{}{0pt}%
\pgfpathmoveto{\pgfqpoint{2.223396in}{10.574340in}}%
\pgfpathlineto{\pgfqpoint{2.311132in}{10.574340in}}%
\pgfpathlineto{\pgfqpoint{2.311132in}{10.486604in}}%
\pgfpathlineto{\pgfqpoint{2.223396in}{10.486604in}}%
\pgfpathlineto{\pgfqpoint{2.223396in}{10.574340in}}%
\pgfusepath{stroke,fill}%
\end{pgfscope}%
\begin{pgfscope}%
\pgfpathrectangle{\pgfqpoint{0.380943in}{9.960189in}}{\pgfqpoint{4.650000in}{0.614151in}}%
\pgfusepath{clip}%
\pgfsetbuttcap%
\pgfsetroundjoin%
\definecolor{currentfill}{rgb}{1.000000,1.000000,0.929412}%
\pgfsetfillcolor{currentfill}%
\pgfsetlinewidth{0.250937pt}%
\definecolor{currentstroke}{rgb}{1.000000,1.000000,1.000000}%
\pgfsetstrokecolor{currentstroke}%
\pgfsetdash{}{0pt}%
\pgfpathmoveto{\pgfqpoint{2.311132in}{10.574340in}}%
\pgfpathlineto{\pgfqpoint{2.398868in}{10.574340in}}%
\pgfpathlineto{\pgfqpoint{2.398868in}{10.486604in}}%
\pgfpathlineto{\pgfqpoint{2.311132in}{10.486604in}}%
\pgfpathlineto{\pgfqpoint{2.311132in}{10.574340in}}%
\pgfusepath{stroke,fill}%
\end{pgfscope}%
\begin{pgfscope}%
\pgfpathrectangle{\pgfqpoint{0.380943in}{9.960189in}}{\pgfqpoint{4.650000in}{0.614151in}}%
\pgfusepath{clip}%
\pgfsetbuttcap%
\pgfsetroundjoin%
\definecolor{currentfill}{rgb}{1.000000,1.000000,0.929412}%
\pgfsetfillcolor{currentfill}%
\pgfsetlinewidth{0.250937pt}%
\definecolor{currentstroke}{rgb}{1.000000,1.000000,1.000000}%
\pgfsetstrokecolor{currentstroke}%
\pgfsetdash{}{0pt}%
\pgfpathmoveto{\pgfqpoint{2.398868in}{10.574340in}}%
\pgfpathlineto{\pgfqpoint{2.486603in}{10.574340in}}%
\pgfpathlineto{\pgfqpoint{2.486603in}{10.486604in}}%
\pgfpathlineto{\pgfqpoint{2.398868in}{10.486604in}}%
\pgfpathlineto{\pgfqpoint{2.398868in}{10.574340in}}%
\pgfusepath{stroke,fill}%
\end{pgfscope}%
\begin{pgfscope}%
\pgfpathrectangle{\pgfqpoint{0.380943in}{9.960189in}}{\pgfqpoint{4.650000in}{0.614151in}}%
\pgfusepath{clip}%
\pgfsetbuttcap%
\pgfsetroundjoin%
\definecolor{currentfill}{rgb}{1.000000,1.000000,0.929412}%
\pgfsetfillcolor{currentfill}%
\pgfsetlinewidth{0.250937pt}%
\definecolor{currentstroke}{rgb}{1.000000,1.000000,1.000000}%
\pgfsetstrokecolor{currentstroke}%
\pgfsetdash{}{0pt}%
\pgfpathmoveto{\pgfqpoint{2.486603in}{10.574340in}}%
\pgfpathlineto{\pgfqpoint{2.574339in}{10.574340in}}%
\pgfpathlineto{\pgfqpoint{2.574339in}{10.486604in}}%
\pgfpathlineto{\pgfqpoint{2.486603in}{10.486604in}}%
\pgfpathlineto{\pgfqpoint{2.486603in}{10.574340in}}%
\pgfusepath{stroke,fill}%
\end{pgfscope}%
\begin{pgfscope}%
\pgfpathrectangle{\pgfqpoint{0.380943in}{9.960189in}}{\pgfqpoint{4.650000in}{0.614151in}}%
\pgfusepath{clip}%
\pgfsetbuttcap%
\pgfsetroundjoin%
\definecolor{currentfill}{rgb}{1.000000,1.000000,0.929412}%
\pgfsetfillcolor{currentfill}%
\pgfsetlinewidth{0.250937pt}%
\definecolor{currentstroke}{rgb}{1.000000,1.000000,1.000000}%
\pgfsetstrokecolor{currentstroke}%
\pgfsetdash{}{0pt}%
\pgfpathmoveto{\pgfqpoint{2.574339in}{10.574340in}}%
\pgfpathlineto{\pgfqpoint{2.662075in}{10.574340in}}%
\pgfpathlineto{\pgfqpoint{2.662075in}{10.486604in}}%
\pgfpathlineto{\pgfqpoint{2.574339in}{10.486604in}}%
\pgfpathlineto{\pgfqpoint{2.574339in}{10.574340in}}%
\pgfusepath{stroke,fill}%
\end{pgfscope}%
\begin{pgfscope}%
\pgfpathrectangle{\pgfqpoint{0.380943in}{9.960189in}}{\pgfqpoint{4.650000in}{0.614151in}}%
\pgfusepath{clip}%
\pgfsetbuttcap%
\pgfsetroundjoin%
\definecolor{currentfill}{rgb}{1.000000,1.000000,0.929412}%
\pgfsetfillcolor{currentfill}%
\pgfsetlinewidth{0.250937pt}%
\definecolor{currentstroke}{rgb}{1.000000,1.000000,1.000000}%
\pgfsetstrokecolor{currentstroke}%
\pgfsetdash{}{0pt}%
\pgfpathmoveto{\pgfqpoint{2.662075in}{10.574340in}}%
\pgfpathlineto{\pgfqpoint{2.749811in}{10.574340in}}%
\pgfpathlineto{\pgfqpoint{2.749811in}{10.486604in}}%
\pgfpathlineto{\pgfqpoint{2.662075in}{10.486604in}}%
\pgfpathlineto{\pgfqpoint{2.662075in}{10.574340in}}%
\pgfusepath{stroke,fill}%
\end{pgfscope}%
\begin{pgfscope}%
\pgfpathrectangle{\pgfqpoint{0.380943in}{9.960189in}}{\pgfqpoint{4.650000in}{0.614151in}}%
\pgfusepath{clip}%
\pgfsetbuttcap%
\pgfsetroundjoin%
\definecolor{currentfill}{rgb}{1.000000,1.000000,0.929412}%
\pgfsetfillcolor{currentfill}%
\pgfsetlinewidth{0.250937pt}%
\definecolor{currentstroke}{rgb}{1.000000,1.000000,1.000000}%
\pgfsetstrokecolor{currentstroke}%
\pgfsetdash{}{0pt}%
\pgfpathmoveto{\pgfqpoint{2.749811in}{10.574340in}}%
\pgfpathlineto{\pgfqpoint{2.837547in}{10.574340in}}%
\pgfpathlineto{\pgfqpoint{2.837547in}{10.486604in}}%
\pgfpathlineto{\pgfqpoint{2.749811in}{10.486604in}}%
\pgfpathlineto{\pgfqpoint{2.749811in}{10.574340in}}%
\pgfusepath{stroke,fill}%
\end{pgfscope}%
\begin{pgfscope}%
\pgfpathrectangle{\pgfqpoint{0.380943in}{9.960189in}}{\pgfqpoint{4.650000in}{0.614151in}}%
\pgfusepath{clip}%
\pgfsetbuttcap%
\pgfsetroundjoin%
\definecolor{currentfill}{rgb}{1.000000,1.000000,0.929412}%
\pgfsetfillcolor{currentfill}%
\pgfsetlinewidth{0.250937pt}%
\definecolor{currentstroke}{rgb}{1.000000,1.000000,1.000000}%
\pgfsetstrokecolor{currentstroke}%
\pgfsetdash{}{0pt}%
\pgfpathmoveto{\pgfqpoint{2.837547in}{10.574340in}}%
\pgfpathlineto{\pgfqpoint{2.925283in}{10.574340in}}%
\pgfpathlineto{\pgfqpoint{2.925283in}{10.486604in}}%
\pgfpathlineto{\pgfqpoint{2.837547in}{10.486604in}}%
\pgfpathlineto{\pgfqpoint{2.837547in}{10.574340in}}%
\pgfusepath{stroke,fill}%
\end{pgfscope}%
\begin{pgfscope}%
\pgfpathrectangle{\pgfqpoint{0.380943in}{9.960189in}}{\pgfqpoint{4.650000in}{0.614151in}}%
\pgfusepath{clip}%
\pgfsetbuttcap%
\pgfsetroundjoin%
\definecolor{currentfill}{rgb}{1.000000,1.000000,0.929412}%
\pgfsetfillcolor{currentfill}%
\pgfsetlinewidth{0.250937pt}%
\definecolor{currentstroke}{rgb}{1.000000,1.000000,1.000000}%
\pgfsetstrokecolor{currentstroke}%
\pgfsetdash{}{0pt}%
\pgfpathmoveto{\pgfqpoint{2.925283in}{10.574340in}}%
\pgfpathlineto{\pgfqpoint{3.013019in}{10.574340in}}%
\pgfpathlineto{\pgfqpoint{3.013019in}{10.486604in}}%
\pgfpathlineto{\pgfqpoint{2.925283in}{10.486604in}}%
\pgfpathlineto{\pgfqpoint{2.925283in}{10.574340in}}%
\pgfusepath{stroke,fill}%
\end{pgfscope}%
\begin{pgfscope}%
\pgfpathrectangle{\pgfqpoint{0.380943in}{9.960189in}}{\pgfqpoint{4.650000in}{0.614151in}}%
\pgfusepath{clip}%
\pgfsetbuttcap%
\pgfsetroundjoin%
\definecolor{currentfill}{rgb}{1.000000,1.000000,0.929412}%
\pgfsetfillcolor{currentfill}%
\pgfsetlinewidth{0.250937pt}%
\definecolor{currentstroke}{rgb}{1.000000,1.000000,1.000000}%
\pgfsetstrokecolor{currentstroke}%
\pgfsetdash{}{0pt}%
\pgfpathmoveto{\pgfqpoint{3.013019in}{10.574340in}}%
\pgfpathlineto{\pgfqpoint{3.100754in}{10.574340in}}%
\pgfpathlineto{\pgfqpoint{3.100754in}{10.486604in}}%
\pgfpathlineto{\pgfqpoint{3.013019in}{10.486604in}}%
\pgfpathlineto{\pgfqpoint{3.013019in}{10.574340in}}%
\pgfusepath{stroke,fill}%
\end{pgfscope}%
\begin{pgfscope}%
\pgfpathrectangle{\pgfqpoint{0.380943in}{9.960189in}}{\pgfqpoint{4.650000in}{0.614151in}}%
\pgfusepath{clip}%
\pgfsetbuttcap%
\pgfsetroundjoin%
\definecolor{currentfill}{rgb}{1.000000,1.000000,0.929412}%
\pgfsetfillcolor{currentfill}%
\pgfsetlinewidth{0.250937pt}%
\definecolor{currentstroke}{rgb}{1.000000,1.000000,1.000000}%
\pgfsetstrokecolor{currentstroke}%
\pgfsetdash{}{0pt}%
\pgfpathmoveto{\pgfqpoint{3.100754in}{10.574340in}}%
\pgfpathlineto{\pgfqpoint{3.188490in}{10.574340in}}%
\pgfpathlineto{\pgfqpoint{3.188490in}{10.486604in}}%
\pgfpathlineto{\pgfqpoint{3.100754in}{10.486604in}}%
\pgfpathlineto{\pgfqpoint{3.100754in}{10.574340in}}%
\pgfusepath{stroke,fill}%
\end{pgfscope}%
\begin{pgfscope}%
\pgfpathrectangle{\pgfqpoint{0.380943in}{9.960189in}}{\pgfqpoint{4.650000in}{0.614151in}}%
\pgfusepath{clip}%
\pgfsetbuttcap%
\pgfsetroundjoin%
\definecolor{currentfill}{rgb}{1.000000,1.000000,0.929412}%
\pgfsetfillcolor{currentfill}%
\pgfsetlinewidth{0.250937pt}%
\definecolor{currentstroke}{rgb}{1.000000,1.000000,1.000000}%
\pgfsetstrokecolor{currentstroke}%
\pgfsetdash{}{0pt}%
\pgfpathmoveto{\pgfqpoint{3.188490in}{10.574340in}}%
\pgfpathlineto{\pgfqpoint{3.276226in}{10.574340in}}%
\pgfpathlineto{\pgfqpoint{3.276226in}{10.486604in}}%
\pgfpathlineto{\pgfqpoint{3.188490in}{10.486604in}}%
\pgfpathlineto{\pgfqpoint{3.188490in}{10.574340in}}%
\pgfusepath{stroke,fill}%
\end{pgfscope}%
\begin{pgfscope}%
\pgfpathrectangle{\pgfqpoint{0.380943in}{9.960189in}}{\pgfqpoint{4.650000in}{0.614151in}}%
\pgfusepath{clip}%
\pgfsetbuttcap%
\pgfsetroundjoin%
\definecolor{currentfill}{rgb}{1.000000,1.000000,0.929412}%
\pgfsetfillcolor{currentfill}%
\pgfsetlinewidth{0.250937pt}%
\definecolor{currentstroke}{rgb}{1.000000,1.000000,1.000000}%
\pgfsetstrokecolor{currentstroke}%
\pgfsetdash{}{0pt}%
\pgfpathmoveto{\pgfqpoint{3.276226in}{10.574340in}}%
\pgfpathlineto{\pgfqpoint{3.363962in}{10.574340in}}%
\pgfpathlineto{\pgfqpoint{3.363962in}{10.486604in}}%
\pgfpathlineto{\pgfqpoint{3.276226in}{10.486604in}}%
\pgfpathlineto{\pgfqpoint{3.276226in}{10.574340in}}%
\pgfusepath{stroke,fill}%
\end{pgfscope}%
\begin{pgfscope}%
\pgfpathrectangle{\pgfqpoint{0.380943in}{9.960189in}}{\pgfqpoint{4.650000in}{0.614151in}}%
\pgfusepath{clip}%
\pgfsetbuttcap%
\pgfsetroundjoin%
\definecolor{currentfill}{rgb}{1.000000,1.000000,0.929412}%
\pgfsetfillcolor{currentfill}%
\pgfsetlinewidth{0.250937pt}%
\definecolor{currentstroke}{rgb}{1.000000,1.000000,1.000000}%
\pgfsetstrokecolor{currentstroke}%
\pgfsetdash{}{0pt}%
\pgfpathmoveto{\pgfqpoint{3.363962in}{10.574340in}}%
\pgfpathlineto{\pgfqpoint{3.451698in}{10.574340in}}%
\pgfpathlineto{\pgfqpoint{3.451698in}{10.486604in}}%
\pgfpathlineto{\pgfqpoint{3.363962in}{10.486604in}}%
\pgfpathlineto{\pgfqpoint{3.363962in}{10.574340in}}%
\pgfusepath{stroke,fill}%
\end{pgfscope}%
\begin{pgfscope}%
\pgfpathrectangle{\pgfqpoint{0.380943in}{9.960189in}}{\pgfqpoint{4.650000in}{0.614151in}}%
\pgfusepath{clip}%
\pgfsetbuttcap%
\pgfsetroundjoin%
\definecolor{currentfill}{rgb}{1.000000,1.000000,0.929412}%
\pgfsetfillcolor{currentfill}%
\pgfsetlinewidth{0.250937pt}%
\definecolor{currentstroke}{rgb}{1.000000,1.000000,1.000000}%
\pgfsetstrokecolor{currentstroke}%
\pgfsetdash{}{0pt}%
\pgfpathmoveto{\pgfqpoint{3.451698in}{10.574340in}}%
\pgfpathlineto{\pgfqpoint{3.539434in}{10.574340in}}%
\pgfpathlineto{\pgfqpoint{3.539434in}{10.486604in}}%
\pgfpathlineto{\pgfqpoint{3.451698in}{10.486604in}}%
\pgfpathlineto{\pgfqpoint{3.451698in}{10.574340in}}%
\pgfusepath{stroke,fill}%
\end{pgfscope}%
\begin{pgfscope}%
\pgfpathrectangle{\pgfqpoint{0.380943in}{9.960189in}}{\pgfqpoint{4.650000in}{0.614151in}}%
\pgfusepath{clip}%
\pgfsetbuttcap%
\pgfsetroundjoin%
\definecolor{currentfill}{rgb}{1.000000,1.000000,0.929412}%
\pgfsetfillcolor{currentfill}%
\pgfsetlinewidth{0.250937pt}%
\definecolor{currentstroke}{rgb}{1.000000,1.000000,1.000000}%
\pgfsetstrokecolor{currentstroke}%
\pgfsetdash{}{0pt}%
\pgfpathmoveto{\pgfqpoint{3.539434in}{10.574340in}}%
\pgfpathlineto{\pgfqpoint{3.627169in}{10.574340in}}%
\pgfpathlineto{\pgfqpoint{3.627169in}{10.486604in}}%
\pgfpathlineto{\pgfqpoint{3.539434in}{10.486604in}}%
\pgfpathlineto{\pgfqpoint{3.539434in}{10.574340in}}%
\pgfusepath{stroke,fill}%
\end{pgfscope}%
\begin{pgfscope}%
\pgfpathrectangle{\pgfqpoint{0.380943in}{9.960189in}}{\pgfqpoint{4.650000in}{0.614151in}}%
\pgfusepath{clip}%
\pgfsetbuttcap%
\pgfsetroundjoin%
\definecolor{currentfill}{rgb}{1.000000,1.000000,0.929412}%
\pgfsetfillcolor{currentfill}%
\pgfsetlinewidth{0.250937pt}%
\definecolor{currentstroke}{rgb}{1.000000,1.000000,1.000000}%
\pgfsetstrokecolor{currentstroke}%
\pgfsetdash{}{0pt}%
\pgfpathmoveto{\pgfqpoint{3.627169in}{10.574340in}}%
\pgfpathlineto{\pgfqpoint{3.714905in}{10.574340in}}%
\pgfpathlineto{\pgfqpoint{3.714905in}{10.486604in}}%
\pgfpathlineto{\pgfqpoint{3.627169in}{10.486604in}}%
\pgfpathlineto{\pgfqpoint{3.627169in}{10.574340in}}%
\pgfusepath{stroke,fill}%
\end{pgfscope}%
\begin{pgfscope}%
\pgfpathrectangle{\pgfqpoint{0.380943in}{9.960189in}}{\pgfqpoint{4.650000in}{0.614151in}}%
\pgfusepath{clip}%
\pgfsetbuttcap%
\pgfsetroundjoin%
\definecolor{currentfill}{rgb}{1.000000,1.000000,0.929412}%
\pgfsetfillcolor{currentfill}%
\pgfsetlinewidth{0.250937pt}%
\definecolor{currentstroke}{rgb}{1.000000,1.000000,1.000000}%
\pgfsetstrokecolor{currentstroke}%
\pgfsetdash{}{0pt}%
\pgfpathmoveto{\pgfqpoint{3.714905in}{10.574340in}}%
\pgfpathlineto{\pgfqpoint{3.802641in}{10.574340in}}%
\pgfpathlineto{\pgfqpoint{3.802641in}{10.486604in}}%
\pgfpathlineto{\pgfqpoint{3.714905in}{10.486604in}}%
\pgfpathlineto{\pgfqpoint{3.714905in}{10.574340in}}%
\pgfusepath{stroke,fill}%
\end{pgfscope}%
\begin{pgfscope}%
\pgfpathrectangle{\pgfqpoint{0.380943in}{9.960189in}}{\pgfqpoint{4.650000in}{0.614151in}}%
\pgfusepath{clip}%
\pgfsetbuttcap%
\pgfsetroundjoin%
\definecolor{currentfill}{rgb}{1.000000,1.000000,0.929412}%
\pgfsetfillcolor{currentfill}%
\pgfsetlinewidth{0.250937pt}%
\definecolor{currentstroke}{rgb}{1.000000,1.000000,1.000000}%
\pgfsetstrokecolor{currentstroke}%
\pgfsetdash{}{0pt}%
\pgfpathmoveto{\pgfqpoint{3.802641in}{10.574340in}}%
\pgfpathlineto{\pgfqpoint{3.890377in}{10.574340in}}%
\pgfpathlineto{\pgfqpoint{3.890377in}{10.486604in}}%
\pgfpathlineto{\pgfqpoint{3.802641in}{10.486604in}}%
\pgfpathlineto{\pgfqpoint{3.802641in}{10.574340in}}%
\pgfusepath{stroke,fill}%
\end{pgfscope}%
\begin{pgfscope}%
\pgfpathrectangle{\pgfqpoint{0.380943in}{9.960189in}}{\pgfqpoint{4.650000in}{0.614151in}}%
\pgfusepath{clip}%
\pgfsetbuttcap%
\pgfsetroundjoin%
\definecolor{currentfill}{rgb}{1.000000,1.000000,0.929412}%
\pgfsetfillcolor{currentfill}%
\pgfsetlinewidth{0.250937pt}%
\definecolor{currentstroke}{rgb}{1.000000,1.000000,1.000000}%
\pgfsetstrokecolor{currentstroke}%
\pgfsetdash{}{0pt}%
\pgfpathmoveto{\pgfqpoint{3.890377in}{10.574340in}}%
\pgfpathlineto{\pgfqpoint{3.978113in}{10.574340in}}%
\pgfpathlineto{\pgfqpoint{3.978113in}{10.486604in}}%
\pgfpathlineto{\pgfqpoint{3.890377in}{10.486604in}}%
\pgfpathlineto{\pgfqpoint{3.890377in}{10.574340in}}%
\pgfusepath{stroke,fill}%
\end{pgfscope}%
\begin{pgfscope}%
\pgfpathrectangle{\pgfqpoint{0.380943in}{9.960189in}}{\pgfqpoint{4.650000in}{0.614151in}}%
\pgfusepath{clip}%
\pgfsetbuttcap%
\pgfsetroundjoin%
\definecolor{currentfill}{rgb}{1.000000,1.000000,0.929412}%
\pgfsetfillcolor{currentfill}%
\pgfsetlinewidth{0.250937pt}%
\definecolor{currentstroke}{rgb}{1.000000,1.000000,1.000000}%
\pgfsetstrokecolor{currentstroke}%
\pgfsetdash{}{0pt}%
\pgfpathmoveto{\pgfqpoint{3.978113in}{10.574340in}}%
\pgfpathlineto{\pgfqpoint{4.065849in}{10.574340in}}%
\pgfpathlineto{\pgfqpoint{4.065849in}{10.486604in}}%
\pgfpathlineto{\pgfqpoint{3.978113in}{10.486604in}}%
\pgfpathlineto{\pgfqpoint{3.978113in}{10.574340in}}%
\pgfusepath{stroke,fill}%
\end{pgfscope}%
\begin{pgfscope}%
\pgfpathrectangle{\pgfqpoint{0.380943in}{9.960189in}}{\pgfqpoint{4.650000in}{0.614151in}}%
\pgfusepath{clip}%
\pgfsetbuttcap%
\pgfsetroundjoin%
\definecolor{currentfill}{rgb}{1.000000,1.000000,0.929412}%
\pgfsetfillcolor{currentfill}%
\pgfsetlinewidth{0.250937pt}%
\definecolor{currentstroke}{rgb}{1.000000,1.000000,1.000000}%
\pgfsetstrokecolor{currentstroke}%
\pgfsetdash{}{0pt}%
\pgfpathmoveto{\pgfqpoint{4.065849in}{10.574340in}}%
\pgfpathlineto{\pgfqpoint{4.153585in}{10.574340in}}%
\pgfpathlineto{\pgfqpoint{4.153585in}{10.486604in}}%
\pgfpathlineto{\pgfqpoint{4.065849in}{10.486604in}}%
\pgfpathlineto{\pgfqpoint{4.065849in}{10.574340in}}%
\pgfusepath{stroke,fill}%
\end{pgfscope}%
\begin{pgfscope}%
\pgfpathrectangle{\pgfqpoint{0.380943in}{9.960189in}}{\pgfqpoint{4.650000in}{0.614151in}}%
\pgfusepath{clip}%
\pgfsetbuttcap%
\pgfsetroundjoin%
\definecolor{currentfill}{rgb}{1.000000,1.000000,0.929412}%
\pgfsetfillcolor{currentfill}%
\pgfsetlinewidth{0.250937pt}%
\definecolor{currentstroke}{rgb}{1.000000,1.000000,1.000000}%
\pgfsetstrokecolor{currentstroke}%
\pgfsetdash{}{0pt}%
\pgfpathmoveto{\pgfqpoint{4.153585in}{10.574340in}}%
\pgfpathlineto{\pgfqpoint{4.241320in}{10.574340in}}%
\pgfpathlineto{\pgfqpoint{4.241320in}{10.486604in}}%
\pgfpathlineto{\pgfqpoint{4.153585in}{10.486604in}}%
\pgfpathlineto{\pgfqpoint{4.153585in}{10.574340in}}%
\pgfusepath{stroke,fill}%
\end{pgfscope}%
\begin{pgfscope}%
\pgfpathrectangle{\pgfqpoint{0.380943in}{9.960189in}}{\pgfqpoint{4.650000in}{0.614151in}}%
\pgfusepath{clip}%
\pgfsetbuttcap%
\pgfsetroundjoin%
\definecolor{currentfill}{rgb}{1.000000,1.000000,0.929412}%
\pgfsetfillcolor{currentfill}%
\pgfsetlinewidth{0.250937pt}%
\definecolor{currentstroke}{rgb}{1.000000,1.000000,1.000000}%
\pgfsetstrokecolor{currentstroke}%
\pgfsetdash{}{0pt}%
\pgfpathmoveto{\pgfqpoint{4.241320in}{10.574340in}}%
\pgfpathlineto{\pgfqpoint{4.329056in}{10.574340in}}%
\pgfpathlineto{\pgfqpoint{4.329056in}{10.486604in}}%
\pgfpathlineto{\pgfqpoint{4.241320in}{10.486604in}}%
\pgfpathlineto{\pgfqpoint{4.241320in}{10.574340in}}%
\pgfusepath{stroke,fill}%
\end{pgfscope}%
\begin{pgfscope}%
\pgfpathrectangle{\pgfqpoint{0.380943in}{9.960189in}}{\pgfqpoint{4.650000in}{0.614151in}}%
\pgfusepath{clip}%
\pgfsetbuttcap%
\pgfsetroundjoin%
\definecolor{currentfill}{rgb}{1.000000,1.000000,0.929412}%
\pgfsetfillcolor{currentfill}%
\pgfsetlinewidth{0.250937pt}%
\definecolor{currentstroke}{rgb}{1.000000,1.000000,1.000000}%
\pgfsetstrokecolor{currentstroke}%
\pgfsetdash{}{0pt}%
\pgfpathmoveto{\pgfqpoint{4.329056in}{10.574340in}}%
\pgfpathlineto{\pgfqpoint{4.416792in}{10.574340in}}%
\pgfpathlineto{\pgfqpoint{4.416792in}{10.486604in}}%
\pgfpathlineto{\pgfqpoint{4.329056in}{10.486604in}}%
\pgfpathlineto{\pgfqpoint{4.329056in}{10.574340in}}%
\pgfusepath{stroke,fill}%
\end{pgfscope}%
\begin{pgfscope}%
\pgfpathrectangle{\pgfqpoint{0.380943in}{9.960189in}}{\pgfqpoint{4.650000in}{0.614151in}}%
\pgfusepath{clip}%
\pgfsetbuttcap%
\pgfsetroundjoin%
\definecolor{currentfill}{rgb}{1.000000,1.000000,0.929412}%
\pgfsetfillcolor{currentfill}%
\pgfsetlinewidth{0.250937pt}%
\definecolor{currentstroke}{rgb}{1.000000,1.000000,1.000000}%
\pgfsetstrokecolor{currentstroke}%
\pgfsetdash{}{0pt}%
\pgfpathmoveto{\pgfqpoint{4.416792in}{10.574340in}}%
\pgfpathlineto{\pgfqpoint{4.504528in}{10.574340in}}%
\pgfpathlineto{\pgfqpoint{4.504528in}{10.486604in}}%
\pgfpathlineto{\pgfqpoint{4.416792in}{10.486604in}}%
\pgfpathlineto{\pgfqpoint{4.416792in}{10.574340in}}%
\pgfusepath{stroke,fill}%
\end{pgfscope}%
\begin{pgfscope}%
\pgfpathrectangle{\pgfqpoint{0.380943in}{9.960189in}}{\pgfqpoint{4.650000in}{0.614151in}}%
\pgfusepath{clip}%
\pgfsetbuttcap%
\pgfsetroundjoin%
\definecolor{currentfill}{rgb}{1.000000,1.000000,0.865975}%
\pgfsetfillcolor{currentfill}%
\pgfsetlinewidth{0.250937pt}%
\definecolor{currentstroke}{rgb}{1.000000,1.000000,1.000000}%
\pgfsetstrokecolor{currentstroke}%
\pgfsetdash{}{0pt}%
\pgfpathmoveto{\pgfqpoint{4.504528in}{10.574340in}}%
\pgfpathlineto{\pgfqpoint{4.592264in}{10.574340in}}%
\pgfpathlineto{\pgfqpoint{4.592264in}{10.486604in}}%
\pgfpathlineto{\pgfqpoint{4.504528in}{10.486604in}}%
\pgfpathlineto{\pgfqpoint{4.504528in}{10.574340in}}%
\pgfusepath{stroke,fill}%
\end{pgfscope}%
\begin{pgfscope}%
\pgfpathrectangle{\pgfqpoint{0.380943in}{9.960189in}}{\pgfqpoint{4.650000in}{0.614151in}}%
\pgfusepath{clip}%
\pgfsetbuttcap%
\pgfsetroundjoin%
\definecolor{currentfill}{rgb}{0.967474,0.895963,0.706344}%
\pgfsetfillcolor{currentfill}%
\pgfsetlinewidth{0.250937pt}%
\definecolor{currentstroke}{rgb}{1.000000,1.000000,1.000000}%
\pgfsetstrokecolor{currentstroke}%
\pgfsetdash{}{0pt}%
\pgfpathmoveto{\pgfqpoint{4.592264in}{10.574340in}}%
\pgfpathlineto{\pgfqpoint{4.680000in}{10.574340in}}%
\pgfpathlineto{\pgfqpoint{4.680000in}{10.486604in}}%
\pgfpathlineto{\pgfqpoint{4.592264in}{10.486604in}}%
\pgfpathlineto{\pgfqpoint{4.592264in}{10.574340in}}%
\pgfusepath{stroke,fill}%
\end{pgfscope}%
\begin{pgfscope}%
\pgfpathrectangle{\pgfqpoint{0.380943in}{9.960189in}}{\pgfqpoint{4.650000in}{0.614151in}}%
\pgfusepath{clip}%
\pgfsetbuttcap%
\pgfsetroundjoin%
\definecolor{currentfill}{rgb}{1.000000,0.522261,0.496886}%
\pgfsetfillcolor{currentfill}%
\pgfsetlinewidth{0.250937pt}%
\definecolor{currentstroke}{rgb}{1.000000,1.000000,1.000000}%
\pgfsetstrokecolor{currentstroke}%
\pgfsetdash{}{0pt}%
\pgfpathmoveto{\pgfqpoint{4.680000in}{10.574340in}}%
\pgfpathlineto{\pgfqpoint{4.767736in}{10.574340in}}%
\pgfpathlineto{\pgfqpoint{4.767736in}{10.486604in}}%
\pgfpathlineto{\pgfqpoint{4.680000in}{10.486604in}}%
\pgfpathlineto{\pgfqpoint{4.680000in}{10.574340in}}%
\pgfusepath{stroke,fill}%
\end{pgfscope}%
\begin{pgfscope}%
\pgfpathrectangle{\pgfqpoint{0.380943in}{9.960189in}}{\pgfqpoint{4.650000in}{0.614151in}}%
\pgfusepath{clip}%
\pgfsetbuttcap%
\pgfsetroundjoin%
\definecolor{currentfill}{rgb}{0.997924,0.685352,0.570242}%
\pgfsetfillcolor{currentfill}%
\pgfsetlinewidth{0.250937pt}%
\definecolor{currentstroke}{rgb}{1.000000,1.000000,1.000000}%
\pgfsetstrokecolor{currentstroke}%
\pgfsetdash{}{0pt}%
\pgfpathmoveto{\pgfqpoint{4.767736in}{10.574340in}}%
\pgfpathlineto{\pgfqpoint{4.855471in}{10.574340in}}%
\pgfpathlineto{\pgfqpoint{4.855471in}{10.486604in}}%
\pgfpathlineto{\pgfqpoint{4.767736in}{10.486604in}}%
\pgfpathlineto{\pgfqpoint{4.767736in}{10.574340in}}%
\pgfusepath{stroke,fill}%
\end{pgfscope}%
\begin{pgfscope}%
\pgfpathrectangle{\pgfqpoint{0.380943in}{9.960189in}}{\pgfqpoint{4.650000in}{0.614151in}}%
\pgfusepath{clip}%
\pgfsetbuttcap%
\pgfsetroundjoin%
\definecolor{currentfill}{rgb}{0.989619,0.788235,0.628374}%
\pgfsetfillcolor{currentfill}%
\pgfsetlinewidth{0.250937pt}%
\definecolor{currentstroke}{rgb}{1.000000,1.000000,1.000000}%
\pgfsetstrokecolor{currentstroke}%
\pgfsetdash{}{0pt}%
\pgfpathmoveto{\pgfqpoint{4.855471in}{10.574340in}}%
\pgfpathlineto{\pgfqpoint{4.943207in}{10.574340in}}%
\pgfpathlineto{\pgfqpoint{4.943207in}{10.486604in}}%
\pgfpathlineto{\pgfqpoint{4.855471in}{10.486604in}}%
\pgfpathlineto{\pgfqpoint{4.855471in}{10.574340in}}%
\pgfusepath{stroke,fill}%
\end{pgfscope}%
\begin{pgfscope}%
\pgfpathrectangle{\pgfqpoint{0.380943in}{9.960189in}}{\pgfqpoint{4.650000in}{0.614151in}}%
\pgfusepath{clip}%
\pgfsetbuttcap%
\pgfsetroundjoin%
\definecolor{currentfill}{rgb}{0.994694,0.745098,0.602999}%
\pgfsetfillcolor{currentfill}%
\pgfsetlinewidth{0.250937pt}%
\definecolor{currentstroke}{rgb}{1.000000,1.000000,1.000000}%
\pgfsetstrokecolor{currentstroke}%
\pgfsetdash{}{0pt}%
\pgfpathmoveto{\pgfqpoint{4.943207in}{10.574340in}}%
\pgfpathlineto{\pgfqpoint{5.030943in}{10.574340in}}%
\pgfpathlineto{\pgfqpoint{5.030943in}{10.486604in}}%
\pgfpathlineto{\pgfqpoint{4.943207in}{10.486604in}}%
\pgfpathlineto{\pgfqpoint{4.943207in}{10.574340in}}%
\pgfusepath{stroke,fill}%
\end{pgfscope}%
\begin{pgfscope}%
\pgfpathrectangle{\pgfqpoint{0.380943in}{9.960189in}}{\pgfqpoint{4.650000in}{0.614151in}}%
\pgfusepath{clip}%
\pgfsetbuttcap%
\pgfsetroundjoin%
\pgfsetlinewidth{0.250937pt}%
\definecolor{currentstroke}{rgb}{1.000000,1.000000,1.000000}%
\pgfsetstrokecolor{currentstroke}%
\pgfsetdash{}{0pt}%
\pgfpathmoveto{\pgfqpoint{0.380943in}{10.486604in}}%
\pgfpathlineto{\pgfqpoint{0.468679in}{10.486604in}}%
\pgfpathlineto{\pgfqpoint{0.468679in}{10.398868in}}%
\pgfpathlineto{\pgfqpoint{0.380943in}{10.398868in}}%
\pgfpathlineto{\pgfqpoint{0.380943in}{10.486604in}}%
\pgfusepath{stroke}%
\end{pgfscope}%
\begin{pgfscope}%
\pgfpathrectangle{\pgfqpoint{0.380943in}{9.960189in}}{\pgfqpoint{4.650000in}{0.614151in}}%
\pgfusepath{clip}%
\pgfsetbuttcap%
\pgfsetroundjoin%
\definecolor{currentfill}{rgb}{1.000000,1.000000,0.929412}%
\pgfsetfillcolor{currentfill}%
\pgfsetlinewidth{0.250937pt}%
\definecolor{currentstroke}{rgb}{1.000000,1.000000,1.000000}%
\pgfsetstrokecolor{currentstroke}%
\pgfsetdash{}{0pt}%
\pgfpathmoveto{\pgfqpoint{0.468679in}{10.486604in}}%
\pgfpathlineto{\pgfqpoint{0.556415in}{10.486604in}}%
\pgfpathlineto{\pgfqpoint{0.556415in}{10.398868in}}%
\pgfpathlineto{\pgfqpoint{0.468679in}{10.398868in}}%
\pgfpathlineto{\pgfqpoint{0.468679in}{10.486604in}}%
\pgfusepath{stroke,fill}%
\end{pgfscope}%
\begin{pgfscope}%
\pgfpathrectangle{\pgfqpoint{0.380943in}{9.960189in}}{\pgfqpoint{4.650000in}{0.614151in}}%
\pgfusepath{clip}%
\pgfsetbuttcap%
\pgfsetroundjoin%
\definecolor{currentfill}{rgb}{1.000000,1.000000,0.929412}%
\pgfsetfillcolor{currentfill}%
\pgfsetlinewidth{0.250937pt}%
\definecolor{currentstroke}{rgb}{1.000000,1.000000,1.000000}%
\pgfsetstrokecolor{currentstroke}%
\pgfsetdash{}{0pt}%
\pgfpathmoveto{\pgfqpoint{0.556415in}{10.486604in}}%
\pgfpathlineto{\pgfqpoint{0.644151in}{10.486604in}}%
\pgfpathlineto{\pgfqpoint{0.644151in}{10.398868in}}%
\pgfpathlineto{\pgfqpoint{0.556415in}{10.398868in}}%
\pgfpathlineto{\pgfqpoint{0.556415in}{10.486604in}}%
\pgfusepath{stroke,fill}%
\end{pgfscope}%
\begin{pgfscope}%
\pgfpathrectangle{\pgfqpoint{0.380943in}{9.960189in}}{\pgfqpoint{4.650000in}{0.614151in}}%
\pgfusepath{clip}%
\pgfsetbuttcap%
\pgfsetroundjoin%
\definecolor{currentfill}{rgb}{1.000000,1.000000,0.929412}%
\pgfsetfillcolor{currentfill}%
\pgfsetlinewidth{0.250937pt}%
\definecolor{currentstroke}{rgb}{1.000000,1.000000,1.000000}%
\pgfsetstrokecolor{currentstroke}%
\pgfsetdash{}{0pt}%
\pgfpathmoveto{\pgfqpoint{0.644151in}{10.486604in}}%
\pgfpathlineto{\pgfqpoint{0.731886in}{10.486604in}}%
\pgfpathlineto{\pgfqpoint{0.731886in}{10.398868in}}%
\pgfpathlineto{\pgfqpoint{0.644151in}{10.398868in}}%
\pgfpathlineto{\pgfqpoint{0.644151in}{10.486604in}}%
\pgfusepath{stroke,fill}%
\end{pgfscope}%
\begin{pgfscope}%
\pgfpathrectangle{\pgfqpoint{0.380943in}{9.960189in}}{\pgfqpoint{4.650000in}{0.614151in}}%
\pgfusepath{clip}%
\pgfsetbuttcap%
\pgfsetroundjoin%
\definecolor{currentfill}{rgb}{1.000000,1.000000,0.929412}%
\pgfsetfillcolor{currentfill}%
\pgfsetlinewidth{0.250937pt}%
\definecolor{currentstroke}{rgb}{1.000000,1.000000,1.000000}%
\pgfsetstrokecolor{currentstroke}%
\pgfsetdash{}{0pt}%
\pgfpathmoveto{\pgfqpoint{0.731886in}{10.486604in}}%
\pgfpathlineto{\pgfqpoint{0.819622in}{10.486604in}}%
\pgfpathlineto{\pgfqpoint{0.819622in}{10.398868in}}%
\pgfpathlineto{\pgfqpoint{0.731886in}{10.398868in}}%
\pgfpathlineto{\pgfqpoint{0.731886in}{10.486604in}}%
\pgfusepath{stroke,fill}%
\end{pgfscope}%
\begin{pgfscope}%
\pgfpathrectangle{\pgfqpoint{0.380943in}{9.960189in}}{\pgfqpoint{4.650000in}{0.614151in}}%
\pgfusepath{clip}%
\pgfsetbuttcap%
\pgfsetroundjoin%
\definecolor{currentfill}{rgb}{1.000000,1.000000,0.929412}%
\pgfsetfillcolor{currentfill}%
\pgfsetlinewidth{0.250937pt}%
\definecolor{currentstroke}{rgb}{1.000000,1.000000,1.000000}%
\pgfsetstrokecolor{currentstroke}%
\pgfsetdash{}{0pt}%
\pgfpathmoveto{\pgfqpoint{0.819622in}{10.486604in}}%
\pgfpathlineto{\pgfqpoint{0.907358in}{10.486604in}}%
\pgfpathlineto{\pgfqpoint{0.907358in}{10.398868in}}%
\pgfpathlineto{\pgfqpoint{0.819622in}{10.398868in}}%
\pgfpathlineto{\pgfqpoint{0.819622in}{10.486604in}}%
\pgfusepath{stroke,fill}%
\end{pgfscope}%
\begin{pgfscope}%
\pgfpathrectangle{\pgfqpoint{0.380943in}{9.960189in}}{\pgfqpoint{4.650000in}{0.614151in}}%
\pgfusepath{clip}%
\pgfsetbuttcap%
\pgfsetroundjoin%
\definecolor{currentfill}{rgb}{1.000000,1.000000,0.929412}%
\pgfsetfillcolor{currentfill}%
\pgfsetlinewidth{0.250937pt}%
\definecolor{currentstroke}{rgb}{1.000000,1.000000,1.000000}%
\pgfsetstrokecolor{currentstroke}%
\pgfsetdash{}{0pt}%
\pgfpathmoveto{\pgfqpoint{0.907358in}{10.486604in}}%
\pgfpathlineto{\pgfqpoint{0.995094in}{10.486604in}}%
\pgfpathlineto{\pgfqpoint{0.995094in}{10.398868in}}%
\pgfpathlineto{\pgfqpoint{0.907358in}{10.398868in}}%
\pgfpathlineto{\pgfqpoint{0.907358in}{10.486604in}}%
\pgfusepath{stroke,fill}%
\end{pgfscope}%
\begin{pgfscope}%
\pgfpathrectangle{\pgfqpoint{0.380943in}{9.960189in}}{\pgfqpoint{4.650000in}{0.614151in}}%
\pgfusepath{clip}%
\pgfsetbuttcap%
\pgfsetroundjoin%
\definecolor{currentfill}{rgb}{1.000000,1.000000,0.929412}%
\pgfsetfillcolor{currentfill}%
\pgfsetlinewidth{0.250937pt}%
\definecolor{currentstroke}{rgb}{1.000000,1.000000,1.000000}%
\pgfsetstrokecolor{currentstroke}%
\pgfsetdash{}{0pt}%
\pgfpathmoveto{\pgfqpoint{0.995094in}{10.486604in}}%
\pgfpathlineto{\pgfqpoint{1.082830in}{10.486604in}}%
\pgfpathlineto{\pgfqpoint{1.082830in}{10.398868in}}%
\pgfpathlineto{\pgfqpoint{0.995094in}{10.398868in}}%
\pgfpathlineto{\pgfqpoint{0.995094in}{10.486604in}}%
\pgfusepath{stroke,fill}%
\end{pgfscope}%
\begin{pgfscope}%
\pgfpathrectangle{\pgfqpoint{0.380943in}{9.960189in}}{\pgfqpoint{4.650000in}{0.614151in}}%
\pgfusepath{clip}%
\pgfsetbuttcap%
\pgfsetroundjoin%
\definecolor{currentfill}{rgb}{1.000000,1.000000,0.929412}%
\pgfsetfillcolor{currentfill}%
\pgfsetlinewidth{0.250937pt}%
\definecolor{currentstroke}{rgb}{1.000000,1.000000,1.000000}%
\pgfsetstrokecolor{currentstroke}%
\pgfsetdash{}{0pt}%
\pgfpathmoveto{\pgfqpoint{1.082830in}{10.486604in}}%
\pgfpathlineto{\pgfqpoint{1.170566in}{10.486604in}}%
\pgfpathlineto{\pgfqpoint{1.170566in}{10.398868in}}%
\pgfpathlineto{\pgfqpoint{1.082830in}{10.398868in}}%
\pgfpathlineto{\pgfqpoint{1.082830in}{10.486604in}}%
\pgfusepath{stroke,fill}%
\end{pgfscope}%
\begin{pgfscope}%
\pgfpathrectangle{\pgfqpoint{0.380943in}{9.960189in}}{\pgfqpoint{4.650000in}{0.614151in}}%
\pgfusepath{clip}%
\pgfsetbuttcap%
\pgfsetroundjoin%
\definecolor{currentfill}{rgb}{1.000000,1.000000,0.929412}%
\pgfsetfillcolor{currentfill}%
\pgfsetlinewidth{0.250937pt}%
\definecolor{currentstroke}{rgb}{1.000000,1.000000,1.000000}%
\pgfsetstrokecolor{currentstroke}%
\pgfsetdash{}{0pt}%
\pgfpathmoveto{\pgfqpoint{1.170566in}{10.486604in}}%
\pgfpathlineto{\pgfqpoint{1.258302in}{10.486604in}}%
\pgfpathlineto{\pgfqpoint{1.258302in}{10.398868in}}%
\pgfpathlineto{\pgfqpoint{1.170566in}{10.398868in}}%
\pgfpathlineto{\pgfqpoint{1.170566in}{10.486604in}}%
\pgfusepath{stroke,fill}%
\end{pgfscope}%
\begin{pgfscope}%
\pgfpathrectangle{\pgfqpoint{0.380943in}{9.960189in}}{\pgfqpoint{4.650000in}{0.614151in}}%
\pgfusepath{clip}%
\pgfsetbuttcap%
\pgfsetroundjoin%
\definecolor{currentfill}{rgb}{1.000000,1.000000,0.929412}%
\pgfsetfillcolor{currentfill}%
\pgfsetlinewidth{0.250937pt}%
\definecolor{currentstroke}{rgb}{1.000000,1.000000,1.000000}%
\pgfsetstrokecolor{currentstroke}%
\pgfsetdash{}{0pt}%
\pgfpathmoveto{\pgfqpoint{1.258302in}{10.486604in}}%
\pgfpathlineto{\pgfqpoint{1.346037in}{10.486604in}}%
\pgfpathlineto{\pgfqpoint{1.346037in}{10.398868in}}%
\pgfpathlineto{\pgfqpoint{1.258302in}{10.398868in}}%
\pgfpathlineto{\pgfqpoint{1.258302in}{10.486604in}}%
\pgfusepath{stroke,fill}%
\end{pgfscope}%
\begin{pgfscope}%
\pgfpathrectangle{\pgfqpoint{0.380943in}{9.960189in}}{\pgfqpoint{4.650000in}{0.614151in}}%
\pgfusepath{clip}%
\pgfsetbuttcap%
\pgfsetroundjoin%
\definecolor{currentfill}{rgb}{1.000000,1.000000,0.929412}%
\pgfsetfillcolor{currentfill}%
\pgfsetlinewidth{0.250937pt}%
\definecolor{currentstroke}{rgb}{1.000000,1.000000,1.000000}%
\pgfsetstrokecolor{currentstroke}%
\pgfsetdash{}{0pt}%
\pgfpathmoveto{\pgfqpoint{1.346037in}{10.486604in}}%
\pgfpathlineto{\pgfqpoint{1.433773in}{10.486604in}}%
\pgfpathlineto{\pgfqpoint{1.433773in}{10.398868in}}%
\pgfpathlineto{\pgfqpoint{1.346037in}{10.398868in}}%
\pgfpathlineto{\pgfqpoint{1.346037in}{10.486604in}}%
\pgfusepath{stroke,fill}%
\end{pgfscope}%
\begin{pgfscope}%
\pgfpathrectangle{\pgfqpoint{0.380943in}{9.960189in}}{\pgfqpoint{4.650000in}{0.614151in}}%
\pgfusepath{clip}%
\pgfsetbuttcap%
\pgfsetroundjoin%
\definecolor{currentfill}{rgb}{1.000000,1.000000,0.929412}%
\pgfsetfillcolor{currentfill}%
\pgfsetlinewidth{0.250937pt}%
\definecolor{currentstroke}{rgb}{1.000000,1.000000,1.000000}%
\pgfsetstrokecolor{currentstroke}%
\pgfsetdash{}{0pt}%
\pgfpathmoveto{\pgfqpoint{1.433773in}{10.486604in}}%
\pgfpathlineto{\pgfqpoint{1.521509in}{10.486604in}}%
\pgfpathlineto{\pgfqpoint{1.521509in}{10.398868in}}%
\pgfpathlineto{\pgfqpoint{1.433773in}{10.398868in}}%
\pgfpathlineto{\pgfqpoint{1.433773in}{10.486604in}}%
\pgfusepath{stroke,fill}%
\end{pgfscope}%
\begin{pgfscope}%
\pgfpathrectangle{\pgfqpoint{0.380943in}{9.960189in}}{\pgfqpoint{4.650000in}{0.614151in}}%
\pgfusepath{clip}%
\pgfsetbuttcap%
\pgfsetroundjoin%
\definecolor{currentfill}{rgb}{1.000000,1.000000,0.929412}%
\pgfsetfillcolor{currentfill}%
\pgfsetlinewidth{0.250937pt}%
\definecolor{currentstroke}{rgb}{1.000000,1.000000,1.000000}%
\pgfsetstrokecolor{currentstroke}%
\pgfsetdash{}{0pt}%
\pgfpathmoveto{\pgfqpoint{1.521509in}{10.486604in}}%
\pgfpathlineto{\pgfqpoint{1.609245in}{10.486604in}}%
\pgfpathlineto{\pgfqpoint{1.609245in}{10.398868in}}%
\pgfpathlineto{\pgfqpoint{1.521509in}{10.398868in}}%
\pgfpathlineto{\pgfqpoint{1.521509in}{10.486604in}}%
\pgfusepath{stroke,fill}%
\end{pgfscope}%
\begin{pgfscope}%
\pgfpathrectangle{\pgfqpoint{0.380943in}{9.960189in}}{\pgfqpoint{4.650000in}{0.614151in}}%
\pgfusepath{clip}%
\pgfsetbuttcap%
\pgfsetroundjoin%
\definecolor{currentfill}{rgb}{1.000000,1.000000,0.929412}%
\pgfsetfillcolor{currentfill}%
\pgfsetlinewidth{0.250937pt}%
\definecolor{currentstroke}{rgb}{1.000000,1.000000,1.000000}%
\pgfsetstrokecolor{currentstroke}%
\pgfsetdash{}{0pt}%
\pgfpathmoveto{\pgfqpoint{1.609245in}{10.486604in}}%
\pgfpathlineto{\pgfqpoint{1.696981in}{10.486604in}}%
\pgfpathlineto{\pgfqpoint{1.696981in}{10.398868in}}%
\pgfpathlineto{\pgfqpoint{1.609245in}{10.398868in}}%
\pgfpathlineto{\pgfqpoint{1.609245in}{10.486604in}}%
\pgfusepath{stroke,fill}%
\end{pgfscope}%
\begin{pgfscope}%
\pgfpathrectangle{\pgfqpoint{0.380943in}{9.960189in}}{\pgfqpoint{4.650000in}{0.614151in}}%
\pgfusepath{clip}%
\pgfsetbuttcap%
\pgfsetroundjoin%
\definecolor{currentfill}{rgb}{1.000000,1.000000,0.929412}%
\pgfsetfillcolor{currentfill}%
\pgfsetlinewidth{0.250937pt}%
\definecolor{currentstroke}{rgb}{1.000000,1.000000,1.000000}%
\pgfsetstrokecolor{currentstroke}%
\pgfsetdash{}{0pt}%
\pgfpathmoveto{\pgfqpoint{1.696981in}{10.486604in}}%
\pgfpathlineto{\pgfqpoint{1.784717in}{10.486604in}}%
\pgfpathlineto{\pgfqpoint{1.784717in}{10.398868in}}%
\pgfpathlineto{\pgfqpoint{1.696981in}{10.398868in}}%
\pgfpathlineto{\pgfqpoint{1.696981in}{10.486604in}}%
\pgfusepath{stroke,fill}%
\end{pgfscope}%
\begin{pgfscope}%
\pgfpathrectangle{\pgfqpoint{0.380943in}{9.960189in}}{\pgfqpoint{4.650000in}{0.614151in}}%
\pgfusepath{clip}%
\pgfsetbuttcap%
\pgfsetroundjoin%
\definecolor{currentfill}{rgb}{1.000000,1.000000,0.929412}%
\pgfsetfillcolor{currentfill}%
\pgfsetlinewidth{0.250937pt}%
\definecolor{currentstroke}{rgb}{1.000000,1.000000,1.000000}%
\pgfsetstrokecolor{currentstroke}%
\pgfsetdash{}{0pt}%
\pgfpathmoveto{\pgfqpoint{1.784717in}{10.486604in}}%
\pgfpathlineto{\pgfqpoint{1.872452in}{10.486604in}}%
\pgfpathlineto{\pgfqpoint{1.872452in}{10.398868in}}%
\pgfpathlineto{\pgfqpoint{1.784717in}{10.398868in}}%
\pgfpathlineto{\pgfqpoint{1.784717in}{10.486604in}}%
\pgfusepath{stroke,fill}%
\end{pgfscope}%
\begin{pgfscope}%
\pgfpathrectangle{\pgfqpoint{0.380943in}{9.960189in}}{\pgfqpoint{4.650000in}{0.614151in}}%
\pgfusepath{clip}%
\pgfsetbuttcap%
\pgfsetroundjoin%
\definecolor{currentfill}{rgb}{1.000000,1.000000,0.929412}%
\pgfsetfillcolor{currentfill}%
\pgfsetlinewidth{0.250937pt}%
\definecolor{currentstroke}{rgb}{1.000000,1.000000,1.000000}%
\pgfsetstrokecolor{currentstroke}%
\pgfsetdash{}{0pt}%
\pgfpathmoveto{\pgfqpoint{1.872452in}{10.486604in}}%
\pgfpathlineto{\pgfqpoint{1.960188in}{10.486604in}}%
\pgfpathlineto{\pgfqpoint{1.960188in}{10.398868in}}%
\pgfpathlineto{\pgfqpoint{1.872452in}{10.398868in}}%
\pgfpathlineto{\pgfqpoint{1.872452in}{10.486604in}}%
\pgfusepath{stroke,fill}%
\end{pgfscope}%
\begin{pgfscope}%
\pgfpathrectangle{\pgfqpoint{0.380943in}{9.960189in}}{\pgfqpoint{4.650000in}{0.614151in}}%
\pgfusepath{clip}%
\pgfsetbuttcap%
\pgfsetroundjoin%
\definecolor{currentfill}{rgb}{1.000000,1.000000,0.929412}%
\pgfsetfillcolor{currentfill}%
\pgfsetlinewidth{0.250937pt}%
\definecolor{currentstroke}{rgb}{1.000000,1.000000,1.000000}%
\pgfsetstrokecolor{currentstroke}%
\pgfsetdash{}{0pt}%
\pgfpathmoveto{\pgfqpoint{1.960188in}{10.486604in}}%
\pgfpathlineto{\pgfqpoint{2.047924in}{10.486604in}}%
\pgfpathlineto{\pgfqpoint{2.047924in}{10.398868in}}%
\pgfpathlineto{\pgfqpoint{1.960188in}{10.398868in}}%
\pgfpathlineto{\pgfqpoint{1.960188in}{10.486604in}}%
\pgfusepath{stroke,fill}%
\end{pgfscope}%
\begin{pgfscope}%
\pgfpathrectangle{\pgfqpoint{0.380943in}{9.960189in}}{\pgfqpoint{4.650000in}{0.614151in}}%
\pgfusepath{clip}%
\pgfsetbuttcap%
\pgfsetroundjoin%
\definecolor{currentfill}{rgb}{1.000000,1.000000,0.929412}%
\pgfsetfillcolor{currentfill}%
\pgfsetlinewidth{0.250937pt}%
\definecolor{currentstroke}{rgb}{1.000000,1.000000,1.000000}%
\pgfsetstrokecolor{currentstroke}%
\pgfsetdash{}{0pt}%
\pgfpathmoveto{\pgfqpoint{2.047924in}{10.486604in}}%
\pgfpathlineto{\pgfqpoint{2.135660in}{10.486604in}}%
\pgfpathlineto{\pgfqpoint{2.135660in}{10.398868in}}%
\pgfpathlineto{\pgfqpoint{2.047924in}{10.398868in}}%
\pgfpathlineto{\pgfqpoint{2.047924in}{10.486604in}}%
\pgfusepath{stroke,fill}%
\end{pgfscope}%
\begin{pgfscope}%
\pgfpathrectangle{\pgfqpoint{0.380943in}{9.960189in}}{\pgfqpoint{4.650000in}{0.614151in}}%
\pgfusepath{clip}%
\pgfsetbuttcap%
\pgfsetroundjoin%
\definecolor{currentfill}{rgb}{1.000000,1.000000,0.929412}%
\pgfsetfillcolor{currentfill}%
\pgfsetlinewidth{0.250937pt}%
\definecolor{currentstroke}{rgb}{1.000000,1.000000,1.000000}%
\pgfsetstrokecolor{currentstroke}%
\pgfsetdash{}{0pt}%
\pgfpathmoveto{\pgfqpoint{2.135660in}{10.486604in}}%
\pgfpathlineto{\pgfqpoint{2.223396in}{10.486604in}}%
\pgfpathlineto{\pgfqpoint{2.223396in}{10.398868in}}%
\pgfpathlineto{\pgfqpoint{2.135660in}{10.398868in}}%
\pgfpathlineto{\pgfqpoint{2.135660in}{10.486604in}}%
\pgfusepath{stroke,fill}%
\end{pgfscope}%
\begin{pgfscope}%
\pgfpathrectangle{\pgfqpoint{0.380943in}{9.960189in}}{\pgfqpoint{4.650000in}{0.614151in}}%
\pgfusepath{clip}%
\pgfsetbuttcap%
\pgfsetroundjoin%
\definecolor{currentfill}{rgb}{1.000000,1.000000,0.929412}%
\pgfsetfillcolor{currentfill}%
\pgfsetlinewidth{0.250937pt}%
\definecolor{currentstroke}{rgb}{1.000000,1.000000,1.000000}%
\pgfsetstrokecolor{currentstroke}%
\pgfsetdash{}{0pt}%
\pgfpathmoveto{\pgfqpoint{2.223396in}{10.486604in}}%
\pgfpathlineto{\pgfqpoint{2.311132in}{10.486604in}}%
\pgfpathlineto{\pgfqpoint{2.311132in}{10.398868in}}%
\pgfpathlineto{\pgfqpoint{2.223396in}{10.398868in}}%
\pgfpathlineto{\pgfqpoint{2.223396in}{10.486604in}}%
\pgfusepath{stroke,fill}%
\end{pgfscope}%
\begin{pgfscope}%
\pgfpathrectangle{\pgfqpoint{0.380943in}{9.960189in}}{\pgfqpoint{4.650000in}{0.614151in}}%
\pgfusepath{clip}%
\pgfsetbuttcap%
\pgfsetroundjoin%
\definecolor{currentfill}{rgb}{1.000000,1.000000,0.929412}%
\pgfsetfillcolor{currentfill}%
\pgfsetlinewidth{0.250937pt}%
\definecolor{currentstroke}{rgb}{1.000000,1.000000,1.000000}%
\pgfsetstrokecolor{currentstroke}%
\pgfsetdash{}{0pt}%
\pgfpathmoveto{\pgfqpoint{2.311132in}{10.486604in}}%
\pgfpathlineto{\pgfqpoint{2.398868in}{10.486604in}}%
\pgfpathlineto{\pgfqpoint{2.398868in}{10.398868in}}%
\pgfpathlineto{\pgfqpoint{2.311132in}{10.398868in}}%
\pgfpathlineto{\pgfqpoint{2.311132in}{10.486604in}}%
\pgfusepath{stroke,fill}%
\end{pgfscope}%
\begin{pgfscope}%
\pgfpathrectangle{\pgfqpoint{0.380943in}{9.960189in}}{\pgfqpoint{4.650000in}{0.614151in}}%
\pgfusepath{clip}%
\pgfsetbuttcap%
\pgfsetroundjoin%
\definecolor{currentfill}{rgb}{1.000000,1.000000,0.929412}%
\pgfsetfillcolor{currentfill}%
\pgfsetlinewidth{0.250937pt}%
\definecolor{currentstroke}{rgb}{1.000000,1.000000,1.000000}%
\pgfsetstrokecolor{currentstroke}%
\pgfsetdash{}{0pt}%
\pgfpathmoveto{\pgfqpoint{2.398868in}{10.486604in}}%
\pgfpathlineto{\pgfqpoint{2.486603in}{10.486604in}}%
\pgfpathlineto{\pgfqpoint{2.486603in}{10.398868in}}%
\pgfpathlineto{\pgfqpoint{2.398868in}{10.398868in}}%
\pgfpathlineto{\pgfqpoint{2.398868in}{10.486604in}}%
\pgfusepath{stroke,fill}%
\end{pgfscope}%
\begin{pgfscope}%
\pgfpathrectangle{\pgfqpoint{0.380943in}{9.960189in}}{\pgfqpoint{4.650000in}{0.614151in}}%
\pgfusepath{clip}%
\pgfsetbuttcap%
\pgfsetroundjoin%
\definecolor{currentfill}{rgb}{1.000000,1.000000,0.929412}%
\pgfsetfillcolor{currentfill}%
\pgfsetlinewidth{0.250937pt}%
\definecolor{currentstroke}{rgb}{1.000000,1.000000,1.000000}%
\pgfsetstrokecolor{currentstroke}%
\pgfsetdash{}{0pt}%
\pgfpathmoveto{\pgfqpoint{2.486603in}{10.486604in}}%
\pgfpathlineto{\pgfqpoint{2.574339in}{10.486604in}}%
\pgfpathlineto{\pgfqpoint{2.574339in}{10.398868in}}%
\pgfpathlineto{\pgfqpoint{2.486603in}{10.398868in}}%
\pgfpathlineto{\pgfqpoint{2.486603in}{10.486604in}}%
\pgfusepath{stroke,fill}%
\end{pgfscope}%
\begin{pgfscope}%
\pgfpathrectangle{\pgfqpoint{0.380943in}{9.960189in}}{\pgfqpoint{4.650000in}{0.614151in}}%
\pgfusepath{clip}%
\pgfsetbuttcap%
\pgfsetroundjoin%
\definecolor{currentfill}{rgb}{1.000000,1.000000,0.929412}%
\pgfsetfillcolor{currentfill}%
\pgfsetlinewidth{0.250937pt}%
\definecolor{currentstroke}{rgb}{1.000000,1.000000,1.000000}%
\pgfsetstrokecolor{currentstroke}%
\pgfsetdash{}{0pt}%
\pgfpathmoveto{\pgfqpoint{2.574339in}{10.486604in}}%
\pgfpathlineto{\pgfqpoint{2.662075in}{10.486604in}}%
\pgfpathlineto{\pgfqpoint{2.662075in}{10.398868in}}%
\pgfpathlineto{\pgfqpoint{2.574339in}{10.398868in}}%
\pgfpathlineto{\pgfqpoint{2.574339in}{10.486604in}}%
\pgfusepath{stroke,fill}%
\end{pgfscope}%
\begin{pgfscope}%
\pgfpathrectangle{\pgfqpoint{0.380943in}{9.960189in}}{\pgfqpoint{4.650000in}{0.614151in}}%
\pgfusepath{clip}%
\pgfsetbuttcap%
\pgfsetroundjoin%
\definecolor{currentfill}{rgb}{1.000000,1.000000,0.929412}%
\pgfsetfillcolor{currentfill}%
\pgfsetlinewidth{0.250937pt}%
\definecolor{currentstroke}{rgb}{1.000000,1.000000,1.000000}%
\pgfsetstrokecolor{currentstroke}%
\pgfsetdash{}{0pt}%
\pgfpathmoveto{\pgfqpoint{2.662075in}{10.486604in}}%
\pgfpathlineto{\pgfqpoint{2.749811in}{10.486604in}}%
\pgfpathlineto{\pgfqpoint{2.749811in}{10.398868in}}%
\pgfpathlineto{\pgfqpoint{2.662075in}{10.398868in}}%
\pgfpathlineto{\pgfqpoint{2.662075in}{10.486604in}}%
\pgfusepath{stroke,fill}%
\end{pgfscope}%
\begin{pgfscope}%
\pgfpathrectangle{\pgfqpoint{0.380943in}{9.960189in}}{\pgfqpoint{4.650000in}{0.614151in}}%
\pgfusepath{clip}%
\pgfsetbuttcap%
\pgfsetroundjoin%
\definecolor{currentfill}{rgb}{1.000000,1.000000,0.929412}%
\pgfsetfillcolor{currentfill}%
\pgfsetlinewidth{0.250937pt}%
\definecolor{currentstroke}{rgb}{1.000000,1.000000,1.000000}%
\pgfsetstrokecolor{currentstroke}%
\pgfsetdash{}{0pt}%
\pgfpathmoveto{\pgfqpoint{2.749811in}{10.486604in}}%
\pgfpathlineto{\pgfqpoint{2.837547in}{10.486604in}}%
\pgfpathlineto{\pgfqpoint{2.837547in}{10.398868in}}%
\pgfpathlineto{\pgfqpoint{2.749811in}{10.398868in}}%
\pgfpathlineto{\pgfqpoint{2.749811in}{10.486604in}}%
\pgfusepath{stroke,fill}%
\end{pgfscope}%
\begin{pgfscope}%
\pgfpathrectangle{\pgfqpoint{0.380943in}{9.960189in}}{\pgfqpoint{4.650000in}{0.614151in}}%
\pgfusepath{clip}%
\pgfsetbuttcap%
\pgfsetroundjoin%
\definecolor{currentfill}{rgb}{1.000000,1.000000,0.929412}%
\pgfsetfillcolor{currentfill}%
\pgfsetlinewidth{0.250937pt}%
\definecolor{currentstroke}{rgb}{1.000000,1.000000,1.000000}%
\pgfsetstrokecolor{currentstroke}%
\pgfsetdash{}{0pt}%
\pgfpathmoveto{\pgfqpoint{2.837547in}{10.486604in}}%
\pgfpathlineto{\pgfqpoint{2.925283in}{10.486604in}}%
\pgfpathlineto{\pgfqpoint{2.925283in}{10.398868in}}%
\pgfpathlineto{\pgfqpoint{2.837547in}{10.398868in}}%
\pgfpathlineto{\pgfqpoint{2.837547in}{10.486604in}}%
\pgfusepath{stroke,fill}%
\end{pgfscope}%
\begin{pgfscope}%
\pgfpathrectangle{\pgfqpoint{0.380943in}{9.960189in}}{\pgfqpoint{4.650000in}{0.614151in}}%
\pgfusepath{clip}%
\pgfsetbuttcap%
\pgfsetroundjoin%
\definecolor{currentfill}{rgb}{1.000000,1.000000,0.929412}%
\pgfsetfillcolor{currentfill}%
\pgfsetlinewidth{0.250937pt}%
\definecolor{currentstroke}{rgb}{1.000000,1.000000,1.000000}%
\pgfsetstrokecolor{currentstroke}%
\pgfsetdash{}{0pt}%
\pgfpathmoveto{\pgfqpoint{2.925283in}{10.486604in}}%
\pgfpathlineto{\pgfqpoint{3.013019in}{10.486604in}}%
\pgfpathlineto{\pgfqpoint{3.013019in}{10.398868in}}%
\pgfpathlineto{\pgfqpoint{2.925283in}{10.398868in}}%
\pgfpathlineto{\pgfqpoint{2.925283in}{10.486604in}}%
\pgfusepath{stroke,fill}%
\end{pgfscope}%
\begin{pgfscope}%
\pgfpathrectangle{\pgfqpoint{0.380943in}{9.960189in}}{\pgfqpoint{4.650000in}{0.614151in}}%
\pgfusepath{clip}%
\pgfsetbuttcap%
\pgfsetroundjoin%
\definecolor{currentfill}{rgb}{1.000000,1.000000,0.929412}%
\pgfsetfillcolor{currentfill}%
\pgfsetlinewidth{0.250937pt}%
\definecolor{currentstroke}{rgb}{1.000000,1.000000,1.000000}%
\pgfsetstrokecolor{currentstroke}%
\pgfsetdash{}{0pt}%
\pgfpathmoveto{\pgfqpoint{3.013019in}{10.486604in}}%
\pgfpathlineto{\pgfqpoint{3.100754in}{10.486604in}}%
\pgfpathlineto{\pgfqpoint{3.100754in}{10.398868in}}%
\pgfpathlineto{\pgfqpoint{3.013019in}{10.398868in}}%
\pgfpathlineto{\pgfqpoint{3.013019in}{10.486604in}}%
\pgfusepath{stroke,fill}%
\end{pgfscope}%
\begin{pgfscope}%
\pgfpathrectangle{\pgfqpoint{0.380943in}{9.960189in}}{\pgfqpoint{4.650000in}{0.614151in}}%
\pgfusepath{clip}%
\pgfsetbuttcap%
\pgfsetroundjoin%
\definecolor{currentfill}{rgb}{1.000000,1.000000,0.929412}%
\pgfsetfillcolor{currentfill}%
\pgfsetlinewidth{0.250937pt}%
\definecolor{currentstroke}{rgb}{1.000000,1.000000,1.000000}%
\pgfsetstrokecolor{currentstroke}%
\pgfsetdash{}{0pt}%
\pgfpathmoveto{\pgfqpoint{3.100754in}{10.486604in}}%
\pgfpathlineto{\pgfqpoint{3.188490in}{10.486604in}}%
\pgfpathlineto{\pgfqpoint{3.188490in}{10.398868in}}%
\pgfpathlineto{\pgfqpoint{3.100754in}{10.398868in}}%
\pgfpathlineto{\pgfqpoint{3.100754in}{10.486604in}}%
\pgfusepath{stroke,fill}%
\end{pgfscope}%
\begin{pgfscope}%
\pgfpathrectangle{\pgfqpoint{0.380943in}{9.960189in}}{\pgfqpoint{4.650000in}{0.614151in}}%
\pgfusepath{clip}%
\pgfsetbuttcap%
\pgfsetroundjoin%
\definecolor{currentfill}{rgb}{1.000000,1.000000,0.929412}%
\pgfsetfillcolor{currentfill}%
\pgfsetlinewidth{0.250937pt}%
\definecolor{currentstroke}{rgb}{1.000000,1.000000,1.000000}%
\pgfsetstrokecolor{currentstroke}%
\pgfsetdash{}{0pt}%
\pgfpathmoveto{\pgfqpoint{3.188490in}{10.486604in}}%
\pgfpathlineto{\pgfqpoint{3.276226in}{10.486604in}}%
\pgfpathlineto{\pgfqpoint{3.276226in}{10.398868in}}%
\pgfpathlineto{\pgfqpoint{3.188490in}{10.398868in}}%
\pgfpathlineto{\pgfqpoint{3.188490in}{10.486604in}}%
\pgfusepath{stroke,fill}%
\end{pgfscope}%
\begin{pgfscope}%
\pgfpathrectangle{\pgfqpoint{0.380943in}{9.960189in}}{\pgfqpoint{4.650000in}{0.614151in}}%
\pgfusepath{clip}%
\pgfsetbuttcap%
\pgfsetroundjoin%
\definecolor{currentfill}{rgb}{1.000000,1.000000,0.929412}%
\pgfsetfillcolor{currentfill}%
\pgfsetlinewidth{0.250937pt}%
\definecolor{currentstroke}{rgb}{1.000000,1.000000,1.000000}%
\pgfsetstrokecolor{currentstroke}%
\pgfsetdash{}{0pt}%
\pgfpathmoveto{\pgfqpoint{3.276226in}{10.486604in}}%
\pgfpathlineto{\pgfqpoint{3.363962in}{10.486604in}}%
\pgfpathlineto{\pgfqpoint{3.363962in}{10.398868in}}%
\pgfpathlineto{\pgfqpoint{3.276226in}{10.398868in}}%
\pgfpathlineto{\pgfqpoint{3.276226in}{10.486604in}}%
\pgfusepath{stroke,fill}%
\end{pgfscope}%
\begin{pgfscope}%
\pgfpathrectangle{\pgfqpoint{0.380943in}{9.960189in}}{\pgfqpoint{4.650000in}{0.614151in}}%
\pgfusepath{clip}%
\pgfsetbuttcap%
\pgfsetroundjoin%
\definecolor{currentfill}{rgb}{1.000000,1.000000,0.929412}%
\pgfsetfillcolor{currentfill}%
\pgfsetlinewidth{0.250937pt}%
\definecolor{currentstroke}{rgb}{1.000000,1.000000,1.000000}%
\pgfsetstrokecolor{currentstroke}%
\pgfsetdash{}{0pt}%
\pgfpathmoveto{\pgfqpoint{3.363962in}{10.486604in}}%
\pgfpathlineto{\pgfqpoint{3.451698in}{10.486604in}}%
\pgfpathlineto{\pgfqpoint{3.451698in}{10.398868in}}%
\pgfpathlineto{\pgfqpoint{3.363962in}{10.398868in}}%
\pgfpathlineto{\pgfqpoint{3.363962in}{10.486604in}}%
\pgfusepath{stroke,fill}%
\end{pgfscope}%
\begin{pgfscope}%
\pgfpathrectangle{\pgfqpoint{0.380943in}{9.960189in}}{\pgfqpoint{4.650000in}{0.614151in}}%
\pgfusepath{clip}%
\pgfsetbuttcap%
\pgfsetroundjoin%
\definecolor{currentfill}{rgb}{1.000000,1.000000,0.929412}%
\pgfsetfillcolor{currentfill}%
\pgfsetlinewidth{0.250937pt}%
\definecolor{currentstroke}{rgb}{1.000000,1.000000,1.000000}%
\pgfsetstrokecolor{currentstroke}%
\pgfsetdash{}{0pt}%
\pgfpathmoveto{\pgfqpoint{3.451698in}{10.486604in}}%
\pgfpathlineto{\pgfqpoint{3.539434in}{10.486604in}}%
\pgfpathlineto{\pgfqpoint{3.539434in}{10.398868in}}%
\pgfpathlineto{\pgfqpoint{3.451698in}{10.398868in}}%
\pgfpathlineto{\pgfqpoint{3.451698in}{10.486604in}}%
\pgfusepath{stroke,fill}%
\end{pgfscope}%
\begin{pgfscope}%
\pgfpathrectangle{\pgfqpoint{0.380943in}{9.960189in}}{\pgfqpoint{4.650000in}{0.614151in}}%
\pgfusepath{clip}%
\pgfsetbuttcap%
\pgfsetroundjoin%
\definecolor{currentfill}{rgb}{1.000000,1.000000,0.929412}%
\pgfsetfillcolor{currentfill}%
\pgfsetlinewidth{0.250937pt}%
\definecolor{currentstroke}{rgb}{1.000000,1.000000,1.000000}%
\pgfsetstrokecolor{currentstroke}%
\pgfsetdash{}{0pt}%
\pgfpathmoveto{\pgfqpoint{3.539434in}{10.486604in}}%
\pgfpathlineto{\pgfqpoint{3.627169in}{10.486604in}}%
\pgfpathlineto{\pgfqpoint{3.627169in}{10.398868in}}%
\pgfpathlineto{\pgfqpoint{3.539434in}{10.398868in}}%
\pgfpathlineto{\pgfqpoint{3.539434in}{10.486604in}}%
\pgfusepath{stroke,fill}%
\end{pgfscope}%
\begin{pgfscope}%
\pgfpathrectangle{\pgfqpoint{0.380943in}{9.960189in}}{\pgfqpoint{4.650000in}{0.614151in}}%
\pgfusepath{clip}%
\pgfsetbuttcap%
\pgfsetroundjoin%
\definecolor{currentfill}{rgb}{1.000000,1.000000,0.929412}%
\pgfsetfillcolor{currentfill}%
\pgfsetlinewidth{0.250937pt}%
\definecolor{currentstroke}{rgb}{1.000000,1.000000,1.000000}%
\pgfsetstrokecolor{currentstroke}%
\pgfsetdash{}{0pt}%
\pgfpathmoveto{\pgfqpoint{3.627169in}{10.486604in}}%
\pgfpathlineto{\pgfqpoint{3.714905in}{10.486604in}}%
\pgfpathlineto{\pgfqpoint{3.714905in}{10.398868in}}%
\pgfpathlineto{\pgfqpoint{3.627169in}{10.398868in}}%
\pgfpathlineto{\pgfqpoint{3.627169in}{10.486604in}}%
\pgfusepath{stroke,fill}%
\end{pgfscope}%
\begin{pgfscope}%
\pgfpathrectangle{\pgfqpoint{0.380943in}{9.960189in}}{\pgfqpoint{4.650000in}{0.614151in}}%
\pgfusepath{clip}%
\pgfsetbuttcap%
\pgfsetroundjoin%
\definecolor{currentfill}{rgb}{1.000000,1.000000,0.929412}%
\pgfsetfillcolor{currentfill}%
\pgfsetlinewidth{0.250937pt}%
\definecolor{currentstroke}{rgb}{1.000000,1.000000,1.000000}%
\pgfsetstrokecolor{currentstroke}%
\pgfsetdash{}{0pt}%
\pgfpathmoveto{\pgfqpoint{3.714905in}{10.486604in}}%
\pgfpathlineto{\pgfqpoint{3.802641in}{10.486604in}}%
\pgfpathlineto{\pgfqpoint{3.802641in}{10.398868in}}%
\pgfpathlineto{\pgfqpoint{3.714905in}{10.398868in}}%
\pgfpathlineto{\pgfqpoint{3.714905in}{10.486604in}}%
\pgfusepath{stroke,fill}%
\end{pgfscope}%
\begin{pgfscope}%
\pgfpathrectangle{\pgfqpoint{0.380943in}{9.960189in}}{\pgfqpoint{4.650000in}{0.614151in}}%
\pgfusepath{clip}%
\pgfsetbuttcap%
\pgfsetroundjoin%
\definecolor{currentfill}{rgb}{1.000000,1.000000,0.929412}%
\pgfsetfillcolor{currentfill}%
\pgfsetlinewidth{0.250937pt}%
\definecolor{currentstroke}{rgb}{1.000000,1.000000,1.000000}%
\pgfsetstrokecolor{currentstroke}%
\pgfsetdash{}{0pt}%
\pgfpathmoveto{\pgfqpoint{3.802641in}{10.486604in}}%
\pgfpathlineto{\pgfqpoint{3.890377in}{10.486604in}}%
\pgfpathlineto{\pgfqpoint{3.890377in}{10.398868in}}%
\pgfpathlineto{\pgfqpoint{3.802641in}{10.398868in}}%
\pgfpathlineto{\pgfqpoint{3.802641in}{10.486604in}}%
\pgfusepath{stroke,fill}%
\end{pgfscope}%
\begin{pgfscope}%
\pgfpathrectangle{\pgfqpoint{0.380943in}{9.960189in}}{\pgfqpoint{4.650000in}{0.614151in}}%
\pgfusepath{clip}%
\pgfsetbuttcap%
\pgfsetroundjoin%
\definecolor{currentfill}{rgb}{1.000000,1.000000,0.929412}%
\pgfsetfillcolor{currentfill}%
\pgfsetlinewidth{0.250937pt}%
\definecolor{currentstroke}{rgb}{1.000000,1.000000,1.000000}%
\pgfsetstrokecolor{currentstroke}%
\pgfsetdash{}{0pt}%
\pgfpathmoveto{\pgfqpoint{3.890377in}{10.486604in}}%
\pgfpathlineto{\pgfqpoint{3.978113in}{10.486604in}}%
\pgfpathlineto{\pgfqpoint{3.978113in}{10.398868in}}%
\pgfpathlineto{\pgfqpoint{3.890377in}{10.398868in}}%
\pgfpathlineto{\pgfqpoint{3.890377in}{10.486604in}}%
\pgfusepath{stroke,fill}%
\end{pgfscope}%
\begin{pgfscope}%
\pgfpathrectangle{\pgfqpoint{0.380943in}{9.960189in}}{\pgfqpoint{4.650000in}{0.614151in}}%
\pgfusepath{clip}%
\pgfsetbuttcap%
\pgfsetroundjoin%
\definecolor{currentfill}{rgb}{1.000000,1.000000,0.929412}%
\pgfsetfillcolor{currentfill}%
\pgfsetlinewidth{0.250937pt}%
\definecolor{currentstroke}{rgb}{1.000000,1.000000,1.000000}%
\pgfsetstrokecolor{currentstroke}%
\pgfsetdash{}{0pt}%
\pgfpathmoveto{\pgfqpoint{3.978113in}{10.486604in}}%
\pgfpathlineto{\pgfqpoint{4.065849in}{10.486604in}}%
\pgfpathlineto{\pgfqpoint{4.065849in}{10.398868in}}%
\pgfpathlineto{\pgfqpoint{3.978113in}{10.398868in}}%
\pgfpathlineto{\pgfqpoint{3.978113in}{10.486604in}}%
\pgfusepath{stroke,fill}%
\end{pgfscope}%
\begin{pgfscope}%
\pgfpathrectangle{\pgfqpoint{0.380943in}{9.960189in}}{\pgfqpoint{4.650000in}{0.614151in}}%
\pgfusepath{clip}%
\pgfsetbuttcap%
\pgfsetroundjoin%
\definecolor{currentfill}{rgb}{1.000000,1.000000,0.929412}%
\pgfsetfillcolor{currentfill}%
\pgfsetlinewidth{0.250937pt}%
\definecolor{currentstroke}{rgb}{1.000000,1.000000,1.000000}%
\pgfsetstrokecolor{currentstroke}%
\pgfsetdash{}{0pt}%
\pgfpathmoveto{\pgfqpoint{4.065849in}{10.486604in}}%
\pgfpathlineto{\pgfqpoint{4.153585in}{10.486604in}}%
\pgfpathlineto{\pgfqpoint{4.153585in}{10.398868in}}%
\pgfpathlineto{\pgfqpoint{4.065849in}{10.398868in}}%
\pgfpathlineto{\pgfqpoint{4.065849in}{10.486604in}}%
\pgfusepath{stroke,fill}%
\end{pgfscope}%
\begin{pgfscope}%
\pgfpathrectangle{\pgfqpoint{0.380943in}{9.960189in}}{\pgfqpoint{4.650000in}{0.614151in}}%
\pgfusepath{clip}%
\pgfsetbuttcap%
\pgfsetroundjoin%
\definecolor{currentfill}{rgb}{1.000000,1.000000,0.929412}%
\pgfsetfillcolor{currentfill}%
\pgfsetlinewidth{0.250937pt}%
\definecolor{currentstroke}{rgb}{1.000000,1.000000,1.000000}%
\pgfsetstrokecolor{currentstroke}%
\pgfsetdash{}{0pt}%
\pgfpathmoveto{\pgfqpoint{4.153585in}{10.486604in}}%
\pgfpathlineto{\pgfqpoint{4.241320in}{10.486604in}}%
\pgfpathlineto{\pgfqpoint{4.241320in}{10.398868in}}%
\pgfpathlineto{\pgfqpoint{4.153585in}{10.398868in}}%
\pgfpathlineto{\pgfqpoint{4.153585in}{10.486604in}}%
\pgfusepath{stroke,fill}%
\end{pgfscope}%
\begin{pgfscope}%
\pgfpathrectangle{\pgfqpoint{0.380943in}{9.960189in}}{\pgfqpoint{4.650000in}{0.614151in}}%
\pgfusepath{clip}%
\pgfsetbuttcap%
\pgfsetroundjoin%
\definecolor{currentfill}{rgb}{1.000000,1.000000,0.929412}%
\pgfsetfillcolor{currentfill}%
\pgfsetlinewidth{0.250937pt}%
\definecolor{currentstroke}{rgb}{1.000000,1.000000,1.000000}%
\pgfsetstrokecolor{currentstroke}%
\pgfsetdash{}{0pt}%
\pgfpathmoveto{\pgfqpoint{4.241320in}{10.486604in}}%
\pgfpathlineto{\pgfqpoint{4.329056in}{10.486604in}}%
\pgfpathlineto{\pgfqpoint{4.329056in}{10.398868in}}%
\pgfpathlineto{\pgfqpoint{4.241320in}{10.398868in}}%
\pgfpathlineto{\pgfqpoint{4.241320in}{10.486604in}}%
\pgfusepath{stroke,fill}%
\end{pgfscope}%
\begin{pgfscope}%
\pgfpathrectangle{\pgfqpoint{0.380943in}{9.960189in}}{\pgfqpoint{4.650000in}{0.614151in}}%
\pgfusepath{clip}%
\pgfsetbuttcap%
\pgfsetroundjoin%
\definecolor{currentfill}{rgb}{1.000000,1.000000,0.929412}%
\pgfsetfillcolor{currentfill}%
\pgfsetlinewidth{0.250937pt}%
\definecolor{currentstroke}{rgb}{1.000000,1.000000,1.000000}%
\pgfsetstrokecolor{currentstroke}%
\pgfsetdash{}{0pt}%
\pgfpathmoveto{\pgfqpoint{4.329056in}{10.486604in}}%
\pgfpathlineto{\pgfqpoint{4.416792in}{10.486604in}}%
\pgfpathlineto{\pgfqpoint{4.416792in}{10.398868in}}%
\pgfpathlineto{\pgfqpoint{4.329056in}{10.398868in}}%
\pgfpathlineto{\pgfqpoint{4.329056in}{10.486604in}}%
\pgfusepath{stroke,fill}%
\end{pgfscope}%
\begin{pgfscope}%
\pgfpathrectangle{\pgfqpoint{0.380943in}{9.960189in}}{\pgfqpoint{4.650000in}{0.614151in}}%
\pgfusepath{clip}%
\pgfsetbuttcap%
\pgfsetroundjoin%
\definecolor{currentfill}{rgb}{1.000000,1.000000,0.929412}%
\pgfsetfillcolor{currentfill}%
\pgfsetlinewidth{0.250937pt}%
\definecolor{currentstroke}{rgb}{1.000000,1.000000,1.000000}%
\pgfsetstrokecolor{currentstroke}%
\pgfsetdash{}{0pt}%
\pgfpathmoveto{\pgfqpoint{4.416792in}{10.486604in}}%
\pgfpathlineto{\pgfqpoint{4.504528in}{10.486604in}}%
\pgfpathlineto{\pgfqpoint{4.504528in}{10.398868in}}%
\pgfpathlineto{\pgfqpoint{4.416792in}{10.398868in}}%
\pgfpathlineto{\pgfqpoint{4.416792in}{10.486604in}}%
\pgfusepath{stroke,fill}%
\end{pgfscope}%
\begin{pgfscope}%
\pgfpathrectangle{\pgfqpoint{0.380943in}{9.960189in}}{\pgfqpoint{4.650000in}{0.614151in}}%
\pgfusepath{clip}%
\pgfsetbuttcap%
\pgfsetroundjoin%
\definecolor{currentfill}{rgb}{0.994694,0.745098,0.602999}%
\pgfsetfillcolor{currentfill}%
\pgfsetlinewidth{0.250937pt}%
\definecolor{currentstroke}{rgb}{1.000000,1.000000,1.000000}%
\pgfsetstrokecolor{currentstroke}%
\pgfsetdash{}{0pt}%
\pgfpathmoveto{\pgfqpoint{4.504528in}{10.486604in}}%
\pgfpathlineto{\pgfqpoint{4.592264in}{10.486604in}}%
\pgfpathlineto{\pgfqpoint{4.592264in}{10.398868in}}%
\pgfpathlineto{\pgfqpoint{4.504528in}{10.398868in}}%
\pgfpathlineto{\pgfqpoint{4.504528in}{10.486604in}}%
\pgfusepath{stroke,fill}%
\end{pgfscope}%
\begin{pgfscope}%
\pgfpathrectangle{\pgfqpoint{0.380943in}{9.960189in}}{\pgfqpoint{4.650000in}{0.614151in}}%
\pgfusepath{clip}%
\pgfsetbuttcap%
\pgfsetroundjoin%
\definecolor{currentfill}{rgb}{0.994694,0.745098,0.602999}%
\pgfsetfillcolor{currentfill}%
\pgfsetlinewidth{0.250937pt}%
\definecolor{currentstroke}{rgb}{1.000000,1.000000,1.000000}%
\pgfsetstrokecolor{currentstroke}%
\pgfsetdash{}{0pt}%
\pgfpathmoveto{\pgfqpoint{4.592264in}{10.486604in}}%
\pgfpathlineto{\pgfqpoint{4.680000in}{10.486604in}}%
\pgfpathlineto{\pgfqpoint{4.680000in}{10.398868in}}%
\pgfpathlineto{\pgfqpoint{4.592264in}{10.398868in}}%
\pgfpathlineto{\pgfqpoint{4.592264in}{10.486604in}}%
\pgfusepath{stroke,fill}%
\end{pgfscope}%
\begin{pgfscope}%
\pgfpathrectangle{\pgfqpoint{0.380943in}{9.960189in}}{\pgfqpoint{4.650000in}{0.614151in}}%
\pgfusepath{clip}%
\pgfsetbuttcap%
\pgfsetroundjoin%
\definecolor{currentfill}{rgb}{1.000000,0.522261,0.496886}%
\pgfsetfillcolor{currentfill}%
\pgfsetlinewidth{0.250937pt}%
\definecolor{currentstroke}{rgb}{1.000000,1.000000,1.000000}%
\pgfsetstrokecolor{currentstroke}%
\pgfsetdash{}{0pt}%
\pgfpathmoveto{\pgfqpoint{4.680000in}{10.486604in}}%
\pgfpathlineto{\pgfqpoint{4.767736in}{10.486604in}}%
\pgfpathlineto{\pgfqpoint{4.767736in}{10.398868in}}%
\pgfpathlineto{\pgfqpoint{4.680000in}{10.398868in}}%
\pgfpathlineto{\pgfqpoint{4.680000in}{10.486604in}}%
\pgfusepath{stroke,fill}%
\end{pgfscope}%
\begin{pgfscope}%
\pgfpathrectangle{\pgfqpoint{0.380943in}{9.960189in}}{\pgfqpoint{4.650000in}{0.614151in}}%
\pgfusepath{clip}%
\pgfsetbuttcap%
\pgfsetroundjoin%
\definecolor{currentfill}{rgb}{1.000000,0.622145,0.537486}%
\pgfsetfillcolor{currentfill}%
\pgfsetlinewidth{0.250937pt}%
\definecolor{currentstroke}{rgb}{1.000000,1.000000,1.000000}%
\pgfsetstrokecolor{currentstroke}%
\pgfsetdash{}{0pt}%
\pgfpathmoveto{\pgfqpoint{4.767736in}{10.486604in}}%
\pgfpathlineto{\pgfqpoint{4.855471in}{10.486604in}}%
\pgfpathlineto{\pgfqpoint{4.855471in}{10.398868in}}%
\pgfpathlineto{\pgfqpoint{4.767736in}{10.398868in}}%
\pgfpathlineto{\pgfqpoint{4.767736in}{10.486604in}}%
\pgfusepath{stroke,fill}%
\end{pgfscope}%
\begin{pgfscope}%
\pgfpathrectangle{\pgfqpoint{0.380943in}{9.960189in}}{\pgfqpoint{4.650000in}{0.614151in}}%
\pgfusepath{clip}%
\pgfsetbuttcap%
\pgfsetroundjoin%
\definecolor{currentfill}{rgb}{0.800000,0.278431,0.278431}%
\pgfsetfillcolor{currentfill}%
\pgfsetlinewidth{0.250937pt}%
\definecolor{currentstroke}{rgb}{1.000000,1.000000,1.000000}%
\pgfsetstrokecolor{currentstroke}%
\pgfsetdash{}{0pt}%
\pgfpathmoveto{\pgfqpoint{4.855471in}{10.486604in}}%
\pgfpathlineto{\pgfqpoint{4.943207in}{10.486604in}}%
\pgfpathlineto{\pgfqpoint{4.943207in}{10.398868in}}%
\pgfpathlineto{\pgfqpoint{4.855471in}{10.398868in}}%
\pgfpathlineto{\pgfqpoint{4.855471in}{10.486604in}}%
\pgfusepath{stroke,fill}%
\end{pgfscope}%
\begin{pgfscope}%
\pgfpathrectangle{\pgfqpoint{0.380943in}{9.960189in}}{\pgfqpoint{4.650000in}{0.614151in}}%
\pgfusepath{clip}%
\pgfsetbuttcap%
\pgfsetroundjoin%
\definecolor{currentfill}{rgb}{0.982699,0.823991,0.657439}%
\pgfsetfillcolor{currentfill}%
\pgfsetlinewidth{0.250937pt}%
\definecolor{currentstroke}{rgb}{1.000000,1.000000,1.000000}%
\pgfsetstrokecolor{currentstroke}%
\pgfsetdash{}{0pt}%
\pgfpathmoveto{\pgfqpoint{4.943207in}{10.486604in}}%
\pgfpathlineto{\pgfqpoint{5.030943in}{10.486604in}}%
\pgfpathlineto{\pgfqpoint{5.030943in}{10.398868in}}%
\pgfpathlineto{\pgfqpoint{4.943207in}{10.398868in}}%
\pgfpathlineto{\pgfqpoint{4.943207in}{10.486604in}}%
\pgfusepath{stroke,fill}%
\end{pgfscope}%
\begin{pgfscope}%
\pgfpathrectangle{\pgfqpoint{0.380943in}{9.960189in}}{\pgfqpoint{4.650000in}{0.614151in}}%
\pgfusepath{clip}%
\pgfsetbuttcap%
\pgfsetroundjoin%
\pgfsetlinewidth{0.250937pt}%
\definecolor{currentstroke}{rgb}{1.000000,1.000000,1.000000}%
\pgfsetstrokecolor{currentstroke}%
\pgfsetdash{}{0pt}%
\pgfpathmoveto{\pgfqpoint{0.380943in}{10.398868in}}%
\pgfpathlineto{\pgfqpoint{0.468679in}{10.398868in}}%
\pgfpathlineto{\pgfqpoint{0.468679in}{10.311132in}}%
\pgfpathlineto{\pgfqpoint{0.380943in}{10.311132in}}%
\pgfpathlineto{\pgfqpoint{0.380943in}{10.398868in}}%
\pgfusepath{stroke}%
\end{pgfscope}%
\begin{pgfscope}%
\pgfpathrectangle{\pgfqpoint{0.380943in}{9.960189in}}{\pgfqpoint{4.650000in}{0.614151in}}%
\pgfusepath{clip}%
\pgfsetbuttcap%
\pgfsetroundjoin%
\definecolor{currentfill}{rgb}{1.000000,1.000000,0.929412}%
\pgfsetfillcolor{currentfill}%
\pgfsetlinewidth{0.250937pt}%
\definecolor{currentstroke}{rgb}{1.000000,1.000000,1.000000}%
\pgfsetstrokecolor{currentstroke}%
\pgfsetdash{}{0pt}%
\pgfpathmoveto{\pgfqpoint{0.468679in}{10.398868in}}%
\pgfpathlineto{\pgfqpoint{0.556415in}{10.398868in}}%
\pgfpathlineto{\pgfqpoint{0.556415in}{10.311132in}}%
\pgfpathlineto{\pgfqpoint{0.468679in}{10.311132in}}%
\pgfpathlineto{\pgfqpoint{0.468679in}{10.398868in}}%
\pgfusepath{stroke,fill}%
\end{pgfscope}%
\begin{pgfscope}%
\pgfpathrectangle{\pgfqpoint{0.380943in}{9.960189in}}{\pgfqpoint{4.650000in}{0.614151in}}%
\pgfusepath{clip}%
\pgfsetbuttcap%
\pgfsetroundjoin%
\definecolor{currentfill}{rgb}{1.000000,1.000000,0.929412}%
\pgfsetfillcolor{currentfill}%
\pgfsetlinewidth{0.250937pt}%
\definecolor{currentstroke}{rgb}{1.000000,1.000000,1.000000}%
\pgfsetstrokecolor{currentstroke}%
\pgfsetdash{}{0pt}%
\pgfpathmoveto{\pgfqpoint{0.556415in}{10.398868in}}%
\pgfpathlineto{\pgfqpoint{0.644151in}{10.398868in}}%
\pgfpathlineto{\pgfqpoint{0.644151in}{10.311132in}}%
\pgfpathlineto{\pgfqpoint{0.556415in}{10.311132in}}%
\pgfpathlineto{\pgfqpoint{0.556415in}{10.398868in}}%
\pgfusepath{stroke,fill}%
\end{pgfscope}%
\begin{pgfscope}%
\pgfpathrectangle{\pgfqpoint{0.380943in}{9.960189in}}{\pgfqpoint{4.650000in}{0.614151in}}%
\pgfusepath{clip}%
\pgfsetbuttcap%
\pgfsetroundjoin%
\definecolor{currentfill}{rgb}{1.000000,1.000000,0.929412}%
\pgfsetfillcolor{currentfill}%
\pgfsetlinewidth{0.250937pt}%
\definecolor{currentstroke}{rgb}{1.000000,1.000000,1.000000}%
\pgfsetstrokecolor{currentstroke}%
\pgfsetdash{}{0pt}%
\pgfpathmoveto{\pgfqpoint{0.644151in}{10.398868in}}%
\pgfpathlineto{\pgfqpoint{0.731886in}{10.398868in}}%
\pgfpathlineto{\pgfqpoint{0.731886in}{10.311132in}}%
\pgfpathlineto{\pgfqpoint{0.644151in}{10.311132in}}%
\pgfpathlineto{\pgfqpoint{0.644151in}{10.398868in}}%
\pgfusepath{stroke,fill}%
\end{pgfscope}%
\begin{pgfscope}%
\pgfpathrectangle{\pgfqpoint{0.380943in}{9.960189in}}{\pgfqpoint{4.650000in}{0.614151in}}%
\pgfusepath{clip}%
\pgfsetbuttcap%
\pgfsetroundjoin%
\definecolor{currentfill}{rgb}{1.000000,1.000000,0.929412}%
\pgfsetfillcolor{currentfill}%
\pgfsetlinewidth{0.250937pt}%
\definecolor{currentstroke}{rgb}{1.000000,1.000000,1.000000}%
\pgfsetstrokecolor{currentstroke}%
\pgfsetdash{}{0pt}%
\pgfpathmoveto{\pgfqpoint{0.731886in}{10.398868in}}%
\pgfpathlineto{\pgfqpoint{0.819622in}{10.398868in}}%
\pgfpathlineto{\pgfqpoint{0.819622in}{10.311132in}}%
\pgfpathlineto{\pgfqpoint{0.731886in}{10.311132in}}%
\pgfpathlineto{\pgfqpoint{0.731886in}{10.398868in}}%
\pgfusepath{stroke,fill}%
\end{pgfscope}%
\begin{pgfscope}%
\pgfpathrectangle{\pgfqpoint{0.380943in}{9.960189in}}{\pgfqpoint{4.650000in}{0.614151in}}%
\pgfusepath{clip}%
\pgfsetbuttcap%
\pgfsetroundjoin%
\definecolor{currentfill}{rgb}{1.000000,1.000000,0.929412}%
\pgfsetfillcolor{currentfill}%
\pgfsetlinewidth{0.250937pt}%
\definecolor{currentstroke}{rgb}{1.000000,1.000000,1.000000}%
\pgfsetstrokecolor{currentstroke}%
\pgfsetdash{}{0pt}%
\pgfpathmoveto{\pgfqpoint{0.819622in}{10.398868in}}%
\pgfpathlineto{\pgfqpoint{0.907358in}{10.398868in}}%
\pgfpathlineto{\pgfqpoint{0.907358in}{10.311132in}}%
\pgfpathlineto{\pgfqpoint{0.819622in}{10.311132in}}%
\pgfpathlineto{\pgfqpoint{0.819622in}{10.398868in}}%
\pgfusepath{stroke,fill}%
\end{pgfscope}%
\begin{pgfscope}%
\pgfpathrectangle{\pgfqpoint{0.380943in}{9.960189in}}{\pgfqpoint{4.650000in}{0.614151in}}%
\pgfusepath{clip}%
\pgfsetbuttcap%
\pgfsetroundjoin%
\definecolor{currentfill}{rgb}{1.000000,1.000000,0.929412}%
\pgfsetfillcolor{currentfill}%
\pgfsetlinewidth{0.250937pt}%
\definecolor{currentstroke}{rgb}{1.000000,1.000000,1.000000}%
\pgfsetstrokecolor{currentstroke}%
\pgfsetdash{}{0pt}%
\pgfpathmoveto{\pgfqpoint{0.907358in}{10.398868in}}%
\pgfpathlineto{\pgfqpoint{0.995094in}{10.398868in}}%
\pgfpathlineto{\pgfqpoint{0.995094in}{10.311132in}}%
\pgfpathlineto{\pgfqpoint{0.907358in}{10.311132in}}%
\pgfpathlineto{\pgfqpoint{0.907358in}{10.398868in}}%
\pgfusepath{stroke,fill}%
\end{pgfscope}%
\begin{pgfscope}%
\pgfpathrectangle{\pgfqpoint{0.380943in}{9.960189in}}{\pgfqpoint{4.650000in}{0.614151in}}%
\pgfusepath{clip}%
\pgfsetbuttcap%
\pgfsetroundjoin%
\definecolor{currentfill}{rgb}{1.000000,1.000000,0.929412}%
\pgfsetfillcolor{currentfill}%
\pgfsetlinewidth{0.250937pt}%
\definecolor{currentstroke}{rgb}{1.000000,1.000000,1.000000}%
\pgfsetstrokecolor{currentstroke}%
\pgfsetdash{}{0pt}%
\pgfpathmoveto{\pgfqpoint{0.995094in}{10.398868in}}%
\pgfpathlineto{\pgfqpoint{1.082830in}{10.398868in}}%
\pgfpathlineto{\pgfqpoint{1.082830in}{10.311132in}}%
\pgfpathlineto{\pgfqpoint{0.995094in}{10.311132in}}%
\pgfpathlineto{\pgfqpoint{0.995094in}{10.398868in}}%
\pgfusepath{stroke,fill}%
\end{pgfscope}%
\begin{pgfscope}%
\pgfpathrectangle{\pgfqpoint{0.380943in}{9.960189in}}{\pgfqpoint{4.650000in}{0.614151in}}%
\pgfusepath{clip}%
\pgfsetbuttcap%
\pgfsetroundjoin%
\definecolor{currentfill}{rgb}{1.000000,1.000000,0.929412}%
\pgfsetfillcolor{currentfill}%
\pgfsetlinewidth{0.250937pt}%
\definecolor{currentstroke}{rgb}{1.000000,1.000000,1.000000}%
\pgfsetstrokecolor{currentstroke}%
\pgfsetdash{}{0pt}%
\pgfpathmoveto{\pgfqpoint{1.082830in}{10.398868in}}%
\pgfpathlineto{\pgfqpoint{1.170566in}{10.398868in}}%
\pgfpathlineto{\pgfqpoint{1.170566in}{10.311132in}}%
\pgfpathlineto{\pgfqpoint{1.082830in}{10.311132in}}%
\pgfpathlineto{\pgfqpoint{1.082830in}{10.398868in}}%
\pgfusepath{stroke,fill}%
\end{pgfscope}%
\begin{pgfscope}%
\pgfpathrectangle{\pgfqpoint{0.380943in}{9.960189in}}{\pgfqpoint{4.650000in}{0.614151in}}%
\pgfusepath{clip}%
\pgfsetbuttcap%
\pgfsetroundjoin%
\definecolor{currentfill}{rgb}{1.000000,1.000000,0.929412}%
\pgfsetfillcolor{currentfill}%
\pgfsetlinewidth{0.250937pt}%
\definecolor{currentstroke}{rgb}{1.000000,1.000000,1.000000}%
\pgfsetstrokecolor{currentstroke}%
\pgfsetdash{}{0pt}%
\pgfpathmoveto{\pgfqpoint{1.170566in}{10.398868in}}%
\pgfpathlineto{\pgfqpoint{1.258302in}{10.398868in}}%
\pgfpathlineto{\pgfqpoint{1.258302in}{10.311132in}}%
\pgfpathlineto{\pgfqpoint{1.170566in}{10.311132in}}%
\pgfpathlineto{\pgfqpoint{1.170566in}{10.398868in}}%
\pgfusepath{stroke,fill}%
\end{pgfscope}%
\begin{pgfscope}%
\pgfpathrectangle{\pgfqpoint{0.380943in}{9.960189in}}{\pgfqpoint{4.650000in}{0.614151in}}%
\pgfusepath{clip}%
\pgfsetbuttcap%
\pgfsetroundjoin%
\definecolor{currentfill}{rgb}{1.000000,1.000000,0.929412}%
\pgfsetfillcolor{currentfill}%
\pgfsetlinewidth{0.250937pt}%
\definecolor{currentstroke}{rgb}{1.000000,1.000000,1.000000}%
\pgfsetstrokecolor{currentstroke}%
\pgfsetdash{}{0pt}%
\pgfpathmoveto{\pgfqpoint{1.258302in}{10.398868in}}%
\pgfpathlineto{\pgfqpoint{1.346037in}{10.398868in}}%
\pgfpathlineto{\pgfqpoint{1.346037in}{10.311132in}}%
\pgfpathlineto{\pgfqpoint{1.258302in}{10.311132in}}%
\pgfpathlineto{\pgfqpoint{1.258302in}{10.398868in}}%
\pgfusepath{stroke,fill}%
\end{pgfscope}%
\begin{pgfscope}%
\pgfpathrectangle{\pgfqpoint{0.380943in}{9.960189in}}{\pgfqpoint{4.650000in}{0.614151in}}%
\pgfusepath{clip}%
\pgfsetbuttcap%
\pgfsetroundjoin%
\definecolor{currentfill}{rgb}{1.000000,1.000000,0.929412}%
\pgfsetfillcolor{currentfill}%
\pgfsetlinewidth{0.250937pt}%
\definecolor{currentstroke}{rgb}{1.000000,1.000000,1.000000}%
\pgfsetstrokecolor{currentstroke}%
\pgfsetdash{}{0pt}%
\pgfpathmoveto{\pgfqpoint{1.346037in}{10.398868in}}%
\pgfpathlineto{\pgfqpoint{1.433773in}{10.398868in}}%
\pgfpathlineto{\pgfqpoint{1.433773in}{10.311132in}}%
\pgfpathlineto{\pgfqpoint{1.346037in}{10.311132in}}%
\pgfpathlineto{\pgfqpoint{1.346037in}{10.398868in}}%
\pgfusepath{stroke,fill}%
\end{pgfscope}%
\begin{pgfscope}%
\pgfpathrectangle{\pgfqpoint{0.380943in}{9.960189in}}{\pgfqpoint{4.650000in}{0.614151in}}%
\pgfusepath{clip}%
\pgfsetbuttcap%
\pgfsetroundjoin%
\definecolor{currentfill}{rgb}{1.000000,1.000000,0.929412}%
\pgfsetfillcolor{currentfill}%
\pgfsetlinewidth{0.250937pt}%
\definecolor{currentstroke}{rgb}{1.000000,1.000000,1.000000}%
\pgfsetstrokecolor{currentstroke}%
\pgfsetdash{}{0pt}%
\pgfpathmoveto{\pgfqpoint{1.433773in}{10.398868in}}%
\pgfpathlineto{\pgfqpoint{1.521509in}{10.398868in}}%
\pgfpathlineto{\pgfqpoint{1.521509in}{10.311132in}}%
\pgfpathlineto{\pgfqpoint{1.433773in}{10.311132in}}%
\pgfpathlineto{\pgfqpoint{1.433773in}{10.398868in}}%
\pgfusepath{stroke,fill}%
\end{pgfscope}%
\begin{pgfscope}%
\pgfpathrectangle{\pgfqpoint{0.380943in}{9.960189in}}{\pgfqpoint{4.650000in}{0.614151in}}%
\pgfusepath{clip}%
\pgfsetbuttcap%
\pgfsetroundjoin%
\definecolor{currentfill}{rgb}{1.000000,1.000000,0.929412}%
\pgfsetfillcolor{currentfill}%
\pgfsetlinewidth{0.250937pt}%
\definecolor{currentstroke}{rgb}{1.000000,1.000000,1.000000}%
\pgfsetstrokecolor{currentstroke}%
\pgfsetdash{}{0pt}%
\pgfpathmoveto{\pgfqpoint{1.521509in}{10.398868in}}%
\pgfpathlineto{\pgfqpoint{1.609245in}{10.398868in}}%
\pgfpathlineto{\pgfqpoint{1.609245in}{10.311132in}}%
\pgfpathlineto{\pgfqpoint{1.521509in}{10.311132in}}%
\pgfpathlineto{\pgfqpoint{1.521509in}{10.398868in}}%
\pgfusepath{stroke,fill}%
\end{pgfscope}%
\begin{pgfscope}%
\pgfpathrectangle{\pgfqpoint{0.380943in}{9.960189in}}{\pgfqpoint{4.650000in}{0.614151in}}%
\pgfusepath{clip}%
\pgfsetbuttcap%
\pgfsetroundjoin%
\definecolor{currentfill}{rgb}{1.000000,1.000000,0.929412}%
\pgfsetfillcolor{currentfill}%
\pgfsetlinewidth{0.250937pt}%
\definecolor{currentstroke}{rgb}{1.000000,1.000000,1.000000}%
\pgfsetstrokecolor{currentstroke}%
\pgfsetdash{}{0pt}%
\pgfpathmoveto{\pgfqpoint{1.609245in}{10.398868in}}%
\pgfpathlineto{\pgfqpoint{1.696981in}{10.398868in}}%
\pgfpathlineto{\pgfqpoint{1.696981in}{10.311132in}}%
\pgfpathlineto{\pgfqpoint{1.609245in}{10.311132in}}%
\pgfpathlineto{\pgfqpoint{1.609245in}{10.398868in}}%
\pgfusepath{stroke,fill}%
\end{pgfscope}%
\begin{pgfscope}%
\pgfpathrectangle{\pgfqpoint{0.380943in}{9.960189in}}{\pgfqpoint{4.650000in}{0.614151in}}%
\pgfusepath{clip}%
\pgfsetbuttcap%
\pgfsetroundjoin%
\definecolor{currentfill}{rgb}{1.000000,1.000000,0.929412}%
\pgfsetfillcolor{currentfill}%
\pgfsetlinewidth{0.250937pt}%
\definecolor{currentstroke}{rgb}{1.000000,1.000000,1.000000}%
\pgfsetstrokecolor{currentstroke}%
\pgfsetdash{}{0pt}%
\pgfpathmoveto{\pgfqpoint{1.696981in}{10.398868in}}%
\pgfpathlineto{\pgfqpoint{1.784717in}{10.398868in}}%
\pgfpathlineto{\pgfqpoint{1.784717in}{10.311132in}}%
\pgfpathlineto{\pgfqpoint{1.696981in}{10.311132in}}%
\pgfpathlineto{\pgfqpoint{1.696981in}{10.398868in}}%
\pgfusepath{stroke,fill}%
\end{pgfscope}%
\begin{pgfscope}%
\pgfpathrectangle{\pgfqpoint{0.380943in}{9.960189in}}{\pgfqpoint{4.650000in}{0.614151in}}%
\pgfusepath{clip}%
\pgfsetbuttcap%
\pgfsetroundjoin%
\definecolor{currentfill}{rgb}{1.000000,1.000000,0.929412}%
\pgfsetfillcolor{currentfill}%
\pgfsetlinewidth{0.250937pt}%
\definecolor{currentstroke}{rgb}{1.000000,1.000000,1.000000}%
\pgfsetstrokecolor{currentstroke}%
\pgfsetdash{}{0pt}%
\pgfpathmoveto{\pgfqpoint{1.784717in}{10.398868in}}%
\pgfpathlineto{\pgfqpoint{1.872452in}{10.398868in}}%
\pgfpathlineto{\pgfqpoint{1.872452in}{10.311132in}}%
\pgfpathlineto{\pgfqpoint{1.784717in}{10.311132in}}%
\pgfpathlineto{\pgfqpoint{1.784717in}{10.398868in}}%
\pgfusepath{stroke,fill}%
\end{pgfscope}%
\begin{pgfscope}%
\pgfpathrectangle{\pgfqpoint{0.380943in}{9.960189in}}{\pgfqpoint{4.650000in}{0.614151in}}%
\pgfusepath{clip}%
\pgfsetbuttcap%
\pgfsetroundjoin%
\definecolor{currentfill}{rgb}{1.000000,1.000000,0.929412}%
\pgfsetfillcolor{currentfill}%
\pgfsetlinewidth{0.250937pt}%
\definecolor{currentstroke}{rgb}{1.000000,1.000000,1.000000}%
\pgfsetstrokecolor{currentstroke}%
\pgfsetdash{}{0pt}%
\pgfpathmoveto{\pgfqpoint{1.872452in}{10.398868in}}%
\pgfpathlineto{\pgfqpoint{1.960188in}{10.398868in}}%
\pgfpathlineto{\pgfqpoint{1.960188in}{10.311132in}}%
\pgfpathlineto{\pgfqpoint{1.872452in}{10.311132in}}%
\pgfpathlineto{\pgfqpoint{1.872452in}{10.398868in}}%
\pgfusepath{stroke,fill}%
\end{pgfscope}%
\begin{pgfscope}%
\pgfpathrectangle{\pgfqpoint{0.380943in}{9.960189in}}{\pgfqpoint{4.650000in}{0.614151in}}%
\pgfusepath{clip}%
\pgfsetbuttcap%
\pgfsetroundjoin%
\definecolor{currentfill}{rgb}{1.000000,1.000000,0.929412}%
\pgfsetfillcolor{currentfill}%
\pgfsetlinewidth{0.250937pt}%
\definecolor{currentstroke}{rgb}{1.000000,1.000000,1.000000}%
\pgfsetstrokecolor{currentstroke}%
\pgfsetdash{}{0pt}%
\pgfpathmoveto{\pgfqpoint{1.960188in}{10.398868in}}%
\pgfpathlineto{\pgfqpoint{2.047924in}{10.398868in}}%
\pgfpathlineto{\pgfqpoint{2.047924in}{10.311132in}}%
\pgfpathlineto{\pgfqpoint{1.960188in}{10.311132in}}%
\pgfpathlineto{\pgfqpoint{1.960188in}{10.398868in}}%
\pgfusepath{stroke,fill}%
\end{pgfscope}%
\begin{pgfscope}%
\pgfpathrectangle{\pgfqpoint{0.380943in}{9.960189in}}{\pgfqpoint{4.650000in}{0.614151in}}%
\pgfusepath{clip}%
\pgfsetbuttcap%
\pgfsetroundjoin%
\definecolor{currentfill}{rgb}{1.000000,1.000000,0.929412}%
\pgfsetfillcolor{currentfill}%
\pgfsetlinewidth{0.250937pt}%
\definecolor{currentstroke}{rgb}{1.000000,1.000000,1.000000}%
\pgfsetstrokecolor{currentstroke}%
\pgfsetdash{}{0pt}%
\pgfpathmoveto{\pgfqpoint{2.047924in}{10.398868in}}%
\pgfpathlineto{\pgfqpoint{2.135660in}{10.398868in}}%
\pgfpathlineto{\pgfqpoint{2.135660in}{10.311132in}}%
\pgfpathlineto{\pgfqpoint{2.047924in}{10.311132in}}%
\pgfpathlineto{\pgfqpoint{2.047924in}{10.398868in}}%
\pgfusepath{stroke,fill}%
\end{pgfscope}%
\begin{pgfscope}%
\pgfpathrectangle{\pgfqpoint{0.380943in}{9.960189in}}{\pgfqpoint{4.650000in}{0.614151in}}%
\pgfusepath{clip}%
\pgfsetbuttcap%
\pgfsetroundjoin%
\definecolor{currentfill}{rgb}{1.000000,1.000000,0.929412}%
\pgfsetfillcolor{currentfill}%
\pgfsetlinewidth{0.250937pt}%
\definecolor{currentstroke}{rgb}{1.000000,1.000000,1.000000}%
\pgfsetstrokecolor{currentstroke}%
\pgfsetdash{}{0pt}%
\pgfpathmoveto{\pgfqpoint{2.135660in}{10.398868in}}%
\pgfpathlineto{\pgfqpoint{2.223396in}{10.398868in}}%
\pgfpathlineto{\pgfqpoint{2.223396in}{10.311132in}}%
\pgfpathlineto{\pgfqpoint{2.135660in}{10.311132in}}%
\pgfpathlineto{\pgfqpoint{2.135660in}{10.398868in}}%
\pgfusepath{stroke,fill}%
\end{pgfscope}%
\begin{pgfscope}%
\pgfpathrectangle{\pgfqpoint{0.380943in}{9.960189in}}{\pgfqpoint{4.650000in}{0.614151in}}%
\pgfusepath{clip}%
\pgfsetbuttcap%
\pgfsetroundjoin%
\definecolor{currentfill}{rgb}{1.000000,1.000000,0.929412}%
\pgfsetfillcolor{currentfill}%
\pgfsetlinewidth{0.250937pt}%
\definecolor{currentstroke}{rgb}{1.000000,1.000000,1.000000}%
\pgfsetstrokecolor{currentstroke}%
\pgfsetdash{}{0pt}%
\pgfpathmoveto{\pgfqpoint{2.223396in}{10.398868in}}%
\pgfpathlineto{\pgfqpoint{2.311132in}{10.398868in}}%
\pgfpathlineto{\pgfqpoint{2.311132in}{10.311132in}}%
\pgfpathlineto{\pgfqpoint{2.223396in}{10.311132in}}%
\pgfpathlineto{\pgfqpoint{2.223396in}{10.398868in}}%
\pgfusepath{stroke,fill}%
\end{pgfscope}%
\begin{pgfscope}%
\pgfpathrectangle{\pgfqpoint{0.380943in}{9.960189in}}{\pgfqpoint{4.650000in}{0.614151in}}%
\pgfusepath{clip}%
\pgfsetbuttcap%
\pgfsetroundjoin%
\definecolor{currentfill}{rgb}{1.000000,1.000000,0.929412}%
\pgfsetfillcolor{currentfill}%
\pgfsetlinewidth{0.250937pt}%
\definecolor{currentstroke}{rgb}{1.000000,1.000000,1.000000}%
\pgfsetstrokecolor{currentstroke}%
\pgfsetdash{}{0pt}%
\pgfpathmoveto{\pgfqpoint{2.311132in}{10.398868in}}%
\pgfpathlineto{\pgfqpoint{2.398868in}{10.398868in}}%
\pgfpathlineto{\pgfqpoint{2.398868in}{10.311132in}}%
\pgfpathlineto{\pgfqpoint{2.311132in}{10.311132in}}%
\pgfpathlineto{\pgfqpoint{2.311132in}{10.398868in}}%
\pgfusepath{stroke,fill}%
\end{pgfscope}%
\begin{pgfscope}%
\pgfpathrectangle{\pgfqpoint{0.380943in}{9.960189in}}{\pgfqpoint{4.650000in}{0.614151in}}%
\pgfusepath{clip}%
\pgfsetbuttcap%
\pgfsetroundjoin%
\definecolor{currentfill}{rgb}{1.000000,1.000000,0.929412}%
\pgfsetfillcolor{currentfill}%
\pgfsetlinewidth{0.250937pt}%
\definecolor{currentstroke}{rgb}{1.000000,1.000000,1.000000}%
\pgfsetstrokecolor{currentstroke}%
\pgfsetdash{}{0pt}%
\pgfpathmoveto{\pgfqpoint{2.398868in}{10.398868in}}%
\pgfpathlineto{\pgfqpoint{2.486603in}{10.398868in}}%
\pgfpathlineto{\pgfqpoint{2.486603in}{10.311132in}}%
\pgfpathlineto{\pgfqpoint{2.398868in}{10.311132in}}%
\pgfpathlineto{\pgfqpoint{2.398868in}{10.398868in}}%
\pgfusepath{stroke,fill}%
\end{pgfscope}%
\begin{pgfscope}%
\pgfpathrectangle{\pgfqpoint{0.380943in}{9.960189in}}{\pgfqpoint{4.650000in}{0.614151in}}%
\pgfusepath{clip}%
\pgfsetbuttcap%
\pgfsetroundjoin%
\definecolor{currentfill}{rgb}{1.000000,1.000000,0.929412}%
\pgfsetfillcolor{currentfill}%
\pgfsetlinewidth{0.250937pt}%
\definecolor{currentstroke}{rgb}{1.000000,1.000000,1.000000}%
\pgfsetstrokecolor{currentstroke}%
\pgfsetdash{}{0pt}%
\pgfpathmoveto{\pgfqpoint{2.486603in}{10.398868in}}%
\pgfpathlineto{\pgfqpoint{2.574339in}{10.398868in}}%
\pgfpathlineto{\pgfqpoint{2.574339in}{10.311132in}}%
\pgfpathlineto{\pgfqpoint{2.486603in}{10.311132in}}%
\pgfpathlineto{\pgfqpoint{2.486603in}{10.398868in}}%
\pgfusepath{stroke,fill}%
\end{pgfscope}%
\begin{pgfscope}%
\pgfpathrectangle{\pgfqpoint{0.380943in}{9.960189in}}{\pgfqpoint{4.650000in}{0.614151in}}%
\pgfusepath{clip}%
\pgfsetbuttcap%
\pgfsetroundjoin%
\definecolor{currentfill}{rgb}{1.000000,1.000000,0.929412}%
\pgfsetfillcolor{currentfill}%
\pgfsetlinewidth{0.250937pt}%
\definecolor{currentstroke}{rgb}{1.000000,1.000000,1.000000}%
\pgfsetstrokecolor{currentstroke}%
\pgfsetdash{}{0pt}%
\pgfpathmoveto{\pgfqpoint{2.574339in}{10.398868in}}%
\pgfpathlineto{\pgfqpoint{2.662075in}{10.398868in}}%
\pgfpathlineto{\pgfqpoint{2.662075in}{10.311132in}}%
\pgfpathlineto{\pgfqpoint{2.574339in}{10.311132in}}%
\pgfpathlineto{\pgfqpoint{2.574339in}{10.398868in}}%
\pgfusepath{stroke,fill}%
\end{pgfscope}%
\begin{pgfscope}%
\pgfpathrectangle{\pgfqpoint{0.380943in}{9.960189in}}{\pgfqpoint{4.650000in}{0.614151in}}%
\pgfusepath{clip}%
\pgfsetbuttcap%
\pgfsetroundjoin%
\definecolor{currentfill}{rgb}{1.000000,1.000000,0.929412}%
\pgfsetfillcolor{currentfill}%
\pgfsetlinewidth{0.250937pt}%
\definecolor{currentstroke}{rgb}{1.000000,1.000000,1.000000}%
\pgfsetstrokecolor{currentstroke}%
\pgfsetdash{}{0pt}%
\pgfpathmoveto{\pgfqpoint{2.662075in}{10.398868in}}%
\pgfpathlineto{\pgfqpoint{2.749811in}{10.398868in}}%
\pgfpathlineto{\pgfqpoint{2.749811in}{10.311132in}}%
\pgfpathlineto{\pgfqpoint{2.662075in}{10.311132in}}%
\pgfpathlineto{\pgfqpoint{2.662075in}{10.398868in}}%
\pgfusepath{stroke,fill}%
\end{pgfscope}%
\begin{pgfscope}%
\pgfpathrectangle{\pgfqpoint{0.380943in}{9.960189in}}{\pgfqpoint{4.650000in}{0.614151in}}%
\pgfusepath{clip}%
\pgfsetbuttcap%
\pgfsetroundjoin%
\definecolor{currentfill}{rgb}{1.000000,1.000000,0.929412}%
\pgfsetfillcolor{currentfill}%
\pgfsetlinewidth{0.250937pt}%
\definecolor{currentstroke}{rgb}{1.000000,1.000000,1.000000}%
\pgfsetstrokecolor{currentstroke}%
\pgfsetdash{}{0pt}%
\pgfpathmoveto{\pgfqpoint{2.749811in}{10.398868in}}%
\pgfpathlineto{\pgfqpoint{2.837547in}{10.398868in}}%
\pgfpathlineto{\pgfqpoint{2.837547in}{10.311132in}}%
\pgfpathlineto{\pgfqpoint{2.749811in}{10.311132in}}%
\pgfpathlineto{\pgfqpoint{2.749811in}{10.398868in}}%
\pgfusepath{stroke,fill}%
\end{pgfscope}%
\begin{pgfscope}%
\pgfpathrectangle{\pgfqpoint{0.380943in}{9.960189in}}{\pgfqpoint{4.650000in}{0.614151in}}%
\pgfusepath{clip}%
\pgfsetbuttcap%
\pgfsetroundjoin%
\definecolor{currentfill}{rgb}{1.000000,1.000000,0.929412}%
\pgfsetfillcolor{currentfill}%
\pgfsetlinewidth{0.250937pt}%
\definecolor{currentstroke}{rgb}{1.000000,1.000000,1.000000}%
\pgfsetstrokecolor{currentstroke}%
\pgfsetdash{}{0pt}%
\pgfpathmoveto{\pgfqpoint{2.837547in}{10.398868in}}%
\pgfpathlineto{\pgfqpoint{2.925283in}{10.398868in}}%
\pgfpathlineto{\pgfqpoint{2.925283in}{10.311132in}}%
\pgfpathlineto{\pgfqpoint{2.837547in}{10.311132in}}%
\pgfpathlineto{\pgfqpoint{2.837547in}{10.398868in}}%
\pgfusepath{stroke,fill}%
\end{pgfscope}%
\begin{pgfscope}%
\pgfpathrectangle{\pgfqpoint{0.380943in}{9.960189in}}{\pgfqpoint{4.650000in}{0.614151in}}%
\pgfusepath{clip}%
\pgfsetbuttcap%
\pgfsetroundjoin%
\definecolor{currentfill}{rgb}{1.000000,1.000000,0.929412}%
\pgfsetfillcolor{currentfill}%
\pgfsetlinewidth{0.250937pt}%
\definecolor{currentstroke}{rgb}{1.000000,1.000000,1.000000}%
\pgfsetstrokecolor{currentstroke}%
\pgfsetdash{}{0pt}%
\pgfpathmoveto{\pgfqpoint{2.925283in}{10.398868in}}%
\pgfpathlineto{\pgfqpoint{3.013019in}{10.398868in}}%
\pgfpathlineto{\pgfqpoint{3.013019in}{10.311132in}}%
\pgfpathlineto{\pgfqpoint{2.925283in}{10.311132in}}%
\pgfpathlineto{\pgfqpoint{2.925283in}{10.398868in}}%
\pgfusepath{stroke,fill}%
\end{pgfscope}%
\begin{pgfscope}%
\pgfpathrectangle{\pgfqpoint{0.380943in}{9.960189in}}{\pgfqpoint{4.650000in}{0.614151in}}%
\pgfusepath{clip}%
\pgfsetbuttcap%
\pgfsetroundjoin%
\definecolor{currentfill}{rgb}{1.000000,1.000000,0.929412}%
\pgfsetfillcolor{currentfill}%
\pgfsetlinewidth{0.250937pt}%
\definecolor{currentstroke}{rgb}{1.000000,1.000000,1.000000}%
\pgfsetstrokecolor{currentstroke}%
\pgfsetdash{}{0pt}%
\pgfpathmoveto{\pgfqpoint{3.013019in}{10.398868in}}%
\pgfpathlineto{\pgfqpoint{3.100754in}{10.398868in}}%
\pgfpathlineto{\pgfqpoint{3.100754in}{10.311132in}}%
\pgfpathlineto{\pgfqpoint{3.013019in}{10.311132in}}%
\pgfpathlineto{\pgfqpoint{3.013019in}{10.398868in}}%
\pgfusepath{stroke,fill}%
\end{pgfscope}%
\begin{pgfscope}%
\pgfpathrectangle{\pgfqpoint{0.380943in}{9.960189in}}{\pgfqpoint{4.650000in}{0.614151in}}%
\pgfusepath{clip}%
\pgfsetbuttcap%
\pgfsetroundjoin%
\definecolor{currentfill}{rgb}{1.000000,1.000000,0.929412}%
\pgfsetfillcolor{currentfill}%
\pgfsetlinewidth{0.250937pt}%
\definecolor{currentstroke}{rgb}{1.000000,1.000000,1.000000}%
\pgfsetstrokecolor{currentstroke}%
\pgfsetdash{}{0pt}%
\pgfpathmoveto{\pgfqpoint{3.100754in}{10.398868in}}%
\pgfpathlineto{\pgfqpoint{3.188490in}{10.398868in}}%
\pgfpathlineto{\pgfqpoint{3.188490in}{10.311132in}}%
\pgfpathlineto{\pgfqpoint{3.100754in}{10.311132in}}%
\pgfpathlineto{\pgfqpoint{3.100754in}{10.398868in}}%
\pgfusepath{stroke,fill}%
\end{pgfscope}%
\begin{pgfscope}%
\pgfpathrectangle{\pgfqpoint{0.380943in}{9.960189in}}{\pgfqpoint{4.650000in}{0.614151in}}%
\pgfusepath{clip}%
\pgfsetbuttcap%
\pgfsetroundjoin%
\definecolor{currentfill}{rgb}{1.000000,1.000000,0.929412}%
\pgfsetfillcolor{currentfill}%
\pgfsetlinewidth{0.250937pt}%
\definecolor{currentstroke}{rgb}{1.000000,1.000000,1.000000}%
\pgfsetstrokecolor{currentstroke}%
\pgfsetdash{}{0pt}%
\pgfpathmoveto{\pgfqpoint{3.188490in}{10.398868in}}%
\pgfpathlineto{\pgfqpoint{3.276226in}{10.398868in}}%
\pgfpathlineto{\pgfqpoint{3.276226in}{10.311132in}}%
\pgfpathlineto{\pgfqpoint{3.188490in}{10.311132in}}%
\pgfpathlineto{\pgfqpoint{3.188490in}{10.398868in}}%
\pgfusepath{stroke,fill}%
\end{pgfscope}%
\begin{pgfscope}%
\pgfpathrectangle{\pgfqpoint{0.380943in}{9.960189in}}{\pgfqpoint{4.650000in}{0.614151in}}%
\pgfusepath{clip}%
\pgfsetbuttcap%
\pgfsetroundjoin%
\definecolor{currentfill}{rgb}{1.000000,1.000000,0.929412}%
\pgfsetfillcolor{currentfill}%
\pgfsetlinewidth{0.250937pt}%
\definecolor{currentstroke}{rgb}{1.000000,1.000000,1.000000}%
\pgfsetstrokecolor{currentstroke}%
\pgfsetdash{}{0pt}%
\pgfpathmoveto{\pgfqpoint{3.276226in}{10.398868in}}%
\pgfpathlineto{\pgfqpoint{3.363962in}{10.398868in}}%
\pgfpathlineto{\pgfqpoint{3.363962in}{10.311132in}}%
\pgfpathlineto{\pgfqpoint{3.276226in}{10.311132in}}%
\pgfpathlineto{\pgfqpoint{3.276226in}{10.398868in}}%
\pgfusepath{stroke,fill}%
\end{pgfscope}%
\begin{pgfscope}%
\pgfpathrectangle{\pgfqpoint{0.380943in}{9.960189in}}{\pgfqpoint{4.650000in}{0.614151in}}%
\pgfusepath{clip}%
\pgfsetbuttcap%
\pgfsetroundjoin%
\definecolor{currentfill}{rgb}{1.000000,1.000000,0.929412}%
\pgfsetfillcolor{currentfill}%
\pgfsetlinewidth{0.250937pt}%
\definecolor{currentstroke}{rgb}{1.000000,1.000000,1.000000}%
\pgfsetstrokecolor{currentstroke}%
\pgfsetdash{}{0pt}%
\pgfpathmoveto{\pgfqpoint{3.363962in}{10.398868in}}%
\pgfpathlineto{\pgfqpoint{3.451698in}{10.398868in}}%
\pgfpathlineto{\pgfqpoint{3.451698in}{10.311132in}}%
\pgfpathlineto{\pgfqpoint{3.363962in}{10.311132in}}%
\pgfpathlineto{\pgfqpoint{3.363962in}{10.398868in}}%
\pgfusepath{stroke,fill}%
\end{pgfscope}%
\begin{pgfscope}%
\pgfpathrectangle{\pgfqpoint{0.380943in}{9.960189in}}{\pgfqpoint{4.650000in}{0.614151in}}%
\pgfusepath{clip}%
\pgfsetbuttcap%
\pgfsetroundjoin%
\definecolor{currentfill}{rgb}{1.000000,1.000000,0.929412}%
\pgfsetfillcolor{currentfill}%
\pgfsetlinewidth{0.250937pt}%
\definecolor{currentstroke}{rgb}{1.000000,1.000000,1.000000}%
\pgfsetstrokecolor{currentstroke}%
\pgfsetdash{}{0pt}%
\pgfpathmoveto{\pgfqpoint{3.451698in}{10.398868in}}%
\pgfpathlineto{\pgfqpoint{3.539434in}{10.398868in}}%
\pgfpathlineto{\pgfqpoint{3.539434in}{10.311132in}}%
\pgfpathlineto{\pgfqpoint{3.451698in}{10.311132in}}%
\pgfpathlineto{\pgfqpoint{3.451698in}{10.398868in}}%
\pgfusepath{stroke,fill}%
\end{pgfscope}%
\begin{pgfscope}%
\pgfpathrectangle{\pgfqpoint{0.380943in}{9.960189in}}{\pgfqpoint{4.650000in}{0.614151in}}%
\pgfusepath{clip}%
\pgfsetbuttcap%
\pgfsetroundjoin%
\definecolor{currentfill}{rgb}{1.000000,1.000000,0.929412}%
\pgfsetfillcolor{currentfill}%
\pgfsetlinewidth{0.250937pt}%
\definecolor{currentstroke}{rgb}{1.000000,1.000000,1.000000}%
\pgfsetstrokecolor{currentstroke}%
\pgfsetdash{}{0pt}%
\pgfpathmoveto{\pgfqpoint{3.539434in}{10.398868in}}%
\pgfpathlineto{\pgfqpoint{3.627169in}{10.398868in}}%
\pgfpathlineto{\pgfqpoint{3.627169in}{10.311132in}}%
\pgfpathlineto{\pgfqpoint{3.539434in}{10.311132in}}%
\pgfpathlineto{\pgfqpoint{3.539434in}{10.398868in}}%
\pgfusepath{stroke,fill}%
\end{pgfscope}%
\begin{pgfscope}%
\pgfpathrectangle{\pgfqpoint{0.380943in}{9.960189in}}{\pgfqpoint{4.650000in}{0.614151in}}%
\pgfusepath{clip}%
\pgfsetbuttcap%
\pgfsetroundjoin%
\definecolor{currentfill}{rgb}{1.000000,1.000000,0.929412}%
\pgfsetfillcolor{currentfill}%
\pgfsetlinewidth{0.250937pt}%
\definecolor{currentstroke}{rgb}{1.000000,1.000000,1.000000}%
\pgfsetstrokecolor{currentstroke}%
\pgfsetdash{}{0pt}%
\pgfpathmoveto{\pgfqpoint{3.627169in}{10.398868in}}%
\pgfpathlineto{\pgfqpoint{3.714905in}{10.398868in}}%
\pgfpathlineto{\pgfqpoint{3.714905in}{10.311132in}}%
\pgfpathlineto{\pgfqpoint{3.627169in}{10.311132in}}%
\pgfpathlineto{\pgfqpoint{3.627169in}{10.398868in}}%
\pgfusepath{stroke,fill}%
\end{pgfscope}%
\begin{pgfscope}%
\pgfpathrectangle{\pgfqpoint{0.380943in}{9.960189in}}{\pgfqpoint{4.650000in}{0.614151in}}%
\pgfusepath{clip}%
\pgfsetbuttcap%
\pgfsetroundjoin%
\definecolor{currentfill}{rgb}{1.000000,1.000000,0.929412}%
\pgfsetfillcolor{currentfill}%
\pgfsetlinewidth{0.250937pt}%
\definecolor{currentstroke}{rgb}{1.000000,1.000000,1.000000}%
\pgfsetstrokecolor{currentstroke}%
\pgfsetdash{}{0pt}%
\pgfpathmoveto{\pgfqpoint{3.714905in}{10.398868in}}%
\pgfpathlineto{\pgfqpoint{3.802641in}{10.398868in}}%
\pgfpathlineto{\pgfqpoint{3.802641in}{10.311132in}}%
\pgfpathlineto{\pgfqpoint{3.714905in}{10.311132in}}%
\pgfpathlineto{\pgfqpoint{3.714905in}{10.398868in}}%
\pgfusepath{stroke,fill}%
\end{pgfscope}%
\begin{pgfscope}%
\pgfpathrectangle{\pgfqpoint{0.380943in}{9.960189in}}{\pgfqpoint{4.650000in}{0.614151in}}%
\pgfusepath{clip}%
\pgfsetbuttcap%
\pgfsetroundjoin%
\definecolor{currentfill}{rgb}{1.000000,1.000000,0.929412}%
\pgfsetfillcolor{currentfill}%
\pgfsetlinewidth{0.250937pt}%
\definecolor{currentstroke}{rgb}{1.000000,1.000000,1.000000}%
\pgfsetstrokecolor{currentstroke}%
\pgfsetdash{}{0pt}%
\pgfpathmoveto{\pgfqpoint{3.802641in}{10.398868in}}%
\pgfpathlineto{\pgfqpoint{3.890377in}{10.398868in}}%
\pgfpathlineto{\pgfqpoint{3.890377in}{10.311132in}}%
\pgfpathlineto{\pgfqpoint{3.802641in}{10.311132in}}%
\pgfpathlineto{\pgfqpoint{3.802641in}{10.398868in}}%
\pgfusepath{stroke,fill}%
\end{pgfscope}%
\begin{pgfscope}%
\pgfpathrectangle{\pgfqpoint{0.380943in}{9.960189in}}{\pgfqpoint{4.650000in}{0.614151in}}%
\pgfusepath{clip}%
\pgfsetbuttcap%
\pgfsetroundjoin%
\definecolor{currentfill}{rgb}{1.000000,1.000000,0.929412}%
\pgfsetfillcolor{currentfill}%
\pgfsetlinewidth{0.250937pt}%
\definecolor{currentstroke}{rgb}{1.000000,1.000000,1.000000}%
\pgfsetstrokecolor{currentstroke}%
\pgfsetdash{}{0pt}%
\pgfpathmoveto{\pgfqpoint{3.890377in}{10.398868in}}%
\pgfpathlineto{\pgfqpoint{3.978113in}{10.398868in}}%
\pgfpathlineto{\pgfqpoint{3.978113in}{10.311132in}}%
\pgfpathlineto{\pgfqpoint{3.890377in}{10.311132in}}%
\pgfpathlineto{\pgfqpoint{3.890377in}{10.398868in}}%
\pgfusepath{stroke,fill}%
\end{pgfscope}%
\begin{pgfscope}%
\pgfpathrectangle{\pgfqpoint{0.380943in}{9.960189in}}{\pgfqpoint{4.650000in}{0.614151in}}%
\pgfusepath{clip}%
\pgfsetbuttcap%
\pgfsetroundjoin%
\definecolor{currentfill}{rgb}{1.000000,1.000000,0.929412}%
\pgfsetfillcolor{currentfill}%
\pgfsetlinewidth{0.250937pt}%
\definecolor{currentstroke}{rgb}{1.000000,1.000000,1.000000}%
\pgfsetstrokecolor{currentstroke}%
\pgfsetdash{}{0pt}%
\pgfpathmoveto{\pgfqpoint{3.978113in}{10.398868in}}%
\pgfpathlineto{\pgfqpoint{4.065849in}{10.398868in}}%
\pgfpathlineto{\pgfqpoint{4.065849in}{10.311132in}}%
\pgfpathlineto{\pgfqpoint{3.978113in}{10.311132in}}%
\pgfpathlineto{\pgfqpoint{3.978113in}{10.398868in}}%
\pgfusepath{stroke,fill}%
\end{pgfscope}%
\begin{pgfscope}%
\pgfpathrectangle{\pgfqpoint{0.380943in}{9.960189in}}{\pgfqpoint{4.650000in}{0.614151in}}%
\pgfusepath{clip}%
\pgfsetbuttcap%
\pgfsetroundjoin%
\definecolor{currentfill}{rgb}{1.000000,1.000000,0.929412}%
\pgfsetfillcolor{currentfill}%
\pgfsetlinewidth{0.250937pt}%
\definecolor{currentstroke}{rgb}{1.000000,1.000000,1.000000}%
\pgfsetstrokecolor{currentstroke}%
\pgfsetdash{}{0pt}%
\pgfpathmoveto{\pgfqpoint{4.065849in}{10.398868in}}%
\pgfpathlineto{\pgfqpoint{4.153585in}{10.398868in}}%
\pgfpathlineto{\pgfqpoint{4.153585in}{10.311132in}}%
\pgfpathlineto{\pgfqpoint{4.065849in}{10.311132in}}%
\pgfpathlineto{\pgfqpoint{4.065849in}{10.398868in}}%
\pgfusepath{stroke,fill}%
\end{pgfscope}%
\begin{pgfscope}%
\pgfpathrectangle{\pgfqpoint{0.380943in}{9.960189in}}{\pgfqpoint{4.650000in}{0.614151in}}%
\pgfusepath{clip}%
\pgfsetbuttcap%
\pgfsetroundjoin%
\definecolor{currentfill}{rgb}{1.000000,1.000000,0.929412}%
\pgfsetfillcolor{currentfill}%
\pgfsetlinewidth{0.250937pt}%
\definecolor{currentstroke}{rgb}{1.000000,1.000000,1.000000}%
\pgfsetstrokecolor{currentstroke}%
\pgfsetdash{}{0pt}%
\pgfpathmoveto{\pgfqpoint{4.153585in}{10.398868in}}%
\pgfpathlineto{\pgfqpoint{4.241320in}{10.398868in}}%
\pgfpathlineto{\pgfqpoint{4.241320in}{10.311132in}}%
\pgfpathlineto{\pgfqpoint{4.153585in}{10.311132in}}%
\pgfpathlineto{\pgfqpoint{4.153585in}{10.398868in}}%
\pgfusepath{stroke,fill}%
\end{pgfscope}%
\begin{pgfscope}%
\pgfpathrectangle{\pgfqpoint{0.380943in}{9.960189in}}{\pgfqpoint{4.650000in}{0.614151in}}%
\pgfusepath{clip}%
\pgfsetbuttcap%
\pgfsetroundjoin%
\definecolor{currentfill}{rgb}{1.000000,1.000000,0.929412}%
\pgfsetfillcolor{currentfill}%
\pgfsetlinewidth{0.250937pt}%
\definecolor{currentstroke}{rgb}{1.000000,1.000000,1.000000}%
\pgfsetstrokecolor{currentstroke}%
\pgfsetdash{}{0pt}%
\pgfpathmoveto{\pgfqpoint{4.241320in}{10.398868in}}%
\pgfpathlineto{\pgfqpoint{4.329056in}{10.398868in}}%
\pgfpathlineto{\pgfqpoint{4.329056in}{10.311132in}}%
\pgfpathlineto{\pgfqpoint{4.241320in}{10.311132in}}%
\pgfpathlineto{\pgfqpoint{4.241320in}{10.398868in}}%
\pgfusepath{stroke,fill}%
\end{pgfscope}%
\begin{pgfscope}%
\pgfpathrectangle{\pgfqpoint{0.380943in}{9.960189in}}{\pgfqpoint{4.650000in}{0.614151in}}%
\pgfusepath{clip}%
\pgfsetbuttcap%
\pgfsetroundjoin%
\definecolor{currentfill}{rgb}{1.000000,1.000000,0.929412}%
\pgfsetfillcolor{currentfill}%
\pgfsetlinewidth{0.250937pt}%
\definecolor{currentstroke}{rgb}{1.000000,1.000000,1.000000}%
\pgfsetstrokecolor{currentstroke}%
\pgfsetdash{}{0pt}%
\pgfpathmoveto{\pgfqpoint{4.329056in}{10.398868in}}%
\pgfpathlineto{\pgfqpoint{4.416792in}{10.398868in}}%
\pgfpathlineto{\pgfqpoint{4.416792in}{10.311132in}}%
\pgfpathlineto{\pgfqpoint{4.329056in}{10.311132in}}%
\pgfpathlineto{\pgfqpoint{4.329056in}{10.398868in}}%
\pgfusepath{stroke,fill}%
\end{pgfscope}%
\begin{pgfscope}%
\pgfpathrectangle{\pgfqpoint{0.380943in}{9.960189in}}{\pgfqpoint{4.650000in}{0.614151in}}%
\pgfusepath{clip}%
\pgfsetbuttcap%
\pgfsetroundjoin%
\definecolor{currentfill}{rgb}{1.000000,1.000000,0.929412}%
\pgfsetfillcolor{currentfill}%
\pgfsetlinewidth{0.250937pt}%
\definecolor{currentstroke}{rgb}{1.000000,1.000000,1.000000}%
\pgfsetstrokecolor{currentstroke}%
\pgfsetdash{}{0pt}%
\pgfpathmoveto{\pgfqpoint{4.416792in}{10.398868in}}%
\pgfpathlineto{\pgfqpoint{4.504528in}{10.398868in}}%
\pgfpathlineto{\pgfqpoint{4.504528in}{10.311132in}}%
\pgfpathlineto{\pgfqpoint{4.416792in}{10.311132in}}%
\pgfpathlineto{\pgfqpoint{4.416792in}{10.398868in}}%
\pgfusepath{stroke,fill}%
\end{pgfscope}%
\begin{pgfscope}%
\pgfpathrectangle{\pgfqpoint{0.380943in}{9.960189in}}{\pgfqpoint{4.650000in}{0.614151in}}%
\pgfusepath{clip}%
\pgfsetbuttcap%
\pgfsetroundjoin%
\definecolor{currentfill}{rgb}{1.000000,0.622145,0.537486}%
\pgfsetfillcolor{currentfill}%
\pgfsetlinewidth{0.250937pt}%
\definecolor{currentstroke}{rgb}{1.000000,1.000000,1.000000}%
\pgfsetstrokecolor{currentstroke}%
\pgfsetdash{}{0pt}%
\pgfpathmoveto{\pgfqpoint{4.504528in}{10.398868in}}%
\pgfpathlineto{\pgfqpoint{4.592264in}{10.398868in}}%
\pgfpathlineto{\pgfqpoint{4.592264in}{10.311132in}}%
\pgfpathlineto{\pgfqpoint{4.504528in}{10.311132in}}%
\pgfpathlineto{\pgfqpoint{4.504528in}{10.398868in}}%
\pgfusepath{stroke,fill}%
\end{pgfscope}%
\begin{pgfscope}%
\pgfpathrectangle{\pgfqpoint{0.380943in}{9.960189in}}{\pgfqpoint{4.650000in}{0.614151in}}%
\pgfusepath{clip}%
\pgfsetbuttcap%
\pgfsetroundjoin%
\definecolor{currentfill}{rgb}{0.989619,0.788235,0.628374}%
\pgfsetfillcolor{currentfill}%
\pgfsetlinewidth{0.250937pt}%
\definecolor{currentstroke}{rgb}{1.000000,1.000000,1.000000}%
\pgfsetstrokecolor{currentstroke}%
\pgfsetdash{}{0pt}%
\pgfpathmoveto{\pgfqpoint{4.592264in}{10.398868in}}%
\pgfpathlineto{\pgfqpoint{4.680000in}{10.398868in}}%
\pgfpathlineto{\pgfqpoint{4.680000in}{10.311132in}}%
\pgfpathlineto{\pgfqpoint{4.592264in}{10.311132in}}%
\pgfpathlineto{\pgfqpoint{4.592264in}{10.398868in}}%
\pgfusepath{stroke,fill}%
\end{pgfscope}%
\begin{pgfscope}%
\pgfpathrectangle{\pgfqpoint{0.380943in}{9.960189in}}{\pgfqpoint{4.650000in}{0.614151in}}%
\pgfusepath{clip}%
\pgfsetbuttcap%
\pgfsetroundjoin%
\definecolor{currentfill}{rgb}{0.982699,0.823991,0.657439}%
\pgfsetfillcolor{currentfill}%
\pgfsetlinewidth{0.250937pt}%
\definecolor{currentstroke}{rgb}{1.000000,1.000000,1.000000}%
\pgfsetstrokecolor{currentstroke}%
\pgfsetdash{}{0pt}%
\pgfpathmoveto{\pgfqpoint{4.680000in}{10.398868in}}%
\pgfpathlineto{\pgfqpoint{4.767736in}{10.398868in}}%
\pgfpathlineto{\pgfqpoint{4.767736in}{10.311132in}}%
\pgfpathlineto{\pgfqpoint{4.680000in}{10.311132in}}%
\pgfpathlineto{\pgfqpoint{4.680000in}{10.398868in}}%
\pgfusepath{stroke,fill}%
\end{pgfscope}%
\begin{pgfscope}%
\pgfpathrectangle{\pgfqpoint{0.380943in}{9.960189in}}{\pgfqpoint{4.650000in}{0.614151in}}%
\pgfusepath{clip}%
\pgfsetbuttcap%
\pgfsetroundjoin%
\definecolor{currentfill}{rgb}{0.997924,0.685352,0.570242}%
\pgfsetfillcolor{currentfill}%
\pgfsetlinewidth{0.250937pt}%
\definecolor{currentstroke}{rgb}{1.000000,1.000000,1.000000}%
\pgfsetstrokecolor{currentstroke}%
\pgfsetdash{}{0pt}%
\pgfpathmoveto{\pgfqpoint{4.767736in}{10.398868in}}%
\pgfpathlineto{\pgfqpoint{4.855471in}{10.398868in}}%
\pgfpathlineto{\pgfqpoint{4.855471in}{10.311132in}}%
\pgfpathlineto{\pgfqpoint{4.767736in}{10.311132in}}%
\pgfpathlineto{\pgfqpoint{4.767736in}{10.398868in}}%
\pgfusepath{stroke,fill}%
\end{pgfscope}%
\begin{pgfscope}%
\pgfpathrectangle{\pgfqpoint{0.380943in}{9.960189in}}{\pgfqpoint{4.650000in}{0.614151in}}%
\pgfusepath{clip}%
\pgfsetbuttcap%
\pgfsetroundjoin%
\definecolor{currentfill}{rgb}{0.989619,0.788235,0.628374}%
\pgfsetfillcolor{currentfill}%
\pgfsetlinewidth{0.250937pt}%
\definecolor{currentstroke}{rgb}{1.000000,1.000000,1.000000}%
\pgfsetstrokecolor{currentstroke}%
\pgfsetdash{}{0pt}%
\pgfpathmoveto{\pgfqpoint{4.855471in}{10.398868in}}%
\pgfpathlineto{\pgfqpoint{4.943207in}{10.398868in}}%
\pgfpathlineto{\pgfqpoint{4.943207in}{10.311132in}}%
\pgfpathlineto{\pgfqpoint{4.855471in}{10.311132in}}%
\pgfpathlineto{\pgfqpoint{4.855471in}{10.398868in}}%
\pgfusepath{stroke,fill}%
\end{pgfscope}%
\begin{pgfscope}%
\pgfpathrectangle{\pgfqpoint{0.380943in}{9.960189in}}{\pgfqpoint{4.650000in}{0.614151in}}%
\pgfusepath{clip}%
\pgfsetbuttcap%
\pgfsetroundjoin%
\definecolor{currentfill}{rgb}{0.963091,0.937255,0.735409}%
\pgfsetfillcolor{currentfill}%
\pgfsetlinewidth{0.250937pt}%
\definecolor{currentstroke}{rgb}{1.000000,1.000000,1.000000}%
\pgfsetstrokecolor{currentstroke}%
\pgfsetdash{}{0pt}%
\pgfpathmoveto{\pgfqpoint{4.943207in}{10.398868in}}%
\pgfpathlineto{\pgfqpoint{5.030943in}{10.398868in}}%
\pgfpathlineto{\pgfqpoint{5.030943in}{10.311132in}}%
\pgfpathlineto{\pgfqpoint{4.943207in}{10.311132in}}%
\pgfpathlineto{\pgfqpoint{4.943207in}{10.398868in}}%
\pgfusepath{stroke,fill}%
\end{pgfscope}%
\begin{pgfscope}%
\pgfpathrectangle{\pgfqpoint{0.380943in}{9.960189in}}{\pgfqpoint{4.650000in}{0.614151in}}%
\pgfusepath{clip}%
\pgfsetbuttcap%
\pgfsetroundjoin%
\pgfsetlinewidth{0.250937pt}%
\definecolor{currentstroke}{rgb}{1.000000,1.000000,1.000000}%
\pgfsetstrokecolor{currentstroke}%
\pgfsetdash{}{0pt}%
\pgfpathmoveto{\pgfqpoint{0.380943in}{10.311132in}}%
\pgfpathlineto{\pgfqpoint{0.468679in}{10.311132in}}%
\pgfpathlineto{\pgfqpoint{0.468679in}{10.223396in}}%
\pgfpathlineto{\pgfqpoint{0.380943in}{10.223396in}}%
\pgfpathlineto{\pgfqpoint{0.380943in}{10.311132in}}%
\pgfusepath{stroke}%
\end{pgfscope}%
\begin{pgfscope}%
\pgfpathrectangle{\pgfqpoint{0.380943in}{9.960189in}}{\pgfqpoint{4.650000in}{0.614151in}}%
\pgfusepath{clip}%
\pgfsetbuttcap%
\pgfsetroundjoin%
\definecolor{currentfill}{rgb}{1.000000,1.000000,0.929412}%
\pgfsetfillcolor{currentfill}%
\pgfsetlinewidth{0.250937pt}%
\definecolor{currentstroke}{rgb}{1.000000,1.000000,1.000000}%
\pgfsetstrokecolor{currentstroke}%
\pgfsetdash{}{0pt}%
\pgfpathmoveto{\pgfqpoint{0.468679in}{10.311132in}}%
\pgfpathlineto{\pgfqpoint{0.556415in}{10.311132in}}%
\pgfpathlineto{\pgfqpoint{0.556415in}{10.223396in}}%
\pgfpathlineto{\pgfqpoint{0.468679in}{10.223396in}}%
\pgfpathlineto{\pgfqpoint{0.468679in}{10.311132in}}%
\pgfusepath{stroke,fill}%
\end{pgfscope}%
\begin{pgfscope}%
\pgfpathrectangle{\pgfqpoint{0.380943in}{9.960189in}}{\pgfqpoint{4.650000in}{0.614151in}}%
\pgfusepath{clip}%
\pgfsetbuttcap%
\pgfsetroundjoin%
\definecolor{currentfill}{rgb}{1.000000,1.000000,0.929412}%
\pgfsetfillcolor{currentfill}%
\pgfsetlinewidth{0.250937pt}%
\definecolor{currentstroke}{rgb}{1.000000,1.000000,1.000000}%
\pgfsetstrokecolor{currentstroke}%
\pgfsetdash{}{0pt}%
\pgfpathmoveto{\pgfqpoint{0.556415in}{10.311132in}}%
\pgfpathlineto{\pgfqpoint{0.644151in}{10.311132in}}%
\pgfpathlineto{\pgfqpoint{0.644151in}{10.223396in}}%
\pgfpathlineto{\pgfqpoint{0.556415in}{10.223396in}}%
\pgfpathlineto{\pgfqpoint{0.556415in}{10.311132in}}%
\pgfusepath{stroke,fill}%
\end{pgfscope}%
\begin{pgfscope}%
\pgfpathrectangle{\pgfqpoint{0.380943in}{9.960189in}}{\pgfqpoint{4.650000in}{0.614151in}}%
\pgfusepath{clip}%
\pgfsetbuttcap%
\pgfsetroundjoin%
\definecolor{currentfill}{rgb}{1.000000,1.000000,0.929412}%
\pgfsetfillcolor{currentfill}%
\pgfsetlinewidth{0.250937pt}%
\definecolor{currentstroke}{rgb}{1.000000,1.000000,1.000000}%
\pgfsetstrokecolor{currentstroke}%
\pgfsetdash{}{0pt}%
\pgfpathmoveto{\pgfqpoint{0.644151in}{10.311132in}}%
\pgfpathlineto{\pgfqpoint{0.731886in}{10.311132in}}%
\pgfpathlineto{\pgfqpoint{0.731886in}{10.223396in}}%
\pgfpathlineto{\pgfqpoint{0.644151in}{10.223396in}}%
\pgfpathlineto{\pgfqpoint{0.644151in}{10.311132in}}%
\pgfusepath{stroke,fill}%
\end{pgfscope}%
\begin{pgfscope}%
\pgfpathrectangle{\pgfqpoint{0.380943in}{9.960189in}}{\pgfqpoint{4.650000in}{0.614151in}}%
\pgfusepath{clip}%
\pgfsetbuttcap%
\pgfsetroundjoin%
\definecolor{currentfill}{rgb}{1.000000,1.000000,0.929412}%
\pgfsetfillcolor{currentfill}%
\pgfsetlinewidth{0.250937pt}%
\definecolor{currentstroke}{rgb}{1.000000,1.000000,1.000000}%
\pgfsetstrokecolor{currentstroke}%
\pgfsetdash{}{0pt}%
\pgfpathmoveto{\pgfqpoint{0.731886in}{10.311132in}}%
\pgfpathlineto{\pgfqpoint{0.819622in}{10.311132in}}%
\pgfpathlineto{\pgfqpoint{0.819622in}{10.223396in}}%
\pgfpathlineto{\pgfqpoint{0.731886in}{10.223396in}}%
\pgfpathlineto{\pgfqpoint{0.731886in}{10.311132in}}%
\pgfusepath{stroke,fill}%
\end{pgfscope}%
\begin{pgfscope}%
\pgfpathrectangle{\pgfqpoint{0.380943in}{9.960189in}}{\pgfqpoint{4.650000in}{0.614151in}}%
\pgfusepath{clip}%
\pgfsetbuttcap%
\pgfsetroundjoin%
\definecolor{currentfill}{rgb}{1.000000,1.000000,0.929412}%
\pgfsetfillcolor{currentfill}%
\pgfsetlinewidth{0.250937pt}%
\definecolor{currentstroke}{rgb}{1.000000,1.000000,1.000000}%
\pgfsetstrokecolor{currentstroke}%
\pgfsetdash{}{0pt}%
\pgfpathmoveto{\pgfqpoint{0.819622in}{10.311132in}}%
\pgfpathlineto{\pgfqpoint{0.907358in}{10.311132in}}%
\pgfpathlineto{\pgfqpoint{0.907358in}{10.223396in}}%
\pgfpathlineto{\pgfqpoint{0.819622in}{10.223396in}}%
\pgfpathlineto{\pgfqpoint{0.819622in}{10.311132in}}%
\pgfusepath{stroke,fill}%
\end{pgfscope}%
\begin{pgfscope}%
\pgfpathrectangle{\pgfqpoint{0.380943in}{9.960189in}}{\pgfqpoint{4.650000in}{0.614151in}}%
\pgfusepath{clip}%
\pgfsetbuttcap%
\pgfsetroundjoin%
\definecolor{currentfill}{rgb}{1.000000,1.000000,0.929412}%
\pgfsetfillcolor{currentfill}%
\pgfsetlinewidth{0.250937pt}%
\definecolor{currentstroke}{rgb}{1.000000,1.000000,1.000000}%
\pgfsetstrokecolor{currentstroke}%
\pgfsetdash{}{0pt}%
\pgfpathmoveto{\pgfqpoint{0.907358in}{10.311132in}}%
\pgfpathlineto{\pgfqpoint{0.995094in}{10.311132in}}%
\pgfpathlineto{\pgfqpoint{0.995094in}{10.223396in}}%
\pgfpathlineto{\pgfqpoint{0.907358in}{10.223396in}}%
\pgfpathlineto{\pgfqpoint{0.907358in}{10.311132in}}%
\pgfusepath{stroke,fill}%
\end{pgfscope}%
\begin{pgfscope}%
\pgfpathrectangle{\pgfqpoint{0.380943in}{9.960189in}}{\pgfqpoint{4.650000in}{0.614151in}}%
\pgfusepath{clip}%
\pgfsetbuttcap%
\pgfsetroundjoin%
\definecolor{currentfill}{rgb}{1.000000,1.000000,0.929412}%
\pgfsetfillcolor{currentfill}%
\pgfsetlinewidth{0.250937pt}%
\definecolor{currentstroke}{rgb}{1.000000,1.000000,1.000000}%
\pgfsetstrokecolor{currentstroke}%
\pgfsetdash{}{0pt}%
\pgfpathmoveto{\pgfqpoint{0.995094in}{10.311132in}}%
\pgfpathlineto{\pgfqpoint{1.082830in}{10.311132in}}%
\pgfpathlineto{\pgfqpoint{1.082830in}{10.223396in}}%
\pgfpathlineto{\pgfqpoint{0.995094in}{10.223396in}}%
\pgfpathlineto{\pgfqpoint{0.995094in}{10.311132in}}%
\pgfusepath{stroke,fill}%
\end{pgfscope}%
\begin{pgfscope}%
\pgfpathrectangle{\pgfqpoint{0.380943in}{9.960189in}}{\pgfqpoint{4.650000in}{0.614151in}}%
\pgfusepath{clip}%
\pgfsetbuttcap%
\pgfsetroundjoin%
\definecolor{currentfill}{rgb}{1.000000,1.000000,0.929412}%
\pgfsetfillcolor{currentfill}%
\pgfsetlinewidth{0.250937pt}%
\definecolor{currentstroke}{rgb}{1.000000,1.000000,1.000000}%
\pgfsetstrokecolor{currentstroke}%
\pgfsetdash{}{0pt}%
\pgfpathmoveto{\pgfqpoint{1.082830in}{10.311132in}}%
\pgfpathlineto{\pgfqpoint{1.170566in}{10.311132in}}%
\pgfpathlineto{\pgfqpoint{1.170566in}{10.223396in}}%
\pgfpathlineto{\pgfqpoint{1.082830in}{10.223396in}}%
\pgfpathlineto{\pgfqpoint{1.082830in}{10.311132in}}%
\pgfusepath{stroke,fill}%
\end{pgfscope}%
\begin{pgfscope}%
\pgfpathrectangle{\pgfqpoint{0.380943in}{9.960189in}}{\pgfqpoint{4.650000in}{0.614151in}}%
\pgfusepath{clip}%
\pgfsetbuttcap%
\pgfsetroundjoin%
\definecolor{currentfill}{rgb}{1.000000,1.000000,0.929412}%
\pgfsetfillcolor{currentfill}%
\pgfsetlinewidth{0.250937pt}%
\definecolor{currentstroke}{rgb}{1.000000,1.000000,1.000000}%
\pgfsetstrokecolor{currentstroke}%
\pgfsetdash{}{0pt}%
\pgfpathmoveto{\pgfqpoint{1.170566in}{10.311132in}}%
\pgfpathlineto{\pgfqpoint{1.258302in}{10.311132in}}%
\pgfpathlineto{\pgfqpoint{1.258302in}{10.223396in}}%
\pgfpathlineto{\pgfqpoint{1.170566in}{10.223396in}}%
\pgfpathlineto{\pgfqpoint{1.170566in}{10.311132in}}%
\pgfusepath{stroke,fill}%
\end{pgfscope}%
\begin{pgfscope}%
\pgfpathrectangle{\pgfqpoint{0.380943in}{9.960189in}}{\pgfqpoint{4.650000in}{0.614151in}}%
\pgfusepath{clip}%
\pgfsetbuttcap%
\pgfsetroundjoin%
\definecolor{currentfill}{rgb}{1.000000,1.000000,0.929412}%
\pgfsetfillcolor{currentfill}%
\pgfsetlinewidth{0.250937pt}%
\definecolor{currentstroke}{rgb}{1.000000,1.000000,1.000000}%
\pgfsetstrokecolor{currentstroke}%
\pgfsetdash{}{0pt}%
\pgfpathmoveto{\pgfqpoint{1.258302in}{10.311132in}}%
\pgfpathlineto{\pgfqpoint{1.346037in}{10.311132in}}%
\pgfpathlineto{\pgfqpoint{1.346037in}{10.223396in}}%
\pgfpathlineto{\pgfqpoint{1.258302in}{10.223396in}}%
\pgfpathlineto{\pgfqpoint{1.258302in}{10.311132in}}%
\pgfusepath{stroke,fill}%
\end{pgfscope}%
\begin{pgfscope}%
\pgfpathrectangle{\pgfqpoint{0.380943in}{9.960189in}}{\pgfqpoint{4.650000in}{0.614151in}}%
\pgfusepath{clip}%
\pgfsetbuttcap%
\pgfsetroundjoin%
\definecolor{currentfill}{rgb}{1.000000,1.000000,0.929412}%
\pgfsetfillcolor{currentfill}%
\pgfsetlinewidth{0.250937pt}%
\definecolor{currentstroke}{rgb}{1.000000,1.000000,1.000000}%
\pgfsetstrokecolor{currentstroke}%
\pgfsetdash{}{0pt}%
\pgfpathmoveto{\pgfqpoint{1.346037in}{10.311132in}}%
\pgfpathlineto{\pgfqpoint{1.433773in}{10.311132in}}%
\pgfpathlineto{\pgfqpoint{1.433773in}{10.223396in}}%
\pgfpathlineto{\pgfqpoint{1.346037in}{10.223396in}}%
\pgfpathlineto{\pgfqpoint{1.346037in}{10.311132in}}%
\pgfusepath{stroke,fill}%
\end{pgfscope}%
\begin{pgfscope}%
\pgfpathrectangle{\pgfqpoint{0.380943in}{9.960189in}}{\pgfqpoint{4.650000in}{0.614151in}}%
\pgfusepath{clip}%
\pgfsetbuttcap%
\pgfsetroundjoin%
\definecolor{currentfill}{rgb}{1.000000,1.000000,0.929412}%
\pgfsetfillcolor{currentfill}%
\pgfsetlinewidth{0.250937pt}%
\definecolor{currentstroke}{rgb}{1.000000,1.000000,1.000000}%
\pgfsetstrokecolor{currentstroke}%
\pgfsetdash{}{0pt}%
\pgfpathmoveto{\pgfqpoint{1.433773in}{10.311132in}}%
\pgfpathlineto{\pgfqpoint{1.521509in}{10.311132in}}%
\pgfpathlineto{\pgfqpoint{1.521509in}{10.223396in}}%
\pgfpathlineto{\pgfqpoint{1.433773in}{10.223396in}}%
\pgfpathlineto{\pgfqpoint{1.433773in}{10.311132in}}%
\pgfusepath{stroke,fill}%
\end{pgfscope}%
\begin{pgfscope}%
\pgfpathrectangle{\pgfqpoint{0.380943in}{9.960189in}}{\pgfqpoint{4.650000in}{0.614151in}}%
\pgfusepath{clip}%
\pgfsetbuttcap%
\pgfsetroundjoin%
\definecolor{currentfill}{rgb}{1.000000,1.000000,0.929412}%
\pgfsetfillcolor{currentfill}%
\pgfsetlinewidth{0.250937pt}%
\definecolor{currentstroke}{rgb}{1.000000,1.000000,1.000000}%
\pgfsetstrokecolor{currentstroke}%
\pgfsetdash{}{0pt}%
\pgfpathmoveto{\pgfqpoint{1.521509in}{10.311132in}}%
\pgfpathlineto{\pgfqpoint{1.609245in}{10.311132in}}%
\pgfpathlineto{\pgfqpoint{1.609245in}{10.223396in}}%
\pgfpathlineto{\pgfqpoint{1.521509in}{10.223396in}}%
\pgfpathlineto{\pgfqpoint{1.521509in}{10.311132in}}%
\pgfusepath{stroke,fill}%
\end{pgfscope}%
\begin{pgfscope}%
\pgfpathrectangle{\pgfqpoint{0.380943in}{9.960189in}}{\pgfqpoint{4.650000in}{0.614151in}}%
\pgfusepath{clip}%
\pgfsetbuttcap%
\pgfsetroundjoin%
\definecolor{currentfill}{rgb}{1.000000,1.000000,0.929412}%
\pgfsetfillcolor{currentfill}%
\pgfsetlinewidth{0.250937pt}%
\definecolor{currentstroke}{rgb}{1.000000,1.000000,1.000000}%
\pgfsetstrokecolor{currentstroke}%
\pgfsetdash{}{0pt}%
\pgfpathmoveto{\pgfqpoint{1.609245in}{10.311132in}}%
\pgfpathlineto{\pgfqpoint{1.696981in}{10.311132in}}%
\pgfpathlineto{\pgfqpoint{1.696981in}{10.223396in}}%
\pgfpathlineto{\pgfqpoint{1.609245in}{10.223396in}}%
\pgfpathlineto{\pgfqpoint{1.609245in}{10.311132in}}%
\pgfusepath{stroke,fill}%
\end{pgfscope}%
\begin{pgfscope}%
\pgfpathrectangle{\pgfqpoint{0.380943in}{9.960189in}}{\pgfqpoint{4.650000in}{0.614151in}}%
\pgfusepath{clip}%
\pgfsetbuttcap%
\pgfsetroundjoin%
\definecolor{currentfill}{rgb}{1.000000,1.000000,0.929412}%
\pgfsetfillcolor{currentfill}%
\pgfsetlinewidth{0.250937pt}%
\definecolor{currentstroke}{rgb}{1.000000,1.000000,1.000000}%
\pgfsetstrokecolor{currentstroke}%
\pgfsetdash{}{0pt}%
\pgfpathmoveto{\pgfqpoint{1.696981in}{10.311132in}}%
\pgfpathlineto{\pgfqpoint{1.784717in}{10.311132in}}%
\pgfpathlineto{\pgfqpoint{1.784717in}{10.223396in}}%
\pgfpathlineto{\pgfqpoint{1.696981in}{10.223396in}}%
\pgfpathlineto{\pgfqpoint{1.696981in}{10.311132in}}%
\pgfusepath{stroke,fill}%
\end{pgfscope}%
\begin{pgfscope}%
\pgfpathrectangle{\pgfqpoint{0.380943in}{9.960189in}}{\pgfqpoint{4.650000in}{0.614151in}}%
\pgfusepath{clip}%
\pgfsetbuttcap%
\pgfsetroundjoin%
\definecolor{currentfill}{rgb}{1.000000,1.000000,0.929412}%
\pgfsetfillcolor{currentfill}%
\pgfsetlinewidth{0.250937pt}%
\definecolor{currentstroke}{rgb}{1.000000,1.000000,1.000000}%
\pgfsetstrokecolor{currentstroke}%
\pgfsetdash{}{0pt}%
\pgfpathmoveto{\pgfqpoint{1.784717in}{10.311132in}}%
\pgfpathlineto{\pgfqpoint{1.872452in}{10.311132in}}%
\pgfpathlineto{\pgfqpoint{1.872452in}{10.223396in}}%
\pgfpathlineto{\pgfqpoint{1.784717in}{10.223396in}}%
\pgfpathlineto{\pgfqpoint{1.784717in}{10.311132in}}%
\pgfusepath{stroke,fill}%
\end{pgfscope}%
\begin{pgfscope}%
\pgfpathrectangle{\pgfqpoint{0.380943in}{9.960189in}}{\pgfqpoint{4.650000in}{0.614151in}}%
\pgfusepath{clip}%
\pgfsetbuttcap%
\pgfsetroundjoin%
\definecolor{currentfill}{rgb}{1.000000,1.000000,0.929412}%
\pgfsetfillcolor{currentfill}%
\pgfsetlinewidth{0.250937pt}%
\definecolor{currentstroke}{rgb}{1.000000,1.000000,1.000000}%
\pgfsetstrokecolor{currentstroke}%
\pgfsetdash{}{0pt}%
\pgfpathmoveto{\pgfqpoint{1.872452in}{10.311132in}}%
\pgfpathlineto{\pgfqpoint{1.960188in}{10.311132in}}%
\pgfpathlineto{\pgfqpoint{1.960188in}{10.223396in}}%
\pgfpathlineto{\pgfqpoint{1.872452in}{10.223396in}}%
\pgfpathlineto{\pgfqpoint{1.872452in}{10.311132in}}%
\pgfusepath{stroke,fill}%
\end{pgfscope}%
\begin{pgfscope}%
\pgfpathrectangle{\pgfqpoint{0.380943in}{9.960189in}}{\pgfqpoint{4.650000in}{0.614151in}}%
\pgfusepath{clip}%
\pgfsetbuttcap%
\pgfsetroundjoin%
\definecolor{currentfill}{rgb}{1.000000,1.000000,0.929412}%
\pgfsetfillcolor{currentfill}%
\pgfsetlinewidth{0.250937pt}%
\definecolor{currentstroke}{rgb}{1.000000,1.000000,1.000000}%
\pgfsetstrokecolor{currentstroke}%
\pgfsetdash{}{0pt}%
\pgfpathmoveto{\pgfqpoint{1.960188in}{10.311132in}}%
\pgfpathlineto{\pgfqpoint{2.047924in}{10.311132in}}%
\pgfpathlineto{\pgfqpoint{2.047924in}{10.223396in}}%
\pgfpathlineto{\pgfqpoint{1.960188in}{10.223396in}}%
\pgfpathlineto{\pgfqpoint{1.960188in}{10.311132in}}%
\pgfusepath{stroke,fill}%
\end{pgfscope}%
\begin{pgfscope}%
\pgfpathrectangle{\pgfqpoint{0.380943in}{9.960189in}}{\pgfqpoint{4.650000in}{0.614151in}}%
\pgfusepath{clip}%
\pgfsetbuttcap%
\pgfsetroundjoin%
\definecolor{currentfill}{rgb}{1.000000,1.000000,0.929412}%
\pgfsetfillcolor{currentfill}%
\pgfsetlinewidth{0.250937pt}%
\definecolor{currentstroke}{rgb}{1.000000,1.000000,1.000000}%
\pgfsetstrokecolor{currentstroke}%
\pgfsetdash{}{0pt}%
\pgfpathmoveto{\pgfqpoint{2.047924in}{10.311132in}}%
\pgfpathlineto{\pgfqpoint{2.135660in}{10.311132in}}%
\pgfpathlineto{\pgfqpoint{2.135660in}{10.223396in}}%
\pgfpathlineto{\pgfqpoint{2.047924in}{10.223396in}}%
\pgfpathlineto{\pgfqpoint{2.047924in}{10.311132in}}%
\pgfusepath{stroke,fill}%
\end{pgfscope}%
\begin{pgfscope}%
\pgfpathrectangle{\pgfqpoint{0.380943in}{9.960189in}}{\pgfqpoint{4.650000in}{0.614151in}}%
\pgfusepath{clip}%
\pgfsetbuttcap%
\pgfsetroundjoin%
\definecolor{currentfill}{rgb}{1.000000,1.000000,0.929412}%
\pgfsetfillcolor{currentfill}%
\pgfsetlinewidth{0.250937pt}%
\definecolor{currentstroke}{rgb}{1.000000,1.000000,1.000000}%
\pgfsetstrokecolor{currentstroke}%
\pgfsetdash{}{0pt}%
\pgfpathmoveto{\pgfqpoint{2.135660in}{10.311132in}}%
\pgfpathlineto{\pgfqpoint{2.223396in}{10.311132in}}%
\pgfpathlineto{\pgfqpoint{2.223396in}{10.223396in}}%
\pgfpathlineto{\pgfqpoint{2.135660in}{10.223396in}}%
\pgfpathlineto{\pgfqpoint{2.135660in}{10.311132in}}%
\pgfusepath{stroke,fill}%
\end{pgfscope}%
\begin{pgfscope}%
\pgfpathrectangle{\pgfqpoint{0.380943in}{9.960189in}}{\pgfqpoint{4.650000in}{0.614151in}}%
\pgfusepath{clip}%
\pgfsetbuttcap%
\pgfsetroundjoin%
\definecolor{currentfill}{rgb}{1.000000,1.000000,0.929412}%
\pgfsetfillcolor{currentfill}%
\pgfsetlinewidth{0.250937pt}%
\definecolor{currentstroke}{rgb}{1.000000,1.000000,1.000000}%
\pgfsetstrokecolor{currentstroke}%
\pgfsetdash{}{0pt}%
\pgfpathmoveto{\pgfqpoint{2.223396in}{10.311132in}}%
\pgfpathlineto{\pgfqpoint{2.311132in}{10.311132in}}%
\pgfpathlineto{\pgfqpoint{2.311132in}{10.223396in}}%
\pgfpathlineto{\pgfqpoint{2.223396in}{10.223396in}}%
\pgfpathlineto{\pgfqpoint{2.223396in}{10.311132in}}%
\pgfusepath{stroke,fill}%
\end{pgfscope}%
\begin{pgfscope}%
\pgfpathrectangle{\pgfqpoint{0.380943in}{9.960189in}}{\pgfqpoint{4.650000in}{0.614151in}}%
\pgfusepath{clip}%
\pgfsetbuttcap%
\pgfsetroundjoin%
\definecolor{currentfill}{rgb}{1.000000,1.000000,0.929412}%
\pgfsetfillcolor{currentfill}%
\pgfsetlinewidth{0.250937pt}%
\definecolor{currentstroke}{rgb}{1.000000,1.000000,1.000000}%
\pgfsetstrokecolor{currentstroke}%
\pgfsetdash{}{0pt}%
\pgfpathmoveto{\pgfqpoint{2.311132in}{10.311132in}}%
\pgfpathlineto{\pgfqpoint{2.398868in}{10.311132in}}%
\pgfpathlineto{\pgfqpoint{2.398868in}{10.223396in}}%
\pgfpathlineto{\pgfqpoint{2.311132in}{10.223396in}}%
\pgfpathlineto{\pgfqpoint{2.311132in}{10.311132in}}%
\pgfusepath{stroke,fill}%
\end{pgfscope}%
\begin{pgfscope}%
\pgfpathrectangle{\pgfqpoint{0.380943in}{9.960189in}}{\pgfqpoint{4.650000in}{0.614151in}}%
\pgfusepath{clip}%
\pgfsetbuttcap%
\pgfsetroundjoin%
\definecolor{currentfill}{rgb}{1.000000,1.000000,0.929412}%
\pgfsetfillcolor{currentfill}%
\pgfsetlinewidth{0.250937pt}%
\definecolor{currentstroke}{rgb}{1.000000,1.000000,1.000000}%
\pgfsetstrokecolor{currentstroke}%
\pgfsetdash{}{0pt}%
\pgfpathmoveto{\pgfqpoint{2.398868in}{10.311132in}}%
\pgfpathlineto{\pgfqpoint{2.486603in}{10.311132in}}%
\pgfpathlineto{\pgfqpoint{2.486603in}{10.223396in}}%
\pgfpathlineto{\pgfqpoint{2.398868in}{10.223396in}}%
\pgfpathlineto{\pgfqpoint{2.398868in}{10.311132in}}%
\pgfusepath{stroke,fill}%
\end{pgfscope}%
\begin{pgfscope}%
\pgfpathrectangle{\pgfqpoint{0.380943in}{9.960189in}}{\pgfqpoint{4.650000in}{0.614151in}}%
\pgfusepath{clip}%
\pgfsetbuttcap%
\pgfsetroundjoin%
\definecolor{currentfill}{rgb}{1.000000,1.000000,0.929412}%
\pgfsetfillcolor{currentfill}%
\pgfsetlinewidth{0.250937pt}%
\definecolor{currentstroke}{rgb}{1.000000,1.000000,1.000000}%
\pgfsetstrokecolor{currentstroke}%
\pgfsetdash{}{0pt}%
\pgfpathmoveto{\pgfqpoint{2.486603in}{10.311132in}}%
\pgfpathlineto{\pgfqpoint{2.574339in}{10.311132in}}%
\pgfpathlineto{\pgfqpoint{2.574339in}{10.223396in}}%
\pgfpathlineto{\pgfqpoint{2.486603in}{10.223396in}}%
\pgfpathlineto{\pgfqpoint{2.486603in}{10.311132in}}%
\pgfusepath{stroke,fill}%
\end{pgfscope}%
\begin{pgfscope}%
\pgfpathrectangle{\pgfqpoint{0.380943in}{9.960189in}}{\pgfqpoint{4.650000in}{0.614151in}}%
\pgfusepath{clip}%
\pgfsetbuttcap%
\pgfsetroundjoin%
\definecolor{currentfill}{rgb}{1.000000,1.000000,0.929412}%
\pgfsetfillcolor{currentfill}%
\pgfsetlinewidth{0.250937pt}%
\definecolor{currentstroke}{rgb}{1.000000,1.000000,1.000000}%
\pgfsetstrokecolor{currentstroke}%
\pgfsetdash{}{0pt}%
\pgfpathmoveto{\pgfqpoint{2.574339in}{10.311132in}}%
\pgfpathlineto{\pgfqpoint{2.662075in}{10.311132in}}%
\pgfpathlineto{\pgfqpoint{2.662075in}{10.223396in}}%
\pgfpathlineto{\pgfqpoint{2.574339in}{10.223396in}}%
\pgfpathlineto{\pgfqpoint{2.574339in}{10.311132in}}%
\pgfusepath{stroke,fill}%
\end{pgfscope}%
\begin{pgfscope}%
\pgfpathrectangle{\pgfqpoint{0.380943in}{9.960189in}}{\pgfqpoint{4.650000in}{0.614151in}}%
\pgfusepath{clip}%
\pgfsetbuttcap%
\pgfsetroundjoin%
\definecolor{currentfill}{rgb}{1.000000,1.000000,0.929412}%
\pgfsetfillcolor{currentfill}%
\pgfsetlinewidth{0.250937pt}%
\definecolor{currentstroke}{rgb}{1.000000,1.000000,1.000000}%
\pgfsetstrokecolor{currentstroke}%
\pgfsetdash{}{0pt}%
\pgfpathmoveto{\pgfqpoint{2.662075in}{10.311132in}}%
\pgfpathlineto{\pgfqpoint{2.749811in}{10.311132in}}%
\pgfpathlineto{\pgfqpoint{2.749811in}{10.223396in}}%
\pgfpathlineto{\pgfqpoint{2.662075in}{10.223396in}}%
\pgfpathlineto{\pgfqpoint{2.662075in}{10.311132in}}%
\pgfusepath{stroke,fill}%
\end{pgfscope}%
\begin{pgfscope}%
\pgfpathrectangle{\pgfqpoint{0.380943in}{9.960189in}}{\pgfqpoint{4.650000in}{0.614151in}}%
\pgfusepath{clip}%
\pgfsetbuttcap%
\pgfsetroundjoin%
\definecolor{currentfill}{rgb}{1.000000,1.000000,0.929412}%
\pgfsetfillcolor{currentfill}%
\pgfsetlinewidth{0.250937pt}%
\definecolor{currentstroke}{rgb}{1.000000,1.000000,1.000000}%
\pgfsetstrokecolor{currentstroke}%
\pgfsetdash{}{0pt}%
\pgfpathmoveto{\pgfqpoint{2.749811in}{10.311132in}}%
\pgfpathlineto{\pgfqpoint{2.837547in}{10.311132in}}%
\pgfpathlineto{\pgfqpoint{2.837547in}{10.223396in}}%
\pgfpathlineto{\pgfqpoint{2.749811in}{10.223396in}}%
\pgfpathlineto{\pgfqpoint{2.749811in}{10.311132in}}%
\pgfusepath{stroke,fill}%
\end{pgfscope}%
\begin{pgfscope}%
\pgfpathrectangle{\pgfqpoint{0.380943in}{9.960189in}}{\pgfqpoint{4.650000in}{0.614151in}}%
\pgfusepath{clip}%
\pgfsetbuttcap%
\pgfsetroundjoin%
\definecolor{currentfill}{rgb}{1.000000,1.000000,0.929412}%
\pgfsetfillcolor{currentfill}%
\pgfsetlinewidth{0.250937pt}%
\definecolor{currentstroke}{rgb}{1.000000,1.000000,1.000000}%
\pgfsetstrokecolor{currentstroke}%
\pgfsetdash{}{0pt}%
\pgfpathmoveto{\pgfqpoint{2.837547in}{10.311132in}}%
\pgfpathlineto{\pgfqpoint{2.925283in}{10.311132in}}%
\pgfpathlineto{\pgfqpoint{2.925283in}{10.223396in}}%
\pgfpathlineto{\pgfqpoint{2.837547in}{10.223396in}}%
\pgfpathlineto{\pgfqpoint{2.837547in}{10.311132in}}%
\pgfusepath{stroke,fill}%
\end{pgfscope}%
\begin{pgfscope}%
\pgfpathrectangle{\pgfqpoint{0.380943in}{9.960189in}}{\pgfqpoint{4.650000in}{0.614151in}}%
\pgfusepath{clip}%
\pgfsetbuttcap%
\pgfsetroundjoin%
\definecolor{currentfill}{rgb}{1.000000,1.000000,0.929412}%
\pgfsetfillcolor{currentfill}%
\pgfsetlinewidth{0.250937pt}%
\definecolor{currentstroke}{rgb}{1.000000,1.000000,1.000000}%
\pgfsetstrokecolor{currentstroke}%
\pgfsetdash{}{0pt}%
\pgfpathmoveto{\pgfqpoint{2.925283in}{10.311132in}}%
\pgfpathlineto{\pgfqpoint{3.013019in}{10.311132in}}%
\pgfpathlineto{\pgfqpoint{3.013019in}{10.223396in}}%
\pgfpathlineto{\pgfqpoint{2.925283in}{10.223396in}}%
\pgfpathlineto{\pgfqpoint{2.925283in}{10.311132in}}%
\pgfusepath{stroke,fill}%
\end{pgfscope}%
\begin{pgfscope}%
\pgfpathrectangle{\pgfqpoint{0.380943in}{9.960189in}}{\pgfqpoint{4.650000in}{0.614151in}}%
\pgfusepath{clip}%
\pgfsetbuttcap%
\pgfsetroundjoin%
\definecolor{currentfill}{rgb}{1.000000,1.000000,0.929412}%
\pgfsetfillcolor{currentfill}%
\pgfsetlinewidth{0.250937pt}%
\definecolor{currentstroke}{rgb}{1.000000,1.000000,1.000000}%
\pgfsetstrokecolor{currentstroke}%
\pgfsetdash{}{0pt}%
\pgfpathmoveto{\pgfqpoint{3.013019in}{10.311132in}}%
\pgfpathlineto{\pgfqpoint{3.100754in}{10.311132in}}%
\pgfpathlineto{\pgfqpoint{3.100754in}{10.223396in}}%
\pgfpathlineto{\pgfqpoint{3.013019in}{10.223396in}}%
\pgfpathlineto{\pgfqpoint{3.013019in}{10.311132in}}%
\pgfusepath{stroke,fill}%
\end{pgfscope}%
\begin{pgfscope}%
\pgfpathrectangle{\pgfqpoint{0.380943in}{9.960189in}}{\pgfqpoint{4.650000in}{0.614151in}}%
\pgfusepath{clip}%
\pgfsetbuttcap%
\pgfsetroundjoin%
\definecolor{currentfill}{rgb}{1.000000,1.000000,0.929412}%
\pgfsetfillcolor{currentfill}%
\pgfsetlinewidth{0.250937pt}%
\definecolor{currentstroke}{rgb}{1.000000,1.000000,1.000000}%
\pgfsetstrokecolor{currentstroke}%
\pgfsetdash{}{0pt}%
\pgfpathmoveto{\pgfqpoint{3.100754in}{10.311132in}}%
\pgfpathlineto{\pgfqpoint{3.188490in}{10.311132in}}%
\pgfpathlineto{\pgfqpoint{3.188490in}{10.223396in}}%
\pgfpathlineto{\pgfqpoint{3.100754in}{10.223396in}}%
\pgfpathlineto{\pgfqpoint{3.100754in}{10.311132in}}%
\pgfusepath{stroke,fill}%
\end{pgfscope}%
\begin{pgfscope}%
\pgfpathrectangle{\pgfqpoint{0.380943in}{9.960189in}}{\pgfqpoint{4.650000in}{0.614151in}}%
\pgfusepath{clip}%
\pgfsetbuttcap%
\pgfsetroundjoin%
\definecolor{currentfill}{rgb}{1.000000,1.000000,0.929412}%
\pgfsetfillcolor{currentfill}%
\pgfsetlinewidth{0.250937pt}%
\definecolor{currentstroke}{rgb}{1.000000,1.000000,1.000000}%
\pgfsetstrokecolor{currentstroke}%
\pgfsetdash{}{0pt}%
\pgfpathmoveto{\pgfqpoint{3.188490in}{10.311132in}}%
\pgfpathlineto{\pgfqpoint{3.276226in}{10.311132in}}%
\pgfpathlineto{\pgfqpoint{3.276226in}{10.223396in}}%
\pgfpathlineto{\pgfqpoint{3.188490in}{10.223396in}}%
\pgfpathlineto{\pgfqpoint{3.188490in}{10.311132in}}%
\pgfusepath{stroke,fill}%
\end{pgfscope}%
\begin{pgfscope}%
\pgfpathrectangle{\pgfqpoint{0.380943in}{9.960189in}}{\pgfqpoint{4.650000in}{0.614151in}}%
\pgfusepath{clip}%
\pgfsetbuttcap%
\pgfsetroundjoin%
\definecolor{currentfill}{rgb}{1.000000,1.000000,0.929412}%
\pgfsetfillcolor{currentfill}%
\pgfsetlinewidth{0.250937pt}%
\definecolor{currentstroke}{rgb}{1.000000,1.000000,1.000000}%
\pgfsetstrokecolor{currentstroke}%
\pgfsetdash{}{0pt}%
\pgfpathmoveto{\pgfqpoint{3.276226in}{10.311132in}}%
\pgfpathlineto{\pgfqpoint{3.363962in}{10.311132in}}%
\pgfpathlineto{\pgfqpoint{3.363962in}{10.223396in}}%
\pgfpathlineto{\pgfqpoint{3.276226in}{10.223396in}}%
\pgfpathlineto{\pgfqpoint{3.276226in}{10.311132in}}%
\pgfusepath{stroke,fill}%
\end{pgfscope}%
\begin{pgfscope}%
\pgfpathrectangle{\pgfqpoint{0.380943in}{9.960189in}}{\pgfqpoint{4.650000in}{0.614151in}}%
\pgfusepath{clip}%
\pgfsetbuttcap%
\pgfsetroundjoin%
\definecolor{currentfill}{rgb}{1.000000,1.000000,0.929412}%
\pgfsetfillcolor{currentfill}%
\pgfsetlinewidth{0.250937pt}%
\definecolor{currentstroke}{rgb}{1.000000,1.000000,1.000000}%
\pgfsetstrokecolor{currentstroke}%
\pgfsetdash{}{0pt}%
\pgfpathmoveto{\pgfqpoint{3.363962in}{10.311132in}}%
\pgfpathlineto{\pgfqpoint{3.451698in}{10.311132in}}%
\pgfpathlineto{\pgfqpoint{3.451698in}{10.223396in}}%
\pgfpathlineto{\pgfqpoint{3.363962in}{10.223396in}}%
\pgfpathlineto{\pgfqpoint{3.363962in}{10.311132in}}%
\pgfusepath{stroke,fill}%
\end{pgfscope}%
\begin{pgfscope}%
\pgfpathrectangle{\pgfqpoint{0.380943in}{9.960189in}}{\pgfqpoint{4.650000in}{0.614151in}}%
\pgfusepath{clip}%
\pgfsetbuttcap%
\pgfsetroundjoin%
\definecolor{currentfill}{rgb}{1.000000,1.000000,0.929412}%
\pgfsetfillcolor{currentfill}%
\pgfsetlinewidth{0.250937pt}%
\definecolor{currentstroke}{rgb}{1.000000,1.000000,1.000000}%
\pgfsetstrokecolor{currentstroke}%
\pgfsetdash{}{0pt}%
\pgfpathmoveto{\pgfqpoint{3.451698in}{10.311132in}}%
\pgfpathlineto{\pgfqpoint{3.539434in}{10.311132in}}%
\pgfpathlineto{\pgfqpoint{3.539434in}{10.223396in}}%
\pgfpathlineto{\pgfqpoint{3.451698in}{10.223396in}}%
\pgfpathlineto{\pgfqpoint{3.451698in}{10.311132in}}%
\pgfusepath{stroke,fill}%
\end{pgfscope}%
\begin{pgfscope}%
\pgfpathrectangle{\pgfqpoint{0.380943in}{9.960189in}}{\pgfqpoint{4.650000in}{0.614151in}}%
\pgfusepath{clip}%
\pgfsetbuttcap%
\pgfsetroundjoin%
\definecolor{currentfill}{rgb}{1.000000,1.000000,0.929412}%
\pgfsetfillcolor{currentfill}%
\pgfsetlinewidth{0.250937pt}%
\definecolor{currentstroke}{rgb}{1.000000,1.000000,1.000000}%
\pgfsetstrokecolor{currentstroke}%
\pgfsetdash{}{0pt}%
\pgfpathmoveto{\pgfqpoint{3.539434in}{10.311132in}}%
\pgfpathlineto{\pgfqpoint{3.627169in}{10.311132in}}%
\pgfpathlineto{\pgfqpoint{3.627169in}{10.223396in}}%
\pgfpathlineto{\pgfqpoint{3.539434in}{10.223396in}}%
\pgfpathlineto{\pgfqpoint{3.539434in}{10.311132in}}%
\pgfusepath{stroke,fill}%
\end{pgfscope}%
\begin{pgfscope}%
\pgfpathrectangle{\pgfqpoint{0.380943in}{9.960189in}}{\pgfqpoint{4.650000in}{0.614151in}}%
\pgfusepath{clip}%
\pgfsetbuttcap%
\pgfsetroundjoin%
\definecolor{currentfill}{rgb}{1.000000,1.000000,0.929412}%
\pgfsetfillcolor{currentfill}%
\pgfsetlinewidth{0.250937pt}%
\definecolor{currentstroke}{rgb}{1.000000,1.000000,1.000000}%
\pgfsetstrokecolor{currentstroke}%
\pgfsetdash{}{0pt}%
\pgfpathmoveto{\pgfqpoint{3.627169in}{10.311132in}}%
\pgfpathlineto{\pgfqpoint{3.714905in}{10.311132in}}%
\pgfpathlineto{\pgfqpoint{3.714905in}{10.223396in}}%
\pgfpathlineto{\pgfqpoint{3.627169in}{10.223396in}}%
\pgfpathlineto{\pgfqpoint{3.627169in}{10.311132in}}%
\pgfusepath{stroke,fill}%
\end{pgfscope}%
\begin{pgfscope}%
\pgfpathrectangle{\pgfqpoint{0.380943in}{9.960189in}}{\pgfqpoint{4.650000in}{0.614151in}}%
\pgfusepath{clip}%
\pgfsetbuttcap%
\pgfsetroundjoin%
\definecolor{currentfill}{rgb}{1.000000,1.000000,0.929412}%
\pgfsetfillcolor{currentfill}%
\pgfsetlinewidth{0.250937pt}%
\definecolor{currentstroke}{rgb}{1.000000,1.000000,1.000000}%
\pgfsetstrokecolor{currentstroke}%
\pgfsetdash{}{0pt}%
\pgfpathmoveto{\pgfqpoint{3.714905in}{10.311132in}}%
\pgfpathlineto{\pgfqpoint{3.802641in}{10.311132in}}%
\pgfpathlineto{\pgfqpoint{3.802641in}{10.223396in}}%
\pgfpathlineto{\pgfqpoint{3.714905in}{10.223396in}}%
\pgfpathlineto{\pgfqpoint{3.714905in}{10.311132in}}%
\pgfusepath{stroke,fill}%
\end{pgfscope}%
\begin{pgfscope}%
\pgfpathrectangle{\pgfqpoint{0.380943in}{9.960189in}}{\pgfqpoint{4.650000in}{0.614151in}}%
\pgfusepath{clip}%
\pgfsetbuttcap%
\pgfsetroundjoin%
\definecolor{currentfill}{rgb}{1.000000,1.000000,0.929412}%
\pgfsetfillcolor{currentfill}%
\pgfsetlinewidth{0.250937pt}%
\definecolor{currentstroke}{rgb}{1.000000,1.000000,1.000000}%
\pgfsetstrokecolor{currentstroke}%
\pgfsetdash{}{0pt}%
\pgfpathmoveto{\pgfqpoint{3.802641in}{10.311132in}}%
\pgfpathlineto{\pgfqpoint{3.890377in}{10.311132in}}%
\pgfpathlineto{\pgfqpoint{3.890377in}{10.223396in}}%
\pgfpathlineto{\pgfqpoint{3.802641in}{10.223396in}}%
\pgfpathlineto{\pgfqpoint{3.802641in}{10.311132in}}%
\pgfusepath{stroke,fill}%
\end{pgfscope}%
\begin{pgfscope}%
\pgfpathrectangle{\pgfqpoint{0.380943in}{9.960189in}}{\pgfqpoint{4.650000in}{0.614151in}}%
\pgfusepath{clip}%
\pgfsetbuttcap%
\pgfsetroundjoin%
\definecolor{currentfill}{rgb}{1.000000,1.000000,0.929412}%
\pgfsetfillcolor{currentfill}%
\pgfsetlinewidth{0.250937pt}%
\definecolor{currentstroke}{rgb}{1.000000,1.000000,1.000000}%
\pgfsetstrokecolor{currentstroke}%
\pgfsetdash{}{0pt}%
\pgfpathmoveto{\pgfqpoint{3.890377in}{10.311132in}}%
\pgfpathlineto{\pgfqpoint{3.978113in}{10.311132in}}%
\pgfpathlineto{\pgfqpoint{3.978113in}{10.223396in}}%
\pgfpathlineto{\pgfqpoint{3.890377in}{10.223396in}}%
\pgfpathlineto{\pgfqpoint{3.890377in}{10.311132in}}%
\pgfusepath{stroke,fill}%
\end{pgfscope}%
\begin{pgfscope}%
\pgfpathrectangle{\pgfqpoint{0.380943in}{9.960189in}}{\pgfqpoint{4.650000in}{0.614151in}}%
\pgfusepath{clip}%
\pgfsetbuttcap%
\pgfsetroundjoin%
\definecolor{currentfill}{rgb}{1.000000,1.000000,0.929412}%
\pgfsetfillcolor{currentfill}%
\pgfsetlinewidth{0.250937pt}%
\definecolor{currentstroke}{rgb}{1.000000,1.000000,1.000000}%
\pgfsetstrokecolor{currentstroke}%
\pgfsetdash{}{0pt}%
\pgfpathmoveto{\pgfqpoint{3.978113in}{10.311132in}}%
\pgfpathlineto{\pgfqpoint{4.065849in}{10.311132in}}%
\pgfpathlineto{\pgfqpoint{4.065849in}{10.223396in}}%
\pgfpathlineto{\pgfqpoint{3.978113in}{10.223396in}}%
\pgfpathlineto{\pgfqpoint{3.978113in}{10.311132in}}%
\pgfusepath{stroke,fill}%
\end{pgfscope}%
\begin{pgfscope}%
\pgfpathrectangle{\pgfqpoint{0.380943in}{9.960189in}}{\pgfqpoint{4.650000in}{0.614151in}}%
\pgfusepath{clip}%
\pgfsetbuttcap%
\pgfsetroundjoin%
\definecolor{currentfill}{rgb}{1.000000,1.000000,0.929412}%
\pgfsetfillcolor{currentfill}%
\pgfsetlinewidth{0.250937pt}%
\definecolor{currentstroke}{rgb}{1.000000,1.000000,1.000000}%
\pgfsetstrokecolor{currentstroke}%
\pgfsetdash{}{0pt}%
\pgfpathmoveto{\pgfqpoint{4.065849in}{10.311132in}}%
\pgfpathlineto{\pgfqpoint{4.153585in}{10.311132in}}%
\pgfpathlineto{\pgfqpoint{4.153585in}{10.223396in}}%
\pgfpathlineto{\pgfqpoint{4.065849in}{10.223396in}}%
\pgfpathlineto{\pgfqpoint{4.065849in}{10.311132in}}%
\pgfusepath{stroke,fill}%
\end{pgfscope}%
\begin{pgfscope}%
\pgfpathrectangle{\pgfqpoint{0.380943in}{9.960189in}}{\pgfqpoint{4.650000in}{0.614151in}}%
\pgfusepath{clip}%
\pgfsetbuttcap%
\pgfsetroundjoin%
\definecolor{currentfill}{rgb}{1.000000,1.000000,0.929412}%
\pgfsetfillcolor{currentfill}%
\pgfsetlinewidth{0.250937pt}%
\definecolor{currentstroke}{rgb}{1.000000,1.000000,1.000000}%
\pgfsetstrokecolor{currentstroke}%
\pgfsetdash{}{0pt}%
\pgfpathmoveto{\pgfqpoint{4.153585in}{10.311132in}}%
\pgfpathlineto{\pgfqpoint{4.241320in}{10.311132in}}%
\pgfpathlineto{\pgfqpoint{4.241320in}{10.223396in}}%
\pgfpathlineto{\pgfqpoint{4.153585in}{10.223396in}}%
\pgfpathlineto{\pgfqpoint{4.153585in}{10.311132in}}%
\pgfusepath{stroke,fill}%
\end{pgfscope}%
\begin{pgfscope}%
\pgfpathrectangle{\pgfqpoint{0.380943in}{9.960189in}}{\pgfqpoint{4.650000in}{0.614151in}}%
\pgfusepath{clip}%
\pgfsetbuttcap%
\pgfsetroundjoin%
\definecolor{currentfill}{rgb}{1.000000,1.000000,0.929412}%
\pgfsetfillcolor{currentfill}%
\pgfsetlinewidth{0.250937pt}%
\definecolor{currentstroke}{rgb}{1.000000,1.000000,1.000000}%
\pgfsetstrokecolor{currentstroke}%
\pgfsetdash{}{0pt}%
\pgfpathmoveto{\pgfqpoint{4.241320in}{10.311132in}}%
\pgfpathlineto{\pgfqpoint{4.329056in}{10.311132in}}%
\pgfpathlineto{\pgfqpoint{4.329056in}{10.223396in}}%
\pgfpathlineto{\pgfqpoint{4.241320in}{10.223396in}}%
\pgfpathlineto{\pgfqpoint{4.241320in}{10.311132in}}%
\pgfusepath{stroke,fill}%
\end{pgfscope}%
\begin{pgfscope}%
\pgfpathrectangle{\pgfqpoint{0.380943in}{9.960189in}}{\pgfqpoint{4.650000in}{0.614151in}}%
\pgfusepath{clip}%
\pgfsetbuttcap%
\pgfsetroundjoin%
\definecolor{currentfill}{rgb}{1.000000,1.000000,0.929412}%
\pgfsetfillcolor{currentfill}%
\pgfsetlinewidth{0.250937pt}%
\definecolor{currentstroke}{rgb}{1.000000,1.000000,1.000000}%
\pgfsetstrokecolor{currentstroke}%
\pgfsetdash{}{0pt}%
\pgfpathmoveto{\pgfqpoint{4.329056in}{10.311132in}}%
\pgfpathlineto{\pgfqpoint{4.416792in}{10.311132in}}%
\pgfpathlineto{\pgfqpoint{4.416792in}{10.223396in}}%
\pgfpathlineto{\pgfqpoint{4.329056in}{10.223396in}}%
\pgfpathlineto{\pgfqpoint{4.329056in}{10.311132in}}%
\pgfusepath{stroke,fill}%
\end{pgfscope}%
\begin{pgfscope}%
\pgfpathrectangle{\pgfqpoint{0.380943in}{9.960189in}}{\pgfqpoint{4.650000in}{0.614151in}}%
\pgfusepath{clip}%
\pgfsetbuttcap%
\pgfsetroundjoin%
\definecolor{currentfill}{rgb}{0.967474,0.895963,0.706344}%
\pgfsetfillcolor{currentfill}%
\pgfsetlinewidth{0.250937pt}%
\definecolor{currentstroke}{rgb}{1.000000,1.000000,1.000000}%
\pgfsetstrokecolor{currentstroke}%
\pgfsetdash{}{0pt}%
\pgfpathmoveto{\pgfqpoint{4.416792in}{10.311132in}}%
\pgfpathlineto{\pgfqpoint{4.504528in}{10.311132in}}%
\pgfpathlineto{\pgfqpoint{4.504528in}{10.223396in}}%
\pgfpathlineto{\pgfqpoint{4.416792in}{10.223396in}}%
\pgfpathlineto{\pgfqpoint{4.416792in}{10.311132in}}%
\pgfusepath{stroke,fill}%
\end{pgfscope}%
\begin{pgfscope}%
\pgfpathrectangle{\pgfqpoint{0.380943in}{9.960189in}}{\pgfqpoint{4.650000in}{0.614151in}}%
\pgfusepath{clip}%
\pgfsetbuttcap%
\pgfsetroundjoin%
\definecolor{currentfill}{rgb}{0.997924,0.685352,0.570242}%
\pgfsetfillcolor{currentfill}%
\pgfsetlinewidth{0.250937pt}%
\definecolor{currentstroke}{rgb}{1.000000,1.000000,1.000000}%
\pgfsetstrokecolor{currentstroke}%
\pgfsetdash{}{0pt}%
\pgfpathmoveto{\pgfqpoint{4.504528in}{10.311132in}}%
\pgfpathlineto{\pgfqpoint{4.592264in}{10.311132in}}%
\pgfpathlineto{\pgfqpoint{4.592264in}{10.223396in}}%
\pgfpathlineto{\pgfqpoint{4.504528in}{10.223396in}}%
\pgfpathlineto{\pgfqpoint{4.504528in}{10.311132in}}%
\pgfusepath{stroke,fill}%
\end{pgfscope}%
\begin{pgfscope}%
\pgfpathrectangle{\pgfqpoint{0.380943in}{9.960189in}}{\pgfqpoint{4.650000in}{0.614151in}}%
\pgfusepath{clip}%
\pgfsetbuttcap%
\pgfsetroundjoin%
\definecolor{currentfill}{rgb}{0.967474,0.895963,0.706344}%
\pgfsetfillcolor{currentfill}%
\pgfsetlinewidth{0.250937pt}%
\definecolor{currentstroke}{rgb}{1.000000,1.000000,1.000000}%
\pgfsetstrokecolor{currentstroke}%
\pgfsetdash{}{0pt}%
\pgfpathmoveto{\pgfqpoint{4.592264in}{10.311132in}}%
\pgfpathlineto{\pgfqpoint{4.680000in}{10.311132in}}%
\pgfpathlineto{\pgfqpoint{4.680000in}{10.223396in}}%
\pgfpathlineto{\pgfqpoint{4.592264in}{10.223396in}}%
\pgfpathlineto{\pgfqpoint{4.592264in}{10.311132in}}%
\pgfusepath{stroke,fill}%
\end{pgfscope}%
\begin{pgfscope}%
\pgfpathrectangle{\pgfqpoint{0.380943in}{9.960189in}}{\pgfqpoint{4.650000in}{0.614151in}}%
\pgfusepath{clip}%
\pgfsetbuttcap%
\pgfsetroundjoin%
\definecolor{currentfill}{rgb}{0.989619,0.788235,0.628374}%
\pgfsetfillcolor{currentfill}%
\pgfsetlinewidth{0.250937pt}%
\definecolor{currentstroke}{rgb}{1.000000,1.000000,1.000000}%
\pgfsetstrokecolor{currentstroke}%
\pgfsetdash{}{0pt}%
\pgfpathmoveto{\pgfqpoint{4.680000in}{10.311132in}}%
\pgfpathlineto{\pgfqpoint{4.767736in}{10.311132in}}%
\pgfpathlineto{\pgfqpoint{4.767736in}{10.223396in}}%
\pgfpathlineto{\pgfqpoint{4.680000in}{10.223396in}}%
\pgfpathlineto{\pgfqpoint{4.680000in}{10.311132in}}%
\pgfusepath{stroke,fill}%
\end{pgfscope}%
\begin{pgfscope}%
\pgfpathrectangle{\pgfqpoint{0.380943in}{9.960189in}}{\pgfqpoint{4.650000in}{0.614151in}}%
\pgfusepath{clip}%
\pgfsetbuttcap%
\pgfsetroundjoin%
\definecolor{currentfill}{rgb}{0.982699,0.823991,0.657439}%
\pgfsetfillcolor{currentfill}%
\pgfsetlinewidth{0.250937pt}%
\definecolor{currentstroke}{rgb}{1.000000,1.000000,1.000000}%
\pgfsetstrokecolor{currentstroke}%
\pgfsetdash{}{0pt}%
\pgfpathmoveto{\pgfqpoint{4.767736in}{10.311132in}}%
\pgfpathlineto{\pgfqpoint{4.855471in}{10.311132in}}%
\pgfpathlineto{\pgfqpoint{4.855471in}{10.223396in}}%
\pgfpathlineto{\pgfqpoint{4.767736in}{10.223396in}}%
\pgfpathlineto{\pgfqpoint{4.767736in}{10.311132in}}%
\pgfusepath{stroke,fill}%
\end{pgfscope}%
\begin{pgfscope}%
\pgfpathrectangle{\pgfqpoint{0.380943in}{9.960189in}}{\pgfqpoint{4.650000in}{0.614151in}}%
\pgfusepath{clip}%
\pgfsetbuttcap%
\pgfsetroundjoin%
\definecolor{currentfill}{rgb}{0.994694,0.745098,0.602999}%
\pgfsetfillcolor{currentfill}%
\pgfsetlinewidth{0.250937pt}%
\definecolor{currentstroke}{rgb}{1.000000,1.000000,1.000000}%
\pgfsetstrokecolor{currentstroke}%
\pgfsetdash{}{0pt}%
\pgfpathmoveto{\pgfqpoint{4.855471in}{10.311132in}}%
\pgfpathlineto{\pgfqpoint{4.943207in}{10.311132in}}%
\pgfpathlineto{\pgfqpoint{4.943207in}{10.223396in}}%
\pgfpathlineto{\pgfqpoint{4.855471in}{10.223396in}}%
\pgfpathlineto{\pgfqpoint{4.855471in}{10.311132in}}%
\pgfusepath{stroke,fill}%
\end{pgfscope}%
\begin{pgfscope}%
\pgfpathrectangle{\pgfqpoint{0.380943in}{9.960189in}}{\pgfqpoint{4.650000in}{0.614151in}}%
\pgfusepath{clip}%
\pgfsetbuttcap%
\pgfsetroundjoin%
\definecolor{currentfill}{rgb}{0.989619,0.788235,0.628374}%
\pgfsetfillcolor{currentfill}%
\pgfsetlinewidth{0.250937pt}%
\definecolor{currentstroke}{rgb}{1.000000,1.000000,1.000000}%
\pgfsetstrokecolor{currentstroke}%
\pgfsetdash{}{0pt}%
\pgfpathmoveto{\pgfqpoint{4.943207in}{10.311132in}}%
\pgfpathlineto{\pgfqpoint{5.030943in}{10.311132in}}%
\pgfpathlineto{\pgfqpoint{5.030943in}{10.223396in}}%
\pgfpathlineto{\pgfqpoint{4.943207in}{10.223396in}}%
\pgfpathlineto{\pgfqpoint{4.943207in}{10.311132in}}%
\pgfusepath{stroke,fill}%
\end{pgfscope}%
\begin{pgfscope}%
\pgfpathrectangle{\pgfqpoint{0.380943in}{9.960189in}}{\pgfqpoint{4.650000in}{0.614151in}}%
\pgfusepath{clip}%
\pgfsetbuttcap%
\pgfsetroundjoin%
\definecolor{currentfill}{rgb}{1.000000,1.000000,0.929412}%
\pgfsetfillcolor{currentfill}%
\pgfsetlinewidth{0.250937pt}%
\definecolor{currentstroke}{rgb}{1.000000,1.000000,1.000000}%
\pgfsetstrokecolor{currentstroke}%
\pgfsetdash{}{0pt}%
\pgfpathmoveto{\pgfqpoint{0.380943in}{10.223396in}}%
\pgfpathlineto{\pgfqpoint{0.468679in}{10.223396in}}%
\pgfpathlineto{\pgfqpoint{0.468679in}{10.135661in}}%
\pgfpathlineto{\pgfqpoint{0.380943in}{10.135661in}}%
\pgfpathlineto{\pgfqpoint{0.380943in}{10.223396in}}%
\pgfusepath{stroke,fill}%
\end{pgfscope}%
\begin{pgfscope}%
\pgfpathrectangle{\pgfqpoint{0.380943in}{9.960189in}}{\pgfqpoint{4.650000in}{0.614151in}}%
\pgfusepath{clip}%
\pgfsetbuttcap%
\pgfsetroundjoin%
\definecolor{currentfill}{rgb}{1.000000,1.000000,0.929412}%
\pgfsetfillcolor{currentfill}%
\pgfsetlinewidth{0.250937pt}%
\definecolor{currentstroke}{rgb}{1.000000,1.000000,1.000000}%
\pgfsetstrokecolor{currentstroke}%
\pgfsetdash{}{0pt}%
\pgfpathmoveto{\pgfqpoint{0.468679in}{10.223396in}}%
\pgfpathlineto{\pgfqpoint{0.556415in}{10.223396in}}%
\pgfpathlineto{\pgfqpoint{0.556415in}{10.135661in}}%
\pgfpathlineto{\pgfqpoint{0.468679in}{10.135661in}}%
\pgfpathlineto{\pgfqpoint{0.468679in}{10.223396in}}%
\pgfusepath{stroke,fill}%
\end{pgfscope}%
\begin{pgfscope}%
\pgfpathrectangle{\pgfqpoint{0.380943in}{9.960189in}}{\pgfqpoint{4.650000in}{0.614151in}}%
\pgfusepath{clip}%
\pgfsetbuttcap%
\pgfsetroundjoin%
\definecolor{currentfill}{rgb}{1.000000,1.000000,0.929412}%
\pgfsetfillcolor{currentfill}%
\pgfsetlinewidth{0.250937pt}%
\definecolor{currentstroke}{rgb}{1.000000,1.000000,1.000000}%
\pgfsetstrokecolor{currentstroke}%
\pgfsetdash{}{0pt}%
\pgfpathmoveto{\pgfqpoint{0.556415in}{10.223396in}}%
\pgfpathlineto{\pgfqpoint{0.644151in}{10.223396in}}%
\pgfpathlineto{\pgfqpoint{0.644151in}{10.135661in}}%
\pgfpathlineto{\pgfqpoint{0.556415in}{10.135661in}}%
\pgfpathlineto{\pgfqpoint{0.556415in}{10.223396in}}%
\pgfusepath{stroke,fill}%
\end{pgfscope}%
\begin{pgfscope}%
\pgfpathrectangle{\pgfqpoint{0.380943in}{9.960189in}}{\pgfqpoint{4.650000in}{0.614151in}}%
\pgfusepath{clip}%
\pgfsetbuttcap%
\pgfsetroundjoin%
\definecolor{currentfill}{rgb}{1.000000,1.000000,0.929412}%
\pgfsetfillcolor{currentfill}%
\pgfsetlinewidth{0.250937pt}%
\definecolor{currentstroke}{rgb}{1.000000,1.000000,1.000000}%
\pgfsetstrokecolor{currentstroke}%
\pgfsetdash{}{0pt}%
\pgfpathmoveto{\pgfqpoint{0.644151in}{10.223396in}}%
\pgfpathlineto{\pgfqpoint{0.731886in}{10.223396in}}%
\pgfpathlineto{\pgfqpoint{0.731886in}{10.135661in}}%
\pgfpathlineto{\pgfqpoint{0.644151in}{10.135661in}}%
\pgfpathlineto{\pgfqpoint{0.644151in}{10.223396in}}%
\pgfusepath{stroke,fill}%
\end{pgfscope}%
\begin{pgfscope}%
\pgfpathrectangle{\pgfqpoint{0.380943in}{9.960189in}}{\pgfqpoint{4.650000in}{0.614151in}}%
\pgfusepath{clip}%
\pgfsetbuttcap%
\pgfsetroundjoin%
\definecolor{currentfill}{rgb}{1.000000,1.000000,0.929412}%
\pgfsetfillcolor{currentfill}%
\pgfsetlinewidth{0.250937pt}%
\definecolor{currentstroke}{rgb}{1.000000,1.000000,1.000000}%
\pgfsetstrokecolor{currentstroke}%
\pgfsetdash{}{0pt}%
\pgfpathmoveto{\pgfqpoint{0.731886in}{10.223396in}}%
\pgfpathlineto{\pgfqpoint{0.819622in}{10.223396in}}%
\pgfpathlineto{\pgfqpoint{0.819622in}{10.135661in}}%
\pgfpathlineto{\pgfqpoint{0.731886in}{10.135661in}}%
\pgfpathlineto{\pgfqpoint{0.731886in}{10.223396in}}%
\pgfusepath{stroke,fill}%
\end{pgfscope}%
\begin{pgfscope}%
\pgfpathrectangle{\pgfqpoint{0.380943in}{9.960189in}}{\pgfqpoint{4.650000in}{0.614151in}}%
\pgfusepath{clip}%
\pgfsetbuttcap%
\pgfsetroundjoin%
\definecolor{currentfill}{rgb}{1.000000,1.000000,0.929412}%
\pgfsetfillcolor{currentfill}%
\pgfsetlinewidth{0.250937pt}%
\definecolor{currentstroke}{rgb}{1.000000,1.000000,1.000000}%
\pgfsetstrokecolor{currentstroke}%
\pgfsetdash{}{0pt}%
\pgfpathmoveto{\pgfqpoint{0.819622in}{10.223396in}}%
\pgfpathlineto{\pgfqpoint{0.907358in}{10.223396in}}%
\pgfpathlineto{\pgfqpoint{0.907358in}{10.135661in}}%
\pgfpathlineto{\pgfqpoint{0.819622in}{10.135661in}}%
\pgfpathlineto{\pgfqpoint{0.819622in}{10.223396in}}%
\pgfusepath{stroke,fill}%
\end{pgfscope}%
\begin{pgfscope}%
\pgfpathrectangle{\pgfqpoint{0.380943in}{9.960189in}}{\pgfqpoint{4.650000in}{0.614151in}}%
\pgfusepath{clip}%
\pgfsetbuttcap%
\pgfsetroundjoin%
\definecolor{currentfill}{rgb}{1.000000,1.000000,0.929412}%
\pgfsetfillcolor{currentfill}%
\pgfsetlinewidth{0.250937pt}%
\definecolor{currentstroke}{rgb}{1.000000,1.000000,1.000000}%
\pgfsetstrokecolor{currentstroke}%
\pgfsetdash{}{0pt}%
\pgfpathmoveto{\pgfqpoint{0.907358in}{10.223396in}}%
\pgfpathlineto{\pgfqpoint{0.995094in}{10.223396in}}%
\pgfpathlineto{\pgfqpoint{0.995094in}{10.135661in}}%
\pgfpathlineto{\pgfqpoint{0.907358in}{10.135661in}}%
\pgfpathlineto{\pgfqpoint{0.907358in}{10.223396in}}%
\pgfusepath{stroke,fill}%
\end{pgfscope}%
\begin{pgfscope}%
\pgfpathrectangle{\pgfqpoint{0.380943in}{9.960189in}}{\pgfqpoint{4.650000in}{0.614151in}}%
\pgfusepath{clip}%
\pgfsetbuttcap%
\pgfsetroundjoin%
\definecolor{currentfill}{rgb}{1.000000,1.000000,0.929412}%
\pgfsetfillcolor{currentfill}%
\pgfsetlinewidth{0.250937pt}%
\definecolor{currentstroke}{rgb}{1.000000,1.000000,1.000000}%
\pgfsetstrokecolor{currentstroke}%
\pgfsetdash{}{0pt}%
\pgfpathmoveto{\pgfqpoint{0.995094in}{10.223396in}}%
\pgfpathlineto{\pgfqpoint{1.082830in}{10.223396in}}%
\pgfpathlineto{\pgfqpoint{1.082830in}{10.135661in}}%
\pgfpathlineto{\pgfqpoint{0.995094in}{10.135661in}}%
\pgfpathlineto{\pgfqpoint{0.995094in}{10.223396in}}%
\pgfusepath{stroke,fill}%
\end{pgfscope}%
\begin{pgfscope}%
\pgfpathrectangle{\pgfqpoint{0.380943in}{9.960189in}}{\pgfqpoint{4.650000in}{0.614151in}}%
\pgfusepath{clip}%
\pgfsetbuttcap%
\pgfsetroundjoin%
\definecolor{currentfill}{rgb}{1.000000,1.000000,0.929412}%
\pgfsetfillcolor{currentfill}%
\pgfsetlinewidth{0.250937pt}%
\definecolor{currentstroke}{rgb}{1.000000,1.000000,1.000000}%
\pgfsetstrokecolor{currentstroke}%
\pgfsetdash{}{0pt}%
\pgfpathmoveto{\pgfqpoint{1.082830in}{10.223396in}}%
\pgfpathlineto{\pgfqpoint{1.170566in}{10.223396in}}%
\pgfpathlineto{\pgfqpoint{1.170566in}{10.135661in}}%
\pgfpathlineto{\pgfqpoint{1.082830in}{10.135661in}}%
\pgfpathlineto{\pgfqpoint{1.082830in}{10.223396in}}%
\pgfusepath{stroke,fill}%
\end{pgfscope}%
\begin{pgfscope}%
\pgfpathrectangle{\pgfqpoint{0.380943in}{9.960189in}}{\pgfqpoint{4.650000in}{0.614151in}}%
\pgfusepath{clip}%
\pgfsetbuttcap%
\pgfsetroundjoin%
\definecolor{currentfill}{rgb}{1.000000,1.000000,0.929412}%
\pgfsetfillcolor{currentfill}%
\pgfsetlinewidth{0.250937pt}%
\definecolor{currentstroke}{rgb}{1.000000,1.000000,1.000000}%
\pgfsetstrokecolor{currentstroke}%
\pgfsetdash{}{0pt}%
\pgfpathmoveto{\pgfqpoint{1.170566in}{10.223396in}}%
\pgfpathlineto{\pgfqpoint{1.258302in}{10.223396in}}%
\pgfpathlineto{\pgfqpoint{1.258302in}{10.135661in}}%
\pgfpathlineto{\pgfqpoint{1.170566in}{10.135661in}}%
\pgfpathlineto{\pgfqpoint{1.170566in}{10.223396in}}%
\pgfusepath{stroke,fill}%
\end{pgfscope}%
\begin{pgfscope}%
\pgfpathrectangle{\pgfqpoint{0.380943in}{9.960189in}}{\pgfqpoint{4.650000in}{0.614151in}}%
\pgfusepath{clip}%
\pgfsetbuttcap%
\pgfsetroundjoin%
\definecolor{currentfill}{rgb}{1.000000,1.000000,0.929412}%
\pgfsetfillcolor{currentfill}%
\pgfsetlinewidth{0.250937pt}%
\definecolor{currentstroke}{rgb}{1.000000,1.000000,1.000000}%
\pgfsetstrokecolor{currentstroke}%
\pgfsetdash{}{0pt}%
\pgfpathmoveto{\pgfqpoint{1.258302in}{10.223396in}}%
\pgfpathlineto{\pgfqpoint{1.346037in}{10.223396in}}%
\pgfpathlineto{\pgfqpoint{1.346037in}{10.135661in}}%
\pgfpathlineto{\pgfqpoint{1.258302in}{10.135661in}}%
\pgfpathlineto{\pgfqpoint{1.258302in}{10.223396in}}%
\pgfusepath{stroke,fill}%
\end{pgfscope}%
\begin{pgfscope}%
\pgfpathrectangle{\pgfqpoint{0.380943in}{9.960189in}}{\pgfqpoint{4.650000in}{0.614151in}}%
\pgfusepath{clip}%
\pgfsetbuttcap%
\pgfsetroundjoin%
\definecolor{currentfill}{rgb}{1.000000,1.000000,0.929412}%
\pgfsetfillcolor{currentfill}%
\pgfsetlinewidth{0.250937pt}%
\definecolor{currentstroke}{rgb}{1.000000,1.000000,1.000000}%
\pgfsetstrokecolor{currentstroke}%
\pgfsetdash{}{0pt}%
\pgfpathmoveto{\pgfqpoint{1.346037in}{10.223396in}}%
\pgfpathlineto{\pgfqpoint{1.433773in}{10.223396in}}%
\pgfpathlineto{\pgfqpoint{1.433773in}{10.135661in}}%
\pgfpathlineto{\pgfqpoint{1.346037in}{10.135661in}}%
\pgfpathlineto{\pgfqpoint{1.346037in}{10.223396in}}%
\pgfusepath{stroke,fill}%
\end{pgfscope}%
\begin{pgfscope}%
\pgfpathrectangle{\pgfqpoint{0.380943in}{9.960189in}}{\pgfqpoint{4.650000in}{0.614151in}}%
\pgfusepath{clip}%
\pgfsetbuttcap%
\pgfsetroundjoin%
\definecolor{currentfill}{rgb}{1.000000,1.000000,0.929412}%
\pgfsetfillcolor{currentfill}%
\pgfsetlinewidth{0.250937pt}%
\definecolor{currentstroke}{rgb}{1.000000,1.000000,1.000000}%
\pgfsetstrokecolor{currentstroke}%
\pgfsetdash{}{0pt}%
\pgfpathmoveto{\pgfqpoint{1.433773in}{10.223396in}}%
\pgfpathlineto{\pgfqpoint{1.521509in}{10.223396in}}%
\pgfpathlineto{\pgfqpoint{1.521509in}{10.135661in}}%
\pgfpathlineto{\pgfqpoint{1.433773in}{10.135661in}}%
\pgfpathlineto{\pgfqpoint{1.433773in}{10.223396in}}%
\pgfusepath{stroke,fill}%
\end{pgfscope}%
\begin{pgfscope}%
\pgfpathrectangle{\pgfqpoint{0.380943in}{9.960189in}}{\pgfqpoint{4.650000in}{0.614151in}}%
\pgfusepath{clip}%
\pgfsetbuttcap%
\pgfsetroundjoin%
\definecolor{currentfill}{rgb}{1.000000,1.000000,0.929412}%
\pgfsetfillcolor{currentfill}%
\pgfsetlinewidth{0.250937pt}%
\definecolor{currentstroke}{rgb}{1.000000,1.000000,1.000000}%
\pgfsetstrokecolor{currentstroke}%
\pgfsetdash{}{0pt}%
\pgfpathmoveto{\pgfqpoint{1.521509in}{10.223396in}}%
\pgfpathlineto{\pgfqpoint{1.609245in}{10.223396in}}%
\pgfpathlineto{\pgfqpoint{1.609245in}{10.135661in}}%
\pgfpathlineto{\pgfqpoint{1.521509in}{10.135661in}}%
\pgfpathlineto{\pgfqpoint{1.521509in}{10.223396in}}%
\pgfusepath{stroke,fill}%
\end{pgfscope}%
\begin{pgfscope}%
\pgfpathrectangle{\pgfqpoint{0.380943in}{9.960189in}}{\pgfqpoint{4.650000in}{0.614151in}}%
\pgfusepath{clip}%
\pgfsetbuttcap%
\pgfsetroundjoin%
\definecolor{currentfill}{rgb}{1.000000,1.000000,0.929412}%
\pgfsetfillcolor{currentfill}%
\pgfsetlinewidth{0.250937pt}%
\definecolor{currentstroke}{rgb}{1.000000,1.000000,1.000000}%
\pgfsetstrokecolor{currentstroke}%
\pgfsetdash{}{0pt}%
\pgfpathmoveto{\pgfqpoint{1.609245in}{10.223396in}}%
\pgfpathlineto{\pgfqpoint{1.696981in}{10.223396in}}%
\pgfpathlineto{\pgfqpoint{1.696981in}{10.135661in}}%
\pgfpathlineto{\pgfqpoint{1.609245in}{10.135661in}}%
\pgfpathlineto{\pgfqpoint{1.609245in}{10.223396in}}%
\pgfusepath{stroke,fill}%
\end{pgfscope}%
\begin{pgfscope}%
\pgfpathrectangle{\pgfqpoint{0.380943in}{9.960189in}}{\pgfqpoint{4.650000in}{0.614151in}}%
\pgfusepath{clip}%
\pgfsetbuttcap%
\pgfsetroundjoin%
\definecolor{currentfill}{rgb}{1.000000,1.000000,0.929412}%
\pgfsetfillcolor{currentfill}%
\pgfsetlinewidth{0.250937pt}%
\definecolor{currentstroke}{rgb}{1.000000,1.000000,1.000000}%
\pgfsetstrokecolor{currentstroke}%
\pgfsetdash{}{0pt}%
\pgfpathmoveto{\pgfqpoint{1.696981in}{10.223396in}}%
\pgfpathlineto{\pgfqpoint{1.784717in}{10.223396in}}%
\pgfpathlineto{\pgfqpoint{1.784717in}{10.135661in}}%
\pgfpathlineto{\pgfqpoint{1.696981in}{10.135661in}}%
\pgfpathlineto{\pgfqpoint{1.696981in}{10.223396in}}%
\pgfusepath{stroke,fill}%
\end{pgfscope}%
\begin{pgfscope}%
\pgfpathrectangle{\pgfqpoint{0.380943in}{9.960189in}}{\pgfqpoint{4.650000in}{0.614151in}}%
\pgfusepath{clip}%
\pgfsetbuttcap%
\pgfsetroundjoin%
\definecolor{currentfill}{rgb}{1.000000,1.000000,0.929412}%
\pgfsetfillcolor{currentfill}%
\pgfsetlinewidth{0.250937pt}%
\definecolor{currentstroke}{rgb}{1.000000,1.000000,1.000000}%
\pgfsetstrokecolor{currentstroke}%
\pgfsetdash{}{0pt}%
\pgfpathmoveto{\pgfqpoint{1.784717in}{10.223396in}}%
\pgfpathlineto{\pgfqpoint{1.872452in}{10.223396in}}%
\pgfpathlineto{\pgfqpoint{1.872452in}{10.135661in}}%
\pgfpathlineto{\pgfqpoint{1.784717in}{10.135661in}}%
\pgfpathlineto{\pgfqpoint{1.784717in}{10.223396in}}%
\pgfusepath{stroke,fill}%
\end{pgfscope}%
\begin{pgfscope}%
\pgfpathrectangle{\pgfqpoint{0.380943in}{9.960189in}}{\pgfqpoint{4.650000in}{0.614151in}}%
\pgfusepath{clip}%
\pgfsetbuttcap%
\pgfsetroundjoin%
\definecolor{currentfill}{rgb}{1.000000,1.000000,0.929412}%
\pgfsetfillcolor{currentfill}%
\pgfsetlinewidth{0.250937pt}%
\definecolor{currentstroke}{rgb}{1.000000,1.000000,1.000000}%
\pgfsetstrokecolor{currentstroke}%
\pgfsetdash{}{0pt}%
\pgfpathmoveto{\pgfqpoint{1.872452in}{10.223396in}}%
\pgfpathlineto{\pgfqpoint{1.960188in}{10.223396in}}%
\pgfpathlineto{\pgfqpoint{1.960188in}{10.135661in}}%
\pgfpathlineto{\pgfqpoint{1.872452in}{10.135661in}}%
\pgfpathlineto{\pgfqpoint{1.872452in}{10.223396in}}%
\pgfusepath{stroke,fill}%
\end{pgfscope}%
\begin{pgfscope}%
\pgfpathrectangle{\pgfqpoint{0.380943in}{9.960189in}}{\pgfqpoint{4.650000in}{0.614151in}}%
\pgfusepath{clip}%
\pgfsetbuttcap%
\pgfsetroundjoin%
\definecolor{currentfill}{rgb}{1.000000,1.000000,0.929412}%
\pgfsetfillcolor{currentfill}%
\pgfsetlinewidth{0.250937pt}%
\definecolor{currentstroke}{rgb}{1.000000,1.000000,1.000000}%
\pgfsetstrokecolor{currentstroke}%
\pgfsetdash{}{0pt}%
\pgfpathmoveto{\pgfqpoint{1.960188in}{10.223396in}}%
\pgfpathlineto{\pgfqpoint{2.047924in}{10.223396in}}%
\pgfpathlineto{\pgfqpoint{2.047924in}{10.135661in}}%
\pgfpathlineto{\pgfqpoint{1.960188in}{10.135661in}}%
\pgfpathlineto{\pgfqpoint{1.960188in}{10.223396in}}%
\pgfusepath{stroke,fill}%
\end{pgfscope}%
\begin{pgfscope}%
\pgfpathrectangle{\pgfqpoint{0.380943in}{9.960189in}}{\pgfqpoint{4.650000in}{0.614151in}}%
\pgfusepath{clip}%
\pgfsetbuttcap%
\pgfsetroundjoin%
\definecolor{currentfill}{rgb}{1.000000,1.000000,0.929412}%
\pgfsetfillcolor{currentfill}%
\pgfsetlinewidth{0.250937pt}%
\definecolor{currentstroke}{rgb}{1.000000,1.000000,1.000000}%
\pgfsetstrokecolor{currentstroke}%
\pgfsetdash{}{0pt}%
\pgfpathmoveto{\pgfqpoint{2.047924in}{10.223396in}}%
\pgfpathlineto{\pgfqpoint{2.135660in}{10.223396in}}%
\pgfpathlineto{\pgfqpoint{2.135660in}{10.135661in}}%
\pgfpathlineto{\pgfqpoint{2.047924in}{10.135661in}}%
\pgfpathlineto{\pgfqpoint{2.047924in}{10.223396in}}%
\pgfusepath{stroke,fill}%
\end{pgfscope}%
\begin{pgfscope}%
\pgfpathrectangle{\pgfqpoint{0.380943in}{9.960189in}}{\pgfqpoint{4.650000in}{0.614151in}}%
\pgfusepath{clip}%
\pgfsetbuttcap%
\pgfsetroundjoin%
\definecolor{currentfill}{rgb}{1.000000,1.000000,0.929412}%
\pgfsetfillcolor{currentfill}%
\pgfsetlinewidth{0.250937pt}%
\definecolor{currentstroke}{rgb}{1.000000,1.000000,1.000000}%
\pgfsetstrokecolor{currentstroke}%
\pgfsetdash{}{0pt}%
\pgfpathmoveto{\pgfqpoint{2.135660in}{10.223396in}}%
\pgfpathlineto{\pgfqpoint{2.223396in}{10.223396in}}%
\pgfpathlineto{\pgfqpoint{2.223396in}{10.135661in}}%
\pgfpathlineto{\pgfqpoint{2.135660in}{10.135661in}}%
\pgfpathlineto{\pgfqpoint{2.135660in}{10.223396in}}%
\pgfusepath{stroke,fill}%
\end{pgfscope}%
\begin{pgfscope}%
\pgfpathrectangle{\pgfqpoint{0.380943in}{9.960189in}}{\pgfqpoint{4.650000in}{0.614151in}}%
\pgfusepath{clip}%
\pgfsetbuttcap%
\pgfsetroundjoin%
\definecolor{currentfill}{rgb}{1.000000,1.000000,0.929412}%
\pgfsetfillcolor{currentfill}%
\pgfsetlinewidth{0.250937pt}%
\definecolor{currentstroke}{rgb}{1.000000,1.000000,1.000000}%
\pgfsetstrokecolor{currentstroke}%
\pgfsetdash{}{0pt}%
\pgfpathmoveto{\pgfqpoint{2.223396in}{10.223396in}}%
\pgfpathlineto{\pgfqpoint{2.311132in}{10.223396in}}%
\pgfpathlineto{\pgfqpoint{2.311132in}{10.135661in}}%
\pgfpathlineto{\pgfqpoint{2.223396in}{10.135661in}}%
\pgfpathlineto{\pgfqpoint{2.223396in}{10.223396in}}%
\pgfusepath{stroke,fill}%
\end{pgfscope}%
\begin{pgfscope}%
\pgfpathrectangle{\pgfqpoint{0.380943in}{9.960189in}}{\pgfqpoint{4.650000in}{0.614151in}}%
\pgfusepath{clip}%
\pgfsetbuttcap%
\pgfsetroundjoin%
\definecolor{currentfill}{rgb}{1.000000,1.000000,0.929412}%
\pgfsetfillcolor{currentfill}%
\pgfsetlinewidth{0.250937pt}%
\definecolor{currentstroke}{rgb}{1.000000,1.000000,1.000000}%
\pgfsetstrokecolor{currentstroke}%
\pgfsetdash{}{0pt}%
\pgfpathmoveto{\pgfqpoint{2.311132in}{10.223396in}}%
\pgfpathlineto{\pgfqpoint{2.398868in}{10.223396in}}%
\pgfpathlineto{\pgfqpoint{2.398868in}{10.135661in}}%
\pgfpathlineto{\pgfqpoint{2.311132in}{10.135661in}}%
\pgfpathlineto{\pgfqpoint{2.311132in}{10.223396in}}%
\pgfusepath{stroke,fill}%
\end{pgfscope}%
\begin{pgfscope}%
\pgfpathrectangle{\pgfqpoint{0.380943in}{9.960189in}}{\pgfqpoint{4.650000in}{0.614151in}}%
\pgfusepath{clip}%
\pgfsetbuttcap%
\pgfsetroundjoin%
\definecolor{currentfill}{rgb}{1.000000,1.000000,0.929412}%
\pgfsetfillcolor{currentfill}%
\pgfsetlinewidth{0.250937pt}%
\definecolor{currentstroke}{rgb}{1.000000,1.000000,1.000000}%
\pgfsetstrokecolor{currentstroke}%
\pgfsetdash{}{0pt}%
\pgfpathmoveto{\pgfqpoint{2.398868in}{10.223396in}}%
\pgfpathlineto{\pgfqpoint{2.486603in}{10.223396in}}%
\pgfpathlineto{\pgfqpoint{2.486603in}{10.135661in}}%
\pgfpathlineto{\pgfqpoint{2.398868in}{10.135661in}}%
\pgfpathlineto{\pgfqpoint{2.398868in}{10.223396in}}%
\pgfusepath{stroke,fill}%
\end{pgfscope}%
\begin{pgfscope}%
\pgfpathrectangle{\pgfqpoint{0.380943in}{9.960189in}}{\pgfqpoint{4.650000in}{0.614151in}}%
\pgfusepath{clip}%
\pgfsetbuttcap%
\pgfsetroundjoin%
\definecolor{currentfill}{rgb}{1.000000,1.000000,0.929412}%
\pgfsetfillcolor{currentfill}%
\pgfsetlinewidth{0.250937pt}%
\definecolor{currentstroke}{rgb}{1.000000,1.000000,1.000000}%
\pgfsetstrokecolor{currentstroke}%
\pgfsetdash{}{0pt}%
\pgfpathmoveto{\pgfqpoint{2.486603in}{10.223396in}}%
\pgfpathlineto{\pgfqpoint{2.574339in}{10.223396in}}%
\pgfpathlineto{\pgfqpoint{2.574339in}{10.135661in}}%
\pgfpathlineto{\pgfqpoint{2.486603in}{10.135661in}}%
\pgfpathlineto{\pgfqpoint{2.486603in}{10.223396in}}%
\pgfusepath{stroke,fill}%
\end{pgfscope}%
\begin{pgfscope}%
\pgfpathrectangle{\pgfqpoint{0.380943in}{9.960189in}}{\pgfqpoint{4.650000in}{0.614151in}}%
\pgfusepath{clip}%
\pgfsetbuttcap%
\pgfsetroundjoin%
\definecolor{currentfill}{rgb}{1.000000,1.000000,0.929412}%
\pgfsetfillcolor{currentfill}%
\pgfsetlinewidth{0.250937pt}%
\definecolor{currentstroke}{rgb}{1.000000,1.000000,1.000000}%
\pgfsetstrokecolor{currentstroke}%
\pgfsetdash{}{0pt}%
\pgfpathmoveto{\pgfqpoint{2.574339in}{10.223396in}}%
\pgfpathlineto{\pgfqpoint{2.662075in}{10.223396in}}%
\pgfpathlineto{\pgfqpoint{2.662075in}{10.135661in}}%
\pgfpathlineto{\pgfqpoint{2.574339in}{10.135661in}}%
\pgfpathlineto{\pgfqpoint{2.574339in}{10.223396in}}%
\pgfusepath{stroke,fill}%
\end{pgfscope}%
\begin{pgfscope}%
\pgfpathrectangle{\pgfqpoint{0.380943in}{9.960189in}}{\pgfqpoint{4.650000in}{0.614151in}}%
\pgfusepath{clip}%
\pgfsetbuttcap%
\pgfsetroundjoin%
\definecolor{currentfill}{rgb}{1.000000,1.000000,0.929412}%
\pgfsetfillcolor{currentfill}%
\pgfsetlinewidth{0.250937pt}%
\definecolor{currentstroke}{rgb}{1.000000,1.000000,1.000000}%
\pgfsetstrokecolor{currentstroke}%
\pgfsetdash{}{0pt}%
\pgfpathmoveto{\pgfqpoint{2.662075in}{10.223396in}}%
\pgfpathlineto{\pgfqpoint{2.749811in}{10.223396in}}%
\pgfpathlineto{\pgfqpoint{2.749811in}{10.135661in}}%
\pgfpathlineto{\pgfqpoint{2.662075in}{10.135661in}}%
\pgfpathlineto{\pgfqpoint{2.662075in}{10.223396in}}%
\pgfusepath{stroke,fill}%
\end{pgfscope}%
\begin{pgfscope}%
\pgfpathrectangle{\pgfqpoint{0.380943in}{9.960189in}}{\pgfqpoint{4.650000in}{0.614151in}}%
\pgfusepath{clip}%
\pgfsetbuttcap%
\pgfsetroundjoin%
\definecolor{currentfill}{rgb}{1.000000,1.000000,0.929412}%
\pgfsetfillcolor{currentfill}%
\pgfsetlinewidth{0.250937pt}%
\definecolor{currentstroke}{rgb}{1.000000,1.000000,1.000000}%
\pgfsetstrokecolor{currentstroke}%
\pgfsetdash{}{0pt}%
\pgfpathmoveto{\pgfqpoint{2.749811in}{10.223396in}}%
\pgfpathlineto{\pgfqpoint{2.837547in}{10.223396in}}%
\pgfpathlineto{\pgfqpoint{2.837547in}{10.135661in}}%
\pgfpathlineto{\pgfqpoint{2.749811in}{10.135661in}}%
\pgfpathlineto{\pgfqpoint{2.749811in}{10.223396in}}%
\pgfusepath{stroke,fill}%
\end{pgfscope}%
\begin{pgfscope}%
\pgfpathrectangle{\pgfqpoint{0.380943in}{9.960189in}}{\pgfqpoint{4.650000in}{0.614151in}}%
\pgfusepath{clip}%
\pgfsetbuttcap%
\pgfsetroundjoin%
\definecolor{currentfill}{rgb}{1.000000,1.000000,0.929412}%
\pgfsetfillcolor{currentfill}%
\pgfsetlinewidth{0.250937pt}%
\definecolor{currentstroke}{rgb}{1.000000,1.000000,1.000000}%
\pgfsetstrokecolor{currentstroke}%
\pgfsetdash{}{0pt}%
\pgfpathmoveto{\pgfqpoint{2.837547in}{10.223396in}}%
\pgfpathlineto{\pgfqpoint{2.925283in}{10.223396in}}%
\pgfpathlineto{\pgfqpoint{2.925283in}{10.135661in}}%
\pgfpathlineto{\pgfqpoint{2.837547in}{10.135661in}}%
\pgfpathlineto{\pgfqpoint{2.837547in}{10.223396in}}%
\pgfusepath{stroke,fill}%
\end{pgfscope}%
\begin{pgfscope}%
\pgfpathrectangle{\pgfqpoint{0.380943in}{9.960189in}}{\pgfqpoint{4.650000in}{0.614151in}}%
\pgfusepath{clip}%
\pgfsetbuttcap%
\pgfsetroundjoin%
\definecolor{currentfill}{rgb}{1.000000,1.000000,0.929412}%
\pgfsetfillcolor{currentfill}%
\pgfsetlinewidth{0.250937pt}%
\definecolor{currentstroke}{rgb}{1.000000,1.000000,1.000000}%
\pgfsetstrokecolor{currentstroke}%
\pgfsetdash{}{0pt}%
\pgfpathmoveto{\pgfqpoint{2.925283in}{10.223396in}}%
\pgfpathlineto{\pgfqpoint{3.013019in}{10.223396in}}%
\pgfpathlineto{\pgfqpoint{3.013019in}{10.135661in}}%
\pgfpathlineto{\pgfqpoint{2.925283in}{10.135661in}}%
\pgfpathlineto{\pgfqpoint{2.925283in}{10.223396in}}%
\pgfusepath{stroke,fill}%
\end{pgfscope}%
\begin{pgfscope}%
\pgfpathrectangle{\pgfqpoint{0.380943in}{9.960189in}}{\pgfqpoint{4.650000in}{0.614151in}}%
\pgfusepath{clip}%
\pgfsetbuttcap%
\pgfsetroundjoin%
\definecolor{currentfill}{rgb}{1.000000,1.000000,0.929412}%
\pgfsetfillcolor{currentfill}%
\pgfsetlinewidth{0.250937pt}%
\definecolor{currentstroke}{rgb}{1.000000,1.000000,1.000000}%
\pgfsetstrokecolor{currentstroke}%
\pgfsetdash{}{0pt}%
\pgfpathmoveto{\pgfqpoint{3.013019in}{10.223396in}}%
\pgfpathlineto{\pgfqpoint{3.100754in}{10.223396in}}%
\pgfpathlineto{\pgfqpoint{3.100754in}{10.135661in}}%
\pgfpathlineto{\pgfqpoint{3.013019in}{10.135661in}}%
\pgfpathlineto{\pgfqpoint{3.013019in}{10.223396in}}%
\pgfusepath{stroke,fill}%
\end{pgfscope}%
\begin{pgfscope}%
\pgfpathrectangle{\pgfqpoint{0.380943in}{9.960189in}}{\pgfqpoint{4.650000in}{0.614151in}}%
\pgfusepath{clip}%
\pgfsetbuttcap%
\pgfsetroundjoin%
\definecolor{currentfill}{rgb}{1.000000,1.000000,0.929412}%
\pgfsetfillcolor{currentfill}%
\pgfsetlinewidth{0.250937pt}%
\definecolor{currentstroke}{rgb}{1.000000,1.000000,1.000000}%
\pgfsetstrokecolor{currentstroke}%
\pgfsetdash{}{0pt}%
\pgfpathmoveto{\pgfqpoint{3.100754in}{10.223396in}}%
\pgfpathlineto{\pgfqpoint{3.188490in}{10.223396in}}%
\pgfpathlineto{\pgfqpoint{3.188490in}{10.135661in}}%
\pgfpathlineto{\pgfqpoint{3.100754in}{10.135661in}}%
\pgfpathlineto{\pgfqpoint{3.100754in}{10.223396in}}%
\pgfusepath{stroke,fill}%
\end{pgfscope}%
\begin{pgfscope}%
\pgfpathrectangle{\pgfqpoint{0.380943in}{9.960189in}}{\pgfqpoint{4.650000in}{0.614151in}}%
\pgfusepath{clip}%
\pgfsetbuttcap%
\pgfsetroundjoin%
\definecolor{currentfill}{rgb}{1.000000,1.000000,0.929412}%
\pgfsetfillcolor{currentfill}%
\pgfsetlinewidth{0.250937pt}%
\definecolor{currentstroke}{rgb}{1.000000,1.000000,1.000000}%
\pgfsetstrokecolor{currentstroke}%
\pgfsetdash{}{0pt}%
\pgfpathmoveto{\pgfqpoint{3.188490in}{10.223396in}}%
\pgfpathlineto{\pgfqpoint{3.276226in}{10.223396in}}%
\pgfpathlineto{\pgfqpoint{3.276226in}{10.135661in}}%
\pgfpathlineto{\pgfqpoint{3.188490in}{10.135661in}}%
\pgfpathlineto{\pgfqpoint{3.188490in}{10.223396in}}%
\pgfusepath{stroke,fill}%
\end{pgfscope}%
\begin{pgfscope}%
\pgfpathrectangle{\pgfqpoint{0.380943in}{9.960189in}}{\pgfqpoint{4.650000in}{0.614151in}}%
\pgfusepath{clip}%
\pgfsetbuttcap%
\pgfsetroundjoin%
\definecolor{currentfill}{rgb}{1.000000,1.000000,0.929412}%
\pgfsetfillcolor{currentfill}%
\pgfsetlinewidth{0.250937pt}%
\definecolor{currentstroke}{rgb}{1.000000,1.000000,1.000000}%
\pgfsetstrokecolor{currentstroke}%
\pgfsetdash{}{0pt}%
\pgfpathmoveto{\pgfqpoint{3.276226in}{10.223396in}}%
\pgfpathlineto{\pgfqpoint{3.363962in}{10.223396in}}%
\pgfpathlineto{\pgfqpoint{3.363962in}{10.135661in}}%
\pgfpathlineto{\pgfqpoint{3.276226in}{10.135661in}}%
\pgfpathlineto{\pgfqpoint{3.276226in}{10.223396in}}%
\pgfusepath{stroke,fill}%
\end{pgfscope}%
\begin{pgfscope}%
\pgfpathrectangle{\pgfqpoint{0.380943in}{9.960189in}}{\pgfqpoint{4.650000in}{0.614151in}}%
\pgfusepath{clip}%
\pgfsetbuttcap%
\pgfsetroundjoin%
\definecolor{currentfill}{rgb}{1.000000,1.000000,0.929412}%
\pgfsetfillcolor{currentfill}%
\pgfsetlinewidth{0.250937pt}%
\definecolor{currentstroke}{rgb}{1.000000,1.000000,1.000000}%
\pgfsetstrokecolor{currentstroke}%
\pgfsetdash{}{0pt}%
\pgfpathmoveto{\pgfqpoint{3.363962in}{10.223396in}}%
\pgfpathlineto{\pgfqpoint{3.451698in}{10.223396in}}%
\pgfpathlineto{\pgfqpoint{3.451698in}{10.135661in}}%
\pgfpathlineto{\pgfqpoint{3.363962in}{10.135661in}}%
\pgfpathlineto{\pgfqpoint{3.363962in}{10.223396in}}%
\pgfusepath{stroke,fill}%
\end{pgfscope}%
\begin{pgfscope}%
\pgfpathrectangle{\pgfqpoint{0.380943in}{9.960189in}}{\pgfqpoint{4.650000in}{0.614151in}}%
\pgfusepath{clip}%
\pgfsetbuttcap%
\pgfsetroundjoin%
\definecolor{currentfill}{rgb}{1.000000,1.000000,0.929412}%
\pgfsetfillcolor{currentfill}%
\pgfsetlinewidth{0.250937pt}%
\definecolor{currentstroke}{rgb}{1.000000,1.000000,1.000000}%
\pgfsetstrokecolor{currentstroke}%
\pgfsetdash{}{0pt}%
\pgfpathmoveto{\pgfqpoint{3.451698in}{10.223396in}}%
\pgfpathlineto{\pgfqpoint{3.539434in}{10.223396in}}%
\pgfpathlineto{\pgfqpoint{3.539434in}{10.135661in}}%
\pgfpathlineto{\pgfqpoint{3.451698in}{10.135661in}}%
\pgfpathlineto{\pgfqpoint{3.451698in}{10.223396in}}%
\pgfusepath{stroke,fill}%
\end{pgfscope}%
\begin{pgfscope}%
\pgfpathrectangle{\pgfqpoint{0.380943in}{9.960189in}}{\pgfqpoint{4.650000in}{0.614151in}}%
\pgfusepath{clip}%
\pgfsetbuttcap%
\pgfsetroundjoin%
\definecolor{currentfill}{rgb}{1.000000,1.000000,0.929412}%
\pgfsetfillcolor{currentfill}%
\pgfsetlinewidth{0.250937pt}%
\definecolor{currentstroke}{rgb}{1.000000,1.000000,1.000000}%
\pgfsetstrokecolor{currentstroke}%
\pgfsetdash{}{0pt}%
\pgfpathmoveto{\pgfqpoint{3.539434in}{10.223396in}}%
\pgfpathlineto{\pgfqpoint{3.627169in}{10.223396in}}%
\pgfpathlineto{\pgfqpoint{3.627169in}{10.135661in}}%
\pgfpathlineto{\pgfqpoint{3.539434in}{10.135661in}}%
\pgfpathlineto{\pgfqpoint{3.539434in}{10.223396in}}%
\pgfusepath{stroke,fill}%
\end{pgfscope}%
\begin{pgfscope}%
\pgfpathrectangle{\pgfqpoint{0.380943in}{9.960189in}}{\pgfqpoint{4.650000in}{0.614151in}}%
\pgfusepath{clip}%
\pgfsetbuttcap%
\pgfsetroundjoin%
\definecolor{currentfill}{rgb}{1.000000,1.000000,0.929412}%
\pgfsetfillcolor{currentfill}%
\pgfsetlinewidth{0.250937pt}%
\definecolor{currentstroke}{rgb}{1.000000,1.000000,1.000000}%
\pgfsetstrokecolor{currentstroke}%
\pgfsetdash{}{0pt}%
\pgfpathmoveto{\pgfqpoint{3.627169in}{10.223396in}}%
\pgfpathlineto{\pgfqpoint{3.714905in}{10.223396in}}%
\pgfpathlineto{\pgfqpoint{3.714905in}{10.135661in}}%
\pgfpathlineto{\pgfqpoint{3.627169in}{10.135661in}}%
\pgfpathlineto{\pgfqpoint{3.627169in}{10.223396in}}%
\pgfusepath{stroke,fill}%
\end{pgfscope}%
\begin{pgfscope}%
\pgfpathrectangle{\pgfqpoint{0.380943in}{9.960189in}}{\pgfqpoint{4.650000in}{0.614151in}}%
\pgfusepath{clip}%
\pgfsetbuttcap%
\pgfsetroundjoin%
\definecolor{currentfill}{rgb}{1.000000,1.000000,0.929412}%
\pgfsetfillcolor{currentfill}%
\pgfsetlinewidth{0.250937pt}%
\definecolor{currentstroke}{rgb}{1.000000,1.000000,1.000000}%
\pgfsetstrokecolor{currentstroke}%
\pgfsetdash{}{0pt}%
\pgfpathmoveto{\pgfqpoint{3.714905in}{10.223396in}}%
\pgfpathlineto{\pgfqpoint{3.802641in}{10.223396in}}%
\pgfpathlineto{\pgfqpoint{3.802641in}{10.135661in}}%
\pgfpathlineto{\pgfqpoint{3.714905in}{10.135661in}}%
\pgfpathlineto{\pgfqpoint{3.714905in}{10.223396in}}%
\pgfusepath{stroke,fill}%
\end{pgfscope}%
\begin{pgfscope}%
\pgfpathrectangle{\pgfqpoint{0.380943in}{9.960189in}}{\pgfqpoint{4.650000in}{0.614151in}}%
\pgfusepath{clip}%
\pgfsetbuttcap%
\pgfsetroundjoin%
\definecolor{currentfill}{rgb}{1.000000,1.000000,0.929412}%
\pgfsetfillcolor{currentfill}%
\pgfsetlinewidth{0.250937pt}%
\definecolor{currentstroke}{rgb}{1.000000,1.000000,1.000000}%
\pgfsetstrokecolor{currentstroke}%
\pgfsetdash{}{0pt}%
\pgfpathmoveto{\pgfqpoint{3.802641in}{10.223396in}}%
\pgfpathlineto{\pgfqpoint{3.890377in}{10.223396in}}%
\pgfpathlineto{\pgfqpoint{3.890377in}{10.135661in}}%
\pgfpathlineto{\pgfqpoint{3.802641in}{10.135661in}}%
\pgfpathlineto{\pgfqpoint{3.802641in}{10.223396in}}%
\pgfusepath{stroke,fill}%
\end{pgfscope}%
\begin{pgfscope}%
\pgfpathrectangle{\pgfqpoint{0.380943in}{9.960189in}}{\pgfqpoint{4.650000in}{0.614151in}}%
\pgfusepath{clip}%
\pgfsetbuttcap%
\pgfsetroundjoin%
\definecolor{currentfill}{rgb}{1.000000,1.000000,0.929412}%
\pgfsetfillcolor{currentfill}%
\pgfsetlinewidth{0.250937pt}%
\definecolor{currentstroke}{rgb}{1.000000,1.000000,1.000000}%
\pgfsetstrokecolor{currentstroke}%
\pgfsetdash{}{0pt}%
\pgfpathmoveto{\pgfqpoint{3.890377in}{10.223396in}}%
\pgfpathlineto{\pgfqpoint{3.978113in}{10.223396in}}%
\pgfpathlineto{\pgfqpoint{3.978113in}{10.135661in}}%
\pgfpathlineto{\pgfqpoint{3.890377in}{10.135661in}}%
\pgfpathlineto{\pgfqpoint{3.890377in}{10.223396in}}%
\pgfusepath{stroke,fill}%
\end{pgfscope}%
\begin{pgfscope}%
\pgfpathrectangle{\pgfqpoint{0.380943in}{9.960189in}}{\pgfqpoint{4.650000in}{0.614151in}}%
\pgfusepath{clip}%
\pgfsetbuttcap%
\pgfsetroundjoin%
\definecolor{currentfill}{rgb}{1.000000,1.000000,0.929412}%
\pgfsetfillcolor{currentfill}%
\pgfsetlinewidth{0.250937pt}%
\definecolor{currentstroke}{rgb}{1.000000,1.000000,1.000000}%
\pgfsetstrokecolor{currentstroke}%
\pgfsetdash{}{0pt}%
\pgfpathmoveto{\pgfqpoint{3.978113in}{10.223396in}}%
\pgfpathlineto{\pgfqpoint{4.065849in}{10.223396in}}%
\pgfpathlineto{\pgfqpoint{4.065849in}{10.135661in}}%
\pgfpathlineto{\pgfqpoint{3.978113in}{10.135661in}}%
\pgfpathlineto{\pgfqpoint{3.978113in}{10.223396in}}%
\pgfusepath{stroke,fill}%
\end{pgfscope}%
\begin{pgfscope}%
\pgfpathrectangle{\pgfqpoint{0.380943in}{9.960189in}}{\pgfqpoint{4.650000in}{0.614151in}}%
\pgfusepath{clip}%
\pgfsetbuttcap%
\pgfsetroundjoin%
\definecolor{currentfill}{rgb}{1.000000,1.000000,0.929412}%
\pgfsetfillcolor{currentfill}%
\pgfsetlinewidth{0.250937pt}%
\definecolor{currentstroke}{rgb}{1.000000,1.000000,1.000000}%
\pgfsetstrokecolor{currentstroke}%
\pgfsetdash{}{0pt}%
\pgfpathmoveto{\pgfqpoint{4.065849in}{10.223396in}}%
\pgfpathlineto{\pgfqpoint{4.153585in}{10.223396in}}%
\pgfpathlineto{\pgfqpoint{4.153585in}{10.135661in}}%
\pgfpathlineto{\pgfqpoint{4.065849in}{10.135661in}}%
\pgfpathlineto{\pgfqpoint{4.065849in}{10.223396in}}%
\pgfusepath{stroke,fill}%
\end{pgfscope}%
\begin{pgfscope}%
\pgfpathrectangle{\pgfqpoint{0.380943in}{9.960189in}}{\pgfqpoint{4.650000in}{0.614151in}}%
\pgfusepath{clip}%
\pgfsetbuttcap%
\pgfsetroundjoin%
\definecolor{currentfill}{rgb}{1.000000,1.000000,0.929412}%
\pgfsetfillcolor{currentfill}%
\pgfsetlinewidth{0.250937pt}%
\definecolor{currentstroke}{rgb}{1.000000,1.000000,1.000000}%
\pgfsetstrokecolor{currentstroke}%
\pgfsetdash{}{0pt}%
\pgfpathmoveto{\pgfqpoint{4.153585in}{10.223396in}}%
\pgfpathlineto{\pgfqpoint{4.241320in}{10.223396in}}%
\pgfpathlineto{\pgfqpoint{4.241320in}{10.135661in}}%
\pgfpathlineto{\pgfqpoint{4.153585in}{10.135661in}}%
\pgfpathlineto{\pgfqpoint{4.153585in}{10.223396in}}%
\pgfusepath{stroke,fill}%
\end{pgfscope}%
\begin{pgfscope}%
\pgfpathrectangle{\pgfqpoint{0.380943in}{9.960189in}}{\pgfqpoint{4.650000in}{0.614151in}}%
\pgfusepath{clip}%
\pgfsetbuttcap%
\pgfsetroundjoin%
\definecolor{currentfill}{rgb}{1.000000,1.000000,0.929412}%
\pgfsetfillcolor{currentfill}%
\pgfsetlinewidth{0.250937pt}%
\definecolor{currentstroke}{rgb}{1.000000,1.000000,1.000000}%
\pgfsetstrokecolor{currentstroke}%
\pgfsetdash{}{0pt}%
\pgfpathmoveto{\pgfqpoint{4.241320in}{10.223396in}}%
\pgfpathlineto{\pgfqpoint{4.329056in}{10.223396in}}%
\pgfpathlineto{\pgfqpoint{4.329056in}{10.135661in}}%
\pgfpathlineto{\pgfqpoint{4.241320in}{10.135661in}}%
\pgfpathlineto{\pgfqpoint{4.241320in}{10.223396in}}%
\pgfusepath{stroke,fill}%
\end{pgfscope}%
\begin{pgfscope}%
\pgfpathrectangle{\pgfqpoint{0.380943in}{9.960189in}}{\pgfqpoint{4.650000in}{0.614151in}}%
\pgfusepath{clip}%
\pgfsetbuttcap%
\pgfsetroundjoin%
\definecolor{currentfill}{rgb}{1.000000,1.000000,0.929412}%
\pgfsetfillcolor{currentfill}%
\pgfsetlinewidth{0.250937pt}%
\definecolor{currentstroke}{rgb}{1.000000,1.000000,1.000000}%
\pgfsetstrokecolor{currentstroke}%
\pgfsetdash{}{0pt}%
\pgfpathmoveto{\pgfqpoint{4.329056in}{10.223396in}}%
\pgfpathlineto{\pgfqpoint{4.416792in}{10.223396in}}%
\pgfpathlineto{\pgfqpoint{4.416792in}{10.135661in}}%
\pgfpathlineto{\pgfqpoint{4.329056in}{10.135661in}}%
\pgfpathlineto{\pgfqpoint{4.329056in}{10.223396in}}%
\pgfusepath{stroke,fill}%
\end{pgfscope}%
\begin{pgfscope}%
\pgfpathrectangle{\pgfqpoint{0.380943in}{9.960189in}}{\pgfqpoint{4.650000in}{0.614151in}}%
\pgfusepath{clip}%
\pgfsetbuttcap%
\pgfsetroundjoin%
\definecolor{currentfill}{rgb}{0.963091,0.919493,0.720185}%
\pgfsetfillcolor{currentfill}%
\pgfsetlinewidth{0.250937pt}%
\definecolor{currentstroke}{rgb}{1.000000,1.000000,1.000000}%
\pgfsetstrokecolor{currentstroke}%
\pgfsetdash{}{0pt}%
\pgfpathmoveto{\pgfqpoint{4.416792in}{10.223396in}}%
\pgfpathlineto{\pgfqpoint{4.504528in}{10.223396in}}%
\pgfpathlineto{\pgfqpoint{4.504528in}{10.135661in}}%
\pgfpathlineto{\pgfqpoint{4.416792in}{10.135661in}}%
\pgfpathlineto{\pgfqpoint{4.416792in}{10.223396in}}%
\pgfusepath{stroke,fill}%
\end{pgfscope}%
\begin{pgfscope}%
\pgfpathrectangle{\pgfqpoint{0.380943in}{9.960189in}}{\pgfqpoint{4.650000in}{0.614151in}}%
\pgfusepath{clip}%
\pgfsetbuttcap%
\pgfsetroundjoin%
\definecolor{currentfill}{rgb}{0.997924,0.685352,0.570242}%
\pgfsetfillcolor{currentfill}%
\pgfsetlinewidth{0.250937pt}%
\definecolor{currentstroke}{rgb}{1.000000,1.000000,1.000000}%
\pgfsetstrokecolor{currentstroke}%
\pgfsetdash{}{0pt}%
\pgfpathmoveto{\pgfqpoint{4.504528in}{10.223396in}}%
\pgfpathlineto{\pgfqpoint{4.592264in}{10.223396in}}%
\pgfpathlineto{\pgfqpoint{4.592264in}{10.135661in}}%
\pgfpathlineto{\pgfqpoint{4.504528in}{10.135661in}}%
\pgfpathlineto{\pgfqpoint{4.504528in}{10.223396in}}%
\pgfusepath{stroke,fill}%
\end{pgfscope}%
\begin{pgfscope}%
\pgfpathrectangle{\pgfqpoint{0.380943in}{9.960189in}}{\pgfqpoint{4.650000in}{0.614151in}}%
\pgfusepath{clip}%
\pgfsetbuttcap%
\pgfsetroundjoin%
\definecolor{currentfill}{rgb}{0.975087,0.857901,0.686044}%
\pgfsetfillcolor{currentfill}%
\pgfsetlinewidth{0.250937pt}%
\definecolor{currentstroke}{rgb}{1.000000,1.000000,1.000000}%
\pgfsetstrokecolor{currentstroke}%
\pgfsetdash{}{0pt}%
\pgfpathmoveto{\pgfqpoint{4.592264in}{10.223396in}}%
\pgfpathlineto{\pgfqpoint{4.680000in}{10.223396in}}%
\pgfpathlineto{\pgfqpoint{4.680000in}{10.135661in}}%
\pgfpathlineto{\pgfqpoint{4.592264in}{10.135661in}}%
\pgfpathlineto{\pgfqpoint{4.592264in}{10.223396in}}%
\pgfusepath{stroke,fill}%
\end{pgfscope}%
\begin{pgfscope}%
\pgfpathrectangle{\pgfqpoint{0.380943in}{9.960189in}}{\pgfqpoint{4.650000in}{0.614151in}}%
\pgfusepath{clip}%
\pgfsetbuttcap%
\pgfsetroundjoin%
\definecolor{currentfill}{rgb}{0.989619,0.788235,0.628374}%
\pgfsetfillcolor{currentfill}%
\pgfsetlinewidth{0.250937pt}%
\definecolor{currentstroke}{rgb}{1.000000,1.000000,1.000000}%
\pgfsetstrokecolor{currentstroke}%
\pgfsetdash{}{0pt}%
\pgfpathmoveto{\pgfqpoint{4.680000in}{10.223396in}}%
\pgfpathlineto{\pgfqpoint{4.767736in}{10.223396in}}%
\pgfpathlineto{\pgfqpoint{4.767736in}{10.135661in}}%
\pgfpathlineto{\pgfqpoint{4.680000in}{10.135661in}}%
\pgfpathlineto{\pgfqpoint{4.680000in}{10.223396in}}%
\pgfusepath{stroke,fill}%
\end{pgfscope}%
\begin{pgfscope}%
\pgfpathrectangle{\pgfqpoint{0.380943in}{9.960189in}}{\pgfqpoint{4.650000in}{0.614151in}}%
\pgfusepath{clip}%
\pgfsetbuttcap%
\pgfsetroundjoin%
\definecolor{currentfill}{rgb}{1.000000,0.522261,0.496886}%
\pgfsetfillcolor{currentfill}%
\pgfsetlinewidth{0.250937pt}%
\definecolor{currentstroke}{rgb}{1.000000,1.000000,1.000000}%
\pgfsetstrokecolor{currentstroke}%
\pgfsetdash{}{0pt}%
\pgfpathmoveto{\pgfqpoint{4.767736in}{10.223396in}}%
\pgfpathlineto{\pgfqpoint{4.855471in}{10.223396in}}%
\pgfpathlineto{\pgfqpoint{4.855471in}{10.135661in}}%
\pgfpathlineto{\pgfqpoint{4.767736in}{10.135661in}}%
\pgfpathlineto{\pgfqpoint{4.767736in}{10.223396in}}%
\pgfusepath{stroke,fill}%
\end{pgfscope}%
\begin{pgfscope}%
\pgfpathrectangle{\pgfqpoint{0.380943in}{9.960189in}}{\pgfqpoint{4.650000in}{0.614151in}}%
\pgfusepath{clip}%
\pgfsetbuttcap%
\pgfsetroundjoin%
\definecolor{currentfill}{rgb}{0.994694,0.745098,0.602999}%
\pgfsetfillcolor{currentfill}%
\pgfsetlinewidth{0.250937pt}%
\definecolor{currentstroke}{rgb}{1.000000,1.000000,1.000000}%
\pgfsetstrokecolor{currentstroke}%
\pgfsetdash{}{0pt}%
\pgfpathmoveto{\pgfqpoint{4.855471in}{10.223396in}}%
\pgfpathlineto{\pgfqpoint{4.943207in}{10.223396in}}%
\pgfpathlineto{\pgfqpoint{4.943207in}{10.135661in}}%
\pgfpathlineto{\pgfqpoint{4.855471in}{10.135661in}}%
\pgfpathlineto{\pgfqpoint{4.855471in}{10.223396in}}%
\pgfusepath{stroke,fill}%
\end{pgfscope}%
\begin{pgfscope}%
\pgfpathrectangle{\pgfqpoint{0.380943in}{9.960189in}}{\pgfqpoint{4.650000in}{0.614151in}}%
\pgfusepath{clip}%
\pgfsetbuttcap%
\pgfsetroundjoin%
\definecolor{currentfill}{rgb}{0.975087,0.857901,0.686044}%
\pgfsetfillcolor{currentfill}%
\pgfsetlinewidth{0.250937pt}%
\definecolor{currentstroke}{rgb}{1.000000,1.000000,1.000000}%
\pgfsetstrokecolor{currentstroke}%
\pgfsetdash{}{0pt}%
\pgfpathmoveto{\pgfqpoint{4.943207in}{10.223396in}}%
\pgfpathlineto{\pgfqpoint{5.030943in}{10.223396in}}%
\pgfpathlineto{\pgfqpoint{5.030943in}{10.135661in}}%
\pgfpathlineto{\pgfqpoint{4.943207in}{10.135661in}}%
\pgfpathlineto{\pgfqpoint{4.943207in}{10.223396in}}%
\pgfusepath{stroke,fill}%
\end{pgfscope}%
\begin{pgfscope}%
\pgfpathrectangle{\pgfqpoint{0.380943in}{9.960189in}}{\pgfqpoint{4.650000in}{0.614151in}}%
\pgfusepath{clip}%
\pgfsetbuttcap%
\pgfsetroundjoin%
\definecolor{currentfill}{rgb}{1.000000,1.000000,0.929412}%
\pgfsetfillcolor{currentfill}%
\pgfsetlinewidth{0.250937pt}%
\definecolor{currentstroke}{rgb}{1.000000,1.000000,1.000000}%
\pgfsetstrokecolor{currentstroke}%
\pgfsetdash{}{0pt}%
\pgfpathmoveto{\pgfqpoint{0.380943in}{10.135661in}}%
\pgfpathlineto{\pgfqpoint{0.468679in}{10.135661in}}%
\pgfpathlineto{\pgfqpoint{0.468679in}{10.047925in}}%
\pgfpathlineto{\pgfqpoint{0.380943in}{10.047925in}}%
\pgfpathlineto{\pgfqpoint{0.380943in}{10.135661in}}%
\pgfusepath{stroke,fill}%
\end{pgfscope}%
\begin{pgfscope}%
\pgfpathrectangle{\pgfqpoint{0.380943in}{9.960189in}}{\pgfqpoint{4.650000in}{0.614151in}}%
\pgfusepath{clip}%
\pgfsetbuttcap%
\pgfsetroundjoin%
\definecolor{currentfill}{rgb}{1.000000,1.000000,0.929412}%
\pgfsetfillcolor{currentfill}%
\pgfsetlinewidth{0.250937pt}%
\definecolor{currentstroke}{rgb}{1.000000,1.000000,1.000000}%
\pgfsetstrokecolor{currentstroke}%
\pgfsetdash{}{0pt}%
\pgfpathmoveto{\pgfqpoint{0.468679in}{10.135661in}}%
\pgfpathlineto{\pgfqpoint{0.556415in}{10.135661in}}%
\pgfpathlineto{\pgfqpoint{0.556415in}{10.047925in}}%
\pgfpathlineto{\pgfqpoint{0.468679in}{10.047925in}}%
\pgfpathlineto{\pgfqpoint{0.468679in}{10.135661in}}%
\pgfusepath{stroke,fill}%
\end{pgfscope}%
\begin{pgfscope}%
\pgfpathrectangle{\pgfqpoint{0.380943in}{9.960189in}}{\pgfqpoint{4.650000in}{0.614151in}}%
\pgfusepath{clip}%
\pgfsetbuttcap%
\pgfsetroundjoin%
\definecolor{currentfill}{rgb}{1.000000,1.000000,0.929412}%
\pgfsetfillcolor{currentfill}%
\pgfsetlinewidth{0.250937pt}%
\definecolor{currentstroke}{rgb}{1.000000,1.000000,1.000000}%
\pgfsetstrokecolor{currentstroke}%
\pgfsetdash{}{0pt}%
\pgfpathmoveto{\pgfqpoint{0.556415in}{10.135661in}}%
\pgfpathlineto{\pgfqpoint{0.644151in}{10.135661in}}%
\pgfpathlineto{\pgfqpoint{0.644151in}{10.047925in}}%
\pgfpathlineto{\pgfqpoint{0.556415in}{10.047925in}}%
\pgfpathlineto{\pgfqpoint{0.556415in}{10.135661in}}%
\pgfusepath{stroke,fill}%
\end{pgfscope}%
\begin{pgfscope}%
\pgfpathrectangle{\pgfqpoint{0.380943in}{9.960189in}}{\pgfqpoint{4.650000in}{0.614151in}}%
\pgfusepath{clip}%
\pgfsetbuttcap%
\pgfsetroundjoin%
\definecolor{currentfill}{rgb}{1.000000,1.000000,0.929412}%
\pgfsetfillcolor{currentfill}%
\pgfsetlinewidth{0.250937pt}%
\definecolor{currentstroke}{rgb}{1.000000,1.000000,1.000000}%
\pgfsetstrokecolor{currentstroke}%
\pgfsetdash{}{0pt}%
\pgfpathmoveto{\pgfqpoint{0.644151in}{10.135661in}}%
\pgfpathlineto{\pgfqpoint{0.731886in}{10.135661in}}%
\pgfpathlineto{\pgfqpoint{0.731886in}{10.047925in}}%
\pgfpathlineto{\pgfqpoint{0.644151in}{10.047925in}}%
\pgfpathlineto{\pgfqpoint{0.644151in}{10.135661in}}%
\pgfusepath{stroke,fill}%
\end{pgfscope}%
\begin{pgfscope}%
\pgfpathrectangle{\pgfqpoint{0.380943in}{9.960189in}}{\pgfqpoint{4.650000in}{0.614151in}}%
\pgfusepath{clip}%
\pgfsetbuttcap%
\pgfsetroundjoin%
\definecolor{currentfill}{rgb}{1.000000,1.000000,0.929412}%
\pgfsetfillcolor{currentfill}%
\pgfsetlinewidth{0.250937pt}%
\definecolor{currentstroke}{rgb}{1.000000,1.000000,1.000000}%
\pgfsetstrokecolor{currentstroke}%
\pgfsetdash{}{0pt}%
\pgfpathmoveto{\pgfqpoint{0.731886in}{10.135661in}}%
\pgfpathlineto{\pgfqpoint{0.819622in}{10.135661in}}%
\pgfpathlineto{\pgfqpoint{0.819622in}{10.047925in}}%
\pgfpathlineto{\pgfqpoint{0.731886in}{10.047925in}}%
\pgfpathlineto{\pgfqpoint{0.731886in}{10.135661in}}%
\pgfusepath{stroke,fill}%
\end{pgfscope}%
\begin{pgfscope}%
\pgfpathrectangle{\pgfqpoint{0.380943in}{9.960189in}}{\pgfqpoint{4.650000in}{0.614151in}}%
\pgfusepath{clip}%
\pgfsetbuttcap%
\pgfsetroundjoin%
\definecolor{currentfill}{rgb}{1.000000,1.000000,0.929412}%
\pgfsetfillcolor{currentfill}%
\pgfsetlinewidth{0.250937pt}%
\definecolor{currentstroke}{rgb}{1.000000,1.000000,1.000000}%
\pgfsetstrokecolor{currentstroke}%
\pgfsetdash{}{0pt}%
\pgfpathmoveto{\pgfqpoint{0.819622in}{10.135661in}}%
\pgfpathlineto{\pgfqpoint{0.907358in}{10.135661in}}%
\pgfpathlineto{\pgfqpoint{0.907358in}{10.047925in}}%
\pgfpathlineto{\pgfqpoint{0.819622in}{10.047925in}}%
\pgfpathlineto{\pgfqpoint{0.819622in}{10.135661in}}%
\pgfusepath{stroke,fill}%
\end{pgfscope}%
\begin{pgfscope}%
\pgfpathrectangle{\pgfqpoint{0.380943in}{9.960189in}}{\pgfqpoint{4.650000in}{0.614151in}}%
\pgfusepath{clip}%
\pgfsetbuttcap%
\pgfsetroundjoin%
\definecolor{currentfill}{rgb}{1.000000,1.000000,0.929412}%
\pgfsetfillcolor{currentfill}%
\pgfsetlinewidth{0.250937pt}%
\definecolor{currentstroke}{rgb}{1.000000,1.000000,1.000000}%
\pgfsetstrokecolor{currentstroke}%
\pgfsetdash{}{0pt}%
\pgfpathmoveto{\pgfqpoint{0.907358in}{10.135661in}}%
\pgfpathlineto{\pgfqpoint{0.995094in}{10.135661in}}%
\pgfpathlineto{\pgfqpoint{0.995094in}{10.047925in}}%
\pgfpathlineto{\pgfqpoint{0.907358in}{10.047925in}}%
\pgfpathlineto{\pgfqpoint{0.907358in}{10.135661in}}%
\pgfusepath{stroke,fill}%
\end{pgfscope}%
\begin{pgfscope}%
\pgfpathrectangle{\pgfqpoint{0.380943in}{9.960189in}}{\pgfqpoint{4.650000in}{0.614151in}}%
\pgfusepath{clip}%
\pgfsetbuttcap%
\pgfsetroundjoin%
\definecolor{currentfill}{rgb}{1.000000,1.000000,0.929412}%
\pgfsetfillcolor{currentfill}%
\pgfsetlinewidth{0.250937pt}%
\definecolor{currentstroke}{rgb}{1.000000,1.000000,1.000000}%
\pgfsetstrokecolor{currentstroke}%
\pgfsetdash{}{0pt}%
\pgfpathmoveto{\pgfqpoint{0.995094in}{10.135661in}}%
\pgfpathlineto{\pgfqpoint{1.082830in}{10.135661in}}%
\pgfpathlineto{\pgfqpoint{1.082830in}{10.047925in}}%
\pgfpathlineto{\pgfqpoint{0.995094in}{10.047925in}}%
\pgfpathlineto{\pgfqpoint{0.995094in}{10.135661in}}%
\pgfusepath{stroke,fill}%
\end{pgfscope}%
\begin{pgfscope}%
\pgfpathrectangle{\pgfqpoint{0.380943in}{9.960189in}}{\pgfqpoint{4.650000in}{0.614151in}}%
\pgfusepath{clip}%
\pgfsetbuttcap%
\pgfsetroundjoin%
\definecolor{currentfill}{rgb}{1.000000,1.000000,0.929412}%
\pgfsetfillcolor{currentfill}%
\pgfsetlinewidth{0.250937pt}%
\definecolor{currentstroke}{rgb}{1.000000,1.000000,1.000000}%
\pgfsetstrokecolor{currentstroke}%
\pgfsetdash{}{0pt}%
\pgfpathmoveto{\pgfqpoint{1.082830in}{10.135661in}}%
\pgfpathlineto{\pgfqpoint{1.170566in}{10.135661in}}%
\pgfpathlineto{\pgfqpoint{1.170566in}{10.047925in}}%
\pgfpathlineto{\pgfqpoint{1.082830in}{10.047925in}}%
\pgfpathlineto{\pgfqpoint{1.082830in}{10.135661in}}%
\pgfusepath{stroke,fill}%
\end{pgfscope}%
\begin{pgfscope}%
\pgfpathrectangle{\pgfqpoint{0.380943in}{9.960189in}}{\pgfqpoint{4.650000in}{0.614151in}}%
\pgfusepath{clip}%
\pgfsetbuttcap%
\pgfsetroundjoin%
\definecolor{currentfill}{rgb}{1.000000,1.000000,0.929412}%
\pgfsetfillcolor{currentfill}%
\pgfsetlinewidth{0.250937pt}%
\definecolor{currentstroke}{rgb}{1.000000,1.000000,1.000000}%
\pgfsetstrokecolor{currentstroke}%
\pgfsetdash{}{0pt}%
\pgfpathmoveto{\pgfqpoint{1.170566in}{10.135661in}}%
\pgfpathlineto{\pgfqpoint{1.258302in}{10.135661in}}%
\pgfpathlineto{\pgfqpoint{1.258302in}{10.047925in}}%
\pgfpathlineto{\pgfqpoint{1.170566in}{10.047925in}}%
\pgfpathlineto{\pgfqpoint{1.170566in}{10.135661in}}%
\pgfusepath{stroke,fill}%
\end{pgfscope}%
\begin{pgfscope}%
\pgfpathrectangle{\pgfqpoint{0.380943in}{9.960189in}}{\pgfqpoint{4.650000in}{0.614151in}}%
\pgfusepath{clip}%
\pgfsetbuttcap%
\pgfsetroundjoin%
\definecolor{currentfill}{rgb}{1.000000,1.000000,0.929412}%
\pgfsetfillcolor{currentfill}%
\pgfsetlinewidth{0.250937pt}%
\definecolor{currentstroke}{rgb}{1.000000,1.000000,1.000000}%
\pgfsetstrokecolor{currentstroke}%
\pgfsetdash{}{0pt}%
\pgfpathmoveto{\pgfqpoint{1.258302in}{10.135661in}}%
\pgfpathlineto{\pgfqpoint{1.346037in}{10.135661in}}%
\pgfpathlineto{\pgfqpoint{1.346037in}{10.047925in}}%
\pgfpathlineto{\pgfqpoint{1.258302in}{10.047925in}}%
\pgfpathlineto{\pgfqpoint{1.258302in}{10.135661in}}%
\pgfusepath{stroke,fill}%
\end{pgfscope}%
\begin{pgfscope}%
\pgfpathrectangle{\pgfqpoint{0.380943in}{9.960189in}}{\pgfqpoint{4.650000in}{0.614151in}}%
\pgfusepath{clip}%
\pgfsetbuttcap%
\pgfsetroundjoin%
\definecolor{currentfill}{rgb}{1.000000,1.000000,0.929412}%
\pgfsetfillcolor{currentfill}%
\pgfsetlinewidth{0.250937pt}%
\definecolor{currentstroke}{rgb}{1.000000,1.000000,1.000000}%
\pgfsetstrokecolor{currentstroke}%
\pgfsetdash{}{0pt}%
\pgfpathmoveto{\pgfqpoint{1.346037in}{10.135661in}}%
\pgfpathlineto{\pgfqpoint{1.433773in}{10.135661in}}%
\pgfpathlineto{\pgfqpoint{1.433773in}{10.047925in}}%
\pgfpathlineto{\pgfqpoint{1.346037in}{10.047925in}}%
\pgfpathlineto{\pgfqpoint{1.346037in}{10.135661in}}%
\pgfusepath{stroke,fill}%
\end{pgfscope}%
\begin{pgfscope}%
\pgfpathrectangle{\pgfqpoint{0.380943in}{9.960189in}}{\pgfqpoint{4.650000in}{0.614151in}}%
\pgfusepath{clip}%
\pgfsetbuttcap%
\pgfsetroundjoin%
\definecolor{currentfill}{rgb}{1.000000,1.000000,0.929412}%
\pgfsetfillcolor{currentfill}%
\pgfsetlinewidth{0.250937pt}%
\definecolor{currentstroke}{rgb}{1.000000,1.000000,1.000000}%
\pgfsetstrokecolor{currentstroke}%
\pgfsetdash{}{0pt}%
\pgfpathmoveto{\pgfqpoint{1.433773in}{10.135661in}}%
\pgfpathlineto{\pgfqpoint{1.521509in}{10.135661in}}%
\pgfpathlineto{\pgfqpoint{1.521509in}{10.047925in}}%
\pgfpathlineto{\pgfqpoint{1.433773in}{10.047925in}}%
\pgfpathlineto{\pgfqpoint{1.433773in}{10.135661in}}%
\pgfusepath{stroke,fill}%
\end{pgfscope}%
\begin{pgfscope}%
\pgfpathrectangle{\pgfqpoint{0.380943in}{9.960189in}}{\pgfqpoint{4.650000in}{0.614151in}}%
\pgfusepath{clip}%
\pgfsetbuttcap%
\pgfsetroundjoin%
\definecolor{currentfill}{rgb}{1.000000,1.000000,0.929412}%
\pgfsetfillcolor{currentfill}%
\pgfsetlinewidth{0.250937pt}%
\definecolor{currentstroke}{rgb}{1.000000,1.000000,1.000000}%
\pgfsetstrokecolor{currentstroke}%
\pgfsetdash{}{0pt}%
\pgfpathmoveto{\pgfqpoint{1.521509in}{10.135661in}}%
\pgfpathlineto{\pgfqpoint{1.609245in}{10.135661in}}%
\pgfpathlineto{\pgfqpoint{1.609245in}{10.047925in}}%
\pgfpathlineto{\pgfqpoint{1.521509in}{10.047925in}}%
\pgfpathlineto{\pgfqpoint{1.521509in}{10.135661in}}%
\pgfusepath{stroke,fill}%
\end{pgfscope}%
\begin{pgfscope}%
\pgfpathrectangle{\pgfqpoint{0.380943in}{9.960189in}}{\pgfqpoint{4.650000in}{0.614151in}}%
\pgfusepath{clip}%
\pgfsetbuttcap%
\pgfsetroundjoin%
\definecolor{currentfill}{rgb}{1.000000,1.000000,0.929412}%
\pgfsetfillcolor{currentfill}%
\pgfsetlinewidth{0.250937pt}%
\definecolor{currentstroke}{rgb}{1.000000,1.000000,1.000000}%
\pgfsetstrokecolor{currentstroke}%
\pgfsetdash{}{0pt}%
\pgfpathmoveto{\pgfqpoint{1.609245in}{10.135661in}}%
\pgfpathlineto{\pgfqpoint{1.696981in}{10.135661in}}%
\pgfpathlineto{\pgfqpoint{1.696981in}{10.047925in}}%
\pgfpathlineto{\pgfqpoint{1.609245in}{10.047925in}}%
\pgfpathlineto{\pgfqpoint{1.609245in}{10.135661in}}%
\pgfusepath{stroke,fill}%
\end{pgfscope}%
\begin{pgfscope}%
\pgfpathrectangle{\pgfqpoint{0.380943in}{9.960189in}}{\pgfqpoint{4.650000in}{0.614151in}}%
\pgfusepath{clip}%
\pgfsetbuttcap%
\pgfsetroundjoin%
\definecolor{currentfill}{rgb}{1.000000,1.000000,0.929412}%
\pgfsetfillcolor{currentfill}%
\pgfsetlinewidth{0.250937pt}%
\definecolor{currentstroke}{rgb}{1.000000,1.000000,1.000000}%
\pgfsetstrokecolor{currentstroke}%
\pgfsetdash{}{0pt}%
\pgfpathmoveto{\pgfqpoint{1.696981in}{10.135661in}}%
\pgfpathlineto{\pgfqpoint{1.784717in}{10.135661in}}%
\pgfpathlineto{\pgfqpoint{1.784717in}{10.047925in}}%
\pgfpathlineto{\pgfqpoint{1.696981in}{10.047925in}}%
\pgfpathlineto{\pgfqpoint{1.696981in}{10.135661in}}%
\pgfusepath{stroke,fill}%
\end{pgfscope}%
\begin{pgfscope}%
\pgfpathrectangle{\pgfqpoint{0.380943in}{9.960189in}}{\pgfqpoint{4.650000in}{0.614151in}}%
\pgfusepath{clip}%
\pgfsetbuttcap%
\pgfsetroundjoin%
\definecolor{currentfill}{rgb}{1.000000,1.000000,0.929412}%
\pgfsetfillcolor{currentfill}%
\pgfsetlinewidth{0.250937pt}%
\definecolor{currentstroke}{rgb}{1.000000,1.000000,1.000000}%
\pgfsetstrokecolor{currentstroke}%
\pgfsetdash{}{0pt}%
\pgfpathmoveto{\pgfqpoint{1.784717in}{10.135661in}}%
\pgfpathlineto{\pgfqpoint{1.872452in}{10.135661in}}%
\pgfpathlineto{\pgfqpoint{1.872452in}{10.047925in}}%
\pgfpathlineto{\pgfqpoint{1.784717in}{10.047925in}}%
\pgfpathlineto{\pgfqpoint{1.784717in}{10.135661in}}%
\pgfusepath{stroke,fill}%
\end{pgfscope}%
\begin{pgfscope}%
\pgfpathrectangle{\pgfqpoint{0.380943in}{9.960189in}}{\pgfqpoint{4.650000in}{0.614151in}}%
\pgfusepath{clip}%
\pgfsetbuttcap%
\pgfsetroundjoin%
\definecolor{currentfill}{rgb}{1.000000,1.000000,0.929412}%
\pgfsetfillcolor{currentfill}%
\pgfsetlinewidth{0.250937pt}%
\definecolor{currentstroke}{rgb}{1.000000,1.000000,1.000000}%
\pgfsetstrokecolor{currentstroke}%
\pgfsetdash{}{0pt}%
\pgfpathmoveto{\pgfqpoint{1.872452in}{10.135661in}}%
\pgfpathlineto{\pgfqpoint{1.960188in}{10.135661in}}%
\pgfpathlineto{\pgfqpoint{1.960188in}{10.047925in}}%
\pgfpathlineto{\pgfqpoint{1.872452in}{10.047925in}}%
\pgfpathlineto{\pgfqpoint{1.872452in}{10.135661in}}%
\pgfusepath{stroke,fill}%
\end{pgfscope}%
\begin{pgfscope}%
\pgfpathrectangle{\pgfqpoint{0.380943in}{9.960189in}}{\pgfqpoint{4.650000in}{0.614151in}}%
\pgfusepath{clip}%
\pgfsetbuttcap%
\pgfsetroundjoin%
\definecolor{currentfill}{rgb}{1.000000,1.000000,0.929412}%
\pgfsetfillcolor{currentfill}%
\pgfsetlinewidth{0.250937pt}%
\definecolor{currentstroke}{rgb}{1.000000,1.000000,1.000000}%
\pgfsetstrokecolor{currentstroke}%
\pgfsetdash{}{0pt}%
\pgfpathmoveto{\pgfqpoint{1.960188in}{10.135661in}}%
\pgfpathlineto{\pgfqpoint{2.047924in}{10.135661in}}%
\pgfpathlineto{\pgfqpoint{2.047924in}{10.047925in}}%
\pgfpathlineto{\pgfqpoint{1.960188in}{10.047925in}}%
\pgfpathlineto{\pgfqpoint{1.960188in}{10.135661in}}%
\pgfusepath{stroke,fill}%
\end{pgfscope}%
\begin{pgfscope}%
\pgfpathrectangle{\pgfqpoint{0.380943in}{9.960189in}}{\pgfqpoint{4.650000in}{0.614151in}}%
\pgfusepath{clip}%
\pgfsetbuttcap%
\pgfsetroundjoin%
\definecolor{currentfill}{rgb}{1.000000,1.000000,0.929412}%
\pgfsetfillcolor{currentfill}%
\pgfsetlinewidth{0.250937pt}%
\definecolor{currentstroke}{rgb}{1.000000,1.000000,1.000000}%
\pgfsetstrokecolor{currentstroke}%
\pgfsetdash{}{0pt}%
\pgfpathmoveto{\pgfqpoint{2.047924in}{10.135661in}}%
\pgfpathlineto{\pgfqpoint{2.135660in}{10.135661in}}%
\pgfpathlineto{\pgfqpoint{2.135660in}{10.047925in}}%
\pgfpathlineto{\pgfqpoint{2.047924in}{10.047925in}}%
\pgfpathlineto{\pgfqpoint{2.047924in}{10.135661in}}%
\pgfusepath{stroke,fill}%
\end{pgfscope}%
\begin{pgfscope}%
\pgfpathrectangle{\pgfqpoint{0.380943in}{9.960189in}}{\pgfqpoint{4.650000in}{0.614151in}}%
\pgfusepath{clip}%
\pgfsetbuttcap%
\pgfsetroundjoin%
\definecolor{currentfill}{rgb}{1.000000,1.000000,0.929412}%
\pgfsetfillcolor{currentfill}%
\pgfsetlinewidth{0.250937pt}%
\definecolor{currentstroke}{rgb}{1.000000,1.000000,1.000000}%
\pgfsetstrokecolor{currentstroke}%
\pgfsetdash{}{0pt}%
\pgfpathmoveto{\pgfqpoint{2.135660in}{10.135661in}}%
\pgfpathlineto{\pgfqpoint{2.223396in}{10.135661in}}%
\pgfpathlineto{\pgfqpoint{2.223396in}{10.047925in}}%
\pgfpathlineto{\pgfqpoint{2.135660in}{10.047925in}}%
\pgfpathlineto{\pgfqpoint{2.135660in}{10.135661in}}%
\pgfusepath{stroke,fill}%
\end{pgfscope}%
\begin{pgfscope}%
\pgfpathrectangle{\pgfqpoint{0.380943in}{9.960189in}}{\pgfqpoint{4.650000in}{0.614151in}}%
\pgfusepath{clip}%
\pgfsetbuttcap%
\pgfsetroundjoin%
\definecolor{currentfill}{rgb}{1.000000,1.000000,0.929412}%
\pgfsetfillcolor{currentfill}%
\pgfsetlinewidth{0.250937pt}%
\definecolor{currentstroke}{rgb}{1.000000,1.000000,1.000000}%
\pgfsetstrokecolor{currentstroke}%
\pgfsetdash{}{0pt}%
\pgfpathmoveto{\pgfqpoint{2.223396in}{10.135661in}}%
\pgfpathlineto{\pgfqpoint{2.311132in}{10.135661in}}%
\pgfpathlineto{\pgfqpoint{2.311132in}{10.047925in}}%
\pgfpathlineto{\pgfqpoint{2.223396in}{10.047925in}}%
\pgfpathlineto{\pgfqpoint{2.223396in}{10.135661in}}%
\pgfusepath{stroke,fill}%
\end{pgfscope}%
\begin{pgfscope}%
\pgfpathrectangle{\pgfqpoint{0.380943in}{9.960189in}}{\pgfqpoint{4.650000in}{0.614151in}}%
\pgfusepath{clip}%
\pgfsetbuttcap%
\pgfsetroundjoin%
\definecolor{currentfill}{rgb}{1.000000,1.000000,0.929412}%
\pgfsetfillcolor{currentfill}%
\pgfsetlinewidth{0.250937pt}%
\definecolor{currentstroke}{rgb}{1.000000,1.000000,1.000000}%
\pgfsetstrokecolor{currentstroke}%
\pgfsetdash{}{0pt}%
\pgfpathmoveto{\pgfqpoint{2.311132in}{10.135661in}}%
\pgfpathlineto{\pgfqpoint{2.398868in}{10.135661in}}%
\pgfpathlineto{\pgfqpoint{2.398868in}{10.047925in}}%
\pgfpathlineto{\pgfqpoint{2.311132in}{10.047925in}}%
\pgfpathlineto{\pgfqpoint{2.311132in}{10.135661in}}%
\pgfusepath{stroke,fill}%
\end{pgfscope}%
\begin{pgfscope}%
\pgfpathrectangle{\pgfqpoint{0.380943in}{9.960189in}}{\pgfqpoint{4.650000in}{0.614151in}}%
\pgfusepath{clip}%
\pgfsetbuttcap%
\pgfsetroundjoin%
\definecolor{currentfill}{rgb}{1.000000,1.000000,0.929412}%
\pgfsetfillcolor{currentfill}%
\pgfsetlinewidth{0.250937pt}%
\definecolor{currentstroke}{rgb}{1.000000,1.000000,1.000000}%
\pgfsetstrokecolor{currentstroke}%
\pgfsetdash{}{0pt}%
\pgfpathmoveto{\pgfqpoint{2.398868in}{10.135661in}}%
\pgfpathlineto{\pgfqpoint{2.486603in}{10.135661in}}%
\pgfpathlineto{\pgfqpoint{2.486603in}{10.047925in}}%
\pgfpathlineto{\pgfqpoint{2.398868in}{10.047925in}}%
\pgfpathlineto{\pgfqpoint{2.398868in}{10.135661in}}%
\pgfusepath{stroke,fill}%
\end{pgfscope}%
\begin{pgfscope}%
\pgfpathrectangle{\pgfqpoint{0.380943in}{9.960189in}}{\pgfqpoint{4.650000in}{0.614151in}}%
\pgfusepath{clip}%
\pgfsetbuttcap%
\pgfsetroundjoin%
\definecolor{currentfill}{rgb}{1.000000,1.000000,0.929412}%
\pgfsetfillcolor{currentfill}%
\pgfsetlinewidth{0.250937pt}%
\definecolor{currentstroke}{rgb}{1.000000,1.000000,1.000000}%
\pgfsetstrokecolor{currentstroke}%
\pgfsetdash{}{0pt}%
\pgfpathmoveto{\pgfqpoint{2.486603in}{10.135661in}}%
\pgfpathlineto{\pgfqpoint{2.574339in}{10.135661in}}%
\pgfpathlineto{\pgfqpoint{2.574339in}{10.047925in}}%
\pgfpathlineto{\pgfqpoint{2.486603in}{10.047925in}}%
\pgfpathlineto{\pgfqpoint{2.486603in}{10.135661in}}%
\pgfusepath{stroke,fill}%
\end{pgfscope}%
\begin{pgfscope}%
\pgfpathrectangle{\pgfqpoint{0.380943in}{9.960189in}}{\pgfqpoint{4.650000in}{0.614151in}}%
\pgfusepath{clip}%
\pgfsetbuttcap%
\pgfsetroundjoin%
\definecolor{currentfill}{rgb}{1.000000,1.000000,0.929412}%
\pgfsetfillcolor{currentfill}%
\pgfsetlinewidth{0.250937pt}%
\definecolor{currentstroke}{rgb}{1.000000,1.000000,1.000000}%
\pgfsetstrokecolor{currentstroke}%
\pgfsetdash{}{0pt}%
\pgfpathmoveto{\pgfqpoint{2.574339in}{10.135661in}}%
\pgfpathlineto{\pgfqpoint{2.662075in}{10.135661in}}%
\pgfpathlineto{\pgfqpoint{2.662075in}{10.047925in}}%
\pgfpathlineto{\pgfqpoint{2.574339in}{10.047925in}}%
\pgfpathlineto{\pgfqpoint{2.574339in}{10.135661in}}%
\pgfusepath{stroke,fill}%
\end{pgfscope}%
\begin{pgfscope}%
\pgfpathrectangle{\pgfqpoint{0.380943in}{9.960189in}}{\pgfqpoint{4.650000in}{0.614151in}}%
\pgfusepath{clip}%
\pgfsetbuttcap%
\pgfsetroundjoin%
\definecolor{currentfill}{rgb}{1.000000,1.000000,0.929412}%
\pgfsetfillcolor{currentfill}%
\pgfsetlinewidth{0.250937pt}%
\definecolor{currentstroke}{rgb}{1.000000,1.000000,1.000000}%
\pgfsetstrokecolor{currentstroke}%
\pgfsetdash{}{0pt}%
\pgfpathmoveto{\pgfqpoint{2.662075in}{10.135661in}}%
\pgfpathlineto{\pgfqpoint{2.749811in}{10.135661in}}%
\pgfpathlineto{\pgfqpoint{2.749811in}{10.047925in}}%
\pgfpathlineto{\pgfqpoint{2.662075in}{10.047925in}}%
\pgfpathlineto{\pgfqpoint{2.662075in}{10.135661in}}%
\pgfusepath{stroke,fill}%
\end{pgfscope}%
\begin{pgfscope}%
\pgfpathrectangle{\pgfqpoint{0.380943in}{9.960189in}}{\pgfqpoint{4.650000in}{0.614151in}}%
\pgfusepath{clip}%
\pgfsetbuttcap%
\pgfsetroundjoin%
\definecolor{currentfill}{rgb}{1.000000,1.000000,0.929412}%
\pgfsetfillcolor{currentfill}%
\pgfsetlinewidth{0.250937pt}%
\definecolor{currentstroke}{rgb}{1.000000,1.000000,1.000000}%
\pgfsetstrokecolor{currentstroke}%
\pgfsetdash{}{0pt}%
\pgfpathmoveto{\pgfqpoint{2.749811in}{10.135661in}}%
\pgfpathlineto{\pgfqpoint{2.837547in}{10.135661in}}%
\pgfpathlineto{\pgfqpoint{2.837547in}{10.047925in}}%
\pgfpathlineto{\pgfqpoint{2.749811in}{10.047925in}}%
\pgfpathlineto{\pgfqpoint{2.749811in}{10.135661in}}%
\pgfusepath{stroke,fill}%
\end{pgfscope}%
\begin{pgfscope}%
\pgfpathrectangle{\pgfqpoint{0.380943in}{9.960189in}}{\pgfqpoint{4.650000in}{0.614151in}}%
\pgfusepath{clip}%
\pgfsetbuttcap%
\pgfsetroundjoin%
\definecolor{currentfill}{rgb}{1.000000,1.000000,0.929412}%
\pgfsetfillcolor{currentfill}%
\pgfsetlinewidth{0.250937pt}%
\definecolor{currentstroke}{rgb}{1.000000,1.000000,1.000000}%
\pgfsetstrokecolor{currentstroke}%
\pgfsetdash{}{0pt}%
\pgfpathmoveto{\pgfqpoint{2.837547in}{10.135661in}}%
\pgfpathlineto{\pgfqpoint{2.925283in}{10.135661in}}%
\pgfpathlineto{\pgfqpoint{2.925283in}{10.047925in}}%
\pgfpathlineto{\pgfqpoint{2.837547in}{10.047925in}}%
\pgfpathlineto{\pgfqpoint{2.837547in}{10.135661in}}%
\pgfusepath{stroke,fill}%
\end{pgfscope}%
\begin{pgfscope}%
\pgfpathrectangle{\pgfqpoint{0.380943in}{9.960189in}}{\pgfqpoint{4.650000in}{0.614151in}}%
\pgfusepath{clip}%
\pgfsetbuttcap%
\pgfsetroundjoin%
\definecolor{currentfill}{rgb}{1.000000,1.000000,0.929412}%
\pgfsetfillcolor{currentfill}%
\pgfsetlinewidth{0.250937pt}%
\definecolor{currentstroke}{rgb}{1.000000,1.000000,1.000000}%
\pgfsetstrokecolor{currentstroke}%
\pgfsetdash{}{0pt}%
\pgfpathmoveto{\pgfqpoint{2.925283in}{10.135661in}}%
\pgfpathlineto{\pgfqpoint{3.013019in}{10.135661in}}%
\pgfpathlineto{\pgfqpoint{3.013019in}{10.047925in}}%
\pgfpathlineto{\pgfqpoint{2.925283in}{10.047925in}}%
\pgfpathlineto{\pgfqpoint{2.925283in}{10.135661in}}%
\pgfusepath{stroke,fill}%
\end{pgfscope}%
\begin{pgfscope}%
\pgfpathrectangle{\pgfqpoint{0.380943in}{9.960189in}}{\pgfqpoint{4.650000in}{0.614151in}}%
\pgfusepath{clip}%
\pgfsetbuttcap%
\pgfsetroundjoin%
\definecolor{currentfill}{rgb}{1.000000,1.000000,0.929412}%
\pgfsetfillcolor{currentfill}%
\pgfsetlinewidth{0.250937pt}%
\definecolor{currentstroke}{rgb}{1.000000,1.000000,1.000000}%
\pgfsetstrokecolor{currentstroke}%
\pgfsetdash{}{0pt}%
\pgfpathmoveto{\pgfqpoint{3.013019in}{10.135661in}}%
\pgfpathlineto{\pgfqpoint{3.100754in}{10.135661in}}%
\pgfpathlineto{\pgfqpoint{3.100754in}{10.047925in}}%
\pgfpathlineto{\pgfqpoint{3.013019in}{10.047925in}}%
\pgfpathlineto{\pgfqpoint{3.013019in}{10.135661in}}%
\pgfusepath{stroke,fill}%
\end{pgfscope}%
\begin{pgfscope}%
\pgfpathrectangle{\pgfqpoint{0.380943in}{9.960189in}}{\pgfqpoint{4.650000in}{0.614151in}}%
\pgfusepath{clip}%
\pgfsetbuttcap%
\pgfsetroundjoin%
\definecolor{currentfill}{rgb}{1.000000,1.000000,0.929412}%
\pgfsetfillcolor{currentfill}%
\pgfsetlinewidth{0.250937pt}%
\definecolor{currentstroke}{rgb}{1.000000,1.000000,1.000000}%
\pgfsetstrokecolor{currentstroke}%
\pgfsetdash{}{0pt}%
\pgfpathmoveto{\pgfqpoint{3.100754in}{10.135661in}}%
\pgfpathlineto{\pgfqpoint{3.188490in}{10.135661in}}%
\pgfpathlineto{\pgfqpoint{3.188490in}{10.047925in}}%
\pgfpathlineto{\pgfqpoint{3.100754in}{10.047925in}}%
\pgfpathlineto{\pgfqpoint{3.100754in}{10.135661in}}%
\pgfusepath{stroke,fill}%
\end{pgfscope}%
\begin{pgfscope}%
\pgfpathrectangle{\pgfqpoint{0.380943in}{9.960189in}}{\pgfqpoint{4.650000in}{0.614151in}}%
\pgfusepath{clip}%
\pgfsetbuttcap%
\pgfsetroundjoin%
\definecolor{currentfill}{rgb}{1.000000,1.000000,0.929412}%
\pgfsetfillcolor{currentfill}%
\pgfsetlinewidth{0.250937pt}%
\definecolor{currentstroke}{rgb}{1.000000,1.000000,1.000000}%
\pgfsetstrokecolor{currentstroke}%
\pgfsetdash{}{0pt}%
\pgfpathmoveto{\pgfqpoint{3.188490in}{10.135661in}}%
\pgfpathlineto{\pgfqpoint{3.276226in}{10.135661in}}%
\pgfpathlineto{\pgfqpoint{3.276226in}{10.047925in}}%
\pgfpathlineto{\pgfqpoint{3.188490in}{10.047925in}}%
\pgfpathlineto{\pgfqpoint{3.188490in}{10.135661in}}%
\pgfusepath{stroke,fill}%
\end{pgfscope}%
\begin{pgfscope}%
\pgfpathrectangle{\pgfqpoint{0.380943in}{9.960189in}}{\pgfqpoint{4.650000in}{0.614151in}}%
\pgfusepath{clip}%
\pgfsetbuttcap%
\pgfsetroundjoin%
\definecolor{currentfill}{rgb}{1.000000,1.000000,0.929412}%
\pgfsetfillcolor{currentfill}%
\pgfsetlinewidth{0.250937pt}%
\definecolor{currentstroke}{rgb}{1.000000,1.000000,1.000000}%
\pgfsetstrokecolor{currentstroke}%
\pgfsetdash{}{0pt}%
\pgfpathmoveto{\pgfqpoint{3.276226in}{10.135661in}}%
\pgfpathlineto{\pgfqpoint{3.363962in}{10.135661in}}%
\pgfpathlineto{\pgfqpoint{3.363962in}{10.047925in}}%
\pgfpathlineto{\pgfqpoint{3.276226in}{10.047925in}}%
\pgfpathlineto{\pgfqpoint{3.276226in}{10.135661in}}%
\pgfusepath{stroke,fill}%
\end{pgfscope}%
\begin{pgfscope}%
\pgfpathrectangle{\pgfqpoint{0.380943in}{9.960189in}}{\pgfqpoint{4.650000in}{0.614151in}}%
\pgfusepath{clip}%
\pgfsetbuttcap%
\pgfsetroundjoin%
\definecolor{currentfill}{rgb}{1.000000,1.000000,0.929412}%
\pgfsetfillcolor{currentfill}%
\pgfsetlinewidth{0.250937pt}%
\definecolor{currentstroke}{rgb}{1.000000,1.000000,1.000000}%
\pgfsetstrokecolor{currentstroke}%
\pgfsetdash{}{0pt}%
\pgfpathmoveto{\pgfqpoint{3.363962in}{10.135661in}}%
\pgfpathlineto{\pgfqpoint{3.451698in}{10.135661in}}%
\pgfpathlineto{\pgfqpoint{3.451698in}{10.047925in}}%
\pgfpathlineto{\pgfqpoint{3.363962in}{10.047925in}}%
\pgfpathlineto{\pgfqpoint{3.363962in}{10.135661in}}%
\pgfusepath{stroke,fill}%
\end{pgfscope}%
\begin{pgfscope}%
\pgfpathrectangle{\pgfqpoint{0.380943in}{9.960189in}}{\pgfqpoint{4.650000in}{0.614151in}}%
\pgfusepath{clip}%
\pgfsetbuttcap%
\pgfsetroundjoin%
\definecolor{currentfill}{rgb}{1.000000,1.000000,0.929412}%
\pgfsetfillcolor{currentfill}%
\pgfsetlinewidth{0.250937pt}%
\definecolor{currentstroke}{rgb}{1.000000,1.000000,1.000000}%
\pgfsetstrokecolor{currentstroke}%
\pgfsetdash{}{0pt}%
\pgfpathmoveto{\pgfqpoint{3.451698in}{10.135661in}}%
\pgfpathlineto{\pgfqpoint{3.539434in}{10.135661in}}%
\pgfpathlineto{\pgfqpoint{3.539434in}{10.047925in}}%
\pgfpathlineto{\pgfqpoint{3.451698in}{10.047925in}}%
\pgfpathlineto{\pgfqpoint{3.451698in}{10.135661in}}%
\pgfusepath{stroke,fill}%
\end{pgfscope}%
\begin{pgfscope}%
\pgfpathrectangle{\pgfqpoint{0.380943in}{9.960189in}}{\pgfqpoint{4.650000in}{0.614151in}}%
\pgfusepath{clip}%
\pgfsetbuttcap%
\pgfsetroundjoin%
\definecolor{currentfill}{rgb}{1.000000,1.000000,0.929412}%
\pgfsetfillcolor{currentfill}%
\pgfsetlinewidth{0.250937pt}%
\definecolor{currentstroke}{rgb}{1.000000,1.000000,1.000000}%
\pgfsetstrokecolor{currentstroke}%
\pgfsetdash{}{0pt}%
\pgfpathmoveto{\pgfqpoint{3.539434in}{10.135661in}}%
\pgfpathlineto{\pgfqpoint{3.627169in}{10.135661in}}%
\pgfpathlineto{\pgfqpoint{3.627169in}{10.047925in}}%
\pgfpathlineto{\pgfqpoint{3.539434in}{10.047925in}}%
\pgfpathlineto{\pgfqpoint{3.539434in}{10.135661in}}%
\pgfusepath{stroke,fill}%
\end{pgfscope}%
\begin{pgfscope}%
\pgfpathrectangle{\pgfqpoint{0.380943in}{9.960189in}}{\pgfqpoint{4.650000in}{0.614151in}}%
\pgfusepath{clip}%
\pgfsetbuttcap%
\pgfsetroundjoin%
\definecolor{currentfill}{rgb}{1.000000,1.000000,0.929412}%
\pgfsetfillcolor{currentfill}%
\pgfsetlinewidth{0.250937pt}%
\definecolor{currentstroke}{rgb}{1.000000,1.000000,1.000000}%
\pgfsetstrokecolor{currentstroke}%
\pgfsetdash{}{0pt}%
\pgfpathmoveto{\pgfqpoint{3.627169in}{10.135661in}}%
\pgfpathlineto{\pgfqpoint{3.714905in}{10.135661in}}%
\pgfpathlineto{\pgfqpoint{3.714905in}{10.047925in}}%
\pgfpathlineto{\pgfqpoint{3.627169in}{10.047925in}}%
\pgfpathlineto{\pgfqpoint{3.627169in}{10.135661in}}%
\pgfusepath{stroke,fill}%
\end{pgfscope}%
\begin{pgfscope}%
\pgfpathrectangle{\pgfqpoint{0.380943in}{9.960189in}}{\pgfqpoint{4.650000in}{0.614151in}}%
\pgfusepath{clip}%
\pgfsetbuttcap%
\pgfsetroundjoin%
\definecolor{currentfill}{rgb}{1.000000,1.000000,0.929412}%
\pgfsetfillcolor{currentfill}%
\pgfsetlinewidth{0.250937pt}%
\definecolor{currentstroke}{rgb}{1.000000,1.000000,1.000000}%
\pgfsetstrokecolor{currentstroke}%
\pgfsetdash{}{0pt}%
\pgfpathmoveto{\pgfqpoint{3.714905in}{10.135661in}}%
\pgfpathlineto{\pgfqpoint{3.802641in}{10.135661in}}%
\pgfpathlineto{\pgfqpoint{3.802641in}{10.047925in}}%
\pgfpathlineto{\pgfqpoint{3.714905in}{10.047925in}}%
\pgfpathlineto{\pgfqpoint{3.714905in}{10.135661in}}%
\pgfusepath{stroke,fill}%
\end{pgfscope}%
\begin{pgfscope}%
\pgfpathrectangle{\pgfqpoint{0.380943in}{9.960189in}}{\pgfqpoint{4.650000in}{0.614151in}}%
\pgfusepath{clip}%
\pgfsetbuttcap%
\pgfsetroundjoin%
\definecolor{currentfill}{rgb}{1.000000,1.000000,0.929412}%
\pgfsetfillcolor{currentfill}%
\pgfsetlinewidth{0.250937pt}%
\definecolor{currentstroke}{rgb}{1.000000,1.000000,1.000000}%
\pgfsetstrokecolor{currentstroke}%
\pgfsetdash{}{0pt}%
\pgfpathmoveto{\pgfqpoint{3.802641in}{10.135661in}}%
\pgfpathlineto{\pgfqpoint{3.890377in}{10.135661in}}%
\pgfpathlineto{\pgfqpoint{3.890377in}{10.047925in}}%
\pgfpathlineto{\pgfqpoint{3.802641in}{10.047925in}}%
\pgfpathlineto{\pgfqpoint{3.802641in}{10.135661in}}%
\pgfusepath{stroke,fill}%
\end{pgfscope}%
\begin{pgfscope}%
\pgfpathrectangle{\pgfqpoint{0.380943in}{9.960189in}}{\pgfqpoint{4.650000in}{0.614151in}}%
\pgfusepath{clip}%
\pgfsetbuttcap%
\pgfsetroundjoin%
\definecolor{currentfill}{rgb}{1.000000,1.000000,0.929412}%
\pgfsetfillcolor{currentfill}%
\pgfsetlinewidth{0.250937pt}%
\definecolor{currentstroke}{rgb}{1.000000,1.000000,1.000000}%
\pgfsetstrokecolor{currentstroke}%
\pgfsetdash{}{0pt}%
\pgfpathmoveto{\pgfqpoint{3.890377in}{10.135661in}}%
\pgfpathlineto{\pgfqpoint{3.978113in}{10.135661in}}%
\pgfpathlineto{\pgfqpoint{3.978113in}{10.047925in}}%
\pgfpathlineto{\pgfqpoint{3.890377in}{10.047925in}}%
\pgfpathlineto{\pgfqpoint{3.890377in}{10.135661in}}%
\pgfusepath{stroke,fill}%
\end{pgfscope}%
\begin{pgfscope}%
\pgfpathrectangle{\pgfqpoint{0.380943in}{9.960189in}}{\pgfqpoint{4.650000in}{0.614151in}}%
\pgfusepath{clip}%
\pgfsetbuttcap%
\pgfsetroundjoin%
\definecolor{currentfill}{rgb}{1.000000,1.000000,0.929412}%
\pgfsetfillcolor{currentfill}%
\pgfsetlinewidth{0.250937pt}%
\definecolor{currentstroke}{rgb}{1.000000,1.000000,1.000000}%
\pgfsetstrokecolor{currentstroke}%
\pgfsetdash{}{0pt}%
\pgfpathmoveto{\pgfqpoint{3.978113in}{10.135661in}}%
\pgfpathlineto{\pgfqpoint{4.065849in}{10.135661in}}%
\pgfpathlineto{\pgfqpoint{4.065849in}{10.047925in}}%
\pgfpathlineto{\pgfqpoint{3.978113in}{10.047925in}}%
\pgfpathlineto{\pgfqpoint{3.978113in}{10.135661in}}%
\pgfusepath{stroke,fill}%
\end{pgfscope}%
\begin{pgfscope}%
\pgfpathrectangle{\pgfqpoint{0.380943in}{9.960189in}}{\pgfqpoint{4.650000in}{0.614151in}}%
\pgfusepath{clip}%
\pgfsetbuttcap%
\pgfsetroundjoin%
\definecolor{currentfill}{rgb}{1.000000,1.000000,0.929412}%
\pgfsetfillcolor{currentfill}%
\pgfsetlinewidth{0.250937pt}%
\definecolor{currentstroke}{rgb}{1.000000,1.000000,1.000000}%
\pgfsetstrokecolor{currentstroke}%
\pgfsetdash{}{0pt}%
\pgfpathmoveto{\pgfqpoint{4.065849in}{10.135661in}}%
\pgfpathlineto{\pgfqpoint{4.153585in}{10.135661in}}%
\pgfpathlineto{\pgfqpoint{4.153585in}{10.047925in}}%
\pgfpathlineto{\pgfqpoint{4.065849in}{10.047925in}}%
\pgfpathlineto{\pgfqpoint{4.065849in}{10.135661in}}%
\pgfusepath{stroke,fill}%
\end{pgfscope}%
\begin{pgfscope}%
\pgfpathrectangle{\pgfqpoint{0.380943in}{9.960189in}}{\pgfqpoint{4.650000in}{0.614151in}}%
\pgfusepath{clip}%
\pgfsetbuttcap%
\pgfsetroundjoin%
\definecolor{currentfill}{rgb}{1.000000,1.000000,0.929412}%
\pgfsetfillcolor{currentfill}%
\pgfsetlinewidth{0.250937pt}%
\definecolor{currentstroke}{rgb}{1.000000,1.000000,1.000000}%
\pgfsetstrokecolor{currentstroke}%
\pgfsetdash{}{0pt}%
\pgfpathmoveto{\pgfqpoint{4.153585in}{10.135661in}}%
\pgfpathlineto{\pgfqpoint{4.241320in}{10.135661in}}%
\pgfpathlineto{\pgfqpoint{4.241320in}{10.047925in}}%
\pgfpathlineto{\pgfqpoint{4.153585in}{10.047925in}}%
\pgfpathlineto{\pgfqpoint{4.153585in}{10.135661in}}%
\pgfusepath{stroke,fill}%
\end{pgfscope}%
\begin{pgfscope}%
\pgfpathrectangle{\pgfqpoint{0.380943in}{9.960189in}}{\pgfqpoint{4.650000in}{0.614151in}}%
\pgfusepath{clip}%
\pgfsetbuttcap%
\pgfsetroundjoin%
\definecolor{currentfill}{rgb}{1.000000,1.000000,0.929412}%
\pgfsetfillcolor{currentfill}%
\pgfsetlinewidth{0.250937pt}%
\definecolor{currentstroke}{rgb}{1.000000,1.000000,1.000000}%
\pgfsetstrokecolor{currentstroke}%
\pgfsetdash{}{0pt}%
\pgfpathmoveto{\pgfqpoint{4.241320in}{10.135661in}}%
\pgfpathlineto{\pgfqpoint{4.329056in}{10.135661in}}%
\pgfpathlineto{\pgfqpoint{4.329056in}{10.047925in}}%
\pgfpathlineto{\pgfqpoint{4.241320in}{10.047925in}}%
\pgfpathlineto{\pgfqpoint{4.241320in}{10.135661in}}%
\pgfusepath{stroke,fill}%
\end{pgfscope}%
\begin{pgfscope}%
\pgfpathrectangle{\pgfqpoint{0.380943in}{9.960189in}}{\pgfqpoint{4.650000in}{0.614151in}}%
\pgfusepath{clip}%
\pgfsetbuttcap%
\pgfsetroundjoin%
\definecolor{currentfill}{rgb}{1.000000,1.000000,0.929412}%
\pgfsetfillcolor{currentfill}%
\pgfsetlinewidth{0.250937pt}%
\definecolor{currentstroke}{rgb}{1.000000,1.000000,1.000000}%
\pgfsetstrokecolor{currentstroke}%
\pgfsetdash{}{0pt}%
\pgfpathmoveto{\pgfqpoint{4.329056in}{10.135661in}}%
\pgfpathlineto{\pgfqpoint{4.416792in}{10.135661in}}%
\pgfpathlineto{\pgfqpoint{4.416792in}{10.047925in}}%
\pgfpathlineto{\pgfqpoint{4.329056in}{10.047925in}}%
\pgfpathlineto{\pgfqpoint{4.329056in}{10.135661in}}%
\pgfusepath{stroke,fill}%
\end{pgfscope}%
\begin{pgfscope}%
\pgfpathrectangle{\pgfqpoint{0.380943in}{9.960189in}}{\pgfqpoint{4.650000in}{0.614151in}}%
\pgfusepath{clip}%
\pgfsetbuttcap%
\pgfsetroundjoin%
\definecolor{currentfill}{rgb}{0.963091,0.919493,0.720185}%
\pgfsetfillcolor{currentfill}%
\pgfsetlinewidth{0.250937pt}%
\definecolor{currentstroke}{rgb}{1.000000,1.000000,1.000000}%
\pgfsetstrokecolor{currentstroke}%
\pgfsetdash{}{0pt}%
\pgfpathmoveto{\pgfqpoint{4.416792in}{10.135661in}}%
\pgfpathlineto{\pgfqpoint{4.504528in}{10.135661in}}%
\pgfpathlineto{\pgfqpoint{4.504528in}{10.047925in}}%
\pgfpathlineto{\pgfqpoint{4.416792in}{10.047925in}}%
\pgfpathlineto{\pgfqpoint{4.416792in}{10.135661in}}%
\pgfusepath{stroke,fill}%
\end{pgfscope}%
\begin{pgfscope}%
\pgfpathrectangle{\pgfqpoint{0.380943in}{9.960189in}}{\pgfqpoint{4.650000in}{0.614151in}}%
\pgfusepath{clip}%
\pgfsetbuttcap%
\pgfsetroundjoin%
\definecolor{currentfill}{rgb}{0.994694,0.745098,0.602999}%
\pgfsetfillcolor{currentfill}%
\pgfsetlinewidth{0.250937pt}%
\definecolor{currentstroke}{rgb}{1.000000,1.000000,1.000000}%
\pgfsetstrokecolor{currentstroke}%
\pgfsetdash{}{0pt}%
\pgfpathmoveto{\pgfqpoint{4.504528in}{10.135661in}}%
\pgfpathlineto{\pgfqpoint{4.592264in}{10.135661in}}%
\pgfpathlineto{\pgfqpoint{4.592264in}{10.047925in}}%
\pgfpathlineto{\pgfqpoint{4.504528in}{10.047925in}}%
\pgfpathlineto{\pgfqpoint{4.504528in}{10.135661in}}%
\pgfusepath{stroke,fill}%
\end{pgfscope}%
\begin{pgfscope}%
\pgfpathrectangle{\pgfqpoint{0.380943in}{9.960189in}}{\pgfqpoint{4.650000in}{0.614151in}}%
\pgfusepath{clip}%
\pgfsetbuttcap%
\pgfsetroundjoin%
\definecolor{currentfill}{rgb}{0.963091,0.937255,0.735409}%
\pgfsetfillcolor{currentfill}%
\pgfsetlinewidth{0.250937pt}%
\definecolor{currentstroke}{rgb}{1.000000,1.000000,1.000000}%
\pgfsetstrokecolor{currentstroke}%
\pgfsetdash{}{0pt}%
\pgfpathmoveto{\pgfqpoint{4.592264in}{10.135661in}}%
\pgfpathlineto{\pgfqpoint{4.680000in}{10.135661in}}%
\pgfpathlineto{\pgfqpoint{4.680000in}{10.047925in}}%
\pgfpathlineto{\pgfqpoint{4.592264in}{10.047925in}}%
\pgfpathlineto{\pgfqpoint{4.592264in}{10.135661in}}%
\pgfusepath{stroke,fill}%
\end{pgfscope}%
\begin{pgfscope}%
\pgfpathrectangle{\pgfqpoint{0.380943in}{9.960189in}}{\pgfqpoint{4.650000in}{0.614151in}}%
\pgfusepath{clip}%
\pgfsetbuttcap%
\pgfsetroundjoin%
\definecolor{currentfill}{rgb}{0.982699,0.823991,0.657439}%
\pgfsetfillcolor{currentfill}%
\pgfsetlinewidth{0.250937pt}%
\definecolor{currentstroke}{rgb}{1.000000,1.000000,1.000000}%
\pgfsetstrokecolor{currentstroke}%
\pgfsetdash{}{0pt}%
\pgfpathmoveto{\pgfqpoint{4.680000in}{10.135661in}}%
\pgfpathlineto{\pgfqpoint{4.767736in}{10.135661in}}%
\pgfpathlineto{\pgfqpoint{4.767736in}{10.047925in}}%
\pgfpathlineto{\pgfqpoint{4.680000in}{10.047925in}}%
\pgfpathlineto{\pgfqpoint{4.680000in}{10.135661in}}%
\pgfusepath{stroke,fill}%
\end{pgfscope}%
\begin{pgfscope}%
\pgfpathrectangle{\pgfqpoint{0.380943in}{9.960189in}}{\pgfqpoint{4.650000in}{0.614151in}}%
\pgfusepath{clip}%
\pgfsetbuttcap%
\pgfsetroundjoin%
\definecolor{currentfill}{rgb}{0.963091,0.919493,0.720185}%
\pgfsetfillcolor{currentfill}%
\pgfsetlinewidth{0.250937pt}%
\definecolor{currentstroke}{rgb}{1.000000,1.000000,1.000000}%
\pgfsetstrokecolor{currentstroke}%
\pgfsetdash{}{0pt}%
\pgfpathmoveto{\pgfqpoint{4.767736in}{10.135661in}}%
\pgfpathlineto{\pgfqpoint{4.855471in}{10.135661in}}%
\pgfpathlineto{\pgfqpoint{4.855471in}{10.047925in}}%
\pgfpathlineto{\pgfqpoint{4.767736in}{10.047925in}}%
\pgfpathlineto{\pgfqpoint{4.767736in}{10.135661in}}%
\pgfusepath{stroke,fill}%
\end{pgfscope}%
\begin{pgfscope}%
\pgfpathrectangle{\pgfqpoint{0.380943in}{9.960189in}}{\pgfqpoint{4.650000in}{0.614151in}}%
\pgfusepath{clip}%
\pgfsetbuttcap%
\pgfsetroundjoin%
\definecolor{currentfill}{rgb}{0.963091,0.919493,0.720185}%
\pgfsetfillcolor{currentfill}%
\pgfsetlinewidth{0.250937pt}%
\definecolor{currentstroke}{rgb}{1.000000,1.000000,1.000000}%
\pgfsetstrokecolor{currentstroke}%
\pgfsetdash{}{0pt}%
\pgfpathmoveto{\pgfqpoint{4.855471in}{10.135661in}}%
\pgfpathlineto{\pgfqpoint{4.943207in}{10.135661in}}%
\pgfpathlineto{\pgfqpoint{4.943207in}{10.047925in}}%
\pgfpathlineto{\pgfqpoint{4.855471in}{10.047925in}}%
\pgfpathlineto{\pgfqpoint{4.855471in}{10.135661in}}%
\pgfusepath{stroke,fill}%
\end{pgfscope}%
\begin{pgfscope}%
\pgfpathrectangle{\pgfqpoint{0.380943in}{9.960189in}}{\pgfqpoint{4.650000in}{0.614151in}}%
\pgfusepath{clip}%
\pgfsetbuttcap%
\pgfsetroundjoin%
\definecolor{currentfill}{rgb}{0.967474,0.895963,0.706344}%
\pgfsetfillcolor{currentfill}%
\pgfsetlinewidth{0.250937pt}%
\definecolor{currentstroke}{rgb}{1.000000,1.000000,1.000000}%
\pgfsetstrokecolor{currentstroke}%
\pgfsetdash{}{0pt}%
\pgfpathmoveto{\pgfqpoint{4.943207in}{10.135661in}}%
\pgfpathlineto{\pgfqpoint{5.030943in}{10.135661in}}%
\pgfpathlineto{\pgfqpoint{5.030943in}{10.047925in}}%
\pgfpathlineto{\pgfqpoint{4.943207in}{10.047925in}}%
\pgfpathlineto{\pgfqpoint{4.943207in}{10.135661in}}%
\pgfusepath{stroke,fill}%
\end{pgfscope}%
\begin{pgfscope}%
\pgfpathrectangle{\pgfqpoint{0.380943in}{9.960189in}}{\pgfqpoint{4.650000in}{0.614151in}}%
\pgfusepath{clip}%
\pgfsetbuttcap%
\pgfsetroundjoin%
\definecolor{currentfill}{rgb}{1.000000,1.000000,0.929412}%
\pgfsetfillcolor{currentfill}%
\pgfsetlinewidth{0.250937pt}%
\definecolor{currentstroke}{rgb}{1.000000,1.000000,1.000000}%
\pgfsetstrokecolor{currentstroke}%
\pgfsetdash{}{0pt}%
\pgfpathmoveto{\pgfqpoint{0.380943in}{10.047925in}}%
\pgfpathlineto{\pgfqpoint{0.468679in}{10.047925in}}%
\pgfpathlineto{\pgfqpoint{0.468679in}{9.960189in}}%
\pgfpathlineto{\pgfqpoint{0.380943in}{9.960189in}}%
\pgfpathlineto{\pgfqpoint{0.380943in}{10.047925in}}%
\pgfusepath{stroke,fill}%
\end{pgfscope}%
\begin{pgfscope}%
\pgfpathrectangle{\pgfqpoint{0.380943in}{9.960189in}}{\pgfqpoint{4.650000in}{0.614151in}}%
\pgfusepath{clip}%
\pgfsetbuttcap%
\pgfsetroundjoin%
\definecolor{currentfill}{rgb}{1.000000,1.000000,0.929412}%
\pgfsetfillcolor{currentfill}%
\pgfsetlinewidth{0.250937pt}%
\definecolor{currentstroke}{rgb}{1.000000,1.000000,1.000000}%
\pgfsetstrokecolor{currentstroke}%
\pgfsetdash{}{0pt}%
\pgfpathmoveto{\pgfqpoint{0.468679in}{10.047925in}}%
\pgfpathlineto{\pgfqpoint{0.556415in}{10.047925in}}%
\pgfpathlineto{\pgfqpoint{0.556415in}{9.960189in}}%
\pgfpathlineto{\pgfqpoint{0.468679in}{9.960189in}}%
\pgfpathlineto{\pgfqpoint{0.468679in}{10.047925in}}%
\pgfusepath{stroke,fill}%
\end{pgfscope}%
\begin{pgfscope}%
\pgfpathrectangle{\pgfqpoint{0.380943in}{9.960189in}}{\pgfqpoint{4.650000in}{0.614151in}}%
\pgfusepath{clip}%
\pgfsetbuttcap%
\pgfsetroundjoin%
\definecolor{currentfill}{rgb}{1.000000,1.000000,0.929412}%
\pgfsetfillcolor{currentfill}%
\pgfsetlinewidth{0.250937pt}%
\definecolor{currentstroke}{rgb}{1.000000,1.000000,1.000000}%
\pgfsetstrokecolor{currentstroke}%
\pgfsetdash{}{0pt}%
\pgfpathmoveto{\pgfqpoint{0.556415in}{10.047925in}}%
\pgfpathlineto{\pgfqpoint{0.644151in}{10.047925in}}%
\pgfpathlineto{\pgfqpoint{0.644151in}{9.960189in}}%
\pgfpathlineto{\pgfqpoint{0.556415in}{9.960189in}}%
\pgfpathlineto{\pgfqpoint{0.556415in}{10.047925in}}%
\pgfusepath{stroke,fill}%
\end{pgfscope}%
\begin{pgfscope}%
\pgfpathrectangle{\pgfqpoint{0.380943in}{9.960189in}}{\pgfqpoint{4.650000in}{0.614151in}}%
\pgfusepath{clip}%
\pgfsetbuttcap%
\pgfsetroundjoin%
\definecolor{currentfill}{rgb}{1.000000,1.000000,0.929412}%
\pgfsetfillcolor{currentfill}%
\pgfsetlinewidth{0.250937pt}%
\definecolor{currentstroke}{rgb}{1.000000,1.000000,1.000000}%
\pgfsetstrokecolor{currentstroke}%
\pgfsetdash{}{0pt}%
\pgfpathmoveto{\pgfqpoint{0.644151in}{10.047925in}}%
\pgfpathlineto{\pgfqpoint{0.731886in}{10.047925in}}%
\pgfpathlineto{\pgfqpoint{0.731886in}{9.960189in}}%
\pgfpathlineto{\pgfqpoint{0.644151in}{9.960189in}}%
\pgfpathlineto{\pgfqpoint{0.644151in}{10.047925in}}%
\pgfusepath{stroke,fill}%
\end{pgfscope}%
\begin{pgfscope}%
\pgfpathrectangle{\pgfqpoint{0.380943in}{9.960189in}}{\pgfqpoint{4.650000in}{0.614151in}}%
\pgfusepath{clip}%
\pgfsetbuttcap%
\pgfsetroundjoin%
\definecolor{currentfill}{rgb}{1.000000,1.000000,0.929412}%
\pgfsetfillcolor{currentfill}%
\pgfsetlinewidth{0.250937pt}%
\definecolor{currentstroke}{rgb}{1.000000,1.000000,1.000000}%
\pgfsetstrokecolor{currentstroke}%
\pgfsetdash{}{0pt}%
\pgfpathmoveto{\pgfqpoint{0.731886in}{10.047925in}}%
\pgfpathlineto{\pgfqpoint{0.819622in}{10.047925in}}%
\pgfpathlineto{\pgfqpoint{0.819622in}{9.960189in}}%
\pgfpathlineto{\pgfqpoint{0.731886in}{9.960189in}}%
\pgfpathlineto{\pgfqpoint{0.731886in}{10.047925in}}%
\pgfusepath{stroke,fill}%
\end{pgfscope}%
\begin{pgfscope}%
\pgfpathrectangle{\pgfqpoint{0.380943in}{9.960189in}}{\pgfqpoint{4.650000in}{0.614151in}}%
\pgfusepath{clip}%
\pgfsetbuttcap%
\pgfsetroundjoin%
\definecolor{currentfill}{rgb}{1.000000,1.000000,0.929412}%
\pgfsetfillcolor{currentfill}%
\pgfsetlinewidth{0.250937pt}%
\definecolor{currentstroke}{rgb}{1.000000,1.000000,1.000000}%
\pgfsetstrokecolor{currentstroke}%
\pgfsetdash{}{0pt}%
\pgfpathmoveto{\pgfqpoint{0.819622in}{10.047925in}}%
\pgfpathlineto{\pgfqpoint{0.907358in}{10.047925in}}%
\pgfpathlineto{\pgfqpoint{0.907358in}{9.960189in}}%
\pgfpathlineto{\pgfqpoint{0.819622in}{9.960189in}}%
\pgfpathlineto{\pgfqpoint{0.819622in}{10.047925in}}%
\pgfusepath{stroke,fill}%
\end{pgfscope}%
\begin{pgfscope}%
\pgfpathrectangle{\pgfqpoint{0.380943in}{9.960189in}}{\pgfqpoint{4.650000in}{0.614151in}}%
\pgfusepath{clip}%
\pgfsetbuttcap%
\pgfsetroundjoin%
\definecolor{currentfill}{rgb}{1.000000,1.000000,0.929412}%
\pgfsetfillcolor{currentfill}%
\pgfsetlinewidth{0.250937pt}%
\definecolor{currentstroke}{rgb}{1.000000,1.000000,1.000000}%
\pgfsetstrokecolor{currentstroke}%
\pgfsetdash{}{0pt}%
\pgfpathmoveto{\pgfqpoint{0.907358in}{10.047925in}}%
\pgfpathlineto{\pgfqpoint{0.995094in}{10.047925in}}%
\pgfpathlineto{\pgfqpoint{0.995094in}{9.960189in}}%
\pgfpathlineto{\pgfqpoint{0.907358in}{9.960189in}}%
\pgfpathlineto{\pgfqpoint{0.907358in}{10.047925in}}%
\pgfusepath{stroke,fill}%
\end{pgfscope}%
\begin{pgfscope}%
\pgfpathrectangle{\pgfqpoint{0.380943in}{9.960189in}}{\pgfqpoint{4.650000in}{0.614151in}}%
\pgfusepath{clip}%
\pgfsetbuttcap%
\pgfsetroundjoin%
\definecolor{currentfill}{rgb}{1.000000,1.000000,0.929412}%
\pgfsetfillcolor{currentfill}%
\pgfsetlinewidth{0.250937pt}%
\definecolor{currentstroke}{rgb}{1.000000,1.000000,1.000000}%
\pgfsetstrokecolor{currentstroke}%
\pgfsetdash{}{0pt}%
\pgfpathmoveto{\pgfqpoint{0.995094in}{10.047925in}}%
\pgfpathlineto{\pgfqpoint{1.082830in}{10.047925in}}%
\pgfpathlineto{\pgfqpoint{1.082830in}{9.960189in}}%
\pgfpathlineto{\pgfqpoint{0.995094in}{9.960189in}}%
\pgfpathlineto{\pgfqpoint{0.995094in}{10.047925in}}%
\pgfusepath{stroke,fill}%
\end{pgfscope}%
\begin{pgfscope}%
\pgfpathrectangle{\pgfqpoint{0.380943in}{9.960189in}}{\pgfqpoint{4.650000in}{0.614151in}}%
\pgfusepath{clip}%
\pgfsetbuttcap%
\pgfsetroundjoin%
\definecolor{currentfill}{rgb}{1.000000,1.000000,0.929412}%
\pgfsetfillcolor{currentfill}%
\pgfsetlinewidth{0.250937pt}%
\definecolor{currentstroke}{rgb}{1.000000,1.000000,1.000000}%
\pgfsetstrokecolor{currentstroke}%
\pgfsetdash{}{0pt}%
\pgfpathmoveto{\pgfqpoint{1.082830in}{10.047925in}}%
\pgfpathlineto{\pgfqpoint{1.170566in}{10.047925in}}%
\pgfpathlineto{\pgfqpoint{1.170566in}{9.960189in}}%
\pgfpathlineto{\pgfqpoint{1.082830in}{9.960189in}}%
\pgfpathlineto{\pgfqpoint{1.082830in}{10.047925in}}%
\pgfusepath{stroke,fill}%
\end{pgfscope}%
\begin{pgfscope}%
\pgfpathrectangle{\pgfqpoint{0.380943in}{9.960189in}}{\pgfqpoint{4.650000in}{0.614151in}}%
\pgfusepath{clip}%
\pgfsetbuttcap%
\pgfsetroundjoin%
\definecolor{currentfill}{rgb}{1.000000,1.000000,0.929412}%
\pgfsetfillcolor{currentfill}%
\pgfsetlinewidth{0.250937pt}%
\definecolor{currentstroke}{rgb}{1.000000,1.000000,1.000000}%
\pgfsetstrokecolor{currentstroke}%
\pgfsetdash{}{0pt}%
\pgfpathmoveto{\pgfqpoint{1.170566in}{10.047925in}}%
\pgfpathlineto{\pgfqpoint{1.258302in}{10.047925in}}%
\pgfpathlineto{\pgfqpoint{1.258302in}{9.960189in}}%
\pgfpathlineto{\pgfqpoint{1.170566in}{9.960189in}}%
\pgfpathlineto{\pgfqpoint{1.170566in}{10.047925in}}%
\pgfusepath{stroke,fill}%
\end{pgfscope}%
\begin{pgfscope}%
\pgfpathrectangle{\pgfqpoint{0.380943in}{9.960189in}}{\pgfqpoint{4.650000in}{0.614151in}}%
\pgfusepath{clip}%
\pgfsetbuttcap%
\pgfsetroundjoin%
\definecolor{currentfill}{rgb}{1.000000,1.000000,0.929412}%
\pgfsetfillcolor{currentfill}%
\pgfsetlinewidth{0.250937pt}%
\definecolor{currentstroke}{rgb}{1.000000,1.000000,1.000000}%
\pgfsetstrokecolor{currentstroke}%
\pgfsetdash{}{0pt}%
\pgfpathmoveto{\pgfqpoint{1.258302in}{10.047925in}}%
\pgfpathlineto{\pgfqpoint{1.346037in}{10.047925in}}%
\pgfpathlineto{\pgfqpoint{1.346037in}{9.960189in}}%
\pgfpathlineto{\pgfqpoint{1.258302in}{9.960189in}}%
\pgfpathlineto{\pgfqpoint{1.258302in}{10.047925in}}%
\pgfusepath{stroke,fill}%
\end{pgfscope}%
\begin{pgfscope}%
\pgfpathrectangle{\pgfqpoint{0.380943in}{9.960189in}}{\pgfqpoint{4.650000in}{0.614151in}}%
\pgfusepath{clip}%
\pgfsetbuttcap%
\pgfsetroundjoin%
\definecolor{currentfill}{rgb}{1.000000,1.000000,0.929412}%
\pgfsetfillcolor{currentfill}%
\pgfsetlinewidth{0.250937pt}%
\definecolor{currentstroke}{rgb}{1.000000,1.000000,1.000000}%
\pgfsetstrokecolor{currentstroke}%
\pgfsetdash{}{0pt}%
\pgfpathmoveto{\pgfqpoint{1.346037in}{10.047925in}}%
\pgfpathlineto{\pgfqpoint{1.433773in}{10.047925in}}%
\pgfpathlineto{\pgfqpoint{1.433773in}{9.960189in}}%
\pgfpathlineto{\pgfqpoint{1.346037in}{9.960189in}}%
\pgfpathlineto{\pgfqpoint{1.346037in}{10.047925in}}%
\pgfusepath{stroke,fill}%
\end{pgfscope}%
\begin{pgfscope}%
\pgfpathrectangle{\pgfqpoint{0.380943in}{9.960189in}}{\pgfqpoint{4.650000in}{0.614151in}}%
\pgfusepath{clip}%
\pgfsetbuttcap%
\pgfsetroundjoin%
\definecolor{currentfill}{rgb}{1.000000,1.000000,0.929412}%
\pgfsetfillcolor{currentfill}%
\pgfsetlinewidth{0.250937pt}%
\definecolor{currentstroke}{rgb}{1.000000,1.000000,1.000000}%
\pgfsetstrokecolor{currentstroke}%
\pgfsetdash{}{0pt}%
\pgfpathmoveto{\pgfqpoint{1.433773in}{10.047925in}}%
\pgfpathlineto{\pgfqpoint{1.521509in}{10.047925in}}%
\pgfpathlineto{\pgfqpoint{1.521509in}{9.960189in}}%
\pgfpathlineto{\pgfqpoint{1.433773in}{9.960189in}}%
\pgfpathlineto{\pgfqpoint{1.433773in}{10.047925in}}%
\pgfusepath{stroke,fill}%
\end{pgfscope}%
\begin{pgfscope}%
\pgfpathrectangle{\pgfqpoint{0.380943in}{9.960189in}}{\pgfqpoint{4.650000in}{0.614151in}}%
\pgfusepath{clip}%
\pgfsetbuttcap%
\pgfsetroundjoin%
\definecolor{currentfill}{rgb}{1.000000,1.000000,0.929412}%
\pgfsetfillcolor{currentfill}%
\pgfsetlinewidth{0.250937pt}%
\definecolor{currentstroke}{rgb}{1.000000,1.000000,1.000000}%
\pgfsetstrokecolor{currentstroke}%
\pgfsetdash{}{0pt}%
\pgfpathmoveto{\pgfqpoint{1.521509in}{10.047925in}}%
\pgfpathlineto{\pgfqpoint{1.609245in}{10.047925in}}%
\pgfpathlineto{\pgfqpoint{1.609245in}{9.960189in}}%
\pgfpathlineto{\pgfqpoint{1.521509in}{9.960189in}}%
\pgfpathlineto{\pgfqpoint{1.521509in}{10.047925in}}%
\pgfusepath{stroke,fill}%
\end{pgfscope}%
\begin{pgfscope}%
\pgfpathrectangle{\pgfqpoint{0.380943in}{9.960189in}}{\pgfqpoint{4.650000in}{0.614151in}}%
\pgfusepath{clip}%
\pgfsetbuttcap%
\pgfsetroundjoin%
\definecolor{currentfill}{rgb}{1.000000,1.000000,0.929412}%
\pgfsetfillcolor{currentfill}%
\pgfsetlinewidth{0.250937pt}%
\definecolor{currentstroke}{rgb}{1.000000,1.000000,1.000000}%
\pgfsetstrokecolor{currentstroke}%
\pgfsetdash{}{0pt}%
\pgfpathmoveto{\pgfqpoint{1.609245in}{10.047925in}}%
\pgfpathlineto{\pgfqpoint{1.696981in}{10.047925in}}%
\pgfpathlineto{\pgfqpoint{1.696981in}{9.960189in}}%
\pgfpathlineto{\pgfqpoint{1.609245in}{9.960189in}}%
\pgfpathlineto{\pgfqpoint{1.609245in}{10.047925in}}%
\pgfusepath{stroke,fill}%
\end{pgfscope}%
\begin{pgfscope}%
\pgfpathrectangle{\pgfqpoint{0.380943in}{9.960189in}}{\pgfqpoint{4.650000in}{0.614151in}}%
\pgfusepath{clip}%
\pgfsetbuttcap%
\pgfsetroundjoin%
\definecolor{currentfill}{rgb}{1.000000,1.000000,0.929412}%
\pgfsetfillcolor{currentfill}%
\pgfsetlinewidth{0.250937pt}%
\definecolor{currentstroke}{rgb}{1.000000,1.000000,1.000000}%
\pgfsetstrokecolor{currentstroke}%
\pgfsetdash{}{0pt}%
\pgfpathmoveto{\pgfqpoint{1.696981in}{10.047925in}}%
\pgfpathlineto{\pgfqpoint{1.784717in}{10.047925in}}%
\pgfpathlineto{\pgfqpoint{1.784717in}{9.960189in}}%
\pgfpathlineto{\pgfqpoint{1.696981in}{9.960189in}}%
\pgfpathlineto{\pgfqpoint{1.696981in}{10.047925in}}%
\pgfusepath{stroke,fill}%
\end{pgfscope}%
\begin{pgfscope}%
\pgfpathrectangle{\pgfqpoint{0.380943in}{9.960189in}}{\pgfqpoint{4.650000in}{0.614151in}}%
\pgfusepath{clip}%
\pgfsetbuttcap%
\pgfsetroundjoin%
\definecolor{currentfill}{rgb}{1.000000,1.000000,0.929412}%
\pgfsetfillcolor{currentfill}%
\pgfsetlinewidth{0.250937pt}%
\definecolor{currentstroke}{rgb}{1.000000,1.000000,1.000000}%
\pgfsetstrokecolor{currentstroke}%
\pgfsetdash{}{0pt}%
\pgfpathmoveto{\pgfqpoint{1.784717in}{10.047925in}}%
\pgfpathlineto{\pgfqpoint{1.872452in}{10.047925in}}%
\pgfpathlineto{\pgfqpoint{1.872452in}{9.960189in}}%
\pgfpathlineto{\pgfqpoint{1.784717in}{9.960189in}}%
\pgfpathlineto{\pgfqpoint{1.784717in}{10.047925in}}%
\pgfusepath{stroke,fill}%
\end{pgfscope}%
\begin{pgfscope}%
\pgfpathrectangle{\pgfqpoint{0.380943in}{9.960189in}}{\pgfqpoint{4.650000in}{0.614151in}}%
\pgfusepath{clip}%
\pgfsetbuttcap%
\pgfsetroundjoin%
\definecolor{currentfill}{rgb}{1.000000,1.000000,0.929412}%
\pgfsetfillcolor{currentfill}%
\pgfsetlinewidth{0.250937pt}%
\definecolor{currentstroke}{rgb}{1.000000,1.000000,1.000000}%
\pgfsetstrokecolor{currentstroke}%
\pgfsetdash{}{0pt}%
\pgfpathmoveto{\pgfqpoint{1.872452in}{10.047925in}}%
\pgfpathlineto{\pgfqpoint{1.960188in}{10.047925in}}%
\pgfpathlineto{\pgfqpoint{1.960188in}{9.960189in}}%
\pgfpathlineto{\pgfqpoint{1.872452in}{9.960189in}}%
\pgfpathlineto{\pgfqpoint{1.872452in}{10.047925in}}%
\pgfusepath{stroke,fill}%
\end{pgfscope}%
\begin{pgfscope}%
\pgfpathrectangle{\pgfqpoint{0.380943in}{9.960189in}}{\pgfqpoint{4.650000in}{0.614151in}}%
\pgfusepath{clip}%
\pgfsetbuttcap%
\pgfsetroundjoin%
\definecolor{currentfill}{rgb}{1.000000,1.000000,0.929412}%
\pgfsetfillcolor{currentfill}%
\pgfsetlinewidth{0.250937pt}%
\definecolor{currentstroke}{rgb}{1.000000,1.000000,1.000000}%
\pgfsetstrokecolor{currentstroke}%
\pgfsetdash{}{0pt}%
\pgfpathmoveto{\pgfqpoint{1.960188in}{10.047925in}}%
\pgfpathlineto{\pgfqpoint{2.047924in}{10.047925in}}%
\pgfpathlineto{\pgfqpoint{2.047924in}{9.960189in}}%
\pgfpathlineto{\pgfqpoint{1.960188in}{9.960189in}}%
\pgfpathlineto{\pgfqpoint{1.960188in}{10.047925in}}%
\pgfusepath{stroke,fill}%
\end{pgfscope}%
\begin{pgfscope}%
\pgfpathrectangle{\pgfqpoint{0.380943in}{9.960189in}}{\pgfqpoint{4.650000in}{0.614151in}}%
\pgfusepath{clip}%
\pgfsetbuttcap%
\pgfsetroundjoin%
\definecolor{currentfill}{rgb}{1.000000,1.000000,0.929412}%
\pgfsetfillcolor{currentfill}%
\pgfsetlinewidth{0.250937pt}%
\definecolor{currentstroke}{rgb}{1.000000,1.000000,1.000000}%
\pgfsetstrokecolor{currentstroke}%
\pgfsetdash{}{0pt}%
\pgfpathmoveto{\pgfqpoint{2.047924in}{10.047925in}}%
\pgfpathlineto{\pgfqpoint{2.135660in}{10.047925in}}%
\pgfpathlineto{\pgfqpoint{2.135660in}{9.960189in}}%
\pgfpathlineto{\pgfqpoint{2.047924in}{9.960189in}}%
\pgfpathlineto{\pgfqpoint{2.047924in}{10.047925in}}%
\pgfusepath{stroke,fill}%
\end{pgfscope}%
\begin{pgfscope}%
\pgfpathrectangle{\pgfqpoint{0.380943in}{9.960189in}}{\pgfqpoint{4.650000in}{0.614151in}}%
\pgfusepath{clip}%
\pgfsetbuttcap%
\pgfsetroundjoin%
\definecolor{currentfill}{rgb}{1.000000,1.000000,0.929412}%
\pgfsetfillcolor{currentfill}%
\pgfsetlinewidth{0.250937pt}%
\definecolor{currentstroke}{rgb}{1.000000,1.000000,1.000000}%
\pgfsetstrokecolor{currentstroke}%
\pgfsetdash{}{0pt}%
\pgfpathmoveto{\pgfqpoint{2.135660in}{10.047925in}}%
\pgfpathlineto{\pgfqpoint{2.223396in}{10.047925in}}%
\pgfpathlineto{\pgfqpoint{2.223396in}{9.960189in}}%
\pgfpathlineto{\pgfqpoint{2.135660in}{9.960189in}}%
\pgfpathlineto{\pgfqpoint{2.135660in}{10.047925in}}%
\pgfusepath{stroke,fill}%
\end{pgfscope}%
\begin{pgfscope}%
\pgfpathrectangle{\pgfqpoint{0.380943in}{9.960189in}}{\pgfqpoint{4.650000in}{0.614151in}}%
\pgfusepath{clip}%
\pgfsetbuttcap%
\pgfsetroundjoin%
\definecolor{currentfill}{rgb}{1.000000,1.000000,0.929412}%
\pgfsetfillcolor{currentfill}%
\pgfsetlinewidth{0.250937pt}%
\definecolor{currentstroke}{rgb}{1.000000,1.000000,1.000000}%
\pgfsetstrokecolor{currentstroke}%
\pgfsetdash{}{0pt}%
\pgfpathmoveto{\pgfqpoint{2.223396in}{10.047925in}}%
\pgfpathlineto{\pgfqpoint{2.311132in}{10.047925in}}%
\pgfpathlineto{\pgfqpoint{2.311132in}{9.960189in}}%
\pgfpathlineto{\pgfqpoint{2.223396in}{9.960189in}}%
\pgfpathlineto{\pgfqpoint{2.223396in}{10.047925in}}%
\pgfusepath{stroke,fill}%
\end{pgfscope}%
\begin{pgfscope}%
\pgfpathrectangle{\pgfqpoint{0.380943in}{9.960189in}}{\pgfqpoint{4.650000in}{0.614151in}}%
\pgfusepath{clip}%
\pgfsetbuttcap%
\pgfsetroundjoin%
\definecolor{currentfill}{rgb}{1.000000,1.000000,0.929412}%
\pgfsetfillcolor{currentfill}%
\pgfsetlinewidth{0.250937pt}%
\definecolor{currentstroke}{rgb}{1.000000,1.000000,1.000000}%
\pgfsetstrokecolor{currentstroke}%
\pgfsetdash{}{0pt}%
\pgfpathmoveto{\pgfqpoint{2.311132in}{10.047925in}}%
\pgfpathlineto{\pgfqpoint{2.398868in}{10.047925in}}%
\pgfpathlineto{\pgfqpoint{2.398868in}{9.960189in}}%
\pgfpathlineto{\pgfqpoint{2.311132in}{9.960189in}}%
\pgfpathlineto{\pgfqpoint{2.311132in}{10.047925in}}%
\pgfusepath{stroke,fill}%
\end{pgfscope}%
\begin{pgfscope}%
\pgfpathrectangle{\pgfqpoint{0.380943in}{9.960189in}}{\pgfqpoint{4.650000in}{0.614151in}}%
\pgfusepath{clip}%
\pgfsetbuttcap%
\pgfsetroundjoin%
\definecolor{currentfill}{rgb}{1.000000,1.000000,0.929412}%
\pgfsetfillcolor{currentfill}%
\pgfsetlinewidth{0.250937pt}%
\definecolor{currentstroke}{rgb}{1.000000,1.000000,1.000000}%
\pgfsetstrokecolor{currentstroke}%
\pgfsetdash{}{0pt}%
\pgfpathmoveto{\pgfqpoint{2.398868in}{10.047925in}}%
\pgfpathlineto{\pgfqpoint{2.486603in}{10.047925in}}%
\pgfpathlineto{\pgfqpoint{2.486603in}{9.960189in}}%
\pgfpathlineto{\pgfqpoint{2.398868in}{9.960189in}}%
\pgfpathlineto{\pgfqpoint{2.398868in}{10.047925in}}%
\pgfusepath{stroke,fill}%
\end{pgfscope}%
\begin{pgfscope}%
\pgfpathrectangle{\pgfqpoint{0.380943in}{9.960189in}}{\pgfqpoint{4.650000in}{0.614151in}}%
\pgfusepath{clip}%
\pgfsetbuttcap%
\pgfsetroundjoin%
\definecolor{currentfill}{rgb}{1.000000,1.000000,0.929412}%
\pgfsetfillcolor{currentfill}%
\pgfsetlinewidth{0.250937pt}%
\definecolor{currentstroke}{rgb}{1.000000,1.000000,1.000000}%
\pgfsetstrokecolor{currentstroke}%
\pgfsetdash{}{0pt}%
\pgfpathmoveto{\pgfqpoint{2.486603in}{10.047925in}}%
\pgfpathlineto{\pgfqpoint{2.574339in}{10.047925in}}%
\pgfpathlineto{\pgfqpoint{2.574339in}{9.960189in}}%
\pgfpathlineto{\pgfqpoint{2.486603in}{9.960189in}}%
\pgfpathlineto{\pgfqpoint{2.486603in}{10.047925in}}%
\pgfusepath{stroke,fill}%
\end{pgfscope}%
\begin{pgfscope}%
\pgfpathrectangle{\pgfqpoint{0.380943in}{9.960189in}}{\pgfqpoint{4.650000in}{0.614151in}}%
\pgfusepath{clip}%
\pgfsetbuttcap%
\pgfsetroundjoin%
\definecolor{currentfill}{rgb}{1.000000,1.000000,0.929412}%
\pgfsetfillcolor{currentfill}%
\pgfsetlinewidth{0.250937pt}%
\definecolor{currentstroke}{rgb}{1.000000,1.000000,1.000000}%
\pgfsetstrokecolor{currentstroke}%
\pgfsetdash{}{0pt}%
\pgfpathmoveto{\pgfqpoint{2.574339in}{10.047925in}}%
\pgfpathlineto{\pgfqpoint{2.662075in}{10.047925in}}%
\pgfpathlineto{\pgfqpoint{2.662075in}{9.960189in}}%
\pgfpathlineto{\pgfqpoint{2.574339in}{9.960189in}}%
\pgfpathlineto{\pgfqpoint{2.574339in}{10.047925in}}%
\pgfusepath{stroke,fill}%
\end{pgfscope}%
\begin{pgfscope}%
\pgfpathrectangle{\pgfqpoint{0.380943in}{9.960189in}}{\pgfqpoint{4.650000in}{0.614151in}}%
\pgfusepath{clip}%
\pgfsetbuttcap%
\pgfsetroundjoin%
\definecolor{currentfill}{rgb}{1.000000,1.000000,0.929412}%
\pgfsetfillcolor{currentfill}%
\pgfsetlinewidth{0.250937pt}%
\definecolor{currentstroke}{rgb}{1.000000,1.000000,1.000000}%
\pgfsetstrokecolor{currentstroke}%
\pgfsetdash{}{0pt}%
\pgfpathmoveto{\pgfqpoint{2.662075in}{10.047925in}}%
\pgfpathlineto{\pgfqpoint{2.749811in}{10.047925in}}%
\pgfpathlineto{\pgfqpoint{2.749811in}{9.960189in}}%
\pgfpathlineto{\pgfqpoint{2.662075in}{9.960189in}}%
\pgfpathlineto{\pgfqpoint{2.662075in}{10.047925in}}%
\pgfusepath{stroke,fill}%
\end{pgfscope}%
\begin{pgfscope}%
\pgfpathrectangle{\pgfqpoint{0.380943in}{9.960189in}}{\pgfqpoint{4.650000in}{0.614151in}}%
\pgfusepath{clip}%
\pgfsetbuttcap%
\pgfsetroundjoin%
\definecolor{currentfill}{rgb}{1.000000,1.000000,0.929412}%
\pgfsetfillcolor{currentfill}%
\pgfsetlinewidth{0.250937pt}%
\definecolor{currentstroke}{rgb}{1.000000,1.000000,1.000000}%
\pgfsetstrokecolor{currentstroke}%
\pgfsetdash{}{0pt}%
\pgfpathmoveto{\pgfqpoint{2.749811in}{10.047925in}}%
\pgfpathlineto{\pgfqpoint{2.837547in}{10.047925in}}%
\pgfpathlineto{\pgfqpoint{2.837547in}{9.960189in}}%
\pgfpathlineto{\pgfqpoint{2.749811in}{9.960189in}}%
\pgfpathlineto{\pgfqpoint{2.749811in}{10.047925in}}%
\pgfusepath{stroke,fill}%
\end{pgfscope}%
\begin{pgfscope}%
\pgfpathrectangle{\pgfqpoint{0.380943in}{9.960189in}}{\pgfqpoint{4.650000in}{0.614151in}}%
\pgfusepath{clip}%
\pgfsetbuttcap%
\pgfsetroundjoin%
\definecolor{currentfill}{rgb}{1.000000,1.000000,0.929412}%
\pgfsetfillcolor{currentfill}%
\pgfsetlinewidth{0.250937pt}%
\definecolor{currentstroke}{rgb}{1.000000,1.000000,1.000000}%
\pgfsetstrokecolor{currentstroke}%
\pgfsetdash{}{0pt}%
\pgfpathmoveto{\pgfqpoint{2.837547in}{10.047925in}}%
\pgfpathlineto{\pgfqpoint{2.925283in}{10.047925in}}%
\pgfpathlineto{\pgfqpoint{2.925283in}{9.960189in}}%
\pgfpathlineto{\pgfqpoint{2.837547in}{9.960189in}}%
\pgfpathlineto{\pgfqpoint{2.837547in}{10.047925in}}%
\pgfusepath{stroke,fill}%
\end{pgfscope}%
\begin{pgfscope}%
\pgfpathrectangle{\pgfqpoint{0.380943in}{9.960189in}}{\pgfqpoint{4.650000in}{0.614151in}}%
\pgfusepath{clip}%
\pgfsetbuttcap%
\pgfsetroundjoin%
\definecolor{currentfill}{rgb}{1.000000,1.000000,0.929412}%
\pgfsetfillcolor{currentfill}%
\pgfsetlinewidth{0.250937pt}%
\definecolor{currentstroke}{rgb}{1.000000,1.000000,1.000000}%
\pgfsetstrokecolor{currentstroke}%
\pgfsetdash{}{0pt}%
\pgfpathmoveto{\pgfqpoint{2.925283in}{10.047925in}}%
\pgfpathlineto{\pgfqpoint{3.013019in}{10.047925in}}%
\pgfpathlineto{\pgfqpoint{3.013019in}{9.960189in}}%
\pgfpathlineto{\pgfqpoint{2.925283in}{9.960189in}}%
\pgfpathlineto{\pgfqpoint{2.925283in}{10.047925in}}%
\pgfusepath{stroke,fill}%
\end{pgfscope}%
\begin{pgfscope}%
\pgfpathrectangle{\pgfqpoint{0.380943in}{9.960189in}}{\pgfqpoint{4.650000in}{0.614151in}}%
\pgfusepath{clip}%
\pgfsetbuttcap%
\pgfsetroundjoin%
\definecolor{currentfill}{rgb}{1.000000,1.000000,0.929412}%
\pgfsetfillcolor{currentfill}%
\pgfsetlinewidth{0.250937pt}%
\definecolor{currentstroke}{rgb}{1.000000,1.000000,1.000000}%
\pgfsetstrokecolor{currentstroke}%
\pgfsetdash{}{0pt}%
\pgfpathmoveto{\pgfqpoint{3.013019in}{10.047925in}}%
\pgfpathlineto{\pgfqpoint{3.100754in}{10.047925in}}%
\pgfpathlineto{\pgfqpoint{3.100754in}{9.960189in}}%
\pgfpathlineto{\pgfqpoint{3.013019in}{9.960189in}}%
\pgfpathlineto{\pgfqpoint{3.013019in}{10.047925in}}%
\pgfusepath{stroke,fill}%
\end{pgfscope}%
\begin{pgfscope}%
\pgfpathrectangle{\pgfqpoint{0.380943in}{9.960189in}}{\pgfqpoint{4.650000in}{0.614151in}}%
\pgfusepath{clip}%
\pgfsetbuttcap%
\pgfsetroundjoin%
\definecolor{currentfill}{rgb}{1.000000,1.000000,0.929412}%
\pgfsetfillcolor{currentfill}%
\pgfsetlinewidth{0.250937pt}%
\definecolor{currentstroke}{rgb}{1.000000,1.000000,1.000000}%
\pgfsetstrokecolor{currentstroke}%
\pgfsetdash{}{0pt}%
\pgfpathmoveto{\pgfqpoint{3.100754in}{10.047925in}}%
\pgfpathlineto{\pgfqpoint{3.188490in}{10.047925in}}%
\pgfpathlineto{\pgfqpoint{3.188490in}{9.960189in}}%
\pgfpathlineto{\pgfqpoint{3.100754in}{9.960189in}}%
\pgfpathlineto{\pgfqpoint{3.100754in}{10.047925in}}%
\pgfusepath{stroke,fill}%
\end{pgfscope}%
\begin{pgfscope}%
\pgfpathrectangle{\pgfqpoint{0.380943in}{9.960189in}}{\pgfqpoint{4.650000in}{0.614151in}}%
\pgfusepath{clip}%
\pgfsetbuttcap%
\pgfsetroundjoin%
\definecolor{currentfill}{rgb}{1.000000,1.000000,0.929412}%
\pgfsetfillcolor{currentfill}%
\pgfsetlinewidth{0.250937pt}%
\definecolor{currentstroke}{rgb}{1.000000,1.000000,1.000000}%
\pgfsetstrokecolor{currentstroke}%
\pgfsetdash{}{0pt}%
\pgfpathmoveto{\pgfqpoint{3.188490in}{10.047925in}}%
\pgfpathlineto{\pgfqpoint{3.276226in}{10.047925in}}%
\pgfpathlineto{\pgfqpoint{3.276226in}{9.960189in}}%
\pgfpathlineto{\pgfqpoint{3.188490in}{9.960189in}}%
\pgfpathlineto{\pgfqpoint{3.188490in}{10.047925in}}%
\pgfusepath{stroke,fill}%
\end{pgfscope}%
\begin{pgfscope}%
\pgfpathrectangle{\pgfqpoint{0.380943in}{9.960189in}}{\pgfqpoint{4.650000in}{0.614151in}}%
\pgfusepath{clip}%
\pgfsetbuttcap%
\pgfsetroundjoin%
\definecolor{currentfill}{rgb}{1.000000,1.000000,0.929412}%
\pgfsetfillcolor{currentfill}%
\pgfsetlinewidth{0.250937pt}%
\definecolor{currentstroke}{rgb}{1.000000,1.000000,1.000000}%
\pgfsetstrokecolor{currentstroke}%
\pgfsetdash{}{0pt}%
\pgfpathmoveto{\pgfqpoint{3.276226in}{10.047925in}}%
\pgfpathlineto{\pgfqpoint{3.363962in}{10.047925in}}%
\pgfpathlineto{\pgfqpoint{3.363962in}{9.960189in}}%
\pgfpathlineto{\pgfqpoint{3.276226in}{9.960189in}}%
\pgfpathlineto{\pgfqpoint{3.276226in}{10.047925in}}%
\pgfusepath{stroke,fill}%
\end{pgfscope}%
\begin{pgfscope}%
\pgfpathrectangle{\pgfqpoint{0.380943in}{9.960189in}}{\pgfqpoint{4.650000in}{0.614151in}}%
\pgfusepath{clip}%
\pgfsetbuttcap%
\pgfsetroundjoin%
\definecolor{currentfill}{rgb}{1.000000,1.000000,0.929412}%
\pgfsetfillcolor{currentfill}%
\pgfsetlinewidth{0.250937pt}%
\definecolor{currentstroke}{rgb}{1.000000,1.000000,1.000000}%
\pgfsetstrokecolor{currentstroke}%
\pgfsetdash{}{0pt}%
\pgfpathmoveto{\pgfqpoint{3.363962in}{10.047925in}}%
\pgfpathlineto{\pgfqpoint{3.451698in}{10.047925in}}%
\pgfpathlineto{\pgfqpoint{3.451698in}{9.960189in}}%
\pgfpathlineto{\pgfqpoint{3.363962in}{9.960189in}}%
\pgfpathlineto{\pgfqpoint{3.363962in}{10.047925in}}%
\pgfusepath{stroke,fill}%
\end{pgfscope}%
\begin{pgfscope}%
\pgfpathrectangle{\pgfqpoint{0.380943in}{9.960189in}}{\pgfqpoint{4.650000in}{0.614151in}}%
\pgfusepath{clip}%
\pgfsetbuttcap%
\pgfsetroundjoin%
\definecolor{currentfill}{rgb}{1.000000,1.000000,0.929412}%
\pgfsetfillcolor{currentfill}%
\pgfsetlinewidth{0.250937pt}%
\definecolor{currentstroke}{rgb}{1.000000,1.000000,1.000000}%
\pgfsetstrokecolor{currentstroke}%
\pgfsetdash{}{0pt}%
\pgfpathmoveto{\pgfqpoint{3.451698in}{10.047925in}}%
\pgfpathlineto{\pgfqpoint{3.539434in}{10.047925in}}%
\pgfpathlineto{\pgfqpoint{3.539434in}{9.960189in}}%
\pgfpathlineto{\pgfqpoint{3.451698in}{9.960189in}}%
\pgfpathlineto{\pgfqpoint{3.451698in}{10.047925in}}%
\pgfusepath{stroke,fill}%
\end{pgfscope}%
\begin{pgfscope}%
\pgfpathrectangle{\pgfqpoint{0.380943in}{9.960189in}}{\pgfqpoint{4.650000in}{0.614151in}}%
\pgfusepath{clip}%
\pgfsetbuttcap%
\pgfsetroundjoin%
\definecolor{currentfill}{rgb}{1.000000,1.000000,0.929412}%
\pgfsetfillcolor{currentfill}%
\pgfsetlinewidth{0.250937pt}%
\definecolor{currentstroke}{rgb}{1.000000,1.000000,1.000000}%
\pgfsetstrokecolor{currentstroke}%
\pgfsetdash{}{0pt}%
\pgfpathmoveto{\pgfqpoint{3.539434in}{10.047925in}}%
\pgfpathlineto{\pgfqpoint{3.627169in}{10.047925in}}%
\pgfpathlineto{\pgfqpoint{3.627169in}{9.960189in}}%
\pgfpathlineto{\pgfqpoint{3.539434in}{9.960189in}}%
\pgfpathlineto{\pgfqpoint{3.539434in}{10.047925in}}%
\pgfusepath{stroke,fill}%
\end{pgfscope}%
\begin{pgfscope}%
\pgfpathrectangle{\pgfqpoint{0.380943in}{9.960189in}}{\pgfqpoint{4.650000in}{0.614151in}}%
\pgfusepath{clip}%
\pgfsetbuttcap%
\pgfsetroundjoin%
\definecolor{currentfill}{rgb}{1.000000,1.000000,0.929412}%
\pgfsetfillcolor{currentfill}%
\pgfsetlinewidth{0.250937pt}%
\definecolor{currentstroke}{rgb}{1.000000,1.000000,1.000000}%
\pgfsetstrokecolor{currentstroke}%
\pgfsetdash{}{0pt}%
\pgfpathmoveto{\pgfqpoint{3.627169in}{10.047925in}}%
\pgfpathlineto{\pgfqpoint{3.714905in}{10.047925in}}%
\pgfpathlineto{\pgfqpoint{3.714905in}{9.960189in}}%
\pgfpathlineto{\pgfqpoint{3.627169in}{9.960189in}}%
\pgfpathlineto{\pgfqpoint{3.627169in}{10.047925in}}%
\pgfusepath{stroke,fill}%
\end{pgfscope}%
\begin{pgfscope}%
\pgfpathrectangle{\pgfqpoint{0.380943in}{9.960189in}}{\pgfqpoint{4.650000in}{0.614151in}}%
\pgfusepath{clip}%
\pgfsetbuttcap%
\pgfsetroundjoin%
\definecolor{currentfill}{rgb}{1.000000,1.000000,0.929412}%
\pgfsetfillcolor{currentfill}%
\pgfsetlinewidth{0.250937pt}%
\definecolor{currentstroke}{rgb}{1.000000,1.000000,1.000000}%
\pgfsetstrokecolor{currentstroke}%
\pgfsetdash{}{0pt}%
\pgfpathmoveto{\pgfqpoint{3.714905in}{10.047925in}}%
\pgfpathlineto{\pgfqpoint{3.802641in}{10.047925in}}%
\pgfpathlineto{\pgfqpoint{3.802641in}{9.960189in}}%
\pgfpathlineto{\pgfqpoint{3.714905in}{9.960189in}}%
\pgfpathlineto{\pgfqpoint{3.714905in}{10.047925in}}%
\pgfusepath{stroke,fill}%
\end{pgfscope}%
\begin{pgfscope}%
\pgfpathrectangle{\pgfqpoint{0.380943in}{9.960189in}}{\pgfqpoint{4.650000in}{0.614151in}}%
\pgfusepath{clip}%
\pgfsetbuttcap%
\pgfsetroundjoin%
\definecolor{currentfill}{rgb}{1.000000,1.000000,0.929412}%
\pgfsetfillcolor{currentfill}%
\pgfsetlinewidth{0.250937pt}%
\definecolor{currentstroke}{rgb}{1.000000,1.000000,1.000000}%
\pgfsetstrokecolor{currentstroke}%
\pgfsetdash{}{0pt}%
\pgfpathmoveto{\pgfqpoint{3.802641in}{10.047925in}}%
\pgfpathlineto{\pgfqpoint{3.890377in}{10.047925in}}%
\pgfpathlineto{\pgfqpoint{3.890377in}{9.960189in}}%
\pgfpathlineto{\pgfqpoint{3.802641in}{9.960189in}}%
\pgfpathlineto{\pgfqpoint{3.802641in}{10.047925in}}%
\pgfusepath{stroke,fill}%
\end{pgfscope}%
\begin{pgfscope}%
\pgfpathrectangle{\pgfqpoint{0.380943in}{9.960189in}}{\pgfqpoint{4.650000in}{0.614151in}}%
\pgfusepath{clip}%
\pgfsetbuttcap%
\pgfsetroundjoin%
\definecolor{currentfill}{rgb}{1.000000,1.000000,0.929412}%
\pgfsetfillcolor{currentfill}%
\pgfsetlinewidth{0.250937pt}%
\definecolor{currentstroke}{rgb}{1.000000,1.000000,1.000000}%
\pgfsetstrokecolor{currentstroke}%
\pgfsetdash{}{0pt}%
\pgfpathmoveto{\pgfqpoint{3.890377in}{10.047925in}}%
\pgfpathlineto{\pgfqpoint{3.978113in}{10.047925in}}%
\pgfpathlineto{\pgfqpoint{3.978113in}{9.960189in}}%
\pgfpathlineto{\pgfqpoint{3.890377in}{9.960189in}}%
\pgfpathlineto{\pgfqpoint{3.890377in}{10.047925in}}%
\pgfusepath{stroke,fill}%
\end{pgfscope}%
\begin{pgfscope}%
\pgfpathrectangle{\pgfqpoint{0.380943in}{9.960189in}}{\pgfqpoint{4.650000in}{0.614151in}}%
\pgfusepath{clip}%
\pgfsetbuttcap%
\pgfsetroundjoin%
\definecolor{currentfill}{rgb}{1.000000,1.000000,0.929412}%
\pgfsetfillcolor{currentfill}%
\pgfsetlinewidth{0.250937pt}%
\definecolor{currentstroke}{rgb}{1.000000,1.000000,1.000000}%
\pgfsetstrokecolor{currentstroke}%
\pgfsetdash{}{0pt}%
\pgfpathmoveto{\pgfqpoint{3.978113in}{10.047925in}}%
\pgfpathlineto{\pgfqpoint{4.065849in}{10.047925in}}%
\pgfpathlineto{\pgfqpoint{4.065849in}{9.960189in}}%
\pgfpathlineto{\pgfqpoint{3.978113in}{9.960189in}}%
\pgfpathlineto{\pgfqpoint{3.978113in}{10.047925in}}%
\pgfusepath{stroke,fill}%
\end{pgfscope}%
\begin{pgfscope}%
\pgfpathrectangle{\pgfqpoint{0.380943in}{9.960189in}}{\pgfqpoint{4.650000in}{0.614151in}}%
\pgfusepath{clip}%
\pgfsetbuttcap%
\pgfsetroundjoin%
\definecolor{currentfill}{rgb}{1.000000,1.000000,0.929412}%
\pgfsetfillcolor{currentfill}%
\pgfsetlinewidth{0.250937pt}%
\definecolor{currentstroke}{rgb}{1.000000,1.000000,1.000000}%
\pgfsetstrokecolor{currentstroke}%
\pgfsetdash{}{0pt}%
\pgfpathmoveto{\pgfqpoint{4.065849in}{10.047925in}}%
\pgfpathlineto{\pgfqpoint{4.153585in}{10.047925in}}%
\pgfpathlineto{\pgfqpoint{4.153585in}{9.960189in}}%
\pgfpathlineto{\pgfqpoint{4.065849in}{9.960189in}}%
\pgfpathlineto{\pgfqpoint{4.065849in}{10.047925in}}%
\pgfusepath{stroke,fill}%
\end{pgfscope}%
\begin{pgfscope}%
\pgfpathrectangle{\pgfqpoint{0.380943in}{9.960189in}}{\pgfqpoint{4.650000in}{0.614151in}}%
\pgfusepath{clip}%
\pgfsetbuttcap%
\pgfsetroundjoin%
\definecolor{currentfill}{rgb}{1.000000,1.000000,0.929412}%
\pgfsetfillcolor{currentfill}%
\pgfsetlinewidth{0.250937pt}%
\definecolor{currentstroke}{rgb}{1.000000,1.000000,1.000000}%
\pgfsetstrokecolor{currentstroke}%
\pgfsetdash{}{0pt}%
\pgfpathmoveto{\pgfqpoint{4.153585in}{10.047925in}}%
\pgfpathlineto{\pgfqpoint{4.241320in}{10.047925in}}%
\pgfpathlineto{\pgfqpoint{4.241320in}{9.960189in}}%
\pgfpathlineto{\pgfqpoint{4.153585in}{9.960189in}}%
\pgfpathlineto{\pgfqpoint{4.153585in}{10.047925in}}%
\pgfusepath{stroke,fill}%
\end{pgfscope}%
\begin{pgfscope}%
\pgfpathrectangle{\pgfqpoint{0.380943in}{9.960189in}}{\pgfqpoint{4.650000in}{0.614151in}}%
\pgfusepath{clip}%
\pgfsetbuttcap%
\pgfsetroundjoin%
\definecolor{currentfill}{rgb}{1.000000,1.000000,0.929412}%
\pgfsetfillcolor{currentfill}%
\pgfsetlinewidth{0.250937pt}%
\definecolor{currentstroke}{rgb}{1.000000,1.000000,1.000000}%
\pgfsetstrokecolor{currentstroke}%
\pgfsetdash{}{0pt}%
\pgfpathmoveto{\pgfqpoint{4.241320in}{10.047925in}}%
\pgfpathlineto{\pgfqpoint{4.329056in}{10.047925in}}%
\pgfpathlineto{\pgfqpoint{4.329056in}{9.960189in}}%
\pgfpathlineto{\pgfqpoint{4.241320in}{9.960189in}}%
\pgfpathlineto{\pgfqpoint{4.241320in}{10.047925in}}%
\pgfusepath{stroke,fill}%
\end{pgfscope}%
\begin{pgfscope}%
\pgfpathrectangle{\pgfqpoint{0.380943in}{9.960189in}}{\pgfqpoint{4.650000in}{0.614151in}}%
\pgfusepath{clip}%
\pgfsetbuttcap%
\pgfsetroundjoin%
\definecolor{currentfill}{rgb}{1.000000,1.000000,0.929412}%
\pgfsetfillcolor{currentfill}%
\pgfsetlinewidth{0.250937pt}%
\definecolor{currentstroke}{rgb}{1.000000,1.000000,1.000000}%
\pgfsetstrokecolor{currentstroke}%
\pgfsetdash{}{0pt}%
\pgfpathmoveto{\pgfqpoint{4.329056in}{10.047925in}}%
\pgfpathlineto{\pgfqpoint{4.416792in}{10.047925in}}%
\pgfpathlineto{\pgfqpoint{4.416792in}{9.960189in}}%
\pgfpathlineto{\pgfqpoint{4.329056in}{9.960189in}}%
\pgfpathlineto{\pgfqpoint{4.329056in}{10.047925in}}%
\pgfusepath{stroke,fill}%
\end{pgfscope}%
\begin{pgfscope}%
\pgfpathrectangle{\pgfqpoint{0.380943in}{9.960189in}}{\pgfqpoint{4.650000in}{0.614151in}}%
\pgfusepath{clip}%
\pgfsetbuttcap%
\pgfsetroundjoin%
\definecolor{currentfill}{rgb}{0.988466,0.980392,0.801384}%
\pgfsetfillcolor{currentfill}%
\pgfsetlinewidth{0.250937pt}%
\definecolor{currentstroke}{rgb}{1.000000,1.000000,1.000000}%
\pgfsetstrokecolor{currentstroke}%
\pgfsetdash{}{0pt}%
\pgfpathmoveto{\pgfqpoint{4.416792in}{10.047925in}}%
\pgfpathlineto{\pgfqpoint{4.504528in}{10.047925in}}%
\pgfpathlineto{\pgfqpoint{4.504528in}{9.960189in}}%
\pgfpathlineto{\pgfqpoint{4.416792in}{9.960189in}}%
\pgfpathlineto{\pgfqpoint{4.416792in}{10.047925in}}%
\pgfusepath{stroke,fill}%
\end{pgfscope}%
\begin{pgfscope}%
\pgfpathrectangle{\pgfqpoint{0.380943in}{9.960189in}}{\pgfqpoint{4.650000in}{0.614151in}}%
\pgfusepath{clip}%
\pgfsetbuttcap%
\pgfsetroundjoin%
\definecolor{currentfill}{rgb}{0.963091,0.919493,0.720185}%
\pgfsetfillcolor{currentfill}%
\pgfsetlinewidth{0.250937pt}%
\definecolor{currentstroke}{rgb}{1.000000,1.000000,1.000000}%
\pgfsetstrokecolor{currentstroke}%
\pgfsetdash{}{0pt}%
\pgfpathmoveto{\pgfqpoint{4.504528in}{10.047925in}}%
\pgfpathlineto{\pgfqpoint{4.592264in}{10.047925in}}%
\pgfpathlineto{\pgfqpoint{4.592264in}{9.960189in}}%
\pgfpathlineto{\pgfqpoint{4.504528in}{9.960189in}}%
\pgfpathlineto{\pgfqpoint{4.504528in}{10.047925in}}%
\pgfusepath{stroke,fill}%
\end{pgfscope}%
\begin{pgfscope}%
\pgfpathrectangle{\pgfqpoint{0.380943in}{9.960189in}}{\pgfqpoint{4.650000in}{0.614151in}}%
\pgfusepath{clip}%
\pgfsetbuttcap%
\pgfsetroundjoin%
\definecolor{currentfill}{rgb}{0.975087,0.857901,0.686044}%
\pgfsetfillcolor{currentfill}%
\pgfsetlinewidth{0.250937pt}%
\definecolor{currentstroke}{rgb}{1.000000,1.000000,1.000000}%
\pgfsetstrokecolor{currentstroke}%
\pgfsetdash{}{0pt}%
\pgfpathmoveto{\pgfqpoint{4.592264in}{10.047925in}}%
\pgfpathlineto{\pgfqpoint{4.680000in}{10.047925in}}%
\pgfpathlineto{\pgfqpoint{4.680000in}{9.960189in}}%
\pgfpathlineto{\pgfqpoint{4.592264in}{9.960189in}}%
\pgfpathlineto{\pgfqpoint{4.592264in}{10.047925in}}%
\pgfusepath{stroke,fill}%
\end{pgfscope}%
\begin{pgfscope}%
\pgfpathrectangle{\pgfqpoint{0.380943in}{9.960189in}}{\pgfqpoint{4.650000in}{0.614151in}}%
\pgfusepath{clip}%
\pgfsetbuttcap%
\pgfsetroundjoin%
\definecolor{currentfill}{rgb}{0.988466,0.980392,0.801384}%
\pgfsetfillcolor{currentfill}%
\pgfsetlinewidth{0.250937pt}%
\definecolor{currentstroke}{rgb}{1.000000,1.000000,1.000000}%
\pgfsetstrokecolor{currentstroke}%
\pgfsetdash{}{0pt}%
\pgfpathmoveto{\pgfqpoint{4.680000in}{10.047925in}}%
\pgfpathlineto{\pgfqpoint{4.767736in}{10.047925in}}%
\pgfpathlineto{\pgfqpoint{4.767736in}{9.960189in}}%
\pgfpathlineto{\pgfqpoint{4.680000in}{9.960189in}}%
\pgfpathlineto{\pgfqpoint{4.680000in}{10.047925in}}%
\pgfusepath{stroke,fill}%
\end{pgfscope}%
\begin{pgfscope}%
\pgfpathrectangle{\pgfqpoint{0.380943in}{9.960189in}}{\pgfqpoint{4.650000in}{0.614151in}}%
\pgfusepath{clip}%
\pgfsetbuttcap%
\pgfsetroundjoin%
\definecolor{currentfill}{rgb}{0.963091,0.919493,0.720185}%
\pgfsetfillcolor{currentfill}%
\pgfsetlinewidth{0.250937pt}%
\definecolor{currentstroke}{rgb}{1.000000,1.000000,1.000000}%
\pgfsetstrokecolor{currentstroke}%
\pgfsetdash{}{0pt}%
\pgfpathmoveto{\pgfqpoint{4.767736in}{10.047925in}}%
\pgfpathlineto{\pgfqpoint{4.855471in}{10.047925in}}%
\pgfpathlineto{\pgfqpoint{4.855471in}{9.960189in}}%
\pgfpathlineto{\pgfqpoint{4.767736in}{9.960189in}}%
\pgfpathlineto{\pgfqpoint{4.767736in}{10.047925in}}%
\pgfusepath{stroke,fill}%
\end{pgfscope}%
\begin{pgfscope}%
\pgfpathrectangle{\pgfqpoint{0.380943in}{9.960189in}}{\pgfqpoint{4.650000in}{0.614151in}}%
\pgfusepath{clip}%
\pgfsetbuttcap%
\pgfsetroundjoin%
\definecolor{currentfill}{rgb}{1.000000,1.000000,0.865975}%
\pgfsetfillcolor{currentfill}%
\pgfsetlinewidth{0.250937pt}%
\definecolor{currentstroke}{rgb}{1.000000,1.000000,1.000000}%
\pgfsetstrokecolor{currentstroke}%
\pgfsetdash{}{0pt}%
\pgfpathmoveto{\pgfqpoint{4.855471in}{10.047925in}}%
\pgfpathlineto{\pgfqpoint{4.943207in}{10.047925in}}%
\pgfpathlineto{\pgfqpoint{4.943207in}{9.960189in}}%
\pgfpathlineto{\pgfqpoint{4.855471in}{9.960189in}}%
\pgfpathlineto{\pgfqpoint{4.855471in}{10.047925in}}%
\pgfusepath{stroke,fill}%
\end{pgfscope}%
\begin{pgfscope}%
\pgfpathrectangle{\pgfqpoint{0.380943in}{9.960189in}}{\pgfqpoint{4.650000in}{0.614151in}}%
\pgfusepath{clip}%
\pgfsetbuttcap%
\pgfsetroundjoin%
\pgfsetlinewidth{0.250937pt}%
\definecolor{currentstroke}{rgb}{1.000000,1.000000,1.000000}%
\pgfsetstrokecolor{currentstroke}%
\pgfsetdash{}{0pt}%
\pgfpathmoveto{\pgfqpoint{4.943207in}{10.047925in}}%
\pgfpathlineto{\pgfqpoint{5.030943in}{10.047925in}}%
\pgfpathlineto{\pgfqpoint{5.030943in}{9.960189in}}%
\pgfpathlineto{\pgfqpoint{4.943207in}{9.960189in}}%
\pgfpathlineto{\pgfqpoint{4.943207in}{10.047925in}}%
\pgfusepath{stroke}%
\end{pgfscope}%
\begin{pgfscope}%
\pgfsetbuttcap%
\pgfsetroundjoin%
\definecolor{currentfill}{rgb}{0.000000,0.000000,0.000000}%
\pgfsetfillcolor{currentfill}%
\pgfsetlinewidth{0.803000pt}%
\definecolor{currentstroke}{rgb}{0.000000,0.000000,0.000000}%
\pgfsetstrokecolor{currentstroke}%
\pgfsetdash{}{0pt}%
\pgfsys@defobject{currentmarker}{\pgfqpoint{0.000000in}{-0.048611in}}{\pgfqpoint{0.000000in}{0.000000in}}{%
\pgfpathmoveto{\pgfqpoint{0.000000in}{0.000000in}}%
\pgfpathlineto{\pgfqpoint{0.000000in}{-0.048611in}}%
\pgfusepath{stroke,fill}%
}%
\begin{pgfscope}%
\pgfsys@transformshift{0.600283in}{9.960189in}%
\pgfsys@useobject{currentmarker}{}%
\end{pgfscope}%
\end{pgfscope}%
\begin{pgfscope}%
\definecolor{textcolor}{rgb}{0.000000,0.000000,0.000000}%
\pgfsetstrokecolor{textcolor}%
\pgfsetfillcolor{textcolor}%
\pgftext[x=0.600283in,y=9.862967in,,top]{\color{textcolor}\rmfamily\fontsize{8.000000}{9.600000}\selectfont Jan}%
\end{pgfscope}%
\begin{pgfscope}%
\pgfsetbuttcap%
\pgfsetroundjoin%
\definecolor{currentfill}{rgb}{0.000000,0.000000,0.000000}%
\pgfsetfillcolor{currentfill}%
\pgfsetlinewidth{0.803000pt}%
\definecolor{currentstroke}{rgb}{0.000000,0.000000,0.000000}%
\pgfsetstrokecolor{currentstroke}%
\pgfsetdash{}{0pt}%
\pgfsys@defobject{currentmarker}{\pgfqpoint{0.000000in}{-0.048611in}}{\pgfqpoint{0.000000in}{0.000000in}}{%
\pgfpathmoveto{\pgfqpoint{0.000000in}{0.000000in}}%
\pgfpathlineto{\pgfqpoint{0.000000in}{-0.048611in}}%
\pgfusepath{stroke,fill}%
}%
\begin{pgfscope}%
\pgfsys@transformshift{1.038962in}{9.960189in}%
\pgfsys@useobject{currentmarker}{}%
\end{pgfscope}%
\end{pgfscope}%
\begin{pgfscope}%
\definecolor{textcolor}{rgb}{0.000000,0.000000,0.000000}%
\pgfsetstrokecolor{textcolor}%
\pgfsetfillcolor{textcolor}%
\pgftext[x=1.038962in,y=9.862967in,,top]{\color{textcolor}\rmfamily\fontsize{8.000000}{9.600000}\selectfont Feb}%
\end{pgfscope}%
\begin{pgfscope}%
\pgfsetbuttcap%
\pgfsetroundjoin%
\definecolor{currentfill}{rgb}{0.000000,0.000000,0.000000}%
\pgfsetfillcolor{currentfill}%
\pgfsetlinewidth{0.803000pt}%
\definecolor{currentstroke}{rgb}{0.000000,0.000000,0.000000}%
\pgfsetstrokecolor{currentstroke}%
\pgfsetdash{}{0pt}%
\pgfsys@defobject{currentmarker}{\pgfqpoint{0.000000in}{-0.048611in}}{\pgfqpoint{0.000000in}{0.000000in}}{%
\pgfpathmoveto{\pgfqpoint{0.000000in}{0.000000in}}%
\pgfpathlineto{\pgfqpoint{0.000000in}{-0.048611in}}%
\pgfusepath{stroke,fill}%
}%
\begin{pgfscope}%
\pgfsys@transformshift{1.389905in}{9.960189in}%
\pgfsys@useobject{currentmarker}{}%
\end{pgfscope}%
\end{pgfscope}%
\begin{pgfscope}%
\definecolor{textcolor}{rgb}{0.000000,0.000000,0.000000}%
\pgfsetstrokecolor{textcolor}%
\pgfsetfillcolor{textcolor}%
\pgftext[x=1.389905in,y=9.862967in,,top]{\color{textcolor}\rmfamily\fontsize{8.000000}{9.600000}\selectfont Mar}%
\end{pgfscope}%
\begin{pgfscope}%
\pgfsetbuttcap%
\pgfsetroundjoin%
\definecolor{currentfill}{rgb}{0.000000,0.000000,0.000000}%
\pgfsetfillcolor{currentfill}%
\pgfsetlinewidth{0.803000pt}%
\definecolor{currentstroke}{rgb}{0.000000,0.000000,0.000000}%
\pgfsetstrokecolor{currentstroke}%
\pgfsetdash{}{0pt}%
\pgfsys@defobject{currentmarker}{\pgfqpoint{0.000000in}{-0.048611in}}{\pgfqpoint{0.000000in}{0.000000in}}{%
\pgfpathmoveto{\pgfqpoint{0.000000in}{0.000000in}}%
\pgfpathlineto{\pgfqpoint{0.000000in}{-0.048611in}}%
\pgfusepath{stroke,fill}%
}%
\begin{pgfscope}%
\pgfsys@transformshift{1.740849in}{9.960189in}%
\pgfsys@useobject{currentmarker}{}%
\end{pgfscope}%
\end{pgfscope}%
\begin{pgfscope}%
\definecolor{textcolor}{rgb}{0.000000,0.000000,0.000000}%
\pgfsetstrokecolor{textcolor}%
\pgfsetfillcolor{textcolor}%
\pgftext[x=1.740849in,y=9.862967in,,top]{\color{textcolor}\rmfamily\fontsize{8.000000}{9.600000}\selectfont Apr}%
\end{pgfscope}%
\begin{pgfscope}%
\pgfsetbuttcap%
\pgfsetroundjoin%
\definecolor{currentfill}{rgb}{0.000000,0.000000,0.000000}%
\pgfsetfillcolor{currentfill}%
\pgfsetlinewidth{0.803000pt}%
\definecolor{currentstroke}{rgb}{0.000000,0.000000,0.000000}%
\pgfsetstrokecolor{currentstroke}%
\pgfsetdash{}{0pt}%
\pgfsys@defobject{currentmarker}{\pgfqpoint{0.000000in}{-0.048611in}}{\pgfqpoint{0.000000in}{0.000000in}}{%
\pgfpathmoveto{\pgfqpoint{0.000000in}{0.000000in}}%
\pgfpathlineto{\pgfqpoint{0.000000in}{-0.048611in}}%
\pgfusepath{stroke,fill}%
}%
\begin{pgfscope}%
\pgfsys@transformshift{2.135660in}{9.960189in}%
\pgfsys@useobject{currentmarker}{}%
\end{pgfscope}%
\end{pgfscope}%
\begin{pgfscope}%
\definecolor{textcolor}{rgb}{0.000000,0.000000,0.000000}%
\pgfsetstrokecolor{textcolor}%
\pgfsetfillcolor{textcolor}%
\pgftext[x=2.135660in,y=9.862967in,,top]{\color{textcolor}\rmfamily\fontsize{8.000000}{9.600000}\selectfont May}%
\end{pgfscope}%
\begin{pgfscope}%
\pgfsetbuttcap%
\pgfsetroundjoin%
\definecolor{currentfill}{rgb}{0.000000,0.000000,0.000000}%
\pgfsetfillcolor{currentfill}%
\pgfsetlinewidth{0.803000pt}%
\definecolor{currentstroke}{rgb}{0.000000,0.000000,0.000000}%
\pgfsetstrokecolor{currentstroke}%
\pgfsetdash{}{0pt}%
\pgfsys@defobject{currentmarker}{\pgfqpoint{0.000000in}{-0.048611in}}{\pgfqpoint{0.000000in}{0.000000in}}{%
\pgfpathmoveto{\pgfqpoint{0.000000in}{0.000000in}}%
\pgfpathlineto{\pgfqpoint{0.000000in}{-0.048611in}}%
\pgfusepath{stroke,fill}%
}%
\begin{pgfscope}%
\pgfsys@transformshift{2.530471in}{9.960189in}%
\pgfsys@useobject{currentmarker}{}%
\end{pgfscope}%
\end{pgfscope}%
\begin{pgfscope}%
\definecolor{textcolor}{rgb}{0.000000,0.000000,0.000000}%
\pgfsetstrokecolor{textcolor}%
\pgfsetfillcolor{textcolor}%
\pgftext[x=2.530471in,y=9.862967in,,top]{\color{textcolor}\rmfamily\fontsize{8.000000}{9.600000}\selectfont Jun}%
\end{pgfscope}%
\begin{pgfscope}%
\pgfsetbuttcap%
\pgfsetroundjoin%
\definecolor{currentfill}{rgb}{0.000000,0.000000,0.000000}%
\pgfsetfillcolor{currentfill}%
\pgfsetlinewidth{0.803000pt}%
\definecolor{currentstroke}{rgb}{0.000000,0.000000,0.000000}%
\pgfsetstrokecolor{currentstroke}%
\pgfsetdash{}{0pt}%
\pgfsys@defobject{currentmarker}{\pgfqpoint{0.000000in}{-0.048611in}}{\pgfqpoint{0.000000in}{0.000000in}}{%
\pgfpathmoveto{\pgfqpoint{0.000000in}{0.000000in}}%
\pgfpathlineto{\pgfqpoint{0.000000in}{-0.048611in}}%
\pgfusepath{stroke,fill}%
}%
\begin{pgfscope}%
\pgfsys@transformshift{2.881415in}{9.960189in}%
\pgfsys@useobject{currentmarker}{}%
\end{pgfscope}%
\end{pgfscope}%
\begin{pgfscope}%
\definecolor{textcolor}{rgb}{0.000000,0.000000,0.000000}%
\pgfsetstrokecolor{textcolor}%
\pgfsetfillcolor{textcolor}%
\pgftext[x=2.881415in,y=9.862967in,,top]{\color{textcolor}\rmfamily\fontsize{8.000000}{9.600000}\selectfont Jul}%
\end{pgfscope}%
\begin{pgfscope}%
\pgfsetbuttcap%
\pgfsetroundjoin%
\definecolor{currentfill}{rgb}{0.000000,0.000000,0.000000}%
\pgfsetfillcolor{currentfill}%
\pgfsetlinewidth{0.803000pt}%
\definecolor{currentstroke}{rgb}{0.000000,0.000000,0.000000}%
\pgfsetstrokecolor{currentstroke}%
\pgfsetdash{}{0pt}%
\pgfsys@defobject{currentmarker}{\pgfqpoint{0.000000in}{-0.048611in}}{\pgfqpoint{0.000000in}{0.000000in}}{%
\pgfpathmoveto{\pgfqpoint{0.000000in}{0.000000in}}%
\pgfpathlineto{\pgfqpoint{0.000000in}{-0.048611in}}%
\pgfusepath{stroke,fill}%
}%
\begin{pgfscope}%
\pgfsys@transformshift{3.320094in}{9.960189in}%
\pgfsys@useobject{currentmarker}{}%
\end{pgfscope}%
\end{pgfscope}%
\begin{pgfscope}%
\definecolor{textcolor}{rgb}{0.000000,0.000000,0.000000}%
\pgfsetstrokecolor{textcolor}%
\pgfsetfillcolor{textcolor}%
\pgftext[x=3.320094in,y=9.862967in,,top]{\color{textcolor}\rmfamily\fontsize{8.000000}{9.600000}\selectfont Aug}%
\end{pgfscope}%
\begin{pgfscope}%
\pgfsetbuttcap%
\pgfsetroundjoin%
\definecolor{currentfill}{rgb}{0.000000,0.000000,0.000000}%
\pgfsetfillcolor{currentfill}%
\pgfsetlinewidth{0.803000pt}%
\definecolor{currentstroke}{rgb}{0.000000,0.000000,0.000000}%
\pgfsetstrokecolor{currentstroke}%
\pgfsetdash{}{0pt}%
\pgfsys@defobject{currentmarker}{\pgfqpoint{0.000000in}{-0.048611in}}{\pgfqpoint{0.000000in}{0.000000in}}{%
\pgfpathmoveto{\pgfqpoint{0.000000in}{0.000000in}}%
\pgfpathlineto{\pgfqpoint{0.000000in}{-0.048611in}}%
\pgfusepath{stroke,fill}%
}%
\begin{pgfscope}%
\pgfsys@transformshift{3.671037in}{9.960189in}%
\pgfsys@useobject{currentmarker}{}%
\end{pgfscope}%
\end{pgfscope}%
\begin{pgfscope}%
\definecolor{textcolor}{rgb}{0.000000,0.000000,0.000000}%
\pgfsetstrokecolor{textcolor}%
\pgfsetfillcolor{textcolor}%
\pgftext[x=3.671037in,y=9.862967in,,top]{\color{textcolor}\rmfamily\fontsize{8.000000}{9.600000}\selectfont Sep}%
\end{pgfscope}%
\begin{pgfscope}%
\pgfsetbuttcap%
\pgfsetroundjoin%
\definecolor{currentfill}{rgb}{0.000000,0.000000,0.000000}%
\pgfsetfillcolor{currentfill}%
\pgfsetlinewidth{0.803000pt}%
\definecolor{currentstroke}{rgb}{0.000000,0.000000,0.000000}%
\pgfsetstrokecolor{currentstroke}%
\pgfsetdash{}{0pt}%
\pgfsys@defobject{currentmarker}{\pgfqpoint{0.000000in}{-0.048611in}}{\pgfqpoint{0.000000in}{0.000000in}}{%
\pgfpathmoveto{\pgfqpoint{0.000000in}{0.000000in}}%
\pgfpathlineto{\pgfqpoint{0.000000in}{-0.048611in}}%
\pgfusepath{stroke,fill}%
}%
\begin{pgfscope}%
\pgfsys@transformshift{4.065849in}{9.960189in}%
\pgfsys@useobject{currentmarker}{}%
\end{pgfscope}%
\end{pgfscope}%
\begin{pgfscope}%
\definecolor{textcolor}{rgb}{0.000000,0.000000,0.000000}%
\pgfsetstrokecolor{textcolor}%
\pgfsetfillcolor{textcolor}%
\pgftext[x=4.065849in,y=9.862967in,,top]{\color{textcolor}\rmfamily\fontsize{8.000000}{9.600000}\selectfont Oct}%
\end{pgfscope}%
\begin{pgfscope}%
\pgfsetbuttcap%
\pgfsetroundjoin%
\definecolor{currentfill}{rgb}{0.000000,0.000000,0.000000}%
\pgfsetfillcolor{currentfill}%
\pgfsetlinewidth{0.803000pt}%
\definecolor{currentstroke}{rgb}{0.000000,0.000000,0.000000}%
\pgfsetstrokecolor{currentstroke}%
\pgfsetdash{}{0pt}%
\pgfsys@defobject{currentmarker}{\pgfqpoint{0.000000in}{-0.048611in}}{\pgfqpoint{0.000000in}{0.000000in}}{%
\pgfpathmoveto{\pgfqpoint{0.000000in}{0.000000in}}%
\pgfpathlineto{\pgfqpoint{0.000000in}{-0.048611in}}%
\pgfusepath{stroke,fill}%
}%
\begin{pgfscope}%
\pgfsys@transformshift{4.460660in}{9.960189in}%
\pgfsys@useobject{currentmarker}{}%
\end{pgfscope}%
\end{pgfscope}%
\begin{pgfscope}%
\definecolor{textcolor}{rgb}{0.000000,0.000000,0.000000}%
\pgfsetstrokecolor{textcolor}%
\pgfsetfillcolor{textcolor}%
\pgftext[x=4.460660in,y=9.862967in,,top]{\color{textcolor}\rmfamily\fontsize{8.000000}{9.600000}\selectfont Nov}%
\end{pgfscope}%
\begin{pgfscope}%
\pgfsetbuttcap%
\pgfsetroundjoin%
\definecolor{currentfill}{rgb}{0.000000,0.000000,0.000000}%
\pgfsetfillcolor{currentfill}%
\pgfsetlinewidth{0.803000pt}%
\definecolor{currentstroke}{rgb}{0.000000,0.000000,0.000000}%
\pgfsetstrokecolor{currentstroke}%
\pgfsetdash{}{0pt}%
\pgfsys@defobject{currentmarker}{\pgfqpoint{0.000000in}{-0.048611in}}{\pgfqpoint{0.000000in}{0.000000in}}{%
\pgfpathmoveto{\pgfqpoint{0.000000in}{0.000000in}}%
\pgfpathlineto{\pgfqpoint{0.000000in}{-0.048611in}}%
\pgfusepath{stroke,fill}%
}%
\begin{pgfscope}%
\pgfsys@transformshift{4.811603in}{9.960189in}%
\pgfsys@useobject{currentmarker}{}%
\end{pgfscope}%
\end{pgfscope}%
\begin{pgfscope}%
\definecolor{textcolor}{rgb}{0.000000,0.000000,0.000000}%
\pgfsetstrokecolor{textcolor}%
\pgfsetfillcolor{textcolor}%
\pgftext[x=4.811603in,y=9.862967in,,top]{\color{textcolor}\rmfamily\fontsize{8.000000}{9.600000}\selectfont Dec}%
\end{pgfscope}%
\begin{pgfscope}%
\pgfsetbuttcap%
\pgfsetroundjoin%
\definecolor{currentfill}{rgb}{0.000000,0.000000,0.000000}%
\pgfsetfillcolor{currentfill}%
\pgfsetlinewidth{0.803000pt}%
\definecolor{currentstroke}{rgb}{0.000000,0.000000,0.000000}%
\pgfsetstrokecolor{currentstroke}%
\pgfsetdash{}{0pt}%
\pgfsys@defobject{currentmarker}{\pgfqpoint{-0.048611in}{0.000000in}}{\pgfqpoint{-0.000000in}{0.000000in}}{%
\pgfpathmoveto{\pgfqpoint{-0.000000in}{0.000000in}}%
\pgfpathlineto{\pgfqpoint{-0.048611in}{0.000000in}}%
\pgfusepath{stroke,fill}%
}%
\begin{pgfscope}%
\pgfsys@transformshift{0.380943in}{10.530472in}%
\pgfsys@useobject{currentmarker}{}%
\end{pgfscope}%
\end{pgfscope}%
\begin{pgfscope}%
\definecolor{textcolor}{rgb}{0.000000,0.000000,0.000000}%
\pgfsetstrokecolor{textcolor}%
\pgfsetfillcolor{textcolor}%
\pgftext[x=0.113117in, y=10.491892in, left, base]{\color{textcolor}\rmfamily\fontsize{8.000000}{9.600000}\selectfont M}%
\end{pgfscope}%
\begin{pgfscope}%
\pgfsetbuttcap%
\pgfsetroundjoin%
\definecolor{currentfill}{rgb}{0.000000,0.000000,0.000000}%
\pgfsetfillcolor{currentfill}%
\pgfsetlinewidth{0.803000pt}%
\definecolor{currentstroke}{rgb}{0.000000,0.000000,0.000000}%
\pgfsetstrokecolor{currentstroke}%
\pgfsetdash{}{0pt}%
\pgfsys@defobject{currentmarker}{\pgfqpoint{-0.048611in}{0.000000in}}{\pgfqpoint{-0.000000in}{0.000000in}}{%
\pgfpathmoveto{\pgfqpoint{-0.000000in}{0.000000in}}%
\pgfpathlineto{\pgfqpoint{-0.048611in}{0.000000in}}%
\pgfusepath{stroke,fill}%
}%
\begin{pgfscope}%
\pgfsys@transformshift{0.380943in}{10.442736in}%
\pgfsys@useobject{currentmarker}{}%
\end{pgfscope}%
\end{pgfscope}%
\begin{pgfscope}%
\definecolor{textcolor}{rgb}{0.000000,0.000000,0.000000}%
\pgfsetstrokecolor{textcolor}%
\pgfsetfillcolor{textcolor}%
\pgftext[x=0.135957in, y=10.404156in, left, base]{\color{textcolor}\rmfamily\fontsize{8.000000}{9.600000}\selectfont T}%
\end{pgfscope}%
\begin{pgfscope}%
\pgfsetbuttcap%
\pgfsetroundjoin%
\definecolor{currentfill}{rgb}{0.000000,0.000000,0.000000}%
\pgfsetfillcolor{currentfill}%
\pgfsetlinewidth{0.803000pt}%
\definecolor{currentstroke}{rgb}{0.000000,0.000000,0.000000}%
\pgfsetstrokecolor{currentstroke}%
\pgfsetdash{}{0pt}%
\pgfsys@defobject{currentmarker}{\pgfqpoint{-0.048611in}{0.000000in}}{\pgfqpoint{-0.000000in}{0.000000in}}{%
\pgfpathmoveto{\pgfqpoint{-0.000000in}{0.000000in}}%
\pgfpathlineto{\pgfqpoint{-0.048611in}{0.000000in}}%
\pgfusepath{stroke,fill}%
}%
\begin{pgfscope}%
\pgfsys@transformshift{0.380943in}{10.355000in}%
\pgfsys@useobject{currentmarker}{}%
\end{pgfscope}%
\end{pgfscope}%
\begin{pgfscope}%
\definecolor{textcolor}{rgb}{0.000000,0.000000,0.000000}%
\pgfsetstrokecolor{textcolor}%
\pgfsetfillcolor{textcolor}%
\pgftext[x=0.100000in, y=10.316420in, left, base]{\color{textcolor}\rmfamily\fontsize{8.000000}{9.600000}\selectfont W}%
\end{pgfscope}%
\begin{pgfscope}%
\pgfsetbuttcap%
\pgfsetroundjoin%
\definecolor{currentfill}{rgb}{0.000000,0.000000,0.000000}%
\pgfsetfillcolor{currentfill}%
\pgfsetlinewidth{0.803000pt}%
\definecolor{currentstroke}{rgb}{0.000000,0.000000,0.000000}%
\pgfsetstrokecolor{currentstroke}%
\pgfsetdash{}{0pt}%
\pgfsys@defobject{currentmarker}{\pgfqpoint{-0.048611in}{0.000000in}}{\pgfqpoint{-0.000000in}{0.000000in}}{%
\pgfpathmoveto{\pgfqpoint{-0.000000in}{0.000000in}}%
\pgfpathlineto{\pgfqpoint{-0.048611in}{0.000000in}}%
\pgfusepath{stroke,fill}%
}%
\begin{pgfscope}%
\pgfsys@transformshift{0.380943in}{10.267264in}%
\pgfsys@useobject{currentmarker}{}%
\end{pgfscope}%
\end{pgfscope}%
\begin{pgfscope}%
\definecolor{textcolor}{rgb}{0.000000,0.000000,0.000000}%
\pgfsetstrokecolor{textcolor}%
\pgfsetfillcolor{textcolor}%
\pgftext[x=0.135957in, y=10.228684in, left, base]{\color{textcolor}\rmfamily\fontsize{8.000000}{9.600000}\selectfont T}%
\end{pgfscope}%
\begin{pgfscope}%
\pgfsetbuttcap%
\pgfsetroundjoin%
\definecolor{currentfill}{rgb}{0.000000,0.000000,0.000000}%
\pgfsetfillcolor{currentfill}%
\pgfsetlinewidth{0.803000pt}%
\definecolor{currentstroke}{rgb}{0.000000,0.000000,0.000000}%
\pgfsetstrokecolor{currentstroke}%
\pgfsetdash{}{0pt}%
\pgfsys@defobject{currentmarker}{\pgfqpoint{-0.048611in}{0.000000in}}{\pgfqpoint{-0.000000in}{0.000000in}}{%
\pgfpathmoveto{\pgfqpoint{-0.000000in}{0.000000in}}%
\pgfpathlineto{\pgfqpoint{-0.048611in}{0.000000in}}%
\pgfusepath{stroke,fill}%
}%
\begin{pgfscope}%
\pgfsys@transformshift{0.380943in}{10.179529in}%
\pgfsys@useobject{currentmarker}{}%
\end{pgfscope}%
\end{pgfscope}%
\begin{pgfscope}%
\definecolor{textcolor}{rgb}{0.000000,0.000000,0.000000}%
\pgfsetstrokecolor{textcolor}%
\pgfsetfillcolor{textcolor}%
\pgftext[x=0.144213in, y=10.140948in, left, base]{\color{textcolor}\rmfamily\fontsize{8.000000}{9.600000}\selectfont F}%
\end{pgfscope}%
\begin{pgfscope}%
\pgfsetbuttcap%
\pgfsetroundjoin%
\definecolor{currentfill}{rgb}{0.000000,0.000000,0.000000}%
\pgfsetfillcolor{currentfill}%
\pgfsetlinewidth{0.803000pt}%
\definecolor{currentstroke}{rgb}{0.000000,0.000000,0.000000}%
\pgfsetstrokecolor{currentstroke}%
\pgfsetdash{}{0pt}%
\pgfsys@defobject{currentmarker}{\pgfqpoint{-0.048611in}{0.000000in}}{\pgfqpoint{-0.000000in}{0.000000in}}{%
\pgfpathmoveto{\pgfqpoint{-0.000000in}{0.000000in}}%
\pgfpathlineto{\pgfqpoint{-0.048611in}{0.000000in}}%
\pgfusepath{stroke,fill}%
}%
\begin{pgfscope}%
\pgfsys@transformshift{0.380943in}{10.091793in}%
\pgfsys@useobject{currentmarker}{}%
\end{pgfscope}%
\end{pgfscope}%
\begin{pgfscope}%
\definecolor{textcolor}{rgb}{0.000000,0.000000,0.000000}%
\pgfsetstrokecolor{textcolor}%
\pgfsetfillcolor{textcolor}%
\pgftext[x=0.155633in, y=10.053212in, left, base]{\color{textcolor}\rmfamily\fontsize{8.000000}{9.600000}\selectfont S}%
\end{pgfscope}%
\begin{pgfscope}%
\pgfsetbuttcap%
\pgfsetroundjoin%
\definecolor{currentfill}{rgb}{0.000000,0.000000,0.000000}%
\pgfsetfillcolor{currentfill}%
\pgfsetlinewidth{0.803000pt}%
\definecolor{currentstroke}{rgb}{0.000000,0.000000,0.000000}%
\pgfsetstrokecolor{currentstroke}%
\pgfsetdash{}{0pt}%
\pgfsys@defobject{currentmarker}{\pgfqpoint{-0.048611in}{0.000000in}}{\pgfqpoint{-0.000000in}{0.000000in}}{%
\pgfpathmoveto{\pgfqpoint{-0.000000in}{0.000000in}}%
\pgfpathlineto{\pgfqpoint{-0.048611in}{0.000000in}}%
\pgfusepath{stroke,fill}%
}%
\begin{pgfscope}%
\pgfsys@transformshift{0.380943in}{10.004057in}%
\pgfsys@useobject{currentmarker}{}%
\end{pgfscope}%
\end{pgfscope}%
\begin{pgfscope}%
\definecolor{textcolor}{rgb}{0.000000,0.000000,0.000000}%
\pgfsetstrokecolor{textcolor}%
\pgfsetfillcolor{textcolor}%
\pgftext[x=0.155633in, y=9.965477in, left, base]{\color{textcolor}\rmfamily\fontsize{8.000000}{9.600000}\selectfont S}%
\end{pgfscope}%
\begin{pgfscope}%
\definecolor{textcolor}{rgb}{0.000000,0.000000,0.000000}%
\pgfsetstrokecolor{textcolor}%
\pgfsetfillcolor{textcolor}%
\pgftext[x=2.705943in,y=10.741007in,,]{\color{textcolor}\ttfamily\fontsize{14.400000}{17.280000}\selectfont 2016}%
\end{pgfscope}%
\begin{pgfscope}%
\pgfpathrectangle{\pgfqpoint{0.380943in}{8.035189in}}{\pgfqpoint{4.650000in}{0.614151in}}%
\pgfusepath{clip}%
\pgfsetbuttcap%
\pgfsetroundjoin%
\pgfsetlinewidth{0.250937pt}%
\definecolor{currentstroke}{rgb}{1.000000,1.000000,1.000000}%
\pgfsetstrokecolor{currentstroke}%
\pgfsetdash{}{0pt}%
\pgfpathmoveto{\pgfqpoint{0.380943in}{8.649340in}}%
\pgfpathlineto{\pgfqpoint{0.468679in}{8.649340in}}%
\pgfpathlineto{\pgfqpoint{0.468679in}{8.561604in}}%
\pgfpathlineto{\pgfqpoint{0.380943in}{8.561604in}}%
\pgfpathlineto{\pgfqpoint{0.380943in}{8.649340in}}%
\pgfusepath{stroke}%
\end{pgfscope}%
\begin{pgfscope}%
\pgfpathrectangle{\pgfqpoint{0.380943in}{8.035189in}}{\pgfqpoint{4.650000in}{0.614151in}}%
\pgfusepath{clip}%
\pgfsetbuttcap%
\pgfsetroundjoin%
\definecolor{currentfill}{rgb}{0.986251,0.808597,0.643230}%
\pgfsetfillcolor{currentfill}%
\pgfsetlinewidth{0.250937pt}%
\definecolor{currentstroke}{rgb}{1.000000,1.000000,1.000000}%
\pgfsetstrokecolor{currentstroke}%
\pgfsetdash{}{0pt}%
\pgfpathmoveto{\pgfqpoint{0.468679in}{8.649340in}}%
\pgfpathlineto{\pgfqpoint{0.556415in}{8.649340in}}%
\pgfpathlineto{\pgfqpoint{0.556415in}{8.561604in}}%
\pgfpathlineto{\pgfqpoint{0.468679in}{8.561604in}}%
\pgfpathlineto{\pgfqpoint{0.468679in}{8.649340in}}%
\pgfusepath{stroke,fill}%
\end{pgfscope}%
\begin{pgfscope}%
\pgfpathrectangle{\pgfqpoint{0.380943in}{8.035189in}}{\pgfqpoint{4.650000in}{0.614151in}}%
\pgfusepath{clip}%
\pgfsetbuttcap%
\pgfsetroundjoin%
\definecolor{currentfill}{rgb}{0.963768,0.915433,0.717478}%
\pgfsetfillcolor{currentfill}%
\pgfsetlinewidth{0.250937pt}%
\definecolor{currentstroke}{rgb}{1.000000,1.000000,1.000000}%
\pgfsetstrokecolor{currentstroke}%
\pgfsetdash{}{0pt}%
\pgfpathmoveto{\pgfqpoint{0.556415in}{8.649340in}}%
\pgfpathlineto{\pgfqpoint{0.644151in}{8.649340in}}%
\pgfpathlineto{\pgfqpoint{0.644151in}{8.561604in}}%
\pgfpathlineto{\pgfqpoint{0.556415in}{8.561604in}}%
\pgfpathlineto{\pgfqpoint{0.556415in}{8.649340in}}%
\pgfusepath{stroke,fill}%
\end{pgfscope}%
\begin{pgfscope}%
\pgfpathrectangle{\pgfqpoint{0.380943in}{8.035189in}}{\pgfqpoint{4.650000in}{0.614151in}}%
\pgfusepath{clip}%
\pgfsetbuttcap%
\pgfsetroundjoin%
\definecolor{currentfill}{rgb}{0.996909,0.711742,0.584452}%
\pgfsetfillcolor{currentfill}%
\pgfsetlinewidth{0.250937pt}%
\definecolor{currentstroke}{rgb}{1.000000,1.000000,1.000000}%
\pgfsetstrokecolor{currentstroke}%
\pgfsetdash{}{0pt}%
\pgfpathmoveto{\pgfqpoint{0.644151in}{8.649340in}}%
\pgfpathlineto{\pgfqpoint{0.731886in}{8.649340in}}%
\pgfpathlineto{\pgfqpoint{0.731886in}{8.561604in}}%
\pgfpathlineto{\pgfqpoint{0.644151in}{8.561604in}}%
\pgfpathlineto{\pgfqpoint{0.644151in}{8.649340in}}%
\pgfusepath{stroke,fill}%
\end{pgfscope}%
\begin{pgfscope}%
\pgfpathrectangle{\pgfqpoint{0.380943in}{8.035189in}}{\pgfqpoint{4.650000in}{0.614151in}}%
\pgfusepath{clip}%
\pgfsetbuttcap%
\pgfsetroundjoin%
\definecolor{currentfill}{rgb}{0.999616,0.641369,0.546559}%
\pgfsetfillcolor{currentfill}%
\pgfsetlinewidth{0.250937pt}%
\definecolor{currentstroke}{rgb}{1.000000,1.000000,1.000000}%
\pgfsetstrokecolor{currentstroke}%
\pgfsetdash{}{0pt}%
\pgfpathmoveto{\pgfqpoint{0.731886in}{8.649340in}}%
\pgfpathlineto{\pgfqpoint{0.819622in}{8.649340in}}%
\pgfpathlineto{\pgfqpoint{0.819622in}{8.561604in}}%
\pgfpathlineto{\pgfqpoint{0.731886in}{8.561604in}}%
\pgfpathlineto{\pgfqpoint{0.731886in}{8.649340in}}%
\pgfusepath{stroke,fill}%
\end{pgfscope}%
\begin{pgfscope}%
\pgfpathrectangle{\pgfqpoint{0.380943in}{8.035189in}}{\pgfqpoint{4.650000in}{0.614151in}}%
\pgfusepath{clip}%
\pgfsetbuttcap%
\pgfsetroundjoin%
\definecolor{currentfill}{rgb}{0.996909,0.711742,0.584452}%
\pgfsetfillcolor{currentfill}%
\pgfsetlinewidth{0.250937pt}%
\definecolor{currentstroke}{rgb}{1.000000,1.000000,1.000000}%
\pgfsetstrokecolor{currentstroke}%
\pgfsetdash{}{0pt}%
\pgfpathmoveto{\pgfqpoint{0.819622in}{8.649340in}}%
\pgfpathlineto{\pgfqpoint{0.907358in}{8.649340in}}%
\pgfpathlineto{\pgfqpoint{0.907358in}{8.561604in}}%
\pgfpathlineto{\pgfqpoint{0.819622in}{8.561604in}}%
\pgfpathlineto{\pgfqpoint{0.819622in}{8.649340in}}%
\pgfusepath{stroke,fill}%
\end{pgfscope}%
\begin{pgfscope}%
\pgfpathrectangle{\pgfqpoint{0.380943in}{8.035189in}}{\pgfqpoint{4.650000in}{0.614151in}}%
\pgfusepath{clip}%
\pgfsetbuttcap%
\pgfsetroundjoin%
\definecolor{currentfill}{rgb}{0.970012,0.883276,0.699577}%
\pgfsetfillcolor{currentfill}%
\pgfsetlinewidth{0.250937pt}%
\definecolor{currentstroke}{rgb}{1.000000,1.000000,1.000000}%
\pgfsetstrokecolor{currentstroke}%
\pgfsetdash{}{0pt}%
\pgfpathmoveto{\pgfqpoint{0.907358in}{8.649340in}}%
\pgfpathlineto{\pgfqpoint{0.995094in}{8.649340in}}%
\pgfpathlineto{\pgfqpoint{0.995094in}{8.561604in}}%
\pgfpathlineto{\pgfqpoint{0.907358in}{8.561604in}}%
\pgfpathlineto{\pgfqpoint{0.907358in}{8.649340in}}%
\pgfusepath{stroke,fill}%
\end{pgfscope}%
\begin{pgfscope}%
\pgfpathrectangle{\pgfqpoint{0.380943in}{8.035189in}}{\pgfqpoint{4.650000in}{0.614151in}}%
\pgfusepath{clip}%
\pgfsetbuttcap%
\pgfsetroundjoin%
\definecolor{currentfill}{rgb}{1.000000,0.531903,0.500946}%
\pgfsetfillcolor{currentfill}%
\pgfsetlinewidth{0.250937pt}%
\definecolor{currentstroke}{rgb}{1.000000,1.000000,1.000000}%
\pgfsetstrokecolor{currentstroke}%
\pgfsetdash{}{0pt}%
\pgfpathmoveto{\pgfqpoint{0.995094in}{8.649340in}}%
\pgfpathlineto{\pgfqpoint{1.082830in}{8.649340in}}%
\pgfpathlineto{\pgfqpoint{1.082830in}{8.561604in}}%
\pgfpathlineto{\pgfqpoint{0.995094in}{8.561604in}}%
\pgfpathlineto{\pgfqpoint{0.995094in}{8.649340in}}%
\pgfusepath{stroke,fill}%
\end{pgfscope}%
\begin{pgfscope}%
\pgfpathrectangle{\pgfqpoint{0.380943in}{8.035189in}}{\pgfqpoint{4.650000in}{0.614151in}}%
\pgfusepath{clip}%
\pgfsetbuttcap%
\pgfsetroundjoin%
\definecolor{currentfill}{rgb}{1.000000,0.531903,0.500946}%
\pgfsetfillcolor{currentfill}%
\pgfsetlinewidth{0.250937pt}%
\definecolor{currentstroke}{rgb}{1.000000,1.000000,1.000000}%
\pgfsetstrokecolor{currentstroke}%
\pgfsetdash{}{0pt}%
\pgfpathmoveto{\pgfqpoint{1.082830in}{8.649340in}}%
\pgfpathlineto{\pgfqpoint{1.170566in}{8.649340in}}%
\pgfpathlineto{\pgfqpoint{1.170566in}{8.561604in}}%
\pgfpathlineto{\pgfqpoint{1.082830in}{8.561604in}}%
\pgfpathlineto{\pgfqpoint{1.082830in}{8.649340in}}%
\pgfusepath{stroke,fill}%
\end{pgfscope}%
\begin{pgfscope}%
\pgfpathrectangle{\pgfqpoint{0.380943in}{8.035189in}}{\pgfqpoint{4.650000in}{0.614151in}}%
\pgfusepath{clip}%
\pgfsetbuttcap%
\pgfsetroundjoin%
\definecolor{currentfill}{rgb}{1.000000,0.584929,0.522599}%
\pgfsetfillcolor{currentfill}%
\pgfsetlinewidth{0.250937pt}%
\definecolor{currentstroke}{rgb}{1.000000,1.000000,1.000000}%
\pgfsetstrokecolor{currentstroke}%
\pgfsetdash{}{0pt}%
\pgfpathmoveto{\pgfqpoint{1.170566in}{8.649340in}}%
\pgfpathlineto{\pgfqpoint{1.258302in}{8.649340in}}%
\pgfpathlineto{\pgfqpoint{1.258302in}{8.561604in}}%
\pgfpathlineto{\pgfqpoint{1.170566in}{8.561604in}}%
\pgfpathlineto{\pgfqpoint{1.170566in}{8.649340in}}%
\pgfusepath{stroke,fill}%
\end{pgfscope}%
\begin{pgfscope}%
\pgfpathrectangle{\pgfqpoint{0.380943in}{8.035189in}}{\pgfqpoint{4.650000in}{0.614151in}}%
\pgfusepath{clip}%
\pgfsetbuttcap%
\pgfsetroundjoin%
\definecolor{currentfill}{rgb}{0.963768,0.915433,0.717478}%
\pgfsetfillcolor{currentfill}%
\pgfsetlinewidth{0.250937pt}%
\definecolor{currentstroke}{rgb}{1.000000,1.000000,1.000000}%
\pgfsetstrokecolor{currentstroke}%
\pgfsetdash{}{0pt}%
\pgfpathmoveto{\pgfqpoint{1.258302in}{8.649340in}}%
\pgfpathlineto{\pgfqpoint{1.346037in}{8.649340in}}%
\pgfpathlineto{\pgfqpoint{1.346037in}{8.561604in}}%
\pgfpathlineto{\pgfqpoint{1.258302in}{8.561604in}}%
\pgfpathlineto{\pgfqpoint{1.258302in}{8.649340in}}%
\pgfusepath{stroke,fill}%
\end{pgfscope}%
\begin{pgfscope}%
\pgfpathrectangle{\pgfqpoint{0.380943in}{8.035189in}}{\pgfqpoint{4.650000in}{0.614151in}}%
\pgfusepath{clip}%
\pgfsetbuttcap%
\pgfsetroundjoin%
\definecolor{currentfill}{rgb}{0.992326,0.765229,0.614840}%
\pgfsetfillcolor{currentfill}%
\pgfsetlinewidth{0.250937pt}%
\definecolor{currentstroke}{rgb}{1.000000,1.000000,1.000000}%
\pgfsetstrokecolor{currentstroke}%
\pgfsetdash{}{0pt}%
\pgfpathmoveto{\pgfqpoint{1.346037in}{8.649340in}}%
\pgfpathlineto{\pgfqpoint{1.433773in}{8.649340in}}%
\pgfpathlineto{\pgfqpoint{1.433773in}{8.561604in}}%
\pgfpathlineto{\pgfqpoint{1.346037in}{8.561604in}}%
\pgfpathlineto{\pgfqpoint{1.346037in}{8.649340in}}%
\pgfusepath{stroke,fill}%
\end{pgfscope}%
\begin{pgfscope}%
\pgfpathrectangle{\pgfqpoint{0.380943in}{8.035189in}}{\pgfqpoint{4.650000in}{0.614151in}}%
\pgfusepath{clip}%
\pgfsetbuttcap%
\pgfsetroundjoin%
\definecolor{currentfill}{rgb}{0.978131,0.843783,0.675709}%
\pgfsetfillcolor{currentfill}%
\pgfsetlinewidth{0.250937pt}%
\definecolor{currentstroke}{rgb}{1.000000,1.000000,1.000000}%
\pgfsetstrokecolor{currentstroke}%
\pgfsetdash{}{0pt}%
\pgfpathmoveto{\pgfqpoint{1.433773in}{8.649340in}}%
\pgfpathlineto{\pgfqpoint{1.521509in}{8.649340in}}%
\pgfpathlineto{\pgfqpoint{1.521509in}{8.561604in}}%
\pgfpathlineto{\pgfqpoint{1.433773in}{8.561604in}}%
\pgfpathlineto{\pgfqpoint{1.433773in}{8.649340in}}%
\pgfusepath{stroke,fill}%
\end{pgfscope}%
\begin{pgfscope}%
\pgfpathrectangle{\pgfqpoint{0.380943in}{8.035189in}}{\pgfqpoint{4.650000in}{0.614151in}}%
\pgfusepath{clip}%
\pgfsetbuttcap%
\pgfsetroundjoin%
\definecolor{currentfill}{rgb}{0.992326,0.765229,0.614840}%
\pgfsetfillcolor{currentfill}%
\pgfsetlinewidth{0.250937pt}%
\definecolor{currentstroke}{rgb}{1.000000,1.000000,1.000000}%
\pgfsetstrokecolor{currentstroke}%
\pgfsetdash{}{0pt}%
\pgfpathmoveto{\pgfqpoint{1.521509in}{8.649340in}}%
\pgfpathlineto{\pgfqpoint{1.609245in}{8.649340in}}%
\pgfpathlineto{\pgfqpoint{1.609245in}{8.561604in}}%
\pgfpathlineto{\pgfqpoint{1.521509in}{8.561604in}}%
\pgfpathlineto{\pgfqpoint{1.521509in}{8.649340in}}%
\pgfusepath{stroke,fill}%
\end{pgfscope}%
\begin{pgfscope}%
\pgfpathrectangle{\pgfqpoint{0.380943in}{8.035189in}}{\pgfqpoint{4.650000in}{0.614151in}}%
\pgfusepath{clip}%
\pgfsetbuttcap%
\pgfsetroundjoin%
\definecolor{currentfill}{rgb}{0.961061,0.931672,0.728304}%
\pgfsetfillcolor{currentfill}%
\pgfsetlinewidth{0.250937pt}%
\definecolor{currentstroke}{rgb}{1.000000,1.000000,1.000000}%
\pgfsetstrokecolor{currentstroke}%
\pgfsetdash{}{0pt}%
\pgfpathmoveto{\pgfqpoint{1.609245in}{8.649340in}}%
\pgfpathlineto{\pgfqpoint{1.696981in}{8.649340in}}%
\pgfpathlineto{\pgfqpoint{1.696981in}{8.561604in}}%
\pgfpathlineto{\pgfqpoint{1.609245in}{8.561604in}}%
\pgfpathlineto{\pgfqpoint{1.609245in}{8.649340in}}%
\pgfusepath{stroke,fill}%
\end{pgfscope}%
\begin{pgfscope}%
\pgfpathrectangle{\pgfqpoint{0.380943in}{8.035189in}}{\pgfqpoint{4.650000in}{0.614151in}}%
\pgfusepath{clip}%
\pgfsetbuttcap%
\pgfsetroundjoin%
\definecolor{currentfill}{rgb}{0.961061,0.931672,0.728304}%
\pgfsetfillcolor{currentfill}%
\pgfsetlinewidth{0.250937pt}%
\definecolor{currentstroke}{rgb}{1.000000,1.000000,1.000000}%
\pgfsetstrokecolor{currentstroke}%
\pgfsetdash{}{0pt}%
\pgfpathmoveto{\pgfqpoint{1.696981in}{8.649340in}}%
\pgfpathlineto{\pgfqpoint{1.784717in}{8.649340in}}%
\pgfpathlineto{\pgfqpoint{1.784717in}{8.561604in}}%
\pgfpathlineto{\pgfqpoint{1.696981in}{8.561604in}}%
\pgfpathlineto{\pgfqpoint{1.696981in}{8.649340in}}%
\pgfusepath{stroke,fill}%
\end{pgfscope}%
\begin{pgfscope}%
\pgfpathrectangle{\pgfqpoint{0.380943in}{8.035189in}}{\pgfqpoint{4.650000in}{0.614151in}}%
\pgfusepath{clip}%
\pgfsetbuttcap%
\pgfsetroundjoin%
\definecolor{currentfill}{rgb}{1.000000,1.000000,0.861745}%
\pgfsetfillcolor{currentfill}%
\pgfsetlinewidth{0.250937pt}%
\definecolor{currentstroke}{rgb}{1.000000,1.000000,1.000000}%
\pgfsetstrokecolor{currentstroke}%
\pgfsetdash{}{0pt}%
\pgfpathmoveto{\pgfqpoint{1.784717in}{8.649340in}}%
\pgfpathlineto{\pgfqpoint{1.872452in}{8.649340in}}%
\pgfpathlineto{\pgfqpoint{1.872452in}{8.561604in}}%
\pgfpathlineto{\pgfqpoint{1.784717in}{8.561604in}}%
\pgfpathlineto{\pgfqpoint{1.784717in}{8.649340in}}%
\pgfusepath{stroke,fill}%
\end{pgfscope}%
\begin{pgfscope}%
\pgfpathrectangle{\pgfqpoint{0.380943in}{8.035189in}}{\pgfqpoint{4.650000in}{0.614151in}}%
\pgfusepath{clip}%
\pgfsetbuttcap%
\pgfsetroundjoin%
\definecolor{currentfill}{rgb}{0.970012,0.883276,0.699577}%
\pgfsetfillcolor{currentfill}%
\pgfsetlinewidth{0.250937pt}%
\definecolor{currentstroke}{rgb}{1.000000,1.000000,1.000000}%
\pgfsetstrokecolor{currentstroke}%
\pgfsetdash{}{0pt}%
\pgfpathmoveto{\pgfqpoint{1.872452in}{8.649340in}}%
\pgfpathlineto{\pgfqpoint{1.960188in}{8.649340in}}%
\pgfpathlineto{\pgfqpoint{1.960188in}{8.561604in}}%
\pgfpathlineto{\pgfqpoint{1.872452in}{8.561604in}}%
\pgfpathlineto{\pgfqpoint{1.872452in}{8.649340in}}%
\pgfusepath{stroke,fill}%
\end{pgfscope}%
\begin{pgfscope}%
\pgfpathrectangle{\pgfqpoint{0.380943in}{8.035189in}}{\pgfqpoint{4.650000in}{0.614151in}}%
\pgfusepath{clip}%
\pgfsetbuttcap%
\pgfsetroundjoin%
\definecolor{currentfill}{rgb}{1.000000,1.000000,0.861745}%
\pgfsetfillcolor{currentfill}%
\pgfsetlinewidth{0.250937pt}%
\definecolor{currentstroke}{rgb}{1.000000,1.000000,1.000000}%
\pgfsetstrokecolor{currentstroke}%
\pgfsetdash{}{0pt}%
\pgfpathmoveto{\pgfqpoint{1.960188in}{8.649340in}}%
\pgfpathlineto{\pgfqpoint{2.047924in}{8.649340in}}%
\pgfpathlineto{\pgfqpoint{2.047924in}{8.561604in}}%
\pgfpathlineto{\pgfqpoint{1.960188in}{8.561604in}}%
\pgfpathlineto{\pgfqpoint{1.960188in}{8.649340in}}%
\pgfusepath{stroke,fill}%
\end{pgfscope}%
\begin{pgfscope}%
\pgfpathrectangle{\pgfqpoint{0.380943in}{8.035189in}}{\pgfqpoint{4.650000in}{0.614151in}}%
\pgfusepath{clip}%
\pgfsetbuttcap%
\pgfsetroundjoin%
\definecolor{currentfill}{rgb}{0.961061,0.931672,0.728304}%
\pgfsetfillcolor{currentfill}%
\pgfsetlinewidth{0.250937pt}%
\definecolor{currentstroke}{rgb}{1.000000,1.000000,1.000000}%
\pgfsetstrokecolor{currentstroke}%
\pgfsetdash{}{0pt}%
\pgfpathmoveto{\pgfqpoint{2.047924in}{8.649340in}}%
\pgfpathlineto{\pgfqpoint{2.135660in}{8.649340in}}%
\pgfpathlineto{\pgfqpoint{2.135660in}{8.561604in}}%
\pgfpathlineto{\pgfqpoint{2.047924in}{8.561604in}}%
\pgfpathlineto{\pgfqpoint{2.047924in}{8.649340in}}%
\pgfusepath{stroke,fill}%
\end{pgfscope}%
\begin{pgfscope}%
\pgfpathrectangle{\pgfqpoint{0.380943in}{8.035189in}}{\pgfqpoint{4.650000in}{0.614151in}}%
\pgfusepath{clip}%
\pgfsetbuttcap%
\pgfsetroundjoin%
\definecolor{currentfill}{rgb}{0.986251,0.808597,0.643230}%
\pgfsetfillcolor{currentfill}%
\pgfsetlinewidth{0.250937pt}%
\definecolor{currentstroke}{rgb}{1.000000,1.000000,1.000000}%
\pgfsetstrokecolor{currentstroke}%
\pgfsetdash{}{0pt}%
\pgfpathmoveto{\pgfqpoint{2.135660in}{8.649340in}}%
\pgfpathlineto{\pgfqpoint{2.223396in}{8.649340in}}%
\pgfpathlineto{\pgfqpoint{2.223396in}{8.561604in}}%
\pgfpathlineto{\pgfqpoint{2.135660in}{8.561604in}}%
\pgfpathlineto{\pgfqpoint{2.135660in}{8.649340in}}%
\pgfusepath{stroke,fill}%
\end{pgfscope}%
\begin{pgfscope}%
\pgfpathrectangle{\pgfqpoint{0.380943in}{8.035189in}}{\pgfqpoint{4.650000in}{0.614151in}}%
\pgfusepath{clip}%
\pgfsetbuttcap%
\pgfsetroundjoin%
\definecolor{currentfill}{rgb}{0.961061,0.931672,0.728304}%
\pgfsetfillcolor{currentfill}%
\pgfsetlinewidth{0.250937pt}%
\definecolor{currentstroke}{rgb}{1.000000,1.000000,1.000000}%
\pgfsetstrokecolor{currentstroke}%
\pgfsetdash{}{0pt}%
\pgfpathmoveto{\pgfqpoint{2.223396in}{8.649340in}}%
\pgfpathlineto{\pgfqpoint{2.311132in}{8.649340in}}%
\pgfpathlineto{\pgfqpoint{2.311132in}{8.561604in}}%
\pgfpathlineto{\pgfqpoint{2.223396in}{8.561604in}}%
\pgfpathlineto{\pgfqpoint{2.223396in}{8.649340in}}%
\pgfusepath{stroke,fill}%
\end{pgfscope}%
\begin{pgfscope}%
\pgfpathrectangle{\pgfqpoint{0.380943in}{8.035189in}}{\pgfqpoint{4.650000in}{0.614151in}}%
\pgfusepath{clip}%
\pgfsetbuttcap%
\pgfsetroundjoin%
\definecolor{currentfill}{rgb}{0.970012,0.883276,0.699577}%
\pgfsetfillcolor{currentfill}%
\pgfsetlinewidth{0.250937pt}%
\definecolor{currentstroke}{rgb}{1.000000,1.000000,1.000000}%
\pgfsetstrokecolor{currentstroke}%
\pgfsetdash{}{0pt}%
\pgfpathmoveto{\pgfqpoint{2.311132in}{8.649340in}}%
\pgfpathlineto{\pgfqpoint{2.398868in}{8.649340in}}%
\pgfpathlineto{\pgfqpoint{2.398868in}{8.561604in}}%
\pgfpathlineto{\pgfqpoint{2.311132in}{8.561604in}}%
\pgfpathlineto{\pgfqpoint{2.311132in}{8.649340in}}%
\pgfusepath{stroke,fill}%
\end{pgfscope}%
\begin{pgfscope}%
\pgfpathrectangle{\pgfqpoint{0.380943in}{8.035189in}}{\pgfqpoint{4.650000in}{0.614151in}}%
\pgfusepath{clip}%
\pgfsetbuttcap%
\pgfsetroundjoin%
\definecolor{currentfill}{rgb}{0.961061,0.931672,0.728304}%
\pgfsetfillcolor{currentfill}%
\pgfsetlinewidth{0.250937pt}%
\definecolor{currentstroke}{rgb}{1.000000,1.000000,1.000000}%
\pgfsetstrokecolor{currentstroke}%
\pgfsetdash{}{0pt}%
\pgfpathmoveto{\pgfqpoint{2.398868in}{8.649340in}}%
\pgfpathlineto{\pgfqpoint{2.486603in}{8.649340in}}%
\pgfpathlineto{\pgfqpoint{2.486603in}{8.561604in}}%
\pgfpathlineto{\pgfqpoint{2.398868in}{8.561604in}}%
\pgfpathlineto{\pgfqpoint{2.398868in}{8.649340in}}%
\pgfusepath{stroke,fill}%
\end{pgfscope}%
\begin{pgfscope}%
\pgfpathrectangle{\pgfqpoint{0.380943in}{8.035189in}}{\pgfqpoint{4.650000in}{0.614151in}}%
\pgfusepath{clip}%
\pgfsetbuttcap%
\pgfsetroundjoin%
\definecolor{currentfill}{rgb}{0.970012,0.883276,0.699577}%
\pgfsetfillcolor{currentfill}%
\pgfsetlinewidth{0.250937pt}%
\definecolor{currentstroke}{rgb}{1.000000,1.000000,1.000000}%
\pgfsetstrokecolor{currentstroke}%
\pgfsetdash{}{0pt}%
\pgfpathmoveto{\pgfqpoint{2.486603in}{8.649340in}}%
\pgfpathlineto{\pgfqpoint{2.574339in}{8.649340in}}%
\pgfpathlineto{\pgfqpoint{2.574339in}{8.561604in}}%
\pgfpathlineto{\pgfqpoint{2.486603in}{8.561604in}}%
\pgfpathlineto{\pgfqpoint{2.486603in}{8.649340in}}%
\pgfusepath{stroke,fill}%
\end{pgfscope}%
\begin{pgfscope}%
\pgfpathrectangle{\pgfqpoint{0.380943in}{8.035189in}}{\pgfqpoint{4.650000in}{0.614151in}}%
\pgfusepath{clip}%
\pgfsetbuttcap%
\pgfsetroundjoin%
\definecolor{currentfill}{rgb}{0.978131,0.843783,0.675709}%
\pgfsetfillcolor{currentfill}%
\pgfsetlinewidth{0.250937pt}%
\definecolor{currentstroke}{rgb}{1.000000,1.000000,1.000000}%
\pgfsetstrokecolor{currentstroke}%
\pgfsetdash{}{0pt}%
\pgfpathmoveto{\pgfqpoint{2.574339in}{8.649340in}}%
\pgfpathlineto{\pgfqpoint{2.662075in}{8.649340in}}%
\pgfpathlineto{\pgfqpoint{2.662075in}{8.561604in}}%
\pgfpathlineto{\pgfqpoint{2.574339in}{8.561604in}}%
\pgfpathlineto{\pgfqpoint{2.574339in}{8.649340in}}%
\pgfusepath{stroke,fill}%
\end{pgfscope}%
\begin{pgfscope}%
\pgfpathrectangle{\pgfqpoint{0.380943in}{8.035189in}}{\pgfqpoint{4.650000in}{0.614151in}}%
\pgfusepath{clip}%
\pgfsetbuttcap%
\pgfsetroundjoin%
\definecolor{currentfill}{rgb}{0.992326,0.765229,0.614840}%
\pgfsetfillcolor{currentfill}%
\pgfsetlinewidth{0.250937pt}%
\definecolor{currentstroke}{rgb}{1.000000,1.000000,1.000000}%
\pgfsetstrokecolor{currentstroke}%
\pgfsetdash{}{0pt}%
\pgfpathmoveto{\pgfqpoint{2.662075in}{8.649340in}}%
\pgfpathlineto{\pgfqpoint{2.749811in}{8.649340in}}%
\pgfpathlineto{\pgfqpoint{2.749811in}{8.561604in}}%
\pgfpathlineto{\pgfqpoint{2.662075in}{8.561604in}}%
\pgfpathlineto{\pgfqpoint{2.662075in}{8.649340in}}%
\pgfusepath{stroke,fill}%
\end{pgfscope}%
\begin{pgfscope}%
\pgfpathrectangle{\pgfqpoint{0.380943in}{8.035189in}}{\pgfqpoint{4.650000in}{0.614151in}}%
\pgfusepath{clip}%
\pgfsetbuttcap%
\pgfsetroundjoin%
\definecolor{currentfill}{rgb}{0.986251,0.808597,0.643230}%
\pgfsetfillcolor{currentfill}%
\pgfsetlinewidth{0.250937pt}%
\definecolor{currentstroke}{rgb}{1.000000,1.000000,1.000000}%
\pgfsetstrokecolor{currentstroke}%
\pgfsetdash{}{0pt}%
\pgfpathmoveto{\pgfqpoint{2.749811in}{8.649340in}}%
\pgfpathlineto{\pgfqpoint{2.837547in}{8.649340in}}%
\pgfpathlineto{\pgfqpoint{2.837547in}{8.561604in}}%
\pgfpathlineto{\pgfqpoint{2.749811in}{8.561604in}}%
\pgfpathlineto{\pgfqpoint{2.749811in}{8.649340in}}%
\pgfusepath{stroke,fill}%
\end{pgfscope}%
\begin{pgfscope}%
\pgfpathrectangle{\pgfqpoint{0.380943in}{8.035189in}}{\pgfqpoint{4.650000in}{0.614151in}}%
\pgfusepath{clip}%
\pgfsetbuttcap%
\pgfsetroundjoin%
\definecolor{currentfill}{rgb}{0.992326,0.765229,0.614840}%
\pgfsetfillcolor{currentfill}%
\pgfsetlinewidth{0.250937pt}%
\definecolor{currentstroke}{rgb}{1.000000,1.000000,1.000000}%
\pgfsetstrokecolor{currentstroke}%
\pgfsetdash{}{0pt}%
\pgfpathmoveto{\pgfqpoint{2.837547in}{8.649340in}}%
\pgfpathlineto{\pgfqpoint{2.925283in}{8.649340in}}%
\pgfpathlineto{\pgfqpoint{2.925283in}{8.561604in}}%
\pgfpathlineto{\pgfqpoint{2.837547in}{8.561604in}}%
\pgfpathlineto{\pgfqpoint{2.837547in}{8.649340in}}%
\pgfusepath{stroke,fill}%
\end{pgfscope}%
\begin{pgfscope}%
\pgfpathrectangle{\pgfqpoint{0.380943in}{8.035189in}}{\pgfqpoint{4.650000in}{0.614151in}}%
\pgfusepath{clip}%
\pgfsetbuttcap%
\pgfsetroundjoin%
\definecolor{currentfill}{rgb}{0.963768,0.915433,0.717478}%
\pgfsetfillcolor{currentfill}%
\pgfsetlinewidth{0.250937pt}%
\definecolor{currentstroke}{rgb}{1.000000,1.000000,1.000000}%
\pgfsetstrokecolor{currentstroke}%
\pgfsetdash{}{0pt}%
\pgfpathmoveto{\pgfqpoint{2.925283in}{8.649340in}}%
\pgfpathlineto{\pgfqpoint{3.013019in}{8.649340in}}%
\pgfpathlineto{\pgfqpoint{3.013019in}{8.561604in}}%
\pgfpathlineto{\pgfqpoint{2.925283in}{8.561604in}}%
\pgfpathlineto{\pgfqpoint{2.925283in}{8.649340in}}%
\pgfusepath{stroke,fill}%
\end{pgfscope}%
\begin{pgfscope}%
\pgfpathrectangle{\pgfqpoint{0.380943in}{8.035189in}}{\pgfqpoint{4.650000in}{0.614151in}}%
\pgfusepath{clip}%
\pgfsetbuttcap%
\pgfsetroundjoin%
\definecolor{currentfill}{rgb}{0.992326,0.765229,0.614840}%
\pgfsetfillcolor{currentfill}%
\pgfsetlinewidth{0.250937pt}%
\definecolor{currentstroke}{rgb}{1.000000,1.000000,1.000000}%
\pgfsetstrokecolor{currentstroke}%
\pgfsetdash{}{0pt}%
\pgfpathmoveto{\pgfqpoint{3.013019in}{8.649340in}}%
\pgfpathlineto{\pgfqpoint{3.100754in}{8.649340in}}%
\pgfpathlineto{\pgfqpoint{3.100754in}{8.561604in}}%
\pgfpathlineto{\pgfqpoint{3.013019in}{8.561604in}}%
\pgfpathlineto{\pgfqpoint{3.013019in}{8.649340in}}%
\pgfusepath{stroke,fill}%
\end{pgfscope}%
\begin{pgfscope}%
\pgfpathrectangle{\pgfqpoint{0.380943in}{8.035189in}}{\pgfqpoint{4.650000in}{0.614151in}}%
\pgfusepath{clip}%
\pgfsetbuttcap%
\pgfsetroundjoin%
\definecolor{currentfill}{rgb}{0.961061,0.931672,0.728304}%
\pgfsetfillcolor{currentfill}%
\pgfsetlinewidth{0.250937pt}%
\definecolor{currentstroke}{rgb}{1.000000,1.000000,1.000000}%
\pgfsetstrokecolor{currentstroke}%
\pgfsetdash{}{0pt}%
\pgfpathmoveto{\pgfqpoint{3.100754in}{8.649340in}}%
\pgfpathlineto{\pgfqpoint{3.188490in}{8.649340in}}%
\pgfpathlineto{\pgfqpoint{3.188490in}{8.561604in}}%
\pgfpathlineto{\pgfqpoint{3.100754in}{8.561604in}}%
\pgfpathlineto{\pgfqpoint{3.100754in}{8.649340in}}%
\pgfusepath{stroke,fill}%
\end{pgfscope}%
\begin{pgfscope}%
\pgfpathrectangle{\pgfqpoint{0.380943in}{8.035189in}}{\pgfqpoint{4.650000in}{0.614151in}}%
\pgfusepath{clip}%
\pgfsetbuttcap%
\pgfsetroundjoin%
\definecolor{currentfill}{rgb}{0.961061,0.931672,0.728304}%
\pgfsetfillcolor{currentfill}%
\pgfsetlinewidth{0.250937pt}%
\definecolor{currentstroke}{rgb}{1.000000,1.000000,1.000000}%
\pgfsetstrokecolor{currentstroke}%
\pgfsetdash{}{0pt}%
\pgfpathmoveto{\pgfqpoint{3.188490in}{8.649340in}}%
\pgfpathlineto{\pgfqpoint{3.276226in}{8.649340in}}%
\pgfpathlineto{\pgfqpoint{3.276226in}{8.561604in}}%
\pgfpathlineto{\pgfqpoint{3.188490in}{8.561604in}}%
\pgfpathlineto{\pgfqpoint{3.188490in}{8.649340in}}%
\pgfusepath{stroke,fill}%
\end{pgfscope}%
\begin{pgfscope}%
\pgfpathrectangle{\pgfqpoint{0.380943in}{8.035189in}}{\pgfqpoint{4.650000in}{0.614151in}}%
\pgfusepath{clip}%
\pgfsetbuttcap%
\pgfsetroundjoin%
\definecolor{currentfill}{rgb}{0.985083,0.974641,0.792587}%
\pgfsetfillcolor{currentfill}%
\pgfsetlinewidth{0.250937pt}%
\definecolor{currentstroke}{rgb}{1.000000,1.000000,1.000000}%
\pgfsetstrokecolor{currentstroke}%
\pgfsetdash{}{0pt}%
\pgfpathmoveto{\pgfqpoint{3.276226in}{8.649340in}}%
\pgfpathlineto{\pgfqpoint{3.363962in}{8.649340in}}%
\pgfpathlineto{\pgfqpoint{3.363962in}{8.561604in}}%
\pgfpathlineto{\pgfqpoint{3.276226in}{8.561604in}}%
\pgfpathlineto{\pgfqpoint{3.276226in}{8.649340in}}%
\pgfusepath{stroke,fill}%
\end{pgfscope}%
\begin{pgfscope}%
\pgfpathrectangle{\pgfqpoint{0.380943in}{8.035189in}}{\pgfqpoint{4.650000in}{0.614151in}}%
\pgfusepath{clip}%
\pgfsetbuttcap%
\pgfsetroundjoin%
\definecolor{currentfill}{rgb}{0.978131,0.843783,0.675709}%
\pgfsetfillcolor{currentfill}%
\pgfsetlinewidth{0.250937pt}%
\definecolor{currentstroke}{rgb}{1.000000,1.000000,1.000000}%
\pgfsetstrokecolor{currentstroke}%
\pgfsetdash{}{0pt}%
\pgfpathmoveto{\pgfqpoint{3.363962in}{8.649340in}}%
\pgfpathlineto{\pgfqpoint{3.451698in}{8.649340in}}%
\pgfpathlineto{\pgfqpoint{3.451698in}{8.561604in}}%
\pgfpathlineto{\pgfqpoint{3.363962in}{8.561604in}}%
\pgfpathlineto{\pgfqpoint{3.363962in}{8.649340in}}%
\pgfusepath{stroke,fill}%
\end{pgfscope}%
\begin{pgfscope}%
\pgfpathrectangle{\pgfqpoint{0.380943in}{8.035189in}}{\pgfqpoint{4.650000in}{0.614151in}}%
\pgfusepath{clip}%
\pgfsetbuttcap%
\pgfsetroundjoin%
\definecolor{currentfill}{rgb}{0.970012,0.883276,0.699577}%
\pgfsetfillcolor{currentfill}%
\pgfsetlinewidth{0.250937pt}%
\definecolor{currentstroke}{rgb}{1.000000,1.000000,1.000000}%
\pgfsetstrokecolor{currentstroke}%
\pgfsetdash{}{0pt}%
\pgfpathmoveto{\pgfqpoint{3.451698in}{8.649340in}}%
\pgfpathlineto{\pgfqpoint{3.539434in}{8.649340in}}%
\pgfpathlineto{\pgfqpoint{3.539434in}{8.561604in}}%
\pgfpathlineto{\pgfqpoint{3.451698in}{8.561604in}}%
\pgfpathlineto{\pgfqpoint{3.451698in}{8.649340in}}%
\pgfusepath{stroke,fill}%
\end{pgfscope}%
\begin{pgfscope}%
\pgfpathrectangle{\pgfqpoint{0.380943in}{8.035189in}}{\pgfqpoint{4.650000in}{0.614151in}}%
\pgfusepath{clip}%
\pgfsetbuttcap%
\pgfsetroundjoin%
\definecolor{currentfill}{rgb}{0.986251,0.808597,0.643230}%
\pgfsetfillcolor{currentfill}%
\pgfsetlinewidth{0.250937pt}%
\definecolor{currentstroke}{rgb}{1.000000,1.000000,1.000000}%
\pgfsetstrokecolor{currentstroke}%
\pgfsetdash{}{0pt}%
\pgfpathmoveto{\pgfqpoint{3.539434in}{8.649340in}}%
\pgfpathlineto{\pgfqpoint{3.627169in}{8.649340in}}%
\pgfpathlineto{\pgfqpoint{3.627169in}{8.561604in}}%
\pgfpathlineto{\pgfqpoint{3.539434in}{8.561604in}}%
\pgfpathlineto{\pgfqpoint{3.539434in}{8.649340in}}%
\pgfusepath{stroke,fill}%
\end{pgfscope}%
\begin{pgfscope}%
\pgfpathrectangle{\pgfqpoint{0.380943in}{8.035189in}}{\pgfqpoint{4.650000in}{0.614151in}}%
\pgfusepath{clip}%
\pgfsetbuttcap%
\pgfsetroundjoin%
\definecolor{currentfill}{rgb}{0.986251,0.808597,0.643230}%
\pgfsetfillcolor{currentfill}%
\pgfsetlinewidth{0.250937pt}%
\definecolor{currentstroke}{rgb}{1.000000,1.000000,1.000000}%
\pgfsetstrokecolor{currentstroke}%
\pgfsetdash{}{0pt}%
\pgfpathmoveto{\pgfqpoint{3.627169in}{8.649340in}}%
\pgfpathlineto{\pgfqpoint{3.714905in}{8.649340in}}%
\pgfpathlineto{\pgfqpoint{3.714905in}{8.561604in}}%
\pgfpathlineto{\pgfqpoint{3.627169in}{8.561604in}}%
\pgfpathlineto{\pgfqpoint{3.627169in}{8.649340in}}%
\pgfusepath{stroke,fill}%
\end{pgfscope}%
\begin{pgfscope}%
\pgfpathrectangle{\pgfqpoint{0.380943in}{8.035189in}}{\pgfqpoint{4.650000in}{0.614151in}}%
\pgfusepath{clip}%
\pgfsetbuttcap%
\pgfsetroundjoin%
\definecolor{currentfill}{rgb}{0.970012,0.883276,0.699577}%
\pgfsetfillcolor{currentfill}%
\pgfsetlinewidth{0.250937pt}%
\definecolor{currentstroke}{rgb}{1.000000,1.000000,1.000000}%
\pgfsetstrokecolor{currentstroke}%
\pgfsetdash{}{0pt}%
\pgfpathmoveto{\pgfqpoint{3.714905in}{8.649340in}}%
\pgfpathlineto{\pgfqpoint{3.802641in}{8.649340in}}%
\pgfpathlineto{\pgfqpoint{3.802641in}{8.561604in}}%
\pgfpathlineto{\pgfqpoint{3.714905in}{8.561604in}}%
\pgfpathlineto{\pgfqpoint{3.714905in}{8.649340in}}%
\pgfusepath{stroke,fill}%
\end{pgfscope}%
\begin{pgfscope}%
\pgfpathrectangle{\pgfqpoint{0.380943in}{8.035189in}}{\pgfqpoint{4.650000in}{0.614151in}}%
\pgfusepath{clip}%
\pgfsetbuttcap%
\pgfsetroundjoin%
\definecolor{currentfill}{rgb}{0.963768,0.915433,0.717478}%
\pgfsetfillcolor{currentfill}%
\pgfsetlinewidth{0.250937pt}%
\definecolor{currentstroke}{rgb}{1.000000,1.000000,1.000000}%
\pgfsetstrokecolor{currentstroke}%
\pgfsetdash{}{0pt}%
\pgfpathmoveto{\pgfqpoint{3.802641in}{8.649340in}}%
\pgfpathlineto{\pgfqpoint{3.890377in}{8.649340in}}%
\pgfpathlineto{\pgfqpoint{3.890377in}{8.561604in}}%
\pgfpathlineto{\pgfqpoint{3.802641in}{8.561604in}}%
\pgfpathlineto{\pgfqpoint{3.802641in}{8.649340in}}%
\pgfusepath{stroke,fill}%
\end{pgfscope}%
\begin{pgfscope}%
\pgfpathrectangle{\pgfqpoint{0.380943in}{8.035189in}}{\pgfqpoint{4.650000in}{0.614151in}}%
\pgfusepath{clip}%
\pgfsetbuttcap%
\pgfsetroundjoin%
\definecolor{currentfill}{rgb}{0.961061,0.931672,0.728304}%
\pgfsetfillcolor{currentfill}%
\pgfsetlinewidth{0.250937pt}%
\definecolor{currentstroke}{rgb}{1.000000,1.000000,1.000000}%
\pgfsetstrokecolor{currentstroke}%
\pgfsetdash{}{0pt}%
\pgfpathmoveto{\pgfqpoint{3.890377in}{8.649340in}}%
\pgfpathlineto{\pgfqpoint{3.978113in}{8.649340in}}%
\pgfpathlineto{\pgfqpoint{3.978113in}{8.561604in}}%
\pgfpathlineto{\pgfqpoint{3.890377in}{8.561604in}}%
\pgfpathlineto{\pgfqpoint{3.890377in}{8.649340in}}%
\pgfusepath{stroke,fill}%
\end{pgfscope}%
\begin{pgfscope}%
\pgfpathrectangle{\pgfqpoint{0.380943in}{8.035189in}}{\pgfqpoint{4.650000in}{0.614151in}}%
\pgfusepath{clip}%
\pgfsetbuttcap%
\pgfsetroundjoin%
\definecolor{currentfill}{rgb}{0.865975,0.344406,0.344406}%
\pgfsetfillcolor{currentfill}%
\pgfsetlinewidth{0.250937pt}%
\definecolor{currentstroke}{rgb}{1.000000,1.000000,1.000000}%
\pgfsetstrokecolor{currentstroke}%
\pgfsetdash{}{0pt}%
\pgfpathmoveto{\pgfqpoint{3.978113in}{8.649340in}}%
\pgfpathlineto{\pgfqpoint{4.065849in}{8.649340in}}%
\pgfpathlineto{\pgfqpoint{4.065849in}{8.561604in}}%
\pgfpathlineto{\pgfqpoint{3.978113in}{8.561604in}}%
\pgfpathlineto{\pgfqpoint{3.978113in}{8.649340in}}%
\pgfusepath{stroke,fill}%
\end{pgfscope}%
\begin{pgfscope}%
\pgfpathrectangle{\pgfqpoint{0.380943in}{8.035189in}}{\pgfqpoint{4.650000in}{0.614151in}}%
\pgfusepath{clip}%
\pgfsetbuttcap%
\pgfsetroundjoin%
\definecolor{currentfill}{rgb}{0.961061,0.931672,0.728304}%
\pgfsetfillcolor{currentfill}%
\pgfsetlinewidth{0.250937pt}%
\definecolor{currentstroke}{rgb}{1.000000,1.000000,1.000000}%
\pgfsetstrokecolor{currentstroke}%
\pgfsetdash{}{0pt}%
\pgfpathmoveto{\pgfqpoint{4.065849in}{8.649340in}}%
\pgfpathlineto{\pgfqpoint{4.153585in}{8.649340in}}%
\pgfpathlineto{\pgfqpoint{4.153585in}{8.561604in}}%
\pgfpathlineto{\pgfqpoint{4.065849in}{8.561604in}}%
\pgfpathlineto{\pgfqpoint{4.065849in}{8.649340in}}%
\pgfusepath{stroke,fill}%
\end{pgfscope}%
\begin{pgfscope}%
\pgfpathrectangle{\pgfqpoint{0.380943in}{8.035189in}}{\pgfqpoint{4.650000in}{0.614151in}}%
\pgfusepath{clip}%
\pgfsetbuttcap%
\pgfsetroundjoin%
\definecolor{currentfill}{rgb}{0.999616,0.641369,0.546559}%
\pgfsetfillcolor{currentfill}%
\pgfsetlinewidth{0.250937pt}%
\definecolor{currentstroke}{rgb}{1.000000,1.000000,1.000000}%
\pgfsetstrokecolor{currentstroke}%
\pgfsetdash{}{0pt}%
\pgfpathmoveto{\pgfqpoint{4.153585in}{8.649340in}}%
\pgfpathlineto{\pgfqpoint{4.241320in}{8.649340in}}%
\pgfpathlineto{\pgfqpoint{4.241320in}{8.561604in}}%
\pgfpathlineto{\pgfqpoint{4.153585in}{8.561604in}}%
\pgfpathlineto{\pgfqpoint{4.153585in}{8.649340in}}%
\pgfusepath{stroke,fill}%
\end{pgfscope}%
\begin{pgfscope}%
\pgfpathrectangle{\pgfqpoint{0.380943in}{8.035189in}}{\pgfqpoint{4.650000in}{0.614151in}}%
\pgfusepath{clip}%
\pgfsetbuttcap%
\pgfsetroundjoin%
\definecolor{currentfill}{rgb}{0.978131,0.843783,0.675709}%
\pgfsetfillcolor{currentfill}%
\pgfsetlinewidth{0.250937pt}%
\definecolor{currentstroke}{rgb}{1.000000,1.000000,1.000000}%
\pgfsetstrokecolor{currentstroke}%
\pgfsetdash{}{0pt}%
\pgfpathmoveto{\pgfqpoint{4.241320in}{8.649340in}}%
\pgfpathlineto{\pgfqpoint{4.329056in}{8.649340in}}%
\pgfpathlineto{\pgfqpoint{4.329056in}{8.561604in}}%
\pgfpathlineto{\pgfqpoint{4.241320in}{8.561604in}}%
\pgfpathlineto{\pgfqpoint{4.241320in}{8.649340in}}%
\pgfusepath{stroke,fill}%
\end{pgfscope}%
\begin{pgfscope}%
\pgfpathrectangle{\pgfqpoint{0.380943in}{8.035189in}}{\pgfqpoint{4.650000in}{0.614151in}}%
\pgfusepath{clip}%
\pgfsetbuttcap%
\pgfsetroundjoin%
\definecolor{currentfill}{rgb}{0.986251,0.808597,0.643230}%
\pgfsetfillcolor{currentfill}%
\pgfsetlinewidth{0.250937pt}%
\definecolor{currentstroke}{rgb}{1.000000,1.000000,1.000000}%
\pgfsetstrokecolor{currentstroke}%
\pgfsetdash{}{0pt}%
\pgfpathmoveto{\pgfqpoint{4.329056in}{8.649340in}}%
\pgfpathlineto{\pgfqpoint{4.416792in}{8.649340in}}%
\pgfpathlineto{\pgfqpoint{4.416792in}{8.561604in}}%
\pgfpathlineto{\pgfqpoint{4.329056in}{8.561604in}}%
\pgfpathlineto{\pgfqpoint{4.329056in}{8.649340in}}%
\pgfusepath{stroke,fill}%
\end{pgfscope}%
\begin{pgfscope}%
\pgfpathrectangle{\pgfqpoint{0.380943in}{8.035189in}}{\pgfqpoint{4.650000in}{0.614151in}}%
\pgfusepath{clip}%
\pgfsetbuttcap%
\pgfsetroundjoin%
\definecolor{currentfill}{rgb}{0.992326,0.765229,0.614840}%
\pgfsetfillcolor{currentfill}%
\pgfsetlinewidth{0.250937pt}%
\definecolor{currentstroke}{rgb}{1.000000,1.000000,1.000000}%
\pgfsetstrokecolor{currentstroke}%
\pgfsetdash{}{0pt}%
\pgfpathmoveto{\pgfqpoint{4.416792in}{8.649340in}}%
\pgfpathlineto{\pgfqpoint{4.504528in}{8.649340in}}%
\pgfpathlineto{\pgfqpoint{4.504528in}{8.561604in}}%
\pgfpathlineto{\pgfqpoint{4.416792in}{8.561604in}}%
\pgfpathlineto{\pgfqpoint{4.416792in}{8.649340in}}%
\pgfusepath{stroke,fill}%
\end{pgfscope}%
\begin{pgfscope}%
\pgfpathrectangle{\pgfqpoint{0.380943in}{8.035189in}}{\pgfqpoint{4.650000in}{0.614151in}}%
\pgfusepath{clip}%
\pgfsetbuttcap%
\pgfsetroundjoin%
\definecolor{currentfill}{rgb}{0.986251,0.808597,0.643230}%
\pgfsetfillcolor{currentfill}%
\pgfsetlinewidth{0.250937pt}%
\definecolor{currentstroke}{rgb}{1.000000,1.000000,1.000000}%
\pgfsetstrokecolor{currentstroke}%
\pgfsetdash{}{0pt}%
\pgfpathmoveto{\pgfqpoint{4.504528in}{8.649340in}}%
\pgfpathlineto{\pgfqpoint{4.592264in}{8.649340in}}%
\pgfpathlineto{\pgfqpoint{4.592264in}{8.561604in}}%
\pgfpathlineto{\pgfqpoint{4.504528in}{8.561604in}}%
\pgfpathlineto{\pgfqpoint{4.504528in}{8.649340in}}%
\pgfusepath{stroke,fill}%
\end{pgfscope}%
\begin{pgfscope}%
\pgfpathrectangle{\pgfqpoint{0.380943in}{8.035189in}}{\pgfqpoint{4.650000in}{0.614151in}}%
\pgfusepath{clip}%
\pgfsetbuttcap%
\pgfsetroundjoin%
\definecolor{currentfill}{rgb}{0.996909,0.711742,0.584452}%
\pgfsetfillcolor{currentfill}%
\pgfsetlinewidth{0.250937pt}%
\definecolor{currentstroke}{rgb}{1.000000,1.000000,1.000000}%
\pgfsetstrokecolor{currentstroke}%
\pgfsetdash{}{0pt}%
\pgfpathmoveto{\pgfqpoint{4.592264in}{8.649340in}}%
\pgfpathlineto{\pgfqpoint{4.680000in}{8.649340in}}%
\pgfpathlineto{\pgfqpoint{4.680000in}{8.561604in}}%
\pgfpathlineto{\pgfqpoint{4.592264in}{8.561604in}}%
\pgfpathlineto{\pgfqpoint{4.592264in}{8.649340in}}%
\pgfusepath{stroke,fill}%
\end{pgfscope}%
\begin{pgfscope}%
\pgfpathrectangle{\pgfqpoint{0.380943in}{8.035189in}}{\pgfqpoint{4.650000in}{0.614151in}}%
\pgfusepath{clip}%
\pgfsetbuttcap%
\pgfsetroundjoin%
\definecolor{currentfill}{rgb}{0.978131,0.843783,0.675709}%
\pgfsetfillcolor{currentfill}%
\pgfsetlinewidth{0.250937pt}%
\definecolor{currentstroke}{rgb}{1.000000,1.000000,1.000000}%
\pgfsetstrokecolor{currentstroke}%
\pgfsetdash{}{0pt}%
\pgfpathmoveto{\pgfqpoint{4.680000in}{8.649340in}}%
\pgfpathlineto{\pgfqpoint{4.767736in}{8.649340in}}%
\pgfpathlineto{\pgfqpoint{4.767736in}{8.561604in}}%
\pgfpathlineto{\pgfqpoint{4.680000in}{8.561604in}}%
\pgfpathlineto{\pgfqpoint{4.680000in}{8.649340in}}%
\pgfusepath{stroke,fill}%
\end{pgfscope}%
\begin{pgfscope}%
\pgfpathrectangle{\pgfqpoint{0.380943in}{8.035189in}}{\pgfqpoint{4.650000in}{0.614151in}}%
\pgfusepath{clip}%
\pgfsetbuttcap%
\pgfsetroundjoin%
\definecolor{currentfill}{rgb}{1.000000,0.584929,0.522599}%
\pgfsetfillcolor{currentfill}%
\pgfsetlinewidth{0.250937pt}%
\definecolor{currentstroke}{rgb}{1.000000,1.000000,1.000000}%
\pgfsetstrokecolor{currentstroke}%
\pgfsetdash{}{0pt}%
\pgfpathmoveto{\pgfqpoint{4.767736in}{8.649340in}}%
\pgfpathlineto{\pgfqpoint{4.855471in}{8.649340in}}%
\pgfpathlineto{\pgfqpoint{4.855471in}{8.561604in}}%
\pgfpathlineto{\pgfqpoint{4.767736in}{8.561604in}}%
\pgfpathlineto{\pgfqpoint{4.767736in}{8.649340in}}%
\pgfusepath{stroke,fill}%
\end{pgfscope}%
\begin{pgfscope}%
\pgfpathrectangle{\pgfqpoint{0.380943in}{8.035189in}}{\pgfqpoint{4.650000in}{0.614151in}}%
\pgfusepath{clip}%
\pgfsetbuttcap%
\pgfsetroundjoin%
\definecolor{currentfill}{rgb}{0.970012,0.883276,0.699577}%
\pgfsetfillcolor{currentfill}%
\pgfsetlinewidth{0.250937pt}%
\definecolor{currentstroke}{rgb}{1.000000,1.000000,1.000000}%
\pgfsetstrokecolor{currentstroke}%
\pgfsetdash{}{0pt}%
\pgfpathmoveto{\pgfqpoint{4.855471in}{8.649340in}}%
\pgfpathlineto{\pgfqpoint{4.943207in}{8.649340in}}%
\pgfpathlineto{\pgfqpoint{4.943207in}{8.561604in}}%
\pgfpathlineto{\pgfqpoint{4.855471in}{8.561604in}}%
\pgfpathlineto{\pgfqpoint{4.855471in}{8.649340in}}%
\pgfusepath{stroke,fill}%
\end{pgfscope}%
\begin{pgfscope}%
\pgfpathrectangle{\pgfqpoint{0.380943in}{8.035189in}}{\pgfqpoint{4.650000in}{0.614151in}}%
\pgfusepath{clip}%
\pgfsetbuttcap%
\pgfsetroundjoin%
\definecolor{currentfill}{rgb}{0.985083,0.974641,0.792587}%
\pgfsetfillcolor{currentfill}%
\pgfsetlinewidth{0.250937pt}%
\definecolor{currentstroke}{rgb}{1.000000,1.000000,1.000000}%
\pgfsetstrokecolor{currentstroke}%
\pgfsetdash{}{0pt}%
\pgfpathmoveto{\pgfqpoint{4.943207in}{8.649340in}}%
\pgfpathlineto{\pgfqpoint{5.030943in}{8.649340in}}%
\pgfpathlineto{\pgfqpoint{5.030943in}{8.561604in}}%
\pgfpathlineto{\pgfqpoint{4.943207in}{8.561604in}}%
\pgfpathlineto{\pgfqpoint{4.943207in}{8.649340in}}%
\pgfusepath{stroke,fill}%
\end{pgfscope}%
\begin{pgfscope}%
\pgfpathrectangle{\pgfqpoint{0.380943in}{8.035189in}}{\pgfqpoint{4.650000in}{0.614151in}}%
\pgfusepath{clip}%
\pgfsetbuttcap%
\pgfsetroundjoin%
\pgfsetlinewidth{0.250937pt}%
\definecolor{currentstroke}{rgb}{1.000000,1.000000,1.000000}%
\pgfsetstrokecolor{currentstroke}%
\pgfsetdash{}{0pt}%
\pgfpathmoveto{\pgfqpoint{0.380943in}{8.561604in}}%
\pgfpathlineto{\pgfqpoint{0.468679in}{8.561604in}}%
\pgfpathlineto{\pgfqpoint{0.468679in}{8.473868in}}%
\pgfpathlineto{\pgfqpoint{0.380943in}{8.473868in}}%
\pgfpathlineto{\pgfqpoint{0.380943in}{8.561604in}}%
\pgfusepath{stroke}%
\end{pgfscope}%
\begin{pgfscope}%
\pgfpathrectangle{\pgfqpoint{0.380943in}{8.035189in}}{\pgfqpoint{4.650000in}{0.614151in}}%
\pgfusepath{clip}%
\pgfsetbuttcap%
\pgfsetroundjoin%
\definecolor{currentfill}{rgb}{1.000000,0.584929,0.522599}%
\pgfsetfillcolor{currentfill}%
\pgfsetlinewidth{0.250937pt}%
\definecolor{currentstroke}{rgb}{1.000000,1.000000,1.000000}%
\pgfsetstrokecolor{currentstroke}%
\pgfsetdash{}{0pt}%
\pgfpathmoveto{\pgfqpoint{0.468679in}{8.561604in}}%
\pgfpathlineto{\pgfqpoint{0.556415in}{8.561604in}}%
\pgfpathlineto{\pgfqpoint{0.556415in}{8.473868in}}%
\pgfpathlineto{\pgfqpoint{0.468679in}{8.473868in}}%
\pgfpathlineto{\pgfqpoint{0.468679in}{8.561604in}}%
\pgfusepath{stroke,fill}%
\end{pgfscope}%
\begin{pgfscope}%
\pgfpathrectangle{\pgfqpoint{0.380943in}{8.035189in}}{\pgfqpoint{4.650000in}{0.614151in}}%
\pgfusepath{clip}%
\pgfsetbuttcap%
\pgfsetroundjoin%
\definecolor{currentfill}{rgb}{0.996909,0.711742,0.584452}%
\pgfsetfillcolor{currentfill}%
\pgfsetlinewidth{0.250937pt}%
\definecolor{currentstroke}{rgb}{1.000000,1.000000,1.000000}%
\pgfsetstrokecolor{currentstroke}%
\pgfsetdash{}{0pt}%
\pgfpathmoveto{\pgfqpoint{0.556415in}{8.561604in}}%
\pgfpathlineto{\pgfqpoint{0.644151in}{8.561604in}}%
\pgfpathlineto{\pgfqpoint{0.644151in}{8.473868in}}%
\pgfpathlineto{\pgfqpoint{0.556415in}{8.473868in}}%
\pgfpathlineto{\pgfqpoint{0.556415in}{8.561604in}}%
\pgfusepath{stroke,fill}%
\end{pgfscope}%
\begin{pgfscope}%
\pgfpathrectangle{\pgfqpoint{0.380943in}{8.035189in}}{\pgfqpoint{4.650000in}{0.614151in}}%
\pgfusepath{clip}%
\pgfsetbuttcap%
\pgfsetroundjoin%
\definecolor{currentfill}{rgb}{0.986251,0.808597,0.643230}%
\pgfsetfillcolor{currentfill}%
\pgfsetlinewidth{0.250937pt}%
\definecolor{currentstroke}{rgb}{1.000000,1.000000,1.000000}%
\pgfsetstrokecolor{currentstroke}%
\pgfsetdash{}{0pt}%
\pgfpathmoveto{\pgfqpoint{0.644151in}{8.561604in}}%
\pgfpathlineto{\pgfqpoint{0.731886in}{8.561604in}}%
\pgfpathlineto{\pgfqpoint{0.731886in}{8.473868in}}%
\pgfpathlineto{\pgfqpoint{0.644151in}{8.473868in}}%
\pgfpathlineto{\pgfqpoint{0.644151in}{8.561604in}}%
\pgfusepath{stroke,fill}%
\end{pgfscope}%
\begin{pgfscope}%
\pgfpathrectangle{\pgfqpoint{0.380943in}{8.035189in}}{\pgfqpoint{4.650000in}{0.614151in}}%
\pgfusepath{clip}%
\pgfsetbuttcap%
\pgfsetroundjoin%
\definecolor{currentfill}{rgb}{0.961061,0.931672,0.728304}%
\pgfsetfillcolor{currentfill}%
\pgfsetlinewidth{0.250937pt}%
\definecolor{currentstroke}{rgb}{1.000000,1.000000,1.000000}%
\pgfsetstrokecolor{currentstroke}%
\pgfsetdash{}{0pt}%
\pgfpathmoveto{\pgfqpoint{0.731886in}{8.561604in}}%
\pgfpathlineto{\pgfqpoint{0.819622in}{8.561604in}}%
\pgfpathlineto{\pgfqpoint{0.819622in}{8.473868in}}%
\pgfpathlineto{\pgfqpoint{0.731886in}{8.473868in}}%
\pgfpathlineto{\pgfqpoint{0.731886in}{8.561604in}}%
\pgfusepath{stroke,fill}%
\end{pgfscope}%
\begin{pgfscope}%
\pgfpathrectangle{\pgfqpoint{0.380943in}{8.035189in}}{\pgfqpoint{4.650000in}{0.614151in}}%
\pgfusepath{clip}%
\pgfsetbuttcap%
\pgfsetroundjoin%
\definecolor{currentfill}{rgb}{0.999616,0.641369,0.546559}%
\pgfsetfillcolor{currentfill}%
\pgfsetlinewidth{0.250937pt}%
\definecolor{currentstroke}{rgb}{1.000000,1.000000,1.000000}%
\pgfsetstrokecolor{currentstroke}%
\pgfsetdash{}{0pt}%
\pgfpathmoveto{\pgfqpoint{0.819622in}{8.561604in}}%
\pgfpathlineto{\pgfqpoint{0.907358in}{8.561604in}}%
\pgfpathlineto{\pgfqpoint{0.907358in}{8.473868in}}%
\pgfpathlineto{\pgfqpoint{0.819622in}{8.473868in}}%
\pgfpathlineto{\pgfqpoint{0.819622in}{8.561604in}}%
\pgfusepath{stroke,fill}%
\end{pgfscope}%
\begin{pgfscope}%
\pgfpathrectangle{\pgfqpoint{0.380943in}{8.035189in}}{\pgfqpoint{4.650000in}{0.614151in}}%
\pgfusepath{clip}%
\pgfsetbuttcap%
\pgfsetroundjoin%
\definecolor{currentfill}{rgb}{1.000000,0.480477,0.479293}%
\pgfsetfillcolor{currentfill}%
\pgfsetlinewidth{0.250937pt}%
\definecolor{currentstroke}{rgb}{1.000000,1.000000,1.000000}%
\pgfsetstrokecolor{currentstroke}%
\pgfsetdash{}{0pt}%
\pgfpathmoveto{\pgfqpoint{0.907358in}{8.561604in}}%
\pgfpathlineto{\pgfqpoint{0.995094in}{8.561604in}}%
\pgfpathlineto{\pgfqpoint{0.995094in}{8.473868in}}%
\pgfpathlineto{\pgfqpoint{0.907358in}{8.473868in}}%
\pgfpathlineto{\pgfqpoint{0.907358in}{8.561604in}}%
\pgfusepath{stroke,fill}%
\end{pgfscope}%
\begin{pgfscope}%
\pgfpathrectangle{\pgfqpoint{0.380943in}{8.035189in}}{\pgfqpoint{4.650000in}{0.614151in}}%
\pgfusepath{clip}%
\pgfsetbuttcap%
\pgfsetroundjoin%
\definecolor{currentfill}{rgb}{0.970012,0.883276,0.699577}%
\pgfsetfillcolor{currentfill}%
\pgfsetlinewidth{0.250937pt}%
\definecolor{currentstroke}{rgb}{1.000000,1.000000,1.000000}%
\pgfsetstrokecolor{currentstroke}%
\pgfsetdash{}{0pt}%
\pgfpathmoveto{\pgfqpoint{0.995094in}{8.561604in}}%
\pgfpathlineto{\pgfqpoint{1.082830in}{8.561604in}}%
\pgfpathlineto{\pgfqpoint{1.082830in}{8.473868in}}%
\pgfpathlineto{\pgfqpoint{0.995094in}{8.473868in}}%
\pgfpathlineto{\pgfqpoint{0.995094in}{8.561604in}}%
\pgfusepath{stroke,fill}%
\end{pgfscope}%
\begin{pgfscope}%
\pgfpathrectangle{\pgfqpoint{0.380943in}{8.035189in}}{\pgfqpoint{4.650000in}{0.614151in}}%
\pgfusepath{clip}%
\pgfsetbuttcap%
\pgfsetroundjoin%
\definecolor{currentfill}{rgb}{0.970012,0.883276,0.699577}%
\pgfsetfillcolor{currentfill}%
\pgfsetlinewidth{0.250937pt}%
\definecolor{currentstroke}{rgb}{1.000000,1.000000,1.000000}%
\pgfsetstrokecolor{currentstroke}%
\pgfsetdash{}{0pt}%
\pgfpathmoveto{\pgfqpoint{1.082830in}{8.561604in}}%
\pgfpathlineto{\pgfqpoint{1.170566in}{8.561604in}}%
\pgfpathlineto{\pgfqpoint{1.170566in}{8.473868in}}%
\pgfpathlineto{\pgfqpoint{1.082830in}{8.473868in}}%
\pgfpathlineto{\pgfqpoint{1.082830in}{8.561604in}}%
\pgfusepath{stroke,fill}%
\end{pgfscope}%
\begin{pgfscope}%
\pgfpathrectangle{\pgfqpoint{0.380943in}{8.035189in}}{\pgfqpoint{4.650000in}{0.614151in}}%
\pgfusepath{clip}%
\pgfsetbuttcap%
\pgfsetroundjoin%
\definecolor{currentfill}{rgb}{1.000000,0.480477,0.479293}%
\pgfsetfillcolor{currentfill}%
\pgfsetlinewidth{0.250937pt}%
\definecolor{currentstroke}{rgb}{1.000000,1.000000,1.000000}%
\pgfsetstrokecolor{currentstroke}%
\pgfsetdash{}{0pt}%
\pgfpathmoveto{\pgfqpoint{1.170566in}{8.561604in}}%
\pgfpathlineto{\pgfqpoint{1.258302in}{8.561604in}}%
\pgfpathlineto{\pgfqpoint{1.258302in}{8.473868in}}%
\pgfpathlineto{\pgfqpoint{1.170566in}{8.473868in}}%
\pgfpathlineto{\pgfqpoint{1.170566in}{8.561604in}}%
\pgfusepath{stroke,fill}%
\end{pgfscope}%
\begin{pgfscope}%
\pgfpathrectangle{\pgfqpoint{0.380943in}{8.035189in}}{\pgfqpoint{4.650000in}{0.614151in}}%
\pgfusepath{clip}%
\pgfsetbuttcap%
\pgfsetroundjoin%
\definecolor{currentfill}{rgb}{0.970012,0.883276,0.699577}%
\pgfsetfillcolor{currentfill}%
\pgfsetlinewidth{0.250937pt}%
\definecolor{currentstroke}{rgb}{1.000000,1.000000,1.000000}%
\pgfsetstrokecolor{currentstroke}%
\pgfsetdash{}{0pt}%
\pgfpathmoveto{\pgfqpoint{1.258302in}{8.561604in}}%
\pgfpathlineto{\pgfqpoint{1.346037in}{8.561604in}}%
\pgfpathlineto{\pgfqpoint{1.346037in}{8.473868in}}%
\pgfpathlineto{\pgfqpoint{1.258302in}{8.473868in}}%
\pgfpathlineto{\pgfqpoint{1.258302in}{8.561604in}}%
\pgfusepath{stroke,fill}%
\end{pgfscope}%
\begin{pgfscope}%
\pgfpathrectangle{\pgfqpoint{0.380943in}{8.035189in}}{\pgfqpoint{4.650000in}{0.614151in}}%
\pgfusepath{clip}%
\pgfsetbuttcap%
\pgfsetroundjoin%
\definecolor{currentfill}{rgb}{0.999616,0.641369,0.546559}%
\pgfsetfillcolor{currentfill}%
\pgfsetlinewidth{0.250937pt}%
\definecolor{currentstroke}{rgb}{1.000000,1.000000,1.000000}%
\pgfsetstrokecolor{currentstroke}%
\pgfsetdash{}{0pt}%
\pgfpathmoveto{\pgfqpoint{1.346037in}{8.561604in}}%
\pgfpathlineto{\pgfqpoint{1.433773in}{8.561604in}}%
\pgfpathlineto{\pgfqpoint{1.433773in}{8.473868in}}%
\pgfpathlineto{\pgfqpoint{1.346037in}{8.473868in}}%
\pgfpathlineto{\pgfqpoint{1.346037in}{8.561604in}}%
\pgfusepath{stroke,fill}%
\end{pgfscope}%
\begin{pgfscope}%
\pgfpathrectangle{\pgfqpoint{0.380943in}{8.035189in}}{\pgfqpoint{4.650000in}{0.614151in}}%
\pgfusepath{clip}%
\pgfsetbuttcap%
\pgfsetroundjoin%
\definecolor{currentfill}{rgb}{0.970012,0.883276,0.699577}%
\pgfsetfillcolor{currentfill}%
\pgfsetlinewidth{0.250937pt}%
\definecolor{currentstroke}{rgb}{1.000000,1.000000,1.000000}%
\pgfsetstrokecolor{currentstroke}%
\pgfsetdash{}{0pt}%
\pgfpathmoveto{\pgfqpoint{1.433773in}{8.561604in}}%
\pgfpathlineto{\pgfqpoint{1.521509in}{8.561604in}}%
\pgfpathlineto{\pgfqpoint{1.521509in}{8.473868in}}%
\pgfpathlineto{\pgfqpoint{1.433773in}{8.473868in}}%
\pgfpathlineto{\pgfqpoint{1.433773in}{8.561604in}}%
\pgfusepath{stroke,fill}%
\end{pgfscope}%
\begin{pgfscope}%
\pgfpathrectangle{\pgfqpoint{0.380943in}{8.035189in}}{\pgfqpoint{4.650000in}{0.614151in}}%
\pgfusepath{clip}%
\pgfsetbuttcap%
\pgfsetroundjoin%
\definecolor{currentfill}{rgb}{0.999616,0.641369,0.546559}%
\pgfsetfillcolor{currentfill}%
\pgfsetlinewidth{0.250937pt}%
\definecolor{currentstroke}{rgb}{1.000000,1.000000,1.000000}%
\pgfsetstrokecolor{currentstroke}%
\pgfsetdash{}{0pt}%
\pgfpathmoveto{\pgfqpoint{1.521509in}{8.561604in}}%
\pgfpathlineto{\pgfqpoint{1.609245in}{8.561604in}}%
\pgfpathlineto{\pgfqpoint{1.609245in}{8.473868in}}%
\pgfpathlineto{\pgfqpoint{1.521509in}{8.473868in}}%
\pgfpathlineto{\pgfqpoint{1.521509in}{8.561604in}}%
\pgfusepath{stroke,fill}%
\end{pgfscope}%
\begin{pgfscope}%
\pgfpathrectangle{\pgfqpoint{0.380943in}{8.035189in}}{\pgfqpoint{4.650000in}{0.614151in}}%
\pgfusepath{clip}%
\pgfsetbuttcap%
\pgfsetroundjoin%
\definecolor{currentfill}{rgb}{0.996909,0.711742,0.584452}%
\pgfsetfillcolor{currentfill}%
\pgfsetlinewidth{0.250937pt}%
\definecolor{currentstroke}{rgb}{1.000000,1.000000,1.000000}%
\pgfsetstrokecolor{currentstroke}%
\pgfsetdash{}{0pt}%
\pgfpathmoveto{\pgfqpoint{1.609245in}{8.561604in}}%
\pgfpathlineto{\pgfqpoint{1.696981in}{8.561604in}}%
\pgfpathlineto{\pgfqpoint{1.696981in}{8.473868in}}%
\pgfpathlineto{\pgfqpoint{1.609245in}{8.473868in}}%
\pgfpathlineto{\pgfqpoint{1.609245in}{8.561604in}}%
\pgfusepath{stroke,fill}%
\end{pgfscope}%
\begin{pgfscope}%
\pgfpathrectangle{\pgfqpoint{0.380943in}{8.035189in}}{\pgfqpoint{4.650000in}{0.614151in}}%
\pgfusepath{clip}%
\pgfsetbuttcap%
\pgfsetroundjoin%
\definecolor{currentfill}{rgb}{0.978131,0.843783,0.675709}%
\pgfsetfillcolor{currentfill}%
\pgfsetlinewidth{0.250937pt}%
\definecolor{currentstroke}{rgb}{1.000000,1.000000,1.000000}%
\pgfsetstrokecolor{currentstroke}%
\pgfsetdash{}{0pt}%
\pgfpathmoveto{\pgfqpoint{1.696981in}{8.561604in}}%
\pgfpathlineto{\pgfqpoint{1.784717in}{8.561604in}}%
\pgfpathlineto{\pgfqpoint{1.784717in}{8.473868in}}%
\pgfpathlineto{\pgfqpoint{1.696981in}{8.473868in}}%
\pgfpathlineto{\pgfqpoint{1.696981in}{8.561604in}}%
\pgfusepath{stroke,fill}%
\end{pgfscope}%
\begin{pgfscope}%
\pgfpathrectangle{\pgfqpoint{0.380943in}{8.035189in}}{\pgfqpoint{4.650000in}{0.614151in}}%
\pgfusepath{clip}%
\pgfsetbuttcap%
\pgfsetroundjoin%
\definecolor{currentfill}{rgb}{0.970012,0.883276,0.699577}%
\pgfsetfillcolor{currentfill}%
\pgfsetlinewidth{0.250937pt}%
\definecolor{currentstroke}{rgb}{1.000000,1.000000,1.000000}%
\pgfsetstrokecolor{currentstroke}%
\pgfsetdash{}{0pt}%
\pgfpathmoveto{\pgfqpoint{1.784717in}{8.561604in}}%
\pgfpathlineto{\pgfqpoint{1.872452in}{8.561604in}}%
\pgfpathlineto{\pgfqpoint{1.872452in}{8.473868in}}%
\pgfpathlineto{\pgfqpoint{1.784717in}{8.473868in}}%
\pgfpathlineto{\pgfqpoint{1.784717in}{8.561604in}}%
\pgfusepath{stroke,fill}%
\end{pgfscope}%
\begin{pgfscope}%
\pgfpathrectangle{\pgfqpoint{0.380943in}{8.035189in}}{\pgfqpoint{4.650000in}{0.614151in}}%
\pgfusepath{clip}%
\pgfsetbuttcap%
\pgfsetroundjoin%
\definecolor{currentfill}{rgb}{0.999616,0.641369,0.546559}%
\pgfsetfillcolor{currentfill}%
\pgfsetlinewidth{0.250937pt}%
\definecolor{currentstroke}{rgb}{1.000000,1.000000,1.000000}%
\pgfsetstrokecolor{currentstroke}%
\pgfsetdash{}{0pt}%
\pgfpathmoveto{\pgfqpoint{1.872452in}{8.561604in}}%
\pgfpathlineto{\pgfqpoint{1.960188in}{8.561604in}}%
\pgfpathlineto{\pgfqpoint{1.960188in}{8.473868in}}%
\pgfpathlineto{\pgfqpoint{1.872452in}{8.473868in}}%
\pgfpathlineto{\pgfqpoint{1.872452in}{8.561604in}}%
\pgfusepath{stroke,fill}%
\end{pgfscope}%
\begin{pgfscope}%
\pgfpathrectangle{\pgfqpoint{0.380943in}{8.035189in}}{\pgfqpoint{4.650000in}{0.614151in}}%
\pgfusepath{clip}%
\pgfsetbuttcap%
\pgfsetroundjoin%
\definecolor{currentfill}{rgb}{0.963768,0.915433,0.717478}%
\pgfsetfillcolor{currentfill}%
\pgfsetlinewidth{0.250937pt}%
\definecolor{currentstroke}{rgb}{1.000000,1.000000,1.000000}%
\pgfsetstrokecolor{currentstroke}%
\pgfsetdash{}{0pt}%
\pgfpathmoveto{\pgfqpoint{1.960188in}{8.561604in}}%
\pgfpathlineto{\pgfqpoint{2.047924in}{8.561604in}}%
\pgfpathlineto{\pgfqpoint{2.047924in}{8.473868in}}%
\pgfpathlineto{\pgfqpoint{1.960188in}{8.473868in}}%
\pgfpathlineto{\pgfqpoint{1.960188in}{8.561604in}}%
\pgfusepath{stroke,fill}%
\end{pgfscope}%
\begin{pgfscope}%
\pgfpathrectangle{\pgfqpoint{0.380943in}{8.035189in}}{\pgfqpoint{4.650000in}{0.614151in}}%
\pgfusepath{clip}%
\pgfsetbuttcap%
\pgfsetroundjoin%
\definecolor{currentfill}{rgb}{0.961061,0.931672,0.728304}%
\pgfsetfillcolor{currentfill}%
\pgfsetlinewidth{0.250937pt}%
\definecolor{currentstroke}{rgb}{1.000000,1.000000,1.000000}%
\pgfsetstrokecolor{currentstroke}%
\pgfsetdash{}{0pt}%
\pgfpathmoveto{\pgfqpoint{2.047924in}{8.561604in}}%
\pgfpathlineto{\pgfqpoint{2.135660in}{8.561604in}}%
\pgfpathlineto{\pgfqpoint{2.135660in}{8.473868in}}%
\pgfpathlineto{\pgfqpoint{2.047924in}{8.473868in}}%
\pgfpathlineto{\pgfqpoint{2.047924in}{8.561604in}}%
\pgfusepath{stroke,fill}%
\end{pgfscope}%
\begin{pgfscope}%
\pgfpathrectangle{\pgfqpoint{0.380943in}{8.035189in}}{\pgfqpoint{4.650000in}{0.614151in}}%
\pgfusepath{clip}%
\pgfsetbuttcap%
\pgfsetroundjoin%
\definecolor{currentfill}{rgb}{0.978131,0.843783,0.675709}%
\pgfsetfillcolor{currentfill}%
\pgfsetlinewidth{0.250937pt}%
\definecolor{currentstroke}{rgb}{1.000000,1.000000,1.000000}%
\pgfsetstrokecolor{currentstroke}%
\pgfsetdash{}{0pt}%
\pgfpathmoveto{\pgfqpoint{2.135660in}{8.561604in}}%
\pgfpathlineto{\pgfqpoint{2.223396in}{8.561604in}}%
\pgfpathlineto{\pgfqpoint{2.223396in}{8.473868in}}%
\pgfpathlineto{\pgfqpoint{2.135660in}{8.473868in}}%
\pgfpathlineto{\pgfqpoint{2.135660in}{8.561604in}}%
\pgfusepath{stroke,fill}%
\end{pgfscope}%
\begin{pgfscope}%
\pgfpathrectangle{\pgfqpoint{0.380943in}{8.035189in}}{\pgfqpoint{4.650000in}{0.614151in}}%
\pgfusepath{clip}%
\pgfsetbuttcap%
\pgfsetroundjoin%
\definecolor{currentfill}{rgb}{1.000000,0.584929,0.522599}%
\pgfsetfillcolor{currentfill}%
\pgfsetlinewidth{0.250937pt}%
\definecolor{currentstroke}{rgb}{1.000000,1.000000,1.000000}%
\pgfsetstrokecolor{currentstroke}%
\pgfsetdash{}{0pt}%
\pgfpathmoveto{\pgfqpoint{2.223396in}{8.561604in}}%
\pgfpathlineto{\pgfqpoint{2.311132in}{8.561604in}}%
\pgfpathlineto{\pgfqpoint{2.311132in}{8.473868in}}%
\pgfpathlineto{\pgfqpoint{2.223396in}{8.473868in}}%
\pgfpathlineto{\pgfqpoint{2.223396in}{8.561604in}}%
\pgfusepath{stroke,fill}%
\end{pgfscope}%
\begin{pgfscope}%
\pgfpathrectangle{\pgfqpoint{0.380943in}{8.035189in}}{\pgfqpoint{4.650000in}{0.614151in}}%
\pgfusepath{clip}%
\pgfsetbuttcap%
\pgfsetroundjoin%
\definecolor{currentfill}{rgb}{0.961061,0.931672,0.728304}%
\pgfsetfillcolor{currentfill}%
\pgfsetlinewidth{0.250937pt}%
\definecolor{currentstroke}{rgb}{1.000000,1.000000,1.000000}%
\pgfsetstrokecolor{currentstroke}%
\pgfsetdash{}{0pt}%
\pgfpathmoveto{\pgfqpoint{2.311132in}{8.561604in}}%
\pgfpathlineto{\pgfqpoint{2.398868in}{8.561604in}}%
\pgfpathlineto{\pgfqpoint{2.398868in}{8.473868in}}%
\pgfpathlineto{\pgfqpoint{2.311132in}{8.473868in}}%
\pgfpathlineto{\pgfqpoint{2.311132in}{8.561604in}}%
\pgfusepath{stroke,fill}%
\end{pgfscope}%
\begin{pgfscope}%
\pgfpathrectangle{\pgfqpoint{0.380943in}{8.035189in}}{\pgfqpoint{4.650000in}{0.614151in}}%
\pgfusepath{clip}%
\pgfsetbuttcap%
\pgfsetroundjoin%
\definecolor{currentfill}{rgb}{0.970012,0.883276,0.699577}%
\pgfsetfillcolor{currentfill}%
\pgfsetlinewidth{0.250937pt}%
\definecolor{currentstroke}{rgb}{1.000000,1.000000,1.000000}%
\pgfsetstrokecolor{currentstroke}%
\pgfsetdash{}{0pt}%
\pgfpathmoveto{\pgfqpoint{2.398868in}{8.561604in}}%
\pgfpathlineto{\pgfqpoint{2.486603in}{8.561604in}}%
\pgfpathlineto{\pgfqpoint{2.486603in}{8.473868in}}%
\pgfpathlineto{\pgfqpoint{2.398868in}{8.473868in}}%
\pgfpathlineto{\pgfqpoint{2.398868in}{8.561604in}}%
\pgfusepath{stroke,fill}%
\end{pgfscope}%
\begin{pgfscope}%
\pgfpathrectangle{\pgfqpoint{0.380943in}{8.035189in}}{\pgfqpoint{4.650000in}{0.614151in}}%
\pgfusepath{clip}%
\pgfsetbuttcap%
\pgfsetroundjoin%
\definecolor{currentfill}{rgb}{0.970012,0.883276,0.699577}%
\pgfsetfillcolor{currentfill}%
\pgfsetlinewidth{0.250937pt}%
\definecolor{currentstroke}{rgb}{1.000000,1.000000,1.000000}%
\pgfsetstrokecolor{currentstroke}%
\pgfsetdash{}{0pt}%
\pgfpathmoveto{\pgfqpoint{2.486603in}{8.561604in}}%
\pgfpathlineto{\pgfqpoint{2.574339in}{8.561604in}}%
\pgfpathlineto{\pgfqpoint{2.574339in}{8.473868in}}%
\pgfpathlineto{\pgfqpoint{2.486603in}{8.473868in}}%
\pgfpathlineto{\pgfqpoint{2.486603in}{8.561604in}}%
\pgfusepath{stroke,fill}%
\end{pgfscope}%
\begin{pgfscope}%
\pgfpathrectangle{\pgfqpoint{0.380943in}{8.035189in}}{\pgfqpoint{4.650000in}{0.614151in}}%
\pgfusepath{clip}%
\pgfsetbuttcap%
\pgfsetroundjoin%
\definecolor{currentfill}{rgb}{0.992326,0.765229,0.614840}%
\pgfsetfillcolor{currentfill}%
\pgfsetlinewidth{0.250937pt}%
\definecolor{currentstroke}{rgb}{1.000000,1.000000,1.000000}%
\pgfsetstrokecolor{currentstroke}%
\pgfsetdash{}{0pt}%
\pgfpathmoveto{\pgfqpoint{2.574339in}{8.561604in}}%
\pgfpathlineto{\pgfqpoint{2.662075in}{8.561604in}}%
\pgfpathlineto{\pgfqpoint{2.662075in}{8.473868in}}%
\pgfpathlineto{\pgfqpoint{2.574339in}{8.473868in}}%
\pgfpathlineto{\pgfqpoint{2.574339in}{8.561604in}}%
\pgfusepath{stroke,fill}%
\end{pgfscope}%
\begin{pgfscope}%
\pgfpathrectangle{\pgfqpoint{0.380943in}{8.035189in}}{\pgfqpoint{4.650000in}{0.614151in}}%
\pgfusepath{clip}%
\pgfsetbuttcap%
\pgfsetroundjoin%
\definecolor{currentfill}{rgb}{0.970012,0.883276,0.699577}%
\pgfsetfillcolor{currentfill}%
\pgfsetlinewidth{0.250937pt}%
\definecolor{currentstroke}{rgb}{1.000000,1.000000,1.000000}%
\pgfsetstrokecolor{currentstroke}%
\pgfsetdash{}{0pt}%
\pgfpathmoveto{\pgfqpoint{2.662075in}{8.561604in}}%
\pgfpathlineto{\pgfqpoint{2.749811in}{8.561604in}}%
\pgfpathlineto{\pgfqpoint{2.749811in}{8.473868in}}%
\pgfpathlineto{\pgfqpoint{2.662075in}{8.473868in}}%
\pgfpathlineto{\pgfqpoint{2.662075in}{8.561604in}}%
\pgfusepath{stroke,fill}%
\end{pgfscope}%
\begin{pgfscope}%
\pgfpathrectangle{\pgfqpoint{0.380943in}{8.035189in}}{\pgfqpoint{4.650000in}{0.614151in}}%
\pgfusepath{clip}%
\pgfsetbuttcap%
\pgfsetroundjoin%
\definecolor{currentfill}{rgb}{0.961061,0.931672,0.728304}%
\pgfsetfillcolor{currentfill}%
\pgfsetlinewidth{0.250937pt}%
\definecolor{currentstroke}{rgb}{1.000000,1.000000,1.000000}%
\pgfsetstrokecolor{currentstroke}%
\pgfsetdash{}{0pt}%
\pgfpathmoveto{\pgfqpoint{2.749811in}{8.561604in}}%
\pgfpathlineto{\pgfqpoint{2.837547in}{8.561604in}}%
\pgfpathlineto{\pgfqpoint{2.837547in}{8.473868in}}%
\pgfpathlineto{\pgfqpoint{2.749811in}{8.473868in}}%
\pgfpathlineto{\pgfqpoint{2.749811in}{8.561604in}}%
\pgfusepath{stroke,fill}%
\end{pgfscope}%
\begin{pgfscope}%
\pgfpathrectangle{\pgfqpoint{0.380943in}{8.035189in}}{\pgfqpoint{4.650000in}{0.614151in}}%
\pgfusepath{clip}%
\pgfsetbuttcap%
\pgfsetroundjoin%
\definecolor{currentfill}{rgb}{0.999616,0.641369,0.546559}%
\pgfsetfillcolor{currentfill}%
\pgfsetlinewidth{0.250937pt}%
\definecolor{currentstroke}{rgb}{1.000000,1.000000,1.000000}%
\pgfsetstrokecolor{currentstroke}%
\pgfsetdash{}{0pt}%
\pgfpathmoveto{\pgfqpoint{2.837547in}{8.561604in}}%
\pgfpathlineto{\pgfqpoint{2.925283in}{8.561604in}}%
\pgfpathlineto{\pgfqpoint{2.925283in}{8.473868in}}%
\pgfpathlineto{\pgfqpoint{2.837547in}{8.473868in}}%
\pgfpathlineto{\pgfqpoint{2.837547in}{8.561604in}}%
\pgfusepath{stroke,fill}%
\end{pgfscope}%
\begin{pgfscope}%
\pgfpathrectangle{\pgfqpoint{0.380943in}{8.035189in}}{\pgfqpoint{4.650000in}{0.614151in}}%
\pgfusepath{clip}%
\pgfsetbuttcap%
\pgfsetroundjoin%
\definecolor{currentfill}{rgb}{0.978131,0.843783,0.675709}%
\pgfsetfillcolor{currentfill}%
\pgfsetlinewidth{0.250937pt}%
\definecolor{currentstroke}{rgb}{1.000000,1.000000,1.000000}%
\pgfsetstrokecolor{currentstroke}%
\pgfsetdash{}{0pt}%
\pgfpathmoveto{\pgfqpoint{2.925283in}{8.561604in}}%
\pgfpathlineto{\pgfqpoint{3.013019in}{8.561604in}}%
\pgfpathlineto{\pgfqpoint{3.013019in}{8.473868in}}%
\pgfpathlineto{\pgfqpoint{2.925283in}{8.473868in}}%
\pgfpathlineto{\pgfqpoint{2.925283in}{8.561604in}}%
\pgfusepath{stroke,fill}%
\end{pgfscope}%
\begin{pgfscope}%
\pgfpathrectangle{\pgfqpoint{0.380943in}{8.035189in}}{\pgfqpoint{4.650000in}{0.614151in}}%
\pgfusepath{clip}%
\pgfsetbuttcap%
\pgfsetroundjoin%
\definecolor{currentfill}{rgb}{0.999616,0.641369,0.546559}%
\pgfsetfillcolor{currentfill}%
\pgfsetlinewidth{0.250937pt}%
\definecolor{currentstroke}{rgb}{1.000000,1.000000,1.000000}%
\pgfsetstrokecolor{currentstroke}%
\pgfsetdash{}{0pt}%
\pgfpathmoveto{\pgfqpoint{3.013019in}{8.561604in}}%
\pgfpathlineto{\pgfqpoint{3.100754in}{8.561604in}}%
\pgfpathlineto{\pgfqpoint{3.100754in}{8.473868in}}%
\pgfpathlineto{\pgfqpoint{3.013019in}{8.473868in}}%
\pgfpathlineto{\pgfqpoint{3.013019in}{8.561604in}}%
\pgfusepath{stroke,fill}%
\end{pgfscope}%
\begin{pgfscope}%
\pgfpathrectangle{\pgfqpoint{0.380943in}{8.035189in}}{\pgfqpoint{4.650000in}{0.614151in}}%
\pgfusepath{clip}%
\pgfsetbuttcap%
\pgfsetroundjoin%
\definecolor{currentfill}{rgb}{0.961061,0.931672,0.728304}%
\pgfsetfillcolor{currentfill}%
\pgfsetlinewidth{0.250937pt}%
\definecolor{currentstroke}{rgb}{1.000000,1.000000,1.000000}%
\pgfsetstrokecolor{currentstroke}%
\pgfsetdash{}{0pt}%
\pgfpathmoveto{\pgfqpoint{3.100754in}{8.561604in}}%
\pgfpathlineto{\pgfqpoint{3.188490in}{8.561604in}}%
\pgfpathlineto{\pgfqpoint{3.188490in}{8.473868in}}%
\pgfpathlineto{\pgfqpoint{3.100754in}{8.473868in}}%
\pgfpathlineto{\pgfqpoint{3.100754in}{8.561604in}}%
\pgfusepath{stroke,fill}%
\end{pgfscope}%
\begin{pgfscope}%
\pgfpathrectangle{\pgfqpoint{0.380943in}{8.035189in}}{\pgfqpoint{4.650000in}{0.614151in}}%
\pgfusepath{clip}%
\pgfsetbuttcap%
\pgfsetroundjoin%
\definecolor{currentfill}{rgb}{0.978131,0.843783,0.675709}%
\pgfsetfillcolor{currentfill}%
\pgfsetlinewidth{0.250937pt}%
\definecolor{currentstroke}{rgb}{1.000000,1.000000,1.000000}%
\pgfsetstrokecolor{currentstroke}%
\pgfsetdash{}{0pt}%
\pgfpathmoveto{\pgfqpoint{3.188490in}{8.561604in}}%
\pgfpathlineto{\pgfqpoint{3.276226in}{8.561604in}}%
\pgfpathlineto{\pgfqpoint{3.276226in}{8.473868in}}%
\pgfpathlineto{\pgfqpoint{3.188490in}{8.473868in}}%
\pgfpathlineto{\pgfqpoint{3.188490in}{8.561604in}}%
\pgfusepath{stroke,fill}%
\end{pgfscope}%
\begin{pgfscope}%
\pgfpathrectangle{\pgfqpoint{0.380943in}{8.035189in}}{\pgfqpoint{4.650000in}{0.614151in}}%
\pgfusepath{clip}%
\pgfsetbuttcap%
\pgfsetroundjoin%
\definecolor{currentfill}{rgb}{0.961061,0.931672,0.728304}%
\pgfsetfillcolor{currentfill}%
\pgfsetlinewidth{0.250937pt}%
\definecolor{currentstroke}{rgb}{1.000000,1.000000,1.000000}%
\pgfsetstrokecolor{currentstroke}%
\pgfsetdash{}{0pt}%
\pgfpathmoveto{\pgfqpoint{3.276226in}{8.561604in}}%
\pgfpathlineto{\pgfqpoint{3.363962in}{8.561604in}}%
\pgfpathlineto{\pgfqpoint{3.363962in}{8.473868in}}%
\pgfpathlineto{\pgfqpoint{3.276226in}{8.473868in}}%
\pgfpathlineto{\pgfqpoint{3.276226in}{8.561604in}}%
\pgfusepath{stroke,fill}%
\end{pgfscope}%
\begin{pgfscope}%
\pgfpathrectangle{\pgfqpoint{0.380943in}{8.035189in}}{\pgfqpoint{4.650000in}{0.614151in}}%
\pgfusepath{clip}%
\pgfsetbuttcap%
\pgfsetroundjoin%
\definecolor{currentfill}{rgb}{0.970012,0.883276,0.699577}%
\pgfsetfillcolor{currentfill}%
\pgfsetlinewidth{0.250937pt}%
\definecolor{currentstroke}{rgb}{1.000000,1.000000,1.000000}%
\pgfsetstrokecolor{currentstroke}%
\pgfsetdash{}{0pt}%
\pgfpathmoveto{\pgfqpoint{3.363962in}{8.561604in}}%
\pgfpathlineto{\pgfqpoint{3.451698in}{8.561604in}}%
\pgfpathlineto{\pgfqpoint{3.451698in}{8.473868in}}%
\pgfpathlineto{\pgfqpoint{3.363962in}{8.473868in}}%
\pgfpathlineto{\pgfqpoint{3.363962in}{8.561604in}}%
\pgfusepath{stroke,fill}%
\end{pgfscope}%
\begin{pgfscope}%
\pgfpathrectangle{\pgfqpoint{0.380943in}{8.035189in}}{\pgfqpoint{4.650000in}{0.614151in}}%
\pgfusepath{clip}%
\pgfsetbuttcap%
\pgfsetroundjoin%
\definecolor{currentfill}{rgb}{0.963768,0.915433,0.717478}%
\pgfsetfillcolor{currentfill}%
\pgfsetlinewidth{0.250937pt}%
\definecolor{currentstroke}{rgb}{1.000000,1.000000,1.000000}%
\pgfsetstrokecolor{currentstroke}%
\pgfsetdash{}{0pt}%
\pgfpathmoveto{\pgfqpoint{3.451698in}{8.561604in}}%
\pgfpathlineto{\pgfqpoint{3.539434in}{8.561604in}}%
\pgfpathlineto{\pgfqpoint{3.539434in}{8.473868in}}%
\pgfpathlineto{\pgfqpoint{3.451698in}{8.473868in}}%
\pgfpathlineto{\pgfqpoint{3.451698in}{8.561604in}}%
\pgfusepath{stroke,fill}%
\end{pgfscope}%
\begin{pgfscope}%
\pgfpathrectangle{\pgfqpoint{0.380943in}{8.035189in}}{\pgfqpoint{4.650000in}{0.614151in}}%
\pgfusepath{clip}%
\pgfsetbuttcap%
\pgfsetroundjoin%
\definecolor{currentfill}{rgb}{0.986251,0.808597,0.643230}%
\pgfsetfillcolor{currentfill}%
\pgfsetlinewidth{0.250937pt}%
\definecolor{currentstroke}{rgb}{1.000000,1.000000,1.000000}%
\pgfsetstrokecolor{currentstroke}%
\pgfsetdash{}{0pt}%
\pgfpathmoveto{\pgfqpoint{3.539434in}{8.561604in}}%
\pgfpathlineto{\pgfqpoint{3.627169in}{8.561604in}}%
\pgfpathlineto{\pgfqpoint{3.627169in}{8.473868in}}%
\pgfpathlineto{\pgfqpoint{3.539434in}{8.473868in}}%
\pgfpathlineto{\pgfqpoint{3.539434in}{8.561604in}}%
\pgfusepath{stroke,fill}%
\end{pgfscope}%
\begin{pgfscope}%
\pgfpathrectangle{\pgfqpoint{0.380943in}{8.035189in}}{\pgfqpoint{4.650000in}{0.614151in}}%
\pgfusepath{clip}%
\pgfsetbuttcap%
\pgfsetroundjoin%
\definecolor{currentfill}{rgb}{1.000000,0.584929,0.522599}%
\pgfsetfillcolor{currentfill}%
\pgfsetlinewidth{0.250937pt}%
\definecolor{currentstroke}{rgb}{1.000000,1.000000,1.000000}%
\pgfsetstrokecolor{currentstroke}%
\pgfsetdash{}{0pt}%
\pgfpathmoveto{\pgfqpoint{3.627169in}{8.561604in}}%
\pgfpathlineto{\pgfqpoint{3.714905in}{8.561604in}}%
\pgfpathlineto{\pgfqpoint{3.714905in}{8.473868in}}%
\pgfpathlineto{\pgfqpoint{3.627169in}{8.473868in}}%
\pgfpathlineto{\pgfqpoint{3.627169in}{8.561604in}}%
\pgfusepath{stroke,fill}%
\end{pgfscope}%
\begin{pgfscope}%
\pgfpathrectangle{\pgfqpoint{0.380943in}{8.035189in}}{\pgfqpoint{4.650000in}{0.614151in}}%
\pgfusepath{clip}%
\pgfsetbuttcap%
\pgfsetroundjoin%
\definecolor{currentfill}{rgb}{0.986251,0.808597,0.643230}%
\pgfsetfillcolor{currentfill}%
\pgfsetlinewidth{0.250937pt}%
\definecolor{currentstroke}{rgb}{1.000000,1.000000,1.000000}%
\pgfsetstrokecolor{currentstroke}%
\pgfsetdash{}{0pt}%
\pgfpathmoveto{\pgfqpoint{3.714905in}{8.561604in}}%
\pgfpathlineto{\pgfqpoint{3.802641in}{8.561604in}}%
\pgfpathlineto{\pgfqpoint{3.802641in}{8.473868in}}%
\pgfpathlineto{\pgfqpoint{3.714905in}{8.473868in}}%
\pgfpathlineto{\pgfqpoint{3.714905in}{8.561604in}}%
\pgfusepath{stroke,fill}%
\end{pgfscope}%
\begin{pgfscope}%
\pgfpathrectangle{\pgfqpoint{0.380943in}{8.035189in}}{\pgfqpoint{4.650000in}{0.614151in}}%
\pgfusepath{clip}%
\pgfsetbuttcap%
\pgfsetroundjoin%
\definecolor{currentfill}{rgb}{0.970012,0.883276,0.699577}%
\pgfsetfillcolor{currentfill}%
\pgfsetlinewidth{0.250937pt}%
\definecolor{currentstroke}{rgb}{1.000000,1.000000,1.000000}%
\pgfsetstrokecolor{currentstroke}%
\pgfsetdash{}{0pt}%
\pgfpathmoveto{\pgfqpoint{3.802641in}{8.561604in}}%
\pgfpathlineto{\pgfqpoint{3.890377in}{8.561604in}}%
\pgfpathlineto{\pgfqpoint{3.890377in}{8.473868in}}%
\pgfpathlineto{\pgfqpoint{3.802641in}{8.473868in}}%
\pgfpathlineto{\pgfqpoint{3.802641in}{8.561604in}}%
\pgfusepath{stroke,fill}%
\end{pgfscope}%
\begin{pgfscope}%
\pgfpathrectangle{\pgfqpoint{0.380943in}{8.035189in}}{\pgfqpoint{4.650000in}{0.614151in}}%
\pgfusepath{clip}%
\pgfsetbuttcap%
\pgfsetroundjoin%
\definecolor{currentfill}{rgb}{0.986251,0.808597,0.643230}%
\pgfsetfillcolor{currentfill}%
\pgfsetlinewidth{0.250937pt}%
\definecolor{currentstroke}{rgb}{1.000000,1.000000,1.000000}%
\pgfsetstrokecolor{currentstroke}%
\pgfsetdash{}{0pt}%
\pgfpathmoveto{\pgfqpoint{3.890377in}{8.561604in}}%
\pgfpathlineto{\pgfqpoint{3.978113in}{8.561604in}}%
\pgfpathlineto{\pgfqpoint{3.978113in}{8.473868in}}%
\pgfpathlineto{\pgfqpoint{3.890377in}{8.473868in}}%
\pgfpathlineto{\pgfqpoint{3.890377in}{8.561604in}}%
\pgfusepath{stroke,fill}%
\end{pgfscope}%
\begin{pgfscope}%
\pgfpathrectangle{\pgfqpoint{0.380943in}{8.035189in}}{\pgfqpoint{4.650000in}{0.614151in}}%
\pgfusepath{clip}%
\pgfsetbuttcap%
\pgfsetroundjoin%
\definecolor{currentfill}{rgb}{0.978131,0.843783,0.675709}%
\pgfsetfillcolor{currentfill}%
\pgfsetlinewidth{0.250937pt}%
\definecolor{currentstroke}{rgb}{1.000000,1.000000,1.000000}%
\pgfsetstrokecolor{currentstroke}%
\pgfsetdash{}{0pt}%
\pgfpathmoveto{\pgfqpoint{3.978113in}{8.561604in}}%
\pgfpathlineto{\pgfqpoint{4.065849in}{8.561604in}}%
\pgfpathlineto{\pgfqpoint{4.065849in}{8.473868in}}%
\pgfpathlineto{\pgfqpoint{3.978113in}{8.473868in}}%
\pgfpathlineto{\pgfqpoint{3.978113in}{8.561604in}}%
\pgfusepath{stroke,fill}%
\end{pgfscope}%
\begin{pgfscope}%
\pgfpathrectangle{\pgfqpoint{0.380943in}{8.035189in}}{\pgfqpoint{4.650000in}{0.614151in}}%
\pgfusepath{clip}%
\pgfsetbuttcap%
\pgfsetroundjoin%
\definecolor{currentfill}{rgb}{0.963768,0.915433,0.717478}%
\pgfsetfillcolor{currentfill}%
\pgfsetlinewidth{0.250937pt}%
\definecolor{currentstroke}{rgb}{1.000000,1.000000,1.000000}%
\pgfsetstrokecolor{currentstroke}%
\pgfsetdash{}{0pt}%
\pgfpathmoveto{\pgfqpoint{4.065849in}{8.561604in}}%
\pgfpathlineto{\pgfqpoint{4.153585in}{8.561604in}}%
\pgfpathlineto{\pgfqpoint{4.153585in}{8.473868in}}%
\pgfpathlineto{\pgfqpoint{4.065849in}{8.473868in}}%
\pgfpathlineto{\pgfqpoint{4.065849in}{8.561604in}}%
\pgfusepath{stroke,fill}%
\end{pgfscope}%
\begin{pgfscope}%
\pgfpathrectangle{\pgfqpoint{0.380943in}{8.035189in}}{\pgfqpoint{4.650000in}{0.614151in}}%
\pgfusepath{clip}%
\pgfsetbuttcap%
\pgfsetroundjoin%
\definecolor{currentfill}{rgb}{0.999616,0.641369,0.546559}%
\pgfsetfillcolor{currentfill}%
\pgfsetlinewidth{0.250937pt}%
\definecolor{currentstroke}{rgb}{1.000000,1.000000,1.000000}%
\pgfsetstrokecolor{currentstroke}%
\pgfsetdash{}{0pt}%
\pgfpathmoveto{\pgfqpoint{4.153585in}{8.561604in}}%
\pgfpathlineto{\pgfqpoint{4.241320in}{8.561604in}}%
\pgfpathlineto{\pgfqpoint{4.241320in}{8.473868in}}%
\pgfpathlineto{\pgfqpoint{4.153585in}{8.473868in}}%
\pgfpathlineto{\pgfqpoint{4.153585in}{8.561604in}}%
\pgfusepath{stroke,fill}%
\end{pgfscope}%
\begin{pgfscope}%
\pgfpathrectangle{\pgfqpoint{0.380943in}{8.035189in}}{\pgfqpoint{4.650000in}{0.614151in}}%
\pgfusepath{clip}%
\pgfsetbuttcap%
\pgfsetroundjoin%
\definecolor{currentfill}{rgb}{0.992326,0.765229,0.614840}%
\pgfsetfillcolor{currentfill}%
\pgfsetlinewidth{0.250937pt}%
\definecolor{currentstroke}{rgb}{1.000000,1.000000,1.000000}%
\pgfsetstrokecolor{currentstroke}%
\pgfsetdash{}{0pt}%
\pgfpathmoveto{\pgfqpoint{4.241320in}{8.561604in}}%
\pgfpathlineto{\pgfqpoint{4.329056in}{8.561604in}}%
\pgfpathlineto{\pgfqpoint{4.329056in}{8.473868in}}%
\pgfpathlineto{\pgfqpoint{4.241320in}{8.473868in}}%
\pgfpathlineto{\pgfqpoint{4.241320in}{8.561604in}}%
\pgfusepath{stroke,fill}%
\end{pgfscope}%
\begin{pgfscope}%
\pgfpathrectangle{\pgfqpoint{0.380943in}{8.035189in}}{\pgfqpoint{4.650000in}{0.614151in}}%
\pgfusepath{clip}%
\pgfsetbuttcap%
\pgfsetroundjoin%
\definecolor{currentfill}{rgb}{0.978131,0.843783,0.675709}%
\pgfsetfillcolor{currentfill}%
\pgfsetlinewidth{0.250937pt}%
\definecolor{currentstroke}{rgb}{1.000000,1.000000,1.000000}%
\pgfsetstrokecolor{currentstroke}%
\pgfsetdash{}{0pt}%
\pgfpathmoveto{\pgfqpoint{4.329056in}{8.561604in}}%
\pgfpathlineto{\pgfqpoint{4.416792in}{8.561604in}}%
\pgfpathlineto{\pgfqpoint{4.416792in}{8.473868in}}%
\pgfpathlineto{\pgfqpoint{4.329056in}{8.473868in}}%
\pgfpathlineto{\pgfqpoint{4.329056in}{8.561604in}}%
\pgfusepath{stroke,fill}%
\end{pgfscope}%
\begin{pgfscope}%
\pgfpathrectangle{\pgfqpoint{0.380943in}{8.035189in}}{\pgfqpoint{4.650000in}{0.614151in}}%
\pgfusepath{clip}%
\pgfsetbuttcap%
\pgfsetroundjoin%
\definecolor{currentfill}{rgb}{0.986251,0.808597,0.643230}%
\pgfsetfillcolor{currentfill}%
\pgfsetlinewidth{0.250937pt}%
\definecolor{currentstroke}{rgb}{1.000000,1.000000,1.000000}%
\pgfsetstrokecolor{currentstroke}%
\pgfsetdash{}{0pt}%
\pgfpathmoveto{\pgfqpoint{4.416792in}{8.561604in}}%
\pgfpathlineto{\pgfqpoint{4.504528in}{8.561604in}}%
\pgfpathlineto{\pgfqpoint{4.504528in}{8.473868in}}%
\pgfpathlineto{\pgfqpoint{4.416792in}{8.473868in}}%
\pgfpathlineto{\pgfqpoint{4.416792in}{8.561604in}}%
\pgfusepath{stroke,fill}%
\end{pgfscope}%
\begin{pgfscope}%
\pgfpathrectangle{\pgfqpoint{0.380943in}{8.035189in}}{\pgfqpoint{4.650000in}{0.614151in}}%
\pgfusepath{clip}%
\pgfsetbuttcap%
\pgfsetroundjoin%
\definecolor{currentfill}{rgb}{0.999616,0.641369,0.546559}%
\pgfsetfillcolor{currentfill}%
\pgfsetlinewidth{0.250937pt}%
\definecolor{currentstroke}{rgb}{1.000000,1.000000,1.000000}%
\pgfsetstrokecolor{currentstroke}%
\pgfsetdash{}{0pt}%
\pgfpathmoveto{\pgfqpoint{4.504528in}{8.561604in}}%
\pgfpathlineto{\pgfqpoint{4.592264in}{8.561604in}}%
\pgfpathlineto{\pgfqpoint{4.592264in}{8.473868in}}%
\pgfpathlineto{\pgfqpoint{4.504528in}{8.473868in}}%
\pgfpathlineto{\pgfqpoint{4.504528in}{8.561604in}}%
\pgfusepath{stroke,fill}%
\end{pgfscope}%
\begin{pgfscope}%
\pgfpathrectangle{\pgfqpoint{0.380943in}{8.035189in}}{\pgfqpoint{4.650000in}{0.614151in}}%
\pgfusepath{clip}%
\pgfsetbuttcap%
\pgfsetroundjoin%
\definecolor{currentfill}{rgb}{0.999616,0.641369,0.546559}%
\pgfsetfillcolor{currentfill}%
\pgfsetlinewidth{0.250937pt}%
\definecolor{currentstroke}{rgb}{1.000000,1.000000,1.000000}%
\pgfsetstrokecolor{currentstroke}%
\pgfsetdash{}{0pt}%
\pgfpathmoveto{\pgfqpoint{4.592264in}{8.561604in}}%
\pgfpathlineto{\pgfqpoint{4.680000in}{8.561604in}}%
\pgfpathlineto{\pgfqpoint{4.680000in}{8.473868in}}%
\pgfpathlineto{\pgfqpoint{4.592264in}{8.473868in}}%
\pgfpathlineto{\pgfqpoint{4.592264in}{8.561604in}}%
\pgfusepath{stroke,fill}%
\end{pgfscope}%
\begin{pgfscope}%
\pgfpathrectangle{\pgfqpoint{0.380943in}{8.035189in}}{\pgfqpoint{4.650000in}{0.614151in}}%
\pgfusepath{clip}%
\pgfsetbuttcap%
\pgfsetroundjoin%
\definecolor{currentfill}{rgb}{0.800000,0.278431,0.278431}%
\pgfsetfillcolor{currentfill}%
\pgfsetlinewidth{0.250937pt}%
\definecolor{currentstroke}{rgb}{1.000000,1.000000,1.000000}%
\pgfsetstrokecolor{currentstroke}%
\pgfsetdash{}{0pt}%
\pgfpathmoveto{\pgfqpoint{4.680000in}{8.561604in}}%
\pgfpathlineto{\pgfqpoint{4.767736in}{8.561604in}}%
\pgfpathlineto{\pgfqpoint{4.767736in}{8.473868in}}%
\pgfpathlineto{\pgfqpoint{4.680000in}{8.473868in}}%
\pgfpathlineto{\pgfqpoint{4.680000in}{8.561604in}}%
\pgfusepath{stroke,fill}%
\end{pgfscope}%
\begin{pgfscope}%
\pgfpathrectangle{\pgfqpoint{0.380943in}{8.035189in}}{\pgfqpoint{4.650000in}{0.614151in}}%
\pgfusepath{clip}%
\pgfsetbuttcap%
\pgfsetroundjoin%
\definecolor{currentfill}{rgb}{0.986251,0.808597,0.643230}%
\pgfsetfillcolor{currentfill}%
\pgfsetlinewidth{0.250937pt}%
\definecolor{currentstroke}{rgb}{1.000000,1.000000,1.000000}%
\pgfsetstrokecolor{currentstroke}%
\pgfsetdash{}{0pt}%
\pgfpathmoveto{\pgfqpoint{4.767736in}{8.561604in}}%
\pgfpathlineto{\pgfqpoint{4.855471in}{8.561604in}}%
\pgfpathlineto{\pgfqpoint{4.855471in}{8.473868in}}%
\pgfpathlineto{\pgfqpoint{4.767736in}{8.473868in}}%
\pgfpathlineto{\pgfqpoint{4.767736in}{8.561604in}}%
\pgfusepath{stroke,fill}%
\end{pgfscope}%
\begin{pgfscope}%
\pgfpathrectangle{\pgfqpoint{0.380943in}{8.035189in}}{\pgfqpoint{4.650000in}{0.614151in}}%
\pgfusepath{clip}%
\pgfsetbuttcap%
\pgfsetroundjoin%
\definecolor{currentfill}{rgb}{0.999616,0.641369,0.546559}%
\pgfsetfillcolor{currentfill}%
\pgfsetlinewidth{0.250937pt}%
\definecolor{currentstroke}{rgb}{1.000000,1.000000,1.000000}%
\pgfsetstrokecolor{currentstroke}%
\pgfsetdash{}{0pt}%
\pgfpathmoveto{\pgfqpoint{4.855471in}{8.561604in}}%
\pgfpathlineto{\pgfqpoint{4.943207in}{8.561604in}}%
\pgfpathlineto{\pgfqpoint{4.943207in}{8.473868in}}%
\pgfpathlineto{\pgfqpoint{4.855471in}{8.473868in}}%
\pgfpathlineto{\pgfqpoint{4.855471in}{8.561604in}}%
\pgfusepath{stroke,fill}%
\end{pgfscope}%
\begin{pgfscope}%
\pgfpathrectangle{\pgfqpoint{0.380943in}{8.035189in}}{\pgfqpoint{4.650000in}{0.614151in}}%
\pgfusepath{clip}%
\pgfsetbuttcap%
\pgfsetroundjoin%
\definecolor{currentfill}{rgb}{0.963768,0.915433,0.717478}%
\pgfsetfillcolor{currentfill}%
\pgfsetlinewidth{0.250937pt}%
\definecolor{currentstroke}{rgb}{1.000000,1.000000,1.000000}%
\pgfsetstrokecolor{currentstroke}%
\pgfsetdash{}{0pt}%
\pgfpathmoveto{\pgfqpoint{4.943207in}{8.561604in}}%
\pgfpathlineto{\pgfqpoint{5.030943in}{8.561604in}}%
\pgfpathlineto{\pgfqpoint{5.030943in}{8.473868in}}%
\pgfpathlineto{\pgfqpoint{4.943207in}{8.473868in}}%
\pgfpathlineto{\pgfqpoint{4.943207in}{8.561604in}}%
\pgfusepath{stroke,fill}%
\end{pgfscope}%
\begin{pgfscope}%
\pgfpathrectangle{\pgfqpoint{0.380943in}{8.035189in}}{\pgfqpoint{4.650000in}{0.614151in}}%
\pgfusepath{clip}%
\pgfsetbuttcap%
\pgfsetroundjoin%
\pgfsetlinewidth{0.250937pt}%
\definecolor{currentstroke}{rgb}{1.000000,1.000000,1.000000}%
\pgfsetstrokecolor{currentstroke}%
\pgfsetdash{}{0pt}%
\pgfpathmoveto{\pgfqpoint{0.380943in}{8.473868in}}%
\pgfpathlineto{\pgfqpoint{0.468679in}{8.473868in}}%
\pgfpathlineto{\pgfqpoint{0.468679in}{8.386132in}}%
\pgfpathlineto{\pgfqpoint{0.380943in}{8.386132in}}%
\pgfpathlineto{\pgfqpoint{0.380943in}{8.473868in}}%
\pgfusepath{stroke}%
\end{pgfscope}%
\begin{pgfscope}%
\pgfpathrectangle{\pgfqpoint{0.380943in}{8.035189in}}{\pgfqpoint{4.650000in}{0.614151in}}%
\pgfusepath{clip}%
\pgfsetbuttcap%
\pgfsetroundjoin%
\definecolor{currentfill}{rgb}{0.996909,0.711742,0.584452}%
\pgfsetfillcolor{currentfill}%
\pgfsetlinewidth{0.250937pt}%
\definecolor{currentstroke}{rgb}{1.000000,1.000000,1.000000}%
\pgfsetstrokecolor{currentstroke}%
\pgfsetdash{}{0pt}%
\pgfpathmoveto{\pgfqpoint{0.468679in}{8.473868in}}%
\pgfpathlineto{\pgfqpoint{0.556415in}{8.473868in}}%
\pgfpathlineto{\pgfqpoint{0.556415in}{8.386132in}}%
\pgfpathlineto{\pgfqpoint{0.468679in}{8.386132in}}%
\pgfpathlineto{\pgfqpoint{0.468679in}{8.473868in}}%
\pgfusepath{stroke,fill}%
\end{pgfscope}%
\begin{pgfscope}%
\pgfpathrectangle{\pgfqpoint{0.380943in}{8.035189in}}{\pgfqpoint{4.650000in}{0.614151in}}%
\pgfusepath{clip}%
\pgfsetbuttcap%
\pgfsetroundjoin%
\definecolor{currentfill}{rgb}{0.970012,0.883276,0.699577}%
\pgfsetfillcolor{currentfill}%
\pgfsetlinewidth{0.250937pt}%
\definecolor{currentstroke}{rgb}{1.000000,1.000000,1.000000}%
\pgfsetstrokecolor{currentstroke}%
\pgfsetdash{}{0pt}%
\pgfpathmoveto{\pgfqpoint{0.556415in}{8.473868in}}%
\pgfpathlineto{\pgfqpoint{0.644151in}{8.473868in}}%
\pgfpathlineto{\pgfqpoint{0.644151in}{8.386132in}}%
\pgfpathlineto{\pgfqpoint{0.556415in}{8.386132in}}%
\pgfpathlineto{\pgfqpoint{0.556415in}{8.473868in}}%
\pgfusepath{stroke,fill}%
\end{pgfscope}%
\begin{pgfscope}%
\pgfpathrectangle{\pgfqpoint{0.380943in}{8.035189in}}{\pgfqpoint{4.650000in}{0.614151in}}%
\pgfusepath{clip}%
\pgfsetbuttcap%
\pgfsetroundjoin%
\definecolor{currentfill}{rgb}{0.996909,0.711742,0.584452}%
\pgfsetfillcolor{currentfill}%
\pgfsetlinewidth{0.250937pt}%
\definecolor{currentstroke}{rgb}{1.000000,1.000000,1.000000}%
\pgfsetstrokecolor{currentstroke}%
\pgfsetdash{}{0pt}%
\pgfpathmoveto{\pgfqpoint{0.644151in}{8.473868in}}%
\pgfpathlineto{\pgfqpoint{0.731886in}{8.473868in}}%
\pgfpathlineto{\pgfqpoint{0.731886in}{8.386132in}}%
\pgfpathlineto{\pgfqpoint{0.644151in}{8.386132in}}%
\pgfpathlineto{\pgfqpoint{0.644151in}{8.473868in}}%
\pgfusepath{stroke,fill}%
\end{pgfscope}%
\begin{pgfscope}%
\pgfpathrectangle{\pgfqpoint{0.380943in}{8.035189in}}{\pgfqpoint{4.650000in}{0.614151in}}%
\pgfusepath{clip}%
\pgfsetbuttcap%
\pgfsetroundjoin%
\definecolor{currentfill}{rgb}{0.978131,0.843783,0.675709}%
\pgfsetfillcolor{currentfill}%
\pgfsetlinewidth{0.250937pt}%
\definecolor{currentstroke}{rgb}{1.000000,1.000000,1.000000}%
\pgfsetstrokecolor{currentstroke}%
\pgfsetdash{}{0pt}%
\pgfpathmoveto{\pgfqpoint{0.731886in}{8.473868in}}%
\pgfpathlineto{\pgfqpoint{0.819622in}{8.473868in}}%
\pgfpathlineto{\pgfqpoint{0.819622in}{8.386132in}}%
\pgfpathlineto{\pgfqpoint{0.731886in}{8.386132in}}%
\pgfpathlineto{\pgfqpoint{0.731886in}{8.473868in}}%
\pgfusepath{stroke,fill}%
\end{pgfscope}%
\begin{pgfscope}%
\pgfpathrectangle{\pgfqpoint{0.380943in}{8.035189in}}{\pgfqpoint{4.650000in}{0.614151in}}%
\pgfusepath{clip}%
\pgfsetbuttcap%
\pgfsetroundjoin%
\definecolor{currentfill}{rgb}{0.992326,0.765229,0.614840}%
\pgfsetfillcolor{currentfill}%
\pgfsetlinewidth{0.250937pt}%
\definecolor{currentstroke}{rgb}{1.000000,1.000000,1.000000}%
\pgfsetstrokecolor{currentstroke}%
\pgfsetdash{}{0pt}%
\pgfpathmoveto{\pgfqpoint{0.819622in}{8.473868in}}%
\pgfpathlineto{\pgfqpoint{0.907358in}{8.473868in}}%
\pgfpathlineto{\pgfqpoint{0.907358in}{8.386132in}}%
\pgfpathlineto{\pgfqpoint{0.819622in}{8.386132in}}%
\pgfpathlineto{\pgfqpoint{0.819622in}{8.473868in}}%
\pgfusepath{stroke,fill}%
\end{pgfscope}%
\begin{pgfscope}%
\pgfpathrectangle{\pgfqpoint{0.380943in}{8.035189in}}{\pgfqpoint{4.650000in}{0.614151in}}%
\pgfusepath{clip}%
\pgfsetbuttcap%
\pgfsetroundjoin%
\definecolor{currentfill}{rgb}{0.996909,0.711742,0.584452}%
\pgfsetfillcolor{currentfill}%
\pgfsetlinewidth{0.250937pt}%
\definecolor{currentstroke}{rgb}{1.000000,1.000000,1.000000}%
\pgfsetstrokecolor{currentstroke}%
\pgfsetdash{}{0pt}%
\pgfpathmoveto{\pgfqpoint{0.907358in}{8.473868in}}%
\pgfpathlineto{\pgfqpoint{0.995094in}{8.473868in}}%
\pgfpathlineto{\pgfqpoint{0.995094in}{8.386132in}}%
\pgfpathlineto{\pgfqpoint{0.907358in}{8.386132in}}%
\pgfpathlineto{\pgfqpoint{0.907358in}{8.473868in}}%
\pgfusepath{stroke,fill}%
\end{pgfscope}%
\begin{pgfscope}%
\pgfpathrectangle{\pgfqpoint{0.380943in}{8.035189in}}{\pgfqpoint{4.650000in}{0.614151in}}%
\pgfusepath{clip}%
\pgfsetbuttcap%
\pgfsetroundjoin%
\definecolor{currentfill}{rgb}{0.986251,0.808597,0.643230}%
\pgfsetfillcolor{currentfill}%
\pgfsetlinewidth{0.250937pt}%
\definecolor{currentstroke}{rgb}{1.000000,1.000000,1.000000}%
\pgfsetstrokecolor{currentstroke}%
\pgfsetdash{}{0pt}%
\pgfpathmoveto{\pgfqpoint{0.995094in}{8.473868in}}%
\pgfpathlineto{\pgfqpoint{1.082830in}{8.473868in}}%
\pgfpathlineto{\pgfqpoint{1.082830in}{8.386132in}}%
\pgfpathlineto{\pgfqpoint{0.995094in}{8.386132in}}%
\pgfpathlineto{\pgfqpoint{0.995094in}{8.473868in}}%
\pgfusepath{stroke,fill}%
\end{pgfscope}%
\begin{pgfscope}%
\pgfpathrectangle{\pgfqpoint{0.380943in}{8.035189in}}{\pgfqpoint{4.650000in}{0.614151in}}%
\pgfusepath{clip}%
\pgfsetbuttcap%
\pgfsetroundjoin%
\definecolor{currentfill}{rgb}{0.963768,0.915433,0.717478}%
\pgfsetfillcolor{currentfill}%
\pgfsetlinewidth{0.250937pt}%
\definecolor{currentstroke}{rgb}{1.000000,1.000000,1.000000}%
\pgfsetstrokecolor{currentstroke}%
\pgfsetdash{}{0pt}%
\pgfpathmoveto{\pgfqpoint{1.082830in}{8.473868in}}%
\pgfpathlineto{\pgfqpoint{1.170566in}{8.473868in}}%
\pgfpathlineto{\pgfqpoint{1.170566in}{8.386132in}}%
\pgfpathlineto{\pgfqpoint{1.082830in}{8.386132in}}%
\pgfpathlineto{\pgfqpoint{1.082830in}{8.473868in}}%
\pgfusepath{stroke,fill}%
\end{pgfscope}%
\begin{pgfscope}%
\pgfpathrectangle{\pgfqpoint{0.380943in}{8.035189in}}{\pgfqpoint{4.650000in}{0.614151in}}%
\pgfusepath{clip}%
\pgfsetbuttcap%
\pgfsetroundjoin%
\definecolor{currentfill}{rgb}{0.961061,0.931672,0.728304}%
\pgfsetfillcolor{currentfill}%
\pgfsetlinewidth{0.250937pt}%
\definecolor{currentstroke}{rgb}{1.000000,1.000000,1.000000}%
\pgfsetstrokecolor{currentstroke}%
\pgfsetdash{}{0pt}%
\pgfpathmoveto{\pgfqpoint{1.170566in}{8.473868in}}%
\pgfpathlineto{\pgfqpoint{1.258302in}{8.473868in}}%
\pgfpathlineto{\pgfqpoint{1.258302in}{8.386132in}}%
\pgfpathlineto{\pgfqpoint{1.170566in}{8.386132in}}%
\pgfpathlineto{\pgfqpoint{1.170566in}{8.473868in}}%
\pgfusepath{stroke,fill}%
\end{pgfscope}%
\begin{pgfscope}%
\pgfpathrectangle{\pgfqpoint{0.380943in}{8.035189in}}{\pgfqpoint{4.650000in}{0.614151in}}%
\pgfusepath{clip}%
\pgfsetbuttcap%
\pgfsetroundjoin%
\definecolor{currentfill}{rgb}{0.978131,0.843783,0.675709}%
\pgfsetfillcolor{currentfill}%
\pgfsetlinewidth{0.250937pt}%
\definecolor{currentstroke}{rgb}{1.000000,1.000000,1.000000}%
\pgfsetstrokecolor{currentstroke}%
\pgfsetdash{}{0pt}%
\pgfpathmoveto{\pgfqpoint{1.258302in}{8.473868in}}%
\pgfpathlineto{\pgfqpoint{1.346037in}{8.473868in}}%
\pgfpathlineto{\pgfqpoint{1.346037in}{8.386132in}}%
\pgfpathlineto{\pgfqpoint{1.258302in}{8.386132in}}%
\pgfpathlineto{\pgfqpoint{1.258302in}{8.473868in}}%
\pgfusepath{stroke,fill}%
\end{pgfscope}%
\begin{pgfscope}%
\pgfpathrectangle{\pgfqpoint{0.380943in}{8.035189in}}{\pgfqpoint{4.650000in}{0.614151in}}%
\pgfusepath{clip}%
\pgfsetbuttcap%
\pgfsetroundjoin%
\definecolor{currentfill}{rgb}{0.986251,0.808597,0.643230}%
\pgfsetfillcolor{currentfill}%
\pgfsetlinewidth{0.250937pt}%
\definecolor{currentstroke}{rgb}{1.000000,1.000000,1.000000}%
\pgfsetstrokecolor{currentstroke}%
\pgfsetdash{}{0pt}%
\pgfpathmoveto{\pgfqpoint{1.346037in}{8.473868in}}%
\pgfpathlineto{\pgfqpoint{1.433773in}{8.473868in}}%
\pgfpathlineto{\pgfqpoint{1.433773in}{8.386132in}}%
\pgfpathlineto{\pgfqpoint{1.346037in}{8.386132in}}%
\pgfpathlineto{\pgfqpoint{1.346037in}{8.473868in}}%
\pgfusepath{stroke,fill}%
\end{pgfscope}%
\begin{pgfscope}%
\pgfpathrectangle{\pgfqpoint{0.380943in}{8.035189in}}{\pgfqpoint{4.650000in}{0.614151in}}%
\pgfusepath{clip}%
\pgfsetbuttcap%
\pgfsetroundjoin%
\definecolor{currentfill}{rgb}{0.800000,0.278431,0.278431}%
\pgfsetfillcolor{currentfill}%
\pgfsetlinewidth{0.250937pt}%
\definecolor{currentstroke}{rgb}{1.000000,1.000000,1.000000}%
\pgfsetstrokecolor{currentstroke}%
\pgfsetdash{}{0pt}%
\pgfpathmoveto{\pgfqpoint{1.433773in}{8.473868in}}%
\pgfpathlineto{\pgfqpoint{1.521509in}{8.473868in}}%
\pgfpathlineto{\pgfqpoint{1.521509in}{8.386132in}}%
\pgfpathlineto{\pgfqpoint{1.433773in}{8.386132in}}%
\pgfpathlineto{\pgfqpoint{1.433773in}{8.473868in}}%
\pgfusepath{stroke,fill}%
\end{pgfscope}%
\begin{pgfscope}%
\pgfpathrectangle{\pgfqpoint{0.380943in}{8.035189in}}{\pgfqpoint{4.650000in}{0.614151in}}%
\pgfusepath{clip}%
\pgfsetbuttcap%
\pgfsetroundjoin%
\definecolor{currentfill}{rgb}{0.970012,0.883276,0.699577}%
\pgfsetfillcolor{currentfill}%
\pgfsetlinewidth{0.250937pt}%
\definecolor{currentstroke}{rgb}{1.000000,1.000000,1.000000}%
\pgfsetstrokecolor{currentstroke}%
\pgfsetdash{}{0pt}%
\pgfpathmoveto{\pgfqpoint{1.521509in}{8.473868in}}%
\pgfpathlineto{\pgfqpoint{1.609245in}{8.473868in}}%
\pgfpathlineto{\pgfqpoint{1.609245in}{8.386132in}}%
\pgfpathlineto{\pgfqpoint{1.521509in}{8.386132in}}%
\pgfpathlineto{\pgfqpoint{1.521509in}{8.473868in}}%
\pgfusepath{stroke,fill}%
\end{pgfscope}%
\begin{pgfscope}%
\pgfpathrectangle{\pgfqpoint{0.380943in}{8.035189in}}{\pgfqpoint{4.650000in}{0.614151in}}%
\pgfusepath{clip}%
\pgfsetbuttcap%
\pgfsetroundjoin%
\definecolor{currentfill}{rgb}{0.978131,0.843783,0.675709}%
\pgfsetfillcolor{currentfill}%
\pgfsetlinewidth{0.250937pt}%
\definecolor{currentstroke}{rgb}{1.000000,1.000000,1.000000}%
\pgfsetstrokecolor{currentstroke}%
\pgfsetdash{}{0pt}%
\pgfpathmoveto{\pgfqpoint{1.609245in}{8.473868in}}%
\pgfpathlineto{\pgfqpoint{1.696981in}{8.473868in}}%
\pgfpathlineto{\pgfqpoint{1.696981in}{8.386132in}}%
\pgfpathlineto{\pgfqpoint{1.609245in}{8.386132in}}%
\pgfpathlineto{\pgfqpoint{1.609245in}{8.473868in}}%
\pgfusepath{stroke,fill}%
\end{pgfscope}%
\begin{pgfscope}%
\pgfpathrectangle{\pgfqpoint{0.380943in}{8.035189in}}{\pgfqpoint{4.650000in}{0.614151in}}%
\pgfusepath{clip}%
\pgfsetbuttcap%
\pgfsetroundjoin%
\definecolor{currentfill}{rgb}{0.963768,0.915433,0.717478}%
\pgfsetfillcolor{currentfill}%
\pgfsetlinewidth{0.250937pt}%
\definecolor{currentstroke}{rgb}{1.000000,1.000000,1.000000}%
\pgfsetstrokecolor{currentstroke}%
\pgfsetdash{}{0pt}%
\pgfpathmoveto{\pgfqpoint{1.696981in}{8.473868in}}%
\pgfpathlineto{\pgfqpoint{1.784717in}{8.473868in}}%
\pgfpathlineto{\pgfqpoint{1.784717in}{8.386132in}}%
\pgfpathlineto{\pgfqpoint{1.696981in}{8.386132in}}%
\pgfpathlineto{\pgfqpoint{1.696981in}{8.473868in}}%
\pgfusepath{stroke,fill}%
\end{pgfscope}%
\begin{pgfscope}%
\pgfpathrectangle{\pgfqpoint{0.380943in}{8.035189in}}{\pgfqpoint{4.650000in}{0.614151in}}%
\pgfusepath{clip}%
\pgfsetbuttcap%
\pgfsetroundjoin%
\definecolor{currentfill}{rgb}{0.986251,0.808597,0.643230}%
\pgfsetfillcolor{currentfill}%
\pgfsetlinewidth{0.250937pt}%
\definecolor{currentstroke}{rgb}{1.000000,1.000000,1.000000}%
\pgfsetstrokecolor{currentstroke}%
\pgfsetdash{}{0pt}%
\pgfpathmoveto{\pgfqpoint{1.784717in}{8.473868in}}%
\pgfpathlineto{\pgfqpoint{1.872452in}{8.473868in}}%
\pgfpathlineto{\pgfqpoint{1.872452in}{8.386132in}}%
\pgfpathlineto{\pgfqpoint{1.784717in}{8.386132in}}%
\pgfpathlineto{\pgfqpoint{1.784717in}{8.473868in}}%
\pgfusepath{stroke,fill}%
\end{pgfscope}%
\begin{pgfscope}%
\pgfpathrectangle{\pgfqpoint{0.380943in}{8.035189in}}{\pgfqpoint{4.650000in}{0.614151in}}%
\pgfusepath{clip}%
\pgfsetbuttcap%
\pgfsetroundjoin%
\definecolor{currentfill}{rgb}{0.986251,0.808597,0.643230}%
\pgfsetfillcolor{currentfill}%
\pgfsetlinewidth{0.250937pt}%
\definecolor{currentstroke}{rgb}{1.000000,1.000000,1.000000}%
\pgfsetstrokecolor{currentstroke}%
\pgfsetdash{}{0pt}%
\pgfpathmoveto{\pgfqpoint{1.872452in}{8.473868in}}%
\pgfpathlineto{\pgfqpoint{1.960188in}{8.473868in}}%
\pgfpathlineto{\pgfqpoint{1.960188in}{8.386132in}}%
\pgfpathlineto{\pgfqpoint{1.872452in}{8.386132in}}%
\pgfpathlineto{\pgfqpoint{1.872452in}{8.473868in}}%
\pgfusepath{stroke,fill}%
\end{pgfscope}%
\begin{pgfscope}%
\pgfpathrectangle{\pgfqpoint{0.380943in}{8.035189in}}{\pgfqpoint{4.650000in}{0.614151in}}%
\pgfusepath{clip}%
\pgfsetbuttcap%
\pgfsetroundjoin%
\definecolor{currentfill}{rgb}{0.999616,0.641369,0.546559}%
\pgfsetfillcolor{currentfill}%
\pgfsetlinewidth{0.250937pt}%
\definecolor{currentstroke}{rgb}{1.000000,1.000000,1.000000}%
\pgfsetstrokecolor{currentstroke}%
\pgfsetdash{}{0pt}%
\pgfpathmoveto{\pgfqpoint{1.960188in}{8.473868in}}%
\pgfpathlineto{\pgfqpoint{2.047924in}{8.473868in}}%
\pgfpathlineto{\pgfqpoint{2.047924in}{8.386132in}}%
\pgfpathlineto{\pgfqpoint{1.960188in}{8.386132in}}%
\pgfpathlineto{\pgfqpoint{1.960188in}{8.473868in}}%
\pgfusepath{stroke,fill}%
\end{pgfscope}%
\begin{pgfscope}%
\pgfpathrectangle{\pgfqpoint{0.380943in}{8.035189in}}{\pgfqpoint{4.650000in}{0.614151in}}%
\pgfusepath{clip}%
\pgfsetbuttcap%
\pgfsetroundjoin%
\definecolor{currentfill}{rgb}{0.992326,0.765229,0.614840}%
\pgfsetfillcolor{currentfill}%
\pgfsetlinewidth{0.250937pt}%
\definecolor{currentstroke}{rgb}{1.000000,1.000000,1.000000}%
\pgfsetstrokecolor{currentstroke}%
\pgfsetdash{}{0pt}%
\pgfpathmoveto{\pgfqpoint{2.047924in}{8.473868in}}%
\pgfpathlineto{\pgfqpoint{2.135660in}{8.473868in}}%
\pgfpathlineto{\pgfqpoint{2.135660in}{8.386132in}}%
\pgfpathlineto{\pgfqpoint{2.047924in}{8.386132in}}%
\pgfpathlineto{\pgfqpoint{2.047924in}{8.473868in}}%
\pgfusepath{stroke,fill}%
\end{pgfscope}%
\begin{pgfscope}%
\pgfpathrectangle{\pgfqpoint{0.380943in}{8.035189in}}{\pgfqpoint{4.650000in}{0.614151in}}%
\pgfusepath{clip}%
\pgfsetbuttcap%
\pgfsetroundjoin%
\definecolor{currentfill}{rgb}{0.986251,0.808597,0.643230}%
\pgfsetfillcolor{currentfill}%
\pgfsetlinewidth{0.250937pt}%
\definecolor{currentstroke}{rgb}{1.000000,1.000000,1.000000}%
\pgfsetstrokecolor{currentstroke}%
\pgfsetdash{}{0pt}%
\pgfpathmoveto{\pgfqpoint{2.135660in}{8.473868in}}%
\pgfpathlineto{\pgfqpoint{2.223396in}{8.473868in}}%
\pgfpathlineto{\pgfqpoint{2.223396in}{8.386132in}}%
\pgfpathlineto{\pgfqpoint{2.135660in}{8.386132in}}%
\pgfpathlineto{\pgfqpoint{2.135660in}{8.473868in}}%
\pgfusepath{stroke,fill}%
\end{pgfscope}%
\begin{pgfscope}%
\pgfpathrectangle{\pgfqpoint{0.380943in}{8.035189in}}{\pgfqpoint{4.650000in}{0.614151in}}%
\pgfusepath{clip}%
\pgfsetbuttcap%
\pgfsetroundjoin%
\definecolor{currentfill}{rgb}{0.986251,0.808597,0.643230}%
\pgfsetfillcolor{currentfill}%
\pgfsetlinewidth{0.250937pt}%
\definecolor{currentstroke}{rgb}{1.000000,1.000000,1.000000}%
\pgfsetstrokecolor{currentstroke}%
\pgfsetdash{}{0pt}%
\pgfpathmoveto{\pgfqpoint{2.223396in}{8.473868in}}%
\pgfpathlineto{\pgfqpoint{2.311132in}{8.473868in}}%
\pgfpathlineto{\pgfqpoint{2.311132in}{8.386132in}}%
\pgfpathlineto{\pgfqpoint{2.223396in}{8.386132in}}%
\pgfpathlineto{\pgfqpoint{2.223396in}{8.473868in}}%
\pgfusepath{stroke,fill}%
\end{pgfscope}%
\begin{pgfscope}%
\pgfpathrectangle{\pgfqpoint{0.380943in}{8.035189in}}{\pgfqpoint{4.650000in}{0.614151in}}%
\pgfusepath{clip}%
\pgfsetbuttcap%
\pgfsetroundjoin%
\definecolor{currentfill}{rgb}{0.992326,0.765229,0.614840}%
\pgfsetfillcolor{currentfill}%
\pgfsetlinewidth{0.250937pt}%
\definecolor{currentstroke}{rgb}{1.000000,1.000000,1.000000}%
\pgfsetstrokecolor{currentstroke}%
\pgfsetdash{}{0pt}%
\pgfpathmoveto{\pgfqpoint{2.311132in}{8.473868in}}%
\pgfpathlineto{\pgfqpoint{2.398868in}{8.473868in}}%
\pgfpathlineto{\pgfqpoint{2.398868in}{8.386132in}}%
\pgfpathlineto{\pgfqpoint{2.311132in}{8.386132in}}%
\pgfpathlineto{\pgfqpoint{2.311132in}{8.473868in}}%
\pgfusepath{stroke,fill}%
\end{pgfscope}%
\begin{pgfscope}%
\pgfpathrectangle{\pgfqpoint{0.380943in}{8.035189in}}{\pgfqpoint{4.650000in}{0.614151in}}%
\pgfusepath{clip}%
\pgfsetbuttcap%
\pgfsetroundjoin%
\definecolor{currentfill}{rgb}{0.963768,0.915433,0.717478}%
\pgfsetfillcolor{currentfill}%
\pgfsetlinewidth{0.250937pt}%
\definecolor{currentstroke}{rgb}{1.000000,1.000000,1.000000}%
\pgfsetstrokecolor{currentstroke}%
\pgfsetdash{}{0pt}%
\pgfpathmoveto{\pgfqpoint{2.398868in}{8.473868in}}%
\pgfpathlineto{\pgfqpoint{2.486603in}{8.473868in}}%
\pgfpathlineto{\pgfqpoint{2.486603in}{8.386132in}}%
\pgfpathlineto{\pgfqpoint{2.398868in}{8.386132in}}%
\pgfpathlineto{\pgfqpoint{2.398868in}{8.473868in}}%
\pgfusepath{stroke,fill}%
\end{pgfscope}%
\begin{pgfscope}%
\pgfpathrectangle{\pgfqpoint{0.380943in}{8.035189in}}{\pgfqpoint{4.650000in}{0.614151in}}%
\pgfusepath{clip}%
\pgfsetbuttcap%
\pgfsetroundjoin%
\definecolor{currentfill}{rgb}{0.961061,0.931672,0.728304}%
\pgfsetfillcolor{currentfill}%
\pgfsetlinewidth{0.250937pt}%
\definecolor{currentstroke}{rgb}{1.000000,1.000000,1.000000}%
\pgfsetstrokecolor{currentstroke}%
\pgfsetdash{}{0pt}%
\pgfpathmoveto{\pgfqpoint{2.486603in}{8.473868in}}%
\pgfpathlineto{\pgfqpoint{2.574339in}{8.473868in}}%
\pgfpathlineto{\pgfqpoint{2.574339in}{8.386132in}}%
\pgfpathlineto{\pgfqpoint{2.486603in}{8.386132in}}%
\pgfpathlineto{\pgfqpoint{2.486603in}{8.473868in}}%
\pgfusepath{stroke,fill}%
\end{pgfscope}%
\begin{pgfscope}%
\pgfpathrectangle{\pgfqpoint{0.380943in}{8.035189in}}{\pgfqpoint{4.650000in}{0.614151in}}%
\pgfusepath{clip}%
\pgfsetbuttcap%
\pgfsetroundjoin%
\definecolor{currentfill}{rgb}{0.986251,0.808597,0.643230}%
\pgfsetfillcolor{currentfill}%
\pgfsetlinewidth{0.250937pt}%
\definecolor{currentstroke}{rgb}{1.000000,1.000000,1.000000}%
\pgfsetstrokecolor{currentstroke}%
\pgfsetdash{}{0pt}%
\pgfpathmoveto{\pgfqpoint{2.574339in}{8.473868in}}%
\pgfpathlineto{\pgfqpoint{2.662075in}{8.473868in}}%
\pgfpathlineto{\pgfqpoint{2.662075in}{8.386132in}}%
\pgfpathlineto{\pgfqpoint{2.574339in}{8.386132in}}%
\pgfpathlineto{\pgfqpoint{2.574339in}{8.473868in}}%
\pgfusepath{stroke,fill}%
\end{pgfscope}%
\begin{pgfscope}%
\pgfpathrectangle{\pgfqpoint{0.380943in}{8.035189in}}{\pgfqpoint{4.650000in}{0.614151in}}%
\pgfusepath{clip}%
\pgfsetbuttcap%
\pgfsetroundjoin%
\definecolor{currentfill}{rgb}{1.000000,0.531903,0.500946}%
\pgfsetfillcolor{currentfill}%
\pgfsetlinewidth{0.250937pt}%
\definecolor{currentstroke}{rgb}{1.000000,1.000000,1.000000}%
\pgfsetstrokecolor{currentstroke}%
\pgfsetdash{}{0pt}%
\pgfpathmoveto{\pgfqpoint{2.662075in}{8.473868in}}%
\pgfpathlineto{\pgfqpoint{2.749811in}{8.473868in}}%
\pgfpathlineto{\pgfqpoint{2.749811in}{8.386132in}}%
\pgfpathlineto{\pgfqpoint{2.662075in}{8.386132in}}%
\pgfpathlineto{\pgfqpoint{2.662075in}{8.473868in}}%
\pgfusepath{stroke,fill}%
\end{pgfscope}%
\begin{pgfscope}%
\pgfpathrectangle{\pgfqpoint{0.380943in}{8.035189in}}{\pgfqpoint{4.650000in}{0.614151in}}%
\pgfusepath{clip}%
\pgfsetbuttcap%
\pgfsetroundjoin%
\definecolor{currentfill}{rgb}{0.996909,0.711742,0.584452}%
\pgfsetfillcolor{currentfill}%
\pgfsetlinewidth{0.250937pt}%
\definecolor{currentstroke}{rgb}{1.000000,1.000000,1.000000}%
\pgfsetstrokecolor{currentstroke}%
\pgfsetdash{}{0pt}%
\pgfpathmoveto{\pgfqpoint{2.749811in}{8.473868in}}%
\pgfpathlineto{\pgfqpoint{2.837547in}{8.473868in}}%
\pgfpathlineto{\pgfqpoint{2.837547in}{8.386132in}}%
\pgfpathlineto{\pgfqpoint{2.749811in}{8.386132in}}%
\pgfpathlineto{\pgfqpoint{2.749811in}{8.473868in}}%
\pgfusepath{stroke,fill}%
\end{pgfscope}%
\begin{pgfscope}%
\pgfpathrectangle{\pgfqpoint{0.380943in}{8.035189in}}{\pgfqpoint{4.650000in}{0.614151in}}%
\pgfusepath{clip}%
\pgfsetbuttcap%
\pgfsetroundjoin%
\definecolor{currentfill}{rgb}{0.970012,0.883276,0.699577}%
\pgfsetfillcolor{currentfill}%
\pgfsetlinewidth{0.250937pt}%
\definecolor{currentstroke}{rgb}{1.000000,1.000000,1.000000}%
\pgfsetstrokecolor{currentstroke}%
\pgfsetdash{}{0pt}%
\pgfpathmoveto{\pgfqpoint{2.837547in}{8.473868in}}%
\pgfpathlineto{\pgfqpoint{2.925283in}{8.473868in}}%
\pgfpathlineto{\pgfqpoint{2.925283in}{8.386132in}}%
\pgfpathlineto{\pgfqpoint{2.837547in}{8.386132in}}%
\pgfpathlineto{\pgfqpoint{2.837547in}{8.473868in}}%
\pgfusepath{stroke,fill}%
\end{pgfscope}%
\begin{pgfscope}%
\pgfpathrectangle{\pgfqpoint{0.380943in}{8.035189in}}{\pgfqpoint{4.650000in}{0.614151in}}%
\pgfusepath{clip}%
\pgfsetbuttcap%
\pgfsetroundjoin%
\definecolor{currentfill}{rgb}{0.978131,0.843783,0.675709}%
\pgfsetfillcolor{currentfill}%
\pgfsetlinewidth{0.250937pt}%
\definecolor{currentstroke}{rgb}{1.000000,1.000000,1.000000}%
\pgfsetstrokecolor{currentstroke}%
\pgfsetdash{}{0pt}%
\pgfpathmoveto{\pgfqpoint{2.925283in}{8.473868in}}%
\pgfpathlineto{\pgfqpoint{3.013019in}{8.473868in}}%
\pgfpathlineto{\pgfqpoint{3.013019in}{8.386132in}}%
\pgfpathlineto{\pgfqpoint{2.925283in}{8.386132in}}%
\pgfpathlineto{\pgfqpoint{2.925283in}{8.473868in}}%
\pgfusepath{stroke,fill}%
\end{pgfscope}%
\begin{pgfscope}%
\pgfpathrectangle{\pgfqpoint{0.380943in}{8.035189in}}{\pgfqpoint{4.650000in}{0.614151in}}%
\pgfusepath{clip}%
\pgfsetbuttcap%
\pgfsetroundjoin%
\definecolor{currentfill}{rgb}{0.865975,0.344406,0.344406}%
\pgfsetfillcolor{currentfill}%
\pgfsetlinewidth{0.250937pt}%
\definecolor{currentstroke}{rgb}{1.000000,1.000000,1.000000}%
\pgfsetstrokecolor{currentstroke}%
\pgfsetdash{}{0pt}%
\pgfpathmoveto{\pgfqpoint{3.013019in}{8.473868in}}%
\pgfpathlineto{\pgfqpoint{3.100754in}{8.473868in}}%
\pgfpathlineto{\pgfqpoint{3.100754in}{8.386132in}}%
\pgfpathlineto{\pgfqpoint{3.013019in}{8.386132in}}%
\pgfpathlineto{\pgfqpoint{3.013019in}{8.473868in}}%
\pgfusepath{stroke,fill}%
\end{pgfscope}%
\begin{pgfscope}%
\pgfpathrectangle{\pgfqpoint{0.380943in}{8.035189in}}{\pgfqpoint{4.650000in}{0.614151in}}%
\pgfusepath{clip}%
\pgfsetbuttcap%
\pgfsetroundjoin%
\definecolor{currentfill}{rgb}{0.978131,0.843783,0.675709}%
\pgfsetfillcolor{currentfill}%
\pgfsetlinewidth{0.250937pt}%
\definecolor{currentstroke}{rgb}{1.000000,1.000000,1.000000}%
\pgfsetstrokecolor{currentstroke}%
\pgfsetdash{}{0pt}%
\pgfpathmoveto{\pgfqpoint{3.100754in}{8.473868in}}%
\pgfpathlineto{\pgfqpoint{3.188490in}{8.473868in}}%
\pgfpathlineto{\pgfqpoint{3.188490in}{8.386132in}}%
\pgfpathlineto{\pgfqpoint{3.100754in}{8.386132in}}%
\pgfpathlineto{\pgfqpoint{3.100754in}{8.473868in}}%
\pgfusepath{stroke,fill}%
\end{pgfscope}%
\begin{pgfscope}%
\pgfpathrectangle{\pgfqpoint{0.380943in}{8.035189in}}{\pgfqpoint{4.650000in}{0.614151in}}%
\pgfusepath{clip}%
\pgfsetbuttcap%
\pgfsetroundjoin%
\definecolor{currentfill}{rgb}{0.978131,0.843783,0.675709}%
\pgfsetfillcolor{currentfill}%
\pgfsetlinewidth{0.250937pt}%
\definecolor{currentstroke}{rgb}{1.000000,1.000000,1.000000}%
\pgfsetstrokecolor{currentstroke}%
\pgfsetdash{}{0pt}%
\pgfpathmoveto{\pgfqpoint{3.188490in}{8.473868in}}%
\pgfpathlineto{\pgfqpoint{3.276226in}{8.473868in}}%
\pgfpathlineto{\pgfqpoint{3.276226in}{8.386132in}}%
\pgfpathlineto{\pgfqpoint{3.188490in}{8.386132in}}%
\pgfpathlineto{\pgfqpoint{3.188490in}{8.473868in}}%
\pgfusepath{stroke,fill}%
\end{pgfscope}%
\begin{pgfscope}%
\pgfpathrectangle{\pgfqpoint{0.380943in}{8.035189in}}{\pgfqpoint{4.650000in}{0.614151in}}%
\pgfusepath{clip}%
\pgfsetbuttcap%
\pgfsetroundjoin%
\definecolor{currentfill}{rgb}{0.992326,0.765229,0.614840}%
\pgfsetfillcolor{currentfill}%
\pgfsetlinewidth{0.250937pt}%
\definecolor{currentstroke}{rgb}{1.000000,1.000000,1.000000}%
\pgfsetstrokecolor{currentstroke}%
\pgfsetdash{}{0pt}%
\pgfpathmoveto{\pgfqpoint{3.276226in}{8.473868in}}%
\pgfpathlineto{\pgfqpoint{3.363962in}{8.473868in}}%
\pgfpathlineto{\pgfqpoint{3.363962in}{8.386132in}}%
\pgfpathlineto{\pgfqpoint{3.276226in}{8.386132in}}%
\pgfpathlineto{\pgfqpoint{3.276226in}{8.473868in}}%
\pgfusepath{stroke,fill}%
\end{pgfscope}%
\begin{pgfscope}%
\pgfpathrectangle{\pgfqpoint{0.380943in}{8.035189in}}{\pgfqpoint{4.650000in}{0.614151in}}%
\pgfusepath{clip}%
\pgfsetbuttcap%
\pgfsetroundjoin%
\definecolor{currentfill}{rgb}{0.992326,0.765229,0.614840}%
\pgfsetfillcolor{currentfill}%
\pgfsetlinewidth{0.250937pt}%
\definecolor{currentstroke}{rgb}{1.000000,1.000000,1.000000}%
\pgfsetstrokecolor{currentstroke}%
\pgfsetdash{}{0pt}%
\pgfpathmoveto{\pgfqpoint{3.363962in}{8.473868in}}%
\pgfpathlineto{\pgfqpoint{3.451698in}{8.473868in}}%
\pgfpathlineto{\pgfqpoint{3.451698in}{8.386132in}}%
\pgfpathlineto{\pgfqpoint{3.363962in}{8.386132in}}%
\pgfpathlineto{\pgfqpoint{3.363962in}{8.473868in}}%
\pgfusepath{stroke,fill}%
\end{pgfscope}%
\begin{pgfscope}%
\pgfpathrectangle{\pgfqpoint{0.380943in}{8.035189in}}{\pgfqpoint{4.650000in}{0.614151in}}%
\pgfusepath{clip}%
\pgfsetbuttcap%
\pgfsetroundjoin%
\definecolor{currentfill}{rgb}{0.963768,0.915433,0.717478}%
\pgfsetfillcolor{currentfill}%
\pgfsetlinewidth{0.250937pt}%
\definecolor{currentstroke}{rgb}{1.000000,1.000000,1.000000}%
\pgfsetstrokecolor{currentstroke}%
\pgfsetdash{}{0pt}%
\pgfpathmoveto{\pgfqpoint{3.451698in}{8.473868in}}%
\pgfpathlineto{\pgfqpoint{3.539434in}{8.473868in}}%
\pgfpathlineto{\pgfqpoint{3.539434in}{8.386132in}}%
\pgfpathlineto{\pgfqpoint{3.451698in}{8.386132in}}%
\pgfpathlineto{\pgfqpoint{3.451698in}{8.473868in}}%
\pgfusepath{stroke,fill}%
\end{pgfscope}%
\begin{pgfscope}%
\pgfpathrectangle{\pgfqpoint{0.380943in}{8.035189in}}{\pgfqpoint{4.650000in}{0.614151in}}%
\pgfusepath{clip}%
\pgfsetbuttcap%
\pgfsetroundjoin%
\definecolor{currentfill}{rgb}{0.978131,0.843783,0.675709}%
\pgfsetfillcolor{currentfill}%
\pgfsetlinewidth{0.250937pt}%
\definecolor{currentstroke}{rgb}{1.000000,1.000000,1.000000}%
\pgfsetstrokecolor{currentstroke}%
\pgfsetdash{}{0pt}%
\pgfpathmoveto{\pgfqpoint{3.539434in}{8.473868in}}%
\pgfpathlineto{\pgfqpoint{3.627169in}{8.473868in}}%
\pgfpathlineto{\pgfqpoint{3.627169in}{8.386132in}}%
\pgfpathlineto{\pgfqpoint{3.539434in}{8.386132in}}%
\pgfpathlineto{\pgfqpoint{3.539434in}{8.473868in}}%
\pgfusepath{stroke,fill}%
\end{pgfscope}%
\begin{pgfscope}%
\pgfpathrectangle{\pgfqpoint{0.380943in}{8.035189in}}{\pgfqpoint{4.650000in}{0.614151in}}%
\pgfusepath{clip}%
\pgfsetbuttcap%
\pgfsetroundjoin%
\definecolor{currentfill}{rgb}{0.970012,0.883276,0.699577}%
\pgfsetfillcolor{currentfill}%
\pgfsetlinewidth{0.250937pt}%
\definecolor{currentstroke}{rgb}{1.000000,1.000000,1.000000}%
\pgfsetstrokecolor{currentstroke}%
\pgfsetdash{}{0pt}%
\pgfpathmoveto{\pgfqpoint{3.627169in}{8.473868in}}%
\pgfpathlineto{\pgfqpoint{3.714905in}{8.473868in}}%
\pgfpathlineto{\pgfqpoint{3.714905in}{8.386132in}}%
\pgfpathlineto{\pgfqpoint{3.627169in}{8.386132in}}%
\pgfpathlineto{\pgfqpoint{3.627169in}{8.473868in}}%
\pgfusepath{stroke,fill}%
\end{pgfscope}%
\begin{pgfscope}%
\pgfpathrectangle{\pgfqpoint{0.380943in}{8.035189in}}{\pgfqpoint{4.650000in}{0.614151in}}%
\pgfusepath{clip}%
\pgfsetbuttcap%
\pgfsetroundjoin%
\definecolor{currentfill}{rgb}{0.986251,0.808597,0.643230}%
\pgfsetfillcolor{currentfill}%
\pgfsetlinewidth{0.250937pt}%
\definecolor{currentstroke}{rgb}{1.000000,1.000000,1.000000}%
\pgfsetstrokecolor{currentstroke}%
\pgfsetdash{}{0pt}%
\pgfpathmoveto{\pgfqpoint{3.714905in}{8.473868in}}%
\pgfpathlineto{\pgfqpoint{3.802641in}{8.473868in}}%
\pgfpathlineto{\pgfqpoint{3.802641in}{8.386132in}}%
\pgfpathlineto{\pgfqpoint{3.714905in}{8.386132in}}%
\pgfpathlineto{\pgfqpoint{3.714905in}{8.473868in}}%
\pgfusepath{stroke,fill}%
\end{pgfscope}%
\begin{pgfscope}%
\pgfpathrectangle{\pgfqpoint{0.380943in}{8.035189in}}{\pgfqpoint{4.650000in}{0.614151in}}%
\pgfusepath{clip}%
\pgfsetbuttcap%
\pgfsetroundjoin%
\definecolor{currentfill}{rgb}{0.970012,0.883276,0.699577}%
\pgfsetfillcolor{currentfill}%
\pgfsetlinewidth{0.250937pt}%
\definecolor{currentstroke}{rgb}{1.000000,1.000000,1.000000}%
\pgfsetstrokecolor{currentstroke}%
\pgfsetdash{}{0pt}%
\pgfpathmoveto{\pgfqpoint{3.802641in}{8.473868in}}%
\pgfpathlineto{\pgfqpoint{3.890377in}{8.473868in}}%
\pgfpathlineto{\pgfqpoint{3.890377in}{8.386132in}}%
\pgfpathlineto{\pgfqpoint{3.802641in}{8.386132in}}%
\pgfpathlineto{\pgfqpoint{3.802641in}{8.473868in}}%
\pgfusepath{stroke,fill}%
\end{pgfscope}%
\begin{pgfscope}%
\pgfpathrectangle{\pgfqpoint{0.380943in}{8.035189in}}{\pgfqpoint{4.650000in}{0.614151in}}%
\pgfusepath{clip}%
\pgfsetbuttcap%
\pgfsetroundjoin%
\definecolor{currentfill}{rgb}{0.986251,0.808597,0.643230}%
\pgfsetfillcolor{currentfill}%
\pgfsetlinewidth{0.250937pt}%
\definecolor{currentstroke}{rgb}{1.000000,1.000000,1.000000}%
\pgfsetstrokecolor{currentstroke}%
\pgfsetdash{}{0pt}%
\pgfpathmoveto{\pgfqpoint{3.890377in}{8.473868in}}%
\pgfpathlineto{\pgfqpoint{3.978113in}{8.473868in}}%
\pgfpathlineto{\pgfqpoint{3.978113in}{8.386132in}}%
\pgfpathlineto{\pgfqpoint{3.890377in}{8.386132in}}%
\pgfpathlineto{\pgfqpoint{3.890377in}{8.473868in}}%
\pgfusepath{stroke,fill}%
\end{pgfscope}%
\begin{pgfscope}%
\pgfpathrectangle{\pgfqpoint{0.380943in}{8.035189in}}{\pgfqpoint{4.650000in}{0.614151in}}%
\pgfusepath{clip}%
\pgfsetbuttcap%
\pgfsetroundjoin%
\definecolor{currentfill}{rgb}{0.986251,0.808597,0.643230}%
\pgfsetfillcolor{currentfill}%
\pgfsetlinewidth{0.250937pt}%
\definecolor{currentstroke}{rgb}{1.000000,1.000000,1.000000}%
\pgfsetstrokecolor{currentstroke}%
\pgfsetdash{}{0pt}%
\pgfpathmoveto{\pgfqpoint{3.978113in}{8.473868in}}%
\pgfpathlineto{\pgfqpoint{4.065849in}{8.473868in}}%
\pgfpathlineto{\pgfqpoint{4.065849in}{8.386132in}}%
\pgfpathlineto{\pgfqpoint{3.978113in}{8.386132in}}%
\pgfpathlineto{\pgfqpoint{3.978113in}{8.473868in}}%
\pgfusepath{stroke,fill}%
\end{pgfscope}%
\begin{pgfscope}%
\pgfpathrectangle{\pgfqpoint{0.380943in}{8.035189in}}{\pgfqpoint{4.650000in}{0.614151in}}%
\pgfusepath{clip}%
\pgfsetbuttcap%
\pgfsetroundjoin%
\definecolor{currentfill}{rgb}{0.992326,0.765229,0.614840}%
\pgfsetfillcolor{currentfill}%
\pgfsetlinewidth{0.250937pt}%
\definecolor{currentstroke}{rgb}{1.000000,1.000000,1.000000}%
\pgfsetstrokecolor{currentstroke}%
\pgfsetdash{}{0pt}%
\pgfpathmoveto{\pgfqpoint{4.065849in}{8.473868in}}%
\pgfpathlineto{\pgfqpoint{4.153585in}{8.473868in}}%
\pgfpathlineto{\pgfqpoint{4.153585in}{8.386132in}}%
\pgfpathlineto{\pgfqpoint{4.065849in}{8.386132in}}%
\pgfpathlineto{\pgfqpoint{4.065849in}{8.473868in}}%
\pgfusepath{stroke,fill}%
\end{pgfscope}%
\begin{pgfscope}%
\pgfpathrectangle{\pgfqpoint{0.380943in}{8.035189in}}{\pgfqpoint{4.650000in}{0.614151in}}%
\pgfusepath{clip}%
\pgfsetbuttcap%
\pgfsetroundjoin%
\definecolor{currentfill}{rgb}{0.978131,0.843783,0.675709}%
\pgfsetfillcolor{currentfill}%
\pgfsetlinewidth{0.250937pt}%
\definecolor{currentstroke}{rgb}{1.000000,1.000000,1.000000}%
\pgfsetstrokecolor{currentstroke}%
\pgfsetdash{}{0pt}%
\pgfpathmoveto{\pgfqpoint{4.153585in}{8.473868in}}%
\pgfpathlineto{\pgfqpoint{4.241320in}{8.473868in}}%
\pgfpathlineto{\pgfqpoint{4.241320in}{8.386132in}}%
\pgfpathlineto{\pgfqpoint{4.153585in}{8.386132in}}%
\pgfpathlineto{\pgfqpoint{4.153585in}{8.473868in}}%
\pgfusepath{stroke,fill}%
\end{pgfscope}%
\begin{pgfscope}%
\pgfpathrectangle{\pgfqpoint{0.380943in}{8.035189in}}{\pgfqpoint{4.650000in}{0.614151in}}%
\pgfusepath{clip}%
\pgfsetbuttcap%
\pgfsetroundjoin%
\definecolor{currentfill}{rgb}{0.963768,0.915433,0.717478}%
\pgfsetfillcolor{currentfill}%
\pgfsetlinewidth{0.250937pt}%
\definecolor{currentstroke}{rgb}{1.000000,1.000000,1.000000}%
\pgfsetstrokecolor{currentstroke}%
\pgfsetdash{}{0pt}%
\pgfpathmoveto{\pgfqpoint{4.241320in}{8.473868in}}%
\pgfpathlineto{\pgfqpoint{4.329056in}{8.473868in}}%
\pgfpathlineto{\pgfqpoint{4.329056in}{8.386132in}}%
\pgfpathlineto{\pgfqpoint{4.241320in}{8.386132in}}%
\pgfpathlineto{\pgfqpoint{4.241320in}{8.473868in}}%
\pgfusepath{stroke,fill}%
\end{pgfscope}%
\begin{pgfscope}%
\pgfpathrectangle{\pgfqpoint{0.380943in}{8.035189in}}{\pgfqpoint{4.650000in}{0.614151in}}%
\pgfusepath{clip}%
\pgfsetbuttcap%
\pgfsetroundjoin%
\definecolor{currentfill}{rgb}{0.963768,0.915433,0.717478}%
\pgfsetfillcolor{currentfill}%
\pgfsetlinewidth{0.250937pt}%
\definecolor{currentstroke}{rgb}{1.000000,1.000000,1.000000}%
\pgfsetstrokecolor{currentstroke}%
\pgfsetdash{}{0pt}%
\pgfpathmoveto{\pgfqpoint{4.329056in}{8.473868in}}%
\pgfpathlineto{\pgfqpoint{4.416792in}{8.473868in}}%
\pgfpathlineto{\pgfqpoint{4.416792in}{8.386132in}}%
\pgfpathlineto{\pgfqpoint{4.329056in}{8.386132in}}%
\pgfpathlineto{\pgfqpoint{4.329056in}{8.473868in}}%
\pgfusepath{stroke,fill}%
\end{pgfscope}%
\begin{pgfscope}%
\pgfpathrectangle{\pgfqpoint{0.380943in}{8.035189in}}{\pgfqpoint{4.650000in}{0.614151in}}%
\pgfusepath{clip}%
\pgfsetbuttcap%
\pgfsetroundjoin%
\definecolor{currentfill}{rgb}{0.970012,0.883276,0.699577}%
\pgfsetfillcolor{currentfill}%
\pgfsetlinewidth{0.250937pt}%
\definecolor{currentstroke}{rgb}{1.000000,1.000000,1.000000}%
\pgfsetstrokecolor{currentstroke}%
\pgfsetdash{}{0pt}%
\pgfpathmoveto{\pgfqpoint{4.416792in}{8.473868in}}%
\pgfpathlineto{\pgfqpoint{4.504528in}{8.473868in}}%
\pgfpathlineto{\pgfqpoint{4.504528in}{8.386132in}}%
\pgfpathlineto{\pgfqpoint{4.416792in}{8.386132in}}%
\pgfpathlineto{\pgfqpoint{4.416792in}{8.473868in}}%
\pgfusepath{stroke,fill}%
\end{pgfscope}%
\begin{pgfscope}%
\pgfpathrectangle{\pgfqpoint{0.380943in}{8.035189in}}{\pgfqpoint{4.650000in}{0.614151in}}%
\pgfusepath{clip}%
\pgfsetbuttcap%
\pgfsetroundjoin%
\definecolor{currentfill}{rgb}{0.992326,0.765229,0.614840}%
\pgfsetfillcolor{currentfill}%
\pgfsetlinewidth{0.250937pt}%
\definecolor{currentstroke}{rgb}{1.000000,1.000000,1.000000}%
\pgfsetstrokecolor{currentstroke}%
\pgfsetdash{}{0pt}%
\pgfpathmoveto{\pgfqpoint{4.504528in}{8.473868in}}%
\pgfpathlineto{\pgfqpoint{4.592264in}{8.473868in}}%
\pgfpathlineto{\pgfqpoint{4.592264in}{8.386132in}}%
\pgfpathlineto{\pgfqpoint{4.504528in}{8.386132in}}%
\pgfpathlineto{\pgfqpoint{4.504528in}{8.473868in}}%
\pgfusepath{stroke,fill}%
\end{pgfscope}%
\begin{pgfscope}%
\pgfpathrectangle{\pgfqpoint{0.380943in}{8.035189in}}{\pgfqpoint{4.650000in}{0.614151in}}%
\pgfusepath{clip}%
\pgfsetbuttcap%
\pgfsetroundjoin%
\definecolor{currentfill}{rgb}{0.978131,0.843783,0.675709}%
\pgfsetfillcolor{currentfill}%
\pgfsetlinewidth{0.250937pt}%
\definecolor{currentstroke}{rgb}{1.000000,1.000000,1.000000}%
\pgfsetstrokecolor{currentstroke}%
\pgfsetdash{}{0pt}%
\pgfpathmoveto{\pgfqpoint{4.592264in}{8.473868in}}%
\pgfpathlineto{\pgfqpoint{4.680000in}{8.473868in}}%
\pgfpathlineto{\pgfqpoint{4.680000in}{8.386132in}}%
\pgfpathlineto{\pgfqpoint{4.592264in}{8.386132in}}%
\pgfpathlineto{\pgfqpoint{4.592264in}{8.473868in}}%
\pgfusepath{stroke,fill}%
\end{pgfscope}%
\begin{pgfscope}%
\pgfpathrectangle{\pgfqpoint{0.380943in}{8.035189in}}{\pgfqpoint{4.650000in}{0.614151in}}%
\pgfusepath{clip}%
\pgfsetbuttcap%
\pgfsetroundjoin%
\definecolor{currentfill}{rgb}{0.978131,0.843783,0.675709}%
\pgfsetfillcolor{currentfill}%
\pgfsetlinewidth{0.250937pt}%
\definecolor{currentstroke}{rgb}{1.000000,1.000000,1.000000}%
\pgfsetstrokecolor{currentstroke}%
\pgfsetdash{}{0pt}%
\pgfpathmoveto{\pgfqpoint{4.680000in}{8.473868in}}%
\pgfpathlineto{\pgfqpoint{4.767736in}{8.473868in}}%
\pgfpathlineto{\pgfqpoint{4.767736in}{8.386132in}}%
\pgfpathlineto{\pgfqpoint{4.680000in}{8.386132in}}%
\pgfpathlineto{\pgfqpoint{4.680000in}{8.473868in}}%
\pgfusepath{stroke,fill}%
\end{pgfscope}%
\begin{pgfscope}%
\pgfpathrectangle{\pgfqpoint{0.380943in}{8.035189in}}{\pgfqpoint{4.650000in}{0.614151in}}%
\pgfusepath{clip}%
\pgfsetbuttcap%
\pgfsetroundjoin%
\definecolor{currentfill}{rgb}{0.996909,0.711742,0.584452}%
\pgfsetfillcolor{currentfill}%
\pgfsetlinewidth{0.250937pt}%
\definecolor{currentstroke}{rgb}{1.000000,1.000000,1.000000}%
\pgfsetstrokecolor{currentstroke}%
\pgfsetdash{}{0pt}%
\pgfpathmoveto{\pgfqpoint{4.767736in}{8.473868in}}%
\pgfpathlineto{\pgfqpoint{4.855471in}{8.473868in}}%
\pgfpathlineto{\pgfqpoint{4.855471in}{8.386132in}}%
\pgfpathlineto{\pgfqpoint{4.767736in}{8.386132in}}%
\pgfpathlineto{\pgfqpoint{4.767736in}{8.473868in}}%
\pgfusepath{stroke,fill}%
\end{pgfscope}%
\begin{pgfscope}%
\pgfpathrectangle{\pgfqpoint{0.380943in}{8.035189in}}{\pgfqpoint{4.650000in}{0.614151in}}%
\pgfusepath{clip}%
\pgfsetbuttcap%
\pgfsetroundjoin%
\definecolor{currentfill}{rgb}{0.992326,0.765229,0.614840}%
\pgfsetfillcolor{currentfill}%
\pgfsetlinewidth{0.250937pt}%
\definecolor{currentstroke}{rgb}{1.000000,1.000000,1.000000}%
\pgfsetstrokecolor{currentstroke}%
\pgfsetdash{}{0pt}%
\pgfpathmoveto{\pgfqpoint{4.855471in}{8.473868in}}%
\pgfpathlineto{\pgfqpoint{4.943207in}{8.473868in}}%
\pgfpathlineto{\pgfqpoint{4.943207in}{8.386132in}}%
\pgfpathlineto{\pgfqpoint{4.855471in}{8.386132in}}%
\pgfpathlineto{\pgfqpoint{4.855471in}{8.473868in}}%
\pgfusepath{stroke,fill}%
\end{pgfscope}%
\begin{pgfscope}%
\pgfpathrectangle{\pgfqpoint{0.380943in}{8.035189in}}{\pgfqpoint{4.650000in}{0.614151in}}%
\pgfusepath{clip}%
\pgfsetbuttcap%
\pgfsetroundjoin%
\definecolor{currentfill}{rgb}{0.986251,0.808597,0.643230}%
\pgfsetfillcolor{currentfill}%
\pgfsetlinewidth{0.250937pt}%
\definecolor{currentstroke}{rgb}{1.000000,1.000000,1.000000}%
\pgfsetstrokecolor{currentstroke}%
\pgfsetdash{}{0pt}%
\pgfpathmoveto{\pgfqpoint{4.943207in}{8.473868in}}%
\pgfpathlineto{\pgfqpoint{5.030943in}{8.473868in}}%
\pgfpathlineto{\pgfqpoint{5.030943in}{8.386132in}}%
\pgfpathlineto{\pgfqpoint{4.943207in}{8.386132in}}%
\pgfpathlineto{\pgfqpoint{4.943207in}{8.473868in}}%
\pgfusepath{stroke,fill}%
\end{pgfscope}%
\begin{pgfscope}%
\pgfpathrectangle{\pgfqpoint{0.380943in}{8.035189in}}{\pgfqpoint{4.650000in}{0.614151in}}%
\pgfusepath{clip}%
\pgfsetbuttcap%
\pgfsetroundjoin%
\pgfsetlinewidth{0.250937pt}%
\definecolor{currentstroke}{rgb}{1.000000,1.000000,1.000000}%
\pgfsetstrokecolor{currentstroke}%
\pgfsetdash{}{0pt}%
\pgfpathmoveto{\pgfqpoint{0.380943in}{8.386132in}}%
\pgfpathlineto{\pgfqpoint{0.468679in}{8.386132in}}%
\pgfpathlineto{\pgfqpoint{0.468679in}{8.298396in}}%
\pgfpathlineto{\pgfqpoint{0.380943in}{8.298396in}}%
\pgfpathlineto{\pgfqpoint{0.380943in}{8.386132in}}%
\pgfusepath{stroke}%
\end{pgfscope}%
\begin{pgfscope}%
\pgfpathrectangle{\pgfqpoint{0.380943in}{8.035189in}}{\pgfqpoint{4.650000in}{0.614151in}}%
\pgfusepath{clip}%
\pgfsetbuttcap%
\pgfsetroundjoin%
\definecolor{currentfill}{rgb}{0.978131,0.843783,0.675709}%
\pgfsetfillcolor{currentfill}%
\pgfsetlinewidth{0.250937pt}%
\definecolor{currentstroke}{rgb}{1.000000,1.000000,1.000000}%
\pgfsetstrokecolor{currentstroke}%
\pgfsetdash{}{0pt}%
\pgfpathmoveto{\pgfqpoint{0.468679in}{8.386132in}}%
\pgfpathlineto{\pgfqpoint{0.556415in}{8.386132in}}%
\pgfpathlineto{\pgfqpoint{0.556415in}{8.298396in}}%
\pgfpathlineto{\pgfqpoint{0.468679in}{8.298396in}}%
\pgfpathlineto{\pgfqpoint{0.468679in}{8.386132in}}%
\pgfusepath{stroke,fill}%
\end{pgfscope}%
\begin{pgfscope}%
\pgfpathrectangle{\pgfqpoint{0.380943in}{8.035189in}}{\pgfqpoint{4.650000in}{0.614151in}}%
\pgfusepath{clip}%
\pgfsetbuttcap%
\pgfsetroundjoin%
\definecolor{currentfill}{rgb}{1.000000,0.531903,0.500946}%
\pgfsetfillcolor{currentfill}%
\pgfsetlinewidth{0.250937pt}%
\definecolor{currentstroke}{rgb}{1.000000,1.000000,1.000000}%
\pgfsetstrokecolor{currentstroke}%
\pgfsetdash{}{0pt}%
\pgfpathmoveto{\pgfqpoint{0.556415in}{8.386132in}}%
\pgfpathlineto{\pgfqpoint{0.644151in}{8.386132in}}%
\pgfpathlineto{\pgfqpoint{0.644151in}{8.298396in}}%
\pgfpathlineto{\pgfqpoint{0.556415in}{8.298396in}}%
\pgfpathlineto{\pgfqpoint{0.556415in}{8.386132in}}%
\pgfusepath{stroke,fill}%
\end{pgfscope}%
\begin{pgfscope}%
\pgfpathrectangle{\pgfqpoint{0.380943in}{8.035189in}}{\pgfqpoint{4.650000in}{0.614151in}}%
\pgfusepath{clip}%
\pgfsetbuttcap%
\pgfsetroundjoin%
\definecolor{currentfill}{rgb}{1.000000,0.531903,0.500946}%
\pgfsetfillcolor{currentfill}%
\pgfsetlinewidth{0.250937pt}%
\definecolor{currentstroke}{rgb}{1.000000,1.000000,1.000000}%
\pgfsetstrokecolor{currentstroke}%
\pgfsetdash{}{0pt}%
\pgfpathmoveto{\pgfqpoint{0.644151in}{8.386132in}}%
\pgfpathlineto{\pgfqpoint{0.731886in}{8.386132in}}%
\pgfpathlineto{\pgfqpoint{0.731886in}{8.298396in}}%
\pgfpathlineto{\pgfqpoint{0.644151in}{8.298396in}}%
\pgfpathlineto{\pgfqpoint{0.644151in}{8.386132in}}%
\pgfusepath{stroke,fill}%
\end{pgfscope}%
\begin{pgfscope}%
\pgfpathrectangle{\pgfqpoint{0.380943in}{8.035189in}}{\pgfqpoint{4.650000in}{0.614151in}}%
\pgfusepath{clip}%
\pgfsetbuttcap%
\pgfsetroundjoin%
\definecolor{currentfill}{rgb}{1.000000,0.480477,0.479293}%
\pgfsetfillcolor{currentfill}%
\pgfsetlinewidth{0.250937pt}%
\definecolor{currentstroke}{rgb}{1.000000,1.000000,1.000000}%
\pgfsetstrokecolor{currentstroke}%
\pgfsetdash{}{0pt}%
\pgfpathmoveto{\pgfqpoint{0.731886in}{8.386132in}}%
\pgfpathlineto{\pgfqpoint{0.819622in}{8.386132in}}%
\pgfpathlineto{\pgfqpoint{0.819622in}{8.298396in}}%
\pgfpathlineto{\pgfqpoint{0.731886in}{8.298396in}}%
\pgfpathlineto{\pgfqpoint{0.731886in}{8.386132in}}%
\pgfusepath{stroke,fill}%
\end{pgfscope}%
\begin{pgfscope}%
\pgfpathrectangle{\pgfqpoint{0.380943in}{8.035189in}}{\pgfqpoint{4.650000in}{0.614151in}}%
\pgfusepath{clip}%
\pgfsetbuttcap%
\pgfsetroundjoin%
\definecolor{currentfill}{rgb}{0.970012,0.883276,0.699577}%
\pgfsetfillcolor{currentfill}%
\pgfsetlinewidth{0.250937pt}%
\definecolor{currentstroke}{rgb}{1.000000,1.000000,1.000000}%
\pgfsetstrokecolor{currentstroke}%
\pgfsetdash{}{0pt}%
\pgfpathmoveto{\pgfqpoint{0.819622in}{8.386132in}}%
\pgfpathlineto{\pgfqpoint{0.907358in}{8.386132in}}%
\pgfpathlineto{\pgfqpoint{0.907358in}{8.298396in}}%
\pgfpathlineto{\pgfqpoint{0.819622in}{8.298396in}}%
\pgfpathlineto{\pgfqpoint{0.819622in}{8.386132in}}%
\pgfusepath{stroke,fill}%
\end{pgfscope}%
\begin{pgfscope}%
\pgfpathrectangle{\pgfqpoint{0.380943in}{8.035189in}}{\pgfqpoint{4.650000in}{0.614151in}}%
\pgfusepath{clip}%
\pgfsetbuttcap%
\pgfsetroundjoin%
\definecolor{currentfill}{rgb}{0.986251,0.808597,0.643230}%
\pgfsetfillcolor{currentfill}%
\pgfsetlinewidth{0.250937pt}%
\definecolor{currentstroke}{rgb}{1.000000,1.000000,1.000000}%
\pgfsetstrokecolor{currentstroke}%
\pgfsetdash{}{0pt}%
\pgfpathmoveto{\pgfqpoint{0.907358in}{8.386132in}}%
\pgfpathlineto{\pgfqpoint{0.995094in}{8.386132in}}%
\pgfpathlineto{\pgfqpoint{0.995094in}{8.298396in}}%
\pgfpathlineto{\pgfqpoint{0.907358in}{8.298396in}}%
\pgfpathlineto{\pgfqpoint{0.907358in}{8.386132in}}%
\pgfusepath{stroke,fill}%
\end{pgfscope}%
\begin{pgfscope}%
\pgfpathrectangle{\pgfqpoint{0.380943in}{8.035189in}}{\pgfqpoint{4.650000in}{0.614151in}}%
\pgfusepath{clip}%
\pgfsetbuttcap%
\pgfsetroundjoin%
\definecolor{currentfill}{rgb}{0.986251,0.808597,0.643230}%
\pgfsetfillcolor{currentfill}%
\pgfsetlinewidth{0.250937pt}%
\definecolor{currentstroke}{rgb}{1.000000,1.000000,1.000000}%
\pgfsetstrokecolor{currentstroke}%
\pgfsetdash{}{0pt}%
\pgfpathmoveto{\pgfqpoint{0.995094in}{8.386132in}}%
\pgfpathlineto{\pgfqpoint{1.082830in}{8.386132in}}%
\pgfpathlineto{\pgfqpoint{1.082830in}{8.298396in}}%
\pgfpathlineto{\pgfqpoint{0.995094in}{8.298396in}}%
\pgfpathlineto{\pgfqpoint{0.995094in}{8.386132in}}%
\pgfusepath{stroke,fill}%
\end{pgfscope}%
\begin{pgfscope}%
\pgfpathrectangle{\pgfqpoint{0.380943in}{8.035189in}}{\pgfqpoint{4.650000in}{0.614151in}}%
\pgfusepath{clip}%
\pgfsetbuttcap%
\pgfsetroundjoin%
\definecolor{currentfill}{rgb}{1.000000,0.480477,0.479293}%
\pgfsetfillcolor{currentfill}%
\pgfsetlinewidth{0.250937pt}%
\definecolor{currentstroke}{rgb}{1.000000,1.000000,1.000000}%
\pgfsetstrokecolor{currentstroke}%
\pgfsetdash{}{0pt}%
\pgfpathmoveto{\pgfqpoint{1.082830in}{8.386132in}}%
\pgfpathlineto{\pgfqpoint{1.170566in}{8.386132in}}%
\pgfpathlineto{\pgfqpoint{1.170566in}{8.298396in}}%
\pgfpathlineto{\pgfqpoint{1.082830in}{8.298396in}}%
\pgfpathlineto{\pgfqpoint{1.082830in}{8.386132in}}%
\pgfusepath{stroke,fill}%
\end{pgfscope}%
\begin{pgfscope}%
\pgfpathrectangle{\pgfqpoint{0.380943in}{8.035189in}}{\pgfqpoint{4.650000in}{0.614151in}}%
\pgfusepath{clip}%
\pgfsetbuttcap%
\pgfsetroundjoin%
\definecolor{currentfill}{rgb}{1.000000,0.531903,0.500946}%
\pgfsetfillcolor{currentfill}%
\pgfsetlinewidth{0.250937pt}%
\definecolor{currentstroke}{rgb}{1.000000,1.000000,1.000000}%
\pgfsetstrokecolor{currentstroke}%
\pgfsetdash{}{0pt}%
\pgfpathmoveto{\pgfqpoint{1.170566in}{8.386132in}}%
\pgfpathlineto{\pgfqpoint{1.258302in}{8.386132in}}%
\pgfpathlineto{\pgfqpoint{1.258302in}{8.298396in}}%
\pgfpathlineto{\pgfqpoint{1.170566in}{8.298396in}}%
\pgfpathlineto{\pgfqpoint{1.170566in}{8.386132in}}%
\pgfusepath{stroke,fill}%
\end{pgfscope}%
\begin{pgfscope}%
\pgfpathrectangle{\pgfqpoint{0.380943in}{8.035189in}}{\pgfqpoint{4.650000in}{0.614151in}}%
\pgfusepath{clip}%
\pgfsetbuttcap%
\pgfsetroundjoin%
\definecolor{currentfill}{rgb}{0.986251,0.808597,0.643230}%
\pgfsetfillcolor{currentfill}%
\pgfsetlinewidth{0.250937pt}%
\definecolor{currentstroke}{rgb}{1.000000,1.000000,1.000000}%
\pgfsetstrokecolor{currentstroke}%
\pgfsetdash{}{0pt}%
\pgfpathmoveto{\pgfqpoint{1.258302in}{8.386132in}}%
\pgfpathlineto{\pgfqpoint{1.346037in}{8.386132in}}%
\pgfpathlineto{\pgfqpoint{1.346037in}{8.298396in}}%
\pgfpathlineto{\pgfqpoint{1.258302in}{8.298396in}}%
\pgfpathlineto{\pgfqpoint{1.258302in}{8.386132in}}%
\pgfusepath{stroke,fill}%
\end{pgfscope}%
\begin{pgfscope}%
\pgfpathrectangle{\pgfqpoint{0.380943in}{8.035189in}}{\pgfqpoint{4.650000in}{0.614151in}}%
\pgfusepath{clip}%
\pgfsetbuttcap%
\pgfsetroundjoin%
\definecolor{currentfill}{rgb}{0.992326,0.765229,0.614840}%
\pgfsetfillcolor{currentfill}%
\pgfsetlinewidth{0.250937pt}%
\definecolor{currentstroke}{rgb}{1.000000,1.000000,1.000000}%
\pgfsetstrokecolor{currentstroke}%
\pgfsetdash{}{0pt}%
\pgfpathmoveto{\pgfqpoint{1.346037in}{8.386132in}}%
\pgfpathlineto{\pgfqpoint{1.433773in}{8.386132in}}%
\pgfpathlineto{\pgfqpoint{1.433773in}{8.298396in}}%
\pgfpathlineto{\pgfqpoint{1.346037in}{8.298396in}}%
\pgfpathlineto{\pgfqpoint{1.346037in}{8.386132in}}%
\pgfusepath{stroke,fill}%
\end{pgfscope}%
\begin{pgfscope}%
\pgfpathrectangle{\pgfqpoint{0.380943in}{8.035189in}}{\pgfqpoint{4.650000in}{0.614151in}}%
\pgfusepath{clip}%
\pgfsetbuttcap%
\pgfsetroundjoin%
\definecolor{currentfill}{rgb}{0.961061,0.931672,0.728304}%
\pgfsetfillcolor{currentfill}%
\pgfsetlinewidth{0.250937pt}%
\definecolor{currentstroke}{rgb}{1.000000,1.000000,1.000000}%
\pgfsetstrokecolor{currentstroke}%
\pgfsetdash{}{0pt}%
\pgfpathmoveto{\pgfqpoint{1.433773in}{8.386132in}}%
\pgfpathlineto{\pgfqpoint{1.521509in}{8.386132in}}%
\pgfpathlineto{\pgfqpoint{1.521509in}{8.298396in}}%
\pgfpathlineto{\pgfqpoint{1.433773in}{8.298396in}}%
\pgfpathlineto{\pgfqpoint{1.433773in}{8.386132in}}%
\pgfusepath{stroke,fill}%
\end{pgfscope}%
\begin{pgfscope}%
\pgfpathrectangle{\pgfqpoint{0.380943in}{8.035189in}}{\pgfqpoint{4.650000in}{0.614151in}}%
\pgfusepath{clip}%
\pgfsetbuttcap%
\pgfsetroundjoin%
\definecolor{currentfill}{rgb}{0.992326,0.765229,0.614840}%
\pgfsetfillcolor{currentfill}%
\pgfsetlinewidth{0.250937pt}%
\definecolor{currentstroke}{rgb}{1.000000,1.000000,1.000000}%
\pgfsetstrokecolor{currentstroke}%
\pgfsetdash{}{0pt}%
\pgfpathmoveto{\pgfqpoint{1.521509in}{8.386132in}}%
\pgfpathlineto{\pgfqpoint{1.609245in}{8.386132in}}%
\pgfpathlineto{\pgfqpoint{1.609245in}{8.298396in}}%
\pgfpathlineto{\pgfqpoint{1.521509in}{8.298396in}}%
\pgfpathlineto{\pgfqpoint{1.521509in}{8.386132in}}%
\pgfusepath{stroke,fill}%
\end{pgfscope}%
\begin{pgfscope}%
\pgfpathrectangle{\pgfqpoint{0.380943in}{8.035189in}}{\pgfqpoint{4.650000in}{0.614151in}}%
\pgfusepath{clip}%
\pgfsetbuttcap%
\pgfsetroundjoin%
\definecolor{currentfill}{rgb}{0.986251,0.808597,0.643230}%
\pgfsetfillcolor{currentfill}%
\pgfsetlinewidth{0.250937pt}%
\definecolor{currentstroke}{rgb}{1.000000,1.000000,1.000000}%
\pgfsetstrokecolor{currentstroke}%
\pgfsetdash{}{0pt}%
\pgfpathmoveto{\pgfqpoint{1.609245in}{8.386132in}}%
\pgfpathlineto{\pgfqpoint{1.696981in}{8.386132in}}%
\pgfpathlineto{\pgfqpoint{1.696981in}{8.298396in}}%
\pgfpathlineto{\pgfqpoint{1.609245in}{8.298396in}}%
\pgfpathlineto{\pgfqpoint{1.609245in}{8.386132in}}%
\pgfusepath{stroke,fill}%
\end{pgfscope}%
\begin{pgfscope}%
\pgfpathrectangle{\pgfqpoint{0.380943in}{8.035189in}}{\pgfqpoint{4.650000in}{0.614151in}}%
\pgfusepath{clip}%
\pgfsetbuttcap%
\pgfsetroundjoin%
\definecolor{currentfill}{rgb}{0.992326,0.765229,0.614840}%
\pgfsetfillcolor{currentfill}%
\pgfsetlinewidth{0.250937pt}%
\definecolor{currentstroke}{rgb}{1.000000,1.000000,1.000000}%
\pgfsetstrokecolor{currentstroke}%
\pgfsetdash{}{0pt}%
\pgfpathmoveto{\pgfqpoint{1.696981in}{8.386132in}}%
\pgfpathlineto{\pgfqpoint{1.784717in}{8.386132in}}%
\pgfpathlineto{\pgfqpoint{1.784717in}{8.298396in}}%
\pgfpathlineto{\pgfqpoint{1.696981in}{8.298396in}}%
\pgfpathlineto{\pgfqpoint{1.696981in}{8.386132in}}%
\pgfusepath{stroke,fill}%
\end{pgfscope}%
\begin{pgfscope}%
\pgfpathrectangle{\pgfqpoint{0.380943in}{8.035189in}}{\pgfqpoint{4.650000in}{0.614151in}}%
\pgfusepath{clip}%
\pgfsetbuttcap%
\pgfsetroundjoin%
\definecolor{currentfill}{rgb}{0.996909,0.711742,0.584452}%
\pgfsetfillcolor{currentfill}%
\pgfsetlinewidth{0.250937pt}%
\definecolor{currentstroke}{rgb}{1.000000,1.000000,1.000000}%
\pgfsetstrokecolor{currentstroke}%
\pgfsetdash{}{0pt}%
\pgfpathmoveto{\pgfqpoint{1.784717in}{8.386132in}}%
\pgfpathlineto{\pgfqpoint{1.872452in}{8.386132in}}%
\pgfpathlineto{\pgfqpoint{1.872452in}{8.298396in}}%
\pgfpathlineto{\pgfqpoint{1.784717in}{8.298396in}}%
\pgfpathlineto{\pgfqpoint{1.784717in}{8.386132in}}%
\pgfusepath{stroke,fill}%
\end{pgfscope}%
\begin{pgfscope}%
\pgfpathrectangle{\pgfqpoint{0.380943in}{8.035189in}}{\pgfqpoint{4.650000in}{0.614151in}}%
\pgfusepath{clip}%
\pgfsetbuttcap%
\pgfsetroundjoin%
\definecolor{currentfill}{rgb}{0.970012,0.883276,0.699577}%
\pgfsetfillcolor{currentfill}%
\pgfsetlinewidth{0.250937pt}%
\definecolor{currentstroke}{rgb}{1.000000,1.000000,1.000000}%
\pgfsetstrokecolor{currentstroke}%
\pgfsetdash{}{0pt}%
\pgfpathmoveto{\pgfqpoint{1.872452in}{8.386132in}}%
\pgfpathlineto{\pgfqpoint{1.960188in}{8.386132in}}%
\pgfpathlineto{\pgfqpoint{1.960188in}{8.298396in}}%
\pgfpathlineto{\pgfqpoint{1.872452in}{8.298396in}}%
\pgfpathlineto{\pgfqpoint{1.872452in}{8.386132in}}%
\pgfusepath{stroke,fill}%
\end{pgfscope}%
\begin{pgfscope}%
\pgfpathrectangle{\pgfqpoint{0.380943in}{8.035189in}}{\pgfqpoint{4.650000in}{0.614151in}}%
\pgfusepath{clip}%
\pgfsetbuttcap%
\pgfsetroundjoin%
\definecolor{currentfill}{rgb}{0.985083,0.974641,0.792587}%
\pgfsetfillcolor{currentfill}%
\pgfsetlinewidth{0.250937pt}%
\definecolor{currentstroke}{rgb}{1.000000,1.000000,1.000000}%
\pgfsetstrokecolor{currentstroke}%
\pgfsetdash{}{0pt}%
\pgfpathmoveto{\pgfqpoint{1.960188in}{8.386132in}}%
\pgfpathlineto{\pgfqpoint{2.047924in}{8.386132in}}%
\pgfpathlineto{\pgfqpoint{2.047924in}{8.298396in}}%
\pgfpathlineto{\pgfqpoint{1.960188in}{8.298396in}}%
\pgfpathlineto{\pgfqpoint{1.960188in}{8.386132in}}%
\pgfusepath{stroke,fill}%
\end{pgfscope}%
\begin{pgfscope}%
\pgfpathrectangle{\pgfqpoint{0.380943in}{8.035189in}}{\pgfqpoint{4.650000in}{0.614151in}}%
\pgfusepath{clip}%
\pgfsetbuttcap%
\pgfsetroundjoin%
\definecolor{currentfill}{rgb}{0.986251,0.808597,0.643230}%
\pgfsetfillcolor{currentfill}%
\pgfsetlinewidth{0.250937pt}%
\definecolor{currentstroke}{rgb}{1.000000,1.000000,1.000000}%
\pgfsetstrokecolor{currentstroke}%
\pgfsetdash{}{0pt}%
\pgfpathmoveto{\pgfqpoint{2.047924in}{8.386132in}}%
\pgfpathlineto{\pgfqpoint{2.135660in}{8.386132in}}%
\pgfpathlineto{\pgfqpoint{2.135660in}{8.298396in}}%
\pgfpathlineto{\pgfqpoint{2.047924in}{8.298396in}}%
\pgfpathlineto{\pgfqpoint{2.047924in}{8.386132in}}%
\pgfusepath{stroke,fill}%
\end{pgfscope}%
\begin{pgfscope}%
\pgfpathrectangle{\pgfqpoint{0.380943in}{8.035189in}}{\pgfqpoint{4.650000in}{0.614151in}}%
\pgfusepath{clip}%
\pgfsetbuttcap%
\pgfsetroundjoin%
\definecolor{currentfill}{rgb}{0.978131,0.843783,0.675709}%
\pgfsetfillcolor{currentfill}%
\pgfsetlinewidth{0.250937pt}%
\definecolor{currentstroke}{rgb}{1.000000,1.000000,1.000000}%
\pgfsetstrokecolor{currentstroke}%
\pgfsetdash{}{0pt}%
\pgfpathmoveto{\pgfqpoint{2.135660in}{8.386132in}}%
\pgfpathlineto{\pgfqpoint{2.223396in}{8.386132in}}%
\pgfpathlineto{\pgfqpoint{2.223396in}{8.298396in}}%
\pgfpathlineto{\pgfqpoint{2.135660in}{8.298396in}}%
\pgfpathlineto{\pgfqpoint{2.135660in}{8.386132in}}%
\pgfusepath{stroke,fill}%
\end{pgfscope}%
\begin{pgfscope}%
\pgfpathrectangle{\pgfqpoint{0.380943in}{8.035189in}}{\pgfqpoint{4.650000in}{0.614151in}}%
\pgfusepath{clip}%
\pgfsetbuttcap%
\pgfsetroundjoin%
\definecolor{currentfill}{rgb}{1.000000,1.000000,0.861745}%
\pgfsetfillcolor{currentfill}%
\pgfsetlinewidth{0.250937pt}%
\definecolor{currentstroke}{rgb}{1.000000,1.000000,1.000000}%
\pgfsetstrokecolor{currentstroke}%
\pgfsetdash{}{0pt}%
\pgfpathmoveto{\pgfqpoint{2.223396in}{8.386132in}}%
\pgfpathlineto{\pgfqpoint{2.311132in}{8.386132in}}%
\pgfpathlineto{\pgfqpoint{2.311132in}{8.298396in}}%
\pgfpathlineto{\pgfqpoint{2.223396in}{8.298396in}}%
\pgfpathlineto{\pgfqpoint{2.223396in}{8.386132in}}%
\pgfusepath{stroke,fill}%
\end{pgfscope}%
\begin{pgfscope}%
\pgfpathrectangle{\pgfqpoint{0.380943in}{8.035189in}}{\pgfqpoint{4.650000in}{0.614151in}}%
\pgfusepath{clip}%
\pgfsetbuttcap%
\pgfsetroundjoin%
\definecolor{currentfill}{rgb}{0.992326,0.765229,0.614840}%
\pgfsetfillcolor{currentfill}%
\pgfsetlinewidth{0.250937pt}%
\definecolor{currentstroke}{rgb}{1.000000,1.000000,1.000000}%
\pgfsetstrokecolor{currentstroke}%
\pgfsetdash{}{0pt}%
\pgfpathmoveto{\pgfqpoint{2.311132in}{8.386132in}}%
\pgfpathlineto{\pgfqpoint{2.398868in}{8.386132in}}%
\pgfpathlineto{\pgfqpoint{2.398868in}{8.298396in}}%
\pgfpathlineto{\pgfqpoint{2.311132in}{8.298396in}}%
\pgfpathlineto{\pgfqpoint{2.311132in}{8.386132in}}%
\pgfusepath{stroke,fill}%
\end{pgfscope}%
\begin{pgfscope}%
\pgfpathrectangle{\pgfqpoint{0.380943in}{8.035189in}}{\pgfqpoint{4.650000in}{0.614151in}}%
\pgfusepath{clip}%
\pgfsetbuttcap%
\pgfsetroundjoin%
\definecolor{currentfill}{rgb}{0.978131,0.843783,0.675709}%
\pgfsetfillcolor{currentfill}%
\pgfsetlinewidth{0.250937pt}%
\definecolor{currentstroke}{rgb}{1.000000,1.000000,1.000000}%
\pgfsetstrokecolor{currentstroke}%
\pgfsetdash{}{0pt}%
\pgfpathmoveto{\pgfqpoint{2.398868in}{8.386132in}}%
\pgfpathlineto{\pgfqpoint{2.486603in}{8.386132in}}%
\pgfpathlineto{\pgfqpoint{2.486603in}{8.298396in}}%
\pgfpathlineto{\pgfqpoint{2.398868in}{8.298396in}}%
\pgfpathlineto{\pgfqpoint{2.398868in}{8.386132in}}%
\pgfusepath{stroke,fill}%
\end{pgfscope}%
\begin{pgfscope}%
\pgfpathrectangle{\pgfqpoint{0.380943in}{8.035189in}}{\pgfqpoint{4.650000in}{0.614151in}}%
\pgfusepath{clip}%
\pgfsetbuttcap%
\pgfsetroundjoin%
\definecolor{currentfill}{rgb}{0.970012,0.883276,0.699577}%
\pgfsetfillcolor{currentfill}%
\pgfsetlinewidth{0.250937pt}%
\definecolor{currentstroke}{rgb}{1.000000,1.000000,1.000000}%
\pgfsetstrokecolor{currentstroke}%
\pgfsetdash{}{0pt}%
\pgfpathmoveto{\pgfqpoint{2.486603in}{8.386132in}}%
\pgfpathlineto{\pgfqpoint{2.574339in}{8.386132in}}%
\pgfpathlineto{\pgfqpoint{2.574339in}{8.298396in}}%
\pgfpathlineto{\pgfqpoint{2.486603in}{8.298396in}}%
\pgfpathlineto{\pgfqpoint{2.486603in}{8.386132in}}%
\pgfusepath{stroke,fill}%
\end{pgfscope}%
\begin{pgfscope}%
\pgfpathrectangle{\pgfqpoint{0.380943in}{8.035189in}}{\pgfqpoint{4.650000in}{0.614151in}}%
\pgfusepath{clip}%
\pgfsetbuttcap%
\pgfsetroundjoin%
\definecolor{currentfill}{rgb}{0.970012,0.883276,0.699577}%
\pgfsetfillcolor{currentfill}%
\pgfsetlinewidth{0.250937pt}%
\definecolor{currentstroke}{rgb}{1.000000,1.000000,1.000000}%
\pgfsetstrokecolor{currentstroke}%
\pgfsetdash{}{0pt}%
\pgfpathmoveto{\pgfqpoint{2.574339in}{8.386132in}}%
\pgfpathlineto{\pgfqpoint{2.662075in}{8.386132in}}%
\pgfpathlineto{\pgfqpoint{2.662075in}{8.298396in}}%
\pgfpathlineto{\pgfqpoint{2.574339in}{8.298396in}}%
\pgfpathlineto{\pgfqpoint{2.574339in}{8.386132in}}%
\pgfusepath{stroke,fill}%
\end{pgfscope}%
\begin{pgfscope}%
\pgfpathrectangle{\pgfqpoint{0.380943in}{8.035189in}}{\pgfqpoint{4.650000in}{0.614151in}}%
\pgfusepath{clip}%
\pgfsetbuttcap%
\pgfsetroundjoin%
\definecolor{currentfill}{rgb}{0.961061,0.931672,0.728304}%
\pgfsetfillcolor{currentfill}%
\pgfsetlinewidth{0.250937pt}%
\definecolor{currentstroke}{rgb}{1.000000,1.000000,1.000000}%
\pgfsetstrokecolor{currentstroke}%
\pgfsetdash{}{0pt}%
\pgfpathmoveto{\pgfqpoint{2.662075in}{8.386132in}}%
\pgfpathlineto{\pgfqpoint{2.749811in}{8.386132in}}%
\pgfpathlineto{\pgfqpoint{2.749811in}{8.298396in}}%
\pgfpathlineto{\pgfqpoint{2.662075in}{8.298396in}}%
\pgfpathlineto{\pgfqpoint{2.662075in}{8.386132in}}%
\pgfusepath{stroke,fill}%
\end{pgfscope}%
\begin{pgfscope}%
\pgfpathrectangle{\pgfqpoint{0.380943in}{8.035189in}}{\pgfqpoint{4.650000in}{0.614151in}}%
\pgfusepath{clip}%
\pgfsetbuttcap%
\pgfsetroundjoin%
\definecolor{currentfill}{rgb}{0.985083,0.974641,0.792587}%
\pgfsetfillcolor{currentfill}%
\pgfsetlinewidth{0.250937pt}%
\definecolor{currentstroke}{rgb}{1.000000,1.000000,1.000000}%
\pgfsetstrokecolor{currentstroke}%
\pgfsetdash{}{0pt}%
\pgfpathmoveto{\pgfqpoint{2.749811in}{8.386132in}}%
\pgfpathlineto{\pgfqpoint{2.837547in}{8.386132in}}%
\pgfpathlineto{\pgfqpoint{2.837547in}{8.298396in}}%
\pgfpathlineto{\pgfqpoint{2.749811in}{8.298396in}}%
\pgfpathlineto{\pgfqpoint{2.749811in}{8.386132in}}%
\pgfusepath{stroke,fill}%
\end{pgfscope}%
\begin{pgfscope}%
\pgfpathrectangle{\pgfqpoint{0.380943in}{8.035189in}}{\pgfqpoint{4.650000in}{0.614151in}}%
\pgfusepath{clip}%
\pgfsetbuttcap%
\pgfsetroundjoin%
\definecolor{currentfill}{rgb}{0.978131,0.843783,0.675709}%
\pgfsetfillcolor{currentfill}%
\pgfsetlinewidth{0.250937pt}%
\definecolor{currentstroke}{rgb}{1.000000,1.000000,1.000000}%
\pgfsetstrokecolor{currentstroke}%
\pgfsetdash{}{0pt}%
\pgfpathmoveto{\pgfqpoint{2.837547in}{8.386132in}}%
\pgfpathlineto{\pgfqpoint{2.925283in}{8.386132in}}%
\pgfpathlineto{\pgfqpoint{2.925283in}{8.298396in}}%
\pgfpathlineto{\pgfqpoint{2.837547in}{8.298396in}}%
\pgfpathlineto{\pgfqpoint{2.837547in}{8.386132in}}%
\pgfusepath{stroke,fill}%
\end{pgfscope}%
\begin{pgfscope}%
\pgfpathrectangle{\pgfqpoint{0.380943in}{8.035189in}}{\pgfqpoint{4.650000in}{0.614151in}}%
\pgfusepath{clip}%
\pgfsetbuttcap%
\pgfsetroundjoin%
\definecolor{currentfill}{rgb}{0.978131,0.843783,0.675709}%
\pgfsetfillcolor{currentfill}%
\pgfsetlinewidth{0.250937pt}%
\definecolor{currentstroke}{rgb}{1.000000,1.000000,1.000000}%
\pgfsetstrokecolor{currentstroke}%
\pgfsetdash{}{0pt}%
\pgfpathmoveto{\pgfqpoint{2.925283in}{8.386132in}}%
\pgfpathlineto{\pgfqpoint{3.013019in}{8.386132in}}%
\pgfpathlineto{\pgfqpoint{3.013019in}{8.298396in}}%
\pgfpathlineto{\pgfqpoint{2.925283in}{8.298396in}}%
\pgfpathlineto{\pgfqpoint{2.925283in}{8.386132in}}%
\pgfusepath{stroke,fill}%
\end{pgfscope}%
\begin{pgfscope}%
\pgfpathrectangle{\pgfqpoint{0.380943in}{8.035189in}}{\pgfqpoint{4.650000in}{0.614151in}}%
\pgfusepath{clip}%
\pgfsetbuttcap%
\pgfsetroundjoin%
\definecolor{currentfill}{rgb}{0.986251,0.808597,0.643230}%
\pgfsetfillcolor{currentfill}%
\pgfsetlinewidth{0.250937pt}%
\definecolor{currentstroke}{rgb}{1.000000,1.000000,1.000000}%
\pgfsetstrokecolor{currentstroke}%
\pgfsetdash{}{0pt}%
\pgfpathmoveto{\pgfqpoint{3.013019in}{8.386132in}}%
\pgfpathlineto{\pgfqpoint{3.100754in}{8.386132in}}%
\pgfpathlineto{\pgfqpoint{3.100754in}{8.298396in}}%
\pgfpathlineto{\pgfqpoint{3.013019in}{8.298396in}}%
\pgfpathlineto{\pgfqpoint{3.013019in}{8.386132in}}%
\pgfusepath{stroke,fill}%
\end{pgfscope}%
\begin{pgfscope}%
\pgfpathrectangle{\pgfqpoint{0.380943in}{8.035189in}}{\pgfqpoint{4.650000in}{0.614151in}}%
\pgfusepath{clip}%
\pgfsetbuttcap%
\pgfsetroundjoin%
\definecolor{currentfill}{rgb}{0.961061,0.931672,0.728304}%
\pgfsetfillcolor{currentfill}%
\pgfsetlinewidth{0.250937pt}%
\definecolor{currentstroke}{rgb}{1.000000,1.000000,1.000000}%
\pgfsetstrokecolor{currentstroke}%
\pgfsetdash{}{0pt}%
\pgfpathmoveto{\pgfqpoint{3.100754in}{8.386132in}}%
\pgfpathlineto{\pgfqpoint{3.188490in}{8.386132in}}%
\pgfpathlineto{\pgfqpoint{3.188490in}{8.298396in}}%
\pgfpathlineto{\pgfqpoint{3.100754in}{8.298396in}}%
\pgfpathlineto{\pgfqpoint{3.100754in}{8.386132in}}%
\pgfusepath{stroke,fill}%
\end{pgfscope}%
\begin{pgfscope}%
\pgfpathrectangle{\pgfqpoint{0.380943in}{8.035189in}}{\pgfqpoint{4.650000in}{0.614151in}}%
\pgfusepath{clip}%
\pgfsetbuttcap%
\pgfsetroundjoin%
\definecolor{currentfill}{rgb}{0.963768,0.915433,0.717478}%
\pgfsetfillcolor{currentfill}%
\pgfsetlinewidth{0.250937pt}%
\definecolor{currentstroke}{rgb}{1.000000,1.000000,1.000000}%
\pgfsetstrokecolor{currentstroke}%
\pgfsetdash{}{0pt}%
\pgfpathmoveto{\pgfqpoint{3.188490in}{8.386132in}}%
\pgfpathlineto{\pgfqpoint{3.276226in}{8.386132in}}%
\pgfpathlineto{\pgfqpoint{3.276226in}{8.298396in}}%
\pgfpathlineto{\pgfqpoint{3.188490in}{8.298396in}}%
\pgfpathlineto{\pgfqpoint{3.188490in}{8.386132in}}%
\pgfusepath{stroke,fill}%
\end{pgfscope}%
\begin{pgfscope}%
\pgfpathrectangle{\pgfqpoint{0.380943in}{8.035189in}}{\pgfqpoint{4.650000in}{0.614151in}}%
\pgfusepath{clip}%
\pgfsetbuttcap%
\pgfsetroundjoin%
\definecolor{currentfill}{rgb}{0.986251,0.808597,0.643230}%
\pgfsetfillcolor{currentfill}%
\pgfsetlinewidth{0.250937pt}%
\definecolor{currentstroke}{rgb}{1.000000,1.000000,1.000000}%
\pgfsetstrokecolor{currentstroke}%
\pgfsetdash{}{0pt}%
\pgfpathmoveto{\pgfqpoint{3.276226in}{8.386132in}}%
\pgfpathlineto{\pgfqpoint{3.363962in}{8.386132in}}%
\pgfpathlineto{\pgfqpoint{3.363962in}{8.298396in}}%
\pgfpathlineto{\pgfqpoint{3.276226in}{8.298396in}}%
\pgfpathlineto{\pgfqpoint{3.276226in}{8.386132in}}%
\pgfusepath{stroke,fill}%
\end{pgfscope}%
\begin{pgfscope}%
\pgfpathrectangle{\pgfqpoint{0.380943in}{8.035189in}}{\pgfqpoint{4.650000in}{0.614151in}}%
\pgfusepath{clip}%
\pgfsetbuttcap%
\pgfsetroundjoin%
\definecolor{currentfill}{rgb}{0.999616,0.641369,0.546559}%
\pgfsetfillcolor{currentfill}%
\pgfsetlinewidth{0.250937pt}%
\definecolor{currentstroke}{rgb}{1.000000,1.000000,1.000000}%
\pgfsetstrokecolor{currentstroke}%
\pgfsetdash{}{0pt}%
\pgfpathmoveto{\pgfqpoint{3.363962in}{8.386132in}}%
\pgfpathlineto{\pgfqpoint{3.451698in}{8.386132in}}%
\pgfpathlineto{\pgfqpoint{3.451698in}{8.298396in}}%
\pgfpathlineto{\pgfqpoint{3.363962in}{8.298396in}}%
\pgfpathlineto{\pgfqpoint{3.363962in}{8.386132in}}%
\pgfusepath{stroke,fill}%
\end{pgfscope}%
\begin{pgfscope}%
\pgfpathrectangle{\pgfqpoint{0.380943in}{8.035189in}}{\pgfqpoint{4.650000in}{0.614151in}}%
\pgfusepath{clip}%
\pgfsetbuttcap%
\pgfsetroundjoin%
\definecolor{currentfill}{rgb}{0.978131,0.843783,0.675709}%
\pgfsetfillcolor{currentfill}%
\pgfsetlinewidth{0.250937pt}%
\definecolor{currentstroke}{rgb}{1.000000,1.000000,1.000000}%
\pgfsetstrokecolor{currentstroke}%
\pgfsetdash{}{0pt}%
\pgfpathmoveto{\pgfqpoint{3.451698in}{8.386132in}}%
\pgfpathlineto{\pgfqpoint{3.539434in}{8.386132in}}%
\pgfpathlineto{\pgfqpoint{3.539434in}{8.298396in}}%
\pgfpathlineto{\pgfqpoint{3.451698in}{8.298396in}}%
\pgfpathlineto{\pgfqpoint{3.451698in}{8.386132in}}%
\pgfusepath{stroke,fill}%
\end{pgfscope}%
\begin{pgfscope}%
\pgfpathrectangle{\pgfqpoint{0.380943in}{8.035189in}}{\pgfqpoint{4.650000in}{0.614151in}}%
\pgfusepath{clip}%
\pgfsetbuttcap%
\pgfsetroundjoin%
\definecolor{currentfill}{rgb}{0.970012,0.883276,0.699577}%
\pgfsetfillcolor{currentfill}%
\pgfsetlinewidth{0.250937pt}%
\definecolor{currentstroke}{rgb}{1.000000,1.000000,1.000000}%
\pgfsetstrokecolor{currentstroke}%
\pgfsetdash{}{0pt}%
\pgfpathmoveto{\pgfqpoint{3.539434in}{8.386132in}}%
\pgfpathlineto{\pgfqpoint{3.627169in}{8.386132in}}%
\pgfpathlineto{\pgfqpoint{3.627169in}{8.298396in}}%
\pgfpathlineto{\pgfqpoint{3.539434in}{8.298396in}}%
\pgfpathlineto{\pgfqpoint{3.539434in}{8.386132in}}%
\pgfusepath{stroke,fill}%
\end{pgfscope}%
\begin{pgfscope}%
\pgfpathrectangle{\pgfqpoint{0.380943in}{8.035189in}}{\pgfqpoint{4.650000in}{0.614151in}}%
\pgfusepath{clip}%
\pgfsetbuttcap%
\pgfsetroundjoin%
\definecolor{currentfill}{rgb}{0.986251,0.808597,0.643230}%
\pgfsetfillcolor{currentfill}%
\pgfsetlinewidth{0.250937pt}%
\definecolor{currentstroke}{rgb}{1.000000,1.000000,1.000000}%
\pgfsetstrokecolor{currentstroke}%
\pgfsetdash{}{0pt}%
\pgfpathmoveto{\pgfqpoint{3.627169in}{8.386132in}}%
\pgfpathlineto{\pgfqpoint{3.714905in}{8.386132in}}%
\pgfpathlineto{\pgfqpoint{3.714905in}{8.298396in}}%
\pgfpathlineto{\pgfqpoint{3.627169in}{8.298396in}}%
\pgfpathlineto{\pgfqpoint{3.627169in}{8.386132in}}%
\pgfusepath{stroke,fill}%
\end{pgfscope}%
\begin{pgfscope}%
\pgfpathrectangle{\pgfqpoint{0.380943in}{8.035189in}}{\pgfqpoint{4.650000in}{0.614151in}}%
\pgfusepath{clip}%
\pgfsetbuttcap%
\pgfsetroundjoin%
\definecolor{currentfill}{rgb}{0.961061,0.931672,0.728304}%
\pgfsetfillcolor{currentfill}%
\pgfsetlinewidth{0.250937pt}%
\definecolor{currentstroke}{rgb}{1.000000,1.000000,1.000000}%
\pgfsetstrokecolor{currentstroke}%
\pgfsetdash{}{0pt}%
\pgfpathmoveto{\pgfqpoint{3.714905in}{8.386132in}}%
\pgfpathlineto{\pgfqpoint{3.802641in}{8.386132in}}%
\pgfpathlineto{\pgfqpoint{3.802641in}{8.298396in}}%
\pgfpathlineto{\pgfqpoint{3.714905in}{8.298396in}}%
\pgfpathlineto{\pgfqpoint{3.714905in}{8.386132in}}%
\pgfusepath{stroke,fill}%
\end{pgfscope}%
\begin{pgfscope}%
\pgfpathrectangle{\pgfqpoint{0.380943in}{8.035189in}}{\pgfqpoint{4.650000in}{0.614151in}}%
\pgfusepath{clip}%
\pgfsetbuttcap%
\pgfsetroundjoin%
\definecolor{currentfill}{rgb}{1.000000,0.584929,0.522599}%
\pgfsetfillcolor{currentfill}%
\pgfsetlinewidth{0.250937pt}%
\definecolor{currentstroke}{rgb}{1.000000,1.000000,1.000000}%
\pgfsetstrokecolor{currentstroke}%
\pgfsetdash{}{0pt}%
\pgfpathmoveto{\pgfqpoint{3.802641in}{8.386132in}}%
\pgfpathlineto{\pgfqpoint{3.890377in}{8.386132in}}%
\pgfpathlineto{\pgfqpoint{3.890377in}{8.298396in}}%
\pgfpathlineto{\pgfqpoint{3.802641in}{8.298396in}}%
\pgfpathlineto{\pgfqpoint{3.802641in}{8.386132in}}%
\pgfusepath{stroke,fill}%
\end{pgfscope}%
\begin{pgfscope}%
\pgfpathrectangle{\pgfqpoint{0.380943in}{8.035189in}}{\pgfqpoint{4.650000in}{0.614151in}}%
\pgfusepath{clip}%
\pgfsetbuttcap%
\pgfsetroundjoin%
\definecolor{currentfill}{rgb}{0.992326,0.765229,0.614840}%
\pgfsetfillcolor{currentfill}%
\pgfsetlinewidth{0.250937pt}%
\definecolor{currentstroke}{rgb}{1.000000,1.000000,1.000000}%
\pgfsetstrokecolor{currentstroke}%
\pgfsetdash{}{0pt}%
\pgfpathmoveto{\pgfqpoint{3.890377in}{8.386132in}}%
\pgfpathlineto{\pgfqpoint{3.978113in}{8.386132in}}%
\pgfpathlineto{\pgfqpoint{3.978113in}{8.298396in}}%
\pgfpathlineto{\pgfqpoint{3.890377in}{8.298396in}}%
\pgfpathlineto{\pgfqpoint{3.890377in}{8.386132in}}%
\pgfusepath{stroke,fill}%
\end{pgfscope}%
\begin{pgfscope}%
\pgfpathrectangle{\pgfqpoint{0.380943in}{8.035189in}}{\pgfqpoint{4.650000in}{0.614151in}}%
\pgfusepath{clip}%
\pgfsetbuttcap%
\pgfsetroundjoin%
\definecolor{currentfill}{rgb}{0.986251,0.808597,0.643230}%
\pgfsetfillcolor{currentfill}%
\pgfsetlinewidth{0.250937pt}%
\definecolor{currentstroke}{rgb}{1.000000,1.000000,1.000000}%
\pgfsetstrokecolor{currentstroke}%
\pgfsetdash{}{0pt}%
\pgfpathmoveto{\pgfqpoint{3.978113in}{8.386132in}}%
\pgfpathlineto{\pgfqpoint{4.065849in}{8.386132in}}%
\pgfpathlineto{\pgfqpoint{4.065849in}{8.298396in}}%
\pgfpathlineto{\pgfqpoint{3.978113in}{8.298396in}}%
\pgfpathlineto{\pgfqpoint{3.978113in}{8.386132in}}%
\pgfusepath{stroke,fill}%
\end{pgfscope}%
\begin{pgfscope}%
\pgfpathrectangle{\pgfqpoint{0.380943in}{8.035189in}}{\pgfqpoint{4.650000in}{0.614151in}}%
\pgfusepath{clip}%
\pgfsetbuttcap%
\pgfsetroundjoin%
\definecolor{currentfill}{rgb}{1.000000,0.480477,0.479293}%
\pgfsetfillcolor{currentfill}%
\pgfsetlinewidth{0.250937pt}%
\definecolor{currentstroke}{rgb}{1.000000,1.000000,1.000000}%
\pgfsetstrokecolor{currentstroke}%
\pgfsetdash{}{0pt}%
\pgfpathmoveto{\pgfqpoint{4.065849in}{8.386132in}}%
\pgfpathlineto{\pgfqpoint{4.153585in}{8.386132in}}%
\pgfpathlineto{\pgfqpoint{4.153585in}{8.298396in}}%
\pgfpathlineto{\pgfqpoint{4.065849in}{8.298396in}}%
\pgfpathlineto{\pgfqpoint{4.065849in}{8.386132in}}%
\pgfusepath{stroke,fill}%
\end{pgfscope}%
\begin{pgfscope}%
\pgfpathrectangle{\pgfqpoint{0.380943in}{8.035189in}}{\pgfqpoint{4.650000in}{0.614151in}}%
\pgfusepath{clip}%
\pgfsetbuttcap%
\pgfsetroundjoin%
\definecolor{currentfill}{rgb}{0.992326,0.765229,0.614840}%
\pgfsetfillcolor{currentfill}%
\pgfsetlinewidth{0.250937pt}%
\definecolor{currentstroke}{rgb}{1.000000,1.000000,1.000000}%
\pgfsetstrokecolor{currentstroke}%
\pgfsetdash{}{0pt}%
\pgfpathmoveto{\pgfqpoint{4.153585in}{8.386132in}}%
\pgfpathlineto{\pgfqpoint{4.241320in}{8.386132in}}%
\pgfpathlineto{\pgfqpoint{4.241320in}{8.298396in}}%
\pgfpathlineto{\pgfqpoint{4.153585in}{8.298396in}}%
\pgfpathlineto{\pgfqpoint{4.153585in}{8.386132in}}%
\pgfusepath{stroke,fill}%
\end{pgfscope}%
\begin{pgfscope}%
\pgfpathrectangle{\pgfqpoint{0.380943in}{8.035189in}}{\pgfqpoint{4.650000in}{0.614151in}}%
\pgfusepath{clip}%
\pgfsetbuttcap%
\pgfsetroundjoin%
\definecolor{currentfill}{rgb}{0.986251,0.808597,0.643230}%
\pgfsetfillcolor{currentfill}%
\pgfsetlinewidth{0.250937pt}%
\definecolor{currentstroke}{rgb}{1.000000,1.000000,1.000000}%
\pgfsetstrokecolor{currentstroke}%
\pgfsetdash{}{0pt}%
\pgfpathmoveto{\pgfqpoint{4.241320in}{8.386132in}}%
\pgfpathlineto{\pgfqpoint{4.329056in}{8.386132in}}%
\pgfpathlineto{\pgfqpoint{4.329056in}{8.298396in}}%
\pgfpathlineto{\pgfqpoint{4.241320in}{8.298396in}}%
\pgfpathlineto{\pgfqpoint{4.241320in}{8.386132in}}%
\pgfusepath{stroke,fill}%
\end{pgfscope}%
\begin{pgfscope}%
\pgfpathrectangle{\pgfqpoint{0.380943in}{8.035189in}}{\pgfqpoint{4.650000in}{0.614151in}}%
\pgfusepath{clip}%
\pgfsetbuttcap%
\pgfsetroundjoin%
\definecolor{currentfill}{rgb}{0.961061,0.931672,0.728304}%
\pgfsetfillcolor{currentfill}%
\pgfsetlinewidth{0.250937pt}%
\definecolor{currentstroke}{rgb}{1.000000,1.000000,1.000000}%
\pgfsetstrokecolor{currentstroke}%
\pgfsetdash{}{0pt}%
\pgfpathmoveto{\pgfqpoint{4.329056in}{8.386132in}}%
\pgfpathlineto{\pgfqpoint{4.416792in}{8.386132in}}%
\pgfpathlineto{\pgfqpoint{4.416792in}{8.298396in}}%
\pgfpathlineto{\pgfqpoint{4.329056in}{8.298396in}}%
\pgfpathlineto{\pgfqpoint{4.329056in}{8.386132in}}%
\pgfusepath{stroke,fill}%
\end{pgfscope}%
\begin{pgfscope}%
\pgfpathrectangle{\pgfqpoint{0.380943in}{8.035189in}}{\pgfqpoint{4.650000in}{0.614151in}}%
\pgfusepath{clip}%
\pgfsetbuttcap%
\pgfsetroundjoin%
\definecolor{currentfill}{rgb}{0.999616,0.641369,0.546559}%
\pgfsetfillcolor{currentfill}%
\pgfsetlinewidth{0.250937pt}%
\definecolor{currentstroke}{rgb}{1.000000,1.000000,1.000000}%
\pgfsetstrokecolor{currentstroke}%
\pgfsetdash{}{0pt}%
\pgfpathmoveto{\pgfqpoint{4.416792in}{8.386132in}}%
\pgfpathlineto{\pgfqpoint{4.504528in}{8.386132in}}%
\pgfpathlineto{\pgfqpoint{4.504528in}{8.298396in}}%
\pgfpathlineto{\pgfqpoint{4.416792in}{8.298396in}}%
\pgfpathlineto{\pgfqpoint{4.416792in}{8.386132in}}%
\pgfusepath{stroke,fill}%
\end{pgfscope}%
\begin{pgfscope}%
\pgfpathrectangle{\pgfqpoint{0.380943in}{8.035189in}}{\pgfqpoint{4.650000in}{0.614151in}}%
\pgfusepath{clip}%
\pgfsetbuttcap%
\pgfsetroundjoin%
\definecolor{currentfill}{rgb}{0.986251,0.808597,0.643230}%
\pgfsetfillcolor{currentfill}%
\pgfsetlinewidth{0.250937pt}%
\definecolor{currentstroke}{rgb}{1.000000,1.000000,1.000000}%
\pgfsetstrokecolor{currentstroke}%
\pgfsetdash{}{0pt}%
\pgfpathmoveto{\pgfqpoint{4.504528in}{8.386132in}}%
\pgfpathlineto{\pgfqpoint{4.592264in}{8.386132in}}%
\pgfpathlineto{\pgfqpoint{4.592264in}{8.298396in}}%
\pgfpathlineto{\pgfqpoint{4.504528in}{8.298396in}}%
\pgfpathlineto{\pgfqpoint{4.504528in}{8.386132in}}%
\pgfusepath{stroke,fill}%
\end{pgfscope}%
\begin{pgfscope}%
\pgfpathrectangle{\pgfqpoint{0.380943in}{8.035189in}}{\pgfqpoint{4.650000in}{0.614151in}}%
\pgfusepath{clip}%
\pgfsetbuttcap%
\pgfsetroundjoin%
\definecolor{currentfill}{rgb}{0.935025,0.413456,0.413456}%
\pgfsetfillcolor{currentfill}%
\pgfsetlinewidth{0.250937pt}%
\definecolor{currentstroke}{rgb}{1.000000,1.000000,1.000000}%
\pgfsetstrokecolor{currentstroke}%
\pgfsetdash{}{0pt}%
\pgfpathmoveto{\pgfqpoint{4.592264in}{8.386132in}}%
\pgfpathlineto{\pgfqpoint{4.680000in}{8.386132in}}%
\pgfpathlineto{\pgfqpoint{4.680000in}{8.298396in}}%
\pgfpathlineto{\pgfqpoint{4.592264in}{8.298396in}}%
\pgfpathlineto{\pgfqpoint{4.592264in}{8.386132in}}%
\pgfusepath{stroke,fill}%
\end{pgfscope}%
\begin{pgfscope}%
\pgfpathrectangle{\pgfqpoint{0.380943in}{8.035189in}}{\pgfqpoint{4.650000in}{0.614151in}}%
\pgfusepath{clip}%
\pgfsetbuttcap%
\pgfsetroundjoin%
\definecolor{currentfill}{rgb}{0.999616,0.641369,0.546559}%
\pgfsetfillcolor{currentfill}%
\pgfsetlinewidth{0.250937pt}%
\definecolor{currentstroke}{rgb}{1.000000,1.000000,1.000000}%
\pgfsetstrokecolor{currentstroke}%
\pgfsetdash{}{0pt}%
\pgfpathmoveto{\pgfqpoint{4.680000in}{8.386132in}}%
\pgfpathlineto{\pgfqpoint{4.767736in}{8.386132in}}%
\pgfpathlineto{\pgfqpoint{4.767736in}{8.298396in}}%
\pgfpathlineto{\pgfqpoint{4.680000in}{8.298396in}}%
\pgfpathlineto{\pgfqpoint{4.680000in}{8.386132in}}%
\pgfusepath{stroke,fill}%
\end{pgfscope}%
\begin{pgfscope}%
\pgfpathrectangle{\pgfqpoint{0.380943in}{8.035189in}}{\pgfqpoint{4.650000in}{0.614151in}}%
\pgfusepath{clip}%
\pgfsetbuttcap%
\pgfsetroundjoin%
\definecolor{currentfill}{rgb}{0.996909,0.711742,0.584452}%
\pgfsetfillcolor{currentfill}%
\pgfsetlinewidth{0.250937pt}%
\definecolor{currentstroke}{rgb}{1.000000,1.000000,1.000000}%
\pgfsetstrokecolor{currentstroke}%
\pgfsetdash{}{0pt}%
\pgfpathmoveto{\pgfqpoint{4.767736in}{8.386132in}}%
\pgfpathlineto{\pgfqpoint{4.855471in}{8.386132in}}%
\pgfpathlineto{\pgfqpoint{4.855471in}{8.298396in}}%
\pgfpathlineto{\pgfqpoint{4.767736in}{8.298396in}}%
\pgfpathlineto{\pgfqpoint{4.767736in}{8.386132in}}%
\pgfusepath{stroke,fill}%
\end{pgfscope}%
\begin{pgfscope}%
\pgfpathrectangle{\pgfqpoint{0.380943in}{8.035189in}}{\pgfqpoint{4.650000in}{0.614151in}}%
\pgfusepath{clip}%
\pgfsetbuttcap%
\pgfsetroundjoin%
\definecolor{currentfill}{rgb}{0.978131,0.843783,0.675709}%
\pgfsetfillcolor{currentfill}%
\pgfsetlinewidth{0.250937pt}%
\definecolor{currentstroke}{rgb}{1.000000,1.000000,1.000000}%
\pgfsetstrokecolor{currentstroke}%
\pgfsetdash{}{0pt}%
\pgfpathmoveto{\pgfqpoint{4.855471in}{8.386132in}}%
\pgfpathlineto{\pgfqpoint{4.943207in}{8.386132in}}%
\pgfpathlineto{\pgfqpoint{4.943207in}{8.298396in}}%
\pgfpathlineto{\pgfqpoint{4.855471in}{8.298396in}}%
\pgfpathlineto{\pgfqpoint{4.855471in}{8.386132in}}%
\pgfusepath{stroke,fill}%
\end{pgfscope}%
\begin{pgfscope}%
\pgfpathrectangle{\pgfqpoint{0.380943in}{8.035189in}}{\pgfqpoint{4.650000in}{0.614151in}}%
\pgfusepath{clip}%
\pgfsetbuttcap%
\pgfsetroundjoin%
\definecolor{currentfill}{rgb}{0.970012,0.883276,0.699577}%
\pgfsetfillcolor{currentfill}%
\pgfsetlinewidth{0.250937pt}%
\definecolor{currentstroke}{rgb}{1.000000,1.000000,1.000000}%
\pgfsetstrokecolor{currentstroke}%
\pgfsetdash{}{0pt}%
\pgfpathmoveto{\pgfqpoint{4.943207in}{8.386132in}}%
\pgfpathlineto{\pgfqpoint{5.030943in}{8.386132in}}%
\pgfpathlineto{\pgfqpoint{5.030943in}{8.298396in}}%
\pgfpathlineto{\pgfqpoint{4.943207in}{8.298396in}}%
\pgfpathlineto{\pgfqpoint{4.943207in}{8.386132in}}%
\pgfusepath{stroke,fill}%
\end{pgfscope}%
\begin{pgfscope}%
\pgfpathrectangle{\pgfqpoint{0.380943in}{8.035189in}}{\pgfqpoint{4.650000in}{0.614151in}}%
\pgfusepath{clip}%
\pgfsetbuttcap%
\pgfsetroundjoin%
\pgfsetlinewidth{0.250937pt}%
\definecolor{currentstroke}{rgb}{1.000000,1.000000,1.000000}%
\pgfsetstrokecolor{currentstroke}%
\pgfsetdash{}{0pt}%
\pgfpathmoveto{\pgfqpoint{0.380943in}{8.298396in}}%
\pgfpathlineto{\pgfqpoint{0.468679in}{8.298396in}}%
\pgfpathlineto{\pgfqpoint{0.468679in}{8.210661in}}%
\pgfpathlineto{\pgfqpoint{0.380943in}{8.210661in}}%
\pgfpathlineto{\pgfqpoint{0.380943in}{8.298396in}}%
\pgfusepath{stroke}%
\end{pgfscope}%
\begin{pgfscope}%
\pgfpathrectangle{\pgfqpoint{0.380943in}{8.035189in}}{\pgfqpoint{4.650000in}{0.614151in}}%
\pgfusepath{clip}%
\pgfsetbuttcap%
\pgfsetroundjoin%
\definecolor{currentfill}{rgb}{0.999616,0.641369,0.546559}%
\pgfsetfillcolor{currentfill}%
\pgfsetlinewidth{0.250937pt}%
\definecolor{currentstroke}{rgb}{1.000000,1.000000,1.000000}%
\pgfsetstrokecolor{currentstroke}%
\pgfsetdash{}{0pt}%
\pgfpathmoveto{\pgfqpoint{0.468679in}{8.298396in}}%
\pgfpathlineto{\pgfqpoint{0.556415in}{8.298396in}}%
\pgfpathlineto{\pgfqpoint{0.556415in}{8.210661in}}%
\pgfpathlineto{\pgfqpoint{0.468679in}{8.210661in}}%
\pgfpathlineto{\pgfqpoint{0.468679in}{8.298396in}}%
\pgfusepath{stroke,fill}%
\end{pgfscope}%
\begin{pgfscope}%
\pgfpathrectangle{\pgfqpoint{0.380943in}{8.035189in}}{\pgfqpoint{4.650000in}{0.614151in}}%
\pgfusepath{clip}%
\pgfsetbuttcap%
\pgfsetroundjoin%
\definecolor{currentfill}{rgb}{0.986251,0.808597,0.643230}%
\pgfsetfillcolor{currentfill}%
\pgfsetlinewidth{0.250937pt}%
\definecolor{currentstroke}{rgb}{1.000000,1.000000,1.000000}%
\pgfsetstrokecolor{currentstroke}%
\pgfsetdash{}{0pt}%
\pgfpathmoveto{\pgfqpoint{0.556415in}{8.298396in}}%
\pgfpathlineto{\pgfqpoint{0.644151in}{8.298396in}}%
\pgfpathlineto{\pgfqpoint{0.644151in}{8.210661in}}%
\pgfpathlineto{\pgfqpoint{0.556415in}{8.210661in}}%
\pgfpathlineto{\pgfqpoint{0.556415in}{8.298396in}}%
\pgfusepath{stroke,fill}%
\end{pgfscope}%
\begin{pgfscope}%
\pgfpathrectangle{\pgfqpoint{0.380943in}{8.035189in}}{\pgfqpoint{4.650000in}{0.614151in}}%
\pgfusepath{clip}%
\pgfsetbuttcap%
\pgfsetroundjoin%
\definecolor{currentfill}{rgb}{0.970012,0.883276,0.699577}%
\pgfsetfillcolor{currentfill}%
\pgfsetlinewidth{0.250937pt}%
\definecolor{currentstroke}{rgb}{1.000000,1.000000,1.000000}%
\pgfsetstrokecolor{currentstroke}%
\pgfsetdash{}{0pt}%
\pgfpathmoveto{\pgfqpoint{0.644151in}{8.298396in}}%
\pgfpathlineto{\pgfqpoint{0.731886in}{8.298396in}}%
\pgfpathlineto{\pgfqpoint{0.731886in}{8.210661in}}%
\pgfpathlineto{\pgfqpoint{0.644151in}{8.210661in}}%
\pgfpathlineto{\pgfqpoint{0.644151in}{8.298396in}}%
\pgfusepath{stroke,fill}%
\end{pgfscope}%
\begin{pgfscope}%
\pgfpathrectangle{\pgfqpoint{0.380943in}{8.035189in}}{\pgfqpoint{4.650000in}{0.614151in}}%
\pgfusepath{clip}%
\pgfsetbuttcap%
\pgfsetroundjoin%
\definecolor{currentfill}{rgb}{0.986251,0.808597,0.643230}%
\pgfsetfillcolor{currentfill}%
\pgfsetlinewidth{0.250937pt}%
\definecolor{currentstroke}{rgb}{1.000000,1.000000,1.000000}%
\pgfsetstrokecolor{currentstroke}%
\pgfsetdash{}{0pt}%
\pgfpathmoveto{\pgfqpoint{0.731886in}{8.298396in}}%
\pgfpathlineto{\pgfqpoint{0.819622in}{8.298396in}}%
\pgfpathlineto{\pgfqpoint{0.819622in}{8.210661in}}%
\pgfpathlineto{\pgfqpoint{0.731886in}{8.210661in}}%
\pgfpathlineto{\pgfqpoint{0.731886in}{8.298396in}}%
\pgfusepath{stroke,fill}%
\end{pgfscope}%
\begin{pgfscope}%
\pgfpathrectangle{\pgfqpoint{0.380943in}{8.035189in}}{\pgfqpoint{4.650000in}{0.614151in}}%
\pgfusepath{clip}%
\pgfsetbuttcap%
\pgfsetroundjoin%
\definecolor{currentfill}{rgb}{0.978131,0.843783,0.675709}%
\pgfsetfillcolor{currentfill}%
\pgfsetlinewidth{0.250937pt}%
\definecolor{currentstroke}{rgb}{1.000000,1.000000,1.000000}%
\pgfsetstrokecolor{currentstroke}%
\pgfsetdash{}{0pt}%
\pgfpathmoveto{\pgfqpoint{0.819622in}{8.298396in}}%
\pgfpathlineto{\pgfqpoint{0.907358in}{8.298396in}}%
\pgfpathlineto{\pgfqpoint{0.907358in}{8.210661in}}%
\pgfpathlineto{\pgfqpoint{0.819622in}{8.210661in}}%
\pgfpathlineto{\pgfqpoint{0.819622in}{8.298396in}}%
\pgfusepath{stroke,fill}%
\end{pgfscope}%
\begin{pgfscope}%
\pgfpathrectangle{\pgfqpoint{0.380943in}{8.035189in}}{\pgfqpoint{4.650000in}{0.614151in}}%
\pgfusepath{clip}%
\pgfsetbuttcap%
\pgfsetroundjoin%
\definecolor{currentfill}{rgb}{0.992326,0.765229,0.614840}%
\pgfsetfillcolor{currentfill}%
\pgfsetlinewidth{0.250937pt}%
\definecolor{currentstroke}{rgb}{1.000000,1.000000,1.000000}%
\pgfsetstrokecolor{currentstroke}%
\pgfsetdash{}{0pt}%
\pgfpathmoveto{\pgfqpoint{0.907358in}{8.298396in}}%
\pgfpathlineto{\pgfqpoint{0.995094in}{8.298396in}}%
\pgfpathlineto{\pgfqpoint{0.995094in}{8.210661in}}%
\pgfpathlineto{\pgfqpoint{0.907358in}{8.210661in}}%
\pgfpathlineto{\pgfqpoint{0.907358in}{8.298396in}}%
\pgfusepath{stroke,fill}%
\end{pgfscope}%
\begin{pgfscope}%
\pgfpathrectangle{\pgfqpoint{0.380943in}{8.035189in}}{\pgfqpoint{4.650000in}{0.614151in}}%
\pgfusepath{clip}%
\pgfsetbuttcap%
\pgfsetroundjoin%
\definecolor{currentfill}{rgb}{0.999616,0.641369,0.546559}%
\pgfsetfillcolor{currentfill}%
\pgfsetlinewidth{0.250937pt}%
\definecolor{currentstroke}{rgb}{1.000000,1.000000,1.000000}%
\pgfsetstrokecolor{currentstroke}%
\pgfsetdash{}{0pt}%
\pgfpathmoveto{\pgfqpoint{0.995094in}{8.298396in}}%
\pgfpathlineto{\pgfqpoint{1.082830in}{8.298396in}}%
\pgfpathlineto{\pgfqpoint{1.082830in}{8.210661in}}%
\pgfpathlineto{\pgfqpoint{0.995094in}{8.210661in}}%
\pgfpathlineto{\pgfqpoint{0.995094in}{8.298396in}}%
\pgfusepath{stroke,fill}%
\end{pgfscope}%
\begin{pgfscope}%
\pgfpathrectangle{\pgfqpoint{0.380943in}{8.035189in}}{\pgfqpoint{4.650000in}{0.614151in}}%
\pgfusepath{clip}%
\pgfsetbuttcap%
\pgfsetroundjoin%
\definecolor{currentfill}{rgb}{0.992326,0.765229,0.614840}%
\pgfsetfillcolor{currentfill}%
\pgfsetlinewidth{0.250937pt}%
\definecolor{currentstroke}{rgb}{1.000000,1.000000,1.000000}%
\pgfsetstrokecolor{currentstroke}%
\pgfsetdash{}{0pt}%
\pgfpathmoveto{\pgfqpoint{1.082830in}{8.298396in}}%
\pgfpathlineto{\pgfqpoint{1.170566in}{8.298396in}}%
\pgfpathlineto{\pgfqpoint{1.170566in}{8.210661in}}%
\pgfpathlineto{\pgfqpoint{1.082830in}{8.210661in}}%
\pgfpathlineto{\pgfqpoint{1.082830in}{8.298396in}}%
\pgfusepath{stroke,fill}%
\end{pgfscope}%
\begin{pgfscope}%
\pgfpathrectangle{\pgfqpoint{0.380943in}{8.035189in}}{\pgfqpoint{4.650000in}{0.614151in}}%
\pgfusepath{clip}%
\pgfsetbuttcap%
\pgfsetroundjoin%
\definecolor{currentfill}{rgb}{0.978131,0.843783,0.675709}%
\pgfsetfillcolor{currentfill}%
\pgfsetlinewidth{0.250937pt}%
\definecolor{currentstroke}{rgb}{1.000000,1.000000,1.000000}%
\pgfsetstrokecolor{currentstroke}%
\pgfsetdash{}{0pt}%
\pgfpathmoveto{\pgfqpoint{1.170566in}{8.298396in}}%
\pgfpathlineto{\pgfqpoint{1.258302in}{8.298396in}}%
\pgfpathlineto{\pgfqpoint{1.258302in}{8.210661in}}%
\pgfpathlineto{\pgfqpoint{1.170566in}{8.210661in}}%
\pgfpathlineto{\pgfqpoint{1.170566in}{8.298396in}}%
\pgfusepath{stroke,fill}%
\end{pgfscope}%
\begin{pgfscope}%
\pgfpathrectangle{\pgfqpoint{0.380943in}{8.035189in}}{\pgfqpoint{4.650000in}{0.614151in}}%
\pgfusepath{clip}%
\pgfsetbuttcap%
\pgfsetroundjoin%
\definecolor{currentfill}{rgb}{0.985083,0.974641,0.792587}%
\pgfsetfillcolor{currentfill}%
\pgfsetlinewidth{0.250937pt}%
\definecolor{currentstroke}{rgb}{1.000000,1.000000,1.000000}%
\pgfsetstrokecolor{currentstroke}%
\pgfsetdash{}{0pt}%
\pgfpathmoveto{\pgfqpoint{1.258302in}{8.298396in}}%
\pgfpathlineto{\pgfqpoint{1.346037in}{8.298396in}}%
\pgfpathlineto{\pgfqpoint{1.346037in}{8.210661in}}%
\pgfpathlineto{\pgfqpoint{1.258302in}{8.210661in}}%
\pgfpathlineto{\pgfqpoint{1.258302in}{8.298396in}}%
\pgfusepath{stroke,fill}%
\end{pgfscope}%
\begin{pgfscope}%
\pgfpathrectangle{\pgfqpoint{0.380943in}{8.035189in}}{\pgfqpoint{4.650000in}{0.614151in}}%
\pgfusepath{clip}%
\pgfsetbuttcap%
\pgfsetroundjoin%
\definecolor{currentfill}{rgb}{0.970012,0.883276,0.699577}%
\pgfsetfillcolor{currentfill}%
\pgfsetlinewidth{0.250937pt}%
\definecolor{currentstroke}{rgb}{1.000000,1.000000,1.000000}%
\pgfsetstrokecolor{currentstroke}%
\pgfsetdash{}{0pt}%
\pgfpathmoveto{\pgfqpoint{1.346037in}{8.298396in}}%
\pgfpathlineto{\pgfqpoint{1.433773in}{8.298396in}}%
\pgfpathlineto{\pgfqpoint{1.433773in}{8.210661in}}%
\pgfpathlineto{\pgfqpoint{1.346037in}{8.210661in}}%
\pgfpathlineto{\pgfqpoint{1.346037in}{8.298396in}}%
\pgfusepath{stroke,fill}%
\end{pgfscope}%
\begin{pgfscope}%
\pgfpathrectangle{\pgfqpoint{0.380943in}{8.035189in}}{\pgfqpoint{4.650000in}{0.614151in}}%
\pgfusepath{clip}%
\pgfsetbuttcap%
\pgfsetroundjoin%
\definecolor{currentfill}{rgb}{0.992326,0.765229,0.614840}%
\pgfsetfillcolor{currentfill}%
\pgfsetlinewidth{0.250937pt}%
\definecolor{currentstroke}{rgb}{1.000000,1.000000,1.000000}%
\pgfsetstrokecolor{currentstroke}%
\pgfsetdash{}{0pt}%
\pgfpathmoveto{\pgfqpoint{1.433773in}{8.298396in}}%
\pgfpathlineto{\pgfqpoint{1.521509in}{8.298396in}}%
\pgfpathlineto{\pgfqpoint{1.521509in}{8.210661in}}%
\pgfpathlineto{\pgfqpoint{1.433773in}{8.210661in}}%
\pgfpathlineto{\pgfqpoint{1.433773in}{8.298396in}}%
\pgfusepath{stroke,fill}%
\end{pgfscope}%
\begin{pgfscope}%
\pgfpathrectangle{\pgfqpoint{0.380943in}{8.035189in}}{\pgfqpoint{4.650000in}{0.614151in}}%
\pgfusepath{clip}%
\pgfsetbuttcap%
\pgfsetroundjoin%
\definecolor{currentfill}{rgb}{0.961061,0.931672,0.728304}%
\pgfsetfillcolor{currentfill}%
\pgfsetlinewidth{0.250937pt}%
\definecolor{currentstroke}{rgb}{1.000000,1.000000,1.000000}%
\pgfsetstrokecolor{currentstroke}%
\pgfsetdash{}{0pt}%
\pgfpathmoveto{\pgfqpoint{1.521509in}{8.298396in}}%
\pgfpathlineto{\pgfqpoint{1.609245in}{8.298396in}}%
\pgfpathlineto{\pgfqpoint{1.609245in}{8.210661in}}%
\pgfpathlineto{\pgfqpoint{1.521509in}{8.210661in}}%
\pgfpathlineto{\pgfqpoint{1.521509in}{8.298396in}}%
\pgfusepath{stroke,fill}%
\end{pgfscope}%
\begin{pgfscope}%
\pgfpathrectangle{\pgfqpoint{0.380943in}{8.035189in}}{\pgfqpoint{4.650000in}{0.614151in}}%
\pgfusepath{clip}%
\pgfsetbuttcap%
\pgfsetroundjoin%
\definecolor{currentfill}{rgb}{0.986251,0.808597,0.643230}%
\pgfsetfillcolor{currentfill}%
\pgfsetlinewidth{0.250937pt}%
\definecolor{currentstroke}{rgb}{1.000000,1.000000,1.000000}%
\pgfsetstrokecolor{currentstroke}%
\pgfsetdash{}{0pt}%
\pgfpathmoveto{\pgfqpoint{1.609245in}{8.298396in}}%
\pgfpathlineto{\pgfqpoint{1.696981in}{8.298396in}}%
\pgfpathlineto{\pgfqpoint{1.696981in}{8.210661in}}%
\pgfpathlineto{\pgfqpoint{1.609245in}{8.210661in}}%
\pgfpathlineto{\pgfqpoint{1.609245in}{8.298396in}}%
\pgfusepath{stroke,fill}%
\end{pgfscope}%
\begin{pgfscope}%
\pgfpathrectangle{\pgfqpoint{0.380943in}{8.035189in}}{\pgfqpoint{4.650000in}{0.614151in}}%
\pgfusepath{clip}%
\pgfsetbuttcap%
\pgfsetroundjoin%
\definecolor{currentfill}{rgb}{0.985083,0.974641,0.792587}%
\pgfsetfillcolor{currentfill}%
\pgfsetlinewidth{0.250937pt}%
\definecolor{currentstroke}{rgb}{1.000000,1.000000,1.000000}%
\pgfsetstrokecolor{currentstroke}%
\pgfsetdash{}{0pt}%
\pgfpathmoveto{\pgfqpoint{1.696981in}{8.298396in}}%
\pgfpathlineto{\pgfqpoint{1.784717in}{8.298396in}}%
\pgfpathlineto{\pgfqpoint{1.784717in}{8.210661in}}%
\pgfpathlineto{\pgfqpoint{1.696981in}{8.210661in}}%
\pgfpathlineto{\pgfqpoint{1.696981in}{8.298396in}}%
\pgfusepath{stroke,fill}%
\end{pgfscope}%
\begin{pgfscope}%
\pgfpathrectangle{\pgfqpoint{0.380943in}{8.035189in}}{\pgfqpoint{4.650000in}{0.614151in}}%
\pgfusepath{clip}%
\pgfsetbuttcap%
\pgfsetroundjoin%
\definecolor{currentfill}{rgb}{0.963768,0.915433,0.717478}%
\pgfsetfillcolor{currentfill}%
\pgfsetlinewidth{0.250937pt}%
\definecolor{currentstroke}{rgb}{1.000000,1.000000,1.000000}%
\pgfsetstrokecolor{currentstroke}%
\pgfsetdash{}{0pt}%
\pgfpathmoveto{\pgfqpoint{1.784717in}{8.298396in}}%
\pgfpathlineto{\pgfqpoint{1.872452in}{8.298396in}}%
\pgfpathlineto{\pgfqpoint{1.872452in}{8.210661in}}%
\pgfpathlineto{\pgfqpoint{1.784717in}{8.210661in}}%
\pgfpathlineto{\pgfqpoint{1.784717in}{8.298396in}}%
\pgfusepath{stroke,fill}%
\end{pgfscope}%
\begin{pgfscope}%
\pgfpathrectangle{\pgfqpoint{0.380943in}{8.035189in}}{\pgfqpoint{4.650000in}{0.614151in}}%
\pgfusepath{clip}%
\pgfsetbuttcap%
\pgfsetroundjoin%
\definecolor{currentfill}{rgb}{0.963768,0.915433,0.717478}%
\pgfsetfillcolor{currentfill}%
\pgfsetlinewidth{0.250937pt}%
\definecolor{currentstroke}{rgb}{1.000000,1.000000,1.000000}%
\pgfsetstrokecolor{currentstroke}%
\pgfsetdash{}{0pt}%
\pgfpathmoveto{\pgfqpoint{1.872452in}{8.298396in}}%
\pgfpathlineto{\pgfqpoint{1.960188in}{8.298396in}}%
\pgfpathlineto{\pgfqpoint{1.960188in}{8.210661in}}%
\pgfpathlineto{\pgfqpoint{1.872452in}{8.210661in}}%
\pgfpathlineto{\pgfqpoint{1.872452in}{8.298396in}}%
\pgfusepath{stroke,fill}%
\end{pgfscope}%
\begin{pgfscope}%
\pgfpathrectangle{\pgfqpoint{0.380943in}{8.035189in}}{\pgfqpoint{4.650000in}{0.614151in}}%
\pgfusepath{clip}%
\pgfsetbuttcap%
\pgfsetroundjoin%
\definecolor{currentfill}{rgb}{0.963768,0.915433,0.717478}%
\pgfsetfillcolor{currentfill}%
\pgfsetlinewidth{0.250937pt}%
\definecolor{currentstroke}{rgb}{1.000000,1.000000,1.000000}%
\pgfsetstrokecolor{currentstroke}%
\pgfsetdash{}{0pt}%
\pgfpathmoveto{\pgfqpoint{1.960188in}{8.298396in}}%
\pgfpathlineto{\pgfqpoint{2.047924in}{8.298396in}}%
\pgfpathlineto{\pgfqpoint{2.047924in}{8.210661in}}%
\pgfpathlineto{\pgfqpoint{1.960188in}{8.210661in}}%
\pgfpathlineto{\pgfqpoint{1.960188in}{8.298396in}}%
\pgfusepath{stroke,fill}%
\end{pgfscope}%
\begin{pgfscope}%
\pgfpathrectangle{\pgfqpoint{0.380943in}{8.035189in}}{\pgfqpoint{4.650000in}{0.614151in}}%
\pgfusepath{clip}%
\pgfsetbuttcap%
\pgfsetroundjoin%
\definecolor{currentfill}{rgb}{0.985083,0.974641,0.792587}%
\pgfsetfillcolor{currentfill}%
\pgfsetlinewidth{0.250937pt}%
\definecolor{currentstroke}{rgb}{1.000000,1.000000,1.000000}%
\pgfsetstrokecolor{currentstroke}%
\pgfsetdash{}{0pt}%
\pgfpathmoveto{\pgfqpoint{2.047924in}{8.298396in}}%
\pgfpathlineto{\pgfqpoint{2.135660in}{8.298396in}}%
\pgfpathlineto{\pgfqpoint{2.135660in}{8.210661in}}%
\pgfpathlineto{\pgfqpoint{2.047924in}{8.210661in}}%
\pgfpathlineto{\pgfqpoint{2.047924in}{8.298396in}}%
\pgfusepath{stroke,fill}%
\end{pgfscope}%
\begin{pgfscope}%
\pgfpathrectangle{\pgfqpoint{0.380943in}{8.035189in}}{\pgfqpoint{4.650000in}{0.614151in}}%
\pgfusepath{clip}%
\pgfsetbuttcap%
\pgfsetroundjoin%
\definecolor{currentfill}{rgb}{0.986251,0.808597,0.643230}%
\pgfsetfillcolor{currentfill}%
\pgfsetlinewidth{0.250937pt}%
\definecolor{currentstroke}{rgb}{1.000000,1.000000,1.000000}%
\pgfsetstrokecolor{currentstroke}%
\pgfsetdash{}{0pt}%
\pgfpathmoveto{\pgfqpoint{2.135660in}{8.298396in}}%
\pgfpathlineto{\pgfqpoint{2.223396in}{8.298396in}}%
\pgfpathlineto{\pgfqpoint{2.223396in}{8.210661in}}%
\pgfpathlineto{\pgfqpoint{2.135660in}{8.210661in}}%
\pgfpathlineto{\pgfqpoint{2.135660in}{8.298396in}}%
\pgfusepath{stroke,fill}%
\end{pgfscope}%
\begin{pgfscope}%
\pgfpathrectangle{\pgfqpoint{0.380943in}{8.035189in}}{\pgfqpoint{4.650000in}{0.614151in}}%
\pgfusepath{clip}%
\pgfsetbuttcap%
\pgfsetroundjoin%
\definecolor{currentfill}{rgb}{0.963768,0.915433,0.717478}%
\pgfsetfillcolor{currentfill}%
\pgfsetlinewidth{0.250937pt}%
\definecolor{currentstroke}{rgb}{1.000000,1.000000,1.000000}%
\pgfsetstrokecolor{currentstroke}%
\pgfsetdash{}{0pt}%
\pgfpathmoveto{\pgfqpoint{2.223396in}{8.298396in}}%
\pgfpathlineto{\pgfqpoint{2.311132in}{8.298396in}}%
\pgfpathlineto{\pgfqpoint{2.311132in}{8.210661in}}%
\pgfpathlineto{\pgfqpoint{2.223396in}{8.210661in}}%
\pgfpathlineto{\pgfqpoint{2.223396in}{8.298396in}}%
\pgfusepath{stroke,fill}%
\end{pgfscope}%
\begin{pgfscope}%
\pgfpathrectangle{\pgfqpoint{0.380943in}{8.035189in}}{\pgfqpoint{4.650000in}{0.614151in}}%
\pgfusepath{clip}%
\pgfsetbuttcap%
\pgfsetroundjoin%
\definecolor{currentfill}{rgb}{0.996909,0.711742,0.584452}%
\pgfsetfillcolor{currentfill}%
\pgfsetlinewidth{0.250937pt}%
\definecolor{currentstroke}{rgb}{1.000000,1.000000,1.000000}%
\pgfsetstrokecolor{currentstroke}%
\pgfsetdash{}{0pt}%
\pgfpathmoveto{\pgfqpoint{2.311132in}{8.298396in}}%
\pgfpathlineto{\pgfqpoint{2.398868in}{8.298396in}}%
\pgfpathlineto{\pgfqpoint{2.398868in}{8.210661in}}%
\pgfpathlineto{\pgfqpoint{2.311132in}{8.210661in}}%
\pgfpathlineto{\pgfqpoint{2.311132in}{8.298396in}}%
\pgfusepath{stroke,fill}%
\end{pgfscope}%
\begin{pgfscope}%
\pgfpathrectangle{\pgfqpoint{0.380943in}{8.035189in}}{\pgfqpoint{4.650000in}{0.614151in}}%
\pgfusepath{clip}%
\pgfsetbuttcap%
\pgfsetroundjoin%
\definecolor{currentfill}{rgb}{0.992326,0.765229,0.614840}%
\pgfsetfillcolor{currentfill}%
\pgfsetlinewidth{0.250937pt}%
\definecolor{currentstroke}{rgb}{1.000000,1.000000,1.000000}%
\pgfsetstrokecolor{currentstroke}%
\pgfsetdash{}{0pt}%
\pgfpathmoveto{\pgfqpoint{2.398868in}{8.298396in}}%
\pgfpathlineto{\pgfqpoint{2.486603in}{8.298396in}}%
\pgfpathlineto{\pgfqpoint{2.486603in}{8.210661in}}%
\pgfpathlineto{\pgfqpoint{2.398868in}{8.210661in}}%
\pgfpathlineto{\pgfqpoint{2.398868in}{8.298396in}}%
\pgfusepath{stroke,fill}%
\end{pgfscope}%
\begin{pgfscope}%
\pgfpathrectangle{\pgfqpoint{0.380943in}{8.035189in}}{\pgfqpoint{4.650000in}{0.614151in}}%
\pgfusepath{clip}%
\pgfsetbuttcap%
\pgfsetroundjoin%
\definecolor{currentfill}{rgb}{0.970012,0.883276,0.699577}%
\pgfsetfillcolor{currentfill}%
\pgfsetlinewidth{0.250937pt}%
\definecolor{currentstroke}{rgb}{1.000000,1.000000,1.000000}%
\pgfsetstrokecolor{currentstroke}%
\pgfsetdash{}{0pt}%
\pgfpathmoveto{\pgfqpoint{2.486603in}{8.298396in}}%
\pgfpathlineto{\pgfqpoint{2.574339in}{8.298396in}}%
\pgfpathlineto{\pgfqpoint{2.574339in}{8.210661in}}%
\pgfpathlineto{\pgfqpoint{2.486603in}{8.210661in}}%
\pgfpathlineto{\pgfqpoint{2.486603in}{8.298396in}}%
\pgfusepath{stroke,fill}%
\end{pgfscope}%
\begin{pgfscope}%
\pgfpathrectangle{\pgfqpoint{0.380943in}{8.035189in}}{\pgfqpoint{4.650000in}{0.614151in}}%
\pgfusepath{clip}%
\pgfsetbuttcap%
\pgfsetroundjoin%
\definecolor{currentfill}{rgb}{0.986251,0.808597,0.643230}%
\pgfsetfillcolor{currentfill}%
\pgfsetlinewidth{0.250937pt}%
\definecolor{currentstroke}{rgb}{1.000000,1.000000,1.000000}%
\pgfsetstrokecolor{currentstroke}%
\pgfsetdash{}{0pt}%
\pgfpathmoveto{\pgfqpoint{2.574339in}{8.298396in}}%
\pgfpathlineto{\pgfqpoint{2.662075in}{8.298396in}}%
\pgfpathlineto{\pgfqpoint{2.662075in}{8.210661in}}%
\pgfpathlineto{\pgfqpoint{2.574339in}{8.210661in}}%
\pgfpathlineto{\pgfqpoint{2.574339in}{8.298396in}}%
\pgfusepath{stroke,fill}%
\end{pgfscope}%
\begin{pgfscope}%
\pgfpathrectangle{\pgfqpoint{0.380943in}{8.035189in}}{\pgfqpoint{4.650000in}{0.614151in}}%
\pgfusepath{clip}%
\pgfsetbuttcap%
\pgfsetroundjoin%
\definecolor{currentfill}{rgb}{0.963768,0.915433,0.717478}%
\pgfsetfillcolor{currentfill}%
\pgfsetlinewidth{0.250937pt}%
\definecolor{currentstroke}{rgb}{1.000000,1.000000,1.000000}%
\pgfsetstrokecolor{currentstroke}%
\pgfsetdash{}{0pt}%
\pgfpathmoveto{\pgfqpoint{2.662075in}{8.298396in}}%
\pgfpathlineto{\pgfqpoint{2.749811in}{8.298396in}}%
\pgfpathlineto{\pgfqpoint{2.749811in}{8.210661in}}%
\pgfpathlineto{\pgfqpoint{2.662075in}{8.210661in}}%
\pgfpathlineto{\pgfqpoint{2.662075in}{8.298396in}}%
\pgfusepath{stroke,fill}%
\end{pgfscope}%
\begin{pgfscope}%
\pgfpathrectangle{\pgfqpoint{0.380943in}{8.035189in}}{\pgfqpoint{4.650000in}{0.614151in}}%
\pgfusepath{clip}%
\pgfsetbuttcap%
\pgfsetroundjoin%
\definecolor{currentfill}{rgb}{0.985083,0.974641,0.792587}%
\pgfsetfillcolor{currentfill}%
\pgfsetlinewidth{0.250937pt}%
\definecolor{currentstroke}{rgb}{1.000000,1.000000,1.000000}%
\pgfsetstrokecolor{currentstroke}%
\pgfsetdash{}{0pt}%
\pgfpathmoveto{\pgfqpoint{2.749811in}{8.298396in}}%
\pgfpathlineto{\pgfqpoint{2.837547in}{8.298396in}}%
\pgfpathlineto{\pgfqpoint{2.837547in}{8.210661in}}%
\pgfpathlineto{\pgfqpoint{2.749811in}{8.210661in}}%
\pgfpathlineto{\pgfqpoint{2.749811in}{8.298396in}}%
\pgfusepath{stroke,fill}%
\end{pgfscope}%
\begin{pgfscope}%
\pgfpathrectangle{\pgfqpoint{0.380943in}{8.035189in}}{\pgfqpoint{4.650000in}{0.614151in}}%
\pgfusepath{clip}%
\pgfsetbuttcap%
\pgfsetroundjoin%
\definecolor{currentfill}{rgb}{0.963768,0.915433,0.717478}%
\pgfsetfillcolor{currentfill}%
\pgfsetlinewidth{0.250937pt}%
\definecolor{currentstroke}{rgb}{1.000000,1.000000,1.000000}%
\pgfsetstrokecolor{currentstroke}%
\pgfsetdash{}{0pt}%
\pgfpathmoveto{\pgfqpoint{2.837547in}{8.298396in}}%
\pgfpathlineto{\pgfqpoint{2.925283in}{8.298396in}}%
\pgfpathlineto{\pgfqpoint{2.925283in}{8.210661in}}%
\pgfpathlineto{\pgfqpoint{2.837547in}{8.210661in}}%
\pgfpathlineto{\pgfqpoint{2.837547in}{8.298396in}}%
\pgfusepath{stroke,fill}%
\end{pgfscope}%
\begin{pgfscope}%
\pgfpathrectangle{\pgfqpoint{0.380943in}{8.035189in}}{\pgfqpoint{4.650000in}{0.614151in}}%
\pgfusepath{clip}%
\pgfsetbuttcap%
\pgfsetroundjoin%
\definecolor{currentfill}{rgb}{0.978131,0.843783,0.675709}%
\pgfsetfillcolor{currentfill}%
\pgfsetlinewidth{0.250937pt}%
\definecolor{currentstroke}{rgb}{1.000000,1.000000,1.000000}%
\pgfsetstrokecolor{currentstroke}%
\pgfsetdash{}{0pt}%
\pgfpathmoveto{\pgfqpoint{2.925283in}{8.298396in}}%
\pgfpathlineto{\pgfqpoint{3.013019in}{8.298396in}}%
\pgfpathlineto{\pgfqpoint{3.013019in}{8.210661in}}%
\pgfpathlineto{\pgfqpoint{2.925283in}{8.210661in}}%
\pgfpathlineto{\pgfqpoint{2.925283in}{8.298396in}}%
\pgfusepath{stroke,fill}%
\end{pgfscope}%
\begin{pgfscope}%
\pgfpathrectangle{\pgfqpoint{0.380943in}{8.035189in}}{\pgfqpoint{4.650000in}{0.614151in}}%
\pgfusepath{clip}%
\pgfsetbuttcap%
\pgfsetroundjoin%
\definecolor{currentfill}{rgb}{0.961061,0.931672,0.728304}%
\pgfsetfillcolor{currentfill}%
\pgfsetlinewidth{0.250937pt}%
\definecolor{currentstroke}{rgb}{1.000000,1.000000,1.000000}%
\pgfsetstrokecolor{currentstroke}%
\pgfsetdash{}{0pt}%
\pgfpathmoveto{\pgfqpoint{3.013019in}{8.298396in}}%
\pgfpathlineto{\pgfqpoint{3.100754in}{8.298396in}}%
\pgfpathlineto{\pgfqpoint{3.100754in}{8.210661in}}%
\pgfpathlineto{\pgfqpoint{3.013019in}{8.210661in}}%
\pgfpathlineto{\pgfqpoint{3.013019in}{8.298396in}}%
\pgfusepath{stroke,fill}%
\end{pgfscope}%
\begin{pgfscope}%
\pgfpathrectangle{\pgfqpoint{0.380943in}{8.035189in}}{\pgfqpoint{4.650000in}{0.614151in}}%
\pgfusepath{clip}%
\pgfsetbuttcap%
\pgfsetroundjoin%
\definecolor{currentfill}{rgb}{0.961061,0.931672,0.728304}%
\pgfsetfillcolor{currentfill}%
\pgfsetlinewidth{0.250937pt}%
\definecolor{currentstroke}{rgb}{1.000000,1.000000,1.000000}%
\pgfsetstrokecolor{currentstroke}%
\pgfsetdash{}{0pt}%
\pgfpathmoveto{\pgfqpoint{3.100754in}{8.298396in}}%
\pgfpathlineto{\pgfqpoint{3.188490in}{8.298396in}}%
\pgfpathlineto{\pgfqpoint{3.188490in}{8.210661in}}%
\pgfpathlineto{\pgfqpoint{3.100754in}{8.210661in}}%
\pgfpathlineto{\pgfqpoint{3.100754in}{8.298396in}}%
\pgfusepath{stroke,fill}%
\end{pgfscope}%
\begin{pgfscope}%
\pgfpathrectangle{\pgfqpoint{0.380943in}{8.035189in}}{\pgfqpoint{4.650000in}{0.614151in}}%
\pgfusepath{clip}%
\pgfsetbuttcap%
\pgfsetroundjoin%
\definecolor{currentfill}{rgb}{0.978131,0.843783,0.675709}%
\pgfsetfillcolor{currentfill}%
\pgfsetlinewidth{0.250937pt}%
\definecolor{currentstroke}{rgb}{1.000000,1.000000,1.000000}%
\pgfsetstrokecolor{currentstroke}%
\pgfsetdash{}{0pt}%
\pgfpathmoveto{\pgfqpoint{3.188490in}{8.298396in}}%
\pgfpathlineto{\pgfqpoint{3.276226in}{8.298396in}}%
\pgfpathlineto{\pgfqpoint{3.276226in}{8.210661in}}%
\pgfpathlineto{\pgfqpoint{3.188490in}{8.210661in}}%
\pgfpathlineto{\pgfqpoint{3.188490in}{8.298396in}}%
\pgfusepath{stroke,fill}%
\end{pgfscope}%
\begin{pgfscope}%
\pgfpathrectangle{\pgfqpoint{0.380943in}{8.035189in}}{\pgfqpoint{4.650000in}{0.614151in}}%
\pgfusepath{clip}%
\pgfsetbuttcap%
\pgfsetroundjoin%
\definecolor{currentfill}{rgb}{0.986251,0.808597,0.643230}%
\pgfsetfillcolor{currentfill}%
\pgfsetlinewidth{0.250937pt}%
\definecolor{currentstroke}{rgb}{1.000000,1.000000,1.000000}%
\pgfsetstrokecolor{currentstroke}%
\pgfsetdash{}{0pt}%
\pgfpathmoveto{\pgfqpoint{3.276226in}{8.298396in}}%
\pgfpathlineto{\pgfqpoint{3.363962in}{8.298396in}}%
\pgfpathlineto{\pgfqpoint{3.363962in}{8.210661in}}%
\pgfpathlineto{\pgfqpoint{3.276226in}{8.210661in}}%
\pgfpathlineto{\pgfqpoint{3.276226in}{8.298396in}}%
\pgfusepath{stroke,fill}%
\end{pgfscope}%
\begin{pgfscope}%
\pgfpathrectangle{\pgfqpoint{0.380943in}{8.035189in}}{\pgfqpoint{4.650000in}{0.614151in}}%
\pgfusepath{clip}%
\pgfsetbuttcap%
\pgfsetroundjoin%
\definecolor{currentfill}{rgb}{0.961061,0.931672,0.728304}%
\pgfsetfillcolor{currentfill}%
\pgfsetlinewidth{0.250937pt}%
\definecolor{currentstroke}{rgb}{1.000000,1.000000,1.000000}%
\pgfsetstrokecolor{currentstroke}%
\pgfsetdash{}{0pt}%
\pgfpathmoveto{\pgfqpoint{3.363962in}{8.298396in}}%
\pgfpathlineto{\pgfqpoint{3.451698in}{8.298396in}}%
\pgfpathlineto{\pgfqpoint{3.451698in}{8.210661in}}%
\pgfpathlineto{\pgfqpoint{3.363962in}{8.210661in}}%
\pgfpathlineto{\pgfqpoint{3.363962in}{8.298396in}}%
\pgfusepath{stroke,fill}%
\end{pgfscope}%
\begin{pgfscope}%
\pgfpathrectangle{\pgfqpoint{0.380943in}{8.035189in}}{\pgfqpoint{4.650000in}{0.614151in}}%
\pgfusepath{clip}%
\pgfsetbuttcap%
\pgfsetroundjoin%
\definecolor{currentfill}{rgb}{0.961061,0.931672,0.728304}%
\pgfsetfillcolor{currentfill}%
\pgfsetlinewidth{0.250937pt}%
\definecolor{currentstroke}{rgb}{1.000000,1.000000,1.000000}%
\pgfsetstrokecolor{currentstroke}%
\pgfsetdash{}{0pt}%
\pgfpathmoveto{\pgfqpoint{3.451698in}{8.298396in}}%
\pgfpathlineto{\pgfqpoint{3.539434in}{8.298396in}}%
\pgfpathlineto{\pgfqpoint{3.539434in}{8.210661in}}%
\pgfpathlineto{\pgfqpoint{3.451698in}{8.210661in}}%
\pgfpathlineto{\pgfqpoint{3.451698in}{8.298396in}}%
\pgfusepath{stroke,fill}%
\end{pgfscope}%
\begin{pgfscope}%
\pgfpathrectangle{\pgfqpoint{0.380943in}{8.035189in}}{\pgfqpoint{4.650000in}{0.614151in}}%
\pgfusepath{clip}%
\pgfsetbuttcap%
\pgfsetroundjoin%
\definecolor{currentfill}{rgb}{1.000000,0.584929,0.522599}%
\pgfsetfillcolor{currentfill}%
\pgfsetlinewidth{0.250937pt}%
\definecolor{currentstroke}{rgb}{1.000000,1.000000,1.000000}%
\pgfsetstrokecolor{currentstroke}%
\pgfsetdash{}{0pt}%
\pgfpathmoveto{\pgfqpoint{3.539434in}{8.298396in}}%
\pgfpathlineto{\pgfqpoint{3.627169in}{8.298396in}}%
\pgfpathlineto{\pgfqpoint{3.627169in}{8.210661in}}%
\pgfpathlineto{\pgfqpoint{3.539434in}{8.210661in}}%
\pgfpathlineto{\pgfqpoint{3.539434in}{8.298396in}}%
\pgfusepath{stroke,fill}%
\end{pgfscope}%
\begin{pgfscope}%
\pgfpathrectangle{\pgfqpoint{0.380943in}{8.035189in}}{\pgfqpoint{4.650000in}{0.614151in}}%
\pgfusepath{clip}%
\pgfsetbuttcap%
\pgfsetroundjoin%
\definecolor{currentfill}{rgb}{0.800000,0.278431,0.278431}%
\pgfsetfillcolor{currentfill}%
\pgfsetlinewidth{0.250937pt}%
\definecolor{currentstroke}{rgb}{1.000000,1.000000,1.000000}%
\pgfsetstrokecolor{currentstroke}%
\pgfsetdash{}{0pt}%
\pgfpathmoveto{\pgfqpoint{3.627169in}{8.298396in}}%
\pgfpathlineto{\pgfqpoint{3.714905in}{8.298396in}}%
\pgfpathlineto{\pgfqpoint{3.714905in}{8.210661in}}%
\pgfpathlineto{\pgfqpoint{3.627169in}{8.210661in}}%
\pgfpathlineto{\pgfqpoint{3.627169in}{8.298396in}}%
\pgfusepath{stroke,fill}%
\end{pgfscope}%
\begin{pgfscope}%
\pgfpathrectangle{\pgfqpoint{0.380943in}{8.035189in}}{\pgfqpoint{4.650000in}{0.614151in}}%
\pgfusepath{clip}%
\pgfsetbuttcap%
\pgfsetroundjoin%
\definecolor{currentfill}{rgb}{0.970012,0.883276,0.699577}%
\pgfsetfillcolor{currentfill}%
\pgfsetlinewidth{0.250937pt}%
\definecolor{currentstroke}{rgb}{1.000000,1.000000,1.000000}%
\pgfsetstrokecolor{currentstroke}%
\pgfsetdash{}{0pt}%
\pgfpathmoveto{\pgfqpoint{3.714905in}{8.298396in}}%
\pgfpathlineto{\pgfqpoint{3.802641in}{8.298396in}}%
\pgfpathlineto{\pgfqpoint{3.802641in}{8.210661in}}%
\pgfpathlineto{\pgfqpoint{3.714905in}{8.210661in}}%
\pgfpathlineto{\pgfqpoint{3.714905in}{8.298396in}}%
\pgfusepath{stroke,fill}%
\end{pgfscope}%
\begin{pgfscope}%
\pgfpathrectangle{\pgfqpoint{0.380943in}{8.035189in}}{\pgfqpoint{4.650000in}{0.614151in}}%
\pgfusepath{clip}%
\pgfsetbuttcap%
\pgfsetroundjoin%
\definecolor{currentfill}{rgb}{0.970012,0.883276,0.699577}%
\pgfsetfillcolor{currentfill}%
\pgfsetlinewidth{0.250937pt}%
\definecolor{currentstroke}{rgb}{1.000000,1.000000,1.000000}%
\pgfsetstrokecolor{currentstroke}%
\pgfsetdash{}{0pt}%
\pgfpathmoveto{\pgfqpoint{3.802641in}{8.298396in}}%
\pgfpathlineto{\pgfqpoint{3.890377in}{8.298396in}}%
\pgfpathlineto{\pgfqpoint{3.890377in}{8.210661in}}%
\pgfpathlineto{\pgfqpoint{3.802641in}{8.210661in}}%
\pgfpathlineto{\pgfqpoint{3.802641in}{8.298396in}}%
\pgfusepath{stroke,fill}%
\end{pgfscope}%
\begin{pgfscope}%
\pgfpathrectangle{\pgfqpoint{0.380943in}{8.035189in}}{\pgfqpoint{4.650000in}{0.614151in}}%
\pgfusepath{clip}%
\pgfsetbuttcap%
\pgfsetroundjoin%
\definecolor{currentfill}{rgb}{0.986251,0.808597,0.643230}%
\pgfsetfillcolor{currentfill}%
\pgfsetlinewidth{0.250937pt}%
\definecolor{currentstroke}{rgb}{1.000000,1.000000,1.000000}%
\pgfsetstrokecolor{currentstroke}%
\pgfsetdash{}{0pt}%
\pgfpathmoveto{\pgfqpoint{3.890377in}{8.298396in}}%
\pgfpathlineto{\pgfqpoint{3.978113in}{8.298396in}}%
\pgfpathlineto{\pgfqpoint{3.978113in}{8.210661in}}%
\pgfpathlineto{\pgfqpoint{3.890377in}{8.210661in}}%
\pgfpathlineto{\pgfqpoint{3.890377in}{8.298396in}}%
\pgfusepath{stroke,fill}%
\end{pgfscope}%
\begin{pgfscope}%
\pgfpathrectangle{\pgfqpoint{0.380943in}{8.035189in}}{\pgfqpoint{4.650000in}{0.614151in}}%
\pgfusepath{clip}%
\pgfsetbuttcap%
\pgfsetroundjoin%
\definecolor{currentfill}{rgb}{0.986251,0.808597,0.643230}%
\pgfsetfillcolor{currentfill}%
\pgfsetlinewidth{0.250937pt}%
\definecolor{currentstroke}{rgb}{1.000000,1.000000,1.000000}%
\pgfsetstrokecolor{currentstroke}%
\pgfsetdash{}{0pt}%
\pgfpathmoveto{\pgfqpoint{3.978113in}{8.298396in}}%
\pgfpathlineto{\pgfqpoint{4.065849in}{8.298396in}}%
\pgfpathlineto{\pgfqpoint{4.065849in}{8.210661in}}%
\pgfpathlineto{\pgfqpoint{3.978113in}{8.210661in}}%
\pgfpathlineto{\pgfqpoint{3.978113in}{8.298396in}}%
\pgfusepath{stroke,fill}%
\end{pgfscope}%
\begin{pgfscope}%
\pgfpathrectangle{\pgfqpoint{0.380943in}{8.035189in}}{\pgfqpoint{4.650000in}{0.614151in}}%
\pgfusepath{clip}%
\pgfsetbuttcap%
\pgfsetroundjoin%
\definecolor{currentfill}{rgb}{0.970012,0.883276,0.699577}%
\pgfsetfillcolor{currentfill}%
\pgfsetlinewidth{0.250937pt}%
\definecolor{currentstroke}{rgb}{1.000000,1.000000,1.000000}%
\pgfsetstrokecolor{currentstroke}%
\pgfsetdash{}{0pt}%
\pgfpathmoveto{\pgfqpoint{4.065849in}{8.298396in}}%
\pgfpathlineto{\pgfqpoint{4.153585in}{8.298396in}}%
\pgfpathlineto{\pgfqpoint{4.153585in}{8.210661in}}%
\pgfpathlineto{\pgfqpoint{4.065849in}{8.210661in}}%
\pgfpathlineto{\pgfqpoint{4.065849in}{8.298396in}}%
\pgfusepath{stroke,fill}%
\end{pgfscope}%
\begin{pgfscope}%
\pgfpathrectangle{\pgfqpoint{0.380943in}{8.035189in}}{\pgfqpoint{4.650000in}{0.614151in}}%
\pgfusepath{clip}%
\pgfsetbuttcap%
\pgfsetroundjoin%
\definecolor{currentfill}{rgb}{0.996909,0.711742,0.584452}%
\pgfsetfillcolor{currentfill}%
\pgfsetlinewidth{0.250937pt}%
\definecolor{currentstroke}{rgb}{1.000000,1.000000,1.000000}%
\pgfsetstrokecolor{currentstroke}%
\pgfsetdash{}{0pt}%
\pgfpathmoveto{\pgfqpoint{4.153585in}{8.298396in}}%
\pgfpathlineto{\pgfqpoint{4.241320in}{8.298396in}}%
\pgfpathlineto{\pgfqpoint{4.241320in}{8.210661in}}%
\pgfpathlineto{\pgfqpoint{4.153585in}{8.210661in}}%
\pgfpathlineto{\pgfqpoint{4.153585in}{8.298396in}}%
\pgfusepath{stroke,fill}%
\end{pgfscope}%
\begin{pgfscope}%
\pgfpathrectangle{\pgfqpoint{0.380943in}{8.035189in}}{\pgfqpoint{4.650000in}{0.614151in}}%
\pgfusepath{clip}%
\pgfsetbuttcap%
\pgfsetroundjoin%
\definecolor{currentfill}{rgb}{0.978131,0.843783,0.675709}%
\pgfsetfillcolor{currentfill}%
\pgfsetlinewidth{0.250937pt}%
\definecolor{currentstroke}{rgb}{1.000000,1.000000,1.000000}%
\pgfsetstrokecolor{currentstroke}%
\pgfsetdash{}{0pt}%
\pgfpathmoveto{\pgfqpoint{4.241320in}{8.298396in}}%
\pgfpathlineto{\pgfqpoint{4.329056in}{8.298396in}}%
\pgfpathlineto{\pgfqpoint{4.329056in}{8.210661in}}%
\pgfpathlineto{\pgfqpoint{4.241320in}{8.210661in}}%
\pgfpathlineto{\pgfqpoint{4.241320in}{8.298396in}}%
\pgfusepath{stroke,fill}%
\end{pgfscope}%
\begin{pgfscope}%
\pgfpathrectangle{\pgfqpoint{0.380943in}{8.035189in}}{\pgfqpoint{4.650000in}{0.614151in}}%
\pgfusepath{clip}%
\pgfsetbuttcap%
\pgfsetroundjoin%
\definecolor{currentfill}{rgb}{0.970012,0.883276,0.699577}%
\pgfsetfillcolor{currentfill}%
\pgfsetlinewidth{0.250937pt}%
\definecolor{currentstroke}{rgb}{1.000000,1.000000,1.000000}%
\pgfsetstrokecolor{currentstroke}%
\pgfsetdash{}{0pt}%
\pgfpathmoveto{\pgfqpoint{4.329056in}{8.298396in}}%
\pgfpathlineto{\pgfqpoint{4.416792in}{8.298396in}}%
\pgfpathlineto{\pgfqpoint{4.416792in}{8.210661in}}%
\pgfpathlineto{\pgfqpoint{4.329056in}{8.210661in}}%
\pgfpathlineto{\pgfqpoint{4.329056in}{8.298396in}}%
\pgfusepath{stroke,fill}%
\end{pgfscope}%
\begin{pgfscope}%
\pgfpathrectangle{\pgfqpoint{0.380943in}{8.035189in}}{\pgfqpoint{4.650000in}{0.614151in}}%
\pgfusepath{clip}%
\pgfsetbuttcap%
\pgfsetroundjoin%
\definecolor{currentfill}{rgb}{0.963768,0.915433,0.717478}%
\pgfsetfillcolor{currentfill}%
\pgfsetlinewidth{0.250937pt}%
\definecolor{currentstroke}{rgb}{1.000000,1.000000,1.000000}%
\pgfsetstrokecolor{currentstroke}%
\pgfsetdash{}{0pt}%
\pgfpathmoveto{\pgfqpoint{4.416792in}{8.298396in}}%
\pgfpathlineto{\pgfqpoint{4.504528in}{8.298396in}}%
\pgfpathlineto{\pgfqpoint{4.504528in}{8.210661in}}%
\pgfpathlineto{\pgfqpoint{4.416792in}{8.210661in}}%
\pgfpathlineto{\pgfqpoint{4.416792in}{8.298396in}}%
\pgfusepath{stroke,fill}%
\end{pgfscope}%
\begin{pgfscope}%
\pgfpathrectangle{\pgfqpoint{0.380943in}{8.035189in}}{\pgfqpoint{4.650000in}{0.614151in}}%
\pgfusepath{clip}%
\pgfsetbuttcap%
\pgfsetroundjoin%
\definecolor{currentfill}{rgb}{0.999616,0.641369,0.546559}%
\pgfsetfillcolor{currentfill}%
\pgfsetlinewidth{0.250937pt}%
\definecolor{currentstroke}{rgb}{1.000000,1.000000,1.000000}%
\pgfsetstrokecolor{currentstroke}%
\pgfsetdash{}{0pt}%
\pgfpathmoveto{\pgfqpoint{4.504528in}{8.298396in}}%
\pgfpathlineto{\pgfqpoint{4.592264in}{8.298396in}}%
\pgfpathlineto{\pgfqpoint{4.592264in}{8.210661in}}%
\pgfpathlineto{\pgfqpoint{4.504528in}{8.210661in}}%
\pgfpathlineto{\pgfqpoint{4.504528in}{8.298396in}}%
\pgfusepath{stroke,fill}%
\end{pgfscope}%
\begin{pgfscope}%
\pgfpathrectangle{\pgfqpoint{0.380943in}{8.035189in}}{\pgfqpoint{4.650000in}{0.614151in}}%
\pgfusepath{clip}%
\pgfsetbuttcap%
\pgfsetroundjoin%
\definecolor{currentfill}{rgb}{0.986251,0.808597,0.643230}%
\pgfsetfillcolor{currentfill}%
\pgfsetlinewidth{0.250937pt}%
\definecolor{currentstroke}{rgb}{1.000000,1.000000,1.000000}%
\pgfsetstrokecolor{currentstroke}%
\pgfsetdash{}{0pt}%
\pgfpathmoveto{\pgfqpoint{4.592264in}{8.298396in}}%
\pgfpathlineto{\pgfqpoint{4.680000in}{8.298396in}}%
\pgfpathlineto{\pgfqpoint{4.680000in}{8.210661in}}%
\pgfpathlineto{\pgfqpoint{4.592264in}{8.210661in}}%
\pgfpathlineto{\pgfqpoint{4.592264in}{8.298396in}}%
\pgfusepath{stroke,fill}%
\end{pgfscope}%
\begin{pgfscope}%
\pgfpathrectangle{\pgfqpoint{0.380943in}{8.035189in}}{\pgfqpoint{4.650000in}{0.614151in}}%
\pgfusepath{clip}%
\pgfsetbuttcap%
\pgfsetroundjoin%
\definecolor{currentfill}{rgb}{0.985083,0.974641,0.792587}%
\pgfsetfillcolor{currentfill}%
\pgfsetlinewidth{0.250937pt}%
\definecolor{currentstroke}{rgb}{1.000000,1.000000,1.000000}%
\pgfsetstrokecolor{currentstroke}%
\pgfsetdash{}{0pt}%
\pgfpathmoveto{\pgfqpoint{4.680000in}{8.298396in}}%
\pgfpathlineto{\pgfqpoint{4.767736in}{8.298396in}}%
\pgfpathlineto{\pgfqpoint{4.767736in}{8.210661in}}%
\pgfpathlineto{\pgfqpoint{4.680000in}{8.210661in}}%
\pgfpathlineto{\pgfqpoint{4.680000in}{8.298396in}}%
\pgfusepath{stroke,fill}%
\end{pgfscope}%
\begin{pgfscope}%
\pgfpathrectangle{\pgfqpoint{0.380943in}{8.035189in}}{\pgfqpoint{4.650000in}{0.614151in}}%
\pgfusepath{clip}%
\pgfsetbuttcap%
\pgfsetroundjoin%
\definecolor{currentfill}{rgb}{0.986251,0.808597,0.643230}%
\pgfsetfillcolor{currentfill}%
\pgfsetlinewidth{0.250937pt}%
\definecolor{currentstroke}{rgb}{1.000000,1.000000,1.000000}%
\pgfsetstrokecolor{currentstroke}%
\pgfsetdash{}{0pt}%
\pgfpathmoveto{\pgfqpoint{4.767736in}{8.298396in}}%
\pgfpathlineto{\pgfqpoint{4.855471in}{8.298396in}}%
\pgfpathlineto{\pgfqpoint{4.855471in}{8.210661in}}%
\pgfpathlineto{\pgfqpoint{4.767736in}{8.210661in}}%
\pgfpathlineto{\pgfqpoint{4.767736in}{8.298396in}}%
\pgfusepath{stroke,fill}%
\end{pgfscope}%
\begin{pgfscope}%
\pgfpathrectangle{\pgfqpoint{0.380943in}{8.035189in}}{\pgfqpoint{4.650000in}{0.614151in}}%
\pgfusepath{clip}%
\pgfsetbuttcap%
\pgfsetroundjoin%
\definecolor{currentfill}{rgb}{1.000000,0.584929,0.522599}%
\pgfsetfillcolor{currentfill}%
\pgfsetlinewidth{0.250937pt}%
\definecolor{currentstroke}{rgb}{1.000000,1.000000,1.000000}%
\pgfsetstrokecolor{currentstroke}%
\pgfsetdash{}{0pt}%
\pgfpathmoveto{\pgfqpoint{4.855471in}{8.298396in}}%
\pgfpathlineto{\pgfqpoint{4.943207in}{8.298396in}}%
\pgfpathlineto{\pgfqpoint{4.943207in}{8.210661in}}%
\pgfpathlineto{\pgfqpoint{4.855471in}{8.210661in}}%
\pgfpathlineto{\pgfqpoint{4.855471in}{8.298396in}}%
\pgfusepath{stroke,fill}%
\end{pgfscope}%
\begin{pgfscope}%
\pgfpathrectangle{\pgfqpoint{0.380943in}{8.035189in}}{\pgfqpoint{4.650000in}{0.614151in}}%
\pgfusepath{clip}%
\pgfsetbuttcap%
\pgfsetroundjoin%
\definecolor{currentfill}{rgb}{0.978131,0.843783,0.675709}%
\pgfsetfillcolor{currentfill}%
\pgfsetlinewidth{0.250937pt}%
\definecolor{currentstroke}{rgb}{1.000000,1.000000,1.000000}%
\pgfsetstrokecolor{currentstroke}%
\pgfsetdash{}{0pt}%
\pgfpathmoveto{\pgfqpoint{4.943207in}{8.298396in}}%
\pgfpathlineto{\pgfqpoint{5.030943in}{8.298396in}}%
\pgfpathlineto{\pgfqpoint{5.030943in}{8.210661in}}%
\pgfpathlineto{\pgfqpoint{4.943207in}{8.210661in}}%
\pgfpathlineto{\pgfqpoint{4.943207in}{8.298396in}}%
\pgfusepath{stroke,fill}%
\end{pgfscope}%
\begin{pgfscope}%
\pgfpathrectangle{\pgfqpoint{0.380943in}{8.035189in}}{\pgfqpoint{4.650000in}{0.614151in}}%
\pgfusepath{clip}%
\pgfsetbuttcap%
\pgfsetroundjoin%
\pgfsetlinewidth{0.250937pt}%
\definecolor{currentstroke}{rgb}{1.000000,1.000000,1.000000}%
\pgfsetstrokecolor{currentstroke}%
\pgfsetdash{}{0pt}%
\pgfpathmoveto{\pgfqpoint{0.380943in}{8.210661in}}%
\pgfpathlineto{\pgfqpoint{0.468679in}{8.210661in}}%
\pgfpathlineto{\pgfqpoint{0.468679in}{8.122925in}}%
\pgfpathlineto{\pgfqpoint{0.380943in}{8.122925in}}%
\pgfpathlineto{\pgfqpoint{0.380943in}{8.210661in}}%
\pgfusepath{stroke}%
\end{pgfscope}%
\begin{pgfscope}%
\pgfpathrectangle{\pgfqpoint{0.380943in}{8.035189in}}{\pgfqpoint{4.650000in}{0.614151in}}%
\pgfusepath{clip}%
\pgfsetbuttcap%
\pgfsetroundjoin%
\definecolor{currentfill}{rgb}{0.961061,0.931672,0.728304}%
\pgfsetfillcolor{currentfill}%
\pgfsetlinewidth{0.250937pt}%
\definecolor{currentstroke}{rgb}{1.000000,1.000000,1.000000}%
\pgfsetstrokecolor{currentstroke}%
\pgfsetdash{}{0pt}%
\pgfpathmoveto{\pgfqpoint{0.468679in}{8.210661in}}%
\pgfpathlineto{\pgfqpoint{0.556415in}{8.210661in}}%
\pgfpathlineto{\pgfqpoint{0.556415in}{8.122925in}}%
\pgfpathlineto{\pgfqpoint{0.468679in}{8.122925in}}%
\pgfpathlineto{\pgfqpoint{0.468679in}{8.210661in}}%
\pgfusepath{stroke,fill}%
\end{pgfscope}%
\begin{pgfscope}%
\pgfpathrectangle{\pgfqpoint{0.380943in}{8.035189in}}{\pgfqpoint{4.650000in}{0.614151in}}%
\pgfusepath{clip}%
\pgfsetbuttcap%
\pgfsetroundjoin%
\definecolor{currentfill}{rgb}{0.978131,0.843783,0.675709}%
\pgfsetfillcolor{currentfill}%
\pgfsetlinewidth{0.250937pt}%
\definecolor{currentstroke}{rgb}{1.000000,1.000000,1.000000}%
\pgfsetstrokecolor{currentstroke}%
\pgfsetdash{}{0pt}%
\pgfpathmoveto{\pgfqpoint{0.556415in}{8.210661in}}%
\pgfpathlineto{\pgfqpoint{0.644151in}{8.210661in}}%
\pgfpathlineto{\pgfqpoint{0.644151in}{8.122925in}}%
\pgfpathlineto{\pgfqpoint{0.556415in}{8.122925in}}%
\pgfpathlineto{\pgfqpoint{0.556415in}{8.210661in}}%
\pgfusepath{stroke,fill}%
\end{pgfscope}%
\begin{pgfscope}%
\pgfpathrectangle{\pgfqpoint{0.380943in}{8.035189in}}{\pgfqpoint{4.650000in}{0.614151in}}%
\pgfusepath{clip}%
\pgfsetbuttcap%
\pgfsetroundjoin%
\definecolor{currentfill}{rgb}{0.996909,0.711742,0.584452}%
\pgfsetfillcolor{currentfill}%
\pgfsetlinewidth{0.250937pt}%
\definecolor{currentstroke}{rgb}{1.000000,1.000000,1.000000}%
\pgfsetstrokecolor{currentstroke}%
\pgfsetdash{}{0pt}%
\pgfpathmoveto{\pgfqpoint{0.644151in}{8.210661in}}%
\pgfpathlineto{\pgfqpoint{0.731886in}{8.210661in}}%
\pgfpathlineto{\pgfqpoint{0.731886in}{8.122925in}}%
\pgfpathlineto{\pgfqpoint{0.644151in}{8.122925in}}%
\pgfpathlineto{\pgfqpoint{0.644151in}{8.210661in}}%
\pgfusepath{stroke,fill}%
\end{pgfscope}%
\begin{pgfscope}%
\pgfpathrectangle{\pgfqpoint{0.380943in}{8.035189in}}{\pgfqpoint{4.650000in}{0.614151in}}%
\pgfusepath{clip}%
\pgfsetbuttcap%
\pgfsetroundjoin%
\definecolor{currentfill}{rgb}{0.978131,0.843783,0.675709}%
\pgfsetfillcolor{currentfill}%
\pgfsetlinewidth{0.250937pt}%
\definecolor{currentstroke}{rgb}{1.000000,1.000000,1.000000}%
\pgfsetstrokecolor{currentstroke}%
\pgfsetdash{}{0pt}%
\pgfpathmoveto{\pgfqpoint{0.731886in}{8.210661in}}%
\pgfpathlineto{\pgfqpoint{0.819622in}{8.210661in}}%
\pgfpathlineto{\pgfqpoint{0.819622in}{8.122925in}}%
\pgfpathlineto{\pgfqpoint{0.731886in}{8.122925in}}%
\pgfpathlineto{\pgfqpoint{0.731886in}{8.210661in}}%
\pgfusepath{stroke,fill}%
\end{pgfscope}%
\begin{pgfscope}%
\pgfpathrectangle{\pgfqpoint{0.380943in}{8.035189in}}{\pgfqpoint{4.650000in}{0.614151in}}%
\pgfusepath{clip}%
\pgfsetbuttcap%
\pgfsetroundjoin%
\definecolor{currentfill}{rgb}{0.978131,0.843783,0.675709}%
\pgfsetfillcolor{currentfill}%
\pgfsetlinewidth{0.250937pt}%
\definecolor{currentstroke}{rgb}{1.000000,1.000000,1.000000}%
\pgfsetstrokecolor{currentstroke}%
\pgfsetdash{}{0pt}%
\pgfpathmoveto{\pgfqpoint{0.819622in}{8.210661in}}%
\pgfpathlineto{\pgfqpoint{0.907358in}{8.210661in}}%
\pgfpathlineto{\pgfqpoint{0.907358in}{8.122925in}}%
\pgfpathlineto{\pgfqpoint{0.819622in}{8.122925in}}%
\pgfpathlineto{\pgfqpoint{0.819622in}{8.210661in}}%
\pgfusepath{stroke,fill}%
\end{pgfscope}%
\begin{pgfscope}%
\pgfpathrectangle{\pgfqpoint{0.380943in}{8.035189in}}{\pgfqpoint{4.650000in}{0.614151in}}%
\pgfusepath{clip}%
\pgfsetbuttcap%
\pgfsetroundjoin%
\definecolor{currentfill}{rgb}{0.985083,0.974641,0.792587}%
\pgfsetfillcolor{currentfill}%
\pgfsetlinewidth{0.250937pt}%
\definecolor{currentstroke}{rgb}{1.000000,1.000000,1.000000}%
\pgfsetstrokecolor{currentstroke}%
\pgfsetdash{}{0pt}%
\pgfpathmoveto{\pgfqpoint{0.907358in}{8.210661in}}%
\pgfpathlineto{\pgfqpoint{0.995094in}{8.210661in}}%
\pgfpathlineto{\pgfqpoint{0.995094in}{8.122925in}}%
\pgfpathlineto{\pgfqpoint{0.907358in}{8.122925in}}%
\pgfpathlineto{\pgfqpoint{0.907358in}{8.210661in}}%
\pgfusepath{stroke,fill}%
\end{pgfscope}%
\begin{pgfscope}%
\pgfpathrectangle{\pgfqpoint{0.380943in}{8.035189in}}{\pgfqpoint{4.650000in}{0.614151in}}%
\pgfusepath{clip}%
\pgfsetbuttcap%
\pgfsetroundjoin%
\definecolor{currentfill}{rgb}{0.985083,0.974641,0.792587}%
\pgfsetfillcolor{currentfill}%
\pgfsetlinewidth{0.250937pt}%
\definecolor{currentstroke}{rgb}{1.000000,1.000000,1.000000}%
\pgfsetstrokecolor{currentstroke}%
\pgfsetdash{}{0pt}%
\pgfpathmoveto{\pgfqpoint{0.995094in}{8.210661in}}%
\pgfpathlineto{\pgfqpoint{1.082830in}{8.210661in}}%
\pgfpathlineto{\pgfqpoint{1.082830in}{8.122925in}}%
\pgfpathlineto{\pgfqpoint{0.995094in}{8.122925in}}%
\pgfpathlineto{\pgfqpoint{0.995094in}{8.210661in}}%
\pgfusepath{stroke,fill}%
\end{pgfscope}%
\begin{pgfscope}%
\pgfpathrectangle{\pgfqpoint{0.380943in}{8.035189in}}{\pgfqpoint{4.650000in}{0.614151in}}%
\pgfusepath{clip}%
\pgfsetbuttcap%
\pgfsetroundjoin%
\definecolor{currentfill}{rgb}{0.961061,0.931672,0.728304}%
\pgfsetfillcolor{currentfill}%
\pgfsetlinewidth{0.250937pt}%
\definecolor{currentstroke}{rgb}{1.000000,1.000000,1.000000}%
\pgfsetstrokecolor{currentstroke}%
\pgfsetdash{}{0pt}%
\pgfpathmoveto{\pgfqpoint{1.082830in}{8.210661in}}%
\pgfpathlineto{\pgfqpoint{1.170566in}{8.210661in}}%
\pgfpathlineto{\pgfqpoint{1.170566in}{8.122925in}}%
\pgfpathlineto{\pgfqpoint{1.082830in}{8.122925in}}%
\pgfpathlineto{\pgfqpoint{1.082830in}{8.210661in}}%
\pgfusepath{stroke,fill}%
\end{pgfscope}%
\begin{pgfscope}%
\pgfpathrectangle{\pgfqpoint{0.380943in}{8.035189in}}{\pgfqpoint{4.650000in}{0.614151in}}%
\pgfusepath{clip}%
\pgfsetbuttcap%
\pgfsetroundjoin%
\definecolor{currentfill}{rgb}{0.978131,0.843783,0.675709}%
\pgfsetfillcolor{currentfill}%
\pgfsetlinewidth{0.250937pt}%
\definecolor{currentstroke}{rgb}{1.000000,1.000000,1.000000}%
\pgfsetstrokecolor{currentstroke}%
\pgfsetdash{}{0pt}%
\pgfpathmoveto{\pgfqpoint{1.170566in}{8.210661in}}%
\pgfpathlineto{\pgfqpoint{1.258302in}{8.210661in}}%
\pgfpathlineto{\pgfqpoint{1.258302in}{8.122925in}}%
\pgfpathlineto{\pgfqpoint{1.170566in}{8.122925in}}%
\pgfpathlineto{\pgfqpoint{1.170566in}{8.210661in}}%
\pgfusepath{stroke,fill}%
\end{pgfscope}%
\begin{pgfscope}%
\pgfpathrectangle{\pgfqpoint{0.380943in}{8.035189in}}{\pgfqpoint{4.650000in}{0.614151in}}%
\pgfusepath{clip}%
\pgfsetbuttcap%
\pgfsetroundjoin%
\definecolor{currentfill}{rgb}{0.978131,0.843783,0.675709}%
\pgfsetfillcolor{currentfill}%
\pgfsetlinewidth{0.250937pt}%
\definecolor{currentstroke}{rgb}{1.000000,1.000000,1.000000}%
\pgfsetstrokecolor{currentstroke}%
\pgfsetdash{}{0pt}%
\pgfpathmoveto{\pgfqpoint{1.258302in}{8.210661in}}%
\pgfpathlineto{\pgfqpoint{1.346037in}{8.210661in}}%
\pgfpathlineto{\pgfqpoint{1.346037in}{8.122925in}}%
\pgfpathlineto{\pgfqpoint{1.258302in}{8.122925in}}%
\pgfpathlineto{\pgfqpoint{1.258302in}{8.210661in}}%
\pgfusepath{stroke,fill}%
\end{pgfscope}%
\begin{pgfscope}%
\pgfpathrectangle{\pgfqpoint{0.380943in}{8.035189in}}{\pgfqpoint{4.650000in}{0.614151in}}%
\pgfusepath{clip}%
\pgfsetbuttcap%
\pgfsetroundjoin%
\definecolor{currentfill}{rgb}{0.961061,0.931672,0.728304}%
\pgfsetfillcolor{currentfill}%
\pgfsetlinewidth{0.250937pt}%
\definecolor{currentstroke}{rgb}{1.000000,1.000000,1.000000}%
\pgfsetstrokecolor{currentstroke}%
\pgfsetdash{}{0pt}%
\pgfpathmoveto{\pgfqpoint{1.346037in}{8.210661in}}%
\pgfpathlineto{\pgfqpoint{1.433773in}{8.210661in}}%
\pgfpathlineto{\pgfqpoint{1.433773in}{8.122925in}}%
\pgfpathlineto{\pgfqpoint{1.346037in}{8.122925in}}%
\pgfpathlineto{\pgfqpoint{1.346037in}{8.210661in}}%
\pgfusepath{stroke,fill}%
\end{pgfscope}%
\begin{pgfscope}%
\pgfpathrectangle{\pgfqpoint{0.380943in}{8.035189in}}{\pgfqpoint{4.650000in}{0.614151in}}%
\pgfusepath{clip}%
\pgfsetbuttcap%
\pgfsetroundjoin%
\definecolor{currentfill}{rgb}{0.985083,0.974641,0.792587}%
\pgfsetfillcolor{currentfill}%
\pgfsetlinewidth{0.250937pt}%
\definecolor{currentstroke}{rgb}{1.000000,1.000000,1.000000}%
\pgfsetstrokecolor{currentstroke}%
\pgfsetdash{}{0pt}%
\pgfpathmoveto{\pgfqpoint{1.433773in}{8.210661in}}%
\pgfpathlineto{\pgfqpoint{1.521509in}{8.210661in}}%
\pgfpathlineto{\pgfqpoint{1.521509in}{8.122925in}}%
\pgfpathlineto{\pgfqpoint{1.433773in}{8.122925in}}%
\pgfpathlineto{\pgfqpoint{1.433773in}{8.210661in}}%
\pgfusepath{stroke,fill}%
\end{pgfscope}%
\begin{pgfscope}%
\pgfpathrectangle{\pgfqpoint{0.380943in}{8.035189in}}{\pgfqpoint{4.650000in}{0.614151in}}%
\pgfusepath{clip}%
\pgfsetbuttcap%
\pgfsetroundjoin%
\definecolor{currentfill}{rgb}{1.000000,1.000000,0.861745}%
\pgfsetfillcolor{currentfill}%
\pgfsetlinewidth{0.250937pt}%
\definecolor{currentstroke}{rgb}{1.000000,1.000000,1.000000}%
\pgfsetstrokecolor{currentstroke}%
\pgfsetdash{}{0pt}%
\pgfpathmoveto{\pgfqpoint{1.521509in}{8.210661in}}%
\pgfpathlineto{\pgfqpoint{1.609245in}{8.210661in}}%
\pgfpathlineto{\pgfqpoint{1.609245in}{8.122925in}}%
\pgfpathlineto{\pgfqpoint{1.521509in}{8.122925in}}%
\pgfpathlineto{\pgfqpoint{1.521509in}{8.210661in}}%
\pgfusepath{stroke,fill}%
\end{pgfscope}%
\begin{pgfscope}%
\pgfpathrectangle{\pgfqpoint{0.380943in}{8.035189in}}{\pgfqpoint{4.650000in}{0.614151in}}%
\pgfusepath{clip}%
\pgfsetbuttcap%
\pgfsetroundjoin%
\definecolor{currentfill}{rgb}{1.000000,1.000000,0.861745}%
\pgfsetfillcolor{currentfill}%
\pgfsetlinewidth{0.250937pt}%
\definecolor{currentstroke}{rgb}{1.000000,1.000000,1.000000}%
\pgfsetstrokecolor{currentstroke}%
\pgfsetdash{}{0pt}%
\pgfpathmoveto{\pgfqpoint{1.609245in}{8.210661in}}%
\pgfpathlineto{\pgfqpoint{1.696981in}{8.210661in}}%
\pgfpathlineto{\pgfqpoint{1.696981in}{8.122925in}}%
\pgfpathlineto{\pgfqpoint{1.609245in}{8.122925in}}%
\pgfpathlineto{\pgfqpoint{1.609245in}{8.210661in}}%
\pgfusepath{stroke,fill}%
\end{pgfscope}%
\begin{pgfscope}%
\pgfpathrectangle{\pgfqpoint{0.380943in}{8.035189in}}{\pgfqpoint{4.650000in}{0.614151in}}%
\pgfusepath{clip}%
\pgfsetbuttcap%
\pgfsetroundjoin%
\definecolor{currentfill}{rgb}{0.985083,0.974641,0.792587}%
\pgfsetfillcolor{currentfill}%
\pgfsetlinewidth{0.250937pt}%
\definecolor{currentstroke}{rgb}{1.000000,1.000000,1.000000}%
\pgfsetstrokecolor{currentstroke}%
\pgfsetdash{}{0pt}%
\pgfpathmoveto{\pgfqpoint{1.696981in}{8.210661in}}%
\pgfpathlineto{\pgfqpoint{1.784717in}{8.210661in}}%
\pgfpathlineto{\pgfqpoint{1.784717in}{8.122925in}}%
\pgfpathlineto{\pgfqpoint{1.696981in}{8.122925in}}%
\pgfpathlineto{\pgfqpoint{1.696981in}{8.210661in}}%
\pgfusepath{stroke,fill}%
\end{pgfscope}%
\begin{pgfscope}%
\pgfpathrectangle{\pgfqpoint{0.380943in}{8.035189in}}{\pgfqpoint{4.650000in}{0.614151in}}%
\pgfusepath{clip}%
\pgfsetbuttcap%
\pgfsetroundjoin%
\definecolor{currentfill}{rgb}{0.985083,0.974641,0.792587}%
\pgfsetfillcolor{currentfill}%
\pgfsetlinewidth{0.250937pt}%
\definecolor{currentstroke}{rgb}{1.000000,1.000000,1.000000}%
\pgfsetstrokecolor{currentstroke}%
\pgfsetdash{}{0pt}%
\pgfpathmoveto{\pgfqpoint{1.784717in}{8.210661in}}%
\pgfpathlineto{\pgfqpoint{1.872452in}{8.210661in}}%
\pgfpathlineto{\pgfqpoint{1.872452in}{8.122925in}}%
\pgfpathlineto{\pgfqpoint{1.784717in}{8.122925in}}%
\pgfpathlineto{\pgfqpoint{1.784717in}{8.210661in}}%
\pgfusepath{stroke,fill}%
\end{pgfscope}%
\begin{pgfscope}%
\pgfpathrectangle{\pgfqpoint{0.380943in}{8.035189in}}{\pgfqpoint{4.650000in}{0.614151in}}%
\pgfusepath{clip}%
\pgfsetbuttcap%
\pgfsetroundjoin%
\definecolor{currentfill}{rgb}{0.978131,0.843783,0.675709}%
\pgfsetfillcolor{currentfill}%
\pgfsetlinewidth{0.250937pt}%
\definecolor{currentstroke}{rgb}{1.000000,1.000000,1.000000}%
\pgfsetstrokecolor{currentstroke}%
\pgfsetdash{}{0pt}%
\pgfpathmoveto{\pgfqpoint{1.872452in}{8.210661in}}%
\pgfpathlineto{\pgfqpoint{1.960188in}{8.210661in}}%
\pgfpathlineto{\pgfqpoint{1.960188in}{8.122925in}}%
\pgfpathlineto{\pgfqpoint{1.872452in}{8.122925in}}%
\pgfpathlineto{\pgfqpoint{1.872452in}{8.210661in}}%
\pgfusepath{stroke,fill}%
\end{pgfscope}%
\begin{pgfscope}%
\pgfpathrectangle{\pgfqpoint{0.380943in}{8.035189in}}{\pgfqpoint{4.650000in}{0.614151in}}%
\pgfusepath{clip}%
\pgfsetbuttcap%
\pgfsetroundjoin%
\definecolor{currentfill}{rgb}{0.985083,0.974641,0.792587}%
\pgfsetfillcolor{currentfill}%
\pgfsetlinewidth{0.250937pt}%
\definecolor{currentstroke}{rgb}{1.000000,1.000000,1.000000}%
\pgfsetstrokecolor{currentstroke}%
\pgfsetdash{}{0pt}%
\pgfpathmoveto{\pgfqpoint{1.960188in}{8.210661in}}%
\pgfpathlineto{\pgfqpoint{2.047924in}{8.210661in}}%
\pgfpathlineto{\pgfqpoint{2.047924in}{8.122925in}}%
\pgfpathlineto{\pgfqpoint{1.960188in}{8.122925in}}%
\pgfpathlineto{\pgfqpoint{1.960188in}{8.210661in}}%
\pgfusepath{stroke,fill}%
\end{pgfscope}%
\begin{pgfscope}%
\pgfpathrectangle{\pgfqpoint{0.380943in}{8.035189in}}{\pgfqpoint{4.650000in}{0.614151in}}%
\pgfusepath{clip}%
\pgfsetbuttcap%
\pgfsetroundjoin%
\definecolor{currentfill}{rgb}{0.963768,0.915433,0.717478}%
\pgfsetfillcolor{currentfill}%
\pgfsetlinewidth{0.250937pt}%
\definecolor{currentstroke}{rgb}{1.000000,1.000000,1.000000}%
\pgfsetstrokecolor{currentstroke}%
\pgfsetdash{}{0pt}%
\pgfpathmoveto{\pgfqpoint{2.047924in}{8.210661in}}%
\pgfpathlineto{\pgfqpoint{2.135660in}{8.210661in}}%
\pgfpathlineto{\pgfqpoint{2.135660in}{8.122925in}}%
\pgfpathlineto{\pgfqpoint{2.047924in}{8.122925in}}%
\pgfpathlineto{\pgfqpoint{2.047924in}{8.210661in}}%
\pgfusepath{stroke,fill}%
\end{pgfscope}%
\begin{pgfscope}%
\pgfpathrectangle{\pgfqpoint{0.380943in}{8.035189in}}{\pgfqpoint{4.650000in}{0.614151in}}%
\pgfusepath{clip}%
\pgfsetbuttcap%
\pgfsetroundjoin%
\definecolor{currentfill}{rgb}{0.970012,0.883276,0.699577}%
\pgfsetfillcolor{currentfill}%
\pgfsetlinewidth{0.250937pt}%
\definecolor{currentstroke}{rgb}{1.000000,1.000000,1.000000}%
\pgfsetstrokecolor{currentstroke}%
\pgfsetdash{}{0pt}%
\pgfpathmoveto{\pgfqpoint{2.135660in}{8.210661in}}%
\pgfpathlineto{\pgfqpoint{2.223396in}{8.210661in}}%
\pgfpathlineto{\pgfqpoint{2.223396in}{8.122925in}}%
\pgfpathlineto{\pgfqpoint{2.135660in}{8.122925in}}%
\pgfpathlineto{\pgfqpoint{2.135660in}{8.210661in}}%
\pgfusepath{stroke,fill}%
\end{pgfscope}%
\begin{pgfscope}%
\pgfpathrectangle{\pgfqpoint{0.380943in}{8.035189in}}{\pgfqpoint{4.650000in}{0.614151in}}%
\pgfusepath{clip}%
\pgfsetbuttcap%
\pgfsetroundjoin%
\definecolor{currentfill}{rgb}{0.985083,0.974641,0.792587}%
\pgfsetfillcolor{currentfill}%
\pgfsetlinewidth{0.250937pt}%
\definecolor{currentstroke}{rgb}{1.000000,1.000000,1.000000}%
\pgfsetstrokecolor{currentstroke}%
\pgfsetdash{}{0pt}%
\pgfpathmoveto{\pgfqpoint{2.223396in}{8.210661in}}%
\pgfpathlineto{\pgfqpoint{2.311132in}{8.210661in}}%
\pgfpathlineto{\pgfqpoint{2.311132in}{8.122925in}}%
\pgfpathlineto{\pgfqpoint{2.223396in}{8.122925in}}%
\pgfpathlineto{\pgfqpoint{2.223396in}{8.210661in}}%
\pgfusepath{stroke,fill}%
\end{pgfscope}%
\begin{pgfscope}%
\pgfpathrectangle{\pgfqpoint{0.380943in}{8.035189in}}{\pgfqpoint{4.650000in}{0.614151in}}%
\pgfusepath{clip}%
\pgfsetbuttcap%
\pgfsetroundjoin%
\definecolor{currentfill}{rgb}{0.970012,0.883276,0.699577}%
\pgfsetfillcolor{currentfill}%
\pgfsetlinewidth{0.250937pt}%
\definecolor{currentstroke}{rgb}{1.000000,1.000000,1.000000}%
\pgfsetstrokecolor{currentstroke}%
\pgfsetdash{}{0pt}%
\pgfpathmoveto{\pgfqpoint{2.311132in}{8.210661in}}%
\pgfpathlineto{\pgfqpoint{2.398868in}{8.210661in}}%
\pgfpathlineto{\pgfqpoint{2.398868in}{8.122925in}}%
\pgfpathlineto{\pgfqpoint{2.311132in}{8.122925in}}%
\pgfpathlineto{\pgfqpoint{2.311132in}{8.210661in}}%
\pgfusepath{stroke,fill}%
\end{pgfscope}%
\begin{pgfscope}%
\pgfpathrectangle{\pgfqpoint{0.380943in}{8.035189in}}{\pgfqpoint{4.650000in}{0.614151in}}%
\pgfusepath{clip}%
\pgfsetbuttcap%
\pgfsetroundjoin%
\definecolor{currentfill}{rgb}{0.978131,0.843783,0.675709}%
\pgfsetfillcolor{currentfill}%
\pgfsetlinewidth{0.250937pt}%
\definecolor{currentstroke}{rgb}{1.000000,1.000000,1.000000}%
\pgfsetstrokecolor{currentstroke}%
\pgfsetdash{}{0pt}%
\pgfpathmoveto{\pgfqpoint{2.398868in}{8.210661in}}%
\pgfpathlineto{\pgfqpoint{2.486603in}{8.210661in}}%
\pgfpathlineto{\pgfqpoint{2.486603in}{8.122925in}}%
\pgfpathlineto{\pgfqpoint{2.398868in}{8.122925in}}%
\pgfpathlineto{\pgfqpoint{2.398868in}{8.210661in}}%
\pgfusepath{stroke,fill}%
\end{pgfscope}%
\begin{pgfscope}%
\pgfpathrectangle{\pgfqpoint{0.380943in}{8.035189in}}{\pgfqpoint{4.650000in}{0.614151in}}%
\pgfusepath{clip}%
\pgfsetbuttcap%
\pgfsetroundjoin%
\definecolor{currentfill}{rgb}{0.985083,0.974641,0.792587}%
\pgfsetfillcolor{currentfill}%
\pgfsetlinewidth{0.250937pt}%
\definecolor{currentstroke}{rgb}{1.000000,1.000000,1.000000}%
\pgfsetstrokecolor{currentstroke}%
\pgfsetdash{}{0pt}%
\pgfpathmoveto{\pgfqpoint{2.486603in}{8.210661in}}%
\pgfpathlineto{\pgfqpoint{2.574339in}{8.210661in}}%
\pgfpathlineto{\pgfqpoint{2.574339in}{8.122925in}}%
\pgfpathlineto{\pgfqpoint{2.486603in}{8.122925in}}%
\pgfpathlineto{\pgfqpoint{2.486603in}{8.210661in}}%
\pgfusepath{stroke,fill}%
\end{pgfscope}%
\begin{pgfscope}%
\pgfpathrectangle{\pgfqpoint{0.380943in}{8.035189in}}{\pgfqpoint{4.650000in}{0.614151in}}%
\pgfusepath{clip}%
\pgfsetbuttcap%
\pgfsetroundjoin%
\definecolor{currentfill}{rgb}{1.000000,1.000000,0.861745}%
\pgfsetfillcolor{currentfill}%
\pgfsetlinewidth{0.250937pt}%
\definecolor{currentstroke}{rgb}{1.000000,1.000000,1.000000}%
\pgfsetstrokecolor{currentstroke}%
\pgfsetdash{}{0pt}%
\pgfpathmoveto{\pgfqpoint{2.574339in}{8.210661in}}%
\pgfpathlineto{\pgfqpoint{2.662075in}{8.210661in}}%
\pgfpathlineto{\pgfqpoint{2.662075in}{8.122925in}}%
\pgfpathlineto{\pgfqpoint{2.574339in}{8.122925in}}%
\pgfpathlineto{\pgfqpoint{2.574339in}{8.210661in}}%
\pgfusepath{stroke,fill}%
\end{pgfscope}%
\begin{pgfscope}%
\pgfpathrectangle{\pgfqpoint{0.380943in}{8.035189in}}{\pgfqpoint{4.650000in}{0.614151in}}%
\pgfusepath{clip}%
\pgfsetbuttcap%
\pgfsetroundjoin%
\definecolor{currentfill}{rgb}{0.970012,0.883276,0.699577}%
\pgfsetfillcolor{currentfill}%
\pgfsetlinewidth{0.250937pt}%
\definecolor{currentstroke}{rgb}{1.000000,1.000000,1.000000}%
\pgfsetstrokecolor{currentstroke}%
\pgfsetdash{}{0pt}%
\pgfpathmoveto{\pgfqpoint{2.662075in}{8.210661in}}%
\pgfpathlineto{\pgfqpoint{2.749811in}{8.210661in}}%
\pgfpathlineto{\pgfqpoint{2.749811in}{8.122925in}}%
\pgfpathlineto{\pgfqpoint{2.662075in}{8.122925in}}%
\pgfpathlineto{\pgfqpoint{2.662075in}{8.210661in}}%
\pgfusepath{stroke,fill}%
\end{pgfscope}%
\begin{pgfscope}%
\pgfpathrectangle{\pgfqpoint{0.380943in}{8.035189in}}{\pgfqpoint{4.650000in}{0.614151in}}%
\pgfusepath{clip}%
\pgfsetbuttcap%
\pgfsetroundjoin%
\definecolor{currentfill}{rgb}{0.961061,0.931672,0.728304}%
\pgfsetfillcolor{currentfill}%
\pgfsetlinewidth{0.250937pt}%
\definecolor{currentstroke}{rgb}{1.000000,1.000000,1.000000}%
\pgfsetstrokecolor{currentstroke}%
\pgfsetdash{}{0pt}%
\pgfpathmoveto{\pgfqpoint{2.749811in}{8.210661in}}%
\pgfpathlineto{\pgfqpoint{2.837547in}{8.210661in}}%
\pgfpathlineto{\pgfqpoint{2.837547in}{8.122925in}}%
\pgfpathlineto{\pgfqpoint{2.749811in}{8.122925in}}%
\pgfpathlineto{\pgfqpoint{2.749811in}{8.210661in}}%
\pgfusepath{stroke,fill}%
\end{pgfscope}%
\begin{pgfscope}%
\pgfpathrectangle{\pgfqpoint{0.380943in}{8.035189in}}{\pgfqpoint{4.650000in}{0.614151in}}%
\pgfusepath{clip}%
\pgfsetbuttcap%
\pgfsetroundjoin%
\definecolor{currentfill}{rgb}{0.985083,0.974641,0.792587}%
\pgfsetfillcolor{currentfill}%
\pgfsetlinewidth{0.250937pt}%
\definecolor{currentstroke}{rgb}{1.000000,1.000000,1.000000}%
\pgfsetstrokecolor{currentstroke}%
\pgfsetdash{}{0pt}%
\pgfpathmoveto{\pgfqpoint{2.837547in}{8.210661in}}%
\pgfpathlineto{\pgfqpoint{2.925283in}{8.210661in}}%
\pgfpathlineto{\pgfqpoint{2.925283in}{8.122925in}}%
\pgfpathlineto{\pgfqpoint{2.837547in}{8.122925in}}%
\pgfpathlineto{\pgfqpoint{2.837547in}{8.210661in}}%
\pgfusepath{stroke,fill}%
\end{pgfscope}%
\begin{pgfscope}%
\pgfpathrectangle{\pgfqpoint{0.380943in}{8.035189in}}{\pgfqpoint{4.650000in}{0.614151in}}%
\pgfusepath{clip}%
\pgfsetbuttcap%
\pgfsetroundjoin%
\definecolor{currentfill}{rgb}{0.963768,0.915433,0.717478}%
\pgfsetfillcolor{currentfill}%
\pgfsetlinewidth{0.250937pt}%
\definecolor{currentstroke}{rgb}{1.000000,1.000000,1.000000}%
\pgfsetstrokecolor{currentstroke}%
\pgfsetdash{}{0pt}%
\pgfpathmoveto{\pgfqpoint{2.925283in}{8.210661in}}%
\pgfpathlineto{\pgfqpoint{3.013019in}{8.210661in}}%
\pgfpathlineto{\pgfqpoint{3.013019in}{8.122925in}}%
\pgfpathlineto{\pgfqpoint{2.925283in}{8.122925in}}%
\pgfpathlineto{\pgfqpoint{2.925283in}{8.210661in}}%
\pgfusepath{stroke,fill}%
\end{pgfscope}%
\begin{pgfscope}%
\pgfpathrectangle{\pgfqpoint{0.380943in}{8.035189in}}{\pgfqpoint{4.650000in}{0.614151in}}%
\pgfusepath{clip}%
\pgfsetbuttcap%
\pgfsetroundjoin%
\definecolor{currentfill}{rgb}{0.961061,0.931672,0.728304}%
\pgfsetfillcolor{currentfill}%
\pgfsetlinewidth{0.250937pt}%
\definecolor{currentstroke}{rgb}{1.000000,1.000000,1.000000}%
\pgfsetstrokecolor{currentstroke}%
\pgfsetdash{}{0pt}%
\pgfpathmoveto{\pgfqpoint{3.013019in}{8.210661in}}%
\pgfpathlineto{\pgfqpoint{3.100754in}{8.210661in}}%
\pgfpathlineto{\pgfqpoint{3.100754in}{8.122925in}}%
\pgfpathlineto{\pgfqpoint{3.013019in}{8.122925in}}%
\pgfpathlineto{\pgfqpoint{3.013019in}{8.210661in}}%
\pgfusepath{stroke,fill}%
\end{pgfscope}%
\begin{pgfscope}%
\pgfpathrectangle{\pgfqpoint{0.380943in}{8.035189in}}{\pgfqpoint{4.650000in}{0.614151in}}%
\pgfusepath{clip}%
\pgfsetbuttcap%
\pgfsetroundjoin%
\definecolor{currentfill}{rgb}{1.000000,1.000000,0.861745}%
\pgfsetfillcolor{currentfill}%
\pgfsetlinewidth{0.250937pt}%
\definecolor{currentstroke}{rgb}{1.000000,1.000000,1.000000}%
\pgfsetstrokecolor{currentstroke}%
\pgfsetdash{}{0pt}%
\pgfpathmoveto{\pgfqpoint{3.100754in}{8.210661in}}%
\pgfpathlineto{\pgfqpoint{3.188490in}{8.210661in}}%
\pgfpathlineto{\pgfqpoint{3.188490in}{8.122925in}}%
\pgfpathlineto{\pgfqpoint{3.100754in}{8.122925in}}%
\pgfpathlineto{\pgfqpoint{3.100754in}{8.210661in}}%
\pgfusepath{stroke,fill}%
\end{pgfscope}%
\begin{pgfscope}%
\pgfpathrectangle{\pgfqpoint{0.380943in}{8.035189in}}{\pgfqpoint{4.650000in}{0.614151in}}%
\pgfusepath{clip}%
\pgfsetbuttcap%
\pgfsetroundjoin%
\definecolor{currentfill}{rgb}{0.963768,0.915433,0.717478}%
\pgfsetfillcolor{currentfill}%
\pgfsetlinewidth{0.250937pt}%
\definecolor{currentstroke}{rgb}{1.000000,1.000000,1.000000}%
\pgfsetstrokecolor{currentstroke}%
\pgfsetdash{}{0pt}%
\pgfpathmoveto{\pgfqpoint{3.188490in}{8.210661in}}%
\pgfpathlineto{\pgfqpoint{3.276226in}{8.210661in}}%
\pgfpathlineto{\pgfqpoint{3.276226in}{8.122925in}}%
\pgfpathlineto{\pgfqpoint{3.188490in}{8.122925in}}%
\pgfpathlineto{\pgfqpoint{3.188490in}{8.210661in}}%
\pgfusepath{stroke,fill}%
\end{pgfscope}%
\begin{pgfscope}%
\pgfpathrectangle{\pgfqpoint{0.380943in}{8.035189in}}{\pgfqpoint{4.650000in}{0.614151in}}%
\pgfusepath{clip}%
\pgfsetbuttcap%
\pgfsetroundjoin%
\definecolor{currentfill}{rgb}{0.961061,0.931672,0.728304}%
\pgfsetfillcolor{currentfill}%
\pgfsetlinewidth{0.250937pt}%
\definecolor{currentstroke}{rgb}{1.000000,1.000000,1.000000}%
\pgfsetstrokecolor{currentstroke}%
\pgfsetdash{}{0pt}%
\pgfpathmoveto{\pgfqpoint{3.276226in}{8.210661in}}%
\pgfpathlineto{\pgfqpoint{3.363962in}{8.210661in}}%
\pgfpathlineto{\pgfqpoint{3.363962in}{8.122925in}}%
\pgfpathlineto{\pgfqpoint{3.276226in}{8.122925in}}%
\pgfpathlineto{\pgfqpoint{3.276226in}{8.210661in}}%
\pgfusepath{stroke,fill}%
\end{pgfscope}%
\begin{pgfscope}%
\pgfpathrectangle{\pgfqpoint{0.380943in}{8.035189in}}{\pgfqpoint{4.650000in}{0.614151in}}%
\pgfusepath{clip}%
\pgfsetbuttcap%
\pgfsetroundjoin%
\definecolor{currentfill}{rgb}{1.000000,1.000000,0.861745}%
\pgfsetfillcolor{currentfill}%
\pgfsetlinewidth{0.250937pt}%
\definecolor{currentstroke}{rgb}{1.000000,1.000000,1.000000}%
\pgfsetstrokecolor{currentstroke}%
\pgfsetdash{}{0pt}%
\pgfpathmoveto{\pgfqpoint{3.363962in}{8.210661in}}%
\pgfpathlineto{\pgfqpoint{3.451698in}{8.210661in}}%
\pgfpathlineto{\pgfqpoint{3.451698in}{8.122925in}}%
\pgfpathlineto{\pgfqpoint{3.363962in}{8.122925in}}%
\pgfpathlineto{\pgfqpoint{3.363962in}{8.210661in}}%
\pgfusepath{stroke,fill}%
\end{pgfscope}%
\begin{pgfscope}%
\pgfpathrectangle{\pgfqpoint{0.380943in}{8.035189in}}{\pgfqpoint{4.650000in}{0.614151in}}%
\pgfusepath{clip}%
\pgfsetbuttcap%
\pgfsetroundjoin%
\definecolor{currentfill}{rgb}{1.000000,1.000000,0.861745}%
\pgfsetfillcolor{currentfill}%
\pgfsetlinewidth{0.250937pt}%
\definecolor{currentstroke}{rgb}{1.000000,1.000000,1.000000}%
\pgfsetstrokecolor{currentstroke}%
\pgfsetdash{}{0pt}%
\pgfpathmoveto{\pgfqpoint{3.451698in}{8.210661in}}%
\pgfpathlineto{\pgfqpoint{3.539434in}{8.210661in}}%
\pgfpathlineto{\pgfqpoint{3.539434in}{8.122925in}}%
\pgfpathlineto{\pgfqpoint{3.451698in}{8.122925in}}%
\pgfpathlineto{\pgfqpoint{3.451698in}{8.210661in}}%
\pgfusepath{stroke,fill}%
\end{pgfscope}%
\begin{pgfscope}%
\pgfpathrectangle{\pgfqpoint{0.380943in}{8.035189in}}{\pgfqpoint{4.650000in}{0.614151in}}%
\pgfusepath{clip}%
\pgfsetbuttcap%
\pgfsetroundjoin%
\definecolor{currentfill}{rgb}{0.961061,0.931672,0.728304}%
\pgfsetfillcolor{currentfill}%
\pgfsetlinewidth{0.250937pt}%
\definecolor{currentstroke}{rgb}{1.000000,1.000000,1.000000}%
\pgfsetstrokecolor{currentstroke}%
\pgfsetdash{}{0pt}%
\pgfpathmoveto{\pgfqpoint{3.539434in}{8.210661in}}%
\pgfpathlineto{\pgfqpoint{3.627169in}{8.210661in}}%
\pgfpathlineto{\pgfqpoint{3.627169in}{8.122925in}}%
\pgfpathlineto{\pgfqpoint{3.539434in}{8.122925in}}%
\pgfpathlineto{\pgfqpoint{3.539434in}{8.210661in}}%
\pgfusepath{stroke,fill}%
\end{pgfscope}%
\begin{pgfscope}%
\pgfpathrectangle{\pgfqpoint{0.380943in}{8.035189in}}{\pgfqpoint{4.650000in}{0.614151in}}%
\pgfusepath{clip}%
\pgfsetbuttcap%
\pgfsetroundjoin%
\definecolor{currentfill}{rgb}{0.978131,0.843783,0.675709}%
\pgfsetfillcolor{currentfill}%
\pgfsetlinewidth{0.250937pt}%
\definecolor{currentstroke}{rgb}{1.000000,1.000000,1.000000}%
\pgfsetstrokecolor{currentstroke}%
\pgfsetdash{}{0pt}%
\pgfpathmoveto{\pgfqpoint{3.627169in}{8.210661in}}%
\pgfpathlineto{\pgfqpoint{3.714905in}{8.210661in}}%
\pgfpathlineto{\pgfqpoint{3.714905in}{8.122925in}}%
\pgfpathlineto{\pgfqpoint{3.627169in}{8.122925in}}%
\pgfpathlineto{\pgfqpoint{3.627169in}{8.210661in}}%
\pgfusepath{stroke,fill}%
\end{pgfscope}%
\begin{pgfscope}%
\pgfpathrectangle{\pgfqpoint{0.380943in}{8.035189in}}{\pgfqpoint{4.650000in}{0.614151in}}%
\pgfusepath{clip}%
\pgfsetbuttcap%
\pgfsetroundjoin%
\definecolor{currentfill}{rgb}{0.963768,0.915433,0.717478}%
\pgfsetfillcolor{currentfill}%
\pgfsetlinewidth{0.250937pt}%
\definecolor{currentstroke}{rgb}{1.000000,1.000000,1.000000}%
\pgfsetstrokecolor{currentstroke}%
\pgfsetdash{}{0pt}%
\pgfpathmoveto{\pgfqpoint{3.714905in}{8.210661in}}%
\pgfpathlineto{\pgfqpoint{3.802641in}{8.210661in}}%
\pgfpathlineto{\pgfqpoint{3.802641in}{8.122925in}}%
\pgfpathlineto{\pgfqpoint{3.714905in}{8.122925in}}%
\pgfpathlineto{\pgfqpoint{3.714905in}{8.210661in}}%
\pgfusepath{stroke,fill}%
\end{pgfscope}%
\begin{pgfscope}%
\pgfpathrectangle{\pgfqpoint{0.380943in}{8.035189in}}{\pgfqpoint{4.650000in}{0.614151in}}%
\pgfusepath{clip}%
\pgfsetbuttcap%
\pgfsetroundjoin%
\definecolor{currentfill}{rgb}{0.985083,0.974641,0.792587}%
\pgfsetfillcolor{currentfill}%
\pgfsetlinewidth{0.250937pt}%
\definecolor{currentstroke}{rgb}{1.000000,1.000000,1.000000}%
\pgfsetstrokecolor{currentstroke}%
\pgfsetdash{}{0pt}%
\pgfpathmoveto{\pgfqpoint{3.802641in}{8.210661in}}%
\pgfpathlineto{\pgfqpoint{3.890377in}{8.210661in}}%
\pgfpathlineto{\pgfqpoint{3.890377in}{8.122925in}}%
\pgfpathlineto{\pgfqpoint{3.802641in}{8.122925in}}%
\pgfpathlineto{\pgfqpoint{3.802641in}{8.210661in}}%
\pgfusepath{stroke,fill}%
\end{pgfscope}%
\begin{pgfscope}%
\pgfpathrectangle{\pgfqpoint{0.380943in}{8.035189in}}{\pgfqpoint{4.650000in}{0.614151in}}%
\pgfusepath{clip}%
\pgfsetbuttcap%
\pgfsetroundjoin%
\definecolor{currentfill}{rgb}{0.985083,0.974641,0.792587}%
\pgfsetfillcolor{currentfill}%
\pgfsetlinewidth{0.250937pt}%
\definecolor{currentstroke}{rgb}{1.000000,1.000000,1.000000}%
\pgfsetstrokecolor{currentstroke}%
\pgfsetdash{}{0pt}%
\pgfpathmoveto{\pgfqpoint{3.890377in}{8.210661in}}%
\pgfpathlineto{\pgfqpoint{3.978113in}{8.210661in}}%
\pgfpathlineto{\pgfqpoint{3.978113in}{8.122925in}}%
\pgfpathlineto{\pgfqpoint{3.890377in}{8.122925in}}%
\pgfpathlineto{\pgfqpoint{3.890377in}{8.210661in}}%
\pgfusepath{stroke,fill}%
\end{pgfscope}%
\begin{pgfscope}%
\pgfpathrectangle{\pgfqpoint{0.380943in}{8.035189in}}{\pgfqpoint{4.650000in}{0.614151in}}%
\pgfusepath{clip}%
\pgfsetbuttcap%
\pgfsetroundjoin%
\definecolor{currentfill}{rgb}{0.978131,0.843783,0.675709}%
\pgfsetfillcolor{currentfill}%
\pgfsetlinewidth{0.250937pt}%
\definecolor{currentstroke}{rgb}{1.000000,1.000000,1.000000}%
\pgfsetstrokecolor{currentstroke}%
\pgfsetdash{}{0pt}%
\pgfpathmoveto{\pgfqpoint{3.978113in}{8.210661in}}%
\pgfpathlineto{\pgfqpoint{4.065849in}{8.210661in}}%
\pgfpathlineto{\pgfqpoint{4.065849in}{8.122925in}}%
\pgfpathlineto{\pgfqpoint{3.978113in}{8.122925in}}%
\pgfpathlineto{\pgfqpoint{3.978113in}{8.210661in}}%
\pgfusepath{stroke,fill}%
\end{pgfscope}%
\begin{pgfscope}%
\pgfpathrectangle{\pgfqpoint{0.380943in}{8.035189in}}{\pgfqpoint{4.650000in}{0.614151in}}%
\pgfusepath{clip}%
\pgfsetbuttcap%
\pgfsetroundjoin%
\definecolor{currentfill}{rgb}{0.985083,0.974641,0.792587}%
\pgfsetfillcolor{currentfill}%
\pgfsetlinewidth{0.250937pt}%
\definecolor{currentstroke}{rgb}{1.000000,1.000000,1.000000}%
\pgfsetstrokecolor{currentstroke}%
\pgfsetdash{}{0pt}%
\pgfpathmoveto{\pgfqpoint{4.065849in}{8.210661in}}%
\pgfpathlineto{\pgfqpoint{4.153585in}{8.210661in}}%
\pgfpathlineto{\pgfqpoint{4.153585in}{8.122925in}}%
\pgfpathlineto{\pgfqpoint{4.065849in}{8.122925in}}%
\pgfpathlineto{\pgfqpoint{4.065849in}{8.210661in}}%
\pgfusepath{stroke,fill}%
\end{pgfscope}%
\begin{pgfscope}%
\pgfpathrectangle{\pgfqpoint{0.380943in}{8.035189in}}{\pgfqpoint{4.650000in}{0.614151in}}%
\pgfusepath{clip}%
\pgfsetbuttcap%
\pgfsetroundjoin%
\definecolor{currentfill}{rgb}{0.970012,0.883276,0.699577}%
\pgfsetfillcolor{currentfill}%
\pgfsetlinewidth{0.250937pt}%
\definecolor{currentstroke}{rgb}{1.000000,1.000000,1.000000}%
\pgfsetstrokecolor{currentstroke}%
\pgfsetdash{}{0pt}%
\pgfpathmoveto{\pgfqpoint{4.153585in}{8.210661in}}%
\pgfpathlineto{\pgfqpoint{4.241320in}{8.210661in}}%
\pgfpathlineto{\pgfqpoint{4.241320in}{8.122925in}}%
\pgfpathlineto{\pgfqpoint{4.153585in}{8.122925in}}%
\pgfpathlineto{\pgfqpoint{4.153585in}{8.210661in}}%
\pgfusepath{stroke,fill}%
\end{pgfscope}%
\begin{pgfscope}%
\pgfpathrectangle{\pgfqpoint{0.380943in}{8.035189in}}{\pgfqpoint{4.650000in}{0.614151in}}%
\pgfusepath{clip}%
\pgfsetbuttcap%
\pgfsetroundjoin%
\definecolor{currentfill}{rgb}{0.963768,0.915433,0.717478}%
\pgfsetfillcolor{currentfill}%
\pgfsetlinewidth{0.250937pt}%
\definecolor{currentstroke}{rgb}{1.000000,1.000000,1.000000}%
\pgfsetstrokecolor{currentstroke}%
\pgfsetdash{}{0pt}%
\pgfpathmoveto{\pgfqpoint{4.241320in}{8.210661in}}%
\pgfpathlineto{\pgfqpoint{4.329056in}{8.210661in}}%
\pgfpathlineto{\pgfqpoint{4.329056in}{8.122925in}}%
\pgfpathlineto{\pgfqpoint{4.241320in}{8.122925in}}%
\pgfpathlineto{\pgfqpoint{4.241320in}{8.210661in}}%
\pgfusepath{stroke,fill}%
\end{pgfscope}%
\begin{pgfscope}%
\pgfpathrectangle{\pgfqpoint{0.380943in}{8.035189in}}{\pgfqpoint{4.650000in}{0.614151in}}%
\pgfusepath{clip}%
\pgfsetbuttcap%
\pgfsetroundjoin%
\definecolor{currentfill}{rgb}{0.985083,0.974641,0.792587}%
\pgfsetfillcolor{currentfill}%
\pgfsetlinewidth{0.250937pt}%
\definecolor{currentstroke}{rgb}{1.000000,1.000000,1.000000}%
\pgfsetstrokecolor{currentstroke}%
\pgfsetdash{}{0pt}%
\pgfpathmoveto{\pgfqpoint{4.329056in}{8.210661in}}%
\pgfpathlineto{\pgfqpoint{4.416792in}{8.210661in}}%
\pgfpathlineto{\pgfqpoint{4.416792in}{8.122925in}}%
\pgfpathlineto{\pgfqpoint{4.329056in}{8.122925in}}%
\pgfpathlineto{\pgfqpoint{4.329056in}{8.210661in}}%
\pgfusepath{stroke,fill}%
\end{pgfscope}%
\begin{pgfscope}%
\pgfpathrectangle{\pgfqpoint{0.380943in}{8.035189in}}{\pgfqpoint{4.650000in}{0.614151in}}%
\pgfusepath{clip}%
\pgfsetbuttcap%
\pgfsetroundjoin%
\definecolor{currentfill}{rgb}{0.963768,0.915433,0.717478}%
\pgfsetfillcolor{currentfill}%
\pgfsetlinewidth{0.250937pt}%
\definecolor{currentstroke}{rgb}{1.000000,1.000000,1.000000}%
\pgfsetstrokecolor{currentstroke}%
\pgfsetdash{}{0pt}%
\pgfpathmoveto{\pgfqpoint{4.416792in}{8.210661in}}%
\pgfpathlineto{\pgfqpoint{4.504528in}{8.210661in}}%
\pgfpathlineto{\pgfqpoint{4.504528in}{8.122925in}}%
\pgfpathlineto{\pgfqpoint{4.416792in}{8.122925in}}%
\pgfpathlineto{\pgfqpoint{4.416792in}{8.210661in}}%
\pgfusepath{stroke,fill}%
\end{pgfscope}%
\begin{pgfscope}%
\pgfpathrectangle{\pgfqpoint{0.380943in}{8.035189in}}{\pgfqpoint{4.650000in}{0.614151in}}%
\pgfusepath{clip}%
\pgfsetbuttcap%
\pgfsetroundjoin%
\definecolor{currentfill}{rgb}{0.961061,0.931672,0.728304}%
\pgfsetfillcolor{currentfill}%
\pgfsetlinewidth{0.250937pt}%
\definecolor{currentstroke}{rgb}{1.000000,1.000000,1.000000}%
\pgfsetstrokecolor{currentstroke}%
\pgfsetdash{}{0pt}%
\pgfpathmoveto{\pgfqpoint{4.504528in}{8.210661in}}%
\pgfpathlineto{\pgfqpoint{4.592264in}{8.210661in}}%
\pgfpathlineto{\pgfqpoint{4.592264in}{8.122925in}}%
\pgfpathlineto{\pgfqpoint{4.504528in}{8.122925in}}%
\pgfpathlineto{\pgfqpoint{4.504528in}{8.210661in}}%
\pgfusepath{stroke,fill}%
\end{pgfscope}%
\begin{pgfscope}%
\pgfpathrectangle{\pgfqpoint{0.380943in}{8.035189in}}{\pgfqpoint{4.650000in}{0.614151in}}%
\pgfusepath{clip}%
\pgfsetbuttcap%
\pgfsetroundjoin%
\definecolor{currentfill}{rgb}{0.985083,0.974641,0.792587}%
\pgfsetfillcolor{currentfill}%
\pgfsetlinewidth{0.250937pt}%
\definecolor{currentstroke}{rgb}{1.000000,1.000000,1.000000}%
\pgfsetstrokecolor{currentstroke}%
\pgfsetdash{}{0pt}%
\pgfpathmoveto{\pgfqpoint{4.592264in}{8.210661in}}%
\pgfpathlineto{\pgfqpoint{4.680000in}{8.210661in}}%
\pgfpathlineto{\pgfqpoint{4.680000in}{8.122925in}}%
\pgfpathlineto{\pgfqpoint{4.592264in}{8.122925in}}%
\pgfpathlineto{\pgfqpoint{4.592264in}{8.210661in}}%
\pgfusepath{stroke,fill}%
\end{pgfscope}%
\begin{pgfscope}%
\pgfpathrectangle{\pgfqpoint{0.380943in}{8.035189in}}{\pgfqpoint{4.650000in}{0.614151in}}%
\pgfusepath{clip}%
\pgfsetbuttcap%
\pgfsetroundjoin%
\definecolor{currentfill}{rgb}{0.970012,0.883276,0.699577}%
\pgfsetfillcolor{currentfill}%
\pgfsetlinewidth{0.250937pt}%
\definecolor{currentstroke}{rgb}{1.000000,1.000000,1.000000}%
\pgfsetstrokecolor{currentstroke}%
\pgfsetdash{}{0pt}%
\pgfpathmoveto{\pgfqpoint{4.680000in}{8.210661in}}%
\pgfpathlineto{\pgfqpoint{4.767736in}{8.210661in}}%
\pgfpathlineto{\pgfqpoint{4.767736in}{8.122925in}}%
\pgfpathlineto{\pgfqpoint{4.680000in}{8.122925in}}%
\pgfpathlineto{\pgfqpoint{4.680000in}{8.210661in}}%
\pgfusepath{stroke,fill}%
\end{pgfscope}%
\begin{pgfscope}%
\pgfpathrectangle{\pgfqpoint{0.380943in}{8.035189in}}{\pgfqpoint{4.650000in}{0.614151in}}%
\pgfusepath{clip}%
\pgfsetbuttcap%
\pgfsetroundjoin%
\definecolor{currentfill}{rgb}{0.970012,0.883276,0.699577}%
\pgfsetfillcolor{currentfill}%
\pgfsetlinewidth{0.250937pt}%
\definecolor{currentstroke}{rgb}{1.000000,1.000000,1.000000}%
\pgfsetstrokecolor{currentstroke}%
\pgfsetdash{}{0pt}%
\pgfpathmoveto{\pgfqpoint{4.767736in}{8.210661in}}%
\pgfpathlineto{\pgfqpoint{4.855471in}{8.210661in}}%
\pgfpathlineto{\pgfqpoint{4.855471in}{8.122925in}}%
\pgfpathlineto{\pgfqpoint{4.767736in}{8.122925in}}%
\pgfpathlineto{\pgfqpoint{4.767736in}{8.210661in}}%
\pgfusepath{stroke,fill}%
\end{pgfscope}%
\begin{pgfscope}%
\pgfpathrectangle{\pgfqpoint{0.380943in}{8.035189in}}{\pgfqpoint{4.650000in}{0.614151in}}%
\pgfusepath{clip}%
\pgfsetbuttcap%
\pgfsetroundjoin%
\definecolor{currentfill}{rgb}{0.963768,0.915433,0.717478}%
\pgfsetfillcolor{currentfill}%
\pgfsetlinewidth{0.250937pt}%
\definecolor{currentstroke}{rgb}{1.000000,1.000000,1.000000}%
\pgfsetstrokecolor{currentstroke}%
\pgfsetdash{}{0pt}%
\pgfpathmoveto{\pgfqpoint{4.855471in}{8.210661in}}%
\pgfpathlineto{\pgfqpoint{4.943207in}{8.210661in}}%
\pgfpathlineto{\pgfqpoint{4.943207in}{8.122925in}}%
\pgfpathlineto{\pgfqpoint{4.855471in}{8.122925in}}%
\pgfpathlineto{\pgfqpoint{4.855471in}{8.210661in}}%
\pgfusepath{stroke,fill}%
\end{pgfscope}%
\begin{pgfscope}%
\pgfpathrectangle{\pgfqpoint{0.380943in}{8.035189in}}{\pgfqpoint{4.650000in}{0.614151in}}%
\pgfusepath{clip}%
\pgfsetbuttcap%
\pgfsetroundjoin%
\definecolor{currentfill}{rgb}{0.970012,0.883276,0.699577}%
\pgfsetfillcolor{currentfill}%
\pgfsetlinewidth{0.250937pt}%
\definecolor{currentstroke}{rgb}{1.000000,1.000000,1.000000}%
\pgfsetstrokecolor{currentstroke}%
\pgfsetdash{}{0pt}%
\pgfpathmoveto{\pgfqpoint{4.943207in}{8.210661in}}%
\pgfpathlineto{\pgfqpoint{5.030943in}{8.210661in}}%
\pgfpathlineto{\pgfqpoint{5.030943in}{8.122925in}}%
\pgfpathlineto{\pgfqpoint{4.943207in}{8.122925in}}%
\pgfpathlineto{\pgfqpoint{4.943207in}{8.210661in}}%
\pgfusepath{stroke,fill}%
\end{pgfscope}%
\begin{pgfscope}%
\pgfpathrectangle{\pgfqpoint{0.380943in}{8.035189in}}{\pgfqpoint{4.650000in}{0.614151in}}%
\pgfusepath{clip}%
\pgfsetbuttcap%
\pgfsetroundjoin%
\definecolor{currentfill}{rgb}{0.970012,0.883276,0.699577}%
\pgfsetfillcolor{currentfill}%
\pgfsetlinewidth{0.250937pt}%
\definecolor{currentstroke}{rgb}{1.000000,1.000000,1.000000}%
\pgfsetstrokecolor{currentstroke}%
\pgfsetdash{}{0pt}%
\pgfpathmoveto{\pgfqpoint{0.380943in}{8.122925in}}%
\pgfpathlineto{\pgfqpoint{0.468679in}{8.122925in}}%
\pgfpathlineto{\pgfqpoint{0.468679in}{8.035189in}}%
\pgfpathlineto{\pgfqpoint{0.380943in}{8.035189in}}%
\pgfpathlineto{\pgfqpoint{0.380943in}{8.122925in}}%
\pgfusepath{stroke,fill}%
\end{pgfscope}%
\begin{pgfscope}%
\pgfpathrectangle{\pgfqpoint{0.380943in}{8.035189in}}{\pgfqpoint{4.650000in}{0.614151in}}%
\pgfusepath{clip}%
\pgfsetbuttcap%
\pgfsetroundjoin%
\definecolor{currentfill}{rgb}{0.985083,0.974641,0.792587}%
\pgfsetfillcolor{currentfill}%
\pgfsetlinewidth{0.250937pt}%
\definecolor{currentstroke}{rgb}{1.000000,1.000000,1.000000}%
\pgfsetstrokecolor{currentstroke}%
\pgfsetdash{}{0pt}%
\pgfpathmoveto{\pgfqpoint{0.468679in}{8.122925in}}%
\pgfpathlineto{\pgfqpoint{0.556415in}{8.122925in}}%
\pgfpathlineto{\pgfqpoint{0.556415in}{8.035189in}}%
\pgfpathlineto{\pgfqpoint{0.468679in}{8.035189in}}%
\pgfpathlineto{\pgfqpoint{0.468679in}{8.122925in}}%
\pgfusepath{stroke,fill}%
\end{pgfscope}%
\begin{pgfscope}%
\pgfpathrectangle{\pgfqpoint{0.380943in}{8.035189in}}{\pgfqpoint{4.650000in}{0.614151in}}%
\pgfusepath{clip}%
\pgfsetbuttcap%
\pgfsetroundjoin%
\definecolor{currentfill}{rgb}{0.985083,0.974641,0.792587}%
\pgfsetfillcolor{currentfill}%
\pgfsetlinewidth{0.250937pt}%
\definecolor{currentstroke}{rgb}{1.000000,1.000000,1.000000}%
\pgfsetstrokecolor{currentstroke}%
\pgfsetdash{}{0pt}%
\pgfpathmoveto{\pgfqpoint{0.556415in}{8.122925in}}%
\pgfpathlineto{\pgfqpoint{0.644151in}{8.122925in}}%
\pgfpathlineto{\pgfqpoint{0.644151in}{8.035189in}}%
\pgfpathlineto{\pgfqpoint{0.556415in}{8.035189in}}%
\pgfpathlineto{\pgfqpoint{0.556415in}{8.122925in}}%
\pgfusepath{stroke,fill}%
\end{pgfscope}%
\begin{pgfscope}%
\pgfpathrectangle{\pgfqpoint{0.380943in}{8.035189in}}{\pgfqpoint{4.650000in}{0.614151in}}%
\pgfusepath{clip}%
\pgfsetbuttcap%
\pgfsetroundjoin%
\definecolor{currentfill}{rgb}{0.963768,0.915433,0.717478}%
\pgfsetfillcolor{currentfill}%
\pgfsetlinewidth{0.250937pt}%
\definecolor{currentstroke}{rgb}{1.000000,1.000000,1.000000}%
\pgfsetstrokecolor{currentstroke}%
\pgfsetdash{}{0pt}%
\pgfpathmoveto{\pgfqpoint{0.644151in}{8.122925in}}%
\pgfpathlineto{\pgfqpoint{0.731886in}{8.122925in}}%
\pgfpathlineto{\pgfqpoint{0.731886in}{8.035189in}}%
\pgfpathlineto{\pgfqpoint{0.644151in}{8.035189in}}%
\pgfpathlineto{\pgfqpoint{0.644151in}{8.122925in}}%
\pgfusepath{stroke,fill}%
\end{pgfscope}%
\begin{pgfscope}%
\pgfpathrectangle{\pgfqpoint{0.380943in}{8.035189in}}{\pgfqpoint{4.650000in}{0.614151in}}%
\pgfusepath{clip}%
\pgfsetbuttcap%
\pgfsetroundjoin%
\definecolor{currentfill}{rgb}{0.970012,0.883276,0.699577}%
\pgfsetfillcolor{currentfill}%
\pgfsetlinewidth{0.250937pt}%
\definecolor{currentstroke}{rgb}{1.000000,1.000000,1.000000}%
\pgfsetstrokecolor{currentstroke}%
\pgfsetdash{}{0pt}%
\pgfpathmoveto{\pgfqpoint{0.731886in}{8.122925in}}%
\pgfpathlineto{\pgfqpoint{0.819622in}{8.122925in}}%
\pgfpathlineto{\pgfqpoint{0.819622in}{8.035189in}}%
\pgfpathlineto{\pgfqpoint{0.731886in}{8.035189in}}%
\pgfpathlineto{\pgfqpoint{0.731886in}{8.122925in}}%
\pgfusepath{stroke,fill}%
\end{pgfscope}%
\begin{pgfscope}%
\pgfpathrectangle{\pgfqpoint{0.380943in}{8.035189in}}{\pgfqpoint{4.650000in}{0.614151in}}%
\pgfusepath{clip}%
\pgfsetbuttcap%
\pgfsetroundjoin%
\definecolor{currentfill}{rgb}{1.000000,1.000000,0.861745}%
\pgfsetfillcolor{currentfill}%
\pgfsetlinewidth{0.250937pt}%
\definecolor{currentstroke}{rgb}{1.000000,1.000000,1.000000}%
\pgfsetstrokecolor{currentstroke}%
\pgfsetdash{}{0pt}%
\pgfpathmoveto{\pgfqpoint{0.819622in}{8.122925in}}%
\pgfpathlineto{\pgfqpoint{0.907358in}{8.122925in}}%
\pgfpathlineto{\pgfqpoint{0.907358in}{8.035189in}}%
\pgfpathlineto{\pgfqpoint{0.819622in}{8.035189in}}%
\pgfpathlineto{\pgfqpoint{0.819622in}{8.122925in}}%
\pgfusepath{stroke,fill}%
\end{pgfscope}%
\begin{pgfscope}%
\pgfpathrectangle{\pgfqpoint{0.380943in}{8.035189in}}{\pgfqpoint{4.650000in}{0.614151in}}%
\pgfusepath{clip}%
\pgfsetbuttcap%
\pgfsetroundjoin%
\definecolor{currentfill}{rgb}{0.970012,0.883276,0.699577}%
\pgfsetfillcolor{currentfill}%
\pgfsetlinewidth{0.250937pt}%
\definecolor{currentstroke}{rgb}{1.000000,1.000000,1.000000}%
\pgfsetstrokecolor{currentstroke}%
\pgfsetdash{}{0pt}%
\pgfpathmoveto{\pgfqpoint{0.907358in}{8.122925in}}%
\pgfpathlineto{\pgfqpoint{0.995094in}{8.122925in}}%
\pgfpathlineto{\pgfqpoint{0.995094in}{8.035189in}}%
\pgfpathlineto{\pgfqpoint{0.907358in}{8.035189in}}%
\pgfpathlineto{\pgfqpoint{0.907358in}{8.122925in}}%
\pgfusepath{stroke,fill}%
\end{pgfscope}%
\begin{pgfscope}%
\pgfpathrectangle{\pgfqpoint{0.380943in}{8.035189in}}{\pgfqpoint{4.650000in}{0.614151in}}%
\pgfusepath{clip}%
\pgfsetbuttcap%
\pgfsetroundjoin%
\definecolor{currentfill}{rgb}{0.985083,0.974641,0.792587}%
\pgfsetfillcolor{currentfill}%
\pgfsetlinewidth{0.250937pt}%
\definecolor{currentstroke}{rgb}{1.000000,1.000000,1.000000}%
\pgfsetstrokecolor{currentstroke}%
\pgfsetdash{}{0pt}%
\pgfpathmoveto{\pgfqpoint{0.995094in}{8.122925in}}%
\pgfpathlineto{\pgfqpoint{1.082830in}{8.122925in}}%
\pgfpathlineto{\pgfqpoint{1.082830in}{8.035189in}}%
\pgfpathlineto{\pgfqpoint{0.995094in}{8.035189in}}%
\pgfpathlineto{\pgfqpoint{0.995094in}{8.122925in}}%
\pgfusepath{stroke,fill}%
\end{pgfscope}%
\begin{pgfscope}%
\pgfpathrectangle{\pgfqpoint{0.380943in}{8.035189in}}{\pgfqpoint{4.650000in}{0.614151in}}%
\pgfusepath{clip}%
\pgfsetbuttcap%
\pgfsetroundjoin%
\definecolor{currentfill}{rgb}{0.978131,0.843783,0.675709}%
\pgfsetfillcolor{currentfill}%
\pgfsetlinewidth{0.250937pt}%
\definecolor{currentstroke}{rgb}{1.000000,1.000000,1.000000}%
\pgfsetstrokecolor{currentstroke}%
\pgfsetdash{}{0pt}%
\pgfpathmoveto{\pgfqpoint{1.082830in}{8.122925in}}%
\pgfpathlineto{\pgfqpoint{1.170566in}{8.122925in}}%
\pgfpathlineto{\pgfqpoint{1.170566in}{8.035189in}}%
\pgfpathlineto{\pgfqpoint{1.082830in}{8.035189in}}%
\pgfpathlineto{\pgfqpoint{1.082830in}{8.122925in}}%
\pgfusepath{stroke,fill}%
\end{pgfscope}%
\begin{pgfscope}%
\pgfpathrectangle{\pgfqpoint{0.380943in}{8.035189in}}{\pgfqpoint{4.650000in}{0.614151in}}%
\pgfusepath{clip}%
\pgfsetbuttcap%
\pgfsetroundjoin%
\definecolor{currentfill}{rgb}{0.961061,0.931672,0.728304}%
\pgfsetfillcolor{currentfill}%
\pgfsetlinewidth{0.250937pt}%
\definecolor{currentstroke}{rgb}{1.000000,1.000000,1.000000}%
\pgfsetstrokecolor{currentstroke}%
\pgfsetdash{}{0pt}%
\pgfpathmoveto{\pgfqpoint{1.170566in}{8.122925in}}%
\pgfpathlineto{\pgfqpoint{1.258302in}{8.122925in}}%
\pgfpathlineto{\pgfqpoint{1.258302in}{8.035189in}}%
\pgfpathlineto{\pgfqpoint{1.170566in}{8.035189in}}%
\pgfpathlineto{\pgfqpoint{1.170566in}{8.122925in}}%
\pgfusepath{stroke,fill}%
\end{pgfscope}%
\begin{pgfscope}%
\pgfpathrectangle{\pgfqpoint{0.380943in}{8.035189in}}{\pgfqpoint{4.650000in}{0.614151in}}%
\pgfusepath{clip}%
\pgfsetbuttcap%
\pgfsetroundjoin%
\definecolor{currentfill}{rgb}{0.961061,0.931672,0.728304}%
\pgfsetfillcolor{currentfill}%
\pgfsetlinewidth{0.250937pt}%
\definecolor{currentstroke}{rgb}{1.000000,1.000000,1.000000}%
\pgfsetstrokecolor{currentstroke}%
\pgfsetdash{}{0pt}%
\pgfpathmoveto{\pgfqpoint{1.258302in}{8.122925in}}%
\pgfpathlineto{\pgfqpoint{1.346037in}{8.122925in}}%
\pgfpathlineto{\pgfqpoint{1.346037in}{8.035189in}}%
\pgfpathlineto{\pgfqpoint{1.258302in}{8.035189in}}%
\pgfpathlineto{\pgfqpoint{1.258302in}{8.122925in}}%
\pgfusepath{stroke,fill}%
\end{pgfscope}%
\begin{pgfscope}%
\pgfpathrectangle{\pgfqpoint{0.380943in}{8.035189in}}{\pgfqpoint{4.650000in}{0.614151in}}%
\pgfusepath{clip}%
\pgfsetbuttcap%
\pgfsetroundjoin%
\definecolor{currentfill}{rgb}{0.963768,0.915433,0.717478}%
\pgfsetfillcolor{currentfill}%
\pgfsetlinewidth{0.250937pt}%
\definecolor{currentstroke}{rgb}{1.000000,1.000000,1.000000}%
\pgfsetstrokecolor{currentstroke}%
\pgfsetdash{}{0pt}%
\pgfpathmoveto{\pgfqpoint{1.346037in}{8.122925in}}%
\pgfpathlineto{\pgfqpoint{1.433773in}{8.122925in}}%
\pgfpathlineto{\pgfqpoint{1.433773in}{8.035189in}}%
\pgfpathlineto{\pgfqpoint{1.346037in}{8.035189in}}%
\pgfpathlineto{\pgfqpoint{1.346037in}{8.122925in}}%
\pgfusepath{stroke,fill}%
\end{pgfscope}%
\begin{pgfscope}%
\pgfpathrectangle{\pgfqpoint{0.380943in}{8.035189in}}{\pgfqpoint{4.650000in}{0.614151in}}%
\pgfusepath{clip}%
\pgfsetbuttcap%
\pgfsetroundjoin%
\definecolor{currentfill}{rgb}{0.961061,0.931672,0.728304}%
\pgfsetfillcolor{currentfill}%
\pgfsetlinewidth{0.250937pt}%
\definecolor{currentstroke}{rgb}{1.000000,1.000000,1.000000}%
\pgfsetstrokecolor{currentstroke}%
\pgfsetdash{}{0pt}%
\pgfpathmoveto{\pgfqpoint{1.433773in}{8.122925in}}%
\pgfpathlineto{\pgfqpoint{1.521509in}{8.122925in}}%
\pgfpathlineto{\pgfqpoint{1.521509in}{8.035189in}}%
\pgfpathlineto{\pgfqpoint{1.433773in}{8.035189in}}%
\pgfpathlineto{\pgfqpoint{1.433773in}{8.122925in}}%
\pgfusepath{stroke,fill}%
\end{pgfscope}%
\begin{pgfscope}%
\pgfpathrectangle{\pgfqpoint{0.380943in}{8.035189in}}{\pgfqpoint{4.650000in}{0.614151in}}%
\pgfusepath{clip}%
\pgfsetbuttcap%
\pgfsetroundjoin%
\definecolor{currentfill}{rgb}{0.963768,0.915433,0.717478}%
\pgfsetfillcolor{currentfill}%
\pgfsetlinewidth{0.250937pt}%
\definecolor{currentstroke}{rgb}{1.000000,1.000000,1.000000}%
\pgfsetstrokecolor{currentstroke}%
\pgfsetdash{}{0pt}%
\pgfpathmoveto{\pgfqpoint{1.521509in}{8.122925in}}%
\pgfpathlineto{\pgfqpoint{1.609245in}{8.122925in}}%
\pgfpathlineto{\pgfqpoint{1.609245in}{8.035189in}}%
\pgfpathlineto{\pgfqpoint{1.521509in}{8.035189in}}%
\pgfpathlineto{\pgfqpoint{1.521509in}{8.122925in}}%
\pgfusepath{stroke,fill}%
\end{pgfscope}%
\begin{pgfscope}%
\pgfpathrectangle{\pgfqpoint{0.380943in}{8.035189in}}{\pgfqpoint{4.650000in}{0.614151in}}%
\pgfusepath{clip}%
\pgfsetbuttcap%
\pgfsetroundjoin%
\definecolor{currentfill}{rgb}{0.961061,0.931672,0.728304}%
\pgfsetfillcolor{currentfill}%
\pgfsetlinewidth{0.250937pt}%
\definecolor{currentstroke}{rgb}{1.000000,1.000000,1.000000}%
\pgfsetstrokecolor{currentstroke}%
\pgfsetdash{}{0pt}%
\pgfpathmoveto{\pgfqpoint{1.609245in}{8.122925in}}%
\pgfpathlineto{\pgfqpoint{1.696981in}{8.122925in}}%
\pgfpathlineto{\pgfqpoint{1.696981in}{8.035189in}}%
\pgfpathlineto{\pgfqpoint{1.609245in}{8.035189in}}%
\pgfpathlineto{\pgfqpoint{1.609245in}{8.122925in}}%
\pgfusepath{stroke,fill}%
\end{pgfscope}%
\begin{pgfscope}%
\pgfpathrectangle{\pgfqpoint{0.380943in}{8.035189in}}{\pgfqpoint{4.650000in}{0.614151in}}%
\pgfusepath{clip}%
\pgfsetbuttcap%
\pgfsetroundjoin%
\definecolor{currentfill}{rgb}{1.000000,1.000000,0.861745}%
\pgfsetfillcolor{currentfill}%
\pgfsetlinewidth{0.250937pt}%
\definecolor{currentstroke}{rgb}{1.000000,1.000000,1.000000}%
\pgfsetstrokecolor{currentstroke}%
\pgfsetdash{}{0pt}%
\pgfpathmoveto{\pgfqpoint{1.696981in}{8.122925in}}%
\pgfpathlineto{\pgfqpoint{1.784717in}{8.122925in}}%
\pgfpathlineto{\pgfqpoint{1.784717in}{8.035189in}}%
\pgfpathlineto{\pgfqpoint{1.696981in}{8.035189in}}%
\pgfpathlineto{\pgfqpoint{1.696981in}{8.122925in}}%
\pgfusepath{stroke,fill}%
\end{pgfscope}%
\begin{pgfscope}%
\pgfpathrectangle{\pgfqpoint{0.380943in}{8.035189in}}{\pgfqpoint{4.650000in}{0.614151in}}%
\pgfusepath{clip}%
\pgfsetbuttcap%
\pgfsetroundjoin%
\definecolor{currentfill}{rgb}{0.963768,0.915433,0.717478}%
\pgfsetfillcolor{currentfill}%
\pgfsetlinewidth{0.250937pt}%
\definecolor{currentstroke}{rgb}{1.000000,1.000000,1.000000}%
\pgfsetstrokecolor{currentstroke}%
\pgfsetdash{}{0pt}%
\pgfpathmoveto{\pgfqpoint{1.784717in}{8.122925in}}%
\pgfpathlineto{\pgfqpoint{1.872452in}{8.122925in}}%
\pgfpathlineto{\pgfqpoint{1.872452in}{8.035189in}}%
\pgfpathlineto{\pgfqpoint{1.784717in}{8.035189in}}%
\pgfpathlineto{\pgfqpoint{1.784717in}{8.122925in}}%
\pgfusepath{stroke,fill}%
\end{pgfscope}%
\begin{pgfscope}%
\pgfpathrectangle{\pgfqpoint{0.380943in}{8.035189in}}{\pgfqpoint{4.650000in}{0.614151in}}%
\pgfusepath{clip}%
\pgfsetbuttcap%
\pgfsetroundjoin%
\definecolor{currentfill}{rgb}{0.961061,0.931672,0.728304}%
\pgfsetfillcolor{currentfill}%
\pgfsetlinewidth{0.250937pt}%
\definecolor{currentstroke}{rgb}{1.000000,1.000000,1.000000}%
\pgfsetstrokecolor{currentstroke}%
\pgfsetdash{}{0pt}%
\pgfpathmoveto{\pgfqpoint{1.872452in}{8.122925in}}%
\pgfpathlineto{\pgfqpoint{1.960188in}{8.122925in}}%
\pgfpathlineto{\pgfqpoint{1.960188in}{8.035189in}}%
\pgfpathlineto{\pgfqpoint{1.872452in}{8.035189in}}%
\pgfpathlineto{\pgfqpoint{1.872452in}{8.122925in}}%
\pgfusepath{stroke,fill}%
\end{pgfscope}%
\begin{pgfscope}%
\pgfpathrectangle{\pgfqpoint{0.380943in}{8.035189in}}{\pgfqpoint{4.650000in}{0.614151in}}%
\pgfusepath{clip}%
\pgfsetbuttcap%
\pgfsetroundjoin%
\definecolor{currentfill}{rgb}{1.000000,1.000000,0.861745}%
\pgfsetfillcolor{currentfill}%
\pgfsetlinewidth{0.250937pt}%
\definecolor{currentstroke}{rgb}{1.000000,1.000000,1.000000}%
\pgfsetstrokecolor{currentstroke}%
\pgfsetdash{}{0pt}%
\pgfpathmoveto{\pgfqpoint{1.960188in}{8.122925in}}%
\pgfpathlineto{\pgfqpoint{2.047924in}{8.122925in}}%
\pgfpathlineto{\pgfqpoint{2.047924in}{8.035189in}}%
\pgfpathlineto{\pgfqpoint{1.960188in}{8.035189in}}%
\pgfpathlineto{\pgfqpoint{1.960188in}{8.122925in}}%
\pgfusepath{stroke,fill}%
\end{pgfscope}%
\begin{pgfscope}%
\pgfpathrectangle{\pgfqpoint{0.380943in}{8.035189in}}{\pgfqpoint{4.650000in}{0.614151in}}%
\pgfusepath{clip}%
\pgfsetbuttcap%
\pgfsetroundjoin%
\definecolor{currentfill}{rgb}{1.000000,1.000000,0.861745}%
\pgfsetfillcolor{currentfill}%
\pgfsetlinewidth{0.250937pt}%
\definecolor{currentstroke}{rgb}{1.000000,1.000000,1.000000}%
\pgfsetstrokecolor{currentstroke}%
\pgfsetdash{}{0pt}%
\pgfpathmoveto{\pgfqpoint{2.047924in}{8.122925in}}%
\pgfpathlineto{\pgfqpoint{2.135660in}{8.122925in}}%
\pgfpathlineto{\pgfqpoint{2.135660in}{8.035189in}}%
\pgfpathlineto{\pgfqpoint{2.047924in}{8.035189in}}%
\pgfpathlineto{\pgfqpoint{2.047924in}{8.122925in}}%
\pgfusepath{stroke,fill}%
\end{pgfscope}%
\begin{pgfscope}%
\pgfpathrectangle{\pgfqpoint{0.380943in}{8.035189in}}{\pgfqpoint{4.650000in}{0.614151in}}%
\pgfusepath{clip}%
\pgfsetbuttcap%
\pgfsetroundjoin%
\definecolor{currentfill}{rgb}{1.000000,1.000000,0.861745}%
\pgfsetfillcolor{currentfill}%
\pgfsetlinewidth{0.250937pt}%
\definecolor{currentstroke}{rgb}{1.000000,1.000000,1.000000}%
\pgfsetstrokecolor{currentstroke}%
\pgfsetdash{}{0pt}%
\pgfpathmoveto{\pgfqpoint{2.135660in}{8.122925in}}%
\pgfpathlineto{\pgfqpoint{2.223396in}{8.122925in}}%
\pgfpathlineto{\pgfqpoint{2.223396in}{8.035189in}}%
\pgfpathlineto{\pgfqpoint{2.135660in}{8.035189in}}%
\pgfpathlineto{\pgfqpoint{2.135660in}{8.122925in}}%
\pgfusepath{stroke,fill}%
\end{pgfscope}%
\begin{pgfscope}%
\pgfpathrectangle{\pgfqpoint{0.380943in}{8.035189in}}{\pgfqpoint{4.650000in}{0.614151in}}%
\pgfusepath{clip}%
\pgfsetbuttcap%
\pgfsetroundjoin%
\definecolor{currentfill}{rgb}{0.963768,0.915433,0.717478}%
\pgfsetfillcolor{currentfill}%
\pgfsetlinewidth{0.250937pt}%
\definecolor{currentstroke}{rgb}{1.000000,1.000000,1.000000}%
\pgfsetstrokecolor{currentstroke}%
\pgfsetdash{}{0pt}%
\pgfpathmoveto{\pgfqpoint{2.223396in}{8.122925in}}%
\pgfpathlineto{\pgfqpoint{2.311132in}{8.122925in}}%
\pgfpathlineto{\pgfqpoint{2.311132in}{8.035189in}}%
\pgfpathlineto{\pgfqpoint{2.223396in}{8.035189in}}%
\pgfpathlineto{\pgfqpoint{2.223396in}{8.122925in}}%
\pgfusepath{stroke,fill}%
\end{pgfscope}%
\begin{pgfscope}%
\pgfpathrectangle{\pgfqpoint{0.380943in}{8.035189in}}{\pgfqpoint{4.650000in}{0.614151in}}%
\pgfusepath{clip}%
\pgfsetbuttcap%
\pgfsetroundjoin%
\definecolor{currentfill}{rgb}{0.961061,0.931672,0.728304}%
\pgfsetfillcolor{currentfill}%
\pgfsetlinewidth{0.250937pt}%
\definecolor{currentstroke}{rgb}{1.000000,1.000000,1.000000}%
\pgfsetstrokecolor{currentstroke}%
\pgfsetdash{}{0pt}%
\pgfpathmoveto{\pgfqpoint{2.311132in}{8.122925in}}%
\pgfpathlineto{\pgfqpoint{2.398868in}{8.122925in}}%
\pgfpathlineto{\pgfqpoint{2.398868in}{8.035189in}}%
\pgfpathlineto{\pgfqpoint{2.311132in}{8.035189in}}%
\pgfpathlineto{\pgfqpoint{2.311132in}{8.122925in}}%
\pgfusepath{stroke,fill}%
\end{pgfscope}%
\begin{pgfscope}%
\pgfpathrectangle{\pgfqpoint{0.380943in}{8.035189in}}{\pgfqpoint{4.650000in}{0.614151in}}%
\pgfusepath{clip}%
\pgfsetbuttcap%
\pgfsetroundjoin%
\definecolor{currentfill}{rgb}{0.961061,0.931672,0.728304}%
\pgfsetfillcolor{currentfill}%
\pgfsetlinewidth{0.250937pt}%
\definecolor{currentstroke}{rgb}{1.000000,1.000000,1.000000}%
\pgfsetstrokecolor{currentstroke}%
\pgfsetdash{}{0pt}%
\pgfpathmoveto{\pgfqpoint{2.398868in}{8.122925in}}%
\pgfpathlineto{\pgfqpoint{2.486603in}{8.122925in}}%
\pgfpathlineto{\pgfqpoint{2.486603in}{8.035189in}}%
\pgfpathlineto{\pgfqpoint{2.398868in}{8.035189in}}%
\pgfpathlineto{\pgfqpoint{2.398868in}{8.122925in}}%
\pgfusepath{stroke,fill}%
\end{pgfscope}%
\begin{pgfscope}%
\pgfpathrectangle{\pgfqpoint{0.380943in}{8.035189in}}{\pgfqpoint{4.650000in}{0.614151in}}%
\pgfusepath{clip}%
\pgfsetbuttcap%
\pgfsetroundjoin%
\definecolor{currentfill}{rgb}{0.963768,0.915433,0.717478}%
\pgfsetfillcolor{currentfill}%
\pgfsetlinewidth{0.250937pt}%
\definecolor{currentstroke}{rgb}{1.000000,1.000000,1.000000}%
\pgfsetstrokecolor{currentstroke}%
\pgfsetdash{}{0pt}%
\pgfpathmoveto{\pgfqpoint{2.486603in}{8.122925in}}%
\pgfpathlineto{\pgfqpoint{2.574339in}{8.122925in}}%
\pgfpathlineto{\pgfqpoint{2.574339in}{8.035189in}}%
\pgfpathlineto{\pgfqpoint{2.486603in}{8.035189in}}%
\pgfpathlineto{\pgfqpoint{2.486603in}{8.122925in}}%
\pgfusepath{stroke,fill}%
\end{pgfscope}%
\begin{pgfscope}%
\pgfpathrectangle{\pgfqpoint{0.380943in}{8.035189in}}{\pgfqpoint{4.650000in}{0.614151in}}%
\pgfusepath{clip}%
\pgfsetbuttcap%
\pgfsetroundjoin%
\definecolor{currentfill}{rgb}{1.000000,1.000000,0.861745}%
\pgfsetfillcolor{currentfill}%
\pgfsetlinewidth{0.250937pt}%
\definecolor{currentstroke}{rgb}{1.000000,1.000000,1.000000}%
\pgfsetstrokecolor{currentstroke}%
\pgfsetdash{}{0pt}%
\pgfpathmoveto{\pgfqpoint{2.574339in}{8.122925in}}%
\pgfpathlineto{\pgfqpoint{2.662075in}{8.122925in}}%
\pgfpathlineto{\pgfqpoint{2.662075in}{8.035189in}}%
\pgfpathlineto{\pgfqpoint{2.574339in}{8.035189in}}%
\pgfpathlineto{\pgfqpoint{2.574339in}{8.122925in}}%
\pgfusepath{stroke,fill}%
\end{pgfscope}%
\begin{pgfscope}%
\pgfpathrectangle{\pgfqpoint{0.380943in}{8.035189in}}{\pgfqpoint{4.650000in}{0.614151in}}%
\pgfusepath{clip}%
\pgfsetbuttcap%
\pgfsetroundjoin%
\definecolor{currentfill}{rgb}{1.000000,1.000000,0.861745}%
\pgfsetfillcolor{currentfill}%
\pgfsetlinewidth{0.250937pt}%
\definecolor{currentstroke}{rgb}{1.000000,1.000000,1.000000}%
\pgfsetstrokecolor{currentstroke}%
\pgfsetdash{}{0pt}%
\pgfpathmoveto{\pgfqpoint{2.662075in}{8.122925in}}%
\pgfpathlineto{\pgfqpoint{2.749811in}{8.122925in}}%
\pgfpathlineto{\pgfqpoint{2.749811in}{8.035189in}}%
\pgfpathlineto{\pgfqpoint{2.662075in}{8.035189in}}%
\pgfpathlineto{\pgfqpoint{2.662075in}{8.122925in}}%
\pgfusepath{stroke,fill}%
\end{pgfscope}%
\begin{pgfscope}%
\pgfpathrectangle{\pgfqpoint{0.380943in}{8.035189in}}{\pgfqpoint{4.650000in}{0.614151in}}%
\pgfusepath{clip}%
\pgfsetbuttcap%
\pgfsetroundjoin%
\definecolor{currentfill}{rgb}{1.000000,1.000000,0.929412}%
\pgfsetfillcolor{currentfill}%
\pgfsetlinewidth{0.250937pt}%
\definecolor{currentstroke}{rgb}{1.000000,1.000000,1.000000}%
\pgfsetstrokecolor{currentstroke}%
\pgfsetdash{}{0pt}%
\pgfpathmoveto{\pgfqpoint{2.749811in}{8.122925in}}%
\pgfpathlineto{\pgfqpoint{2.837547in}{8.122925in}}%
\pgfpathlineto{\pgfqpoint{2.837547in}{8.035189in}}%
\pgfpathlineto{\pgfqpoint{2.749811in}{8.035189in}}%
\pgfpathlineto{\pgfqpoint{2.749811in}{8.122925in}}%
\pgfusepath{stroke,fill}%
\end{pgfscope}%
\begin{pgfscope}%
\pgfpathrectangle{\pgfqpoint{0.380943in}{8.035189in}}{\pgfqpoint{4.650000in}{0.614151in}}%
\pgfusepath{clip}%
\pgfsetbuttcap%
\pgfsetroundjoin%
\definecolor{currentfill}{rgb}{1.000000,1.000000,0.861745}%
\pgfsetfillcolor{currentfill}%
\pgfsetlinewidth{0.250937pt}%
\definecolor{currentstroke}{rgb}{1.000000,1.000000,1.000000}%
\pgfsetstrokecolor{currentstroke}%
\pgfsetdash{}{0pt}%
\pgfpathmoveto{\pgfqpoint{2.837547in}{8.122925in}}%
\pgfpathlineto{\pgfqpoint{2.925283in}{8.122925in}}%
\pgfpathlineto{\pgfqpoint{2.925283in}{8.035189in}}%
\pgfpathlineto{\pgfqpoint{2.837547in}{8.035189in}}%
\pgfpathlineto{\pgfqpoint{2.837547in}{8.122925in}}%
\pgfusepath{stroke,fill}%
\end{pgfscope}%
\begin{pgfscope}%
\pgfpathrectangle{\pgfqpoint{0.380943in}{8.035189in}}{\pgfqpoint{4.650000in}{0.614151in}}%
\pgfusepath{clip}%
\pgfsetbuttcap%
\pgfsetroundjoin%
\definecolor{currentfill}{rgb}{0.970012,0.883276,0.699577}%
\pgfsetfillcolor{currentfill}%
\pgfsetlinewidth{0.250937pt}%
\definecolor{currentstroke}{rgb}{1.000000,1.000000,1.000000}%
\pgfsetstrokecolor{currentstroke}%
\pgfsetdash{}{0pt}%
\pgfpathmoveto{\pgfqpoint{2.925283in}{8.122925in}}%
\pgfpathlineto{\pgfqpoint{3.013019in}{8.122925in}}%
\pgfpathlineto{\pgfqpoint{3.013019in}{8.035189in}}%
\pgfpathlineto{\pgfqpoint{2.925283in}{8.035189in}}%
\pgfpathlineto{\pgfqpoint{2.925283in}{8.122925in}}%
\pgfusepath{stroke,fill}%
\end{pgfscope}%
\begin{pgfscope}%
\pgfpathrectangle{\pgfqpoint{0.380943in}{8.035189in}}{\pgfqpoint{4.650000in}{0.614151in}}%
\pgfusepath{clip}%
\pgfsetbuttcap%
\pgfsetroundjoin%
\definecolor{currentfill}{rgb}{0.961061,0.931672,0.728304}%
\pgfsetfillcolor{currentfill}%
\pgfsetlinewidth{0.250937pt}%
\definecolor{currentstroke}{rgb}{1.000000,1.000000,1.000000}%
\pgfsetstrokecolor{currentstroke}%
\pgfsetdash{}{0pt}%
\pgfpathmoveto{\pgfqpoint{3.013019in}{8.122925in}}%
\pgfpathlineto{\pgfqpoint{3.100754in}{8.122925in}}%
\pgfpathlineto{\pgfqpoint{3.100754in}{8.035189in}}%
\pgfpathlineto{\pgfqpoint{3.013019in}{8.035189in}}%
\pgfpathlineto{\pgfqpoint{3.013019in}{8.122925in}}%
\pgfusepath{stroke,fill}%
\end{pgfscope}%
\begin{pgfscope}%
\pgfpathrectangle{\pgfqpoint{0.380943in}{8.035189in}}{\pgfqpoint{4.650000in}{0.614151in}}%
\pgfusepath{clip}%
\pgfsetbuttcap%
\pgfsetroundjoin%
\definecolor{currentfill}{rgb}{1.000000,1.000000,0.929412}%
\pgfsetfillcolor{currentfill}%
\pgfsetlinewidth{0.250937pt}%
\definecolor{currentstroke}{rgb}{1.000000,1.000000,1.000000}%
\pgfsetstrokecolor{currentstroke}%
\pgfsetdash{}{0pt}%
\pgfpathmoveto{\pgfqpoint{3.100754in}{8.122925in}}%
\pgfpathlineto{\pgfqpoint{3.188490in}{8.122925in}}%
\pgfpathlineto{\pgfqpoint{3.188490in}{8.035189in}}%
\pgfpathlineto{\pgfqpoint{3.100754in}{8.035189in}}%
\pgfpathlineto{\pgfqpoint{3.100754in}{8.122925in}}%
\pgfusepath{stroke,fill}%
\end{pgfscope}%
\begin{pgfscope}%
\pgfpathrectangle{\pgfqpoint{0.380943in}{8.035189in}}{\pgfqpoint{4.650000in}{0.614151in}}%
\pgfusepath{clip}%
\pgfsetbuttcap%
\pgfsetroundjoin%
\definecolor{currentfill}{rgb}{1.000000,1.000000,0.861745}%
\pgfsetfillcolor{currentfill}%
\pgfsetlinewidth{0.250937pt}%
\definecolor{currentstroke}{rgb}{1.000000,1.000000,1.000000}%
\pgfsetstrokecolor{currentstroke}%
\pgfsetdash{}{0pt}%
\pgfpathmoveto{\pgfqpoint{3.188490in}{8.122925in}}%
\pgfpathlineto{\pgfqpoint{3.276226in}{8.122925in}}%
\pgfpathlineto{\pgfqpoint{3.276226in}{8.035189in}}%
\pgfpathlineto{\pgfqpoint{3.188490in}{8.035189in}}%
\pgfpathlineto{\pgfqpoint{3.188490in}{8.122925in}}%
\pgfusepath{stroke,fill}%
\end{pgfscope}%
\begin{pgfscope}%
\pgfpathrectangle{\pgfqpoint{0.380943in}{8.035189in}}{\pgfqpoint{4.650000in}{0.614151in}}%
\pgfusepath{clip}%
\pgfsetbuttcap%
\pgfsetroundjoin%
\definecolor{currentfill}{rgb}{0.963768,0.915433,0.717478}%
\pgfsetfillcolor{currentfill}%
\pgfsetlinewidth{0.250937pt}%
\definecolor{currentstroke}{rgb}{1.000000,1.000000,1.000000}%
\pgfsetstrokecolor{currentstroke}%
\pgfsetdash{}{0pt}%
\pgfpathmoveto{\pgfqpoint{3.276226in}{8.122925in}}%
\pgfpathlineto{\pgfqpoint{3.363962in}{8.122925in}}%
\pgfpathlineto{\pgfqpoint{3.363962in}{8.035189in}}%
\pgfpathlineto{\pgfqpoint{3.276226in}{8.035189in}}%
\pgfpathlineto{\pgfqpoint{3.276226in}{8.122925in}}%
\pgfusepath{stroke,fill}%
\end{pgfscope}%
\begin{pgfscope}%
\pgfpathrectangle{\pgfqpoint{0.380943in}{8.035189in}}{\pgfqpoint{4.650000in}{0.614151in}}%
\pgfusepath{clip}%
\pgfsetbuttcap%
\pgfsetroundjoin%
\definecolor{currentfill}{rgb}{0.978131,0.843783,0.675709}%
\pgfsetfillcolor{currentfill}%
\pgfsetlinewidth{0.250937pt}%
\definecolor{currentstroke}{rgb}{1.000000,1.000000,1.000000}%
\pgfsetstrokecolor{currentstroke}%
\pgfsetdash{}{0pt}%
\pgfpathmoveto{\pgfqpoint{3.363962in}{8.122925in}}%
\pgfpathlineto{\pgfqpoint{3.451698in}{8.122925in}}%
\pgfpathlineto{\pgfqpoint{3.451698in}{8.035189in}}%
\pgfpathlineto{\pgfqpoint{3.363962in}{8.035189in}}%
\pgfpathlineto{\pgfqpoint{3.363962in}{8.122925in}}%
\pgfusepath{stroke,fill}%
\end{pgfscope}%
\begin{pgfscope}%
\pgfpathrectangle{\pgfqpoint{0.380943in}{8.035189in}}{\pgfqpoint{4.650000in}{0.614151in}}%
\pgfusepath{clip}%
\pgfsetbuttcap%
\pgfsetroundjoin%
\definecolor{currentfill}{rgb}{0.961061,0.931672,0.728304}%
\pgfsetfillcolor{currentfill}%
\pgfsetlinewidth{0.250937pt}%
\definecolor{currentstroke}{rgb}{1.000000,1.000000,1.000000}%
\pgfsetstrokecolor{currentstroke}%
\pgfsetdash{}{0pt}%
\pgfpathmoveto{\pgfqpoint{3.451698in}{8.122925in}}%
\pgfpathlineto{\pgfqpoint{3.539434in}{8.122925in}}%
\pgfpathlineto{\pgfqpoint{3.539434in}{8.035189in}}%
\pgfpathlineto{\pgfqpoint{3.451698in}{8.035189in}}%
\pgfpathlineto{\pgfqpoint{3.451698in}{8.122925in}}%
\pgfusepath{stroke,fill}%
\end{pgfscope}%
\begin{pgfscope}%
\pgfpathrectangle{\pgfqpoint{0.380943in}{8.035189in}}{\pgfqpoint{4.650000in}{0.614151in}}%
\pgfusepath{clip}%
\pgfsetbuttcap%
\pgfsetroundjoin%
\definecolor{currentfill}{rgb}{1.000000,1.000000,0.861745}%
\pgfsetfillcolor{currentfill}%
\pgfsetlinewidth{0.250937pt}%
\definecolor{currentstroke}{rgb}{1.000000,1.000000,1.000000}%
\pgfsetstrokecolor{currentstroke}%
\pgfsetdash{}{0pt}%
\pgfpathmoveto{\pgfqpoint{3.539434in}{8.122925in}}%
\pgfpathlineto{\pgfqpoint{3.627169in}{8.122925in}}%
\pgfpathlineto{\pgfqpoint{3.627169in}{8.035189in}}%
\pgfpathlineto{\pgfqpoint{3.539434in}{8.035189in}}%
\pgfpathlineto{\pgfqpoint{3.539434in}{8.122925in}}%
\pgfusepath{stroke,fill}%
\end{pgfscope}%
\begin{pgfscope}%
\pgfpathrectangle{\pgfqpoint{0.380943in}{8.035189in}}{\pgfqpoint{4.650000in}{0.614151in}}%
\pgfusepath{clip}%
\pgfsetbuttcap%
\pgfsetroundjoin%
\definecolor{currentfill}{rgb}{0.999616,0.641369,0.546559}%
\pgfsetfillcolor{currentfill}%
\pgfsetlinewidth{0.250937pt}%
\definecolor{currentstroke}{rgb}{1.000000,1.000000,1.000000}%
\pgfsetstrokecolor{currentstroke}%
\pgfsetdash{}{0pt}%
\pgfpathmoveto{\pgfqpoint{3.627169in}{8.122925in}}%
\pgfpathlineto{\pgfqpoint{3.714905in}{8.122925in}}%
\pgfpathlineto{\pgfqpoint{3.714905in}{8.035189in}}%
\pgfpathlineto{\pgfqpoint{3.627169in}{8.035189in}}%
\pgfpathlineto{\pgfqpoint{3.627169in}{8.122925in}}%
\pgfusepath{stroke,fill}%
\end{pgfscope}%
\begin{pgfscope}%
\pgfpathrectangle{\pgfqpoint{0.380943in}{8.035189in}}{\pgfqpoint{4.650000in}{0.614151in}}%
\pgfusepath{clip}%
\pgfsetbuttcap%
\pgfsetroundjoin%
\definecolor{currentfill}{rgb}{0.978131,0.843783,0.675709}%
\pgfsetfillcolor{currentfill}%
\pgfsetlinewidth{0.250937pt}%
\definecolor{currentstroke}{rgb}{1.000000,1.000000,1.000000}%
\pgfsetstrokecolor{currentstroke}%
\pgfsetdash{}{0pt}%
\pgfpathmoveto{\pgfqpoint{3.714905in}{8.122925in}}%
\pgfpathlineto{\pgfqpoint{3.802641in}{8.122925in}}%
\pgfpathlineto{\pgfqpoint{3.802641in}{8.035189in}}%
\pgfpathlineto{\pgfqpoint{3.714905in}{8.035189in}}%
\pgfpathlineto{\pgfqpoint{3.714905in}{8.122925in}}%
\pgfusepath{stroke,fill}%
\end{pgfscope}%
\begin{pgfscope}%
\pgfpathrectangle{\pgfqpoint{0.380943in}{8.035189in}}{\pgfqpoint{4.650000in}{0.614151in}}%
\pgfusepath{clip}%
\pgfsetbuttcap%
\pgfsetroundjoin%
\definecolor{currentfill}{rgb}{0.985083,0.974641,0.792587}%
\pgfsetfillcolor{currentfill}%
\pgfsetlinewidth{0.250937pt}%
\definecolor{currentstroke}{rgb}{1.000000,1.000000,1.000000}%
\pgfsetstrokecolor{currentstroke}%
\pgfsetdash{}{0pt}%
\pgfpathmoveto{\pgfqpoint{3.802641in}{8.122925in}}%
\pgfpathlineto{\pgfqpoint{3.890377in}{8.122925in}}%
\pgfpathlineto{\pgfqpoint{3.890377in}{8.035189in}}%
\pgfpathlineto{\pgfqpoint{3.802641in}{8.035189in}}%
\pgfpathlineto{\pgfqpoint{3.802641in}{8.122925in}}%
\pgfusepath{stroke,fill}%
\end{pgfscope}%
\begin{pgfscope}%
\pgfpathrectangle{\pgfqpoint{0.380943in}{8.035189in}}{\pgfqpoint{4.650000in}{0.614151in}}%
\pgfusepath{clip}%
\pgfsetbuttcap%
\pgfsetroundjoin%
\definecolor{currentfill}{rgb}{0.961061,0.931672,0.728304}%
\pgfsetfillcolor{currentfill}%
\pgfsetlinewidth{0.250937pt}%
\definecolor{currentstroke}{rgb}{1.000000,1.000000,1.000000}%
\pgfsetstrokecolor{currentstroke}%
\pgfsetdash{}{0pt}%
\pgfpathmoveto{\pgfqpoint{3.890377in}{8.122925in}}%
\pgfpathlineto{\pgfqpoint{3.978113in}{8.122925in}}%
\pgfpathlineto{\pgfqpoint{3.978113in}{8.035189in}}%
\pgfpathlineto{\pgfqpoint{3.890377in}{8.035189in}}%
\pgfpathlineto{\pgfqpoint{3.890377in}{8.122925in}}%
\pgfusepath{stroke,fill}%
\end{pgfscope}%
\begin{pgfscope}%
\pgfpathrectangle{\pgfqpoint{0.380943in}{8.035189in}}{\pgfqpoint{4.650000in}{0.614151in}}%
\pgfusepath{clip}%
\pgfsetbuttcap%
\pgfsetroundjoin%
\definecolor{currentfill}{rgb}{0.978131,0.843783,0.675709}%
\pgfsetfillcolor{currentfill}%
\pgfsetlinewidth{0.250937pt}%
\definecolor{currentstroke}{rgb}{1.000000,1.000000,1.000000}%
\pgfsetstrokecolor{currentstroke}%
\pgfsetdash{}{0pt}%
\pgfpathmoveto{\pgfqpoint{3.978113in}{8.122925in}}%
\pgfpathlineto{\pgfqpoint{4.065849in}{8.122925in}}%
\pgfpathlineto{\pgfqpoint{4.065849in}{8.035189in}}%
\pgfpathlineto{\pgfqpoint{3.978113in}{8.035189in}}%
\pgfpathlineto{\pgfqpoint{3.978113in}{8.122925in}}%
\pgfusepath{stroke,fill}%
\end{pgfscope}%
\begin{pgfscope}%
\pgfpathrectangle{\pgfqpoint{0.380943in}{8.035189in}}{\pgfqpoint{4.650000in}{0.614151in}}%
\pgfusepath{clip}%
\pgfsetbuttcap%
\pgfsetroundjoin%
\definecolor{currentfill}{rgb}{0.985083,0.974641,0.792587}%
\pgfsetfillcolor{currentfill}%
\pgfsetlinewidth{0.250937pt}%
\definecolor{currentstroke}{rgb}{1.000000,1.000000,1.000000}%
\pgfsetstrokecolor{currentstroke}%
\pgfsetdash{}{0pt}%
\pgfpathmoveto{\pgfqpoint{4.065849in}{8.122925in}}%
\pgfpathlineto{\pgfqpoint{4.153585in}{8.122925in}}%
\pgfpathlineto{\pgfqpoint{4.153585in}{8.035189in}}%
\pgfpathlineto{\pgfqpoint{4.065849in}{8.035189in}}%
\pgfpathlineto{\pgfqpoint{4.065849in}{8.122925in}}%
\pgfusepath{stroke,fill}%
\end{pgfscope}%
\begin{pgfscope}%
\pgfpathrectangle{\pgfqpoint{0.380943in}{8.035189in}}{\pgfqpoint{4.650000in}{0.614151in}}%
\pgfusepath{clip}%
\pgfsetbuttcap%
\pgfsetroundjoin%
\definecolor{currentfill}{rgb}{0.970012,0.883276,0.699577}%
\pgfsetfillcolor{currentfill}%
\pgfsetlinewidth{0.250937pt}%
\definecolor{currentstroke}{rgb}{1.000000,1.000000,1.000000}%
\pgfsetstrokecolor{currentstroke}%
\pgfsetdash{}{0pt}%
\pgfpathmoveto{\pgfqpoint{4.153585in}{8.122925in}}%
\pgfpathlineto{\pgfqpoint{4.241320in}{8.122925in}}%
\pgfpathlineto{\pgfqpoint{4.241320in}{8.035189in}}%
\pgfpathlineto{\pgfqpoint{4.153585in}{8.035189in}}%
\pgfpathlineto{\pgfqpoint{4.153585in}{8.122925in}}%
\pgfusepath{stroke,fill}%
\end{pgfscope}%
\begin{pgfscope}%
\pgfpathrectangle{\pgfqpoint{0.380943in}{8.035189in}}{\pgfqpoint{4.650000in}{0.614151in}}%
\pgfusepath{clip}%
\pgfsetbuttcap%
\pgfsetroundjoin%
\definecolor{currentfill}{rgb}{0.963768,0.915433,0.717478}%
\pgfsetfillcolor{currentfill}%
\pgfsetlinewidth{0.250937pt}%
\definecolor{currentstroke}{rgb}{1.000000,1.000000,1.000000}%
\pgfsetstrokecolor{currentstroke}%
\pgfsetdash{}{0pt}%
\pgfpathmoveto{\pgfqpoint{4.241320in}{8.122925in}}%
\pgfpathlineto{\pgfqpoint{4.329056in}{8.122925in}}%
\pgfpathlineto{\pgfqpoint{4.329056in}{8.035189in}}%
\pgfpathlineto{\pgfqpoint{4.241320in}{8.035189in}}%
\pgfpathlineto{\pgfqpoint{4.241320in}{8.122925in}}%
\pgfusepath{stroke,fill}%
\end{pgfscope}%
\begin{pgfscope}%
\pgfpathrectangle{\pgfqpoint{0.380943in}{8.035189in}}{\pgfqpoint{4.650000in}{0.614151in}}%
\pgfusepath{clip}%
\pgfsetbuttcap%
\pgfsetroundjoin%
\definecolor{currentfill}{rgb}{0.985083,0.974641,0.792587}%
\pgfsetfillcolor{currentfill}%
\pgfsetlinewidth{0.250937pt}%
\definecolor{currentstroke}{rgb}{1.000000,1.000000,1.000000}%
\pgfsetstrokecolor{currentstroke}%
\pgfsetdash{}{0pt}%
\pgfpathmoveto{\pgfqpoint{4.329056in}{8.122925in}}%
\pgfpathlineto{\pgfqpoint{4.416792in}{8.122925in}}%
\pgfpathlineto{\pgfqpoint{4.416792in}{8.035189in}}%
\pgfpathlineto{\pgfqpoint{4.329056in}{8.035189in}}%
\pgfpathlineto{\pgfqpoint{4.329056in}{8.122925in}}%
\pgfusepath{stroke,fill}%
\end{pgfscope}%
\begin{pgfscope}%
\pgfpathrectangle{\pgfqpoint{0.380943in}{8.035189in}}{\pgfqpoint{4.650000in}{0.614151in}}%
\pgfusepath{clip}%
\pgfsetbuttcap%
\pgfsetroundjoin%
\definecolor{currentfill}{rgb}{0.985083,0.974641,0.792587}%
\pgfsetfillcolor{currentfill}%
\pgfsetlinewidth{0.250937pt}%
\definecolor{currentstroke}{rgb}{1.000000,1.000000,1.000000}%
\pgfsetstrokecolor{currentstroke}%
\pgfsetdash{}{0pt}%
\pgfpathmoveto{\pgfqpoint{4.416792in}{8.122925in}}%
\pgfpathlineto{\pgfqpoint{4.504528in}{8.122925in}}%
\pgfpathlineto{\pgfqpoint{4.504528in}{8.035189in}}%
\pgfpathlineto{\pgfqpoint{4.416792in}{8.035189in}}%
\pgfpathlineto{\pgfqpoint{4.416792in}{8.122925in}}%
\pgfusepath{stroke,fill}%
\end{pgfscope}%
\begin{pgfscope}%
\pgfpathrectangle{\pgfqpoint{0.380943in}{8.035189in}}{\pgfqpoint{4.650000in}{0.614151in}}%
\pgfusepath{clip}%
\pgfsetbuttcap%
\pgfsetroundjoin%
\definecolor{currentfill}{rgb}{0.985083,0.974641,0.792587}%
\pgfsetfillcolor{currentfill}%
\pgfsetlinewidth{0.250937pt}%
\definecolor{currentstroke}{rgb}{1.000000,1.000000,1.000000}%
\pgfsetstrokecolor{currentstroke}%
\pgfsetdash{}{0pt}%
\pgfpathmoveto{\pgfqpoint{4.504528in}{8.122925in}}%
\pgfpathlineto{\pgfqpoint{4.592264in}{8.122925in}}%
\pgfpathlineto{\pgfqpoint{4.592264in}{8.035189in}}%
\pgfpathlineto{\pgfqpoint{4.504528in}{8.035189in}}%
\pgfpathlineto{\pgfqpoint{4.504528in}{8.122925in}}%
\pgfusepath{stroke,fill}%
\end{pgfscope}%
\begin{pgfscope}%
\pgfpathrectangle{\pgfqpoint{0.380943in}{8.035189in}}{\pgfqpoint{4.650000in}{0.614151in}}%
\pgfusepath{clip}%
\pgfsetbuttcap%
\pgfsetroundjoin%
\definecolor{currentfill}{rgb}{0.963768,0.915433,0.717478}%
\pgfsetfillcolor{currentfill}%
\pgfsetlinewidth{0.250937pt}%
\definecolor{currentstroke}{rgb}{1.000000,1.000000,1.000000}%
\pgfsetstrokecolor{currentstroke}%
\pgfsetdash{}{0pt}%
\pgfpathmoveto{\pgfqpoint{4.592264in}{8.122925in}}%
\pgfpathlineto{\pgfqpoint{4.680000in}{8.122925in}}%
\pgfpathlineto{\pgfqpoint{4.680000in}{8.035189in}}%
\pgfpathlineto{\pgfqpoint{4.592264in}{8.035189in}}%
\pgfpathlineto{\pgfqpoint{4.592264in}{8.122925in}}%
\pgfusepath{stroke,fill}%
\end{pgfscope}%
\begin{pgfscope}%
\pgfpathrectangle{\pgfqpoint{0.380943in}{8.035189in}}{\pgfqpoint{4.650000in}{0.614151in}}%
\pgfusepath{clip}%
\pgfsetbuttcap%
\pgfsetroundjoin%
\definecolor{currentfill}{rgb}{0.978131,0.843783,0.675709}%
\pgfsetfillcolor{currentfill}%
\pgfsetlinewidth{0.250937pt}%
\definecolor{currentstroke}{rgb}{1.000000,1.000000,1.000000}%
\pgfsetstrokecolor{currentstroke}%
\pgfsetdash{}{0pt}%
\pgfpathmoveto{\pgfqpoint{4.680000in}{8.122925in}}%
\pgfpathlineto{\pgfqpoint{4.767736in}{8.122925in}}%
\pgfpathlineto{\pgfqpoint{4.767736in}{8.035189in}}%
\pgfpathlineto{\pgfqpoint{4.680000in}{8.035189in}}%
\pgfpathlineto{\pgfqpoint{4.680000in}{8.122925in}}%
\pgfusepath{stroke,fill}%
\end{pgfscope}%
\begin{pgfscope}%
\pgfpathrectangle{\pgfqpoint{0.380943in}{8.035189in}}{\pgfqpoint{4.650000in}{0.614151in}}%
\pgfusepath{clip}%
\pgfsetbuttcap%
\pgfsetroundjoin%
\definecolor{currentfill}{rgb}{1.000000,1.000000,0.861745}%
\pgfsetfillcolor{currentfill}%
\pgfsetlinewidth{0.250937pt}%
\definecolor{currentstroke}{rgb}{1.000000,1.000000,1.000000}%
\pgfsetstrokecolor{currentstroke}%
\pgfsetdash{}{0pt}%
\pgfpathmoveto{\pgfqpoint{4.767736in}{8.122925in}}%
\pgfpathlineto{\pgfqpoint{4.855471in}{8.122925in}}%
\pgfpathlineto{\pgfqpoint{4.855471in}{8.035189in}}%
\pgfpathlineto{\pgfqpoint{4.767736in}{8.035189in}}%
\pgfpathlineto{\pgfqpoint{4.767736in}{8.122925in}}%
\pgfusepath{stroke,fill}%
\end{pgfscope}%
\begin{pgfscope}%
\pgfpathrectangle{\pgfqpoint{0.380943in}{8.035189in}}{\pgfqpoint{4.650000in}{0.614151in}}%
\pgfusepath{clip}%
\pgfsetbuttcap%
\pgfsetroundjoin%
\definecolor{currentfill}{rgb}{1.000000,1.000000,0.861745}%
\pgfsetfillcolor{currentfill}%
\pgfsetlinewidth{0.250937pt}%
\definecolor{currentstroke}{rgb}{1.000000,1.000000,1.000000}%
\pgfsetstrokecolor{currentstroke}%
\pgfsetdash{}{0pt}%
\pgfpathmoveto{\pgfqpoint{4.855471in}{8.122925in}}%
\pgfpathlineto{\pgfqpoint{4.943207in}{8.122925in}}%
\pgfpathlineto{\pgfqpoint{4.943207in}{8.035189in}}%
\pgfpathlineto{\pgfqpoint{4.855471in}{8.035189in}}%
\pgfpathlineto{\pgfqpoint{4.855471in}{8.122925in}}%
\pgfusepath{stroke,fill}%
\end{pgfscope}%
\begin{pgfscope}%
\pgfpathrectangle{\pgfqpoint{0.380943in}{8.035189in}}{\pgfqpoint{4.650000in}{0.614151in}}%
\pgfusepath{clip}%
\pgfsetbuttcap%
\pgfsetroundjoin%
\definecolor{currentfill}{rgb}{0.985083,0.974641,0.792587}%
\pgfsetfillcolor{currentfill}%
\pgfsetlinewidth{0.250937pt}%
\definecolor{currentstroke}{rgb}{1.000000,1.000000,1.000000}%
\pgfsetstrokecolor{currentstroke}%
\pgfsetdash{}{0pt}%
\pgfpathmoveto{\pgfqpoint{4.943207in}{8.122925in}}%
\pgfpathlineto{\pgfqpoint{5.030943in}{8.122925in}}%
\pgfpathlineto{\pgfqpoint{5.030943in}{8.035189in}}%
\pgfpathlineto{\pgfqpoint{4.943207in}{8.035189in}}%
\pgfpathlineto{\pgfqpoint{4.943207in}{8.122925in}}%
\pgfusepath{stroke,fill}%
\end{pgfscope}%
\begin{pgfscope}%
\pgfsetbuttcap%
\pgfsetroundjoin%
\definecolor{currentfill}{rgb}{0.000000,0.000000,0.000000}%
\pgfsetfillcolor{currentfill}%
\pgfsetlinewidth{0.803000pt}%
\definecolor{currentstroke}{rgb}{0.000000,0.000000,0.000000}%
\pgfsetstrokecolor{currentstroke}%
\pgfsetdash{}{0pt}%
\pgfsys@defobject{currentmarker}{\pgfqpoint{0.000000in}{-0.048611in}}{\pgfqpoint{0.000000in}{0.000000in}}{%
\pgfpathmoveto{\pgfqpoint{0.000000in}{0.000000in}}%
\pgfpathlineto{\pgfqpoint{0.000000in}{-0.048611in}}%
\pgfusepath{stroke,fill}%
}%
\begin{pgfscope}%
\pgfsys@transformshift{0.644151in}{8.035189in}%
\pgfsys@useobject{currentmarker}{}%
\end{pgfscope}%
\end{pgfscope}%
\begin{pgfscope}%
\definecolor{textcolor}{rgb}{0.000000,0.000000,0.000000}%
\pgfsetstrokecolor{textcolor}%
\pgfsetfillcolor{textcolor}%
\pgftext[x=0.644151in,y=7.937967in,,top]{\color{textcolor}\rmfamily\fontsize{8.000000}{9.600000}\selectfont Jan}%
\end{pgfscope}%
\begin{pgfscope}%
\pgfsetbuttcap%
\pgfsetroundjoin%
\definecolor{currentfill}{rgb}{0.000000,0.000000,0.000000}%
\pgfsetfillcolor{currentfill}%
\pgfsetlinewidth{0.803000pt}%
\definecolor{currentstroke}{rgb}{0.000000,0.000000,0.000000}%
\pgfsetstrokecolor{currentstroke}%
\pgfsetdash{}{0pt}%
\pgfsys@defobject{currentmarker}{\pgfqpoint{0.000000in}{-0.048611in}}{\pgfqpoint{0.000000in}{0.000000in}}{%
\pgfpathmoveto{\pgfqpoint{0.000000in}{0.000000in}}%
\pgfpathlineto{\pgfqpoint{0.000000in}{-0.048611in}}%
\pgfusepath{stroke,fill}%
}%
\begin{pgfscope}%
\pgfsys@transformshift{1.038962in}{8.035189in}%
\pgfsys@useobject{currentmarker}{}%
\end{pgfscope}%
\end{pgfscope}%
\begin{pgfscope}%
\definecolor{textcolor}{rgb}{0.000000,0.000000,0.000000}%
\pgfsetstrokecolor{textcolor}%
\pgfsetfillcolor{textcolor}%
\pgftext[x=1.038962in,y=7.937967in,,top]{\color{textcolor}\rmfamily\fontsize{8.000000}{9.600000}\selectfont Feb}%
\end{pgfscope}%
\begin{pgfscope}%
\pgfsetbuttcap%
\pgfsetroundjoin%
\definecolor{currentfill}{rgb}{0.000000,0.000000,0.000000}%
\pgfsetfillcolor{currentfill}%
\pgfsetlinewidth{0.803000pt}%
\definecolor{currentstroke}{rgb}{0.000000,0.000000,0.000000}%
\pgfsetstrokecolor{currentstroke}%
\pgfsetdash{}{0pt}%
\pgfsys@defobject{currentmarker}{\pgfqpoint{0.000000in}{-0.048611in}}{\pgfqpoint{0.000000in}{0.000000in}}{%
\pgfpathmoveto{\pgfqpoint{0.000000in}{0.000000in}}%
\pgfpathlineto{\pgfqpoint{0.000000in}{-0.048611in}}%
\pgfusepath{stroke,fill}%
}%
\begin{pgfscope}%
\pgfsys@transformshift{1.389905in}{8.035189in}%
\pgfsys@useobject{currentmarker}{}%
\end{pgfscope}%
\end{pgfscope}%
\begin{pgfscope}%
\definecolor{textcolor}{rgb}{0.000000,0.000000,0.000000}%
\pgfsetstrokecolor{textcolor}%
\pgfsetfillcolor{textcolor}%
\pgftext[x=1.389905in,y=7.937967in,,top]{\color{textcolor}\rmfamily\fontsize{8.000000}{9.600000}\selectfont Mar}%
\end{pgfscope}%
\begin{pgfscope}%
\pgfsetbuttcap%
\pgfsetroundjoin%
\definecolor{currentfill}{rgb}{0.000000,0.000000,0.000000}%
\pgfsetfillcolor{currentfill}%
\pgfsetlinewidth{0.803000pt}%
\definecolor{currentstroke}{rgb}{0.000000,0.000000,0.000000}%
\pgfsetstrokecolor{currentstroke}%
\pgfsetdash{}{0pt}%
\pgfsys@defobject{currentmarker}{\pgfqpoint{0.000000in}{-0.048611in}}{\pgfqpoint{0.000000in}{0.000000in}}{%
\pgfpathmoveto{\pgfqpoint{0.000000in}{0.000000in}}%
\pgfpathlineto{\pgfqpoint{0.000000in}{-0.048611in}}%
\pgfusepath{stroke,fill}%
}%
\begin{pgfscope}%
\pgfsys@transformshift{1.740849in}{8.035189in}%
\pgfsys@useobject{currentmarker}{}%
\end{pgfscope}%
\end{pgfscope}%
\begin{pgfscope}%
\definecolor{textcolor}{rgb}{0.000000,0.000000,0.000000}%
\pgfsetstrokecolor{textcolor}%
\pgfsetfillcolor{textcolor}%
\pgftext[x=1.740849in,y=7.937967in,,top]{\color{textcolor}\rmfamily\fontsize{8.000000}{9.600000}\selectfont Apr}%
\end{pgfscope}%
\begin{pgfscope}%
\pgfsetbuttcap%
\pgfsetroundjoin%
\definecolor{currentfill}{rgb}{0.000000,0.000000,0.000000}%
\pgfsetfillcolor{currentfill}%
\pgfsetlinewidth{0.803000pt}%
\definecolor{currentstroke}{rgb}{0.000000,0.000000,0.000000}%
\pgfsetstrokecolor{currentstroke}%
\pgfsetdash{}{0pt}%
\pgfsys@defobject{currentmarker}{\pgfqpoint{0.000000in}{-0.048611in}}{\pgfqpoint{0.000000in}{0.000000in}}{%
\pgfpathmoveto{\pgfqpoint{0.000000in}{0.000000in}}%
\pgfpathlineto{\pgfqpoint{0.000000in}{-0.048611in}}%
\pgfusepath{stroke,fill}%
}%
\begin{pgfscope}%
\pgfsys@transformshift{2.179528in}{8.035189in}%
\pgfsys@useobject{currentmarker}{}%
\end{pgfscope}%
\end{pgfscope}%
\begin{pgfscope}%
\definecolor{textcolor}{rgb}{0.000000,0.000000,0.000000}%
\pgfsetstrokecolor{textcolor}%
\pgfsetfillcolor{textcolor}%
\pgftext[x=2.179528in,y=7.937967in,,top]{\color{textcolor}\rmfamily\fontsize{8.000000}{9.600000}\selectfont May}%
\end{pgfscope}%
\begin{pgfscope}%
\pgfsetbuttcap%
\pgfsetroundjoin%
\definecolor{currentfill}{rgb}{0.000000,0.000000,0.000000}%
\pgfsetfillcolor{currentfill}%
\pgfsetlinewidth{0.803000pt}%
\definecolor{currentstroke}{rgb}{0.000000,0.000000,0.000000}%
\pgfsetstrokecolor{currentstroke}%
\pgfsetdash{}{0pt}%
\pgfsys@defobject{currentmarker}{\pgfqpoint{0.000000in}{-0.048611in}}{\pgfqpoint{0.000000in}{0.000000in}}{%
\pgfpathmoveto{\pgfqpoint{0.000000in}{0.000000in}}%
\pgfpathlineto{\pgfqpoint{0.000000in}{-0.048611in}}%
\pgfusepath{stroke,fill}%
}%
\begin{pgfscope}%
\pgfsys@transformshift{2.530471in}{8.035189in}%
\pgfsys@useobject{currentmarker}{}%
\end{pgfscope}%
\end{pgfscope}%
\begin{pgfscope}%
\definecolor{textcolor}{rgb}{0.000000,0.000000,0.000000}%
\pgfsetstrokecolor{textcolor}%
\pgfsetfillcolor{textcolor}%
\pgftext[x=2.530471in,y=7.937967in,,top]{\color{textcolor}\rmfamily\fontsize{8.000000}{9.600000}\selectfont Jun}%
\end{pgfscope}%
\begin{pgfscope}%
\pgfsetbuttcap%
\pgfsetroundjoin%
\definecolor{currentfill}{rgb}{0.000000,0.000000,0.000000}%
\pgfsetfillcolor{currentfill}%
\pgfsetlinewidth{0.803000pt}%
\definecolor{currentstroke}{rgb}{0.000000,0.000000,0.000000}%
\pgfsetstrokecolor{currentstroke}%
\pgfsetdash{}{0pt}%
\pgfsys@defobject{currentmarker}{\pgfqpoint{0.000000in}{-0.048611in}}{\pgfqpoint{0.000000in}{0.000000in}}{%
\pgfpathmoveto{\pgfqpoint{0.000000in}{0.000000in}}%
\pgfpathlineto{\pgfqpoint{0.000000in}{-0.048611in}}%
\pgfusepath{stroke,fill}%
}%
\begin{pgfscope}%
\pgfsys@transformshift{2.925283in}{8.035189in}%
\pgfsys@useobject{currentmarker}{}%
\end{pgfscope}%
\end{pgfscope}%
\begin{pgfscope}%
\definecolor{textcolor}{rgb}{0.000000,0.000000,0.000000}%
\pgfsetstrokecolor{textcolor}%
\pgfsetfillcolor{textcolor}%
\pgftext[x=2.925283in,y=7.937967in,,top]{\color{textcolor}\rmfamily\fontsize{8.000000}{9.600000}\selectfont Jul}%
\end{pgfscope}%
\begin{pgfscope}%
\pgfsetbuttcap%
\pgfsetroundjoin%
\definecolor{currentfill}{rgb}{0.000000,0.000000,0.000000}%
\pgfsetfillcolor{currentfill}%
\pgfsetlinewidth{0.803000pt}%
\definecolor{currentstroke}{rgb}{0.000000,0.000000,0.000000}%
\pgfsetstrokecolor{currentstroke}%
\pgfsetdash{}{0pt}%
\pgfsys@defobject{currentmarker}{\pgfqpoint{0.000000in}{-0.048611in}}{\pgfqpoint{0.000000in}{0.000000in}}{%
\pgfpathmoveto{\pgfqpoint{0.000000in}{0.000000in}}%
\pgfpathlineto{\pgfqpoint{0.000000in}{-0.048611in}}%
\pgfusepath{stroke,fill}%
}%
\begin{pgfscope}%
\pgfsys@transformshift{3.320094in}{8.035189in}%
\pgfsys@useobject{currentmarker}{}%
\end{pgfscope}%
\end{pgfscope}%
\begin{pgfscope}%
\definecolor{textcolor}{rgb}{0.000000,0.000000,0.000000}%
\pgfsetstrokecolor{textcolor}%
\pgfsetfillcolor{textcolor}%
\pgftext[x=3.320094in,y=7.937967in,,top]{\color{textcolor}\rmfamily\fontsize{8.000000}{9.600000}\selectfont Aug}%
\end{pgfscope}%
\begin{pgfscope}%
\pgfsetbuttcap%
\pgfsetroundjoin%
\definecolor{currentfill}{rgb}{0.000000,0.000000,0.000000}%
\pgfsetfillcolor{currentfill}%
\pgfsetlinewidth{0.803000pt}%
\definecolor{currentstroke}{rgb}{0.000000,0.000000,0.000000}%
\pgfsetstrokecolor{currentstroke}%
\pgfsetdash{}{0pt}%
\pgfsys@defobject{currentmarker}{\pgfqpoint{0.000000in}{-0.048611in}}{\pgfqpoint{0.000000in}{0.000000in}}{%
\pgfpathmoveto{\pgfqpoint{0.000000in}{0.000000in}}%
\pgfpathlineto{\pgfqpoint{0.000000in}{-0.048611in}}%
\pgfusepath{stroke,fill}%
}%
\begin{pgfscope}%
\pgfsys@transformshift{3.671037in}{8.035189in}%
\pgfsys@useobject{currentmarker}{}%
\end{pgfscope}%
\end{pgfscope}%
\begin{pgfscope}%
\definecolor{textcolor}{rgb}{0.000000,0.000000,0.000000}%
\pgfsetstrokecolor{textcolor}%
\pgfsetfillcolor{textcolor}%
\pgftext[x=3.671037in,y=7.937967in,,top]{\color{textcolor}\rmfamily\fontsize{8.000000}{9.600000}\selectfont Sep}%
\end{pgfscope}%
\begin{pgfscope}%
\pgfsetbuttcap%
\pgfsetroundjoin%
\definecolor{currentfill}{rgb}{0.000000,0.000000,0.000000}%
\pgfsetfillcolor{currentfill}%
\pgfsetlinewidth{0.803000pt}%
\definecolor{currentstroke}{rgb}{0.000000,0.000000,0.000000}%
\pgfsetstrokecolor{currentstroke}%
\pgfsetdash{}{0pt}%
\pgfsys@defobject{currentmarker}{\pgfqpoint{0.000000in}{-0.048611in}}{\pgfqpoint{0.000000in}{0.000000in}}{%
\pgfpathmoveto{\pgfqpoint{0.000000in}{0.000000in}}%
\pgfpathlineto{\pgfqpoint{0.000000in}{-0.048611in}}%
\pgfusepath{stroke,fill}%
}%
\begin{pgfscope}%
\pgfsys@transformshift{4.065849in}{8.035189in}%
\pgfsys@useobject{currentmarker}{}%
\end{pgfscope}%
\end{pgfscope}%
\begin{pgfscope}%
\definecolor{textcolor}{rgb}{0.000000,0.000000,0.000000}%
\pgfsetstrokecolor{textcolor}%
\pgfsetfillcolor{textcolor}%
\pgftext[x=4.065849in,y=7.937967in,,top]{\color{textcolor}\rmfamily\fontsize{8.000000}{9.600000}\selectfont Oct}%
\end{pgfscope}%
\begin{pgfscope}%
\pgfsetbuttcap%
\pgfsetroundjoin%
\definecolor{currentfill}{rgb}{0.000000,0.000000,0.000000}%
\pgfsetfillcolor{currentfill}%
\pgfsetlinewidth{0.803000pt}%
\definecolor{currentstroke}{rgb}{0.000000,0.000000,0.000000}%
\pgfsetstrokecolor{currentstroke}%
\pgfsetdash{}{0pt}%
\pgfsys@defobject{currentmarker}{\pgfqpoint{0.000000in}{-0.048611in}}{\pgfqpoint{0.000000in}{0.000000in}}{%
\pgfpathmoveto{\pgfqpoint{0.000000in}{0.000000in}}%
\pgfpathlineto{\pgfqpoint{0.000000in}{-0.048611in}}%
\pgfusepath{stroke,fill}%
}%
\begin{pgfscope}%
\pgfsys@transformshift{4.460660in}{8.035189in}%
\pgfsys@useobject{currentmarker}{}%
\end{pgfscope}%
\end{pgfscope}%
\begin{pgfscope}%
\definecolor{textcolor}{rgb}{0.000000,0.000000,0.000000}%
\pgfsetstrokecolor{textcolor}%
\pgfsetfillcolor{textcolor}%
\pgftext[x=4.460660in,y=7.937967in,,top]{\color{textcolor}\rmfamily\fontsize{8.000000}{9.600000}\selectfont Nov}%
\end{pgfscope}%
\begin{pgfscope}%
\pgfsetbuttcap%
\pgfsetroundjoin%
\definecolor{currentfill}{rgb}{0.000000,0.000000,0.000000}%
\pgfsetfillcolor{currentfill}%
\pgfsetlinewidth{0.803000pt}%
\definecolor{currentstroke}{rgb}{0.000000,0.000000,0.000000}%
\pgfsetstrokecolor{currentstroke}%
\pgfsetdash{}{0pt}%
\pgfsys@defobject{currentmarker}{\pgfqpoint{0.000000in}{-0.048611in}}{\pgfqpoint{0.000000in}{0.000000in}}{%
\pgfpathmoveto{\pgfqpoint{0.000000in}{0.000000in}}%
\pgfpathlineto{\pgfqpoint{0.000000in}{-0.048611in}}%
\pgfusepath{stroke,fill}%
}%
\begin{pgfscope}%
\pgfsys@transformshift{4.811603in}{8.035189in}%
\pgfsys@useobject{currentmarker}{}%
\end{pgfscope}%
\end{pgfscope}%
\begin{pgfscope}%
\definecolor{textcolor}{rgb}{0.000000,0.000000,0.000000}%
\pgfsetstrokecolor{textcolor}%
\pgfsetfillcolor{textcolor}%
\pgftext[x=4.811603in,y=7.937967in,,top]{\color{textcolor}\rmfamily\fontsize{8.000000}{9.600000}\selectfont Dec}%
\end{pgfscope}%
\begin{pgfscope}%
\pgfsetbuttcap%
\pgfsetroundjoin%
\definecolor{currentfill}{rgb}{0.000000,0.000000,0.000000}%
\pgfsetfillcolor{currentfill}%
\pgfsetlinewidth{0.803000pt}%
\definecolor{currentstroke}{rgb}{0.000000,0.000000,0.000000}%
\pgfsetstrokecolor{currentstroke}%
\pgfsetdash{}{0pt}%
\pgfsys@defobject{currentmarker}{\pgfqpoint{-0.048611in}{0.000000in}}{\pgfqpoint{-0.000000in}{0.000000in}}{%
\pgfpathmoveto{\pgfqpoint{-0.000000in}{0.000000in}}%
\pgfpathlineto{\pgfqpoint{-0.048611in}{0.000000in}}%
\pgfusepath{stroke,fill}%
}%
\begin{pgfscope}%
\pgfsys@transformshift{0.380943in}{8.605472in}%
\pgfsys@useobject{currentmarker}{}%
\end{pgfscope}%
\end{pgfscope}%
\begin{pgfscope}%
\definecolor{textcolor}{rgb}{0.000000,0.000000,0.000000}%
\pgfsetstrokecolor{textcolor}%
\pgfsetfillcolor{textcolor}%
\pgftext[x=0.113117in, y=8.566892in, left, base]{\color{textcolor}\rmfamily\fontsize{8.000000}{9.600000}\selectfont M}%
\end{pgfscope}%
\begin{pgfscope}%
\pgfsetbuttcap%
\pgfsetroundjoin%
\definecolor{currentfill}{rgb}{0.000000,0.000000,0.000000}%
\pgfsetfillcolor{currentfill}%
\pgfsetlinewidth{0.803000pt}%
\definecolor{currentstroke}{rgb}{0.000000,0.000000,0.000000}%
\pgfsetstrokecolor{currentstroke}%
\pgfsetdash{}{0pt}%
\pgfsys@defobject{currentmarker}{\pgfqpoint{-0.048611in}{0.000000in}}{\pgfqpoint{-0.000000in}{0.000000in}}{%
\pgfpathmoveto{\pgfqpoint{-0.000000in}{0.000000in}}%
\pgfpathlineto{\pgfqpoint{-0.048611in}{0.000000in}}%
\pgfusepath{stroke,fill}%
}%
\begin{pgfscope}%
\pgfsys@transformshift{0.380943in}{8.517736in}%
\pgfsys@useobject{currentmarker}{}%
\end{pgfscope}%
\end{pgfscope}%
\begin{pgfscope}%
\definecolor{textcolor}{rgb}{0.000000,0.000000,0.000000}%
\pgfsetstrokecolor{textcolor}%
\pgfsetfillcolor{textcolor}%
\pgftext[x=0.135957in, y=8.479156in, left, base]{\color{textcolor}\rmfamily\fontsize{8.000000}{9.600000}\selectfont T}%
\end{pgfscope}%
\begin{pgfscope}%
\pgfsetbuttcap%
\pgfsetroundjoin%
\definecolor{currentfill}{rgb}{0.000000,0.000000,0.000000}%
\pgfsetfillcolor{currentfill}%
\pgfsetlinewidth{0.803000pt}%
\definecolor{currentstroke}{rgb}{0.000000,0.000000,0.000000}%
\pgfsetstrokecolor{currentstroke}%
\pgfsetdash{}{0pt}%
\pgfsys@defobject{currentmarker}{\pgfqpoint{-0.048611in}{0.000000in}}{\pgfqpoint{-0.000000in}{0.000000in}}{%
\pgfpathmoveto{\pgfqpoint{-0.000000in}{0.000000in}}%
\pgfpathlineto{\pgfqpoint{-0.048611in}{0.000000in}}%
\pgfusepath{stroke,fill}%
}%
\begin{pgfscope}%
\pgfsys@transformshift{0.380943in}{8.430000in}%
\pgfsys@useobject{currentmarker}{}%
\end{pgfscope}%
\end{pgfscope}%
\begin{pgfscope}%
\definecolor{textcolor}{rgb}{0.000000,0.000000,0.000000}%
\pgfsetstrokecolor{textcolor}%
\pgfsetfillcolor{textcolor}%
\pgftext[x=0.100000in, y=8.391420in, left, base]{\color{textcolor}\rmfamily\fontsize{8.000000}{9.600000}\selectfont W}%
\end{pgfscope}%
\begin{pgfscope}%
\pgfsetbuttcap%
\pgfsetroundjoin%
\definecolor{currentfill}{rgb}{0.000000,0.000000,0.000000}%
\pgfsetfillcolor{currentfill}%
\pgfsetlinewidth{0.803000pt}%
\definecolor{currentstroke}{rgb}{0.000000,0.000000,0.000000}%
\pgfsetstrokecolor{currentstroke}%
\pgfsetdash{}{0pt}%
\pgfsys@defobject{currentmarker}{\pgfqpoint{-0.048611in}{0.000000in}}{\pgfqpoint{-0.000000in}{0.000000in}}{%
\pgfpathmoveto{\pgfqpoint{-0.000000in}{0.000000in}}%
\pgfpathlineto{\pgfqpoint{-0.048611in}{0.000000in}}%
\pgfusepath{stroke,fill}%
}%
\begin{pgfscope}%
\pgfsys@transformshift{0.380943in}{8.342264in}%
\pgfsys@useobject{currentmarker}{}%
\end{pgfscope}%
\end{pgfscope}%
\begin{pgfscope}%
\definecolor{textcolor}{rgb}{0.000000,0.000000,0.000000}%
\pgfsetstrokecolor{textcolor}%
\pgfsetfillcolor{textcolor}%
\pgftext[x=0.135957in, y=8.303684in, left, base]{\color{textcolor}\rmfamily\fontsize{8.000000}{9.600000}\selectfont T}%
\end{pgfscope}%
\begin{pgfscope}%
\pgfsetbuttcap%
\pgfsetroundjoin%
\definecolor{currentfill}{rgb}{0.000000,0.000000,0.000000}%
\pgfsetfillcolor{currentfill}%
\pgfsetlinewidth{0.803000pt}%
\definecolor{currentstroke}{rgb}{0.000000,0.000000,0.000000}%
\pgfsetstrokecolor{currentstroke}%
\pgfsetdash{}{0pt}%
\pgfsys@defobject{currentmarker}{\pgfqpoint{-0.048611in}{0.000000in}}{\pgfqpoint{-0.000000in}{0.000000in}}{%
\pgfpathmoveto{\pgfqpoint{-0.000000in}{0.000000in}}%
\pgfpathlineto{\pgfqpoint{-0.048611in}{0.000000in}}%
\pgfusepath{stroke,fill}%
}%
\begin{pgfscope}%
\pgfsys@transformshift{0.380943in}{8.254529in}%
\pgfsys@useobject{currentmarker}{}%
\end{pgfscope}%
\end{pgfscope}%
\begin{pgfscope}%
\definecolor{textcolor}{rgb}{0.000000,0.000000,0.000000}%
\pgfsetstrokecolor{textcolor}%
\pgfsetfillcolor{textcolor}%
\pgftext[x=0.144213in, y=8.215948in, left, base]{\color{textcolor}\rmfamily\fontsize{8.000000}{9.600000}\selectfont F}%
\end{pgfscope}%
\begin{pgfscope}%
\pgfsetbuttcap%
\pgfsetroundjoin%
\definecolor{currentfill}{rgb}{0.000000,0.000000,0.000000}%
\pgfsetfillcolor{currentfill}%
\pgfsetlinewidth{0.803000pt}%
\definecolor{currentstroke}{rgb}{0.000000,0.000000,0.000000}%
\pgfsetstrokecolor{currentstroke}%
\pgfsetdash{}{0pt}%
\pgfsys@defobject{currentmarker}{\pgfqpoint{-0.048611in}{0.000000in}}{\pgfqpoint{-0.000000in}{0.000000in}}{%
\pgfpathmoveto{\pgfqpoint{-0.000000in}{0.000000in}}%
\pgfpathlineto{\pgfqpoint{-0.048611in}{0.000000in}}%
\pgfusepath{stroke,fill}%
}%
\begin{pgfscope}%
\pgfsys@transformshift{0.380943in}{8.166793in}%
\pgfsys@useobject{currentmarker}{}%
\end{pgfscope}%
\end{pgfscope}%
\begin{pgfscope}%
\definecolor{textcolor}{rgb}{0.000000,0.000000,0.000000}%
\pgfsetstrokecolor{textcolor}%
\pgfsetfillcolor{textcolor}%
\pgftext[x=0.155633in, y=8.128212in, left, base]{\color{textcolor}\rmfamily\fontsize{8.000000}{9.600000}\selectfont S}%
\end{pgfscope}%
\begin{pgfscope}%
\pgfsetbuttcap%
\pgfsetroundjoin%
\definecolor{currentfill}{rgb}{0.000000,0.000000,0.000000}%
\pgfsetfillcolor{currentfill}%
\pgfsetlinewidth{0.803000pt}%
\definecolor{currentstroke}{rgb}{0.000000,0.000000,0.000000}%
\pgfsetstrokecolor{currentstroke}%
\pgfsetdash{}{0pt}%
\pgfsys@defobject{currentmarker}{\pgfqpoint{-0.048611in}{0.000000in}}{\pgfqpoint{-0.000000in}{0.000000in}}{%
\pgfpathmoveto{\pgfqpoint{-0.000000in}{0.000000in}}%
\pgfpathlineto{\pgfqpoint{-0.048611in}{0.000000in}}%
\pgfusepath{stroke,fill}%
}%
\begin{pgfscope}%
\pgfsys@transformshift{0.380943in}{8.079057in}%
\pgfsys@useobject{currentmarker}{}%
\end{pgfscope}%
\end{pgfscope}%
\begin{pgfscope}%
\definecolor{textcolor}{rgb}{0.000000,0.000000,0.000000}%
\pgfsetstrokecolor{textcolor}%
\pgfsetfillcolor{textcolor}%
\pgftext[x=0.155633in, y=8.040477in, left, base]{\color{textcolor}\rmfamily\fontsize{8.000000}{9.600000}\selectfont S}%
\end{pgfscope}%
\begin{pgfscope}%
\definecolor{textcolor}{rgb}{0.000000,0.000000,0.000000}%
\pgfsetstrokecolor{textcolor}%
\pgfsetfillcolor{textcolor}%
\pgftext[x=2.705943in,y=8.816007in,,]{\color{textcolor}\ttfamily\fontsize{14.400000}{17.280000}\selectfont 2017}%
\end{pgfscope}%
\begin{pgfscope}%
\pgfpathrectangle{\pgfqpoint{0.380943in}{6.110189in}}{\pgfqpoint{4.650000in}{0.614151in}}%
\pgfusepath{clip}%
\pgfsetbuttcap%
\pgfsetroundjoin%
\definecolor{currentfill}{rgb}{0.991849,0.986144,0.810181}%
\pgfsetfillcolor{currentfill}%
\pgfsetlinewidth{0.250937pt}%
\definecolor{currentstroke}{rgb}{1.000000,1.000000,1.000000}%
\pgfsetstrokecolor{currentstroke}%
\pgfsetdash{}{0pt}%
\pgfpathmoveto{\pgfqpoint{0.380943in}{6.724340in}}%
\pgfpathlineto{\pgfqpoint{0.468679in}{6.724340in}}%
\pgfpathlineto{\pgfqpoint{0.468679in}{6.636604in}}%
\pgfpathlineto{\pgfqpoint{0.380943in}{6.636604in}}%
\pgfpathlineto{\pgfqpoint{0.380943in}{6.724340in}}%
\pgfusepath{stroke,fill}%
\end{pgfscope}%
\begin{pgfscope}%
\pgfpathrectangle{\pgfqpoint{0.380943in}{6.110189in}}{\pgfqpoint{4.650000in}{0.614151in}}%
\pgfusepath{clip}%
\pgfsetbuttcap%
\pgfsetroundjoin%
\definecolor{currentfill}{rgb}{0.992326,0.765229,0.614840}%
\pgfsetfillcolor{currentfill}%
\pgfsetlinewidth{0.250937pt}%
\definecolor{currentstroke}{rgb}{1.000000,1.000000,1.000000}%
\pgfsetstrokecolor{currentstroke}%
\pgfsetdash{}{0pt}%
\pgfpathmoveto{\pgfqpoint{0.468679in}{6.724340in}}%
\pgfpathlineto{\pgfqpoint{0.556415in}{6.724340in}}%
\pgfpathlineto{\pgfqpoint{0.556415in}{6.636604in}}%
\pgfpathlineto{\pgfqpoint{0.468679in}{6.636604in}}%
\pgfpathlineto{\pgfqpoint{0.468679in}{6.724340in}}%
\pgfusepath{stroke,fill}%
\end{pgfscope}%
\begin{pgfscope}%
\pgfpathrectangle{\pgfqpoint{0.380943in}{6.110189in}}{\pgfqpoint{4.650000in}{0.614151in}}%
\pgfusepath{clip}%
\pgfsetbuttcap%
\pgfsetroundjoin%
\definecolor{currentfill}{rgb}{0.981546,0.459977,0.459977}%
\pgfsetfillcolor{currentfill}%
\pgfsetlinewidth{0.250937pt}%
\definecolor{currentstroke}{rgb}{1.000000,1.000000,1.000000}%
\pgfsetstrokecolor{currentstroke}%
\pgfsetdash{}{0pt}%
\pgfpathmoveto{\pgfqpoint{0.556415in}{6.724340in}}%
\pgfpathlineto{\pgfqpoint{0.644151in}{6.724340in}}%
\pgfpathlineto{\pgfqpoint{0.644151in}{6.636604in}}%
\pgfpathlineto{\pgfqpoint{0.556415in}{6.636604in}}%
\pgfpathlineto{\pgfqpoint{0.556415in}{6.724340in}}%
\pgfusepath{stroke,fill}%
\end{pgfscope}%
\begin{pgfscope}%
\pgfpathrectangle{\pgfqpoint{0.380943in}{6.110189in}}{\pgfqpoint{4.650000in}{0.614151in}}%
\pgfusepath{clip}%
\pgfsetbuttcap%
\pgfsetroundjoin%
\definecolor{currentfill}{rgb}{0.968166,0.945882,0.748604}%
\pgfsetfillcolor{currentfill}%
\pgfsetlinewidth{0.250937pt}%
\definecolor{currentstroke}{rgb}{1.000000,1.000000,1.000000}%
\pgfsetstrokecolor{currentstroke}%
\pgfsetdash{}{0pt}%
\pgfpathmoveto{\pgfqpoint{0.644151in}{6.724340in}}%
\pgfpathlineto{\pgfqpoint{0.731886in}{6.724340in}}%
\pgfpathlineto{\pgfqpoint{0.731886in}{6.636604in}}%
\pgfpathlineto{\pgfqpoint{0.644151in}{6.636604in}}%
\pgfpathlineto{\pgfqpoint{0.644151in}{6.724340in}}%
\pgfusepath{stroke,fill}%
\end{pgfscope}%
\begin{pgfscope}%
\pgfpathrectangle{\pgfqpoint{0.380943in}{6.110189in}}{\pgfqpoint{4.650000in}{0.614151in}}%
\pgfusepath{clip}%
\pgfsetbuttcap%
\pgfsetroundjoin%
\definecolor{currentfill}{rgb}{0.962414,0.923552,0.722891}%
\pgfsetfillcolor{currentfill}%
\pgfsetlinewidth{0.250937pt}%
\definecolor{currentstroke}{rgb}{1.000000,1.000000,1.000000}%
\pgfsetstrokecolor{currentstroke}%
\pgfsetdash{}{0pt}%
\pgfpathmoveto{\pgfqpoint{0.731886in}{6.724340in}}%
\pgfpathlineto{\pgfqpoint{0.819622in}{6.724340in}}%
\pgfpathlineto{\pgfqpoint{0.819622in}{6.636604in}}%
\pgfpathlineto{\pgfqpoint{0.731886in}{6.636604in}}%
\pgfpathlineto{\pgfqpoint{0.731886in}{6.724340in}}%
\pgfusepath{stroke,fill}%
\end{pgfscope}%
\begin{pgfscope}%
\pgfpathrectangle{\pgfqpoint{0.380943in}{6.110189in}}{\pgfqpoint{4.650000in}{0.614151in}}%
\pgfusepath{clip}%
\pgfsetbuttcap%
\pgfsetroundjoin%
\definecolor{currentfill}{rgb}{0.992326,0.765229,0.614840}%
\pgfsetfillcolor{currentfill}%
\pgfsetlinewidth{0.250937pt}%
\definecolor{currentstroke}{rgb}{1.000000,1.000000,1.000000}%
\pgfsetstrokecolor{currentstroke}%
\pgfsetdash{}{0pt}%
\pgfpathmoveto{\pgfqpoint{0.819622in}{6.724340in}}%
\pgfpathlineto{\pgfqpoint{0.907358in}{6.724340in}}%
\pgfpathlineto{\pgfqpoint{0.907358in}{6.636604in}}%
\pgfpathlineto{\pgfqpoint{0.819622in}{6.636604in}}%
\pgfpathlineto{\pgfqpoint{0.819622in}{6.724340in}}%
\pgfusepath{stroke,fill}%
\end{pgfscope}%
\begin{pgfscope}%
\pgfpathrectangle{\pgfqpoint{0.380943in}{6.110189in}}{\pgfqpoint{4.650000in}{0.614151in}}%
\pgfusepath{clip}%
\pgfsetbuttcap%
\pgfsetroundjoin%
\definecolor{currentfill}{rgb}{0.996571,0.720538,0.589189}%
\pgfsetfillcolor{currentfill}%
\pgfsetlinewidth{0.250937pt}%
\definecolor{currentstroke}{rgb}{1.000000,1.000000,1.000000}%
\pgfsetstrokecolor{currentstroke}%
\pgfsetdash{}{0pt}%
\pgfpathmoveto{\pgfqpoint{0.907358in}{6.724340in}}%
\pgfpathlineto{\pgfqpoint{0.995094in}{6.724340in}}%
\pgfpathlineto{\pgfqpoint{0.995094in}{6.636604in}}%
\pgfpathlineto{\pgfqpoint{0.907358in}{6.636604in}}%
\pgfpathlineto{\pgfqpoint{0.907358in}{6.724340in}}%
\pgfusepath{stroke,fill}%
\end{pgfscope}%
\begin{pgfscope}%
\pgfpathrectangle{\pgfqpoint{0.380943in}{6.110189in}}{\pgfqpoint{4.650000in}{0.614151in}}%
\pgfusepath{clip}%
\pgfsetbuttcap%
\pgfsetroundjoin%
\definecolor{currentfill}{rgb}{0.972549,0.870588,0.692810}%
\pgfsetfillcolor{currentfill}%
\pgfsetlinewidth{0.250937pt}%
\definecolor{currentstroke}{rgb}{1.000000,1.000000,1.000000}%
\pgfsetstrokecolor{currentstroke}%
\pgfsetdash{}{0pt}%
\pgfpathmoveto{\pgfqpoint{0.995094in}{6.724340in}}%
\pgfpathlineto{\pgfqpoint{1.082830in}{6.724340in}}%
\pgfpathlineto{\pgfqpoint{1.082830in}{6.636604in}}%
\pgfpathlineto{\pgfqpoint{0.995094in}{6.636604in}}%
\pgfpathlineto{\pgfqpoint{0.995094in}{6.724340in}}%
\pgfusepath{stroke,fill}%
\end{pgfscope}%
\begin{pgfscope}%
\pgfpathrectangle{\pgfqpoint{0.380943in}{6.110189in}}{\pgfqpoint{4.650000in}{0.614151in}}%
\pgfusepath{clip}%
\pgfsetbuttcap%
\pgfsetroundjoin%
\definecolor{currentfill}{rgb}{1.000000,0.557862,0.511772}%
\pgfsetfillcolor{currentfill}%
\pgfsetlinewidth{0.250937pt}%
\definecolor{currentstroke}{rgb}{1.000000,1.000000,1.000000}%
\pgfsetstrokecolor{currentstroke}%
\pgfsetdash{}{0pt}%
\pgfpathmoveto{\pgfqpoint{1.082830in}{6.724340in}}%
\pgfpathlineto{\pgfqpoint{1.170566in}{6.724340in}}%
\pgfpathlineto{\pgfqpoint{1.170566in}{6.636604in}}%
\pgfpathlineto{\pgfqpoint{1.082830in}{6.636604in}}%
\pgfpathlineto{\pgfqpoint{1.082830in}{6.724340in}}%
\pgfusepath{stroke,fill}%
\end{pgfscope}%
\begin{pgfscope}%
\pgfpathrectangle{\pgfqpoint{0.380943in}{6.110189in}}{\pgfqpoint{4.650000in}{0.614151in}}%
\pgfusepath{clip}%
\pgfsetbuttcap%
\pgfsetroundjoin%
\definecolor{currentfill}{rgb}{0.986759,0.806398,0.641200}%
\pgfsetfillcolor{currentfill}%
\pgfsetlinewidth{0.250937pt}%
\definecolor{currentstroke}{rgb}{1.000000,1.000000,1.000000}%
\pgfsetstrokecolor{currentstroke}%
\pgfsetdash{}{0pt}%
\pgfpathmoveto{\pgfqpoint{1.170566in}{6.724340in}}%
\pgfpathlineto{\pgfqpoint{1.258302in}{6.724340in}}%
\pgfpathlineto{\pgfqpoint{1.258302in}{6.636604in}}%
\pgfpathlineto{\pgfqpoint{1.170566in}{6.636604in}}%
\pgfpathlineto{\pgfqpoint{1.170566in}{6.724340in}}%
\pgfusepath{stroke,fill}%
\end{pgfscope}%
\begin{pgfscope}%
\pgfpathrectangle{\pgfqpoint{0.380943in}{6.110189in}}{\pgfqpoint{4.650000in}{0.614151in}}%
\pgfusepath{clip}%
\pgfsetbuttcap%
\pgfsetroundjoin%
\definecolor{currentfill}{rgb}{0.979654,0.837186,0.669619}%
\pgfsetfillcolor{currentfill}%
\pgfsetlinewidth{0.250937pt}%
\definecolor{currentstroke}{rgb}{1.000000,1.000000,1.000000}%
\pgfsetstrokecolor{currentstroke}%
\pgfsetdash{}{0pt}%
\pgfpathmoveto{\pgfqpoint{1.258302in}{6.724340in}}%
\pgfpathlineto{\pgfqpoint{1.346037in}{6.724340in}}%
\pgfpathlineto{\pgfqpoint{1.346037in}{6.636604in}}%
\pgfpathlineto{\pgfqpoint{1.258302in}{6.636604in}}%
\pgfpathlineto{\pgfqpoint{1.258302in}{6.724340in}}%
\pgfusepath{stroke,fill}%
\end{pgfscope}%
\begin{pgfscope}%
\pgfpathrectangle{\pgfqpoint{0.380943in}{6.110189in}}{\pgfqpoint{4.650000in}{0.614151in}}%
\pgfusepath{clip}%
\pgfsetbuttcap%
\pgfsetroundjoin%
\definecolor{currentfill}{rgb}{0.996571,0.720538,0.589189}%
\pgfsetfillcolor{currentfill}%
\pgfsetlinewidth{0.250937pt}%
\definecolor{currentstroke}{rgb}{1.000000,1.000000,1.000000}%
\pgfsetstrokecolor{currentstroke}%
\pgfsetdash{}{0pt}%
\pgfpathmoveto{\pgfqpoint{1.346037in}{6.724340in}}%
\pgfpathlineto{\pgfqpoint{1.433773in}{6.724340in}}%
\pgfpathlineto{\pgfqpoint{1.433773in}{6.636604in}}%
\pgfpathlineto{\pgfqpoint{1.346037in}{6.636604in}}%
\pgfpathlineto{\pgfqpoint{1.346037in}{6.724340in}}%
\pgfusepath{stroke,fill}%
\end{pgfscope}%
\begin{pgfscope}%
\pgfpathrectangle{\pgfqpoint{0.380943in}{6.110189in}}{\pgfqpoint{4.650000in}{0.614151in}}%
\pgfusepath{clip}%
\pgfsetbuttcap%
\pgfsetroundjoin%
\definecolor{currentfill}{rgb}{0.979654,0.837186,0.669619}%
\pgfsetfillcolor{currentfill}%
\pgfsetlinewidth{0.250937pt}%
\definecolor{currentstroke}{rgb}{1.000000,1.000000,1.000000}%
\pgfsetstrokecolor{currentstroke}%
\pgfsetdash{}{0pt}%
\pgfpathmoveto{\pgfqpoint{1.433773in}{6.724340in}}%
\pgfpathlineto{\pgfqpoint{1.521509in}{6.724340in}}%
\pgfpathlineto{\pgfqpoint{1.521509in}{6.636604in}}%
\pgfpathlineto{\pgfqpoint{1.433773in}{6.636604in}}%
\pgfpathlineto{\pgfqpoint{1.433773in}{6.724340in}}%
\pgfusepath{stroke,fill}%
\end{pgfscope}%
\begin{pgfscope}%
\pgfpathrectangle{\pgfqpoint{0.380943in}{6.110189in}}{\pgfqpoint{4.650000in}{0.614151in}}%
\pgfusepath{clip}%
\pgfsetbuttcap%
\pgfsetroundjoin%
\definecolor{currentfill}{rgb}{0.991849,0.986144,0.810181}%
\pgfsetfillcolor{currentfill}%
\pgfsetlinewidth{0.250937pt}%
\definecolor{currentstroke}{rgb}{1.000000,1.000000,1.000000}%
\pgfsetstrokecolor{currentstroke}%
\pgfsetdash{}{0pt}%
\pgfpathmoveto{\pgfqpoint{1.521509in}{6.724340in}}%
\pgfpathlineto{\pgfqpoint{1.609245in}{6.724340in}}%
\pgfpathlineto{\pgfqpoint{1.609245in}{6.636604in}}%
\pgfpathlineto{\pgfqpoint{1.521509in}{6.636604in}}%
\pgfpathlineto{\pgfqpoint{1.521509in}{6.724340in}}%
\pgfusepath{stroke,fill}%
\end{pgfscope}%
\begin{pgfscope}%
\pgfpathrectangle{\pgfqpoint{0.380943in}{6.110189in}}{\pgfqpoint{4.650000in}{0.614151in}}%
\pgfusepath{clip}%
\pgfsetbuttcap%
\pgfsetroundjoin%
\definecolor{currentfill}{rgb}{0.979654,0.837186,0.669619}%
\pgfsetfillcolor{currentfill}%
\pgfsetlinewidth{0.250937pt}%
\definecolor{currentstroke}{rgb}{1.000000,1.000000,1.000000}%
\pgfsetstrokecolor{currentstroke}%
\pgfsetdash{}{0pt}%
\pgfpathmoveto{\pgfqpoint{1.609245in}{6.724340in}}%
\pgfpathlineto{\pgfqpoint{1.696981in}{6.724340in}}%
\pgfpathlineto{\pgfqpoint{1.696981in}{6.636604in}}%
\pgfpathlineto{\pgfqpoint{1.609245in}{6.636604in}}%
\pgfpathlineto{\pgfqpoint{1.609245in}{6.724340in}}%
\pgfusepath{stroke,fill}%
\end{pgfscope}%
\begin{pgfscope}%
\pgfpathrectangle{\pgfqpoint{0.380943in}{6.110189in}}{\pgfqpoint{4.650000in}{0.614151in}}%
\pgfusepath{clip}%
\pgfsetbuttcap%
\pgfsetroundjoin%
\definecolor{currentfill}{rgb}{0.986759,0.806398,0.641200}%
\pgfsetfillcolor{currentfill}%
\pgfsetlinewidth{0.250937pt}%
\definecolor{currentstroke}{rgb}{1.000000,1.000000,1.000000}%
\pgfsetstrokecolor{currentstroke}%
\pgfsetdash{}{0pt}%
\pgfpathmoveto{\pgfqpoint{1.696981in}{6.724340in}}%
\pgfpathlineto{\pgfqpoint{1.784717in}{6.724340in}}%
\pgfpathlineto{\pgfqpoint{1.784717in}{6.636604in}}%
\pgfpathlineto{\pgfqpoint{1.696981in}{6.636604in}}%
\pgfpathlineto{\pgfqpoint{1.696981in}{6.724340in}}%
\pgfusepath{stroke,fill}%
\end{pgfscope}%
\begin{pgfscope}%
\pgfpathrectangle{\pgfqpoint{0.380943in}{6.110189in}}{\pgfqpoint{4.650000in}{0.614151in}}%
\pgfusepath{clip}%
\pgfsetbuttcap%
\pgfsetroundjoin%
\definecolor{currentfill}{rgb}{0.986759,0.806398,0.641200}%
\pgfsetfillcolor{currentfill}%
\pgfsetlinewidth{0.250937pt}%
\definecolor{currentstroke}{rgb}{1.000000,1.000000,1.000000}%
\pgfsetstrokecolor{currentstroke}%
\pgfsetdash{}{0pt}%
\pgfpathmoveto{\pgfqpoint{1.784717in}{6.724340in}}%
\pgfpathlineto{\pgfqpoint{1.872452in}{6.724340in}}%
\pgfpathlineto{\pgfqpoint{1.872452in}{6.636604in}}%
\pgfpathlineto{\pgfqpoint{1.784717in}{6.636604in}}%
\pgfpathlineto{\pgfqpoint{1.784717in}{6.724340in}}%
\pgfusepath{stroke,fill}%
\end{pgfscope}%
\begin{pgfscope}%
\pgfpathrectangle{\pgfqpoint{0.380943in}{6.110189in}}{\pgfqpoint{4.650000in}{0.614151in}}%
\pgfusepath{clip}%
\pgfsetbuttcap%
\pgfsetroundjoin%
\definecolor{currentfill}{rgb}{1.000000,0.557862,0.511772}%
\pgfsetfillcolor{currentfill}%
\pgfsetlinewidth{0.250937pt}%
\definecolor{currentstroke}{rgb}{1.000000,1.000000,1.000000}%
\pgfsetstrokecolor{currentstroke}%
\pgfsetdash{}{0pt}%
\pgfpathmoveto{\pgfqpoint{1.872452in}{6.724340in}}%
\pgfpathlineto{\pgfqpoint{1.960188in}{6.724340in}}%
\pgfpathlineto{\pgfqpoint{1.960188in}{6.636604in}}%
\pgfpathlineto{\pgfqpoint{1.872452in}{6.636604in}}%
\pgfpathlineto{\pgfqpoint{1.872452in}{6.724340in}}%
\pgfusepath{stroke,fill}%
\end{pgfscope}%
\begin{pgfscope}%
\pgfpathrectangle{\pgfqpoint{0.380943in}{6.110189in}}{\pgfqpoint{4.650000in}{0.614151in}}%
\pgfusepath{clip}%
\pgfsetbuttcap%
\pgfsetroundjoin%
\definecolor{currentfill}{rgb}{0.965444,0.906113,0.711757}%
\pgfsetfillcolor{currentfill}%
\pgfsetlinewidth{0.250937pt}%
\definecolor{currentstroke}{rgb}{1.000000,1.000000,1.000000}%
\pgfsetstrokecolor{currentstroke}%
\pgfsetdash{}{0pt}%
\pgfpathmoveto{\pgfqpoint{1.960188in}{6.724340in}}%
\pgfpathlineto{\pgfqpoint{2.047924in}{6.724340in}}%
\pgfpathlineto{\pgfqpoint{2.047924in}{6.636604in}}%
\pgfpathlineto{\pgfqpoint{1.960188in}{6.636604in}}%
\pgfpathlineto{\pgfqpoint{1.960188in}{6.724340in}}%
\pgfusepath{stroke,fill}%
\end{pgfscope}%
\begin{pgfscope}%
\pgfpathrectangle{\pgfqpoint{0.380943in}{6.110189in}}{\pgfqpoint{4.650000in}{0.614151in}}%
\pgfusepath{clip}%
\pgfsetbuttcap%
\pgfsetroundjoin%
\definecolor{currentfill}{rgb}{0.996571,0.720538,0.589189}%
\pgfsetfillcolor{currentfill}%
\pgfsetlinewidth{0.250937pt}%
\definecolor{currentstroke}{rgb}{1.000000,1.000000,1.000000}%
\pgfsetstrokecolor{currentstroke}%
\pgfsetdash{}{0pt}%
\pgfpathmoveto{\pgfqpoint{2.047924in}{6.724340in}}%
\pgfpathlineto{\pgfqpoint{2.135660in}{6.724340in}}%
\pgfpathlineto{\pgfqpoint{2.135660in}{6.636604in}}%
\pgfpathlineto{\pgfqpoint{2.047924in}{6.636604in}}%
\pgfpathlineto{\pgfqpoint{2.047924in}{6.724340in}}%
\pgfusepath{stroke,fill}%
\end{pgfscope}%
\begin{pgfscope}%
\pgfpathrectangle{\pgfqpoint{0.380943in}{6.110189in}}{\pgfqpoint{4.650000in}{0.614151in}}%
\pgfusepath{clip}%
\pgfsetbuttcap%
\pgfsetroundjoin%
\definecolor{currentfill}{rgb}{1.000000,1.000000,0.870204}%
\pgfsetfillcolor{currentfill}%
\pgfsetlinewidth{0.250937pt}%
\definecolor{currentstroke}{rgb}{1.000000,1.000000,1.000000}%
\pgfsetstrokecolor{currentstroke}%
\pgfsetdash{}{0pt}%
\pgfpathmoveto{\pgfqpoint{2.135660in}{6.724340in}}%
\pgfpathlineto{\pgfqpoint{2.223396in}{6.724340in}}%
\pgfpathlineto{\pgfqpoint{2.223396in}{6.636604in}}%
\pgfpathlineto{\pgfqpoint{2.135660in}{6.636604in}}%
\pgfpathlineto{\pgfqpoint{2.135660in}{6.724340in}}%
\pgfusepath{stroke,fill}%
\end{pgfscope}%
\begin{pgfscope}%
\pgfpathrectangle{\pgfqpoint{0.380943in}{6.110189in}}{\pgfqpoint{4.650000in}{0.614151in}}%
\pgfusepath{clip}%
\pgfsetbuttcap%
\pgfsetroundjoin%
\definecolor{currentfill}{rgb}{0.962414,0.923552,0.722891}%
\pgfsetfillcolor{currentfill}%
\pgfsetlinewidth{0.250937pt}%
\definecolor{currentstroke}{rgb}{1.000000,1.000000,1.000000}%
\pgfsetstrokecolor{currentstroke}%
\pgfsetdash{}{0pt}%
\pgfpathmoveto{\pgfqpoint{2.223396in}{6.724340in}}%
\pgfpathlineto{\pgfqpoint{2.311132in}{6.724340in}}%
\pgfpathlineto{\pgfqpoint{2.311132in}{6.636604in}}%
\pgfpathlineto{\pgfqpoint{2.223396in}{6.636604in}}%
\pgfpathlineto{\pgfqpoint{2.223396in}{6.724340in}}%
\pgfusepath{stroke,fill}%
\end{pgfscope}%
\begin{pgfscope}%
\pgfpathrectangle{\pgfqpoint{0.380943in}{6.110189in}}{\pgfqpoint{4.650000in}{0.614151in}}%
\pgfusepath{clip}%
\pgfsetbuttcap%
\pgfsetroundjoin%
\definecolor{currentfill}{rgb}{0.992326,0.765229,0.614840}%
\pgfsetfillcolor{currentfill}%
\pgfsetlinewidth{0.250937pt}%
\definecolor{currentstroke}{rgb}{1.000000,1.000000,1.000000}%
\pgfsetstrokecolor{currentstroke}%
\pgfsetdash{}{0pt}%
\pgfpathmoveto{\pgfqpoint{2.311132in}{6.724340in}}%
\pgfpathlineto{\pgfqpoint{2.398868in}{6.724340in}}%
\pgfpathlineto{\pgfqpoint{2.398868in}{6.636604in}}%
\pgfpathlineto{\pgfqpoint{2.311132in}{6.636604in}}%
\pgfpathlineto{\pgfqpoint{2.311132in}{6.724340in}}%
\pgfusepath{stroke,fill}%
\end{pgfscope}%
\begin{pgfscope}%
\pgfpathrectangle{\pgfqpoint{0.380943in}{6.110189in}}{\pgfqpoint{4.650000in}{0.614151in}}%
\pgfusepath{clip}%
\pgfsetbuttcap%
\pgfsetroundjoin%
\definecolor{currentfill}{rgb}{0.992326,0.765229,0.614840}%
\pgfsetfillcolor{currentfill}%
\pgfsetlinewidth{0.250937pt}%
\definecolor{currentstroke}{rgb}{1.000000,1.000000,1.000000}%
\pgfsetstrokecolor{currentstroke}%
\pgfsetdash{}{0pt}%
\pgfpathmoveto{\pgfqpoint{2.398868in}{6.724340in}}%
\pgfpathlineto{\pgfqpoint{2.486603in}{6.724340in}}%
\pgfpathlineto{\pgfqpoint{2.486603in}{6.636604in}}%
\pgfpathlineto{\pgfqpoint{2.398868in}{6.636604in}}%
\pgfpathlineto{\pgfqpoint{2.398868in}{6.724340in}}%
\pgfusepath{stroke,fill}%
\end{pgfscope}%
\begin{pgfscope}%
\pgfpathrectangle{\pgfqpoint{0.380943in}{6.110189in}}{\pgfqpoint{4.650000in}{0.614151in}}%
\pgfusepath{clip}%
\pgfsetbuttcap%
\pgfsetroundjoin%
\definecolor{currentfill}{rgb}{0.996571,0.720538,0.589189}%
\pgfsetfillcolor{currentfill}%
\pgfsetlinewidth{0.250937pt}%
\definecolor{currentstroke}{rgb}{1.000000,1.000000,1.000000}%
\pgfsetstrokecolor{currentstroke}%
\pgfsetdash{}{0pt}%
\pgfpathmoveto{\pgfqpoint{2.486603in}{6.724340in}}%
\pgfpathlineto{\pgfqpoint{2.574339in}{6.724340in}}%
\pgfpathlineto{\pgfqpoint{2.574339in}{6.636604in}}%
\pgfpathlineto{\pgfqpoint{2.486603in}{6.636604in}}%
\pgfpathlineto{\pgfqpoint{2.486603in}{6.724340in}}%
\pgfusepath{stroke,fill}%
\end{pgfscope}%
\begin{pgfscope}%
\pgfpathrectangle{\pgfqpoint{0.380943in}{6.110189in}}{\pgfqpoint{4.650000in}{0.614151in}}%
\pgfusepath{clip}%
\pgfsetbuttcap%
\pgfsetroundjoin%
\definecolor{currentfill}{rgb}{0.986759,0.806398,0.641200}%
\pgfsetfillcolor{currentfill}%
\pgfsetlinewidth{0.250937pt}%
\definecolor{currentstroke}{rgb}{1.000000,1.000000,1.000000}%
\pgfsetstrokecolor{currentstroke}%
\pgfsetdash{}{0pt}%
\pgfpathmoveto{\pgfqpoint{2.574339in}{6.724340in}}%
\pgfpathlineto{\pgfqpoint{2.662075in}{6.724340in}}%
\pgfpathlineto{\pgfqpoint{2.662075in}{6.636604in}}%
\pgfpathlineto{\pgfqpoint{2.574339in}{6.636604in}}%
\pgfpathlineto{\pgfqpoint{2.574339in}{6.724340in}}%
\pgfusepath{stroke,fill}%
\end{pgfscope}%
\begin{pgfscope}%
\pgfpathrectangle{\pgfqpoint{0.380943in}{6.110189in}}{\pgfqpoint{4.650000in}{0.614151in}}%
\pgfusepath{clip}%
\pgfsetbuttcap%
\pgfsetroundjoin%
\definecolor{currentfill}{rgb}{0.972549,0.870588,0.692810}%
\pgfsetfillcolor{currentfill}%
\pgfsetlinewidth{0.250937pt}%
\definecolor{currentstroke}{rgb}{1.000000,1.000000,1.000000}%
\pgfsetstrokecolor{currentstroke}%
\pgfsetdash{}{0pt}%
\pgfpathmoveto{\pgfqpoint{2.662075in}{6.724340in}}%
\pgfpathlineto{\pgfqpoint{2.749811in}{6.724340in}}%
\pgfpathlineto{\pgfqpoint{2.749811in}{6.636604in}}%
\pgfpathlineto{\pgfqpoint{2.662075in}{6.636604in}}%
\pgfpathlineto{\pgfqpoint{2.662075in}{6.724340in}}%
\pgfusepath{stroke,fill}%
\end{pgfscope}%
\begin{pgfscope}%
\pgfpathrectangle{\pgfqpoint{0.380943in}{6.110189in}}{\pgfqpoint{4.650000in}{0.614151in}}%
\pgfusepath{clip}%
\pgfsetbuttcap%
\pgfsetroundjoin%
\definecolor{currentfill}{rgb}{0.992326,0.765229,0.614840}%
\pgfsetfillcolor{currentfill}%
\pgfsetlinewidth{0.250937pt}%
\definecolor{currentstroke}{rgb}{1.000000,1.000000,1.000000}%
\pgfsetstrokecolor{currentstroke}%
\pgfsetdash{}{0pt}%
\pgfpathmoveto{\pgfqpoint{2.749811in}{6.724340in}}%
\pgfpathlineto{\pgfqpoint{2.837547in}{6.724340in}}%
\pgfpathlineto{\pgfqpoint{2.837547in}{6.636604in}}%
\pgfpathlineto{\pgfqpoint{2.749811in}{6.636604in}}%
\pgfpathlineto{\pgfqpoint{2.749811in}{6.724340in}}%
\pgfusepath{stroke,fill}%
\end{pgfscope}%
\begin{pgfscope}%
\pgfpathrectangle{\pgfqpoint{0.380943in}{6.110189in}}{\pgfqpoint{4.650000in}{0.614151in}}%
\pgfusepath{clip}%
\pgfsetbuttcap%
\pgfsetroundjoin%
\definecolor{currentfill}{rgb}{0.986759,0.806398,0.641200}%
\pgfsetfillcolor{currentfill}%
\pgfsetlinewidth{0.250937pt}%
\definecolor{currentstroke}{rgb}{1.000000,1.000000,1.000000}%
\pgfsetstrokecolor{currentstroke}%
\pgfsetdash{}{0pt}%
\pgfpathmoveto{\pgfqpoint{2.837547in}{6.724340in}}%
\pgfpathlineto{\pgfqpoint{2.925283in}{6.724340in}}%
\pgfpathlineto{\pgfqpoint{2.925283in}{6.636604in}}%
\pgfpathlineto{\pgfqpoint{2.837547in}{6.636604in}}%
\pgfpathlineto{\pgfqpoint{2.837547in}{6.724340in}}%
\pgfusepath{stroke,fill}%
\end{pgfscope}%
\begin{pgfscope}%
\pgfpathrectangle{\pgfqpoint{0.380943in}{6.110189in}}{\pgfqpoint{4.650000in}{0.614151in}}%
\pgfusepath{clip}%
\pgfsetbuttcap%
\pgfsetroundjoin%
\definecolor{currentfill}{rgb}{0.965444,0.906113,0.711757}%
\pgfsetfillcolor{currentfill}%
\pgfsetlinewidth{0.250937pt}%
\definecolor{currentstroke}{rgb}{1.000000,1.000000,1.000000}%
\pgfsetstrokecolor{currentstroke}%
\pgfsetdash{}{0pt}%
\pgfpathmoveto{\pgfqpoint{2.925283in}{6.724340in}}%
\pgfpathlineto{\pgfqpoint{3.013019in}{6.724340in}}%
\pgfpathlineto{\pgfqpoint{3.013019in}{6.636604in}}%
\pgfpathlineto{\pgfqpoint{2.925283in}{6.636604in}}%
\pgfpathlineto{\pgfqpoint{2.925283in}{6.724340in}}%
\pgfusepath{stroke,fill}%
\end{pgfscope}%
\begin{pgfscope}%
\pgfpathrectangle{\pgfqpoint{0.380943in}{6.110189in}}{\pgfqpoint{4.650000in}{0.614151in}}%
\pgfusepath{clip}%
\pgfsetbuttcap%
\pgfsetroundjoin%
\definecolor{currentfill}{rgb}{0.979654,0.837186,0.669619}%
\pgfsetfillcolor{currentfill}%
\pgfsetlinewidth{0.250937pt}%
\definecolor{currentstroke}{rgb}{1.000000,1.000000,1.000000}%
\pgfsetstrokecolor{currentstroke}%
\pgfsetdash{}{0pt}%
\pgfpathmoveto{\pgfqpoint{3.013019in}{6.724340in}}%
\pgfpathlineto{\pgfqpoint{3.100754in}{6.724340in}}%
\pgfpathlineto{\pgfqpoint{3.100754in}{6.636604in}}%
\pgfpathlineto{\pgfqpoint{3.013019in}{6.636604in}}%
\pgfpathlineto{\pgfqpoint{3.013019in}{6.724340in}}%
\pgfusepath{stroke,fill}%
\end{pgfscope}%
\begin{pgfscope}%
\pgfpathrectangle{\pgfqpoint{0.380943in}{6.110189in}}{\pgfqpoint{4.650000in}{0.614151in}}%
\pgfusepath{clip}%
\pgfsetbuttcap%
\pgfsetroundjoin%
\definecolor{currentfill}{rgb}{0.962414,0.923552,0.722891}%
\pgfsetfillcolor{currentfill}%
\pgfsetlinewidth{0.250937pt}%
\definecolor{currentstroke}{rgb}{1.000000,1.000000,1.000000}%
\pgfsetstrokecolor{currentstroke}%
\pgfsetdash{}{0pt}%
\pgfpathmoveto{\pgfqpoint{3.100754in}{6.724340in}}%
\pgfpathlineto{\pgfqpoint{3.188490in}{6.724340in}}%
\pgfpathlineto{\pgfqpoint{3.188490in}{6.636604in}}%
\pgfpathlineto{\pgfqpoint{3.100754in}{6.636604in}}%
\pgfpathlineto{\pgfqpoint{3.100754in}{6.724340in}}%
\pgfusepath{stroke,fill}%
\end{pgfscope}%
\begin{pgfscope}%
\pgfpathrectangle{\pgfqpoint{0.380943in}{6.110189in}}{\pgfqpoint{4.650000in}{0.614151in}}%
\pgfusepath{clip}%
\pgfsetbuttcap%
\pgfsetroundjoin%
\definecolor{currentfill}{rgb}{0.992326,0.765229,0.614840}%
\pgfsetfillcolor{currentfill}%
\pgfsetlinewidth{0.250937pt}%
\definecolor{currentstroke}{rgb}{1.000000,1.000000,1.000000}%
\pgfsetstrokecolor{currentstroke}%
\pgfsetdash{}{0pt}%
\pgfpathmoveto{\pgfqpoint{3.188490in}{6.724340in}}%
\pgfpathlineto{\pgfqpoint{3.276226in}{6.724340in}}%
\pgfpathlineto{\pgfqpoint{3.276226in}{6.636604in}}%
\pgfpathlineto{\pgfqpoint{3.188490in}{6.636604in}}%
\pgfpathlineto{\pgfqpoint{3.188490in}{6.724340in}}%
\pgfusepath{stroke,fill}%
\end{pgfscope}%
\begin{pgfscope}%
\pgfpathrectangle{\pgfqpoint{0.380943in}{6.110189in}}{\pgfqpoint{4.650000in}{0.614151in}}%
\pgfusepath{clip}%
\pgfsetbuttcap%
\pgfsetroundjoin%
\definecolor{currentfill}{rgb}{0.968166,0.945882,0.748604}%
\pgfsetfillcolor{currentfill}%
\pgfsetlinewidth{0.250937pt}%
\definecolor{currentstroke}{rgb}{1.000000,1.000000,1.000000}%
\pgfsetstrokecolor{currentstroke}%
\pgfsetdash{}{0pt}%
\pgfpathmoveto{\pgfqpoint{3.276226in}{6.724340in}}%
\pgfpathlineto{\pgfqpoint{3.363962in}{6.724340in}}%
\pgfpathlineto{\pgfqpoint{3.363962in}{6.636604in}}%
\pgfpathlineto{\pgfqpoint{3.276226in}{6.636604in}}%
\pgfpathlineto{\pgfqpoint{3.276226in}{6.724340in}}%
\pgfusepath{stroke,fill}%
\end{pgfscope}%
\begin{pgfscope}%
\pgfpathrectangle{\pgfqpoint{0.380943in}{6.110189in}}{\pgfqpoint{4.650000in}{0.614151in}}%
\pgfusepath{clip}%
\pgfsetbuttcap%
\pgfsetroundjoin%
\definecolor{currentfill}{rgb}{0.991849,0.986144,0.810181}%
\pgfsetfillcolor{currentfill}%
\pgfsetlinewidth{0.250937pt}%
\definecolor{currentstroke}{rgb}{1.000000,1.000000,1.000000}%
\pgfsetstrokecolor{currentstroke}%
\pgfsetdash{}{0pt}%
\pgfpathmoveto{\pgfqpoint{3.363962in}{6.724340in}}%
\pgfpathlineto{\pgfqpoint{3.451698in}{6.724340in}}%
\pgfpathlineto{\pgfqpoint{3.451698in}{6.636604in}}%
\pgfpathlineto{\pgfqpoint{3.363962in}{6.636604in}}%
\pgfpathlineto{\pgfqpoint{3.363962in}{6.724340in}}%
\pgfusepath{stroke,fill}%
\end{pgfscope}%
\begin{pgfscope}%
\pgfpathrectangle{\pgfqpoint{0.380943in}{6.110189in}}{\pgfqpoint{4.650000in}{0.614151in}}%
\pgfusepath{clip}%
\pgfsetbuttcap%
\pgfsetroundjoin%
\definecolor{currentfill}{rgb}{0.962414,0.923552,0.722891}%
\pgfsetfillcolor{currentfill}%
\pgfsetlinewidth{0.250937pt}%
\definecolor{currentstroke}{rgb}{1.000000,1.000000,1.000000}%
\pgfsetstrokecolor{currentstroke}%
\pgfsetdash{}{0pt}%
\pgfpathmoveto{\pgfqpoint{3.451698in}{6.724340in}}%
\pgfpathlineto{\pgfqpoint{3.539434in}{6.724340in}}%
\pgfpathlineto{\pgfqpoint{3.539434in}{6.636604in}}%
\pgfpathlineto{\pgfqpoint{3.451698in}{6.636604in}}%
\pgfpathlineto{\pgfqpoint{3.451698in}{6.724340in}}%
\pgfusepath{stroke,fill}%
\end{pgfscope}%
\begin{pgfscope}%
\pgfpathrectangle{\pgfqpoint{0.380943in}{6.110189in}}{\pgfqpoint{4.650000in}{0.614151in}}%
\pgfusepath{clip}%
\pgfsetbuttcap%
\pgfsetroundjoin%
\definecolor{currentfill}{rgb}{0.965444,0.906113,0.711757}%
\pgfsetfillcolor{currentfill}%
\pgfsetlinewidth{0.250937pt}%
\definecolor{currentstroke}{rgb}{1.000000,1.000000,1.000000}%
\pgfsetstrokecolor{currentstroke}%
\pgfsetdash{}{0pt}%
\pgfpathmoveto{\pgfqpoint{3.539434in}{6.724340in}}%
\pgfpathlineto{\pgfqpoint{3.627169in}{6.724340in}}%
\pgfpathlineto{\pgfqpoint{3.627169in}{6.636604in}}%
\pgfpathlineto{\pgfqpoint{3.539434in}{6.636604in}}%
\pgfpathlineto{\pgfqpoint{3.539434in}{6.724340in}}%
\pgfusepath{stroke,fill}%
\end{pgfscope}%
\begin{pgfscope}%
\pgfpathrectangle{\pgfqpoint{0.380943in}{6.110189in}}{\pgfqpoint{4.650000in}{0.614151in}}%
\pgfusepath{clip}%
\pgfsetbuttcap%
\pgfsetroundjoin%
\definecolor{currentfill}{rgb}{0.986759,0.806398,0.641200}%
\pgfsetfillcolor{currentfill}%
\pgfsetlinewidth{0.250937pt}%
\definecolor{currentstroke}{rgb}{1.000000,1.000000,1.000000}%
\pgfsetstrokecolor{currentstroke}%
\pgfsetdash{}{0pt}%
\pgfpathmoveto{\pgfqpoint{3.627169in}{6.724340in}}%
\pgfpathlineto{\pgfqpoint{3.714905in}{6.724340in}}%
\pgfpathlineto{\pgfqpoint{3.714905in}{6.636604in}}%
\pgfpathlineto{\pgfqpoint{3.627169in}{6.636604in}}%
\pgfpathlineto{\pgfqpoint{3.627169in}{6.724340in}}%
\pgfusepath{stroke,fill}%
\end{pgfscope}%
\begin{pgfscope}%
\pgfpathrectangle{\pgfqpoint{0.380943in}{6.110189in}}{\pgfqpoint{4.650000in}{0.614151in}}%
\pgfusepath{clip}%
\pgfsetbuttcap%
\pgfsetroundjoin%
\definecolor{currentfill}{rgb}{0.968166,0.945882,0.748604}%
\pgfsetfillcolor{currentfill}%
\pgfsetlinewidth{0.250937pt}%
\definecolor{currentstroke}{rgb}{1.000000,1.000000,1.000000}%
\pgfsetstrokecolor{currentstroke}%
\pgfsetdash{}{0pt}%
\pgfpathmoveto{\pgfqpoint{3.714905in}{6.724340in}}%
\pgfpathlineto{\pgfqpoint{3.802641in}{6.724340in}}%
\pgfpathlineto{\pgfqpoint{3.802641in}{6.636604in}}%
\pgfpathlineto{\pgfqpoint{3.714905in}{6.636604in}}%
\pgfpathlineto{\pgfqpoint{3.714905in}{6.724340in}}%
\pgfusepath{stroke,fill}%
\end{pgfscope}%
\begin{pgfscope}%
\pgfpathrectangle{\pgfqpoint{0.380943in}{6.110189in}}{\pgfqpoint{4.650000in}{0.614151in}}%
\pgfusepath{clip}%
\pgfsetbuttcap%
\pgfsetroundjoin%
\definecolor{currentfill}{rgb}{0.996571,0.720538,0.589189}%
\pgfsetfillcolor{currentfill}%
\pgfsetlinewidth{0.250937pt}%
\definecolor{currentstroke}{rgb}{1.000000,1.000000,1.000000}%
\pgfsetstrokecolor{currentstroke}%
\pgfsetdash{}{0pt}%
\pgfpathmoveto{\pgfqpoint{3.802641in}{6.724340in}}%
\pgfpathlineto{\pgfqpoint{3.890377in}{6.724340in}}%
\pgfpathlineto{\pgfqpoint{3.890377in}{6.636604in}}%
\pgfpathlineto{\pgfqpoint{3.802641in}{6.636604in}}%
\pgfpathlineto{\pgfqpoint{3.802641in}{6.724340in}}%
\pgfusepath{stroke,fill}%
\end{pgfscope}%
\begin{pgfscope}%
\pgfpathrectangle{\pgfqpoint{0.380943in}{6.110189in}}{\pgfqpoint{4.650000in}{0.614151in}}%
\pgfusepath{clip}%
\pgfsetbuttcap%
\pgfsetroundjoin%
\definecolor{currentfill}{rgb}{0.861576,0.340008,0.340008}%
\pgfsetfillcolor{currentfill}%
\pgfsetlinewidth{0.250937pt}%
\definecolor{currentstroke}{rgb}{1.000000,1.000000,1.000000}%
\pgfsetstrokecolor{currentstroke}%
\pgfsetdash{}{0pt}%
\pgfpathmoveto{\pgfqpoint{3.890377in}{6.724340in}}%
\pgfpathlineto{\pgfqpoint{3.978113in}{6.724340in}}%
\pgfpathlineto{\pgfqpoint{3.978113in}{6.636604in}}%
\pgfpathlineto{\pgfqpoint{3.890377in}{6.636604in}}%
\pgfpathlineto{\pgfqpoint{3.890377in}{6.724340in}}%
\pgfusepath{stroke,fill}%
\end{pgfscope}%
\begin{pgfscope}%
\pgfpathrectangle{\pgfqpoint{0.380943in}{6.110189in}}{\pgfqpoint{4.650000in}{0.614151in}}%
\pgfusepath{clip}%
\pgfsetbuttcap%
\pgfsetroundjoin%
\definecolor{currentfill}{rgb}{0.979654,0.837186,0.669619}%
\pgfsetfillcolor{currentfill}%
\pgfsetlinewidth{0.250937pt}%
\definecolor{currentstroke}{rgb}{1.000000,1.000000,1.000000}%
\pgfsetstrokecolor{currentstroke}%
\pgfsetdash{}{0pt}%
\pgfpathmoveto{\pgfqpoint{3.978113in}{6.724340in}}%
\pgfpathlineto{\pgfqpoint{4.065849in}{6.724340in}}%
\pgfpathlineto{\pgfqpoint{4.065849in}{6.636604in}}%
\pgfpathlineto{\pgfqpoint{3.978113in}{6.636604in}}%
\pgfpathlineto{\pgfqpoint{3.978113in}{6.724340in}}%
\pgfusepath{stroke,fill}%
\end{pgfscope}%
\begin{pgfscope}%
\pgfpathrectangle{\pgfqpoint{0.380943in}{6.110189in}}{\pgfqpoint{4.650000in}{0.614151in}}%
\pgfusepath{clip}%
\pgfsetbuttcap%
\pgfsetroundjoin%
\definecolor{currentfill}{rgb}{0.992326,0.765229,0.614840}%
\pgfsetfillcolor{currentfill}%
\pgfsetlinewidth{0.250937pt}%
\definecolor{currentstroke}{rgb}{1.000000,1.000000,1.000000}%
\pgfsetstrokecolor{currentstroke}%
\pgfsetdash{}{0pt}%
\pgfpathmoveto{\pgfqpoint{4.065849in}{6.724340in}}%
\pgfpathlineto{\pgfqpoint{4.153585in}{6.724340in}}%
\pgfpathlineto{\pgfqpoint{4.153585in}{6.636604in}}%
\pgfpathlineto{\pgfqpoint{4.065849in}{6.636604in}}%
\pgfpathlineto{\pgfqpoint{4.065849in}{6.724340in}}%
\pgfusepath{stroke,fill}%
\end{pgfscope}%
\begin{pgfscope}%
\pgfpathrectangle{\pgfqpoint{0.380943in}{6.110189in}}{\pgfqpoint{4.650000in}{0.614151in}}%
\pgfusepath{clip}%
\pgfsetbuttcap%
\pgfsetroundjoin%
\definecolor{currentfill}{rgb}{0.972549,0.870588,0.692810}%
\pgfsetfillcolor{currentfill}%
\pgfsetlinewidth{0.250937pt}%
\definecolor{currentstroke}{rgb}{1.000000,1.000000,1.000000}%
\pgfsetstrokecolor{currentstroke}%
\pgfsetdash{}{0pt}%
\pgfpathmoveto{\pgfqpoint{4.153585in}{6.724340in}}%
\pgfpathlineto{\pgfqpoint{4.241320in}{6.724340in}}%
\pgfpathlineto{\pgfqpoint{4.241320in}{6.636604in}}%
\pgfpathlineto{\pgfqpoint{4.153585in}{6.636604in}}%
\pgfpathlineto{\pgfqpoint{4.153585in}{6.724340in}}%
\pgfusepath{stroke,fill}%
\end{pgfscope}%
\begin{pgfscope}%
\pgfpathrectangle{\pgfqpoint{0.380943in}{6.110189in}}{\pgfqpoint{4.650000in}{0.614151in}}%
\pgfusepath{clip}%
\pgfsetbuttcap%
\pgfsetroundjoin%
\definecolor{currentfill}{rgb}{0.965444,0.906113,0.711757}%
\pgfsetfillcolor{currentfill}%
\pgfsetlinewidth{0.250937pt}%
\definecolor{currentstroke}{rgb}{1.000000,1.000000,1.000000}%
\pgfsetstrokecolor{currentstroke}%
\pgfsetdash{}{0pt}%
\pgfpathmoveto{\pgfqpoint{4.241320in}{6.724340in}}%
\pgfpathlineto{\pgfqpoint{4.329056in}{6.724340in}}%
\pgfpathlineto{\pgfqpoint{4.329056in}{6.636604in}}%
\pgfpathlineto{\pgfqpoint{4.241320in}{6.636604in}}%
\pgfpathlineto{\pgfqpoint{4.241320in}{6.724340in}}%
\pgfusepath{stroke,fill}%
\end{pgfscope}%
\begin{pgfscope}%
\pgfpathrectangle{\pgfqpoint{0.380943in}{6.110189in}}{\pgfqpoint{4.650000in}{0.614151in}}%
\pgfusepath{clip}%
\pgfsetbuttcap%
\pgfsetroundjoin%
\definecolor{currentfill}{rgb}{0.965444,0.906113,0.711757}%
\pgfsetfillcolor{currentfill}%
\pgfsetlinewidth{0.250937pt}%
\definecolor{currentstroke}{rgb}{1.000000,1.000000,1.000000}%
\pgfsetstrokecolor{currentstroke}%
\pgfsetdash{}{0pt}%
\pgfpathmoveto{\pgfqpoint{4.329056in}{6.724340in}}%
\pgfpathlineto{\pgfqpoint{4.416792in}{6.724340in}}%
\pgfpathlineto{\pgfqpoint{4.416792in}{6.636604in}}%
\pgfpathlineto{\pgfqpoint{4.329056in}{6.636604in}}%
\pgfpathlineto{\pgfqpoint{4.329056in}{6.724340in}}%
\pgfusepath{stroke,fill}%
\end{pgfscope}%
\begin{pgfscope}%
\pgfpathrectangle{\pgfqpoint{0.380943in}{6.110189in}}{\pgfqpoint{4.650000in}{0.614151in}}%
\pgfusepath{clip}%
\pgfsetbuttcap%
\pgfsetroundjoin%
\definecolor{currentfill}{rgb}{0.962414,0.923552,0.722891}%
\pgfsetfillcolor{currentfill}%
\pgfsetlinewidth{0.250937pt}%
\definecolor{currentstroke}{rgb}{1.000000,1.000000,1.000000}%
\pgfsetstrokecolor{currentstroke}%
\pgfsetdash{}{0pt}%
\pgfpathmoveto{\pgfqpoint{4.416792in}{6.724340in}}%
\pgfpathlineto{\pgfqpoint{4.504528in}{6.724340in}}%
\pgfpathlineto{\pgfqpoint{4.504528in}{6.636604in}}%
\pgfpathlineto{\pgfqpoint{4.416792in}{6.636604in}}%
\pgfpathlineto{\pgfqpoint{4.416792in}{6.724340in}}%
\pgfusepath{stroke,fill}%
\end{pgfscope}%
\begin{pgfscope}%
\pgfpathrectangle{\pgfqpoint{0.380943in}{6.110189in}}{\pgfqpoint{4.650000in}{0.614151in}}%
\pgfusepath{clip}%
\pgfsetbuttcap%
\pgfsetroundjoin%
\definecolor{currentfill}{rgb}{0.972549,0.870588,0.692810}%
\pgfsetfillcolor{currentfill}%
\pgfsetlinewidth{0.250937pt}%
\definecolor{currentstroke}{rgb}{1.000000,1.000000,1.000000}%
\pgfsetstrokecolor{currentstroke}%
\pgfsetdash{}{0pt}%
\pgfpathmoveto{\pgfqpoint{4.504528in}{6.724340in}}%
\pgfpathlineto{\pgfqpoint{4.592264in}{6.724340in}}%
\pgfpathlineto{\pgfqpoint{4.592264in}{6.636604in}}%
\pgfpathlineto{\pgfqpoint{4.504528in}{6.636604in}}%
\pgfpathlineto{\pgfqpoint{4.504528in}{6.724340in}}%
\pgfusepath{stroke,fill}%
\end{pgfscope}%
\begin{pgfscope}%
\pgfpathrectangle{\pgfqpoint{0.380943in}{6.110189in}}{\pgfqpoint{4.650000in}{0.614151in}}%
\pgfusepath{clip}%
\pgfsetbuttcap%
\pgfsetroundjoin%
\definecolor{currentfill}{rgb}{0.986759,0.806398,0.641200}%
\pgfsetfillcolor{currentfill}%
\pgfsetlinewidth{0.250937pt}%
\definecolor{currentstroke}{rgb}{1.000000,1.000000,1.000000}%
\pgfsetstrokecolor{currentstroke}%
\pgfsetdash{}{0pt}%
\pgfpathmoveto{\pgfqpoint{4.592264in}{6.724340in}}%
\pgfpathlineto{\pgfqpoint{4.680000in}{6.724340in}}%
\pgfpathlineto{\pgfqpoint{4.680000in}{6.636604in}}%
\pgfpathlineto{\pgfqpoint{4.592264in}{6.636604in}}%
\pgfpathlineto{\pgfqpoint{4.592264in}{6.724340in}}%
\pgfusepath{stroke,fill}%
\end{pgfscope}%
\begin{pgfscope}%
\pgfpathrectangle{\pgfqpoint{0.380943in}{6.110189in}}{\pgfqpoint{4.650000in}{0.614151in}}%
\pgfusepath{clip}%
\pgfsetbuttcap%
\pgfsetroundjoin%
\definecolor{currentfill}{rgb}{0.968166,0.945882,0.748604}%
\pgfsetfillcolor{currentfill}%
\pgfsetlinewidth{0.250937pt}%
\definecolor{currentstroke}{rgb}{1.000000,1.000000,1.000000}%
\pgfsetstrokecolor{currentstroke}%
\pgfsetdash{}{0pt}%
\pgfpathmoveto{\pgfqpoint{4.680000in}{6.724340in}}%
\pgfpathlineto{\pgfqpoint{4.767736in}{6.724340in}}%
\pgfpathlineto{\pgfqpoint{4.767736in}{6.636604in}}%
\pgfpathlineto{\pgfqpoint{4.680000in}{6.636604in}}%
\pgfpathlineto{\pgfqpoint{4.680000in}{6.724340in}}%
\pgfusepath{stroke,fill}%
\end{pgfscope}%
\begin{pgfscope}%
\pgfpathrectangle{\pgfqpoint{0.380943in}{6.110189in}}{\pgfqpoint{4.650000in}{0.614151in}}%
\pgfusepath{clip}%
\pgfsetbuttcap%
\pgfsetroundjoin%
\definecolor{currentfill}{rgb}{0.979654,0.837186,0.669619}%
\pgfsetfillcolor{currentfill}%
\pgfsetlinewidth{0.250937pt}%
\definecolor{currentstroke}{rgb}{1.000000,1.000000,1.000000}%
\pgfsetstrokecolor{currentstroke}%
\pgfsetdash{}{0pt}%
\pgfpathmoveto{\pgfqpoint{4.767736in}{6.724340in}}%
\pgfpathlineto{\pgfqpoint{4.855471in}{6.724340in}}%
\pgfpathlineto{\pgfqpoint{4.855471in}{6.636604in}}%
\pgfpathlineto{\pgfqpoint{4.767736in}{6.636604in}}%
\pgfpathlineto{\pgfqpoint{4.767736in}{6.724340in}}%
\pgfusepath{stroke,fill}%
\end{pgfscope}%
\begin{pgfscope}%
\pgfpathrectangle{\pgfqpoint{0.380943in}{6.110189in}}{\pgfqpoint{4.650000in}{0.614151in}}%
\pgfusepath{clip}%
\pgfsetbuttcap%
\pgfsetroundjoin%
\definecolor{currentfill}{rgb}{0.962414,0.923552,0.722891}%
\pgfsetfillcolor{currentfill}%
\pgfsetlinewidth{0.250937pt}%
\definecolor{currentstroke}{rgb}{1.000000,1.000000,1.000000}%
\pgfsetstrokecolor{currentstroke}%
\pgfsetdash{}{0pt}%
\pgfpathmoveto{\pgfqpoint{4.855471in}{6.724340in}}%
\pgfpathlineto{\pgfqpoint{4.943207in}{6.724340in}}%
\pgfpathlineto{\pgfqpoint{4.943207in}{6.636604in}}%
\pgfpathlineto{\pgfqpoint{4.855471in}{6.636604in}}%
\pgfpathlineto{\pgfqpoint{4.855471in}{6.724340in}}%
\pgfusepath{stroke,fill}%
\end{pgfscope}%
\begin{pgfscope}%
\pgfpathrectangle{\pgfqpoint{0.380943in}{6.110189in}}{\pgfqpoint{4.650000in}{0.614151in}}%
\pgfusepath{clip}%
\pgfsetbuttcap%
\pgfsetroundjoin%
\definecolor{currentfill}{rgb}{0.986759,0.806398,0.641200}%
\pgfsetfillcolor{currentfill}%
\pgfsetlinewidth{0.250937pt}%
\definecolor{currentstroke}{rgb}{1.000000,1.000000,1.000000}%
\pgfsetstrokecolor{currentstroke}%
\pgfsetdash{}{0pt}%
\pgfpathmoveto{\pgfqpoint{4.943207in}{6.724340in}}%
\pgfpathlineto{\pgfqpoint{5.030943in}{6.724340in}}%
\pgfpathlineto{\pgfqpoint{5.030943in}{6.636604in}}%
\pgfpathlineto{\pgfqpoint{4.943207in}{6.636604in}}%
\pgfpathlineto{\pgfqpoint{4.943207in}{6.724340in}}%
\pgfusepath{stroke,fill}%
\end{pgfscope}%
\begin{pgfscope}%
\pgfpathrectangle{\pgfqpoint{0.380943in}{6.110189in}}{\pgfqpoint{4.650000in}{0.614151in}}%
\pgfusepath{clip}%
\pgfsetbuttcap%
\pgfsetroundjoin%
\definecolor{currentfill}{rgb}{0.998939,0.658962,0.556032}%
\pgfsetfillcolor{currentfill}%
\pgfsetlinewidth{0.250937pt}%
\definecolor{currentstroke}{rgb}{1.000000,1.000000,1.000000}%
\pgfsetstrokecolor{currentstroke}%
\pgfsetdash{}{0pt}%
\pgfpathmoveto{\pgfqpoint{0.380943in}{6.636604in}}%
\pgfpathlineto{\pgfqpoint{0.468679in}{6.636604in}}%
\pgfpathlineto{\pgfqpoint{0.468679in}{6.548868in}}%
\pgfpathlineto{\pgfqpoint{0.380943in}{6.548868in}}%
\pgfpathlineto{\pgfqpoint{0.380943in}{6.636604in}}%
\pgfusepath{stroke,fill}%
\end{pgfscope}%
\begin{pgfscope}%
\pgfpathrectangle{\pgfqpoint{0.380943in}{6.110189in}}{\pgfqpoint{4.650000in}{0.614151in}}%
\pgfusepath{clip}%
\pgfsetbuttcap%
\pgfsetroundjoin%
\definecolor{currentfill}{rgb}{0.992326,0.765229,0.614840}%
\pgfsetfillcolor{currentfill}%
\pgfsetlinewidth{0.250937pt}%
\definecolor{currentstroke}{rgb}{1.000000,1.000000,1.000000}%
\pgfsetstrokecolor{currentstroke}%
\pgfsetdash{}{0pt}%
\pgfpathmoveto{\pgfqpoint{0.468679in}{6.636604in}}%
\pgfpathlineto{\pgfqpoint{0.556415in}{6.636604in}}%
\pgfpathlineto{\pgfqpoint{0.556415in}{6.548868in}}%
\pgfpathlineto{\pgfqpoint{0.468679in}{6.548868in}}%
\pgfpathlineto{\pgfqpoint{0.468679in}{6.636604in}}%
\pgfusepath{stroke,fill}%
\end{pgfscope}%
\begin{pgfscope}%
\pgfpathrectangle{\pgfqpoint{0.380943in}{6.110189in}}{\pgfqpoint{4.650000in}{0.614151in}}%
\pgfusepath{clip}%
\pgfsetbuttcap%
\pgfsetroundjoin%
\definecolor{currentfill}{rgb}{0.972549,0.870588,0.692810}%
\pgfsetfillcolor{currentfill}%
\pgfsetlinewidth{0.250937pt}%
\definecolor{currentstroke}{rgb}{1.000000,1.000000,1.000000}%
\pgfsetstrokecolor{currentstroke}%
\pgfsetdash{}{0pt}%
\pgfpathmoveto{\pgfqpoint{0.556415in}{6.636604in}}%
\pgfpathlineto{\pgfqpoint{0.644151in}{6.636604in}}%
\pgfpathlineto{\pgfqpoint{0.644151in}{6.548868in}}%
\pgfpathlineto{\pgfqpoint{0.556415in}{6.548868in}}%
\pgfpathlineto{\pgfqpoint{0.556415in}{6.636604in}}%
\pgfusepath{stroke,fill}%
\end{pgfscope}%
\begin{pgfscope}%
\pgfpathrectangle{\pgfqpoint{0.380943in}{6.110189in}}{\pgfqpoint{4.650000in}{0.614151in}}%
\pgfusepath{clip}%
\pgfsetbuttcap%
\pgfsetroundjoin%
\definecolor{currentfill}{rgb}{0.979654,0.837186,0.669619}%
\pgfsetfillcolor{currentfill}%
\pgfsetlinewidth{0.250937pt}%
\definecolor{currentstroke}{rgb}{1.000000,1.000000,1.000000}%
\pgfsetstrokecolor{currentstroke}%
\pgfsetdash{}{0pt}%
\pgfpathmoveto{\pgfqpoint{0.644151in}{6.636604in}}%
\pgfpathlineto{\pgfqpoint{0.731886in}{6.636604in}}%
\pgfpathlineto{\pgfqpoint{0.731886in}{6.548868in}}%
\pgfpathlineto{\pgfqpoint{0.644151in}{6.548868in}}%
\pgfpathlineto{\pgfqpoint{0.644151in}{6.636604in}}%
\pgfusepath{stroke,fill}%
\end{pgfscope}%
\begin{pgfscope}%
\pgfpathrectangle{\pgfqpoint{0.380943in}{6.110189in}}{\pgfqpoint{4.650000in}{0.614151in}}%
\pgfusepath{clip}%
\pgfsetbuttcap%
\pgfsetroundjoin%
\definecolor{currentfill}{rgb}{0.998939,0.658962,0.556032}%
\pgfsetfillcolor{currentfill}%
\pgfsetlinewidth{0.250937pt}%
\definecolor{currentstroke}{rgb}{1.000000,1.000000,1.000000}%
\pgfsetstrokecolor{currentstroke}%
\pgfsetdash{}{0pt}%
\pgfpathmoveto{\pgfqpoint{0.731886in}{6.636604in}}%
\pgfpathlineto{\pgfqpoint{0.819622in}{6.636604in}}%
\pgfpathlineto{\pgfqpoint{0.819622in}{6.548868in}}%
\pgfpathlineto{\pgfqpoint{0.731886in}{6.548868in}}%
\pgfpathlineto{\pgfqpoint{0.731886in}{6.636604in}}%
\pgfusepath{stroke,fill}%
\end{pgfscope}%
\begin{pgfscope}%
\pgfpathrectangle{\pgfqpoint{0.380943in}{6.110189in}}{\pgfqpoint{4.650000in}{0.614151in}}%
\pgfusepath{clip}%
\pgfsetbuttcap%
\pgfsetroundjoin%
\definecolor{currentfill}{rgb}{0.996571,0.720538,0.589189}%
\pgfsetfillcolor{currentfill}%
\pgfsetlinewidth{0.250937pt}%
\definecolor{currentstroke}{rgb}{1.000000,1.000000,1.000000}%
\pgfsetstrokecolor{currentstroke}%
\pgfsetdash{}{0pt}%
\pgfpathmoveto{\pgfqpoint{0.819622in}{6.636604in}}%
\pgfpathlineto{\pgfqpoint{0.907358in}{6.636604in}}%
\pgfpathlineto{\pgfqpoint{0.907358in}{6.548868in}}%
\pgfpathlineto{\pgfqpoint{0.819622in}{6.548868in}}%
\pgfpathlineto{\pgfqpoint{0.819622in}{6.636604in}}%
\pgfusepath{stroke,fill}%
\end{pgfscope}%
\begin{pgfscope}%
\pgfpathrectangle{\pgfqpoint{0.380943in}{6.110189in}}{\pgfqpoint{4.650000in}{0.614151in}}%
\pgfusepath{clip}%
\pgfsetbuttcap%
\pgfsetroundjoin%
\definecolor{currentfill}{rgb}{0.979654,0.837186,0.669619}%
\pgfsetfillcolor{currentfill}%
\pgfsetlinewidth{0.250937pt}%
\definecolor{currentstroke}{rgb}{1.000000,1.000000,1.000000}%
\pgfsetstrokecolor{currentstroke}%
\pgfsetdash{}{0pt}%
\pgfpathmoveto{\pgfqpoint{0.907358in}{6.636604in}}%
\pgfpathlineto{\pgfqpoint{0.995094in}{6.636604in}}%
\pgfpathlineto{\pgfqpoint{0.995094in}{6.548868in}}%
\pgfpathlineto{\pgfqpoint{0.907358in}{6.548868in}}%
\pgfpathlineto{\pgfqpoint{0.907358in}{6.636604in}}%
\pgfusepath{stroke,fill}%
\end{pgfscope}%
\begin{pgfscope}%
\pgfpathrectangle{\pgfqpoint{0.380943in}{6.110189in}}{\pgfqpoint{4.650000in}{0.614151in}}%
\pgfusepath{clip}%
\pgfsetbuttcap%
\pgfsetroundjoin%
\definecolor{currentfill}{rgb}{0.979654,0.837186,0.669619}%
\pgfsetfillcolor{currentfill}%
\pgfsetlinewidth{0.250937pt}%
\definecolor{currentstroke}{rgb}{1.000000,1.000000,1.000000}%
\pgfsetstrokecolor{currentstroke}%
\pgfsetdash{}{0pt}%
\pgfpathmoveto{\pgfqpoint{0.995094in}{6.636604in}}%
\pgfpathlineto{\pgfqpoint{1.082830in}{6.636604in}}%
\pgfpathlineto{\pgfqpoint{1.082830in}{6.548868in}}%
\pgfpathlineto{\pgfqpoint{0.995094in}{6.548868in}}%
\pgfpathlineto{\pgfqpoint{0.995094in}{6.636604in}}%
\pgfusepath{stroke,fill}%
\end{pgfscope}%
\begin{pgfscope}%
\pgfpathrectangle{\pgfqpoint{0.380943in}{6.110189in}}{\pgfqpoint{4.650000in}{0.614151in}}%
\pgfusepath{clip}%
\pgfsetbuttcap%
\pgfsetroundjoin%
\definecolor{currentfill}{rgb}{0.965444,0.906113,0.711757}%
\pgfsetfillcolor{currentfill}%
\pgfsetlinewidth{0.250937pt}%
\definecolor{currentstroke}{rgb}{1.000000,1.000000,1.000000}%
\pgfsetstrokecolor{currentstroke}%
\pgfsetdash{}{0pt}%
\pgfpathmoveto{\pgfqpoint{1.082830in}{6.636604in}}%
\pgfpathlineto{\pgfqpoint{1.170566in}{6.636604in}}%
\pgfpathlineto{\pgfqpoint{1.170566in}{6.548868in}}%
\pgfpathlineto{\pgfqpoint{1.082830in}{6.548868in}}%
\pgfpathlineto{\pgfqpoint{1.082830in}{6.636604in}}%
\pgfusepath{stroke,fill}%
\end{pgfscope}%
\begin{pgfscope}%
\pgfpathrectangle{\pgfqpoint{0.380943in}{6.110189in}}{\pgfqpoint{4.650000in}{0.614151in}}%
\pgfusepath{clip}%
\pgfsetbuttcap%
\pgfsetroundjoin%
\definecolor{currentfill}{rgb}{0.992326,0.765229,0.614840}%
\pgfsetfillcolor{currentfill}%
\pgfsetlinewidth{0.250937pt}%
\definecolor{currentstroke}{rgb}{1.000000,1.000000,1.000000}%
\pgfsetstrokecolor{currentstroke}%
\pgfsetdash{}{0pt}%
\pgfpathmoveto{\pgfqpoint{1.170566in}{6.636604in}}%
\pgfpathlineto{\pgfqpoint{1.258302in}{6.636604in}}%
\pgfpathlineto{\pgfqpoint{1.258302in}{6.548868in}}%
\pgfpathlineto{\pgfqpoint{1.170566in}{6.548868in}}%
\pgfpathlineto{\pgfqpoint{1.170566in}{6.636604in}}%
\pgfusepath{stroke,fill}%
\end{pgfscope}%
\begin{pgfscope}%
\pgfpathrectangle{\pgfqpoint{0.380943in}{6.110189in}}{\pgfqpoint{4.650000in}{0.614151in}}%
\pgfusepath{clip}%
\pgfsetbuttcap%
\pgfsetroundjoin%
\definecolor{currentfill}{rgb}{1.000000,0.509404,0.491473}%
\pgfsetfillcolor{currentfill}%
\pgfsetlinewidth{0.250937pt}%
\definecolor{currentstroke}{rgb}{1.000000,1.000000,1.000000}%
\pgfsetstrokecolor{currentstroke}%
\pgfsetdash{}{0pt}%
\pgfpathmoveto{\pgfqpoint{1.258302in}{6.636604in}}%
\pgfpathlineto{\pgfqpoint{1.346037in}{6.636604in}}%
\pgfpathlineto{\pgfqpoint{1.346037in}{6.548868in}}%
\pgfpathlineto{\pgfqpoint{1.258302in}{6.548868in}}%
\pgfpathlineto{\pgfqpoint{1.258302in}{6.636604in}}%
\pgfusepath{stroke,fill}%
\end{pgfscope}%
\begin{pgfscope}%
\pgfpathrectangle{\pgfqpoint{0.380943in}{6.110189in}}{\pgfqpoint{4.650000in}{0.614151in}}%
\pgfusepath{clip}%
\pgfsetbuttcap%
\pgfsetroundjoin%
\definecolor{currentfill}{rgb}{0.972549,0.870588,0.692810}%
\pgfsetfillcolor{currentfill}%
\pgfsetlinewidth{0.250937pt}%
\definecolor{currentstroke}{rgb}{1.000000,1.000000,1.000000}%
\pgfsetstrokecolor{currentstroke}%
\pgfsetdash{}{0pt}%
\pgfpathmoveto{\pgfqpoint{1.346037in}{6.636604in}}%
\pgfpathlineto{\pgfqpoint{1.433773in}{6.636604in}}%
\pgfpathlineto{\pgfqpoint{1.433773in}{6.548868in}}%
\pgfpathlineto{\pgfqpoint{1.346037in}{6.548868in}}%
\pgfpathlineto{\pgfqpoint{1.346037in}{6.636604in}}%
\pgfusepath{stroke,fill}%
\end{pgfscope}%
\begin{pgfscope}%
\pgfpathrectangle{\pgfqpoint{0.380943in}{6.110189in}}{\pgfqpoint{4.650000in}{0.614151in}}%
\pgfusepath{clip}%
\pgfsetbuttcap%
\pgfsetroundjoin%
\definecolor{currentfill}{rgb}{0.996571,0.720538,0.589189}%
\pgfsetfillcolor{currentfill}%
\pgfsetlinewidth{0.250937pt}%
\definecolor{currentstroke}{rgb}{1.000000,1.000000,1.000000}%
\pgfsetstrokecolor{currentstroke}%
\pgfsetdash{}{0pt}%
\pgfpathmoveto{\pgfqpoint{1.433773in}{6.636604in}}%
\pgfpathlineto{\pgfqpoint{1.521509in}{6.636604in}}%
\pgfpathlineto{\pgfqpoint{1.521509in}{6.548868in}}%
\pgfpathlineto{\pgfqpoint{1.433773in}{6.548868in}}%
\pgfpathlineto{\pgfqpoint{1.433773in}{6.636604in}}%
\pgfusepath{stroke,fill}%
\end{pgfscope}%
\begin{pgfscope}%
\pgfpathrectangle{\pgfqpoint{0.380943in}{6.110189in}}{\pgfqpoint{4.650000in}{0.614151in}}%
\pgfusepath{clip}%
\pgfsetbuttcap%
\pgfsetroundjoin%
\definecolor{currentfill}{rgb}{0.996571,0.720538,0.589189}%
\pgfsetfillcolor{currentfill}%
\pgfsetlinewidth{0.250937pt}%
\definecolor{currentstroke}{rgb}{1.000000,1.000000,1.000000}%
\pgfsetstrokecolor{currentstroke}%
\pgfsetdash{}{0pt}%
\pgfpathmoveto{\pgfqpoint{1.521509in}{6.636604in}}%
\pgfpathlineto{\pgfqpoint{1.609245in}{6.636604in}}%
\pgfpathlineto{\pgfqpoint{1.609245in}{6.548868in}}%
\pgfpathlineto{\pgfqpoint{1.521509in}{6.548868in}}%
\pgfpathlineto{\pgfqpoint{1.521509in}{6.636604in}}%
\pgfusepath{stroke,fill}%
\end{pgfscope}%
\begin{pgfscope}%
\pgfpathrectangle{\pgfqpoint{0.380943in}{6.110189in}}{\pgfqpoint{4.650000in}{0.614151in}}%
\pgfusepath{clip}%
\pgfsetbuttcap%
\pgfsetroundjoin%
\definecolor{currentfill}{rgb}{1.000000,0.509404,0.491473}%
\pgfsetfillcolor{currentfill}%
\pgfsetlinewidth{0.250937pt}%
\definecolor{currentstroke}{rgb}{1.000000,1.000000,1.000000}%
\pgfsetstrokecolor{currentstroke}%
\pgfsetdash{}{0pt}%
\pgfpathmoveto{\pgfqpoint{1.609245in}{6.636604in}}%
\pgfpathlineto{\pgfqpoint{1.696981in}{6.636604in}}%
\pgfpathlineto{\pgfqpoint{1.696981in}{6.548868in}}%
\pgfpathlineto{\pgfqpoint{1.609245in}{6.548868in}}%
\pgfpathlineto{\pgfqpoint{1.609245in}{6.636604in}}%
\pgfusepath{stroke,fill}%
\end{pgfscope}%
\begin{pgfscope}%
\pgfpathrectangle{\pgfqpoint{0.380943in}{6.110189in}}{\pgfqpoint{4.650000in}{0.614151in}}%
\pgfusepath{clip}%
\pgfsetbuttcap%
\pgfsetroundjoin%
\definecolor{currentfill}{rgb}{0.986759,0.806398,0.641200}%
\pgfsetfillcolor{currentfill}%
\pgfsetlinewidth{0.250937pt}%
\definecolor{currentstroke}{rgb}{1.000000,1.000000,1.000000}%
\pgfsetstrokecolor{currentstroke}%
\pgfsetdash{}{0pt}%
\pgfpathmoveto{\pgfqpoint{1.696981in}{6.636604in}}%
\pgfpathlineto{\pgfqpoint{1.784717in}{6.636604in}}%
\pgfpathlineto{\pgfqpoint{1.784717in}{6.548868in}}%
\pgfpathlineto{\pgfqpoint{1.696981in}{6.548868in}}%
\pgfpathlineto{\pgfqpoint{1.696981in}{6.636604in}}%
\pgfusepath{stroke,fill}%
\end{pgfscope}%
\begin{pgfscope}%
\pgfpathrectangle{\pgfqpoint{0.380943in}{6.110189in}}{\pgfqpoint{4.650000in}{0.614151in}}%
\pgfusepath{clip}%
\pgfsetbuttcap%
\pgfsetroundjoin%
\definecolor{currentfill}{rgb}{0.992326,0.765229,0.614840}%
\pgfsetfillcolor{currentfill}%
\pgfsetlinewidth{0.250937pt}%
\definecolor{currentstroke}{rgb}{1.000000,1.000000,1.000000}%
\pgfsetstrokecolor{currentstroke}%
\pgfsetdash{}{0pt}%
\pgfpathmoveto{\pgfqpoint{1.784717in}{6.636604in}}%
\pgfpathlineto{\pgfqpoint{1.872452in}{6.636604in}}%
\pgfpathlineto{\pgfqpoint{1.872452in}{6.548868in}}%
\pgfpathlineto{\pgfqpoint{1.784717in}{6.548868in}}%
\pgfpathlineto{\pgfqpoint{1.784717in}{6.636604in}}%
\pgfusepath{stroke,fill}%
\end{pgfscope}%
\begin{pgfscope}%
\pgfpathrectangle{\pgfqpoint{0.380943in}{6.110189in}}{\pgfqpoint{4.650000in}{0.614151in}}%
\pgfusepath{clip}%
\pgfsetbuttcap%
\pgfsetroundjoin%
\definecolor{currentfill}{rgb}{0.991849,0.986144,0.810181}%
\pgfsetfillcolor{currentfill}%
\pgfsetlinewidth{0.250937pt}%
\definecolor{currentstroke}{rgb}{1.000000,1.000000,1.000000}%
\pgfsetstrokecolor{currentstroke}%
\pgfsetdash{}{0pt}%
\pgfpathmoveto{\pgfqpoint{1.872452in}{6.636604in}}%
\pgfpathlineto{\pgfqpoint{1.960188in}{6.636604in}}%
\pgfpathlineto{\pgfqpoint{1.960188in}{6.548868in}}%
\pgfpathlineto{\pgfqpoint{1.872452in}{6.548868in}}%
\pgfpathlineto{\pgfqpoint{1.872452in}{6.636604in}}%
\pgfusepath{stroke,fill}%
\end{pgfscope}%
\begin{pgfscope}%
\pgfpathrectangle{\pgfqpoint{0.380943in}{6.110189in}}{\pgfqpoint{4.650000in}{0.614151in}}%
\pgfusepath{clip}%
\pgfsetbuttcap%
\pgfsetroundjoin%
\definecolor{currentfill}{rgb}{0.991849,0.986144,0.810181}%
\pgfsetfillcolor{currentfill}%
\pgfsetlinewidth{0.250937pt}%
\definecolor{currentstroke}{rgb}{1.000000,1.000000,1.000000}%
\pgfsetstrokecolor{currentstroke}%
\pgfsetdash{}{0pt}%
\pgfpathmoveto{\pgfqpoint{1.960188in}{6.636604in}}%
\pgfpathlineto{\pgfqpoint{2.047924in}{6.636604in}}%
\pgfpathlineto{\pgfqpoint{2.047924in}{6.548868in}}%
\pgfpathlineto{\pgfqpoint{1.960188in}{6.548868in}}%
\pgfpathlineto{\pgfqpoint{1.960188in}{6.636604in}}%
\pgfusepath{stroke,fill}%
\end{pgfscope}%
\begin{pgfscope}%
\pgfpathrectangle{\pgfqpoint{0.380943in}{6.110189in}}{\pgfqpoint{4.650000in}{0.614151in}}%
\pgfusepath{clip}%
\pgfsetbuttcap%
\pgfsetroundjoin%
\definecolor{currentfill}{rgb}{0.986759,0.806398,0.641200}%
\pgfsetfillcolor{currentfill}%
\pgfsetlinewidth{0.250937pt}%
\definecolor{currentstroke}{rgb}{1.000000,1.000000,1.000000}%
\pgfsetstrokecolor{currentstroke}%
\pgfsetdash{}{0pt}%
\pgfpathmoveto{\pgfqpoint{2.047924in}{6.636604in}}%
\pgfpathlineto{\pgfqpoint{2.135660in}{6.636604in}}%
\pgfpathlineto{\pgfqpoint{2.135660in}{6.548868in}}%
\pgfpathlineto{\pgfqpoint{2.047924in}{6.548868in}}%
\pgfpathlineto{\pgfqpoint{2.047924in}{6.636604in}}%
\pgfusepath{stroke,fill}%
\end{pgfscope}%
\begin{pgfscope}%
\pgfpathrectangle{\pgfqpoint{0.380943in}{6.110189in}}{\pgfqpoint{4.650000in}{0.614151in}}%
\pgfusepath{clip}%
\pgfsetbuttcap%
\pgfsetroundjoin%
\definecolor{currentfill}{rgb}{0.972549,0.870588,0.692810}%
\pgfsetfillcolor{currentfill}%
\pgfsetlinewidth{0.250937pt}%
\definecolor{currentstroke}{rgb}{1.000000,1.000000,1.000000}%
\pgfsetstrokecolor{currentstroke}%
\pgfsetdash{}{0pt}%
\pgfpathmoveto{\pgfqpoint{2.135660in}{6.636604in}}%
\pgfpathlineto{\pgfqpoint{2.223396in}{6.636604in}}%
\pgfpathlineto{\pgfqpoint{2.223396in}{6.548868in}}%
\pgfpathlineto{\pgfqpoint{2.135660in}{6.548868in}}%
\pgfpathlineto{\pgfqpoint{2.135660in}{6.636604in}}%
\pgfusepath{stroke,fill}%
\end{pgfscope}%
\begin{pgfscope}%
\pgfpathrectangle{\pgfqpoint{0.380943in}{6.110189in}}{\pgfqpoint{4.650000in}{0.614151in}}%
\pgfusepath{clip}%
\pgfsetbuttcap%
\pgfsetroundjoin%
\definecolor{currentfill}{rgb}{0.979654,0.837186,0.669619}%
\pgfsetfillcolor{currentfill}%
\pgfsetlinewidth{0.250937pt}%
\definecolor{currentstroke}{rgb}{1.000000,1.000000,1.000000}%
\pgfsetstrokecolor{currentstroke}%
\pgfsetdash{}{0pt}%
\pgfpathmoveto{\pgfqpoint{2.223396in}{6.636604in}}%
\pgfpathlineto{\pgfqpoint{2.311132in}{6.636604in}}%
\pgfpathlineto{\pgfqpoint{2.311132in}{6.548868in}}%
\pgfpathlineto{\pgfqpoint{2.223396in}{6.548868in}}%
\pgfpathlineto{\pgfqpoint{2.223396in}{6.636604in}}%
\pgfusepath{stroke,fill}%
\end{pgfscope}%
\begin{pgfscope}%
\pgfpathrectangle{\pgfqpoint{0.380943in}{6.110189in}}{\pgfqpoint{4.650000in}{0.614151in}}%
\pgfusepath{clip}%
\pgfsetbuttcap%
\pgfsetroundjoin%
\definecolor{currentfill}{rgb}{0.861576,0.340008,0.340008}%
\pgfsetfillcolor{currentfill}%
\pgfsetlinewidth{0.250937pt}%
\definecolor{currentstroke}{rgb}{1.000000,1.000000,1.000000}%
\pgfsetstrokecolor{currentstroke}%
\pgfsetdash{}{0pt}%
\pgfpathmoveto{\pgfqpoint{2.311132in}{6.636604in}}%
\pgfpathlineto{\pgfqpoint{2.398868in}{6.636604in}}%
\pgfpathlineto{\pgfqpoint{2.398868in}{6.548868in}}%
\pgfpathlineto{\pgfqpoint{2.311132in}{6.548868in}}%
\pgfpathlineto{\pgfqpoint{2.311132in}{6.636604in}}%
\pgfusepath{stroke,fill}%
\end{pgfscope}%
\begin{pgfscope}%
\pgfpathrectangle{\pgfqpoint{0.380943in}{6.110189in}}{\pgfqpoint{4.650000in}{0.614151in}}%
\pgfusepath{clip}%
\pgfsetbuttcap%
\pgfsetroundjoin%
\definecolor{currentfill}{rgb}{1.000000,0.509404,0.491473}%
\pgfsetfillcolor{currentfill}%
\pgfsetlinewidth{0.250937pt}%
\definecolor{currentstroke}{rgb}{1.000000,1.000000,1.000000}%
\pgfsetstrokecolor{currentstroke}%
\pgfsetdash{}{0pt}%
\pgfpathmoveto{\pgfqpoint{2.398868in}{6.636604in}}%
\pgfpathlineto{\pgfqpoint{2.486603in}{6.636604in}}%
\pgfpathlineto{\pgfqpoint{2.486603in}{6.548868in}}%
\pgfpathlineto{\pgfqpoint{2.398868in}{6.548868in}}%
\pgfpathlineto{\pgfqpoint{2.398868in}{6.636604in}}%
\pgfusepath{stroke,fill}%
\end{pgfscope}%
\begin{pgfscope}%
\pgfpathrectangle{\pgfqpoint{0.380943in}{6.110189in}}{\pgfqpoint{4.650000in}{0.614151in}}%
\pgfusepath{clip}%
\pgfsetbuttcap%
\pgfsetroundjoin%
\definecolor{currentfill}{rgb}{1.000000,0.605229,0.530719}%
\pgfsetfillcolor{currentfill}%
\pgfsetlinewidth{0.250937pt}%
\definecolor{currentstroke}{rgb}{1.000000,1.000000,1.000000}%
\pgfsetstrokecolor{currentstroke}%
\pgfsetdash{}{0pt}%
\pgfpathmoveto{\pgfqpoint{2.486603in}{6.636604in}}%
\pgfpathlineto{\pgfqpoint{2.574339in}{6.636604in}}%
\pgfpathlineto{\pgfqpoint{2.574339in}{6.548868in}}%
\pgfpathlineto{\pgfqpoint{2.486603in}{6.548868in}}%
\pgfpathlineto{\pgfqpoint{2.486603in}{6.636604in}}%
\pgfusepath{stroke,fill}%
\end{pgfscope}%
\begin{pgfscope}%
\pgfpathrectangle{\pgfqpoint{0.380943in}{6.110189in}}{\pgfqpoint{4.650000in}{0.614151in}}%
\pgfusepath{clip}%
\pgfsetbuttcap%
\pgfsetroundjoin%
\definecolor{currentfill}{rgb}{0.972549,0.870588,0.692810}%
\pgfsetfillcolor{currentfill}%
\pgfsetlinewidth{0.250937pt}%
\definecolor{currentstroke}{rgb}{1.000000,1.000000,1.000000}%
\pgfsetstrokecolor{currentstroke}%
\pgfsetdash{}{0pt}%
\pgfpathmoveto{\pgfqpoint{2.574339in}{6.636604in}}%
\pgfpathlineto{\pgfqpoint{2.662075in}{6.636604in}}%
\pgfpathlineto{\pgfqpoint{2.662075in}{6.548868in}}%
\pgfpathlineto{\pgfqpoint{2.574339in}{6.548868in}}%
\pgfpathlineto{\pgfqpoint{2.574339in}{6.636604in}}%
\pgfusepath{stroke,fill}%
\end{pgfscope}%
\begin{pgfscope}%
\pgfpathrectangle{\pgfqpoint{0.380943in}{6.110189in}}{\pgfqpoint{4.650000in}{0.614151in}}%
\pgfusepath{clip}%
\pgfsetbuttcap%
\pgfsetroundjoin%
\definecolor{currentfill}{rgb}{0.962414,0.923552,0.722891}%
\pgfsetfillcolor{currentfill}%
\pgfsetlinewidth{0.250937pt}%
\definecolor{currentstroke}{rgb}{1.000000,1.000000,1.000000}%
\pgfsetstrokecolor{currentstroke}%
\pgfsetdash{}{0pt}%
\pgfpathmoveto{\pgfqpoint{2.662075in}{6.636604in}}%
\pgfpathlineto{\pgfqpoint{2.749811in}{6.636604in}}%
\pgfpathlineto{\pgfqpoint{2.749811in}{6.548868in}}%
\pgfpathlineto{\pgfqpoint{2.662075in}{6.548868in}}%
\pgfpathlineto{\pgfqpoint{2.662075in}{6.636604in}}%
\pgfusepath{stroke,fill}%
\end{pgfscope}%
\begin{pgfscope}%
\pgfpathrectangle{\pgfqpoint{0.380943in}{6.110189in}}{\pgfqpoint{4.650000in}{0.614151in}}%
\pgfusepath{clip}%
\pgfsetbuttcap%
\pgfsetroundjoin%
\definecolor{currentfill}{rgb}{0.972549,0.870588,0.692810}%
\pgfsetfillcolor{currentfill}%
\pgfsetlinewidth{0.250937pt}%
\definecolor{currentstroke}{rgb}{1.000000,1.000000,1.000000}%
\pgfsetstrokecolor{currentstroke}%
\pgfsetdash{}{0pt}%
\pgfpathmoveto{\pgfqpoint{2.749811in}{6.636604in}}%
\pgfpathlineto{\pgfqpoint{2.837547in}{6.636604in}}%
\pgfpathlineto{\pgfqpoint{2.837547in}{6.548868in}}%
\pgfpathlineto{\pgfqpoint{2.749811in}{6.548868in}}%
\pgfpathlineto{\pgfqpoint{2.749811in}{6.636604in}}%
\pgfusepath{stroke,fill}%
\end{pgfscope}%
\begin{pgfscope}%
\pgfpathrectangle{\pgfqpoint{0.380943in}{6.110189in}}{\pgfqpoint{4.650000in}{0.614151in}}%
\pgfusepath{clip}%
\pgfsetbuttcap%
\pgfsetroundjoin%
\definecolor{currentfill}{rgb}{0.992326,0.765229,0.614840}%
\pgfsetfillcolor{currentfill}%
\pgfsetlinewidth{0.250937pt}%
\definecolor{currentstroke}{rgb}{1.000000,1.000000,1.000000}%
\pgfsetstrokecolor{currentstroke}%
\pgfsetdash{}{0pt}%
\pgfpathmoveto{\pgfqpoint{2.837547in}{6.636604in}}%
\pgfpathlineto{\pgfqpoint{2.925283in}{6.636604in}}%
\pgfpathlineto{\pgfqpoint{2.925283in}{6.548868in}}%
\pgfpathlineto{\pgfqpoint{2.837547in}{6.548868in}}%
\pgfpathlineto{\pgfqpoint{2.837547in}{6.636604in}}%
\pgfusepath{stroke,fill}%
\end{pgfscope}%
\begin{pgfscope}%
\pgfpathrectangle{\pgfqpoint{0.380943in}{6.110189in}}{\pgfqpoint{4.650000in}{0.614151in}}%
\pgfusepath{clip}%
\pgfsetbuttcap%
\pgfsetroundjoin%
\definecolor{currentfill}{rgb}{0.968166,0.945882,0.748604}%
\pgfsetfillcolor{currentfill}%
\pgfsetlinewidth{0.250937pt}%
\definecolor{currentstroke}{rgb}{1.000000,1.000000,1.000000}%
\pgfsetstrokecolor{currentstroke}%
\pgfsetdash{}{0pt}%
\pgfpathmoveto{\pgfqpoint{2.925283in}{6.636604in}}%
\pgfpathlineto{\pgfqpoint{3.013019in}{6.636604in}}%
\pgfpathlineto{\pgfqpoint{3.013019in}{6.548868in}}%
\pgfpathlineto{\pgfqpoint{2.925283in}{6.548868in}}%
\pgfpathlineto{\pgfqpoint{2.925283in}{6.636604in}}%
\pgfusepath{stroke,fill}%
\end{pgfscope}%
\begin{pgfscope}%
\pgfpathrectangle{\pgfqpoint{0.380943in}{6.110189in}}{\pgfqpoint{4.650000in}{0.614151in}}%
\pgfusepath{clip}%
\pgfsetbuttcap%
\pgfsetroundjoin%
\definecolor{currentfill}{rgb}{0.979654,0.837186,0.669619}%
\pgfsetfillcolor{currentfill}%
\pgfsetlinewidth{0.250937pt}%
\definecolor{currentstroke}{rgb}{1.000000,1.000000,1.000000}%
\pgfsetstrokecolor{currentstroke}%
\pgfsetdash{}{0pt}%
\pgfpathmoveto{\pgfqpoint{3.013019in}{6.636604in}}%
\pgfpathlineto{\pgfqpoint{3.100754in}{6.636604in}}%
\pgfpathlineto{\pgfqpoint{3.100754in}{6.548868in}}%
\pgfpathlineto{\pgfqpoint{3.013019in}{6.548868in}}%
\pgfpathlineto{\pgfqpoint{3.013019in}{6.636604in}}%
\pgfusepath{stroke,fill}%
\end{pgfscope}%
\begin{pgfscope}%
\pgfpathrectangle{\pgfqpoint{0.380943in}{6.110189in}}{\pgfqpoint{4.650000in}{0.614151in}}%
\pgfusepath{clip}%
\pgfsetbuttcap%
\pgfsetroundjoin%
\definecolor{currentfill}{rgb}{0.972549,0.870588,0.692810}%
\pgfsetfillcolor{currentfill}%
\pgfsetlinewidth{0.250937pt}%
\definecolor{currentstroke}{rgb}{1.000000,1.000000,1.000000}%
\pgfsetstrokecolor{currentstroke}%
\pgfsetdash{}{0pt}%
\pgfpathmoveto{\pgfqpoint{3.100754in}{6.636604in}}%
\pgfpathlineto{\pgfqpoint{3.188490in}{6.636604in}}%
\pgfpathlineto{\pgfqpoint{3.188490in}{6.548868in}}%
\pgfpathlineto{\pgfqpoint{3.100754in}{6.548868in}}%
\pgfpathlineto{\pgfqpoint{3.100754in}{6.636604in}}%
\pgfusepath{stroke,fill}%
\end{pgfscope}%
\begin{pgfscope}%
\pgfpathrectangle{\pgfqpoint{0.380943in}{6.110189in}}{\pgfqpoint{4.650000in}{0.614151in}}%
\pgfusepath{clip}%
\pgfsetbuttcap%
\pgfsetroundjoin%
\definecolor{currentfill}{rgb}{0.972549,0.870588,0.692810}%
\pgfsetfillcolor{currentfill}%
\pgfsetlinewidth{0.250937pt}%
\definecolor{currentstroke}{rgb}{1.000000,1.000000,1.000000}%
\pgfsetstrokecolor{currentstroke}%
\pgfsetdash{}{0pt}%
\pgfpathmoveto{\pgfqpoint{3.188490in}{6.636604in}}%
\pgfpathlineto{\pgfqpoint{3.276226in}{6.636604in}}%
\pgfpathlineto{\pgfqpoint{3.276226in}{6.548868in}}%
\pgfpathlineto{\pgfqpoint{3.188490in}{6.548868in}}%
\pgfpathlineto{\pgfqpoint{3.188490in}{6.636604in}}%
\pgfusepath{stroke,fill}%
\end{pgfscope}%
\begin{pgfscope}%
\pgfpathrectangle{\pgfqpoint{0.380943in}{6.110189in}}{\pgfqpoint{4.650000in}{0.614151in}}%
\pgfusepath{clip}%
\pgfsetbuttcap%
\pgfsetroundjoin%
\definecolor{currentfill}{rgb}{0.972549,0.870588,0.692810}%
\pgfsetfillcolor{currentfill}%
\pgfsetlinewidth{0.250937pt}%
\definecolor{currentstroke}{rgb}{1.000000,1.000000,1.000000}%
\pgfsetstrokecolor{currentstroke}%
\pgfsetdash{}{0pt}%
\pgfpathmoveto{\pgfqpoint{3.276226in}{6.636604in}}%
\pgfpathlineto{\pgfqpoint{3.363962in}{6.636604in}}%
\pgfpathlineto{\pgfqpoint{3.363962in}{6.548868in}}%
\pgfpathlineto{\pgfqpoint{3.276226in}{6.548868in}}%
\pgfpathlineto{\pgfqpoint{3.276226in}{6.636604in}}%
\pgfusepath{stroke,fill}%
\end{pgfscope}%
\begin{pgfscope}%
\pgfpathrectangle{\pgfqpoint{0.380943in}{6.110189in}}{\pgfqpoint{4.650000in}{0.614151in}}%
\pgfusepath{clip}%
\pgfsetbuttcap%
\pgfsetroundjoin%
\definecolor{currentfill}{rgb}{0.965444,0.906113,0.711757}%
\pgfsetfillcolor{currentfill}%
\pgfsetlinewidth{0.250937pt}%
\definecolor{currentstroke}{rgb}{1.000000,1.000000,1.000000}%
\pgfsetstrokecolor{currentstroke}%
\pgfsetdash{}{0pt}%
\pgfpathmoveto{\pgfqpoint{3.363962in}{6.636604in}}%
\pgfpathlineto{\pgfqpoint{3.451698in}{6.636604in}}%
\pgfpathlineto{\pgfqpoint{3.451698in}{6.548868in}}%
\pgfpathlineto{\pgfqpoint{3.363962in}{6.548868in}}%
\pgfpathlineto{\pgfqpoint{3.363962in}{6.636604in}}%
\pgfusepath{stroke,fill}%
\end{pgfscope}%
\begin{pgfscope}%
\pgfpathrectangle{\pgfqpoint{0.380943in}{6.110189in}}{\pgfqpoint{4.650000in}{0.614151in}}%
\pgfusepath{clip}%
\pgfsetbuttcap%
\pgfsetroundjoin%
\definecolor{currentfill}{rgb}{0.992326,0.765229,0.614840}%
\pgfsetfillcolor{currentfill}%
\pgfsetlinewidth{0.250937pt}%
\definecolor{currentstroke}{rgb}{1.000000,1.000000,1.000000}%
\pgfsetstrokecolor{currentstroke}%
\pgfsetdash{}{0pt}%
\pgfpathmoveto{\pgfqpoint{3.451698in}{6.636604in}}%
\pgfpathlineto{\pgfqpoint{3.539434in}{6.636604in}}%
\pgfpathlineto{\pgfqpoint{3.539434in}{6.548868in}}%
\pgfpathlineto{\pgfqpoint{3.451698in}{6.548868in}}%
\pgfpathlineto{\pgfqpoint{3.451698in}{6.636604in}}%
\pgfusepath{stroke,fill}%
\end{pgfscope}%
\begin{pgfscope}%
\pgfpathrectangle{\pgfqpoint{0.380943in}{6.110189in}}{\pgfqpoint{4.650000in}{0.614151in}}%
\pgfusepath{clip}%
\pgfsetbuttcap%
\pgfsetroundjoin%
\definecolor{currentfill}{rgb}{0.962414,0.923552,0.722891}%
\pgfsetfillcolor{currentfill}%
\pgfsetlinewidth{0.250937pt}%
\definecolor{currentstroke}{rgb}{1.000000,1.000000,1.000000}%
\pgfsetstrokecolor{currentstroke}%
\pgfsetdash{}{0pt}%
\pgfpathmoveto{\pgfqpoint{3.539434in}{6.636604in}}%
\pgfpathlineto{\pgfqpoint{3.627169in}{6.636604in}}%
\pgfpathlineto{\pgfqpoint{3.627169in}{6.548868in}}%
\pgfpathlineto{\pgfqpoint{3.539434in}{6.548868in}}%
\pgfpathlineto{\pgfqpoint{3.539434in}{6.636604in}}%
\pgfusepath{stroke,fill}%
\end{pgfscope}%
\begin{pgfscope}%
\pgfpathrectangle{\pgfqpoint{0.380943in}{6.110189in}}{\pgfqpoint{4.650000in}{0.614151in}}%
\pgfusepath{clip}%
\pgfsetbuttcap%
\pgfsetroundjoin%
\definecolor{currentfill}{rgb}{0.992326,0.765229,0.614840}%
\pgfsetfillcolor{currentfill}%
\pgfsetlinewidth{0.250937pt}%
\definecolor{currentstroke}{rgb}{1.000000,1.000000,1.000000}%
\pgfsetstrokecolor{currentstroke}%
\pgfsetdash{}{0pt}%
\pgfpathmoveto{\pgfqpoint{3.627169in}{6.636604in}}%
\pgfpathlineto{\pgfqpoint{3.714905in}{6.636604in}}%
\pgfpathlineto{\pgfqpoint{3.714905in}{6.548868in}}%
\pgfpathlineto{\pgfqpoint{3.627169in}{6.548868in}}%
\pgfpathlineto{\pgfqpoint{3.627169in}{6.636604in}}%
\pgfusepath{stroke,fill}%
\end{pgfscope}%
\begin{pgfscope}%
\pgfpathrectangle{\pgfqpoint{0.380943in}{6.110189in}}{\pgfqpoint{4.650000in}{0.614151in}}%
\pgfusepath{clip}%
\pgfsetbuttcap%
\pgfsetroundjoin%
\definecolor{currentfill}{rgb}{0.986759,0.806398,0.641200}%
\pgfsetfillcolor{currentfill}%
\pgfsetlinewidth{0.250937pt}%
\definecolor{currentstroke}{rgb}{1.000000,1.000000,1.000000}%
\pgfsetstrokecolor{currentstroke}%
\pgfsetdash{}{0pt}%
\pgfpathmoveto{\pgfqpoint{3.714905in}{6.636604in}}%
\pgfpathlineto{\pgfqpoint{3.802641in}{6.636604in}}%
\pgfpathlineto{\pgfqpoint{3.802641in}{6.548868in}}%
\pgfpathlineto{\pgfqpoint{3.714905in}{6.548868in}}%
\pgfpathlineto{\pgfqpoint{3.714905in}{6.636604in}}%
\pgfusepath{stroke,fill}%
\end{pgfscope}%
\begin{pgfscope}%
\pgfpathrectangle{\pgfqpoint{0.380943in}{6.110189in}}{\pgfqpoint{4.650000in}{0.614151in}}%
\pgfusepath{clip}%
\pgfsetbuttcap%
\pgfsetroundjoin%
\definecolor{currentfill}{rgb}{0.972549,0.870588,0.692810}%
\pgfsetfillcolor{currentfill}%
\pgfsetlinewidth{0.250937pt}%
\definecolor{currentstroke}{rgb}{1.000000,1.000000,1.000000}%
\pgfsetstrokecolor{currentstroke}%
\pgfsetdash{}{0pt}%
\pgfpathmoveto{\pgfqpoint{3.802641in}{6.636604in}}%
\pgfpathlineto{\pgfqpoint{3.890377in}{6.636604in}}%
\pgfpathlineto{\pgfqpoint{3.890377in}{6.548868in}}%
\pgfpathlineto{\pgfqpoint{3.802641in}{6.548868in}}%
\pgfpathlineto{\pgfqpoint{3.802641in}{6.636604in}}%
\pgfusepath{stroke,fill}%
\end{pgfscope}%
\begin{pgfscope}%
\pgfpathrectangle{\pgfqpoint{0.380943in}{6.110189in}}{\pgfqpoint{4.650000in}{0.614151in}}%
\pgfusepath{clip}%
\pgfsetbuttcap%
\pgfsetroundjoin%
\definecolor{currentfill}{rgb}{0.992326,0.765229,0.614840}%
\pgfsetfillcolor{currentfill}%
\pgfsetlinewidth{0.250937pt}%
\definecolor{currentstroke}{rgb}{1.000000,1.000000,1.000000}%
\pgfsetstrokecolor{currentstroke}%
\pgfsetdash{}{0pt}%
\pgfpathmoveto{\pgfqpoint{3.890377in}{6.636604in}}%
\pgfpathlineto{\pgfqpoint{3.978113in}{6.636604in}}%
\pgfpathlineto{\pgfqpoint{3.978113in}{6.548868in}}%
\pgfpathlineto{\pgfqpoint{3.890377in}{6.548868in}}%
\pgfpathlineto{\pgfqpoint{3.890377in}{6.636604in}}%
\pgfusepath{stroke,fill}%
\end{pgfscope}%
\begin{pgfscope}%
\pgfpathrectangle{\pgfqpoint{0.380943in}{6.110189in}}{\pgfqpoint{4.650000in}{0.614151in}}%
\pgfusepath{clip}%
\pgfsetbuttcap%
\pgfsetroundjoin%
\definecolor{currentfill}{rgb}{0.800000,0.278431,0.278431}%
\pgfsetfillcolor{currentfill}%
\pgfsetlinewidth{0.250937pt}%
\definecolor{currentstroke}{rgb}{1.000000,1.000000,1.000000}%
\pgfsetstrokecolor{currentstroke}%
\pgfsetdash{}{0pt}%
\pgfpathmoveto{\pgfqpoint{3.978113in}{6.636604in}}%
\pgfpathlineto{\pgfqpoint{4.065849in}{6.636604in}}%
\pgfpathlineto{\pgfqpoint{4.065849in}{6.548868in}}%
\pgfpathlineto{\pgfqpoint{3.978113in}{6.548868in}}%
\pgfpathlineto{\pgfqpoint{3.978113in}{6.636604in}}%
\pgfusepath{stroke,fill}%
\end{pgfscope}%
\begin{pgfscope}%
\pgfpathrectangle{\pgfqpoint{0.380943in}{6.110189in}}{\pgfqpoint{4.650000in}{0.614151in}}%
\pgfusepath{clip}%
\pgfsetbuttcap%
\pgfsetroundjoin%
\definecolor{currentfill}{rgb}{0.979654,0.837186,0.669619}%
\pgfsetfillcolor{currentfill}%
\pgfsetlinewidth{0.250937pt}%
\definecolor{currentstroke}{rgb}{1.000000,1.000000,1.000000}%
\pgfsetstrokecolor{currentstroke}%
\pgfsetdash{}{0pt}%
\pgfpathmoveto{\pgfqpoint{4.065849in}{6.636604in}}%
\pgfpathlineto{\pgfqpoint{4.153585in}{6.636604in}}%
\pgfpathlineto{\pgfqpoint{4.153585in}{6.548868in}}%
\pgfpathlineto{\pgfqpoint{4.065849in}{6.548868in}}%
\pgfpathlineto{\pgfqpoint{4.065849in}{6.636604in}}%
\pgfusepath{stroke,fill}%
\end{pgfscope}%
\begin{pgfscope}%
\pgfpathrectangle{\pgfqpoint{0.380943in}{6.110189in}}{\pgfqpoint{4.650000in}{0.614151in}}%
\pgfusepath{clip}%
\pgfsetbuttcap%
\pgfsetroundjoin%
\definecolor{currentfill}{rgb}{0.992326,0.765229,0.614840}%
\pgfsetfillcolor{currentfill}%
\pgfsetlinewidth{0.250937pt}%
\definecolor{currentstroke}{rgb}{1.000000,1.000000,1.000000}%
\pgfsetstrokecolor{currentstroke}%
\pgfsetdash{}{0pt}%
\pgfpathmoveto{\pgfqpoint{4.153585in}{6.636604in}}%
\pgfpathlineto{\pgfqpoint{4.241320in}{6.636604in}}%
\pgfpathlineto{\pgfqpoint{4.241320in}{6.548868in}}%
\pgfpathlineto{\pgfqpoint{4.153585in}{6.548868in}}%
\pgfpathlineto{\pgfqpoint{4.153585in}{6.636604in}}%
\pgfusepath{stroke,fill}%
\end{pgfscope}%
\begin{pgfscope}%
\pgfpathrectangle{\pgfqpoint{0.380943in}{6.110189in}}{\pgfqpoint{4.650000in}{0.614151in}}%
\pgfusepath{clip}%
\pgfsetbuttcap%
\pgfsetroundjoin%
\definecolor{currentfill}{rgb}{0.972549,0.870588,0.692810}%
\pgfsetfillcolor{currentfill}%
\pgfsetlinewidth{0.250937pt}%
\definecolor{currentstroke}{rgb}{1.000000,1.000000,1.000000}%
\pgfsetstrokecolor{currentstroke}%
\pgfsetdash{}{0pt}%
\pgfpathmoveto{\pgfqpoint{4.241320in}{6.636604in}}%
\pgfpathlineto{\pgfqpoint{4.329056in}{6.636604in}}%
\pgfpathlineto{\pgfqpoint{4.329056in}{6.548868in}}%
\pgfpathlineto{\pgfqpoint{4.241320in}{6.548868in}}%
\pgfpathlineto{\pgfqpoint{4.241320in}{6.636604in}}%
\pgfusepath{stroke,fill}%
\end{pgfscope}%
\begin{pgfscope}%
\pgfpathrectangle{\pgfqpoint{0.380943in}{6.110189in}}{\pgfqpoint{4.650000in}{0.614151in}}%
\pgfusepath{clip}%
\pgfsetbuttcap%
\pgfsetroundjoin%
\definecolor{currentfill}{rgb}{0.965444,0.906113,0.711757}%
\pgfsetfillcolor{currentfill}%
\pgfsetlinewidth{0.250937pt}%
\definecolor{currentstroke}{rgb}{1.000000,1.000000,1.000000}%
\pgfsetstrokecolor{currentstroke}%
\pgfsetdash{}{0pt}%
\pgfpathmoveto{\pgfqpoint{4.329056in}{6.636604in}}%
\pgfpathlineto{\pgfqpoint{4.416792in}{6.636604in}}%
\pgfpathlineto{\pgfqpoint{4.416792in}{6.548868in}}%
\pgfpathlineto{\pgfqpoint{4.329056in}{6.548868in}}%
\pgfpathlineto{\pgfqpoint{4.329056in}{6.636604in}}%
\pgfusepath{stroke,fill}%
\end{pgfscope}%
\begin{pgfscope}%
\pgfpathrectangle{\pgfqpoint{0.380943in}{6.110189in}}{\pgfqpoint{4.650000in}{0.614151in}}%
\pgfusepath{clip}%
\pgfsetbuttcap%
\pgfsetroundjoin%
\definecolor{currentfill}{rgb}{0.992326,0.765229,0.614840}%
\pgfsetfillcolor{currentfill}%
\pgfsetlinewidth{0.250937pt}%
\definecolor{currentstroke}{rgb}{1.000000,1.000000,1.000000}%
\pgfsetstrokecolor{currentstroke}%
\pgfsetdash{}{0pt}%
\pgfpathmoveto{\pgfqpoint{4.416792in}{6.636604in}}%
\pgfpathlineto{\pgfqpoint{4.504528in}{6.636604in}}%
\pgfpathlineto{\pgfqpoint{4.504528in}{6.548868in}}%
\pgfpathlineto{\pgfqpoint{4.416792in}{6.548868in}}%
\pgfpathlineto{\pgfqpoint{4.416792in}{6.636604in}}%
\pgfusepath{stroke,fill}%
\end{pgfscope}%
\begin{pgfscope}%
\pgfpathrectangle{\pgfqpoint{0.380943in}{6.110189in}}{\pgfqpoint{4.650000in}{0.614151in}}%
\pgfusepath{clip}%
\pgfsetbuttcap%
\pgfsetroundjoin%
\definecolor{currentfill}{rgb}{0.998939,0.658962,0.556032}%
\pgfsetfillcolor{currentfill}%
\pgfsetlinewidth{0.250937pt}%
\definecolor{currentstroke}{rgb}{1.000000,1.000000,1.000000}%
\pgfsetstrokecolor{currentstroke}%
\pgfsetdash{}{0pt}%
\pgfpathmoveto{\pgfqpoint{4.504528in}{6.636604in}}%
\pgfpathlineto{\pgfqpoint{4.592264in}{6.636604in}}%
\pgfpathlineto{\pgfqpoint{4.592264in}{6.548868in}}%
\pgfpathlineto{\pgfqpoint{4.504528in}{6.548868in}}%
\pgfpathlineto{\pgfqpoint{4.504528in}{6.636604in}}%
\pgfusepath{stroke,fill}%
\end{pgfscope}%
\begin{pgfscope}%
\pgfpathrectangle{\pgfqpoint{0.380943in}{6.110189in}}{\pgfqpoint{4.650000in}{0.614151in}}%
\pgfusepath{clip}%
\pgfsetbuttcap%
\pgfsetroundjoin%
\definecolor{currentfill}{rgb}{0.986759,0.806398,0.641200}%
\pgfsetfillcolor{currentfill}%
\pgfsetlinewidth{0.250937pt}%
\definecolor{currentstroke}{rgb}{1.000000,1.000000,1.000000}%
\pgfsetstrokecolor{currentstroke}%
\pgfsetdash{}{0pt}%
\pgfpathmoveto{\pgfqpoint{4.592264in}{6.636604in}}%
\pgfpathlineto{\pgfqpoint{4.680000in}{6.636604in}}%
\pgfpathlineto{\pgfqpoint{4.680000in}{6.548868in}}%
\pgfpathlineto{\pgfqpoint{4.592264in}{6.548868in}}%
\pgfpathlineto{\pgfqpoint{4.592264in}{6.636604in}}%
\pgfusepath{stroke,fill}%
\end{pgfscope}%
\begin{pgfscope}%
\pgfpathrectangle{\pgfqpoint{0.380943in}{6.110189in}}{\pgfqpoint{4.650000in}{0.614151in}}%
\pgfusepath{clip}%
\pgfsetbuttcap%
\pgfsetroundjoin%
\definecolor{currentfill}{rgb}{1.000000,0.605229,0.530719}%
\pgfsetfillcolor{currentfill}%
\pgfsetlinewidth{0.250937pt}%
\definecolor{currentstroke}{rgb}{1.000000,1.000000,1.000000}%
\pgfsetstrokecolor{currentstroke}%
\pgfsetdash{}{0pt}%
\pgfpathmoveto{\pgfqpoint{4.680000in}{6.636604in}}%
\pgfpathlineto{\pgfqpoint{4.767736in}{6.636604in}}%
\pgfpathlineto{\pgfqpoint{4.767736in}{6.548868in}}%
\pgfpathlineto{\pgfqpoint{4.680000in}{6.548868in}}%
\pgfpathlineto{\pgfqpoint{4.680000in}{6.636604in}}%
\pgfusepath{stroke,fill}%
\end{pgfscope}%
\begin{pgfscope}%
\pgfpathrectangle{\pgfqpoint{0.380943in}{6.110189in}}{\pgfqpoint{4.650000in}{0.614151in}}%
\pgfusepath{clip}%
\pgfsetbuttcap%
\pgfsetroundjoin%
\definecolor{currentfill}{rgb}{0.965444,0.906113,0.711757}%
\pgfsetfillcolor{currentfill}%
\pgfsetlinewidth{0.250937pt}%
\definecolor{currentstroke}{rgb}{1.000000,1.000000,1.000000}%
\pgfsetstrokecolor{currentstroke}%
\pgfsetdash{}{0pt}%
\pgfpathmoveto{\pgfqpoint{4.767736in}{6.636604in}}%
\pgfpathlineto{\pgfqpoint{4.855471in}{6.636604in}}%
\pgfpathlineto{\pgfqpoint{4.855471in}{6.548868in}}%
\pgfpathlineto{\pgfqpoint{4.767736in}{6.548868in}}%
\pgfpathlineto{\pgfqpoint{4.767736in}{6.636604in}}%
\pgfusepath{stroke,fill}%
\end{pgfscope}%
\begin{pgfscope}%
\pgfpathrectangle{\pgfqpoint{0.380943in}{6.110189in}}{\pgfqpoint{4.650000in}{0.614151in}}%
\pgfusepath{clip}%
\pgfsetbuttcap%
\pgfsetroundjoin%
\definecolor{currentfill}{rgb}{1.000000,1.000000,0.870204}%
\pgfsetfillcolor{currentfill}%
\pgfsetlinewidth{0.250937pt}%
\definecolor{currentstroke}{rgb}{1.000000,1.000000,1.000000}%
\pgfsetstrokecolor{currentstroke}%
\pgfsetdash{}{0pt}%
\pgfpathmoveto{\pgfqpoint{4.855471in}{6.636604in}}%
\pgfpathlineto{\pgfqpoint{4.943207in}{6.636604in}}%
\pgfpathlineto{\pgfqpoint{4.943207in}{6.548868in}}%
\pgfpathlineto{\pgfqpoint{4.855471in}{6.548868in}}%
\pgfpathlineto{\pgfqpoint{4.855471in}{6.636604in}}%
\pgfusepath{stroke,fill}%
\end{pgfscope}%
\begin{pgfscope}%
\pgfpathrectangle{\pgfqpoint{0.380943in}{6.110189in}}{\pgfqpoint{4.650000in}{0.614151in}}%
\pgfusepath{clip}%
\pgfsetbuttcap%
\pgfsetroundjoin%
\pgfsetlinewidth{0.250937pt}%
\definecolor{currentstroke}{rgb}{1.000000,1.000000,1.000000}%
\pgfsetstrokecolor{currentstroke}%
\pgfsetdash{}{0pt}%
\pgfpathmoveto{\pgfqpoint{4.943207in}{6.636604in}}%
\pgfpathlineto{\pgfqpoint{5.030943in}{6.636604in}}%
\pgfpathlineto{\pgfqpoint{5.030943in}{6.548868in}}%
\pgfpathlineto{\pgfqpoint{4.943207in}{6.548868in}}%
\pgfpathlineto{\pgfqpoint{4.943207in}{6.636604in}}%
\pgfusepath{stroke}%
\end{pgfscope}%
\begin{pgfscope}%
\pgfpathrectangle{\pgfqpoint{0.380943in}{6.110189in}}{\pgfqpoint{4.650000in}{0.614151in}}%
\pgfusepath{clip}%
\pgfsetbuttcap%
\pgfsetroundjoin%
\definecolor{currentfill}{rgb}{0.981546,0.459977,0.459977}%
\pgfsetfillcolor{currentfill}%
\pgfsetlinewidth{0.250937pt}%
\definecolor{currentstroke}{rgb}{1.000000,1.000000,1.000000}%
\pgfsetstrokecolor{currentstroke}%
\pgfsetdash{}{0pt}%
\pgfpathmoveto{\pgfqpoint{0.380943in}{6.548868in}}%
\pgfpathlineto{\pgfqpoint{0.468679in}{6.548868in}}%
\pgfpathlineto{\pgfqpoint{0.468679in}{6.461132in}}%
\pgfpathlineto{\pgfqpoint{0.380943in}{6.461132in}}%
\pgfpathlineto{\pgfqpoint{0.380943in}{6.548868in}}%
\pgfusepath{stroke,fill}%
\end{pgfscope}%
\begin{pgfscope}%
\pgfpathrectangle{\pgfqpoint{0.380943in}{6.110189in}}{\pgfqpoint{4.650000in}{0.614151in}}%
\pgfusepath{clip}%
\pgfsetbuttcap%
\pgfsetroundjoin%
\definecolor{currentfill}{rgb}{0.965444,0.906113,0.711757}%
\pgfsetfillcolor{currentfill}%
\pgfsetlinewidth{0.250937pt}%
\definecolor{currentstroke}{rgb}{1.000000,1.000000,1.000000}%
\pgfsetstrokecolor{currentstroke}%
\pgfsetdash{}{0pt}%
\pgfpathmoveto{\pgfqpoint{0.468679in}{6.548868in}}%
\pgfpathlineto{\pgfqpoint{0.556415in}{6.548868in}}%
\pgfpathlineto{\pgfqpoint{0.556415in}{6.461132in}}%
\pgfpathlineto{\pgfqpoint{0.468679in}{6.461132in}}%
\pgfpathlineto{\pgfqpoint{0.468679in}{6.548868in}}%
\pgfusepath{stroke,fill}%
\end{pgfscope}%
\begin{pgfscope}%
\pgfpathrectangle{\pgfqpoint{0.380943in}{6.110189in}}{\pgfqpoint{4.650000in}{0.614151in}}%
\pgfusepath{clip}%
\pgfsetbuttcap%
\pgfsetroundjoin%
\definecolor{currentfill}{rgb}{0.986759,0.806398,0.641200}%
\pgfsetfillcolor{currentfill}%
\pgfsetlinewidth{0.250937pt}%
\definecolor{currentstroke}{rgb}{1.000000,1.000000,1.000000}%
\pgfsetstrokecolor{currentstroke}%
\pgfsetdash{}{0pt}%
\pgfpathmoveto{\pgfqpoint{0.556415in}{6.548868in}}%
\pgfpathlineto{\pgfqpoint{0.644151in}{6.548868in}}%
\pgfpathlineto{\pgfqpoint{0.644151in}{6.461132in}}%
\pgfpathlineto{\pgfqpoint{0.556415in}{6.461132in}}%
\pgfpathlineto{\pgfqpoint{0.556415in}{6.548868in}}%
\pgfusepath{stroke,fill}%
\end{pgfscope}%
\begin{pgfscope}%
\pgfpathrectangle{\pgfqpoint{0.380943in}{6.110189in}}{\pgfqpoint{4.650000in}{0.614151in}}%
\pgfusepath{clip}%
\pgfsetbuttcap%
\pgfsetroundjoin%
\definecolor{currentfill}{rgb}{1.000000,0.509404,0.491473}%
\pgfsetfillcolor{currentfill}%
\pgfsetlinewidth{0.250937pt}%
\definecolor{currentstroke}{rgb}{1.000000,1.000000,1.000000}%
\pgfsetstrokecolor{currentstroke}%
\pgfsetdash{}{0pt}%
\pgfpathmoveto{\pgfqpoint{0.644151in}{6.548868in}}%
\pgfpathlineto{\pgfqpoint{0.731886in}{6.548868in}}%
\pgfpathlineto{\pgfqpoint{0.731886in}{6.461132in}}%
\pgfpathlineto{\pgfqpoint{0.644151in}{6.461132in}}%
\pgfpathlineto{\pgfqpoint{0.644151in}{6.548868in}}%
\pgfusepath{stroke,fill}%
\end{pgfscope}%
\begin{pgfscope}%
\pgfpathrectangle{\pgfqpoint{0.380943in}{6.110189in}}{\pgfqpoint{4.650000in}{0.614151in}}%
\pgfusepath{clip}%
\pgfsetbuttcap%
\pgfsetroundjoin%
\definecolor{currentfill}{rgb}{0.965444,0.906113,0.711757}%
\pgfsetfillcolor{currentfill}%
\pgfsetlinewidth{0.250937pt}%
\definecolor{currentstroke}{rgb}{1.000000,1.000000,1.000000}%
\pgfsetstrokecolor{currentstroke}%
\pgfsetdash{}{0pt}%
\pgfpathmoveto{\pgfqpoint{0.731886in}{6.548868in}}%
\pgfpathlineto{\pgfqpoint{0.819622in}{6.548868in}}%
\pgfpathlineto{\pgfqpoint{0.819622in}{6.461132in}}%
\pgfpathlineto{\pgfqpoint{0.731886in}{6.461132in}}%
\pgfpathlineto{\pgfqpoint{0.731886in}{6.548868in}}%
\pgfusepath{stroke,fill}%
\end{pgfscope}%
\begin{pgfscope}%
\pgfpathrectangle{\pgfqpoint{0.380943in}{6.110189in}}{\pgfqpoint{4.650000in}{0.614151in}}%
\pgfusepath{clip}%
\pgfsetbuttcap%
\pgfsetroundjoin%
\definecolor{currentfill}{rgb}{1.000000,0.605229,0.530719}%
\pgfsetfillcolor{currentfill}%
\pgfsetlinewidth{0.250937pt}%
\definecolor{currentstroke}{rgb}{1.000000,1.000000,1.000000}%
\pgfsetstrokecolor{currentstroke}%
\pgfsetdash{}{0pt}%
\pgfpathmoveto{\pgfqpoint{0.819622in}{6.548868in}}%
\pgfpathlineto{\pgfqpoint{0.907358in}{6.548868in}}%
\pgfpathlineto{\pgfqpoint{0.907358in}{6.461132in}}%
\pgfpathlineto{\pgfqpoint{0.819622in}{6.461132in}}%
\pgfpathlineto{\pgfqpoint{0.819622in}{6.548868in}}%
\pgfusepath{stroke,fill}%
\end{pgfscope}%
\begin{pgfscope}%
\pgfpathrectangle{\pgfqpoint{0.380943in}{6.110189in}}{\pgfqpoint{4.650000in}{0.614151in}}%
\pgfusepath{clip}%
\pgfsetbuttcap%
\pgfsetroundjoin%
\definecolor{currentfill}{rgb}{0.996571,0.720538,0.589189}%
\pgfsetfillcolor{currentfill}%
\pgfsetlinewidth{0.250937pt}%
\definecolor{currentstroke}{rgb}{1.000000,1.000000,1.000000}%
\pgfsetstrokecolor{currentstroke}%
\pgfsetdash{}{0pt}%
\pgfpathmoveto{\pgfqpoint{0.907358in}{6.548868in}}%
\pgfpathlineto{\pgfqpoint{0.995094in}{6.548868in}}%
\pgfpathlineto{\pgfqpoint{0.995094in}{6.461132in}}%
\pgfpathlineto{\pgfqpoint{0.907358in}{6.461132in}}%
\pgfpathlineto{\pgfqpoint{0.907358in}{6.548868in}}%
\pgfusepath{stroke,fill}%
\end{pgfscope}%
\begin{pgfscope}%
\pgfpathrectangle{\pgfqpoint{0.380943in}{6.110189in}}{\pgfqpoint{4.650000in}{0.614151in}}%
\pgfusepath{clip}%
\pgfsetbuttcap%
\pgfsetroundjoin%
\definecolor{currentfill}{rgb}{0.996571,0.720538,0.589189}%
\pgfsetfillcolor{currentfill}%
\pgfsetlinewidth{0.250937pt}%
\definecolor{currentstroke}{rgb}{1.000000,1.000000,1.000000}%
\pgfsetstrokecolor{currentstroke}%
\pgfsetdash{}{0pt}%
\pgfpathmoveto{\pgfqpoint{0.995094in}{6.548868in}}%
\pgfpathlineto{\pgfqpoint{1.082830in}{6.548868in}}%
\pgfpathlineto{\pgfqpoint{1.082830in}{6.461132in}}%
\pgfpathlineto{\pgfqpoint{0.995094in}{6.461132in}}%
\pgfpathlineto{\pgfqpoint{0.995094in}{6.548868in}}%
\pgfusepath{stroke,fill}%
\end{pgfscope}%
\begin{pgfscope}%
\pgfpathrectangle{\pgfqpoint{0.380943in}{6.110189in}}{\pgfqpoint{4.650000in}{0.614151in}}%
\pgfusepath{clip}%
\pgfsetbuttcap%
\pgfsetroundjoin%
\definecolor{currentfill}{rgb}{0.986759,0.806398,0.641200}%
\pgfsetfillcolor{currentfill}%
\pgfsetlinewidth{0.250937pt}%
\definecolor{currentstroke}{rgb}{1.000000,1.000000,1.000000}%
\pgfsetstrokecolor{currentstroke}%
\pgfsetdash{}{0pt}%
\pgfpathmoveto{\pgfqpoint{1.082830in}{6.548868in}}%
\pgfpathlineto{\pgfqpoint{1.170566in}{6.548868in}}%
\pgfpathlineto{\pgfqpoint{1.170566in}{6.461132in}}%
\pgfpathlineto{\pgfqpoint{1.082830in}{6.461132in}}%
\pgfpathlineto{\pgfqpoint{1.082830in}{6.548868in}}%
\pgfusepath{stroke,fill}%
\end{pgfscope}%
\begin{pgfscope}%
\pgfpathrectangle{\pgfqpoint{0.380943in}{6.110189in}}{\pgfqpoint{4.650000in}{0.614151in}}%
\pgfusepath{clip}%
\pgfsetbuttcap%
\pgfsetroundjoin%
\definecolor{currentfill}{rgb}{0.998939,0.658962,0.556032}%
\pgfsetfillcolor{currentfill}%
\pgfsetlinewidth{0.250937pt}%
\definecolor{currentstroke}{rgb}{1.000000,1.000000,1.000000}%
\pgfsetstrokecolor{currentstroke}%
\pgfsetdash{}{0pt}%
\pgfpathmoveto{\pgfqpoint{1.170566in}{6.548868in}}%
\pgfpathlineto{\pgfqpoint{1.258302in}{6.548868in}}%
\pgfpathlineto{\pgfqpoint{1.258302in}{6.461132in}}%
\pgfpathlineto{\pgfqpoint{1.170566in}{6.461132in}}%
\pgfpathlineto{\pgfqpoint{1.170566in}{6.548868in}}%
\pgfusepath{stroke,fill}%
\end{pgfscope}%
\begin{pgfscope}%
\pgfpathrectangle{\pgfqpoint{0.380943in}{6.110189in}}{\pgfqpoint{4.650000in}{0.614151in}}%
\pgfusepath{clip}%
\pgfsetbuttcap%
\pgfsetroundjoin%
\definecolor{currentfill}{rgb}{0.922338,0.400769,0.400769}%
\pgfsetfillcolor{currentfill}%
\pgfsetlinewidth{0.250937pt}%
\definecolor{currentstroke}{rgb}{1.000000,1.000000,1.000000}%
\pgfsetstrokecolor{currentstroke}%
\pgfsetdash{}{0pt}%
\pgfpathmoveto{\pgfqpoint{1.258302in}{6.548868in}}%
\pgfpathlineto{\pgfqpoint{1.346037in}{6.548868in}}%
\pgfpathlineto{\pgfqpoint{1.346037in}{6.461132in}}%
\pgfpathlineto{\pgfqpoint{1.258302in}{6.461132in}}%
\pgfpathlineto{\pgfqpoint{1.258302in}{6.548868in}}%
\pgfusepath{stroke,fill}%
\end{pgfscope}%
\begin{pgfscope}%
\pgfpathrectangle{\pgfqpoint{0.380943in}{6.110189in}}{\pgfqpoint{4.650000in}{0.614151in}}%
\pgfusepath{clip}%
\pgfsetbuttcap%
\pgfsetroundjoin%
\definecolor{currentfill}{rgb}{0.998939,0.658962,0.556032}%
\pgfsetfillcolor{currentfill}%
\pgfsetlinewidth{0.250937pt}%
\definecolor{currentstroke}{rgb}{1.000000,1.000000,1.000000}%
\pgfsetstrokecolor{currentstroke}%
\pgfsetdash{}{0pt}%
\pgfpathmoveto{\pgfqpoint{1.346037in}{6.548868in}}%
\pgfpathlineto{\pgfqpoint{1.433773in}{6.548868in}}%
\pgfpathlineto{\pgfqpoint{1.433773in}{6.461132in}}%
\pgfpathlineto{\pgfqpoint{1.346037in}{6.461132in}}%
\pgfpathlineto{\pgfqpoint{1.346037in}{6.548868in}}%
\pgfusepath{stroke,fill}%
\end{pgfscope}%
\begin{pgfscope}%
\pgfpathrectangle{\pgfqpoint{0.380943in}{6.110189in}}{\pgfqpoint{4.650000in}{0.614151in}}%
\pgfusepath{clip}%
\pgfsetbuttcap%
\pgfsetroundjoin%
\definecolor{currentfill}{rgb}{0.962414,0.923552,0.722891}%
\pgfsetfillcolor{currentfill}%
\pgfsetlinewidth{0.250937pt}%
\definecolor{currentstroke}{rgb}{1.000000,1.000000,1.000000}%
\pgfsetstrokecolor{currentstroke}%
\pgfsetdash{}{0pt}%
\pgfpathmoveto{\pgfqpoint{1.433773in}{6.548868in}}%
\pgfpathlineto{\pgfqpoint{1.521509in}{6.548868in}}%
\pgfpathlineto{\pgfqpoint{1.521509in}{6.461132in}}%
\pgfpathlineto{\pgfqpoint{1.433773in}{6.461132in}}%
\pgfpathlineto{\pgfqpoint{1.433773in}{6.548868in}}%
\pgfusepath{stroke,fill}%
\end{pgfscope}%
\begin{pgfscope}%
\pgfpathrectangle{\pgfqpoint{0.380943in}{6.110189in}}{\pgfqpoint{4.650000in}{0.614151in}}%
\pgfusepath{clip}%
\pgfsetbuttcap%
\pgfsetroundjoin%
\definecolor{currentfill}{rgb}{0.922338,0.400769,0.400769}%
\pgfsetfillcolor{currentfill}%
\pgfsetlinewidth{0.250937pt}%
\definecolor{currentstroke}{rgb}{1.000000,1.000000,1.000000}%
\pgfsetstrokecolor{currentstroke}%
\pgfsetdash{}{0pt}%
\pgfpathmoveto{\pgfqpoint{1.521509in}{6.548868in}}%
\pgfpathlineto{\pgfqpoint{1.609245in}{6.548868in}}%
\pgfpathlineto{\pgfqpoint{1.609245in}{6.461132in}}%
\pgfpathlineto{\pgfqpoint{1.521509in}{6.461132in}}%
\pgfpathlineto{\pgfqpoint{1.521509in}{6.548868in}}%
\pgfusepath{stroke,fill}%
\end{pgfscope}%
\begin{pgfscope}%
\pgfpathrectangle{\pgfqpoint{0.380943in}{6.110189in}}{\pgfqpoint{4.650000in}{0.614151in}}%
\pgfusepath{clip}%
\pgfsetbuttcap%
\pgfsetroundjoin%
\definecolor{currentfill}{rgb}{1.000000,0.605229,0.530719}%
\pgfsetfillcolor{currentfill}%
\pgfsetlinewidth{0.250937pt}%
\definecolor{currentstroke}{rgb}{1.000000,1.000000,1.000000}%
\pgfsetstrokecolor{currentstroke}%
\pgfsetdash{}{0pt}%
\pgfpathmoveto{\pgfqpoint{1.609245in}{6.548868in}}%
\pgfpathlineto{\pgfqpoint{1.696981in}{6.548868in}}%
\pgfpathlineto{\pgfqpoint{1.696981in}{6.461132in}}%
\pgfpathlineto{\pgfqpoint{1.609245in}{6.461132in}}%
\pgfpathlineto{\pgfqpoint{1.609245in}{6.548868in}}%
\pgfusepath{stroke,fill}%
\end{pgfscope}%
\begin{pgfscope}%
\pgfpathrectangle{\pgfqpoint{0.380943in}{6.110189in}}{\pgfqpoint{4.650000in}{0.614151in}}%
\pgfusepath{clip}%
\pgfsetbuttcap%
\pgfsetroundjoin%
\definecolor{currentfill}{rgb}{1.000000,0.509404,0.491473}%
\pgfsetfillcolor{currentfill}%
\pgfsetlinewidth{0.250937pt}%
\definecolor{currentstroke}{rgb}{1.000000,1.000000,1.000000}%
\pgfsetstrokecolor{currentstroke}%
\pgfsetdash{}{0pt}%
\pgfpathmoveto{\pgfqpoint{1.696981in}{6.548868in}}%
\pgfpathlineto{\pgfqpoint{1.784717in}{6.548868in}}%
\pgfpathlineto{\pgfqpoint{1.784717in}{6.461132in}}%
\pgfpathlineto{\pgfqpoint{1.696981in}{6.461132in}}%
\pgfpathlineto{\pgfqpoint{1.696981in}{6.548868in}}%
\pgfusepath{stroke,fill}%
\end{pgfscope}%
\begin{pgfscope}%
\pgfpathrectangle{\pgfqpoint{0.380943in}{6.110189in}}{\pgfqpoint{4.650000in}{0.614151in}}%
\pgfusepath{clip}%
\pgfsetbuttcap%
\pgfsetroundjoin%
\definecolor{currentfill}{rgb}{0.992326,0.765229,0.614840}%
\pgfsetfillcolor{currentfill}%
\pgfsetlinewidth{0.250937pt}%
\definecolor{currentstroke}{rgb}{1.000000,1.000000,1.000000}%
\pgfsetstrokecolor{currentstroke}%
\pgfsetdash{}{0pt}%
\pgfpathmoveto{\pgfqpoint{1.784717in}{6.548868in}}%
\pgfpathlineto{\pgfqpoint{1.872452in}{6.548868in}}%
\pgfpathlineto{\pgfqpoint{1.872452in}{6.461132in}}%
\pgfpathlineto{\pgfqpoint{1.784717in}{6.461132in}}%
\pgfpathlineto{\pgfqpoint{1.784717in}{6.548868in}}%
\pgfusepath{stroke,fill}%
\end{pgfscope}%
\begin{pgfscope}%
\pgfpathrectangle{\pgfqpoint{0.380943in}{6.110189in}}{\pgfqpoint{4.650000in}{0.614151in}}%
\pgfusepath{clip}%
\pgfsetbuttcap%
\pgfsetroundjoin%
\definecolor{currentfill}{rgb}{0.996571,0.720538,0.589189}%
\pgfsetfillcolor{currentfill}%
\pgfsetlinewidth{0.250937pt}%
\definecolor{currentstroke}{rgb}{1.000000,1.000000,1.000000}%
\pgfsetstrokecolor{currentstroke}%
\pgfsetdash{}{0pt}%
\pgfpathmoveto{\pgfqpoint{1.872452in}{6.548868in}}%
\pgfpathlineto{\pgfqpoint{1.960188in}{6.548868in}}%
\pgfpathlineto{\pgfqpoint{1.960188in}{6.461132in}}%
\pgfpathlineto{\pgfqpoint{1.872452in}{6.461132in}}%
\pgfpathlineto{\pgfqpoint{1.872452in}{6.548868in}}%
\pgfusepath{stroke,fill}%
\end{pgfscope}%
\begin{pgfscope}%
\pgfpathrectangle{\pgfqpoint{0.380943in}{6.110189in}}{\pgfqpoint{4.650000in}{0.614151in}}%
\pgfusepath{clip}%
\pgfsetbuttcap%
\pgfsetroundjoin%
\definecolor{currentfill}{rgb}{0.965444,0.906113,0.711757}%
\pgfsetfillcolor{currentfill}%
\pgfsetlinewidth{0.250937pt}%
\definecolor{currentstroke}{rgb}{1.000000,1.000000,1.000000}%
\pgfsetstrokecolor{currentstroke}%
\pgfsetdash{}{0pt}%
\pgfpathmoveto{\pgfqpoint{1.960188in}{6.548868in}}%
\pgfpathlineto{\pgfqpoint{2.047924in}{6.548868in}}%
\pgfpathlineto{\pgfqpoint{2.047924in}{6.461132in}}%
\pgfpathlineto{\pgfqpoint{1.960188in}{6.461132in}}%
\pgfpathlineto{\pgfqpoint{1.960188in}{6.548868in}}%
\pgfusepath{stroke,fill}%
\end{pgfscope}%
\begin{pgfscope}%
\pgfpathrectangle{\pgfqpoint{0.380943in}{6.110189in}}{\pgfqpoint{4.650000in}{0.614151in}}%
\pgfusepath{clip}%
\pgfsetbuttcap%
\pgfsetroundjoin%
\definecolor{currentfill}{rgb}{0.968166,0.945882,0.748604}%
\pgfsetfillcolor{currentfill}%
\pgfsetlinewidth{0.250937pt}%
\definecolor{currentstroke}{rgb}{1.000000,1.000000,1.000000}%
\pgfsetstrokecolor{currentstroke}%
\pgfsetdash{}{0pt}%
\pgfpathmoveto{\pgfqpoint{2.047924in}{6.548868in}}%
\pgfpathlineto{\pgfqpoint{2.135660in}{6.548868in}}%
\pgfpathlineto{\pgfqpoint{2.135660in}{6.461132in}}%
\pgfpathlineto{\pgfqpoint{2.047924in}{6.461132in}}%
\pgfpathlineto{\pgfqpoint{2.047924in}{6.548868in}}%
\pgfusepath{stroke,fill}%
\end{pgfscope}%
\begin{pgfscope}%
\pgfpathrectangle{\pgfqpoint{0.380943in}{6.110189in}}{\pgfqpoint{4.650000in}{0.614151in}}%
\pgfusepath{clip}%
\pgfsetbuttcap%
\pgfsetroundjoin%
\definecolor{currentfill}{rgb}{0.968166,0.945882,0.748604}%
\pgfsetfillcolor{currentfill}%
\pgfsetlinewidth{0.250937pt}%
\definecolor{currentstroke}{rgb}{1.000000,1.000000,1.000000}%
\pgfsetstrokecolor{currentstroke}%
\pgfsetdash{}{0pt}%
\pgfpathmoveto{\pgfqpoint{2.135660in}{6.548868in}}%
\pgfpathlineto{\pgfqpoint{2.223396in}{6.548868in}}%
\pgfpathlineto{\pgfqpoint{2.223396in}{6.461132in}}%
\pgfpathlineto{\pgfqpoint{2.135660in}{6.461132in}}%
\pgfpathlineto{\pgfqpoint{2.135660in}{6.548868in}}%
\pgfusepath{stroke,fill}%
\end{pgfscope}%
\begin{pgfscope}%
\pgfpathrectangle{\pgfqpoint{0.380943in}{6.110189in}}{\pgfqpoint{4.650000in}{0.614151in}}%
\pgfusepath{clip}%
\pgfsetbuttcap%
\pgfsetroundjoin%
\definecolor{currentfill}{rgb}{1.000000,0.605229,0.530719}%
\pgfsetfillcolor{currentfill}%
\pgfsetlinewidth{0.250937pt}%
\definecolor{currentstroke}{rgb}{1.000000,1.000000,1.000000}%
\pgfsetstrokecolor{currentstroke}%
\pgfsetdash{}{0pt}%
\pgfpathmoveto{\pgfqpoint{2.223396in}{6.548868in}}%
\pgfpathlineto{\pgfqpoint{2.311132in}{6.548868in}}%
\pgfpathlineto{\pgfqpoint{2.311132in}{6.461132in}}%
\pgfpathlineto{\pgfqpoint{2.223396in}{6.461132in}}%
\pgfpathlineto{\pgfqpoint{2.223396in}{6.548868in}}%
\pgfusepath{stroke,fill}%
\end{pgfscope}%
\begin{pgfscope}%
\pgfpathrectangle{\pgfqpoint{0.380943in}{6.110189in}}{\pgfqpoint{4.650000in}{0.614151in}}%
\pgfusepath{clip}%
\pgfsetbuttcap%
\pgfsetroundjoin%
\definecolor{currentfill}{rgb}{1.000000,0.605229,0.530719}%
\pgfsetfillcolor{currentfill}%
\pgfsetlinewidth{0.250937pt}%
\definecolor{currentstroke}{rgb}{1.000000,1.000000,1.000000}%
\pgfsetstrokecolor{currentstroke}%
\pgfsetdash{}{0pt}%
\pgfpathmoveto{\pgfqpoint{2.311132in}{6.548868in}}%
\pgfpathlineto{\pgfqpoint{2.398868in}{6.548868in}}%
\pgfpathlineto{\pgfqpoint{2.398868in}{6.461132in}}%
\pgfpathlineto{\pgfqpoint{2.311132in}{6.461132in}}%
\pgfpathlineto{\pgfqpoint{2.311132in}{6.548868in}}%
\pgfusepath{stroke,fill}%
\end{pgfscope}%
\begin{pgfscope}%
\pgfpathrectangle{\pgfqpoint{0.380943in}{6.110189in}}{\pgfqpoint{4.650000in}{0.614151in}}%
\pgfusepath{clip}%
\pgfsetbuttcap%
\pgfsetroundjoin%
\definecolor{currentfill}{rgb}{0.992326,0.765229,0.614840}%
\pgfsetfillcolor{currentfill}%
\pgfsetlinewidth{0.250937pt}%
\definecolor{currentstroke}{rgb}{1.000000,1.000000,1.000000}%
\pgfsetstrokecolor{currentstroke}%
\pgfsetdash{}{0pt}%
\pgfpathmoveto{\pgfqpoint{2.398868in}{6.548868in}}%
\pgfpathlineto{\pgfqpoint{2.486603in}{6.548868in}}%
\pgfpathlineto{\pgfqpoint{2.486603in}{6.461132in}}%
\pgfpathlineto{\pgfqpoint{2.398868in}{6.461132in}}%
\pgfpathlineto{\pgfqpoint{2.398868in}{6.548868in}}%
\pgfusepath{stroke,fill}%
\end{pgfscope}%
\begin{pgfscope}%
\pgfpathrectangle{\pgfqpoint{0.380943in}{6.110189in}}{\pgfqpoint{4.650000in}{0.614151in}}%
\pgfusepath{clip}%
\pgfsetbuttcap%
\pgfsetroundjoin%
\definecolor{currentfill}{rgb}{0.968166,0.945882,0.748604}%
\pgfsetfillcolor{currentfill}%
\pgfsetlinewidth{0.250937pt}%
\definecolor{currentstroke}{rgb}{1.000000,1.000000,1.000000}%
\pgfsetstrokecolor{currentstroke}%
\pgfsetdash{}{0pt}%
\pgfpathmoveto{\pgfqpoint{2.486603in}{6.548868in}}%
\pgfpathlineto{\pgfqpoint{2.574339in}{6.548868in}}%
\pgfpathlineto{\pgfqpoint{2.574339in}{6.461132in}}%
\pgfpathlineto{\pgfqpoint{2.486603in}{6.461132in}}%
\pgfpathlineto{\pgfqpoint{2.486603in}{6.548868in}}%
\pgfusepath{stroke,fill}%
\end{pgfscope}%
\begin{pgfscope}%
\pgfpathrectangle{\pgfqpoint{0.380943in}{6.110189in}}{\pgfqpoint{4.650000in}{0.614151in}}%
\pgfusepath{clip}%
\pgfsetbuttcap%
\pgfsetroundjoin%
\definecolor{currentfill}{rgb}{0.962414,0.923552,0.722891}%
\pgfsetfillcolor{currentfill}%
\pgfsetlinewidth{0.250937pt}%
\definecolor{currentstroke}{rgb}{1.000000,1.000000,1.000000}%
\pgfsetstrokecolor{currentstroke}%
\pgfsetdash{}{0pt}%
\pgfpathmoveto{\pgfqpoint{2.574339in}{6.548868in}}%
\pgfpathlineto{\pgfqpoint{2.662075in}{6.548868in}}%
\pgfpathlineto{\pgfqpoint{2.662075in}{6.461132in}}%
\pgfpathlineto{\pgfqpoint{2.574339in}{6.461132in}}%
\pgfpathlineto{\pgfqpoint{2.574339in}{6.548868in}}%
\pgfusepath{stroke,fill}%
\end{pgfscope}%
\begin{pgfscope}%
\pgfpathrectangle{\pgfqpoint{0.380943in}{6.110189in}}{\pgfqpoint{4.650000in}{0.614151in}}%
\pgfusepath{clip}%
\pgfsetbuttcap%
\pgfsetroundjoin%
\definecolor{currentfill}{rgb}{0.962414,0.923552,0.722891}%
\pgfsetfillcolor{currentfill}%
\pgfsetlinewidth{0.250937pt}%
\definecolor{currentstroke}{rgb}{1.000000,1.000000,1.000000}%
\pgfsetstrokecolor{currentstroke}%
\pgfsetdash{}{0pt}%
\pgfpathmoveto{\pgfqpoint{2.662075in}{6.548868in}}%
\pgfpathlineto{\pgfqpoint{2.749811in}{6.548868in}}%
\pgfpathlineto{\pgfqpoint{2.749811in}{6.461132in}}%
\pgfpathlineto{\pgfqpoint{2.662075in}{6.461132in}}%
\pgfpathlineto{\pgfqpoint{2.662075in}{6.548868in}}%
\pgfusepath{stroke,fill}%
\end{pgfscope}%
\begin{pgfscope}%
\pgfpathrectangle{\pgfqpoint{0.380943in}{6.110189in}}{\pgfqpoint{4.650000in}{0.614151in}}%
\pgfusepath{clip}%
\pgfsetbuttcap%
\pgfsetroundjoin%
\definecolor{currentfill}{rgb}{0.972549,0.870588,0.692810}%
\pgfsetfillcolor{currentfill}%
\pgfsetlinewidth{0.250937pt}%
\definecolor{currentstroke}{rgb}{1.000000,1.000000,1.000000}%
\pgfsetstrokecolor{currentstroke}%
\pgfsetdash{}{0pt}%
\pgfpathmoveto{\pgfqpoint{2.749811in}{6.548868in}}%
\pgfpathlineto{\pgfqpoint{2.837547in}{6.548868in}}%
\pgfpathlineto{\pgfqpoint{2.837547in}{6.461132in}}%
\pgfpathlineto{\pgfqpoint{2.749811in}{6.461132in}}%
\pgfpathlineto{\pgfqpoint{2.749811in}{6.548868in}}%
\pgfusepath{stroke,fill}%
\end{pgfscope}%
\begin{pgfscope}%
\pgfpathrectangle{\pgfqpoint{0.380943in}{6.110189in}}{\pgfqpoint{4.650000in}{0.614151in}}%
\pgfusepath{clip}%
\pgfsetbuttcap%
\pgfsetroundjoin%
\definecolor{currentfill}{rgb}{0.965444,0.906113,0.711757}%
\pgfsetfillcolor{currentfill}%
\pgfsetlinewidth{0.250937pt}%
\definecolor{currentstroke}{rgb}{1.000000,1.000000,1.000000}%
\pgfsetstrokecolor{currentstroke}%
\pgfsetdash{}{0pt}%
\pgfpathmoveto{\pgfqpoint{2.837547in}{6.548868in}}%
\pgfpathlineto{\pgfqpoint{2.925283in}{6.548868in}}%
\pgfpathlineto{\pgfqpoint{2.925283in}{6.461132in}}%
\pgfpathlineto{\pgfqpoint{2.837547in}{6.461132in}}%
\pgfpathlineto{\pgfqpoint{2.837547in}{6.548868in}}%
\pgfusepath{stroke,fill}%
\end{pgfscope}%
\begin{pgfscope}%
\pgfpathrectangle{\pgfqpoint{0.380943in}{6.110189in}}{\pgfqpoint{4.650000in}{0.614151in}}%
\pgfusepath{clip}%
\pgfsetbuttcap%
\pgfsetroundjoin%
\definecolor{currentfill}{rgb}{0.979654,0.837186,0.669619}%
\pgfsetfillcolor{currentfill}%
\pgfsetlinewidth{0.250937pt}%
\definecolor{currentstroke}{rgb}{1.000000,1.000000,1.000000}%
\pgfsetstrokecolor{currentstroke}%
\pgfsetdash{}{0pt}%
\pgfpathmoveto{\pgfqpoint{2.925283in}{6.548868in}}%
\pgfpathlineto{\pgfqpoint{3.013019in}{6.548868in}}%
\pgfpathlineto{\pgfqpoint{3.013019in}{6.461132in}}%
\pgfpathlineto{\pgfqpoint{2.925283in}{6.461132in}}%
\pgfpathlineto{\pgfqpoint{2.925283in}{6.548868in}}%
\pgfusepath{stroke,fill}%
\end{pgfscope}%
\begin{pgfscope}%
\pgfpathrectangle{\pgfqpoint{0.380943in}{6.110189in}}{\pgfqpoint{4.650000in}{0.614151in}}%
\pgfusepath{clip}%
\pgfsetbuttcap%
\pgfsetroundjoin%
\definecolor{currentfill}{rgb}{0.968166,0.945882,0.748604}%
\pgfsetfillcolor{currentfill}%
\pgfsetlinewidth{0.250937pt}%
\definecolor{currentstroke}{rgb}{1.000000,1.000000,1.000000}%
\pgfsetstrokecolor{currentstroke}%
\pgfsetdash{}{0pt}%
\pgfpathmoveto{\pgfqpoint{3.013019in}{6.548868in}}%
\pgfpathlineto{\pgfqpoint{3.100754in}{6.548868in}}%
\pgfpathlineto{\pgfqpoint{3.100754in}{6.461132in}}%
\pgfpathlineto{\pgfqpoint{3.013019in}{6.461132in}}%
\pgfpathlineto{\pgfqpoint{3.013019in}{6.548868in}}%
\pgfusepath{stroke,fill}%
\end{pgfscope}%
\begin{pgfscope}%
\pgfpathrectangle{\pgfqpoint{0.380943in}{6.110189in}}{\pgfqpoint{4.650000in}{0.614151in}}%
\pgfusepath{clip}%
\pgfsetbuttcap%
\pgfsetroundjoin%
\definecolor{currentfill}{rgb}{1.000000,0.605229,0.530719}%
\pgfsetfillcolor{currentfill}%
\pgfsetlinewidth{0.250937pt}%
\definecolor{currentstroke}{rgb}{1.000000,1.000000,1.000000}%
\pgfsetstrokecolor{currentstroke}%
\pgfsetdash{}{0pt}%
\pgfpathmoveto{\pgfqpoint{3.100754in}{6.548868in}}%
\pgfpathlineto{\pgfqpoint{3.188490in}{6.548868in}}%
\pgfpathlineto{\pgfqpoint{3.188490in}{6.461132in}}%
\pgfpathlineto{\pgfqpoint{3.100754in}{6.461132in}}%
\pgfpathlineto{\pgfqpoint{3.100754in}{6.548868in}}%
\pgfusepath{stroke,fill}%
\end{pgfscope}%
\begin{pgfscope}%
\pgfpathrectangle{\pgfqpoint{0.380943in}{6.110189in}}{\pgfqpoint{4.650000in}{0.614151in}}%
\pgfusepath{clip}%
\pgfsetbuttcap%
\pgfsetroundjoin%
\definecolor{currentfill}{rgb}{1.000000,1.000000,0.870204}%
\pgfsetfillcolor{currentfill}%
\pgfsetlinewidth{0.250937pt}%
\definecolor{currentstroke}{rgb}{1.000000,1.000000,1.000000}%
\pgfsetstrokecolor{currentstroke}%
\pgfsetdash{}{0pt}%
\pgfpathmoveto{\pgfqpoint{3.188490in}{6.548868in}}%
\pgfpathlineto{\pgfqpoint{3.276226in}{6.548868in}}%
\pgfpathlineto{\pgfqpoint{3.276226in}{6.461132in}}%
\pgfpathlineto{\pgfqpoint{3.188490in}{6.461132in}}%
\pgfpathlineto{\pgfqpoint{3.188490in}{6.548868in}}%
\pgfusepath{stroke,fill}%
\end{pgfscope}%
\begin{pgfscope}%
\pgfpathrectangle{\pgfqpoint{0.380943in}{6.110189in}}{\pgfqpoint{4.650000in}{0.614151in}}%
\pgfusepath{clip}%
\pgfsetbuttcap%
\pgfsetroundjoin%
\definecolor{currentfill}{rgb}{0.962414,0.923552,0.722891}%
\pgfsetfillcolor{currentfill}%
\pgfsetlinewidth{0.250937pt}%
\definecolor{currentstroke}{rgb}{1.000000,1.000000,1.000000}%
\pgfsetstrokecolor{currentstroke}%
\pgfsetdash{}{0pt}%
\pgfpathmoveto{\pgfqpoint{3.276226in}{6.548868in}}%
\pgfpathlineto{\pgfqpoint{3.363962in}{6.548868in}}%
\pgfpathlineto{\pgfqpoint{3.363962in}{6.461132in}}%
\pgfpathlineto{\pgfqpoint{3.276226in}{6.461132in}}%
\pgfpathlineto{\pgfqpoint{3.276226in}{6.548868in}}%
\pgfusepath{stroke,fill}%
\end{pgfscope}%
\begin{pgfscope}%
\pgfpathrectangle{\pgfqpoint{0.380943in}{6.110189in}}{\pgfqpoint{4.650000in}{0.614151in}}%
\pgfusepath{clip}%
\pgfsetbuttcap%
\pgfsetroundjoin%
\definecolor{currentfill}{rgb}{0.992326,0.765229,0.614840}%
\pgfsetfillcolor{currentfill}%
\pgfsetlinewidth{0.250937pt}%
\definecolor{currentstroke}{rgb}{1.000000,1.000000,1.000000}%
\pgfsetstrokecolor{currentstroke}%
\pgfsetdash{}{0pt}%
\pgfpathmoveto{\pgfqpoint{3.363962in}{6.548868in}}%
\pgfpathlineto{\pgfqpoint{3.451698in}{6.548868in}}%
\pgfpathlineto{\pgfqpoint{3.451698in}{6.461132in}}%
\pgfpathlineto{\pgfqpoint{3.363962in}{6.461132in}}%
\pgfpathlineto{\pgfqpoint{3.363962in}{6.548868in}}%
\pgfusepath{stroke,fill}%
\end{pgfscope}%
\begin{pgfscope}%
\pgfpathrectangle{\pgfqpoint{0.380943in}{6.110189in}}{\pgfqpoint{4.650000in}{0.614151in}}%
\pgfusepath{clip}%
\pgfsetbuttcap%
\pgfsetroundjoin%
\definecolor{currentfill}{rgb}{0.996571,0.720538,0.589189}%
\pgfsetfillcolor{currentfill}%
\pgfsetlinewidth{0.250937pt}%
\definecolor{currentstroke}{rgb}{1.000000,1.000000,1.000000}%
\pgfsetstrokecolor{currentstroke}%
\pgfsetdash{}{0pt}%
\pgfpathmoveto{\pgfqpoint{3.451698in}{6.548868in}}%
\pgfpathlineto{\pgfqpoint{3.539434in}{6.548868in}}%
\pgfpathlineto{\pgfqpoint{3.539434in}{6.461132in}}%
\pgfpathlineto{\pgfqpoint{3.451698in}{6.461132in}}%
\pgfpathlineto{\pgfqpoint{3.451698in}{6.548868in}}%
\pgfusepath{stroke,fill}%
\end{pgfscope}%
\begin{pgfscope}%
\pgfpathrectangle{\pgfqpoint{0.380943in}{6.110189in}}{\pgfqpoint{4.650000in}{0.614151in}}%
\pgfusepath{clip}%
\pgfsetbuttcap%
\pgfsetroundjoin%
\definecolor{currentfill}{rgb}{0.962414,0.923552,0.722891}%
\pgfsetfillcolor{currentfill}%
\pgfsetlinewidth{0.250937pt}%
\definecolor{currentstroke}{rgb}{1.000000,1.000000,1.000000}%
\pgfsetstrokecolor{currentstroke}%
\pgfsetdash{}{0pt}%
\pgfpathmoveto{\pgfqpoint{3.539434in}{6.548868in}}%
\pgfpathlineto{\pgfqpoint{3.627169in}{6.548868in}}%
\pgfpathlineto{\pgfqpoint{3.627169in}{6.461132in}}%
\pgfpathlineto{\pgfqpoint{3.539434in}{6.461132in}}%
\pgfpathlineto{\pgfqpoint{3.539434in}{6.548868in}}%
\pgfusepath{stroke,fill}%
\end{pgfscope}%
\begin{pgfscope}%
\pgfpathrectangle{\pgfqpoint{0.380943in}{6.110189in}}{\pgfqpoint{4.650000in}{0.614151in}}%
\pgfusepath{clip}%
\pgfsetbuttcap%
\pgfsetroundjoin%
\definecolor{currentfill}{rgb}{0.979654,0.837186,0.669619}%
\pgfsetfillcolor{currentfill}%
\pgfsetlinewidth{0.250937pt}%
\definecolor{currentstroke}{rgb}{1.000000,1.000000,1.000000}%
\pgfsetstrokecolor{currentstroke}%
\pgfsetdash{}{0pt}%
\pgfpathmoveto{\pgfqpoint{3.627169in}{6.548868in}}%
\pgfpathlineto{\pgfqpoint{3.714905in}{6.548868in}}%
\pgfpathlineto{\pgfqpoint{3.714905in}{6.461132in}}%
\pgfpathlineto{\pgfqpoint{3.627169in}{6.461132in}}%
\pgfpathlineto{\pgfqpoint{3.627169in}{6.548868in}}%
\pgfusepath{stroke,fill}%
\end{pgfscope}%
\begin{pgfscope}%
\pgfpathrectangle{\pgfqpoint{0.380943in}{6.110189in}}{\pgfqpoint{4.650000in}{0.614151in}}%
\pgfusepath{clip}%
\pgfsetbuttcap%
\pgfsetroundjoin%
\definecolor{currentfill}{rgb}{0.992326,0.765229,0.614840}%
\pgfsetfillcolor{currentfill}%
\pgfsetlinewidth{0.250937pt}%
\definecolor{currentstroke}{rgb}{1.000000,1.000000,1.000000}%
\pgfsetstrokecolor{currentstroke}%
\pgfsetdash{}{0pt}%
\pgfpathmoveto{\pgfqpoint{3.714905in}{6.548868in}}%
\pgfpathlineto{\pgfqpoint{3.802641in}{6.548868in}}%
\pgfpathlineto{\pgfqpoint{3.802641in}{6.461132in}}%
\pgfpathlineto{\pgfqpoint{3.714905in}{6.461132in}}%
\pgfpathlineto{\pgfqpoint{3.714905in}{6.548868in}}%
\pgfusepath{stroke,fill}%
\end{pgfscope}%
\begin{pgfscope}%
\pgfpathrectangle{\pgfqpoint{0.380943in}{6.110189in}}{\pgfqpoint{4.650000in}{0.614151in}}%
\pgfusepath{clip}%
\pgfsetbuttcap%
\pgfsetroundjoin%
\definecolor{currentfill}{rgb}{0.965444,0.906113,0.711757}%
\pgfsetfillcolor{currentfill}%
\pgfsetlinewidth{0.250937pt}%
\definecolor{currentstroke}{rgb}{1.000000,1.000000,1.000000}%
\pgfsetstrokecolor{currentstroke}%
\pgfsetdash{}{0pt}%
\pgfpathmoveto{\pgfqpoint{3.802641in}{6.548868in}}%
\pgfpathlineto{\pgfqpoint{3.890377in}{6.548868in}}%
\pgfpathlineto{\pgfqpoint{3.890377in}{6.461132in}}%
\pgfpathlineto{\pgfqpoint{3.802641in}{6.461132in}}%
\pgfpathlineto{\pgfqpoint{3.802641in}{6.548868in}}%
\pgfusepath{stroke,fill}%
\end{pgfscope}%
\begin{pgfscope}%
\pgfpathrectangle{\pgfqpoint{0.380943in}{6.110189in}}{\pgfqpoint{4.650000in}{0.614151in}}%
\pgfusepath{clip}%
\pgfsetbuttcap%
\pgfsetroundjoin%
\definecolor{currentfill}{rgb}{0.992326,0.765229,0.614840}%
\pgfsetfillcolor{currentfill}%
\pgfsetlinewidth{0.250937pt}%
\definecolor{currentstroke}{rgb}{1.000000,1.000000,1.000000}%
\pgfsetstrokecolor{currentstroke}%
\pgfsetdash{}{0pt}%
\pgfpathmoveto{\pgfqpoint{3.890377in}{6.548868in}}%
\pgfpathlineto{\pgfqpoint{3.978113in}{6.548868in}}%
\pgfpathlineto{\pgfqpoint{3.978113in}{6.461132in}}%
\pgfpathlineto{\pgfqpoint{3.890377in}{6.461132in}}%
\pgfpathlineto{\pgfqpoint{3.890377in}{6.548868in}}%
\pgfusepath{stroke,fill}%
\end{pgfscope}%
\begin{pgfscope}%
\pgfpathrectangle{\pgfqpoint{0.380943in}{6.110189in}}{\pgfqpoint{4.650000in}{0.614151in}}%
\pgfusepath{clip}%
\pgfsetbuttcap%
\pgfsetroundjoin%
\definecolor{currentfill}{rgb}{0.998939,0.658962,0.556032}%
\pgfsetfillcolor{currentfill}%
\pgfsetlinewidth{0.250937pt}%
\definecolor{currentstroke}{rgb}{1.000000,1.000000,1.000000}%
\pgfsetstrokecolor{currentstroke}%
\pgfsetdash{}{0pt}%
\pgfpathmoveto{\pgfqpoint{3.978113in}{6.548868in}}%
\pgfpathlineto{\pgfqpoint{4.065849in}{6.548868in}}%
\pgfpathlineto{\pgfqpoint{4.065849in}{6.461132in}}%
\pgfpathlineto{\pgfqpoint{3.978113in}{6.461132in}}%
\pgfpathlineto{\pgfqpoint{3.978113in}{6.548868in}}%
\pgfusepath{stroke,fill}%
\end{pgfscope}%
\begin{pgfscope}%
\pgfpathrectangle{\pgfqpoint{0.380943in}{6.110189in}}{\pgfqpoint{4.650000in}{0.614151in}}%
\pgfusepath{clip}%
\pgfsetbuttcap%
\pgfsetroundjoin%
\definecolor{currentfill}{rgb}{0.986759,0.806398,0.641200}%
\pgfsetfillcolor{currentfill}%
\pgfsetlinewidth{0.250937pt}%
\definecolor{currentstroke}{rgb}{1.000000,1.000000,1.000000}%
\pgfsetstrokecolor{currentstroke}%
\pgfsetdash{}{0pt}%
\pgfpathmoveto{\pgfqpoint{4.065849in}{6.548868in}}%
\pgfpathlineto{\pgfqpoint{4.153585in}{6.548868in}}%
\pgfpathlineto{\pgfqpoint{4.153585in}{6.461132in}}%
\pgfpathlineto{\pgfqpoint{4.065849in}{6.461132in}}%
\pgfpathlineto{\pgfqpoint{4.065849in}{6.548868in}}%
\pgfusepath{stroke,fill}%
\end{pgfscope}%
\begin{pgfscope}%
\pgfpathrectangle{\pgfqpoint{0.380943in}{6.110189in}}{\pgfqpoint{4.650000in}{0.614151in}}%
\pgfusepath{clip}%
\pgfsetbuttcap%
\pgfsetroundjoin%
\definecolor{currentfill}{rgb}{0.986759,0.806398,0.641200}%
\pgfsetfillcolor{currentfill}%
\pgfsetlinewidth{0.250937pt}%
\definecolor{currentstroke}{rgb}{1.000000,1.000000,1.000000}%
\pgfsetstrokecolor{currentstroke}%
\pgfsetdash{}{0pt}%
\pgfpathmoveto{\pgfqpoint{4.153585in}{6.548868in}}%
\pgfpathlineto{\pgfqpoint{4.241320in}{6.548868in}}%
\pgfpathlineto{\pgfqpoint{4.241320in}{6.461132in}}%
\pgfpathlineto{\pgfqpoint{4.153585in}{6.461132in}}%
\pgfpathlineto{\pgfqpoint{4.153585in}{6.548868in}}%
\pgfusepath{stroke,fill}%
\end{pgfscope}%
\begin{pgfscope}%
\pgfpathrectangle{\pgfqpoint{0.380943in}{6.110189in}}{\pgfqpoint{4.650000in}{0.614151in}}%
\pgfusepath{clip}%
\pgfsetbuttcap%
\pgfsetroundjoin%
\definecolor{currentfill}{rgb}{0.992326,0.765229,0.614840}%
\pgfsetfillcolor{currentfill}%
\pgfsetlinewidth{0.250937pt}%
\definecolor{currentstroke}{rgb}{1.000000,1.000000,1.000000}%
\pgfsetstrokecolor{currentstroke}%
\pgfsetdash{}{0pt}%
\pgfpathmoveto{\pgfqpoint{4.241320in}{6.548868in}}%
\pgfpathlineto{\pgfqpoint{4.329056in}{6.548868in}}%
\pgfpathlineto{\pgfqpoint{4.329056in}{6.461132in}}%
\pgfpathlineto{\pgfqpoint{4.241320in}{6.461132in}}%
\pgfpathlineto{\pgfqpoint{4.241320in}{6.548868in}}%
\pgfusepath{stroke,fill}%
\end{pgfscope}%
\begin{pgfscope}%
\pgfpathrectangle{\pgfqpoint{0.380943in}{6.110189in}}{\pgfqpoint{4.650000in}{0.614151in}}%
\pgfusepath{clip}%
\pgfsetbuttcap%
\pgfsetroundjoin%
\definecolor{currentfill}{rgb}{0.979654,0.837186,0.669619}%
\pgfsetfillcolor{currentfill}%
\pgfsetlinewidth{0.250937pt}%
\definecolor{currentstroke}{rgb}{1.000000,1.000000,1.000000}%
\pgfsetstrokecolor{currentstroke}%
\pgfsetdash{}{0pt}%
\pgfpathmoveto{\pgfqpoint{4.329056in}{6.548868in}}%
\pgfpathlineto{\pgfqpoint{4.416792in}{6.548868in}}%
\pgfpathlineto{\pgfqpoint{4.416792in}{6.461132in}}%
\pgfpathlineto{\pgfqpoint{4.329056in}{6.461132in}}%
\pgfpathlineto{\pgfqpoint{4.329056in}{6.548868in}}%
\pgfusepath{stroke,fill}%
\end{pgfscope}%
\begin{pgfscope}%
\pgfpathrectangle{\pgfqpoint{0.380943in}{6.110189in}}{\pgfqpoint{4.650000in}{0.614151in}}%
\pgfusepath{clip}%
\pgfsetbuttcap%
\pgfsetroundjoin%
\definecolor{currentfill}{rgb}{0.979654,0.837186,0.669619}%
\pgfsetfillcolor{currentfill}%
\pgfsetlinewidth{0.250937pt}%
\definecolor{currentstroke}{rgb}{1.000000,1.000000,1.000000}%
\pgfsetstrokecolor{currentstroke}%
\pgfsetdash{}{0pt}%
\pgfpathmoveto{\pgfqpoint{4.416792in}{6.548868in}}%
\pgfpathlineto{\pgfqpoint{4.504528in}{6.548868in}}%
\pgfpathlineto{\pgfqpoint{4.504528in}{6.461132in}}%
\pgfpathlineto{\pgfqpoint{4.416792in}{6.461132in}}%
\pgfpathlineto{\pgfqpoint{4.416792in}{6.548868in}}%
\pgfusepath{stroke,fill}%
\end{pgfscope}%
\begin{pgfscope}%
\pgfpathrectangle{\pgfqpoint{0.380943in}{6.110189in}}{\pgfqpoint{4.650000in}{0.614151in}}%
\pgfusepath{clip}%
\pgfsetbuttcap%
\pgfsetroundjoin%
\definecolor{currentfill}{rgb}{0.979654,0.837186,0.669619}%
\pgfsetfillcolor{currentfill}%
\pgfsetlinewidth{0.250937pt}%
\definecolor{currentstroke}{rgb}{1.000000,1.000000,1.000000}%
\pgfsetstrokecolor{currentstroke}%
\pgfsetdash{}{0pt}%
\pgfpathmoveto{\pgfqpoint{4.504528in}{6.548868in}}%
\pgfpathlineto{\pgfqpoint{4.592264in}{6.548868in}}%
\pgfpathlineto{\pgfqpoint{4.592264in}{6.461132in}}%
\pgfpathlineto{\pgfqpoint{4.504528in}{6.461132in}}%
\pgfpathlineto{\pgfqpoint{4.504528in}{6.548868in}}%
\pgfusepath{stroke,fill}%
\end{pgfscope}%
\begin{pgfscope}%
\pgfpathrectangle{\pgfqpoint{0.380943in}{6.110189in}}{\pgfqpoint{4.650000in}{0.614151in}}%
\pgfusepath{clip}%
\pgfsetbuttcap%
\pgfsetroundjoin%
\definecolor{currentfill}{rgb}{0.962414,0.923552,0.722891}%
\pgfsetfillcolor{currentfill}%
\pgfsetlinewidth{0.250937pt}%
\definecolor{currentstroke}{rgb}{1.000000,1.000000,1.000000}%
\pgfsetstrokecolor{currentstroke}%
\pgfsetdash{}{0pt}%
\pgfpathmoveto{\pgfqpoint{4.592264in}{6.548868in}}%
\pgfpathlineto{\pgfqpoint{4.680000in}{6.548868in}}%
\pgfpathlineto{\pgfqpoint{4.680000in}{6.461132in}}%
\pgfpathlineto{\pgfqpoint{4.592264in}{6.461132in}}%
\pgfpathlineto{\pgfqpoint{4.592264in}{6.548868in}}%
\pgfusepath{stroke,fill}%
\end{pgfscope}%
\begin{pgfscope}%
\pgfpathrectangle{\pgfqpoint{0.380943in}{6.110189in}}{\pgfqpoint{4.650000in}{0.614151in}}%
\pgfusepath{clip}%
\pgfsetbuttcap%
\pgfsetroundjoin%
\definecolor{currentfill}{rgb}{0.965444,0.906113,0.711757}%
\pgfsetfillcolor{currentfill}%
\pgfsetlinewidth{0.250937pt}%
\definecolor{currentstroke}{rgb}{1.000000,1.000000,1.000000}%
\pgfsetstrokecolor{currentstroke}%
\pgfsetdash{}{0pt}%
\pgfpathmoveto{\pgfqpoint{4.680000in}{6.548868in}}%
\pgfpathlineto{\pgfqpoint{4.767736in}{6.548868in}}%
\pgfpathlineto{\pgfqpoint{4.767736in}{6.461132in}}%
\pgfpathlineto{\pgfqpoint{4.680000in}{6.461132in}}%
\pgfpathlineto{\pgfqpoint{4.680000in}{6.548868in}}%
\pgfusepath{stroke,fill}%
\end{pgfscope}%
\begin{pgfscope}%
\pgfpathrectangle{\pgfqpoint{0.380943in}{6.110189in}}{\pgfqpoint{4.650000in}{0.614151in}}%
\pgfusepath{clip}%
\pgfsetbuttcap%
\pgfsetroundjoin%
\definecolor{currentfill}{rgb}{1.000000,0.605229,0.530719}%
\pgfsetfillcolor{currentfill}%
\pgfsetlinewidth{0.250937pt}%
\definecolor{currentstroke}{rgb}{1.000000,1.000000,1.000000}%
\pgfsetstrokecolor{currentstroke}%
\pgfsetdash{}{0pt}%
\pgfpathmoveto{\pgfqpoint{4.767736in}{6.548868in}}%
\pgfpathlineto{\pgfqpoint{4.855471in}{6.548868in}}%
\pgfpathlineto{\pgfqpoint{4.855471in}{6.461132in}}%
\pgfpathlineto{\pgfqpoint{4.767736in}{6.461132in}}%
\pgfpathlineto{\pgfqpoint{4.767736in}{6.548868in}}%
\pgfusepath{stroke,fill}%
\end{pgfscope}%
\begin{pgfscope}%
\pgfpathrectangle{\pgfqpoint{0.380943in}{6.110189in}}{\pgfqpoint{4.650000in}{0.614151in}}%
\pgfusepath{clip}%
\pgfsetbuttcap%
\pgfsetroundjoin%
\definecolor{currentfill}{rgb}{0.986759,0.806398,0.641200}%
\pgfsetfillcolor{currentfill}%
\pgfsetlinewidth{0.250937pt}%
\definecolor{currentstroke}{rgb}{1.000000,1.000000,1.000000}%
\pgfsetstrokecolor{currentstroke}%
\pgfsetdash{}{0pt}%
\pgfpathmoveto{\pgfqpoint{4.855471in}{6.548868in}}%
\pgfpathlineto{\pgfqpoint{4.943207in}{6.548868in}}%
\pgfpathlineto{\pgfqpoint{4.943207in}{6.461132in}}%
\pgfpathlineto{\pgfqpoint{4.855471in}{6.461132in}}%
\pgfpathlineto{\pgfqpoint{4.855471in}{6.548868in}}%
\pgfusepath{stroke,fill}%
\end{pgfscope}%
\begin{pgfscope}%
\pgfpathrectangle{\pgfqpoint{0.380943in}{6.110189in}}{\pgfqpoint{4.650000in}{0.614151in}}%
\pgfusepath{clip}%
\pgfsetbuttcap%
\pgfsetroundjoin%
\pgfsetlinewidth{0.250937pt}%
\definecolor{currentstroke}{rgb}{1.000000,1.000000,1.000000}%
\pgfsetstrokecolor{currentstroke}%
\pgfsetdash{}{0pt}%
\pgfpathmoveto{\pgfqpoint{4.943207in}{6.548868in}}%
\pgfpathlineto{\pgfqpoint{5.030943in}{6.548868in}}%
\pgfpathlineto{\pgfqpoint{5.030943in}{6.461132in}}%
\pgfpathlineto{\pgfqpoint{4.943207in}{6.461132in}}%
\pgfpathlineto{\pgfqpoint{4.943207in}{6.548868in}}%
\pgfusepath{stroke}%
\end{pgfscope}%
\begin{pgfscope}%
\pgfpathrectangle{\pgfqpoint{0.380943in}{6.110189in}}{\pgfqpoint{4.650000in}{0.614151in}}%
\pgfusepath{clip}%
\pgfsetbuttcap%
\pgfsetroundjoin%
\definecolor{currentfill}{rgb}{0.986759,0.806398,0.641200}%
\pgfsetfillcolor{currentfill}%
\pgfsetlinewidth{0.250937pt}%
\definecolor{currentstroke}{rgb}{1.000000,1.000000,1.000000}%
\pgfsetstrokecolor{currentstroke}%
\pgfsetdash{}{0pt}%
\pgfpathmoveto{\pgfqpoint{0.380943in}{6.461132in}}%
\pgfpathlineto{\pgfqpoint{0.468679in}{6.461132in}}%
\pgfpathlineto{\pgfqpoint{0.468679in}{6.373396in}}%
\pgfpathlineto{\pgfqpoint{0.380943in}{6.373396in}}%
\pgfpathlineto{\pgfqpoint{0.380943in}{6.461132in}}%
\pgfusepath{stroke,fill}%
\end{pgfscope}%
\begin{pgfscope}%
\pgfpathrectangle{\pgfqpoint{0.380943in}{6.110189in}}{\pgfqpoint{4.650000in}{0.614151in}}%
\pgfusepath{clip}%
\pgfsetbuttcap%
\pgfsetroundjoin%
\definecolor{currentfill}{rgb}{0.981546,0.459977,0.459977}%
\pgfsetfillcolor{currentfill}%
\pgfsetlinewidth{0.250937pt}%
\definecolor{currentstroke}{rgb}{1.000000,1.000000,1.000000}%
\pgfsetstrokecolor{currentstroke}%
\pgfsetdash{}{0pt}%
\pgfpathmoveto{\pgfqpoint{0.468679in}{6.461132in}}%
\pgfpathlineto{\pgfqpoint{0.556415in}{6.461132in}}%
\pgfpathlineto{\pgfqpoint{0.556415in}{6.373396in}}%
\pgfpathlineto{\pgfqpoint{0.468679in}{6.373396in}}%
\pgfpathlineto{\pgfqpoint{0.468679in}{6.461132in}}%
\pgfusepath{stroke,fill}%
\end{pgfscope}%
\begin{pgfscope}%
\pgfpathrectangle{\pgfqpoint{0.380943in}{6.110189in}}{\pgfqpoint{4.650000in}{0.614151in}}%
\pgfusepath{clip}%
\pgfsetbuttcap%
\pgfsetroundjoin%
\definecolor{currentfill}{rgb}{0.979654,0.837186,0.669619}%
\pgfsetfillcolor{currentfill}%
\pgfsetlinewidth{0.250937pt}%
\definecolor{currentstroke}{rgb}{1.000000,1.000000,1.000000}%
\pgfsetstrokecolor{currentstroke}%
\pgfsetdash{}{0pt}%
\pgfpathmoveto{\pgfqpoint{0.556415in}{6.461132in}}%
\pgfpathlineto{\pgfqpoint{0.644151in}{6.461132in}}%
\pgfpathlineto{\pgfqpoint{0.644151in}{6.373396in}}%
\pgfpathlineto{\pgfqpoint{0.556415in}{6.373396in}}%
\pgfpathlineto{\pgfqpoint{0.556415in}{6.461132in}}%
\pgfusepath{stroke,fill}%
\end{pgfscope}%
\begin{pgfscope}%
\pgfpathrectangle{\pgfqpoint{0.380943in}{6.110189in}}{\pgfqpoint{4.650000in}{0.614151in}}%
\pgfusepath{clip}%
\pgfsetbuttcap%
\pgfsetroundjoin%
\definecolor{currentfill}{rgb}{0.979654,0.837186,0.669619}%
\pgfsetfillcolor{currentfill}%
\pgfsetlinewidth{0.250937pt}%
\definecolor{currentstroke}{rgb}{1.000000,1.000000,1.000000}%
\pgfsetstrokecolor{currentstroke}%
\pgfsetdash{}{0pt}%
\pgfpathmoveto{\pgfqpoint{0.644151in}{6.461132in}}%
\pgfpathlineto{\pgfqpoint{0.731886in}{6.461132in}}%
\pgfpathlineto{\pgfqpoint{0.731886in}{6.373396in}}%
\pgfpathlineto{\pgfqpoint{0.644151in}{6.373396in}}%
\pgfpathlineto{\pgfqpoint{0.644151in}{6.461132in}}%
\pgfusepath{stroke,fill}%
\end{pgfscope}%
\begin{pgfscope}%
\pgfpathrectangle{\pgfqpoint{0.380943in}{6.110189in}}{\pgfqpoint{4.650000in}{0.614151in}}%
\pgfusepath{clip}%
\pgfsetbuttcap%
\pgfsetroundjoin%
\definecolor{currentfill}{rgb}{0.979654,0.837186,0.669619}%
\pgfsetfillcolor{currentfill}%
\pgfsetlinewidth{0.250937pt}%
\definecolor{currentstroke}{rgb}{1.000000,1.000000,1.000000}%
\pgfsetstrokecolor{currentstroke}%
\pgfsetdash{}{0pt}%
\pgfpathmoveto{\pgfqpoint{0.731886in}{6.461132in}}%
\pgfpathlineto{\pgfqpoint{0.819622in}{6.461132in}}%
\pgfpathlineto{\pgfqpoint{0.819622in}{6.373396in}}%
\pgfpathlineto{\pgfqpoint{0.731886in}{6.373396in}}%
\pgfpathlineto{\pgfqpoint{0.731886in}{6.461132in}}%
\pgfusepath{stroke,fill}%
\end{pgfscope}%
\begin{pgfscope}%
\pgfpathrectangle{\pgfqpoint{0.380943in}{6.110189in}}{\pgfqpoint{4.650000in}{0.614151in}}%
\pgfusepath{clip}%
\pgfsetbuttcap%
\pgfsetroundjoin%
\definecolor{currentfill}{rgb}{0.972549,0.870588,0.692810}%
\pgfsetfillcolor{currentfill}%
\pgfsetlinewidth{0.250937pt}%
\definecolor{currentstroke}{rgb}{1.000000,1.000000,1.000000}%
\pgfsetstrokecolor{currentstroke}%
\pgfsetdash{}{0pt}%
\pgfpathmoveto{\pgfqpoint{0.819622in}{6.461132in}}%
\pgfpathlineto{\pgfqpoint{0.907358in}{6.461132in}}%
\pgfpathlineto{\pgfqpoint{0.907358in}{6.373396in}}%
\pgfpathlineto{\pgfqpoint{0.819622in}{6.373396in}}%
\pgfpathlineto{\pgfqpoint{0.819622in}{6.461132in}}%
\pgfusepath{stroke,fill}%
\end{pgfscope}%
\begin{pgfscope}%
\pgfpathrectangle{\pgfqpoint{0.380943in}{6.110189in}}{\pgfqpoint{4.650000in}{0.614151in}}%
\pgfusepath{clip}%
\pgfsetbuttcap%
\pgfsetroundjoin%
\definecolor{currentfill}{rgb}{0.986759,0.806398,0.641200}%
\pgfsetfillcolor{currentfill}%
\pgfsetlinewidth{0.250937pt}%
\definecolor{currentstroke}{rgb}{1.000000,1.000000,1.000000}%
\pgfsetstrokecolor{currentstroke}%
\pgfsetdash{}{0pt}%
\pgfpathmoveto{\pgfqpoint{0.907358in}{6.461132in}}%
\pgfpathlineto{\pgfqpoint{0.995094in}{6.461132in}}%
\pgfpathlineto{\pgfqpoint{0.995094in}{6.373396in}}%
\pgfpathlineto{\pgfqpoint{0.907358in}{6.373396in}}%
\pgfpathlineto{\pgfqpoint{0.907358in}{6.461132in}}%
\pgfusepath{stroke,fill}%
\end{pgfscope}%
\begin{pgfscope}%
\pgfpathrectangle{\pgfqpoint{0.380943in}{6.110189in}}{\pgfqpoint{4.650000in}{0.614151in}}%
\pgfusepath{clip}%
\pgfsetbuttcap%
\pgfsetroundjoin%
\definecolor{currentfill}{rgb}{0.986759,0.806398,0.641200}%
\pgfsetfillcolor{currentfill}%
\pgfsetlinewidth{0.250937pt}%
\definecolor{currentstroke}{rgb}{1.000000,1.000000,1.000000}%
\pgfsetstrokecolor{currentstroke}%
\pgfsetdash{}{0pt}%
\pgfpathmoveto{\pgfqpoint{0.995094in}{6.461132in}}%
\pgfpathlineto{\pgfqpoint{1.082830in}{6.461132in}}%
\pgfpathlineto{\pgfqpoint{1.082830in}{6.373396in}}%
\pgfpathlineto{\pgfqpoint{0.995094in}{6.373396in}}%
\pgfpathlineto{\pgfqpoint{0.995094in}{6.461132in}}%
\pgfusepath{stroke,fill}%
\end{pgfscope}%
\begin{pgfscope}%
\pgfpathrectangle{\pgfqpoint{0.380943in}{6.110189in}}{\pgfqpoint{4.650000in}{0.614151in}}%
\pgfusepath{clip}%
\pgfsetbuttcap%
\pgfsetroundjoin%
\definecolor{currentfill}{rgb}{0.981546,0.459977,0.459977}%
\pgfsetfillcolor{currentfill}%
\pgfsetlinewidth{0.250937pt}%
\definecolor{currentstroke}{rgb}{1.000000,1.000000,1.000000}%
\pgfsetstrokecolor{currentstroke}%
\pgfsetdash{}{0pt}%
\pgfpathmoveto{\pgfqpoint{1.082830in}{6.461132in}}%
\pgfpathlineto{\pgfqpoint{1.170566in}{6.461132in}}%
\pgfpathlineto{\pgfqpoint{1.170566in}{6.373396in}}%
\pgfpathlineto{\pgfqpoint{1.082830in}{6.373396in}}%
\pgfpathlineto{\pgfqpoint{1.082830in}{6.461132in}}%
\pgfusepath{stroke,fill}%
\end{pgfscope}%
\begin{pgfscope}%
\pgfpathrectangle{\pgfqpoint{0.380943in}{6.110189in}}{\pgfqpoint{4.650000in}{0.614151in}}%
\pgfusepath{clip}%
\pgfsetbuttcap%
\pgfsetroundjoin%
\definecolor{currentfill}{rgb}{0.979654,0.837186,0.669619}%
\pgfsetfillcolor{currentfill}%
\pgfsetlinewidth{0.250937pt}%
\definecolor{currentstroke}{rgb}{1.000000,1.000000,1.000000}%
\pgfsetstrokecolor{currentstroke}%
\pgfsetdash{}{0pt}%
\pgfpathmoveto{\pgfqpoint{1.170566in}{6.461132in}}%
\pgfpathlineto{\pgfqpoint{1.258302in}{6.461132in}}%
\pgfpathlineto{\pgfqpoint{1.258302in}{6.373396in}}%
\pgfpathlineto{\pgfqpoint{1.170566in}{6.373396in}}%
\pgfpathlineto{\pgfqpoint{1.170566in}{6.461132in}}%
\pgfusepath{stroke,fill}%
\end{pgfscope}%
\begin{pgfscope}%
\pgfpathrectangle{\pgfqpoint{0.380943in}{6.110189in}}{\pgfqpoint{4.650000in}{0.614151in}}%
\pgfusepath{clip}%
\pgfsetbuttcap%
\pgfsetroundjoin%
\definecolor{currentfill}{rgb}{1.000000,0.557862,0.511772}%
\pgfsetfillcolor{currentfill}%
\pgfsetlinewidth{0.250937pt}%
\definecolor{currentstroke}{rgb}{1.000000,1.000000,1.000000}%
\pgfsetstrokecolor{currentstroke}%
\pgfsetdash{}{0pt}%
\pgfpathmoveto{\pgfqpoint{1.258302in}{6.461132in}}%
\pgfpathlineto{\pgfqpoint{1.346037in}{6.461132in}}%
\pgfpathlineto{\pgfqpoint{1.346037in}{6.373396in}}%
\pgfpathlineto{\pgfqpoint{1.258302in}{6.373396in}}%
\pgfpathlineto{\pgfqpoint{1.258302in}{6.461132in}}%
\pgfusepath{stroke,fill}%
\end{pgfscope}%
\begin{pgfscope}%
\pgfpathrectangle{\pgfqpoint{0.380943in}{6.110189in}}{\pgfqpoint{4.650000in}{0.614151in}}%
\pgfusepath{clip}%
\pgfsetbuttcap%
\pgfsetroundjoin%
\definecolor{currentfill}{rgb}{0.998939,0.658962,0.556032}%
\pgfsetfillcolor{currentfill}%
\pgfsetlinewidth{0.250937pt}%
\definecolor{currentstroke}{rgb}{1.000000,1.000000,1.000000}%
\pgfsetstrokecolor{currentstroke}%
\pgfsetdash{}{0pt}%
\pgfpathmoveto{\pgfqpoint{1.346037in}{6.461132in}}%
\pgfpathlineto{\pgfqpoint{1.433773in}{6.461132in}}%
\pgfpathlineto{\pgfqpoint{1.433773in}{6.373396in}}%
\pgfpathlineto{\pgfqpoint{1.346037in}{6.373396in}}%
\pgfpathlineto{\pgfqpoint{1.346037in}{6.461132in}}%
\pgfusepath{stroke,fill}%
\end{pgfscope}%
\begin{pgfscope}%
\pgfpathrectangle{\pgfqpoint{0.380943in}{6.110189in}}{\pgfqpoint{4.650000in}{0.614151in}}%
\pgfusepath{clip}%
\pgfsetbuttcap%
\pgfsetroundjoin%
\definecolor{currentfill}{rgb}{0.986759,0.806398,0.641200}%
\pgfsetfillcolor{currentfill}%
\pgfsetlinewidth{0.250937pt}%
\definecolor{currentstroke}{rgb}{1.000000,1.000000,1.000000}%
\pgfsetstrokecolor{currentstroke}%
\pgfsetdash{}{0pt}%
\pgfpathmoveto{\pgfqpoint{1.433773in}{6.461132in}}%
\pgfpathlineto{\pgfqpoint{1.521509in}{6.461132in}}%
\pgfpathlineto{\pgfqpoint{1.521509in}{6.373396in}}%
\pgfpathlineto{\pgfqpoint{1.433773in}{6.373396in}}%
\pgfpathlineto{\pgfqpoint{1.433773in}{6.461132in}}%
\pgfusepath{stroke,fill}%
\end{pgfscope}%
\begin{pgfscope}%
\pgfpathrectangle{\pgfqpoint{0.380943in}{6.110189in}}{\pgfqpoint{4.650000in}{0.614151in}}%
\pgfusepath{clip}%
\pgfsetbuttcap%
\pgfsetroundjoin%
\definecolor{currentfill}{rgb}{0.992326,0.765229,0.614840}%
\pgfsetfillcolor{currentfill}%
\pgfsetlinewidth{0.250937pt}%
\definecolor{currentstroke}{rgb}{1.000000,1.000000,1.000000}%
\pgfsetstrokecolor{currentstroke}%
\pgfsetdash{}{0pt}%
\pgfpathmoveto{\pgfqpoint{1.521509in}{6.461132in}}%
\pgfpathlineto{\pgfqpoint{1.609245in}{6.461132in}}%
\pgfpathlineto{\pgfqpoint{1.609245in}{6.373396in}}%
\pgfpathlineto{\pgfqpoint{1.521509in}{6.373396in}}%
\pgfpathlineto{\pgfqpoint{1.521509in}{6.461132in}}%
\pgfusepath{stroke,fill}%
\end{pgfscope}%
\begin{pgfscope}%
\pgfpathrectangle{\pgfqpoint{0.380943in}{6.110189in}}{\pgfqpoint{4.650000in}{0.614151in}}%
\pgfusepath{clip}%
\pgfsetbuttcap%
\pgfsetroundjoin%
\definecolor{currentfill}{rgb}{0.800000,0.278431,0.278431}%
\pgfsetfillcolor{currentfill}%
\pgfsetlinewidth{0.250937pt}%
\definecolor{currentstroke}{rgb}{1.000000,1.000000,1.000000}%
\pgfsetstrokecolor{currentstroke}%
\pgfsetdash{}{0pt}%
\pgfpathmoveto{\pgfqpoint{1.609245in}{6.461132in}}%
\pgfpathlineto{\pgfqpoint{1.696981in}{6.461132in}}%
\pgfpathlineto{\pgfqpoint{1.696981in}{6.373396in}}%
\pgfpathlineto{\pgfqpoint{1.609245in}{6.373396in}}%
\pgfpathlineto{\pgfqpoint{1.609245in}{6.461132in}}%
\pgfusepath{stroke,fill}%
\end{pgfscope}%
\begin{pgfscope}%
\pgfpathrectangle{\pgfqpoint{0.380943in}{6.110189in}}{\pgfqpoint{4.650000in}{0.614151in}}%
\pgfusepath{clip}%
\pgfsetbuttcap%
\pgfsetroundjoin%
\definecolor{currentfill}{rgb}{0.992326,0.765229,0.614840}%
\pgfsetfillcolor{currentfill}%
\pgfsetlinewidth{0.250937pt}%
\definecolor{currentstroke}{rgb}{1.000000,1.000000,1.000000}%
\pgfsetstrokecolor{currentstroke}%
\pgfsetdash{}{0pt}%
\pgfpathmoveto{\pgfqpoint{1.696981in}{6.461132in}}%
\pgfpathlineto{\pgfqpoint{1.784717in}{6.461132in}}%
\pgfpathlineto{\pgfqpoint{1.784717in}{6.373396in}}%
\pgfpathlineto{\pgfqpoint{1.696981in}{6.373396in}}%
\pgfpathlineto{\pgfqpoint{1.696981in}{6.461132in}}%
\pgfusepath{stroke,fill}%
\end{pgfscope}%
\begin{pgfscope}%
\pgfpathrectangle{\pgfqpoint{0.380943in}{6.110189in}}{\pgfqpoint{4.650000in}{0.614151in}}%
\pgfusepath{clip}%
\pgfsetbuttcap%
\pgfsetroundjoin%
\definecolor{currentfill}{rgb}{0.992326,0.765229,0.614840}%
\pgfsetfillcolor{currentfill}%
\pgfsetlinewidth{0.250937pt}%
\definecolor{currentstroke}{rgb}{1.000000,1.000000,1.000000}%
\pgfsetstrokecolor{currentstroke}%
\pgfsetdash{}{0pt}%
\pgfpathmoveto{\pgfqpoint{1.784717in}{6.461132in}}%
\pgfpathlineto{\pgfqpoint{1.872452in}{6.461132in}}%
\pgfpathlineto{\pgfqpoint{1.872452in}{6.373396in}}%
\pgfpathlineto{\pgfqpoint{1.784717in}{6.373396in}}%
\pgfpathlineto{\pgfqpoint{1.784717in}{6.461132in}}%
\pgfusepath{stroke,fill}%
\end{pgfscope}%
\begin{pgfscope}%
\pgfpathrectangle{\pgfqpoint{0.380943in}{6.110189in}}{\pgfqpoint{4.650000in}{0.614151in}}%
\pgfusepath{clip}%
\pgfsetbuttcap%
\pgfsetroundjoin%
\definecolor{currentfill}{rgb}{1.000000,0.605229,0.530719}%
\pgfsetfillcolor{currentfill}%
\pgfsetlinewidth{0.250937pt}%
\definecolor{currentstroke}{rgb}{1.000000,1.000000,1.000000}%
\pgfsetstrokecolor{currentstroke}%
\pgfsetdash{}{0pt}%
\pgfpathmoveto{\pgfqpoint{1.872452in}{6.461132in}}%
\pgfpathlineto{\pgfqpoint{1.960188in}{6.461132in}}%
\pgfpathlineto{\pgfqpoint{1.960188in}{6.373396in}}%
\pgfpathlineto{\pgfqpoint{1.872452in}{6.373396in}}%
\pgfpathlineto{\pgfqpoint{1.872452in}{6.461132in}}%
\pgfusepath{stroke,fill}%
\end{pgfscope}%
\begin{pgfscope}%
\pgfpathrectangle{\pgfqpoint{0.380943in}{6.110189in}}{\pgfqpoint{4.650000in}{0.614151in}}%
\pgfusepath{clip}%
\pgfsetbuttcap%
\pgfsetroundjoin%
\definecolor{currentfill}{rgb}{1.000000,1.000000,0.929412}%
\pgfsetfillcolor{currentfill}%
\pgfsetlinewidth{0.250937pt}%
\definecolor{currentstroke}{rgb}{1.000000,1.000000,1.000000}%
\pgfsetstrokecolor{currentstroke}%
\pgfsetdash{}{0pt}%
\pgfpathmoveto{\pgfqpoint{1.960188in}{6.461132in}}%
\pgfpathlineto{\pgfqpoint{2.047924in}{6.461132in}}%
\pgfpathlineto{\pgfqpoint{2.047924in}{6.373396in}}%
\pgfpathlineto{\pgfqpoint{1.960188in}{6.373396in}}%
\pgfpathlineto{\pgfqpoint{1.960188in}{6.461132in}}%
\pgfusepath{stroke,fill}%
\end{pgfscope}%
\begin{pgfscope}%
\pgfpathrectangle{\pgfqpoint{0.380943in}{6.110189in}}{\pgfqpoint{4.650000in}{0.614151in}}%
\pgfusepath{clip}%
\pgfsetbuttcap%
\pgfsetroundjoin%
\definecolor{currentfill}{rgb}{0.965444,0.906113,0.711757}%
\pgfsetfillcolor{currentfill}%
\pgfsetlinewidth{0.250937pt}%
\definecolor{currentstroke}{rgb}{1.000000,1.000000,1.000000}%
\pgfsetstrokecolor{currentstroke}%
\pgfsetdash{}{0pt}%
\pgfpathmoveto{\pgfqpoint{2.047924in}{6.461132in}}%
\pgfpathlineto{\pgfqpoint{2.135660in}{6.461132in}}%
\pgfpathlineto{\pgfqpoint{2.135660in}{6.373396in}}%
\pgfpathlineto{\pgfqpoint{2.047924in}{6.373396in}}%
\pgfpathlineto{\pgfqpoint{2.047924in}{6.461132in}}%
\pgfusepath{stroke,fill}%
\end{pgfscope}%
\begin{pgfscope}%
\pgfpathrectangle{\pgfqpoint{0.380943in}{6.110189in}}{\pgfqpoint{4.650000in}{0.614151in}}%
\pgfusepath{clip}%
\pgfsetbuttcap%
\pgfsetroundjoin%
\definecolor{currentfill}{rgb}{0.996571,0.720538,0.589189}%
\pgfsetfillcolor{currentfill}%
\pgfsetlinewidth{0.250937pt}%
\definecolor{currentstroke}{rgb}{1.000000,1.000000,1.000000}%
\pgfsetstrokecolor{currentstroke}%
\pgfsetdash{}{0pt}%
\pgfpathmoveto{\pgfqpoint{2.135660in}{6.461132in}}%
\pgfpathlineto{\pgfqpoint{2.223396in}{6.461132in}}%
\pgfpathlineto{\pgfqpoint{2.223396in}{6.373396in}}%
\pgfpathlineto{\pgfqpoint{2.135660in}{6.373396in}}%
\pgfpathlineto{\pgfqpoint{2.135660in}{6.461132in}}%
\pgfusepath{stroke,fill}%
\end{pgfscope}%
\begin{pgfscope}%
\pgfpathrectangle{\pgfqpoint{0.380943in}{6.110189in}}{\pgfqpoint{4.650000in}{0.614151in}}%
\pgfusepath{clip}%
\pgfsetbuttcap%
\pgfsetroundjoin%
\definecolor{currentfill}{rgb}{0.996571,0.720538,0.589189}%
\pgfsetfillcolor{currentfill}%
\pgfsetlinewidth{0.250937pt}%
\definecolor{currentstroke}{rgb}{1.000000,1.000000,1.000000}%
\pgfsetstrokecolor{currentstroke}%
\pgfsetdash{}{0pt}%
\pgfpathmoveto{\pgfqpoint{2.223396in}{6.461132in}}%
\pgfpathlineto{\pgfqpoint{2.311132in}{6.461132in}}%
\pgfpathlineto{\pgfqpoint{2.311132in}{6.373396in}}%
\pgfpathlineto{\pgfqpoint{2.223396in}{6.373396in}}%
\pgfpathlineto{\pgfqpoint{2.223396in}{6.461132in}}%
\pgfusepath{stroke,fill}%
\end{pgfscope}%
\begin{pgfscope}%
\pgfpathrectangle{\pgfqpoint{0.380943in}{6.110189in}}{\pgfqpoint{4.650000in}{0.614151in}}%
\pgfusepath{clip}%
\pgfsetbuttcap%
\pgfsetroundjoin%
\definecolor{currentfill}{rgb}{0.992326,0.765229,0.614840}%
\pgfsetfillcolor{currentfill}%
\pgfsetlinewidth{0.250937pt}%
\definecolor{currentstroke}{rgb}{1.000000,1.000000,1.000000}%
\pgfsetstrokecolor{currentstroke}%
\pgfsetdash{}{0pt}%
\pgfpathmoveto{\pgfqpoint{2.311132in}{6.461132in}}%
\pgfpathlineto{\pgfqpoint{2.398868in}{6.461132in}}%
\pgfpathlineto{\pgfqpoint{2.398868in}{6.373396in}}%
\pgfpathlineto{\pgfqpoint{2.311132in}{6.373396in}}%
\pgfpathlineto{\pgfqpoint{2.311132in}{6.461132in}}%
\pgfusepath{stroke,fill}%
\end{pgfscope}%
\begin{pgfscope}%
\pgfpathrectangle{\pgfqpoint{0.380943in}{6.110189in}}{\pgfqpoint{4.650000in}{0.614151in}}%
\pgfusepath{clip}%
\pgfsetbuttcap%
\pgfsetroundjoin%
\definecolor{currentfill}{rgb}{0.972549,0.870588,0.692810}%
\pgfsetfillcolor{currentfill}%
\pgfsetlinewidth{0.250937pt}%
\definecolor{currentstroke}{rgb}{1.000000,1.000000,1.000000}%
\pgfsetstrokecolor{currentstroke}%
\pgfsetdash{}{0pt}%
\pgfpathmoveto{\pgfqpoint{2.398868in}{6.461132in}}%
\pgfpathlineto{\pgfqpoint{2.486603in}{6.461132in}}%
\pgfpathlineto{\pgfqpoint{2.486603in}{6.373396in}}%
\pgfpathlineto{\pgfqpoint{2.398868in}{6.373396in}}%
\pgfpathlineto{\pgfqpoint{2.398868in}{6.461132in}}%
\pgfusepath{stroke,fill}%
\end{pgfscope}%
\begin{pgfscope}%
\pgfpathrectangle{\pgfqpoint{0.380943in}{6.110189in}}{\pgfqpoint{4.650000in}{0.614151in}}%
\pgfusepath{clip}%
\pgfsetbuttcap%
\pgfsetroundjoin%
\definecolor{currentfill}{rgb}{0.979654,0.837186,0.669619}%
\pgfsetfillcolor{currentfill}%
\pgfsetlinewidth{0.250937pt}%
\definecolor{currentstroke}{rgb}{1.000000,1.000000,1.000000}%
\pgfsetstrokecolor{currentstroke}%
\pgfsetdash{}{0pt}%
\pgfpathmoveto{\pgfqpoint{2.486603in}{6.461132in}}%
\pgfpathlineto{\pgfqpoint{2.574339in}{6.461132in}}%
\pgfpathlineto{\pgfqpoint{2.574339in}{6.373396in}}%
\pgfpathlineto{\pgfqpoint{2.486603in}{6.373396in}}%
\pgfpathlineto{\pgfqpoint{2.486603in}{6.461132in}}%
\pgfusepath{stroke,fill}%
\end{pgfscope}%
\begin{pgfscope}%
\pgfpathrectangle{\pgfqpoint{0.380943in}{6.110189in}}{\pgfqpoint{4.650000in}{0.614151in}}%
\pgfusepath{clip}%
\pgfsetbuttcap%
\pgfsetroundjoin%
\definecolor{currentfill}{rgb}{0.996571,0.720538,0.589189}%
\pgfsetfillcolor{currentfill}%
\pgfsetlinewidth{0.250937pt}%
\definecolor{currentstroke}{rgb}{1.000000,1.000000,1.000000}%
\pgfsetstrokecolor{currentstroke}%
\pgfsetdash{}{0pt}%
\pgfpathmoveto{\pgfqpoint{2.574339in}{6.461132in}}%
\pgfpathlineto{\pgfqpoint{2.662075in}{6.461132in}}%
\pgfpathlineto{\pgfqpoint{2.662075in}{6.373396in}}%
\pgfpathlineto{\pgfqpoint{2.574339in}{6.373396in}}%
\pgfpathlineto{\pgfqpoint{2.574339in}{6.461132in}}%
\pgfusepath{stroke,fill}%
\end{pgfscope}%
\begin{pgfscope}%
\pgfpathrectangle{\pgfqpoint{0.380943in}{6.110189in}}{\pgfqpoint{4.650000in}{0.614151in}}%
\pgfusepath{clip}%
\pgfsetbuttcap%
\pgfsetroundjoin%
\definecolor{currentfill}{rgb}{0.986759,0.806398,0.641200}%
\pgfsetfillcolor{currentfill}%
\pgfsetlinewidth{0.250937pt}%
\definecolor{currentstroke}{rgb}{1.000000,1.000000,1.000000}%
\pgfsetstrokecolor{currentstroke}%
\pgfsetdash{}{0pt}%
\pgfpathmoveto{\pgfqpoint{2.662075in}{6.461132in}}%
\pgfpathlineto{\pgfqpoint{2.749811in}{6.461132in}}%
\pgfpathlineto{\pgfqpoint{2.749811in}{6.373396in}}%
\pgfpathlineto{\pgfqpoint{2.662075in}{6.373396in}}%
\pgfpathlineto{\pgfqpoint{2.662075in}{6.461132in}}%
\pgfusepath{stroke,fill}%
\end{pgfscope}%
\begin{pgfscope}%
\pgfpathrectangle{\pgfqpoint{0.380943in}{6.110189in}}{\pgfqpoint{4.650000in}{0.614151in}}%
\pgfusepath{clip}%
\pgfsetbuttcap%
\pgfsetroundjoin%
\definecolor{currentfill}{rgb}{0.972549,0.870588,0.692810}%
\pgfsetfillcolor{currentfill}%
\pgfsetlinewidth{0.250937pt}%
\definecolor{currentstroke}{rgb}{1.000000,1.000000,1.000000}%
\pgfsetstrokecolor{currentstroke}%
\pgfsetdash{}{0pt}%
\pgfpathmoveto{\pgfqpoint{2.749811in}{6.461132in}}%
\pgfpathlineto{\pgfqpoint{2.837547in}{6.461132in}}%
\pgfpathlineto{\pgfqpoint{2.837547in}{6.373396in}}%
\pgfpathlineto{\pgfqpoint{2.749811in}{6.373396in}}%
\pgfpathlineto{\pgfqpoint{2.749811in}{6.461132in}}%
\pgfusepath{stroke,fill}%
\end{pgfscope}%
\begin{pgfscope}%
\pgfpathrectangle{\pgfqpoint{0.380943in}{6.110189in}}{\pgfqpoint{4.650000in}{0.614151in}}%
\pgfusepath{clip}%
\pgfsetbuttcap%
\pgfsetroundjoin%
\definecolor{currentfill}{rgb}{0.998939,0.658962,0.556032}%
\pgfsetfillcolor{currentfill}%
\pgfsetlinewidth{0.250937pt}%
\definecolor{currentstroke}{rgb}{1.000000,1.000000,1.000000}%
\pgfsetstrokecolor{currentstroke}%
\pgfsetdash{}{0pt}%
\pgfpathmoveto{\pgfqpoint{2.837547in}{6.461132in}}%
\pgfpathlineto{\pgfqpoint{2.925283in}{6.461132in}}%
\pgfpathlineto{\pgfqpoint{2.925283in}{6.373396in}}%
\pgfpathlineto{\pgfqpoint{2.837547in}{6.373396in}}%
\pgfpathlineto{\pgfqpoint{2.837547in}{6.461132in}}%
\pgfusepath{stroke,fill}%
\end{pgfscope}%
\begin{pgfscope}%
\pgfpathrectangle{\pgfqpoint{0.380943in}{6.110189in}}{\pgfqpoint{4.650000in}{0.614151in}}%
\pgfusepath{clip}%
\pgfsetbuttcap%
\pgfsetroundjoin%
\definecolor{currentfill}{rgb}{0.992326,0.765229,0.614840}%
\pgfsetfillcolor{currentfill}%
\pgfsetlinewidth{0.250937pt}%
\definecolor{currentstroke}{rgb}{1.000000,1.000000,1.000000}%
\pgfsetstrokecolor{currentstroke}%
\pgfsetdash{}{0pt}%
\pgfpathmoveto{\pgfqpoint{2.925283in}{6.461132in}}%
\pgfpathlineto{\pgfqpoint{3.013019in}{6.461132in}}%
\pgfpathlineto{\pgfqpoint{3.013019in}{6.373396in}}%
\pgfpathlineto{\pgfqpoint{2.925283in}{6.373396in}}%
\pgfpathlineto{\pgfqpoint{2.925283in}{6.461132in}}%
\pgfusepath{stroke,fill}%
\end{pgfscope}%
\begin{pgfscope}%
\pgfpathrectangle{\pgfqpoint{0.380943in}{6.110189in}}{\pgfqpoint{4.650000in}{0.614151in}}%
\pgfusepath{clip}%
\pgfsetbuttcap%
\pgfsetroundjoin%
\definecolor{currentfill}{rgb}{0.965444,0.906113,0.711757}%
\pgfsetfillcolor{currentfill}%
\pgfsetlinewidth{0.250937pt}%
\definecolor{currentstroke}{rgb}{1.000000,1.000000,1.000000}%
\pgfsetstrokecolor{currentstroke}%
\pgfsetdash{}{0pt}%
\pgfpathmoveto{\pgfqpoint{3.013019in}{6.461132in}}%
\pgfpathlineto{\pgfqpoint{3.100754in}{6.461132in}}%
\pgfpathlineto{\pgfqpoint{3.100754in}{6.373396in}}%
\pgfpathlineto{\pgfqpoint{3.013019in}{6.373396in}}%
\pgfpathlineto{\pgfqpoint{3.013019in}{6.461132in}}%
\pgfusepath{stroke,fill}%
\end{pgfscope}%
\begin{pgfscope}%
\pgfpathrectangle{\pgfqpoint{0.380943in}{6.110189in}}{\pgfqpoint{4.650000in}{0.614151in}}%
\pgfusepath{clip}%
\pgfsetbuttcap%
\pgfsetroundjoin%
\definecolor{currentfill}{rgb}{0.965444,0.906113,0.711757}%
\pgfsetfillcolor{currentfill}%
\pgfsetlinewidth{0.250937pt}%
\definecolor{currentstroke}{rgb}{1.000000,1.000000,1.000000}%
\pgfsetstrokecolor{currentstroke}%
\pgfsetdash{}{0pt}%
\pgfpathmoveto{\pgfqpoint{3.100754in}{6.461132in}}%
\pgfpathlineto{\pgfqpoint{3.188490in}{6.461132in}}%
\pgfpathlineto{\pgfqpoint{3.188490in}{6.373396in}}%
\pgfpathlineto{\pgfqpoint{3.100754in}{6.373396in}}%
\pgfpathlineto{\pgfqpoint{3.100754in}{6.461132in}}%
\pgfusepath{stroke,fill}%
\end{pgfscope}%
\begin{pgfscope}%
\pgfpathrectangle{\pgfqpoint{0.380943in}{6.110189in}}{\pgfqpoint{4.650000in}{0.614151in}}%
\pgfusepath{clip}%
\pgfsetbuttcap%
\pgfsetroundjoin%
\definecolor{currentfill}{rgb}{0.972549,0.870588,0.692810}%
\pgfsetfillcolor{currentfill}%
\pgfsetlinewidth{0.250937pt}%
\definecolor{currentstroke}{rgb}{1.000000,1.000000,1.000000}%
\pgfsetstrokecolor{currentstroke}%
\pgfsetdash{}{0pt}%
\pgfpathmoveto{\pgfqpoint{3.188490in}{6.461132in}}%
\pgfpathlineto{\pgfqpoint{3.276226in}{6.461132in}}%
\pgfpathlineto{\pgfqpoint{3.276226in}{6.373396in}}%
\pgfpathlineto{\pgfqpoint{3.188490in}{6.373396in}}%
\pgfpathlineto{\pgfqpoint{3.188490in}{6.461132in}}%
\pgfusepath{stroke,fill}%
\end{pgfscope}%
\begin{pgfscope}%
\pgfpathrectangle{\pgfqpoint{0.380943in}{6.110189in}}{\pgfqpoint{4.650000in}{0.614151in}}%
\pgfusepath{clip}%
\pgfsetbuttcap%
\pgfsetroundjoin%
\definecolor{currentfill}{rgb}{0.962414,0.923552,0.722891}%
\pgfsetfillcolor{currentfill}%
\pgfsetlinewidth{0.250937pt}%
\definecolor{currentstroke}{rgb}{1.000000,1.000000,1.000000}%
\pgfsetstrokecolor{currentstroke}%
\pgfsetdash{}{0pt}%
\pgfpathmoveto{\pgfqpoint{3.276226in}{6.461132in}}%
\pgfpathlineto{\pgfqpoint{3.363962in}{6.461132in}}%
\pgfpathlineto{\pgfqpoint{3.363962in}{6.373396in}}%
\pgfpathlineto{\pgfqpoint{3.276226in}{6.373396in}}%
\pgfpathlineto{\pgfqpoint{3.276226in}{6.461132in}}%
\pgfusepath{stroke,fill}%
\end{pgfscope}%
\begin{pgfscope}%
\pgfpathrectangle{\pgfqpoint{0.380943in}{6.110189in}}{\pgfqpoint{4.650000in}{0.614151in}}%
\pgfusepath{clip}%
\pgfsetbuttcap%
\pgfsetroundjoin%
\definecolor{currentfill}{rgb}{0.968166,0.945882,0.748604}%
\pgfsetfillcolor{currentfill}%
\pgfsetlinewidth{0.250937pt}%
\definecolor{currentstroke}{rgb}{1.000000,1.000000,1.000000}%
\pgfsetstrokecolor{currentstroke}%
\pgfsetdash{}{0pt}%
\pgfpathmoveto{\pgfqpoint{3.363962in}{6.461132in}}%
\pgfpathlineto{\pgfqpoint{3.451698in}{6.461132in}}%
\pgfpathlineto{\pgfqpoint{3.451698in}{6.373396in}}%
\pgfpathlineto{\pgfqpoint{3.363962in}{6.373396in}}%
\pgfpathlineto{\pgfqpoint{3.363962in}{6.461132in}}%
\pgfusepath{stroke,fill}%
\end{pgfscope}%
\begin{pgfscope}%
\pgfpathrectangle{\pgfqpoint{0.380943in}{6.110189in}}{\pgfqpoint{4.650000in}{0.614151in}}%
\pgfusepath{clip}%
\pgfsetbuttcap%
\pgfsetroundjoin%
\definecolor{currentfill}{rgb}{0.972549,0.870588,0.692810}%
\pgfsetfillcolor{currentfill}%
\pgfsetlinewidth{0.250937pt}%
\definecolor{currentstroke}{rgb}{1.000000,1.000000,1.000000}%
\pgfsetstrokecolor{currentstroke}%
\pgfsetdash{}{0pt}%
\pgfpathmoveto{\pgfqpoint{3.451698in}{6.461132in}}%
\pgfpathlineto{\pgfqpoint{3.539434in}{6.461132in}}%
\pgfpathlineto{\pgfqpoint{3.539434in}{6.373396in}}%
\pgfpathlineto{\pgfqpoint{3.451698in}{6.373396in}}%
\pgfpathlineto{\pgfqpoint{3.451698in}{6.461132in}}%
\pgfusepath{stroke,fill}%
\end{pgfscope}%
\begin{pgfscope}%
\pgfpathrectangle{\pgfqpoint{0.380943in}{6.110189in}}{\pgfqpoint{4.650000in}{0.614151in}}%
\pgfusepath{clip}%
\pgfsetbuttcap%
\pgfsetroundjoin%
\definecolor{currentfill}{rgb}{0.962414,0.923552,0.722891}%
\pgfsetfillcolor{currentfill}%
\pgfsetlinewidth{0.250937pt}%
\definecolor{currentstroke}{rgb}{1.000000,1.000000,1.000000}%
\pgfsetstrokecolor{currentstroke}%
\pgfsetdash{}{0pt}%
\pgfpathmoveto{\pgfqpoint{3.539434in}{6.461132in}}%
\pgfpathlineto{\pgfqpoint{3.627169in}{6.461132in}}%
\pgfpathlineto{\pgfqpoint{3.627169in}{6.373396in}}%
\pgfpathlineto{\pgfqpoint{3.539434in}{6.373396in}}%
\pgfpathlineto{\pgfqpoint{3.539434in}{6.461132in}}%
\pgfusepath{stroke,fill}%
\end{pgfscope}%
\begin{pgfscope}%
\pgfpathrectangle{\pgfqpoint{0.380943in}{6.110189in}}{\pgfqpoint{4.650000in}{0.614151in}}%
\pgfusepath{clip}%
\pgfsetbuttcap%
\pgfsetroundjoin%
\definecolor{currentfill}{rgb}{0.979654,0.837186,0.669619}%
\pgfsetfillcolor{currentfill}%
\pgfsetlinewidth{0.250937pt}%
\definecolor{currentstroke}{rgb}{1.000000,1.000000,1.000000}%
\pgfsetstrokecolor{currentstroke}%
\pgfsetdash{}{0pt}%
\pgfpathmoveto{\pgfqpoint{3.627169in}{6.461132in}}%
\pgfpathlineto{\pgfqpoint{3.714905in}{6.461132in}}%
\pgfpathlineto{\pgfqpoint{3.714905in}{6.373396in}}%
\pgfpathlineto{\pgfqpoint{3.627169in}{6.373396in}}%
\pgfpathlineto{\pgfqpoint{3.627169in}{6.461132in}}%
\pgfusepath{stroke,fill}%
\end{pgfscope}%
\begin{pgfscope}%
\pgfpathrectangle{\pgfqpoint{0.380943in}{6.110189in}}{\pgfqpoint{4.650000in}{0.614151in}}%
\pgfusepath{clip}%
\pgfsetbuttcap%
\pgfsetroundjoin%
\definecolor{currentfill}{rgb}{0.968166,0.945882,0.748604}%
\pgfsetfillcolor{currentfill}%
\pgfsetlinewidth{0.250937pt}%
\definecolor{currentstroke}{rgb}{1.000000,1.000000,1.000000}%
\pgfsetstrokecolor{currentstroke}%
\pgfsetdash{}{0pt}%
\pgfpathmoveto{\pgfqpoint{3.714905in}{6.461132in}}%
\pgfpathlineto{\pgfqpoint{3.802641in}{6.461132in}}%
\pgfpathlineto{\pgfqpoint{3.802641in}{6.373396in}}%
\pgfpathlineto{\pgfqpoint{3.714905in}{6.373396in}}%
\pgfpathlineto{\pgfqpoint{3.714905in}{6.461132in}}%
\pgfusepath{stroke,fill}%
\end{pgfscope}%
\begin{pgfscope}%
\pgfpathrectangle{\pgfqpoint{0.380943in}{6.110189in}}{\pgfqpoint{4.650000in}{0.614151in}}%
\pgfusepath{clip}%
\pgfsetbuttcap%
\pgfsetroundjoin%
\definecolor{currentfill}{rgb}{0.998939,0.658962,0.556032}%
\pgfsetfillcolor{currentfill}%
\pgfsetlinewidth{0.250937pt}%
\definecolor{currentstroke}{rgb}{1.000000,1.000000,1.000000}%
\pgfsetstrokecolor{currentstroke}%
\pgfsetdash{}{0pt}%
\pgfpathmoveto{\pgfqpoint{3.802641in}{6.461132in}}%
\pgfpathlineto{\pgfqpoint{3.890377in}{6.461132in}}%
\pgfpathlineto{\pgfqpoint{3.890377in}{6.373396in}}%
\pgfpathlineto{\pgfqpoint{3.802641in}{6.373396in}}%
\pgfpathlineto{\pgfqpoint{3.802641in}{6.461132in}}%
\pgfusepath{stroke,fill}%
\end{pgfscope}%
\begin{pgfscope}%
\pgfpathrectangle{\pgfqpoint{0.380943in}{6.110189in}}{\pgfqpoint{4.650000in}{0.614151in}}%
\pgfusepath{clip}%
\pgfsetbuttcap%
\pgfsetroundjoin%
\definecolor{currentfill}{rgb}{0.972549,0.870588,0.692810}%
\pgfsetfillcolor{currentfill}%
\pgfsetlinewidth{0.250937pt}%
\definecolor{currentstroke}{rgb}{1.000000,1.000000,1.000000}%
\pgfsetstrokecolor{currentstroke}%
\pgfsetdash{}{0pt}%
\pgfpathmoveto{\pgfqpoint{3.890377in}{6.461132in}}%
\pgfpathlineto{\pgfqpoint{3.978113in}{6.461132in}}%
\pgfpathlineto{\pgfqpoint{3.978113in}{6.373396in}}%
\pgfpathlineto{\pgfqpoint{3.890377in}{6.373396in}}%
\pgfpathlineto{\pgfqpoint{3.890377in}{6.461132in}}%
\pgfusepath{stroke,fill}%
\end{pgfscope}%
\begin{pgfscope}%
\pgfpathrectangle{\pgfqpoint{0.380943in}{6.110189in}}{\pgfqpoint{4.650000in}{0.614151in}}%
\pgfusepath{clip}%
\pgfsetbuttcap%
\pgfsetroundjoin%
\definecolor{currentfill}{rgb}{1.000000,0.557862,0.511772}%
\pgfsetfillcolor{currentfill}%
\pgfsetlinewidth{0.250937pt}%
\definecolor{currentstroke}{rgb}{1.000000,1.000000,1.000000}%
\pgfsetstrokecolor{currentstroke}%
\pgfsetdash{}{0pt}%
\pgfpathmoveto{\pgfqpoint{3.978113in}{6.461132in}}%
\pgfpathlineto{\pgfqpoint{4.065849in}{6.461132in}}%
\pgfpathlineto{\pgfqpoint{4.065849in}{6.373396in}}%
\pgfpathlineto{\pgfqpoint{3.978113in}{6.373396in}}%
\pgfpathlineto{\pgfqpoint{3.978113in}{6.461132in}}%
\pgfusepath{stroke,fill}%
\end{pgfscope}%
\begin{pgfscope}%
\pgfpathrectangle{\pgfqpoint{0.380943in}{6.110189in}}{\pgfqpoint{4.650000in}{0.614151in}}%
\pgfusepath{clip}%
\pgfsetbuttcap%
\pgfsetroundjoin%
\definecolor{currentfill}{rgb}{0.992326,0.765229,0.614840}%
\pgfsetfillcolor{currentfill}%
\pgfsetlinewidth{0.250937pt}%
\definecolor{currentstroke}{rgb}{1.000000,1.000000,1.000000}%
\pgfsetstrokecolor{currentstroke}%
\pgfsetdash{}{0pt}%
\pgfpathmoveto{\pgfqpoint{4.065849in}{6.461132in}}%
\pgfpathlineto{\pgfqpoint{4.153585in}{6.461132in}}%
\pgfpathlineto{\pgfqpoint{4.153585in}{6.373396in}}%
\pgfpathlineto{\pgfqpoint{4.065849in}{6.373396in}}%
\pgfpathlineto{\pgfqpoint{4.065849in}{6.461132in}}%
\pgfusepath{stroke,fill}%
\end{pgfscope}%
\begin{pgfscope}%
\pgfpathrectangle{\pgfqpoint{0.380943in}{6.110189in}}{\pgfqpoint{4.650000in}{0.614151in}}%
\pgfusepath{clip}%
\pgfsetbuttcap%
\pgfsetroundjoin%
\definecolor{currentfill}{rgb}{0.968166,0.945882,0.748604}%
\pgfsetfillcolor{currentfill}%
\pgfsetlinewidth{0.250937pt}%
\definecolor{currentstroke}{rgb}{1.000000,1.000000,1.000000}%
\pgfsetstrokecolor{currentstroke}%
\pgfsetdash{}{0pt}%
\pgfpathmoveto{\pgfqpoint{4.153585in}{6.461132in}}%
\pgfpathlineto{\pgfqpoint{4.241320in}{6.461132in}}%
\pgfpathlineto{\pgfqpoint{4.241320in}{6.373396in}}%
\pgfpathlineto{\pgfqpoint{4.153585in}{6.373396in}}%
\pgfpathlineto{\pgfqpoint{4.153585in}{6.461132in}}%
\pgfusepath{stroke,fill}%
\end{pgfscope}%
\begin{pgfscope}%
\pgfpathrectangle{\pgfqpoint{0.380943in}{6.110189in}}{\pgfqpoint{4.650000in}{0.614151in}}%
\pgfusepath{clip}%
\pgfsetbuttcap%
\pgfsetroundjoin%
\definecolor{currentfill}{rgb}{0.992326,0.765229,0.614840}%
\pgfsetfillcolor{currentfill}%
\pgfsetlinewidth{0.250937pt}%
\definecolor{currentstroke}{rgb}{1.000000,1.000000,1.000000}%
\pgfsetstrokecolor{currentstroke}%
\pgfsetdash{}{0pt}%
\pgfpathmoveto{\pgfqpoint{4.241320in}{6.461132in}}%
\pgfpathlineto{\pgfqpoint{4.329056in}{6.461132in}}%
\pgfpathlineto{\pgfqpoint{4.329056in}{6.373396in}}%
\pgfpathlineto{\pgfqpoint{4.241320in}{6.373396in}}%
\pgfpathlineto{\pgfqpoint{4.241320in}{6.461132in}}%
\pgfusepath{stroke,fill}%
\end{pgfscope}%
\begin{pgfscope}%
\pgfpathrectangle{\pgfqpoint{0.380943in}{6.110189in}}{\pgfqpoint{4.650000in}{0.614151in}}%
\pgfusepath{clip}%
\pgfsetbuttcap%
\pgfsetroundjoin%
\definecolor{currentfill}{rgb}{1.000000,0.605229,0.530719}%
\pgfsetfillcolor{currentfill}%
\pgfsetlinewidth{0.250937pt}%
\definecolor{currentstroke}{rgb}{1.000000,1.000000,1.000000}%
\pgfsetstrokecolor{currentstroke}%
\pgfsetdash{}{0pt}%
\pgfpathmoveto{\pgfqpoint{4.329056in}{6.461132in}}%
\pgfpathlineto{\pgfqpoint{4.416792in}{6.461132in}}%
\pgfpathlineto{\pgfqpoint{4.416792in}{6.373396in}}%
\pgfpathlineto{\pgfqpoint{4.329056in}{6.373396in}}%
\pgfpathlineto{\pgfqpoint{4.329056in}{6.461132in}}%
\pgfusepath{stroke,fill}%
\end{pgfscope}%
\begin{pgfscope}%
\pgfpathrectangle{\pgfqpoint{0.380943in}{6.110189in}}{\pgfqpoint{4.650000in}{0.614151in}}%
\pgfusepath{clip}%
\pgfsetbuttcap%
\pgfsetroundjoin%
\definecolor{currentfill}{rgb}{0.972549,0.870588,0.692810}%
\pgfsetfillcolor{currentfill}%
\pgfsetlinewidth{0.250937pt}%
\definecolor{currentstroke}{rgb}{1.000000,1.000000,1.000000}%
\pgfsetstrokecolor{currentstroke}%
\pgfsetdash{}{0pt}%
\pgfpathmoveto{\pgfqpoint{4.416792in}{6.461132in}}%
\pgfpathlineto{\pgfqpoint{4.504528in}{6.461132in}}%
\pgfpathlineto{\pgfqpoint{4.504528in}{6.373396in}}%
\pgfpathlineto{\pgfqpoint{4.416792in}{6.373396in}}%
\pgfpathlineto{\pgfqpoint{4.416792in}{6.461132in}}%
\pgfusepath{stroke,fill}%
\end{pgfscope}%
\begin{pgfscope}%
\pgfpathrectangle{\pgfqpoint{0.380943in}{6.110189in}}{\pgfqpoint{4.650000in}{0.614151in}}%
\pgfusepath{clip}%
\pgfsetbuttcap%
\pgfsetroundjoin%
\definecolor{currentfill}{rgb}{0.996571,0.720538,0.589189}%
\pgfsetfillcolor{currentfill}%
\pgfsetlinewidth{0.250937pt}%
\definecolor{currentstroke}{rgb}{1.000000,1.000000,1.000000}%
\pgfsetstrokecolor{currentstroke}%
\pgfsetdash{}{0pt}%
\pgfpathmoveto{\pgfqpoint{4.504528in}{6.461132in}}%
\pgfpathlineto{\pgfqpoint{4.592264in}{6.461132in}}%
\pgfpathlineto{\pgfqpoint{4.592264in}{6.373396in}}%
\pgfpathlineto{\pgfqpoint{4.504528in}{6.373396in}}%
\pgfpathlineto{\pgfqpoint{4.504528in}{6.461132in}}%
\pgfusepath{stroke,fill}%
\end{pgfscope}%
\begin{pgfscope}%
\pgfpathrectangle{\pgfqpoint{0.380943in}{6.110189in}}{\pgfqpoint{4.650000in}{0.614151in}}%
\pgfusepath{clip}%
\pgfsetbuttcap%
\pgfsetroundjoin%
\definecolor{currentfill}{rgb}{1.000000,1.000000,0.929412}%
\pgfsetfillcolor{currentfill}%
\pgfsetlinewidth{0.250937pt}%
\definecolor{currentstroke}{rgb}{1.000000,1.000000,1.000000}%
\pgfsetstrokecolor{currentstroke}%
\pgfsetdash{}{0pt}%
\pgfpathmoveto{\pgfqpoint{4.592264in}{6.461132in}}%
\pgfpathlineto{\pgfqpoint{4.680000in}{6.461132in}}%
\pgfpathlineto{\pgfqpoint{4.680000in}{6.373396in}}%
\pgfpathlineto{\pgfqpoint{4.592264in}{6.373396in}}%
\pgfpathlineto{\pgfqpoint{4.592264in}{6.461132in}}%
\pgfusepath{stroke,fill}%
\end{pgfscope}%
\begin{pgfscope}%
\pgfpathrectangle{\pgfqpoint{0.380943in}{6.110189in}}{\pgfqpoint{4.650000in}{0.614151in}}%
\pgfusepath{clip}%
\pgfsetbuttcap%
\pgfsetroundjoin%
\definecolor{currentfill}{rgb}{0.979654,0.837186,0.669619}%
\pgfsetfillcolor{currentfill}%
\pgfsetlinewidth{0.250937pt}%
\definecolor{currentstroke}{rgb}{1.000000,1.000000,1.000000}%
\pgfsetstrokecolor{currentstroke}%
\pgfsetdash{}{0pt}%
\pgfpathmoveto{\pgfqpoint{4.680000in}{6.461132in}}%
\pgfpathlineto{\pgfqpoint{4.767736in}{6.461132in}}%
\pgfpathlineto{\pgfqpoint{4.767736in}{6.373396in}}%
\pgfpathlineto{\pgfqpoint{4.680000in}{6.373396in}}%
\pgfpathlineto{\pgfqpoint{4.680000in}{6.461132in}}%
\pgfusepath{stroke,fill}%
\end{pgfscope}%
\begin{pgfscope}%
\pgfpathrectangle{\pgfqpoint{0.380943in}{6.110189in}}{\pgfqpoint{4.650000in}{0.614151in}}%
\pgfusepath{clip}%
\pgfsetbuttcap%
\pgfsetroundjoin%
\definecolor{currentfill}{rgb}{0.972549,0.870588,0.692810}%
\pgfsetfillcolor{currentfill}%
\pgfsetlinewidth{0.250937pt}%
\definecolor{currentstroke}{rgb}{1.000000,1.000000,1.000000}%
\pgfsetstrokecolor{currentstroke}%
\pgfsetdash{}{0pt}%
\pgfpathmoveto{\pgfqpoint{4.767736in}{6.461132in}}%
\pgfpathlineto{\pgfqpoint{4.855471in}{6.461132in}}%
\pgfpathlineto{\pgfqpoint{4.855471in}{6.373396in}}%
\pgfpathlineto{\pgfqpoint{4.767736in}{6.373396in}}%
\pgfpathlineto{\pgfqpoint{4.767736in}{6.461132in}}%
\pgfusepath{stroke,fill}%
\end{pgfscope}%
\begin{pgfscope}%
\pgfpathrectangle{\pgfqpoint{0.380943in}{6.110189in}}{\pgfqpoint{4.650000in}{0.614151in}}%
\pgfusepath{clip}%
\pgfsetbuttcap%
\pgfsetroundjoin%
\definecolor{currentfill}{rgb}{0.979654,0.837186,0.669619}%
\pgfsetfillcolor{currentfill}%
\pgfsetlinewidth{0.250937pt}%
\definecolor{currentstroke}{rgb}{1.000000,1.000000,1.000000}%
\pgfsetstrokecolor{currentstroke}%
\pgfsetdash{}{0pt}%
\pgfpathmoveto{\pgfqpoint{4.855471in}{6.461132in}}%
\pgfpathlineto{\pgfqpoint{4.943207in}{6.461132in}}%
\pgfpathlineto{\pgfqpoint{4.943207in}{6.373396in}}%
\pgfpathlineto{\pgfqpoint{4.855471in}{6.373396in}}%
\pgfpathlineto{\pgfqpoint{4.855471in}{6.461132in}}%
\pgfusepath{stroke,fill}%
\end{pgfscope}%
\begin{pgfscope}%
\pgfpathrectangle{\pgfqpoint{0.380943in}{6.110189in}}{\pgfqpoint{4.650000in}{0.614151in}}%
\pgfusepath{clip}%
\pgfsetbuttcap%
\pgfsetroundjoin%
\pgfsetlinewidth{0.250937pt}%
\definecolor{currentstroke}{rgb}{1.000000,1.000000,1.000000}%
\pgfsetstrokecolor{currentstroke}%
\pgfsetdash{}{0pt}%
\pgfpathmoveto{\pgfqpoint{4.943207in}{6.461132in}}%
\pgfpathlineto{\pgfqpoint{5.030943in}{6.461132in}}%
\pgfpathlineto{\pgfqpoint{5.030943in}{6.373396in}}%
\pgfpathlineto{\pgfqpoint{4.943207in}{6.373396in}}%
\pgfpathlineto{\pgfqpoint{4.943207in}{6.461132in}}%
\pgfusepath{stroke}%
\end{pgfscope}%
\begin{pgfscope}%
\pgfpathrectangle{\pgfqpoint{0.380943in}{6.110189in}}{\pgfqpoint{4.650000in}{0.614151in}}%
\pgfusepath{clip}%
\pgfsetbuttcap%
\pgfsetroundjoin%
\definecolor{currentfill}{rgb}{1.000000,0.605229,0.530719}%
\pgfsetfillcolor{currentfill}%
\pgfsetlinewidth{0.250937pt}%
\definecolor{currentstroke}{rgb}{1.000000,1.000000,1.000000}%
\pgfsetstrokecolor{currentstroke}%
\pgfsetdash{}{0pt}%
\pgfpathmoveto{\pgfqpoint{0.380943in}{6.373396in}}%
\pgfpathlineto{\pgfqpoint{0.468679in}{6.373396in}}%
\pgfpathlineto{\pgfqpoint{0.468679in}{6.285661in}}%
\pgfpathlineto{\pgfqpoint{0.380943in}{6.285661in}}%
\pgfpathlineto{\pgfqpoint{0.380943in}{6.373396in}}%
\pgfusepath{stroke,fill}%
\end{pgfscope}%
\begin{pgfscope}%
\pgfpathrectangle{\pgfqpoint{0.380943in}{6.110189in}}{\pgfqpoint{4.650000in}{0.614151in}}%
\pgfusepath{clip}%
\pgfsetbuttcap%
\pgfsetroundjoin%
\definecolor{currentfill}{rgb}{0.998939,0.658962,0.556032}%
\pgfsetfillcolor{currentfill}%
\pgfsetlinewidth{0.250937pt}%
\definecolor{currentstroke}{rgb}{1.000000,1.000000,1.000000}%
\pgfsetstrokecolor{currentstroke}%
\pgfsetdash{}{0pt}%
\pgfpathmoveto{\pgfqpoint{0.468679in}{6.373396in}}%
\pgfpathlineto{\pgfqpoint{0.556415in}{6.373396in}}%
\pgfpathlineto{\pgfqpoint{0.556415in}{6.285661in}}%
\pgfpathlineto{\pgfqpoint{0.468679in}{6.285661in}}%
\pgfpathlineto{\pgfqpoint{0.468679in}{6.373396in}}%
\pgfusepath{stroke,fill}%
\end{pgfscope}%
\begin{pgfscope}%
\pgfpathrectangle{\pgfqpoint{0.380943in}{6.110189in}}{\pgfqpoint{4.650000in}{0.614151in}}%
\pgfusepath{clip}%
\pgfsetbuttcap%
\pgfsetroundjoin%
\definecolor{currentfill}{rgb}{0.992326,0.765229,0.614840}%
\pgfsetfillcolor{currentfill}%
\pgfsetlinewidth{0.250937pt}%
\definecolor{currentstroke}{rgb}{1.000000,1.000000,1.000000}%
\pgfsetstrokecolor{currentstroke}%
\pgfsetdash{}{0pt}%
\pgfpathmoveto{\pgfqpoint{0.556415in}{6.373396in}}%
\pgfpathlineto{\pgfqpoint{0.644151in}{6.373396in}}%
\pgfpathlineto{\pgfqpoint{0.644151in}{6.285661in}}%
\pgfpathlineto{\pgfqpoint{0.556415in}{6.285661in}}%
\pgfpathlineto{\pgfqpoint{0.556415in}{6.373396in}}%
\pgfusepath{stroke,fill}%
\end{pgfscope}%
\begin{pgfscope}%
\pgfpathrectangle{\pgfqpoint{0.380943in}{6.110189in}}{\pgfqpoint{4.650000in}{0.614151in}}%
\pgfusepath{clip}%
\pgfsetbuttcap%
\pgfsetroundjoin%
\definecolor{currentfill}{rgb}{1.000000,0.605229,0.530719}%
\pgfsetfillcolor{currentfill}%
\pgfsetlinewidth{0.250937pt}%
\definecolor{currentstroke}{rgb}{1.000000,1.000000,1.000000}%
\pgfsetstrokecolor{currentstroke}%
\pgfsetdash{}{0pt}%
\pgfpathmoveto{\pgfqpoint{0.644151in}{6.373396in}}%
\pgfpathlineto{\pgfqpoint{0.731886in}{6.373396in}}%
\pgfpathlineto{\pgfqpoint{0.731886in}{6.285661in}}%
\pgfpathlineto{\pgfqpoint{0.644151in}{6.285661in}}%
\pgfpathlineto{\pgfqpoint{0.644151in}{6.373396in}}%
\pgfusepath{stroke,fill}%
\end{pgfscope}%
\begin{pgfscope}%
\pgfpathrectangle{\pgfqpoint{0.380943in}{6.110189in}}{\pgfqpoint{4.650000in}{0.614151in}}%
\pgfusepath{clip}%
\pgfsetbuttcap%
\pgfsetroundjoin%
\definecolor{currentfill}{rgb}{0.981546,0.459977,0.459977}%
\pgfsetfillcolor{currentfill}%
\pgfsetlinewidth{0.250937pt}%
\definecolor{currentstroke}{rgb}{1.000000,1.000000,1.000000}%
\pgfsetstrokecolor{currentstroke}%
\pgfsetdash{}{0pt}%
\pgfpathmoveto{\pgfqpoint{0.731886in}{6.373396in}}%
\pgfpathlineto{\pgfqpoint{0.819622in}{6.373396in}}%
\pgfpathlineto{\pgfqpoint{0.819622in}{6.285661in}}%
\pgfpathlineto{\pgfqpoint{0.731886in}{6.285661in}}%
\pgfpathlineto{\pgfqpoint{0.731886in}{6.373396in}}%
\pgfusepath{stroke,fill}%
\end{pgfscope}%
\begin{pgfscope}%
\pgfpathrectangle{\pgfqpoint{0.380943in}{6.110189in}}{\pgfqpoint{4.650000in}{0.614151in}}%
\pgfusepath{clip}%
\pgfsetbuttcap%
\pgfsetroundjoin%
\definecolor{currentfill}{rgb}{0.992326,0.765229,0.614840}%
\pgfsetfillcolor{currentfill}%
\pgfsetlinewidth{0.250937pt}%
\definecolor{currentstroke}{rgb}{1.000000,1.000000,1.000000}%
\pgfsetstrokecolor{currentstroke}%
\pgfsetdash{}{0pt}%
\pgfpathmoveto{\pgfqpoint{0.819622in}{6.373396in}}%
\pgfpathlineto{\pgfqpoint{0.907358in}{6.373396in}}%
\pgfpathlineto{\pgfqpoint{0.907358in}{6.285661in}}%
\pgfpathlineto{\pgfqpoint{0.819622in}{6.285661in}}%
\pgfpathlineto{\pgfqpoint{0.819622in}{6.373396in}}%
\pgfusepath{stroke,fill}%
\end{pgfscope}%
\begin{pgfscope}%
\pgfpathrectangle{\pgfqpoint{0.380943in}{6.110189in}}{\pgfqpoint{4.650000in}{0.614151in}}%
\pgfusepath{clip}%
\pgfsetbuttcap%
\pgfsetroundjoin%
\definecolor{currentfill}{rgb}{0.968166,0.945882,0.748604}%
\pgfsetfillcolor{currentfill}%
\pgfsetlinewidth{0.250937pt}%
\definecolor{currentstroke}{rgb}{1.000000,1.000000,1.000000}%
\pgfsetstrokecolor{currentstroke}%
\pgfsetdash{}{0pt}%
\pgfpathmoveto{\pgfqpoint{0.907358in}{6.373396in}}%
\pgfpathlineto{\pgfqpoint{0.995094in}{6.373396in}}%
\pgfpathlineto{\pgfqpoint{0.995094in}{6.285661in}}%
\pgfpathlineto{\pgfqpoint{0.907358in}{6.285661in}}%
\pgfpathlineto{\pgfqpoint{0.907358in}{6.373396in}}%
\pgfusepath{stroke,fill}%
\end{pgfscope}%
\begin{pgfscope}%
\pgfpathrectangle{\pgfqpoint{0.380943in}{6.110189in}}{\pgfqpoint{4.650000in}{0.614151in}}%
\pgfusepath{clip}%
\pgfsetbuttcap%
\pgfsetroundjoin%
\definecolor{currentfill}{rgb}{0.979654,0.837186,0.669619}%
\pgfsetfillcolor{currentfill}%
\pgfsetlinewidth{0.250937pt}%
\definecolor{currentstroke}{rgb}{1.000000,1.000000,1.000000}%
\pgfsetstrokecolor{currentstroke}%
\pgfsetdash{}{0pt}%
\pgfpathmoveto{\pgfqpoint{0.995094in}{6.373396in}}%
\pgfpathlineto{\pgfqpoint{1.082830in}{6.373396in}}%
\pgfpathlineto{\pgfqpoint{1.082830in}{6.285661in}}%
\pgfpathlineto{\pgfqpoint{0.995094in}{6.285661in}}%
\pgfpathlineto{\pgfqpoint{0.995094in}{6.373396in}}%
\pgfusepath{stroke,fill}%
\end{pgfscope}%
\begin{pgfscope}%
\pgfpathrectangle{\pgfqpoint{0.380943in}{6.110189in}}{\pgfqpoint{4.650000in}{0.614151in}}%
\pgfusepath{clip}%
\pgfsetbuttcap%
\pgfsetroundjoin%
\definecolor{currentfill}{rgb}{0.992326,0.765229,0.614840}%
\pgfsetfillcolor{currentfill}%
\pgfsetlinewidth{0.250937pt}%
\definecolor{currentstroke}{rgb}{1.000000,1.000000,1.000000}%
\pgfsetstrokecolor{currentstroke}%
\pgfsetdash{}{0pt}%
\pgfpathmoveto{\pgfqpoint{1.082830in}{6.373396in}}%
\pgfpathlineto{\pgfqpoint{1.170566in}{6.373396in}}%
\pgfpathlineto{\pgfqpoint{1.170566in}{6.285661in}}%
\pgfpathlineto{\pgfqpoint{1.082830in}{6.285661in}}%
\pgfpathlineto{\pgfqpoint{1.082830in}{6.373396in}}%
\pgfusepath{stroke,fill}%
\end{pgfscope}%
\begin{pgfscope}%
\pgfpathrectangle{\pgfqpoint{0.380943in}{6.110189in}}{\pgfqpoint{4.650000in}{0.614151in}}%
\pgfusepath{clip}%
\pgfsetbuttcap%
\pgfsetroundjoin%
\definecolor{currentfill}{rgb}{0.972549,0.870588,0.692810}%
\pgfsetfillcolor{currentfill}%
\pgfsetlinewidth{0.250937pt}%
\definecolor{currentstroke}{rgb}{1.000000,1.000000,1.000000}%
\pgfsetstrokecolor{currentstroke}%
\pgfsetdash{}{0pt}%
\pgfpathmoveto{\pgfqpoint{1.170566in}{6.373396in}}%
\pgfpathlineto{\pgfqpoint{1.258302in}{6.373396in}}%
\pgfpathlineto{\pgfqpoint{1.258302in}{6.285661in}}%
\pgfpathlineto{\pgfqpoint{1.170566in}{6.285661in}}%
\pgfpathlineto{\pgfqpoint{1.170566in}{6.373396in}}%
\pgfusepath{stroke,fill}%
\end{pgfscope}%
\begin{pgfscope}%
\pgfpathrectangle{\pgfqpoint{0.380943in}{6.110189in}}{\pgfqpoint{4.650000in}{0.614151in}}%
\pgfusepath{clip}%
\pgfsetbuttcap%
\pgfsetroundjoin%
\definecolor{currentfill}{rgb}{0.992326,0.765229,0.614840}%
\pgfsetfillcolor{currentfill}%
\pgfsetlinewidth{0.250937pt}%
\definecolor{currentstroke}{rgb}{1.000000,1.000000,1.000000}%
\pgfsetstrokecolor{currentstroke}%
\pgfsetdash{}{0pt}%
\pgfpathmoveto{\pgfqpoint{1.258302in}{6.373396in}}%
\pgfpathlineto{\pgfqpoint{1.346037in}{6.373396in}}%
\pgfpathlineto{\pgfqpoint{1.346037in}{6.285661in}}%
\pgfpathlineto{\pgfqpoint{1.258302in}{6.285661in}}%
\pgfpathlineto{\pgfqpoint{1.258302in}{6.373396in}}%
\pgfusepath{stroke,fill}%
\end{pgfscope}%
\begin{pgfscope}%
\pgfpathrectangle{\pgfqpoint{0.380943in}{6.110189in}}{\pgfqpoint{4.650000in}{0.614151in}}%
\pgfusepath{clip}%
\pgfsetbuttcap%
\pgfsetroundjoin%
\definecolor{currentfill}{rgb}{0.992326,0.765229,0.614840}%
\pgfsetfillcolor{currentfill}%
\pgfsetlinewidth{0.250937pt}%
\definecolor{currentstroke}{rgb}{1.000000,1.000000,1.000000}%
\pgfsetstrokecolor{currentstroke}%
\pgfsetdash{}{0pt}%
\pgfpathmoveto{\pgfqpoint{1.346037in}{6.373396in}}%
\pgfpathlineto{\pgfqpoint{1.433773in}{6.373396in}}%
\pgfpathlineto{\pgfqpoint{1.433773in}{6.285661in}}%
\pgfpathlineto{\pgfqpoint{1.346037in}{6.285661in}}%
\pgfpathlineto{\pgfqpoint{1.346037in}{6.373396in}}%
\pgfusepath{stroke,fill}%
\end{pgfscope}%
\begin{pgfscope}%
\pgfpathrectangle{\pgfqpoint{0.380943in}{6.110189in}}{\pgfqpoint{4.650000in}{0.614151in}}%
\pgfusepath{clip}%
\pgfsetbuttcap%
\pgfsetroundjoin%
\definecolor{currentfill}{rgb}{0.998939,0.658962,0.556032}%
\pgfsetfillcolor{currentfill}%
\pgfsetlinewidth{0.250937pt}%
\definecolor{currentstroke}{rgb}{1.000000,1.000000,1.000000}%
\pgfsetstrokecolor{currentstroke}%
\pgfsetdash{}{0pt}%
\pgfpathmoveto{\pgfqpoint{1.433773in}{6.373396in}}%
\pgfpathlineto{\pgfqpoint{1.521509in}{6.373396in}}%
\pgfpathlineto{\pgfqpoint{1.521509in}{6.285661in}}%
\pgfpathlineto{\pgfqpoint{1.433773in}{6.285661in}}%
\pgfpathlineto{\pgfqpoint{1.433773in}{6.373396in}}%
\pgfusepath{stroke,fill}%
\end{pgfscope}%
\begin{pgfscope}%
\pgfpathrectangle{\pgfqpoint{0.380943in}{6.110189in}}{\pgfqpoint{4.650000in}{0.614151in}}%
\pgfusepath{clip}%
\pgfsetbuttcap%
\pgfsetroundjoin%
\definecolor{currentfill}{rgb}{0.986759,0.806398,0.641200}%
\pgfsetfillcolor{currentfill}%
\pgfsetlinewidth{0.250937pt}%
\definecolor{currentstroke}{rgb}{1.000000,1.000000,1.000000}%
\pgfsetstrokecolor{currentstroke}%
\pgfsetdash{}{0pt}%
\pgfpathmoveto{\pgfqpoint{1.521509in}{6.373396in}}%
\pgfpathlineto{\pgfqpoint{1.609245in}{6.373396in}}%
\pgfpathlineto{\pgfqpoint{1.609245in}{6.285661in}}%
\pgfpathlineto{\pgfqpoint{1.521509in}{6.285661in}}%
\pgfpathlineto{\pgfqpoint{1.521509in}{6.373396in}}%
\pgfusepath{stroke,fill}%
\end{pgfscope}%
\begin{pgfscope}%
\pgfpathrectangle{\pgfqpoint{0.380943in}{6.110189in}}{\pgfqpoint{4.650000in}{0.614151in}}%
\pgfusepath{clip}%
\pgfsetbuttcap%
\pgfsetroundjoin%
\definecolor{currentfill}{rgb}{0.996571,0.720538,0.589189}%
\pgfsetfillcolor{currentfill}%
\pgfsetlinewidth{0.250937pt}%
\definecolor{currentstroke}{rgb}{1.000000,1.000000,1.000000}%
\pgfsetstrokecolor{currentstroke}%
\pgfsetdash{}{0pt}%
\pgfpathmoveto{\pgfqpoint{1.609245in}{6.373396in}}%
\pgfpathlineto{\pgfqpoint{1.696981in}{6.373396in}}%
\pgfpathlineto{\pgfqpoint{1.696981in}{6.285661in}}%
\pgfpathlineto{\pgfqpoint{1.609245in}{6.285661in}}%
\pgfpathlineto{\pgfqpoint{1.609245in}{6.373396in}}%
\pgfusepath{stroke,fill}%
\end{pgfscope}%
\begin{pgfscope}%
\pgfpathrectangle{\pgfqpoint{0.380943in}{6.110189in}}{\pgfqpoint{4.650000in}{0.614151in}}%
\pgfusepath{clip}%
\pgfsetbuttcap%
\pgfsetroundjoin%
\definecolor{currentfill}{rgb}{0.986759,0.806398,0.641200}%
\pgfsetfillcolor{currentfill}%
\pgfsetlinewidth{0.250937pt}%
\definecolor{currentstroke}{rgb}{1.000000,1.000000,1.000000}%
\pgfsetstrokecolor{currentstroke}%
\pgfsetdash{}{0pt}%
\pgfpathmoveto{\pgfqpoint{1.696981in}{6.373396in}}%
\pgfpathlineto{\pgfqpoint{1.784717in}{6.373396in}}%
\pgfpathlineto{\pgfqpoint{1.784717in}{6.285661in}}%
\pgfpathlineto{\pgfqpoint{1.696981in}{6.285661in}}%
\pgfpathlineto{\pgfqpoint{1.696981in}{6.373396in}}%
\pgfusepath{stroke,fill}%
\end{pgfscope}%
\begin{pgfscope}%
\pgfpathrectangle{\pgfqpoint{0.380943in}{6.110189in}}{\pgfqpoint{4.650000in}{0.614151in}}%
\pgfusepath{clip}%
\pgfsetbuttcap%
\pgfsetroundjoin%
\definecolor{currentfill}{rgb}{0.962414,0.923552,0.722891}%
\pgfsetfillcolor{currentfill}%
\pgfsetlinewidth{0.250937pt}%
\definecolor{currentstroke}{rgb}{1.000000,1.000000,1.000000}%
\pgfsetstrokecolor{currentstroke}%
\pgfsetdash{}{0pt}%
\pgfpathmoveto{\pgfqpoint{1.784717in}{6.373396in}}%
\pgfpathlineto{\pgfqpoint{1.872452in}{6.373396in}}%
\pgfpathlineto{\pgfqpoint{1.872452in}{6.285661in}}%
\pgfpathlineto{\pgfqpoint{1.784717in}{6.285661in}}%
\pgfpathlineto{\pgfqpoint{1.784717in}{6.373396in}}%
\pgfusepath{stroke,fill}%
\end{pgfscope}%
\begin{pgfscope}%
\pgfpathrectangle{\pgfqpoint{0.380943in}{6.110189in}}{\pgfqpoint{4.650000in}{0.614151in}}%
\pgfusepath{clip}%
\pgfsetbuttcap%
\pgfsetroundjoin%
\definecolor{currentfill}{rgb}{0.986759,0.806398,0.641200}%
\pgfsetfillcolor{currentfill}%
\pgfsetlinewidth{0.250937pt}%
\definecolor{currentstroke}{rgb}{1.000000,1.000000,1.000000}%
\pgfsetstrokecolor{currentstroke}%
\pgfsetdash{}{0pt}%
\pgfpathmoveto{\pgfqpoint{1.872452in}{6.373396in}}%
\pgfpathlineto{\pgfqpoint{1.960188in}{6.373396in}}%
\pgfpathlineto{\pgfqpoint{1.960188in}{6.285661in}}%
\pgfpathlineto{\pgfqpoint{1.872452in}{6.285661in}}%
\pgfpathlineto{\pgfqpoint{1.872452in}{6.373396in}}%
\pgfusepath{stroke,fill}%
\end{pgfscope}%
\begin{pgfscope}%
\pgfpathrectangle{\pgfqpoint{0.380943in}{6.110189in}}{\pgfqpoint{4.650000in}{0.614151in}}%
\pgfusepath{clip}%
\pgfsetbuttcap%
\pgfsetroundjoin%
\definecolor{currentfill}{rgb}{0.986759,0.806398,0.641200}%
\pgfsetfillcolor{currentfill}%
\pgfsetlinewidth{0.250937pt}%
\definecolor{currentstroke}{rgb}{1.000000,1.000000,1.000000}%
\pgfsetstrokecolor{currentstroke}%
\pgfsetdash{}{0pt}%
\pgfpathmoveto{\pgfqpoint{1.960188in}{6.373396in}}%
\pgfpathlineto{\pgfqpoint{2.047924in}{6.373396in}}%
\pgfpathlineto{\pgfqpoint{2.047924in}{6.285661in}}%
\pgfpathlineto{\pgfqpoint{1.960188in}{6.285661in}}%
\pgfpathlineto{\pgfqpoint{1.960188in}{6.373396in}}%
\pgfusepath{stroke,fill}%
\end{pgfscope}%
\begin{pgfscope}%
\pgfpathrectangle{\pgfqpoint{0.380943in}{6.110189in}}{\pgfqpoint{4.650000in}{0.614151in}}%
\pgfusepath{clip}%
\pgfsetbuttcap%
\pgfsetroundjoin%
\definecolor{currentfill}{rgb}{0.972549,0.870588,0.692810}%
\pgfsetfillcolor{currentfill}%
\pgfsetlinewidth{0.250937pt}%
\definecolor{currentstroke}{rgb}{1.000000,1.000000,1.000000}%
\pgfsetstrokecolor{currentstroke}%
\pgfsetdash{}{0pt}%
\pgfpathmoveto{\pgfqpoint{2.047924in}{6.373396in}}%
\pgfpathlineto{\pgfqpoint{2.135660in}{6.373396in}}%
\pgfpathlineto{\pgfqpoint{2.135660in}{6.285661in}}%
\pgfpathlineto{\pgfqpoint{2.047924in}{6.285661in}}%
\pgfpathlineto{\pgfqpoint{2.047924in}{6.373396in}}%
\pgfusepath{stroke,fill}%
\end{pgfscope}%
\begin{pgfscope}%
\pgfpathrectangle{\pgfqpoint{0.380943in}{6.110189in}}{\pgfqpoint{4.650000in}{0.614151in}}%
\pgfusepath{clip}%
\pgfsetbuttcap%
\pgfsetroundjoin%
\definecolor{currentfill}{rgb}{0.996571,0.720538,0.589189}%
\pgfsetfillcolor{currentfill}%
\pgfsetlinewidth{0.250937pt}%
\definecolor{currentstroke}{rgb}{1.000000,1.000000,1.000000}%
\pgfsetstrokecolor{currentstroke}%
\pgfsetdash{}{0pt}%
\pgfpathmoveto{\pgfqpoint{2.135660in}{6.373396in}}%
\pgfpathlineto{\pgfqpoint{2.223396in}{6.373396in}}%
\pgfpathlineto{\pgfqpoint{2.223396in}{6.285661in}}%
\pgfpathlineto{\pgfqpoint{2.135660in}{6.285661in}}%
\pgfpathlineto{\pgfqpoint{2.135660in}{6.373396in}}%
\pgfusepath{stroke,fill}%
\end{pgfscope}%
\begin{pgfscope}%
\pgfpathrectangle{\pgfqpoint{0.380943in}{6.110189in}}{\pgfqpoint{4.650000in}{0.614151in}}%
\pgfusepath{clip}%
\pgfsetbuttcap%
\pgfsetroundjoin%
\definecolor{currentfill}{rgb}{0.992326,0.765229,0.614840}%
\pgfsetfillcolor{currentfill}%
\pgfsetlinewidth{0.250937pt}%
\definecolor{currentstroke}{rgb}{1.000000,1.000000,1.000000}%
\pgfsetstrokecolor{currentstroke}%
\pgfsetdash{}{0pt}%
\pgfpathmoveto{\pgfqpoint{2.223396in}{6.373396in}}%
\pgfpathlineto{\pgfqpoint{2.311132in}{6.373396in}}%
\pgfpathlineto{\pgfqpoint{2.311132in}{6.285661in}}%
\pgfpathlineto{\pgfqpoint{2.223396in}{6.285661in}}%
\pgfpathlineto{\pgfqpoint{2.223396in}{6.373396in}}%
\pgfusepath{stroke,fill}%
\end{pgfscope}%
\begin{pgfscope}%
\pgfpathrectangle{\pgfqpoint{0.380943in}{6.110189in}}{\pgfqpoint{4.650000in}{0.614151in}}%
\pgfusepath{clip}%
\pgfsetbuttcap%
\pgfsetroundjoin%
\definecolor{currentfill}{rgb}{0.986759,0.806398,0.641200}%
\pgfsetfillcolor{currentfill}%
\pgfsetlinewidth{0.250937pt}%
\definecolor{currentstroke}{rgb}{1.000000,1.000000,1.000000}%
\pgfsetstrokecolor{currentstroke}%
\pgfsetdash{}{0pt}%
\pgfpathmoveto{\pgfqpoint{2.311132in}{6.373396in}}%
\pgfpathlineto{\pgfqpoint{2.398868in}{6.373396in}}%
\pgfpathlineto{\pgfqpoint{2.398868in}{6.285661in}}%
\pgfpathlineto{\pgfqpoint{2.311132in}{6.285661in}}%
\pgfpathlineto{\pgfqpoint{2.311132in}{6.373396in}}%
\pgfusepath{stroke,fill}%
\end{pgfscope}%
\begin{pgfscope}%
\pgfpathrectangle{\pgfqpoint{0.380943in}{6.110189in}}{\pgfqpoint{4.650000in}{0.614151in}}%
\pgfusepath{clip}%
\pgfsetbuttcap%
\pgfsetroundjoin%
\definecolor{currentfill}{rgb}{0.968166,0.945882,0.748604}%
\pgfsetfillcolor{currentfill}%
\pgfsetlinewidth{0.250937pt}%
\definecolor{currentstroke}{rgb}{1.000000,1.000000,1.000000}%
\pgfsetstrokecolor{currentstroke}%
\pgfsetdash{}{0pt}%
\pgfpathmoveto{\pgfqpoint{2.398868in}{6.373396in}}%
\pgfpathlineto{\pgfqpoint{2.486603in}{6.373396in}}%
\pgfpathlineto{\pgfqpoint{2.486603in}{6.285661in}}%
\pgfpathlineto{\pgfqpoint{2.398868in}{6.285661in}}%
\pgfpathlineto{\pgfqpoint{2.398868in}{6.373396in}}%
\pgfusepath{stroke,fill}%
\end{pgfscope}%
\begin{pgfscope}%
\pgfpathrectangle{\pgfqpoint{0.380943in}{6.110189in}}{\pgfqpoint{4.650000in}{0.614151in}}%
\pgfusepath{clip}%
\pgfsetbuttcap%
\pgfsetroundjoin%
\definecolor{currentfill}{rgb}{0.965444,0.906113,0.711757}%
\pgfsetfillcolor{currentfill}%
\pgfsetlinewidth{0.250937pt}%
\definecolor{currentstroke}{rgb}{1.000000,1.000000,1.000000}%
\pgfsetstrokecolor{currentstroke}%
\pgfsetdash{}{0pt}%
\pgfpathmoveto{\pgfqpoint{2.486603in}{6.373396in}}%
\pgfpathlineto{\pgfqpoint{2.574339in}{6.373396in}}%
\pgfpathlineto{\pgfqpoint{2.574339in}{6.285661in}}%
\pgfpathlineto{\pgfqpoint{2.486603in}{6.285661in}}%
\pgfpathlineto{\pgfqpoint{2.486603in}{6.373396in}}%
\pgfusepath{stroke,fill}%
\end{pgfscope}%
\begin{pgfscope}%
\pgfpathrectangle{\pgfqpoint{0.380943in}{6.110189in}}{\pgfqpoint{4.650000in}{0.614151in}}%
\pgfusepath{clip}%
\pgfsetbuttcap%
\pgfsetroundjoin%
\definecolor{currentfill}{rgb}{0.979654,0.837186,0.669619}%
\pgfsetfillcolor{currentfill}%
\pgfsetlinewidth{0.250937pt}%
\definecolor{currentstroke}{rgb}{1.000000,1.000000,1.000000}%
\pgfsetstrokecolor{currentstroke}%
\pgfsetdash{}{0pt}%
\pgfpathmoveto{\pgfqpoint{2.574339in}{6.373396in}}%
\pgfpathlineto{\pgfqpoint{2.662075in}{6.373396in}}%
\pgfpathlineto{\pgfqpoint{2.662075in}{6.285661in}}%
\pgfpathlineto{\pgfqpoint{2.574339in}{6.285661in}}%
\pgfpathlineto{\pgfqpoint{2.574339in}{6.373396in}}%
\pgfusepath{stroke,fill}%
\end{pgfscope}%
\begin{pgfscope}%
\pgfpathrectangle{\pgfqpoint{0.380943in}{6.110189in}}{\pgfqpoint{4.650000in}{0.614151in}}%
\pgfusepath{clip}%
\pgfsetbuttcap%
\pgfsetroundjoin%
\definecolor{currentfill}{rgb}{0.962414,0.923552,0.722891}%
\pgfsetfillcolor{currentfill}%
\pgfsetlinewidth{0.250937pt}%
\definecolor{currentstroke}{rgb}{1.000000,1.000000,1.000000}%
\pgfsetstrokecolor{currentstroke}%
\pgfsetdash{}{0pt}%
\pgfpathmoveto{\pgfqpoint{2.662075in}{6.373396in}}%
\pgfpathlineto{\pgfqpoint{2.749811in}{6.373396in}}%
\pgfpathlineto{\pgfqpoint{2.749811in}{6.285661in}}%
\pgfpathlineto{\pgfqpoint{2.662075in}{6.285661in}}%
\pgfpathlineto{\pgfqpoint{2.662075in}{6.373396in}}%
\pgfusepath{stroke,fill}%
\end{pgfscope}%
\begin{pgfscope}%
\pgfpathrectangle{\pgfqpoint{0.380943in}{6.110189in}}{\pgfqpoint{4.650000in}{0.614151in}}%
\pgfusepath{clip}%
\pgfsetbuttcap%
\pgfsetroundjoin%
\definecolor{currentfill}{rgb}{0.992326,0.765229,0.614840}%
\pgfsetfillcolor{currentfill}%
\pgfsetlinewidth{0.250937pt}%
\definecolor{currentstroke}{rgb}{1.000000,1.000000,1.000000}%
\pgfsetstrokecolor{currentstroke}%
\pgfsetdash{}{0pt}%
\pgfpathmoveto{\pgfqpoint{2.749811in}{6.373396in}}%
\pgfpathlineto{\pgfqpoint{2.837547in}{6.373396in}}%
\pgfpathlineto{\pgfqpoint{2.837547in}{6.285661in}}%
\pgfpathlineto{\pgfqpoint{2.749811in}{6.285661in}}%
\pgfpathlineto{\pgfqpoint{2.749811in}{6.373396in}}%
\pgfusepath{stroke,fill}%
\end{pgfscope}%
\begin{pgfscope}%
\pgfpathrectangle{\pgfqpoint{0.380943in}{6.110189in}}{\pgfqpoint{4.650000in}{0.614151in}}%
\pgfusepath{clip}%
\pgfsetbuttcap%
\pgfsetroundjoin%
\definecolor{currentfill}{rgb}{1.000000,1.000000,0.870204}%
\pgfsetfillcolor{currentfill}%
\pgfsetlinewidth{0.250937pt}%
\definecolor{currentstroke}{rgb}{1.000000,1.000000,1.000000}%
\pgfsetstrokecolor{currentstroke}%
\pgfsetdash{}{0pt}%
\pgfpathmoveto{\pgfqpoint{2.837547in}{6.373396in}}%
\pgfpathlineto{\pgfqpoint{2.925283in}{6.373396in}}%
\pgfpathlineto{\pgfqpoint{2.925283in}{6.285661in}}%
\pgfpathlineto{\pgfqpoint{2.837547in}{6.285661in}}%
\pgfpathlineto{\pgfqpoint{2.837547in}{6.373396in}}%
\pgfusepath{stroke,fill}%
\end{pgfscope}%
\begin{pgfscope}%
\pgfpathrectangle{\pgfqpoint{0.380943in}{6.110189in}}{\pgfqpoint{4.650000in}{0.614151in}}%
\pgfusepath{clip}%
\pgfsetbuttcap%
\pgfsetroundjoin%
\definecolor{currentfill}{rgb}{0.972549,0.870588,0.692810}%
\pgfsetfillcolor{currentfill}%
\pgfsetlinewidth{0.250937pt}%
\definecolor{currentstroke}{rgb}{1.000000,1.000000,1.000000}%
\pgfsetstrokecolor{currentstroke}%
\pgfsetdash{}{0pt}%
\pgfpathmoveto{\pgfqpoint{2.925283in}{6.373396in}}%
\pgfpathlineto{\pgfqpoint{3.013019in}{6.373396in}}%
\pgfpathlineto{\pgfqpoint{3.013019in}{6.285661in}}%
\pgfpathlineto{\pgfqpoint{2.925283in}{6.285661in}}%
\pgfpathlineto{\pgfqpoint{2.925283in}{6.373396in}}%
\pgfusepath{stroke,fill}%
\end{pgfscope}%
\begin{pgfscope}%
\pgfpathrectangle{\pgfqpoint{0.380943in}{6.110189in}}{\pgfqpoint{4.650000in}{0.614151in}}%
\pgfusepath{clip}%
\pgfsetbuttcap%
\pgfsetroundjoin%
\definecolor{currentfill}{rgb}{0.962414,0.923552,0.722891}%
\pgfsetfillcolor{currentfill}%
\pgfsetlinewidth{0.250937pt}%
\definecolor{currentstroke}{rgb}{1.000000,1.000000,1.000000}%
\pgfsetstrokecolor{currentstroke}%
\pgfsetdash{}{0pt}%
\pgfpathmoveto{\pgfqpoint{3.013019in}{6.373396in}}%
\pgfpathlineto{\pgfqpoint{3.100754in}{6.373396in}}%
\pgfpathlineto{\pgfqpoint{3.100754in}{6.285661in}}%
\pgfpathlineto{\pgfqpoint{3.013019in}{6.285661in}}%
\pgfpathlineto{\pgfqpoint{3.013019in}{6.373396in}}%
\pgfusepath{stroke,fill}%
\end{pgfscope}%
\begin{pgfscope}%
\pgfpathrectangle{\pgfqpoint{0.380943in}{6.110189in}}{\pgfqpoint{4.650000in}{0.614151in}}%
\pgfusepath{clip}%
\pgfsetbuttcap%
\pgfsetroundjoin%
\definecolor{currentfill}{rgb}{0.962414,0.923552,0.722891}%
\pgfsetfillcolor{currentfill}%
\pgfsetlinewidth{0.250937pt}%
\definecolor{currentstroke}{rgb}{1.000000,1.000000,1.000000}%
\pgfsetstrokecolor{currentstroke}%
\pgfsetdash{}{0pt}%
\pgfpathmoveto{\pgfqpoint{3.100754in}{6.373396in}}%
\pgfpathlineto{\pgfqpoint{3.188490in}{6.373396in}}%
\pgfpathlineto{\pgfqpoint{3.188490in}{6.285661in}}%
\pgfpathlineto{\pgfqpoint{3.100754in}{6.285661in}}%
\pgfpathlineto{\pgfqpoint{3.100754in}{6.373396in}}%
\pgfusepath{stroke,fill}%
\end{pgfscope}%
\begin{pgfscope}%
\pgfpathrectangle{\pgfqpoint{0.380943in}{6.110189in}}{\pgfqpoint{4.650000in}{0.614151in}}%
\pgfusepath{clip}%
\pgfsetbuttcap%
\pgfsetroundjoin%
\definecolor{currentfill}{rgb}{0.986759,0.806398,0.641200}%
\pgfsetfillcolor{currentfill}%
\pgfsetlinewidth{0.250937pt}%
\definecolor{currentstroke}{rgb}{1.000000,1.000000,1.000000}%
\pgfsetstrokecolor{currentstroke}%
\pgfsetdash{}{0pt}%
\pgfpathmoveto{\pgfqpoint{3.188490in}{6.373396in}}%
\pgfpathlineto{\pgfqpoint{3.276226in}{6.373396in}}%
\pgfpathlineto{\pgfqpoint{3.276226in}{6.285661in}}%
\pgfpathlineto{\pgfqpoint{3.188490in}{6.285661in}}%
\pgfpathlineto{\pgfqpoint{3.188490in}{6.373396in}}%
\pgfusepath{stroke,fill}%
\end{pgfscope}%
\begin{pgfscope}%
\pgfpathrectangle{\pgfqpoint{0.380943in}{6.110189in}}{\pgfqpoint{4.650000in}{0.614151in}}%
\pgfusepath{clip}%
\pgfsetbuttcap%
\pgfsetroundjoin%
\definecolor{currentfill}{rgb}{0.962414,0.923552,0.722891}%
\pgfsetfillcolor{currentfill}%
\pgfsetlinewidth{0.250937pt}%
\definecolor{currentstroke}{rgb}{1.000000,1.000000,1.000000}%
\pgfsetstrokecolor{currentstroke}%
\pgfsetdash{}{0pt}%
\pgfpathmoveto{\pgfqpoint{3.276226in}{6.373396in}}%
\pgfpathlineto{\pgfqpoint{3.363962in}{6.373396in}}%
\pgfpathlineto{\pgfqpoint{3.363962in}{6.285661in}}%
\pgfpathlineto{\pgfqpoint{3.276226in}{6.285661in}}%
\pgfpathlineto{\pgfqpoint{3.276226in}{6.373396in}}%
\pgfusepath{stroke,fill}%
\end{pgfscope}%
\begin{pgfscope}%
\pgfpathrectangle{\pgfqpoint{0.380943in}{6.110189in}}{\pgfqpoint{4.650000in}{0.614151in}}%
\pgfusepath{clip}%
\pgfsetbuttcap%
\pgfsetroundjoin%
\definecolor{currentfill}{rgb}{0.979654,0.837186,0.669619}%
\pgfsetfillcolor{currentfill}%
\pgfsetlinewidth{0.250937pt}%
\definecolor{currentstroke}{rgb}{1.000000,1.000000,1.000000}%
\pgfsetstrokecolor{currentstroke}%
\pgfsetdash{}{0pt}%
\pgfpathmoveto{\pgfqpoint{3.363962in}{6.373396in}}%
\pgfpathlineto{\pgfqpoint{3.451698in}{6.373396in}}%
\pgfpathlineto{\pgfqpoint{3.451698in}{6.285661in}}%
\pgfpathlineto{\pgfqpoint{3.363962in}{6.285661in}}%
\pgfpathlineto{\pgfqpoint{3.363962in}{6.373396in}}%
\pgfusepath{stroke,fill}%
\end{pgfscope}%
\begin{pgfscope}%
\pgfpathrectangle{\pgfqpoint{0.380943in}{6.110189in}}{\pgfqpoint{4.650000in}{0.614151in}}%
\pgfusepath{clip}%
\pgfsetbuttcap%
\pgfsetroundjoin%
\definecolor{currentfill}{rgb}{0.962414,0.923552,0.722891}%
\pgfsetfillcolor{currentfill}%
\pgfsetlinewidth{0.250937pt}%
\definecolor{currentstroke}{rgb}{1.000000,1.000000,1.000000}%
\pgfsetstrokecolor{currentstroke}%
\pgfsetdash{}{0pt}%
\pgfpathmoveto{\pgfqpoint{3.451698in}{6.373396in}}%
\pgfpathlineto{\pgfqpoint{3.539434in}{6.373396in}}%
\pgfpathlineto{\pgfqpoint{3.539434in}{6.285661in}}%
\pgfpathlineto{\pgfqpoint{3.451698in}{6.285661in}}%
\pgfpathlineto{\pgfqpoint{3.451698in}{6.373396in}}%
\pgfusepath{stroke,fill}%
\end{pgfscope}%
\begin{pgfscope}%
\pgfpathrectangle{\pgfqpoint{0.380943in}{6.110189in}}{\pgfqpoint{4.650000in}{0.614151in}}%
\pgfusepath{clip}%
\pgfsetbuttcap%
\pgfsetroundjoin%
\definecolor{currentfill}{rgb}{0.992326,0.765229,0.614840}%
\pgfsetfillcolor{currentfill}%
\pgfsetlinewidth{0.250937pt}%
\definecolor{currentstroke}{rgb}{1.000000,1.000000,1.000000}%
\pgfsetstrokecolor{currentstroke}%
\pgfsetdash{}{0pt}%
\pgfpathmoveto{\pgfqpoint{3.539434in}{6.373396in}}%
\pgfpathlineto{\pgfqpoint{3.627169in}{6.373396in}}%
\pgfpathlineto{\pgfqpoint{3.627169in}{6.285661in}}%
\pgfpathlineto{\pgfqpoint{3.539434in}{6.285661in}}%
\pgfpathlineto{\pgfqpoint{3.539434in}{6.373396in}}%
\pgfusepath{stroke,fill}%
\end{pgfscope}%
\begin{pgfscope}%
\pgfpathrectangle{\pgfqpoint{0.380943in}{6.110189in}}{\pgfqpoint{4.650000in}{0.614151in}}%
\pgfusepath{clip}%
\pgfsetbuttcap%
\pgfsetroundjoin%
\definecolor{currentfill}{rgb}{0.972549,0.870588,0.692810}%
\pgfsetfillcolor{currentfill}%
\pgfsetlinewidth{0.250937pt}%
\definecolor{currentstroke}{rgb}{1.000000,1.000000,1.000000}%
\pgfsetstrokecolor{currentstroke}%
\pgfsetdash{}{0pt}%
\pgfpathmoveto{\pgfqpoint{3.627169in}{6.373396in}}%
\pgfpathlineto{\pgfqpoint{3.714905in}{6.373396in}}%
\pgfpathlineto{\pgfqpoint{3.714905in}{6.285661in}}%
\pgfpathlineto{\pgfqpoint{3.627169in}{6.285661in}}%
\pgfpathlineto{\pgfqpoint{3.627169in}{6.373396in}}%
\pgfusepath{stroke,fill}%
\end{pgfscope}%
\begin{pgfscope}%
\pgfpathrectangle{\pgfqpoint{0.380943in}{6.110189in}}{\pgfqpoint{4.650000in}{0.614151in}}%
\pgfusepath{clip}%
\pgfsetbuttcap%
\pgfsetroundjoin%
\definecolor{currentfill}{rgb}{1.000000,0.605229,0.530719}%
\pgfsetfillcolor{currentfill}%
\pgfsetlinewidth{0.250937pt}%
\definecolor{currentstroke}{rgb}{1.000000,1.000000,1.000000}%
\pgfsetstrokecolor{currentstroke}%
\pgfsetdash{}{0pt}%
\pgfpathmoveto{\pgfqpoint{3.714905in}{6.373396in}}%
\pgfpathlineto{\pgfqpoint{3.802641in}{6.373396in}}%
\pgfpathlineto{\pgfqpoint{3.802641in}{6.285661in}}%
\pgfpathlineto{\pgfqpoint{3.714905in}{6.285661in}}%
\pgfpathlineto{\pgfqpoint{3.714905in}{6.373396in}}%
\pgfusepath{stroke,fill}%
\end{pgfscope}%
\begin{pgfscope}%
\pgfpathrectangle{\pgfqpoint{0.380943in}{6.110189in}}{\pgfqpoint{4.650000in}{0.614151in}}%
\pgfusepath{clip}%
\pgfsetbuttcap%
\pgfsetroundjoin%
\definecolor{currentfill}{rgb}{0.998939,0.658962,0.556032}%
\pgfsetfillcolor{currentfill}%
\pgfsetlinewidth{0.250937pt}%
\definecolor{currentstroke}{rgb}{1.000000,1.000000,1.000000}%
\pgfsetstrokecolor{currentstroke}%
\pgfsetdash{}{0pt}%
\pgfpathmoveto{\pgfqpoint{3.802641in}{6.373396in}}%
\pgfpathlineto{\pgfqpoint{3.890377in}{6.373396in}}%
\pgfpathlineto{\pgfqpoint{3.890377in}{6.285661in}}%
\pgfpathlineto{\pgfqpoint{3.802641in}{6.285661in}}%
\pgfpathlineto{\pgfqpoint{3.802641in}{6.373396in}}%
\pgfusepath{stroke,fill}%
\end{pgfscope}%
\begin{pgfscope}%
\pgfpathrectangle{\pgfqpoint{0.380943in}{6.110189in}}{\pgfqpoint{4.650000in}{0.614151in}}%
\pgfusepath{clip}%
\pgfsetbuttcap%
\pgfsetroundjoin%
\definecolor{currentfill}{rgb}{0.992326,0.765229,0.614840}%
\pgfsetfillcolor{currentfill}%
\pgfsetlinewidth{0.250937pt}%
\definecolor{currentstroke}{rgb}{1.000000,1.000000,1.000000}%
\pgfsetstrokecolor{currentstroke}%
\pgfsetdash{}{0pt}%
\pgfpathmoveto{\pgfqpoint{3.890377in}{6.373396in}}%
\pgfpathlineto{\pgfqpoint{3.978113in}{6.373396in}}%
\pgfpathlineto{\pgfqpoint{3.978113in}{6.285661in}}%
\pgfpathlineto{\pgfqpoint{3.890377in}{6.285661in}}%
\pgfpathlineto{\pgfqpoint{3.890377in}{6.373396in}}%
\pgfusepath{stroke,fill}%
\end{pgfscope}%
\begin{pgfscope}%
\pgfpathrectangle{\pgfqpoint{0.380943in}{6.110189in}}{\pgfqpoint{4.650000in}{0.614151in}}%
\pgfusepath{clip}%
\pgfsetbuttcap%
\pgfsetroundjoin%
\definecolor{currentfill}{rgb}{0.996571,0.720538,0.589189}%
\pgfsetfillcolor{currentfill}%
\pgfsetlinewidth{0.250937pt}%
\definecolor{currentstroke}{rgb}{1.000000,1.000000,1.000000}%
\pgfsetstrokecolor{currentstroke}%
\pgfsetdash{}{0pt}%
\pgfpathmoveto{\pgfqpoint{3.978113in}{6.373396in}}%
\pgfpathlineto{\pgfqpoint{4.065849in}{6.373396in}}%
\pgfpathlineto{\pgfqpoint{4.065849in}{6.285661in}}%
\pgfpathlineto{\pgfqpoint{3.978113in}{6.285661in}}%
\pgfpathlineto{\pgfqpoint{3.978113in}{6.373396in}}%
\pgfusepath{stroke,fill}%
\end{pgfscope}%
\begin{pgfscope}%
\pgfpathrectangle{\pgfqpoint{0.380943in}{6.110189in}}{\pgfqpoint{4.650000in}{0.614151in}}%
\pgfusepath{clip}%
\pgfsetbuttcap%
\pgfsetroundjoin%
\definecolor{currentfill}{rgb}{0.996571,0.720538,0.589189}%
\pgfsetfillcolor{currentfill}%
\pgfsetlinewidth{0.250937pt}%
\definecolor{currentstroke}{rgb}{1.000000,1.000000,1.000000}%
\pgfsetstrokecolor{currentstroke}%
\pgfsetdash{}{0pt}%
\pgfpathmoveto{\pgfqpoint{4.065849in}{6.373396in}}%
\pgfpathlineto{\pgfqpoint{4.153585in}{6.373396in}}%
\pgfpathlineto{\pgfqpoint{4.153585in}{6.285661in}}%
\pgfpathlineto{\pgfqpoint{4.065849in}{6.285661in}}%
\pgfpathlineto{\pgfqpoint{4.065849in}{6.373396in}}%
\pgfusepath{stroke,fill}%
\end{pgfscope}%
\begin{pgfscope}%
\pgfpathrectangle{\pgfqpoint{0.380943in}{6.110189in}}{\pgfqpoint{4.650000in}{0.614151in}}%
\pgfusepath{clip}%
\pgfsetbuttcap%
\pgfsetroundjoin%
\definecolor{currentfill}{rgb}{0.979654,0.837186,0.669619}%
\pgfsetfillcolor{currentfill}%
\pgfsetlinewidth{0.250937pt}%
\definecolor{currentstroke}{rgb}{1.000000,1.000000,1.000000}%
\pgfsetstrokecolor{currentstroke}%
\pgfsetdash{}{0pt}%
\pgfpathmoveto{\pgfqpoint{4.153585in}{6.373396in}}%
\pgfpathlineto{\pgfqpoint{4.241320in}{6.373396in}}%
\pgfpathlineto{\pgfqpoint{4.241320in}{6.285661in}}%
\pgfpathlineto{\pgfqpoint{4.153585in}{6.285661in}}%
\pgfpathlineto{\pgfqpoint{4.153585in}{6.373396in}}%
\pgfusepath{stroke,fill}%
\end{pgfscope}%
\begin{pgfscope}%
\pgfpathrectangle{\pgfqpoint{0.380943in}{6.110189in}}{\pgfqpoint{4.650000in}{0.614151in}}%
\pgfusepath{clip}%
\pgfsetbuttcap%
\pgfsetroundjoin%
\definecolor{currentfill}{rgb}{0.986759,0.806398,0.641200}%
\pgfsetfillcolor{currentfill}%
\pgfsetlinewidth{0.250937pt}%
\definecolor{currentstroke}{rgb}{1.000000,1.000000,1.000000}%
\pgfsetstrokecolor{currentstroke}%
\pgfsetdash{}{0pt}%
\pgfpathmoveto{\pgfqpoint{4.241320in}{6.373396in}}%
\pgfpathlineto{\pgfqpoint{4.329056in}{6.373396in}}%
\pgfpathlineto{\pgfqpoint{4.329056in}{6.285661in}}%
\pgfpathlineto{\pgfqpoint{4.241320in}{6.285661in}}%
\pgfpathlineto{\pgfqpoint{4.241320in}{6.373396in}}%
\pgfusepath{stroke,fill}%
\end{pgfscope}%
\begin{pgfscope}%
\pgfpathrectangle{\pgfqpoint{0.380943in}{6.110189in}}{\pgfqpoint{4.650000in}{0.614151in}}%
\pgfusepath{clip}%
\pgfsetbuttcap%
\pgfsetroundjoin%
\definecolor{currentfill}{rgb}{0.968166,0.945882,0.748604}%
\pgfsetfillcolor{currentfill}%
\pgfsetlinewidth{0.250937pt}%
\definecolor{currentstroke}{rgb}{1.000000,1.000000,1.000000}%
\pgfsetstrokecolor{currentstroke}%
\pgfsetdash{}{0pt}%
\pgfpathmoveto{\pgfqpoint{4.329056in}{6.373396in}}%
\pgfpathlineto{\pgfqpoint{4.416792in}{6.373396in}}%
\pgfpathlineto{\pgfqpoint{4.416792in}{6.285661in}}%
\pgfpathlineto{\pgfqpoint{4.329056in}{6.285661in}}%
\pgfpathlineto{\pgfqpoint{4.329056in}{6.373396in}}%
\pgfusepath{stroke,fill}%
\end{pgfscope}%
\begin{pgfscope}%
\pgfpathrectangle{\pgfqpoint{0.380943in}{6.110189in}}{\pgfqpoint{4.650000in}{0.614151in}}%
\pgfusepath{clip}%
\pgfsetbuttcap%
\pgfsetroundjoin%
\definecolor{currentfill}{rgb}{0.968166,0.945882,0.748604}%
\pgfsetfillcolor{currentfill}%
\pgfsetlinewidth{0.250937pt}%
\definecolor{currentstroke}{rgb}{1.000000,1.000000,1.000000}%
\pgfsetstrokecolor{currentstroke}%
\pgfsetdash{}{0pt}%
\pgfpathmoveto{\pgfqpoint{4.416792in}{6.373396in}}%
\pgfpathlineto{\pgfqpoint{4.504528in}{6.373396in}}%
\pgfpathlineto{\pgfqpoint{4.504528in}{6.285661in}}%
\pgfpathlineto{\pgfqpoint{4.416792in}{6.285661in}}%
\pgfpathlineto{\pgfqpoint{4.416792in}{6.373396in}}%
\pgfusepath{stroke,fill}%
\end{pgfscope}%
\begin{pgfscope}%
\pgfpathrectangle{\pgfqpoint{0.380943in}{6.110189in}}{\pgfqpoint{4.650000in}{0.614151in}}%
\pgfusepath{clip}%
\pgfsetbuttcap%
\pgfsetroundjoin%
\definecolor{currentfill}{rgb}{0.986759,0.806398,0.641200}%
\pgfsetfillcolor{currentfill}%
\pgfsetlinewidth{0.250937pt}%
\definecolor{currentstroke}{rgb}{1.000000,1.000000,1.000000}%
\pgfsetstrokecolor{currentstroke}%
\pgfsetdash{}{0pt}%
\pgfpathmoveto{\pgfqpoint{4.504528in}{6.373396in}}%
\pgfpathlineto{\pgfqpoint{4.592264in}{6.373396in}}%
\pgfpathlineto{\pgfqpoint{4.592264in}{6.285661in}}%
\pgfpathlineto{\pgfqpoint{4.504528in}{6.285661in}}%
\pgfpathlineto{\pgfqpoint{4.504528in}{6.373396in}}%
\pgfusepath{stroke,fill}%
\end{pgfscope}%
\begin{pgfscope}%
\pgfpathrectangle{\pgfqpoint{0.380943in}{6.110189in}}{\pgfqpoint{4.650000in}{0.614151in}}%
\pgfusepath{clip}%
\pgfsetbuttcap%
\pgfsetroundjoin%
\definecolor{currentfill}{rgb}{0.972549,0.870588,0.692810}%
\pgfsetfillcolor{currentfill}%
\pgfsetlinewidth{0.250937pt}%
\definecolor{currentstroke}{rgb}{1.000000,1.000000,1.000000}%
\pgfsetstrokecolor{currentstroke}%
\pgfsetdash{}{0pt}%
\pgfpathmoveto{\pgfqpoint{4.592264in}{6.373396in}}%
\pgfpathlineto{\pgfqpoint{4.680000in}{6.373396in}}%
\pgfpathlineto{\pgfqpoint{4.680000in}{6.285661in}}%
\pgfpathlineto{\pgfqpoint{4.592264in}{6.285661in}}%
\pgfpathlineto{\pgfqpoint{4.592264in}{6.373396in}}%
\pgfusepath{stroke,fill}%
\end{pgfscope}%
\begin{pgfscope}%
\pgfpathrectangle{\pgfqpoint{0.380943in}{6.110189in}}{\pgfqpoint{4.650000in}{0.614151in}}%
\pgfusepath{clip}%
\pgfsetbuttcap%
\pgfsetroundjoin%
\definecolor{currentfill}{rgb}{0.965444,0.906113,0.711757}%
\pgfsetfillcolor{currentfill}%
\pgfsetlinewidth{0.250937pt}%
\definecolor{currentstroke}{rgb}{1.000000,1.000000,1.000000}%
\pgfsetstrokecolor{currentstroke}%
\pgfsetdash{}{0pt}%
\pgfpathmoveto{\pgfqpoint{4.680000in}{6.373396in}}%
\pgfpathlineto{\pgfqpoint{4.767736in}{6.373396in}}%
\pgfpathlineto{\pgfqpoint{4.767736in}{6.285661in}}%
\pgfpathlineto{\pgfqpoint{4.680000in}{6.285661in}}%
\pgfpathlineto{\pgfqpoint{4.680000in}{6.373396in}}%
\pgfusepath{stroke,fill}%
\end{pgfscope}%
\begin{pgfscope}%
\pgfpathrectangle{\pgfqpoint{0.380943in}{6.110189in}}{\pgfqpoint{4.650000in}{0.614151in}}%
\pgfusepath{clip}%
\pgfsetbuttcap%
\pgfsetroundjoin%
\definecolor{currentfill}{rgb}{0.986759,0.806398,0.641200}%
\pgfsetfillcolor{currentfill}%
\pgfsetlinewidth{0.250937pt}%
\definecolor{currentstroke}{rgb}{1.000000,1.000000,1.000000}%
\pgfsetstrokecolor{currentstroke}%
\pgfsetdash{}{0pt}%
\pgfpathmoveto{\pgfqpoint{4.767736in}{6.373396in}}%
\pgfpathlineto{\pgfqpoint{4.855471in}{6.373396in}}%
\pgfpathlineto{\pgfqpoint{4.855471in}{6.285661in}}%
\pgfpathlineto{\pgfqpoint{4.767736in}{6.285661in}}%
\pgfpathlineto{\pgfqpoint{4.767736in}{6.373396in}}%
\pgfusepath{stroke,fill}%
\end{pgfscope}%
\begin{pgfscope}%
\pgfpathrectangle{\pgfqpoint{0.380943in}{6.110189in}}{\pgfqpoint{4.650000in}{0.614151in}}%
\pgfusepath{clip}%
\pgfsetbuttcap%
\pgfsetroundjoin%
\definecolor{currentfill}{rgb}{0.962414,0.923552,0.722891}%
\pgfsetfillcolor{currentfill}%
\pgfsetlinewidth{0.250937pt}%
\definecolor{currentstroke}{rgb}{1.000000,1.000000,1.000000}%
\pgfsetstrokecolor{currentstroke}%
\pgfsetdash{}{0pt}%
\pgfpathmoveto{\pgfqpoint{4.855471in}{6.373396in}}%
\pgfpathlineto{\pgfqpoint{4.943207in}{6.373396in}}%
\pgfpathlineto{\pgfqpoint{4.943207in}{6.285661in}}%
\pgfpathlineto{\pgfqpoint{4.855471in}{6.285661in}}%
\pgfpathlineto{\pgfqpoint{4.855471in}{6.373396in}}%
\pgfusepath{stroke,fill}%
\end{pgfscope}%
\begin{pgfscope}%
\pgfpathrectangle{\pgfqpoint{0.380943in}{6.110189in}}{\pgfqpoint{4.650000in}{0.614151in}}%
\pgfusepath{clip}%
\pgfsetbuttcap%
\pgfsetroundjoin%
\pgfsetlinewidth{0.250937pt}%
\definecolor{currentstroke}{rgb}{1.000000,1.000000,1.000000}%
\pgfsetstrokecolor{currentstroke}%
\pgfsetdash{}{0pt}%
\pgfpathmoveto{\pgfqpoint{4.943207in}{6.373396in}}%
\pgfpathlineto{\pgfqpoint{5.030943in}{6.373396in}}%
\pgfpathlineto{\pgfqpoint{5.030943in}{6.285661in}}%
\pgfpathlineto{\pgfqpoint{4.943207in}{6.285661in}}%
\pgfpathlineto{\pgfqpoint{4.943207in}{6.373396in}}%
\pgfusepath{stroke}%
\end{pgfscope}%
\begin{pgfscope}%
\pgfpathrectangle{\pgfqpoint{0.380943in}{6.110189in}}{\pgfqpoint{4.650000in}{0.614151in}}%
\pgfusepath{clip}%
\pgfsetbuttcap%
\pgfsetroundjoin%
\definecolor{currentfill}{rgb}{0.972549,0.870588,0.692810}%
\pgfsetfillcolor{currentfill}%
\pgfsetlinewidth{0.250937pt}%
\definecolor{currentstroke}{rgb}{1.000000,1.000000,1.000000}%
\pgfsetstrokecolor{currentstroke}%
\pgfsetdash{}{0pt}%
\pgfpathmoveto{\pgfqpoint{0.380943in}{6.285661in}}%
\pgfpathlineto{\pgfqpoint{0.468679in}{6.285661in}}%
\pgfpathlineto{\pgfqpoint{0.468679in}{6.197925in}}%
\pgfpathlineto{\pgfqpoint{0.380943in}{6.197925in}}%
\pgfpathlineto{\pgfqpoint{0.380943in}{6.285661in}}%
\pgfusepath{stroke,fill}%
\end{pgfscope}%
\begin{pgfscope}%
\pgfpathrectangle{\pgfqpoint{0.380943in}{6.110189in}}{\pgfqpoint{4.650000in}{0.614151in}}%
\pgfusepath{clip}%
\pgfsetbuttcap%
\pgfsetroundjoin%
\definecolor{currentfill}{rgb}{0.965444,0.906113,0.711757}%
\pgfsetfillcolor{currentfill}%
\pgfsetlinewidth{0.250937pt}%
\definecolor{currentstroke}{rgb}{1.000000,1.000000,1.000000}%
\pgfsetstrokecolor{currentstroke}%
\pgfsetdash{}{0pt}%
\pgfpathmoveto{\pgfqpoint{0.468679in}{6.285661in}}%
\pgfpathlineto{\pgfqpoint{0.556415in}{6.285661in}}%
\pgfpathlineto{\pgfqpoint{0.556415in}{6.197925in}}%
\pgfpathlineto{\pgfqpoint{0.468679in}{6.197925in}}%
\pgfpathlineto{\pgfqpoint{0.468679in}{6.285661in}}%
\pgfusepath{stroke,fill}%
\end{pgfscope}%
\begin{pgfscope}%
\pgfpathrectangle{\pgfqpoint{0.380943in}{6.110189in}}{\pgfqpoint{4.650000in}{0.614151in}}%
\pgfusepath{clip}%
\pgfsetbuttcap%
\pgfsetroundjoin%
\definecolor{currentfill}{rgb}{0.991849,0.986144,0.810181}%
\pgfsetfillcolor{currentfill}%
\pgfsetlinewidth{0.250937pt}%
\definecolor{currentstroke}{rgb}{1.000000,1.000000,1.000000}%
\pgfsetstrokecolor{currentstroke}%
\pgfsetdash{}{0pt}%
\pgfpathmoveto{\pgfqpoint{0.556415in}{6.285661in}}%
\pgfpathlineto{\pgfqpoint{0.644151in}{6.285661in}}%
\pgfpathlineto{\pgfqpoint{0.644151in}{6.197925in}}%
\pgfpathlineto{\pgfqpoint{0.556415in}{6.197925in}}%
\pgfpathlineto{\pgfqpoint{0.556415in}{6.285661in}}%
\pgfusepath{stroke,fill}%
\end{pgfscope}%
\begin{pgfscope}%
\pgfpathrectangle{\pgfqpoint{0.380943in}{6.110189in}}{\pgfqpoint{4.650000in}{0.614151in}}%
\pgfusepath{clip}%
\pgfsetbuttcap%
\pgfsetroundjoin%
\definecolor{currentfill}{rgb}{0.965444,0.906113,0.711757}%
\pgfsetfillcolor{currentfill}%
\pgfsetlinewidth{0.250937pt}%
\definecolor{currentstroke}{rgb}{1.000000,1.000000,1.000000}%
\pgfsetstrokecolor{currentstroke}%
\pgfsetdash{}{0pt}%
\pgfpathmoveto{\pgfqpoint{0.644151in}{6.285661in}}%
\pgfpathlineto{\pgfqpoint{0.731886in}{6.285661in}}%
\pgfpathlineto{\pgfqpoint{0.731886in}{6.197925in}}%
\pgfpathlineto{\pgfqpoint{0.644151in}{6.197925in}}%
\pgfpathlineto{\pgfqpoint{0.644151in}{6.285661in}}%
\pgfusepath{stroke,fill}%
\end{pgfscope}%
\begin{pgfscope}%
\pgfpathrectangle{\pgfqpoint{0.380943in}{6.110189in}}{\pgfqpoint{4.650000in}{0.614151in}}%
\pgfusepath{clip}%
\pgfsetbuttcap%
\pgfsetroundjoin%
\definecolor{currentfill}{rgb}{1.000000,1.000000,0.870204}%
\pgfsetfillcolor{currentfill}%
\pgfsetlinewidth{0.250937pt}%
\definecolor{currentstroke}{rgb}{1.000000,1.000000,1.000000}%
\pgfsetstrokecolor{currentstroke}%
\pgfsetdash{}{0pt}%
\pgfpathmoveto{\pgfqpoint{0.731886in}{6.285661in}}%
\pgfpathlineto{\pgfqpoint{0.819622in}{6.285661in}}%
\pgfpathlineto{\pgfqpoint{0.819622in}{6.197925in}}%
\pgfpathlineto{\pgfqpoint{0.731886in}{6.197925in}}%
\pgfpathlineto{\pgfqpoint{0.731886in}{6.285661in}}%
\pgfusepath{stroke,fill}%
\end{pgfscope}%
\begin{pgfscope}%
\pgfpathrectangle{\pgfqpoint{0.380943in}{6.110189in}}{\pgfqpoint{4.650000in}{0.614151in}}%
\pgfusepath{clip}%
\pgfsetbuttcap%
\pgfsetroundjoin%
\definecolor{currentfill}{rgb}{0.968166,0.945882,0.748604}%
\pgfsetfillcolor{currentfill}%
\pgfsetlinewidth{0.250937pt}%
\definecolor{currentstroke}{rgb}{1.000000,1.000000,1.000000}%
\pgfsetstrokecolor{currentstroke}%
\pgfsetdash{}{0pt}%
\pgfpathmoveto{\pgfqpoint{0.819622in}{6.285661in}}%
\pgfpathlineto{\pgfqpoint{0.907358in}{6.285661in}}%
\pgfpathlineto{\pgfqpoint{0.907358in}{6.197925in}}%
\pgfpathlineto{\pgfqpoint{0.819622in}{6.197925in}}%
\pgfpathlineto{\pgfqpoint{0.819622in}{6.285661in}}%
\pgfusepath{stroke,fill}%
\end{pgfscope}%
\begin{pgfscope}%
\pgfpathrectangle{\pgfqpoint{0.380943in}{6.110189in}}{\pgfqpoint{4.650000in}{0.614151in}}%
\pgfusepath{clip}%
\pgfsetbuttcap%
\pgfsetroundjoin%
\definecolor{currentfill}{rgb}{0.965444,0.906113,0.711757}%
\pgfsetfillcolor{currentfill}%
\pgfsetlinewidth{0.250937pt}%
\definecolor{currentstroke}{rgb}{1.000000,1.000000,1.000000}%
\pgfsetstrokecolor{currentstroke}%
\pgfsetdash{}{0pt}%
\pgfpathmoveto{\pgfqpoint{0.907358in}{6.285661in}}%
\pgfpathlineto{\pgfqpoint{0.995094in}{6.285661in}}%
\pgfpathlineto{\pgfqpoint{0.995094in}{6.197925in}}%
\pgfpathlineto{\pgfqpoint{0.907358in}{6.197925in}}%
\pgfpathlineto{\pgfqpoint{0.907358in}{6.285661in}}%
\pgfusepath{stroke,fill}%
\end{pgfscope}%
\begin{pgfscope}%
\pgfpathrectangle{\pgfqpoint{0.380943in}{6.110189in}}{\pgfqpoint{4.650000in}{0.614151in}}%
\pgfusepath{clip}%
\pgfsetbuttcap%
\pgfsetroundjoin%
\definecolor{currentfill}{rgb}{0.965444,0.906113,0.711757}%
\pgfsetfillcolor{currentfill}%
\pgfsetlinewidth{0.250937pt}%
\definecolor{currentstroke}{rgb}{1.000000,1.000000,1.000000}%
\pgfsetstrokecolor{currentstroke}%
\pgfsetdash{}{0pt}%
\pgfpathmoveto{\pgfqpoint{0.995094in}{6.285661in}}%
\pgfpathlineto{\pgfqpoint{1.082830in}{6.285661in}}%
\pgfpathlineto{\pgfqpoint{1.082830in}{6.197925in}}%
\pgfpathlineto{\pgfqpoint{0.995094in}{6.197925in}}%
\pgfpathlineto{\pgfqpoint{0.995094in}{6.285661in}}%
\pgfusepath{stroke,fill}%
\end{pgfscope}%
\begin{pgfscope}%
\pgfpathrectangle{\pgfqpoint{0.380943in}{6.110189in}}{\pgfqpoint{4.650000in}{0.614151in}}%
\pgfusepath{clip}%
\pgfsetbuttcap%
\pgfsetroundjoin%
\definecolor{currentfill}{rgb}{0.965444,0.906113,0.711757}%
\pgfsetfillcolor{currentfill}%
\pgfsetlinewidth{0.250937pt}%
\definecolor{currentstroke}{rgb}{1.000000,1.000000,1.000000}%
\pgfsetstrokecolor{currentstroke}%
\pgfsetdash{}{0pt}%
\pgfpathmoveto{\pgfqpoint{1.082830in}{6.285661in}}%
\pgfpathlineto{\pgfqpoint{1.170566in}{6.285661in}}%
\pgfpathlineto{\pgfqpoint{1.170566in}{6.197925in}}%
\pgfpathlineto{\pgfqpoint{1.082830in}{6.197925in}}%
\pgfpathlineto{\pgfqpoint{1.082830in}{6.285661in}}%
\pgfusepath{stroke,fill}%
\end{pgfscope}%
\begin{pgfscope}%
\pgfpathrectangle{\pgfqpoint{0.380943in}{6.110189in}}{\pgfqpoint{4.650000in}{0.614151in}}%
\pgfusepath{clip}%
\pgfsetbuttcap%
\pgfsetroundjoin%
\definecolor{currentfill}{rgb}{0.968166,0.945882,0.748604}%
\pgfsetfillcolor{currentfill}%
\pgfsetlinewidth{0.250937pt}%
\definecolor{currentstroke}{rgb}{1.000000,1.000000,1.000000}%
\pgfsetstrokecolor{currentstroke}%
\pgfsetdash{}{0pt}%
\pgfpathmoveto{\pgfqpoint{1.170566in}{6.285661in}}%
\pgfpathlineto{\pgfqpoint{1.258302in}{6.285661in}}%
\pgfpathlineto{\pgfqpoint{1.258302in}{6.197925in}}%
\pgfpathlineto{\pgfqpoint{1.170566in}{6.197925in}}%
\pgfpathlineto{\pgfqpoint{1.170566in}{6.285661in}}%
\pgfusepath{stroke,fill}%
\end{pgfscope}%
\begin{pgfscope}%
\pgfpathrectangle{\pgfqpoint{0.380943in}{6.110189in}}{\pgfqpoint{4.650000in}{0.614151in}}%
\pgfusepath{clip}%
\pgfsetbuttcap%
\pgfsetroundjoin%
\definecolor{currentfill}{rgb}{0.991849,0.986144,0.810181}%
\pgfsetfillcolor{currentfill}%
\pgfsetlinewidth{0.250937pt}%
\definecolor{currentstroke}{rgb}{1.000000,1.000000,1.000000}%
\pgfsetstrokecolor{currentstroke}%
\pgfsetdash{}{0pt}%
\pgfpathmoveto{\pgfqpoint{1.258302in}{6.285661in}}%
\pgfpathlineto{\pgfqpoint{1.346037in}{6.285661in}}%
\pgfpathlineto{\pgfqpoint{1.346037in}{6.197925in}}%
\pgfpathlineto{\pgfqpoint{1.258302in}{6.197925in}}%
\pgfpathlineto{\pgfqpoint{1.258302in}{6.285661in}}%
\pgfusepath{stroke,fill}%
\end{pgfscope}%
\begin{pgfscope}%
\pgfpathrectangle{\pgfqpoint{0.380943in}{6.110189in}}{\pgfqpoint{4.650000in}{0.614151in}}%
\pgfusepath{clip}%
\pgfsetbuttcap%
\pgfsetroundjoin%
\definecolor{currentfill}{rgb}{0.972549,0.870588,0.692810}%
\pgfsetfillcolor{currentfill}%
\pgfsetlinewidth{0.250937pt}%
\definecolor{currentstroke}{rgb}{1.000000,1.000000,1.000000}%
\pgfsetstrokecolor{currentstroke}%
\pgfsetdash{}{0pt}%
\pgfpathmoveto{\pgfqpoint{1.346037in}{6.285661in}}%
\pgfpathlineto{\pgfqpoint{1.433773in}{6.285661in}}%
\pgfpathlineto{\pgfqpoint{1.433773in}{6.197925in}}%
\pgfpathlineto{\pgfqpoint{1.346037in}{6.197925in}}%
\pgfpathlineto{\pgfqpoint{1.346037in}{6.285661in}}%
\pgfusepath{stroke,fill}%
\end{pgfscope}%
\begin{pgfscope}%
\pgfpathrectangle{\pgfqpoint{0.380943in}{6.110189in}}{\pgfqpoint{4.650000in}{0.614151in}}%
\pgfusepath{clip}%
\pgfsetbuttcap%
\pgfsetroundjoin%
\definecolor{currentfill}{rgb}{0.965444,0.906113,0.711757}%
\pgfsetfillcolor{currentfill}%
\pgfsetlinewidth{0.250937pt}%
\definecolor{currentstroke}{rgb}{1.000000,1.000000,1.000000}%
\pgfsetstrokecolor{currentstroke}%
\pgfsetdash{}{0pt}%
\pgfpathmoveto{\pgfqpoint{1.433773in}{6.285661in}}%
\pgfpathlineto{\pgfqpoint{1.521509in}{6.285661in}}%
\pgfpathlineto{\pgfqpoint{1.521509in}{6.197925in}}%
\pgfpathlineto{\pgfqpoint{1.433773in}{6.197925in}}%
\pgfpathlineto{\pgfqpoint{1.433773in}{6.285661in}}%
\pgfusepath{stroke,fill}%
\end{pgfscope}%
\begin{pgfscope}%
\pgfpathrectangle{\pgfqpoint{0.380943in}{6.110189in}}{\pgfqpoint{4.650000in}{0.614151in}}%
\pgfusepath{clip}%
\pgfsetbuttcap%
\pgfsetroundjoin%
\definecolor{currentfill}{rgb}{0.991849,0.986144,0.810181}%
\pgfsetfillcolor{currentfill}%
\pgfsetlinewidth{0.250937pt}%
\definecolor{currentstroke}{rgb}{1.000000,1.000000,1.000000}%
\pgfsetstrokecolor{currentstroke}%
\pgfsetdash{}{0pt}%
\pgfpathmoveto{\pgfqpoint{1.521509in}{6.285661in}}%
\pgfpathlineto{\pgfqpoint{1.609245in}{6.285661in}}%
\pgfpathlineto{\pgfqpoint{1.609245in}{6.197925in}}%
\pgfpathlineto{\pgfqpoint{1.521509in}{6.197925in}}%
\pgfpathlineto{\pgfqpoint{1.521509in}{6.285661in}}%
\pgfusepath{stroke,fill}%
\end{pgfscope}%
\begin{pgfscope}%
\pgfpathrectangle{\pgfqpoint{0.380943in}{6.110189in}}{\pgfqpoint{4.650000in}{0.614151in}}%
\pgfusepath{clip}%
\pgfsetbuttcap%
\pgfsetroundjoin%
\definecolor{currentfill}{rgb}{0.965444,0.906113,0.711757}%
\pgfsetfillcolor{currentfill}%
\pgfsetlinewidth{0.250937pt}%
\definecolor{currentstroke}{rgb}{1.000000,1.000000,1.000000}%
\pgfsetstrokecolor{currentstroke}%
\pgfsetdash{}{0pt}%
\pgfpathmoveto{\pgfqpoint{1.609245in}{6.285661in}}%
\pgfpathlineto{\pgfqpoint{1.696981in}{6.285661in}}%
\pgfpathlineto{\pgfqpoint{1.696981in}{6.197925in}}%
\pgfpathlineto{\pgfqpoint{1.609245in}{6.197925in}}%
\pgfpathlineto{\pgfqpoint{1.609245in}{6.285661in}}%
\pgfusepath{stroke,fill}%
\end{pgfscope}%
\begin{pgfscope}%
\pgfpathrectangle{\pgfqpoint{0.380943in}{6.110189in}}{\pgfqpoint{4.650000in}{0.614151in}}%
\pgfusepath{clip}%
\pgfsetbuttcap%
\pgfsetroundjoin%
\definecolor{currentfill}{rgb}{1.000000,1.000000,0.870204}%
\pgfsetfillcolor{currentfill}%
\pgfsetlinewidth{0.250937pt}%
\definecolor{currentstroke}{rgb}{1.000000,1.000000,1.000000}%
\pgfsetstrokecolor{currentstroke}%
\pgfsetdash{}{0pt}%
\pgfpathmoveto{\pgfqpoint{1.696981in}{6.285661in}}%
\pgfpathlineto{\pgfqpoint{1.784717in}{6.285661in}}%
\pgfpathlineto{\pgfqpoint{1.784717in}{6.197925in}}%
\pgfpathlineto{\pgfqpoint{1.696981in}{6.197925in}}%
\pgfpathlineto{\pgfqpoint{1.696981in}{6.285661in}}%
\pgfusepath{stroke,fill}%
\end{pgfscope}%
\begin{pgfscope}%
\pgfpathrectangle{\pgfqpoint{0.380943in}{6.110189in}}{\pgfqpoint{4.650000in}{0.614151in}}%
\pgfusepath{clip}%
\pgfsetbuttcap%
\pgfsetroundjoin%
\definecolor{currentfill}{rgb}{1.000000,1.000000,0.870204}%
\pgfsetfillcolor{currentfill}%
\pgfsetlinewidth{0.250937pt}%
\definecolor{currentstroke}{rgb}{1.000000,1.000000,1.000000}%
\pgfsetstrokecolor{currentstroke}%
\pgfsetdash{}{0pt}%
\pgfpathmoveto{\pgfqpoint{1.784717in}{6.285661in}}%
\pgfpathlineto{\pgfqpoint{1.872452in}{6.285661in}}%
\pgfpathlineto{\pgfqpoint{1.872452in}{6.197925in}}%
\pgfpathlineto{\pgfqpoint{1.784717in}{6.197925in}}%
\pgfpathlineto{\pgfqpoint{1.784717in}{6.285661in}}%
\pgfusepath{stroke,fill}%
\end{pgfscope}%
\begin{pgfscope}%
\pgfpathrectangle{\pgfqpoint{0.380943in}{6.110189in}}{\pgfqpoint{4.650000in}{0.614151in}}%
\pgfusepath{clip}%
\pgfsetbuttcap%
\pgfsetroundjoin%
\definecolor{currentfill}{rgb}{0.968166,0.945882,0.748604}%
\pgfsetfillcolor{currentfill}%
\pgfsetlinewidth{0.250937pt}%
\definecolor{currentstroke}{rgb}{1.000000,1.000000,1.000000}%
\pgfsetstrokecolor{currentstroke}%
\pgfsetdash{}{0pt}%
\pgfpathmoveto{\pgfqpoint{1.872452in}{6.285661in}}%
\pgfpathlineto{\pgfqpoint{1.960188in}{6.285661in}}%
\pgfpathlineto{\pgfqpoint{1.960188in}{6.197925in}}%
\pgfpathlineto{\pgfqpoint{1.872452in}{6.197925in}}%
\pgfpathlineto{\pgfqpoint{1.872452in}{6.285661in}}%
\pgfusepath{stroke,fill}%
\end{pgfscope}%
\begin{pgfscope}%
\pgfpathrectangle{\pgfqpoint{0.380943in}{6.110189in}}{\pgfqpoint{4.650000in}{0.614151in}}%
\pgfusepath{clip}%
\pgfsetbuttcap%
\pgfsetroundjoin%
\definecolor{currentfill}{rgb}{0.962414,0.923552,0.722891}%
\pgfsetfillcolor{currentfill}%
\pgfsetlinewidth{0.250937pt}%
\definecolor{currentstroke}{rgb}{1.000000,1.000000,1.000000}%
\pgfsetstrokecolor{currentstroke}%
\pgfsetdash{}{0pt}%
\pgfpathmoveto{\pgfqpoint{1.960188in}{6.285661in}}%
\pgfpathlineto{\pgfqpoint{2.047924in}{6.285661in}}%
\pgfpathlineto{\pgfqpoint{2.047924in}{6.197925in}}%
\pgfpathlineto{\pgfqpoint{1.960188in}{6.197925in}}%
\pgfpathlineto{\pgfqpoint{1.960188in}{6.285661in}}%
\pgfusepath{stroke,fill}%
\end{pgfscope}%
\begin{pgfscope}%
\pgfpathrectangle{\pgfqpoint{0.380943in}{6.110189in}}{\pgfqpoint{4.650000in}{0.614151in}}%
\pgfusepath{clip}%
\pgfsetbuttcap%
\pgfsetroundjoin%
\definecolor{currentfill}{rgb}{0.991849,0.986144,0.810181}%
\pgfsetfillcolor{currentfill}%
\pgfsetlinewidth{0.250937pt}%
\definecolor{currentstroke}{rgb}{1.000000,1.000000,1.000000}%
\pgfsetstrokecolor{currentstroke}%
\pgfsetdash{}{0pt}%
\pgfpathmoveto{\pgfqpoint{2.047924in}{6.285661in}}%
\pgfpathlineto{\pgfqpoint{2.135660in}{6.285661in}}%
\pgfpathlineto{\pgfqpoint{2.135660in}{6.197925in}}%
\pgfpathlineto{\pgfqpoint{2.047924in}{6.197925in}}%
\pgfpathlineto{\pgfqpoint{2.047924in}{6.285661in}}%
\pgfusepath{stroke,fill}%
\end{pgfscope}%
\begin{pgfscope}%
\pgfpathrectangle{\pgfqpoint{0.380943in}{6.110189in}}{\pgfqpoint{4.650000in}{0.614151in}}%
\pgfusepath{clip}%
\pgfsetbuttcap%
\pgfsetroundjoin%
\definecolor{currentfill}{rgb}{0.991849,0.986144,0.810181}%
\pgfsetfillcolor{currentfill}%
\pgfsetlinewidth{0.250937pt}%
\definecolor{currentstroke}{rgb}{1.000000,1.000000,1.000000}%
\pgfsetstrokecolor{currentstroke}%
\pgfsetdash{}{0pt}%
\pgfpathmoveto{\pgfqpoint{2.135660in}{6.285661in}}%
\pgfpathlineto{\pgfqpoint{2.223396in}{6.285661in}}%
\pgfpathlineto{\pgfqpoint{2.223396in}{6.197925in}}%
\pgfpathlineto{\pgfqpoint{2.135660in}{6.197925in}}%
\pgfpathlineto{\pgfqpoint{2.135660in}{6.285661in}}%
\pgfusepath{stroke,fill}%
\end{pgfscope}%
\begin{pgfscope}%
\pgfpathrectangle{\pgfqpoint{0.380943in}{6.110189in}}{\pgfqpoint{4.650000in}{0.614151in}}%
\pgfusepath{clip}%
\pgfsetbuttcap%
\pgfsetroundjoin%
\definecolor{currentfill}{rgb}{0.962414,0.923552,0.722891}%
\pgfsetfillcolor{currentfill}%
\pgfsetlinewidth{0.250937pt}%
\definecolor{currentstroke}{rgb}{1.000000,1.000000,1.000000}%
\pgfsetstrokecolor{currentstroke}%
\pgfsetdash{}{0pt}%
\pgfpathmoveto{\pgfqpoint{2.223396in}{6.285661in}}%
\pgfpathlineto{\pgfqpoint{2.311132in}{6.285661in}}%
\pgfpathlineto{\pgfqpoint{2.311132in}{6.197925in}}%
\pgfpathlineto{\pgfqpoint{2.223396in}{6.197925in}}%
\pgfpathlineto{\pgfqpoint{2.223396in}{6.285661in}}%
\pgfusepath{stroke,fill}%
\end{pgfscope}%
\begin{pgfscope}%
\pgfpathrectangle{\pgfqpoint{0.380943in}{6.110189in}}{\pgfqpoint{4.650000in}{0.614151in}}%
\pgfusepath{clip}%
\pgfsetbuttcap%
\pgfsetroundjoin%
\definecolor{currentfill}{rgb}{0.979654,0.837186,0.669619}%
\pgfsetfillcolor{currentfill}%
\pgfsetlinewidth{0.250937pt}%
\definecolor{currentstroke}{rgb}{1.000000,1.000000,1.000000}%
\pgfsetstrokecolor{currentstroke}%
\pgfsetdash{}{0pt}%
\pgfpathmoveto{\pgfqpoint{2.311132in}{6.285661in}}%
\pgfpathlineto{\pgfqpoint{2.398868in}{6.285661in}}%
\pgfpathlineto{\pgfqpoint{2.398868in}{6.197925in}}%
\pgfpathlineto{\pgfqpoint{2.311132in}{6.197925in}}%
\pgfpathlineto{\pgfqpoint{2.311132in}{6.285661in}}%
\pgfusepath{stroke,fill}%
\end{pgfscope}%
\begin{pgfscope}%
\pgfpathrectangle{\pgfqpoint{0.380943in}{6.110189in}}{\pgfqpoint{4.650000in}{0.614151in}}%
\pgfusepath{clip}%
\pgfsetbuttcap%
\pgfsetroundjoin%
\definecolor{currentfill}{rgb}{0.968166,0.945882,0.748604}%
\pgfsetfillcolor{currentfill}%
\pgfsetlinewidth{0.250937pt}%
\definecolor{currentstroke}{rgb}{1.000000,1.000000,1.000000}%
\pgfsetstrokecolor{currentstroke}%
\pgfsetdash{}{0pt}%
\pgfpathmoveto{\pgfqpoint{2.398868in}{6.285661in}}%
\pgfpathlineto{\pgfqpoint{2.486603in}{6.285661in}}%
\pgfpathlineto{\pgfqpoint{2.486603in}{6.197925in}}%
\pgfpathlineto{\pgfqpoint{2.398868in}{6.197925in}}%
\pgfpathlineto{\pgfqpoint{2.398868in}{6.285661in}}%
\pgfusepath{stroke,fill}%
\end{pgfscope}%
\begin{pgfscope}%
\pgfpathrectangle{\pgfqpoint{0.380943in}{6.110189in}}{\pgfqpoint{4.650000in}{0.614151in}}%
\pgfusepath{clip}%
\pgfsetbuttcap%
\pgfsetroundjoin%
\definecolor{currentfill}{rgb}{1.000000,1.000000,0.870204}%
\pgfsetfillcolor{currentfill}%
\pgfsetlinewidth{0.250937pt}%
\definecolor{currentstroke}{rgb}{1.000000,1.000000,1.000000}%
\pgfsetstrokecolor{currentstroke}%
\pgfsetdash{}{0pt}%
\pgfpathmoveto{\pgfqpoint{2.486603in}{6.285661in}}%
\pgfpathlineto{\pgfqpoint{2.574339in}{6.285661in}}%
\pgfpathlineto{\pgfqpoint{2.574339in}{6.197925in}}%
\pgfpathlineto{\pgfqpoint{2.486603in}{6.197925in}}%
\pgfpathlineto{\pgfqpoint{2.486603in}{6.285661in}}%
\pgfusepath{stroke,fill}%
\end{pgfscope}%
\begin{pgfscope}%
\pgfpathrectangle{\pgfqpoint{0.380943in}{6.110189in}}{\pgfqpoint{4.650000in}{0.614151in}}%
\pgfusepath{clip}%
\pgfsetbuttcap%
\pgfsetroundjoin%
\definecolor{currentfill}{rgb}{0.991849,0.986144,0.810181}%
\pgfsetfillcolor{currentfill}%
\pgfsetlinewidth{0.250937pt}%
\definecolor{currentstroke}{rgb}{1.000000,1.000000,1.000000}%
\pgfsetstrokecolor{currentstroke}%
\pgfsetdash{}{0pt}%
\pgfpathmoveto{\pgfqpoint{2.574339in}{6.285661in}}%
\pgfpathlineto{\pgfqpoint{2.662075in}{6.285661in}}%
\pgfpathlineto{\pgfqpoint{2.662075in}{6.197925in}}%
\pgfpathlineto{\pgfqpoint{2.574339in}{6.197925in}}%
\pgfpathlineto{\pgfqpoint{2.574339in}{6.285661in}}%
\pgfusepath{stroke,fill}%
\end{pgfscope}%
\begin{pgfscope}%
\pgfpathrectangle{\pgfqpoint{0.380943in}{6.110189in}}{\pgfqpoint{4.650000in}{0.614151in}}%
\pgfusepath{clip}%
\pgfsetbuttcap%
\pgfsetroundjoin%
\definecolor{currentfill}{rgb}{0.991849,0.986144,0.810181}%
\pgfsetfillcolor{currentfill}%
\pgfsetlinewidth{0.250937pt}%
\definecolor{currentstroke}{rgb}{1.000000,1.000000,1.000000}%
\pgfsetstrokecolor{currentstroke}%
\pgfsetdash{}{0pt}%
\pgfpathmoveto{\pgfqpoint{2.662075in}{6.285661in}}%
\pgfpathlineto{\pgfqpoint{2.749811in}{6.285661in}}%
\pgfpathlineto{\pgfqpoint{2.749811in}{6.197925in}}%
\pgfpathlineto{\pgfqpoint{2.662075in}{6.197925in}}%
\pgfpathlineto{\pgfqpoint{2.662075in}{6.285661in}}%
\pgfusepath{stroke,fill}%
\end{pgfscope}%
\begin{pgfscope}%
\pgfpathrectangle{\pgfqpoint{0.380943in}{6.110189in}}{\pgfqpoint{4.650000in}{0.614151in}}%
\pgfusepath{clip}%
\pgfsetbuttcap%
\pgfsetroundjoin%
\definecolor{currentfill}{rgb}{1.000000,1.000000,0.929412}%
\pgfsetfillcolor{currentfill}%
\pgfsetlinewidth{0.250937pt}%
\definecolor{currentstroke}{rgb}{1.000000,1.000000,1.000000}%
\pgfsetstrokecolor{currentstroke}%
\pgfsetdash{}{0pt}%
\pgfpathmoveto{\pgfqpoint{2.749811in}{6.285661in}}%
\pgfpathlineto{\pgfqpoint{2.837547in}{6.285661in}}%
\pgfpathlineto{\pgfqpoint{2.837547in}{6.197925in}}%
\pgfpathlineto{\pgfqpoint{2.749811in}{6.197925in}}%
\pgfpathlineto{\pgfqpoint{2.749811in}{6.285661in}}%
\pgfusepath{stroke,fill}%
\end{pgfscope}%
\begin{pgfscope}%
\pgfpathrectangle{\pgfqpoint{0.380943in}{6.110189in}}{\pgfqpoint{4.650000in}{0.614151in}}%
\pgfusepath{clip}%
\pgfsetbuttcap%
\pgfsetroundjoin%
\definecolor{currentfill}{rgb}{0.968166,0.945882,0.748604}%
\pgfsetfillcolor{currentfill}%
\pgfsetlinewidth{0.250937pt}%
\definecolor{currentstroke}{rgb}{1.000000,1.000000,1.000000}%
\pgfsetstrokecolor{currentstroke}%
\pgfsetdash{}{0pt}%
\pgfpathmoveto{\pgfqpoint{2.837547in}{6.285661in}}%
\pgfpathlineto{\pgfqpoint{2.925283in}{6.285661in}}%
\pgfpathlineto{\pgfqpoint{2.925283in}{6.197925in}}%
\pgfpathlineto{\pgfqpoint{2.837547in}{6.197925in}}%
\pgfpathlineto{\pgfqpoint{2.837547in}{6.285661in}}%
\pgfusepath{stroke,fill}%
\end{pgfscope}%
\begin{pgfscope}%
\pgfpathrectangle{\pgfqpoint{0.380943in}{6.110189in}}{\pgfqpoint{4.650000in}{0.614151in}}%
\pgfusepath{clip}%
\pgfsetbuttcap%
\pgfsetroundjoin%
\definecolor{currentfill}{rgb}{1.000000,1.000000,0.870204}%
\pgfsetfillcolor{currentfill}%
\pgfsetlinewidth{0.250937pt}%
\definecolor{currentstroke}{rgb}{1.000000,1.000000,1.000000}%
\pgfsetstrokecolor{currentstroke}%
\pgfsetdash{}{0pt}%
\pgfpathmoveto{\pgfqpoint{2.925283in}{6.285661in}}%
\pgfpathlineto{\pgfqpoint{3.013019in}{6.285661in}}%
\pgfpathlineto{\pgfqpoint{3.013019in}{6.197925in}}%
\pgfpathlineto{\pgfqpoint{2.925283in}{6.197925in}}%
\pgfpathlineto{\pgfqpoint{2.925283in}{6.285661in}}%
\pgfusepath{stroke,fill}%
\end{pgfscope}%
\begin{pgfscope}%
\pgfpathrectangle{\pgfqpoint{0.380943in}{6.110189in}}{\pgfqpoint{4.650000in}{0.614151in}}%
\pgfusepath{clip}%
\pgfsetbuttcap%
\pgfsetroundjoin%
\definecolor{currentfill}{rgb}{0.968166,0.945882,0.748604}%
\pgfsetfillcolor{currentfill}%
\pgfsetlinewidth{0.250937pt}%
\definecolor{currentstroke}{rgb}{1.000000,1.000000,1.000000}%
\pgfsetstrokecolor{currentstroke}%
\pgfsetdash{}{0pt}%
\pgfpathmoveto{\pgfqpoint{3.013019in}{6.285661in}}%
\pgfpathlineto{\pgfqpoint{3.100754in}{6.285661in}}%
\pgfpathlineto{\pgfqpoint{3.100754in}{6.197925in}}%
\pgfpathlineto{\pgfqpoint{3.013019in}{6.197925in}}%
\pgfpathlineto{\pgfqpoint{3.013019in}{6.285661in}}%
\pgfusepath{stroke,fill}%
\end{pgfscope}%
\begin{pgfscope}%
\pgfpathrectangle{\pgfqpoint{0.380943in}{6.110189in}}{\pgfqpoint{4.650000in}{0.614151in}}%
\pgfusepath{clip}%
\pgfsetbuttcap%
\pgfsetroundjoin%
\definecolor{currentfill}{rgb}{1.000000,1.000000,0.929412}%
\pgfsetfillcolor{currentfill}%
\pgfsetlinewidth{0.250937pt}%
\definecolor{currentstroke}{rgb}{1.000000,1.000000,1.000000}%
\pgfsetstrokecolor{currentstroke}%
\pgfsetdash{}{0pt}%
\pgfpathmoveto{\pgfqpoint{3.100754in}{6.285661in}}%
\pgfpathlineto{\pgfqpoint{3.188490in}{6.285661in}}%
\pgfpathlineto{\pgfqpoint{3.188490in}{6.197925in}}%
\pgfpathlineto{\pgfqpoint{3.100754in}{6.197925in}}%
\pgfpathlineto{\pgfqpoint{3.100754in}{6.285661in}}%
\pgfusepath{stroke,fill}%
\end{pgfscope}%
\begin{pgfscope}%
\pgfpathrectangle{\pgfqpoint{0.380943in}{6.110189in}}{\pgfqpoint{4.650000in}{0.614151in}}%
\pgfusepath{clip}%
\pgfsetbuttcap%
\pgfsetroundjoin%
\definecolor{currentfill}{rgb}{1.000000,1.000000,0.870204}%
\pgfsetfillcolor{currentfill}%
\pgfsetlinewidth{0.250937pt}%
\definecolor{currentstroke}{rgb}{1.000000,1.000000,1.000000}%
\pgfsetstrokecolor{currentstroke}%
\pgfsetdash{}{0pt}%
\pgfpathmoveto{\pgfqpoint{3.188490in}{6.285661in}}%
\pgfpathlineto{\pgfqpoint{3.276226in}{6.285661in}}%
\pgfpathlineto{\pgfqpoint{3.276226in}{6.197925in}}%
\pgfpathlineto{\pgfqpoint{3.188490in}{6.197925in}}%
\pgfpathlineto{\pgfqpoint{3.188490in}{6.285661in}}%
\pgfusepath{stroke,fill}%
\end{pgfscope}%
\begin{pgfscope}%
\pgfpathrectangle{\pgfqpoint{0.380943in}{6.110189in}}{\pgfqpoint{4.650000in}{0.614151in}}%
\pgfusepath{clip}%
\pgfsetbuttcap%
\pgfsetroundjoin%
\definecolor{currentfill}{rgb}{0.991849,0.986144,0.810181}%
\pgfsetfillcolor{currentfill}%
\pgfsetlinewidth{0.250937pt}%
\definecolor{currentstroke}{rgb}{1.000000,1.000000,1.000000}%
\pgfsetstrokecolor{currentstroke}%
\pgfsetdash{}{0pt}%
\pgfpathmoveto{\pgfqpoint{3.276226in}{6.285661in}}%
\pgfpathlineto{\pgfqpoint{3.363962in}{6.285661in}}%
\pgfpathlineto{\pgfqpoint{3.363962in}{6.197925in}}%
\pgfpathlineto{\pgfqpoint{3.276226in}{6.197925in}}%
\pgfpathlineto{\pgfqpoint{3.276226in}{6.285661in}}%
\pgfusepath{stroke,fill}%
\end{pgfscope}%
\begin{pgfscope}%
\pgfpathrectangle{\pgfqpoint{0.380943in}{6.110189in}}{\pgfqpoint{4.650000in}{0.614151in}}%
\pgfusepath{clip}%
\pgfsetbuttcap%
\pgfsetroundjoin%
\definecolor{currentfill}{rgb}{0.972549,0.870588,0.692810}%
\pgfsetfillcolor{currentfill}%
\pgfsetlinewidth{0.250937pt}%
\definecolor{currentstroke}{rgb}{1.000000,1.000000,1.000000}%
\pgfsetstrokecolor{currentstroke}%
\pgfsetdash{}{0pt}%
\pgfpathmoveto{\pgfqpoint{3.363962in}{6.285661in}}%
\pgfpathlineto{\pgfqpoint{3.451698in}{6.285661in}}%
\pgfpathlineto{\pgfqpoint{3.451698in}{6.197925in}}%
\pgfpathlineto{\pgfqpoint{3.363962in}{6.197925in}}%
\pgfpathlineto{\pgfqpoint{3.363962in}{6.285661in}}%
\pgfusepath{stroke,fill}%
\end{pgfscope}%
\begin{pgfscope}%
\pgfpathrectangle{\pgfqpoint{0.380943in}{6.110189in}}{\pgfqpoint{4.650000in}{0.614151in}}%
\pgfusepath{clip}%
\pgfsetbuttcap%
\pgfsetroundjoin%
\definecolor{currentfill}{rgb}{0.962414,0.923552,0.722891}%
\pgfsetfillcolor{currentfill}%
\pgfsetlinewidth{0.250937pt}%
\definecolor{currentstroke}{rgb}{1.000000,1.000000,1.000000}%
\pgfsetstrokecolor{currentstroke}%
\pgfsetdash{}{0pt}%
\pgfpathmoveto{\pgfqpoint{3.451698in}{6.285661in}}%
\pgfpathlineto{\pgfqpoint{3.539434in}{6.285661in}}%
\pgfpathlineto{\pgfqpoint{3.539434in}{6.197925in}}%
\pgfpathlineto{\pgfqpoint{3.451698in}{6.197925in}}%
\pgfpathlineto{\pgfqpoint{3.451698in}{6.285661in}}%
\pgfusepath{stroke,fill}%
\end{pgfscope}%
\begin{pgfscope}%
\pgfpathrectangle{\pgfqpoint{0.380943in}{6.110189in}}{\pgfqpoint{4.650000in}{0.614151in}}%
\pgfusepath{clip}%
\pgfsetbuttcap%
\pgfsetroundjoin%
\definecolor{currentfill}{rgb}{0.991849,0.986144,0.810181}%
\pgfsetfillcolor{currentfill}%
\pgfsetlinewidth{0.250937pt}%
\definecolor{currentstroke}{rgb}{1.000000,1.000000,1.000000}%
\pgfsetstrokecolor{currentstroke}%
\pgfsetdash{}{0pt}%
\pgfpathmoveto{\pgfqpoint{3.539434in}{6.285661in}}%
\pgfpathlineto{\pgfqpoint{3.627169in}{6.285661in}}%
\pgfpathlineto{\pgfqpoint{3.627169in}{6.197925in}}%
\pgfpathlineto{\pgfqpoint{3.539434in}{6.197925in}}%
\pgfpathlineto{\pgfqpoint{3.539434in}{6.285661in}}%
\pgfusepath{stroke,fill}%
\end{pgfscope}%
\begin{pgfscope}%
\pgfpathrectangle{\pgfqpoint{0.380943in}{6.110189in}}{\pgfqpoint{4.650000in}{0.614151in}}%
\pgfusepath{clip}%
\pgfsetbuttcap%
\pgfsetroundjoin%
\definecolor{currentfill}{rgb}{0.991849,0.986144,0.810181}%
\pgfsetfillcolor{currentfill}%
\pgfsetlinewidth{0.250937pt}%
\definecolor{currentstroke}{rgb}{1.000000,1.000000,1.000000}%
\pgfsetstrokecolor{currentstroke}%
\pgfsetdash{}{0pt}%
\pgfpathmoveto{\pgfqpoint{3.627169in}{6.285661in}}%
\pgfpathlineto{\pgfqpoint{3.714905in}{6.285661in}}%
\pgfpathlineto{\pgfqpoint{3.714905in}{6.197925in}}%
\pgfpathlineto{\pgfqpoint{3.627169in}{6.197925in}}%
\pgfpathlineto{\pgfqpoint{3.627169in}{6.285661in}}%
\pgfusepath{stroke,fill}%
\end{pgfscope}%
\begin{pgfscope}%
\pgfpathrectangle{\pgfqpoint{0.380943in}{6.110189in}}{\pgfqpoint{4.650000in}{0.614151in}}%
\pgfusepath{clip}%
\pgfsetbuttcap%
\pgfsetroundjoin%
\definecolor{currentfill}{rgb}{0.991849,0.986144,0.810181}%
\pgfsetfillcolor{currentfill}%
\pgfsetlinewidth{0.250937pt}%
\definecolor{currentstroke}{rgb}{1.000000,1.000000,1.000000}%
\pgfsetstrokecolor{currentstroke}%
\pgfsetdash{}{0pt}%
\pgfpathmoveto{\pgfqpoint{3.714905in}{6.285661in}}%
\pgfpathlineto{\pgfqpoint{3.802641in}{6.285661in}}%
\pgfpathlineto{\pgfqpoint{3.802641in}{6.197925in}}%
\pgfpathlineto{\pgfqpoint{3.714905in}{6.197925in}}%
\pgfpathlineto{\pgfqpoint{3.714905in}{6.285661in}}%
\pgfusepath{stroke,fill}%
\end{pgfscope}%
\begin{pgfscope}%
\pgfpathrectangle{\pgfqpoint{0.380943in}{6.110189in}}{\pgfqpoint{4.650000in}{0.614151in}}%
\pgfusepath{clip}%
\pgfsetbuttcap%
\pgfsetroundjoin%
\definecolor{currentfill}{rgb}{0.991849,0.986144,0.810181}%
\pgfsetfillcolor{currentfill}%
\pgfsetlinewidth{0.250937pt}%
\definecolor{currentstroke}{rgb}{1.000000,1.000000,1.000000}%
\pgfsetstrokecolor{currentstroke}%
\pgfsetdash{}{0pt}%
\pgfpathmoveto{\pgfqpoint{3.802641in}{6.285661in}}%
\pgfpathlineto{\pgfqpoint{3.890377in}{6.285661in}}%
\pgfpathlineto{\pgfqpoint{3.890377in}{6.197925in}}%
\pgfpathlineto{\pgfqpoint{3.802641in}{6.197925in}}%
\pgfpathlineto{\pgfqpoint{3.802641in}{6.285661in}}%
\pgfusepath{stroke,fill}%
\end{pgfscope}%
\begin{pgfscope}%
\pgfpathrectangle{\pgfqpoint{0.380943in}{6.110189in}}{\pgfqpoint{4.650000in}{0.614151in}}%
\pgfusepath{clip}%
\pgfsetbuttcap%
\pgfsetroundjoin%
\definecolor{currentfill}{rgb}{0.962414,0.923552,0.722891}%
\pgfsetfillcolor{currentfill}%
\pgfsetlinewidth{0.250937pt}%
\definecolor{currentstroke}{rgb}{1.000000,1.000000,1.000000}%
\pgfsetstrokecolor{currentstroke}%
\pgfsetdash{}{0pt}%
\pgfpathmoveto{\pgfqpoint{3.890377in}{6.285661in}}%
\pgfpathlineto{\pgfqpoint{3.978113in}{6.285661in}}%
\pgfpathlineto{\pgfqpoint{3.978113in}{6.197925in}}%
\pgfpathlineto{\pgfqpoint{3.890377in}{6.197925in}}%
\pgfpathlineto{\pgfqpoint{3.890377in}{6.285661in}}%
\pgfusepath{stroke,fill}%
\end{pgfscope}%
\begin{pgfscope}%
\pgfpathrectangle{\pgfqpoint{0.380943in}{6.110189in}}{\pgfqpoint{4.650000in}{0.614151in}}%
\pgfusepath{clip}%
\pgfsetbuttcap%
\pgfsetroundjoin%
\definecolor{currentfill}{rgb}{0.979654,0.837186,0.669619}%
\pgfsetfillcolor{currentfill}%
\pgfsetlinewidth{0.250937pt}%
\definecolor{currentstroke}{rgb}{1.000000,1.000000,1.000000}%
\pgfsetstrokecolor{currentstroke}%
\pgfsetdash{}{0pt}%
\pgfpathmoveto{\pgfqpoint{3.978113in}{6.285661in}}%
\pgfpathlineto{\pgfqpoint{4.065849in}{6.285661in}}%
\pgfpathlineto{\pgfqpoint{4.065849in}{6.197925in}}%
\pgfpathlineto{\pgfqpoint{3.978113in}{6.197925in}}%
\pgfpathlineto{\pgfqpoint{3.978113in}{6.285661in}}%
\pgfusepath{stroke,fill}%
\end{pgfscope}%
\begin{pgfscope}%
\pgfpathrectangle{\pgfqpoint{0.380943in}{6.110189in}}{\pgfqpoint{4.650000in}{0.614151in}}%
\pgfusepath{clip}%
\pgfsetbuttcap%
\pgfsetroundjoin%
\definecolor{currentfill}{rgb}{0.998939,0.658962,0.556032}%
\pgfsetfillcolor{currentfill}%
\pgfsetlinewidth{0.250937pt}%
\definecolor{currentstroke}{rgb}{1.000000,1.000000,1.000000}%
\pgfsetstrokecolor{currentstroke}%
\pgfsetdash{}{0pt}%
\pgfpathmoveto{\pgfqpoint{4.065849in}{6.285661in}}%
\pgfpathlineto{\pgfqpoint{4.153585in}{6.285661in}}%
\pgfpathlineto{\pgfqpoint{4.153585in}{6.197925in}}%
\pgfpathlineto{\pgfqpoint{4.065849in}{6.197925in}}%
\pgfpathlineto{\pgfqpoint{4.065849in}{6.285661in}}%
\pgfusepath{stroke,fill}%
\end{pgfscope}%
\begin{pgfscope}%
\pgfpathrectangle{\pgfqpoint{0.380943in}{6.110189in}}{\pgfqpoint{4.650000in}{0.614151in}}%
\pgfusepath{clip}%
\pgfsetbuttcap%
\pgfsetroundjoin%
\definecolor{currentfill}{rgb}{0.972549,0.870588,0.692810}%
\pgfsetfillcolor{currentfill}%
\pgfsetlinewidth{0.250937pt}%
\definecolor{currentstroke}{rgb}{1.000000,1.000000,1.000000}%
\pgfsetstrokecolor{currentstroke}%
\pgfsetdash{}{0pt}%
\pgfpathmoveto{\pgfqpoint{4.153585in}{6.285661in}}%
\pgfpathlineto{\pgfqpoint{4.241320in}{6.285661in}}%
\pgfpathlineto{\pgfqpoint{4.241320in}{6.197925in}}%
\pgfpathlineto{\pgfqpoint{4.153585in}{6.197925in}}%
\pgfpathlineto{\pgfqpoint{4.153585in}{6.285661in}}%
\pgfusepath{stroke,fill}%
\end{pgfscope}%
\begin{pgfscope}%
\pgfpathrectangle{\pgfqpoint{0.380943in}{6.110189in}}{\pgfqpoint{4.650000in}{0.614151in}}%
\pgfusepath{clip}%
\pgfsetbuttcap%
\pgfsetroundjoin%
\definecolor{currentfill}{rgb}{0.965444,0.906113,0.711757}%
\pgfsetfillcolor{currentfill}%
\pgfsetlinewidth{0.250937pt}%
\definecolor{currentstroke}{rgb}{1.000000,1.000000,1.000000}%
\pgfsetstrokecolor{currentstroke}%
\pgfsetdash{}{0pt}%
\pgfpathmoveto{\pgfqpoint{4.241320in}{6.285661in}}%
\pgfpathlineto{\pgfqpoint{4.329056in}{6.285661in}}%
\pgfpathlineto{\pgfqpoint{4.329056in}{6.197925in}}%
\pgfpathlineto{\pgfqpoint{4.241320in}{6.197925in}}%
\pgfpathlineto{\pgfqpoint{4.241320in}{6.285661in}}%
\pgfusepath{stroke,fill}%
\end{pgfscope}%
\begin{pgfscope}%
\pgfpathrectangle{\pgfqpoint{0.380943in}{6.110189in}}{\pgfqpoint{4.650000in}{0.614151in}}%
\pgfusepath{clip}%
\pgfsetbuttcap%
\pgfsetroundjoin%
\definecolor{currentfill}{rgb}{0.968166,0.945882,0.748604}%
\pgfsetfillcolor{currentfill}%
\pgfsetlinewidth{0.250937pt}%
\definecolor{currentstroke}{rgb}{1.000000,1.000000,1.000000}%
\pgfsetstrokecolor{currentstroke}%
\pgfsetdash{}{0pt}%
\pgfpathmoveto{\pgfqpoint{4.329056in}{6.285661in}}%
\pgfpathlineto{\pgfqpoint{4.416792in}{6.285661in}}%
\pgfpathlineto{\pgfqpoint{4.416792in}{6.197925in}}%
\pgfpathlineto{\pgfqpoint{4.329056in}{6.197925in}}%
\pgfpathlineto{\pgfqpoint{4.329056in}{6.285661in}}%
\pgfusepath{stroke,fill}%
\end{pgfscope}%
\begin{pgfscope}%
\pgfpathrectangle{\pgfqpoint{0.380943in}{6.110189in}}{\pgfqpoint{4.650000in}{0.614151in}}%
\pgfusepath{clip}%
\pgfsetbuttcap%
\pgfsetroundjoin%
\definecolor{currentfill}{rgb}{0.968166,0.945882,0.748604}%
\pgfsetfillcolor{currentfill}%
\pgfsetlinewidth{0.250937pt}%
\definecolor{currentstroke}{rgb}{1.000000,1.000000,1.000000}%
\pgfsetstrokecolor{currentstroke}%
\pgfsetdash{}{0pt}%
\pgfpathmoveto{\pgfqpoint{4.416792in}{6.285661in}}%
\pgfpathlineto{\pgfqpoint{4.504528in}{6.285661in}}%
\pgfpathlineto{\pgfqpoint{4.504528in}{6.197925in}}%
\pgfpathlineto{\pgfqpoint{4.416792in}{6.197925in}}%
\pgfpathlineto{\pgfqpoint{4.416792in}{6.285661in}}%
\pgfusepath{stroke,fill}%
\end{pgfscope}%
\begin{pgfscope}%
\pgfpathrectangle{\pgfqpoint{0.380943in}{6.110189in}}{\pgfqpoint{4.650000in}{0.614151in}}%
\pgfusepath{clip}%
\pgfsetbuttcap%
\pgfsetroundjoin%
\definecolor{currentfill}{rgb}{1.000000,1.000000,0.870204}%
\pgfsetfillcolor{currentfill}%
\pgfsetlinewidth{0.250937pt}%
\definecolor{currentstroke}{rgb}{1.000000,1.000000,1.000000}%
\pgfsetstrokecolor{currentstroke}%
\pgfsetdash{}{0pt}%
\pgfpathmoveto{\pgfqpoint{4.504528in}{6.285661in}}%
\pgfpathlineto{\pgfqpoint{4.592264in}{6.285661in}}%
\pgfpathlineto{\pgfqpoint{4.592264in}{6.197925in}}%
\pgfpathlineto{\pgfqpoint{4.504528in}{6.197925in}}%
\pgfpathlineto{\pgfqpoint{4.504528in}{6.285661in}}%
\pgfusepath{stroke,fill}%
\end{pgfscope}%
\begin{pgfscope}%
\pgfpathrectangle{\pgfqpoint{0.380943in}{6.110189in}}{\pgfqpoint{4.650000in}{0.614151in}}%
\pgfusepath{clip}%
\pgfsetbuttcap%
\pgfsetroundjoin%
\definecolor{currentfill}{rgb}{1.000000,1.000000,0.929412}%
\pgfsetfillcolor{currentfill}%
\pgfsetlinewidth{0.250937pt}%
\definecolor{currentstroke}{rgb}{1.000000,1.000000,1.000000}%
\pgfsetstrokecolor{currentstroke}%
\pgfsetdash{}{0pt}%
\pgfpathmoveto{\pgfqpoint{4.592264in}{6.285661in}}%
\pgfpathlineto{\pgfqpoint{4.680000in}{6.285661in}}%
\pgfpathlineto{\pgfqpoint{4.680000in}{6.197925in}}%
\pgfpathlineto{\pgfqpoint{4.592264in}{6.197925in}}%
\pgfpathlineto{\pgfqpoint{4.592264in}{6.285661in}}%
\pgfusepath{stroke,fill}%
\end{pgfscope}%
\begin{pgfscope}%
\pgfpathrectangle{\pgfqpoint{0.380943in}{6.110189in}}{\pgfqpoint{4.650000in}{0.614151in}}%
\pgfusepath{clip}%
\pgfsetbuttcap%
\pgfsetroundjoin%
\definecolor{currentfill}{rgb}{0.991849,0.986144,0.810181}%
\pgfsetfillcolor{currentfill}%
\pgfsetlinewidth{0.250937pt}%
\definecolor{currentstroke}{rgb}{1.000000,1.000000,1.000000}%
\pgfsetstrokecolor{currentstroke}%
\pgfsetdash{}{0pt}%
\pgfpathmoveto{\pgfqpoint{4.680000in}{6.285661in}}%
\pgfpathlineto{\pgfqpoint{4.767736in}{6.285661in}}%
\pgfpathlineto{\pgfqpoint{4.767736in}{6.197925in}}%
\pgfpathlineto{\pgfqpoint{4.680000in}{6.197925in}}%
\pgfpathlineto{\pgfqpoint{4.680000in}{6.285661in}}%
\pgfusepath{stroke,fill}%
\end{pgfscope}%
\begin{pgfscope}%
\pgfpathrectangle{\pgfqpoint{0.380943in}{6.110189in}}{\pgfqpoint{4.650000in}{0.614151in}}%
\pgfusepath{clip}%
\pgfsetbuttcap%
\pgfsetroundjoin%
\definecolor{currentfill}{rgb}{0.965444,0.906113,0.711757}%
\pgfsetfillcolor{currentfill}%
\pgfsetlinewidth{0.250937pt}%
\definecolor{currentstroke}{rgb}{1.000000,1.000000,1.000000}%
\pgfsetstrokecolor{currentstroke}%
\pgfsetdash{}{0pt}%
\pgfpathmoveto{\pgfqpoint{4.767736in}{6.285661in}}%
\pgfpathlineto{\pgfqpoint{4.855471in}{6.285661in}}%
\pgfpathlineto{\pgfqpoint{4.855471in}{6.197925in}}%
\pgfpathlineto{\pgfqpoint{4.767736in}{6.197925in}}%
\pgfpathlineto{\pgfqpoint{4.767736in}{6.285661in}}%
\pgfusepath{stroke,fill}%
\end{pgfscope}%
\begin{pgfscope}%
\pgfpathrectangle{\pgfqpoint{0.380943in}{6.110189in}}{\pgfqpoint{4.650000in}{0.614151in}}%
\pgfusepath{clip}%
\pgfsetbuttcap%
\pgfsetroundjoin%
\definecolor{currentfill}{rgb}{0.965444,0.906113,0.711757}%
\pgfsetfillcolor{currentfill}%
\pgfsetlinewidth{0.250937pt}%
\definecolor{currentstroke}{rgb}{1.000000,1.000000,1.000000}%
\pgfsetstrokecolor{currentstroke}%
\pgfsetdash{}{0pt}%
\pgfpathmoveto{\pgfqpoint{4.855471in}{6.285661in}}%
\pgfpathlineto{\pgfqpoint{4.943207in}{6.285661in}}%
\pgfpathlineto{\pgfqpoint{4.943207in}{6.197925in}}%
\pgfpathlineto{\pgfqpoint{4.855471in}{6.197925in}}%
\pgfpathlineto{\pgfqpoint{4.855471in}{6.285661in}}%
\pgfusepath{stroke,fill}%
\end{pgfscope}%
\begin{pgfscope}%
\pgfpathrectangle{\pgfqpoint{0.380943in}{6.110189in}}{\pgfqpoint{4.650000in}{0.614151in}}%
\pgfusepath{clip}%
\pgfsetbuttcap%
\pgfsetroundjoin%
\pgfsetlinewidth{0.250937pt}%
\definecolor{currentstroke}{rgb}{1.000000,1.000000,1.000000}%
\pgfsetstrokecolor{currentstroke}%
\pgfsetdash{}{0pt}%
\pgfpathmoveto{\pgfqpoint{4.943207in}{6.285661in}}%
\pgfpathlineto{\pgfqpoint{5.030943in}{6.285661in}}%
\pgfpathlineto{\pgfqpoint{5.030943in}{6.197925in}}%
\pgfpathlineto{\pgfqpoint{4.943207in}{6.197925in}}%
\pgfpathlineto{\pgfqpoint{4.943207in}{6.285661in}}%
\pgfusepath{stroke}%
\end{pgfscope}%
\begin{pgfscope}%
\pgfpathrectangle{\pgfqpoint{0.380943in}{6.110189in}}{\pgfqpoint{4.650000in}{0.614151in}}%
\pgfusepath{clip}%
\pgfsetbuttcap%
\pgfsetroundjoin%
\definecolor{currentfill}{rgb}{0.965444,0.906113,0.711757}%
\pgfsetfillcolor{currentfill}%
\pgfsetlinewidth{0.250937pt}%
\definecolor{currentstroke}{rgb}{1.000000,1.000000,1.000000}%
\pgfsetstrokecolor{currentstroke}%
\pgfsetdash{}{0pt}%
\pgfpathmoveto{\pgfqpoint{0.380943in}{6.197925in}}%
\pgfpathlineto{\pgfqpoint{0.468679in}{6.197925in}}%
\pgfpathlineto{\pgfqpoint{0.468679in}{6.110189in}}%
\pgfpathlineto{\pgfqpoint{0.380943in}{6.110189in}}%
\pgfpathlineto{\pgfqpoint{0.380943in}{6.197925in}}%
\pgfusepath{stroke,fill}%
\end{pgfscope}%
\begin{pgfscope}%
\pgfpathrectangle{\pgfqpoint{0.380943in}{6.110189in}}{\pgfqpoint{4.650000in}{0.614151in}}%
\pgfusepath{clip}%
\pgfsetbuttcap%
\pgfsetroundjoin%
\definecolor{currentfill}{rgb}{0.968166,0.945882,0.748604}%
\pgfsetfillcolor{currentfill}%
\pgfsetlinewidth{0.250937pt}%
\definecolor{currentstroke}{rgb}{1.000000,1.000000,1.000000}%
\pgfsetstrokecolor{currentstroke}%
\pgfsetdash{}{0pt}%
\pgfpathmoveto{\pgfqpoint{0.468679in}{6.197925in}}%
\pgfpathlineto{\pgfqpoint{0.556415in}{6.197925in}}%
\pgfpathlineto{\pgfqpoint{0.556415in}{6.110189in}}%
\pgfpathlineto{\pgfqpoint{0.468679in}{6.110189in}}%
\pgfpathlineto{\pgfqpoint{0.468679in}{6.197925in}}%
\pgfusepath{stroke,fill}%
\end{pgfscope}%
\begin{pgfscope}%
\pgfpathrectangle{\pgfqpoint{0.380943in}{6.110189in}}{\pgfqpoint{4.650000in}{0.614151in}}%
\pgfusepath{clip}%
\pgfsetbuttcap%
\pgfsetroundjoin%
\definecolor{currentfill}{rgb}{1.000000,1.000000,0.870204}%
\pgfsetfillcolor{currentfill}%
\pgfsetlinewidth{0.250937pt}%
\definecolor{currentstroke}{rgb}{1.000000,1.000000,1.000000}%
\pgfsetstrokecolor{currentstroke}%
\pgfsetdash{}{0pt}%
\pgfpathmoveto{\pgfqpoint{0.556415in}{6.197925in}}%
\pgfpathlineto{\pgfqpoint{0.644151in}{6.197925in}}%
\pgfpathlineto{\pgfqpoint{0.644151in}{6.110189in}}%
\pgfpathlineto{\pgfqpoint{0.556415in}{6.110189in}}%
\pgfpathlineto{\pgfqpoint{0.556415in}{6.197925in}}%
\pgfusepath{stroke,fill}%
\end{pgfscope}%
\begin{pgfscope}%
\pgfpathrectangle{\pgfqpoint{0.380943in}{6.110189in}}{\pgfqpoint{4.650000in}{0.614151in}}%
\pgfusepath{clip}%
\pgfsetbuttcap%
\pgfsetroundjoin%
\definecolor{currentfill}{rgb}{0.968166,0.945882,0.748604}%
\pgfsetfillcolor{currentfill}%
\pgfsetlinewidth{0.250937pt}%
\definecolor{currentstroke}{rgb}{1.000000,1.000000,1.000000}%
\pgfsetstrokecolor{currentstroke}%
\pgfsetdash{}{0pt}%
\pgfpathmoveto{\pgfqpoint{0.644151in}{6.197925in}}%
\pgfpathlineto{\pgfqpoint{0.731886in}{6.197925in}}%
\pgfpathlineto{\pgfqpoint{0.731886in}{6.110189in}}%
\pgfpathlineto{\pgfqpoint{0.644151in}{6.110189in}}%
\pgfpathlineto{\pgfqpoint{0.644151in}{6.197925in}}%
\pgfusepath{stroke,fill}%
\end{pgfscope}%
\begin{pgfscope}%
\pgfpathrectangle{\pgfqpoint{0.380943in}{6.110189in}}{\pgfqpoint{4.650000in}{0.614151in}}%
\pgfusepath{clip}%
\pgfsetbuttcap%
\pgfsetroundjoin%
\definecolor{currentfill}{rgb}{0.991849,0.986144,0.810181}%
\pgfsetfillcolor{currentfill}%
\pgfsetlinewidth{0.250937pt}%
\definecolor{currentstroke}{rgb}{1.000000,1.000000,1.000000}%
\pgfsetstrokecolor{currentstroke}%
\pgfsetdash{}{0pt}%
\pgfpathmoveto{\pgfqpoint{0.731886in}{6.197925in}}%
\pgfpathlineto{\pgfqpoint{0.819622in}{6.197925in}}%
\pgfpathlineto{\pgfqpoint{0.819622in}{6.110189in}}%
\pgfpathlineto{\pgfqpoint{0.731886in}{6.110189in}}%
\pgfpathlineto{\pgfqpoint{0.731886in}{6.197925in}}%
\pgfusepath{stroke,fill}%
\end{pgfscope}%
\begin{pgfscope}%
\pgfpathrectangle{\pgfqpoint{0.380943in}{6.110189in}}{\pgfqpoint{4.650000in}{0.614151in}}%
\pgfusepath{clip}%
\pgfsetbuttcap%
\pgfsetroundjoin%
\definecolor{currentfill}{rgb}{0.965444,0.906113,0.711757}%
\pgfsetfillcolor{currentfill}%
\pgfsetlinewidth{0.250937pt}%
\definecolor{currentstroke}{rgb}{1.000000,1.000000,1.000000}%
\pgfsetstrokecolor{currentstroke}%
\pgfsetdash{}{0pt}%
\pgfpathmoveto{\pgfqpoint{0.819622in}{6.197925in}}%
\pgfpathlineto{\pgfqpoint{0.907358in}{6.197925in}}%
\pgfpathlineto{\pgfqpoint{0.907358in}{6.110189in}}%
\pgfpathlineto{\pgfqpoint{0.819622in}{6.110189in}}%
\pgfpathlineto{\pgfqpoint{0.819622in}{6.197925in}}%
\pgfusepath{stroke,fill}%
\end{pgfscope}%
\begin{pgfscope}%
\pgfpathrectangle{\pgfqpoint{0.380943in}{6.110189in}}{\pgfqpoint{4.650000in}{0.614151in}}%
\pgfusepath{clip}%
\pgfsetbuttcap%
\pgfsetroundjoin%
\definecolor{currentfill}{rgb}{0.972549,0.870588,0.692810}%
\pgfsetfillcolor{currentfill}%
\pgfsetlinewidth{0.250937pt}%
\definecolor{currentstroke}{rgb}{1.000000,1.000000,1.000000}%
\pgfsetstrokecolor{currentstroke}%
\pgfsetdash{}{0pt}%
\pgfpathmoveto{\pgfqpoint{0.907358in}{6.197925in}}%
\pgfpathlineto{\pgfqpoint{0.995094in}{6.197925in}}%
\pgfpathlineto{\pgfqpoint{0.995094in}{6.110189in}}%
\pgfpathlineto{\pgfqpoint{0.907358in}{6.110189in}}%
\pgfpathlineto{\pgfqpoint{0.907358in}{6.197925in}}%
\pgfusepath{stroke,fill}%
\end{pgfscope}%
\begin{pgfscope}%
\pgfpathrectangle{\pgfqpoint{0.380943in}{6.110189in}}{\pgfqpoint{4.650000in}{0.614151in}}%
\pgfusepath{clip}%
\pgfsetbuttcap%
\pgfsetroundjoin%
\definecolor{currentfill}{rgb}{0.979654,0.837186,0.669619}%
\pgfsetfillcolor{currentfill}%
\pgfsetlinewidth{0.250937pt}%
\definecolor{currentstroke}{rgb}{1.000000,1.000000,1.000000}%
\pgfsetstrokecolor{currentstroke}%
\pgfsetdash{}{0pt}%
\pgfpathmoveto{\pgfqpoint{0.995094in}{6.197925in}}%
\pgfpathlineto{\pgfqpoint{1.082830in}{6.197925in}}%
\pgfpathlineto{\pgfqpoint{1.082830in}{6.110189in}}%
\pgfpathlineto{\pgfqpoint{0.995094in}{6.110189in}}%
\pgfpathlineto{\pgfqpoint{0.995094in}{6.197925in}}%
\pgfusepath{stroke,fill}%
\end{pgfscope}%
\begin{pgfscope}%
\pgfpathrectangle{\pgfqpoint{0.380943in}{6.110189in}}{\pgfqpoint{4.650000in}{0.614151in}}%
\pgfusepath{clip}%
\pgfsetbuttcap%
\pgfsetroundjoin%
\definecolor{currentfill}{rgb}{0.968166,0.945882,0.748604}%
\pgfsetfillcolor{currentfill}%
\pgfsetlinewidth{0.250937pt}%
\definecolor{currentstroke}{rgb}{1.000000,1.000000,1.000000}%
\pgfsetstrokecolor{currentstroke}%
\pgfsetdash{}{0pt}%
\pgfpathmoveto{\pgfqpoint{1.082830in}{6.197925in}}%
\pgfpathlineto{\pgfqpoint{1.170566in}{6.197925in}}%
\pgfpathlineto{\pgfqpoint{1.170566in}{6.110189in}}%
\pgfpathlineto{\pgfqpoint{1.082830in}{6.110189in}}%
\pgfpathlineto{\pgfqpoint{1.082830in}{6.197925in}}%
\pgfusepath{stroke,fill}%
\end{pgfscope}%
\begin{pgfscope}%
\pgfpathrectangle{\pgfqpoint{0.380943in}{6.110189in}}{\pgfqpoint{4.650000in}{0.614151in}}%
\pgfusepath{clip}%
\pgfsetbuttcap%
\pgfsetroundjoin%
\definecolor{currentfill}{rgb}{0.972549,0.870588,0.692810}%
\pgfsetfillcolor{currentfill}%
\pgfsetlinewidth{0.250937pt}%
\definecolor{currentstroke}{rgb}{1.000000,1.000000,1.000000}%
\pgfsetstrokecolor{currentstroke}%
\pgfsetdash{}{0pt}%
\pgfpathmoveto{\pgfqpoint{1.170566in}{6.197925in}}%
\pgfpathlineto{\pgfqpoint{1.258302in}{6.197925in}}%
\pgfpathlineto{\pgfqpoint{1.258302in}{6.110189in}}%
\pgfpathlineto{\pgfqpoint{1.170566in}{6.110189in}}%
\pgfpathlineto{\pgfqpoint{1.170566in}{6.197925in}}%
\pgfusepath{stroke,fill}%
\end{pgfscope}%
\begin{pgfscope}%
\pgfpathrectangle{\pgfqpoint{0.380943in}{6.110189in}}{\pgfqpoint{4.650000in}{0.614151in}}%
\pgfusepath{clip}%
\pgfsetbuttcap%
\pgfsetroundjoin%
\definecolor{currentfill}{rgb}{1.000000,1.000000,0.870204}%
\pgfsetfillcolor{currentfill}%
\pgfsetlinewidth{0.250937pt}%
\definecolor{currentstroke}{rgb}{1.000000,1.000000,1.000000}%
\pgfsetstrokecolor{currentstroke}%
\pgfsetdash{}{0pt}%
\pgfpathmoveto{\pgfqpoint{1.258302in}{6.197925in}}%
\pgfpathlineto{\pgfqpoint{1.346037in}{6.197925in}}%
\pgfpathlineto{\pgfqpoint{1.346037in}{6.110189in}}%
\pgfpathlineto{\pgfqpoint{1.258302in}{6.110189in}}%
\pgfpathlineto{\pgfqpoint{1.258302in}{6.197925in}}%
\pgfusepath{stroke,fill}%
\end{pgfscope}%
\begin{pgfscope}%
\pgfpathrectangle{\pgfqpoint{0.380943in}{6.110189in}}{\pgfqpoint{4.650000in}{0.614151in}}%
\pgfusepath{clip}%
\pgfsetbuttcap%
\pgfsetroundjoin%
\definecolor{currentfill}{rgb}{1.000000,1.000000,0.870204}%
\pgfsetfillcolor{currentfill}%
\pgfsetlinewidth{0.250937pt}%
\definecolor{currentstroke}{rgb}{1.000000,1.000000,1.000000}%
\pgfsetstrokecolor{currentstroke}%
\pgfsetdash{}{0pt}%
\pgfpathmoveto{\pgfqpoint{1.346037in}{6.197925in}}%
\pgfpathlineto{\pgfqpoint{1.433773in}{6.197925in}}%
\pgfpathlineto{\pgfqpoint{1.433773in}{6.110189in}}%
\pgfpathlineto{\pgfqpoint{1.346037in}{6.110189in}}%
\pgfpathlineto{\pgfqpoint{1.346037in}{6.197925in}}%
\pgfusepath{stroke,fill}%
\end{pgfscope}%
\begin{pgfscope}%
\pgfpathrectangle{\pgfqpoint{0.380943in}{6.110189in}}{\pgfqpoint{4.650000in}{0.614151in}}%
\pgfusepath{clip}%
\pgfsetbuttcap%
\pgfsetroundjoin%
\definecolor{currentfill}{rgb}{1.000000,1.000000,0.870204}%
\pgfsetfillcolor{currentfill}%
\pgfsetlinewidth{0.250937pt}%
\definecolor{currentstroke}{rgb}{1.000000,1.000000,1.000000}%
\pgfsetstrokecolor{currentstroke}%
\pgfsetdash{}{0pt}%
\pgfpathmoveto{\pgfqpoint{1.433773in}{6.197925in}}%
\pgfpathlineto{\pgfqpoint{1.521509in}{6.197925in}}%
\pgfpathlineto{\pgfqpoint{1.521509in}{6.110189in}}%
\pgfpathlineto{\pgfqpoint{1.433773in}{6.110189in}}%
\pgfpathlineto{\pgfqpoint{1.433773in}{6.197925in}}%
\pgfusepath{stroke,fill}%
\end{pgfscope}%
\begin{pgfscope}%
\pgfpathrectangle{\pgfqpoint{0.380943in}{6.110189in}}{\pgfqpoint{4.650000in}{0.614151in}}%
\pgfusepath{clip}%
\pgfsetbuttcap%
\pgfsetroundjoin%
\definecolor{currentfill}{rgb}{0.965444,0.906113,0.711757}%
\pgfsetfillcolor{currentfill}%
\pgfsetlinewidth{0.250937pt}%
\definecolor{currentstroke}{rgb}{1.000000,1.000000,1.000000}%
\pgfsetstrokecolor{currentstroke}%
\pgfsetdash{}{0pt}%
\pgfpathmoveto{\pgfqpoint{1.521509in}{6.197925in}}%
\pgfpathlineto{\pgfqpoint{1.609245in}{6.197925in}}%
\pgfpathlineto{\pgfqpoint{1.609245in}{6.110189in}}%
\pgfpathlineto{\pgfqpoint{1.521509in}{6.110189in}}%
\pgfpathlineto{\pgfqpoint{1.521509in}{6.197925in}}%
\pgfusepath{stroke,fill}%
\end{pgfscope}%
\begin{pgfscope}%
\pgfpathrectangle{\pgfqpoint{0.380943in}{6.110189in}}{\pgfqpoint{4.650000in}{0.614151in}}%
\pgfusepath{clip}%
\pgfsetbuttcap%
\pgfsetroundjoin%
\definecolor{currentfill}{rgb}{0.991849,0.986144,0.810181}%
\pgfsetfillcolor{currentfill}%
\pgfsetlinewidth{0.250937pt}%
\definecolor{currentstroke}{rgb}{1.000000,1.000000,1.000000}%
\pgfsetstrokecolor{currentstroke}%
\pgfsetdash{}{0pt}%
\pgfpathmoveto{\pgfqpoint{1.609245in}{6.197925in}}%
\pgfpathlineto{\pgfqpoint{1.696981in}{6.197925in}}%
\pgfpathlineto{\pgfqpoint{1.696981in}{6.110189in}}%
\pgfpathlineto{\pgfqpoint{1.609245in}{6.110189in}}%
\pgfpathlineto{\pgfqpoint{1.609245in}{6.197925in}}%
\pgfusepath{stroke,fill}%
\end{pgfscope}%
\begin{pgfscope}%
\pgfpathrectangle{\pgfqpoint{0.380943in}{6.110189in}}{\pgfqpoint{4.650000in}{0.614151in}}%
\pgfusepath{clip}%
\pgfsetbuttcap%
\pgfsetroundjoin%
\definecolor{currentfill}{rgb}{0.991849,0.986144,0.810181}%
\pgfsetfillcolor{currentfill}%
\pgfsetlinewidth{0.250937pt}%
\definecolor{currentstroke}{rgb}{1.000000,1.000000,1.000000}%
\pgfsetstrokecolor{currentstroke}%
\pgfsetdash{}{0pt}%
\pgfpathmoveto{\pgfqpoint{1.696981in}{6.197925in}}%
\pgfpathlineto{\pgfqpoint{1.784717in}{6.197925in}}%
\pgfpathlineto{\pgfqpoint{1.784717in}{6.110189in}}%
\pgfpathlineto{\pgfqpoint{1.696981in}{6.110189in}}%
\pgfpathlineto{\pgfqpoint{1.696981in}{6.197925in}}%
\pgfusepath{stroke,fill}%
\end{pgfscope}%
\begin{pgfscope}%
\pgfpathrectangle{\pgfqpoint{0.380943in}{6.110189in}}{\pgfqpoint{4.650000in}{0.614151in}}%
\pgfusepath{clip}%
\pgfsetbuttcap%
\pgfsetroundjoin%
\definecolor{currentfill}{rgb}{0.979654,0.837186,0.669619}%
\pgfsetfillcolor{currentfill}%
\pgfsetlinewidth{0.250937pt}%
\definecolor{currentstroke}{rgb}{1.000000,1.000000,1.000000}%
\pgfsetstrokecolor{currentstroke}%
\pgfsetdash{}{0pt}%
\pgfpathmoveto{\pgfqpoint{1.784717in}{6.197925in}}%
\pgfpathlineto{\pgfqpoint{1.872452in}{6.197925in}}%
\pgfpathlineto{\pgfqpoint{1.872452in}{6.110189in}}%
\pgfpathlineto{\pgfqpoint{1.784717in}{6.110189in}}%
\pgfpathlineto{\pgfqpoint{1.784717in}{6.197925in}}%
\pgfusepath{stroke,fill}%
\end{pgfscope}%
\begin{pgfscope}%
\pgfpathrectangle{\pgfqpoint{0.380943in}{6.110189in}}{\pgfqpoint{4.650000in}{0.614151in}}%
\pgfusepath{clip}%
\pgfsetbuttcap%
\pgfsetroundjoin%
\definecolor{currentfill}{rgb}{0.962414,0.923552,0.722891}%
\pgfsetfillcolor{currentfill}%
\pgfsetlinewidth{0.250937pt}%
\definecolor{currentstroke}{rgb}{1.000000,1.000000,1.000000}%
\pgfsetstrokecolor{currentstroke}%
\pgfsetdash{}{0pt}%
\pgfpathmoveto{\pgfqpoint{1.872452in}{6.197925in}}%
\pgfpathlineto{\pgfqpoint{1.960188in}{6.197925in}}%
\pgfpathlineto{\pgfqpoint{1.960188in}{6.110189in}}%
\pgfpathlineto{\pgfqpoint{1.872452in}{6.110189in}}%
\pgfpathlineto{\pgfqpoint{1.872452in}{6.197925in}}%
\pgfusepath{stroke,fill}%
\end{pgfscope}%
\begin{pgfscope}%
\pgfpathrectangle{\pgfqpoint{0.380943in}{6.110189in}}{\pgfqpoint{4.650000in}{0.614151in}}%
\pgfusepath{clip}%
\pgfsetbuttcap%
\pgfsetroundjoin%
\definecolor{currentfill}{rgb}{0.968166,0.945882,0.748604}%
\pgfsetfillcolor{currentfill}%
\pgfsetlinewidth{0.250937pt}%
\definecolor{currentstroke}{rgb}{1.000000,1.000000,1.000000}%
\pgfsetstrokecolor{currentstroke}%
\pgfsetdash{}{0pt}%
\pgfpathmoveto{\pgfqpoint{1.960188in}{6.197925in}}%
\pgfpathlineto{\pgfqpoint{2.047924in}{6.197925in}}%
\pgfpathlineto{\pgfqpoint{2.047924in}{6.110189in}}%
\pgfpathlineto{\pgfqpoint{1.960188in}{6.110189in}}%
\pgfpathlineto{\pgfqpoint{1.960188in}{6.197925in}}%
\pgfusepath{stroke,fill}%
\end{pgfscope}%
\begin{pgfscope}%
\pgfpathrectangle{\pgfqpoint{0.380943in}{6.110189in}}{\pgfqpoint{4.650000in}{0.614151in}}%
\pgfusepath{clip}%
\pgfsetbuttcap%
\pgfsetroundjoin%
\definecolor{currentfill}{rgb}{0.962414,0.923552,0.722891}%
\pgfsetfillcolor{currentfill}%
\pgfsetlinewidth{0.250937pt}%
\definecolor{currentstroke}{rgb}{1.000000,1.000000,1.000000}%
\pgfsetstrokecolor{currentstroke}%
\pgfsetdash{}{0pt}%
\pgfpathmoveto{\pgfqpoint{2.047924in}{6.197925in}}%
\pgfpathlineto{\pgfqpoint{2.135660in}{6.197925in}}%
\pgfpathlineto{\pgfqpoint{2.135660in}{6.110189in}}%
\pgfpathlineto{\pgfqpoint{2.047924in}{6.110189in}}%
\pgfpathlineto{\pgfqpoint{2.047924in}{6.197925in}}%
\pgfusepath{stroke,fill}%
\end{pgfscope}%
\begin{pgfscope}%
\pgfpathrectangle{\pgfqpoint{0.380943in}{6.110189in}}{\pgfqpoint{4.650000in}{0.614151in}}%
\pgfusepath{clip}%
\pgfsetbuttcap%
\pgfsetroundjoin%
\definecolor{currentfill}{rgb}{0.991849,0.986144,0.810181}%
\pgfsetfillcolor{currentfill}%
\pgfsetlinewidth{0.250937pt}%
\definecolor{currentstroke}{rgb}{1.000000,1.000000,1.000000}%
\pgfsetstrokecolor{currentstroke}%
\pgfsetdash{}{0pt}%
\pgfpathmoveto{\pgfqpoint{2.135660in}{6.197925in}}%
\pgfpathlineto{\pgfqpoint{2.223396in}{6.197925in}}%
\pgfpathlineto{\pgfqpoint{2.223396in}{6.110189in}}%
\pgfpathlineto{\pgfqpoint{2.135660in}{6.110189in}}%
\pgfpathlineto{\pgfqpoint{2.135660in}{6.197925in}}%
\pgfusepath{stroke,fill}%
\end{pgfscope}%
\begin{pgfscope}%
\pgfpathrectangle{\pgfqpoint{0.380943in}{6.110189in}}{\pgfqpoint{4.650000in}{0.614151in}}%
\pgfusepath{clip}%
\pgfsetbuttcap%
\pgfsetroundjoin%
\definecolor{currentfill}{rgb}{0.991849,0.986144,0.810181}%
\pgfsetfillcolor{currentfill}%
\pgfsetlinewidth{0.250937pt}%
\definecolor{currentstroke}{rgb}{1.000000,1.000000,1.000000}%
\pgfsetstrokecolor{currentstroke}%
\pgfsetdash{}{0pt}%
\pgfpathmoveto{\pgfqpoint{2.223396in}{6.197925in}}%
\pgfpathlineto{\pgfqpoint{2.311132in}{6.197925in}}%
\pgfpathlineto{\pgfqpoint{2.311132in}{6.110189in}}%
\pgfpathlineto{\pgfqpoint{2.223396in}{6.110189in}}%
\pgfpathlineto{\pgfqpoint{2.223396in}{6.197925in}}%
\pgfusepath{stroke,fill}%
\end{pgfscope}%
\begin{pgfscope}%
\pgfpathrectangle{\pgfqpoint{0.380943in}{6.110189in}}{\pgfqpoint{4.650000in}{0.614151in}}%
\pgfusepath{clip}%
\pgfsetbuttcap%
\pgfsetroundjoin%
\definecolor{currentfill}{rgb}{0.991849,0.986144,0.810181}%
\pgfsetfillcolor{currentfill}%
\pgfsetlinewidth{0.250937pt}%
\definecolor{currentstroke}{rgb}{1.000000,1.000000,1.000000}%
\pgfsetstrokecolor{currentstroke}%
\pgfsetdash{}{0pt}%
\pgfpathmoveto{\pgfqpoint{2.311132in}{6.197925in}}%
\pgfpathlineto{\pgfqpoint{2.398868in}{6.197925in}}%
\pgfpathlineto{\pgfqpoint{2.398868in}{6.110189in}}%
\pgfpathlineto{\pgfqpoint{2.311132in}{6.110189in}}%
\pgfpathlineto{\pgfqpoint{2.311132in}{6.197925in}}%
\pgfusepath{stroke,fill}%
\end{pgfscope}%
\begin{pgfscope}%
\pgfpathrectangle{\pgfqpoint{0.380943in}{6.110189in}}{\pgfqpoint{4.650000in}{0.614151in}}%
\pgfusepath{clip}%
\pgfsetbuttcap%
\pgfsetroundjoin%
\definecolor{currentfill}{rgb}{0.965444,0.906113,0.711757}%
\pgfsetfillcolor{currentfill}%
\pgfsetlinewidth{0.250937pt}%
\definecolor{currentstroke}{rgb}{1.000000,1.000000,1.000000}%
\pgfsetstrokecolor{currentstroke}%
\pgfsetdash{}{0pt}%
\pgfpathmoveto{\pgfqpoint{2.398868in}{6.197925in}}%
\pgfpathlineto{\pgfqpoint{2.486603in}{6.197925in}}%
\pgfpathlineto{\pgfqpoint{2.486603in}{6.110189in}}%
\pgfpathlineto{\pgfqpoint{2.398868in}{6.110189in}}%
\pgfpathlineto{\pgfqpoint{2.398868in}{6.197925in}}%
\pgfusepath{stroke,fill}%
\end{pgfscope}%
\begin{pgfscope}%
\pgfpathrectangle{\pgfqpoint{0.380943in}{6.110189in}}{\pgfqpoint{4.650000in}{0.614151in}}%
\pgfusepath{clip}%
\pgfsetbuttcap%
\pgfsetroundjoin%
\definecolor{currentfill}{rgb}{0.991849,0.986144,0.810181}%
\pgfsetfillcolor{currentfill}%
\pgfsetlinewidth{0.250937pt}%
\definecolor{currentstroke}{rgb}{1.000000,1.000000,1.000000}%
\pgfsetstrokecolor{currentstroke}%
\pgfsetdash{}{0pt}%
\pgfpathmoveto{\pgfqpoint{2.486603in}{6.197925in}}%
\pgfpathlineto{\pgfqpoint{2.574339in}{6.197925in}}%
\pgfpathlineto{\pgfqpoint{2.574339in}{6.110189in}}%
\pgfpathlineto{\pgfqpoint{2.486603in}{6.110189in}}%
\pgfpathlineto{\pgfqpoint{2.486603in}{6.197925in}}%
\pgfusepath{stroke,fill}%
\end{pgfscope}%
\begin{pgfscope}%
\pgfpathrectangle{\pgfqpoint{0.380943in}{6.110189in}}{\pgfqpoint{4.650000in}{0.614151in}}%
\pgfusepath{clip}%
\pgfsetbuttcap%
\pgfsetroundjoin%
\definecolor{currentfill}{rgb}{1.000000,1.000000,0.929412}%
\pgfsetfillcolor{currentfill}%
\pgfsetlinewidth{0.250937pt}%
\definecolor{currentstroke}{rgb}{1.000000,1.000000,1.000000}%
\pgfsetstrokecolor{currentstroke}%
\pgfsetdash{}{0pt}%
\pgfpathmoveto{\pgfqpoint{2.574339in}{6.197925in}}%
\pgfpathlineto{\pgfqpoint{2.662075in}{6.197925in}}%
\pgfpathlineto{\pgfqpoint{2.662075in}{6.110189in}}%
\pgfpathlineto{\pgfqpoint{2.574339in}{6.110189in}}%
\pgfpathlineto{\pgfqpoint{2.574339in}{6.197925in}}%
\pgfusepath{stroke,fill}%
\end{pgfscope}%
\begin{pgfscope}%
\pgfpathrectangle{\pgfqpoint{0.380943in}{6.110189in}}{\pgfqpoint{4.650000in}{0.614151in}}%
\pgfusepath{clip}%
\pgfsetbuttcap%
\pgfsetroundjoin%
\definecolor{currentfill}{rgb}{0.968166,0.945882,0.748604}%
\pgfsetfillcolor{currentfill}%
\pgfsetlinewidth{0.250937pt}%
\definecolor{currentstroke}{rgb}{1.000000,1.000000,1.000000}%
\pgfsetstrokecolor{currentstroke}%
\pgfsetdash{}{0pt}%
\pgfpathmoveto{\pgfqpoint{2.662075in}{6.197925in}}%
\pgfpathlineto{\pgfqpoint{2.749811in}{6.197925in}}%
\pgfpathlineto{\pgfqpoint{2.749811in}{6.110189in}}%
\pgfpathlineto{\pgfqpoint{2.662075in}{6.110189in}}%
\pgfpathlineto{\pgfqpoint{2.662075in}{6.197925in}}%
\pgfusepath{stroke,fill}%
\end{pgfscope}%
\begin{pgfscope}%
\pgfpathrectangle{\pgfqpoint{0.380943in}{6.110189in}}{\pgfqpoint{4.650000in}{0.614151in}}%
\pgfusepath{clip}%
\pgfsetbuttcap%
\pgfsetroundjoin%
\definecolor{currentfill}{rgb}{0.965444,0.906113,0.711757}%
\pgfsetfillcolor{currentfill}%
\pgfsetlinewidth{0.250937pt}%
\definecolor{currentstroke}{rgb}{1.000000,1.000000,1.000000}%
\pgfsetstrokecolor{currentstroke}%
\pgfsetdash{}{0pt}%
\pgfpathmoveto{\pgfqpoint{2.749811in}{6.197925in}}%
\pgfpathlineto{\pgfqpoint{2.837547in}{6.197925in}}%
\pgfpathlineto{\pgfqpoint{2.837547in}{6.110189in}}%
\pgfpathlineto{\pgfqpoint{2.749811in}{6.110189in}}%
\pgfpathlineto{\pgfqpoint{2.749811in}{6.197925in}}%
\pgfusepath{stroke,fill}%
\end{pgfscope}%
\begin{pgfscope}%
\pgfpathrectangle{\pgfqpoint{0.380943in}{6.110189in}}{\pgfqpoint{4.650000in}{0.614151in}}%
\pgfusepath{clip}%
\pgfsetbuttcap%
\pgfsetroundjoin%
\definecolor{currentfill}{rgb}{1.000000,1.000000,0.929412}%
\pgfsetfillcolor{currentfill}%
\pgfsetlinewidth{0.250937pt}%
\definecolor{currentstroke}{rgb}{1.000000,1.000000,1.000000}%
\pgfsetstrokecolor{currentstroke}%
\pgfsetdash{}{0pt}%
\pgfpathmoveto{\pgfqpoint{2.837547in}{6.197925in}}%
\pgfpathlineto{\pgfqpoint{2.925283in}{6.197925in}}%
\pgfpathlineto{\pgfqpoint{2.925283in}{6.110189in}}%
\pgfpathlineto{\pgfqpoint{2.837547in}{6.110189in}}%
\pgfpathlineto{\pgfqpoint{2.837547in}{6.197925in}}%
\pgfusepath{stroke,fill}%
\end{pgfscope}%
\begin{pgfscope}%
\pgfpathrectangle{\pgfqpoint{0.380943in}{6.110189in}}{\pgfqpoint{4.650000in}{0.614151in}}%
\pgfusepath{clip}%
\pgfsetbuttcap%
\pgfsetroundjoin%
\definecolor{currentfill}{rgb}{1.000000,1.000000,0.870204}%
\pgfsetfillcolor{currentfill}%
\pgfsetlinewidth{0.250937pt}%
\definecolor{currentstroke}{rgb}{1.000000,1.000000,1.000000}%
\pgfsetstrokecolor{currentstroke}%
\pgfsetdash{}{0pt}%
\pgfpathmoveto{\pgfqpoint{2.925283in}{6.197925in}}%
\pgfpathlineto{\pgfqpoint{3.013019in}{6.197925in}}%
\pgfpathlineto{\pgfqpoint{3.013019in}{6.110189in}}%
\pgfpathlineto{\pgfqpoint{2.925283in}{6.110189in}}%
\pgfpathlineto{\pgfqpoint{2.925283in}{6.197925in}}%
\pgfusepath{stroke,fill}%
\end{pgfscope}%
\begin{pgfscope}%
\pgfpathrectangle{\pgfqpoint{0.380943in}{6.110189in}}{\pgfqpoint{4.650000in}{0.614151in}}%
\pgfusepath{clip}%
\pgfsetbuttcap%
\pgfsetroundjoin%
\definecolor{currentfill}{rgb}{0.962414,0.923552,0.722891}%
\pgfsetfillcolor{currentfill}%
\pgfsetlinewidth{0.250937pt}%
\definecolor{currentstroke}{rgb}{1.000000,1.000000,1.000000}%
\pgfsetstrokecolor{currentstroke}%
\pgfsetdash{}{0pt}%
\pgfpathmoveto{\pgfqpoint{3.013019in}{6.197925in}}%
\pgfpathlineto{\pgfqpoint{3.100754in}{6.197925in}}%
\pgfpathlineto{\pgfqpoint{3.100754in}{6.110189in}}%
\pgfpathlineto{\pgfqpoint{3.013019in}{6.110189in}}%
\pgfpathlineto{\pgfqpoint{3.013019in}{6.197925in}}%
\pgfusepath{stroke,fill}%
\end{pgfscope}%
\begin{pgfscope}%
\pgfpathrectangle{\pgfqpoint{0.380943in}{6.110189in}}{\pgfqpoint{4.650000in}{0.614151in}}%
\pgfusepath{clip}%
\pgfsetbuttcap%
\pgfsetroundjoin%
\definecolor{currentfill}{rgb}{1.000000,1.000000,0.870204}%
\pgfsetfillcolor{currentfill}%
\pgfsetlinewidth{0.250937pt}%
\definecolor{currentstroke}{rgb}{1.000000,1.000000,1.000000}%
\pgfsetstrokecolor{currentstroke}%
\pgfsetdash{}{0pt}%
\pgfpathmoveto{\pgfqpoint{3.100754in}{6.197925in}}%
\pgfpathlineto{\pgfqpoint{3.188490in}{6.197925in}}%
\pgfpathlineto{\pgfqpoint{3.188490in}{6.110189in}}%
\pgfpathlineto{\pgfqpoint{3.100754in}{6.110189in}}%
\pgfpathlineto{\pgfqpoint{3.100754in}{6.197925in}}%
\pgfusepath{stroke,fill}%
\end{pgfscope}%
\begin{pgfscope}%
\pgfpathrectangle{\pgfqpoint{0.380943in}{6.110189in}}{\pgfqpoint{4.650000in}{0.614151in}}%
\pgfusepath{clip}%
\pgfsetbuttcap%
\pgfsetroundjoin%
\definecolor{currentfill}{rgb}{1.000000,1.000000,0.929412}%
\pgfsetfillcolor{currentfill}%
\pgfsetlinewidth{0.250937pt}%
\definecolor{currentstroke}{rgb}{1.000000,1.000000,1.000000}%
\pgfsetstrokecolor{currentstroke}%
\pgfsetdash{}{0pt}%
\pgfpathmoveto{\pgfqpoint{3.188490in}{6.197925in}}%
\pgfpathlineto{\pgfqpoint{3.276226in}{6.197925in}}%
\pgfpathlineto{\pgfqpoint{3.276226in}{6.110189in}}%
\pgfpathlineto{\pgfqpoint{3.188490in}{6.110189in}}%
\pgfpathlineto{\pgfqpoint{3.188490in}{6.197925in}}%
\pgfusepath{stroke,fill}%
\end{pgfscope}%
\begin{pgfscope}%
\pgfpathrectangle{\pgfqpoint{0.380943in}{6.110189in}}{\pgfqpoint{4.650000in}{0.614151in}}%
\pgfusepath{clip}%
\pgfsetbuttcap%
\pgfsetroundjoin%
\definecolor{currentfill}{rgb}{0.968166,0.945882,0.748604}%
\pgfsetfillcolor{currentfill}%
\pgfsetlinewidth{0.250937pt}%
\definecolor{currentstroke}{rgb}{1.000000,1.000000,1.000000}%
\pgfsetstrokecolor{currentstroke}%
\pgfsetdash{}{0pt}%
\pgfpathmoveto{\pgfqpoint{3.276226in}{6.197925in}}%
\pgfpathlineto{\pgfqpoint{3.363962in}{6.197925in}}%
\pgfpathlineto{\pgfqpoint{3.363962in}{6.110189in}}%
\pgfpathlineto{\pgfqpoint{3.276226in}{6.110189in}}%
\pgfpathlineto{\pgfqpoint{3.276226in}{6.197925in}}%
\pgfusepath{stroke,fill}%
\end{pgfscope}%
\begin{pgfscope}%
\pgfpathrectangle{\pgfqpoint{0.380943in}{6.110189in}}{\pgfqpoint{4.650000in}{0.614151in}}%
\pgfusepath{clip}%
\pgfsetbuttcap%
\pgfsetroundjoin%
\definecolor{currentfill}{rgb}{1.000000,1.000000,0.870204}%
\pgfsetfillcolor{currentfill}%
\pgfsetlinewidth{0.250937pt}%
\definecolor{currentstroke}{rgb}{1.000000,1.000000,1.000000}%
\pgfsetstrokecolor{currentstroke}%
\pgfsetdash{}{0pt}%
\pgfpathmoveto{\pgfqpoint{3.363962in}{6.197925in}}%
\pgfpathlineto{\pgfqpoint{3.451698in}{6.197925in}}%
\pgfpathlineto{\pgfqpoint{3.451698in}{6.110189in}}%
\pgfpathlineto{\pgfqpoint{3.363962in}{6.110189in}}%
\pgfpathlineto{\pgfqpoint{3.363962in}{6.197925in}}%
\pgfusepath{stroke,fill}%
\end{pgfscope}%
\begin{pgfscope}%
\pgfpathrectangle{\pgfqpoint{0.380943in}{6.110189in}}{\pgfqpoint{4.650000in}{0.614151in}}%
\pgfusepath{clip}%
\pgfsetbuttcap%
\pgfsetroundjoin%
\definecolor{currentfill}{rgb}{1.000000,1.000000,0.870204}%
\pgfsetfillcolor{currentfill}%
\pgfsetlinewidth{0.250937pt}%
\definecolor{currentstroke}{rgb}{1.000000,1.000000,1.000000}%
\pgfsetstrokecolor{currentstroke}%
\pgfsetdash{}{0pt}%
\pgfpathmoveto{\pgfqpoint{3.451698in}{6.197925in}}%
\pgfpathlineto{\pgfqpoint{3.539434in}{6.197925in}}%
\pgfpathlineto{\pgfqpoint{3.539434in}{6.110189in}}%
\pgfpathlineto{\pgfqpoint{3.451698in}{6.110189in}}%
\pgfpathlineto{\pgfqpoint{3.451698in}{6.197925in}}%
\pgfusepath{stroke,fill}%
\end{pgfscope}%
\begin{pgfscope}%
\pgfpathrectangle{\pgfqpoint{0.380943in}{6.110189in}}{\pgfqpoint{4.650000in}{0.614151in}}%
\pgfusepath{clip}%
\pgfsetbuttcap%
\pgfsetroundjoin%
\definecolor{currentfill}{rgb}{0.991849,0.986144,0.810181}%
\pgfsetfillcolor{currentfill}%
\pgfsetlinewidth{0.250937pt}%
\definecolor{currentstroke}{rgb}{1.000000,1.000000,1.000000}%
\pgfsetstrokecolor{currentstroke}%
\pgfsetdash{}{0pt}%
\pgfpathmoveto{\pgfqpoint{3.539434in}{6.197925in}}%
\pgfpathlineto{\pgfqpoint{3.627169in}{6.197925in}}%
\pgfpathlineto{\pgfqpoint{3.627169in}{6.110189in}}%
\pgfpathlineto{\pgfqpoint{3.539434in}{6.110189in}}%
\pgfpathlineto{\pgfqpoint{3.539434in}{6.197925in}}%
\pgfusepath{stroke,fill}%
\end{pgfscope}%
\begin{pgfscope}%
\pgfpathrectangle{\pgfqpoint{0.380943in}{6.110189in}}{\pgfqpoint{4.650000in}{0.614151in}}%
\pgfusepath{clip}%
\pgfsetbuttcap%
\pgfsetroundjoin%
\definecolor{currentfill}{rgb}{0.968166,0.945882,0.748604}%
\pgfsetfillcolor{currentfill}%
\pgfsetlinewidth{0.250937pt}%
\definecolor{currentstroke}{rgb}{1.000000,1.000000,1.000000}%
\pgfsetstrokecolor{currentstroke}%
\pgfsetdash{}{0pt}%
\pgfpathmoveto{\pgfqpoint{3.627169in}{6.197925in}}%
\pgfpathlineto{\pgfqpoint{3.714905in}{6.197925in}}%
\pgfpathlineto{\pgfqpoint{3.714905in}{6.110189in}}%
\pgfpathlineto{\pgfqpoint{3.627169in}{6.110189in}}%
\pgfpathlineto{\pgfqpoint{3.627169in}{6.197925in}}%
\pgfusepath{stroke,fill}%
\end{pgfscope}%
\begin{pgfscope}%
\pgfpathrectangle{\pgfqpoint{0.380943in}{6.110189in}}{\pgfqpoint{4.650000in}{0.614151in}}%
\pgfusepath{clip}%
\pgfsetbuttcap%
\pgfsetroundjoin%
\definecolor{currentfill}{rgb}{0.965444,0.906113,0.711757}%
\pgfsetfillcolor{currentfill}%
\pgfsetlinewidth{0.250937pt}%
\definecolor{currentstroke}{rgb}{1.000000,1.000000,1.000000}%
\pgfsetstrokecolor{currentstroke}%
\pgfsetdash{}{0pt}%
\pgfpathmoveto{\pgfqpoint{3.714905in}{6.197925in}}%
\pgfpathlineto{\pgfqpoint{3.802641in}{6.197925in}}%
\pgfpathlineto{\pgfqpoint{3.802641in}{6.110189in}}%
\pgfpathlineto{\pgfqpoint{3.714905in}{6.110189in}}%
\pgfpathlineto{\pgfqpoint{3.714905in}{6.197925in}}%
\pgfusepath{stroke,fill}%
\end{pgfscope}%
\begin{pgfscope}%
\pgfpathrectangle{\pgfqpoint{0.380943in}{6.110189in}}{\pgfqpoint{4.650000in}{0.614151in}}%
\pgfusepath{clip}%
\pgfsetbuttcap%
\pgfsetroundjoin%
\definecolor{currentfill}{rgb}{0.991849,0.986144,0.810181}%
\pgfsetfillcolor{currentfill}%
\pgfsetlinewidth{0.250937pt}%
\definecolor{currentstroke}{rgb}{1.000000,1.000000,1.000000}%
\pgfsetstrokecolor{currentstroke}%
\pgfsetdash{}{0pt}%
\pgfpathmoveto{\pgfqpoint{3.802641in}{6.197925in}}%
\pgfpathlineto{\pgfqpoint{3.890377in}{6.197925in}}%
\pgfpathlineto{\pgfqpoint{3.890377in}{6.110189in}}%
\pgfpathlineto{\pgfqpoint{3.802641in}{6.110189in}}%
\pgfpathlineto{\pgfqpoint{3.802641in}{6.197925in}}%
\pgfusepath{stroke,fill}%
\end{pgfscope}%
\begin{pgfscope}%
\pgfpathrectangle{\pgfqpoint{0.380943in}{6.110189in}}{\pgfqpoint{4.650000in}{0.614151in}}%
\pgfusepath{clip}%
\pgfsetbuttcap%
\pgfsetroundjoin%
\definecolor{currentfill}{rgb}{0.986759,0.806398,0.641200}%
\pgfsetfillcolor{currentfill}%
\pgfsetlinewidth{0.250937pt}%
\definecolor{currentstroke}{rgb}{1.000000,1.000000,1.000000}%
\pgfsetstrokecolor{currentstroke}%
\pgfsetdash{}{0pt}%
\pgfpathmoveto{\pgfqpoint{3.890377in}{6.197925in}}%
\pgfpathlineto{\pgfqpoint{3.978113in}{6.197925in}}%
\pgfpathlineto{\pgfqpoint{3.978113in}{6.110189in}}%
\pgfpathlineto{\pgfqpoint{3.890377in}{6.110189in}}%
\pgfpathlineto{\pgfqpoint{3.890377in}{6.197925in}}%
\pgfusepath{stroke,fill}%
\end{pgfscope}%
\begin{pgfscope}%
\pgfpathrectangle{\pgfqpoint{0.380943in}{6.110189in}}{\pgfqpoint{4.650000in}{0.614151in}}%
\pgfusepath{clip}%
\pgfsetbuttcap%
\pgfsetroundjoin%
\definecolor{currentfill}{rgb}{0.965444,0.906113,0.711757}%
\pgfsetfillcolor{currentfill}%
\pgfsetlinewidth{0.250937pt}%
\definecolor{currentstroke}{rgb}{1.000000,1.000000,1.000000}%
\pgfsetstrokecolor{currentstroke}%
\pgfsetdash{}{0pt}%
\pgfpathmoveto{\pgfqpoint{3.978113in}{6.197925in}}%
\pgfpathlineto{\pgfqpoint{4.065849in}{6.197925in}}%
\pgfpathlineto{\pgfqpoint{4.065849in}{6.110189in}}%
\pgfpathlineto{\pgfqpoint{3.978113in}{6.110189in}}%
\pgfpathlineto{\pgfqpoint{3.978113in}{6.197925in}}%
\pgfusepath{stroke,fill}%
\end{pgfscope}%
\begin{pgfscope}%
\pgfpathrectangle{\pgfqpoint{0.380943in}{6.110189in}}{\pgfqpoint{4.650000in}{0.614151in}}%
\pgfusepath{clip}%
\pgfsetbuttcap%
\pgfsetroundjoin%
\definecolor{currentfill}{rgb}{0.968166,0.945882,0.748604}%
\pgfsetfillcolor{currentfill}%
\pgfsetlinewidth{0.250937pt}%
\definecolor{currentstroke}{rgb}{1.000000,1.000000,1.000000}%
\pgfsetstrokecolor{currentstroke}%
\pgfsetdash{}{0pt}%
\pgfpathmoveto{\pgfqpoint{4.065849in}{6.197925in}}%
\pgfpathlineto{\pgfqpoint{4.153585in}{6.197925in}}%
\pgfpathlineto{\pgfqpoint{4.153585in}{6.110189in}}%
\pgfpathlineto{\pgfqpoint{4.065849in}{6.110189in}}%
\pgfpathlineto{\pgfqpoint{4.065849in}{6.197925in}}%
\pgfusepath{stroke,fill}%
\end{pgfscope}%
\begin{pgfscope}%
\pgfpathrectangle{\pgfqpoint{0.380943in}{6.110189in}}{\pgfqpoint{4.650000in}{0.614151in}}%
\pgfusepath{clip}%
\pgfsetbuttcap%
\pgfsetroundjoin%
\definecolor{currentfill}{rgb}{0.962414,0.923552,0.722891}%
\pgfsetfillcolor{currentfill}%
\pgfsetlinewidth{0.250937pt}%
\definecolor{currentstroke}{rgb}{1.000000,1.000000,1.000000}%
\pgfsetstrokecolor{currentstroke}%
\pgfsetdash{}{0pt}%
\pgfpathmoveto{\pgfqpoint{4.153585in}{6.197925in}}%
\pgfpathlineto{\pgfqpoint{4.241320in}{6.197925in}}%
\pgfpathlineto{\pgfqpoint{4.241320in}{6.110189in}}%
\pgfpathlineto{\pgfqpoint{4.153585in}{6.110189in}}%
\pgfpathlineto{\pgfqpoint{4.153585in}{6.197925in}}%
\pgfusepath{stroke,fill}%
\end{pgfscope}%
\begin{pgfscope}%
\pgfpathrectangle{\pgfqpoint{0.380943in}{6.110189in}}{\pgfqpoint{4.650000in}{0.614151in}}%
\pgfusepath{clip}%
\pgfsetbuttcap%
\pgfsetroundjoin%
\definecolor{currentfill}{rgb}{1.000000,1.000000,0.870204}%
\pgfsetfillcolor{currentfill}%
\pgfsetlinewidth{0.250937pt}%
\definecolor{currentstroke}{rgb}{1.000000,1.000000,1.000000}%
\pgfsetstrokecolor{currentstroke}%
\pgfsetdash{}{0pt}%
\pgfpathmoveto{\pgfqpoint{4.241320in}{6.197925in}}%
\pgfpathlineto{\pgfqpoint{4.329056in}{6.197925in}}%
\pgfpathlineto{\pgfqpoint{4.329056in}{6.110189in}}%
\pgfpathlineto{\pgfqpoint{4.241320in}{6.110189in}}%
\pgfpathlineto{\pgfqpoint{4.241320in}{6.197925in}}%
\pgfusepath{stroke,fill}%
\end{pgfscope}%
\begin{pgfscope}%
\pgfpathrectangle{\pgfqpoint{0.380943in}{6.110189in}}{\pgfqpoint{4.650000in}{0.614151in}}%
\pgfusepath{clip}%
\pgfsetbuttcap%
\pgfsetroundjoin%
\definecolor{currentfill}{rgb}{0.991849,0.986144,0.810181}%
\pgfsetfillcolor{currentfill}%
\pgfsetlinewidth{0.250937pt}%
\definecolor{currentstroke}{rgb}{1.000000,1.000000,1.000000}%
\pgfsetstrokecolor{currentstroke}%
\pgfsetdash{}{0pt}%
\pgfpathmoveto{\pgfqpoint{4.329056in}{6.197925in}}%
\pgfpathlineto{\pgfqpoint{4.416792in}{6.197925in}}%
\pgfpathlineto{\pgfqpoint{4.416792in}{6.110189in}}%
\pgfpathlineto{\pgfqpoint{4.329056in}{6.110189in}}%
\pgfpathlineto{\pgfqpoint{4.329056in}{6.197925in}}%
\pgfusepath{stroke,fill}%
\end{pgfscope}%
\begin{pgfscope}%
\pgfpathrectangle{\pgfqpoint{0.380943in}{6.110189in}}{\pgfqpoint{4.650000in}{0.614151in}}%
\pgfusepath{clip}%
\pgfsetbuttcap%
\pgfsetroundjoin%
\definecolor{currentfill}{rgb}{0.968166,0.945882,0.748604}%
\pgfsetfillcolor{currentfill}%
\pgfsetlinewidth{0.250937pt}%
\definecolor{currentstroke}{rgb}{1.000000,1.000000,1.000000}%
\pgfsetstrokecolor{currentstroke}%
\pgfsetdash{}{0pt}%
\pgfpathmoveto{\pgfqpoint{4.416792in}{6.197925in}}%
\pgfpathlineto{\pgfqpoint{4.504528in}{6.197925in}}%
\pgfpathlineto{\pgfqpoint{4.504528in}{6.110189in}}%
\pgfpathlineto{\pgfqpoint{4.416792in}{6.110189in}}%
\pgfpathlineto{\pgfqpoint{4.416792in}{6.197925in}}%
\pgfusepath{stroke,fill}%
\end{pgfscope}%
\begin{pgfscope}%
\pgfpathrectangle{\pgfqpoint{0.380943in}{6.110189in}}{\pgfqpoint{4.650000in}{0.614151in}}%
\pgfusepath{clip}%
\pgfsetbuttcap%
\pgfsetroundjoin%
\definecolor{currentfill}{rgb}{0.968166,0.945882,0.748604}%
\pgfsetfillcolor{currentfill}%
\pgfsetlinewidth{0.250937pt}%
\definecolor{currentstroke}{rgb}{1.000000,1.000000,1.000000}%
\pgfsetstrokecolor{currentstroke}%
\pgfsetdash{}{0pt}%
\pgfpathmoveto{\pgfqpoint{4.504528in}{6.197925in}}%
\pgfpathlineto{\pgfqpoint{4.592264in}{6.197925in}}%
\pgfpathlineto{\pgfqpoint{4.592264in}{6.110189in}}%
\pgfpathlineto{\pgfqpoint{4.504528in}{6.110189in}}%
\pgfpathlineto{\pgfqpoint{4.504528in}{6.197925in}}%
\pgfusepath{stroke,fill}%
\end{pgfscope}%
\begin{pgfscope}%
\pgfpathrectangle{\pgfqpoint{0.380943in}{6.110189in}}{\pgfqpoint{4.650000in}{0.614151in}}%
\pgfusepath{clip}%
\pgfsetbuttcap%
\pgfsetroundjoin%
\definecolor{currentfill}{rgb}{0.962414,0.923552,0.722891}%
\pgfsetfillcolor{currentfill}%
\pgfsetlinewidth{0.250937pt}%
\definecolor{currentstroke}{rgb}{1.000000,1.000000,1.000000}%
\pgfsetstrokecolor{currentstroke}%
\pgfsetdash{}{0pt}%
\pgfpathmoveto{\pgfqpoint{4.592264in}{6.197925in}}%
\pgfpathlineto{\pgfqpoint{4.680000in}{6.197925in}}%
\pgfpathlineto{\pgfqpoint{4.680000in}{6.110189in}}%
\pgfpathlineto{\pgfqpoint{4.592264in}{6.110189in}}%
\pgfpathlineto{\pgfqpoint{4.592264in}{6.197925in}}%
\pgfusepath{stroke,fill}%
\end{pgfscope}%
\begin{pgfscope}%
\pgfpathrectangle{\pgfqpoint{0.380943in}{6.110189in}}{\pgfqpoint{4.650000in}{0.614151in}}%
\pgfusepath{clip}%
\pgfsetbuttcap%
\pgfsetroundjoin%
\definecolor{currentfill}{rgb}{1.000000,1.000000,0.870204}%
\pgfsetfillcolor{currentfill}%
\pgfsetlinewidth{0.250937pt}%
\definecolor{currentstroke}{rgb}{1.000000,1.000000,1.000000}%
\pgfsetstrokecolor{currentstroke}%
\pgfsetdash{}{0pt}%
\pgfpathmoveto{\pgfqpoint{4.680000in}{6.197925in}}%
\pgfpathlineto{\pgfqpoint{4.767736in}{6.197925in}}%
\pgfpathlineto{\pgfqpoint{4.767736in}{6.110189in}}%
\pgfpathlineto{\pgfqpoint{4.680000in}{6.110189in}}%
\pgfpathlineto{\pgfqpoint{4.680000in}{6.197925in}}%
\pgfusepath{stroke,fill}%
\end{pgfscope}%
\begin{pgfscope}%
\pgfpathrectangle{\pgfqpoint{0.380943in}{6.110189in}}{\pgfqpoint{4.650000in}{0.614151in}}%
\pgfusepath{clip}%
\pgfsetbuttcap%
\pgfsetroundjoin%
\definecolor{currentfill}{rgb}{1.000000,1.000000,0.929412}%
\pgfsetfillcolor{currentfill}%
\pgfsetlinewidth{0.250937pt}%
\definecolor{currentstroke}{rgb}{1.000000,1.000000,1.000000}%
\pgfsetstrokecolor{currentstroke}%
\pgfsetdash{}{0pt}%
\pgfpathmoveto{\pgfqpoint{4.767736in}{6.197925in}}%
\pgfpathlineto{\pgfqpoint{4.855471in}{6.197925in}}%
\pgfpathlineto{\pgfqpoint{4.855471in}{6.110189in}}%
\pgfpathlineto{\pgfqpoint{4.767736in}{6.110189in}}%
\pgfpathlineto{\pgfqpoint{4.767736in}{6.197925in}}%
\pgfusepath{stroke,fill}%
\end{pgfscope}%
\begin{pgfscope}%
\pgfpathrectangle{\pgfqpoint{0.380943in}{6.110189in}}{\pgfqpoint{4.650000in}{0.614151in}}%
\pgfusepath{clip}%
\pgfsetbuttcap%
\pgfsetroundjoin%
\definecolor{currentfill}{rgb}{0.991849,0.986144,0.810181}%
\pgfsetfillcolor{currentfill}%
\pgfsetlinewidth{0.250937pt}%
\definecolor{currentstroke}{rgb}{1.000000,1.000000,1.000000}%
\pgfsetstrokecolor{currentstroke}%
\pgfsetdash{}{0pt}%
\pgfpathmoveto{\pgfqpoint{4.855471in}{6.197925in}}%
\pgfpathlineto{\pgfqpoint{4.943207in}{6.197925in}}%
\pgfpathlineto{\pgfqpoint{4.943207in}{6.110189in}}%
\pgfpathlineto{\pgfqpoint{4.855471in}{6.110189in}}%
\pgfpathlineto{\pgfqpoint{4.855471in}{6.197925in}}%
\pgfusepath{stroke,fill}%
\end{pgfscope}%
\begin{pgfscope}%
\pgfpathrectangle{\pgfqpoint{0.380943in}{6.110189in}}{\pgfqpoint{4.650000in}{0.614151in}}%
\pgfusepath{clip}%
\pgfsetbuttcap%
\pgfsetroundjoin%
\pgfsetlinewidth{0.250937pt}%
\definecolor{currentstroke}{rgb}{1.000000,1.000000,1.000000}%
\pgfsetstrokecolor{currentstroke}%
\pgfsetdash{}{0pt}%
\pgfpathmoveto{\pgfqpoint{4.943207in}{6.197925in}}%
\pgfpathlineto{\pgfqpoint{5.030943in}{6.197925in}}%
\pgfpathlineto{\pgfqpoint{5.030943in}{6.110189in}}%
\pgfpathlineto{\pgfqpoint{4.943207in}{6.110189in}}%
\pgfpathlineto{\pgfqpoint{4.943207in}{6.197925in}}%
\pgfusepath{stroke}%
\end{pgfscope}%
\begin{pgfscope}%
\pgfsetbuttcap%
\pgfsetroundjoin%
\definecolor{currentfill}{rgb}{0.000000,0.000000,0.000000}%
\pgfsetfillcolor{currentfill}%
\pgfsetlinewidth{0.803000pt}%
\definecolor{currentstroke}{rgb}{0.000000,0.000000,0.000000}%
\pgfsetstrokecolor{currentstroke}%
\pgfsetdash{}{0pt}%
\pgfsys@defobject{currentmarker}{\pgfqpoint{0.000000in}{-0.048611in}}{\pgfqpoint{0.000000in}{0.000000in}}{%
\pgfpathmoveto{\pgfqpoint{0.000000in}{0.000000in}}%
\pgfpathlineto{\pgfqpoint{0.000000in}{-0.048611in}}%
\pgfusepath{stroke,fill}%
}%
\begin{pgfscope}%
\pgfsys@transformshift{0.600283in}{6.110189in}%
\pgfsys@useobject{currentmarker}{}%
\end{pgfscope}%
\end{pgfscope}%
\begin{pgfscope}%
\definecolor{textcolor}{rgb}{0.000000,0.000000,0.000000}%
\pgfsetstrokecolor{textcolor}%
\pgfsetfillcolor{textcolor}%
\pgftext[x=0.600283in,y=6.012967in,,top]{\color{textcolor}\rmfamily\fontsize{8.000000}{9.600000}\selectfont Jan}%
\end{pgfscope}%
\begin{pgfscope}%
\pgfsetbuttcap%
\pgfsetroundjoin%
\definecolor{currentfill}{rgb}{0.000000,0.000000,0.000000}%
\pgfsetfillcolor{currentfill}%
\pgfsetlinewidth{0.803000pt}%
\definecolor{currentstroke}{rgb}{0.000000,0.000000,0.000000}%
\pgfsetstrokecolor{currentstroke}%
\pgfsetdash{}{0pt}%
\pgfsys@defobject{currentmarker}{\pgfqpoint{0.000000in}{-0.048611in}}{\pgfqpoint{0.000000in}{0.000000in}}{%
\pgfpathmoveto{\pgfqpoint{0.000000in}{0.000000in}}%
\pgfpathlineto{\pgfqpoint{0.000000in}{-0.048611in}}%
\pgfusepath{stroke,fill}%
}%
\begin{pgfscope}%
\pgfsys@transformshift{0.951226in}{6.110189in}%
\pgfsys@useobject{currentmarker}{}%
\end{pgfscope}%
\end{pgfscope}%
\begin{pgfscope}%
\definecolor{textcolor}{rgb}{0.000000,0.000000,0.000000}%
\pgfsetstrokecolor{textcolor}%
\pgfsetfillcolor{textcolor}%
\pgftext[x=0.951226in,y=6.012967in,,top]{\color{textcolor}\rmfamily\fontsize{8.000000}{9.600000}\selectfont Feb}%
\end{pgfscope}%
\begin{pgfscope}%
\pgfsetbuttcap%
\pgfsetroundjoin%
\definecolor{currentfill}{rgb}{0.000000,0.000000,0.000000}%
\pgfsetfillcolor{currentfill}%
\pgfsetlinewidth{0.803000pt}%
\definecolor{currentstroke}{rgb}{0.000000,0.000000,0.000000}%
\pgfsetstrokecolor{currentstroke}%
\pgfsetdash{}{0pt}%
\pgfsys@defobject{currentmarker}{\pgfqpoint{0.000000in}{-0.048611in}}{\pgfqpoint{0.000000in}{0.000000in}}{%
\pgfpathmoveto{\pgfqpoint{0.000000in}{0.000000in}}%
\pgfpathlineto{\pgfqpoint{0.000000in}{-0.048611in}}%
\pgfusepath{stroke,fill}%
}%
\begin{pgfscope}%
\pgfsys@transformshift{1.302169in}{6.110189in}%
\pgfsys@useobject{currentmarker}{}%
\end{pgfscope}%
\end{pgfscope}%
\begin{pgfscope}%
\definecolor{textcolor}{rgb}{0.000000,0.000000,0.000000}%
\pgfsetstrokecolor{textcolor}%
\pgfsetfillcolor{textcolor}%
\pgftext[x=1.302169in,y=6.012967in,,top]{\color{textcolor}\rmfamily\fontsize{8.000000}{9.600000}\selectfont Mar}%
\end{pgfscope}%
\begin{pgfscope}%
\pgfsetbuttcap%
\pgfsetroundjoin%
\definecolor{currentfill}{rgb}{0.000000,0.000000,0.000000}%
\pgfsetfillcolor{currentfill}%
\pgfsetlinewidth{0.803000pt}%
\definecolor{currentstroke}{rgb}{0.000000,0.000000,0.000000}%
\pgfsetstrokecolor{currentstroke}%
\pgfsetdash{}{0pt}%
\pgfsys@defobject{currentmarker}{\pgfqpoint{0.000000in}{-0.048611in}}{\pgfqpoint{0.000000in}{0.000000in}}{%
\pgfpathmoveto{\pgfqpoint{0.000000in}{0.000000in}}%
\pgfpathlineto{\pgfqpoint{0.000000in}{-0.048611in}}%
\pgfusepath{stroke,fill}%
}%
\begin{pgfscope}%
\pgfsys@transformshift{1.696981in}{6.110189in}%
\pgfsys@useobject{currentmarker}{}%
\end{pgfscope}%
\end{pgfscope}%
\begin{pgfscope}%
\definecolor{textcolor}{rgb}{0.000000,0.000000,0.000000}%
\pgfsetstrokecolor{textcolor}%
\pgfsetfillcolor{textcolor}%
\pgftext[x=1.696981in,y=6.012967in,,top]{\color{textcolor}\rmfamily\fontsize{8.000000}{9.600000}\selectfont Apr}%
\end{pgfscope}%
\begin{pgfscope}%
\pgfsetbuttcap%
\pgfsetroundjoin%
\definecolor{currentfill}{rgb}{0.000000,0.000000,0.000000}%
\pgfsetfillcolor{currentfill}%
\pgfsetlinewidth{0.803000pt}%
\definecolor{currentstroke}{rgb}{0.000000,0.000000,0.000000}%
\pgfsetstrokecolor{currentstroke}%
\pgfsetdash{}{0pt}%
\pgfsys@defobject{currentmarker}{\pgfqpoint{0.000000in}{-0.048611in}}{\pgfqpoint{0.000000in}{0.000000in}}{%
\pgfpathmoveto{\pgfqpoint{0.000000in}{0.000000in}}%
\pgfpathlineto{\pgfqpoint{0.000000in}{-0.048611in}}%
\pgfusepath{stroke,fill}%
}%
\begin{pgfscope}%
\pgfsys@transformshift{2.091792in}{6.110189in}%
\pgfsys@useobject{currentmarker}{}%
\end{pgfscope}%
\end{pgfscope}%
\begin{pgfscope}%
\definecolor{textcolor}{rgb}{0.000000,0.000000,0.000000}%
\pgfsetstrokecolor{textcolor}%
\pgfsetfillcolor{textcolor}%
\pgftext[x=2.091792in,y=6.012967in,,top]{\color{textcolor}\rmfamily\fontsize{8.000000}{9.600000}\selectfont May}%
\end{pgfscope}%
\begin{pgfscope}%
\pgfsetbuttcap%
\pgfsetroundjoin%
\definecolor{currentfill}{rgb}{0.000000,0.000000,0.000000}%
\pgfsetfillcolor{currentfill}%
\pgfsetlinewidth{0.803000pt}%
\definecolor{currentstroke}{rgb}{0.000000,0.000000,0.000000}%
\pgfsetstrokecolor{currentstroke}%
\pgfsetdash{}{0pt}%
\pgfsys@defobject{currentmarker}{\pgfqpoint{0.000000in}{-0.048611in}}{\pgfqpoint{0.000000in}{0.000000in}}{%
\pgfpathmoveto{\pgfqpoint{0.000000in}{0.000000in}}%
\pgfpathlineto{\pgfqpoint{0.000000in}{-0.048611in}}%
\pgfusepath{stroke,fill}%
}%
\begin{pgfscope}%
\pgfsys@transformshift{2.442736in}{6.110189in}%
\pgfsys@useobject{currentmarker}{}%
\end{pgfscope}%
\end{pgfscope}%
\begin{pgfscope}%
\definecolor{textcolor}{rgb}{0.000000,0.000000,0.000000}%
\pgfsetstrokecolor{textcolor}%
\pgfsetfillcolor{textcolor}%
\pgftext[x=2.442736in,y=6.012967in,,top]{\color{textcolor}\rmfamily\fontsize{8.000000}{9.600000}\selectfont Jun}%
\end{pgfscope}%
\begin{pgfscope}%
\pgfsetbuttcap%
\pgfsetroundjoin%
\definecolor{currentfill}{rgb}{0.000000,0.000000,0.000000}%
\pgfsetfillcolor{currentfill}%
\pgfsetlinewidth{0.803000pt}%
\definecolor{currentstroke}{rgb}{0.000000,0.000000,0.000000}%
\pgfsetstrokecolor{currentstroke}%
\pgfsetdash{}{0pt}%
\pgfsys@defobject{currentmarker}{\pgfqpoint{0.000000in}{-0.048611in}}{\pgfqpoint{0.000000in}{0.000000in}}{%
\pgfpathmoveto{\pgfqpoint{0.000000in}{0.000000in}}%
\pgfpathlineto{\pgfqpoint{0.000000in}{-0.048611in}}%
\pgfusepath{stroke,fill}%
}%
\begin{pgfscope}%
\pgfsys@transformshift{2.837547in}{6.110189in}%
\pgfsys@useobject{currentmarker}{}%
\end{pgfscope}%
\end{pgfscope}%
\begin{pgfscope}%
\definecolor{textcolor}{rgb}{0.000000,0.000000,0.000000}%
\pgfsetstrokecolor{textcolor}%
\pgfsetfillcolor{textcolor}%
\pgftext[x=2.837547in,y=6.012967in,,top]{\color{textcolor}\rmfamily\fontsize{8.000000}{9.600000}\selectfont Jul}%
\end{pgfscope}%
\begin{pgfscope}%
\pgfsetbuttcap%
\pgfsetroundjoin%
\definecolor{currentfill}{rgb}{0.000000,0.000000,0.000000}%
\pgfsetfillcolor{currentfill}%
\pgfsetlinewidth{0.803000pt}%
\definecolor{currentstroke}{rgb}{0.000000,0.000000,0.000000}%
\pgfsetstrokecolor{currentstroke}%
\pgfsetdash{}{0pt}%
\pgfsys@defobject{currentmarker}{\pgfqpoint{0.000000in}{-0.048611in}}{\pgfqpoint{0.000000in}{0.000000in}}{%
\pgfpathmoveto{\pgfqpoint{0.000000in}{0.000000in}}%
\pgfpathlineto{\pgfqpoint{0.000000in}{-0.048611in}}%
\pgfusepath{stroke,fill}%
}%
\begin{pgfscope}%
\pgfsys@transformshift{3.232358in}{6.110189in}%
\pgfsys@useobject{currentmarker}{}%
\end{pgfscope}%
\end{pgfscope}%
\begin{pgfscope}%
\definecolor{textcolor}{rgb}{0.000000,0.000000,0.000000}%
\pgfsetstrokecolor{textcolor}%
\pgfsetfillcolor{textcolor}%
\pgftext[x=3.232358in,y=6.012967in,,top]{\color{textcolor}\rmfamily\fontsize{8.000000}{9.600000}\selectfont Aug}%
\end{pgfscope}%
\begin{pgfscope}%
\pgfsetbuttcap%
\pgfsetroundjoin%
\definecolor{currentfill}{rgb}{0.000000,0.000000,0.000000}%
\pgfsetfillcolor{currentfill}%
\pgfsetlinewidth{0.803000pt}%
\definecolor{currentstroke}{rgb}{0.000000,0.000000,0.000000}%
\pgfsetstrokecolor{currentstroke}%
\pgfsetdash{}{0pt}%
\pgfsys@defobject{currentmarker}{\pgfqpoint{0.000000in}{-0.048611in}}{\pgfqpoint{0.000000in}{0.000000in}}{%
\pgfpathmoveto{\pgfqpoint{0.000000in}{0.000000in}}%
\pgfpathlineto{\pgfqpoint{0.000000in}{-0.048611in}}%
\pgfusepath{stroke,fill}%
}%
\begin{pgfscope}%
\pgfsys@transformshift{3.583302in}{6.110189in}%
\pgfsys@useobject{currentmarker}{}%
\end{pgfscope}%
\end{pgfscope}%
\begin{pgfscope}%
\definecolor{textcolor}{rgb}{0.000000,0.000000,0.000000}%
\pgfsetstrokecolor{textcolor}%
\pgfsetfillcolor{textcolor}%
\pgftext[x=3.583302in,y=6.012967in,,top]{\color{textcolor}\rmfamily\fontsize{8.000000}{9.600000}\selectfont Sep}%
\end{pgfscope}%
\begin{pgfscope}%
\pgfsetbuttcap%
\pgfsetroundjoin%
\definecolor{currentfill}{rgb}{0.000000,0.000000,0.000000}%
\pgfsetfillcolor{currentfill}%
\pgfsetlinewidth{0.803000pt}%
\definecolor{currentstroke}{rgb}{0.000000,0.000000,0.000000}%
\pgfsetstrokecolor{currentstroke}%
\pgfsetdash{}{0pt}%
\pgfsys@defobject{currentmarker}{\pgfqpoint{0.000000in}{-0.048611in}}{\pgfqpoint{0.000000in}{0.000000in}}{%
\pgfpathmoveto{\pgfqpoint{0.000000in}{0.000000in}}%
\pgfpathlineto{\pgfqpoint{0.000000in}{-0.048611in}}%
\pgfusepath{stroke,fill}%
}%
\begin{pgfscope}%
\pgfsys@transformshift{4.021981in}{6.110189in}%
\pgfsys@useobject{currentmarker}{}%
\end{pgfscope}%
\end{pgfscope}%
\begin{pgfscope}%
\definecolor{textcolor}{rgb}{0.000000,0.000000,0.000000}%
\pgfsetstrokecolor{textcolor}%
\pgfsetfillcolor{textcolor}%
\pgftext[x=4.021981in,y=6.012967in,,top]{\color{textcolor}\rmfamily\fontsize{8.000000}{9.600000}\selectfont Oct}%
\end{pgfscope}%
\begin{pgfscope}%
\pgfsetbuttcap%
\pgfsetroundjoin%
\definecolor{currentfill}{rgb}{0.000000,0.000000,0.000000}%
\pgfsetfillcolor{currentfill}%
\pgfsetlinewidth{0.803000pt}%
\definecolor{currentstroke}{rgb}{0.000000,0.000000,0.000000}%
\pgfsetstrokecolor{currentstroke}%
\pgfsetdash{}{0pt}%
\pgfsys@defobject{currentmarker}{\pgfqpoint{0.000000in}{-0.048611in}}{\pgfqpoint{0.000000in}{0.000000in}}{%
\pgfpathmoveto{\pgfqpoint{0.000000in}{0.000000in}}%
\pgfpathlineto{\pgfqpoint{0.000000in}{-0.048611in}}%
\pgfusepath{stroke,fill}%
}%
\begin{pgfscope}%
\pgfsys@transformshift{4.372924in}{6.110189in}%
\pgfsys@useobject{currentmarker}{}%
\end{pgfscope}%
\end{pgfscope}%
\begin{pgfscope}%
\definecolor{textcolor}{rgb}{0.000000,0.000000,0.000000}%
\pgfsetstrokecolor{textcolor}%
\pgfsetfillcolor{textcolor}%
\pgftext[x=4.372924in,y=6.012967in,,top]{\color{textcolor}\rmfamily\fontsize{8.000000}{9.600000}\selectfont Nov}%
\end{pgfscope}%
\begin{pgfscope}%
\pgfsetbuttcap%
\pgfsetroundjoin%
\definecolor{currentfill}{rgb}{0.000000,0.000000,0.000000}%
\pgfsetfillcolor{currentfill}%
\pgfsetlinewidth{0.803000pt}%
\definecolor{currentstroke}{rgb}{0.000000,0.000000,0.000000}%
\pgfsetstrokecolor{currentstroke}%
\pgfsetdash{}{0pt}%
\pgfsys@defobject{currentmarker}{\pgfqpoint{0.000000in}{-0.048611in}}{\pgfqpoint{0.000000in}{0.000000in}}{%
\pgfpathmoveto{\pgfqpoint{0.000000in}{0.000000in}}%
\pgfpathlineto{\pgfqpoint{0.000000in}{-0.048611in}}%
\pgfusepath{stroke,fill}%
}%
\begin{pgfscope}%
\pgfsys@transformshift{4.767736in}{6.110189in}%
\pgfsys@useobject{currentmarker}{}%
\end{pgfscope}%
\end{pgfscope}%
\begin{pgfscope}%
\definecolor{textcolor}{rgb}{0.000000,0.000000,0.000000}%
\pgfsetstrokecolor{textcolor}%
\pgfsetfillcolor{textcolor}%
\pgftext[x=4.767736in,y=6.012967in,,top]{\color{textcolor}\rmfamily\fontsize{8.000000}{9.600000}\selectfont Dec}%
\end{pgfscope}%
\begin{pgfscope}%
\pgfsetbuttcap%
\pgfsetroundjoin%
\definecolor{currentfill}{rgb}{0.000000,0.000000,0.000000}%
\pgfsetfillcolor{currentfill}%
\pgfsetlinewidth{0.803000pt}%
\definecolor{currentstroke}{rgb}{0.000000,0.000000,0.000000}%
\pgfsetstrokecolor{currentstroke}%
\pgfsetdash{}{0pt}%
\pgfsys@defobject{currentmarker}{\pgfqpoint{-0.048611in}{0.000000in}}{\pgfqpoint{-0.000000in}{0.000000in}}{%
\pgfpathmoveto{\pgfqpoint{-0.000000in}{0.000000in}}%
\pgfpathlineto{\pgfqpoint{-0.048611in}{0.000000in}}%
\pgfusepath{stroke,fill}%
}%
\begin{pgfscope}%
\pgfsys@transformshift{0.380943in}{6.680472in}%
\pgfsys@useobject{currentmarker}{}%
\end{pgfscope}%
\end{pgfscope}%
\begin{pgfscope}%
\definecolor{textcolor}{rgb}{0.000000,0.000000,0.000000}%
\pgfsetstrokecolor{textcolor}%
\pgfsetfillcolor{textcolor}%
\pgftext[x=0.113117in, y=6.641892in, left, base]{\color{textcolor}\rmfamily\fontsize{8.000000}{9.600000}\selectfont M}%
\end{pgfscope}%
\begin{pgfscope}%
\pgfsetbuttcap%
\pgfsetroundjoin%
\definecolor{currentfill}{rgb}{0.000000,0.000000,0.000000}%
\pgfsetfillcolor{currentfill}%
\pgfsetlinewidth{0.803000pt}%
\definecolor{currentstroke}{rgb}{0.000000,0.000000,0.000000}%
\pgfsetstrokecolor{currentstroke}%
\pgfsetdash{}{0pt}%
\pgfsys@defobject{currentmarker}{\pgfqpoint{-0.048611in}{0.000000in}}{\pgfqpoint{-0.000000in}{0.000000in}}{%
\pgfpathmoveto{\pgfqpoint{-0.000000in}{0.000000in}}%
\pgfpathlineto{\pgfqpoint{-0.048611in}{0.000000in}}%
\pgfusepath{stroke,fill}%
}%
\begin{pgfscope}%
\pgfsys@transformshift{0.380943in}{6.592736in}%
\pgfsys@useobject{currentmarker}{}%
\end{pgfscope}%
\end{pgfscope}%
\begin{pgfscope}%
\definecolor{textcolor}{rgb}{0.000000,0.000000,0.000000}%
\pgfsetstrokecolor{textcolor}%
\pgfsetfillcolor{textcolor}%
\pgftext[x=0.135957in, y=6.554156in, left, base]{\color{textcolor}\rmfamily\fontsize{8.000000}{9.600000}\selectfont T}%
\end{pgfscope}%
\begin{pgfscope}%
\pgfsetbuttcap%
\pgfsetroundjoin%
\definecolor{currentfill}{rgb}{0.000000,0.000000,0.000000}%
\pgfsetfillcolor{currentfill}%
\pgfsetlinewidth{0.803000pt}%
\definecolor{currentstroke}{rgb}{0.000000,0.000000,0.000000}%
\pgfsetstrokecolor{currentstroke}%
\pgfsetdash{}{0pt}%
\pgfsys@defobject{currentmarker}{\pgfqpoint{-0.048611in}{0.000000in}}{\pgfqpoint{-0.000000in}{0.000000in}}{%
\pgfpathmoveto{\pgfqpoint{-0.000000in}{0.000000in}}%
\pgfpathlineto{\pgfqpoint{-0.048611in}{0.000000in}}%
\pgfusepath{stroke,fill}%
}%
\begin{pgfscope}%
\pgfsys@transformshift{0.380943in}{6.505000in}%
\pgfsys@useobject{currentmarker}{}%
\end{pgfscope}%
\end{pgfscope}%
\begin{pgfscope}%
\definecolor{textcolor}{rgb}{0.000000,0.000000,0.000000}%
\pgfsetstrokecolor{textcolor}%
\pgfsetfillcolor{textcolor}%
\pgftext[x=0.100000in, y=6.466420in, left, base]{\color{textcolor}\rmfamily\fontsize{8.000000}{9.600000}\selectfont W}%
\end{pgfscope}%
\begin{pgfscope}%
\pgfsetbuttcap%
\pgfsetroundjoin%
\definecolor{currentfill}{rgb}{0.000000,0.000000,0.000000}%
\pgfsetfillcolor{currentfill}%
\pgfsetlinewidth{0.803000pt}%
\definecolor{currentstroke}{rgb}{0.000000,0.000000,0.000000}%
\pgfsetstrokecolor{currentstroke}%
\pgfsetdash{}{0pt}%
\pgfsys@defobject{currentmarker}{\pgfqpoint{-0.048611in}{0.000000in}}{\pgfqpoint{-0.000000in}{0.000000in}}{%
\pgfpathmoveto{\pgfqpoint{-0.000000in}{0.000000in}}%
\pgfpathlineto{\pgfqpoint{-0.048611in}{0.000000in}}%
\pgfusepath{stroke,fill}%
}%
\begin{pgfscope}%
\pgfsys@transformshift{0.380943in}{6.417264in}%
\pgfsys@useobject{currentmarker}{}%
\end{pgfscope}%
\end{pgfscope}%
\begin{pgfscope}%
\definecolor{textcolor}{rgb}{0.000000,0.000000,0.000000}%
\pgfsetstrokecolor{textcolor}%
\pgfsetfillcolor{textcolor}%
\pgftext[x=0.135957in, y=6.378684in, left, base]{\color{textcolor}\rmfamily\fontsize{8.000000}{9.600000}\selectfont T}%
\end{pgfscope}%
\begin{pgfscope}%
\pgfsetbuttcap%
\pgfsetroundjoin%
\definecolor{currentfill}{rgb}{0.000000,0.000000,0.000000}%
\pgfsetfillcolor{currentfill}%
\pgfsetlinewidth{0.803000pt}%
\definecolor{currentstroke}{rgb}{0.000000,0.000000,0.000000}%
\pgfsetstrokecolor{currentstroke}%
\pgfsetdash{}{0pt}%
\pgfsys@defobject{currentmarker}{\pgfqpoint{-0.048611in}{0.000000in}}{\pgfqpoint{-0.000000in}{0.000000in}}{%
\pgfpathmoveto{\pgfqpoint{-0.000000in}{0.000000in}}%
\pgfpathlineto{\pgfqpoint{-0.048611in}{0.000000in}}%
\pgfusepath{stroke,fill}%
}%
\begin{pgfscope}%
\pgfsys@transformshift{0.380943in}{6.329529in}%
\pgfsys@useobject{currentmarker}{}%
\end{pgfscope}%
\end{pgfscope}%
\begin{pgfscope}%
\definecolor{textcolor}{rgb}{0.000000,0.000000,0.000000}%
\pgfsetstrokecolor{textcolor}%
\pgfsetfillcolor{textcolor}%
\pgftext[x=0.144213in, y=6.290948in, left, base]{\color{textcolor}\rmfamily\fontsize{8.000000}{9.600000}\selectfont F}%
\end{pgfscope}%
\begin{pgfscope}%
\pgfsetbuttcap%
\pgfsetroundjoin%
\definecolor{currentfill}{rgb}{0.000000,0.000000,0.000000}%
\pgfsetfillcolor{currentfill}%
\pgfsetlinewidth{0.803000pt}%
\definecolor{currentstroke}{rgb}{0.000000,0.000000,0.000000}%
\pgfsetstrokecolor{currentstroke}%
\pgfsetdash{}{0pt}%
\pgfsys@defobject{currentmarker}{\pgfqpoint{-0.048611in}{0.000000in}}{\pgfqpoint{-0.000000in}{0.000000in}}{%
\pgfpathmoveto{\pgfqpoint{-0.000000in}{0.000000in}}%
\pgfpathlineto{\pgfqpoint{-0.048611in}{0.000000in}}%
\pgfusepath{stroke,fill}%
}%
\begin{pgfscope}%
\pgfsys@transformshift{0.380943in}{6.241793in}%
\pgfsys@useobject{currentmarker}{}%
\end{pgfscope}%
\end{pgfscope}%
\begin{pgfscope}%
\definecolor{textcolor}{rgb}{0.000000,0.000000,0.000000}%
\pgfsetstrokecolor{textcolor}%
\pgfsetfillcolor{textcolor}%
\pgftext[x=0.155633in, y=6.203212in, left, base]{\color{textcolor}\rmfamily\fontsize{8.000000}{9.600000}\selectfont S}%
\end{pgfscope}%
\begin{pgfscope}%
\pgfsetbuttcap%
\pgfsetroundjoin%
\definecolor{currentfill}{rgb}{0.000000,0.000000,0.000000}%
\pgfsetfillcolor{currentfill}%
\pgfsetlinewidth{0.803000pt}%
\definecolor{currentstroke}{rgb}{0.000000,0.000000,0.000000}%
\pgfsetstrokecolor{currentstroke}%
\pgfsetdash{}{0pt}%
\pgfsys@defobject{currentmarker}{\pgfqpoint{-0.048611in}{0.000000in}}{\pgfqpoint{-0.000000in}{0.000000in}}{%
\pgfpathmoveto{\pgfqpoint{-0.000000in}{0.000000in}}%
\pgfpathlineto{\pgfqpoint{-0.048611in}{0.000000in}}%
\pgfusepath{stroke,fill}%
}%
\begin{pgfscope}%
\pgfsys@transformshift{0.380943in}{6.154057in}%
\pgfsys@useobject{currentmarker}{}%
\end{pgfscope}%
\end{pgfscope}%
\begin{pgfscope}%
\definecolor{textcolor}{rgb}{0.000000,0.000000,0.000000}%
\pgfsetstrokecolor{textcolor}%
\pgfsetfillcolor{textcolor}%
\pgftext[x=0.155633in, y=6.115477in, left, base]{\color{textcolor}\rmfamily\fontsize{8.000000}{9.600000}\selectfont S}%
\end{pgfscope}%
\begin{pgfscope}%
\definecolor{textcolor}{rgb}{0.000000,0.000000,0.000000}%
\pgfsetstrokecolor{textcolor}%
\pgfsetfillcolor{textcolor}%
\pgftext[x=2.705943in,y=6.891007in,,]{\color{textcolor}\ttfamily\fontsize{14.400000}{17.280000}\selectfont 2018}%
\end{pgfscope}%
\begin{pgfscope}%
\pgfpathrectangle{\pgfqpoint{0.380943in}{4.185189in}}{\pgfqpoint{4.650000in}{0.614151in}}%
\pgfusepath{clip}%
\pgfsetbuttcap%
\pgfsetroundjoin%
\pgfsetlinewidth{0.250937pt}%
\definecolor{currentstroke}{rgb}{1.000000,1.000000,1.000000}%
\pgfsetstrokecolor{currentstroke}%
\pgfsetdash{}{0pt}%
\pgfpathmoveto{\pgfqpoint{0.380943in}{4.799340in}}%
\pgfpathlineto{\pgfqpoint{0.468679in}{4.799340in}}%
\pgfpathlineto{\pgfqpoint{0.468679in}{4.711604in}}%
\pgfpathlineto{\pgfqpoint{0.380943in}{4.711604in}}%
\pgfpathlineto{\pgfqpoint{0.380943in}{4.799340in}}%
\pgfusepath{stroke}%
\end{pgfscope}%
\begin{pgfscope}%
\pgfpathrectangle{\pgfqpoint{0.380943in}{4.185189in}}{\pgfqpoint{4.650000in}{0.614151in}}%
\pgfusepath{clip}%
\pgfsetbuttcap%
\pgfsetroundjoin%
\definecolor{currentfill}{rgb}{0.979654,0.837186,0.669619}%
\pgfsetfillcolor{currentfill}%
\pgfsetlinewidth{0.250937pt}%
\definecolor{currentstroke}{rgb}{1.000000,1.000000,1.000000}%
\pgfsetstrokecolor{currentstroke}%
\pgfsetdash{}{0pt}%
\pgfpathmoveto{\pgfqpoint{0.468679in}{4.799340in}}%
\pgfpathlineto{\pgfqpoint{0.556415in}{4.799340in}}%
\pgfpathlineto{\pgfqpoint{0.556415in}{4.711604in}}%
\pgfpathlineto{\pgfqpoint{0.468679in}{4.711604in}}%
\pgfpathlineto{\pgfqpoint{0.468679in}{4.799340in}}%
\pgfusepath{stroke,fill}%
\end{pgfscope}%
\begin{pgfscope}%
\pgfpathrectangle{\pgfqpoint{0.380943in}{4.185189in}}{\pgfqpoint{4.650000in}{0.614151in}}%
\pgfusepath{clip}%
\pgfsetbuttcap%
\pgfsetroundjoin%
\definecolor{currentfill}{rgb}{0.968166,0.945882,0.748604}%
\pgfsetfillcolor{currentfill}%
\pgfsetlinewidth{0.250937pt}%
\definecolor{currentstroke}{rgb}{1.000000,1.000000,1.000000}%
\pgfsetstrokecolor{currentstroke}%
\pgfsetdash{}{0pt}%
\pgfpathmoveto{\pgfqpoint{0.556415in}{4.799340in}}%
\pgfpathlineto{\pgfqpoint{0.644151in}{4.799340in}}%
\pgfpathlineto{\pgfqpoint{0.644151in}{4.711604in}}%
\pgfpathlineto{\pgfqpoint{0.556415in}{4.711604in}}%
\pgfpathlineto{\pgfqpoint{0.556415in}{4.799340in}}%
\pgfusepath{stroke,fill}%
\end{pgfscope}%
\begin{pgfscope}%
\pgfpathrectangle{\pgfqpoint{0.380943in}{4.185189in}}{\pgfqpoint{4.650000in}{0.614151in}}%
\pgfusepath{clip}%
\pgfsetbuttcap%
\pgfsetroundjoin%
\definecolor{currentfill}{rgb}{1.000000,0.557862,0.511772}%
\pgfsetfillcolor{currentfill}%
\pgfsetlinewidth{0.250937pt}%
\definecolor{currentstroke}{rgb}{1.000000,1.000000,1.000000}%
\pgfsetstrokecolor{currentstroke}%
\pgfsetdash{}{0pt}%
\pgfpathmoveto{\pgfqpoint{0.644151in}{4.799340in}}%
\pgfpathlineto{\pgfqpoint{0.731886in}{4.799340in}}%
\pgfpathlineto{\pgfqpoint{0.731886in}{4.711604in}}%
\pgfpathlineto{\pgfqpoint{0.644151in}{4.711604in}}%
\pgfpathlineto{\pgfqpoint{0.644151in}{4.799340in}}%
\pgfusepath{stroke,fill}%
\end{pgfscope}%
\begin{pgfscope}%
\pgfpathrectangle{\pgfqpoint{0.380943in}{4.185189in}}{\pgfqpoint{4.650000in}{0.614151in}}%
\pgfusepath{clip}%
\pgfsetbuttcap%
\pgfsetroundjoin%
\definecolor{currentfill}{rgb}{0.962414,0.923552,0.722891}%
\pgfsetfillcolor{currentfill}%
\pgfsetlinewidth{0.250937pt}%
\definecolor{currentstroke}{rgb}{1.000000,1.000000,1.000000}%
\pgfsetstrokecolor{currentstroke}%
\pgfsetdash{}{0pt}%
\pgfpathmoveto{\pgfqpoint{0.731886in}{4.799340in}}%
\pgfpathlineto{\pgfqpoint{0.819622in}{4.799340in}}%
\pgfpathlineto{\pgfqpoint{0.819622in}{4.711604in}}%
\pgfpathlineto{\pgfqpoint{0.731886in}{4.711604in}}%
\pgfpathlineto{\pgfqpoint{0.731886in}{4.799340in}}%
\pgfusepath{stroke,fill}%
\end{pgfscope}%
\begin{pgfscope}%
\pgfpathrectangle{\pgfqpoint{0.380943in}{4.185189in}}{\pgfqpoint{4.650000in}{0.614151in}}%
\pgfusepath{clip}%
\pgfsetbuttcap%
\pgfsetroundjoin%
\definecolor{currentfill}{rgb}{1.000000,0.557862,0.511772}%
\pgfsetfillcolor{currentfill}%
\pgfsetlinewidth{0.250937pt}%
\definecolor{currentstroke}{rgb}{1.000000,1.000000,1.000000}%
\pgfsetstrokecolor{currentstroke}%
\pgfsetdash{}{0pt}%
\pgfpathmoveto{\pgfqpoint{0.819622in}{4.799340in}}%
\pgfpathlineto{\pgfqpoint{0.907358in}{4.799340in}}%
\pgfpathlineto{\pgfqpoint{0.907358in}{4.711604in}}%
\pgfpathlineto{\pgfqpoint{0.819622in}{4.711604in}}%
\pgfpathlineto{\pgfqpoint{0.819622in}{4.799340in}}%
\pgfusepath{stroke,fill}%
\end{pgfscope}%
\begin{pgfscope}%
\pgfpathrectangle{\pgfqpoint{0.380943in}{4.185189in}}{\pgfqpoint{4.650000in}{0.614151in}}%
\pgfusepath{clip}%
\pgfsetbuttcap%
\pgfsetroundjoin%
\definecolor{currentfill}{rgb}{0.996571,0.720538,0.589189}%
\pgfsetfillcolor{currentfill}%
\pgfsetlinewidth{0.250937pt}%
\definecolor{currentstroke}{rgb}{1.000000,1.000000,1.000000}%
\pgfsetstrokecolor{currentstroke}%
\pgfsetdash{}{0pt}%
\pgfpathmoveto{\pgfqpoint{0.907358in}{4.799340in}}%
\pgfpathlineto{\pgfqpoint{0.995094in}{4.799340in}}%
\pgfpathlineto{\pgfqpoint{0.995094in}{4.711604in}}%
\pgfpathlineto{\pgfqpoint{0.907358in}{4.711604in}}%
\pgfpathlineto{\pgfqpoint{0.907358in}{4.799340in}}%
\pgfusepath{stroke,fill}%
\end{pgfscope}%
\begin{pgfscope}%
\pgfpathrectangle{\pgfqpoint{0.380943in}{4.185189in}}{\pgfqpoint{4.650000in}{0.614151in}}%
\pgfusepath{clip}%
\pgfsetbuttcap%
\pgfsetroundjoin%
\definecolor{currentfill}{rgb}{0.800000,0.278431,0.278431}%
\pgfsetfillcolor{currentfill}%
\pgfsetlinewidth{0.250937pt}%
\definecolor{currentstroke}{rgb}{1.000000,1.000000,1.000000}%
\pgfsetstrokecolor{currentstroke}%
\pgfsetdash{}{0pt}%
\pgfpathmoveto{\pgfqpoint{0.995094in}{4.799340in}}%
\pgfpathlineto{\pgfqpoint{1.082830in}{4.799340in}}%
\pgfpathlineto{\pgfqpoint{1.082830in}{4.711604in}}%
\pgfpathlineto{\pgfqpoint{0.995094in}{4.711604in}}%
\pgfpathlineto{\pgfqpoint{0.995094in}{4.799340in}}%
\pgfusepath{stroke,fill}%
\end{pgfscope}%
\begin{pgfscope}%
\pgfpathrectangle{\pgfqpoint{0.380943in}{4.185189in}}{\pgfqpoint{4.650000in}{0.614151in}}%
\pgfusepath{clip}%
\pgfsetbuttcap%
\pgfsetroundjoin%
\definecolor{currentfill}{rgb}{1.000000,0.605229,0.530719}%
\pgfsetfillcolor{currentfill}%
\pgfsetlinewidth{0.250937pt}%
\definecolor{currentstroke}{rgb}{1.000000,1.000000,1.000000}%
\pgfsetstrokecolor{currentstroke}%
\pgfsetdash{}{0pt}%
\pgfpathmoveto{\pgfqpoint{1.082830in}{4.799340in}}%
\pgfpathlineto{\pgfqpoint{1.170566in}{4.799340in}}%
\pgfpathlineto{\pgfqpoint{1.170566in}{4.711604in}}%
\pgfpathlineto{\pgfqpoint{1.082830in}{4.711604in}}%
\pgfpathlineto{\pgfqpoint{1.082830in}{4.799340in}}%
\pgfusepath{stroke,fill}%
\end{pgfscope}%
\begin{pgfscope}%
\pgfpathrectangle{\pgfqpoint{0.380943in}{4.185189in}}{\pgfqpoint{4.650000in}{0.614151in}}%
\pgfusepath{clip}%
\pgfsetbuttcap%
\pgfsetroundjoin%
\definecolor{currentfill}{rgb}{0.981546,0.459977,0.459977}%
\pgfsetfillcolor{currentfill}%
\pgfsetlinewidth{0.250937pt}%
\definecolor{currentstroke}{rgb}{1.000000,1.000000,1.000000}%
\pgfsetstrokecolor{currentstroke}%
\pgfsetdash{}{0pt}%
\pgfpathmoveto{\pgfqpoint{1.170566in}{4.799340in}}%
\pgfpathlineto{\pgfqpoint{1.258302in}{4.799340in}}%
\pgfpathlineto{\pgfqpoint{1.258302in}{4.711604in}}%
\pgfpathlineto{\pgfqpoint{1.170566in}{4.711604in}}%
\pgfpathlineto{\pgfqpoint{1.170566in}{4.799340in}}%
\pgfusepath{stroke,fill}%
\end{pgfscope}%
\begin{pgfscope}%
\pgfpathrectangle{\pgfqpoint{0.380943in}{4.185189in}}{\pgfqpoint{4.650000in}{0.614151in}}%
\pgfusepath{clip}%
\pgfsetbuttcap%
\pgfsetroundjoin%
\definecolor{currentfill}{rgb}{0.979654,0.837186,0.669619}%
\pgfsetfillcolor{currentfill}%
\pgfsetlinewidth{0.250937pt}%
\definecolor{currentstroke}{rgb}{1.000000,1.000000,1.000000}%
\pgfsetstrokecolor{currentstroke}%
\pgfsetdash{}{0pt}%
\pgfpathmoveto{\pgfqpoint{1.258302in}{4.799340in}}%
\pgfpathlineto{\pgfqpoint{1.346037in}{4.799340in}}%
\pgfpathlineto{\pgfqpoint{1.346037in}{4.711604in}}%
\pgfpathlineto{\pgfqpoint{1.258302in}{4.711604in}}%
\pgfpathlineto{\pgfqpoint{1.258302in}{4.799340in}}%
\pgfusepath{stroke,fill}%
\end{pgfscope}%
\begin{pgfscope}%
\pgfpathrectangle{\pgfqpoint{0.380943in}{4.185189in}}{\pgfqpoint{4.650000in}{0.614151in}}%
\pgfusepath{clip}%
\pgfsetbuttcap%
\pgfsetroundjoin%
\definecolor{currentfill}{rgb}{0.992326,0.765229,0.614840}%
\pgfsetfillcolor{currentfill}%
\pgfsetlinewidth{0.250937pt}%
\definecolor{currentstroke}{rgb}{1.000000,1.000000,1.000000}%
\pgfsetstrokecolor{currentstroke}%
\pgfsetdash{}{0pt}%
\pgfpathmoveto{\pgfqpoint{1.346037in}{4.799340in}}%
\pgfpathlineto{\pgfqpoint{1.433773in}{4.799340in}}%
\pgfpathlineto{\pgfqpoint{1.433773in}{4.711604in}}%
\pgfpathlineto{\pgfqpoint{1.346037in}{4.711604in}}%
\pgfpathlineto{\pgfqpoint{1.346037in}{4.799340in}}%
\pgfusepath{stroke,fill}%
\end{pgfscope}%
\begin{pgfscope}%
\pgfpathrectangle{\pgfqpoint{0.380943in}{4.185189in}}{\pgfqpoint{4.650000in}{0.614151in}}%
\pgfusepath{clip}%
\pgfsetbuttcap%
\pgfsetroundjoin%
\definecolor{currentfill}{rgb}{0.998939,0.658962,0.556032}%
\pgfsetfillcolor{currentfill}%
\pgfsetlinewidth{0.250937pt}%
\definecolor{currentstroke}{rgb}{1.000000,1.000000,1.000000}%
\pgfsetstrokecolor{currentstroke}%
\pgfsetdash{}{0pt}%
\pgfpathmoveto{\pgfqpoint{1.433773in}{4.799340in}}%
\pgfpathlineto{\pgfqpoint{1.521509in}{4.799340in}}%
\pgfpathlineto{\pgfqpoint{1.521509in}{4.711604in}}%
\pgfpathlineto{\pgfqpoint{1.433773in}{4.711604in}}%
\pgfpathlineto{\pgfqpoint{1.433773in}{4.799340in}}%
\pgfusepath{stroke,fill}%
\end{pgfscope}%
\begin{pgfscope}%
\pgfpathrectangle{\pgfqpoint{0.380943in}{4.185189in}}{\pgfqpoint{4.650000in}{0.614151in}}%
\pgfusepath{clip}%
\pgfsetbuttcap%
\pgfsetroundjoin%
\definecolor{currentfill}{rgb}{0.979654,0.837186,0.669619}%
\pgfsetfillcolor{currentfill}%
\pgfsetlinewidth{0.250937pt}%
\definecolor{currentstroke}{rgb}{1.000000,1.000000,1.000000}%
\pgfsetstrokecolor{currentstroke}%
\pgfsetdash{}{0pt}%
\pgfpathmoveto{\pgfqpoint{1.521509in}{4.799340in}}%
\pgfpathlineto{\pgfqpoint{1.609245in}{4.799340in}}%
\pgfpathlineto{\pgfqpoint{1.609245in}{4.711604in}}%
\pgfpathlineto{\pgfqpoint{1.521509in}{4.711604in}}%
\pgfpathlineto{\pgfqpoint{1.521509in}{4.799340in}}%
\pgfusepath{stroke,fill}%
\end{pgfscope}%
\begin{pgfscope}%
\pgfpathrectangle{\pgfqpoint{0.380943in}{4.185189in}}{\pgfqpoint{4.650000in}{0.614151in}}%
\pgfusepath{clip}%
\pgfsetbuttcap%
\pgfsetroundjoin%
\definecolor{currentfill}{rgb}{0.979654,0.837186,0.669619}%
\pgfsetfillcolor{currentfill}%
\pgfsetlinewidth{0.250937pt}%
\definecolor{currentstroke}{rgb}{1.000000,1.000000,1.000000}%
\pgfsetstrokecolor{currentstroke}%
\pgfsetdash{}{0pt}%
\pgfpathmoveto{\pgfqpoint{1.609245in}{4.799340in}}%
\pgfpathlineto{\pgfqpoint{1.696981in}{4.799340in}}%
\pgfpathlineto{\pgfqpoint{1.696981in}{4.711604in}}%
\pgfpathlineto{\pgfqpoint{1.609245in}{4.711604in}}%
\pgfpathlineto{\pgfqpoint{1.609245in}{4.799340in}}%
\pgfusepath{stroke,fill}%
\end{pgfscope}%
\begin{pgfscope}%
\pgfpathrectangle{\pgfqpoint{0.380943in}{4.185189in}}{\pgfqpoint{4.650000in}{0.614151in}}%
\pgfusepath{clip}%
\pgfsetbuttcap%
\pgfsetroundjoin%
\definecolor{currentfill}{rgb}{0.996571,0.720538,0.589189}%
\pgfsetfillcolor{currentfill}%
\pgfsetlinewidth{0.250937pt}%
\definecolor{currentstroke}{rgb}{1.000000,1.000000,1.000000}%
\pgfsetstrokecolor{currentstroke}%
\pgfsetdash{}{0pt}%
\pgfpathmoveto{\pgfqpoint{1.696981in}{4.799340in}}%
\pgfpathlineto{\pgfqpoint{1.784717in}{4.799340in}}%
\pgfpathlineto{\pgfqpoint{1.784717in}{4.711604in}}%
\pgfpathlineto{\pgfqpoint{1.696981in}{4.711604in}}%
\pgfpathlineto{\pgfqpoint{1.696981in}{4.799340in}}%
\pgfusepath{stroke,fill}%
\end{pgfscope}%
\begin{pgfscope}%
\pgfpathrectangle{\pgfqpoint{0.380943in}{4.185189in}}{\pgfqpoint{4.650000in}{0.614151in}}%
\pgfusepath{clip}%
\pgfsetbuttcap%
\pgfsetroundjoin%
\definecolor{currentfill}{rgb}{0.965444,0.906113,0.711757}%
\pgfsetfillcolor{currentfill}%
\pgfsetlinewidth{0.250937pt}%
\definecolor{currentstroke}{rgb}{1.000000,1.000000,1.000000}%
\pgfsetstrokecolor{currentstroke}%
\pgfsetdash{}{0pt}%
\pgfpathmoveto{\pgfqpoint{1.784717in}{4.799340in}}%
\pgfpathlineto{\pgfqpoint{1.872452in}{4.799340in}}%
\pgfpathlineto{\pgfqpoint{1.872452in}{4.711604in}}%
\pgfpathlineto{\pgfqpoint{1.784717in}{4.711604in}}%
\pgfpathlineto{\pgfqpoint{1.784717in}{4.799340in}}%
\pgfusepath{stroke,fill}%
\end{pgfscope}%
\begin{pgfscope}%
\pgfpathrectangle{\pgfqpoint{0.380943in}{4.185189in}}{\pgfqpoint{4.650000in}{0.614151in}}%
\pgfusepath{clip}%
\pgfsetbuttcap%
\pgfsetroundjoin%
\definecolor{currentfill}{rgb}{0.996571,0.720538,0.589189}%
\pgfsetfillcolor{currentfill}%
\pgfsetlinewidth{0.250937pt}%
\definecolor{currentstroke}{rgb}{1.000000,1.000000,1.000000}%
\pgfsetstrokecolor{currentstroke}%
\pgfsetdash{}{0pt}%
\pgfpathmoveto{\pgfqpoint{1.872452in}{4.799340in}}%
\pgfpathlineto{\pgfqpoint{1.960188in}{4.799340in}}%
\pgfpathlineto{\pgfqpoint{1.960188in}{4.711604in}}%
\pgfpathlineto{\pgfqpoint{1.872452in}{4.711604in}}%
\pgfpathlineto{\pgfqpoint{1.872452in}{4.799340in}}%
\pgfusepath{stroke,fill}%
\end{pgfscope}%
\begin{pgfscope}%
\pgfpathrectangle{\pgfqpoint{0.380943in}{4.185189in}}{\pgfqpoint{4.650000in}{0.614151in}}%
\pgfusepath{clip}%
\pgfsetbuttcap%
\pgfsetroundjoin%
\definecolor{currentfill}{rgb}{0.992326,0.765229,0.614840}%
\pgfsetfillcolor{currentfill}%
\pgfsetlinewidth{0.250937pt}%
\definecolor{currentstroke}{rgb}{1.000000,1.000000,1.000000}%
\pgfsetstrokecolor{currentstroke}%
\pgfsetdash{}{0pt}%
\pgfpathmoveto{\pgfqpoint{1.960188in}{4.799340in}}%
\pgfpathlineto{\pgfqpoint{2.047924in}{4.799340in}}%
\pgfpathlineto{\pgfqpoint{2.047924in}{4.711604in}}%
\pgfpathlineto{\pgfqpoint{1.960188in}{4.711604in}}%
\pgfpathlineto{\pgfqpoint{1.960188in}{4.799340in}}%
\pgfusepath{stroke,fill}%
\end{pgfscope}%
\begin{pgfscope}%
\pgfpathrectangle{\pgfqpoint{0.380943in}{4.185189in}}{\pgfqpoint{4.650000in}{0.614151in}}%
\pgfusepath{clip}%
\pgfsetbuttcap%
\pgfsetroundjoin%
\definecolor{currentfill}{rgb}{0.998939,0.658962,0.556032}%
\pgfsetfillcolor{currentfill}%
\pgfsetlinewidth{0.250937pt}%
\definecolor{currentstroke}{rgb}{1.000000,1.000000,1.000000}%
\pgfsetstrokecolor{currentstroke}%
\pgfsetdash{}{0pt}%
\pgfpathmoveto{\pgfqpoint{2.047924in}{4.799340in}}%
\pgfpathlineto{\pgfqpoint{2.135660in}{4.799340in}}%
\pgfpathlineto{\pgfqpoint{2.135660in}{4.711604in}}%
\pgfpathlineto{\pgfqpoint{2.047924in}{4.711604in}}%
\pgfpathlineto{\pgfqpoint{2.047924in}{4.799340in}}%
\pgfusepath{stroke,fill}%
\end{pgfscope}%
\begin{pgfscope}%
\pgfpathrectangle{\pgfqpoint{0.380943in}{4.185189in}}{\pgfqpoint{4.650000in}{0.614151in}}%
\pgfusepath{clip}%
\pgfsetbuttcap%
\pgfsetroundjoin%
\definecolor{currentfill}{rgb}{0.979654,0.837186,0.669619}%
\pgfsetfillcolor{currentfill}%
\pgfsetlinewidth{0.250937pt}%
\definecolor{currentstroke}{rgb}{1.000000,1.000000,1.000000}%
\pgfsetstrokecolor{currentstroke}%
\pgfsetdash{}{0pt}%
\pgfpathmoveto{\pgfqpoint{2.135660in}{4.799340in}}%
\pgfpathlineto{\pgfqpoint{2.223396in}{4.799340in}}%
\pgfpathlineto{\pgfqpoint{2.223396in}{4.711604in}}%
\pgfpathlineto{\pgfqpoint{2.135660in}{4.711604in}}%
\pgfpathlineto{\pgfqpoint{2.135660in}{4.799340in}}%
\pgfusepath{stroke,fill}%
\end{pgfscope}%
\begin{pgfscope}%
\pgfpathrectangle{\pgfqpoint{0.380943in}{4.185189in}}{\pgfqpoint{4.650000in}{0.614151in}}%
\pgfusepath{clip}%
\pgfsetbuttcap%
\pgfsetroundjoin%
\definecolor{currentfill}{rgb}{0.972549,0.870588,0.692810}%
\pgfsetfillcolor{currentfill}%
\pgfsetlinewidth{0.250937pt}%
\definecolor{currentstroke}{rgb}{1.000000,1.000000,1.000000}%
\pgfsetstrokecolor{currentstroke}%
\pgfsetdash{}{0pt}%
\pgfpathmoveto{\pgfqpoint{2.223396in}{4.799340in}}%
\pgfpathlineto{\pgfqpoint{2.311132in}{4.799340in}}%
\pgfpathlineto{\pgfqpoint{2.311132in}{4.711604in}}%
\pgfpathlineto{\pgfqpoint{2.223396in}{4.711604in}}%
\pgfpathlineto{\pgfqpoint{2.223396in}{4.799340in}}%
\pgfusepath{stroke,fill}%
\end{pgfscope}%
\begin{pgfscope}%
\pgfpathrectangle{\pgfqpoint{0.380943in}{4.185189in}}{\pgfqpoint{4.650000in}{0.614151in}}%
\pgfusepath{clip}%
\pgfsetbuttcap%
\pgfsetroundjoin%
\definecolor{currentfill}{rgb}{0.992326,0.765229,0.614840}%
\pgfsetfillcolor{currentfill}%
\pgfsetlinewidth{0.250937pt}%
\definecolor{currentstroke}{rgb}{1.000000,1.000000,1.000000}%
\pgfsetstrokecolor{currentstroke}%
\pgfsetdash{}{0pt}%
\pgfpathmoveto{\pgfqpoint{2.311132in}{4.799340in}}%
\pgfpathlineto{\pgfqpoint{2.398868in}{4.799340in}}%
\pgfpathlineto{\pgfqpoint{2.398868in}{4.711604in}}%
\pgfpathlineto{\pgfqpoint{2.311132in}{4.711604in}}%
\pgfpathlineto{\pgfqpoint{2.311132in}{4.799340in}}%
\pgfusepath{stroke,fill}%
\end{pgfscope}%
\begin{pgfscope}%
\pgfpathrectangle{\pgfqpoint{0.380943in}{4.185189in}}{\pgfqpoint{4.650000in}{0.614151in}}%
\pgfusepath{clip}%
\pgfsetbuttcap%
\pgfsetroundjoin%
\definecolor{currentfill}{rgb}{0.968166,0.945882,0.748604}%
\pgfsetfillcolor{currentfill}%
\pgfsetlinewidth{0.250937pt}%
\definecolor{currentstroke}{rgb}{1.000000,1.000000,1.000000}%
\pgfsetstrokecolor{currentstroke}%
\pgfsetdash{}{0pt}%
\pgfpathmoveto{\pgfqpoint{2.398868in}{4.799340in}}%
\pgfpathlineto{\pgfqpoint{2.486603in}{4.799340in}}%
\pgfpathlineto{\pgfqpoint{2.486603in}{4.711604in}}%
\pgfpathlineto{\pgfqpoint{2.398868in}{4.711604in}}%
\pgfpathlineto{\pgfqpoint{2.398868in}{4.799340in}}%
\pgfusepath{stroke,fill}%
\end{pgfscope}%
\begin{pgfscope}%
\pgfpathrectangle{\pgfqpoint{0.380943in}{4.185189in}}{\pgfqpoint{4.650000in}{0.614151in}}%
\pgfusepath{clip}%
\pgfsetbuttcap%
\pgfsetroundjoin%
\definecolor{currentfill}{rgb}{0.979654,0.837186,0.669619}%
\pgfsetfillcolor{currentfill}%
\pgfsetlinewidth{0.250937pt}%
\definecolor{currentstroke}{rgb}{1.000000,1.000000,1.000000}%
\pgfsetstrokecolor{currentstroke}%
\pgfsetdash{}{0pt}%
\pgfpathmoveto{\pgfqpoint{2.486603in}{4.799340in}}%
\pgfpathlineto{\pgfqpoint{2.574339in}{4.799340in}}%
\pgfpathlineto{\pgfqpoint{2.574339in}{4.711604in}}%
\pgfpathlineto{\pgfqpoint{2.486603in}{4.711604in}}%
\pgfpathlineto{\pgfqpoint{2.486603in}{4.799340in}}%
\pgfusepath{stroke,fill}%
\end{pgfscope}%
\begin{pgfscope}%
\pgfpathrectangle{\pgfqpoint{0.380943in}{4.185189in}}{\pgfqpoint{4.650000in}{0.614151in}}%
\pgfusepath{clip}%
\pgfsetbuttcap%
\pgfsetroundjoin%
\definecolor{currentfill}{rgb}{0.979654,0.837186,0.669619}%
\pgfsetfillcolor{currentfill}%
\pgfsetlinewidth{0.250937pt}%
\definecolor{currentstroke}{rgb}{1.000000,1.000000,1.000000}%
\pgfsetstrokecolor{currentstroke}%
\pgfsetdash{}{0pt}%
\pgfpathmoveto{\pgfqpoint{2.574339in}{4.799340in}}%
\pgfpathlineto{\pgfqpoint{2.662075in}{4.799340in}}%
\pgfpathlineto{\pgfqpoint{2.662075in}{4.711604in}}%
\pgfpathlineto{\pgfqpoint{2.574339in}{4.711604in}}%
\pgfpathlineto{\pgfqpoint{2.574339in}{4.799340in}}%
\pgfusepath{stroke,fill}%
\end{pgfscope}%
\begin{pgfscope}%
\pgfpathrectangle{\pgfqpoint{0.380943in}{4.185189in}}{\pgfqpoint{4.650000in}{0.614151in}}%
\pgfusepath{clip}%
\pgfsetbuttcap%
\pgfsetroundjoin%
\definecolor{currentfill}{rgb}{0.965444,0.906113,0.711757}%
\pgfsetfillcolor{currentfill}%
\pgfsetlinewidth{0.250937pt}%
\definecolor{currentstroke}{rgb}{1.000000,1.000000,1.000000}%
\pgfsetstrokecolor{currentstroke}%
\pgfsetdash{}{0pt}%
\pgfpathmoveto{\pgfqpoint{2.662075in}{4.799340in}}%
\pgfpathlineto{\pgfqpoint{2.749811in}{4.799340in}}%
\pgfpathlineto{\pgfqpoint{2.749811in}{4.711604in}}%
\pgfpathlineto{\pgfqpoint{2.662075in}{4.711604in}}%
\pgfpathlineto{\pgfqpoint{2.662075in}{4.799340in}}%
\pgfusepath{stroke,fill}%
\end{pgfscope}%
\begin{pgfscope}%
\pgfpathrectangle{\pgfqpoint{0.380943in}{4.185189in}}{\pgfqpoint{4.650000in}{0.614151in}}%
\pgfusepath{clip}%
\pgfsetbuttcap%
\pgfsetroundjoin%
\definecolor{currentfill}{rgb}{0.996571,0.720538,0.589189}%
\pgfsetfillcolor{currentfill}%
\pgfsetlinewidth{0.250937pt}%
\definecolor{currentstroke}{rgb}{1.000000,1.000000,1.000000}%
\pgfsetstrokecolor{currentstroke}%
\pgfsetdash{}{0pt}%
\pgfpathmoveto{\pgfqpoint{2.749811in}{4.799340in}}%
\pgfpathlineto{\pgfqpoint{2.837547in}{4.799340in}}%
\pgfpathlineto{\pgfqpoint{2.837547in}{4.711604in}}%
\pgfpathlineto{\pgfqpoint{2.749811in}{4.711604in}}%
\pgfpathlineto{\pgfqpoint{2.749811in}{4.799340in}}%
\pgfusepath{stroke,fill}%
\end{pgfscope}%
\begin{pgfscope}%
\pgfpathrectangle{\pgfqpoint{0.380943in}{4.185189in}}{\pgfqpoint{4.650000in}{0.614151in}}%
\pgfusepath{clip}%
\pgfsetbuttcap%
\pgfsetroundjoin%
\definecolor{currentfill}{rgb}{0.996571,0.720538,0.589189}%
\pgfsetfillcolor{currentfill}%
\pgfsetlinewidth{0.250937pt}%
\definecolor{currentstroke}{rgb}{1.000000,1.000000,1.000000}%
\pgfsetstrokecolor{currentstroke}%
\pgfsetdash{}{0pt}%
\pgfpathmoveto{\pgfqpoint{2.837547in}{4.799340in}}%
\pgfpathlineto{\pgfqpoint{2.925283in}{4.799340in}}%
\pgfpathlineto{\pgfqpoint{2.925283in}{4.711604in}}%
\pgfpathlineto{\pgfqpoint{2.837547in}{4.711604in}}%
\pgfpathlineto{\pgfqpoint{2.837547in}{4.799340in}}%
\pgfusepath{stroke,fill}%
\end{pgfscope}%
\begin{pgfscope}%
\pgfpathrectangle{\pgfqpoint{0.380943in}{4.185189in}}{\pgfqpoint{4.650000in}{0.614151in}}%
\pgfusepath{clip}%
\pgfsetbuttcap%
\pgfsetroundjoin%
\definecolor{currentfill}{rgb}{0.922338,0.400769,0.400769}%
\pgfsetfillcolor{currentfill}%
\pgfsetlinewidth{0.250937pt}%
\definecolor{currentstroke}{rgb}{1.000000,1.000000,1.000000}%
\pgfsetstrokecolor{currentstroke}%
\pgfsetdash{}{0pt}%
\pgfpathmoveto{\pgfqpoint{2.925283in}{4.799340in}}%
\pgfpathlineto{\pgfqpoint{3.013019in}{4.799340in}}%
\pgfpathlineto{\pgfqpoint{3.013019in}{4.711604in}}%
\pgfpathlineto{\pgfqpoint{2.925283in}{4.711604in}}%
\pgfpathlineto{\pgfqpoint{2.925283in}{4.799340in}}%
\pgfusepath{stroke,fill}%
\end{pgfscope}%
\begin{pgfscope}%
\pgfpathrectangle{\pgfqpoint{0.380943in}{4.185189in}}{\pgfqpoint{4.650000in}{0.614151in}}%
\pgfusepath{clip}%
\pgfsetbuttcap%
\pgfsetroundjoin%
\definecolor{currentfill}{rgb}{0.979654,0.837186,0.669619}%
\pgfsetfillcolor{currentfill}%
\pgfsetlinewidth{0.250937pt}%
\definecolor{currentstroke}{rgb}{1.000000,1.000000,1.000000}%
\pgfsetstrokecolor{currentstroke}%
\pgfsetdash{}{0pt}%
\pgfpathmoveto{\pgfqpoint{3.013019in}{4.799340in}}%
\pgfpathlineto{\pgfqpoint{3.100754in}{4.799340in}}%
\pgfpathlineto{\pgfqpoint{3.100754in}{4.711604in}}%
\pgfpathlineto{\pgfqpoint{3.013019in}{4.711604in}}%
\pgfpathlineto{\pgfqpoint{3.013019in}{4.799340in}}%
\pgfusepath{stroke,fill}%
\end{pgfscope}%
\begin{pgfscope}%
\pgfpathrectangle{\pgfqpoint{0.380943in}{4.185189in}}{\pgfqpoint{4.650000in}{0.614151in}}%
\pgfusepath{clip}%
\pgfsetbuttcap%
\pgfsetroundjoin%
\definecolor{currentfill}{rgb}{0.986759,0.806398,0.641200}%
\pgfsetfillcolor{currentfill}%
\pgfsetlinewidth{0.250937pt}%
\definecolor{currentstroke}{rgb}{1.000000,1.000000,1.000000}%
\pgfsetstrokecolor{currentstroke}%
\pgfsetdash{}{0pt}%
\pgfpathmoveto{\pgfqpoint{3.100754in}{4.799340in}}%
\pgfpathlineto{\pgfqpoint{3.188490in}{4.799340in}}%
\pgfpathlineto{\pgfqpoint{3.188490in}{4.711604in}}%
\pgfpathlineto{\pgfqpoint{3.100754in}{4.711604in}}%
\pgfpathlineto{\pgfqpoint{3.100754in}{4.799340in}}%
\pgfusepath{stroke,fill}%
\end{pgfscope}%
\begin{pgfscope}%
\pgfpathrectangle{\pgfqpoint{0.380943in}{4.185189in}}{\pgfqpoint{4.650000in}{0.614151in}}%
\pgfusepath{clip}%
\pgfsetbuttcap%
\pgfsetroundjoin%
\definecolor{currentfill}{rgb}{0.972549,0.870588,0.692810}%
\pgfsetfillcolor{currentfill}%
\pgfsetlinewidth{0.250937pt}%
\definecolor{currentstroke}{rgb}{1.000000,1.000000,1.000000}%
\pgfsetstrokecolor{currentstroke}%
\pgfsetdash{}{0pt}%
\pgfpathmoveto{\pgfqpoint{3.188490in}{4.799340in}}%
\pgfpathlineto{\pgfqpoint{3.276226in}{4.799340in}}%
\pgfpathlineto{\pgfqpoint{3.276226in}{4.711604in}}%
\pgfpathlineto{\pgfqpoint{3.188490in}{4.711604in}}%
\pgfpathlineto{\pgfqpoint{3.188490in}{4.799340in}}%
\pgfusepath{stroke,fill}%
\end{pgfscope}%
\begin{pgfscope}%
\pgfpathrectangle{\pgfqpoint{0.380943in}{4.185189in}}{\pgfqpoint{4.650000in}{0.614151in}}%
\pgfusepath{clip}%
\pgfsetbuttcap%
\pgfsetroundjoin%
\definecolor{currentfill}{rgb}{1.000000,0.605229,0.530719}%
\pgfsetfillcolor{currentfill}%
\pgfsetlinewidth{0.250937pt}%
\definecolor{currentstroke}{rgb}{1.000000,1.000000,1.000000}%
\pgfsetstrokecolor{currentstroke}%
\pgfsetdash{}{0pt}%
\pgfpathmoveto{\pgfqpoint{3.276226in}{4.799340in}}%
\pgfpathlineto{\pgfqpoint{3.363962in}{4.799340in}}%
\pgfpathlineto{\pgfqpoint{3.363962in}{4.711604in}}%
\pgfpathlineto{\pgfqpoint{3.276226in}{4.711604in}}%
\pgfpathlineto{\pgfqpoint{3.276226in}{4.799340in}}%
\pgfusepath{stroke,fill}%
\end{pgfscope}%
\begin{pgfscope}%
\pgfpathrectangle{\pgfqpoint{0.380943in}{4.185189in}}{\pgfqpoint{4.650000in}{0.614151in}}%
\pgfusepath{clip}%
\pgfsetbuttcap%
\pgfsetroundjoin%
\definecolor{currentfill}{rgb}{0.986759,0.806398,0.641200}%
\pgfsetfillcolor{currentfill}%
\pgfsetlinewidth{0.250937pt}%
\definecolor{currentstroke}{rgb}{1.000000,1.000000,1.000000}%
\pgfsetstrokecolor{currentstroke}%
\pgfsetdash{}{0pt}%
\pgfpathmoveto{\pgfqpoint{3.363962in}{4.799340in}}%
\pgfpathlineto{\pgfqpoint{3.451698in}{4.799340in}}%
\pgfpathlineto{\pgfqpoint{3.451698in}{4.711604in}}%
\pgfpathlineto{\pgfqpoint{3.363962in}{4.711604in}}%
\pgfpathlineto{\pgfqpoint{3.363962in}{4.799340in}}%
\pgfusepath{stroke,fill}%
\end{pgfscope}%
\begin{pgfscope}%
\pgfpathrectangle{\pgfqpoint{0.380943in}{4.185189in}}{\pgfqpoint{4.650000in}{0.614151in}}%
\pgfusepath{clip}%
\pgfsetbuttcap%
\pgfsetroundjoin%
\definecolor{currentfill}{rgb}{0.965444,0.906113,0.711757}%
\pgfsetfillcolor{currentfill}%
\pgfsetlinewidth{0.250937pt}%
\definecolor{currentstroke}{rgb}{1.000000,1.000000,1.000000}%
\pgfsetstrokecolor{currentstroke}%
\pgfsetdash{}{0pt}%
\pgfpathmoveto{\pgfqpoint{3.451698in}{4.799340in}}%
\pgfpathlineto{\pgfqpoint{3.539434in}{4.799340in}}%
\pgfpathlineto{\pgfqpoint{3.539434in}{4.711604in}}%
\pgfpathlineto{\pgfqpoint{3.451698in}{4.711604in}}%
\pgfpathlineto{\pgfqpoint{3.451698in}{4.799340in}}%
\pgfusepath{stroke,fill}%
\end{pgfscope}%
\begin{pgfscope}%
\pgfpathrectangle{\pgfqpoint{0.380943in}{4.185189in}}{\pgfqpoint{4.650000in}{0.614151in}}%
\pgfusepath{clip}%
\pgfsetbuttcap%
\pgfsetroundjoin%
\definecolor{currentfill}{rgb}{0.968166,0.945882,0.748604}%
\pgfsetfillcolor{currentfill}%
\pgfsetlinewidth{0.250937pt}%
\definecolor{currentstroke}{rgb}{1.000000,1.000000,1.000000}%
\pgfsetstrokecolor{currentstroke}%
\pgfsetdash{}{0pt}%
\pgfpathmoveto{\pgfqpoint{3.539434in}{4.799340in}}%
\pgfpathlineto{\pgfqpoint{3.627169in}{4.799340in}}%
\pgfpathlineto{\pgfqpoint{3.627169in}{4.711604in}}%
\pgfpathlineto{\pgfqpoint{3.539434in}{4.711604in}}%
\pgfpathlineto{\pgfqpoint{3.539434in}{4.799340in}}%
\pgfusepath{stroke,fill}%
\end{pgfscope}%
\begin{pgfscope}%
\pgfpathrectangle{\pgfqpoint{0.380943in}{4.185189in}}{\pgfqpoint{4.650000in}{0.614151in}}%
\pgfusepath{clip}%
\pgfsetbuttcap%
\pgfsetroundjoin%
\definecolor{currentfill}{rgb}{0.965444,0.906113,0.711757}%
\pgfsetfillcolor{currentfill}%
\pgfsetlinewidth{0.250937pt}%
\definecolor{currentstroke}{rgb}{1.000000,1.000000,1.000000}%
\pgfsetstrokecolor{currentstroke}%
\pgfsetdash{}{0pt}%
\pgfpathmoveto{\pgfqpoint{3.627169in}{4.799340in}}%
\pgfpathlineto{\pgfqpoint{3.714905in}{4.799340in}}%
\pgfpathlineto{\pgfqpoint{3.714905in}{4.711604in}}%
\pgfpathlineto{\pgfqpoint{3.627169in}{4.711604in}}%
\pgfpathlineto{\pgfqpoint{3.627169in}{4.799340in}}%
\pgfusepath{stroke,fill}%
\end{pgfscope}%
\begin{pgfscope}%
\pgfpathrectangle{\pgfqpoint{0.380943in}{4.185189in}}{\pgfqpoint{4.650000in}{0.614151in}}%
\pgfusepath{clip}%
\pgfsetbuttcap%
\pgfsetroundjoin%
\definecolor{currentfill}{rgb}{0.972549,0.870588,0.692810}%
\pgfsetfillcolor{currentfill}%
\pgfsetlinewidth{0.250937pt}%
\definecolor{currentstroke}{rgb}{1.000000,1.000000,1.000000}%
\pgfsetstrokecolor{currentstroke}%
\pgfsetdash{}{0pt}%
\pgfpathmoveto{\pgfqpoint{3.714905in}{4.799340in}}%
\pgfpathlineto{\pgfqpoint{3.802641in}{4.799340in}}%
\pgfpathlineto{\pgfqpoint{3.802641in}{4.711604in}}%
\pgfpathlineto{\pgfqpoint{3.714905in}{4.711604in}}%
\pgfpathlineto{\pgfqpoint{3.714905in}{4.799340in}}%
\pgfusepath{stroke,fill}%
\end{pgfscope}%
\begin{pgfscope}%
\pgfpathrectangle{\pgfqpoint{0.380943in}{4.185189in}}{\pgfqpoint{4.650000in}{0.614151in}}%
\pgfusepath{clip}%
\pgfsetbuttcap%
\pgfsetroundjoin%
\definecolor{currentfill}{rgb}{0.979654,0.837186,0.669619}%
\pgfsetfillcolor{currentfill}%
\pgfsetlinewidth{0.250937pt}%
\definecolor{currentstroke}{rgb}{1.000000,1.000000,1.000000}%
\pgfsetstrokecolor{currentstroke}%
\pgfsetdash{}{0pt}%
\pgfpathmoveto{\pgfqpoint{3.802641in}{4.799340in}}%
\pgfpathlineto{\pgfqpoint{3.890377in}{4.799340in}}%
\pgfpathlineto{\pgfqpoint{3.890377in}{4.711604in}}%
\pgfpathlineto{\pgfqpoint{3.802641in}{4.711604in}}%
\pgfpathlineto{\pgfqpoint{3.802641in}{4.799340in}}%
\pgfusepath{stroke,fill}%
\end{pgfscope}%
\begin{pgfscope}%
\pgfpathrectangle{\pgfqpoint{0.380943in}{4.185189in}}{\pgfqpoint{4.650000in}{0.614151in}}%
\pgfusepath{clip}%
\pgfsetbuttcap%
\pgfsetroundjoin%
\definecolor{currentfill}{rgb}{0.972549,0.870588,0.692810}%
\pgfsetfillcolor{currentfill}%
\pgfsetlinewidth{0.250937pt}%
\definecolor{currentstroke}{rgb}{1.000000,1.000000,1.000000}%
\pgfsetstrokecolor{currentstroke}%
\pgfsetdash{}{0pt}%
\pgfpathmoveto{\pgfqpoint{3.890377in}{4.799340in}}%
\pgfpathlineto{\pgfqpoint{3.978113in}{4.799340in}}%
\pgfpathlineto{\pgfqpoint{3.978113in}{4.711604in}}%
\pgfpathlineto{\pgfqpoint{3.890377in}{4.711604in}}%
\pgfpathlineto{\pgfqpoint{3.890377in}{4.799340in}}%
\pgfusepath{stroke,fill}%
\end{pgfscope}%
\begin{pgfscope}%
\pgfpathrectangle{\pgfqpoint{0.380943in}{4.185189in}}{\pgfqpoint{4.650000in}{0.614151in}}%
\pgfusepath{clip}%
\pgfsetbuttcap%
\pgfsetroundjoin%
\definecolor{currentfill}{rgb}{0.861576,0.340008,0.340008}%
\pgfsetfillcolor{currentfill}%
\pgfsetlinewidth{0.250937pt}%
\definecolor{currentstroke}{rgb}{1.000000,1.000000,1.000000}%
\pgfsetstrokecolor{currentstroke}%
\pgfsetdash{}{0pt}%
\pgfpathmoveto{\pgfqpoint{3.978113in}{4.799340in}}%
\pgfpathlineto{\pgfqpoint{4.065849in}{4.799340in}}%
\pgfpathlineto{\pgfqpoint{4.065849in}{4.711604in}}%
\pgfpathlineto{\pgfqpoint{3.978113in}{4.711604in}}%
\pgfpathlineto{\pgfqpoint{3.978113in}{4.799340in}}%
\pgfusepath{stroke,fill}%
\end{pgfscope}%
\begin{pgfscope}%
\pgfpathrectangle{\pgfqpoint{0.380943in}{4.185189in}}{\pgfqpoint{4.650000in}{0.614151in}}%
\pgfusepath{clip}%
\pgfsetbuttcap%
\pgfsetroundjoin%
\definecolor{currentfill}{rgb}{0.986759,0.806398,0.641200}%
\pgfsetfillcolor{currentfill}%
\pgfsetlinewidth{0.250937pt}%
\definecolor{currentstroke}{rgb}{1.000000,1.000000,1.000000}%
\pgfsetstrokecolor{currentstroke}%
\pgfsetdash{}{0pt}%
\pgfpathmoveto{\pgfqpoint{4.065849in}{4.799340in}}%
\pgfpathlineto{\pgfqpoint{4.153585in}{4.799340in}}%
\pgfpathlineto{\pgfqpoint{4.153585in}{4.711604in}}%
\pgfpathlineto{\pgfqpoint{4.065849in}{4.711604in}}%
\pgfpathlineto{\pgfqpoint{4.065849in}{4.799340in}}%
\pgfusepath{stroke,fill}%
\end{pgfscope}%
\begin{pgfscope}%
\pgfpathrectangle{\pgfqpoint{0.380943in}{4.185189in}}{\pgfqpoint{4.650000in}{0.614151in}}%
\pgfusepath{clip}%
\pgfsetbuttcap%
\pgfsetroundjoin%
\definecolor{currentfill}{rgb}{0.981546,0.459977,0.459977}%
\pgfsetfillcolor{currentfill}%
\pgfsetlinewidth{0.250937pt}%
\definecolor{currentstroke}{rgb}{1.000000,1.000000,1.000000}%
\pgfsetstrokecolor{currentstroke}%
\pgfsetdash{}{0pt}%
\pgfpathmoveto{\pgfqpoint{4.153585in}{4.799340in}}%
\pgfpathlineto{\pgfqpoint{4.241320in}{4.799340in}}%
\pgfpathlineto{\pgfqpoint{4.241320in}{4.711604in}}%
\pgfpathlineto{\pgfqpoint{4.153585in}{4.711604in}}%
\pgfpathlineto{\pgfqpoint{4.153585in}{4.799340in}}%
\pgfusepath{stroke,fill}%
\end{pgfscope}%
\begin{pgfscope}%
\pgfpathrectangle{\pgfqpoint{0.380943in}{4.185189in}}{\pgfqpoint{4.650000in}{0.614151in}}%
\pgfusepath{clip}%
\pgfsetbuttcap%
\pgfsetroundjoin%
\definecolor{currentfill}{rgb}{0.996571,0.720538,0.589189}%
\pgfsetfillcolor{currentfill}%
\pgfsetlinewidth{0.250937pt}%
\definecolor{currentstroke}{rgb}{1.000000,1.000000,1.000000}%
\pgfsetstrokecolor{currentstroke}%
\pgfsetdash{}{0pt}%
\pgfpathmoveto{\pgfqpoint{4.241320in}{4.799340in}}%
\pgfpathlineto{\pgfqpoint{4.329056in}{4.799340in}}%
\pgfpathlineto{\pgfqpoint{4.329056in}{4.711604in}}%
\pgfpathlineto{\pgfqpoint{4.241320in}{4.711604in}}%
\pgfpathlineto{\pgfqpoint{4.241320in}{4.799340in}}%
\pgfusepath{stroke,fill}%
\end{pgfscope}%
\begin{pgfscope}%
\pgfpathrectangle{\pgfqpoint{0.380943in}{4.185189in}}{\pgfqpoint{4.650000in}{0.614151in}}%
\pgfusepath{clip}%
\pgfsetbuttcap%
\pgfsetroundjoin%
\definecolor{currentfill}{rgb}{0.962414,0.923552,0.722891}%
\pgfsetfillcolor{currentfill}%
\pgfsetlinewidth{0.250937pt}%
\definecolor{currentstroke}{rgb}{1.000000,1.000000,1.000000}%
\pgfsetstrokecolor{currentstroke}%
\pgfsetdash{}{0pt}%
\pgfpathmoveto{\pgfqpoint{4.329056in}{4.799340in}}%
\pgfpathlineto{\pgfqpoint{4.416792in}{4.799340in}}%
\pgfpathlineto{\pgfqpoint{4.416792in}{4.711604in}}%
\pgfpathlineto{\pgfqpoint{4.329056in}{4.711604in}}%
\pgfpathlineto{\pgfqpoint{4.329056in}{4.799340in}}%
\pgfusepath{stroke,fill}%
\end{pgfscope}%
\begin{pgfscope}%
\pgfpathrectangle{\pgfqpoint{0.380943in}{4.185189in}}{\pgfqpoint{4.650000in}{0.614151in}}%
\pgfusepath{clip}%
\pgfsetbuttcap%
\pgfsetroundjoin%
\definecolor{currentfill}{rgb}{0.996571,0.720538,0.589189}%
\pgfsetfillcolor{currentfill}%
\pgfsetlinewidth{0.250937pt}%
\definecolor{currentstroke}{rgb}{1.000000,1.000000,1.000000}%
\pgfsetstrokecolor{currentstroke}%
\pgfsetdash{}{0pt}%
\pgfpathmoveto{\pgfqpoint{4.416792in}{4.799340in}}%
\pgfpathlineto{\pgfqpoint{4.504528in}{4.799340in}}%
\pgfpathlineto{\pgfqpoint{4.504528in}{4.711604in}}%
\pgfpathlineto{\pgfqpoint{4.416792in}{4.711604in}}%
\pgfpathlineto{\pgfqpoint{4.416792in}{4.799340in}}%
\pgfusepath{stroke,fill}%
\end{pgfscope}%
\begin{pgfscope}%
\pgfpathrectangle{\pgfqpoint{0.380943in}{4.185189in}}{\pgfqpoint{4.650000in}{0.614151in}}%
\pgfusepath{clip}%
\pgfsetbuttcap%
\pgfsetroundjoin%
\definecolor{currentfill}{rgb}{0.972549,0.870588,0.692810}%
\pgfsetfillcolor{currentfill}%
\pgfsetlinewidth{0.250937pt}%
\definecolor{currentstroke}{rgb}{1.000000,1.000000,1.000000}%
\pgfsetstrokecolor{currentstroke}%
\pgfsetdash{}{0pt}%
\pgfpathmoveto{\pgfqpoint{4.504528in}{4.799340in}}%
\pgfpathlineto{\pgfqpoint{4.592264in}{4.799340in}}%
\pgfpathlineto{\pgfqpoint{4.592264in}{4.711604in}}%
\pgfpathlineto{\pgfqpoint{4.504528in}{4.711604in}}%
\pgfpathlineto{\pgfqpoint{4.504528in}{4.799340in}}%
\pgfusepath{stroke,fill}%
\end{pgfscope}%
\begin{pgfscope}%
\pgfpathrectangle{\pgfqpoint{0.380943in}{4.185189in}}{\pgfqpoint{4.650000in}{0.614151in}}%
\pgfusepath{clip}%
\pgfsetbuttcap%
\pgfsetroundjoin%
\definecolor{currentfill}{rgb}{0.996571,0.720538,0.589189}%
\pgfsetfillcolor{currentfill}%
\pgfsetlinewidth{0.250937pt}%
\definecolor{currentstroke}{rgb}{1.000000,1.000000,1.000000}%
\pgfsetstrokecolor{currentstroke}%
\pgfsetdash{}{0pt}%
\pgfpathmoveto{\pgfqpoint{4.592264in}{4.799340in}}%
\pgfpathlineto{\pgfqpoint{4.680000in}{4.799340in}}%
\pgfpathlineto{\pgfqpoint{4.680000in}{4.711604in}}%
\pgfpathlineto{\pgfqpoint{4.592264in}{4.711604in}}%
\pgfpathlineto{\pgfqpoint{4.592264in}{4.799340in}}%
\pgfusepath{stroke,fill}%
\end{pgfscope}%
\begin{pgfscope}%
\pgfpathrectangle{\pgfqpoint{0.380943in}{4.185189in}}{\pgfqpoint{4.650000in}{0.614151in}}%
\pgfusepath{clip}%
\pgfsetbuttcap%
\pgfsetroundjoin%
\definecolor{currentfill}{rgb}{0.968166,0.945882,0.748604}%
\pgfsetfillcolor{currentfill}%
\pgfsetlinewidth{0.250937pt}%
\definecolor{currentstroke}{rgb}{1.000000,1.000000,1.000000}%
\pgfsetstrokecolor{currentstroke}%
\pgfsetdash{}{0pt}%
\pgfpathmoveto{\pgfqpoint{4.680000in}{4.799340in}}%
\pgfpathlineto{\pgfqpoint{4.767736in}{4.799340in}}%
\pgfpathlineto{\pgfqpoint{4.767736in}{4.711604in}}%
\pgfpathlineto{\pgfqpoint{4.680000in}{4.711604in}}%
\pgfpathlineto{\pgfqpoint{4.680000in}{4.799340in}}%
\pgfusepath{stroke,fill}%
\end{pgfscope}%
\begin{pgfscope}%
\pgfpathrectangle{\pgfqpoint{0.380943in}{4.185189in}}{\pgfqpoint{4.650000in}{0.614151in}}%
\pgfusepath{clip}%
\pgfsetbuttcap%
\pgfsetroundjoin%
\definecolor{currentfill}{rgb}{1.000000,0.605229,0.530719}%
\pgfsetfillcolor{currentfill}%
\pgfsetlinewidth{0.250937pt}%
\definecolor{currentstroke}{rgb}{1.000000,1.000000,1.000000}%
\pgfsetstrokecolor{currentstroke}%
\pgfsetdash{}{0pt}%
\pgfpathmoveto{\pgfqpoint{4.767736in}{4.799340in}}%
\pgfpathlineto{\pgfqpoint{4.855471in}{4.799340in}}%
\pgfpathlineto{\pgfqpoint{4.855471in}{4.711604in}}%
\pgfpathlineto{\pgfqpoint{4.767736in}{4.711604in}}%
\pgfpathlineto{\pgfqpoint{4.767736in}{4.799340in}}%
\pgfusepath{stroke,fill}%
\end{pgfscope}%
\begin{pgfscope}%
\pgfpathrectangle{\pgfqpoint{0.380943in}{4.185189in}}{\pgfqpoint{4.650000in}{0.614151in}}%
\pgfusepath{clip}%
\pgfsetbuttcap%
\pgfsetroundjoin%
\definecolor{currentfill}{rgb}{0.996571,0.720538,0.589189}%
\pgfsetfillcolor{currentfill}%
\pgfsetlinewidth{0.250937pt}%
\definecolor{currentstroke}{rgb}{1.000000,1.000000,1.000000}%
\pgfsetstrokecolor{currentstroke}%
\pgfsetdash{}{0pt}%
\pgfpathmoveto{\pgfqpoint{4.855471in}{4.799340in}}%
\pgfpathlineto{\pgfqpoint{4.943207in}{4.799340in}}%
\pgfpathlineto{\pgfqpoint{4.943207in}{4.711604in}}%
\pgfpathlineto{\pgfqpoint{4.855471in}{4.711604in}}%
\pgfpathlineto{\pgfqpoint{4.855471in}{4.799340in}}%
\pgfusepath{stroke,fill}%
\end{pgfscope}%
\begin{pgfscope}%
\pgfpathrectangle{\pgfqpoint{0.380943in}{4.185189in}}{\pgfqpoint{4.650000in}{0.614151in}}%
\pgfusepath{clip}%
\pgfsetbuttcap%
\pgfsetroundjoin%
\definecolor{currentfill}{rgb}{0.986759,0.806398,0.641200}%
\pgfsetfillcolor{currentfill}%
\pgfsetlinewidth{0.250937pt}%
\definecolor{currentstroke}{rgb}{1.000000,1.000000,1.000000}%
\pgfsetstrokecolor{currentstroke}%
\pgfsetdash{}{0pt}%
\pgfpathmoveto{\pgfqpoint{4.943207in}{4.799340in}}%
\pgfpathlineto{\pgfqpoint{5.030943in}{4.799340in}}%
\pgfpathlineto{\pgfqpoint{5.030943in}{4.711604in}}%
\pgfpathlineto{\pgfqpoint{4.943207in}{4.711604in}}%
\pgfpathlineto{\pgfqpoint{4.943207in}{4.799340in}}%
\pgfusepath{stroke,fill}%
\end{pgfscope}%
\begin{pgfscope}%
\pgfpathrectangle{\pgfqpoint{0.380943in}{4.185189in}}{\pgfqpoint{4.650000in}{0.614151in}}%
\pgfusepath{clip}%
\pgfsetbuttcap%
\pgfsetroundjoin%
\definecolor{currentfill}{rgb}{0.972549,0.870588,0.692810}%
\pgfsetfillcolor{currentfill}%
\pgfsetlinewidth{0.250937pt}%
\definecolor{currentstroke}{rgb}{1.000000,1.000000,1.000000}%
\pgfsetstrokecolor{currentstroke}%
\pgfsetdash{}{0pt}%
\pgfpathmoveto{\pgfqpoint{0.380943in}{4.711604in}}%
\pgfpathlineto{\pgfqpoint{0.468679in}{4.711604in}}%
\pgfpathlineto{\pgfqpoint{0.468679in}{4.623868in}}%
\pgfpathlineto{\pgfqpoint{0.380943in}{4.623868in}}%
\pgfpathlineto{\pgfqpoint{0.380943in}{4.711604in}}%
\pgfusepath{stroke,fill}%
\end{pgfscope}%
\begin{pgfscope}%
\pgfpathrectangle{\pgfqpoint{0.380943in}{4.185189in}}{\pgfqpoint{4.650000in}{0.614151in}}%
\pgfusepath{clip}%
\pgfsetbuttcap%
\pgfsetroundjoin%
\definecolor{currentfill}{rgb}{0.979654,0.837186,0.669619}%
\pgfsetfillcolor{currentfill}%
\pgfsetlinewidth{0.250937pt}%
\definecolor{currentstroke}{rgb}{1.000000,1.000000,1.000000}%
\pgfsetstrokecolor{currentstroke}%
\pgfsetdash{}{0pt}%
\pgfpathmoveto{\pgfqpoint{0.468679in}{4.711604in}}%
\pgfpathlineto{\pgfqpoint{0.556415in}{4.711604in}}%
\pgfpathlineto{\pgfqpoint{0.556415in}{4.623868in}}%
\pgfpathlineto{\pgfqpoint{0.468679in}{4.623868in}}%
\pgfpathlineto{\pgfqpoint{0.468679in}{4.711604in}}%
\pgfusepath{stroke,fill}%
\end{pgfscope}%
\begin{pgfscope}%
\pgfpathrectangle{\pgfqpoint{0.380943in}{4.185189in}}{\pgfqpoint{4.650000in}{0.614151in}}%
\pgfusepath{clip}%
\pgfsetbuttcap%
\pgfsetroundjoin%
\definecolor{currentfill}{rgb}{0.998939,0.658962,0.556032}%
\pgfsetfillcolor{currentfill}%
\pgfsetlinewidth{0.250937pt}%
\definecolor{currentstroke}{rgb}{1.000000,1.000000,1.000000}%
\pgfsetstrokecolor{currentstroke}%
\pgfsetdash{}{0pt}%
\pgfpathmoveto{\pgfqpoint{0.556415in}{4.711604in}}%
\pgfpathlineto{\pgfqpoint{0.644151in}{4.711604in}}%
\pgfpathlineto{\pgfqpoint{0.644151in}{4.623868in}}%
\pgfpathlineto{\pgfqpoint{0.556415in}{4.623868in}}%
\pgfpathlineto{\pgfqpoint{0.556415in}{4.711604in}}%
\pgfusepath{stroke,fill}%
\end{pgfscope}%
\begin{pgfscope}%
\pgfpathrectangle{\pgfqpoint{0.380943in}{4.185189in}}{\pgfqpoint{4.650000in}{0.614151in}}%
\pgfusepath{clip}%
\pgfsetbuttcap%
\pgfsetroundjoin%
\definecolor{currentfill}{rgb}{0.979654,0.837186,0.669619}%
\pgfsetfillcolor{currentfill}%
\pgfsetlinewidth{0.250937pt}%
\definecolor{currentstroke}{rgb}{1.000000,1.000000,1.000000}%
\pgfsetstrokecolor{currentstroke}%
\pgfsetdash{}{0pt}%
\pgfpathmoveto{\pgfqpoint{0.644151in}{4.711604in}}%
\pgfpathlineto{\pgfqpoint{0.731886in}{4.711604in}}%
\pgfpathlineto{\pgfqpoint{0.731886in}{4.623868in}}%
\pgfpathlineto{\pgfqpoint{0.644151in}{4.623868in}}%
\pgfpathlineto{\pgfqpoint{0.644151in}{4.711604in}}%
\pgfusepath{stroke,fill}%
\end{pgfscope}%
\begin{pgfscope}%
\pgfpathrectangle{\pgfqpoint{0.380943in}{4.185189in}}{\pgfqpoint{4.650000in}{0.614151in}}%
\pgfusepath{clip}%
\pgfsetbuttcap%
\pgfsetroundjoin%
\definecolor{currentfill}{rgb}{0.992326,0.765229,0.614840}%
\pgfsetfillcolor{currentfill}%
\pgfsetlinewidth{0.250937pt}%
\definecolor{currentstroke}{rgb}{1.000000,1.000000,1.000000}%
\pgfsetstrokecolor{currentstroke}%
\pgfsetdash{}{0pt}%
\pgfpathmoveto{\pgfqpoint{0.731886in}{4.711604in}}%
\pgfpathlineto{\pgfqpoint{0.819622in}{4.711604in}}%
\pgfpathlineto{\pgfqpoint{0.819622in}{4.623868in}}%
\pgfpathlineto{\pgfqpoint{0.731886in}{4.623868in}}%
\pgfpathlineto{\pgfqpoint{0.731886in}{4.711604in}}%
\pgfusepath{stroke,fill}%
\end{pgfscope}%
\begin{pgfscope}%
\pgfpathrectangle{\pgfqpoint{0.380943in}{4.185189in}}{\pgfqpoint{4.650000in}{0.614151in}}%
\pgfusepath{clip}%
\pgfsetbuttcap%
\pgfsetroundjoin%
\definecolor{currentfill}{rgb}{0.996571,0.720538,0.589189}%
\pgfsetfillcolor{currentfill}%
\pgfsetlinewidth{0.250937pt}%
\definecolor{currentstroke}{rgb}{1.000000,1.000000,1.000000}%
\pgfsetstrokecolor{currentstroke}%
\pgfsetdash{}{0pt}%
\pgfpathmoveto{\pgfqpoint{0.819622in}{4.711604in}}%
\pgfpathlineto{\pgfqpoint{0.907358in}{4.711604in}}%
\pgfpathlineto{\pgfqpoint{0.907358in}{4.623868in}}%
\pgfpathlineto{\pgfqpoint{0.819622in}{4.623868in}}%
\pgfpathlineto{\pgfqpoint{0.819622in}{4.711604in}}%
\pgfusepath{stroke,fill}%
\end{pgfscope}%
\begin{pgfscope}%
\pgfpathrectangle{\pgfqpoint{0.380943in}{4.185189in}}{\pgfqpoint{4.650000in}{0.614151in}}%
\pgfusepath{clip}%
\pgfsetbuttcap%
\pgfsetroundjoin%
\definecolor{currentfill}{rgb}{0.996571,0.720538,0.589189}%
\pgfsetfillcolor{currentfill}%
\pgfsetlinewidth{0.250937pt}%
\definecolor{currentstroke}{rgb}{1.000000,1.000000,1.000000}%
\pgfsetstrokecolor{currentstroke}%
\pgfsetdash{}{0pt}%
\pgfpathmoveto{\pgfqpoint{0.907358in}{4.711604in}}%
\pgfpathlineto{\pgfqpoint{0.995094in}{4.711604in}}%
\pgfpathlineto{\pgfqpoint{0.995094in}{4.623868in}}%
\pgfpathlineto{\pgfqpoint{0.907358in}{4.623868in}}%
\pgfpathlineto{\pgfqpoint{0.907358in}{4.711604in}}%
\pgfusepath{stroke,fill}%
\end{pgfscope}%
\begin{pgfscope}%
\pgfpathrectangle{\pgfqpoint{0.380943in}{4.185189in}}{\pgfqpoint{4.650000in}{0.614151in}}%
\pgfusepath{clip}%
\pgfsetbuttcap%
\pgfsetroundjoin%
\definecolor{currentfill}{rgb}{0.996571,0.720538,0.589189}%
\pgfsetfillcolor{currentfill}%
\pgfsetlinewidth{0.250937pt}%
\definecolor{currentstroke}{rgb}{1.000000,1.000000,1.000000}%
\pgfsetstrokecolor{currentstroke}%
\pgfsetdash{}{0pt}%
\pgfpathmoveto{\pgfqpoint{0.995094in}{4.711604in}}%
\pgfpathlineto{\pgfqpoint{1.082830in}{4.711604in}}%
\pgfpathlineto{\pgfqpoint{1.082830in}{4.623868in}}%
\pgfpathlineto{\pgfqpoint{0.995094in}{4.623868in}}%
\pgfpathlineto{\pgfqpoint{0.995094in}{4.711604in}}%
\pgfusepath{stroke,fill}%
\end{pgfscope}%
\begin{pgfscope}%
\pgfpathrectangle{\pgfqpoint{0.380943in}{4.185189in}}{\pgfqpoint{4.650000in}{0.614151in}}%
\pgfusepath{clip}%
\pgfsetbuttcap%
\pgfsetroundjoin%
\definecolor{currentfill}{rgb}{0.986759,0.806398,0.641200}%
\pgfsetfillcolor{currentfill}%
\pgfsetlinewidth{0.250937pt}%
\definecolor{currentstroke}{rgb}{1.000000,1.000000,1.000000}%
\pgfsetstrokecolor{currentstroke}%
\pgfsetdash{}{0pt}%
\pgfpathmoveto{\pgfqpoint{1.082830in}{4.711604in}}%
\pgfpathlineto{\pgfqpoint{1.170566in}{4.711604in}}%
\pgfpathlineto{\pgfqpoint{1.170566in}{4.623868in}}%
\pgfpathlineto{\pgfqpoint{1.082830in}{4.623868in}}%
\pgfpathlineto{\pgfqpoint{1.082830in}{4.711604in}}%
\pgfusepath{stroke,fill}%
\end{pgfscope}%
\begin{pgfscope}%
\pgfpathrectangle{\pgfqpoint{0.380943in}{4.185189in}}{\pgfqpoint{4.650000in}{0.614151in}}%
\pgfusepath{clip}%
\pgfsetbuttcap%
\pgfsetroundjoin%
\definecolor{currentfill}{rgb}{0.986759,0.806398,0.641200}%
\pgfsetfillcolor{currentfill}%
\pgfsetlinewidth{0.250937pt}%
\definecolor{currentstroke}{rgb}{1.000000,1.000000,1.000000}%
\pgfsetstrokecolor{currentstroke}%
\pgfsetdash{}{0pt}%
\pgfpathmoveto{\pgfqpoint{1.170566in}{4.711604in}}%
\pgfpathlineto{\pgfqpoint{1.258302in}{4.711604in}}%
\pgfpathlineto{\pgfqpoint{1.258302in}{4.623868in}}%
\pgfpathlineto{\pgfqpoint{1.170566in}{4.623868in}}%
\pgfpathlineto{\pgfqpoint{1.170566in}{4.711604in}}%
\pgfusepath{stroke,fill}%
\end{pgfscope}%
\begin{pgfscope}%
\pgfpathrectangle{\pgfqpoint{0.380943in}{4.185189in}}{\pgfqpoint{4.650000in}{0.614151in}}%
\pgfusepath{clip}%
\pgfsetbuttcap%
\pgfsetroundjoin%
\definecolor{currentfill}{rgb}{0.992326,0.765229,0.614840}%
\pgfsetfillcolor{currentfill}%
\pgfsetlinewidth{0.250937pt}%
\definecolor{currentstroke}{rgb}{1.000000,1.000000,1.000000}%
\pgfsetstrokecolor{currentstroke}%
\pgfsetdash{}{0pt}%
\pgfpathmoveto{\pgfqpoint{1.258302in}{4.711604in}}%
\pgfpathlineto{\pgfqpoint{1.346037in}{4.711604in}}%
\pgfpathlineto{\pgfqpoint{1.346037in}{4.623868in}}%
\pgfpathlineto{\pgfqpoint{1.258302in}{4.623868in}}%
\pgfpathlineto{\pgfqpoint{1.258302in}{4.711604in}}%
\pgfusepath{stroke,fill}%
\end{pgfscope}%
\begin{pgfscope}%
\pgfpathrectangle{\pgfqpoint{0.380943in}{4.185189in}}{\pgfqpoint{4.650000in}{0.614151in}}%
\pgfusepath{clip}%
\pgfsetbuttcap%
\pgfsetroundjoin%
\definecolor{currentfill}{rgb}{0.992326,0.765229,0.614840}%
\pgfsetfillcolor{currentfill}%
\pgfsetlinewidth{0.250937pt}%
\definecolor{currentstroke}{rgb}{1.000000,1.000000,1.000000}%
\pgfsetstrokecolor{currentstroke}%
\pgfsetdash{}{0pt}%
\pgfpathmoveto{\pgfqpoint{1.346037in}{4.711604in}}%
\pgfpathlineto{\pgfqpoint{1.433773in}{4.711604in}}%
\pgfpathlineto{\pgfqpoint{1.433773in}{4.623868in}}%
\pgfpathlineto{\pgfqpoint{1.346037in}{4.623868in}}%
\pgfpathlineto{\pgfqpoint{1.346037in}{4.711604in}}%
\pgfusepath{stroke,fill}%
\end{pgfscope}%
\begin{pgfscope}%
\pgfpathrectangle{\pgfqpoint{0.380943in}{4.185189in}}{\pgfqpoint{4.650000in}{0.614151in}}%
\pgfusepath{clip}%
\pgfsetbuttcap%
\pgfsetroundjoin%
\definecolor{currentfill}{rgb}{0.992326,0.765229,0.614840}%
\pgfsetfillcolor{currentfill}%
\pgfsetlinewidth{0.250937pt}%
\definecolor{currentstroke}{rgb}{1.000000,1.000000,1.000000}%
\pgfsetstrokecolor{currentstroke}%
\pgfsetdash{}{0pt}%
\pgfpathmoveto{\pgfqpoint{1.433773in}{4.711604in}}%
\pgfpathlineto{\pgfqpoint{1.521509in}{4.711604in}}%
\pgfpathlineto{\pgfqpoint{1.521509in}{4.623868in}}%
\pgfpathlineto{\pgfqpoint{1.433773in}{4.623868in}}%
\pgfpathlineto{\pgfqpoint{1.433773in}{4.711604in}}%
\pgfusepath{stroke,fill}%
\end{pgfscope}%
\begin{pgfscope}%
\pgfpathrectangle{\pgfqpoint{0.380943in}{4.185189in}}{\pgfqpoint{4.650000in}{0.614151in}}%
\pgfusepath{clip}%
\pgfsetbuttcap%
\pgfsetroundjoin%
\definecolor{currentfill}{rgb}{0.979654,0.837186,0.669619}%
\pgfsetfillcolor{currentfill}%
\pgfsetlinewidth{0.250937pt}%
\definecolor{currentstroke}{rgb}{1.000000,1.000000,1.000000}%
\pgfsetstrokecolor{currentstroke}%
\pgfsetdash{}{0pt}%
\pgfpathmoveto{\pgfqpoint{1.521509in}{4.711604in}}%
\pgfpathlineto{\pgfqpoint{1.609245in}{4.711604in}}%
\pgfpathlineto{\pgfqpoint{1.609245in}{4.623868in}}%
\pgfpathlineto{\pgfqpoint{1.521509in}{4.623868in}}%
\pgfpathlineto{\pgfqpoint{1.521509in}{4.711604in}}%
\pgfusepath{stroke,fill}%
\end{pgfscope}%
\begin{pgfscope}%
\pgfpathrectangle{\pgfqpoint{0.380943in}{4.185189in}}{\pgfqpoint{4.650000in}{0.614151in}}%
\pgfusepath{clip}%
\pgfsetbuttcap%
\pgfsetroundjoin%
\definecolor{currentfill}{rgb}{1.000000,0.557862,0.511772}%
\pgfsetfillcolor{currentfill}%
\pgfsetlinewidth{0.250937pt}%
\definecolor{currentstroke}{rgb}{1.000000,1.000000,1.000000}%
\pgfsetstrokecolor{currentstroke}%
\pgfsetdash{}{0pt}%
\pgfpathmoveto{\pgfqpoint{1.609245in}{4.711604in}}%
\pgfpathlineto{\pgfqpoint{1.696981in}{4.711604in}}%
\pgfpathlineto{\pgfqpoint{1.696981in}{4.623868in}}%
\pgfpathlineto{\pgfqpoint{1.609245in}{4.623868in}}%
\pgfpathlineto{\pgfqpoint{1.609245in}{4.711604in}}%
\pgfusepath{stroke,fill}%
\end{pgfscope}%
\begin{pgfscope}%
\pgfpathrectangle{\pgfqpoint{0.380943in}{4.185189in}}{\pgfqpoint{4.650000in}{0.614151in}}%
\pgfusepath{clip}%
\pgfsetbuttcap%
\pgfsetroundjoin%
\definecolor{currentfill}{rgb}{0.998939,0.658962,0.556032}%
\pgfsetfillcolor{currentfill}%
\pgfsetlinewidth{0.250937pt}%
\definecolor{currentstroke}{rgb}{1.000000,1.000000,1.000000}%
\pgfsetstrokecolor{currentstroke}%
\pgfsetdash{}{0pt}%
\pgfpathmoveto{\pgfqpoint{1.696981in}{4.711604in}}%
\pgfpathlineto{\pgfqpoint{1.784717in}{4.711604in}}%
\pgfpathlineto{\pgfqpoint{1.784717in}{4.623868in}}%
\pgfpathlineto{\pgfqpoint{1.696981in}{4.623868in}}%
\pgfpathlineto{\pgfqpoint{1.696981in}{4.711604in}}%
\pgfusepath{stroke,fill}%
\end{pgfscope}%
\begin{pgfscope}%
\pgfpathrectangle{\pgfqpoint{0.380943in}{4.185189in}}{\pgfqpoint{4.650000in}{0.614151in}}%
\pgfusepath{clip}%
\pgfsetbuttcap%
\pgfsetroundjoin%
\definecolor{currentfill}{rgb}{0.992326,0.765229,0.614840}%
\pgfsetfillcolor{currentfill}%
\pgfsetlinewidth{0.250937pt}%
\definecolor{currentstroke}{rgb}{1.000000,1.000000,1.000000}%
\pgfsetstrokecolor{currentstroke}%
\pgfsetdash{}{0pt}%
\pgfpathmoveto{\pgfqpoint{1.784717in}{4.711604in}}%
\pgfpathlineto{\pgfqpoint{1.872452in}{4.711604in}}%
\pgfpathlineto{\pgfqpoint{1.872452in}{4.623868in}}%
\pgfpathlineto{\pgfqpoint{1.784717in}{4.623868in}}%
\pgfpathlineto{\pgfqpoint{1.784717in}{4.711604in}}%
\pgfusepath{stroke,fill}%
\end{pgfscope}%
\begin{pgfscope}%
\pgfpathrectangle{\pgfqpoint{0.380943in}{4.185189in}}{\pgfqpoint{4.650000in}{0.614151in}}%
\pgfusepath{clip}%
\pgfsetbuttcap%
\pgfsetroundjoin%
\definecolor{currentfill}{rgb}{0.998939,0.658962,0.556032}%
\pgfsetfillcolor{currentfill}%
\pgfsetlinewidth{0.250937pt}%
\definecolor{currentstroke}{rgb}{1.000000,1.000000,1.000000}%
\pgfsetstrokecolor{currentstroke}%
\pgfsetdash{}{0pt}%
\pgfpathmoveto{\pgfqpoint{1.872452in}{4.711604in}}%
\pgfpathlineto{\pgfqpoint{1.960188in}{4.711604in}}%
\pgfpathlineto{\pgfqpoint{1.960188in}{4.623868in}}%
\pgfpathlineto{\pgfqpoint{1.872452in}{4.623868in}}%
\pgfpathlineto{\pgfqpoint{1.872452in}{4.711604in}}%
\pgfusepath{stroke,fill}%
\end{pgfscope}%
\begin{pgfscope}%
\pgfpathrectangle{\pgfqpoint{0.380943in}{4.185189in}}{\pgfqpoint{4.650000in}{0.614151in}}%
\pgfusepath{clip}%
\pgfsetbuttcap%
\pgfsetroundjoin%
\definecolor{currentfill}{rgb}{0.998939,0.658962,0.556032}%
\pgfsetfillcolor{currentfill}%
\pgfsetlinewidth{0.250937pt}%
\definecolor{currentstroke}{rgb}{1.000000,1.000000,1.000000}%
\pgfsetstrokecolor{currentstroke}%
\pgfsetdash{}{0pt}%
\pgfpathmoveto{\pgfqpoint{1.960188in}{4.711604in}}%
\pgfpathlineto{\pgfqpoint{2.047924in}{4.711604in}}%
\pgfpathlineto{\pgfqpoint{2.047924in}{4.623868in}}%
\pgfpathlineto{\pgfqpoint{1.960188in}{4.623868in}}%
\pgfpathlineto{\pgfqpoint{1.960188in}{4.711604in}}%
\pgfusepath{stroke,fill}%
\end{pgfscope}%
\begin{pgfscope}%
\pgfpathrectangle{\pgfqpoint{0.380943in}{4.185189in}}{\pgfqpoint{4.650000in}{0.614151in}}%
\pgfusepath{clip}%
\pgfsetbuttcap%
\pgfsetroundjoin%
\definecolor{currentfill}{rgb}{1.000000,0.509404,0.491473}%
\pgfsetfillcolor{currentfill}%
\pgfsetlinewidth{0.250937pt}%
\definecolor{currentstroke}{rgb}{1.000000,1.000000,1.000000}%
\pgfsetstrokecolor{currentstroke}%
\pgfsetdash{}{0pt}%
\pgfpathmoveto{\pgfqpoint{2.047924in}{4.711604in}}%
\pgfpathlineto{\pgfqpoint{2.135660in}{4.711604in}}%
\pgfpathlineto{\pgfqpoint{2.135660in}{4.623868in}}%
\pgfpathlineto{\pgfqpoint{2.047924in}{4.623868in}}%
\pgfpathlineto{\pgfqpoint{2.047924in}{4.711604in}}%
\pgfusepath{stroke,fill}%
\end{pgfscope}%
\begin{pgfscope}%
\pgfpathrectangle{\pgfqpoint{0.380943in}{4.185189in}}{\pgfqpoint{4.650000in}{0.614151in}}%
\pgfusepath{clip}%
\pgfsetbuttcap%
\pgfsetroundjoin%
\definecolor{currentfill}{rgb}{1.000000,0.509404,0.491473}%
\pgfsetfillcolor{currentfill}%
\pgfsetlinewidth{0.250937pt}%
\definecolor{currentstroke}{rgb}{1.000000,1.000000,1.000000}%
\pgfsetstrokecolor{currentstroke}%
\pgfsetdash{}{0pt}%
\pgfpathmoveto{\pgfqpoint{2.135660in}{4.711604in}}%
\pgfpathlineto{\pgfqpoint{2.223396in}{4.711604in}}%
\pgfpathlineto{\pgfqpoint{2.223396in}{4.623868in}}%
\pgfpathlineto{\pgfqpoint{2.135660in}{4.623868in}}%
\pgfpathlineto{\pgfqpoint{2.135660in}{4.711604in}}%
\pgfusepath{stroke,fill}%
\end{pgfscope}%
\begin{pgfscope}%
\pgfpathrectangle{\pgfqpoint{0.380943in}{4.185189in}}{\pgfqpoint{4.650000in}{0.614151in}}%
\pgfusepath{clip}%
\pgfsetbuttcap%
\pgfsetroundjoin%
\definecolor{currentfill}{rgb}{0.979654,0.837186,0.669619}%
\pgfsetfillcolor{currentfill}%
\pgfsetlinewidth{0.250937pt}%
\definecolor{currentstroke}{rgb}{1.000000,1.000000,1.000000}%
\pgfsetstrokecolor{currentstroke}%
\pgfsetdash{}{0pt}%
\pgfpathmoveto{\pgfqpoint{2.223396in}{4.711604in}}%
\pgfpathlineto{\pgfqpoint{2.311132in}{4.711604in}}%
\pgfpathlineto{\pgfqpoint{2.311132in}{4.623868in}}%
\pgfpathlineto{\pgfqpoint{2.223396in}{4.623868in}}%
\pgfpathlineto{\pgfqpoint{2.223396in}{4.711604in}}%
\pgfusepath{stroke,fill}%
\end{pgfscope}%
\begin{pgfscope}%
\pgfpathrectangle{\pgfqpoint{0.380943in}{4.185189in}}{\pgfqpoint{4.650000in}{0.614151in}}%
\pgfusepath{clip}%
\pgfsetbuttcap%
\pgfsetroundjoin%
\definecolor{currentfill}{rgb}{0.996571,0.720538,0.589189}%
\pgfsetfillcolor{currentfill}%
\pgfsetlinewidth{0.250937pt}%
\definecolor{currentstroke}{rgb}{1.000000,1.000000,1.000000}%
\pgfsetstrokecolor{currentstroke}%
\pgfsetdash{}{0pt}%
\pgfpathmoveto{\pgfqpoint{2.311132in}{4.711604in}}%
\pgfpathlineto{\pgfqpoint{2.398868in}{4.711604in}}%
\pgfpathlineto{\pgfqpoint{2.398868in}{4.623868in}}%
\pgfpathlineto{\pgfqpoint{2.311132in}{4.623868in}}%
\pgfpathlineto{\pgfqpoint{2.311132in}{4.711604in}}%
\pgfusepath{stroke,fill}%
\end{pgfscope}%
\begin{pgfscope}%
\pgfpathrectangle{\pgfqpoint{0.380943in}{4.185189in}}{\pgfqpoint{4.650000in}{0.614151in}}%
\pgfusepath{clip}%
\pgfsetbuttcap%
\pgfsetroundjoin%
\definecolor{currentfill}{rgb}{0.992326,0.765229,0.614840}%
\pgfsetfillcolor{currentfill}%
\pgfsetlinewidth{0.250937pt}%
\definecolor{currentstroke}{rgb}{1.000000,1.000000,1.000000}%
\pgfsetstrokecolor{currentstroke}%
\pgfsetdash{}{0pt}%
\pgfpathmoveto{\pgfqpoint{2.398868in}{4.711604in}}%
\pgfpathlineto{\pgfqpoint{2.486603in}{4.711604in}}%
\pgfpathlineto{\pgfqpoint{2.486603in}{4.623868in}}%
\pgfpathlineto{\pgfqpoint{2.398868in}{4.623868in}}%
\pgfpathlineto{\pgfqpoint{2.398868in}{4.711604in}}%
\pgfusepath{stroke,fill}%
\end{pgfscope}%
\begin{pgfscope}%
\pgfpathrectangle{\pgfqpoint{0.380943in}{4.185189in}}{\pgfqpoint{4.650000in}{0.614151in}}%
\pgfusepath{clip}%
\pgfsetbuttcap%
\pgfsetroundjoin%
\definecolor{currentfill}{rgb}{0.981546,0.459977,0.459977}%
\pgfsetfillcolor{currentfill}%
\pgfsetlinewidth{0.250937pt}%
\definecolor{currentstroke}{rgb}{1.000000,1.000000,1.000000}%
\pgfsetstrokecolor{currentstroke}%
\pgfsetdash{}{0pt}%
\pgfpathmoveto{\pgfqpoint{2.486603in}{4.711604in}}%
\pgfpathlineto{\pgfqpoint{2.574339in}{4.711604in}}%
\pgfpathlineto{\pgfqpoint{2.574339in}{4.623868in}}%
\pgfpathlineto{\pgfqpoint{2.486603in}{4.623868in}}%
\pgfpathlineto{\pgfqpoint{2.486603in}{4.711604in}}%
\pgfusepath{stroke,fill}%
\end{pgfscope}%
\begin{pgfscope}%
\pgfpathrectangle{\pgfqpoint{0.380943in}{4.185189in}}{\pgfqpoint{4.650000in}{0.614151in}}%
\pgfusepath{clip}%
\pgfsetbuttcap%
\pgfsetroundjoin%
\definecolor{currentfill}{rgb}{0.979654,0.837186,0.669619}%
\pgfsetfillcolor{currentfill}%
\pgfsetlinewidth{0.250937pt}%
\definecolor{currentstroke}{rgb}{1.000000,1.000000,1.000000}%
\pgfsetstrokecolor{currentstroke}%
\pgfsetdash{}{0pt}%
\pgfpathmoveto{\pgfqpoint{2.574339in}{4.711604in}}%
\pgfpathlineto{\pgfqpoint{2.662075in}{4.711604in}}%
\pgfpathlineto{\pgfqpoint{2.662075in}{4.623868in}}%
\pgfpathlineto{\pgfqpoint{2.574339in}{4.623868in}}%
\pgfpathlineto{\pgfqpoint{2.574339in}{4.711604in}}%
\pgfusepath{stroke,fill}%
\end{pgfscope}%
\begin{pgfscope}%
\pgfpathrectangle{\pgfqpoint{0.380943in}{4.185189in}}{\pgfqpoint{4.650000in}{0.614151in}}%
\pgfusepath{clip}%
\pgfsetbuttcap%
\pgfsetroundjoin%
\definecolor{currentfill}{rgb}{1.000000,0.557862,0.511772}%
\pgfsetfillcolor{currentfill}%
\pgfsetlinewidth{0.250937pt}%
\definecolor{currentstroke}{rgb}{1.000000,1.000000,1.000000}%
\pgfsetstrokecolor{currentstroke}%
\pgfsetdash{}{0pt}%
\pgfpathmoveto{\pgfqpoint{2.662075in}{4.711604in}}%
\pgfpathlineto{\pgfqpoint{2.749811in}{4.711604in}}%
\pgfpathlineto{\pgfqpoint{2.749811in}{4.623868in}}%
\pgfpathlineto{\pgfqpoint{2.662075in}{4.623868in}}%
\pgfpathlineto{\pgfqpoint{2.662075in}{4.711604in}}%
\pgfusepath{stroke,fill}%
\end{pgfscope}%
\begin{pgfscope}%
\pgfpathrectangle{\pgfqpoint{0.380943in}{4.185189in}}{\pgfqpoint{4.650000in}{0.614151in}}%
\pgfusepath{clip}%
\pgfsetbuttcap%
\pgfsetroundjoin%
\definecolor{currentfill}{rgb}{0.996571,0.720538,0.589189}%
\pgfsetfillcolor{currentfill}%
\pgfsetlinewidth{0.250937pt}%
\definecolor{currentstroke}{rgb}{1.000000,1.000000,1.000000}%
\pgfsetstrokecolor{currentstroke}%
\pgfsetdash{}{0pt}%
\pgfpathmoveto{\pgfqpoint{2.749811in}{4.711604in}}%
\pgfpathlineto{\pgfqpoint{2.837547in}{4.711604in}}%
\pgfpathlineto{\pgfqpoint{2.837547in}{4.623868in}}%
\pgfpathlineto{\pgfqpoint{2.749811in}{4.623868in}}%
\pgfpathlineto{\pgfqpoint{2.749811in}{4.711604in}}%
\pgfusepath{stroke,fill}%
\end{pgfscope}%
\begin{pgfscope}%
\pgfpathrectangle{\pgfqpoint{0.380943in}{4.185189in}}{\pgfqpoint{4.650000in}{0.614151in}}%
\pgfusepath{clip}%
\pgfsetbuttcap%
\pgfsetroundjoin%
\definecolor{currentfill}{rgb}{0.965444,0.906113,0.711757}%
\pgfsetfillcolor{currentfill}%
\pgfsetlinewidth{0.250937pt}%
\definecolor{currentstroke}{rgb}{1.000000,1.000000,1.000000}%
\pgfsetstrokecolor{currentstroke}%
\pgfsetdash{}{0pt}%
\pgfpathmoveto{\pgfqpoint{2.837547in}{4.711604in}}%
\pgfpathlineto{\pgfqpoint{2.925283in}{4.711604in}}%
\pgfpathlineto{\pgfqpoint{2.925283in}{4.623868in}}%
\pgfpathlineto{\pgfqpoint{2.837547in}{4.623868in}}%
\pgfpathlineto{\pgfqpoint{2.837547in}{4.711604in}}%
\pgfusepath{stroke,fill}%
\end{pgfscope}%
\begin{pgfscope}%
\pgfpathrectangle{\pgfqpoint{0.380943in}{4.185189in}}{\pgfqpoint{4.650000in}{0.614151in}}%
\pgfusepath{clip}%
\pgfsetbuttcap%
\pgfsetroundjoin%
\definecolor{currentfill}{rgb}{0.962414,0.923552,0.722891}%
\pgfsetfillcolor{currentfill}%
\pgfsetlinewidth{0.250937pt}%
\definecolor{currentstroke}{rgb}{1.000000,1.000000,1.000000}%
\pgfsetstrokecolor{currentstroke}%
\pgfsetdash{}{0pt}%
\pgfpathmoveto{\pgfqpoint{2.925283in}{4.711604in}}%
\pgfpathlineto{\pgfqpoint{3.013019in}{4.711604in}}%
\pgfpathlineto{\pgfqpoint{3.013019in}{4.623868in}}%
\pgfpathlineto{\pgfqpoint{2.925283in}{4.623868in}}%
\pgfpathlineto{\pgfqpoint{2.925283in}{4.711604in}}%
\pgfusepath{stroke,fill}%
\end{pgfscope}%
\begin{pgfscope}%
\pgfpathrectangle{\pgfqpoint{0.380943in}{4.185189in}}{\pgfqpoint{4.650000in}{0.614151in}}%
\pgfusepath{clip}%
\pgfsetbuttcap%
\pgfsetroundjoin%
\definecolor{currentfill}{rgb}{0.996571,0.720538,0.589189}%
\pgfsetfillcolor{currentfill}%
\pgfsetlinewidth{0.250937pt}%
\definecolor{currentstroke}{rgb}{1.000000,1.000000,1.000000}%
\pgfsetstrokecolor{currentstroke}%
\pgfsetdash{}{0pt}%
\pgfpathmoveto{\pgfqpoint{3.013019in}{4.711604in}}%
\pgfpathlineto{\pgfqpoint{3.100754in}{4.711604in}}%
\pgfpathlineto{\pgfqpoint{3.100754in}{4.623868in}}%
\pgfpathlineto{\pgfqpoint{3.013019in}{4.623868in}}%
\pgfpathlineto{\pgfqpoint{3.013019in}{4.711604in}}%
\pgfusepath{stroke,fill}%
\end{pgfscope}%
\begin{pgfscope}%
\pgfpathrectangle{\pgfqpoint{0.380943in}{4.185189in}}{\pgfqpoint{4.650000in}{0.614151in}}%
\pgfusepath{clip}%
\pgfsetbuttcap%
\pgfsetroundjoin%
\definecolor{currentfill}{rgb}{0.972549,0.870588,0.692810}%
\pgfsetfillcolor{currentfill}%
\pgfsetlinewidth{0.250937pt}%
\definecolor{currentstroke}{rgb}{1.000000,1.000000,1.000000}%
\pgfsetstrokecolor{currentstroke}%
\pgfsetdash{}{0pt}%
\pgfpathmoveto{\pgfqpoint{3.100754in}{4.711604in}}%
\pgfpathlineto{\pgfqpoint{3.188490in}{4.711604in}}%
\pgfpathlineto{\pgfqpoint{3.188490in}{4.623868in}}%
\pgfpathlineto{\pgfqpoint{3.100754in}{4.623868in}}%
\pgfpathlineto{\pgfqpoint{3.100754in}{4.711604in}}%
\pgfusepath{stroke,fill}%
\end{pgfscope}%
\begin{pgfscope}%
\pgfpathrectangle{\pgfqpoint{0.380943in}{4.185189in}}{\pgfqpoint{4.650000in}{0.614151in}}%
\pgfusepath{clip}%
\pgfsetbuttcap%
\pgfsetroundjoin%
\definecolor{currentfill}{rgb}{0.965444,0.906113,0.711757}%
\pgfsetfillcolor{currentfill}%
\pgfsetlinewidth{0.250937pt}%
\definecolor{currentstroke}{rgb}{1.000000,1.000000,1.000000}%
\pgfsetstrokecolor{currentstroke}%
\pgfsetdash{}{0pt}%
\pgfpathmoveto{\pgfqpoint{3.188490in}{4.711604in}}%
\pgfpathlineto{\pgfqpoint{3.276226in}{4.711604in}}%
\pgfpathlineto{\pgfqpoint{3.276226in}{4.623868in}}%
\pgfpathlineto{\pgfqpoint{3.188490in}{4.623868in}}%
\pgfpathlineto{\pgfqpoint{3.188490in}{4.711604in}}%
\pgfusepath{stroke,fill}%
\end{pgfscope}%
\begin{pgfscope}%
\pgfpathrectangle{\pgfqpoint{0.380943in}{4.185189in}}{\pgfqpoint{4.650000in}{0.614151in}}%
\pgfusepath{clip}%
\pgfsetbuttcap%
\pgfsetroundjoin%
\definecolor{currentfill}{rgb}{0.968166,0.945882,0.748604}%
\pgfsetfillcolor{currentfill}%
\pgfsetlinewidth{0.250937pt}%
\definecolor{currentstroke}{rgb}{1.000000,1.000000,1.000000}%
\pgfsetstrokecolor{currentstroke}%
\pgfsetdash{}{0pt}%
\pgfpathmoveto{\pgfqpoint{3.276226in}{4.711604in}}%
\pgfpathlineto{\pgfqpoint{3.363962in}{4.711604in}}%
\pgfpathlineto{\pgfqpoint{3.363962in}{4.623868in}}%
\pgfpathlineto{\pgfqpoint{3.276226in}{4.623868in}}%
\pgfpathlineto{\pgfqpoint{3.276226in}{4.711604in}}%
\pgfusepath{stroke,fill}%
\end{pgfscope}%
\begin{pgfscope}%
\pgfpathrectangle{\pgfqpoint{0.380943in}{4.185189in}}{\pgfqpoint{4.650000in}{0.614151in}}%
\pgfusepath{clip}%
\pgfsetbuttcap%
\pgfsetroundjoin%
\definecolor{currentfill}{rgb}{0.998939,0.658962,0.556032}%
\pgfsetfillcolor{currentfill}%
\pgfsetlinewidth{0.250937pt}%
\definecolor{currentstroke}{rgb}{1.000000,1.000000,1.000000}%
\pgfsetstrokecolor{currentstroke}%
\pgfsetdash{}{0pt}%
\pgfpathmoveto{\pgfqpoint{3.363962in}{4.711604in}}%
\pgfpathlineto{\pgfqpoint{3.451698in}{4.711604in}}%
\pgfpathlineto{\pgfqpoint{3.451698in}{4.623868in}}%
\pgfpathlineto{\pgfqpoint{3.363962in}{4.623868in}}%
\pgfpathlineto{\pgfqpoint{3.363962in}{4.711604in}}%
\pgfusepath{stroke,fill}%
\end{pgfscope}%
\begin{pgfscope}%
\pgfpathrectangle{\pgfqpoint{0.380943in}{4.185189in}}{\pgfqpoint{4.650000in}{0.614151in}}%
\pgfusepath{clip}%
\pgfsetbuttcap%
\pgfsetroundjoin%
\definecolor{currentfill}{rgb}{0.979654,0.837186,0.669619}%
\pgfsetfillcolor{currentfill}%
\pgfsetlinewidth{0.250937pt}%
\definecolor{currentstroke}{rgb}{1.000000,1.000000,1.000000}%
\pgfsetstrokecolor{currentstroke}%
\pgfsetdash{}{0pt}%
\pgfpathmoveto{\pgfqpoint{3.451698in}{4.711604in}}%
\pgfpathlineto{\pgfqpoint{3.539434in}{4.711604in}}%
\pgfpathlineto{\pgfqpoint{3.539434in}{4.623868in}}%
\pgfpathlineto{\pgfqpoint{3.451698in}{4.623868in}}%
\pgfpathlineto{\pgfqpoint{3.451698in}{4.711604in}}%
\pgfusepath{stroke,fill}%
\end{pgfscope}%
\begin{pgfscope}%
\pgfpathrectangle{\pgfqpoint{0.380943in}{4.185189in}}{\pgfqpoint{4.650000in}{0.614151in}}%
\pgfusepath{clip}%
\pgfsetbuttcap%
\pgfsetroundjoin%
\definecolor{currentfill}{rgb}{0.965444,0.906113,0.711757}%
\pgfsetfillcolor{currentfill}%
\pgfsetlinewidth{0.250937pt}%
\definecolor{currentstroke}{rgb}{1.000000,1.000000,1.000000}%
\pgfsetstrokecolor{currentstroke}%
\pgfsetdash{}{0pt}%
\pgfpathmoveto{\pgfqpoint{3.539434in}{4.711604in}}%
\pgfpathlineto{\pgfqpoint{3.627169in}{4.711604in}}%
\pgfpathlineto{\pgfqpoint{3.627169in}{4.623868in}}%
\pgfpathlineto{\pgfqpoint{3.539434in}{4.623868in}}%
\pgfpathlineto{\pgfqpoint{3.539434in}{4.711604in}}%
\pgfusepath{stroke,fill}%
\end{pgfscope}%
\begin{pgfscope}%
\pgfpathrectangle{\pgfqpoint{0.380943in}{4.185189in}}{\pgfqpoint{4.650000in}{0.614151in}}%
\pgfusepath{clip}%
\pgfsetbuttcap%
\pgfsetroundjoin%
\definecolor{currentfill}{rgb}{0.986759,0.806398,0.641200}%
\pgfsetfillcolor{currentfill}%
\pgfsetlinewidth{0.250937pt}%
\definecolor{currentstroke}{rgb}{1.000000,1.000000,1.000000}%
\pgfsetstrokecolor{currentstroke}%
\pgfsetdash{}{0pt}%
\pgfpathmoveto{\pgfqpoint{3.627169in}{4.711604in}}%
\pgfpathlineto{\pgfqpoint{3.714905in}{4.711604in}}%
\pgfpathlineto{\pgfqpoint{3.714905in}{4.623868in}}%
\pgfpathlineto{\pgfqpoint{3.627169in}{4.623868in}}%
\pgfpathlineto{\pgfqpoint{3.627169in}{4.711604in}}%
\pgfusepath{stroke,fill}%
\end{pgfscope}%
\begin{pgfscope}%
\pgfpathrectangle{\pgfqpoint{0.380943in}{4.185189in}}{\pgfqpoint{4.650000in}{0.614151in}}%
\pgfusepath{clip}%
\pgfsetbuttcap%
\pgfsetroundjoin%
\definecolor{currentfill}{rgb}{0.979654,0.837186,0.669619}%
\pgfsetfillcolor{currentfill}%
\pgfsetlinewidth{0.250937pt}%
\definecolor{currentstroke}{rgb}{1.000000,1.000000,1.000000}%
\pgfsetstrokecolor{currentstroke}%
\pgfsetdash{}{0pt}%
\pgfpathmoveto{\pgfqpoint{3.714905in}{4.711604in}}%
\pgfpathlineto{\pgfqpoint{3.802641in}{4.711604in}}%
\pgfpathlineto{\pgfqpoint{3.802641in}{4.623868in}}%
\pgfpathlineto{\pgfqpoint{3.714905in}{4.623868in}}%
\pgfpathlineto{\pgfqpoint{3.714905in}{4.711604in}}%
\pgfusepath{stroke,fill}%
\end{pgfscope}%
\begin{pgfscope}%
\pgfpathrectangle{\pgfqpoint{0.380943in}{4.185189in}}{\pgfqpoint{4.650000in}{0.614151in}}%
\pgfusepath{clip}%
\pgfsetbuttcap%
\pgfsetroundjoin%
\definecolor{currentfill}{rgb}{1.000000,0.605229,0.530719}%
\pgfsetfillcolor{currentfill}%
\pgfsetlinewidth{0.250937pt}%
\definecolor{currentstroke}{rgb}{1.000000,1.000000,1.000000}%
\pgfsetstrokecolor{currentstroke}%
\pgfsetdash{}{0pt}%
\pgfpathmoveto{\pgfqpoint{3.802641in}{4.711604in}}%
\pgfpathlineto{\pgfqpoint{3.890377in}{4.711604in}}%
\pgfpathlineto{\pgfqpoint{3.890377in}{4.623868in}}%
\pgfpathlineto{\pgfqpoint{3.802641in}{4.623868in}}%
\pgfpathlineto{\pgfqpoint{3.802641in}{4.711604in}}%
\pgfusepath{stroke,fill}%
\end{pgfscope}%
\begin{pgfscope}%
\pgfpathrectangle{\pgfqpoint{0.380943in}{4.185189in}}{\pgfqpoint{4.650000in}{0.614151in}}%
\pgfusepath{clip}%
\pgfsetbuttcap%
\pgfsetroundjoin%
\definecolor{currentfill}{rgb}{0.986759,0.806398,0.641200}%
\pgfsetfillcolor{currentfill}%
\pgfsetlinewidth{0.250937pt}%
\definecolor{currentstroke}{rgb}{1.000000,1.000000,1.000000}%
\pgfsetstrokecolor{currentstroke}%
\pgfsetdash{}{0pt}%
\pgfpathmoveto{\pgfqpoint{3.890377in}{4.711604in}}%
\pgfpathlineto{\pgfqpoint{3.978113in}{4.711604in}}%
\pgfpathlineto{\pgfqpoint{3.978113in}{4.623868in}}%
\pgfpathlineto{\pgfqpoint{3.890377in}{4.623868in}}%
\pgfpathlineto{\pgfqpoint{3.890377in}{4.711604in}}%
\pgfusepath{stroke,fill}%
\end{pgfscope}%
\begin{pgfscope}%
\pgfpathrectangle{\pgfqpoint{0.380943in}{4.185189in}}{\pgfqpoint{4.650000in}{0.614151in}}%
\pgfusepath{clip}%
\pgfsetbuttcap%
\pgfsetroundjoin%
\definecolor{currentfill}{rgb}{1.000000,0.557862,0.511772}%
\pgfsetfillcolor{currentfill}%
\pgfsetlinewidth{0.250937pt}%
\definecolor{currentstroke}{rgb}{1.000000,1.000000,1.000000}%
\pgfsetstrokecolor{currentstroke}%
\pgfsetdash{}{0pt}%
\pgfpathmoveto{\pgfqpoint{3.978113in}{4.711604in}}%
\pgfpathlineto{\pgfqpoint{4.065849in}{4.711604in}}%
\pgfpathlineto{\pgfqpoint{4.065849in}{4.623868in}}%
\pgfpathlineto{\pgfqpoint{3.978113in}{4.623868in}}%
\pgfpathlineto{\pgfqpoint{3.978113in}{4.711604in}}%
\pgfusepath{stroke,fill}%
\end{pgfscope}%
\begin{pgfscope}%
\pgfpathrectangle{\pgfqpoint{0.380943in}{4.185189in}}{\pgfqpoint{4.650000in}{0.614151in}}%
\pgfusepath{clip}%
\pgfsetbuttcap%
\pgfsetroundjoin%
\definecolor{currentfill}{rgb}{0.979654,0.837186,0.669619}%
\pgfsetfillcolor{currentfill}%
\pgfsetlinewidth{0.250937pt}%
\definecolor{currentstroke}{rgb}{1.000000,1.000000,1.000000}%
\pgfsetstrokecolor{currentstroke}%
\pgfsetdash{}{0pt}%
\pgfpathmoveto{\pgfqpoint{4.065849in}{4.711604in}}%
\pgfpathlineto{\pgfqpoint{4.153585in}{4.711604in}}%
\pgfpathlineto{\pgfqpoint{4.153585in}{4.623868in}}%
\pgfpathlineto{\pgfqpoint{4.065849in}{4.623868in}}%
\pgfpathlineto{\pgfqpoint{4.065849in}{4.711604in}}%
\pgfusepath{stroke,fill}%
\end{pgfscope}%
\begin{pgfscope}%
\pgfpathrectangle{\pgfqpoint{0.380943in}{4.185189in}}{\pgfqpoint{4.650000in}{0.614151in}}%
\pgfusepath{clip}%
\pgfsetbuttcap%
\pgfsetroundjoin%
\definecolor{currentfill}{rgb}{0.979654,0.837186,0.669619}%
\pgfsetfillcolor{currentfill}%
\pgfsetlinewidth{0.250937pt}%
\definecolor{currentstroke}{rgb}{1.000000,1.000000,1.000000}%
\pgfsetstrokecolor{currentstroke}%
\pgfsetdash{}{0pt}%
\pgfpathmoveto{\pgfqpoint{4.153585in}{4.711604in}}%
\pgfpathlineto{\pgfqpoint{4.241320in}{4.711604in}}%
\pgfpathlineto{\pgfqpoint{4.241320in}{4.623868in}}%
\pgfpathlineto{\pgfqpoint{4.153585in}{4.623868in}}%
\pgfpathlineto{\pgfqpoint{4.153585in}{4.711604in}}%
\pgfusepath{stroke,fill}%
\end{pgfscope}%
\begin{pgfscope}%
\pgfpathrectangle{\pgfqpoint{0.380943in}{4.185189in}}{\pgfqpoint{4.650000in}{0.614151in}}%
\pgfusepath{clip}%
\pgfsetbuttcap%
\pgfsetroundjoin%
\definecolor{currentfill}{rgb}{0.962414,0.923552,0.722891}%
\pgfsetfillcolor{currentfill}%
\pgfsetlinewidth{0.250937pt}%
\definecolor{currentstroke}{rgb}{1.000000,1.000000,1.000000}%
\pgfsetstrokecolor{currentstroke}%
\pgfsetdash{}{0pt}%
\pgfpathmoveto{\pgfqpoint{4.241320in}{4.711604in}}%
\pgfpathlineto{\pgfqpoint{4.329056in}{4.711604in}}%
\pgfpathlineto{\pgfqpoint{4.329056in}{4.623868in}}%
\pgfpathlineto{\pgfqpoint{4.241320in}{4.623868in}}%
\pgfpathlineto{\pgfqpoint{4.241320in}{4.711604in}}%
\pgfusepath{stroke,fill}%
\end{pgfscope}%
\begin{pgfscope}%
\pgfpathrectangle{\pgfqpoint{0.380943in}{4.185189in}}{\pgfqpoint{4.650000in}{0.614151in}}%
\pgfusepath{clip}%
\pgfsetbuttcap%
\pgfsetroundjoin%
\definecolor{currentfill}{rgb}{1.000000,0.509404,0.491473}%
\pgfsetfillcolor{currentfill}%
\pgfsetlinewidth{0.250937pt}%
\definecolor{currentstroke}{rgb}{1.000000,1.000000,1.000000}%
\pgfsetstrokecolor{currentstroke}%
\pgfsetdash{}{0pt}%
\pgfpathmoveto{\pgfqpoint{4.329056in}{4.711604in}}%
\pgfpathlineto{\pgfqpoint{4.416792in}{4.711604in}}%
\pgfpathlineto{\pgfqpoint{4.416792in}{4.623868in}}%
\pgfpathlineto{\pgfqpoint{4.329056in}{4.623868in}}%
\pgfpathlineto{\pgfqpoint{4.329056in}{4.711604in}}%
\pgfusepath{stroke,fill}%
\end{pgfscope}%
\begin{pgfscope}%
\pgfpathrectangle{\pgfqpoint{0.380943in}{4.185189in}}{\pgfqpoint{4.650000in}{0.614151in}}%
\pgfusepath{clip}%
\pgfsetbuttcap%
\pgfsetroundjoin%
\definecolor{currentfill}{rgb}{0.992326,0.765229,0.614840}%
\pgfsetfillcolor{currentfill}%
\pgfsetlinewidth{0.250937pt}%
\definecolor{currentstroke}{rgb}{1.000000,1.000000,1.000000}%
\pgfsetstrokecolor{currentstroke}%
\pgfsetdash{}{0pt}%
\pgfpathmoveto{\pgfqpoint{4.416792in}{4.711604in}}%
\pgfpathlineto{\pgfqpoint{4.504528in}{4.711604in}}%
\pgfpathlineto{\pgfqpoint{4.504528in}{4.623868in}}%
\pgfpathlineto{\pgfqpoint{4.416792in}{4.623868in}}%
\pgfpathlineto{\pgfqpoint{4.416792in}{4.711604in}}%
\pgfusepath{stroke,fill}%
\end{pgfscope}%
\begin{pgfscope}%
\pgfpathrectangle{\pgfqpoint{0.380943in}{4.185189in}}{\pgfqpoint{4.650000in}{0.614151in}}%
\pgfusepath{clip}%
\pgfsetbuttcap%
\pgfsetroundjoin%
\definecolor{currentfill}{rgb}{0.986759,0.806398,0.641200}%
\pgfsetfillcolor{currentfill}%
\pgfsetlinewidth{0.250937pt}%
\definecolor{currentstroke}{rgb}{1.000000,1.000000,1.000000}%
\pgfsetstrokecolor{currentstroke}%
\pgfsetdash{}{0pt}%
\pgfpathmoveto{\pgfqpoint{4.504528in}{4.711604in}}%
\pgfpathlineto{\pgfqpoint{4.592264in}{4.711604in}}%
\pgfpathlineto{\pgfqpoint{4.592264in}{4.623868in}}%
\pgfpathlineto{\pgfqpoint{4.504528in}{4.623868in}}%
\pgfpathlineto{\pgfqpoint{4.504528in}{4.711604in}}%
\pgfusepath{stroke,fill}%
\end{pgfscope}%
\begin{pgfscope}%
\pgfpathrectangle{\pgfqpoint{0.380943in}{4.185189in}}{\pgfqpoint{4.650000in}{0.614151in}}%
\pgfusepath{clip}%
\pgfsetbuttcap%
\pgfsetroundjoin%
\definecolor{currentfill}{rgb}{0.996571,0.720538,0.589189}%
\pgfsetfillcolor{currentfill}%
\pgfsetlinewidth{0.250937pt}%
\definecolor{currentstroke}{rgb}{1.000000,1.000000,1.000000}%
\pgfsetstrokecolor{currentstroke}%
\pgfsetdash{}{0pt}%
\pgfpathmoveto{\pgfqpoint{4.592264in}{4.711604in}}%
\pgfpathlineto{\pgfqpoint{4.680000in}{4.711604in}}%
\pgfpathlineto{\pgfqpoint{4.680000in}{4.623868in}}%
\pgfpathlineto{\pgfqpoint{4.592264in}{4.623868in}}%
\pgfpathlineto{\pgfqpoint{4.592264in}{4.711604in}}%
\pgfusepath{stroke,fill}%
\end{pgfscope}%
\begin{pgfscope}%
\pgfpathrectangle{\pgfqpoint{0.380943in}{4.185189in}}{\pgfqpoint{4.650000in}{0.614151in}}%
\pgfusepath{clip}%
\pgfsetbuttcap%
\pgfsetroundjoin%
\definecolor{currentfill}{rgb}{0.979654,0.837186,0.669619}%
\pgfsetfillcolor{currentfill}%
\pgfsetlinewidth{0.250937pt}%
\definecolor{currentstroke}{rgb}{1.000000,1.000000,1.000000}%
\pgfsetstrokecolor{currentstroke}%
\pgfsetdash{}{0pt}%
\pgfpathmoveto{\pgfqpoint{4.680000in}{4.711604in}}%
\pgfpathlineto{\pgfqpoint{4.767736in}{4.711604in}}%
\pgfpathlineto{\pgfqpoint{4.767736in}{4.623868in}}%
\pgfpathlineto{\pgfqpoint{4.680000in}{4.623868in}}%
\pgfpathlineto{\pgfqpoint{4.680000in}{4.711604in}}%
\pgfusepath{stroke,fill}%
\end{pgfscope}%
\begin{pgfscope}%
\pgfpathrectangle{\pgfqpoint{0.380943in}{4.185189in}}{\pgfqpoint{4.650000in}{0.614151in}}%
\pgfusepath{clip}%
\pgfsetbuttcap%
\pgfsetroundjoin%
\definecolor{currentfill}{rgb}{0.996571,0.720538,0.589189}%
\pgfsetfillcolor{currentfill}%
\pgfsetlinewidth{0.250937pt}%
\definecolor{currentstroke}{rgb}{1.000000,1.000000,1.000000}%
\pgfsetstrokecolor{currentstroke}%
\pgfsetdash{}{0pt}%
\pgfpathmoveto{\pgfqpoint{4.767736in}{4.711604in}}%
\pgfpathlineto{\pgfqpoint{4.855471in}{4.711604in}}%
\pgfpathlineto{\pgfqpoint{4.855471in}{4.623868in}}%
\pgfpathlineto{\pgfqpoint{4.767736in}{4.623868in}}%
\pgfpathlineto{\pgfqpoint{4.767736in}{4.711604in}}%
\pgfusepath{stroke,fill}%
\end{pgfscope}%
\begin{pgfscope}%
\pgfpathrectangle{\pgfqpoint{0.380943in}{4.185189in}}{\pgfqpoint{4.650000in}{0.614151in}}%
\pgfusepath{clip}%
\pgfsetbuttcap%
\pgfsetroundjoin%
\definecolor{currentfill}{rgb}{1.000000,1.000000,0.929412}%
\pgfsetfillcolor{currentfill}%
\pgfsetlinewidth{0.250937pt}%
\definecolor{currentstroke}{rgb}{1.000000,1.000000,1.000000}%
\pgfsetstrokecolor{currentstroke}%
\pgfsetdash{}{0pt}%
\pgfpathmoveto{\pgfqpoint{4.855471in}{4.711604in}}%
\pgfpathlineto{\pgfqpoint{4.943207in}{4.711604in}}%
\pgfpathlineto{\pgfqpoint{4.943207in}{4.623868in}}%
\pgfpathlineto{\pgfqpoint{4.855471in}{4.623868in}}%
\pgfpathlineto{\pgfqpoint{4.855471in}{4.711604in}}%
\pgfusepath{stroke,fill}%
\end{pgfscope}%
\begin{pgfscope}%
\pgfpathrectangle{\pgfqpoint{0.380943in}{4.185189in}}{\pgfqpoint{4.650000in}{0.614151in}}%
\pgfusepath{clip}%
\pgfsetbuttcap%
\pgfsetroundjoin%
\definecolor{currentfill}{rgb}{0.968166,0.945882,0.748604}%
\pgfsetfillcolor{currentfill}%
\pgfsetlinewidth{0.250937pt}%
\definecolor{currentstroke}{rgb}{1.000000,1.000000,1.000000}%
\pgfsetstrokecolor{currentstroke}%
\pgfsetdash{}{0pt}%
\pgfpathmoveto{\pgfqpoint{4.943207in}{4.711604in}}%
\pgfpathlineto{\pgfqpoint{5.030943in}{4.711604in}}%
\pgfpathlineto{\pgfqpoint{5.030943in}{4.623868in}}%
\pgfpathlineto{\pgfqpoint{4.943207in}{4.623868in}}%
\pgfpathlineto{\pgfqpoint{4.943207in}{4.711604in}}%
\pgfusepath{stroke,fill}%
\end{pgfscope}%
\begin{pgfscope}%
\pgfpathrectangle{\pgfqpoint{0.380943in}{4.185189in}}{\pgfqpoint{4.650000in}{0.614151in}}%
\pgfusepath{clip}%
\pgfsetbuttcap%
\pgfsetroundjoin%
\definecolor{currentfill}{rgb}{0.986759,0.806398,0.641200}%
\pgfsetfillcolor{currentfill}%
\pgfsetlinewidth{0.250937pt}%
\definecolor{currentstroke}{rgb}{1.000000,1.000000,1.000000}%
\pgfsetstrokecolor{currentstroke}%
\pgfsetdash{}{0pt}%
\pgfpathmoveto{\pgfqpoint{0.380943in}{4.623868in}}%
\pgfpathlineto{\pgfqpoint{0.468679in}{4.623868in}}%
\pgfpathlineto{\pgfqpoint{0.468679in}{4.536132in}}%
\pgfpathlineto{\pgfqpoint{0.380943in}{4.536132in}}%
\pgfpathlineto{\pgfqpoint{0.380943in}{4.623868in}}%
\pgfusepath{stroke,fill}%
\end{pgfscope}%
\begin{pgfscope}%
\pgfpathrectangle{\pgfqpoint{0.380943in}{4.185189in}}{\pgfqpoint{4.650000in}{0.614151in}}%
\pgfusepath{clip}%
\pgfsetbuttcap%
\pgfsetroundjoin%
\definecolor{currentfill}{rgb}{0.972549,0.870588,0.692810}%
\pgfsetfillcolor{currentfill}%
\pgfsetlinewidth{0.250937pt}%
\definecolor{currentstroke}{rgb}{1.000000,1.000000,1.000000}%
\pgfsetstrokecolor{currentstroke}%
\pgfsetdash{}{0pt}%
\pgfpathmoveto{\pgfqpoint{0.468679in}{4.623868in}}%
\pgfpathlineto{\pgfqpoint{0.556415in}{4.623868in}}%
\pgfpathlineto{\pgfqpoint{0.556415in}{4.536132in}}%
\pgfpathlineto{\pgfqpoint{0.468679in}{4.536132in}}%
\pgfpathlineto{\pgfqpoint{0.468679in}{4.623868in}}%
\pgfusepath{stroke,fill}%
\end{pgfscope}%
\begin{pgfscope}%
\pgfpathrectangle{\pgfqpoint{0.380943in}{4.185189in}}{\pgfqpoint{4.650000in}{0.614151in}}%
\pgfusepath{clip}%
\pgfsetbuttcap%
\pgfsetroundjoin%
\definecolor{currentfill}{rgb}{0.972549,0.870588,0.692810}%
\pgfsetfillcolor{currentfill}%
\pgfsetlinewidth{0.250937pt}%
\definecolor{currentstroke}{rgb}{1.000000,1.000000,1.000000}%
\pgfsetstrokecolor{currentstroke}%
\pgfsetdash{}{0pt}%
\pgfpathmoveto{\pgfqpoint{0.556415in}{4.623868in}}%
\pgfpathlineto{\pgfqpoint{0.644151in}{4.623868in}}%
\pgfpathlineto{\pgfqpoint{0.644151in}{4.536132in}}%
\pgfpathlineto{\pgfqpoint{0.556415in}{4.536132in}}%
\pgfpathlineto{\pgfqpoint{0.556415in}{4.623868in}}%
\pgfusepath{stroke,fill}%
\end{pgfscope}%
\begin{pgfscope}%
\pgfpathrectangle{\pgfqpoint{0.380943in}{4.185189in}}{\pgfqpoint{4.650000in}{0.614151in}}%
\pgfusepath{clip}%
\pgfsetbuttcap%
\pgfsetroundjoin%
\definecolor{currentfill}{rgb}{0.996571,0.720538,0.589189}%
\pgfsetfillcolor{currentfill}%
\pgfsetlinewidth{0.250937pt}%
\definecolor{currentstroke}{rgb}{1.000000,1.000000,1.000000}%
\pgfsetstrokecolor{currentstroke}%
\pgfsetdash{}{0pt}%
\pgfpathmoveto{\pgfqpoint{0.644151in}{4.623868in}}%
\pgfpathlineto{\pgfqpoint{0.731886in}{4.623868in}}%
\pgfpathlineto{\pgfqpoint{0.731886in}{4.536132in}}%
\pgfpathlineto{\pgfqpoint{0.644151in}{4.536132in}}%
\pgfpathlineto{\pgfqpoint{0.644151in}{4.623868in}}%
\pgfusepath{stroke,fill}%
\end{pgfscope}%
\begin{pgfscope}%
\pgfpathrectangle{\pgfqpoint{0.380943in}{4.185189in}}{\pgfqpoint{4.650000in}{0.614151in}}%
\pgfusepath{clip}%
\pgfsetbuttcap%
\pgfsetroundjoin%
\definecolor{currentfill}{rgb}{1.000000,0.509404,0.491473}%
\pgfsetfillcolor{currentfill}%
\pgfsetlinewidth{0.250937pt}%
\definecolor{currentstroke}{rgb}{1.000000,1.000000,1.000000}%
\pgfsetstrokecolor{currentstroke}%
\pgfsetdash{}{0pt}%
\pgfpathmoveto{\pgfqpoint{0.731886in}{4.623868in}}%
\pgfpathlineto{\pgfqpoint{0.819622in}{4.623868in}}%
\pgfpathlineto{\pgfqpoint{0.819622in}{4.536132in}}%
\pgfpathlineto{\pgfqpoint{0.731886in}{4.536132in}}%
\pgfpathlineto{\pgfqpoint{0.731886in}{4.623868in}}%
\pgfusepath{stroke,fill}%
\end{pgfscope}%
\begin{pgfscope}%
\pgfpathrectangle{\pgfqpoint{0.380943in}{4.185189in}}{\pgfqpoint{4.650000in}{0.614151in}}%
\pgfusepath{clip}%
\pgfsetbuttcap%
\pgfsetroundjoin%
\definecolor{currentfill}{rgb}{0.996571,0.720538,0.589189}%
\pgfsetfillcolor{currentfill}%
\pgfsetlinewidth{0.250937pt}%
\definecolor{currentstroke}{rgb}{1.000000,1.000000,1.000000}%
\pgfsetstrokecolor{currentstroke}%
\pgfsetdash{}{0pt}%
\pgfpathmoveto{\pgfqpoint{0.819622in}{4.623868in}}%
\pgfpathlineto{\pgfqpoint{0.907358in}{4.623868in}}%
\pgfpathlineto{\pgfqpoint{0.907358in}{4.536132in}}%
\pgfpathlineto{\pgfqpoint{0.819622in}{4.536132in}}%
\pgfpathlineto{\pgfqpoint{0.819622in}{4.623868in}}%
\pgfusepath{stroke,fill}%
\end{pgfscope}%
\begin{pgfscope}%
\pgfpathrectangle{\pgfqpoint{0.380943in}{4.185189in}}{\pgfqpoint{4.650000in}{0.614151in}}%
\pgfusepath{clip}%
\pgfsetbuttcap%
\pgfsetroundjoin%
\definecolor{currentfill}{rgb}{0.972549,0.870588,0.692810}%
\pgfsetfillcolor{currentfill}%
\pgfsetlinewidth{0.250937pt}%
\definecolor{currentstroke}{rgb}{1.000000,1.000000,1.000000}%
\pgfsetstrokecolor{currentstroke}%
\pgfsetdash{}{0pt}%
\pgfpathmoveto{\pgfqpoint{0.907358in}{4.623868in}}%
\pgfpathlineto{\pgfqpoint{0.995094in}{4.623868in}}%
\pgfpathlineto{\pgfqpoint{0.995094in}{4.536132in}}%
\pgfpathlineto{\pgfqpoint{0.907358in}{4.536132in}}%
\pgfpathlineto{\pgfqpoint{0.907358in}{4.623868in}}%
\pgfusepath{stroke,fill}%
\end{pgfscope}%
\begin{pgfscope}%
\pgfpathrectangle{\pgfqpoint{0.380943in}{4.185189in}}{\pgfqpoint{4.650000in}{0.614151in}}%
\pgfusepath{clip}%
\pgfsetbuttcap%
\pgfsetroundjoin%
\definecolor{currentfill}{rgb}{0.998939,0.658962,0.556032}%
\pgfsetfillcolor{currentfill}%
\pgfsetlinewidth{0.250937pt}%
\definecolor{currentstroke}{rgb}{1.000000,1.000000,1.000000}%
\pgfsetstrokecolor{currentstroke}%
\pgfsetdash{}{0pt}%
\pgfpathmoveto{\pgfqpoint{0.995094in}{4.623868in}}%
\pgfpathlineto{\pgfqpoint{1.082830in}{4.623868in}}%
\pgfpathlineto{\pgfqpoint{1.082830in}{4.536132in}}%
\pgfpathlineto{\pgfqpoint{0.995094in}{4.536132in}}%
\pgfpathlineto{\pgfqpoint{0.995094in}{4.623868in}}%
\pgfusepath{stroke,fill}%
\end{pgfscope}%
\begin{pgfscope}%
\pgfpathrectangle{\pgfqpoint{0.380943in}{4.185189in}}{\pgfqpoint{4.650000in}{0.614151in}}%
\pgfusepath{clip}%
\pgfsetbuttcap%
\pgfsetroundjoin%
\definecolor{currentfill}{rgb}{0.998939,0.658962,0.556032}%
\pgfsetfillcolor{currentfill}%
\pgfsetlinewidth{0.250937pt}%
\definecolor{currentstroke}{rgb}{1.000000,1.000000,1.000000}%
\pgfsetstrokecolor{currentstroke}%
\pgfsetdash{}{0pt}%
\pgfpathmoveto{\pgfqpoint{1.082830in}{4.623868in}}%
\pgfpathlineto{\pgfqpoint{1.170566in}{4.623868in}}%
\pgfpathlineto{\pgfqpoint{1.170566in}{4.536132in}}%
\pgfpathlineto{\pgfqpoint{1.082830in}{4.536132in}}%
\pgfpathlineto{\pgfqpoint{1.082830in}{4.623868in}}%
\pgfusepath{stroke,fill}%
\end{pgfscope}%
\begin{pgfscope}%
\pgfpathrectangle{\pgfqpoint{0.380943in}{4.185189in}}{\pgfqpoint{4.650000in}{0.614151in}}%
\pgfusepath{clip}%
\pgfsetbuttcap%
\pgfsetroundjoin%
\definecolor{currentfill}{rgb}{0.986759,0.806398,0.641200}%
\pgfsetfillcolor{currentfill}%
\pgfsetlinewidth{0.250937pt}%
\definecolor{currentstroke}{rgb}{1.000000,1.000000,1.000000}%
\pgfsetstrokecolor{currentstroke}%
\pgfsetdash{}{0pt}%
\pgfpathmoveto{\pgfqpoint{1.170566in}{4.623868in}}%
\pgfpathlineto{\pgfqpoint{1.258302in}{4.623868in}}%
\pgfpathlineto{\pgfqpoint{1.258302in}{4.536132in}}%
\pgfpathlineto{\pgfqpoint{1.170566in}{4.536132in}}%
\pgfpathlineto{\pgfqpoint{1.170566in}{4.623868in}}%
\pgfusepath{stroke,fill}%
\end{pgfscope}%
\begin{pgfscope}%
\pgfpathrectangle{\pgfqpoint{0.380943in}{4.185189in}}{\pgfqpoint{4.650000in}{0.614151in}}%
\pgfusepath{clip}%
\pgfsetbuttcap%
\pgfsetroundjoin%
\definecolor{currentfill}{rgb}{1.000000,0.605229,0.530719}%
\pgfsetfillcolor{currentfill}%
\pgfsetlinewidth{0.250937pt}%
\definecolor{currentstroke}{rgb}{1.000000,1.000000,1.000000}%
\pgfsetstrokecolor{currentstroke}%
\pgfsetdash{}{0pt}%
\pgfpathmoveto{\pgfqpoint{1.258302in}{4.623868in}}%
\pgfpathlineto{\pgfqpoint{1.346037in}{4.623868in}}%
\pgfpathlineto{\pgfqpoint{1.346037in}{4.536132in}}%
\pgfpathlineto{\pgfqpoint{1.258302in}{4.536132in}}%
\pgfpathlineto{\pgfqpoint{1.258302in}{4.623868in}}%
\pgfusepath{stroke,fill}%
\end{pgfscope}%
\begin{pgfscope}%
\pgfpathrectangle{\pgfqpoint{0.380943in}{4.185189in}}{\pgfqpoint{4.650000in}{0.614151in}}%
\pgfusepath{clip}%
\pgfsetbuttcap%
\pgfsetroundjoin%
\definecolor{currentfill}{rgb}{0.996571,0.720538,0.589189}%
\pgfsetfillcolor{currentfill}%
\pgfsetlinewidth{0.250937pt}%
\definecolor{currentstroke}{rgb}{1.000000,1.000000,1.000000}%
\pgfsetstrokecolor{currentstroke}%
\pgfsetdash{}{0pt}%
\pgfpathmoveto{\pgfqpoint{1.346037in}{4.623868in}}%
\pgfpathlineto{\pgfqpoint{1.433773in}{4.623868in}}%
\pgfpathlineto{\pgfqpoint{1.433773in}{4.536132in}}%
\pgfpathlineto{\pgfqpoint{1.346037in}{4.536132in}}%
\pgfpathlineto{\pgfqpoint{1.346037in}{4.623868in}}%
\pgfusepath{stroke,fill}%
\end{pgfscope}%
\begin{pgfscope}%
\pgfpathrectangle{\pgfqpoint{0.380943in}{4.185189in}}{\pgfqpoint{4.650000in}{0.614151in}}%
\pgfusepath{clip}%
\pgfsetbuttcap%
\pgfsetroundjoin%
\definecolor{currentfill}{rgb}{0.965444,0.906113,0.711757}%
\pgfsetfillcolor{currentfill}%
\pgfsetlinewidth{0.250937pt}%
\definecolor{currentstroke}{rgb}{1.000000,1.000000,1.000000}%
\pgfsetstrokecolor{currentstroke}%
\pgfsetdash{}{0pt}%
\pgfpathmoveto{\pgfqpoint{1.433773in}{4.623868in}}%
\pgfpathlineto{\pgfqpoint{1.521509in}{4.623868in}}%
\pgfpathlineto{\pgfqpoint{1.521509in}{4.536132in}}%
\pgfpathlineto{\pgfqpoint{1.433773in}{4.536132in}}%
\pgfpathlineto{\pgfqpoint{1.433773in}{4.623868in}}%
\pgfusepath{stroke,fill}%
\end{pgfscope}%
\begin{pgfscope}%
\pgfpathrectangle{\pgfqpoint{0.380943in}{4.185189in}}{\pgfqpoint{4.650000in}{0.614151in}}%
\pgfusepath{clip}%
\pgfsetbuttcap%
\pgfsetroundjoin%
\definecolor{currentfill}{rgb}{0.992326,0.765229,0.614840}%
\pgfsetfillcolor{currentfill}%
\pgfsetlinewidth{0.250937pt}%
\definecolor{currentstroke}{rgb}{1.000000,1.000000,1.000000}%
\pgfsetstrokecolor{currentstroke}%
\pgfsetdash{}{0pt}%
\pgfpathmoveto{\pgfqpoint{1.521509in}{4.623868in}}%
\pgfpathlineto{\pgfqpoint{1.609245in}{4.623868in}}%
\pgfpathlineto{\pgfqpoint{1.609245in}{4.536132in}}%
\pgfpathlineto{\pgfqpoint{1.521509in}{4.536132in}}%
\pgfpathlineto{\pgfqpoint{1.521509in}{4.623868in}}%
\pgfusepath{stroke,fill}%
\end{pgfscope}%
\begin{pgfscope}%
\pgfpathrectangle{\pgfqpoint{0.380943in}{4.185189in}}{\pgfqpoint{4.650000in}{0.614151in}}%
\pgfusepath{clip}%
\pgfsetbuttcap%
\pgfsetroundjoin%
\definecolor{currentfill}{rgb}{0.986759,0.806398,0.641200}%
\pgfsetfillcolor{currentfill}%
\pgfsetlinewidth{0.250937pt}%
\definecolor{currentstroke}{rgb}{1.000000,1.000000,1.000000}%
\pgfsetstrokecolor{currentstroke}%
\pgfsetdash{}{0pt}%
\pgfpathmoveto{\pgfqpoint{1.609245in}{4.623868in}}%
\pgfpathlineto{\pgfqpoint{1.696981in}{4.623868in}}%
\pgfpathlineto{\pgfqpoint{1.696981in}{4.536132in}}%
\pgfpathlineto{\pgfqpoint{1.609245in}{4.536132in}}%
\pgfpathlineto{\pgfqpoint{1.609245in}{4.623868in}}%
\pgfusepath{stroke,fill}%
\end{pgfscope}%
\begin{pgfscope}%
\pgfpathrectangle{\pgfqpoint{0.380943in}{4.185189in}}{\pgfqpoint{4.650000in}{0.614151in}}%
\pgfusepath{clip}%
\pgfsetbuttcap%
\pgfsetroundjoin%
\definecolor{currentfill}{rgb}{0.992326,0.765229,0.614840}%
\pgfsetfillcolor{currentfill}%
\pgfsetlinewidth{0.250937pt}%
\definecolor{currentstroke}{rgb}{1.000000,1.000000,1.000000}%
\pgfsetstrokecolor{currentstroke}%
\pgfsetdash{}{0pt}%
\pgfpathmoveto{\pgfqpoint{1.696981in}{4.623868in}}%
\pgfpathlineto{\pgfqpoint{1.784717in}{4.623868in}}%
\pgfpathlineto{\pgfqpoint{1.784717in}{4.536132in}}%
\pgfpathlineto{\pgfqpoint{1.696981in}{4.536132in}}%
\pgfpathlineto{\pgfqpoint{1.696981in}{4.623868in}}%
\pgfusepath{stroke,fill}%
\end{pgfscope}%
\begin{pgfscope}%
\pgfpathrectangle{\pgfqpoint{0.380943in}{4.185189in}}{\pgfqpoint{4.650000in}{0.614151in}}%
\pgfusepath{clip}%
\pgfsetbuttcap%
\pgfsetroundjoin%
\definecolor{currentfill}{rgb}{0.979654,0.837186,0.669619}%
\pgfsetfillcolor{currentfill}%
\pgfsetlinewidth{0.250937pt}%
\definecolor{currentstroke}{rgb}{1.000000,1.000000,1.000000}%
\pgfsetstrokecolor{currentstroke}%
\pgfsetdash{}{0pt}%
\pgfpathmoveto{\pgfqpoint{1.784717in}{4.623868in}}%
\pgfpathlineto{\pgfqpoint{1.872452in}{4.623868in}}%
\pgfpathlineto{\pgfqpoint{1.872452in}{4.536132in}}%
\pgfpathlineto{\pgfqpoint{1.784717in}{4.536132in}}%
\pgfpathlineto{\pgfqpoint{1.784717in}{4.623868in}}%
\pgfusepath{stroke,fill}%
\end{pgfscope}%
\begin{pgfscope}%
\pgfpathrectangle{\pgfqpoint{0.380943in}{4.185189in}}{\pgfqpoint{4.650000in}{0.614151in}}%
\pgfusepath{clip}%
\pgfsetbuttcap%
\pgfsetroundjoin%
\definecolor{currentfill}{rgb}{0.965444,0.906113,0.711757}%
\pgfsetfillcolor{currentfill}%
\pgfsetlinewidth{0.250937pt}%
\definecolor{currentstroke}{rgb}{1.000000,1.000000,1.000000}%
\pgfsetstrokecolor{currentstroke}%
\pgfsetdash{}{0pt}%
\pgfpathmoveto{\pgfqpoint{1.872452in}{4.623868in}}%
\pgfpathlineto{\pgfqpoint{1.960188in}{4.623868in}}%
\pgfpathlineto{\pgfqpoint{1.960188in}{4.536132in}}%
\pgfpathlineto{\pgfqpoint{1.872452in}{4.536132in}}%
\pgfpathlineto{\pgfqpoint{1.872452in}{4.623868in}}%
\pgfusepath{stroke,fill}%
\end{pgfscope}%
\begin{pgfscope}%
\pgfpathrectangle{\pgfqpoint{0.380943in}{4.185189in}}{\pgfqpoint{4.650000in}{0.614151in}}%
\pgfusepath{clip}%
\pgfsetbuttcap%
\pgfsetroundjoin%
\definecolor{currentfill}{rgb}{0.962414,0.923552,0.722891}%
\pgfsetfillcolor{currentfill}%
\pgfsetlinewidth{0.250937pt}%
\definecolor{currentstroke}{rgb}{1.000000,1.000000,1.000000}%
\pgfsetstrokecolor{currentstroke}%
\pgfsetdash{}{0pt}%
\pgfpathmoveto{\pgfqpoint{1.960188in}{4.623868in}}%
\pgfpathlineto{\pgfqpoint{2.047924in}{4.623868in}}%
\pgfpathlineto{\pgfqpoint{2.047924in}{4.536132in}}%
\pgfpathlineto{\pgfqpoint{1.960188in}{4.536132in}}%
\pgfpathlineto{\pgfqpoint{1.960188in}{4.623868in}}%
\pgfusepath{stroke,fill}%
\end{pgfscope}%
\begin{pgfscope}%
\pgfpathrectangle{\pgfqpoint{0.380943in}{4.185189in}}{\pgfqpoint{4.650000in}{0.614151in}}%
\pgfusepath{clip}%
\pgfsetbuttcap%
\pgfsetroundjoin%
\definecolor{currentfill}{rgb}{0.986759,0.806398,0.641200}%
\pgfsetfillcolor{currentfill}%
\pgfsetlinewidth{0.250937pt}%
\definecolor{currentstroke}{rgb}{1.000000,1.000000,1.000000}%
\pgfsetstrokecolor{currentstroke}%
\pgfsetdash{}{0pt}%
\pgfpathmoveto{\pgfqpoint{2.047924in}{4.623868in}}%
\pgfpathlineto{\pgfqpoint{2.135660in}{4.623868in}}%
\pgfpathlineto{\pgfqpoint{2.135660in}{4.536132in}}%
\pgfpathlineto{\pgfqpoint{2.047924in}{4.536132in}}%
\pgfpathlineto{\pgfqpoint{2.047924in}{4.623868in}}%
\pgfusepath{stroke,fill}%
\end{pgfscope}%
\begin{pgfscope}%
\pgfpathrectangle{\pgfqpoint{0.380943in}{4.185189in}}{\pgfqpoint{4.650000in}{0.614151in}}%
\pgfusepath{clip}%
\pgfsetbuttcap%
\pgfsetroundjoin%
\definecolor{currentfill}{rgb}{0.962414,0.923552,0.722891}%
\pgfsetfillcolor{currentfill}%
\pgfsetlinewidth{0.250937pt}%
\definecolor{currentstroke}{rgb}{1.000000,1.000000,1.000000}%
\pgfsetstrokecolor{currentstroke}%
\pgfsetdash{}{0pt}%
\pgfpathmoveto{\pgfqpoint{2.135660in}{4.623868in}}%
\pgfpathlineto{\pgfqpoint{2.223396in}{4.623868in}}%
\pgfpathlineto{\pgfqpoint{2.223396in}{4.536132in}}%
\pgfpathlineto{\pgfqpoint{2.135660in}{4.536132in}}%
\pgfpathlineto{\pgfqpoint{2.135660in}{4.623868in}}%
\pgfusepath{stroke,fill}%
\end{pgfscope}%
\begin{pgfscope}%
\pgfpathrectangle{\pgfqpoint{0.380943in}{4.185189in}}{\pgfqpoint{4.650000in}{0.614151in}}%
\pgfusepath{clip}%
\pgfsetbuttcap%
\pgfsetroundjoin%
\definecolor{currentfill}{rgb}{0.992326,0.765229,0.614840}%
\pgfsetfillcolor{currentfill}%
\pgfsetlinewidth{0.250937pt}%
\definecolor{currentstroke}{rgb}{1.000000,1.000000,1.000000}%
\pgfsetstrokecolor{currentstroke}%
\pgfsetdash{}{0pt}%
\pgfpathmoveto{\pgfqpoint{2.223396in}{4.623868in}}%
\pgfpathlineto{\pgfqpoint{2.311132in}{4.623868in}}%
\pgfpathlineto{\pgfqpoint{2.311132in}{4.536132in}}%
\pgfpathlineto{\pgfqpoint{2.223396in}{4.536132in}}%
\pgfpathlineto{\pgfqpoint{2.223396in}{4.623868in}}%
\pgfusepath{stroke,fill}%
\end{pgfscope}%
\begin{pgfscope}%
\pgfpathrectangle{\pgfqpoint{0.380943in}{4.185189in}}{\pgfqpoint{4.650000in}{0.614151in}}%
\pgfusepath{clip}%
\pgfsetbuttcap%
\pgfsetroundjoin%
\definecolor{currentfill}{rgb}{0.996571,0.720538,0.589189}%
\pgfsetfillcolor{currentfill}%
\pgfsetlinewidth{0.250937pt}%
\definecolor{currentstroke}{rgb}{1.000000,1.000000,1.000000}%
\pgfsetstrokecolor{currentstroke}%
\pgfsetdash{}{0pt}%
\pgfpathmoveto{\pgfqpoint{2.311132in}{4.623868in}}%
\pgfpathlineto{\pgfqpoint{2.398868in}{4.623868in}}%
\pgfpathlineto{\pgfqpoint{2.398868in}{4.536132in}}%
\pgfpathlineto{\pgfqpoint{2.311132in}{4.536132in}}%
\pgfpathlineto{\pgfqpoint{2.311132in}{4.623868in}}%
\pgfusepath{stroke,fill}%
\end{pgfscope}%
\begin{pgfscope}%
\pgfpathrectangle{\pgfqpoint{0.380943in}{4.185189in}}{\pgfqpoint{4.650000in}{0.614151in}}%
\pgfusepath{clip}%
\pgfsetbuttcap%
\pgfsetroundjoin%
\definecolor{currentfill}{rgb}{1.000000,0.557862,0.511772}%
\pgfsetfillcolor{currentfill}%
\pgfsetlinewidth{0.250937pt}%
\definecolor{currentstroke}{rgb}{1.000000,1.000000,1.000000}%
\pgfsetstrokecolor{currentstroke}%
\pgfsetdash{}{0pt}%
\pgfpathmoveto{\pgfqpoint{2.398868in}{4.623868in}}%
\pgfpathlineto{\pgfqpoint{2.486603in}{4.623868in}}%
\pgfpathlineto{\pgfqpoint{2.486603in}{4.536132in}}%
\pgfpathlineto{\pgfqpoint{2.398868in}{4.536132in}}%
\pgfpathlineto{\pgfqpoint{2.398868in}{4.623868in}}%
\pgfusepath{stroke,fill}%
\end{pgfscope}%
\begin{pgfscope}%
\pgfpathrectangle{\pgfqpoint{0.380943in}{4.185189in}}{\pgfqpoint{4.650000in}{0.614151in}}%
\pgfusepath{clip}%
\pgfsetbuttcap%
\pgfsetroundjoin%
\definecolor{currentfill}{rgb}{0.986759,0.806398,0.641200}%
\pgfsetfillcolor{currentfill}%
\pgfsetlinewidth{0.250937pt}%
\definecolor{currentstroke}{rgb}{1.000000,1.000000,1.000000}%
\pgfsetstrokecolor{currentstroke}%
\pgfsetdash{}{0pt}%
\pgfpathmoveto{\pgfqpoint{2.486603in}{4.623868in}}%
\pgfpathlineto{\pgfqpoint{2.574339in}{4.623868in}}%
\pgfpathlineto{\pgfqpoint{2.574339in}{4.536132in}}%
\pgfpathlineto{\pgfqpoint{2.486603in}{4.536132in}}%
\pgfpathlineto{\pgfqpoint{2.486603in}{4.623868in}}%
\pgfusepath{stroke,fill}%
\end{pgfscope}%
\begin{pgfscope}%
\pgfpathrectangle{\pgfqpoint{0.380943in}{4.185189in}}{\pgfqpoint{4.650000in}{0.614151in}}%
\pgfusepath{clip}%
\pgfsetbuttcap%
\pgfsetroundjoin%
\definecolor{currentfill}{rgb}{0.986759,0.806398,0.641200}%
\pgfsetfillcolor{currentfill}%
\pgfsetlinewidth{0.250937pt}%
\definecolor{currentstroke}{rgb}{1.000000,1.000000,1.000000}%
\pgfsetstrokecolor{currentstroke}%
\pgfsetdash{}{0pt}%
\pgfpathmoveto{\pgfqpoint{2.574339in}{4.623868in}}%
\pgfpathlineto{\pgfqpoint{2.662075in}{4.623868in}}%
\pgfpathlineto{\pgfqpoint{2.662075in}{4.536132in}}%
\pgfpathlineto{\pgfqpoint{2.574339in}{4.536132in}}%
\pgfpathlineto{\pgfqpoint{2.574339in}{4.623868in}}%
\pgfusepath{stroke,fill}%
\end{pgfscope}%
\begin{pgfscope}%
\pgfpathrectangle{\pgfqpoint{0.380943in}{4.185189in}}{\pgfqpoint{4.650000in}{0.614151in}}%
\pgfusepath{clip}%
\pgfsetbuttcap%
\pgfsetroundjoin%
\definecolor{currentfill}{rgb}{0.991849,0.986144,0.810181}%
\pgfsetfillcolor{currentfill}%
\pgfsetlinewidth{0.250937pt}%
\definecolor{currentstroke}{rgb}{1.000000,1.000000,1.000000}%
\pgfsetstrokecolor{currentstroke}%
\pgfsetdash{}{0pt}%
\pgfpathmoveto{\pgfqpoint{2.662075in}{4.623868in}}%
\pgfpathlineto{\pgfqpoint{2.749811in}{4.623868in}}%
\pgfpathlineto{\pgfqpoint{2.749811in}{4.536132in}}%
\pgfpathlineto{\pgfqpoint{2.662075in}{4.536132in}}%
\pgfpathlineto{\pgfqpoint{2.662075in}{4.623868in}}%
\pgfusepath{stroke,fill}%
\end{pgfscope}%
\begin{pgfscope}%
\pgfpathrectangle{\pgfqpoint{0.380943in}{4.185189in}}{\pgfqpoint{4.650000in}{0.614151in}}%
\pgfusepath{clip}%
\pgfsetbuttcap%
\pgfsetroundjoin%
\definecolor{currentfill}{rgb}{0.922338,0.400769,0.400769}%
\pgfsetfillcolor{currentfill}%
\pgfsetlinewidth{0.250937pt}%
\definecolor{currentstroke}{rgb}{1.000000,1.000000,1.000000}%
\pgfsetstrokecolor{currentstroke}%
\pgfsetdash{}{0pt}%
\pgfpathmoveto{\pgfqpoint{2.749811in}{4.623868in}}%
\pgfpathlineto{\pgfqpoint{2.837547in}{4.623868in}}%
\pgfpathlineto{\pgfqpoint{2.837547in}{4.536132in}}%
\pgfpathlineto{\pgfqpoint{2.749811in}{4.536132in}}%
\pgfpathlineto{\pgfqpoint{2.749811in}{4.623868in}}%
\pgfusepath{stroke,fill}%
\end{pgfscope}%
\begin{pgfscope}%
\pgfpathrectangle{\pgfqpoint{0.380943in}{4.185189in}}{\pgfqpoint{4.650000in}{0.614151in}}%
\pgfusepath{clip}%
\pgfsetbuttcap%
\pgfsetroundjoin%
\definecolor{currentfill}{rgb}{0.996571,0.720538,0.589189}%
\pgfsetfillcolor{currentfill}%
\pgfsetlinewidth{0.250937pt}%
\definecolor{currentstroke}{rgb}{1.000000,1.000000,1.000000}%
\pgfsetstrokecolor{currentstroke}%
\pgfsetdash{}{0pt}%
\pgfpathmoveto{\pgfqpoint{2.837547in}{4.623868in}}%
\pgfpathlineto{\pgfqpoint{2.925283in}{4.623868in}}%
\pgfpathlineto{\pgfqpoint{2.925283in}{4.536132in}}%
\pgfpathlineto{\pgfqpoint{2.837547in}{4.536132in}}%
\pgfpathlineto{\pgfqpoint{2.837547in}{4.623868in}}%
\pgfusepath{stroke,fill}%
\end{pgfscope}%
\begin{pgfscope}%
\pgfpathrectangle{\pgfqpoint{0.380943in}{4.185189in}}{\pgfqpoint{4.650000in}{0.614151in}}%
\pgfusepath{clip}%
\pgfsetbuttcap%
\pgfsetroundjoin%
\definecolor{currentfill}{rgb}{0.998939,0.658962,0.556032}%
\pgfsetfillcolor{currentfill}%
\pgfsetlinewidth{0.250937pt}%
\definecolor{currentstroke}{rgb}{1.000000,1.000000,1.000000}%
\pgfsetstrokecolor{currentstroke}%
\pgfsetdash{}{0pt}%
\pgfpathmoveto{\pgfqpoint{2.925283in}{4.623868in}}%
\pgfpathlineto{\pgfqpoint{3.013019in}{4.623868in}}%
\pgfpathlineto{\pgfqpoint{3.013019in}{4.536132in}}%
\pgfpathlineto{\pgfqpoint{2.925283in}{4.536132in}}%
\pgfpathlineto{\pgfqpoint{2.925283in}{4.623868in}}%
\pgfusepath{stroke,fill}%
\end{pgfscope}%
\begin{pgfscope}%
\pgfpathrectangle{\pgfqpoint{0.380943in}{4.185189in}}{\pgfqpoint{4.650000in}{0.614151in}}%
\pgfusepath{clip}%
\pgfsetbuttcap%
\pgfsetroundjoin%
\definecolor{currentfill}{rgb}{0.992326,0.765229,0.614840}%
\pgfsetfillcolor{currentfill}%
\pgfsetlinewidth{0.250937pt}%
\definecolor{currentstroke}{rgb}{1.000000,1.000000,1.000000}%
\pgfsetstrokecolor{currentstroke}%
\pgfsetdash{}{0pt}%
\pgfpathmoveto{\pgfqpoint{3.013019in}{4.623868in}}%
\pgfpathlineto{\pgfqpoint{3.100754in}{4.623868in}}%
\pgfpathlineto{\pgfqpoint{3.100754in}{4.536132in}}%
\pgfpathlineto{\pgfqpoint{3.013019in}{4.536132in}}%
\pgfpathlineto{\pgfqpoint{3.013019in}{4.623868in}}%
\pgfusepath{stroke,fill}%
\end{pgfscope}%
\begin{pgfscope}%
\pgfpathrectangle{\pgfqpoint{0.380943in}{4.185189in}}{\pgfqpoint{4.650000in}{0.614151in}}%
\pgfusepath{clip}%
\pgfsetbuttcap%
\pgfsetroundjoin%
\definecolor{currentfill}{rgb}{0.992326,0.765229,0.614840}%
\pgfsetfillcolor{currentfill}%
\pgfsetlinewidth{0.250937pt}%
\definecolor{currentstroke}{rgb}{1.000000,1.000000,1.000000}%
\pgfsetstrokecolor{currentstroke}%
\pgfsetdash{}{0pt}%
\pgfpathmoveto{\pgfqpoint{3.100754in}{4.623868in}}%
\pgfpathlineto{\pgfqpoint{3.188490in}{4.623868in}}%
\pgfpathlineto{\pgfqpoint{3.188490in}{4.536132in}}%
\pgfpathlineto{\pgfqpoint{3.100754in}{4.536132in}}%
\pgfpathlineto{\pgfqpoint{3.100754in}{4.623868in}}%
\pgfusepath{stroke,fill}%
\end{pgfscope}%
\begin{pgfscope}%
\pgfpathrectangle{\pgfqpoint{0.380943in}{4.185189in}}{\pgfqpoint{4.650000in}{0.614151in}}%
\pgfusepath{clip}%
\pgfsetbuttcap%
\pgfsetroundjoin%
\definecolor{currentfill}{rgb}{0.972549,0.870588,0.692810}%
\pgfsetfillcolor{currentfill}%
\pgfsetlinewidth{0.250937pt}%
\definecolor{currentstroke}{rgb}{1.000000,1.000000,1.000000}%
\pgfsetstrokecolor{currentstroke}%
\pgfsetdash{}{0pt}%
\pgfpathmoveto{\pgfqpoint{3.188490in}{4.623868in}}%
\pgfpathlineto{\pgfqpoint{3.276226in}{4.623868in}}%
\pgfpathlineto{\pgfqpoint{3.276226in}{4.536132in}}%
\pgfpathlineto{\pgfqpoint{3.188490in}{4.536132in}}%
\pgfpathlineto{\pgfqpoint{3.188490in}{4.623868in}}%
\pgfusepath{stroke,fill}%
\end{pgfscope}%
\begin{pgfscope}%
\pgfpathrectangle{\pgfqpoint{0.380943in}{4.185189in}}{\pgfqpoint{4.650000in}{0.614151in}}%
\pgfusepath{clip}%
\pgfsetbuttcap%
\pgfsetroundjoin%
\definecolor{currentfill}{rgb}{0.965444,0.906113,0.711757}%
\pgfsetfillcolor{currentfill}%
\pgfsetlinewidth{0.250937pt}%
\definecolor{currentstroke}{rgb}{1.000000,1.000000,1.000000}%
\pgfsetstrokecolor{currentstroke}%
\pgfsetdash{}{0pt}%
\pgfpathmoveto{\pgfqpoint{3.276226in}{4.623868in}}%
\pgfpathlineto{\pgfqpoint{3.363962in}{4.623868in}}%
\pgfpathlineto{\pgfqpoint{3.363962in}{4.536132in}}%
\pgfpathlineto{\pgfqpoint{3.276226in}{4.536132in}}%
\pgfpathlineto{\pgfqpoint{3.276226in}{4.623868in}}%
\pgfusepath{stroke,fill}%
\end{pgfscope}%
\begin{pgfscope}%
\pgfpathrectangle{\pgfqpoint{0.380943in}{4.185189in}}{\pgfqpoint{4.650000in}{0.614151in}}%
\pgfusepath{clip}%
\pgfsetbuttcap%
\pgfsetroundjoin%
\definecolor{currentfill}{rgb}{0.996571,0.720538,0.589189}%
\pgfsetfillcolor{currentfill}%
\pgfsetlinewidth{0.250937pt}%
\definecolor{currentstroke}{rgb}{1.000000,1.000000,1.000000}%
\pgfsetstrokecolor{currentstroke}%
\pgfsetdash{}{0pt}%
\pgfpathmoveto{\pgfqpoint{3.363962in}{4.623868in}}%
\pgfpathlineto{\pgfqpoint{3.451698in}{4.623868in}}%
\pgfpathlineto{\pgfqpoint{3.451698in}{4.536132in}}%
\pgfpathlineto{\pgfqpoint{3.363962in}{4.536132in}}%
\pgfpathlineto{\pgfqpoint{3.363962in}{4.623868in}}%
\pgfusepath{stroke,fill}%
\end{pgfscope}%
\begin{pgfscope}%
\pgfpathrectangle{\pgfqpoint{0.380943in}{4.185189in}}{\pgfqpoint{4.650000in}{0.614151in}}%
\pgfusepath{clip}%
\pgfsetbuttcap%
\pgfsetroundjoin%
\definecolor{currentfill}{rgb}{0.979654,0.837186,0.669619}%
\pgfsetfillcolor{currentfill}%
\pgfsetlinewidth{0.250937pt}%
\definecolor{currentstroke}{rgb}{1.000000,1.000000,1.000000}%
\pgfsetstrokecolor{currentstroke}%
\pgfsetdash{}{0pt}%
\pgfpathmoveto{\pgfqpoint{3.451698in}{4.623868in}}%
\pgfpathlineto{\pgfqpoint{3.539434in}{4.623868in}}%
\pgfpathlineto{\pgfqpoint{3.539434in}{4.536132in}}%
\pgfpathlineto{\pgfqpoint{3.451698in}{4.536132in}}%
\pgfpathlineto{\pgfqpoint{3.451698in}{4.623868in}}%
\pgfusepath{stroke,fill}%
\end{pgfscope}%
\begin{pgfscope}%
\pgfpathrectangle{\pgfqpoint{0.380943in}{4.185189in}}{\pgfqpoint{4.650000in}{0.614151in}}%
\pgfusepath{clip}%
\pgfsetbuttcap%
\pgfsetroundjoin%
\definecolor{currentfill}{rgb}{0.992326,0.765229,0.614840}%
\pgfsetfillcolor{currentfill}%
\pgfsetlinewidth{0.250937pt}%
\definecolor{currentstroke}{rgb}{1.000000,1.000000,1.000000}%
\pgfsetstrokecolor{currentstroke}%
\pgfsetdash{}{0pt}%
\pgfpathmoveto{\pgfqpoint{3.539434in}{4.623868in}}%
\pgfpathlineto{\pgfqpoint{3.627169in}{4.623868in}}%
\pgfpathlineto{\pgfqpoint{3.627169in}{4.536132in}}%
\pgfpathlineto{\pgfqpoint{3.539434in}{4.536132in}}%
\pgfpathlineto{\pgfqpoint{3.539434in}{4.623868in}}%
\pgfusepath{stroke,fill}%
\end{pgfscope}%
\begin{pgfscope}%
\pgfpathrectangle{\pgfqpoint{0.380943in}{4.185189in}}{\pgfqpoint{4.650000in}{0.614151in}}%
\pgfusepath{clip}%
\pgfsetbuttcap%
\pgfsetroundjoin%
\definecolor{currentfill}{rgb}{0.996571,0.720538,0.589189}%
\pgfsetfillcolor{currentfill}%
\pgfsetlinewidth{0.250937pt}%
\definecolor{currentstroke}{rgb}{1.000000,1.000000,1.000000}%
\pgfsetstrokecolor{currentstroke}%
\pgfsetdash{}{0pt}%
\pgfpathmoveto{\pgfqpoint{3.627169in}{4.623868in}}%
\pgfpathlineto{\pgfqpoint{3.714905in}{4.623868in}}%
\pgfpathlineto{\pgfqpoint{3.714905in}{4.536132in}}%
\pgfpathlineto{\pgfqpoint{3.627169in}{4.536132in}}%
\pgfpathlineto{\pgfqpoint{3.627169in}{4.623868in}}%
\pgfusepath{stroke,fill}%
\end{pgfscope}%
\begin{pgfscope}%
\pgfpathrectangle{\pgfqpoint{0.380943in}{4.185189in}}{\pgfqpoint{4.650000in}{0.614151in}}%
\pgfusepath{clip}%
\pgfsetbuttcap%
\pgfsetroundjoin%
\definecolor{currentfill}{rgb}{0.979654,0.837186,0.669619}%
\pgfsetfillcolor{currentfill}%
\pgfsetlinewidth{0.250937pt}%
\definecolor{currentstroke}{rgb}{1.000000,1.000000,1.000000}%
\pgfsetstrokecolor{currentstroke}%
\pgfsetdash{}{0pt}%
\pgfpathmoveto{\pgfqpoint{3.714905in}{4.623868in}}%
\pgfpathlineto{\pgfqpoint{3.802641in}{4.623868in}}%
\pgfpathlineto{\pgfqpoint{3.802641in}{4.536132in}}%
\pgfpathlineto{\pgfqpoint{3.714905in}{4.536132in}}%
\pgfpathlineto{\pgfqpoint{3.714905in}{4.623868in}}%
\pgfusepath{stroke,fill}%
\end{pgfscope}%
\begin{pgfscope}%
\pgfpathrectangle{\pgfqpoint{0.380943in}{4.185189in}}{\pgfqpoint{4.650000in}{0.614151in}}%
\pgfusepath{clip}%
\pgfsetbuttcap%
\pgfsetroundjoin%
\definecolor{currentfill}{rgb}{0.998939,0.658962,0.556032}%
\pgfsetfillcolor{currentfill}%
\pgfsetlinewidth{0.250937pt}%
\definecolor{currentstroke}{rgb}{1.000000,1.000000,1.000000}%
\pgfsetstrokecolor{currentstroke}%
\pgfsetdash{}{0pt}%
\pgfpathmoveto{\pgfqpoint{3.802641in}{4.623868in}}%
\pgfpathlineto{\pgfqpoint{3.890377in}{4.623868in}}%
\pgfpathlineto{\pgfqpoint{3.890377in}{4.536132in}}%
\pgfpathlineto{\pgfqpoint{3.802641in}{4.536132in}}%
\pgfpathlineto{\pgfqpoint{3.802641in}{4.623868in}}%
\pgfusepath{stroke,fill}%
\end{pgfscope}%
\begin{pgfscope}%
\pgfpathrectangle{\pgfqpoint{0.380943in}{4.185189in}}{\pgfqpoint{4.650000in}{0.614151in}}%
\pgfusepath{clip}%
\pgfsetbuttcap%
\pgfsetroundjoin%
\definecolor{currentfill}{rgb}{1.000000,0.509404,0.491473}%
\pgfsetfillcolor{currentfill}%
\pgfsetlinewidth{0.250937pt}%
\definecolor{currentstroke}{rgb}{1.000000,1.000000,1.000000}%
\pgfsetstrokecolor{currentstroke}%
\pgfsetdash{}{0pt}%
\pgfpathmoveto{\pgfqpoint{3.890377in}{4.623868in}}%
\pgfpathlineto{\pgfqpoint{3.978113in}{4.623868in}}%
\pgfpathlineto{\pgfqpoint{3.978113in}{4.536132in}}%
\pgfpathlineto{\pgfqpoint{3.890377in}{4.536132in}}%
\pgfpathlineto{\pgfqpoint{3.890377in}{4.623868in}}%
\pgfusepath{stroke,fill}%
\end{pgfscope}%
\begin{pgfscope}%
\pgfpathrectangle{\pgfqpoint{0.380943in}{4.185189in}}{\pgfqpoint{4.650000in}{0.614151in}}%
\pgfusepath{clip}%
\pgfsetbuttcap%
\pgfsetroundjoin%
\definecolor{currentfill}{rgb}{0.922338,0.400769,0.400769}%
\pgfsetfillcolor{currentfill}%
\pgfsetlinewidth{0.250937pt}%
\definecolor{currentstroke}{rgb}{1.000000,1.000000,1.000000}%
\pgfsetstrokecolor{currentstroke}%
\pgfsetdash{}{0pt}%
\pgfpathmoveto{\pgfqpoint{3.978113in}{4.623868in}}%
\pgfpathlineto{\pgfqpoint{4.065849in}{4.623868in}}%
\pgfpathlineto{\pgfqpoint{4.065849in}{4.536132in}}%
\pgfpathlineto{\pgfqpoint{3.978113in}{4.536132in}}%
\pgfpathlineto{\pgfqpoint{3.978113in}{4.623868in}}%
\pgfusepath{stroke,fill}%
\end{pgfscope}%
\begin{pgfscope}%
\pgfpathrectangle{\pgfqpoint{0.380943in}{4.185189in}}{\pgfqpoint{4.650000in}{0.614151in}}%
\pgfusepath{clip}%
\pgfsetbuttcap%
\pgfsetroundjoin%
\definecolor{currentfill}{rgb}{0.992326,0.765229,0.614840}%
\pgfsetfillcolor{currentfill}%
\pgfsetlinewidth{0.250937pt}%
\definecolor{currentstroke}{rgb}{1.000000,1.000000,1.000000}%
\pgfsetstrokecolor{currentstroke}%
\pgfsetdash{}{0pt}%
\pgfpathmoveto{\pgfqpoint{4.065849in}{4.623868in}}%
\pgfpathlineto{\pgfqpoint{4.153585in}{4.623868in}}%
\pgfpathlineto{\pgfqpoint{4.153585in}{4.536132in}}%
\pgfpathlineto{\pgfqpoint{4.065849in}{4.536132in}}%
\pgfpathlineto{\pgfqpoint{4.065849in}{4.623868in}}%
\pgfusepath{stroke,fill}%
\end{pgfscope}%
\begin{pgfscope}%
\pgfpathrectangle{\pgfqpoint{0.380943in}{4.185189in}}{\pgfqpoint{4.650000in}{0.614151in}}%
\pgfusepath{clip}%
\pgfsetbuttcap%
\pgfsetroundjoin%
\definecolor{currentfill}{rgb}{0.998939,0.658962,0.556032}%
\pgfsetfillcolor{currentfill}%
\pgfsetlinewidth{0.250937pt}%
\definecolor{currentstroke}{rgb}{1.000000,1.000000,1.000000}%
\pgfsetstrokecolor{currentstroke}%
\pgfsetdash{}{0pt}%
\pgfpathmoveto{\pgfqpoint{4.153585in}{4.623868in}}%
\pgfpathlineto{\pgfqpoint{4.241320in}{4.623868in}}%
\pgfpathlineto{\pgfqpoint{4.241320in}{4.536132in}}%
\pgfpathlineto{\pgfqpoint{4.153585in}{4.536132in}}%
\pgfpathlineto{\pgfqpoint{4.153585in}{4.623868in}}%
\pgfusepath{stroke,fill}%
\end{pgfscope}%
\begin{pgfscope}%
\pgfpathrectangle{\pgfqpoint{0.380943in}{4.185189in}}{\pgfqpoint{4.650000in}{0.614151in}}%
\pgfusepath{clip}%
\pgfsetbuttcap%
\pgfsetroundjoin%
\definecolor{currentfill}{rgb}{0.986759,0.806398,0.641200}%
\pgfsetfillcolor{currentfill}%
\pgfsetlinewidth{0.250937pt}%
\definecolor{currentstroke}{rgb}{1.000000,1.000000,1.000000}%
\pgfsetstrokecolor{currentstroke}%
\pgfsetdash{}{0pt}%
\pgfpathmoveto{\pgfqpoint{4.241320in}{4.623868in}}%
\pgfpathlineto{\pgfqpoint{4.329056in}{4.623868in}}%
\pgfpathlineto{\pgfqpoint{4.329056in}{4.536132in}}%
\pgfpathlineto{\pgfqpoint{4.241320in}{4.536132in}}%
\pgfpathlineto{\pgfqpoint{4.241320in}{4.623868in}}%
\pgfusepath{stroke,fill}%
\end{pgfscope}%
\begin{pgfscope}%
\pgfpathrectangle{\pgfqpoint{0.380943in}{4.185189in}}{\pgfqpoint{4.650000in}{0.614151in}}%
\pgfusepath{clip}%
\pgfsetbuttcap%
\pgfsetroundjoin%
\definecolor{currentfill}{rgb}{0.965444,0.906113,0.711757}%
\pgfsetfillcolor{currentfill}%
\pgfsetlinewidth{0.250937pt}%
\definecolor{currentstroke}{rgb}{1.000000,1.000000,1.000000}%
\pgfsetstrokecolor{currentstroke}%
\pgfsetdash{}{0pt}%
\pgfpathmoveto{\pgfqpoint{4.329056in}{4.623868in}}%
\pgfpathlineto{\pgfqpoint{4.416792in}{4.623868in}}%
\pgfpathlineto{\pgfqpoint{4.416792in}{4.536132in}}%
\pgfpathlineto{\pgfqpoint{4.329056in}{4.536132in}}%
\pgfpathlineto{\pgfqpoint{4.329056in}{4.623868in}}%
\pgfusepath{stroke,fill}%
\end{pgfscope}%
\begin{pgfscope}%
\pgfpathrectangle{\pgfqpoint{0.380943in}{4.185189in}}{\pgfqpoint{4.650000in}{0.614151in}}%
\pgfusepath{clip}%
\pgfsetbuttcap%
\pgfsetroundjoin%
\definecolor{currentfill}{rgb}{1.000000,0.605229,0.530719}%
\pgfsetfillcolor{currentfill}%
\pgfsetlinewidth{0.250937pt}%
\definecolor{currentstroke}{rgb}{1.000000,1.000000,1.000000}%
\pgfsetstrokecolor{currentstroke}%
\pgfsetdash{}{0pt}%
\pgfpathmoveto{\pgfqpoint{4.416792in}{4.623868in}}%
\pgfpathlineto{\pgfqpoint{4.504528in}{4.623868in}}%
\pgfpathlineto{\pgfqpoint{4.504528in}{4.536132in}}%
\pgfpathlineto{\pgfqpoint{4.416792in}{4.536132in}}%
\pgfpathlineto{\pgfqpoint{4.416792in}{4.623868in}}%
\pgfusepath{stroke,fill}%
\end{pgfscope}%
\begin{pgfscope}%
\pgfpathrectangle{\pgfqpoint{0.380943in}{4.185189in}}{\pgfqpoint{4.650000in}{0.614151in}}%
\pgfusepath{clip}%
\pgfsetbuttcap%
\pgfsetroundjoin%
\definecolor{currentfill}{rgb}{0.992326,0.765229,0.614840}%
\pgfsetfillcolor{currentfill}%
\pgfsetlinewidth{0.250937pt}%
\definecolor{currentstroke}{rgb}{1.000000,1.000000,1.000000}%
\pgfsetstrokecolor{currentstroke}%
\pgfsetdash{}{0pt}%
\pgfpathmoveto{\pgfqpoint{4.504528in}{4.623868in}}%
\pgfpathlineto{\pgfqpoint{4.592264in}{4.623868in}}%
\pgfpathlineto{\pgfqpoint{4.592264in}{4.536132in}}%
\pgfpathlineto{\pgfqpoint{4.504528in}{4.536132in}}%
\pgfpathlineto{\pgfqpoint{4.504528in}{4.623868in}}%
\pgfusepath{stroke,fill}%
\end{pgfscope}%
\begin{pgfscope}%
\pgfpathrectangle{\pgfqpoint{0.380943in}{4.185189in}}{\pgfqpoint{4.650000in}{0.614151in}}%
\pgfusepath{clip}%
\pgfsetbuttcap%
\pgfsetroundjoin%
\definecolor{currentfill}{rgb}{0.979654,0.837186,0.669619}%
\pgfsetfillcolor{currentfill}%
\pgfsetlinewidth{0.250937pt}%
\definecolor{currentstroke}{rgb}{1.000000,1.000000,1.000000}%
\pgfsetstrokecolor{currentstroke}%
\pgfsetdash{}{0pt}%
\pgfpathmoveto{\pgfqpoint{4.592264in}{4.623868in}}%
\pgfpathlineto{\pgfqpoint{4.680000in}{4.623868in}}%
\pgfpathlineto{\pgfqpoint{4.680000in}{4.536132in}}%
\pgfpathlineto{\pgfqpoint{4.592264in}{4.536132in}}%
\pgfpathlineto{\pgfqpoint{4.592264in}{4.623868in}}%
\pgfusepath{stroke,fill}%
\end{pgfscope}%
\begin{pgfscope}%
\pgfpathrectangle{\pgfqpoint{0.380943in}{4.185189in}}{\pgfqpoint{4.650000in}{0.614151in}}%
\pgfusepath{clip}%
\pgfsetbuttcap%
\pgfsetroundjoin%
\definecolor{currentfill}{rgb}{0.962414,0.923552,0.722891}%
\pgfsetfillcolor{currentfill}%
\pgfsetlinewidth{0.250937pt}%
\definecolor{currentstroke}{rgb}{1.000000,1.000000,1.000000}%
\pgfsetstrokecolor{currentstroke}%
\pgfsetdash{}{0pt}%
\pgfpathmoveto{\pgfqpoint{4.680000in}{4.623868in}}%
\pgfpathlineto{\pgfqpoint{4.767736in}{4.623868in}}%
\pgfpathlineto{\pgfqpoint{4.767736in}{4.536132in}}%
\pgfpathlineto{\pgfqpoint{4.680000in}{4.536132in}}%
\pgfpathlineto{\pgfqpoint{4.680000in}{4.623868in}}%
\pgfusepath{stroke,fill}%
\end{pgfscope}%
\begin{pgfscope}%
\pgfpathrectangle{\pgfqpoint{0.380943in}{4.185189in}}{\pgfqpoint{4.650000in}{0.614151in}}%
\pgfusepath{clip}%
\pgfsetbuttcap%
\pgfsetroundjoin%
\definecolor{currentfill}{rgb}{0.979654,0.837186,0.669619}%
\pgfsetfillcolor{currentfill}%
\pgfsetlinewidth{0.250937pt}%
\definecolor{currentstroke}{rgb}{1.000000,1.000000,1.000000}%
\pgfsetstrokecolor{currentstroke}%
\pgfsetdash{}{0pt}%
\pgfpathmoveto{\pgfqpoint{4.767736in}{4.623868in}}%
\pgfpathlineto{\pgfqpoint{4.855471in}{4.623868in}}%
\pgfpathlineto{\pgfqpoint{4.855471in}{4.536132in}}%
\pgfpathlineto{\pgfqpoint{4.767736in}{4.536132in}}%
\pgfpathlineto{\pgfqpoint{4.767736in}{4.623868in}}%
\pgfusepath{stroke,fill}%
\end{pgfscope}%
\begin{pgfscope}%
\pgfpathrectangle{\pgfqpoint{0.380943in}{4.185189in}}{\pgfqpoint{4.650000in}{0.614151in}}%
\pgfusepath{clip}%
\pgfsetbuttcap%
\pgfsetroundjoin%
\definecolor{currentfill}{rgb}{1.000000,1.000000,0.929412}%
\pgfsetfillcolor{currentfill}%
\pgfsetlinewidth{0.250937pt}%
\definecolor{currentstroke}{rgb}{1.000000,1.000000,1.000000}%
\pgfsetstrokecolor{currentstroke}%
\pgfsetdash{}{0pt}%
\pgfpathmoveto{\pgfqpoint{4.855471in}{4.623868in}}%
\pgfpathlineto{\pgfqpoint{4.943207in}{4.623868in}}%
\pgfpathlineto{\pgfqpoint{4.943207in}{4.536132in}}%
\pgfpathlineto{\pgfqpoint{4.855471in}{4.536132in}}%
\pgfpathlineto{\pgfqpoint{4.855471in}{4.623868in}}%
\pgfusepath{stroke,fill}%
\end{pgfscope}%
\begin{pgfscope}%
\pgfpathrectangle{\pgfqpoint{0.380943in}{4.185189in}}{\pgfqpoint{4.650000in}{0.614151in}}%
\pgfusepath{clip}%
\pgfsetbuttcap%
\pgfsetroundjoin%
\pgfsetlinewidth{0.250937pt}%
\definecolor{currentstroke}{rgb}{1.000000,1.000000,1.000000}%
\pgfsetstrokecolor{currentstroke}%
\pgfsetdash{}{0pt}%
\pgfpathmoveto{\pgfqpoint{4.943207in}{4.623868in}}%
\pgfpathlineto{\pgfqpoint{5.030943in}{4.623868in}}%
\pgfpathlineto{\pgfqpoint{5.030943in}{4.536132in}}%
\pgfpathlineto{\pgfqpoint{4.943207in}{4.536132in}}%
\pgfpathlineto{\pgfqpoint{4.943207in}{4.623868in}}%
\pgfusepath{stroke}%
\end{pgfscope}%
\begin{pgfscope}%
\pgfpathrectangle{\pgfqpoint{0.380943in}{4.185189in}}{\pgfqpoint{4.650000in}{0.614151in}}%
\pgfusepath{clip}%
\pgfsetbuttcap%
\pgfsetroundjoin%
\definecolor{currentfill}{rgb}{0.996571,0.720538,0.589189}%
\pgfsetfillcolor{currentfill}%
\pgfsetlinewidth{0.250937pt}%
\definecolor{currentstroke}{rgb}{1.000000,1.000000,1.000000}%
\pgfsetstrokecolor{currentstroke}%
\pgfsetdash{}{0pt}%
\pgfpathmoveto{\pgfqpoint{0.380943in}{4.536132in}}%
\pgfpathlineto{\pgfqpoint{0.468679in}{4.536132in}}%
\pgfpathlineto{\pgfqpoint{0.468679in}{4.448396in}}%
\pgfpathlineto{\pgfqpoint{0.380943in}{4.448396in}}%
\pgfpathlineto{\pgfqpoint{0.380943in}{4.536132in}}%
\pgfusepath{stroke,fill}%
\end{pgfscope}%
\begin{pgfscope}%
\pgfpathrectangle{\pgfqpoint{0.380943in}{4.185189in}}{\pgfqpoint{4.650000in}{0.614151in}}%
\pgfusepath{clip}%
\pgfsetbuttcap%
\pgfsetroundjoin%
\definecolor{currentfill}{rgb}{0.972549,0.870588,0.692810}%
\pgfsetfillcolor{currentfill}%
\pgfsetlinewidth{0.250937pt}%
\definecolor{currentstroke}{rgb}{1.000000,1.000000,1.000000}%
\pgfsetstrokecolor{currentstroke}%
\pgfsetdash{}{0pt}%
\pgfpathmoveto{\pgfqpoint{0.468679in}{4.536132in}}%
\pgfpathlineto{\pgfqpoint{0.556415in}{4.536132in}}%
\pgfpathlineto{\pgfqpoint{0.556415in}{4.448396in}}%
\pgfpathlineto{\pgfqpoint{0.468679in}{4.448396in}}%
\pgfpathlineto{\pgfqpoint{0.468679in}{4.536132in}}%
\pgfusepath{stroke,fill}%
\end{pgfscope}%
\begin{pgfscope}%
\pgfpathrectangle{\pgfqpoint{0.380943in}{4.185189in}}{\pgfqpoint{4.650000in}{0.614151in}}%
\pgfusepath{clip}%
\pgfsetbuttcap%
\pgfsetroundjoin%
\definecolor{currentfill}{rgb}{0.996571,0.720538,0.589189}%
\pgfsetfillcolor{currentfill}%
\pgfsetlinewidth{0.250937pt}%
\definecolor{currentstroke}{rgb}{1.000000,1.000000,1.000000}%
\pgfsetstrokecolor{currentstroke}%
\pgfsetdash{}{0pt}%
\pgfpathmoveto{\pgfqpoint{0.556415in}{4.536132in}}%
\pgfpathlineto{\pgfqpoint{0.644151in}{4.536132in}}%
\pgfpathlineto{\pgfqpoint{0.644151in}{4.448396in}}%
\pgfpathlineto{\pgfqpoint{0.556415in}{4.448396in}}%
\pgfpathlineto{\pgfqpoint{0.556415in}{4.536132in}}%
\pgfusepath{stroke,fill}%
\end{pgfscope}%
\begin{pgfscope}%
\pgfpathrectangle{\pgfqpoint{0.380943in}{4.185189in}}{\pgfqpoint{4.650000in}{0.614151in}}%
\pgfusepath{clip}%
\pgfsetbuttcap%
\pgfsetroundjoin%
\definecolor{currentfill}{rgb}{0.992326,0.765229,0.614840}%
\pgfsetfillcolor{currentfill}%
\pgfsetlinewidth{0.250937pt}%
\definecolor{currentstroke}{rgb}{1.000000,1.000000,1.000000}%
\pgfsetstrokecolor{currentstroke}%
\pgfsetdash{}{0pt}%
\pgfpathmoveto{\pgfqpoint{0.644151in}{4.536132in}}%
\pgfpathlineto{\pgfqpoint{0.731886in}{4.536132in}}%
\pgfpathlineto{\pgfqpoint{0.731886in}{4.448396in}}%
\pgfpathlineto{\pgfqpoint{0.644151in}{4.448396in}}%
\pgfpathlineto{\pgfqpoint{0.644151in}{4.536132in}}%
\pgfusepath{stroke,fill}%
\end{pgfscope}%
\begin{pgfscope}%
\pgfpathrectangle{\pgfqpoint{0.380943in}{4.185189in}}{\pgfqpoint{4.650000in}{0.614151in}}%
\pgfusepath{clip}%
\pgfsetbuttcap%
\pgfsetroundjoin%
\definecolor{currentfill}{rgb}{0.986759,0.806398,0.641200}%
\pgfsetfillcolor{currentfill}%
\pgfsetlinewidth{0.250937pt}%
\definecolor{currentstroke}{rgb}{1.000000,1.000000,1.000000}%
\pgfsetstrokecolor{currentstroke}%
\pgfsetdash{}{0pt}%
\pgfpathmoveto{\pgfqpoint{0.731886in}{4.536132in}}%
\pgfpathlineto{\pgfqpoint{0.819622in}{4.536132in}}%
\pgfpathlineto{\pgfqpoint{0.819622in}{4.448396in}}%
\pgfpathlineto{\pgfqpoint{0.731886in}{4.448396in}}%
\pgfpathlineto{\pgfqpoint{0.731886in}{4.536132in}}%
\pgfusepath{stroke,fill}%
\end{pgfscope}%
\begin{pgfscope}%
\pgfpathrectangle{\pgfqpoint{0.380943in}{4.185189in}}{\pgfqpoint{4.650000in}{0.614151in}}%
\pgfusepath{clip}%
\pgfsetbuttcap%
\pgfsetroundjoin%
\definecolor{currentfill}{rgb}{1.000000,0.557862,0.511772}%
\pgfsetfillcolor{currentfill}%
\pgfsetlinewidth{0.250937pt}%
\definecolor{currentstroke}{rgb}{1.000000,1.000000,1.000000}%
\pgfsetstrokecolor{currentstroke}%
\pgfsetdash{}{0pt}%
\pgfpathmoveto{\pgfqpoint{0.819622in}{4.536132in}}%
\pgfpathlineto{\pgfqpoint{0.907358in}{4.536132in}}%
\pgfpathlineto{\pgfqpoint{0.907358in}{4.448396in}}%
\pgfpathlineto{\pgfqpoint{0.819622in}{4.448396in}}%
\pgfpathlineto{\pgfqpoint{0.819622in}{4.536132in}}%
\pgfusepath{stroke,fill}%
\end{pgfscope}%
\begin{pgfscope}%
\pgfpathrectangle{\pgfqpoint{0.380943in}{4.185189in}}{\pgfqpoint{4.650000in}{0.614151in}}%
\pgfusepath{clip}%
\pgfsetbuttcap%
\pgfsetroundjoin%
\definecolor{currentfill}{rgb}{0.986759,0.806398,0.641200}%
\pgfsetfillcolor{currentfill}%
\pgfsetlinewidth{0.250937pt}%
\definecolor{currentstroke}{rgb}{1.000000,1.000000,1.000000}%
\pgfsetstrokecolor{currentstroke}%
\pgfsetdash{}{0pt}%
\pgfpathmoveto{\pgfqpoint{0.907358in}{4.536132in}}%
\pgfpathlineto{\pgfqpoint{0.995094in}{4.536132in}}%
\pgfpathlineto{\pgfqpoint{0.995094in}{4.448396in}}%
\pgfpathlineto{\pgfqpoint{0.907358in}{4.448396in}}%
\pgfpathlineto{\pgfqpoint{0.907358in}{4.536132in}}%
\pgfusepath{stroke,fill}%
\end{pgfscope}%
\begin{pgfscope}%
\pgfpathrectangle{\pgfqpoint{0.380943in}{4.185189in}}{\pgfqpoint{4.650000in}{0.614151in}}%
\pgfusepath{clip}%
\pgfsetbuttcap%
\pgfsetroundjoin%
\definecolor{currentfill}{rgb}{0.996571,0.720538,0.589189}%
\pgfsetfillcolor{currentfill}%
\pgfsetlinewidth{0.250937pt}%
\definecolor{currentstroke}{rgb}{1.000000,1.000000,1.000000}%
\pgfsetstrokecolor{currentstroke}%
\pgfsetdash{}{0pt}%
\pgfpathmoveto{\pgfqpoint{0.995094in}{4.536132in}}%
\pgfpathlineto{\pgfqpoint{1.082830in}{4.536132in}}%
\pgfpathlineto{\pgfqpoint{1.082830in}{4.448396in}}%
\pgfpathlineto{\pgfqpoint{0.995094in}{4.448396in}}%
\pgfpathlineto{\pgfqpoint{0.995094in}{4.536132in}}%
\pgfusepath{stroke,fill}%
\end{pgfscope}%
\begin{pgfscope}%
\pgfpathrectangle{\pgfqpoint{0.380943in}{4.185189in}}{\pgfqpoint{4.650000in}{0.614151in}}%
\pgfusepath{clip}%
\pgfsetbuttcap%
\pgfsetroundjoin%
\definecolor{currentfill}{rgb}{0.979654,0.837186,0.669619}%
\pgfsetfillcolor{currentfill}%
\pgfsetlinewidth{0.250937pt}%
\definecolor{currentstroke}{rgb}{1.000000,1.000000,1.000000}%
\pgfsetstrokecolor{currentstroke}%
\pgfsetdash{}{0pt}%
\pgfpathmoveto{\pgfqpoint{1.082830in}{4.536132in}}%
\pgfpathlineto{\pgfqpoint{1.170566in}{4.536132in}}%
\pgfpathlineto{\pgfqpoint{1.170566in}{4.448396in}}%
\pgfpathlineto{\pgfqpoint{1.082830in}{4.448396in}}%
\pgfpathlineto{\pgfqpoint{1.082830in}{4.536132in}}%
\pgfusepath{stroke,fill}%
\end{pgfscope}%
\begin{pgfscope}%
\pgfpathrectangle{\pgfqpoint{0.380943in}{4.185189in}}{\pgfqpoint{4.650000in}{0.614151in}}%
\pgfusepath{clip}%
\pgfsetbuttcap%
\pgfsetroundjoin%
\definecolor{currentfill}{rgb}{0.979654,0.837186,0.669619}%
\pgfsetfillcolor{currentfill}%
\pgfsetlinewidth{0.250937pt}%
\definecolor{currentstroke}{rgb}{1.000000,1.000000,1.000000}%
\pgfsetstrokecolor{currentstroke}%
\pgfsetdash{}{0pt}%
\pgfpathmoveto{\pgfqpoint{1.170566in}{4.536132in}}%
\pgfpathlineto{\pgfqpoint{1.258302in}{4.536132in}}%
\pgfpathlineto{\pgfqpoint{1.258302in}{4.448396in}}%
\pgfpathlineto{\pgfqpoint{1.170566in}{4.448396in}}%
\pgfpathlineto{\pgfqpoint{1.170566in}{4.536132in}}%
\pgfusepath{stroke,fill}%
\end{pgfscope}%
\begin{pgfscope}%
\pgfpathrectangle{\pgfqpoint{0.380943in}{4.185189in}}{\pgfqpoint{4.650000in}{0.614151in}}%
\pgfusepath{clip}%
\pgfsetbuttcap%
\pgfsetroundjoin%
\definecolor{currentfill}{rgb}{1.000000,0.509404,0.491473}%
\pgfsetfillcolor{currentfill}%
\pgfsetlinewidth{0.250937pt}%
\definecolor{currentstroke}{rgb}{1.000000,1.000000,1.000000}%
\pgfsetstrokecolor{currentstroke}%
\pgfsetdash{}{0pt}%
\pgfpathmoveto{\pgfqpoint{1.258302in}{4.536132in}}%
\pgfpathlineto{\pgfqpoint{1.346037in}{4.536132in}}%
\pgfpathlineto{\pgfqpoint{1.346037in}{4.448396in}}%
\pgfpathlineto{\pgfqpoint{1.258302in}{4.448396in}}%
\pgfpathlineto{\pgfqpoint{1.258302in}{4.536132in}}%
\pgfusepath{stroke,fill}%
\end{pgfscope}%
\begin{pgfscope}%
\pgfpathrectangle{\pgfqpoint{0.380943in}{4.185189in}}{\pgfqpoint{4.650000in}{0.614151in}}%
\pgfusepath{clip}%
\pgfsetbuttcap%
\pgfsetroundjoin%
\definecolor{currentfill}{rgb}{0.986759,0.806398,0.641200}%
\pgfsetfillcolor{currentfill}%
\pgfsetlinewidth{0.250937pt}%
\definecolor{currentstroke}{rgb}{1.000000,1.000000,1.000000}%
\pgfsetstrokecolor{currentstroke}%
\pgfsetdash{}{0pt}%
\pgfpathmoveto{\pgfqpoint{1.346037in}{4.536132in}}%
\pgfpathlineto{\pgfqpoint{1.433773in}{4.536132in}}%
\pgfpathlineto{\pgfqpoint{1.433773in}{4.448396in}}%
\pgfpathlineto{\pgfqpoint{1.346037in}{4.448396in}}%
\pgfpathlineto{\pgfqpoint{1.346037in}{4.536132in}}%
\pgfusepath{stroke,fill}%
\end{pgfscope}%
\begin{pgfscope}%
\pgfpathrectangle{\pgfqpoint{0.380943in}{4.185189in}}{\pgfqpoint{4.650000in}{0.614151in}}%
\pgfusepath{clip}%
\pgfsetbuttcap%
\pgfsetroundjoin%
\definecolor{currentfill}{rgb}{0.972549,0.870588,0.692810}%
\pgfsetfillcolor{currentfill}%
\pgfsetlinewidth{0.250937pt}%
\definecolor{currentstroke}{rgb}{1.000000,1.000000,1.000000}%
\pgfsetstrokecolor{currentstroke}%
\pgfsetdash{}{0pt}%
\pgfpathmoveto{\pgfqpoint{1.433773in}{4.536132in}}%
\pgfpathlineto{\pgfqpoint{1.521509in}{4.536132in}}%
\pgfpathlineto{\pgfqpoint{1.521509in}{4.448396in}}%
\pgfpathlineto{\pgfqpoint{1.433773in}{4.448396in}}%
\pgfpathlineto{\pgfqpoint{1.433773in}{4.536132in}}%
\pgfusepath{stroke,fill}%
\end{pgfscope}%
\begin{pgfscope}%
\pgfpathrectangle{\pgfqpoint{0.380943in}{4.185189in}}{\pgfqpoint{4.650000in}{0.614151in}}%
\pgfusepath{clip}%
\pgfsetbuttcap%
\pgfsetroundjoin%
\definecolor{currentfill}{rgb}{0.965444,0.906113,0.711757}%
\pgfsetfillcolor{currentfill}%
\pgfsetlinewidth{0.250937pt}%
\definecolor{currentstroke}{rgb}{1.000000,1.000000,1.000000}%
\pgfsetstrokecolor{currentstroke}%
\pgfsetdash{}{0pt}%
\pgfpathmoveto{\pgfqpoint{1.521509in}{4.536132in}}%
\pgfpathlineto{\pgfqpoint{1.609245in}{4.536132in}}%
\pgfpathlineto{\pgfqpoint{1.609245in}{4.448396in}}%
\pgfpathlineto{\pgfqpoint{1.521509in}{4.448396in}}%
\pgfpathlineto{\pgfqpoint{1.521509in}{4.536132in}}%
\pgfusepath{stroke,fill}%
\end{pgfscope}%
\begin{pgfscope}%
\pgfpathrectangle{\pgfqpoint{0.380943in}{4.185189in}}{\pgfqpoint{4.650000in}{0.614151in}}%
\pgfusepath{clip}%
\pgfsetbuttcap%
\pgfsetroundjoin%
\definecolor{currentfill}{rgb}{0.992326,0.765229,0.614840}%
\pgfsetfillcolor{currentfill}%
\pgfsetlinewidth{0.250937pt}%
\definecolor{currentstroke}{rgb}{1.000000,1.000000,1.000000}%
\pgfsetstrokecolor{currentstroke}%
\pgfsetdash{}{0pt}%
\pgfpathmoveto{\pgfqpoint{1.609245in}{4.536132in}}%
\pgfpathlineto{\pgfqpoint{1.696981in}{4.536132in}}%
\pgfpathlineto{\pgfqpoint{1.696981in}{4.448396in}}%
\pgfpathlineto{\pgfqpoint{1.609245in}{4.448396in}}%
\pgfpathlineto{\pgfqpoint{1.609245in}{4.536132in}}%
\pgfusepath{stroke,fill}%
\end{pgfscope}%
\begin{pgfscope}%
\pgfpathrectangle{\pgfqpoint{0.380943in}{4.185189in}}{\pgfqpoint{4.650000in}{0.614151in}}%
\pgfusepath{clip}%
\pgfsetbuttcap%
\pgfsetroundjoin%
\definecolor{currentfill}{rgb}{0.986759,0.806398,0.641200}%
\pgfsetfillcolor{currentfill}%
\pgfsetlinewidth{0.250937pt}%
\definecolor{currentstroke}{rgb}{1.000000,1.000000,1.000000}%
\pgfsetstrokecolor{currentstroke}%
\pgfsetdash{}{0pt}%
\pgfpathmoveto{\pgfqpoint{1.696981in}{4.536132in}}%
\pgfpathlineto{\pgfqpoint{1.784717in}{4.536132in}}%
\pgfpathlineto{\pgfqpoint{1.784717in}{4.448396in}}%
\pgfpathlineto{\pgfqpoint{1.696981in}{4.448396in}}%
\pgfpathlineto{\pgfqpoint{1.696981in}{4.536132in}}%
\pgfusepath{stroke,fill}%
\end{pgfscope}%
\begin{pgfscope}%
\pgfpathrectangle{\pgfqpoint{0.380943in}{4.185189in}}{\pgfqpoint{4.650000in}{0.614151in}}%
\pgfusepath{clip}%
\pgfsetbuttcap%
\pgfsetroundjoin%
\definecolor{currentfill}{rgb}{1.000000,0.605229,0.530719}%
\pgfsetfillcolor{currentfill}%
\pgfsetlinewidth{0.250937pt}%
\definecolor{currentstroke}{rgb}{1.000000,1.000000,1.000000}%
\pgfsetstrokecolor{currentstroke}%
\pgfsetdash{}{0pt}%
\pgfpathmoveto{\pgfqpoint{1.784717in}{4.536132in}}%
\pgfpathlineto{\pgfqpoint{1.872452in}{4.536132in}}%
\pgfpathlineto{\pgfqpoint{1.872452in}{4.448396in}}%
\pgfpathlineto{\pgfqpoint{1.784717in}{4.448396in}}%
\pgfpathlineto{\pgfqpoint{1.784717in}{4.536132in}}%
\pgfusepath{stroke,fill}%
\end{pgfscope}%
\begin{pgfscope}%
\pgfpathrectangle{\pgfqpoint{0.380943in}{4.185189in}}{\pgfqpoint{4.650000in}{0.614151in}}%
\pgfusepath{clip}%
\pgfsetbuttcap%
\pgfsetroundjoin%
\definecolor{currentfill}{rgb}{0.972549,0.870588,0.692810}%
\pgfsetfillcolor{currentfill}%
\pgfsetlinewidth{0.250937pt}%
\definecolor{currentstroke}{rgb}{1.000000,1.000000,1.000000}%
\pgfsetstrokecolor{currentstroke}%
\pgfsetdash{}{0pt}%
\pgfpathmoveto{\pgfqpoint{1.872452in}{4.536132in}}%
\pgfpathlineto{\pgfqpoint{1.960188in}{4.536132in}}%
\pgfpathlineto{\pgfqpoint{1.960188in}{4.448396in}}%
\pgfpathlineto{\pgfqpoint{1.872452in}{4.448396in}}%
\pgfpathlineto{\pgfqpoint{1.872452in}{4.536132in}}%
\pgfusepath{stroke,fill}%
\end{pgfscope}%
\begin{pgfscope}%
\pgfpathrectangle{\pgfqpoint{0.380943in}{4.185189in}}{\pgfqpoint{4.650000in}{0.614151in}}%
\pgfusepath{clip}%
\pgfsetbuttcap%
\pgfsetroundjoin%
\definecolor{currentfill}{rgb}{0.996571,0.720538,0.589189}%
\pgfsetfillcolor{currentfill}%
\pgfsetlinewidth{0.250937pt}%
\definecolor{currentstroke}{rgb}{1.000000,1.000000,1.000000}%
\pgfsetstrokecolor{currentstroke}%
\pgfsetdash{}{0pt}%
\pgfpathmoveto{\pgfqpoint{1.960188in}{4.536132in}}%
\pgfpathlineto{\pgfqpoint{2.047924in}{4.536132in}}%
\pgfpathlineto{\pgfqpoint{2.047924in}{4.448396in}}%
\pgfpathlineto{\pgfqpoint{1.960188in}{4.448396in}}%
\pgfpathlineto{\pgfqpoint{1.960188in}{4.536132in}}%
\pgfusepath{stroke,fill}%
\end{pgfscope}%
\begin{pgfscope}%
\pgfpathrectangle{\pgfqpoint{0.380943in}{4.185189in}}{\pgfqpoint{4.650000in}{0.614151in}}%
\pgfusepath{clip}%
\pgfsetbuttcap%
\pgfsetroundjoin%
\definecolor{currentfill}{rgb}{1.000000,0.509404,0.491473}%
\pgfsetfillcolor{currentfill}%
\pgfsetlinewidth{0.250937pt}%
\definecolor{currentstroke}{rgb}{1.000000,1.000000,1.000000}%
\pgfsetstrokecolor{currentstroke}%
\pgfsetdash{}{0pt}%
\pgfpathmoveto{\pgfqpoint{2.047924in}{4.536132in}}%
\pgfpathlineto{\pgfqpoint{2.135660in}{4.536132in}}%
\pgfpathlineto{\pgfqpoint{2.135660in}{4.448396in}}%
\pgfpathlineto{\pgfqpoint{2.047924in}{4.448396in}}%
\pgfpathlineto{\pgfqpoint{2.047924in}{4.536132in}}%
\pgfusepath{stroke,fill}%
\end{pgfscope}%
\begin{pgfscope}%
\pgfpathrectangle{\pgfqpoint{0.380943in}{4.185189in}}{\pgfqpoint{4.650000in}{0.614151in}}%
\pgfusepath{clip}%
\pgfsetbuttcap%
\pgfsetroundjoin%
\definecolor{currentfill}{rgb}{0.965444,0.906113,0.711757}%
\pgfsetfillcolor{currentfill}%
\pgfsetlinewidth{0.250937pt}%
\definecolor{currentstroke}{rgb}{1.000000,1.000000,1.000000}%
\pgfsetstrokecolor{currentstroke}%
\pgfsetdash{}{0pt}%
\pgfpathmoveto{\pgfqpoint{2.135660in}{4.536132in}}%
\pgfpathlineto{\pgfqpoint{2.223396in}{4.536132in}}%
\pgfpathlineto{\pgfqpoint{2.223396in}{4.448396in}}%
\pgfpathlineto{\pgfqpoint{2.135660in}{4.448396in}}%
\pgfpathlineto{\pgfqpoint{2.135660in}{4.536132in}}%
\pgfusepath{stroke,fill}%
\end{pgfscope}%
\begin{pgfscope}%
\pgfpathrectangle{\pgfqpoint{0.380943in}{4.185189in}}{\pgfqpoint{4.650000in}{0.614151in}}%
\pgfusepath{clip}%
\pgfsetbuttcap%
\pgfsetroundjoin%
\definecolor{currentfill}{rgb}{0.968166,0.945882,0.748604}%
\pgfsetfillcolor{currentfill}%
\pgfsetlinewidth{0.250937pt}%
\definecolor{currentstroke}{rgb}{1.000000,1.000000,1.000000}%
\pgfsetstrokecolor{currentstroke}%
\pgfsetdash{}{0pt}%
\pgfpathmoveto{\pgfqpoint{2.223396in}{4.536132in}}%
\pgfpathlineto{\pgfqpoint{2.311132in}{4.536132in}}%
\pgfpathlineto{\pgfqpoint{2.311132in}{4.448396in}}%
\pgfpathlineto{\pgfqpoint{2.223396in}{4.448396in}}%
\pgfpathlineto{\pgfqpoint{2.223396in}{4.536132in}}%
\pgfusepath{stroke,fill}%
\end{pgfscope}%
\begin{pgfscope}%
\pgfpathrectangle{\pgfqpoint{0.380943in}{4.185189in}}{\pgfqpoint{4.650000in}{0.614151in}}%
\pgfusepath{clip}%
\pgfsetbuttcap%
\pgfsetroundjoin%
\definecolor{currentfill}{rgb}{0.965444,0.906113,0.711757}%
\pgfsetfillcolor{currentfill}%
\pgfsetlinewidth{0.250937pt}%
\definecolor{currentstroke}{rgb}{1.000000,1.000000,1.000000}%
\pgfsetstrokecolor{currentstroke}%
\pgfsetdash{}{0pt}%
\pgfpathmoveto{\pgfqpoint{2.311132in}{4.536132in}}%
\pgfpathlineto{\pgfqpoint{2.398868in}{4.536132in}}%
\pgfpathlineto{\pgfqpoint{2.398868in}{4.448396in}}%
\pgfpathlineto{\pgfqpoint{2.311132in}{4.448396in}}%
\pgfpathlineto{\pgfqpoint{2.311132in}{4.536132in}}%
\pgfusepath{stroke,fill}%
\end{pgfscope}%
\begin{pgfscope}%
\pgfpathrectangle{\pgfqpoint{0.380943in}{4.185189in}}{\pgfqpoint{4.650000in}{0.614151in}}%
\pgfusepath{clip}%
\pgfsetbuttcap%
\pgfsetroundjoin%
\definecolor{currentfill}{rgb}{0.992326,0.765229,0.614840}%
\pgfsetfillcolor{currentfill}%
\pgfsetlinewidth{0.250937pt}%
\definecolor{currentstroke}{rgb}{1.000000,1.000000,1.000000}%
\pgfsetstrokecolor{currentstroke}%
\pgfsetdash{}{0pt}%
\pgfpathmoveto{\pgfqpoint{2.398868in}{4.536132in}}%
\pgfpathlineto{\pgfqpoint{2.486603in}{4.536132in}}%
\pgfpathlineto{\pgfqpoint{2.486603in}{4.448396in}}%
\pgfpathlineto{\pgfqpoint{2.398868in}{4.448396in}}%
\pgfpathlineto{\pgfqpoint{2.398868in}{4.536132in}}%
\pgfusepath{stroke,fill}%
\end{pgfscope}%
\begin{pgfscope}%
\pgfpathrectangle{\pgfqpoint{0.380943in}{4.185189in}}{\pgfqpoint{4.650000in}{0.614151in}}%
\pgfusepath{clip}%
\pgfsetbuttcap%
\pgfsetroundjoin%
\definecolor{currentfill}{rgb}{0.998939,0.658962,0.556032}%
\pgfsetfillcolor{currentfill}%
\pgfsetlinewidth{0.250937pt}%
\definecolor{currentstroke}{rgb}{1.000000,1.000000,1.000000}%
\pgfsetstrokecolor{currentstroke}%
\pgfsetdash{}{0pt}%
\pgfpathmoveto{\pgfqpoint{2.486603in}{4.536132in}}%
\pgfpathlineto{\pgfqpoint{2.574339in}{4.536132in}}%
\pgfpathlineto{\pgfqpoint{2.574339in}{4.448396in}}%
\pgfpathlineto{\pgfqpoint{2.486603in}{4.448396in}}%
\pgfpathlineto{\pgfqpoint{2.486603in}{4.536132in}}%
\pgfusepath{stroke,fill}%
\end{pgfscope}%
\begin{pgfscope}%
\pgfpathrectangle{\pgfqpoint{0.380943in}{4.185189in}}{\pgfqpoint{4.650000in}{0.614151in}}%
\pgfusepath{clip}%
\pgfsetbuttcap%
\pgfsetroundjoin%
\definecolor{currentfill}{rgb}{0.979654,0.837186,0.669619}%
\pgfsetfillcolor{currentfill}%
\pgfsetlinewidth{0.250937pt}%
\definecolor{currentstroke}{rgb}{1.000000,1.000000,1.000000}%
\pgfsetstrokecolor{currentstroke}%
\pgfsetdash{}{0pt}%
\pgfpathmoveto{\pgfqpoint{2.574339in}{4.536132in}}%
\pgfpathlineto{\pgfqpoint{2.662075in}{4.536132in}}%
\pgfpathlineto{\pgfqpoint{2.662075in}{4.448396in}}%
\pgfpathlineto{\pgfqpoint{2.574339in}{4.448396in}}%
\pgfpathlineto{\pgfqpoint{2.574339in}{4.536132in}}%
\pgfusepath{stroke,fill}%
\end{pgfscope}%
\begin{pgfscope}%
\pgfpathrectangle{\pgfqpoint{0.380943in}{4.185189in}}{\pgfqpoint{4.650000in}{0.614151in}}%
\pgfusepath{clip}%
\pgfsetbuttcap%
\pgfsetroundjoin%
\definecolor{currentfill}{rgb}{0.992326,0.765229,0.614840}%
\pgfsetfillcolor{currentfill}%
\pgfsetlinewidth{0.250937pt}%
\definecolor{currentstroke}{rgb}{1.000000,1.000000,1.000000}%
\pgfsetstrokecolor{currentstroke}%
\pgfsetdash{}{0pt}%
\pgfpathmoveto{\pgfqpoint{2.662075in}{4.536132in}}%
\pgfpathlineto{\pgfqpoint{2.749811in}{4.536132in}}%
\pgfpathlineto{\pgfqpoint{2.749811in}{4.448396in}}%
\pgfpathlineto{\pgfqpoint{2.662075in}{4.448396in}}%
\pgfpathlineto{\pgfqpoint{2.662075in}{4.536132in}}%
\pgfusepath{stroke,fill}%
\end{pgfscope}%
\begin{pgfscope}%
\pgfpathrectangle{\pgfqpoint{0.380943in}{4.185189in}}{\pgfqpoint{4.650000in}{0.614151in}}%
\pgfusepath{clip}%
\pgfsetbuttcap%
\pgfsetroundjoin%
\definecolor{currentfill}{rgb}{0.979654,0.837186,0.669619}%
\pgfsetfillcolor{currentfill}%
\pgfsetlinewidth{0.250937pt}%
\definecolor{currentstroke}{rgb}{1.000000,1.000000,1.000000}%
\pgfsetstrokecolor{currentstroke}%
\pgfsetdash{}{0pt}%
\pgfpathmoveto{\pgfqpoint{2.749811in}{4.536132in}}%
\pgfpathlineto{\pgfqpoint{2.837547in}{4.536132in}}%
\pgfpathlineto{\pgfqpoint{2.837547in}{4.448396in}}%
\pgfpathlineto{\pgfqpoint{2.749811in}{4.448396in}}%
\pgfpathlineto{\pgfqpoint{2.749811in}{4.536132in}}%
\pgfusepath{stroke,fill}%
\end{pgfscope}%
\begin{pgfscope}%
\pgfpathrectangle{\pgfqpoint{0.380943in}{4.185189in}}{\pgfqpoint{4.650000in}{0.614151in}}%
\pgfusepath{clip}%
\pgfsetbuttcap%
\pgfsetroundjoin%
\definecolor{currentfill}{rgb}{0.992326,0.765229,0.614840}%
\pgfsetfillcolor{currentfill}%
\pgfsetlinewidth{0.250937pt}%
\definecolor{currentstroke}{rgb}{1.000000,1.000000,1.000000}%
\pgfsetstrokecolor{currentstroke}%
\pgfsetdash{}{0pt}%
\pgfpathmoveto{\pgfqpoint{2.837547in}{4.536132in}}%
\pgfpathlineto{\pgfqpoint{2.925283in}{4.536132in}}%
\pgfpathlineto{\pgfqpoint{2.925283in}{4.448396in}}%
\pgfpathlineto{\pgfqpoint{2.837547in}{4.448396in}}%
\pgfpathlineto{\pgfqpoint{2.837547in}{4.536132in}}%
\pgfusepath{stroke,fill}%
\end{pgfscope}%
\begin{pgfscope}%
\pgfpathrectangle{\pgfqpoint{0.380943in}{4.185189in}}{\pgfqpoint{4.650000in}{0.614151in}}%
\pgfusepath{clip}%
\pgfsetbuttcap%
\pgfsetroundjoin%
\definecolor{currentfill}{rgb}{0.972549,0.870588,0.692810}%
\pgfsetfillcolor{currentfill}%
\pgfsetlinewidth{0.250937pt}%
\definecolor{currentstroke}{rgb}{1.000000,1.000000,1.000000}%
\pgfsetstrokecolor{currentstroke}%
\pgfsetdash{}{0pt}%
\pgfpathmoveto{\pgfqpoint{2.925283in}{4.536132in}}%
\pgfpathlineto{\pgfqpoint{3.013019in}{4.536132in}}%
\pgfpathlineto{\pgfqpoint{3.013019in}{4.448396in}}%
\pgfpathlineto{\pgfqpoint{2.925283in}{4.448396in}}%
\pgfpathlineto{\pgfqpoint{2.925283in}{4.536132in}}%
\pgfusepath{stroke,fill}%
\end{pgfscope}%
\begin{pgfscope}%
\pgfpathrectangle{\pgfqpoint{0.380943in}{4.185189in}}{\pgfqpoint{4.650000in}{0.614151in}}%
\pgfusepath{clip}%
\pgfsetbuttcap%
\pgfsetroundjoin%
\definecolor{currentfill}{rgb}{0.979654,0.837186,0.669619}%
\pgfsetfillcolor{currentfill}%
\pgfsetlinewidth{0.250937pt}%
\definecolor{currentstroke}{rgb}{1.000000,1.000000,1.000000}%
\pgfsetstrokecolor{currentstroke}%
\pgfsetdash{}{0pt}%
\pgfpathmoveto{\pgfqpoint{3.013019in}{4.536132in}}%
\pgfpathlineto{\pgfqpoint{3.100754in}{4.536132in}}%
\pgfpathlineto{\pgfqpoint{3.100754in}{4.448396in}}%
\pgfpathlineto{\pgfqpoint{3.013019in}{4.448396in}}%
\pgfpathlineto{\pgfqpoint{3.013019in}{4.536132in}}%
\pgfusepath{stroke,fill}%
\end{pgfscope}%
\begin{pgfscope}%
\pgfpathrectangle{\pgfqpoint{0.380943in}{4.185189in}}{\pgfqpoint{4.650000in}{0.614151in}}%
\pgfusepath{clip}%
\pgfsetbuttcap%
\pgfsetroundjoin%
\definecolor{currentfill}{rgb}{0.996571,0.720538,0.589189}%
\pgfsetfillcolor{currentfill}%
\pgfsetlinewidth{0.250937pt}%
\definecolor{currentstroke}{rgb}{1.000000,1.000000,1.000000}%
\pgfsetstrokecolor{currentstroke}%
\pgfsetdash{}{0pt}%
\pgfpathmoveto{\pgfqpoint{3.100754in}{4.536132in}}%
\pgfpathlineto{\pgfqpoint{3.188490in}{4.536132in}}%
\pgfpathlineto{\pgfqpoint{3.188490in}{4.448396in}}%
\pgfpathlineto{\pgfqpoint{3.100754in}{4.448396in}}%
\pgfpathlineto{\pgfqpoint{3.100754in}{4.536132in}}%
\pgfusepath{stroke,fill}%
\end{pgfscope}%
\begin{pgfscope}%
\pgfpathrectangle{\pgfqpoint{0.380943in}{4.185189in}}{\pgfqpoint{4.650000in}{0.614151in}}%
\pgfusepath{clip}%
\pgfsetbuttcap%
\pgfsetroundjoin%
\definecolor{currentfill}{rgb}{0.968166,0.945882,0.748604}%
\pgfsetfillcolor{currentfill}%
\pgfsetlinewidth{0.250937pt}%
\definecolor{currentstroke}{rgb}{1.000000,1.000000,1.000000}%
\pgfsetstrokecolor{currentstroke}%
\pgfsetdash{}{0pt}%
\pgfpathmoveto{\pgfqpoint{3.188490in}{4.536132in}}%
\pgfpathlineto{\pgfqpoint{3.276226in}{4.536132in}}%
\pgfpathlineto{\pgfqpoint{3.276226in}{4.448396in}}%
\pgfpathlineto{\pgfqpoint{3.188490in}{4.448396in}}%
\pgfpathlineto{\pgfqpoint{3.188490in}{4.536132in}}%
\pgfusepath{stroke,fill}%
\end{pgfscope}%
\begin{pgfscope}%
\pgfpathrectangle{\pgfqpoint{0.380943in}{4.185189in}}{\pgfqpoint{4.650000in}{0.614151in}}%
\pgfusepath{clip}%
\pgfsetbuttcap%
\pgfsetroundjoin%
\definecolor{currentfill}{rgb}{0.968166,0.945882,0.748604}%
\pgfsetfillcolor{currentfill}%
\pgfsetlinewidth{0.250937pt}%
\definecolor{currentstroke}{rgb}{1.000000,1.000000,1.000000}%
\pgfsetstrokecolor{currentstroke}%
\pgfsetdash{}{0pt}%
\pgfpathmoveto{\pgfqpoint{3.276226in}{4.536132in}}%
\pgfpathlineto{\pgfqpoint{3.363962in}{4.536132in}}%
\pgfpathlineto{\pgfqpoint{3.363962in}{4.448396in}}%
\pgfpathlineto{\pgfqpoint{3.276226in}{4.448396in}}%
\pgfpathlineto{\pgfqpoint{3.276226in}{4.536132in}}%
\pgfusepath{stroke,fill}%
\end{pgfscope}%
\begin{pgfscope}%
\pgfpathrectangle{\pgfqpoint{0.380943in}{4.185189in}}{\pgfqpoint{4.650000in}{0.614151in}}%
\pgfusepath{clip}%
\pgfsetbuttcap%
\pgfsetroundjoin%
\definecolor{currentfill}{rgb}{0.972549,0.870588,0.692810}%
\pgfsetfillcolor{currentfill}%
\pgfsetlinewidth{0.250937pt}%
\definecolor{currentstroke}{rgb}{1.000000,1.000000,1.000000}%
\pgfsetstrokecolor{currentstroke}%
\pgfsetdash{}{0pt}%
\pgfpathmoveto{\pgfqpoint{3.363962in}{4.536132in}}%
\pgfpathlineto{\pgfqpoint{3.451698in}{4.536132in}}%
\pgfpathlineto{\pgfqpoint{3.451698in}{4.448396in}}%
\pgfpathlineto{\pgfqpoint{3.363962in}{4.448396in}}%
\pgfpathlineto{\pgfqpoint{3.363962in}{4.536132in}}%
\pgfusepath{stroke,fill}%
\end{pgfscope}%
\begin{pgfscope}%
\pgfpathrectangle{\pgfqpoint{0.380943in}{4.185189in}}{\pgfqpoint{4.650000in}{0.614151in}}%
\pgfusepath{clip}%
\pgfsetbuttcap%
\pgfsetroundjoin%
\definecolor{currentfill}{rgb}{0.968166,0.945882,0.748604}%
\pgfsetfillcolor{currentfill}%
\pgfsetlinewidth{0.250937pt}%
\definecolor{currentstroke}{rgb}{1.000000,1.000000,1.000000}%
\pgfsetstrokecolor{currentstroke}%
\pgfsetdash{}{0pt}%
\pgfpathmoveto{\pgfqpoint{3.451698in}{4.536132in}}%
\pgfpathlineto{\pgfqpoint{3.539434in}{4.536132in}}%
\pgfpathlineto{\pgfqpoint{3.539434in}{4.448396in}}%
\pgfpathlineto{\pgfqpoint{3.451698in}{4.448396in}}%
\pgfpathlineto{\pgfqpoint{3.451698in}{4.536132in}}%
\pgfusepath{stroke,fill}%
\end{pgfscope}%
\begin{pgfscope}%
\pgfpathrectangle{\pgfqpoint{0.380943in}{4.185189in}}{\pgfqpoint{4.650000in}{0.614151in}}%
\pgfusepath{clip}%
\pgfsetbuttcap%
\pgfsetroundjoin%
\definecolor{currentfill}{rgb}{0.965444,0.906113,0.711757}%
\pgfsetfillcolor{currentfill}%
\pgfsetlinewidth{0.250937pt}%
\definecolor{currentstroke}{rgb}{1.000000,1.000000,1.000000}%
\pgfsetstrokecolor{currentstroke}%
\pgfsetdash{}{0pt}%
\pgfpathmoveto{\pgfqpoint{3.539434in}{4.536132in}}%
\pgfpathlineto{\pgfqpoint{3.627169in}{4.536132in}}%
\pgfpathlineto{\pgfqpoint{3.627169in}{4.448396in}}%
\pgfpathlineto{\pgfqpoint{3.539434in}{4.448396in}}%
\pgfpathlineto{\pgfqpoint{3.539434in}{4.536132in}}%
\pgfusepath{stroke,fill}%
\end{pgfscope}%
\begin{pgfscope}%
\pgfpathrectangle{\pgfqpoint{0.380943in}{4.185189in}}{\pgfqpoint{4.650000in}{0.614151in}}%
\pgfusepath{clip}%
\pgfsetbuttcap%
\pgfsetroundjoin%
\definecolor{currentfill}{rgb}{0.962414,0.923552,0.722891}%
\pgfsetfillcolor{currentfill}%
\pgfsetlinewidth{0.250937pt}%
\definecolor{currentstroke}{rgb}{1.000000,1.000000,1.000000}%
\pgfsetstrokecolor{currentstroke}%
\pgfsetdash{}{0pt}%
\pgfpathmoveto{\pgfqpoint{3.627169in}{4.536132in}}%
\pgfpathlineto{\pgfqpoint{3.714905in}{4.536132in}}%
\pgfpathlineto{\pgfqpoint{3.714905in}{4.448396in}}%
\pgfpathlineto{\pgfqpoint{3.627169in}{4.448396in}}%
\pgfpathlineto{\pgfqpoint{3.627169in}{4.536132in}}%
\pgfusepath{stroke,fill}%
\end{pgfscope}%
\begin{pgfscope}%
\pgfpathrectangle{\pgfqpoint{0.380943in}{4.185189in}}{\pgfqpoint{4.650000in}{0.614151in}}%
\pgfusepath{clip}%
\pgfsetbuttcap%
\pgfsetroundjoin%
\definecolor{currentfill}{rgb}{0.979654,0.837186,0.669619}%
\pgfsetfillcolor{currentfill}%
\pgfsetlinewidth{0.250937pt}%
\definecolor{currentstroke}{rgb}{1.000000,1.000000,1.000000}%
\pgfsetstrokecolor{currentstroke}%
\pgfsetdash{}{0pt}%
\pgfpathmoveto{\pgfqpoint{3.714905in}{4.536132in}}%
\pgfpathlineto{\pgfqpoint{3.802641in}{4.536132in}}%
\pgfpathlineto{\pgfqpoint{3.802641in}{4.448396in}}%
\pgfpathlineto{\pgfqpoint{3.714905in}{4.448396in}}%
\pgfpathlineto{\pgfqpoint{3.714905in}{4.536132in}}%
\pgfusepath{stroke,fill}%
\end{pgfscope}%
\begin{pgfscope}%
\pgfpathrectangle{\pgfqpoint{0.380943in}{4.185189in}}{\pgfqpoint{4.650000in}{0.614151in}}%
\pgfusepath{clip}%
\pgfsetbuttcap%
\pgfsetroundjoin%
\definecolor{currentfill}{rgb}{0.972549,0.870588,0.692810}%
\pgfsetfillcolor{currentfill}%
\pgfsetlinewidth{0.250937pt}%
\definecolor{currentstroke}{rgb}{1.000000,1.000000,1.000000}%
\pgfsetstrokecolor{currentstroke}%
\pgfsetdash{}{0pt}%
\pgfpathmoveto{\pgfqpoint{3.802641in}{4.536132in}}%
\pgfpathlineto{\pgfqpoint{3.890377in}{4.536132in}}%
\pgfpathlineto{\pgfqpoint{3.890377in}{4.448396in}}%
\pgfpathlineto{\pgfqpoint{3.802641in}{4.448396in}}%
\pgfpathlineto{\pgfqpoint{3.802641in}{4.536132in}}%
\pgfusepath{stroke,fill}%
\end{pgfscope}%
\begin{pgfscope}%
\pgfpathrectangle{\pgfqpoint{0.380943in}{4.185189in}}{\pgfqpoint{4.650000in}{0.614151in}}%
\pgfusepath{clip}%
\pgfsetbuttcap%
\pgfsetroundjoin%
\definecolor{currentfill}{rgb}{0.979654,0.837186,0.669619}%
\pgfsetfillcolor{currentfill}%
\pgfsetlinewidth{0.250937pt}%
\definecolor{currentstroke}{rgb}{1.000000,1.000000,1.000000}%
\pgfsetstrokecolor{currentstroke}%
\pgfsetdash{}{0pt}%
\pgfpathmoveto{\pgfqpoint{3.890377in}{4.536132in}}%
\pgfpathlineto{\pgfqpoint{3.978113in}{4.536132in}}%
\pgfpathlineto{\pgfqpoint{3.978113in}{4.448396in}}%
\pgfpathlineto{\pgfqpoint{3.890377in}{4.448396in}}%
\pgfpathlineto{\pgfqpoint{3.890377in}{4.536132in}}%
\pgfusepath{stroke,fill}%
\end{pgfscope}%
\begin{pgfscope}%
\pgfpathrectangle{\pgfqpoint{0.380943in}{4.185189in}}{\pgfqpoint{4.650000in}{0.614151in}}%
\pgfusepath{clip}%
\pgfsetbuttcap%
\pgfsetroundjoin%
\definecolor{currentfill}{rgb}{1.000000,0.557862,0.511772}%
\pgfsetfillcolor{currentfill}%
\pgfsetlinewidth{0.250937pt}%
\definecolor{currentstroke}{rgb}{1.000000,1.000000,1.000000}%
\pgfsetstrokecolor{currentstroke}%
\pgfsetdash{}{0pt}%
\pgfpathmoveto{\pgfqpoint{3.978113in}{4.536132in}}%
\pgfpathlineto{\pgfqpoint{4.065849in}{4.536132in}}%
\pgfpathlineto{\pgfqpoint{4.065849in}{4.448396in}}%
\pgfpathlineto{\pgfqpoint{3.978113in}{4.448396in}}%
\pgfpathlineto{\pgfqpoint{3.978113in}{4.536132in}}%
\pgfusepath{stroke,fill}%
\end{pgfscope}%
\begin{pgfscope}%
\pgfpathrectangle{\pgfqpoint{0.380943in}{4.185189in}}{\pgfqpoint{4.650000in}{0.614151in}}%
\pgfusepath{clip}%
\pgfsetbuttcap%
\pgfsetroundjoin%
\definecolor{currentfill}{rgb}{0.992326,0.765229,0.614840}%
\pgfsetfillcolor{currentfill}%
\pgfsetlinewidth{0.250937pt}%
\definecolor{currentstroke}{rgb}{1.000000,1.000000,1.000000}%
\pgfsetstrokecolor{currentstroke}%
\pgfsetdash{}{0pt}%
\pgfpathmoveto{\pgfqpoint{4.065849in}{4.536132in}}%
\pgfpathlineto{\pgfqpoint{4.153585in}{4.536132in}}%
\pgfpathlineto{\pgfqpoint{4.153585in}{4.448396in}}%
\pgfpathlineto{\pgfqpoint{4.065849in}{4.448396in}}%
\pgfpathlineto{\pgfqpoint{4.065849in}{4.536132in}}%
\pgfusepath{stroke,fill}%
\end{pgfscope}%
\begin{pgfscope}%
\pgfpathrectangle{\pgfqpoint{0.380943in}{4.185189in}}{\pgfqpoint{4.650000in}{0.614151in}}%
\pgfusepath{clip}%
\pgfsetbuttcap%
\pgfsetroundjoin%
\definecolor{currentfill}{rgb}{0.996571,0.720538,0.589189}%
\pgfsetfillcolor{currentfill}%
\pgfsetlinewidth{0.250937pt}%
\definecolor{currentstroke}{rgb}{1.000000,1.000000,1.000000}%
\pgfsetstrokecolor{currentstroke}%
\pgfsetdash{}{0pt}%
\pgfpathmoveto{\pgfqpoint{4.153585in}{4.536132in}}%
\pgfpathlineto{\pgfqpoint{4.241320in}{4.536132in}}%
\pgfpathlineto{\pgfqpoint{4.241320in}{4.448396in}}%
\pgfpathlineto{\pgfqpoint{4.153585in}{4.448396in}}%
\pgfpathlineto{\pgfqpoint{4.153585in}{4.536132in}}%
\pgfusepath{stroke,fill}%
\end{pgfscope}%
\begin{pgfscope}%
\pgfpathrectangle{\pgfqpoint{0.380943in}{4.185189in}}{\pgfqpoint{4.650000in}{0.614151in}}%
\pgfusepath{clip}%
\pgfsetbuttcap%
\pgfsetroundjoin%
\definecolor{currentfill}{rgb}{0.998939,0.658962,0.556032}%
\pgfsetfillcolor{currentfill}%
\pgfsetlinewidth{0.250937pt}%
\definecolor{currentstroke}{rgb}{1.000000,1.000000,1.000000}%
\pgfsetstrokecolor{currentstroke}%
\pgfsetdash{}{0pt}%
\pgfpathmoveto{\pgfqpoint{4.241320in}{4.536132in}}%
\pgfpathlineto{\pgfqpoint{4.329056in}{4.536132in}}%
\pgfpathlineto{\pgfqpoint{4.329056in}{4.448396in}}%
\pgfpathlineto{\pgfqpoint{4.241320in}{4.448396in}}%
\pgfpathlineto{\pgfqpoint{4.241320in}{4.536132in}}%
\pgfusepath{stroke,fill}%
\end{pgfscope}%
\begin{pgfscope}%
\pgfpathrectangle{\pgfqpoint{0.380943in}{4.185189in}}{\pgfqpoint{4.650000in}{0.614151in}}%
\pgfusepath{clip}%
\pgfsetbuttcap%
\pgfsetroundjoin%
\definecolor{currentfill}{rgb}{0.979654,0.837186,0.669619}%
\pgfsetfillcolor{currentfill}%
\pgfsetlinewidth{0.250937pt}%
\definecolor{currentstroke}{rgb}{1.000000,1.000000,1.000000}%
\pgfsetstrokecolor{currentstroke}%
\pgfsetdash{}{0pt}%
\pgfpathmoveto{\pgfqpoint{4.329056in}{4.536132in}}%
\pgfpathlineto{\pgfqpoint{4.416792in}{4.536132in}}%
\pgfpathlineto{\pgfqpoint{4.416792in}{4.448396in}}%
\pgfpathlineto{\pgfqpoint{4.329056in}{4.448396in}}%
\pgfpathlineto{\pgfqpoint{4.329056in}{4.536132in}}%
\pgfusepath{stroke,fill}%
\end{pgfscope}%
\begin{pgfscope}%
\pgfpathrectangle{\pgfqpoint{0.380943in}{4.185189in}}{\pgfqpoint{4.650000in}{0.614151in}}%
\pgfusepath{clip}%
\pgfsetbuttcap%
\pgfsetroundjoin%
\definecolor{currentfill}{rgb}{0.979654,0.837186,0.669619}%
\pgfsetfillcolor{currentfill}%
\pgfsetlinewidth{0.250937pt}%
\definecolor{currentstroke}{rgb}{1.000000,1.000000,1.000000}%
\pgfsetstrokecolor{currentstroke}%
\pgfsetdash{}{0pt}%
\pgfpathmoveto{\pgfqpoint{4.416792in}{4.536132in}}%
\pgfpathlineto{\pgfqpoint{4.504528in}{4.536132in}}%
\pgfpathlineto{\pgfqpoint{4.504528in}{4.448396in}}%
\pgfpathlineto{\pgfqpoint{4.416792in}{4.448396in}}%
\pgfpathlineto{\pgfqpoint{4.416792in}{4.536132in}}%
\pgfusepath{stroke,fill}%
\end{pgfscope}%
\begin{pgfscope}%
\pgfpathrectangle{\pgfqpoint{0.380943in}{4.185189in}}{\pgfqpoint{4.650000in}{0.614151in}}%
\pgfusepath{clip}%
\pgfsetbuttcap%
\pgfsetroundjoin%
\definecolor{currentfill}{rgb}{0.986759,0.806398,0.641200}%
\pgfsetfillcolor{currentfill}%
\pgfsetlinewidth{0.250937pt}%
\definecolor{currentstroke}{rgb}{1.000000,1.000000,1.000000}%
\pgfsetstrokecolor{currentstroke}%
\pgfsetdash{}{0pt}%
\pgfpathmoveto{\pgfqpoint{4.504528in}{4.536132in}}%
\pgfpathlineto{\pgfqpoint{4.592264in}{4.536132in}}%
\pgfpathlineto{\pgfqpoint{4.592264in}{4.448396in}}%
\pgfpathlineto{\pgfqpoint{4.504528in}{4.448396in}}%
\pgfpathlineto{\pgfqpoint{4.504528in}{4.536132in}}%
\pgfusepath{stroke,fill}%
\end{pgfscope}%
\begin{pgfscope}%
\pgfpathrectangle{\pgfqpoint{0.380943in}{4.185189in}}{\pgfqpoint{4.650000in}{0.614151in}}%
\pgfusepath{clip}%
\pgfsetbuttcap%
\pgfsetroundjoin%
\definecolor{currentfill}{rgb}{0.800000,0.278431,0.278431}%
\pgfsetfillcolor{currentfill}%
\pgfsetlinewidth{0.250937pt}%
\definecolor{currentstroke}{rgb}{1.000000,1.000000,1.000000}%
\pgfsetstrokecolor{currentstroke}%
\pgfsetdash{}{0pt}%
\pgfpathmoveto{\pgfqpoint{4.592264in}{4.536132in}}%
\pgfpathlineto{\pgfqpoint{4.680000in}{4.536132in}}%
\pgfpathlineto{\pgfqpoint{4.680000in}{4.448396in}}%
\pgfpathlineto{\pgfqpoint{4.592264in}{4.448396in}}%
\pgfpathlineto{\pgfqpoint{4.592264in}{4.536132in}}%
\pgfusepath{stroke,fill}%
\end{pgfscope}%
\begin{pgfscope}%
\pgfpathrectangle{\pgfqpoint{0.380943in}{4.185189in}}{\pgfqpoint{4.650000in}{0.614151in}}%
\pgfusepath{clip}%
\pgfsetbuttcap%
\pgfsetroundjoin%
\definecolor{currentfill}{rgb}{0.986759,0.806398,0.641200}%
\pgfsetfillcolor{currentfill}%
\pgfsetlinewidth{0.250937pt}%
\definecolor{currentstroke}{rgb}{1.000000,1.000000,1.000000}%
\pgfsetstrokecolor{currentstroke}%
\pgfsetdash{}{0pt}%
\pgfpathmoveto{\pgfqpoint{4.680000in}{4.536132in}}%
\pgfpathlineto{\pgfqpoint{4.767736in}{4.536132in}}%
\pgfpathlineto{\pgfqpoint{4.767736in}{4.448396in}}%
\pgfpathlineto{\pgfqpoint{4.680000in}{4.448396in}}%
\pgfpathlineto{\pgfqpoint{4.680000in}{4.536132in}}%
\pgfusepath{stroke,fill}%
\end{pgfscope}%
\begin{pgfscope}%
\pgfpathrectangle{\pgfqpoint{0.380943in}{4.185189in}}{\pgfqpoint{4.650000in}{0.614151in}}%
\pgfusepath{clip}%
\pgfsetbuttcap%
\pgfsetroundjoin%
\definecolor{currentfill}{rgb}{0.965444,0.906113,0.711757}%
\pgfsetfillcolor{currentfill}%
\pgfsetlinewidth{0.250937pt}%
\definecolor{currentstroke}{rgb}{1.000000,1.000000,1.000000}%
\pgfsetstrokecolor{currentstroke}%
\pgfsetdash{}{0pt}%
\pgfpathmoveto{\pgfqpoint{4.767736in}{4.536132in}}%
\pgfpathlineto{\pgfqpoint{4.855471in}{4.536132in}}%
\pgfpathlineto{\pgfqpoint{4.855471in}{4.448396in}}%
\pgfpathlineto{\pgfqpoint{4.767736in}{4.448396in}}%
\pgfpathlineto{\pgfqpoint{4.767736in}{4.536132in}}%
\pgfusepath{stroke,fill}%
\end{pgfscope}%
\begin{pgfscope}%
\pgfpathrectangle{\pgfqpoint{0.380943in}{4.185189in}}{\pgfqpoint{4.650000in}{0.614151in}}%
\pgfusepath{clip}%
\pgfsetbuttcap%
\pgfsetroundjoin%
\definecolor{currentfill}{rgb}{0.968166,0.945882,0.748604}%
\pgfsetfillcolor{currentfill}%
\pgfsetlinewidth{0.250937pt}%
\definecolor{currentstroke}{rgb}{1.000000,1.000000,1.000000}%
\pgfsetstrokecolor{currentstroke}%
\pgfsetdash{}{0pt}%
\pgfpathmoveto{\pgfqpoint{4.855471in}{4.536132in}}%
\pgfpathlineto{\pgfqpoint{4.943207in}{4.536132in}}%
\pgfpathlineto{\pgfqpoint{4.943207in}{4.448396in}}%
\pgfpathlineto{\pgfqpoint{4.855471in}{4.448396in}}%
\pgfpathlineto{\pgfqpoint{4.855471in}{4.536132in}}%
\pgfusepath{stroke,fill}%
\end{pgfscope}%
\begin{pgfscope}%
\pgfpathrectangle{\pgfqpoint{0.380943in}{4.185189in}}{\pgfqpoint{4.650000in}{0.614151in}}%
\pgfusepath{clip}%
\pgfsetbuttcap%
\pgfsetroundjoin%
\pgfsetlinewidth{0.250937pt}%
\definecolor{currentstroke}{rgb}{1.000000,1.000000,1.000000}%
\pgfsetstrokecolor{currentstroke}%
\pgfsetdash{}{0pt}%
\pgfpathmoveto{\pgfqpoint{4.943207in}{4.536132in}}%
\pgfpathlineto{\pgfqpoint{5.030943in}{4.536132in}}%
\pgfpathlineto{\pgfqpoint{5.030943in}{4.448396in}}%
\pgfpathlineto{\pgfqpoint{4.943207in}{4.448396in}}%
\pgfpathlineto{\pgfqpoint{4.943207in}{4.536132in}}%
\pgfusepath{stroke}%
\end{pgfscope}%
\begin{pgfscope}%
\pgfpathrectangle{\pgfqpoint{0.380943in}{4.185189in}}{\pgfqpoint{4.650000in}{0.614151in}}%
\pgfusepath{clip}%
\pgfsetbuttcap%
\pgfsetroundjoin%
\definecolor{currentfill}{rgb}{0.992326,0.765229,0.614840}%
\pgfsetfillcolor{currentfill}%
\pgfsetlinewidth{0.250937pt}%
\definecolor{currentstroke}{rgb}{1.000000,1.000000,1.000000}%
\pgfsetstrokecolor{currentstroke}%
\pgfsetdash{}{0pt}%
\pgfpathmoveto{\pgfqpoint{0.380943in}{4.448396in}}%
\pgfpathlineto{\pgfqpoint{0.468679in}{4.448396in}}%
\pgfpathlineto{\pgfqpoint{0.468679in}{4.360661in}}%
\pgfpathlineto{\pgfqpoint{0.380943in}{4.360661in}}%
\pgfpathlineto{\pgfqpoint{0.380943in}{4.448396in}}%
\pgfusepath{stroke,fill}%
\end{pgfscope}%
\begin{pgfscope}%
\pgfpathrectangle{\pgfqpoint{0.380943in}{4.185189in}}{\pgfqpoint{4.650000in}{0.614151in}}%
\pgfusepath{clip}%
\pgfsetbuttcap%
\pgfsetroundjoin%
\definecolor{currentfill}{rgb}{0.992326,0.765229,0.614840}%
\pgfsetfillcolor{currentfill}%
\pgfsetlinewidth{0.250937pt}%
\definecolor{currentstroke}{rgb}{1.000000,1.000000,1.000000}%
\pgfsetstrokecolor{currentstroke}%
\pgfsetdash{}{0pt}%
\pgfpathmoveto{\pgfqpoint{0.468679in}{4.448396in}}%
\pgfpathlineto{\pgfqpoint{0.556415in}{4.448396in}}%
\pgfpathlineto{\pgfqpoint{0.556415in}{4.360661in}}%
\pgfpathlineto{\pgfqpoint{0.468679in}{4.360661in}}%
\pgfpathlineto{\pgfqpoint{0.468679in}{4.448396in}}%
\pgfusepath{stroke,fill}%
\end{pgfscope}%
\begin{pgfscope}%
\pgfpathrectangle{\pgfqpoint{0.380943in}{4.185189in}}{\pgfqpoint{4.650000in}{0.614151in}}%
\pgfusepath{clip}%
\pgfsetbuttcap%
\pgfsetroundjoin%
\definecolor{currentfill}{rgb}{0.996571,0.720538,0.589189}%
\pgfsetfillcolor{currentfill}%
\pgfsetlinewidth{0.250937pt}%
\definecolor{currentstroke}{rgb}{1.000000,1.000000,1.000000}%
\pgfsetstrokecolor{currentstroke}%
\pgfsetdash{}{0pt}%
\pgfpathmoveto{\pgfqpoint{0.556415in}{4.448396in}}%
\pgfpathlineto{\pgfqpoint{0.644151in}{4.448396in}}%
\pgfpathlineto{\pgfqpoint{0.644151in}{4.360661in}}%
\pgfpathlineto{\pgfqpoint{0.556415in}{4.360661in}}%
\pgfpathlineto{\pgfqpoint{0.556415in}{4.448396in}}%
\pgfusepath{stroke,fill}%
\end{pgfscope}%
\begin{pgfscope}%
\pgfpathrectangle{\pgfqpoint{0.380943in}{4.185189in}}{\pgfqpoint{4.650000in}{0.614151in}}%
\pgfusepath{clip}%
\pgfsetbuttcap%
\pgfsetroundjoin%
\definecolor{currentfill}{rgb}{0.992326,0.765229,0.614840}%
\pgfsetfillcolor{currentfill}%
\pgfsetlinewidth{0.250937pt}%
\definecolor{currentstroke}{rgb}{1.000000,1.000000,1.000000}%
\pgfsetstrokecolor{currentstroke}%
\pgfsetdash{}{0pt}%
\pgfpathmoveto{\pgfqpoint{0.644151in}{4.448396in}}%
\pgfpathlineto{\pgfqpoint{0.731886in}{4.448396in}}%
\pgfpathlineto{\pgfqpoint{0.731886in}{4.360661in}}%
\pgfpathlineto{\pgfqpoint{0.644151in}{4.360661in}}%
\pgfpathlineto{\pgfqpoint{0.644151in}{4.448396in}}%
\pgfusepath{stroke,fill}%
\end{pgfscope}%
\begin{pgfscope}%
\pgfpathrectangle{\pgfqpoint{0.380943in}{4.185189in}}{\pgfqpoint{4.650000in}{0.614151in}}%
\pgfusepath{clip}%
\pgfsetbuttcap%
\pgfsetroundjoin%
\definecolor{currentfill}{rgb}{0.972549,0.870588,0.692810}%
\pgfsetfillcolor{currentfill}%
\pgfsetlinewidth{0.250937pt}%
\definecolor{currentstroke}{rgb}{1.000000,1.000000,1.000000}%
\pgfsetstrokecolor{currentstroke}%
\pgfsetdash{}{0pt}%
\pgfpathmoveto{\pgfqpoint{0.731886in}{4.448396in}}%
\pgfpathlineto{\pgfqpoint{0.819622in}{4.448396in}}%
\pgfpathlineto{\pgfqpoint{0.819622in}{4.360661in}}%
\pgfpathlineto{\pgfqpoint{0.731886in}{4.360661in}}%
\pgfpathlineto{\pgfqpoint{0.731886in}{4.448396in}}%
\pgfusepath{stroke,fill}%
\end{pgfscope}%
\begin{pgfscope}%
\pgfpathrectangle{\pgfqpoint{0.380943in}{4.185189in}}{\pgfqpoint{4.650000in}{0.614151in}}%
\pgfusepath{clip}%
\pgfsetbuttcap%
\pgfsetroundjoin%
\definecolor{currentfill}{rgb}{0.922338,0.400769,0.400769}%
\pgfsetfillcolor{currentfill}%
\pgfsetlinewidth{0.250937pt}%
\definecolor{currentstroke}{rgb}{1.000000,1.000000,1.000000}%
\pgfsetstrokecolor{currentstroke}%
\pgfsetdash{}{0pt}%
\pgfpathmoveto{\pgfqpoint{0.819622in}{4.448396in}}%
\pgfpathlineto{\pgfqpoint{0.907358in}{4.448396in}}%
\pgfpathlineto{\pgfqpoint{0.907358in}{4.360661in}}%
\pgfpathlineto{\pgfqpoint{0.819622in}{4.360661in}}%
\pgfpathlineto{\pgfqpoint{0.819622in}{4.448396in}}%
\pgfusepath{stroke,fill}%
\end{pgfscope}%
\begin{pgfscope}%
\pgfpathrectangle{\pgfqpoint{0.380943in}{4.185189in}}{\pgfqpoint{4.650000in}{0.614151in}}%
\pgfusepath{clip}%
\pgfsetbuttcap%
\pgfsetroundjoin%
\definecolor{currentfill}{rgb}{0.986759,0.806398,0.641200}%
\pgfsetfillcolor{currentfill}%
\pgfsetlinewidth{0.250937pt}%
\definecolor{currentstroke}{rgb}{1.000000,1.000000,1.000000}%
\pgfsetstrokecolor{currentstroke}%
\pgfsetdash{}{0pt}%
\pgfpathmoveto{\pgfqpoint{0.907358in}{4.448396in}}%
\pgfpathlineto{\pgfqpoint{0.995094in}{4.448396in}}%
\pgfpathlineto{\pgfqpoint{0.995094in}{4.360661in}}%
\pgfpathlineto{\pgfqpoint{0.907358in}{4.360661in}}%
\pgfpathlineto{\pgfqpoint{0.907358in}{4.448396in}}%
\pgfusepath{stroke,fill}%
\end{pgfscope}%
\begin{pgfscope}%
\pgfpathrectangle{\pgfqpoint{0.380943in}{4.185189in}}{\pgfqpoint{4.650000in}{0.614151in}}%
\pgfusepath{clip}%
\pgfsetbuttcap%
\pgfsetroundjoin%
\definecolor{currentfill}{rgb}{0.992326,0.765229,0.614840}%
\pgfsetfillcolor{currentfill}%
\pgfsetlinewidth{0.250937pt}%
\definecolor{currentstroke}{rgb}{1.000000,1.000000,1.000000}%
\pgfsetstrokecolor{currentstroke}%
\pgfsetdash{}{0pt}%
\pgfpathmoveto{\pgfqpoint{0.995094in}{4.448396in}}%
\pgfpathlineto{\pgfqpoint{1.082830in}{4.448396in}}%
\pgfpathlineto{\pgfqpoint{1.082830in}{4.360661in}}%
\pgfpathlineto{\pgfqpoint{0.995094in}{4.360661in}}%
\pgfpathlineto{\pgfqpoint{0.995094in}{4.448396in}}%
\pgfusepath{stroke,fill}%
\end{pgfscope}%
\begin{pgfscope}%
\pgfpathrectangle{\pgfqpoint{0.380943in}{4.185189in}}{\pgfqpoint{4.650000in}{0.614151in}}%
\pgfusepath{clip}%
\pgfsetbuttcap%
\pgfsetroundjoin%
\definecolor{currentfill}{rgb}{0.979654,0.837186,0.669619}%
\pgfsetfillcolor{currentfill}%
\pgfsetlinewidth{0.250937pt}%
\definecolor{currentstroke}{rgb}{1.000000,1.000000,1.000000}%
\pgfsetstrokecolor{currentstroke}%
\pgfsetdash{}{0pt}%
\pgfpathmoveto{\pgfqpoint{1.082830in}{4.448396in}}%
\pgfpathlineto{\pgfqpoint{1.170566in}{4.448396in}}%
\pgfpathlineto{\pgfqpoint{1.170566in}{4.360661in}}%
\pgfpathlineto{\pgfqpoint{1.082830in}{4.360661in}}%
\pgfpathlineto{\pgfqpoint{1.082830in}{4.448396in}}%
\pgfusepath{stroke,fill}%
\end{pgfscope}%
\begin{pgfscope}%
\pgfpathrectangle{\pgfqpoint{0.380943in}{4.185189in}}{\pgfqpoint{4.650000in}{0.614151in}}%
\pgfusepath{clip}%
\pgfsetbuttcap%
\pgfsetroundjoin%
\definecolor{currentfill}{rgb}{0.996571,0.720538,0.589189}%
\pgfsetfillcolor{currentfill}%
\pgfsetlinewidth{0.250937pt}%
\definecolor{currentstroke}{rgb}{1.000000,1.000000,1.000000}%
\pgfsetstrokecolor{currentstroke}%
\pgfsetdash{}{0pt}%
\pgfpathmoveto{\pgfqpoint{1.170566in}{4.448396in}}%
\pgfpathlineto{\pgfqpoint{1.258302in}{4.448396in}}%
\pgfpathlineto{\pgfqpoint{1.258302in}{4.360661in}}%
\pgfpathlineto{\pgfqpoint{1.170566in}{4.360661in}}%
\pgfpathlineto{\pgfqpoint{1.170566in}{4.448396in}}%
\pgfusepath{stroke,fill}%
\end{pgfscope}%
\begin{pgfscope}%
\pgfpathrectangle{\pgfqpoint{0.380943in}{4.185189in}}{\pgfqpoint{4.650000in}{0.614151in}}%
\pgfusepath{clip}%
\pgfsetbuttcap%
\pgfsetroundjoin%
\definecolor{currentfill}{rgb}{0.965444,0.906113,0.711757}%
\pgfsetfillcolor{currentfill}%
\pgfsetlinewidth{0.250937pt}%
\definecolor{currentstroke}{rgb}{1.000000,1.000000,1.000000}%
\pgfsetstrokecolor{currentstroke}%
\pgfsetdash{}{0pt}%
\pgfpathmoveto{\pgfqpoint{1.258302in}{4.448396in}}%
\pgfpathlineto{\pgfqpoint{1.346037in}{4.448396in}}%
\pgfpathlineto{\pgfqpoint{1.346037in}{4.360661in}}%
\pgfpathlineto{\pgfqpoint{1.258302in}{4.360661in}}%
\pgfpathlineto{\pgfqpoint{1.258302in}{4.448396in}}%
\pgfusepath{stroke,fill}%
\end{pgfscope}%
\begin{pgfscope}%
\pgfpathrectangle{\pgfqpoint{0.380943in}{4.185189in}}{\pgfqpoint{4.650000in}{0.614151in}}%
\pgfusepath{clip}%
\pgfsetbuttcap%
\pgfsetroundjoin%
\definecolor{currentfill}{rgb}{0.979654,0.837186,0.669619}%
\pgfsetfillcolor{currentfill}%
\pgfsetlinewidth{0.250937pt}%
\definecolor{currentstroke}{rgb}{1.000000,1.000000,1.000000}%
\pgfsetstrokecolor{currentstroke}%
\pgfsetdash{}{0pt}%
\pgfpathmoveto{\pgfqpoint{1.346037in}{4.448396in}}%
\pgfpathlineto{\pgfqpoint{1.433773in}{4.448396in}}%
\pgfpathlineto{\pgfqpoint{1.433773in}{4.360661in}}%
\pgfpathlineto{\pgfqpoint{1.346037in}{4.360661in}}%
\pgfpathlineto{\pgfqpoint{1.346037in}{4.448396in}}%
\pgfusepath{stroke,fill}%
\end{pgfscope}%
\begin{pgfscope}%
\pgfpathrectangle{\pgfqpoint{0.380943in}{4.185189in}}{\pgfqpoint{4.650000in}{0.614151in}}%
\pgfusepath{clip}%
\pgfsetbuttcap%
\pgfsetroundjoin%
\definecolor{currentfill}{rgb}{0.986759,0.806398,0.641200}%
\pgfsetfillcolor{currentfill}%
\pgfsetlinewidth{0.250937pt}%
\definecolor{currentstroke}{rgb}{1.000000,1.000000,1.000000}%
\pgfsetstrokecolor{currentstroke}%
\pgfsetdash{}{0pt}%
\pgfpathmoveto{\pgfqpoint{1.433773in}{4.448396in}}%
\pgfpathlineto{\pgfqpoint{1.521509in}{4.448396in}}%
\pgfpathlineto{\pgfqpoint{1.521509in}{4.360661in}}%
\pgfpathlineto{\pgfqpoint{1.433773in}{4.360661in}}%
\pgfpathlineto{\pgfqpoint{1.433773in}{4.448396in}}%
\pgfusepath{stroke,fill}%
\end{pgfscope}%
\begin{pgfscope}%
\pgfpathrectangle{\pgfqpoint{0.380943in}{4.185189in}}{\pgfqpoint{4.650000in}{0.614151in}}%
\pgfusepath{clip}%
\pgfsetbuttcap%
\pgfsetroundjoin%
\definecolor{currentfill}{rgb}{0.986759,0.806398,0.641200}%
\pgfsetfillcolor{currentfill}%
\pgfsetlinewidth{0.250937pt}%
\definecolor{currentstroke}{rgb}{1.000000,1.000000,1.000000}%
\pgfsetstrokecolor{currentstroke}%
\pgfsetdash{}{0pt}%
\pgfpathmoveto{\pgfqpoint{1.521509in}{4.448396in}}%
\pgfpathlineto{\pgfqpoint{1.609245in}{4.448396in}}%
\pgfpathlineto{\pgfqpoint{1.609245in}{4.360661in}}%
\pgfpathlineto{\pgfqpoint{1.521509in}{4.360661in}}%
\pgfpathlineto{\pgfqpoint{1.521509in}{4.448396in}}%
\pgfusepath{stroke,fill}%
\end{pgfscope}%
\begin{pgfscope}%
\pgfpathrectangle{\pgfqpoint{0.380943in}{4.185189in}}{\pgfqpoint{4.650000in}{0.614151in}}%
\pgfusepath{clip}%
\pgfsetbuttcap%
\pgfsetroundjoin%
\definecolor{currentfill}{rgb}{0.972549,0.870588,0.692810}%
\pgfsetfillcolor{currentfill}%
\pgfsetlinewidth{0.250937pt}%
\definecolor{currentstroke}{rgb}{1.000000,1.000000,1.000000}%
\pgfsetstrokecolor{currentstroke}%
\pgfsetdash{}{0pt}%
\pgfpathmoveto{\pgfqpoint{1.609245in}{4.448396in}}%
\pgfpathlineto{\pgfqpoint{1.696981in}{4.448396in}}%
\pgfpathlineto{\pgfqpoint{1.696981in}{4.360661in}}%
\pgfpathlineto{\pgfqpoint{1.609245in}{4.360661in}}%
\pgfpathlineto{\pgfqpoint{1.609245in}{4.448396in}}%
\pgfusepath{stroke,fill}%
\end{pgfscope}%
\begin{pgfscope}%
\pgfpathrectangle{\pgfqpoint{0.380943in}{4.185189in}}{\pgfqpoint{4.650000in}{0.614151in}}%
\pgfusepath{clip}%
\pgfsetbuttcap%
\pgfsetroundjoin%
\definecolor{currentfill}{rgb}{0.996571,0.720538,0.589189}%
\pgfsetfillcolor{currentfill}%
\pgfsetlinewidth{0.250937pt}%
\definecolor{currentstroke}{rgb}{1.000000,1.000000,1.000000}%
\pgfsetstrokecolor{currentstroke}%
\pgfsetdash{}{0pt}%
\pgfpathmoveto{\pgfqpoint{1.696981in}{4.448396in}}%
\pgfpathlineto{\pgfqpoint{1.784717in}{4.448396in}}%
\pgfpathlineto{\pgfqpoint{1.784717in}{4.360661in}}%
\pgfpathlineto{\pgfqpoint{1.696981in}{4.360661in}}%
\pgfpathlineto{\pgfqpoint{1.696981in}{4.448396in}}%
\pgfusepath{stroke,fill}%
\end{pgfscope}%
\begin{pgfscope}%
\pgfpathrectangle{\pgfqpoint{0.380943in}{4.185189in}}{\pgfqpoint{4.650000in}{0.614151in}}%
\pgfusepath{clip}%
\pgfsetbuttcap%
\pgfsetroundjoin%
\definecolor{currentfill}{rgb}{1.000000,0.605229,0.530719}%
\pgfsetfillcolor{currentfill}%
\pgfsetlinewidth{0.250937pt}%
\definecolor{currentstroke}{rgb}{1.000000,1.000000,1.000000}%
\pgfsetstrokecolor{currentstroke}%
\pgfsetdash{}{0pt}%
\pgfpathmoveto{\pgfqpoint{1.784717in}{4.448396in}}%
\pgfpathlineto{\pgfqpoint{1.872452in}{4.448396in}}%
\pgfpathlineto{\pgfqpoint{1.872452in}{4.360661in}}%
\pgfpathlineto{\pgfqpoint{1.784717in}{4.360661in}}%
\pgfpathlineto{\pgfqpoint{1.784717in}{4.448396in}}%
\pgfusepath{stroke,fill}%
\end{pgfscope}%
\begin{pgfscope}%
\pgfpathrectangle{\pgfqpoint{0.380943in}{4.185189in}}{\pgfqpoint{4.650000in}{0.614151in}}%
\pgfusepath{clip}%
\pgfsetbuttcap%
\pgfsetroundjoin%
\definecolor{currentfill}{rgb}{0.979654,0.837186,0.669619}%
\pgfsetfillcolor{currentfill}%
\pgfsetlinewidth{0.250937pt}%
\definecolor{currentstroke}{rgb}{1.000000,1.000000,1.000000}%
\pgfsetstrokecolor{currentstroke}%
\pgfsetdash{}{0pt}%
\pgfpathmoveto{\pgfqpoint{1.872452in}{4.448396in}}%
\pgfpathlineto{\pgfqpoint{1.960188in}{4.448396in}}%
\pgfpathlineto{\pgfqpoint{1.960188in}{4.360661in}}%
\pgfpathlineto{\pgfqpoint{1.872452in}{4.360661in}}%
\pgfpathlineto{\pgfqpoint{1.872452in}{4.448396in}}%
\pgfusepath{stroke,fill}%
\end{pgfscope}%
\begin{pgfscope}%
\pgfpathrectangle{\pgfqpoint{0.380943in}{4.185189in}}{\pgfqpoint{4.650000in}{0.614151in}}%
\pgfusepath{clip}%
\pgfsetbuttcap%
\pgfsetroundjoin%
\definecolor{currentfill}{rgb}{0.986759,0.806398,0.641200}%
\pgfsetfillcolor{currentfill}%
\pgfsetlinewidth{0.250937pt}%
\definecolor{currentstroke}{rgb}{1.000000,1.000000,1.000000}%
\pgfsetstrokecolor{currentstroke}%
\pgfsetdash{}{0pt}%
\pgfpathmoveto{\pgfqpoint{1.960188in}{4.448396in}}%
\pgfpathlineto{\pgfqpoint{2.047924in}{4.448396in}}%
\pgfpathlineto{\pgfqpoint{2.047924in}{4.360661in}}%
\pgfpathlineto{\pgfqpoint{1.960188in}{4.360661in}}%
\pgfpathlineto{\pgfqpoint{1.960188in}{4.448396in}}%
\pgfusepath{stroke,fill}%
\end{pgfscope}%
\begin{pgfscope}%
\pgfpathrectangle{\pgfqpoint{0.380943in}{4.185189in}}{\pgfqpoint{4.650000in}{0.614151in}}%
\pgfusepath{clip}%
\pgfsetbuttcap%
\pgfsetroundjoin%
\definecolor{currentfill}{rgb}{0.996571,0.720538,0.589189}%
\pgfsetfillcolor{currentfill}%
\pgfsetlinewidth{0.250937pt}%
\definecolor{currentstroke}{rgb}{1.000000,1.000000,1.000000}%
\pgfsetstrokecolor{currentstroke}%
\pgfsetdash{}{0pt}%
\pgfpathmoveto{\pgfqpoint{2.047924in}{4.448396in}}%
\pgfpathlineto{\pgfqpoint{2.135660in}{4.448396in}}%
\pgfpathlineto{\pgfqpoint{2.135660in}{4.360661in}}%
\pgfpathlineto{\pgfqpoint{2.047924in}{4.360661in}}%
\pgfpathlineto{\pgfqpoint{2.047924in}{4.448396in}}%
\pgfusepath{stroke,fill}%
\end{pgfscope}%
\begin{pgfscope}%
\pgfpathrectangle{\pgfqpoint{0.380943in}{4.185189in}}{\pgfqpoint{4.650000in}{0.614151in}}%
\pgfusepath{clip}%
\pgfsetbuttcap%
\pgfsetroundjoin%
\definecolor{currentfill}{rgb}{0.962414,0.923552,0.722891}%
\pgfsetfillcolor{currentfill}%
\pgfsetlinewidth{0.250937pt}%
\definecolor{currentstroke}{rgb}{1.000000,1.000000,1.000000}%
\pgfsetstrokecolor{currentstroke}%
\pgfsetdash{}{0pt}%
\pgfpathmoveto{\pgfqpoint{2.135660in}{4.448396in}}%
\pgfpathlineto{\pgfqpoint{2.223396in}{4.448396in}}%
\pgfpathlineto{\pgfqpoint{2.223396in}{4.360661in}}%
\pgfpathlineto{\pgfqpoint{2.135660in}{4.360661in}}%
\pgfpathlineto{\pgfqpoint{2.135660in}{4.448396in}}%
\pgfusepath{stroke,fill}%
\end{pgfscope}%
\begin{pgfscope}%
\pgfpathrectangle{\pgfqpoint{0.380943in}{4.185189in}}{\pgfqpoint{4.650000in}{0.614151in}}%
\pgfusepath{clip}%
\pgfsetbuttcap%
\pgfsetroundjoin%
\definecolor{currentfill}{rgb}{0.972549,0.870588,0.692810}%
\pgfsetfillcolor{currentfill}%
\pgfsetlinewidth{0.250937pt}%
\definecolor{currentstroke}{rgb}{1.000000,1.000000,1.000000}%
\pgfsetstrokecolor{currentstroke}%
\pgfsetdash{}{0pt}%
\pgfpathmoveto{\pgfqpoint{2.223396in}{4.448396in}}%
\pgfpathlineto{\pgfqpoint{2.311132in}{4.448396in}}%
\pgfpathlineto{\pgfqpoint{2.311132in}{4.360661in}}%
\pgfpathlineto{\pgfqpoint{2.223396in}{4.360661in}}%
\pgfpathlineto{\pgfqpoint{2.223396in}{4.448396in}}%
\pgfusepath{stroke,fill}%
\end{pgfscope}%
\begin{pgfscope}%
\pgfpathrectangle{\pgfqpoint{0.380943in}{4.185189in}}{\pgfqpoint{4.650000in}{0.614151in}}%
\pgfusepath{clip}%
\pgfsetbuttcap%
\pgfsetroundjoin%
\definecolor{currentfill}{rgb}{0.986759,0.806398,0.641200}%
\pgfsetfillcolor{currentfill}%
\pgfsetlinewidth{0.250937pt}%
\definecolor{currentstroke}{rgb}{1.000000,1.000000,1.000000}%
\pgfsetstrokecolor{currentstroke}%
\pgfsetdash{}{0pt}%
\pgfpathmoveto{\pgfqpoint{2.311132in}{4.448396in}}%
\pgfpathlineto{\pgfqpoint{2.398868in}{4.448396in}}%
\pgfpathlineto{\pgfqpoint{2.398868in}{4.360661in}}%
\pgfpathlineto{\pgfqpoint{2.311132in}{4.360661in}}%
\pgfpathlineto{\pgfqpoint{2.311132in}{4.448396in}}%
\pgfusepath{stroke,fill}%
\end{pgfscope}%
\begin{pgfscope}%
\pgfpathrectangle{\pgfqpoint{0.380943in}{4.185189in}}{\pgfqpoint{4.650000in}{0.614151in}}%
\pgfusepath{clip}%
\pgfsetbuttcap%
\pgfsetroundjoin%
\definecolor{currentfill}{rgb}{0.965444,0.906113,0.711757}%
\pgfsetfillcolor{currentfill}%
\pgfsetlinewidth{0.250937pt}%
\definecolor{currentstroke}{rgb}{1.000000,1.000000,1.000000}%
\pgfsetstrokecolor{currentstroke}%
\pgfsetdash{}{0pt}%
\pgfpathmoveto{\pgfqpoint{2.398868in}{4.448396in}}%
\pgfpathlineto{\pgfqpoint{2.486603in}{4.448396in}}%
\pgfpathlineto{\pgfqpoint{2.486603in}{4.360661in}}%
\pgfpathlineto{\pgfqpoint{2.398868in}{4.360661in}}%
\pgfpathlineto{\pgfqpoint{2.398868in}{4.448396in}}%
\pgfusepath{stroke,fill}%
\end{pgfscope}%
\begin{pgfscope}%
\pgfpathrectangle{\pgfqpoint{0.380943in}{4.185189in}}{\pgfqpoint{4.650000in}{0.614151in}}%
\pgfusepath{clip}%
\pgfsetbuttcap%
\pgfsetroundjoin%
\definecolor{currentfill}{rgb}{0.979654,0.837186,0.669619}%
\pgfsetfillcolor{currentfill}%
\pgfsetlinewidth{0.250937pt}%
\definecolor{currentstroke}{rgb}{1.000000,1.000000,1.000000}%
\pgfsetstrokecolor{currentstroke}%
\pgfsetdash{}{0pt}%
\pgfpathmoveto{\pgfqpoint{2.486603in}{4.448396in}}%
\pgfpathlineto{\pgfqpoint{2.574339in}{4.448396in}}%
\pgfpathlineto{\pgfqpoint{2.574339in}{4.360661in}}%
\pgfpathlineto{\pgfqpoint{2.486603in}{4.360661in}}%
\pgfpathlineto{\pgfqpoint{2.486603in}{4.448396in}}%
\pgfusepath{stroke,fill}%
\end{pgfscope}%
\begin{pgfscope}%
\pgfpathrectangle{\pgfqpoint{0.380943in}{4.185189in}}{\pgfqpoint{4.650000in}{0.614151in}}%
\pgfusepath{clip}%
\pgfsetbuttcap%
\pgfsetroundjoin%
\definecolor{currentfill}{rgb}{0.972549,0.870588,0.692810}%
\pgfsetfillcolor{currentfill}%
\pgfsetlinewidth{0.250937pt}%
\definecolor{currentstroke}{rgb}{1.000000,1.000000,1.000000}%
\pgfsetstrokecolor{currentstroke}%
\pgfsetdash{}{0pt}%
\pgfpathmoveto{\pgfqpoint{2.574339in}{4.448396in}}%
\pgfpathlineto{\pgfqpoint{2.662075in}{4.448396in}}%
\pgfpathlineto{\pgfqpoint{2.662075in}{4.360661in}}%
\pgfpathlineto{\pgfqpoint{2.574339in}{4.360661in}}%
\pgfpathlineto{\pgfqpoint{2.574339in}{4.448396in}}%
\pgfusepath{stroke,fill}%
\end{pgfscope}%
\begin{pgfscope}%
\pgfpathrectangle{\pgfqpoint{0.380943in}{4.185189in}}{\pgfqpoint{4.650000in}{0.614151in}}%
\pgfusepath{clip}%
\pgfsetbuttcap%
\pgfsetroundjoin%
\definecolor{currentfill}{rgb}{0.962414,0.923552,0.722891}%
\pgfsetfillcolor{currentfill}%
\pgfsetlinewidth{0.250937pt}%
\definecolor{currentstroke}{rgb}{1.000000,1.000000,1.000000}%
\pgfsetstrokecolor{currentstroke}%
\pgfsetdash{}{0pt}%
\pgfpathmoveto{\pgfqpoint{2.662075in}{4.448396in}}%
\pgfpathlineto{\pgfqpoint{2.749811in}{4.448396in}}%
\pgfpathlineto{\pgfqpoint{2.749811in}{4.360661in}}%
\pgfpathlineto{\pgfqpoint{2.662075in}{4.360661in}}%
\pgfpathlineto{\pgfqpoint{2.662075in}{4.448396in}}%
\pgfusepath{stroke,fill}%
\end{pgfscope}%
\begin{pgfscope}%
\pgfpathrectangle{\pgfqpoint{0.380943in}{4.185189in}}{\pgfqpoint{4.650000in}{0.614151in}}%
\pgfusepath{clip}%
\pgfsetbuttcap%
\pgfsetroundjoin%
\definecolor{currentfill}{rgb}{0.996571,0.720538,0.589189}%
\pgfsetfillcolor{currentfill}%
\pgfsetlinewidth{0.250937pt}%
\definecolor{currentstroke}{rgb}{1.000000,1.000000,1.000000}%
\pgfsetstrokecolor{currentstroke}%
\pgfsetdash{}{0pt}%
\pgfpathmoveto{\pgfqpoint{2.749811in}{4.448396in}}%
\pgfpathlineto{\pgfqpoint{2.837547in}{4.448396in}}%
\pgfpathlineto{\pgfqpoint{2.837547in}{4.360661in}}%
\pgfpathlineto{\pgfqpoint{2.749811in}{4.360661in}}%
\pgfpathlineto{\pgfqpoint{2.749811in}{4.448396in}}%
\pgfusepath{stroke,fill}%
\end{pgfscope}%
\begin{pgfscope}%
\pgfpathrectangle{\pgfqpoint{0.380943in}{4.185189in}}{\pgfqpoint{4.650000in}{0.614151in}}%
\pgfusepath{clip}%
\pgfsetbuttcap%
\pgfsetroundjoin%
\definecolor{currentfill}{rgb}{0.962414,0.923552,0.722891}%
\pgfsetfillcolor{currentfill}%
\pgfsetlinewidth{0.250937pt}%
\definecolor{currentstroke}{rgb}{1.000000,1.000000,1.000000}%
\pgfsetstrokecolor{currentstroke}%
\pgfsetdash{}{0pt}%
\pgfpathmoveto{\pgfqpoint{2.837547in}{4.448396in}}%
\pgfpathlineto{\pgfqpoint{2.925283in}{4.448396in}}%
\pgfpathlineto{\pgfqpoint{2.925283in}{4.360661in}}%
\pgfpathlineto{\pgfqpoint{2.837547in}{4.360661in}}%
\pgfpathlineto{\pgfqpoint{2.837547in}{4.448396in}}%
\pgfusepath{stroke,fill}%
\end{pgfscope}%
\begin{pgfscope}%
\pgfpathrectangle{\pgfqpoint{0.380943in}{4.185189in}}{\pgfqpoint{4.650000in}{0.614151in}}%
\pgfusepath{clip}%
\pgfsetbuttcap%
\pgfsetroundjoin%
\definecolor{currentfill}{rgb}{0.965444,0.906113,0.711757}%
\pgfsetfillcolor{currentfill}%
\pgfsetlinewidth{0.250937pt}%
\definecolor{currentstroke}{rgb}{1.000000,1.000000,1.000000}%
\pgfsetstrokecolor{currentstroke}%
\pgfsetdash{}{0pt}%
\pgfpathmoveto{\pgfqpoint{2.925283in}{4.448396in}}%
\pgfpathlineto{\pgfqpoint{3.013019in}{4.448396in}}%
\pgfpathlineto{\pgfqpoint{3.013019in}{4.360661in}}%
\pgfpathlineto{\pgfqpoint{2.925283in}{4.360661in}}%
\pgfpathlineto{\pgfqpoint{2.925283in}{4.448396in}}%
\pgfusepath{stroke,fill}%
\end{pgfscope}%
\begin{pgfscope}%
\pgfpathrectangle{\pgfqpoint{0.380943in}{4.185189in}}{\pgfqpoint{4.650000in}{0.614151in}}%
\pgfusepath{clip}%
\pgfsetbuttcap%
\pgfsetroundjoin%
\definecolor{currentfill}{rgb}{0.965444,0.906113,0.711757}%
\pgfsetfillcolor{currentfill}%
\pgfsetlinewidth{0.250937pt}%
\definecolor{currentstroke}{rgb}{1.000000,1.000000,1.000000}%
\pgfsetstrokecolor{currentstroke}%
\pgfsetdash{}{0pt}%
\pgfpathmoveto{\pgfqpoint{3.013019in}{4.448396in}}%
\pgfpathlineto{\pgfqpoint{3.100754in}{4.448396in}}%
\pgfpathlineto{\pgfqpoint{3.100754in}{4.360661in}}%
\pgfpathlineto{\pgfqpoint{3.013019in}{4.360661in}}%
\pgfpathlineto{\pgfqpoint{3.013019in}{4.448396in}}%
\pgfusepath{stroke,fill}%
\end{pgfscope}%
\begin{pgfscope}%
\pgfpathrectangle{\pgfqpoint{0.380943in}{4.185189in}}{\pgfqpoint{4.650000in}{0.614151in}}%
\pgfusepath{clip}%
\pgfsetbuttcap%
\pgfsetroundjoin%
\definecolor{currentfill}{rgb}{0.996571,0.720538,0.589189}%
\pgfsetfillcolor{currentfill}%
\pgfsetlinewidth{0.250937pt}%
\definecolor{currentstroke}{rgb}{1.000000,1.000000,1.000000}%
\pgfsetstrokecolor{currentstroke}%
\pgfsetdash{}{0pt}%
\pgfpathmoveto{\pgfqpoint{3.100754in}{4.448396in}}%
\pgfpathlineto{\pgfqpoint{3.188490in}{4.448396in}}%
\pgfpathlineto{\pgfqpoint{3.188490in}{4.360661in}}%
\pgfpathlineto{\pgfqpoint{3.100754in}{4.360661in}}%
\pgfpathlineto{\pgfqpoint{3.100754in}{4.448396in}}%
\pgfusepath{stroke,fill}%
\end{pgfscope}%
\begin{pgfscope}%
\pgfpathrectangle{\pgfqpoint{0.380943in}{4.185189in}}{\pgfqpoint{4.650000in}{0.614151in}}%
\pgfusepath{clip}%
\pgfsetbuttcap%
\pgfsetroundjoin%
\definecolor{currentfill}{rgb}{0.979654,0.837186,0.669619}%
\pgfsetfillcolor{currentfill}%
\pgfsetlinewidth{0.250937pt}%
\definecolor{currentstroke}{rgb}{1.000000,1.000000,1.000000}%
\pgfsetstrokecolor{currentstroke}%
\pgfsetdash{}{0pt}%
\pgfpathmoveto{\pgfqpoint{3.188490in}{4.448396in}}%
\pgfpathlineto{\pgfqpoint{3.276226in}{4.448396in}}%
\pgfpathlineto{\pgfqpoint{3.276226in}{4.360661in}}%
\pgfpathlineto{\pgfqpoint{3.188490in}{4.360661in}}%
\pgfpathlineto{\pgfqpoint{3.188490in}{4.448396in}}%
\pgfusepath{stroke,fill}%
\end{pgfscope}%
\begin{pgfscope}%
\pgfpathrectangle{\pgfqpoint{0.380943in}{4.185189in}}{\pgfqpoint{4.650000in}{0.614151in}}%
\pgfusepath{clip}%
\pgfsetbuttcap%
\pgfsetroundjoin%
\definecolor{currentfill}{rgb}{0.979654,0.837186,0.669619}%
\pgfsetfillcolor{currentfill}%
\pgfsetlinewidth{0.250937pt}%
\definecolor{currentstroke}{rgb}{1.000000,1.000000,1.000000}%
\pgfsetstrokecolor{currentstroke}%
\pgfsetdash{}{0pt}%
\pgfpathmoveto{\pgfqpoint{3.276226in}{4.448396in}}%
\pgfpathlineto{\pgfqpoint{3.363962in}{4.448396in}}%
\pgfpathlineto{\pgfqpoint{3.363962in}{4.360661in}}%
\pgfpathlineto{\pgfqpoint{3.276226in}{4.360661in}}%
\pgfpathlineto{\pgfqpoint{3.276226in}{4.448396in}}%
\pgfusepath{stroke,fill}%
\end{pgfscope}%
\begin{pgfscope}%
\pgfpathrectangle{\pgfqpoint{0.380943in}{4.185189in}}{\pgfqpoint{4.650000in}{0.614151in}}%
\pgfusepath{clip}%
\pgfsetbuttcap%
\pgfsetroundjoin%
\definecolor{currentfill}{rgb}{0.992326,0.765229,0.614840}%
\pgfsetfillcolor{currentfill}%
\pgfsetlinewidth{0.250937pt}%
\definecolor{currentstroke}{rgb}{1.000000,1.000000,1.000000}%
\pgfsetstrokecolor{currentstroke}%
\pgfsetdash{}{0pt}%
\pgfpathmoveto{\pgfqpoint{3.363962in}{4.448396in}}%
\pgfpathlineto{\pgfqpoint{3.451698in}{4.448396in}}%
\pgfpathlineto{\pgfqpoint{3.451698in}{4.360661in}}%
\pgfpathlineto{\pgfqpoint{3.363962in}{4.360661in}}%
\pgfpathlineto{\pgfqpoint{3.363962in}{4.448396in}}%
\pgfusepath{stroke,fill}%
\end{pgfscope}%
\begin{pgfscope}%
\pgfpathrectangle{\pgfqpoint{0.380943in}{4.185189in}}{\pgfqpoint{4.650000in}{0.614151in}}%
\pgfusepath{clip}%
\pgfsetbuttcap%
\pgfsetroundjoin%
\definecolor{currentfill}{rgb}{0.996571,0.720538,0.589189}%
\pgfsetfillcolor{currentfill}%
\pgfsetlinewidth{0.250937pt}%
\definecolor{currentstroke}{rgb}{1.000000,1.000000,1.000000}%
\pgfsetstrokecolor{currentstroke}%
\pgfsetdash{}{0pt}%
\pgfpathmoveto{\pgfqpoint{3.451698in}{4.448396in}}%
\pgfpathlineto{\pgfqpoint{3.539434in}{4.448396in}}%
\pgfpathlineto{\pgfqpoint{3.539434in}{4.360661in}}%
\pgfpathlineto{\pgfqpoint{3.451698in}{4.360661in}}%
\pgfpathlineto{\pgfqpoint{3.451698in}{4.448396in}}%
\pgfusepath{stroke,fill}%
\end{pgfscope}%
\begin{pgfscope}%
\pgfpathrectangle{\pgfqpoint{0.380943in}{4.185189in}}{\pgfqpoint{4.650000in}{0.614151in}}%
\pgfusepath{clip}%
\pgfsetbuttcap%
\pgfsetroundjoin%
\definecolor{currentfill}{rgb}{0.972549,0.870588,0.692810}%
\pgfsetfillcolor{currentfill}%
\pgfsetlinewidth{0.250937pt}%
\definecolor{currentstroke}{rgb}{1.000000,1.000000,1.000000}%
\pgfsetstrokecolor{currentstroke}%
\pgfsetdash{}{0pt}%
\pgfpathmoveto{\pgfqpoint{3.539434in}{4.448396in}}%
\pgfpathlineto{\pgfqpoint{3.627169in}{4.448396in}}%
\pgfpathlineto{\pgfqpoint{3.627169in}{4.360661in}}%
\pgfpathlineto{\pgfqpoint{3.539434in}{4.360661in}}%
\pgfpathlineto{\pgfqpoint{3.539434in}{4.448396in}}%
\pgfusepath{stroke,fill}%
\end{pgfscope}%
\begin{pgfscope}%
\pgfpathrectangle{\pgfqpoint{0.380943in}{4.185189in}}{\pgfqpoint{4.650000in}{0.614151in}}%
\pgfusepath{clip}%
\pgfsetbuttcap%
\pgfsetroundjoin%
\definecolor{currentfill}{rgb}{0.972549,0.870588,0.692810}%
\pgfsetfillcolor{currentfill}%
\pgfsetlinewidth{0.250937pt}%
\definecolor{currentstroke}{rgb}{1.000000,1.000000,1.000000}%
\pgfsetstrokecolor{currentstroke}%
\pgfsetdash{}{0pt}%
\pgfpathmoveto{\pgfqpoint{3.627169in}{4.448396in}}%
\pgfpathlineto{\pgfqpoint{3.714905in}{4.448396in}}%
\pgfpathlineto{\pgfqpoint{3.714905in}{4.360661in}}%
\pgfpathlineto{\pgfqpoint{3.627169in}{4.360661in}}%
\pgfpathlineto{\pgfqpoint{3.627169in}{4.448396in}}%
\pgfusepath{stroke,fill}%
\end{pgfscope}%
\begin{pgfscope}%
\pgfpathrectangle{\pgfqpoint{0.380943in}{4.185189in}}{\pgfqpoint{4.650000in}{0.614151in}}%
\pgfusepath{clip}%
\pgfsetbuttcap%
\pgfsetroundjoin%
\definecolor{currentfill}{rgb}{1.000000,0.509404,0.491473}%
\pgfsetfillcolor{currentfill}%
\pgfsetlinewidth{0.250937pt}%
\definecolor{currentstroke}{rgb}{1.000000,1.000000,1.000000}%
\pgfsetstrokecolor{currentstroke}%
\pgfsetdash{}{0pt}%
\pgfpathmoveto{\pgfqpoint{3.714905in}{4.448396in}}%
\pgfpathlineto{\pgfqpoint{3.802641in}{4.448396in}}%
\pgfpathlineto{\pgfqpoint{3.802641in}{4.360661in}}%
\pgfpathlineto{\pgfqpoint{3.714905in}{4.360661in}}%
\pgfpathlineto{\pgfqpoint{3.714905in}{4.448396in}}%
\pgfusepath{stroke,fill}%
\end{pgfscope}%
\begin{pgfscope}%
\pgfpathrectangle{\pgfqpoint{0.380943in}{4.185189in}}{\pgfqpoint{4.650000in}{0.614151in}}%
\pgfusepath{clip}%
\pgfsetbuttcap%
\pgfsetroundjoin%
\definecolor{currentfill}{rgb}{0.972549,0.870588,0.692810}%
\pgfsetfillcolor{currentfill}%
\pgfsetlinewidth{0.250937pt}%
\definecolor{currentstroke}{rgb}{1.000000,1.000000,1.000000}%
\pgfsetstrokecolor{currentstroke}%
\pgfsetdash{}{0pt}%
\pgfpathmoveto{\pgfqpoint{3.802641in}{4.448396in}}%
\pgfpathlineto{\pgfqpoint{3.890377in}{4.448396in}}%
\pgfpathlineto{\pgfqpoint{3.890377in}{4.360661in}}%
\pgfpathlineto{\pgfqpoint{3.802641in}{4.360661in}}%
\pgfpathlineto{\pgfqpoint{3.802641in}{4.448396in}}%
\pgfusepath{stroke,fill}%
\end{pgfscope}%
\begin{pgfscope}%
\pgfpathrectangle{\pgfqpoint{0.380943in}{4.185189in}}{\pgfqpoint{4.650000in}{0.614151in}}%
\pgfusepath{clip}%
\pgfsetbuttcap%
\pgfsetroundjoin%
\definecolor{currentfill}{rgb}{0.972549,0.870588,0.692810}%
\pgfsetfillcolor{currentfill}%
\pgfsetlinewidth{0.250937pt}%
\definecolor{currentstroke}{rgb}{1.000000,1.000000,1.000000}%
\pgfsetstrokecolor{currentstroke}%
\pgfsetdash{}{0pt}%
\pgfpathmoveto{\pgfqpoint{3.890377in}{4.448396in}}%
\pgfpathlineto{\pgfqpoint{3.978113in}{4.448396in}}%
\pgfpathlineto{\pgfqpoint{3.978113in}{4.360661in}}%
\pgfpathlineto{\pgfqpoint{3.890377in}{4.360661in}}%
\pgfpathlineto{\pgfqpoint{3.890377in}{4.448396in}}%
\pgfusepath{stroke,fill}%
\end{pgfscope}%
\begin{pgfscope}%
\pgfpathrectangle{\pgfqpoint{0.380943in}{4.185189in}}{\pgfqpoint{4.650000in}{0.614151in}}%
\pgfusepath{clip}%
\pgfsetbuttcap%
\pgfsetroundjoin%
\definecolor{currentfill}{rgb}{0.861576,0.340008,0.340008}%
\pgfsetfillcolor{currentfill}%
\pgfsetlinewidth{0.250937pt}%
\definecolor{currentstroke}{rgb}{1.000000,1.000000,1.000000}%
\pgfsetstrokecolor{currentstroke}%
\pgfsetdash{}{0pt}%
\pgfpathmoveto{\pgfqpoint{3.978113in}{4.448396in}}%
\pgfpathlineto{\pgfqpoint{4.065849in}{4.448396in}}%
\pgfpathlineto{\pgfqpoint{4.065849in}{4.360661in}}%
\pgfpathlineto{\pgfqpoint{3.978113in}{4.360661in}}%
\pgfpathlineto{\pgfqpoint{3.978113in}{4.448396in}}%
\pgfusepath{stroke,fill}%
\end{pgfscope}%
\begin{pgfscope}%
\pgfpathrectangle{\pgfqpoint{0.380943in}{4.185189in}}{\pgfqpoint{4.650000in}{0.614151in}}%
\pgfusepath{clip}%
\pgfsetbuttcap%
\pgfsetroundjoin%
\definecolor{currentfill}{rgb}{0.962414,0.923552,0.722891}%
\pgfsetfillcolor{currentfill}%
\pgfsetlinewidth{0.250937pt}%
\definecolor{currentstroke}{rgb}{1.000000,1.000000,1.000000}%
\pgfsetstrokecolor{currentstroke}%
\pgfsetdash{}{0pt}%
\pgfpathmoveto{\pgfqpoint{4.065849in}{4.448396in}}%
\pgfpathlineto{\pgfqpoint{4.153585in}{4.448396in}}%
\pgfpathlineto{\pgfqpoint{4.153585in}{4.360661in}}%
\pgfpathlineto{\pgfqpoint{4.065849in}{4.360661in}}%
\pgfpathlineto{\pgfqpoint{4.065849in}{4.448396in}}%
\pgfusepath{stroke,fill}%
\end{pgfscope}%
\begin{pgfscope}%
\pgfpathrectangle{\pgfqpoint{0.380943in}{4.185189in}}{\pgfqpoint{4.650000in}{0.614151in}}%
\pgfusepath{clip}%
\pgfsetbuttcap%
\pgfsetroundjoin%
\definecolor{currentfill}{rgb}{0.968166,0.945882,0.748604}%
\pgfsetfillcolor{currentfill}%
\pgfsetlinewidth{0.250937pt}%
\definecolor{currentstroke}{rgb}{1.000000,1.000000,1.000000}%
\pgfsetstrokecolor{currentstroke}%
\pgfsetdash{}{0pt}%
\pgfpathmoveto{\pgfqpoint{4.153585in}{4.448396in}}%
\pgfpathlineto{\pgfqpoint{4.241320in}{4.448396in}}%
\pgfpathlineto{\pgfqpoint{4.241320in}{4.360661in}}%
\pgfpathlineto{\pgfqpoint{4.153585in}{4.360661in}}%
\pgfpathlineto{\pgfqpoint{4.153585in}{4.448396in}}%
\pgfusepath{stroke,fill}%
\end{pgfscope}%
\begin{pgfscope}%
\pgfpathrectangle{\pgfqpoint{0.380943in}{4.185189in}}{\pgfqpoint{4.650000in}{0.614151in}}%
\pgfusepath{clip}%
\pgfsetbuttcap%
\pgfsetroundjoin%
\definecolor{currentfill}{rgb}{0.968166,0.945882,0.748604}%
\pgfsetfillcolor{currentfill}%
\pgfsetlinewidth{0.250937pt}%
\definecolor{currentstroke}{rgb}{1.000000,1.000000,1.000000}%
\pgfsetstrokecolor{currentstroke}%
\pgfsetdash{}{0pt}%
\pgfpathmoveto{\pgfqpoint{4.241320in}{4.448396in}}%
\pgfpathlineto{\pgfqpoint{4.329056in}{4.448396in}}%
\pgfpathlineto{\pgfqpoint{4.329056in}{4.360661in}}%
\pgfpathlineto{\pgfqpoint{4.241320in}{4.360661in}}%
\pgfpathlineto{\pgfqpoint{4.241320in}{4.448396in}}%
\pgfusepath{stroke,fill}%
\end{pgfscope}%
\begin{pgfscope}%
\pgfpathrectangle{\pgfqpoint{0.380943in}{4.185189in}}{\pgfqpoint{4.650000in}{0.614151in}}%
\pgfusepath{clip}%
\pgfsetbuttcap%
\pgfsetroundjoin%
\definecolor{currentfill}{rgb}{0.962414,0.923552,0.722891}%
\pgfsetfillcolor{currentfill}%
\pgfsetlinewidth{0.250937pt}%
\definecolor{currentstroke}{rgb}{1.000000,1.000000,1.000000}%
\pgfsetstrokecolor{currentstroke}%
\pgfsetdash{}{0pt}%
\pgfpathmoveto{\pgfqpoint{4.329056in}{4.448396in}}%
\pgfpathlineto{\pgfqpoint{4.416792in}{4.448396in}}%
\pgfpathlineto{\pgfqpoint{4.416792in}{4.360661in}}%
\pgfpathlineto{\pgfqpoint{4.329056in}{4.360661in}}%
\pgfpathlineto{\pgfqpoint{4.329056in}{4.448396in}}%
\pgfusepath{stroke,fill}%
\end{pgfscope}%
\begin{pgfscope}%
\pgfpathrectangle{\pgfqpoint{0.380943in}{4.185189in}}{\pgfqpoint{4.650000in}{0.614151in}}%
\pgfusepath{clip}%
\pgfsetbuttcap%
\pgfsetroundjoin%
\definecolor{currentfill}{rgb}{0.979654,0.837186,0.669619}%
\pgfsetfillcolor{currentfill}%
\pgfsetlinewidth{0.250937pt}%
\definecolor{currentstroke}{rgb}{1.000000,1.000000,1.000000}%
\pgfsetstrokecolor{currentstroke}%
\pgfsetdash{}{0pt}%
\pgfpathmoveto{\pgfqpoint{4.416792in}{4.448396in}}%
\pgfpathlineto{\pgfqpoint{4.504528in}{4.448396in}}%
\pgfpathlineto{\pgfqpoint{4.504528in}{4.360661in}}%
\pgfpathlineto{\pgfqpoint{4.416792in}{4.360661in}}%
\pgfpathlineto{\pgfqpoint{4.416792in}{4.448396in}}%
\pgfusepath{stroke,fill}%
\end{pgfscope}%
\begin{pgfscope}%
\pgfpathrectangle{\pgfqpoint{0.380943in}{4.185189in}}{\pgfqpoint{4.650000in}{0.614151in}}%
\pgfusepath{clip}%
\pgfsetbuttcap%
\pgfsetroundjoin%
\definecolor{currentfill}{rgb}{0.972549,0.870588,0.692810}%
\pgfsetfillcolor{currentfill}%
\pgfsetlinewidth{0.250937pt}%
\definecolor{currentstroke}{rgb}{1.000000,1.000000,1.000000}%
\pgfsetstrokecolor{currentstroke}%
\pgfsetdash{}{0pt}%
\pgfpathmoveto{\pgfqpoint{4.504528in}{4.448396in}}%
\pgfpathlineto{\pgfqpoint{4.592264in}{4.448396in}}%
\pgfpathlineto{\pgfqpoint{4.592264in}{4.360661in}}%
\pgfpathlineto{\pgfqpoint{4.504528in}{4.360661in}}%
\pgfpathlineto{\pgfqpoint{4.504528in}{4.448396in}}%
\pgfusepath{stroke,fill}%
\end{pgfscope}%
\begin{pgfscope}%
\pgfpathrectangle{\pgfqpoint{0.380943in}{4.185189in}}{\pgfqpoint{4.650000in}{0.614151in}}%
\pgfusepath{clip}%
\pgfsetbuttcap%
\pgfsetroundjoin%
\definecolor{currentfill}{rgb}{1.000000,0.605229,0.530719}%
\pgfsetfillcolor{currentfill}%
\pgfsetlinewidth{0.250937pt}%
\definecolor{currentstroke}{rgb}{1.000000,1.000000,1.000000}%
\pgfsetstrokecolor{currentstroke}%
\pgfsetdash{}{0pt}%
\pgfpathmoveto{\pgfqpoint{4.592264in}{4.448396in}}%
\pgfpathlineto{\pgfqpoint{4.680000in}{4.448396in}}%
\pgfpathlineto{\pgfqpoint{4.680000in}{4.360661in}}%
\pgfpathlineto{\pgfqpoint{4.592264in}{4.360661in}}%
\pgfpathlineto{\pgfqpoint{4.592264in}{4.448396in}}%
\pgfusepath{stroke,fill}%
\end{pgfscope}%
\begin{pgfscope}%
\pgfpathrectangle{\pgfqpoint{0.380943in}{4.185189in}}{\pgfqpoint{4.650000in}{0.614151in}}%
\pgfusepath{clip}%
\pgfsetbuttcap%
\pgfsetroundjoin%
\definecolor{currentfill}{rgb}{0.962414,0.923552,0.722891}%
\pgfsetfillcolor{currentfill}%
\pgfsetlinewidth{0.250937pt}%
\definecolor{currentstroke}{rgb}{1.000000,1.000000,1.000000}%
\pgfsetstrokecolor{currentstroke}%
\pgfsetdash{}{0pt}%
\pgfpathmoveto{\pgfqpoint{4.680000in}{4.448396in}}%
\pgfpathlineto{\pgfqpoint{4.767736in}{4.448396in}}%
\pgfpathlineto{\pgfqpoint{4.767736in}{4.360661in}}%
\pgfpathlineto{\pgfqpoint{4.680000in}{4.360661in}}%
\pgfpathlineto{\pgfqpoint{4.680000in}{4.448396in}}%
\pgfusepath{stroke,fill}%
\end{pgfscope}%
\begin{pgfscope}%
\pgfpathrectangle{\pgfqpoint{0.380943in}{4.185189in}}{\pgfqpoint{4.650000in}{0.614151in}}%
\pgfusepath{clip}%
\pgfsetbuttcap%
\pgfsetroundjoin%
\definecolor{currentfill}{rgb}{0.992326,0.765229,0.614840}%
\pgfsetfillcolor{currentfill}%
\pgfsetlinewidth{0.250937pt}%
\definecolor{currentstroke}{rgb}{1.000000,1.000000,1.000000}%
\pgfsetstrokecolor{currentstroke}%
\pgfsetdash{}{0pt}%
\pgfpathmoveto{\pgfqpoint{4.767736in}{4.448396in}}%
\pgfpathlineto{\pgfqpoint{4.855471in}{4.448396in}}%
\pgfpathlineto{\pgfqpoint{4.855471in}{4.360661in}}%
\pgfpathlineto{\pgfqpoint{4.767736in}{4.360661in}}%
\pgfpathlineto{\pgfqpoint{4.767736in}{4.448396in}}%
\pgfusepath{stroke,fill}%
\end{pgfscope}%
\begin{pgfscope}%
\pgfpathrectangle{\pgfqpoint{0.380943in}{4.185189in}}{\pgfqpoint{4.650000in}{0.614151in}}%
\pgfusepath{clip}%
\pgfsetbuttcap%
\pgfsetroundjoin%
\definecolor{currentfill}{rgb}{0.962414,0.923552,0.722891}%
\pgfsetfillcolor{currentfill}%
\pgfsetlinewidth{0.250937pt}%
\definecolor{currentstroke}{rgb}{1.000000,1.000000,1.000000}%
\pgfsetstrokecolor{currentstroke}%
\pgfsetdash{}{0pt}%
\pgfpathmoveto{\pgfqpoint{4.855471in}{4.448396in}}%
\pgfpathlineto{\pgfqpoint{4.943207in}{4.448396in}}%
\pgfpathlineto{\pgfqpoint{4.943207in}{4.360661in}}%
\pgfpathlineto{\pgfqpoint{4.855471in}{4.360661in}}%
\pgfpathlineto{\pgfqpoint{4.855471in}{4.448396in}}%
\pgfusepath{stroke,fill}%
\end{pgfscope}%
\begin{pgfscope}%
\pgfpathrectangle{\pgfqpoint{0.380943in}{4.185189in}}{\pgfqpoint{4.650000in}{0.614151in}}%
\pgfusepath{clip}%
\pgfsetbuttcap%
\pgfsetroundjoin%
\pgfsetlinewidth{0.250937pt}%
\definecolor{currentstroke}{rgb}{1.000000,1.000000,1.000000}%
\pgfsetstrokecolor{currentstroke}%
\pgfsetdash{}{0pt}%
\pgfpathmoveto{\pgfqpoint{4.943207in}{4.448396in}}%
\pgfpathlineto{\pgfqpoint{5.030943in}{4.448396in}}%
\pgfpathlineto{\pgfqpoint{5.030943in}{4.360661in}}%
\pgfpathlineto{\pgfqpoint{4.943207in}{4.360661in}}%
\pgfpathlineto{\pgfqpoint{4.943207in}{4.448396in}}%
\pgfusepath{stroke}%
\end{pgfscope}%
\begin{pgfscope}%
\pgfpathrectangle{\pgfqpoint{0.380943in}{4.185189in}}{\pgfqpoint{4.650000in}{0.614151in}}%
\pgfusepath{clip}%
\pgfsetbuttcap%
\pgfsetroundjoin%
\definecolor{currentfill}{rgb}{0.965444,0.906113,0.711757}%
\pgfsetfillcolor{currentfill}%
\pgfsetlinewidth{0.250937pt}%
\definecolor{currentstroke}{rgb}{1.000000,1.000000,1.000000}%
\pgfsetstrokecolor{currentstroke}%
\pgfsetdash{}{0pt}%
\pgfpathmoveto{\pgfqpoint{0.380943in}{4.360661in}}%
\pgfpathlineto{\pgfqpoint{0.468679in}{4.360661in}}%
\pgfpathlineto{\pgfqpoint{0.468679in}{4.272925in}}%
\pgfpathlineto{\pgfqpoint{0.380943in}{4.272925in}}%
\pgfpathlineto{\pgfqpoint{0.380943in}{4.360661in}}%
\pgfusepath{stroke,fill}%
\end{pgfscope}%
\begin{pgfscope}%
\pgfpathrectangle{\pgfqpoint{0.380943in}{4.185189in}}{\pgfqpoint{4.650000in}{0.614151in}}%
\pgfusepath{clip}%
\pgfsetbuttcap%
\pgfsetroundjoin%
\definecolor{currentfill}{rgb}{0.968166,0.945882,0.748604}%
\pgfsetfillcolor{currentfill}%
\pgfsetlinewidth{0.250937pt}%
\definecolor{currentstroke}{rgb}{1.000000,1.000000,1.000000}%
\pgfsetstrokecolor{currentstroke}%
\pgfsetdash{}{0pt}%
\pgfpathmoveto{\pgfqpoint{0.468679in}{4.360661in}}%
\pgfpathlineto{\pgfqpoint{0.556415in}{4.360661in}}%
\pgfpathlineto{\pgfqpoint{0.556415in}{4.272925in}}%
\pgfpathlineto{\pgfqpoint{0.468679in}{4.272925in}}%
\pgfpathlineto{\pgfqpoint{0.468679in}{4.360661in}}%
\pgfusepath{stroke,fill}%
\end{pgfscope}%
\begin{pgfscope}%
\pgfpathrectangle{\pgfqpoint{0.380943in}{4.185189in}}{\pgfqpoint{4.650000in}{0.614151in}}%
\pgfusepath{clip}%
\pgfsetbuttcap%
\pgfsetroundjoin%
\definecolor{currentfill}{rgb}{0.968166,0.945882,0.748604}%
\pgfsetfillcolor{currentfill}%
\pgfsetlinewidth{0.250937pt}%
\definecolor{currentstroke}{rgb}{1.000000,1.000000,1.000000}%
\pgfsetstrokecolor{currentstroke}%
\pgfsetdash{}{0pt}%
\pgfpathmoveto{\pgfqpoint{0.556415in}{4.360661in}}%
\pgfpathlineto{\pgfqpoint{0.644151in}{4.360661in}}%
\pgfpathlineto{\pgfqpoint{0.644151in}{4.272925in}}%
\pgfpathlineto{\pgfqpoint{0.556415in}{4.272925in}}%
\pgfpathlineto{\pgfqpoint{0.556415in}{4.360661in}}%
\pgfusepath{stroke,fill}%
\end{pgfscope}%
\begin{pgfscope}%
\pgfpathrectangle{\pgfqpoint{0.380943in}{4.185189in}}{\pgfqpoint{4.650000in}{0.614151in}}%
\pgfusepath{clip}%
\pgfsetbuttcap%
\pgfsetroundjoin%
\definecolor{currentfill}{rgb}{1.000000,0.605229,0.530719}%
\pgfsetfillcolor{currentfill}%
\pgfsetlinewidth{0.250937pt}%
\definecolor{currentstroke}{rgb}{1.000000,1.000000,1.000000}%
\pgfsetstrokecolor{currentstroke}%
\pgfsetdash{}{0pt}%
\pgfpathmoveto{\pgfqpoint{0.644151in}{4.360661in}}%
\pgfpathlineto{\pgfqpoint{0.731886in}{4.360661in}}%
\pgfpathlineto{\pgfqpoint{0.731886in}{4.272925in}}%
\pgfpathlineto{\pgfqpoint{0.644151in}{4.272925in}}%
\pgfpathlineto{\pgfqpoint{0.644151in}{4.360661in}}%
\pgfusepath{stroke,fill}%
\end{pgfscope}%
\begin{pgfscope}%
\pgfpathrectangle{\pgfqpoint{0.380943in}{4.185189in}}{\pgfqpoint{4.650000in}{0.614151in}}%
\pgfusepath{clip}%
\pgfsetbuttcap%
\pgfsetroundjoin%
\definecolor{currentfill}{rgb}{0.979654,0.837186,0.669619}%
\pgfsetfillcolor{currentfill}%
\pgfsetlinewidth{0.250937pt}%
\definecolor{currentstroke}{rgb}{1.000000,1.000000,1.000000}%
\pgfsetstrokecolor{currentstroke}%
\pgfsetdash{}{0pt}%
\pgfpathmoveto{\pgfqpoint{0.731886in}{4.360661in}}%
\pgfpathlineto{\pgfqpoint{0.819622in}{4.360661in}}%
\pgfpathlineto{\pgfqpoint{0.819622in}{4.272925in}}%
\pgfpathlineto{\pgfqpoint{0.731886in}{4.272925in}}%
\pgfpathlineto{\pgfqpoint{0.731886in}{4.360661in}}%
\pgfusepath{stroke,fill}%
\end{pgfscope}%
\begin{pgfscope}%
\pgfpathrectangle{\pgfqpoint{0.380943in}{4.185189in}}{\pgfqpoint{4.650000in}{0.614151in}}%
\pgfusepath{clip}%
\pgfsetbuttcap%
\pgfsetroundjoin%
\definecolor{currentfill}{rgb}{0.986759,0.806398,0.641200}%
\pgfsetfillcolor{currentfill}%
\pgfsetlinewidth{0.250937pt}%
\definecolor{currentstroke}{rgb}{1.000000,1.000000,1.000000}%
\pgfsetstrokecolor{currentstroke}%
\pgfsetdash{}{0pt}%
\pgfpathmoveto{\pgfqpoint{0.819622in}{4.360661in}}%
\pgfpathlineto{\pgfqpoint{0.907358in}{4.360661in}}%
\pgfpathlineto{\pgfqpoint{0.907358in}{4.272925in}}%
\pgfpathlineto{\pgfqpoint{0.819622in}{4.272925in}}%
\pgfpathlineto{\pgfqpoint{0.819622in}{4.360661in}}%
\pgfusepath{stroke,fill}%
\end{pgfscope}%
\begin{pgfscope}%
\pgfpathrectangle{\pgfqpoint{0.380943in}{4.185189in}}{\pgfqpoint{4.650000in}{0.614151in}}%
\pgfusepath{clip}%
\pgfsetbuttcap%
\pgfsetroundjoin%
\definecolor{currentfill}{rgb}{1.000000,1.000000,0.870204}%
\pgfsetfillcolor{currentfill}%
\pgfsetlinewidth{0.250937pt}%
\definecolor{currentstroke}{rgb}{1.000000,1.000000,1.000000}%
\pgfsetstrokecolor{currentstroke}%
\pgfsetdash{}{0pt}%
\pgfpathmoveto{\pgfqpoint{0.907358in}{4.360661in}}%
\pgfpathlineto{\pgfqpoint{0.995094in}{4.360661in}}%
\pgfpathlineto{\pgfqpoint{0.995094in}{4.272925in}}%
\pgfpathlineto{\pgfqpoint{0.907358in}{4.272925in}}%
\pgfpathlineto{\pgfqpoint{0.907358in}{4.360661in}}%
\pgfusepath{stroke,fill}%
\end{pgfscope}%
\begin{pgfscope}%
\pgfpathrectangle{\pgfqpoint{0.380943in}{4.185189in}}{\pgfqpoint{4.650000in}{0.614151in}}%
\pgfusepath{clip}%
\pgfsetbuttcap%
\pgfsetroundjoin%
\definecolor{currentfill}{rgb}{0.972549,0.870588,0.692810}%
\pgfsetfillcolor{currentfill}%
\pgfsetlinewidth{0.250937pt}%
\definecolor{currentstroke}{rgb}{1.000000,1.000000,1.000000}%
\pgfsetstrokecolor{currentstroke}%
\pgfsetdash{}{0pt}%
\pgfpathmoveto{\pgfqpoint{0.995094in}{4.360661in}}%
\pgfpathlineto{\pgfqpoint{1.082830in}{4.360661in}}%
\pgfpathlineto{\pgfqpoint{1.082830in}{4.272925in}}%
\pgfpathlineto{\pgfqpoint{0.995094in}{4.272925in}}%
\pgfpathlineto{\pgfqpoint{0.995094in}{4.360661in}}%
\pgfusepath{stroke,fill}%
\end{pgfscope}%
\begin{pgfscope}%
\pgfpathrectangle{\pgfqpoint{0.380943in}{4.185189in}}{\pgfqpoint{4.650000in}{0.614151in}}%
\pgfusepath{clip}%
\pgfsetbuttcap%
\pgfsetroundjoin%
\definecolor{currentfill}{rgb}{0.962414,0.923552,0.722891}%
\pgfsetfillcolor{currentfill}%
\pgfsetlinewidth{0.250937pt}%
\definecolor{currentstroke}{rgb}{1.000000,1.000000,1.000000}%
\pgfsetstrokecolor{currentstroke}%
\pgfsetdash{}{0pt}%
\pgfpathmoveto{\pgfqpoint{1.082830in}{4.360661in}}%
\pgfpathlineto{\pgfqpoint{1.170566in}{4.360661in}}%
\pgfpathlineto{\pgfqpoint{1.170566in}{4.272925in}}%
\pgfpathlineto{\pgfqpoint{1.082830in}{4.272925in}}%
\pgfpathlineto{\pgfqpoint{1.082830in}{4.360661in}}%
\pgfusepath{stroke,fill}%
\end{pgfscope}%
\begin{pgfscope}%
\pgfpathrectangle{\pgfqpoint{0.380943in}{4.185189in}}{\pgfqpoint{4.650000in}{0.614151in}}%
\pgfusepath{clip}%
\pgfsetbuttcap%
\pgfsetroundjoin%
\definecolor{currentfill}{rgb}{0.965444,0.906113,0.711757}%
\pgfsetfillcolor{currentfill}%
\pgfsetlinewidth{0.250937pt}%
\definecolor{currentstroke}{rgb}{1.000000,1.000000,1.000000}%
\pgfsetstrokecolor{currentstroke}%
\pgfsetdash{}{0pt}%
\pgfpathmoveto{\pgfqpoint{1.170566in}{4.360661in}}%
\pgfpathlineto{\pgfqpoint{1.258302in}{4.360661in}}%
\pgfpathlineto{\pgfqpoint{1.258302in}{4.272925in}}%
\pgfpathlineto{\pgfqpoint{1.170566in}{4.272925in}}%
\pgfpathlineto{\pgfqpoint{1.170566in}{4.360661in}}%
\pgfusepath{stroke,fill}%
\end{pgfscope}%
\begin{pgfscope}%
\pgfpathrectangle{\pgfqpoint{0.380943in}{4.185189in}}{\pgfqpoint{4.650000in}{0.614151in}}%
\pgfusepath{clip}%
\pgfsetbuttcap%
\pgfsetroundjoin%
\definecolor{currentfill}{rgb}{0.968166,0.945882,0.748604}%
\pgfsetfillcolor{currentfill}%
\pgfsetlinewidth{0.250937pt}%
\definecolor{currentstroke}{rgb}{1.000000,1.000000,1.000000}%
\pgfsetstrokecolor{currentstroke}%
\pgfsetdash{}{0pt}%
\pgfpathmoveto{\pgfqpoint{1.258302in}{4.360661in}}%
\pgfpathlineto{\pgfqpoint{1.346037in}{4.360661in}}%
\pgfpathlineto{\pgfqpoint{1.346037in}{4.272925in}}%
\pgfpathlineto{\pgfqpoint{1.258302in}{4.272925in}}%
\pgfpathlineto{\pgfqpoint{1.258302in}{4.360661in}}%
\pgfusepath{stroke,fill}%
\end{pgfscope}%
\begin{pgfscope}%
\pgfpathrectangle{\pgfqpoint{0.380943in}{4.185189in}}{\pgfqpoint{4.650000in}{0.614151in}}%
\pgfusepath{clip}%
\pgfsetbuttcap%
\pgfsetroundjoin%
\definecolor{currentfill}{rgb}{0.962414,0.923552,0.722891}%
\pgfsetfillcolor{currentfill}%
\pgfsetlinewidth{0.250937pt}%
\definecolor{currentstroke}{rgb}{1.000000,1.000000,1.000000}%
\pgfsetstrokecolor{currentstroke}%
\pgfsetdash{}{0pt}%
\pgfpathmoveto{\pgfqpoint{1.346037in}{4.360661in}}%
\pgfpathlineto{\pgfqpoint{1.433773in}{4.360661in}}%
\pgfpathlineto{\pgfqpoint{1.433773in}{4.272925in}}%
\pgfpathlineto{\pgfqpoint{1.346037in}{4.272925in}}%
\pgfpathlineto{\pgfqpoint{1.346037in}{4.360661in}}%
\pgfusepath{stroke,fill}%
\end{pgfscope}%
\begin{pgfscope}%
\pgfpathrectangle{\pgfqpoint{0.380943in}{4.185189in}}{\pgfqpoint{4.650000in}{0.614151in}}%
\pgfusepath{clip}%
\pgfsetbuttcap%
\pgfsetroundjoin%
\definecolor{currentfill}{rgb}{0.992326,0.765229,0.614840}%
\pgfsetfillcolor{currentfill}%
\pgfsetlinewidth{0.250937pt}%
\definecolor{currentstroke}{rgb}{1.000000,1.000000,1.000000}%
\pgfsetstrokecolor{currentstroke}%
\pgfsetdash{}{0pt}%
\pgfpathmoveto{\pgfqpoint{1.433773in}{4.360661in}}%
\pgfpathlineto{\pgfqpoint{1.521509in}{4.360661in}}%
\pgfpathlineto{\pgfqpoint{1.521509in}{4.272925in}}%
\pgfpathlineto{\pgfqpoint{1.433773in}{4.272925in}}%
\pgfpathlineto{\pgfqpoint{1.433773in}{4.360661in}}%
\pgfusepath{stroke,fill}%
\end{pgfscope}%
\begin{pgfscope}%
\pgfpathrectangle{\pgfqpoint{0.380943in}{4.185189in}}{\pgfqpoint{4.650000in}{0.614151in}}%
\pgfusepath{clip}%
\pgfsetbuttcap%
\pgfsetroundjoin%
\definecolor{currentfill}{rgb}{0.965444,0.906113,0.711757}%
\pgfsetfillcolor{currentfill}%
\pgfsetlinewidth{0.250937pt}%
\definecolor{currentstroke}{rgb}{1.000000,1.000000,1.000000}%
\pgfsetstrokecolor{currentstroke}%
\pgfsetdash{}{0pt}%
\pgfpathmoveto{\pgfqpoint{1.521509in}{4.360661in}}%
\pgfpathlineto{\pgfqpoint{1.609245in}{4.360661in}}%
\pgfpathlineto{\pgfqpoint{1.609245in}{4.272925in}}%
\pgfpathlineto{\pgfqpoint{1.521509in}{4.272925in}}%
\pgfpathlineto{\pgfqpoint{1.521509in}{4.360661in}}%
\pgfusepath{stroke,fill}%
\end{pgfscope}%
\begin{pgfscope}%
\pgfpathrectangle{\pgfqpoint{0.380943in}{4.185189in}}{\pgfqpoint{4.650000in}{0.614151in}}%
\pgfusepath{clip}%
\pgfsetbuttcap%
\pgfsetroundjoin%
\definecolor{currentfill}{rgb}{0.979654,0.837186,0.669619}%
\pgfsetfillcolor{currentfill}%
\pgfsetlinewidth{0.250937pt}%
\definecolor{currentstroke}{rgb}{1.000000,1.000000,1.000000}%
\pgfsetstrokecolor{currentstroke}%
\pgfsetdash{}{0pt}%
\pgfpathmoveto{\pgfqpoint{1.609245in}{4.360661in}}%
\pgfpathlineto{\pgfqpoint{1.696981in}{4.360661in}}%
\pgfpathlineto{\pgfqpoint{1.696981in}{4.272925in}}%
\pgfpathlineto{\pgfqpoint{1.609245in}{4.272925in}}%
\pgfpathlineto{\pgfqpoint{1.609245in}{4.360661in}}%
\pgfusepath{stroke,fill}%
\end{pgfscope}%
\begin{pgfscope}%
\pgfpathrectangle{\pgfqpoint{0.380943in}{4.185189in}}{\pgfqpoint{4.650000in}{0.614151in}}%
\pgfusepath{clip}%
\pgfsetbuttcap%
\pgfsetroundjoin%
\definecolor{currentfill}{rgb}{1.000000,1.000000,0.870204}%
\pgfsetfillcolor{currentfill}%
\pgfsetlinewidth{0.250937pt}%
\definecolor{currentstroke}{rgb}{1.000000,1.000000,1.000000}%
\pgfsetstrokecolor{currentstroke}%
\pgfsetdash{}{0pt}%
\pgfpathmoveto{\pgfqpoint{1.696981in}{4.360661in}}%
\pgfpathlineto{\pgfqpoint{1.784717in}{4.360661in}}%
\pgfpathlineto{\pgfqpoint{1.784717in}{4.272925in}}%
\pgfpathlineto{\pgfqpoint{1.696981in}{4.272925in}}%
\pgfpathlineto{\pgfqpoint{1.696981in}{4.360661in}}%
\pgfusepath{stroke,fill}%
\end{pgfscope}%
\begin{pgfscope}%
\pgfpathrectangle{\pgfqpoint{0.380943in}{4.185189in}}{\pgfqpoint{4.650000in}{0.614151in}}%
\pgfusepath{clip}%
\pgfsetbuttcap%
\pgfsetroundjoin%
\definecolor{currentfill}{rgb}{0.972549,0.870588,0.692810}%
\pgfsetfillcolor{currentfill}%
\pgfsetlinewidth{0.250937pt}%
\definecolor{currentstroke}{rgb}{1.000000,1.000000,1.000000}%
\pgfsetstrokecolor{currentstroke}%
\pgfsetdash{}{0pt}%
\pgfpathmoveto{\pgfqpoint{1.784717in}{4.360661in}}%
\pgfpathlineto{\pgfqpoint{1.872452in}{4.360661in}}%
\pgfpathlineto{\pgfqpoint{1.872452in}{4.272925in}}%
\pgfpathlineto{\pgfqpoint{1.784717in}{4.272925in}}%
\pgfpathlineto{\pgfqpoint{1.784717in}{4.360661in}}%
\pgfusepath{stroke,fill}%
\end{pgfscope}%
\begin{pgfscope}%
\pgfpathrectangle{\pgfqpoint{0.380943in}{4.185189in}}{\pgfqpoint{4.650000in}{0.614151in}}%
\pgfusepath{clip}%
\pgfsetbuttcap%
\pgfsetroundjoin%
\definecolor{currentfill}{rgb}{0.968166,0.945882,0.748604}%
\pgfsetfillcolor{currentfill}%
\pgfsetlinewidth{0.250937pt}%
\definecolor{currentstroke}{rgb}{1.000000,1.000000,1.000000}%
\pgfsetstrokecolor{currentstroke}%
\pgfsetdash{}{0pt}%
\pgfpathmoveto{\pgfqpoint{1.872452in}{4.360661in}}%
\pgfpathlineto{\pgfqpoint{1.960188in}{4.360661in}}%
\pgfpathlineto{\pgfqpoint{1.960188in}{4.272925in}}%
\pgfpathlineto{\pgfqpoint{1.872452in}{4.272925in}}%
\pgfpathlineto{\pgfqpoint{1.872452in}{4.360661in}}%
\pgfusepath{stroke,fill}%
\end{pgfscope}%
\begin{pgfscope}%
\pgfpathrectangle{\pgfqpoint{0.380943in}{4.185189in}}{\pgfqpoint{4.650000in}{0.614151in}}%
\pgfusepath{clip}%
\pgfsetbuttcap%
\pgfsetroundjoin%
\definecolor{currentfill}{rgb}{0.962414,0.923552,0.722891}%
\pgfsetfillcolor{currentfill}%
\pgfsetlinewidth{0.250937pt}%
\definecolor{currentstroke}{rgb}{1.000000,1.000000,1.000000}%
\pgfsetstrokecolor{currentstroke}%
\pgfsetdash{}{0pt}%
\pgfpathmoveto{\pgfqpoint{1.960188in}{4.360661in}}%
\pgfpathlineto{\pgfqpoint{2.047924in}{4.360661in}}%
\pgfpathlineto{\pgfqpoint{2.047924in}{4.272925in}}%
\pgfpathlineto{\pgfqpoint{1.960188in}{4.272925in}}%
\pgfpathlineto{\pgfqpoint{1.960188in}{4.360661in}}%
\pgfusepath{stroke,fill}%
\end{pgfscope}%
\begin{pgfscope}%
\pgfpathrectangle{\pgfqpoint{0.380943in}{4.185189in}}{\pgfqpoint{4.650000in}{0.614151in}}%
\pgfusepath{clip}%
\pgfsetbuttcap%
\pgfsetroundjoin%
\definecolor{currentfill}{rgb}{0.979654,0.837186,0.669619}%
\pgfsetfillcolor{currentfill}%
\pgfsetlinewidth{0.250937pt}%
\definecolor{currentstroke}{rgb}{1.000000,1.000000,1.000000}%
\pgfsetstrokecolor{currentstroke}%
\pgfsetdash{}{0pt}%
\pgfpathmoveto{\pgfqpoint{2.047924in}{4.360661in}}%
\pgfpathlineto{\pgfqpoint{2.135660in}{4.360661in}}%
\pgfpathlineto{\pgfqpoint{2.135660in}{4.272925in}}%
\pgfpathlineto{\pgfqpoint{2.047924in}{4.272925in}}%
\pgfpathlineto{\pgfqpoint{2.047924in}{4.360661in}}%
\pgfusepath{stroke,fill}%
\end{pgfscope}%
\begin{pgfscope}%
\pgfpathrectangle{\pgfqpoint{0.380943in}{4.185189in}}{\pgfqpoint{4.650000in}{0.614151in}}%
\pgfusepath{clip}%
\pgfsetbuttcap%
\pgfsetroundjoin%
\definecolor{currentfill}{rgb}{1.000000,1.000000,0.929412}%
\pgfsetfillcolor{currentfill}%
\pgfsetlinewidth{0.250937pt}%
\definecolor{currentstroke}{rgb}{1.000000,1.000000,1.000000}%
\pgfsetstrokecolor{currentstroke}%
\pgfsetdash{}{0pt}%
\pgfpathmoveto{\pgfqpoint{2.135660in}{4.360661in}}%
\pgfpathlineto{\pgfqpoint{2.223396in}{4.360661in}}%
\pgfpathlineto{\pgfqpoint{2.223396in}{4.272925in}}%
\pgfpathlineto{\pgfqpoint{2.135660in}{4.272925in}}%
\pgfpathlineto{\pgfqpoint{2.135660in}{4.360661in}}%
\pgfusepath{stroke,fill}%
\end{pgfscope}%
\begin{pgfscope}%
\pgfpathrectangle{\pgfqpoint{0.380943in}{4.185189in}}{\pgfqpoint{4.650000in}{0.614151in}}%
\pgfusepath{clip}%
\pgfsetbuttcap%
\pgfsetroundjoin%
\definecolor{currentfill}{rgb}{0.968166,0.945882,0.748604}%
\pgfsetfillcolor{currentfill}%
\pgfsetlinewidth{0.250937pt}%
\definecolor{currentstroke}{rgb}{1.000000,1.000000,1.000000}%
\pgfsetstrokecolor{currentstroke}%
\pgfsetdash{}{0pt}%
\pgfpathmoveto{\pgfqpoint{2.223396in}{4.360661in}}%
\pgfpathlineto{\pgfqpoint{2.311132in}{4.360661in}}%
\pgfpathlineto{\pgfqpoint{2.311132in}{4.272925in}}%
\pgfpathlineto{\pgfqpoint{2.223396in}{4.272925in}}%
\pgfpathlineto{\pgfqpoint{2.223396in}{4.360661in}}%
\pgfusepath{stroke,fill}%
\end{pgfscope}%
\begin{pgfscope}%
\pgfpathrectangle{\pgfqpoint{0.380943in}{4.185189in}}{\pgfqpoint{4.650000in}{0.614151in}}%
\pgfusepath{clip}%
\pgfsetbuttcap%
\pgfsetroundjoin%
\definecolor{currentfill}{rgb}{0.968166,0.945882,0.748604}%
\pgfsetfillcolor{currentfill}%
\pgfsetlinewidth{0.250937pt}%
\definecolor{currentstroke}{rgb}{1.000000,1.000000,1.000000}%
\pgfsetstrokecolor{currentstroke}%
\pgfsetdash{}{0pt}%
\pgfpathmoveto{\pgfqpoint{2.311132in}{4.360661in}}%
\pgfpathlineto{\pgfqpoint{2.398868in}{4.360661in}}%
\pgfpathlineto{\pgfqpoint{2.398868in}{4.272925in}}%
\pgfpathlineto{\pgfqpoint{2.311132in}{4.272925in}}%
\pgfpathlineto{\pgfqpoint{2.311132in}{4.360661in}}%
\pgfusepath{stroke,fill}%
\end{pgfscope}%
\begin{pgfscope}%
\pgfpathrectangle{\pgfqpoint{0.380943in}{4.185189in}}{\pgfqpoint{4.650000in}{0.614151in}}%
\pgfusepath{clip}%
\pgfsetbuttcap%
\pgfsetroundjoin%
\definecolor{currentfill}{rgb}{0.979654,0.837186,0.669619}%
\pgfsetfillcolor{currentfill}%
\pgfsetlinewidth{0.250937pt}%
\definecolor{currentstroke}{rgb}{1.000000,1.000000,1.000000}%
\pgfsetstrokecolor{currentstroke}%
\pgfsetdash{}{0pt}%
\pgfpathmoveto{\pgfqpoint{2.398868in}{4.360661in}}%
\pgfpathlineto{\pgfqpoint{2.486603in}{4.360661in}}%
\pgfpathlineto{\pgfqpoint{2.486603in}{4.272925in}}%
\pgfpathlineto{\pgfqpoint{2.398868in}{4.272925in}}%
\pgfpathlineto{\pgfqpoint{2.398868in}{4.360661in}}%
\pgfusepath{stroke,fill}%
\end{pgfscope}%
\begin{pgfscope}%
\pgfpathrectangle{\pgfqpoint{0.380943in}{4.185189in}}{\pgfqpoint{4.650000in}{0.614151in}}%
\pgfusepath{clip}%
\pgfsetbuttcap%
\pgfsetroundjoin%
\definecolor{currentfill}{rgb}{0.979654,0.837186,0.669619}%
\pgfsetfillcolor{currentfill}%
\pgfsetlinewidth{0.250937pt}%
\definecolor{currentstroke}{rgb}{1.000000,1.000000,1.000000}%
\pgfsetstrokecolor{currentstroke}%
\pgfsetdash{}{0pt}%
\pgfpathmoveto{\pgfqpoint{2.486603in}{4.360661in}}%
\pgfpathlineto{\pgfqpoint{2.574339in}{4.360661in}}%
\pgfpathlineto{\pgfqpoint{2.574339in}{4.272925in}}%
\pgfpathlineto{\pgfqpoint{2.486603in}{4.272925in}}%
\pgfpathlineto{\pgfqpoint{2.486603in}{4.360661in}}%
\pgfusepath{stroke,fill}%
\end{pgfscope}%
\begin{pgfscope}%
\pgfpathrectangle{\pgfqpoint{0.380943in}{4.185189in}}{\pgfqpoint{4.650000in}{0.614151in}}%
\pgfusepath{clip}%
\pgfsetbuttcap%
\pgfsetroundjoin%
\definecolor{currentfill}{rgb}{0.991849,0.986144,0.810181}%
\pgfsetfillcolor{currentfill}%
\pgfsetlinewidth{0.250937pt}%
\definecolor{currentstroke}{rgb}{1.000000,1.000000,1.000000}%
\pgfsetstrokecolor{currentstroke}%
\pgfsetdash{}{0pt}%
\pgfpathmoveto{\pgfqpoint{2.574339in}{4.360661in}}%
\pgfpathlineto{\pgfqpoint{2.662075in}{4.360661in}}%
\pgfpathlineto{\pgfqpoint{2.662075in}{4.272925in}}%
\pgfpathlineto{\pgfqpoint{2.574339in}{4.272925in}}%
\pgfpathlineto{\pgfqpoint{2.574339in}{4.360661in}}%
\pgfusepath{stroke,fill}%
\end{pgfscope}%
\begin{pgfscope}%
\pgfpathrectangle{\pgfqpoint{0.380943in}{4.185189in}}{\pgfqpoint{4.650000in}{0.614151in}}%
\pgfusepath{clip}%
\pgfsetbuttcap%
\pgfsetroundjoin%
\definecolor{currentfill}{rgb}{0.991849,0.986144,0.810181}%
\pgfsetfillcolor{currentfill}%
\pgfsetlinewidth{0.250937pt}%
\definecolor{currentstroke}{rgb}{1.000000,1.000000,1.000000}%
\pgfsetstrokecolor{currentstroke}%
\pgfsetdash{}{0pt}%
\pgfpathmoveto{\pgfqpoint{2.662075in}{4.360661in}}%
\pgfpathlineto{\pgfqpoint{2.749811in}{4.360661in}}%
\pgfpathlineto{\pgfqpoint{2.749811in}{4.272925in}}%
\pgfpathlineto{\pgfqpoint{2.662075in}{4.272925in}}%
\pgfpathlineto{\pgfqpoint{2.662075in}{4.360661in}}%
\pgfusepath{stroke,fill}%
\end{pgfscope}%
\begin{pgfscope}%
\pgfpathrectangle{\pgfqpoint{0.380943in}{4.185189in}}{\pgfqpoint{4.650000in}{0.614151in}}%
\pgfusepath{clip}%
\pgfsetbuttcap%
\pgfsetroundjoin%
\definecolor{currentfill}{rgb}{0.991849,0.986144,0.810181}%
\pgfsetfillcolor{currentfill}%
\pgfsetlinewidth{0.250937pt}%
\definecolor{currentstroke}{rgb}{1.000000,1.000000,1.000000}%
\pgfsetstrokecolor{currentstroke}%
\pgfsetdash{}{0pt}%
\pgfpathmoveto{\pgfqpoint{2.749811in}{4.360661in}}%
\pgfpathlineto{\pgfqpoint{2.837547in}{4.360661in}}%
\pgfpathlineto{\pgfqpoint{2.837547in}{4.272925in}}%
\pgfpathlineto{\pgfqpoint{2.749811in}{4.272925in}}%
\pgfpathlineto{\pgfqpoint{2.749811in}{4.360661in}}%
\pgfusepath{stroke,fill}%
\end{pgfscope}%
\begin{pgfscope}%
\pgfpathrectangle{\pgfqpoint{0.380943in}{4.185189in}}{\pgfqpoint{4.650000in}{0.614151in}}%
\pgfusepath{clip}%
\pgfsetbuttcap%
\pgfsetroundjoin%
\definecolor{currentfill}{rgb}{0.965444,0.906113,0.711757}%
\pgfsetfillcolor{currentfill}%
\pgfsetlinewidth{0.250937pt}%
\definecolor{currentstroke}{rgb}{1.000000,1.000000,1.000000}%
\pgfsetstrokecolor{currentstroke}%
\pgfsetdash{}{0pt}%
\pgfpathmoveto{\pgfqpoint{2.837547in}{4.360661in}}%
\pgfpathlineto{\pgfqpoint{2.925283in}{4.360661in}}%
\pgfpathlineto{\pgfqpoint{2.925283in}{4.272925in}}%
\pgfpathlineto{\pgfqpoint{2.837547in}{4.272925in}}%
\pgfpathlineto{\pgfqpoint{2.837547in}{4.360661in}}%
\pgfusepath{stroke,fill}%
\end{pgfscope}%
\begin{pgfscope}%
\pgfpathrectangle{\pgfqpoint{0.380943in}{4.185189in}}{\pgfqpoint{4.650000in}{0.614151in}}%
\pgfusepath{clip}%
\pgfsetbuttcap%
\pgfsetroundjoin%
\definecolor{currentfill}{rgb}{1.000000,1.000000,0.870204}%
\pgfsetfillcolor{currentfill}%
\pgfsetlinewidth{0.250937pt}%
\definecolor{currentstroke}{rgb}{1.000000,1.000000,1.000000}%
\pgfsetstrokecolor{currentstroke}%
\pgfsetdash{}{0pt}%
\pgfpathmoveto{\pgfqpoint{2.925283in}{4.360661in}}%
\pgfpathlineto{\pgfqpoint{3.013019in}{4.360661in}}%
\pgfpathlineto{\pgfqpoint{3.013019in}{4.272925in}}%
\pgfpathlineto{\pgfqpoint{2.925283in}{4.272925in}}%
\pgfpathlineto{\pgfqpoint{2.925283in}{4.360661in}}%
\pgfusepath{stroke,fill}%
\end{pgfscope}%
\begin{pgfscope}%
\pgfpathrectangle{\pgfqpoint{0.380943in}{4.185189in}}{\pgfqpoint{4.650000in}{0.614151in}}%
\pgfusepath{clip}%
\pgfsetbuttcap%
\pgfsetroundjoin%
\definecolor{currentfill}{rgb}{0.965444,0.906113,0.711757}%
\pgfsetfillcolor{currentfill}%
\pgfsetlinewidth{0.250937pt}%
\definecolor{currentstroke}{rgb}{1.000000,1.000000,1.000000}%
\pgfsetstrokecolor{currentstroke}%
\pgfsetdash{}{0pt}%
\pgfpathmoveto{\pgfqpoint{3.013019in}{4.360661in}}%
\pgfpathlineto{\pgfqpoint{3.100754in}{4.360661in}}%
\pgfpathlineto{\pgfqpoint{3.100754in}{4.272925in}}%
\pgfpathlineto{\pgfqpoint{3.013019in}{4.272925in}}%
\pgfpathlineto{\pgfqpoint{3.013019in}{4.360661in}}%
\pgfusepath{stroke,fill}%
\end{pgfscope}%
\begin{pgfscope}%
\pgfpathrectangle{\pgfqpoint{0.380943in}{4.185189in}}{\pgfqpoint{4.650000in}{0.614151in}}%
\pgfusepath{clip}%
\pgfsetbuttcap%
\pgfsetroundjoin%
\definecolor{currentfill}{rgb}{0.968166,0.945882,0.748604}%
\pgfsetfillcolor{currentfill}%
\pgfsetlinewidth{0.250937pt}%
\definecolor{currentstroke}{rgb}{1.000000,1.000000,1.000000}%
\pgfsetstrokecolor{currentstroke}%
\pgfsetdash{}{0pt}%
\pgfpathmoveto{\pgfqpoint{3.100754in}{4.360661in}}%
\pgfpathlineto{\pgfqpoint{3.188490in}{4.360661in}}%
\pgfpathlineto{\pgfqpoint{3.188490in}{4.272925in}}%
\pgfpathlineto{\pgfqpoint{3.100754in}{4.272925in}}%
\pgfpathlineto{\pgfqpoint{3.100754in}{4.360661in}}%
\pgfusepath{stroke,fill}%
\end{pgfscope}%
\begin{pgfscope}%
\pgfpathrectangle{\pgfqpoint{0.380943in}{4.185189in}}{\pgfqpoint{4.650000in}{0.614151in}}%
\pgfusepath{clip}%
\pgfsetbuttcap%
\pgfsetroundjoin%
\definecolor{currentfill}{rgb}{1.000000,1.000000,0.870204}%
\pgfsetfillcolor{currentfill}%
\pgfsetlinewidth{0.250937pt}%
\definecolor{currentstroke}{rgb}{1.000000,1.000000,1.000000}%
\pgfsetstrokecolor{currentstroke}%
\pgfsetdash{}{0pt}%
\pgfpathmoveto{\pgfqpoint{3.188490in}{4.360661in}}%
\pgfpathlineto{\pgfqpoint{3.276226in}{4.360661in}}%
\pgfpathlineto{\pgfqpoint{3.276226in}{4.272925in}}%
\pgfpathlineto{\pgfqpoint{3.188490in}{4.272925in}}%
\pgfpathlineto{\pgfqpoint{3.188490in}{4.360661in}}%
\pgfusepath{stroke,fill}%
\end{pgfscope}%
\begin{pgfscope}%
\pgfpathrectangle{\pgfqpoint{0.380943in}{4.185189in}}{\pgfqpoint{4.650000in}{0.614151in}}%
\pgfusepath{clip}%
\pgfsetbuttcap%
\pgfsetroundjoin%
\definecolor{currentfill}{rgb}{0.968166,0.945882,0.748604}%
\pgfsetfillcolor{currentfill}%
\pgfsetlinewidth{0.250937pt}%
\definecolor{currentstroke}{rgb}{1.000000,1.000000,1.000000}%
\pgfsetstrokecolor{currentstroke}%
\pgfsetdash{}{0pt}%
\pgfpathmoveto{\pgfqpoint{3.276226in}{4.360661in}}%
\pgfpathlineto{\pgfqpoint{3.363962in}{4.360661in}}%
\pgfpathlineto{\pgfqpoint{3.363962in}{4.272925in}}%
\pgfpathlineto{\pgfqpoint{3.276226in}{4.272925in}}%
\pgfpathlineto{\pgfqpoint{3.276226in}{4.360661in}}%
\pgfusepath{stroke,fill}%
\end{pgfscope}%
\begin{pgfscope}%
\pgfpathrectangle{\pgfqpoint{0.380943in}{4.185189in}}{\pgfqpoint{4.650000in}{0.614151in}}%
\pgfusepath{clip}%
\pgfsetbuttcap%
\pgfsetroundjoin%
\definecolor{currentfill}{rgb}{0.962414,0.923552,0.722891}%
\pgfsetfillcolor{currentfill}%
\pgfsetlinewidth{0.250937pt}%
\definecolor{currentstroke}{rgb}{1.000000,1.000000,1.000000}%
\pgfsetstrokecolor{currentstroke}%
\pgfsetdash{}{0pt}%
\pgfpathmoveto{\pgfqpoint{3.363962in}{4.360661in}}%
\pgfpathlineto{\pgfqpoint{3.451698in}{4.360661in}}%
\pgfpathlineto{\pgfqpoint{3.451698in}{4.272925in}}%
\pgfpathlineto{\pgfqpoint{3.363962in}{4.272925in}}%
\pgfpathlineto{\pgfqpoint{3.363962in}{4.360661in}}%
\pgfusepath{stroke,fill}%
\end{pgfscope}%
\begin{pgfscope}%
\pgfpathrectangle{\pgfqpoint{0.380943in}{4.185189in}}{\pgfqpoint{4.650000in}{0.614151in}}%
\pgfusepath{clip}%
\pgfsetbuttcap%
\pgfsetroundjoin%
\definecolor{currentfill}{rgb}{0.962414,0.923552,0.722891}%
\pgfsetfillcolor{currentfill}%
\pgfsetlinewidth{0.250937pt}%
\definecolor{currentstroke}{rgb}{1.000000,1.000000,1.000000}%
\pgfsetstrokecolor{currentstroke}%
\pgfsetdash{}{0pt}%
\pgfpathmoveto{\pgfqpoint{3.451698in}{4.360661in}}%
\pgfpathlineto{\pgfqpoint{3.539434in}{4.360661in}}%
\pgfpathlineto{\pgfqpoint{3.539434in}{4.272925in}}%
\pgfpathlineto{\pgfqpoint{3.451698in}{4.272925in}}%
\pgfpathlineto{\pgfqpoint{3.451698in}{4.360661in}}%
\pgfusepath{stroke,fill}%
\end{pgfscope}%
\begin{pgfscope}%
\pgfpathrectangle{\pgfqpoint{0.380943in}{4.185189in}}{\pgfqpoint{4.650000in}{0.614151in}}%
\pgfusepath{clip}%
\pgfsetbuttcap%
\pgfsetroundjoin%
\definecolor{currentfill}{rgb}{0.968166,0.945882,0.748604}%
\pgfsetfillcolor{currentfill}%
\pgfsetlinewidth{0.250937pt}%
\definecolor{currentstroke}{rgb}{1.000000,1.000000,1.000000}%
\pgfsetstrokecolor{currentstroke}%
\pgfsetdash{}{0pt}%
\pgfpathmoveto{\pgfqpoint{3.539434in}{4.360661in}}%
\pgfpathlineto{\pgfqpoint{3.627169in}{4.360661in}}%
\pgfpathlineto{\pgfqpoint{3.627169in}{4.272925in}}%
\pgfpathlineto{\pgfqpoint{3.539434in}{4.272925in}}%
\pgfpathlineto{\pgfqpoint{3.539434in}{4.360661in}}%
\pgfusepath{stroke,fill}%
\end{pgfscope}%
\begin{pgfscope}%
\pgfpathrectangle{\pgfqpoint{0.380943in}{4.185189in}}{\pgfqpoint{4.650000in}{0.614151in}}%
\pgfusepath{clip}%
\pgfsetbuttcap%
\pgfsetroundjoin%
\definecolor{currentfill}{rgb}{0.979654,0.837186,0.669619}%
\pgfsetfillcolor{currentfill}%
\pgfsetlinewidth{0.250937pt}%
\definecolor{currentstroke}{rgb}{1.000000,1.000000,1.000000}%
\pgfsetstrokecolor{currentstroke}%
\pgfsetdash{}{0pt}%
\pgfpathmoveto{\pgfqpoint{3.627169in}{4.360661in}}%
\pgfpathlineto{\pgfqpoint{3.714905in}{4.360661in}}%
\pgfpathlineto{\pgfqpoint{3.714905in}{4.272925in}}%
\pgfpathlineto{\pgfqpoint{3.627169in}{4.272925in}}%
\pgfpathlineto{\pgfqpoint{3.627169in}{4.360661in}}%
\pgfusepath{stroke,fill}%
\end{pgfscope}%
\begin{pgfscope}%
\pgfpathrectangle{\pgfqpoint{0.380943in}{4.185189in}}{\pgfqpoint{4.650000in}{0.614151in}}%
\pgfusepath{clip}%
\pgfsetbuttcap%
\pgfsetroundjoin%
\definecolor{currentfill}{rgb}{0.991849,0.986144,0.810181}%
\pgfsetfillcolor{currentfill}%
\pgfsetlinewidth{0.250937pt}%
\definecolor{currentstroke}{rgb}{1.000000,1.000000,1.000000}%
\pgfsetstrokecolor{currentstroke}%
\pgfsetdash{}{0pt}%
\pgfpathmoveto{\pgfqpoint{3.714905in}{4.360661in}}%
\pgfpathlineto{\pgfqpoint{3.802641in}{4.360661in}}%
\pgfpathlineto{\pgfqpoint{3.802641in}{4.272925in}}%
\pgfpathlineto{\pgfqpoint{3.714905in}{4.272925in}}%
\pgfpathlineto{\pgfqpoint{3.714905in}{4.360661in}}%
\pgfusepath{stroke,fill}%
\end{pgfscope}%
\begin{pgfscope}%
\pgfpathrectangle{\pgfqpoint{0.380943in}{4.185189in}}{\pgfqpoint{4.650000in}{0.614151in}}%
\pgfusepath{clip}%
\pgfsetbuttcap%
\pgfsetroundjoin%
\definecolor{currentfill}{rgb}{0.962414,0.923552,0.722891}%
\pgfsetfillcolor{currentfill}%
\pgfsetlinewidth{0.250937pt}%
\definecolor{currentstroke}{rgb}{1.000000,1.000000,1.000000}%
\pgfsetstrokecolor{currentstroke}%
\pgfsetdash{}{0pt}%
\pgfpathmoveto{\pgfqpoint{3.802641in}{4.360661in}}%
\pgfpathlineto{\pgfqpoint{3.890377in}{4.360661in}}%
\pgfpathlineto{\pgfqpoint{3.890377in}{4.272925in}}%
\pgfpathlineto{\pgfqpoint{3.802641in}{4.272925in}}%
\pgfpathlineto{\pgfqpoint{3.802641in}{4.360661in}}%
\pgfusepath{stroke,fill}%
\end{pgfscope}%
\begin{pgfscope}%
\pgfpathrectangle{\pgfqpoint{0.380943in}{4.185189in}}{\pgfqpoint{4.650000in}{0.614151in}}%
\pgfusepath{clip}%
\pgfsetbuttcap%
\pgfsetroundjoin%
\definecolor{currentfill}{rgb}{0.992326,0.765229,0.614840}%
\pgfsetfillcolor{currentfill}%
\pgfsetlinewidth{0.250937pt}%
\definecolor{currentstroke}{rgb}{1.000000,1.000000,1.000000}%
\pgfsetstrokecolor{currentstroke}%
\pgfsetdash{}{0pt}%
\pgfpathmoveto{\pgfqpoint{3.890377in}{4.360661in}}%
\pgfpathlineto{\pgfqpoint{3.978113in}{4.360661in}}%
\pgfpathlineto{\pgfqpoint{3.978113in}{4.272925in}}%
\pgfpathlineto{\pgfqpoint{3.890377in}{4.272925in}}%
\pgfpathlineto{\pgfqpoint{3.890377in}{4.360661in}}%
\pgfusepath{stroke,fill}%
\end{pgfscope}%
\begin{pgfscope}%
\pgfpathrectangle{\pgfqpoint{0.380943in}{4.185189in}}{\pgfqpoint{4.650000in}{0.614151in}}%
\pgfusepath{clip}%
\pgfsetbuttcap%
\pgfsetroundjoin%
\definecolor{currentfill}{rgb}{0.986759,0.806398,0.641200}%
\pgfsetfillcolor{currentfill}%
\pgfsetlinewidth{0.250937pt}%
\definecolor{currentstroke}{rgb}{1.000000,1.000000,1.000000}%
\pgfsetstrokecolor{currentstroke}%
\pgfsetdash{}{0pt}%
\pgfpathmoveto{\pgfqpoint{3.978113in}{4.360661in}}%
\pgfpathlineto{\pgfqpoint{4.065849in}{4.360661in}}%
\pgfpathlineto{\pgfqpoint{4.065849in}{4.272925in}}%
\pgfpathlineto{\pgfqpoint{3.978113in}{4.272925in}}%
\pgfpathlineto{\pgfqpoint{3.978113in}{4.360661in}}%
\pgfusepath{stroke,fill}%
\end{pgfscope}%
\begin{pgfscope}%
\pgfpathrectangle{\pgfqpoint{0.380943in}{4.185189in}}{\pgfqpoint{4.650000in}{0.614151in}}%
\pgfusepath{clip}%
\pgfsetbuttcap%
\pgfsetroundjoin%
\definecolor{currentfill}{rgb}{0.962414,0.923552,0.722891}%
\pgfsetfillcolor{currentfill}%
\pgfsetlinewidth{0.250937pt}%
\definecolor{currentstroke}{rgb}{1.000000,1.000000,1.000000}%
\pgfsetstrokecolor{currentstroke}%
\pgfsetdash{}{0pt}%
\pgfpathmoveto{\pgfqpoint{4.065849in}{4.360661in}}%
\pgfpathlineto{\pgfqpoint{4.153585in}{4.360661in}}%
\pgfpathlineto{\pgfqpoint{4.153585in}{4.272925in}}%
\pgfpathlineto{\pgfqpoint{4.065849in}{4.272925in}}%
\pgfpathlineto{\pgfqpoint{4.065849in}{4.360661in}}%
\pgfusepath{stroke,fill}%
\end{pgfscope}%
\begin{pgfscope}%
\pgfpathrectangle{\pgfqpoint{0.380943in}{4.185189in}}{\pgfqpoint{4.650000in}{0.614151in}}%
\pgfusepath{clip}%
\pgfsetbuttcap%
\pgfsetroundjoin%
\definecolor{currentfill}{rgb}{0.965444,0.906113,0.711757}%
\pgfsetfillcolor{currentfill}%
\pgfsetlinewidth{0.250937pt}%
\definecolor{currentstroke}{rgb}{1.000000,1.000000,1.000000}%
\pgfsetstrokecolor{currentstroke}%
\pgfsetdash{}{0pt}%
\pgfpathmoveto{\pgfqpoint{4.153585in}{4.360661in}}%
\pgfpathlineto{\pgfqpoint{4.241320in}{4.360661in}}%
\pgfpathlineto{\pgfqpoint{4.241320in}{4.272925in}}%
\pgfpathlineto{\pgfqpoint{4.153585in}{4.272925in}}%
\pgfpathlineto{\pgfqpoint{4.153585in}{4.360661in}}%
\pgfusepath{stroke,fill}%
\end{pgfscope}%
\begin{pgfscope}%
\pgfpathrectangle{\pgfqpoint{0.380943in}{4.185189in}}{\pgfqpoint{4.650000in}{0.614151in}}%
\pgfusepath{clip}%
\pgfsetbuttcap%
\pgfsetroundjoin%
\definecolor{currentfill}{rgb}{0.965444,0.906113,0.711757}%
\pgfsetfillcolor{currentfill}%
\pgfsetlinewidth{0.250937pt}%
\definecolor{currentstroke}{rgb}{1.000000,1.000000,1.000000}%
\pgfsetstrokecolor{currentstroke}%
\pgfsetdash{}{0pt}%
\pgfpathmoveto{\pgfqpoint{4.241320in}{4.360661in}}%
\pgfpathlineto{\pgfqpoint{4.329056in}{4.360661in}}%
\pgfpathlineto{\pgfqpoint{4.329056in}{4.272925in}}%
\pgfpathlineto{\pgfqpoint{4.241320in}{4.272925in}}%
\pgfpathlineto{\pgfqpoint{4.241320in}{4.360661in}}%
\pgfusepath{stroke,fill}%
\end{pgfscope}%
\begin{pgfscope}%
\pgfpathrectangle{\pgfqpoint{0.380943in}{4.185189in}}{\pgfqpoint{4.650000in}{0.614151in}}%
\pgfusepath{clip}%
\pgfsetbuttcap%
\pgfsetroundjoin%
\definecolor{currentfill}{rgb}{0.965444,0.906113,0.711757}%
\pgfsetfillcolor{currentfill}%
\pgfsetlinewidth{0.250937pt}%
\definecolor{currentstroke}{rgb}{1.000000,1.000000,1.000000}%
\pgfsetstrokecolor{currentstroke}%
\pgfsetdash{}{0pt}%
\pgfpathmoveto{\pgfqpoint{4.329056in}{4.360661in}}%
\pgfpathlineto{\pgfqpoint{4.416792in}{4.360661in}}%
\pgfpathlineto{\pgfqpoint{4.416792in}{4.272925in}}%
\pgfpathlineto{\pgfqpoint{4.329056in}{4.272925in}}%
\pgfpathlineto{\pgfqpoint{4.329056in}{4.360661in}}%
\pgfusepath{stroke,fill}%
\end{pgfscope}%
\begin{pgfscope}%
\pgfpathrectangle{\pgfqpoint{0.380943in}{4.185189in}}{\pgfqpoint{4.650000in}{0.614151in}}%
\pgfusepath{clip}%
\pgfsetbuttcap%
\pgfsetroundjoin%
\definecolor{currentfill}{rgb}{0.965444,0.906113,0.711757}%
\pgfsetfillcolor{currentfill}%
\pgfsetlinewidth{0.250937pt}%
\definecolor{currentstroke}{rgb}{1.000000,1.000000,1.000000}%
\pgfsetstrokecolor{currentstroke}%
\pgfsetdash{}{0pt}%
\pgfpathmoveto{\pgfqpoint{4.416792in}{4.360661in}}%
\pgfpathlineto{\pgfqpoint{4.504528in}{4.360661in}}%
\pgfpathlineto{\pgfqpoint{4.504528in}{4.272925in}}%
\pgfpathlineto{\pgfqpoint{4.416792in}{4.272925in}}%
\pgfpathlineto{\pgfqpoint{4.416792in}{4.360661in}}%
\pgfusepath{stroke,fill}%
\end{pgfscope}%
\begin{pgfscope}%
\pgfpathrectangle{\pgfqpoint{0.380943in}{4.185189in}}{\pgfqpoint{4.650000in}{0.614151in}}%
\pgfusepath{clip}%
\pgfsetbuttcap%
\pgfsetroundjoin%
\definecolor{currentfill}{rgb}{0.965444,0.906113,0.711757}%
\pgfsetfillcolor{currentfill}%
\pgfsetlinewidth{0.250937pt}%
\definecolor{currentstroke}{rgb}{1.000000,1.000000,1.000000}%
\pgfsetstrokecolor{currentstroke}%
\pgfsetdash{}{0pt}%
\pgfpathmoveto{\pgfqpoint{4.504528in}{4.360661in}}%
\pgfpathlineto{\pgfqpoint{4.592264in}{4.360661in}}%
\pgfpathlineto{\pgfqpoint{4.592264in}{4.272925in}}%
\pgfpathlineto{\pgfqpoint{4.504528in}{4.272925in}}%
\pgfpathlineto{\pgfqpoint{4.504528in}{4.360661in}}%
\pgfusepath{stroke,fill}%
\end{pgfscope}%
\begin{pgfscope}%
\pgfpathrectangle{\pgfqpoint{0.380943in}{4.185189in}}{\pgfqpoint{4.650000in}{0.614151in}}%
\pgfusepath{clip}%
\pgfsetbuttcap%
\pgfsetroundjoin%
\definecolor{currentfill}{rgb}{0.972549,0.870588,0.692810}%
\pgfsetfillcolor{currentfill}%
\pgfsetlinewidth{0.250937pt}%
\definecolor{currentstroke}{rgb}{1.000000,1.000000,1.000000}%
\pgfsetstrokecolor{currentstroke}%
\pgfsetdash{}{0pt}%
\pgfpathmoveto{\pgfqpoint{4.592264in}{4.360661in}}%
\pgfpathlineto{\pgfqpoint{4.680000in}{4.360661in}}%
\pgfpathlineto{\pgfqpoint{4.680000in}{4.272925in}}%
\pgfpathlineto{\pgfqpoint{4.592264in}{4.272925in}}%
\pgfpathlineto{\pgfqpoint{4.592264in}{4.360661in}}%
\pgfusepath{stroke,fill}%
\end{pgfscope}%
\begin{pgfscope}%
\pgfpathrectangle{\pgfqpoint{0.380943in}{4.185189in}}{\pgfqpoint{4.650000in}{0.614151in}}%
\pgfusepath{clip}%
\pgfsetbuttcap%
\pgfsetroundjoin%
\definecolor{currentfill}{rgb}{0.972549,0.870588,0.692810}%
\pgfsetfillcolor{currentfill}%
\pgfsetlinewidth{0.250937pt}%
\definecolor{currentstroke}{rgb}{1.000000,1.000000,1.000000}%
\pgfsetstrokecolor{currentstroke}%
\pgfsetdash{}{0pt}%
\pgfpathmoveto{\pgfqpoint{4.680000in}{4.360661in}}%
\pgfpathlineto{\pgfqpoint{4.767736in}{4.360661in}}%
\pgfpathlineto{\pgfqpoint{4.767736in}{4.272925in}}%
\pgfpathlineto{\pgfqpoint{4.680000in}{4.272925in}}%
\pgfpathlineto{\pgfqpoint{4.680000in}{4.360661in}}%
\pgfusepath{stroke,fill}%
\end{pgfscope}%
\begin{pgfscope}%
\pgfpathrectangle{\pgfqpoint{0.380943in}{4.185189in}}{\pgfqpoint{4.650000in}{0.614151in}}%
\pgfusepath{clip}%
\pgfsetbuttcap%
\pgfsetroundjoin%
\definecolor{currentfill}{rgb}{0.979654,0.837186,0.669619}%
\pgfsetfillcolor{currentfill}%
\pgfsetlinewidth{0.250937pt}%
\definecolor{currentstroke}{rgb}{1.000000,1.000000,1.000000}%
\pgfsetstrokecolor{currentstroke}%
\pgfsetdash{}{0pt}%
\pgfpathmoveto{\pgfqpoint{4.767736in}{4.360661in}}%
\pgfpathlineto{\pgfqpoint{4.855471in}{4.360661in}}%
\pgfpathlineto{\pgfqpoint{4.855471in}{4.272925in}}%
\pgfpathlineto{\pgfqpoint{4.767736in}{4.272925in}}%
\pgfpathlineto{\pgfqpoint{4.767736in}{4.360661in}}%
\pgfusepath{stroke,fill}%
\end{pgfscope}%
\begin{pgfscope}%
\pgfpathrectangle{\pgfqpoint{0.380943in}{4.185189in}}{\pgfqpoint{4.650000in}{0.614151in}}%
\pgfusepath{clip}%
\pgfsetbuttcap%
\pgfsetroundjoin%
\definecolor{currentfill}{rgb}{0.962414,0.923552,0.722891}%
\pgfsetfillcolor{currentfill}%
\pgfsetlinewidth{0.250937pt}%
\definecolor{currentstroke}{rgb}{1.000000,1.000000,1.000000}%
\pgfsetstrokecolor{currentstroke}%
\pgfsetdash{}{0pt}%
\pgfpathmoveto{\pgfqpoint{4.855471in}{4.360661in}}%
\pgfpathlineto{\pgfqpoint{4.943207in}{4.360661in}}%
\pgfpathlineto{\pgfqpoint{4.943207in}{4.272925in}}%
\pgfpathlineto{\pgfqpoint{4.855471in}{4.272925in}}%
\pgfpathlineto{\pgfqpoint{4.855471in}{4.360661in}}%
\pgfusepath{stroke,fill}%
\end{pgfscope}%
\begin{pgfscope}%
\pgfpathrectangle{\pgfqpoint{0.380943in}{4.185189in}}{\pgfqpoint{4.650000in}{0.614151in}}%
\pgfusepath{clip}%
\pgfsetbuttcap%
\pgfsetroundjoin%
\pgfsetlinewidth{0.250937pt}%
\definecolor{currentstroke}{rgb}{1.000000,1.000000,1.000000}%
\pgfsetstrokecolor{currentstroke}%
\pgfsetdash{}{0pt}%
\pgfpathmoveto{\pgfqpoint{4.943207in}{4.360661in}}%
\pgfpathlineto{\pgfqpoint{5.030943in}{4.360661in}}%
\pgfpathlineto{\pgfqpoint{5.030943in}{4.272925in}}%
\pgfpathlineto{\pgfqpoint{4.943207in}{4.272925in}}%
\pgfpathlineto{\pgfqpoint{4.943207in}{4.360661in}}%
\pgfusepath{stroke}%
\end{pgfscope}%
\begin{pgfscope}%
\pgfpathrectangle{\pgfqpoint{0.380943in}{4.185189in}}{\pgfqpoint{4.650000in}{0.614151in}}%
\pgfusepath{clip}%
\pgfsetbuttcap%
\pgfsetroundjoin%
\definecolor{currentfill}{rgb}{0.965444,0.906113,0.711757}%
\pgfsetfillcolor{currentfill}%
\pgfsetlinewidth{0.250937pt}%
\definecolor{currentstroke}{rgb}{1.000000,1.000000,1.000000}%
\pgfsetstrokecolor{currentstroke}%
\pgfsetdash{}{0pt}%
\pgfpathmoveto{\pgfqpoint{0.380943in}{4.272925in}}%
\pgfpathlineto{\pgfqpoint{0.468679in}{4.272925in}}%
\pgfpathlineto{\pgfqpoint{0.468679in}{4.185189in}}%
\pgfpathlineto{\pgfqpoint{0.380943in}{4.185189in}}%
\pgfpathlineto{\pgfqpoint{0.380943in}{4.272925in}}%
\pgfusepath{stroke,fill}%
\end{pgfscope}%
\begin{pgfscope}%
\pgfpathrectangle{\pgfqpoint{0.380943in}{4.185189in}}{\pgfqpoint{4.650000in}{0.614151in}}%
\pgfusepath{clip}%
\pgfsetbuttcap%
\pgfsetroundjoin%
\definecolor{currentfill}{rgb}{0.968166,0.945882,0.748604}%
\pgfsetfillcolor{currentfill}%
\pgfsetlinewidth{0.250937pt}%
\definecolor{currentstroke}{rgb}{1.000000,1.000000,1.000000}%
\pgfsetstrokecolor{currentstroke}%
\pgfsetdash{}{0pt}%
\pgfpathmoveto{\pgfqpoint{0.468679in}{4.272925in}}%
\pgfpathlineto{\pgfqpoint{0.556415in}{4.272925in}}%
\pgfpathlineto{\pgfqpoint{0.556415in}{4.185189in}}%
\pgfpathlineto{\pgfqpoint{0.468679in}{4.185189in}}%
\pgfpathlineto{\pgfqpoint{0.468679in}{4.272925in}}%
\pgfusepath{stroke,fill}%
\end{pgfscope}%
\begin{pgfscope}%
\pgfpathrectangle{\pgfqpoint{0.380943in}{4.185189in}}{\pgfqpoint{4.650000in}{0.614151in}}%
\pgfusepath{clip}%
\pgfsetbuttcap%
\pgfsetroundjoin%
\definecolor{currentfill}{rgb}{0.965444,0.906113,0.711757}%
\pgfsetfillcolor{currentfill}%
\pgfsetlinewidth{0.250937pt}%
\definecolor{currentstroke}{rgb}{1.000000,1.000000,1.000000}%
\pgfsetstrokecolor{currentstroke}%
\pgfsetdash{}{0pt}%
\pgfpathmoveto{\pgfqpoint{0.556415in}{4.272925in}}%
\pgfpathlineto{\pgfqpoint{0.644151in}{4.272925in}}%
\pgfpathlineto{\pgfqpoint{0.644151in}{4.185189in}}%
\pgfpathlineto{\pgfqpoint{0.556415in}{4.185189in}}%
\pgfpathlineto{\pgfqpoint{0.556415in}{4.272925in}}%
\pgfusepath{stroke,fill}%
\end{pgfscope}%
\begin{pgfscope}%
\pgfpathrectangle{\pgfqpoint{0.380943in}{4.185189in}}{\pgfqpoint{4.650000in}{0.614151in}}%
\pgfusepath{clip}%
\pgfsetbuttcap%
\pgfsetroundjoin%
\definecolor{currentfill}{rgb}{0.962414,0.923552,0.722891}%
\pgfsetfillcolor{currentfill}%
\pgfsetlinewidth{0.250937pt}%
\definecolor{currentstroke}{rgb}{1.000000,1.000000,1.000000}%
\pgfsetstrokecolor{currentstroke}%
\pgfsetdash{}{0pt}%
\pgfpathmoveto{\pgfqpoint{0.644151in}{4.272925in}}%
\pgfpathlineto{\pgfqpoint{0.731886in}{4.272925in}}%
\pgfpathlineto{\pgfqpoint{0.731886in}{4.185189in}}%
\pgfpathlineto{\pgfqpoint{0.644151in}{4.185189in}}%
\pgfpathlineto{\pgfqpoint{0.644151in}{4.272925in}}%
\pgfusepath{stroke,fill}%
\end{pgfscope}%
\begin{pgfscope}%
\pgfpathrectangle{\pgfqpoint{0.380943in}{4.185189in}}{\pgfqpoint{4.650000in}{0.614151in}}%
\pgfusepath{clip}%
\pgfsetbuttcap%
\pgfsetroundjoin%
\definecolor{currentfill}{rgb}{0.979654,0.837186,0.669619}%
\pgfsetfillcolor{currentfill}%
\pgfsetlinewidth{0.250937pt}%
\definecolor{currentstroke}{rgb}{1.000000,1.000000,1.000000}%
\pgfsetstrokecolor{currentstroke}%
\pgfsetdash{}{0pt}%
\pgfpathmoveto{\pgfqpoint{0.731886in}{4.272925in}}%
\pgfpathlineto{\pgfqpoint{0.819622in}{4.272925in}}%
\pgfpathlineto{\pgfqpoint{0.819622in}{4.185189in}}%
\pgfpathlineto{\pgfqpoint{0.731886in}{4.185189in}}%
\pgfpathlineto{\pgfqpoint{0.731886in}{4.272925in}}%
\pgfusepath{stroke,fill}%
\end{pgfscope}%
\begin{pgfscope}%
\pgfpathrectangle{\pgfqpoint{0.380943in}{4.185189in}}{\pgfqpoint{4.650000in}{0.614151in}}%
\pgfusepath{clip}%
\pgfsetbuttcap%
\pgfsetroundjoin%
\definecolor{currentfill}{rgb}{0.968166,0.945882,0.748604}%
\pgfsetfillcolor{currentfill}%
\pgfsetlinewidth{0.250937pt}%
\definecolor{currentstroke}{rgb}{1.000000,1.000000,1.000000}%
\pgfsetstrokecolor{currentstroke}%
\pgfsetdash{}{0pt}%
\pgfpathmoveto{\pgfqpoint{0.819622in}{4.272925in}}%
\pgfpathlineto{\pgfqpoint{0.907358in}{4.272925in}}%
\pgfpathlineto{\pgfqpoint{0.907358in}{4.185189in}}%
\pgfpathlineto{\pgfqpoint{0.819622in}{4.185189in}}%
\pgfpathlineto{\pgfqpoint{0.819622in}{4.272925in}}%
\pgfusepath{stroke,fill}%
\end{pgfscope}%
\begin{pgfscope}%
\pgfpathrectangle{\pgfqpoint{0.380943in}{4.185189in}}{\pgfqpoint{4.650000in}{0.614151in}}%
\pgfusepath{clip}%
\pgfsetbuttcap%
\pgfsetroundjoin%
\definecolor{currentfill}{rgb}{0.991849,0.986144,0.810181}%
\pgfsetfillcolor{currentfill}%
\pgfsetlinewidth{0.250937pt}%
\definecolor{currentstroke}{rgb}{1.000000,1.000000,1.000000}%
\pgfsetstrokecolor{currentstroke}%
\pgfsetdash{}{0pt}%
\pgfpathmoveto{\pgfqpoint{0.907358in}{4.272925in}}%
\pgfpathlineto{\pgfqpoint{0.995094in}{4.272925in}}%
\pgfpathlineto{\pgfqpoint{0.995094in}{4.185189in}}%
\pgfpathlineto{\pgfqpoint{0.907358in}{4.185189in}}%
\pgfpathlineto{\pgfqpoint{0.907358in}{4.272925in}}%
\pgfusepath{stroke,fill}%
\end{pgfscope}%
\begin{pgfscope}%
\pgfpathrectangle{\pgfqpoint{0.380943in}{4.185189in}}{\pgfqpoint{4.650000in}{0.614151in}}%
\pgfusepath{clip}%
\pgfsetbuttcap%
\pgfsetroundjoin%
\definecolor{currentfill}{rgb}{0.991849,0.986144,0.810181}%
\pgfsetfillcolor{currentfill}%
\pgfsetlinewidth{0.250937pt}%
\definecolor{currentstroke}{rgb}{1.000000,1.000000,1.000000}%
\pgfsetstrokecolor{currentstroke}%
\pgfsetdash{}{0pt}%
\pgfpathmoveto{\pgfqpoint{0.995094in}{4.272925in}}%
\pgfpathlineto{\pgfqpoint{1.082830in}{4.272925in}}%
\pgfpathlineto{\pgfqpoint{1.082830in}{4.185189in}}%
\pgfpathlineto{\pgfqpoint{0.995094in}{4.185189in}}%
\pgfpathlineto{\pgfqpoint{0.995094in}{4.272925in}}%
\pgfusepath{stroke,fill}%
\end{pgfscope}%
\begin{pgfscope}%
\pgfpathrectangle{\pgfqpoint{0.380943in}{4.185189in}}{\pgfqpoint{4.650000in}{0.614151in}}%
\pgfusepath{clip}%
\pgfsetbuttcap%
\pgfsetroundjoin%
\definecolor{currentfill}{rgb}{1.000000,1.000000,0.870204}%
\pgfsetfillcolor{currentfill}%
\pgfsetlinewidth{0.250937pt}%
\definecolor{currentstroke}{rgb}{1.000000,1.000000,1.000000}%
\pgfsetstrokecolor{currentstroke}%
\pgfsetdash{}{0pt}%
\pgfpathmoveto{\pgfqpoint{1.082830in}{4.272925in}}%
\pgfpathlineto{\pgfqpoint{1.170566in}{4.272925in}}%
\pgfpathlineto{\pgfqpoint{1.170566in}{4.185189in}}%
\pgfpathlineto{\pgfqpoint{1.082830in}{4.185189in}}%
\pgfpathlineto{\pgfqpoint{1.082830in}{4.272925in}}%
\pgfusepath{stroke,fill}%
\end{pgfscope}%
\begin{pgfscope}%
\pgfpathrectangle{\pgfqpoint{0.380943in}{4.185189in}}{\pgfqpoint{4.650000in}{0.614151in}}%
\pgfusepath{clip}%
\pgfsetbuttcap%
\pgfsetroundjoin%
\definecolor{currentfill}{rgb}{0.962414,0.923552,0.722891}%
\pgfsetfillcolor{currentfill}%
\pgfsetlinewidth{0.250937pt}%
\definecolor{currentstroke}{rgb}{1.000000,1.000000,1.000000}%
\pgfsetstrokecolor{currentstroke}%
\pgfsetdash{}{0pt}%
\pgfpathmoveto{\pgfqpoint{1.170566in}{4.272925in}}%
\pgfpathlineto{\pgfqpoint{1.258302in}{4.272925in}}%
\pgfpathlineto{\pgfqpoint{1.258302in}{4.185189in}}%
\pgfpathlineto{\pgfqpoint{1.170566in}{4.185189in}}%
\pgfpathlineto{\pgfqpoint{1.170566in}{4.272925in}}%
\pgfusepath{stroke,fill}%
\end{pgfscope}%
\begin{pgfscope}%
\pgfpathrectangle{\pgfqpoint{0.380943in}{4.185189in}}{\pgfqpoint{4.650000in}{0.614151in}}%
\pgfusepath{clip}%
\pgfsetbuttcap%
\pgfsetroundjoin%
\definecolor{currentfill}{rgb}{0.991849,0.986144,0.810181}%
\pgfsetfillcolor{currentfill}%
\pgfsetlinewidth{0.250937pt}%
\definecolor{currentstroke}{rgb}{1.000000,1.000000,1.000000}%
\pgfsetstrokecolor{currentstroke}%
\pgfsetdash{}{0pt}%
\pgfpathmoveto{\pgfqpoint{1.258302in}{4.272925in}}%
\pgfpathlineto{\pgfqpoint{1.346037in}{4.272925in}}%
\pgfpathlineto{\pgfqpoint{1.346037in}{4.185189in}}%
\pgfpathlineto{\pgfqpoint{1.258302in}{4.185189in}}%
\pgfpathlineto{\pgfqpoint{1.258302in}{4.272925in}}%
\pgfusepath{stroke,fill}%
\end{pgfscope}%
\begin{pgfscope}%
\pgfpathrectangle{\pgfqpoint{0.380943in}{4.185189in}}{\pgfqpoint{4.650000in}{0.614151in}}%
\pgfusepath{clip}%
\pgfsetbuttcap%
\pgfsetroundjoin%
\definecolor{currentfill}{rgb}{0.962414,0.923552,0.722891}%
\pgfsetfillcolor{currentfill}%
\pgfsetlinewidth{0.250937pt}%
\definecolor{currentstroke}{rgb}{1.000000,1.000000,1.000000}%
\pgfsetstrokecolor{currentstroke}%
\pgfsetdash{}{0pt}%
\pgfpathmoveto{\pgfqpoint{1.346037in}{4.272925in}}%
\pgfpathlineto{\pgfqpoint{1.433773in}{4.272925in}}%
\pgfpathlineto{\pgfqpoint{1.433773in}{4.185189in}}%
\pgfpathlineto{\pgfqpoint{1.346037in}{4.185189in}}%
\pgfpathlineto{\pgfqpoint{1.346037in}{4.272925in}}%
\pgfusepath{stroke,fill}%
\end{pgfscope}%
\begin{pgfscope}%
\pgfpathrectangle{\pgfqpoint{0.380943in}{4.185189in}}{\pgfqpoint{4.650000in}{0.614151in}}%
\pgfusepath{clip}%
\pgfsetbuttcap%
\pgfsetroundjoin%
\definecolor{currentfill}{rgb}{0.991849,0.986144,0.810181}%
\pgfsetfillcolor{currentfill}%
\pgfsetlinewidth{0.250937pt}%
\definecolor{currentstroke}{rgb}{1.000000,1.000000,1.000000}%
\pgfsetstrokecolor{currentstroke}%
\pgfsetdash{}{0pt}%
\pgfpathmoveto{\pgfqpoint{1.433773in}{4.272925in}}%
\pgfpathlineto{\pgfqpoint{1.521509in}{4.272925in}}%
\pgfpathlineto{\pgfqpoint{1.521509in}{4.185189in}}%
\pgfpathlineto{\pgfqpoint{1.433773in}{4.185189in}}%
\pgfpathlineto{\pgfqpoint{1.433773in}{4.272925in}}%
\pgfusepath{stroke,fill}%
\end{pgfscope}%
\begin{pgfscope}%
\pgfpathrectangle{\pgfqpoint{0.380943in}{4.185189in}}{\pgfqpoint{4.650000in}{0.614151in}}%
\pgfusepath{clip}%
\pgfsetbuttcap%
\pgfsetroundjoin%
\definecolor{currentfill}{rgb}{0.991849,0.986144,0.810181}%
\pgfsetfillcolor{currentfill}%
\pgfsetlinewidth{0.250937pt}%
\definecolor{currentstroke}{rgb}{1.000000,1.000000,1.000000}%
\pgfsetstrokecolor{currentstroke}%
\pgfsetdash{}{0pt}%
\pgfpathmoveto{\pgfqpoint{1.521509in}{4.272925in}}%
\pgfpathlineto{\pgfqpoint{1.609245in}{4.272925in}}%
\pgfpathlineto{\pgfqpoint{1.609245in}{4.185189in}}%
\pgfpathlineto{\pgfqpoint{1.521509in}{4.185189in}}%
\pgfpathlineto{\pgfqpoint{1.521509in}{4.272925in}}%
\pgfusepath{stroke,fill}%
\end{pgfscope}%
\begin{pgfscope}%
\pgfpathrectangle{\pgfqpoint{0.380943in}{4.185189in}}{\pgfqpoint{4.650000in}{0.614151in}}%
\pgfusepath{clip}%
\pgfsetbuttcap%
\pgfsetroundjoin%
\definecolor{currentfill}{rgb}{0.972549,0.870588,0.692810}%
\pgfsetfillcolor{currentfill}%
\pgfsetlinewidth{0.250937pt}%
\definecolor{currentstroke}{rgb}{1.000000,1.000000,1.000000}%
\pgfsetstrokecolor{currentstroke}%
\pgfsetdash{}{0pt}%
\pgfpathmoveto{\pgfqpoint{1.609245in}{4.272925in}}%
\pgfpathlineto{\pgfqpoint{1.696981in}{4.272925in}}%
\pgfpathlineto{\pgfqpoint{1.696981in}{4.185189in}}%
\pgfpathlineto{\pgfqpoint{1.609245in}{4.185189in}}%
\pgfpathlineto{\pgfqpoint{1.609245in}{4.272925in}}%
\pgfusepath{stroke,fill}%
\end{pgfscope}%
\begin{pgfscope}%
\pgfpathrectangle{\pgfqpoint{0.380943in}{4.185189in}}{\pgfqpoint{4.650000in}{0.614151in}}%
\pgfusepath{clip}%
\pgfsetbuttcap%
\pgfsetroundjoin%
\definecolor{currentfill}{rgb}{1.000000,1.000000,0.929412}%
\pgfsetfillcolor{currentfill}%
\pgfsetlinewidth{0.250937pt}%
\definecolor{currentstroke}{rgb}{1.000000,1.000000,1.000000}%
\pgfsetstrokecolor{currentstroke}%
\pgfsetdash{}{0pt}%
\pgfpathmoveto{\pgfqpoint{1.696981in}{4.272925in}}%
\pgfpathlineto{\pgfqpoint{1.784717in}{4.272925in}}%
\pgfpathlineto{\pgfqpoint{1.784717in}{4.185189in}}%
\pgfpathlineto{\pgfqpoint{1.696981in}{4.185189in}}%
\pgfpathlineto{\pgfqpoint{1.696981in}{4.272925in}}%
\pgfusepath{stroke,fill}%
\end{pgfscope}%
\begin{pgfscope}%
\pgfpathrectangle{\pgfqpoint{0.380943in}{4.185189in}}{\pgfqpoint{4.650000in}{0.614151in}}%
\pgfusepath{clip}%
\pgfsetbuttcap%
\pgfsetroundjoin%
\definecolor{currentfill}{rgb}{0.991849,0.986144,0.810181}%
\pgfsetfillcolor{currentfill}%
\pgfsetlinewidth{0.250937pt}%
\definecolor{currentstroke}{rgb}{1.000000,1.000000,1.000000}%
\pgfsetstrokecolor{currentstroke}%
\pgfsetdash{}{0pt}%
\pgfpathmoveto{\pgfqpoint{1.784717in}{4.272925in}}%
\pgfpathlineto{\pgfqpoint{1.872452in}{4.272925in}}%
\pgfpathlineto{\pgfqpoint{1.872452in}{4.185189in}}%
\pgfpathlineto{\pgfqpoint{1.784717in}{4.185189in}}%
\pgfpathlineto{\pgfqpoint{1.784717in}{4.272925in}}%
\pgfusepath{stroke,fill}%
\end{pgfscope}%
\begin{pgfscope}%
\pgfpathrectangle{\pgfqpoint{0.380943in}{4.185189in}}{\pgfqpoint{4.650000in}{0.614151in}}%
\pgfusepath{clip}%
\pgfsetbuttcap%
\pgfsetroundjoin%
\definecolor{currentfill}{rgb}{0.962414,0.923552,0.722891}%
\pgfsetfillcolor{currentfill}%
\pgfsetlinewidth{0.250937pt}%
\definecolor{currentstroke}{rgb}{1.000000,1.000000,1.000000}%
\pgfsetstrokecolor{currentstroke}%
\pgfsetdash{}{0pt}%
\pgfpathmoveto{\pgfqpoint{1.872452in}{4.272925in}}%
\pgfpathlineto{\pgfqpoint{1.960188in}{4.272925in}}%
\pgfpathlineto{\pgfqpoint{1.960188in}{4.185189in}}%
\pgfpathlineto{\pgfqpoint{1.872452in}{4.185189in}}%
\pgfpathlineto{\pgfqpoint{1.872452in}{4.272925in}}%
\pgfusepath{stroke,fill}%
\end{pgfscope}%
\begin{pgfscope}%
\pgfpathrectangle{\pgfqpoint{0.380943in}{4.185189in}}{\pgfqpoint{4.650000in}{0.614151in}}%
\pgfusepath{clip}%
\pgfsetbuttcap%
\pgfsetroundjoin%
\definecolor{currentfill}{rgb}{0.968166,0.945882,0.748604}%
\pgfsetfillcolor{currentfill}%
\pgfsetlinewidth{0.250937pt}%
\definecolor{currentstroke}{rgb}{1.000000,1.000000,1.000000}%
\pgfsetstrokecolor{currentstroke}%
\pgfsetdash{}{0pt}%
\pgfpathmoveto{\pgfqpoint{1.960188in}{4.272925in}}%
\pgfpathlineto{\pgfqpoint{2.047924in}{4.272925in}}%
\pgfpathlineto{\pgfqpoint{2.047924in}{4.185189in}}%
\pgfpathlineto{\pgfqpoint{1.960188in}{4.185189in}}%
\pgfpathlineto{\pgfqpoint{1.960188in}{4.272925in}}%
\pgfusepath{stroke,fill}%
\end{pgfscope}%
\begin{pgfscope}%
\pgfpathrectangle{\pgfqpoint{0.380943in}{4.185189in}}{\pgfqpoint{4.650000in}{0.614151in}}%
\pgfusepath{clip}%
\pgfsetbuttcap%
\pgfsetroundjoin%
\definecolor{currentfill}{rgb}{1.000000,1.000000,0.870204}%
\pgfsetfillcolor{currentfill}%
\pgfsetlinewidth{0.250937pt}%
\definecolor{currentstroke}{rgb}{1.000000,1.000000,1.000000}%
\pgfsetstrokecolor{currentstroke}%
\pgfsetdash{}{0pt}%
\pgfpathmoveto{\pgfqpoint{2.047924in}{4.272925in}}%
\pgfpathlineto{\pgfqpoint{2.135660in}{4.272925in}}%
\pgfpathlineto{\pgfqpoint{2.135660in}{4.185189in}}%
\pgfpathlineto{\pgfqpoint{2.047924in}{4.185189in}}%
\pgfpathlineto{\pgfqpoint{2.047924in}{4.272925in}}%
\pgfusepath{stroke,fill}%
\end{pgfscope}%
\begin{pgfscope}%
\pgfpathrectangle{\pgfqpoint{0.380943in}{4.185189in}}{\pgfqpoint{4.650000in}{0.614151in}}%
\pgfusepath{clip}%
\pgfsetbuttcap%
\pgfsetroundjoin%
\definecolor{currentfill}{rgb}{1.000000,1.000000,0.870204}%
\pgfsetfillcolor{currentfill}%
\pgfsetlinewidth{0.250937pt}%
\definecolor{currentstroke}{rgb}{1.000000,1.000000,1.000000}%
\pgfsetstrokecolor{currentstroke}%
\pgfsetdash{}{0pt}%
\pgfpathmoveto{\pgfqpoint{2.135660in}{4.272925in}}%
\pgfpathlineto{\pgfqpoint{2.223396in}{4.272925in}}%
\pgfpathlineto{\pgfqpoint{2.223396in}{4.185189in}}%
\pgfpathlineto{\pgfqpoint{2.135660in}{4.185189in}}%
\pgfpathlineto{\pgfqpoint{2.135660in}{4.272925in}}%
\pgfusepath{stroke,fill}%
\end{pgfscope}%
\begin{pgfscope}%
\pgfpathrectangle{\pgfqpoint{0.380943in}{4.185189in}}{\pgfqpoint{4.650000in}{0.614151in}}%
\pgfusepath{clip}%
\pgfsetbuttcap%
\pgfsetroundjoin%
\definecolor{currentfill}{rgb}{0.991849,0.986144,0.810181}%
\pgfsetfillcolor{currentfill}%
\pgfsetlinewidth{0.250937pt}%
\definecolor{currentstroke}{rgb}{1.000000,1.000000,1.000000}%
\pgfsetstrokecolor{currentstroke}%
\pgfsetdash{}{0pt}%
\pgfpathmoveto{\pgfqpoint{2.223396in}{4.272925in}}%
\pgfpathlineto{\pgfqpoint{2.311132in}{4.272925in}}%
\pgfpathlineto{\pgfqpoint{2.311132in}{4.185189in}}%
\pgfpathlineto{\pgfqpoint{2.223396in}{4.185189in}}%
\pgfpathlineto{\pgfqpoint{2.223396in}{4.272925in}}%
\pgfusepath{stroke,fill}%
\end{pgfscope}%
\begin{pgfscope}%
\pgfpathrectangle{\pgfqpoint{0.380943in}{4.185189in}}{\pgfqpoint{4.650000in}{0.614151in}}%
\pgfusepath{clip}%
\pgfsetbuttcap%
\pgfsetroundjoin%
\definecolor{currentfill}{rgb}{0.991849,0.986144,0.810181}%
\pgfsetfillcolor{currentfill}%
\pgfsetlinewidth{0.250937pt}%
\definecolor{currentstroke}{rgb}{1.000000,1.000000,1.000000}%
\pgfsetstrokecolor{currentstroke}%
\pgfsetdash{}{0pt}%
\pgfpathmoveto{\pgfqpoint{2.311132in}{4.272925in}}%
\pgfpathlineto{\pgfqpoint{2.398868in}{4.272925in}}%
\pgfpathlineto{\pgfqpoint{2.398868in}{4.185189in}}%
\pgfpathlineto{\pgfqpoint{2.311132in}{4.185189in}}%
\pgfpathlineto{\pgfqpoint{2.311132in}{4.272925in}}%
\pgfusepath{stroke,fill}%
\end{pgfscope}%
\begin{pgfscope}%
\pgfpathrectangle{\pgfqpoint{0.380943in}{4.185189in}}{\pgfqpoint{4.650000in}{0.614151in}}%
\pgfusepath{clip}%
\pgfsetbuttcap%
\pgfsetroundjoin%
\definecolor{currentfill}{rgb}{0.991849,0.986144,0.810181}%
\pgfsetfillcolor{currentfill}%
\pgfsetlinewidth{0.250937pt}%
\definecolor{currentstroke}{rgb}{1.000000,1.000000,1.000000}%
\pgfsetstrokecolor{currentstroke}%
\pgfsetdash{}{0pt}%
\pgfpathmoveto{\pgfqpoint{2.398868in}{4.272925in}}%
\pgfpathlineto{\pgfqpoint{2.486603in}{4.272925in}}%
\pgfpathlineto{\pgfqpoint{2.486603in}{4.185189in}}%
\pgfpathlineto{\pgfqpoint{2.398868in}{4.185189in}}%
\pgfpathlineto{\pgfqpoint{2.398868in}{4.272925in}}%
\pgfusepath{stroke,fill}%
\end{pgfscope}%
\begin{pgfscope}%
\pgfpathrectangle{\pgfqpoint{0.380943in}{4.185189in}}{\pgfqpoint{4.650000in}{0.614151in}}%
\pgfusepath{clip}%
\pgfsetbuttcap%
\pgfsetroundjoin%
\definecolor{currentfill}{rgb}{1.000000,1.000000,0.929412}%
\pgfsetfillcolor{currentfill}%
\pgfsetlinewidth{0.250937pt}%
\definecolor{currentstroke}{rgb}{1.000000,1.000000,1.000000}%
\pgfsetstrokecolor{currentstroke}%
\pgfsetdash{}{0pt}%
\pgfpathmoveto{\pgfqpoint{2.486603in}{4.272925in}}%
\pgfpathlineto{\pgfqpoint{2.574339in}{4.272925in}}%
\pgfpathlineto{\pgfqpoint{2.574339in}{4.185189in}}%
\pgfpathlineto{\pgfqpoint{2.486603in}{4.185189in}}%
\pgfpathlineto{\pgfqpoint{2.486603in}{4.272925in}}%
\pgfusepath{stroke,fill}%
\end{pgfscope}%
\begin{pgfscope}%
\pgfpathrectangle{\pgfqpoint{0.380943in}{4.185189in}}{\pgfqpoint{4.650000in}{0.614151in}}%
\pgfusepath{clip}%
\pgfsetbuttcap%
\pgfsetroundjoin%
\definecolor{currentfill}{rgb}{0.991849,0.986144,0.810181}%
\pgfsetfillcolor{currentfill}%
\pgfsetlinewidth{0.250937pt}%
\definecolor{currentstroke}{rgb}{1.000000,1.000000,1.000000}%
\pgfsetstrokecolor{currentstroke}%
\pgfsetdash{}{0pt}%
\pgfpathmoveto{\pgfqpoint{2.574339in}{4.272925in}}%
\pgfpathlineto{\pgfqpoint{2.662075in}{4.272925in}}%
\pgfpathlineto{\pgfqpoint{2.662075in}{4.185189in}}%
\pgfpathlineto{\pgfqpoint{2.574339in}{4.185189in}}%
\pgfpathlineto{\pgfqpoint{2.574339in}{4.272925in}}%
\pgfusepath{stroke,fill}%
\end{pgfscope}%
\begin{pgfscope}%
\pgfpathrectangle{\pgfqpoint{0.380943in}{4.185189in}}{\pgfqpoint{4.650000in}{0.614151in}}%
\pgfusepath{clip}%
\pgfsetbuttcap%
\pgfsetroundjoin%
\definecolor{currentfill}{rgb}{1.000000,1.000000,0.870204}%
\pgfsetfillcolor{currentfill}%
\pgfsetlinewidth{0.250937pt}%
\definecolor{currentstroke}{rgb}{1.000000,1.000000,1.000000}%
\pgfsetstrokecolor{currentstroke}%
\pgfsetdash{}{0pt}%
\pgfpathmoveto{\pgfqpoint{2.662075in}{4.272925in}}%
\pgfpathlineto{\pgfqpoint{2.749811in}{4.272925in}}%
\pgfpathlineto{\pgfqpoint{2.749811in}{4.185189in}}%
\pgfpathlineto{\pgfqpoint{2.662075in}{4.185189in}}%
\pgfpathlineto{\pgfqpoint{2.662075in}{4.272925in}}%
\pgfusepath{stroke,fill}%
\end{pgfscope}%
\begin{pgfscope}%
\pgfpathrectangle{\pgfqpoint{0.380943in}{4.185189in}}{\pgfqpoint{4.650000in}{0.614151in}}%
\pgfusepath{clip}%
\pgfsetbuttcap%
\pgfsetroundjoin%
\definecolor{currentfill}{rgb}{1.000000,1.000000,0.870204}%
\pgfsetfillcolor{currentfill}%
\pgfsetlinewidth{0.250937pt}%
\definecolor{currentstroke}{rgb}{1.000000,1.000000,1.000000}%
\pgfsetstrokecolor{currentstroke}%
\pgfsetdash{}{0pt}%
\pgfpathmoveto{\pgfqpoint{2.749811in}{4.272925in}}%
\pgfpathlineto{\pgfqpoint{2.837547in}{4.272925in}}%
\pgfpathlineto{\pgfqpoint{2.837547in}{4.185189in}}%
\pgfpathlineto{\pgfqpoint{2.749811in}{4.185189in}}%
\pgfpathlineto{\pgfqpoint{2.749811in}{4.272925in}}%
\pgfusepath{stroke,fill}%
\end{pgfscope}%
\begin{pgfscope}%
\pgfpathrectangle{\pgfqpoint{0.380943in}{4.185189in}}{\pgfqpoint{4.650000in}{0.614151in}}%
\pgfusepath{clip}%
\pgfsetbuttcap%
\pgfsetroundjoin%
\definecolor{currentfill}{rgb}{0.991849,0.986144,0.810181}%
\pgfsetfillcolor{currentfill}%
\pgfsetlinewidth{0.250937pt}%
\definecolor{currentstroke}{rgb}{1.000000,1.000000,1.000000}%
\pgfsetstrokecolor{currentstroke}%
\pgfsetdash{}{0pt}%
\pgfpathmoveto{\pgfqpoint{2.837547in}{4.272925in}}%
\pgfpathlineto{\pgfqpoint{2.925283in}{4.272925in}}%
\pgfpathlineto{\pgfqpoint{2.925283in}{4.185189in}}%
\pgfpathlineto{\pgfqpoint{2.837547in}{4.185189in}}%
\pgfpathlineto{\pgfqpoint{2.837547in}{4.272925in}}%
\pgfusepath{stroke,fill}%
\end{pgfscope}%
\begin{pgfscope}%
\pgfpathrectangle{\pgfqpoint{0.380943in}{4.185189in}}{\pgfqpoint{4.650000in}{0.614151in}}%
\pgfusepath{clip}%
\pgfsetbuttcap%
\pgfsetroundjoin%
\definecolor{currentfill}{rgb}{1.000000,1.000000,0.870204}%
\pgfsetfillcolor{currentfill}%
\pgfsetlinewidth{0.250937pt}%
\definecolor{currentstroke}{rgb}{1.000000,1.000000,1.000000}%
\pgfsetstrokecolor{currentstroke}%
\pgfsetdash{}{0pt}%
\pgfpathmoveto{\pgfqpoint{2.925283in}{4.272925in}}%
\pgfpathlineto{\pgfqpoint{3.013019in}{4.272925in}}%
\pgfpathlineto{\pgfqpoint{3.013019in}{4.185189in}}%
\pgfpathlineto{\pgfqpoint{2.925283in}{4.185189in}}%
\pgfpathlineto{\pgfqpoint{2.925283in}{4.272925in}}%
\pgfusepath{stroke,fill}%
\end{pgfscope}%
\begin{pgfscope}%
\pgfpathrectangle{\pgfqpoint{0.380943in}{4.185189in}}{\pgfqpoint{4.650000in}{0.614151in}}%
\pgfusepath{clip}%
\pgfsetbuttcap%
\pgfsetroundjoin%
\definecolor{currentfill}{rgb}{1.000000,1.000000,0.870204}%
\pgfsetfillcolor{currentfill}%
\pgfsetlinewidth{0.250937pt}%
\definecolor{currentstroke}{rgb}{1.000000,1.000000,1.000000}%
\pgfsetstrokecolor{currentstroke}%
\pgfsetdash{}{0pt}%
\pgfpathmoveto{\pgfqpoint{3.013019in}{4.272925in}}%
\pgfpathlineto{\pgfqpoint{3.100754in}{4.272925in}}%
\pgfpathlineto{\pgfqpoint{3.100754in}{4.185189in}}%
\pgfpathlineto{\pgfqpoint{3.013019in}{4.185189in}}%
\pgfpathlineto{\pgfqpoint{3.013019in}{4.272925in}}%
\pgfusepath{stroke,fill}%
\end{pgfscope}%
\begin{pgfscope}%
\pgfpathrectangle{\pgfqpoint{0.380943in}{4.185189in}}{\pgfqpoint{4.650000in}{0.614151in}}%
\pgfusepath{clip}%
\pgfsetbuttcap%
\pgfsetroundjoin%
\definecolor{currentfill}{rgb}{1.000000,1.000000,0.870204}%
\pgfsetfillcolor{currentfill}%
\pgfsetlinewidth{0.250937pt}%
\definecolor{currentstroke}{rgb}{1.000000,1.000000,1.000000}%
\pgfsetstrokecolor{currentstroke}%
\pgfsetdash{}{0pt}%
\pgfpathmoveto{\pgfqpoint{3.100754in}{4.272925in}}%
\pgfpathlineto{\pgfqpoint{3.188490in}{4.272925in}}%
\pgfpathlineto{\pgfqpoint{3.188490in}{4.185189in}}%
\pgfpathlineto{\pgfqpoint{3.100754in}{4.185189in}}%
\pgfpathlineto{\pgfqpoint{3.100754in}{4.272925in}}%
\pgfusepath{stroke,fill}%
\end{pgfscope}%
\begin{pgfscope}%
\pgfpathrectangle{\pgfqpoint{0.380943in}{4.185189in}}{\pgfqpoint{4.650000in}{0.614151in}}%
\pgfusepath{clip}%
\pgfsetbuttcap%
\pgfsetroundjoin%
\definecolor{currentfill}{rgb}{0.991849,0.986144,0.810181}%
\pgfsetfillcolor{currentfill}%
\pgfsetlinewidth{0.250937pt}%
\definecolor{currentstroke}{rgb}{1.000000,1.000000,1.000000}%
\pgfsetstrokecolor{currentstroke}%
\pgfsetdash{}{0pt}%
\pgfpathmoveto{\pgfqpoint{3.188490in}{4.272925in}}%
\pgfpathlineto{\pgfqpoint{3.276226in}{4.272925in}}%
\pgfpathlineto{\pgfqpoint{3.276226in}{4.185189in}}%
\pgfpathlineto{\pgfqpoint{3.188490in}{4.185189in}}%
\pgfpathlineto{\pgfqpoint{3.188490in}{4.272925in}}%
\pgfusepath{stroke,fill}%
\end{pgfscope}%
\begin{pgfscope}%
\pgfpathrectangle{\pgfqpoint{0.380943in}{4.185189in}}{\pgfqpoint{4.650000in}{0.614151in}}%
\pgfusepath{clip}%
\pgfsetbuttcap%
\pgfsetroundjoin%
\definecolor{currentfill}{rgb}{1.000000,1.000000,0.929412}%
\pgfsetfillcolor{currentfill}%
\pgfsetlinewidth{0.250937pt}%
\definecolor{currentstroke}{rgb}{1.000000,1.000000,1.000000}%
\pgfsetstrokecolor{currentstroke}%
\pgfsetdash{}{0pt}%
\pgfpathmoveto{\pgfqpoint{3.276226in}{4.272925in}}%
\pgfpathlineto{\pgfqpoint{3.363962in}{4.272925in}}%
\pgfpathlineto{\pgfqpoint{3.363962in}{4.185189in}}%
\pgfpathlineto{\pgfqpoint{3.276226in}{4.185189in}}%
\pgfpathlineto{\pgfqpoint{3.276226in}{4.272925in}}%
\pgfusepath{stroke,fill}%
\end{pgfscope}%
\begin{pgfscope}%
\pgfpathrectangle{\pgfqpoint{0.380943in}{4.185189in}}{\pgfqpoint{4.650000in}{0.614151in}}%
\pgfusepath{clip}%
\pgfsetbuttcap%
\pgfsetroundjoin%
\definecolor{currentfill}{rgb}{0.968166,0.945882,0.748604}%
\pgfsetfillcolor{currentfill}%
\pgfsetlinewidth{0.250937pt}%
\definecolor{currentstroke}{rgb}{1.000000,1.000000,1.000000}%
\pgfsetstrokecolor{currentstroke}%
\pgfsetdash{}{0pt}%
\pgfpathmoveto{\pgfqpoint{3.363962in}{4.272925in}}%
\pgfpathlineto{\pgfqpoint{3.451698in}{4.272925in}}%
\pgfpathlineto{\pgfqpoint{3.451698in}{4.185189in}}%
\pgfpathlineto{\pgfqpoint{3.363962in}{4.185189in}}%
\pgfpathlineto{\pgfqpoint{3.363962in}{4.272925in}}%
\pgfusepath{stroke,fill}%
\end{pgfscope}%
\begin{pgfscope}%
\pgfpathrectangle{\pgfqpoint{0.380943in}{4.185189in}}{\pgfqpoint{4.650000in}{0.614151in}}%
\pgfusepath{clip}%
\pgfsetbuttcap%
\pgfsetroundjoin%
\definecolor{currentfill}{rgb}{0.962414,0.923552,0.722891}%
\pgfsetfillcolor{currentfill}%
\pgfsetlinewidth{0.250937pt}%
\definecolor{currentstroke}{rgb}{1.000000,1.000000,1.000000}%
\pgfsetstrokecolor{currentstroke}%
\pgfsetdash{}{0pt}%
\pgfpathmoveto{\pgfqpoint{3.451698in}{4.272925in}}%
\pgfpathlineto{\pgfqpoint{3.539434in}{4.272925in}}%
\pgfpathlineto{\pgfqpoint{3.539434in}{4.185189in}}%
\pgfpathlineto{\pgfqpoint{3.451698in}{4.185189in}}%
\pgfpathlineto{\pgfqpoint{3.451698in}{4.272925in}}%
\pgfusepath{stroke,fill}%
\end{pgfscope}%
\begin{pgfscope}%
\pgfpathrectangle{\pgfqpoint{0.380943in}{4.185189in}}{\pgfqpoint{4.650000in}{0.614151in}}%
\pgfusepath{clip}%
\pgfsetbuttcap%
\pgfsetroundjoin%
\definecolor{currentfill}{rgb}{0.962414,0.923552,0.722891}%
\pgfsetfillcolor{currentfill}%
\pgfsetlinewidth{0.250937pt}%
\definecolor{currentstroke}{rgb}{1.000000,1.000000,1.000000}%
\pgfsetstrokecolor{currentstroke}%
\pgfsetdash{}{0pt}%
\pgfpathmoveto{\pgfqpoint{3.539434in}{4.272925in}}%
\pgfpathlineto{\pgfqpoint{3.627169in}{4.272925in}}%
\pgfpathlineto{\pgfqpoint{3.627169in}{4.185189in}}%
\pgfpathlineto{\pgfqpoint{3.539434in}{4.185189in}}%
\pgfpathlineto{\pgfqpoint{3.539434in}{4.272925in}}%
\pgfusepath{stroke,fill}%
\end{pgfscope}%
\begin{pgfscope}%
\pgfpathrectangle{\pgfqpoint{0.380943in}{4.185189in}}{\pgfqpoint{4.650000in}{0.614151in}}%
\pgfusepath{clip}%
\pgfsetbuttcap%
\pgfsetroundjoin%
\definecolor{currentfill}{rgb}{0.962414,0.923552,0.722891}%
\pgfsetfillcolor{currentfill}%
\pgfsetlinewidth{0.250937pt}%
\definecolor{currentstroke}{rgb}{1.000000,1.000000,1.000000}%
\pgfsetstrokecolor{currentstroke}%
\pgfsetdash{}{0pt}%
\pgfpathmoveto{\pgfqpoint{3.627169in}{4.272925in}}%
\pgfpathlineto{\pgfqpoint{3.714905in}{4.272925in}}%
\pgfpathlineto{\pgfqpoint{3.714905in}{4.185189in}}%
\pgfpathlineto{\pgfqpoint{3.627169in}{4.185189in}}%
\pgfpathlineto{\pgfqpoint{3.627169in}{4.272925in}}%
\pgfusepath{stroke,fill}%
\end{pgfscope}%
\begin{pgfscope}%
\pgfpathrectangle{\pgfqpoint{0.380943in}{4.185189in}}{\pgfqpoint{4.650000in}{0.614151in}}%
\pgfusepath{clip}%
\pgfsetbuttcap%
\pgfsetroundjoin%
\definecolor{currentfill}{rgb}{0.991849,0.986144,0.810181}%
\pgfsetfillcolor{currentfill}%
\pgfsetlinewidth{0.250937pt}%
\definecolor{currentstroke}{rgb}{1.000000,1.000000,1.000000}%
\pgfsetstrokecolor{currentstroke}%
\pgfsetdash{}{0pt}%
\pgfpathmoveto{\pgfqpoint{3.714905in}{4.272925in}}%
\pgfpathlineto{\pgfqpoint{3.802641in}{4.272925in}}%
\pgfpathlineto{\pgfqpoint{3.802641in}{4.185189in}}%
\pgfpathlineto{\pgfqpoint{3.714905in}{4.185189in}}%
\pgfpathlineto{\pgfqpoint{3.714905in}{4.272925in}}%
\pgfusepath{stroke,fill}%
\end{pgfscope}%
\begin{pgfscope}%
\pgfpathrectangle{\pgfqpoint{0.380943in}{4.185189in}}{\pgfqpoint{4.650000in}{0.614151in}}%
\pgfusepath{clip}%
\pgfsetbuttcap%
\pgfsetroundjoin%
\definecolor{currentfill}{rgb}{0.991849,0.986144,0.810181}%
\pgfsetfillcolor{currentfill}%
\pgfsetlinewidth{0.250937pt}%
\definecolor{currentstroke}{rgb}{1.000000,1.000000,1.000000}%
\pgfsetstrokecolor{currentstroke}%
\pgfsetdash{}{0pt}%
\pgfpathmoveto{\pgfqpoint{3.802641in}{4.272925in}}%
\pgfpathlineto{\pgfqpoint{3.890377in}{4.272925in}}%
\pgfpathlineto{\pgfqpoint{3.890377in}{4.185189in}}%
\pgfpathlineto{\pgfqpoint{3.802641in}{4.185189in}}%
\pgfpathlineto{\pgfqpoint{3.802641in}{4.272925in}}%
\pgfusepath{stroke,fill}%
\end{pgfscope}%
\begin{pgfscope}%
\pgfpathrectangle{\pgfqpoint{0.380943in}{4.185189in}}{\pgfqpoint{4.650000in}{0.614151in}}%
\pgfusepath{clip}%
\pgfsetbuttcap%
\pgfsetroundjoin%
\definecolor{currentfill}{rgb}{0.968166,0.945882,0.748604}%
\pgfsetfillcolor{currentfill}%
\pgfsetlinewidth{0.250937pt}%
\definecolor{currentstroke}{rgb}{1.000000,1.000000,1.000000}%
\pgfsetstrokecolor{currentstroke}%
\pgfsetdash{}{0pt}%
\pgfpathmoveto{\pgfqpoint{3.890377in}{4.272925in}}%
\pgfpathlineto{\pgfqpoint{3.978113in}{4.272925in}}%
\pgfpathlineto{\pgfqpoint{3.978113in}{4.185189in}}%
\pgfpathlineto{\pgfqpoint{3.890377in}{4.185189in}}%
\pgfpathlineto{\pgfqpoint{3.890377in}{4.272925in}}%
\pgfusepath{stroke,fill}%
\end{pgfscope}%
\begin{pgfscope}%
\pgfpathrectangle{\pgfqpoint{0.380943in}{4.185189in}}{\pgfqpoint{4.650000in}{0.614151in}}%
\pgfusepath{clip}%
\pgfsetbuttcap%
\pgfsetroundjoin%
\definecolor{currentfill}{rgb}{0.991849,0.986144,0.810181}%
\pgfsetfillcolor{currentfill}%
\pgfsetlinewidth{0.250937pt}%
\definecolor{currentstroke}{rgb}{1.000000,1.000000,1.000000}%
\pgfsetstrokecolor{currentstroke}%
\pgfsetdash{}{0pt}%
\pgfpathmoveto{\pgfqpoint{3.978113in}{4.272925in}}%
\pgfpathlineto{\pgfqpoint{4.065849in}{4.272925in}}%
\pgfpathlineto{\pgfqpoint{4.065849in}{4.185189in}}%
\pgfpathlineto{\pgfqpoint{3.978113in}{4.185189in}}%
\pgfpathlineto{\pgfqpoint{3.978113in}{4.272925in}}%
\pgfusepath{stroke,fill}%
\end{pgfscope}%
\begin{pgfscope}%
\pgfpathrectangle{\pgfqpoint{0.380943in}{4.185189in}}{\pgfqpoint{4.650000in}{0.614151in}}%
\pgfusepath{clip}%
\pgfsetbuttcap%
\pgfsetroundjoin%
\definecolor{currentfill}{rgb}{0.965444,0.906113,0.711757}%
\pgfsetfillcolor{currentfill}%
\pgfsetlinewidth{0.250937pt}%
\definecolor{currentstroke}{rgb}{1.000000,1.000000,1.000000}%
\pgfsetstrokecolor{currentstroke}%
\pgfsetdash{}{0pt}%
\pgfpathmoveto{\pgfqpoint{4.065849in}{4.272925in}}%
\pgfpathlineto{\pgfqpoint{4.153585in}{4.272925in}}%
\pgfpathlineto{\pgfqpoint{4.153585in}{4.185189in}}%
\pgfpathlineto{\pgfqpoint{4.065849in}{4.185189in}}%
\pgfpathlineto{\pgfqpoint{4.065849in}{4.272925in}}%
\pgfusepath{stroke,fill}%
\end{pgfscope}%
\begin{pgfscope}%
\pgfpathrectangle{\pgfqpoint{0.380943in}{4.185189in}}{\pgfqpoint{4.650000in}{0.614151in}}%
\pgfusepath{clip}%
\pgfsetbuttcap%
\pgfsetroundjoin%
\definecolor{currentfill}{rgb}{1.000000,1.000000,0.870204}%
\pgfsetfillcolor{currentfill}%
\pgfsetlinewidth{0.250937pt}%
\definecolor{currentstroke}{rgb}{1.000000,1.000000,1.000000}%
\pgfsetstrokecolor{currentstroke}%
\pgfsetdash{}{0pt}%
\pgfpathmoveto{\pgfqpoint{4.153585in}{4.272925in}}%
\pgfpathlineto{\pgfqpoint{4.241320in}{4.272925in}}%
\pgfpathlineto{\pgfqpoint{4.241320in}{4.185189in}}%
\pgfpathlineto{\pgfqpoint{4.153585in}{4.185189in}}%
\pgfpathlineto{\pgfqpoint{4.153585in}{4.272925in}}%
\pgfusepath{stroke,fill}%
\end{pgfscope}%
\begin{pgfscope}%
\pgfpathrectangle{\pgfqpoint{0.380943in}{4.185189in}}{\pgfqpoint{4.650000in}{0.614151in}}%
\pgfusepath{clip}%
\pgfsetbuttcap%
\pgfsetroundjoin%
\definecolor{currentfill}{rgb}{0.991849,0.986144,0.810181}%
\pgfsetfillcolor{currentfill}%
\pgfsetlinewidth{0.250937pt}%
\definecolor{currentstroke}{rgb}{1.000000,1.000000,1.000000}%
\pgfsetstrokecolor{currentstroke}%
\pgfsetdash{}{0pt}%
\pgfpathmoveto{\pgfqpoint{4.241320in}{4.272925in}}%
\pgfpathlineto{\pgfqpoint{4.329056in}{4.272925in}}%
\pgfpathlineto{\pgfqpoint{4.329056in}{4.185189in}}%
\pgfpathlineto{\pgfqpoint{4.241320in}{4.185189in}}%
\pgfpathlineto{\pgfqpoint{4.241320in}{4.272925in}}%
\pgfusepath{stroke,fill}%
\end{pgfscope}%
\begin{pgfscope}%
\pgfpathrectangle{\pgfqpoint{0.380943in}{4.185189in}}{\pgfqpoint{4.650000in}{0.614151in}}%
\pgfusepath{clip}%
\pgfsetbuttcap%
\pgfsetroundjoin%
\definecolor{currentfill}{rgb}{0.991849,0.986144,0.810181}%
\pgfsetfillcolor{currentfill}%
\pgfsetlinewidth{0.250937pt}%
\definecolor{currentstroke}{rgb}{1.000000,1.000000,1.000000}%
\pgfsetstrokecolor{currentstroke}%
\pgfsetdash{}{0pt}%
\pgfpathmoveto{\pgfqpoint{4.329056in}{4.272925in}}%
\pgfpathlineto{\pgfqpoint{4.416792in}{4.272925in}}%
\pgfpathlineto{\pgfqpoint{4.416792in}{4.185189in}}%
\pgfpathlineto{\pgfqpoint{4.329056in}{4.185189in}}%
\pgfpathlineto{\pgfqpoint{4.329056in}{4.272925in}}%
\pgfusepath{stroke,fill}%
\end{pgfscope}%
\begin{pgfscope}%
\pgfpathrectangle{\pgfqpoint{0.380943in}{4.185189in}}{\pgfqpoint{4.650000in}{0.614151in}}%
\pgfusepath{clip}%
\pgfsetbuttcap%
\pgfsetroundjoin%
\definecolor{currentfill}{rgb}{1.000000,1.000000,0.929412}%
\pgfsetfillcolor{currentfill}%
\pgfsetlinewidth{0.250937pt}%
\definecolor{currentstroke}{rgb}{1.000000,1.000000,1.000000}%
\pgfsetstrokecolor{currentstroke}%
\pgfsetdash{}{0pt}%
\pgfpathmoveto{\pgfqpoint{4.416792in}{4.272925in}}%
\pgfpathlineto{\pgfqpoint{4.504528in}{4.272925in}}%
\pgfpathlineto{\pgfqpoint{4.504528in}{4.185189in}}%
\pgfpathlineto{\pgfqpoint{4.416792in}{4.185189in}}%
\pgfpathlineto{\pgfqpoint{4.416792in}{4.272925in}}%
\pgfusepath{stroke,fill}%
\end{pgfscope}%
\begin{pgfscope}%
\pgfpathrectangle{\pgfqpoint{0.380943in}{4.185189in}}{\pgfqpoint{4.650000in}{0.614151in}}%
\pgfusepath{clip}%
\pgfsetbuttcap%
\pgfsetroundjoin%
\definecolor{currentfill}{rgb}{0.991849,0.986144,0.810181}%
\pgfsetfillcolor{currentfill}%
\pgfsetlinewidth{0.250937pt}%
\definecolor{currentstroke}{rgb}{1.000000,1.000000,1.000000}%
\pgfsetstrokecolor{currentstroke}%
\pgfsetdash{}{0pt}%
\pgfpathmoveto{\pgfqpoint{4.504528in}{4.272925in}}%
\pgfpathlineto{\pgfqpoint{4.592264in}{4.272925in}}%
\pgfpathlineto{\pgfqpoint{4.592264in}{4.185189in}}%
\pgfpathlineto{\pgfqpoint{4.504528in}{4.185189in}}%
\pgfpathlineto{\pgfqpoint{4.504528in}{4.272925in}}%
\pgfusepath{stroke,fill}%
\end{pgfscope}%
\begin{pgfscope}%
\pgfpathrectangle{\pgfqpoint{0.380943in}{4.185189in}}{\pgfqpoint{4.650000in}{0.614151in}}%
\pgfusepath{clip}%
\pgfsetbuttcap%
\pgfsetroundjoin%
\definecolor{currentfill}{rgb}{1.000000,1.000000,0.929412}%
\pgfsetfillcolor{currentfill}%
\pgfsetlinewidth{0.250937pt}%
\definecolor{currentstroke}{rgb}{1.000000,1.000000,1.000000}%
\pgfsetstrokecolor{currentstroke}%
\pgfsetdash{}{0pt}%
\pgfpathmoveto{\pgfqpoint{4.592264in}{4.272925in}}%
\pgfpathlineto{\pgfqpoint{4.680000in}{4.272925in}}%
\pgfpathlineto{\pgfqpoint{4.680000in}{4.185189in}}%
\pgfpathlineto{\pgfqpoint{4.592264in}{4.185189in}}%
\pgfpathlineto{\pgfqpoint{4.592264in}{4.272925in}}%
\pgfusepath{stroke,fill}%
\end{pgfscope}%
\begin{pgfscope}%
\pgfpathrectangle{\pgfqpoint{0.380943in}{4.185189in}}{\pgfqpoint{4.650000in}{0.614151in}}%
\pgfusepath{clip}%
\pgfsetbuttcap%
\pgfsetroundjoin%
\definecolor{currentfill}{rgb}{0.962414,0.923552,0.722891}%
\pgfsetfillcolor{currentfill}%
\pgfsetlinewidth{0.250937pt}%
\definecolor{currentstroke}{rgb}{1.000000,1.000000,1.000000}%
\pgfsetstrokecolor{currentstroke}%
\pgfsetdash{}{0pt}%
\pgfpathmoveto{\pgfqpoint{4.680000in}{4.272925in}}%
\pgfpathlineto{\pgfqpoint{4.767736in}{4.272925in}}%
\pgfpathlineto{\pgfqpoint{4.767736in}{4.185189in}}%
\pgfpathlineto{\pgfqpoint{4.680000in}{4.185189in}}%
\pgfpathlineto{\pgfqpoint{4.680000in}{4.272925in}}%
\pgfusepath{stroke,fill}%
\end{pgfscope}%
\begin{pgfscope}%
\pgfpathrectangle{\pgfqpoint{0.380943in}{4.185189in}}{\pgfqpoint{4.650000in}{0.614151in}}%
\pgfusepath{clip}%
\pgfsetbuttcap%
\pgfsetroundjoin%
\definecolor{currentfill}{rgb}{0.962414,0.923552,0.722891}%
\pgfsetfillcolor{currentfill}%
\pgfsetlinewidth{0.250937pt}%
\definecolor{currentstroke}{rgb}{1.000000,1.000000,1.000000}%
\pgfsetstrokecolor{currentstroke}%
\pgfsetdash{}{0pt}%
\pgfpathmoveto{\pgfqpoint{4.767736in}{4.272925in}}%
\pgfpathlineto{\pgfqpoint{4.855471in}{4.272925in}}%
\pgfpathlineto{\pgfqpoint{4.855471in}{4.185189in}}%
\pgfpathlineto{\pgfqpoint{4.767736in}{4.185189in}}%
\pgfpathlineto{\pgfqpoint{4.767736in}{4.272925in}}%
\pgfusepath{stroke,fill}%
\end{pgfscope}%
\begin{pgfscope}%
\pgfpathrectangle{\pgfqpoint{0.380943in}{4.185189in}}{\pgfqpoint{4.650000in}{0.614151in}}%
\pgfusepath{clip}%
\pgfsetbuttcap%
\pgfsetroundjoin%
\definecolor{currentfill}{rgb}{0.991849,0.986144,0.810181}%
\pgfsetfillcolor{currentfill}%
\pgfsetlinewidth{0.250937pt}%
\definecolor{currentstroke}{rgb}{1.000000,1.000000,1.000000}%
\pgfsetstrokecolor{currentstroke}%
\pgfsetdash{}{0pt}%
\pgfpathmoveto{\pgfqpoint{4.855471in}{4.272925in}}%
\pgfpathlineto{\pgfqpoint{4.943207in}{4.272925in}}%
\pgfpathlineto{\pgfqpoint{4.943207in}{4.185189in}}%
\pgfpathlineto{\pgfqpoint{4.855471in}{4.185189in}}%
\pgfpathlineto{\pgfqpoint{4.855471in}{4.272925in}}%
\pgfusepath{stroke,fill}%
\end{pgfscope}%
\begin{pgfscope}%
\pgfpathrectangle{\pgfqpoint{0.380943in}{4.185189in}}{\pgfqpoint{4.650000in}{0.614151in}}%
\pgfusepath{clip}%
\pgfsetbuttcap%
\pgfsetroundjoin%
\pgfsetlinewidth{0.250937pt}%
\definecolor{currentstroke}{rgb}{1.000000,1.000000,1.000000}%
\pgfsetstrokecolor{currentstroke}%
\pgfsetdash{}{0pt}%
\pgfpathmoveto{\pgfqpoint{4.943207in}{4.272925in}}%
\pgfpathlineto{\pgfqpoint{5.030943in}{4.272925in}}%
\pgfpathlineto{\pgfqpoint{5.030943in}{4.185189in}}%
\pgfpathlineto{\pgfqpoint{4.943207in}{4.185189in}}%
\pgfpathlineto{\pgfqpoint{4.943207in}{4.272925in}}%
\pgfusepath{stroke}%
\end{pgfscope}%
\begin{pgfscope}%
\pgfsetbuttcap%
\pgfsetroundjoin%
\definecolor{currentfill}{rgb}{0.000000,0.000000,0.000000}%
\pgfsetfillcolor{currentfill}%
\pgfsetlinewidth{0.803000pt}%
\definecolor{currentstroke}{rgb}{0.000000,0.000000,0.000000}%
\pgfsetstrokecolor{currentstroke}%
\pgfsetdash{}{0pt}%
\pgfsys@defobject{currentmarker}{\pgfqpoint{0.000000in}{-0.048611in}}{\pgfqpoint{0.000000in}{0.000000in}}{%
\pgfpathmoveto{\pgfqpoint{0.000000in}{0.000000in}}%
\pgfpathlineto{\pgfqpoint{0.000000in}{-0.048611in}}%
\pgfusepath{stroke,fill}%
}%
\begin{pgfscope}%
\pgfsys@transformshift{0.600283in}{4.185189in}%
\pgfsys@useobject{currentmarker}{}%
\end{pgfscope}%
\end{pgfscope}%
\begin{pgfscope}%
\definecolor{textcolor}{rgb}{0.000000,0.000000,0.000000}%
\pgfsetstrokecolor{textcolor}%
\pgfsetfillcolor{textcolor}%
\pgftext[x=0.600283in,y=4.087967in,,top]{\color{textcolor}\rmfamily\fontsize{8.000000}{9.600000}\selectfont Jan}%
\end{pgfscope}%
\begin{pgfscope}%
\pgfsetbuttcap%
\pgfsetroundjoin%
\definecolor{currentfill}{rgb}{0.000000,0.000000,0.000000}%
\pgfsetfillcolor{currentfill}%
\pgfsetlinewidth{0.803000pt}%
\definecolor{currentstroke}{rgb}{0.000000,0.000000,0.000000}%
\pgfsetstrokecolor{currentstroke}%
\pgfsetdash{}{0pt}%
\pgfsys@defobject{currentmarker}{\pgfqpoint{0.000000in}{-0.048611in}}{\pgfqpoint{0.000000in}{0.000000in}}{%
\pgfpathmoveto{\pgfqpoint{0.000000in}{0.000000in}}%
\pgfpathlineto{\pgfqpoint{0.000000in}{-0.048611in}}%
\pgfusepath{stroke,fill}%
}%
\begin{pgfscope}%
\pgfsys@transformshift{0.951226in}{4.185189in}%
\pgfsys@useobject{currentmarker}{}%
\end{pgfscope}%
\end{pgfscope}%
\begin{pgfscope}%
\definecolor{textcolor}{rgb}{0.000000,0.000000,0.000000}%
\pgfsetstrokecolor{textcolor}%
\pgfsetfillcolor{textcolor}%
\pgftext[x=0.951226in,y=4.087967in,,top]{\color{textcolor}\rmfamily\fontsize{8.000000}{9.600000}\selectfont Feb}%
\end{pgfscope}%
\begin{pgfscope}%
\pgfsetbuttcap%
\pgfsetroundjoin%
\definecolor{currentfill}{rgb}{0.000000,0.000000,0.000000}%
\pgfsetfillcolor{currentfill}%
\pgfsetlinewidth{0.803000pt}%
\definecolor{currentstroke}{rgb}{0.000000,0.000000,0.000000}%
\pgfsetstrokecolor{currentstroke}%
\pgfsetdash{}{0pt}%
\pgfsys@defobject{currentmarker}{\pgfqpoint{0.000000in}{-0.048611in}}{\pgfqpoint{0.000000in}{0.000000in}}{%
\pgfpathmoveto{\pgfqpoint{0.000000in}{0.000000in}}%
\pgfpathlineto{\pgfqpoint{0.000000in}{-0.048611in}}%
\pgfusepath{stroke,fill}%
}%
\begin{pgfscope}%
\pgfsys@transformshift{1.302169in}{4.185189in}%
\pgfsys@useobject{currentmarker}{}%
\end{pgfscope}%
\end{pgfscope}%
\begin{pgfscope}%
\definecolor{textcolor}{rgb}{0.000000,0.000000,0.000000}%
\pgfsetstrokecolor{textcolor}%
\pgfsetfillcolor{textcolor}%
\pgftext[x=1.302169in,y=4.087967in,,top]{\color{textcolor}\rmfamily\fontsize{8.000000}{9.600000}\selectfont Mar}%
\end{pgfscope}%
\begin{pgfscope}%
\pgfsetbuttcap%
\pgfsetroundjoin%
\definecolor{currentfill}{rgb}{0.000000,0.000000,0.000000}%
\pgfsetfillcolor{currentfill}%
\pgfsetlinewidth{0.803000pt}%
\definecolor{currentstroke}{rgb}{0.000000,0.000000,0.000000}%
\pgfsetstrokecolor{currentstroke}%
\pgfsetdash{}{0pt}%
\pgfsys@defobject{currentmarker}{\pgfqpoint{0.000000in}{-0.048611in}}{\pgfqpoint{0.000000in}{0.000000in}}{%
\pgfpathmoveto{\pgfqpoint{0.000000in}{0.000000in}}%
\pgfpathlineto{\pgfqpoint{0.000000in}{-0.048611in}}%
\pgfusepath{stroke,fill}%
}%
\begin{pgfscope}%
\pgfsys@transformshift{1.740849in}{4.185189in}%
\pgfsys@useobject{currentmarker}{}%
\end{pgfscope}%
\end{pgfscope}%
\begin{pgfscope}%
\definecolor{textcolor}{rgb}{0.000000,0.000000,0.000000}%
\pgfsetstrokecolor{textcolor}%
\pgfsetfillcolor{textcolor}%
\pgftext[x=1.740849in,y=4.087967in,,top]{\color{textcolor}\rmfamily\fontsize{8.000000}{9.600000}\selectfont Apr}%
\end{pgfscope}%
\begin{pgfscope}%
\pgfsetbuttcap%
\pgfsetroundjoin%
\definecolor{currentfill}{rgb}{0.000000,0.000000,0.000000}%
\pgfsetfillcolor{currentfill}%
\pgfsetlinewidth{0.803000pt}%
\definecolor{currentstroke}{rgb}{0.000000,0.000000,0.000000}%
\pgfsetstrokecolor{currentstroke}%
\pgfsetdash{}{0pt}%
\pgfsys@defobject{currentmarker}{\pgfqpoint{0.000000in}{-0.048611in}}{\pgfqpoint{0.000000in}{0.000000in}}{%
\pgfpathmoveto{\pgfqpoint{0.000000in}{0.000000in}}%
\pgfpathlineto{\pgfqpoint{0.000000in}{-0.048611in}}%
\pgfusepath{stroke,fill}%
}%
\begin{pgfscope}%
\pgfsys@transformshift{2.091792in}{4.185189in}%
\pgfsys@useobject{currentmarker}{}%
\end{pgfscope}%
\end{pgfscope}%
\begin{pgfscope}%
\definecolor{textcolor}{rgb}{0.000000,0.000000,0.000000}%
\pgfsetstrokecolor{textcolor}%
\pgfsetfillcolor{textcolor}%
\pgftext[x=2.091792in,y=4.087967in,,top]{\color{textcolor}\rmfamily\fontsize{8.000000}{9.600000}\selectfont May}%
\end{pgfscope}%
\begin{pgfscope}%
\pgfsetbuttcap%
\pgfsetroundjoin%
\definecolor{currentfill}{rgb}{0.000000,0.000000,0.000000}%
\pgfsetfillcolor{currentfill}%
\pgfsetlinewidth{0.803000pt}%
\definecolor{currentstroke}{rgb}{0.000000,0.000000,0.000000}%
\pgfsetstrokecolor{currentstroke}%
\pgfsetdash{}{0pt}%
\pgfsys@defobject{currentmarker}{\pgfqpoint{0.000000in}{-0.048611in}}{\pgfqpoint{0.000000in}{0.000000in}}{%
\pgfpathmoveto{\pgfqpoint{0.000000in}{0.000000in}}%
\pgfpathlineto{\pgfqpoint{0.000000in}{-0.048611in}}%
\pgfusepath{stroke,fill}%
}%
\begin{pgfscope}%
\pgfsys@transformshift{2.442736in}{4.185189in}%
\pgfsys@useobject{currentmarker}{}%
\end{pgfscope}%
\end{pgfscope}%
\begin{pgfscope}%
\definecolor{textcolor}{rgb}{0.000000,0.000000,0.000000}%
\pgfsetstrokecolor{textcolor}%
\pgfsetfillcolor{textcolor}%
\pgftext[x=2.442736in,y=4.087967in,,top]{\color{textcolor}\rmfamily\fontsize{8.000000}{9.600000}\selectfont Jun}%
\end{pgfscope}%
\begin{pgfscope}%
\pgfsetbuttcap%
\pgfsetroundjoin%
\definecolor{currentfill}{rgb}{0.000000,0.000000,0.000000}%
\pgfsetfillcolor{currentfill}%
\pgfsetlinewidth{0.803000pt}%
\definecolor{currentstroke}{rgb}{0.000000,0.000000,0.000000}%
\pgfsetstrokecolor{currentstroke}%
\pgfsetdash{}{0pt}%
\pgfsys@defobject{currentmarker}{\pgfqpoint{0.000000in}{-0.048611in}}{\pgfqpoint{0.000000in}{0.000000in}}{%
\pgfpathmoveto{\pgfqpoint{0.000000in}{0.000000in}}%
\pgfpathlineto{\pgfqpoint{0.000000in}{-0.048611in}}%
\pgfusepath{stroke,fill}%
}%
\begin{pgfscope}%
\pgfsys@transformshift{2.881415in}{4.185189in}%
\pgfsys@useobject{currentmarker}{}%
\end{pgfscope}%
\end{pgfscope}%
\begin{pgfscope}%
\definecolor{textcolor}{rgb}{0.000000,0.000000,0.000000}%
\pgfsetstrokecolor{textcolor}%
\pgfsetfillcolor{textcolor}%
\pgftext[x=2.881415in,y=4.087967in,,top]{\color{textcolor}\rmfamily\fontsize{8.000000}{9.600000}\selectfont Jul}%
\end{pgfscope}%
\begin{pgfscope}%
\pgfsetbuttcap%
\pgfsetroundjoin%
\definecolor{currentfill}{rgb}{0.000000,0.000000,0.000000}%
\pgfsetfillcolor{currentfill}%
\pgfsetlinewidth{0.803000pt}%
\definecolor{currentstroke}{rgb}{0.000000,0.000000,0.000000}%
\pgfsetstrokecolor{currentstroke}%
\pgfsetdash{}{0pt}%
\pgfsys@defobject{currentmarker}{\pgfqpoint{0.000000in}{-0.048611in}}{\pgfqpoint{0.000000in}{0.000000in}}{%
\pgfpathmoveto{\pgfqpoint{0.000000in}{0.000000in}}%
\pgfpathlineto{\pgfqpoint{0.000000in}{-0.048611in}}%
\pgfusepath{stroke,fill}%
}%
\begin{pgfscope}%
\pgfsys@transformshift{3.232358in}{4.185189in}%
\pgfsys@useobject{currentmarker}{}%
\end{pgfscope}%
\end{pgfscope}%
\begin{pgfscope}%
\definecolor{textcolor}{rgb}{0.000000,0.000000,0.000000}%
\pgfsetstrokecolor{textcolor}%
\pgfsetfillcolor{textcolor}%
\pgftext[x=3.232358in,y=4.087967in,,top]{\color{textcolor}\rmfamily\fontsize{8.000000}{9.600000}\selectfont Aug}%
\end{pgfscope}%
\begin{pgfscope}%
\pgfsetbuttcap%
\pgfsetroundjoin%
\definecolor{currentfill}{rgb}{0.000000,0.000000,0.000000}%
\pgfsetfillcolor{currentfill}%
\pgfsetlinewidth{0.803000pt}%
\definecolor{currentstroke}{rgb}{0.000000,0.000000,0.000000}%
\pgfsetstrokecolor{currentstroke}%
\pgfsetdash{}{0pt}%
\pgfsys@defobject{currentmarker}{\pgfqpoint{0.000000in}{-0.048611in}}{\pgfqpoint{0.000000in}{0.000000in}}{%
\pgfpathmoveto{\pgfqpoint{0.000000in}{0.000000in}}%
\pgfpathlineto{\pgfqpoint{0.000000in}{-0.048611in}}%
\pgfusepath{stroke,fill}%
}%
\begin{pgfscope}%
\pgfsys@transformshift{3.627169in}{4.185189in}%
\pgfsys@useobject{currentmarker}{}%
\end{pgfscope}%
\end{pgfscope}%
\begin{pgfscope}%
\definecolor{textcolor}{rgb}{0.000000,0.000000,0.000000}%
\pgfsetstrokecolor{textcolor}%
\pgfsetfillcolor{textcolor}%
\pgftext[x=3.627169in,y=4.087967in,,top]{\color{textcolor}\rmfamily\fontsize{8.000000}{9.600000}\selectfont Sep}%
\end{pgfscope}%
\begin{pgfscope}%
\pgfsetbuttcap%
\pgfsetroundjoin%
\definecolor{currentfill}{rgb}{0.000000,0.000000,0.000000}%
\pgfsetfillcolor{currentfill}%
\pgfsetlinewidth{0.803000pt}%
\definecolor{currentstroke}{rgb}{0.000000,0.000000,0.000000}%
\pgfsetstrokecolor{currentstroke}%
\pgfsetdash{}{0pt}%
\pgfsys@defobject{currentmarker}{\pgfqpoint{0.000000in}{-0.048611in}}{\pgfqpoint{0.000000in}{0.000000in}}{%
\pgfpathmoveto{\pgfqpoint{0.000000in}{0.000000in}}%
\pgfpathlineto{\pgfqpoint{0.000000in}{-0.048611in}}%
\pgfusepath{stroke,fill}%
}%
\begin{pgfscope}%
\pgfsys@transformshift{4.021981in}{4.185189in}%
\pgfsys@useobject{currentmarker}{}%
\end{pgfscope}%
\end{pgfscope}%
\begin{pgfscope}%
\definecolor{textcolor}{rgb}{0.000000,0.000000,0.000000}%
\pgfsetstrokecolor{textcolor}%
\pgfsetfillcolor{textcolor}%
\pgftext[x=4.021981in,y=4.087967in,,top]{\color{textcolor}\rmfamily\fontsize{8.000000}{9.600000}\selectfont Oct}%
\end{pgfscope}%
\begin{pgfscope}%
\pgfsetbuttcap%
\pgfsetroundjoin%
\definecolor{currentfill}{rgb}{0.000000,0.000000,0.000000}%
\pgfsetfillcolor{currentfill}%
\pgfsetlinewidth{0.803000pt}%
\definecolor{currentstroke}{rgb}{0.000000,0.000000,0.000000}%
\pgfsetstrokecolor{currentstroke}%
\pgfsetdash{}{0pt}%
\pgfsys@defobject{currentmarker}{\pgfqpoint{0.000000in}{-0.048611in}}{\pgfqpoint{0.000000in}{0.000000in}}{%
\pgfpathmoveto{\pgfqpoint{0.000000in}{0.000000in}}%
\pgfpathlineto{\pgfqpoint{0.000000in}{-0.048611in}}%
\pgfusepath{stroke,fill}%
}%
\begin{pgfscope}%
\pgfsys@transformshift{4.372924in}{4.185189in}%
\pgfsys@useobject{currentmarker}{}%
\end{pgfscope}%
\end{pgfscope}%
\begin{pgfscope}%
\definecolor{textcolor}{rgb}{0.000000,0.000000,0.000000}%
\pgfsetstrokecolor{textcolor}%
\pgfsetfillcolor{textcolor}%
\pgftext[x=4.372924in,y=4.087967in,,top]{\color{textcolor}\rmfamily\fontsize{8.000000}{9.600000}\selectfont Nov}%
\end{pgfscope}%
\begin{pgfscope}%
\pgfsetbuttcap%
\pgfsetroundjoin%
\definecolor{currentfill}{rgb}{0.000000,0.000000,0.000000}%
\pgfsetfillcolor{currentfill}%
\pgfsetlinewidth{0.803000pt}%
\definecolor{currentstroke}{rgb}{0.000000,0.000000,0.000000}%
\pgfsetstrokecolor{currentstroke}%
\pgfsetdash{}{0pt}%
\pgfsys@defobject{currentmarker}{\pgfqpoint{0.000000in}{-0.048611in}}{\pgfqpoint{0.000000in}{0.000000in}}{%
\pgfpathmoveto{\pgfqpoint{0.000000in}{0.000000in}}%
\pgfpathlineto{\pgfqpoint{0.000000in}{-0.048611in}}%
\pgfusepath{stroke,fill}%
}%
\begin{pgfscope}%
\pgfsys@transformshift{4.767736in}{4.185189in}%
\pgfsys@useobject{currentmarker}{}%
\end{pgfscope}%
\end{pgfscope}%
\begin{pgfscope}%
\definecolor{textcolor}{rgb}{0.000000,0.000000,0.000000}%
\pgfsetstrokecolor{textcolor}%
\pgfsetfillcolor{textcolor}%
\pgftext[x=4.767736in,y=4.087967in,,top]{\color{textcolor}\rmfamily\fontsize{8.000000}{9.600000}\selectfont Dec}%
\end{pgfscope}%
\begin{pgfscope}%
\pgfsetbuttcap%
\pgfsetroundjoin%
\definecolor{currentfill}{rgb}{0.000000,0.000000,0.000000}%
\pgfsetfillcolor{currentfill}%
\pgfsetlinewidth{0.803000pt}%
\definecolor{currentstroke}{rgb}{0.000000,0.000000,0.000000}%
\pgfsetstrokecolor{currentstroke}%
\pgfsetdash{}{0pt}%
\pgfsys@defobject{currentmarker}{\pgfqpoint{-0.048611in}{0.000000in}}{\pgfqpoint{-0.000000in}{0.000000in}}{%
\pgfpathmoveto{\pgfqpoint{-0.000000in}{0.000000in}}%
\pgfpathlineto{\pgfqpoint{-0.048611in}{0.000000in}}%
\pgfusepath{stroke,fill}%
}%
\begin{pgfscope}%
\pgfsys@transformshift{0.380943in}{4.755472in}%
\pgfsys@useobject{currentmarker}{}%
\end{pgfscope}%
\end{pgfscope}%
\begin{pgfscope}%
\definecolor{textcolor}{rgb}{0.000000,0.000000,0.000000}%
\pgfsetstrokecolor{textcolor}%
\pgfsetfillcolor{textcolor}%
\pgftext[x=0.113117in, y=4.716892in, left, base]{\color{textcolor}\rmfamily\fontsize{8.000000}{9.600000}\selectfont M}%
\end{pgfscope}%
\begin{pgfscope}%
\pgfsetbuttcap%
\pgfsetroundjoin%
\definecolor{currentfill}{rgb}{0.000000,0.000000,0.000000}%
\pgfsetfillcolor{currentfill}%
\pgfsetlinewidth{0.803000pt}%
\definecolor{currentstroke}{rgb}{0.000000,0.000000,0.000000}%
\pgfsetstrokecolor{currentstroke}%
\pgfsetdash{}{0pt}%
\pgfsys@defobject{currentmarker}{\pgfqpoint{-0.048611in}{0.000000in}}{\pgfqpoint{-0.000000in}{0.000000in}}{%
\pgfpathmoveto{\pgfqpoint{-0.000000in}{0.000000in}}%
\pgfpathlineto{\pgfqpoint{-0.048611in}{0.000000in}}%
\pgfusepath{stroke,fill}%
}%
\begin{pgfscope}%
\pgfsys@transformshift{0.380943in}{4.667736in}%
\pgfsys@useobject{currentmarker}{}%
\end{pgfscope}%
\end{pgfscope}%
\begin{pgfscope}%
\definecolor{textcolor}{rgb}{0.000000,0.000000,0.000000}%
\pgfsetstrokecolor{textcolor}%
\pgfsetfillcolor{textcolor}%
\pgftext[x=0.135957in, y=4.629156in, left, base]{\color{textcolor}\rmfamily\fontsize{8.000000}{9.600000}\selectfont T}%
\end{pgfscope}%
\begin{pgfscope}%
\pgfsetbuttcap%
\pgfsetroundjoin%
\definecolor{currentfill}{rgb}{0.000000,0.000000,0.000000}%
\pgfsetfillcolor{currentfill}%
\pgfsetlinewidth{0.803000pt}%
\definecolor{currentstroke}{rgb}{0.000000,0.000000,0.000000}%
\pgfsetstrokecolor{currentstroke}%
\pgfsetdash{}{0pt}%
\pgfsys@defobject{currentmarker}{\pgfqpoint{-0.048611in}{0.000000in}}{\pgfqpoint{-0.000000in}{0.000000in}}{%
\pgfpathmoveto{\pgfqpoint{-0.000000in}{0.000000in}}%
\pgfpathlineto{\pgfqpoint{-0.048611in}{0.000000in}}%
\pgfusepath{stroke,fill}%
}%
\begin{pgfscope}%
\pgfsys@transformshift{0.380943in}{4.580000in}%
\pgfsys@useobject{currentmarker}{}%
\end{pgfscope}%
\end{pgfscope}%
\begin{pgfscope}%
\definecolor{textcolor}{rgb}{0.000000,0.000000,0.000000}%
\pgfsetstrokecolor{textcolor}%
\pgfsetfillcolor{textcolor}%
\pgftext[x=0.100000in, y=4.541420in, left, base]{\color{textcolor}\rmfamily\fontsize{8.000000}{9.600000}\selectfont W}%
\end{pgfscope}%
\begin{pgfscope}%
\pgfsetbuttcap%
\pgfsetroundjoin%
\definecolor{currentfill}{rgb}{0.000000,0.000000,0.000000}%
\pgfsetfillcolor{currentfill}%
\pgfsetlinewidth{0.803000pt}%
\definecolor{currentstroke}{rgb}{0.000000,0.000000,0.000000}%
\pgfsetstrokecolor{currentstroke}%
\pgfsetdash{}{0pt}%
\pgfsys@defobject{currentmarker}{\pgfqpoint{-0.048611in}{0.000000in}}{\pgfqpoint{-0.000000in}{0.000000in}}{%
\pgfpathmoveto{\pgfqpoint{-0.000000in}{0.000000in}}%
\pgfpathlineto{\pgfqpoint{-0.048611in}{0.000000in}}%
\pgfusepath{stroke,fill}%
}%
\begin{pgfscope}%
\pgfsys@transformshift{0.380943in}{4.492264in}%
\pgfsys@useobject{currentmarker}{}%
\end{pgfscope}%
\end{pgfscope}%
\begin{pgfscope}%
\definecolor{textcolor}{rgb}{0.000000,0.000000,0.000000}%
\pgfsetstrokecolor{textcolor}%
\pgfsetfillcolor{textcolor}%
\pgftext[x=0.135957in, y=4.453684in, left, base]{\color{textcolor}\rmfamily\fontsize{8.000000}{9.600000}\selectfont T}%
\end{pgfscope}%
\begin{pgfscope}%
\pgfsetbuttcap%
\pgfsetroundjoin%
\definecolor{currentfill}{rgb}{0.000000,0.000000,0.000000}%
\pgfsetfillcolor{currentfill}%
\pgfsetlinewidth{0.803000pt}%
\definecolor{currentstroke}{rgb}{0.000000,0.000000,0.000000}%
\pgfsetstrokecolor{currentstroke}%
\pgfsetdash{}{0pt}%
\pgfsys@defobject{currentmarker}{\pgfqpoint{-0.048611in}{0.000000in}}{\pgfqpoint{-0.000000in}{0.000000in}}{%
\pgfpathmoveto{\pgfqpoint{-0.000000in}{0.000000in}}%
\pgfpathlineto{\pgfqpoint{-0.048611in}{0.000000in}}%
\pgfusepath{stroke,fill}%
}%
\begin{pgfscope}%
\pgfsys@transformshift{0.380943in}{4.404529in}%
\pgfsys@useobject{currentmarker}{}%
\end{pgfscope}%
\end{pgfscope}%
\begin{pgfscope}%
\definecolor{textcolor}{rgb}{0.000000,0.000000,0.000000}%
\pgfsetstrokecolor{textcolor}%
\pgfsetfillcolor{textcolor}%
\pgftext[x=0.144213in, y=4.365948in, left, base]{\color{textcolor}\rmfamily\fontsize{8.000000}{9.600000}\selectfont F}%
\end{pgfscope}%
\begin{pgfscope}%
\pgfsetbuttcap%
\pgfsetroundjoin%
\definecolor{currentfill}{rgb}{0.000000,0.000000,0.000000}%
\pgfsetfillcolor{currentfill}%
\pgfsetlinewidth{0.803000pt}%
\definecolor{currentstroke}{rgb}{0.000000,0.000000,0.000000}%
\pgfsetstrokecolor{currentstroke}%
\pgfsetdash{}{0pt}%
\pgfsys@defobject{currentmarker}{\pgfqpoint{-0.048611in}{0.000000in}}{\pgfqpoint{-0.000000in}{0.000000in}}{%
\pgfpathmoveto{\pgfqpoint{-0.000000in}{0.000000in}}%
\pgfpathlineto{\pgfqpoint{-0.048611in}{0.000000in}}%
\pgfusepath{stroke,fill}%
}%
\begin{pgfscope}%
\pgfsys@transformshift{0.380943in}{4.316793in}%
\pgfsys@useobject{currentmarker}{}%
\end{pgfscope}%
\end{pgfscope}%
\begin{pgfscope}%
\definecolor{textcolor}{rgb}{0.000000,0.000000,0.000000}%
\pgfsetstrokecolor{textcolor}%
\pgfsetfillcolor{textcolor}%
\pgftext[x=0.155633in, y=4.278212in, left, base]{\color{textcolor}\rmfamily\fontsize{8.000000}{9.600000}\selectfont S}%
\end{pgfscope}%
\begin{pgfscope}%
\pgfsetbuttcap%
\pgfsetroundjoin%
\definecolor{currentfill}{rgb}{0.000000,0.000000,0.000000}%
\pgfsetfillcolor{currentfill}%
\pgfsetlinewidth{0.803000pt}%
\definecolor{currentstroke}{rgb}{0.000000,0.000000,0.000000}%
\pgfsetstrokecolor{currentstroke}%
\pgfsetdash{}{0pt}%
\pgfsys@defobject{currentmarker}{\pgfqpoint{-0.048611in}{0.000000in}}{\pgfqpoint{-0.000000in}{0.000000in}}{%
\pgfpathmoveto{\pgfqpoint{-0.000000in}{0.000000in}}%
\pgfpathlineto{\pgfqpoint{-0.048611in}{0.000000in}}%
\pgfusepath{stroke,fill}%
}%
\begin{pgfscope}%
\pgfsys@transformshift{0.380943in}{4.229057in}%
\pgfsys@useobject{currentmarker}{}%
\end{pgfscope}%
\end{pgfscope}%
\begin{pgfscope}%
\definecolor{textcolor}{rgb}{0.000000,0.000000,0.000000}%
\pgfsetstrokecolor{textcolor}%
\pgfsetfillcolor{textcolor}%
\pgftext[x=0.155633in, y=4.190477in, left, base]{\color{textcolor}\rmfamily\fontsize{8.000000}{9.600000}\selectfont S}%
\end{pgfscope}%
\begin{pgfscope}%
\definecolor{textcolor}{rgb}{0.000000,0.000000,0.000000}%
\pgfsetstrokecolor{textcolor}%
\pgfsetfillcolor{textcolor}%
\pgftext[x=2.705943in,y=4.966007in,,]{\color{textcolor}\ttfamily\fontsize{14.400000}{17.280000}\selectfont 2019}%
\end{pgfscope}%
\begin{pgfscope}%
\pgfpathrectangle{\pgfqpoint{0.380943in}{2.260189in}}{\pgfqpoint{4.650000in}{0.614151in}}%
\pgfusepath{clip}%
\pgfsetbuttcap%
\pgfsetroundjoin%
\pgfsetlinewidth{0.250937pt}%
\definecolor{currentstroke}{rgb}{1.000000,1.000000,1.000000}%
\pgfsetstrokecolor{currentstroke}%
\pgfsetdash{}{0pt}%
\pgfpathmoveto{\pgfqpoint{0.380943in}{2.874340in}}%
\pgfpathlineto{\pgfqpoint{0.468679in}{2.874340in}}%
\pgfpathlineto{\pgfqpoint{0.468679in}{2.786604in}}%
\pgfpathlineto{\pgfqpoint{0.380943in}{2.786604in}}%
\pgfpathlineto{\pgfqpoint{0.380943in}{2.874340in}}%
\pgfusepath{stroke}%
\end{pgfscope}%
\begin{pgfscope}%
\pgfpathrectangle{\pgfqpoint{0.380943in}{2.260189in}}{\pgfqpoint{4.650000in}{0.614151in}}%
\pgfusepath{clip}%
\pgfsetbuttcap%
\pgfsetroundjoin%
\definecolor{currentfill}{rgb}{0.974072,0.862976,0.688750}%
\pgfsetfillcolor{currentfill}%
\pgfsetlinewidth{0.250937pt}%
\definecolor{currentstroke}{rgb}{1.000000,1.000000,1.000000}%
\pgfsetstrokecolor{currentstroke}%
\pgfsetdash{}{0pt}%
\pgfpathmoveto{\pgfqpoint{0.468679in}{2.874340in}}%
\pgfpathlineto{\pgfqpoint{0.556415in}{2.874340in}}%
\pgfpathlineto{\pgfqpoint{0.556415in}{2.786604in}}%
\pgfpathlineto{\pgfqpoint{0.468679in}{2.786604in}}%
\pgfpathlineto{\pgfqpoint{0.468679in}{2.874340in}}%
\pgfusepath{stroke,fill}%
\end{pgfscope}%
\begin{pgfscope}%
\pgfpathrectangle{\pgfqpoint{0.380943in}{2.260189in}}{\pgfqpoint{4.650000in}{0.614151in}}%
\pgfusepath{clip}%
\pgfsetbuttcap%
\pgfsetroundjoin%
\definecolor{currentfill}{rgb}{1.000000,0.615379,0.534779}%
\pgfsetfillcolor{currentfill}%
\pgfsetlinewidth{0.250937pt}%
\definecolor{currentstroke}{rgb}{1.000000,1.000000,1.000000}%
\pgfsetstrokecolor{currentstroke}%
\pgfsetdash{}{0pt}%
\pgfpathmoveto{\pgfqpoint{0.556415in}{2.874340in}}%
\pgfpathlineto{\pgfqpoint{0.644151in}{2.874340in}}%
\pgfpathlineto{\pgfqpoint{0.644151in}{2.786604in}}%
\pgfpathlineto{\pgfqpoint{0.556415in}{2.786604in}}%
\pgfpathlineto{\pgfqpoint{0.556415in}{2.874340in}}%
\pgfusepath{stroke,fill}%
\end{pgfscope}%
\begin{pgfscope}%
\pgfpathrectangle{\pgfqpoint{0.380943in}{2.260189in}}{\pgfqpoint{4.650000in}{0.614151in}}%
\pgfusepath{clip}%
\pgfsetbuttcap%
\pgfsetroundjoin%
\definecolor{currentfill}{rgb}{0.974072,0.862976,0.688750}%
\pgfsetfillcolor{currentfill}%
\pgfsetlinewidth{0.250937pt}%
\definecolor{currentstroke}{rgb}{1.000000,1.000000,1.000000}%
\pgfsetstrokecolor{currentstroke}%
\pgfsetdash{}{0pt}%
\pgfpathmoveto{\pgfqpoint{0.644151in}{2.874340in}}%
\pgfpathlineto{\pgfqpoint{0.731886in}{2.874340in}}%
\pgfpathlineto{\pgfqpoint{0.731886in}{2.786604in}}%
\pgfpathlineto{\pgfqpoint{0.644151in}{2.786604in}}%
\pgfpathlineto{\pgfqpoint{0.644151in}{2.874340in}}%
\pgfusepath{stroke,fill}%
\end{pgfscope}%
\begin{pgfscope}%
\pgfpathrectangle{\pgfqpoint{0.380943in}{2.260189in}}{\pgfqpoint{4.650000in}{0.614151in}}%
\pgfusepath{clip}%
\pgfsetbuttcap%
\pgfsetroundjoin%
\definecolor{currentfill}{rgb}{0.964937,0.908651,0.713110}%
\pgfsetfillcolor{currentfill}%
\pgfsetlinewidth{0.250937pt}%
\definecolor{currentstroke}{rgb}{1.000000,1.000000,1.000000}%
\pgfsetstrokecolor{currentstroke}%
\pgfsetdash{}{0pt}%
\pgfpathmoveto{\pgfqpoint{0.731886in}{2.874340in}}%
\pgfpathlineto{\pgfqpoint{0.819622in}{2.874340in}}%
\pgfpathlineto{\pgfqpoint{0.819622in}{2.786604in}}%
\pgfpathlineto{\pgfqpoint{0.731886in}{2.786604in}}%
\pgfpathlineto{\pgfqpoint{0.731886in}{2.874340in}}%
\pgfusepath{stroke,fill}%
\end{pgfscope}%
\begin{pgfscope}%
\pgfpathrectangle{\pgfqpoint{0.380943in}{2.260189in}}{\pgfqpoint{4.650000in}{0.614151in}}%
\pgfusepath{clip}%
\pgfsetbuttcap%
\pgfsetroundjoin%
\definecolor{currentfill}{rgb}{0.990634,0.779608,0.623299}%
\pgfsetfillcolor{currentfill}%
\pgfsetlinewidth{0.250937pt}%
\definecolor{currentstroke}{rgb}{1.000000,1.000000,1.000000}%
\pgfsetstrokecolor{currentstroke}%
\pgfsetdash{}{0pt}%
\pgfpathmoveto{\pgfqpoint{0.819622in}{2.874340in}}%
\pgfpathlineto{\pgfqpoint{0.907358in}{2.874340in}}%
\pgfpathlineto{\pgfqpoint{0.907358in}{2.786604in}}%
\pgfpathlineto{\pgfqpoint{0.819622in}{2.786604in}}%
\pgfpathlineto{\pgfqpoint{0.819622in}{2.874340in}}%
\pgfusepath{stroke,fill}%
\end{pgfscope}%
\begin{pgfscope}%
\pgfpathrectangle{\pgfqpoint{0.380943in}{2.260189in}}{\pgfqpoint{4.650000in}{0.614151in}}%
\pgfusepath{clip}%
\pgfsetbuttcap%
\pgfsetroundjoin%
\definecolor{currentfill}{rgb}{0.987266,0.804198,0.639170}%
\pgfsetfillcolor{currentfill}%
\pgfsetlinewidth{0.250937pt}%
\definecolor{currentstroke}{rgb}{1.000000,1.000000,1.000000}%
\pgfsetstrokecolor{currentstroke}%
\pgfsetdash{}{0pt}%
\pgfpathmoveto{\pgfqpoint{0.907358in}{2.874340in}}%
\pgfpathlineto{\pgfqpoint{0.995094in}{2.874340in}}%
\pgfpathlineto{\pgfqpoint{0.995094in}{2.786604in}}%
\pgfpathlineto{\pgfqpoint{0.907358in}{2.786604in}}%
\pgfpathlineto{\pgfqpoint{0.907358in}{2.874340in}}%
\pgfusepath{stroke,fill}%
\end{pgfscope}%
\begin{pgfscope}%
\pgfpathrectangle{\pgfqpoint{0.380943in}{2.260189in}}{\pgfqpoint{4.650000in}{0.614151in}}%
\pgfusepath{clip}%
\pgfsetbuttcap%
\pgfsetroundjoin%
\definecolor{currentfill}{rgb}{0.964937,0.908651,0.713110}%
\pgfsetfillcolor{currentfill}%
\pgfsetlinewidth{0.250937pt}%
\definecolor{currentstroke}{rgb}{1.000000,1.000000,1.000000}%
\pgfsetstrokecolor{currentstroke}%
\pgfsetdash{}{0pt}%
\pgfpathmoveto{\pgfqpoint{0.995094in}{2.874340in}}%
\pgfpathlineto{\pgfqpoint{1.082830in}{2.874340in}}%
\pgfpathlineto{\pgfqpoint{1.082830in}{2.786604in}}%
\pgfpathlineto{\pgfqpoint{0.995094in}{2.786604in}}%
\pgfpathlineto{\pgfqpoint{0.995094in}{2.874340in}}%
\pgfusepath{stroke,fill}%
\end{pgfscope}%
\begin{pgfscope}%
\pgfpathrectangle{\pgfqpoint{0.380943in}{2.260189in}}{\pgfqpoint{4.650000in}{0.614151in}}%
\pgfusepath{clip}%
\pgfsetbuttcap%
\pgfsetroundjoin%
\definecolor{currentfill}{rgb}{0.978639,0.841584,0.673679}%
\pgfsetfillcolor{currentfill}%
\pgfsetlinewidth{0.250937pt}%
\definecolor{currentstroke}{rgb}{1.000000,1.000000,1.000000}%
\pgfsetstrokecolor{currentstroke}%
\pgfsetdash{}{0pt}%
\pgfpathmoveto{\pgfqpoint{1.082830in}{2.874340in}}%
\pgfpathlineto{\pgfqpoint{1.170566in}{2.874340in}}%
\pgfpathlineto{\pgfqpoint{1.170566in}{2.786604in}}%
\pgfpathlineto{\pgfqpoint{1.082830in}{2.786604in}}%
\pgfpathlineto{\pgfqpoint{1.082830in}{2.874340in}}%
\pgfusepath{stroke,fill}%
\end{pgfscope}%
\begin{pgfscope}%
\pgfpathrectangle{\pgfqpoint{0.380943in}{2.260189in}}{\pgfqpoint{4.650000in}{0.614151in}}%
\pgfusepath{clip}%
\pgfsetbuttcap%
\pgfsetroundjoin%
\definecolor{currentfill}{rgb}{0.964937,0.908651,0.713110}%
\pgfsetfillcolor{currentfill}%
\pgfsetlinewidth{0.250937pt}%
\definecolor{currentstroke}{rgb}{1.000000,1.000000,1.000000}%
\pgfsetstrokecolor{currentstroke}%
\pgfsetdash{}{0pt}%
\pgfpathmoveto{\pgfqpoint{1.170566in}{2.874340in}}%
\pgfpathlineto{\pgfqpoint{1.258302in}{2.874340in}}%
\pgfpathlineto{\pgfqpoint{1.258302in}{2.786604in}}%
\pgfpathlineto{\pgfqpoint{1.170566in}{2.786604in}}%
\pgfpathlineto{\pgfqpoint{1.170566in}{2.874340in}}%
\pgfusepath{stroke,fill}%
\end{pgfscope}%
\begin{pgfscope}%
\pgfpathrectangle{\pgfqpoint{0.380943in}{2.260189in}}{\pgfqpoint{4.650000in}{0.614151in}}%
\pgfusepath{clip}%
\pgfsetbuttcap%
\pgfsetroundjoin%
\definecolor{currentfill}{rgb}{0.961738,0.927612,0.725598}%
\pgfsetfillcolor{currentfill}%
\pgfsetlinewidth{0.250937pt}%
\definecolor{currentstroke}{rgb}{1.000000,1.000000,1.000000}%
\pgfsetstrokecolor{currentstroke}%
\pgfsetdash{}{0pt}%
\pgfpathmoveto{\pgfqpoint{1.258302in}{2.874340in}}%
\pgfpathlineto{\pgfqpoint{1.346037in}{2.874340in}}%
\pgfpathlineto{\pgfqpoint{1.346037in}{2.786604in}}%
\pgfpathlineto{\pgfqpoint{1.258302in}{2.786604in}}%
\pgfpathlineto{\pgfqpoint{1.258302in}{2.874340in}}%
\pgfusepath{stroke,fill}%
\end{pgfscope}%
\begin{pgfscope}%
\pgfpathrectangle{\pgfqpoint{0.380943in}{2.260189in}}{\pgfqpoint{4.650000in}{0.614151in}}%
\pgfusepath{clip}%
\pgfsetbuttcap%
\pgfsetroundjoin%
\definecolor{currentfill}{rgb}{1.000000,1.000000,0.895579}%
\pgfsetfillcolor{currentfill}%
\pgfsetlinewidth{0.250937pt}%
\definecolor{currentstroke}{rgb}{1.000000,1.000000,1.000000}%
\pgfsetstrokecolor{currentstroke}%
\pgfsetdash{}{0pt}%
\pgfpathmoveto{\pgfqpoint{1.346037in}{2.874340in}}%
\pgfpathlineto{\pgfqpoint{1.433773in}{2.874340in}}%
\pgfpathlineto{\pgfqpoint{1.433773in}{2.786604in}}%
\pgfpathlineto{\pgfqpoint{1.346037in}{2.786604in}}%
\pgfpathlineto{\pgfqpoint{1.346037in}{2.874340in}}%
\pgfusepath{stroke,fill}%
\end{pgfscope}%
\begin{pgfscope}%
\pgfpathrectangle{\pgfqpoint{0.380943in}{2.260189in}}{\pgfqpoint{4.650000in}{0.614151in}}%
\pgfusepath{clip}%
\pgfsetbuttcap%
\pgfsetroundjoin%
\definecolor{currentfill}{rgb}{0.995233,0.991895,0.818977}%
\pgfsetfillcolor{currentfill}%
\pgfsetlinewidth{0.250937pt}%
\definecolor{currentstroke}{rgb}{1.000000,1.000000,1.000000}%
\pgfsetstrokecolor{currentstroke}%
\pgfsetdash{}{0pt}%
\pgfpathmoveto{\pgfqpoint{1.433773in}{2.874340in}}%
\pgfpathlineto{\pgfqpoint{1.521509in}{2.874340in}}%
\pgfpathlineto{\pgfqpoint{1.521509in}{2.786604in}}%
\pgfpathlineto{\pgfqpoint{1.433773in}{2.786604in}}%
\pgfpathlineto{\pgfqpoint{1.433773in}{2.874340in}}%
\pgfusepath{stroke,fill}%
\end{pgfscope}%
\begin{pgfscope}%
\pgfpathrectangle{\pgfqpoint{0.380943in}{2.260189in}}{\pgfqpoint{4.650000in}{0.614151in}}%
\pgfusepath{clip}%
\pgfsetbuttcap%
\pgfsetroundjoin%
\definecolor{currentfill}{rgb}{0.995233,0.991895,0.818977}%
\pgfsetfillcolor{currentfill}%
\pgfsetlinewidth{0.250937pt}%
\definecolor{currentstroke}{rgb}{1.000000,1.000000,1.000000}%
\pgfsetstrokecolor{currentstroke}%
\pgfsetdash{}{0pt}%
\pgfpathmoveto{\pgfqpoint{1.521509in}{2.874340in}}%
\pgfpathlineto{\pgfqpoint{1.609245in}{2.874340in}}%
\pgfpathlineto{\pgfqpoint{1.609245in}{2.786604in}}%
\pgfpathlineto{\pgfqpoint{1.521509in}{2.786604in}}%
\pgfpathlineto{\pgfqpoint{1.521509in}{2.874340in}}%
\pgfusepath{stroke,fill}%
\end{pgfscope}%
\begin{pgfscope}%
\pgfpathrectangle{\pgfqpoint{0.380943in}{2.260189in}}{\pgfqpoint{4.650000in}{0.614151in}}%
\pgfusepath{clip}%
\pgfsetbuttcap%
\pgfsetroundjoin%
\definecolor{currentfill}{rgb}{0.961738,0.927612,0.725598}%
\pgfsetfillcolor{currentfill}%
\pgfsetlinewidth{0.250937pt}%
\definecolor{currentstroke}{rgb}{1.000000,1.000000,1.000000}%
\pgfsetstrokecolor{currentstroke}%
\pgfsetdash{}{0pt}%
\pgfpathmoveto{\pgfqpoint{1.609245in}{2.874340in}}%
\pgfpathlineto{\pgfqpoint{1.696981in}{2.874340in}}%
\pgfpathlineto{\pgfqpoint{1.696981in}{2.786604in}}%
\pgfpathlineto{\pgfqpoint{1.609245in}{2.786604in}}%
\pgfpathlineto{\pgfqpoint{1.609245in}{2.874340in}}%
\pgfusepath{stroke,fill}%
\end{pgfscope}%
\begin{pgfscope}%
\pgfpathrectangle{\pgfqpoint{0.380943in}{2.260189in}}{\pgfqpoint{4.650000in}{0.614151in}}%
\pgfusepath{clip}%
\pgfsetbuttcap%
\pgfsetroundjoin%
\definecolor{currentfill}{rgb}{1.000000,1.000000,0.895579}%
\pgfsetfillcolor{currentfill}%
\pgfsetlinewidth{0.250937pt}%
\definecolor{currentstroke}{rgb}{1.000000,1.000000,1.000000}%
\pgfsetstrokecolor{currentstroke}%
\pgfsetdash{}{0pt}%
\pgfpathmoveto{\pgfqpoint{1.696981in}{2.874340in}}%
\pgfpathlineto{\pgfqpoint{1.784717in}{2.874340in}}%
\pgfpathlineto{\pgfqpoint{1.784717in}{2.786604in}}%
\pgfpathlineto{\pgfqpoint{1.696981in}{2.786604in}}%
\pgfpathlineto{\pgfqpoint{1.696981in}{2.874340in}}%
\pgfusepath{stroke,fill}%
\end{pgfscope}%
\begin{pgfscope}%
\pgfpathrectangle{\pgfqpoint{0.380943in}{2.260189in}}{\pgfqpoint{4.650000in}{0.614151in}}%
\pgfusepath{clip}%
\pgfsetbuttcap%
\pgfsetroundjoin%
\definecolor{currentfill}{rgb}{0.980008,0.966013,0.779393}%
\pgfsetfillcolor{currentfill}%
\pgfsetlinewidth{0.250937pt}%
\definecolor{currentstroke}{rgb}{1.000000,1.000000,1.000000}%
\pgfsetstrokecolor{currentstroke}%
\pgfsetdash{}{0pt}%
\pgfpathmoveto{\pgfqpoint{1.784717in}{2.874340in}}%
\pgfpathlineto{\pgfqpoint{1.872452in}{2.874340in}}%
\pgfpathlineto{\pgfqpoint{1.872452in}{2.786604in}}%
\pgfpathlineto{\pgfqpoint{1.784717in}{2.786604in}}%
\pgfpathlineto{\pgfqpoint{1.784717in}{2.874340in}}%
\pgfusepath{stroke,fill}%
\end{pgfscope}%
\begin{pgfscope}%
\pgfpathrectangle{\pgfqpoint{0.380943in}{2.260189in}}{\pgfqpoint{4.650000in}{0.614151in}}%
\pgfusepath{clip}%
\pgfsetbuttcap%
\pgfsetroundjoin%
\definecolor{currentfill}{rgb}{0.961738,0.927612,0.725598}%
\pgfsetfillcolor{currentfill}%
\pgfsetlinewidth{0.250937pt}%
\definecolor{currentstroke}{rgb}{1.000000,1.000000,1.000000}%
\pgfsetstrokecolor{currentstroke}%
\pgfsetdash{}{0pt}%
\pgfpathmoveto{\pgfqpoint{1.872452in}{2.874340in}}%
\pgfpathlineto{\pgfqpoint{1.960188in}{2.874340in}}%
\pgfpathlineto{\pgfqpoint{1.960188in}{2.786604in}}%
\pgfpathlineto{\pgfqpoint{1.872452in}{2.786604in}}%
\pgfpathlineto{\pgfqpoint{1.872452in}{2.874340in}}%
\pgfusepath{stroke,fill}%
\end{pgfscope}%
\begin{pgfscope}%
\pgfpathrectangle{\pgfqpoint{0.380943in}{2.260189in}}{\pgfqpoint{4.650000in}{0.614151in}}%
\pgfusepath{clip}%
\pgfsetbuttcap%
\pgfsetroundjoin%
\definecolor{currentfill}{rgb}{0.961738,0.927612,0.725598}%
\pgfsetfillcolor{currentfill}%
\pgfsetlinewidth{0.250937pt}%
\definecolor{currentstroke}{rgb}{1.000000,1.000000,1.000000}%
\pgfsetstrokecolor{currentstroke}%
\pgfsetdash{}{0pt}%
\pgfpathmoveto{\pgfqpoint{1.960188in}{2.874340in}}%
\pgfpathlineto{\pgfqpoint{2.047924in}{2.874340in}}%
\pgfpathlineto{\pgfqpoint{2.047924in}{2.786604in}}%
\pgfpathlineto{\pgfqpoint{1.960188in}{2.786604in}}%
\pgfpathlineto{\pgfqpoint{1.960188in}{2.874340in}}%
\pgfusepath{stroke,fill}%
\end{pgfscope}%
\begin{pgfscope}%
\pgfpathrectangle{\pgfqpoint{0.380943in}{2.260189in}}{\pgfqpoint{4.650000in}{0.614151in}}%
\pgfusepath{clip}%
\pgfsetbuttcap%
\pgfsetroundjoin%
\definecolor{currentfill}{rgb}{0.961738,0.927612,0.725598}%
\pgfsetfillcolor{currentfill}%
\pgfsetlinewidth{0.250937pt}%
\definecolor{currentstroke}{rgb}{1.000000,1.000000,1.000000}%
\pgfsetstrokecolor{currentstroke}%
\pgfsetdash{}{0pt}%
\pgfpathmoveto{\pgfqpoint{2.047924in}{2.874340in}}%
\pgfpathlineto{\pgfqpoint{2.135660in}{2.874340in}}%
\pgfpathlineto{\pgfqpoint{2.135660in}{2.786604in}}%
\pgfpathlineto{\pgfqpoint{2.047924in}{2.786604in}}%
\pgfpathlineto{\pgfqpoint{2.047924in}{2.874340in}}%
\pgfusepath{stroke,fill}%
\end{pgfscope}%
\begin{pgfscope}%
\pgfpathrectangle{\pgfqpoint{0.380943in}{2.260189in}}{\pgfqpoint{4.650000in}{0.614151in}}%
\pgfusepath{clip}%
\pgfsetbuttcap%
\pgfsetroundjoin%
\definecolor{currentfill}{rgb}{0.964783,0.940131,0.739808}%
\pgfsetfillcolor{currentfill}%
\pgfsetlinewidth{0.250937pt}%
\definecolor{currentstroke}{rgb}{1.000000,1.000000,1.000000}%
\pgfsetstrokecolor{currentstroke}%
\pgfsetdash{}{0pt}%
\pgfpathmoveto{\pgfqpoint{2.135660in}{2.874340in}}%
\pgfpathlineto{\pgfqpoint{2.223396in}{2.874340in}}%
\pgfpathlineto{\pgfqpoint{2.223396in}{2.786604in}}%
\pgfpathlineto{\pgfqpoint{2.135660in}{2.786604in}}%
\pgfpathlineto{\pgfqpoint{2.135660in}{2.874340in}}%
\pgfusepath{stroke,fill}%
\end{pgfscope}%
\begin{pgfscope}%
\pgfpathrectangle{\pgfqpoint{0.380943in}{2.260189in}}{\pgfqpoint{4.650000in}{0.614151in}}%
\pgfusepath{clip}%
\pgfsetbuttcap%
\pgfsetroundjoin%
\definecolor{currentfill}{rgb}{0.964937,0.908651,0.713110}%
\pgfsetfillcolor{currentfill}%
\pgfsetlinewidth{0.250937pt}%
\definecolor{currentstroke}{rgb}{1.000000,1.000000,1.000000}%
\pgfsetstrokecolor{currentstroke}%
\pgfsetdash{}{0pt}%
\pgfpathmoveto{\pgfqpoint{2.223396in}{2.874340in}}%
\pgfpathlineto{\pgfqpoint{2.311132in}{2.874340in}}%
\pgfpathlineto{\pgfqpoint{2.311132in}{2.786604in}}%
\pgfpathlineto{\pgfqpoint{2.223396in}{2.786604in}}%
\pgfpathlineto{\pgfqpoint{2.223396in}{2.874340in}}%
\pgfusepath{stroke,fill}%
\end{pgfscope}%
\begin{pgfscope}%
\pgfpathrectangle{\pgfqpoint{0.380943in}{2.260189in}}{\pgfqpoint{4.650000in}{0.614151in}}%
\pgfusepath{clip}%
\pgfsetbuttcap%
\pgfsetroundjoin%
\definecolor{currentfill}{rgb}{1.000000,1.000000,0.857516}%
\pgfsetfillcolor{currentfill}%
\pgfsetlinewidth{0.250937pt}%
\definecolor{currentstroke}{rgb}{1.000000,1.000000,1.000000}%
\pgfsetstrokecolor{currentstroke}%
\pgfsetdash{}{0pt}%
\pgfpathmoveto{\pgfqpoint{2.311132in}{2.874340in}}%
\pgfpathlineto{\pgfqpoint{2.398868in}{2.874340in}}%
\pgfpathlineto{\pgfqpoint{2.398868in}{2.786604in}}%
\pgfpathlineto{\pgfqpoint{2.311132in}{2.786604in}}%
\pgfpathlineto{\pgfqpoint{2.311132in}{2.874340in}}%
\pgfusepath{stroke,fill}%
\end{pgfscope}%
\begin{pgfscope}%
\pgfpathrectangle{\pgfqpoint{0.380943in}{2.260189in}}{\pgfqpoint{4.650000in}{0.614151in}}%
\pgfusepath{clip}%
\pgfsetbuttcap%
\pgfsetroundjoin%
\definecolor{currentfill}{rgb}{0.995233,0.991895,0.818977}%
\pgfsetfillcolor{currentfill}%
\pgfsetlinewidth{0.250937pt}%
\definecolor{currentstroke}{rgb}{1.000000,1.000000,1.000000}%
\pgfsetstrokecolor{currentstroke}%
\pgfsetdash{}{0pt}%
\pgfpathmoveto{\pgfqpoint{2.398868in}{2.874340in}}%
\pgfpathlineto{\pgfqpoint{2.486603in}{2.874340in}}%
\pgfpathlineto{\pgfqpoint{2.486603in}{2.786604in}}%
\pgfpathlineto{\pgfqpoint{2.398868in}{2.786604in}}%
\pgfpathlineto{\pgfqpoint{2.398868in}{2.874340in}}%
\pgfusepath{stroke,fill}%
\end{pgfscope}%
\begin{pgfscope}%
\pgfpathrectangle{\pgfqpoint{0.380943in}{2.260189in}}{\pgfqpoint{4.650000in}{0.614151in}}%
\pgfusepath{clip}%
\pgfsetbuttcap%
\pgfsetroundjoin%
\definecolor{currentfill}{rgb}{0.990004,0.468435,0.468435}%
\pgfsetfillcolor{currentfill}%
\pgfsetlinewidth{0.250937pt}%
\definecolor{currentstroke}{rgb}{1.000000,1.000000,1.000000}%
\pgfsetstrokecolor{currentstroke}%
\pgfsetdash{}{0pt}%
\pgfpathmoveto{\pgfqpoint{2.486603in}{2.874340in}}%
\pgfpathlineto{\pgfqpoint{2.574339in}{2.874340in}}%
\pgfpathlineto{\pgfqpoint{2.574339in}{2.786604in}}%
\pgfpathlineto{\pgfqpoint{2.486603in}{2.786604in}}%
\pgfpathlineto{\pgfqpoint{2.486603in}{2.874340in}}%
\pgfusepath{stroke,fill}%
\end{pgfscope}%
\begin{pgfscope}%
\pgfpathrectangle{\pgfqpoint{0.380943in}{2.260189in}}{\pgfqpoint{4.650000in}{0.614151in}}%
\pgfusepath{clip}%
\pgfsetbuttcap%
\pgfsetroundjoin%
\definecolor{currentfill}{rgb}{0.990634,0.779608,0.623299}%
\pgfsetfillcolor{currentfill}%
\pgfsetlinewidth{0.250937pt}%
\definecolor{currentstroke}{rgb}{1.000000,1.000000,1.000000}%
\pgfsetstrokecolor{currentstroke}%
\pgfsetdash{}{0pt}%
\pgfpathmoveto{\pgfqpoint{2.574339in}{2.874340in}}%
\pgfpathlineto{\pgfqpoint{2.662075in}{2.874340in}}%
\pgfpathlineto{\pgfqpoint{2.662075in}{2.786604in}}%
\pgfpathlineto{\pgfqpoint{2.574339in}{2.786604in}}%
\pgfpathlineto{\pgfqpoint{2.574339in}{2.874340in}}%
\pgfusepath{stroke,fill}%
\end{pgfscope}%
\begin{pgfscope}%
\pgfpathrectangle{\pgfqpoint{0.380943in}{2.260189in}}{\pgfqpoint{4.650000in}{0.614151in}}%
\pgfusepath{clip}%
\pgfsetbuttcap%
\pgfsetroundjoin%
\definecolor{currentfill}{rgb}{1.000000,0.554479,0.510419}%
\pgfsetfillcolor{currentfill}%
\pgfsetlinewidth{0.250937pt}%
\definecolor{currentstroke}{rgb}{1.000000,1.000000,1.000000}%
\pgfsetstrokecolor{currentstroke}%
\pgfsetdash{}{0pt}%
\pgfpathmoveto{\pgfqpoint{2.662075in}{2.874340in}}%
\pgfpathlineto{\pgfqpoint{2.749811in}{2.874340in}}%
\pgfpathlineto{\pgfqpoint{2.749811in}{2.786604in}}%
\pgfpathlineto{\pgfqpoint{2.662075in}{2.786604in}}%
\pgfpathlineto{\pgfqpoint{2.662075in}{2.874340in}}%
\pgfusepath{stroke,fill}%
\end{pgfscope}%
\begin{pgfscope}%
\pgfpathrectangle{\pgfqpoint{0.380943in}{2.260189in}}{\pgfqpoint{4.650000in}{0.614151in}}%
\pgfusepath{clip}%
\pgfsetbuttcap%
\pgfsetroundjoin%
\definecolor{currentfill}{rgb}{0.987266,0.804198,0.639170}%
\pgfsetfillcolor{currentfill}%
\pgfsetlinewidth{0.250937pt}%
\definecolor{currentstroke}{rgb}{1.000000,1.000000,1.000000}%
\pgfsetstrokecolor{currentstroke}%
\pgfsetdash{}{0pt}%
\pgfpathmoveto{\pgfqpoint{2.749811in}{2.874340in}}%
\pgfpathlineto{\pgfqpoint{2.837547in}{2.874340in}}%
\pgfpathlineto{\pgfqpoint{2.837547in}{2.786604in}}%
\pgfpathlineto{\pgfqpoint{2.749811in}{2.786604in}}%
\pgfpathlineto{\pgfqpoint{2.749811in}{2.874340in}}%
\pgfusepath{stroke,fill}%
\end{pgfscope}%
\begin{pgfscope}%
\pgfpathrectangle{\pgfqpoint{0.380943in}{2.260189in}}{\pgfqpoint{4.650000in}{0.614151in}}%
\pgfusepath{clip}%
\pgfsetbuttcap%
\pgfsetroundjoin%
\definecolor{currentfill}{rgb}{1.000000,0.615379,0.534779}%
\pgfsetfillcolor{currentfill}%
\pgfsetlinewidth{0.250937pt}%
\definecolor{currentstroke}{rgb}{1.000000,1.000000,1.000000}%
\pgfsetstrokecolor{currentstroke}%
\pgfsetdash{}{0pt}%
\pgfpathmoveto{\pgfqpoint{2.837547in}{2.874340in}}%
\pgfpathlineto{\pgfqpoint{2.925283in}{2.874340in}}%
\pgfpathlineto{\pgfqpoint{2.925283in}{2.786604in}}%
\pgfpathlineto{\pgfqpoint{2.837547in}{2.786604in}}%
\pgfpathlineto{\pgfqpoint{2.837547in}{2.874340in}}%
\pgfusepath{stroke,fill}%
\end{pgfscope}%
\begin{pgfscope}%
\pgfpathrectangle{\pgfqpoint{0.380943in}{2.260189in}}{\pgfqpoint{4.650000in}{0.614151in}}%
\pgfusepath{clip}%
\pgfsetbuttcap%
\pgfsetroundjoin%
\definecolor{currentfill}{rgb}{0.993679,0.753725,0.608074}%
\pgfsetfillcolor{currentfill}%
\pgfsetlinewidth{0.250937pt}%
\definecolor{currentstroke}{rgb}{1.000000,1.000000,1.000000}%
\pgfsetstrokecolor{currentstroke}%
\pgfsetdash{}{0pt}%
\pgfpathmoveto{\pgfqpoint{2.925283in}{2.874340in}}%
\pgfpathlineto{\pgfqpoint{3.013019in}{2.874340in}}%
\pgfpathlineto{\pgfqpoint{3.013019in}{2.786604in}}%
\pgfpathlineto{\pgfqpoint{2.925283in}{2.786604in}}%
\pgfpathlineto{\pgfqpoint{2.925283in}{2.874340in}}%
\pgfusepath{stroke,fill}%
\end{pgfscope}%
\begin{pgfscope}%
\pgfpathrectangle{\pgfqpoint{0.380943in}{2.260189in}}{\pgfqpoint{4.650000in}{0.614151in}}%
\pgfusepath{clip}%
\pgfsetbuttcap%
\pgfsetroundjoin%
\definecolor{currentfill}{rgb}{0.969504,0.885813,0.700930}%
\pgfsetfillcolor{currentfill}%
\pgfsetlinewidth{0.250937pt}%
\definecolor{currentstroke}{rgb}{1.000000,1.000000,1.000000}%
\pgfsetstrokecolor{currentstroke}%
\pgfsetdash{}{0pt}%
\pgfpathmoveto{\pgfqpoint{3.013019in}{2.874340in}}%
\pgfpathlineto{\pgfqpoint{3.100754in}{2.874340in}}%
\pgfpathlineto{\pgfqpoint{3.100754in}{2.786604in}}%
\pgfpathlineto{\pgfqpoint{3.013019in}{2.786604in}}%
\pgfpathlineto{\pgfqpoint{3.013019in}{2.874340in}}%
\pgfusepath{stroke,fill}%
\end{pgfscope}%
\begin{pgfscope}%
\pgfpathrectangle{\pgfqpoint{0.380943in}{2.260189in}}{\pgfqpoint{4.650000in}{0.614151in}}%
\pgfusepath{clip}%
\pgfsetbuttcap%
\pgfsetroundjoin%
\definecolor{currentfill}{rgb}{0.963260,0.918478,0.719508}%
\pgfsetfillcolor{currentfill}%
\pgfsetlinewidth{0.250937pt}%
\definecolor{currentstroke}{rgb}{1.000000,1.000000,1.000000}%
\pgfsetstrokecolor{currentstroke}%
\pgfsetdash{}{0pt}%
\pgfpathmoveto{\pgfqpoint{3.100754in}{2.874340in}}%
\pgfpathlineto{\pgfqpoint{3.188490in}{2.874340in}}%
\pgfpathlineto{\pgfqpoint{3.188490in}{2.786604in}}%
\pgfpathlineto{\pgfqpoint{3.100754in}{2.786604in}}%
\pgfpathlineto{\pgfqpoint{3.100754in}{2.874340in}}%
\pgfusepath{stroke,fill}%
\end{pgfscope}%
\begin{pgfscope}%
\pgfpathrectangle{\pgfqpoint{0.380943in}{2.260189in}}{\pgfqpoint{4.650000in}{0.614151in}}%
\pgfusepath{clip}%
\pgfsetbuttcap%
\pgfsetroundjoin%
\definecolor{currentfill}{rgb}{0.974072,0.862976,0.688750}%
\pgfsetfillcolor{currentfill}%
\pgfsetlinewidth{0.250937pt}%
\definecolor{currentstroke}{rgb}{1.000000,1.000000,1.000000}%
\pgfsetstrokecolor{currentstroke}%
\pgfsetdash{}{0pt}%
\pgfpathmoveto{\pgfqpoint{3.188490in}{2.874340in}}%
\pgfpathlineto{\pgfqpoint{3.276226in}{2.874340in}}%
\pgfpathlineto{\pgfqpoint{3.276226in}{2.786604in}}%
\pgfpathlineto{\pgfqpoint{3.188490in}{2.786604in}}%
\pgfpathlineto{\pgfqpoint{3.188490in}{2.874340in}}%
\pgfusepath{stroke,fill}%
\end{pgfscope}%
\begin{pgfscope}%
\pgfpathrectangle{\pgfqpoint{0.380943in}{2.260189in}}{\pgfqpoint{4.650000in}{0.614151in}}%
\pgfusepath{clip}%
\pgfsetbuttcap%
\pgfsetroundjoin%
\definecolor{currentfill}{rgb}{0.993679,0.753725,0.608074}%
\pgfsetfillcolor{currentfill}%
\pgfsetlinewidth{0.250937pt}%
\definecolor{currentstroke}{rgb}{1.000000,1.000000,1.000000}%
\pgfsetstrokecolor{currentstroke}%
\pgfsetdash{}{0pt}%
\pgfpathmoveto{\pgfqpoint{3.276226in}{2.874340in}}%
\pgfpathlineto{\pgfqpoint{3.363962in}{2.874340in}}%
\pgfpathlineto{\pgfqpoint{3.363962in}{2.786604in}}%
\pgfpathlineto{\pgfqpoint{3.276226in}{2.786604in}}%
\pgfpathlineto{\pgfqpoint{3.276226in}{2.874340in}}%
\pgfusepath{stroke,fill}%
\end{pgfscope}%
\begin{pgfscope}%
\pgfpathrectangle{\pgfqpoint{0.380943in}{2.260189in}}{\pgfqpoint{4.650000in}{0.614151in}}%
\pgfusepath{clip}%
\pgfsetbuttcap%
\pgfsetroundjoin%
\definecolor{currentfill}{rgb}{0.969504,0.885813,0.700930}%
\pgfsetfillcolor{currentfill}%
\pgfsetlinewidth{0.250937pt}%
\definecolor{currentstroke}{rgb}{1.000000,1.000000,1.000000}%
\pgfsetstrokecolor{currentstroke}%
\pgfsetdash{}{0pt}%
\pgfpathmoveto{\pgfqpoint{3.363962in}{2.874340in}}%
\pgfpathlineto{\pgfqpoint{3.451698in}{2.874340in}}%
\pgfpathlineto{\pgfqpoint{3.451698in}{2.786604in}}%
\pgfpathlineto{\pgfqpoint{3.363962in}{2.786604in}}%
\pgfpathlineto{\pgfqpoint{3.363962in}{2.874340in}}%
\pgfusepath{stroke,fill}%
\end{pgfscope}%
\begin{pgfscope}%
\pgfpathrectangle{\pgfqpoint{0.380943in}{2.260189in}}{\pgfqpoint{4.650000in}{0.614151in}}%
\pgfusepath{clip}%
\pgfsetbuttcap%
\pgfsetroundjoin%
\definecolor{currentfill}{rgb}{1.000000,0.584929,0.522599}%
\pgfsetfillcolor{currentfill}%
\pgfsetlinewidth{0.250937pt}%
\definecolor{currentstroke}{rgb}{1.000000,1.000000,1.000000}%
\pgfsetstrokecolor{currentstroke}%
\pgfsetdash{}{0pt}%
\pgfpathmoveto{\pgfqpoint{3.451698in}{2.874340in}}%
\pgfpathlineto{\pgfqpoint{3.539434in}{2.874340in}}%
\pgfpathlineto{\pgfqpoint{3.539434in}{2.786604in}}%
\pgfpathlineto{\pgfqpoint{3.451698in}{2.786604in}}%
\pgfpathlineto{\pgfqpoint{3.451698in}{2.874340in}}%
\pgfusepath{stroke,fill}%
\end{pgfscope}%
\begin{pgfscope}%
\pgfpathrectangle{\pgfqpoint{0.380943in}{2.260189in}}{\pgfqpoint{4.650000in}{0.614151in}}%
\pgfusepath{clip}%
\pgfsetbuttcap%
\pgfsetroundjoin%
\definecolor{currentfill}{rgb}{0.982699,0.823991,0.657439}%
\pgfsetfillcolor{currentfill}%
\pgfsetlinewidth{0.250937pt}%
\definecolor{currentstroke}{rgb}{1.000000,1.000000,1.000000}%
\pgfsetstrokecolor{currentstroke}%
\pgfsetdash{}{0pt}%
\pgfpathmoveto{\pgfqpoint{3.539434in}{2.874340in}}%
\pgfpathlineto{\pgfqpoint{3.627169in}{2.874340in}}%
\pgfpathlineto{\pgfqpoint{3.627169in}{2.786604in}}%
\pgfpathlineto{\pgfqpoint{3.539434in}{2.786604in}}%
\pgfpathlineto{\pgfqpoint{3.539434in}{2.874340in}}%
\pgfusepath{stroke,fill}%
\end{pgfscope}%
\begin{pgfscope}%
\pgfpathrectangle{\pgfqpoint{0.380943in}{2.260189in}}{\pgfqpoint{4.650000in}{0.614151in}}%
\pgfusepath{clip}%
\pgfsetbuttcap%
\pgfsetroundjoin%
\definecolor{currentfill}{rgb}{0.961738,0.927612,0.725598}%
\pgfsetfillcolor{currentfill}%
\pgfsetlinewidth{0.250937pt}%
\definecolor{currentstroke}{rgb}{1.000000,1.000000,1.000000}%
\pgfsetstrokecolor{currentstroke}%
\pgfsetdash{}{0pt}%
\pgfpathmoveto{\pgfqpoint{3.627169in}{2.874340in}}%
\pgfpathlineto{\pgfqpoint{3.714905in}{2.874340in}}%
\pgfpathlineto{\pgfqpoint{3.714905in}{2.786604in}}%
\pgfpathlineto{\pgfqpoint{3.627169in}{2.786604in}}%
\pgfpathlineto{\pgfqpoint{3.627169in}{2.874340in}}%
\pgfusepath{stroke,fill}%
\end{pgfscope}%
\begin{pgfscope}%
\pgfpathrectangle{\pgfqpoint{0.380943in}{2.260189in}}{\pgfqpoint{4.650000in}{0.614151in}}%
\pgfusepath{clip}%
\pgfsetbuttcap%
\pgfsetroundjoin%
\definecolor{currentfill}{rgb}{0.990634,0.779608,0.623299}%
\pgfsetfillcolor{currentfill}%
\pgfsetlinewidth{0.250937pt}%
\definecolor{currentstroke}{rgb}{1.000000,1.000000,1.000000}%
\pgfsetstrokecolor{currentstroke}%
\pgfsetdash{}{0pt}%
\pgfpathmoveto{\pgfqpoint{3.714905in}{2.874340in}}%
\pgfpathlineto{\pgfqpoint{3.802641in}{2.874340in}}%
\pgfpathlineto{\pgfqpoint{3.802641in}{2.786604in}}%
\pgfpathlineto{\pgfqpoint{3.714905in}{2.786604in}}%
\pgfpathlineto{\pgfqpoint{3.714905in}{2.874340in}}%
\pgfusepath{stroke,fill}%
\end{pgfscope}%
\begin{pgfscope}%
\pgfpathrectangle{\pgfqpoint{0.380943in}{2.260189in}}{\pgfqpoint{4.650000in}{0.614151in}}%
\pgfusepath{clip}%
\pgfsetbuttcap%
\pgfsetroundjoin%
\definecolor{currentfill}{rgb}{0.982699,0.823991,0.657439}%
\pgfsetfillcolor{currentfill}%
\pgfsetlinewidth{0.250937pt}%
\definecolor{currentstroke}{rgb}{1.000000,1.000000,1.000000}%
\pgfsetstrokecolor{currentstroke}%
\pgfsetdash{}{0pt}%
\pgfpathmoveto{\pgfqpoint{3.802641in}{2.874340in}}%
\pgfpathlineto{\pgfqpoint{3.890377in}{2.874340in}}%
\pgfpathlineto{\pgfqpoint{3.890377in}{2.786604in}}%
\pgfpathlineto{\pgfqpoint{3.802641in}{2.786604in}}%
\pgfpathlineto{\pgfqpoint{3.802641in}{2.874340in}}%
\pgfusepath{stroke,fill}%
\end{pgfscope}%
\begin{pgfscope}%
\pgfpathrectangle{\pgfqpoint{0.380943in}{2.260189in}}{\pgfqpoint{4.650000in}{0.614151in}}%
\pgfusepath{clip}%
\pgfsetbuttcap%
\pgfsetroundjoin%
\definecolor{currentfill}{rgb}{0.987266,0.804198,0.639170}%
\pgfsetfillcolor{currentfill}%
\pgfsetlinewidth{0.250937pt}%
\definecolor{currentstroke}{rgb}{1.000000,1.000000,1.000000}%
\pgfsetstrokecolor{currentstroke}%
\pgfsetdash{}{0pt}%
\pgfpathmoveto{\pgfqpoint{3.890377in}{2.874340in}}%
\pgfpathlineto{\pgfqpoint{3.978113in}{2.874340in}}%
\pgfpathlineto{\pgfqpoint{3.978113in}{2.786604in}}%
\pgfpathlineto{\pgfqpoint{3.890377in}{2.786604in}}%
\pgfpathlineto{\pgfqpoint{3.890377in}{2.874340in}}%
\pgfusepath{stroke,fill}%
\end{pgfscope}%
\begin{pgfscope}%
\pgfpathrectangle{\pgfqpoint{0.380943in}{2.260189in}}{\pgfqpoint{4.650000in}{0.614151in}}%
\pgfusepath{clip}%
\pgfsetbuttcap%
\pgfsetroundjoin%
\definecolor{currentfill}{rgb}{0.982699,0.823991,0.657439}%
\pgfsetfillcolor{currentfill}%
\pgfsetlinewidth{0.250937pt}%
\definecolor{currentstroke}{rgb}{1.000000,1.000000,1.000000}%
\pgfsetstrokecolor{currentstroke}%
\pgfsetdash{}{0pt}%
\pgfpathmoveto{\pgfqpoint{3.978113in}{2.874340in}}%
\pgfpathlineto{\pgfqpoint{4.065849in}{2.874340in}}%
\pgfpathlineto{\pgfqpoint{4.065849in}{2.786604in}}%
\pgfpathlineto{\pgfqpoint{3.978113in}{2.786604in}}%
\pgfpathlineto{\pgfqpoint{3.978113in}{2.874340in}}%
\pgfusepath{stroke,fill}%
\end{pgfscope}%
\begin{pgfscope}%
\pgfpathrectangle{\pgfqpoint{0.380943in}{2.260189in}}{\pgfqpoint{4.650000in}{0.614151in}}%
\pgfusepath{clip}%
\pgfsetbuttcap%
\pgfsetroundjoin%
\definecolor{currentfill}{rgb}{0.978639,0.841584,0.673679}%
\pgfsetfillcolor{currentfill}%
\pgfsetlinewidth{0.250937pt}%
\definecolor{currentstroke}{rgb}{1.000000,1.000000,1.000000}%
\pgfsetstrokecolor{currentstroke}%
\pgfsetdash{}{0pt}%
\pgfpathmoveto{\pgfqpoint{4.065849in}{2.874340in}}%
\pgfpathlineto{\pgfqpoint{4.153585in}{2.874340in}}%
\pgfpathlineto{\pgfqpoint{4.153585in}{2.786604in}}%
\pgfpathlineto{\pgfqpoint{4.065849in}{2.786604in}}%
\pgfpathlineto{\pgfqpoint{4.065849in}{2.874340in}}%
\pgfusepath{stroke,fill}%
\end{pgfscope}%
\begin{pgfscope}%
\pgfpathrectangle{\pgfqpoint{0.380943in}{2.260189in}}{\pgfqpoint{4.650000in}{0.614151in}}%
\pgfusepath{clip}%
\pgfsetbuttcap%
\pgfsetroundjoin%
\definecolor{currentfill}{rgb}{0.978639,0.841584,0.673679}%
\pgfsetfillcolor{currentfill}%
\pgfsetlinewidth{0.250937pt}%
\definecolor{currentstroke}{rgb}{1.000000,1.000000,1.000000}%
\pgfsetstrokecolor{currentstroke}%
\pgfsetdash{}{0pt}%
\pgfpathmoveto{\pgfqpoint{4.153585in}{2.874340in}}%
\pgfpathlineto{\pgfqpoint{4.241320in}{2.874340in}}%
\pgfpathlineto{\pgfqpoint{4.241320in}{2.786604in}}%
\pgfpathlineto{\pgfqpoint{4.153585in}{2.786604in}}%
\pgfpathlineto{\pgfqpoint{4.153585in}{2.874340in}}%
\pgfusepath{stroke,fill}%
\end{pgfscope}%
\begin{pgfscope}%
\pgfpathrectangle{\pgfqpoint{0.380943in}{2.260189in}}{\pgfqpoint{4.650000in}{0.614151in}}%
\pgfusepath{clip}%
\pgfsetbuttcap%
\pgfsetroundjoin%
\definecolor{currentfill}{rgb}{0.997924,0.685352,0.570242}%
\pgfsetfillcolor{currentfill}%
\pgfsetlinewidth{0.250937pt}%
\definecolor{currentstroke}{rgb}{1.000000,1.000000,1.000000}%
\pgfsetstrokecolor{currentstroke}%
\pgfsetdash{}{0pt}%
\pgfpathmoveto{\pgfqpoint{4.241320in}{2.874340in}}%
\pgfpathlineto{\pgfqpoint{4.329056in}{2.874340in}}%
\pgfpathlineto{\pgfqpoint{4.329056in}{2.786604in}}%
\pgfpathlineto{\pgfqpoint{4.241320in}{2.786604in}}%
\pgfpathlineto{\pgfqpoint{4.241320in}{2.874340in}}%
\pgfusepath{stroke,fill}%
\end{pgfscope}%
\begin{pgfscope}%
\pgfpathrectangle{\pgfqpoint{0.380943in}{2.260189in}}{\pgfqpoint{4.650000in}{0.614151in}}%
\pgfusepath{clip}%
\pgfsetbuttcap%
\pgfsetroundjoin%
\definecolor{currentfill}{rgb}{0.987266,0.804198,0.639170}%
\pgfsetfillcolor{currentfill}%
\pgfsetlinewidth{0.250937pt}%
\definecolor{currentstroke}{rgb}{1.000000,1.000000,1.000000}%
\pgfsetstrokecolor{currentstroke}%
\pgfsetdash{}{0pt}%
\pgfpathmoveto{\pgfqpoint{4.329056in}{2.874340in}}%
\pgfpathlineto{\pgfqpoint{4.416792in}{2.874340in}}%
\pgfpathlineto{\pgfqpoint{4.416792in}{2.786604in}}%
\pgfpathlineto{\pgfqpoint{4.329056in}{2.786604in}}%
\pgfpathlineto{\pgfqpoint{4.329056in}{2.874340in}}%
\pgfusepath{stroke,fill}%
\end{pgfscope}%
\begin{pgfscope}%
\pgfpathrectangle{\pgfqpoint{0.380943in}{2.260189in}}{\pgfqpoint{4.650000in}{0.614151in}}%
\pgfusepath{clip}%
\pgfsetbuttcap%
\pgfsetroundjoin%
\definecolor{currentfill}{rgb}{0.993679,0.753725,0.608074}%
\pgfsetfillcolor{currentfill}%
\pgfsetlinewidth{0.250937pt}%
\definecolor{currentstroke}{rgb}{1.000000,1.000000,1.000000}%
\pgfsetstrokecolor{currentstroke}%
\pgfsetdash{}{0pt}%
\pgfpathmoveto{\pgfqpoint{4.416792in}{2.874340in}}%
\pgfpathlineto{\pgfqpoint{4.504528in}{2.874340in}}%
\pgfpathlineto{\pgfqpoint{4.504528in}{2.786604in}}%
\pgfpathlineto{\pgfqpoint{4.416792in}{2.786604in}}%
\pgfpathlineto{\pgfqpoint{4.416792in}{2.874340in}}%
\pgfusepath{stroke,fill}%
\end{pgfscope}%
\begin{pgfscope}%
\pgfpathrectangle{\pgfqpoint{0.380943in}{2.260189in}}{\pgfqpoint{4.650000in}{0.614151in}}%
\pgfusepath{clip}%
\pgfsetbuttcap%
\pgfsetroundjoin%
\definecolor{currentfill}{rgb}{0.987266,0.804198,0.639170}%
\pgfsetfillcolor{currentfill}%
\pgfsetlinewidth{0.250937pt}%
\definecolor{currentstroke}{rgb}{1.000000,1.000000,1.000000}%
\pgfsetstrokecolor{currentstroke}%
\pgfsetdash{}{0pt}%
\pgfpathmoveto{\pgfqpoint{4.504528in}{2.874340in}}%
\pgfpathlineto{\pgfqpoint{4.592264in}{2.874340in}}%
\pgfpathlineto{\pgfqpoint{4.592264in}{2.786604in}}%
\pgfpathlineto{\pgfqpoint{4.504528in}{2.786604in}}%
\pgfpathlineto{\pgfqpoint{4.504528in}{2.874340in}}%
\pgfusepath{stroke,fill}%
\end{pgfscope}%
\begin{pgfscope}%
\pgfpathrectangle{\pgfqpoint{0.380943in}{2.260189in}}{\pgfqpoint{4.650000in}{0.614151in}}%
\pgfusepath{clip}%
\pgfsetbuttcap%
\pgfsetroundjoin%
\definecolor{currentfill}{rgb}{0.964937,0.908651,0.713110}%
\pgfsetfillcolor{currentfill}%
\pgfsetlinewidth{0.250937pt}%
\definecolor{currentstroke}{rgb}{1.000000,1.000000,1.000000}%
\pgfsetstrokecolor{currentstroke}%
\pgfsetdash{}{0pt}%
\pgfpathmoveto{\pgfqpoint{4.592264in}{2.874340in}}%
\pgfpathlineto{\pgfqpoint{4.680000in}{2.874340in}}%
\pgfpathlineto{\pgfqpoint{4.680000in}{2.786604in}}%
\pgfpathlineto{\pgfqpoint{4.592264in}{2.786604in}}%
\pgfpathlineto{\pgfqpoint{4.592264in}{2.874340in}}%
\pgfusepath{stroke,fill}%
\end{pgfscope}%
\begin{pgfscope}%
\pgfpathrectangle{\pgfqpoint{0.380943in}{2.260189in}}{\pgfqpoint{4.650000in}{0.614151in}}%
\pgfusepath{clip}%
\pgfsetbuttcap%
\pgfsetroundjoin%
\definecolor{currentfill}{rgb}{0.987266,0.804198,0.639170}%
\pgfsetfillcolor{currentfill}%
\pgfsetlinewidth{0.250937pt}%
\definecolor{currentstroke}{rgb}{1.000000,1.000000,1.000000}%
\pgfsetstrokecolor{currentstroke}%
\pgfsetdash{}{0pt}%
\pgfpathmoveto{\pgfqpoint{4.680000in}{2.874340in}}%
\pgfpathlineto{\pgfqpoint{4.767736in}{2.874340in}}%
\pgfpathlineto{\pgfqpoint{4.767736in}{2.786604in}}%
\pgfpathlineto{\pgfqpoint{4.680000in}{2.786604in}}%
\pgfpathlineto{\pgfqpoint{4.680000in}{2.874340in}}%
\pgfusepath{stroke,fill}%
\end{pgfscope}%
\begin{pgfscope}%
\pgfpathrectangle{\pgfqpoint{0.380943in}{2.260189in}}{\pgfqpoint{4.650000in}{0.614151in}}%
\pgfusepath{clip}%
\pgfsetbuttcap%
\pgfsetroundjoin%
\definecolor{currentfill}{rgb}{0.978639,0.841584,0.673679}%
\pgfsetfillcolor{currentfill}%
\pgfsetlinewidth{0.250937pt}%
\definecolor{currentstroke}{rgb}{1.000000,1.000000,1.000000}%
\pgfsetstrokecolor{currentstroke}%
\pgfsetdash{}{0pt}%
\pgfpathmoveto{\pgfqpoint{4.767736in}{2.874340in}}%
\pgfpathlineto{\pgfqpoint{4.855471in}{2.874340in}}%
\pgfpathlineto{\pgfqpoint{4.855471in}{2.786604in}}%
\pgfpathlineto{\pgfqpoint{4.767736in}{2.786604in}}%
\pgfpathlineto{\pgfqpoint{4.767736in}{2.874340in}}%
\pgfusepath{stroke,fill}%
\end{pgfscope}%
\begin{pgfscope}%
\pgfpathrectangle{\pgfqpoint{0.380943in}{2.260189in}}{\pgfqpoint{4.650000in}{0.614151in}}%
\pgfusepath{clip}%
\pgfsetbuttcap%
\pgfsetroundjoin%
\definecolor{currentfill}{rgb}{0.969504,0.885813,0.700930}%
\pgfsetfillcolor{currentfill}%
\pgfsetlinewidth{0.250937pt}%
\definecolor{currentstroke}{rgb}{1.000000,1.000000,1.000000}%
\pgfsetstrokecolor{currentstroke}%
\pgfsetdash{}{0pt}%
\pgfpathmoveto{\pgfqpoint{4.855471in}{2.874340in}}%
\pgfpathlineto{\pgfqpoint{4.943207in}{2.874340in}}%
\pgfpathlineto{\pgfqpoint{4.943207in}{2.786604in}}%
\pgfpathlineto{\pgfqpoint{4.855471in}{2.786604in}}%
\pgfpathlineto{\pgfqpoint{4.855471in}{2.874340in}}%
\pgfusepath{stroke,fill}%
\end{pgfscope}%
\begin{pgfscope}%
\pgfpathrectangle{\pgfqpoint{0.380943in}{2.260189in}}{\pgfqpoint{4.650000in}{0.614151in}}%
\pgfusepath{clip}%
\pgfsetbuttcap%
\pgfsetroundjoin%
\definecolor{currentfill}{rgb}{0.987266,0.804198,0.639170}%
\pgfsetfillcolor{currentfill}%
\pgfsetlinewidth{0.250937pt}%
\definecolor{currentstroke}{rgb}{1.000000,1.000000,1.000000}%
\pgfsetstrokecolor{currentstroke}%
\pgfsetdash{}{0pt}%
\pgfpathmoveto{\pgfqpoint{4.943207in}{2.874340in}}%
\pgfpathlineto{\pgfqpoint{5.030943in}{2.874340in}}%
\pgfpathlineto{\pgfqpoint{5.030943in}{2.786604in}}%
\pgfpathlineto{\pgfqpoint{4.943207in}{2.786604in}}%
\pgfpathlineto{\pgfqpoint{4.943207in}{2.874340in}}%
\pgfusepath{stroke,fill}%
\end{pgfscope}%
\begin{pgfscope}%
\pgfpathrectangle{\pgfqpoint{0.380943in}{2.260189in}}{\pgfqpoint{4.650000in}{0.614151in}}%
\pgfusepath{clip}%
\pgfsetbuttcap%
\pgfsetroundjoin%
\pgfsetlinewidth{0.250937pt}%
\definecolor{currentstroke}{rgb}{1.000000,1.000000,1.000000}%
\pgfsetstrokecolor{currentstroke}%
\pgfsetdash{}{0pt}%
\pgfpathmoveto{\pgfqpoint{0.380943in}{2.786604in}}%
\pgfpathlineto{\pgfqpoint{0.468679in}{2.786604in}}%
\pgfpathlineto{\pgfqpoint{0.468679in}{2.698868in}}%
\pgfpathlineto{\pgfqpoint{0.380943in}{2.698868in}}%
\pgfpathlineto{\pgfqpoint{0.380943in}{2.786604in}}%
\pgfusepath{stroke}%
\end{pgfscope}%
\begin{pgfscope}%
\pgfpathrectangle{\pgfqpoint{0.380943in}{2.260189in}}{\pgfqpoint{4.650000in}{0.614151in}}%
\pgfusepath{clip}%
\pgfsetbuttcap%
\pgfsetroundjoin%
\definecolor{currentfill}{rgb}{0.996401,0.724937,0.591557}%
\pgfsetfillcolor{currentfill}%
\pgfsetlinewidth{0.250937pt}%
\definecolor{currentstroke}{rgb}{1.000000,1.000000,1.000000}%
\pgfsetstrokecolor{currentstroke}%
\pgfsetdash{}{0pt}%
\pgfpathmoveto{\pgfqpoint{0.468679in}{2.786604in}}%
\pgfpathlineto{\pgfqpoint{0.556415in}{2.786604in}}%
\pgfpathlineto{\pgfqpoint{0.556415in}{2.698868in}}%
\pgfpathlineto{\pgfqpoint{0.468679in}{2.698868in}}%
\pgfpathlineto{\pgfqpoint{0.468679in}{2.786604in}}%
\pgfusepath{stroke,fill}%
\end{pgfscope}%
\begin{pgfscope}%
\pgfpathrectangle{\pgfqpoint{0.380943in}{2.260189in}}{\pgfqpoint{4.650000in}{0.614151in}}%
\pgfusepath{clip}%
\pgfsetbuttcap%
\pgfsetroundjoin%
\definecolor{currentfill}{rgb}{0.978639,0.841584,0.673679}%
\pgfsetfillcolor{currentfill}%
\pgfsetlinewidth{0.250937pt}%
\definecolor{currentstroke}{rgb}{1.000000,1.000000,1.000000}%
\pgfsetstrokecolor{currentstroke}%
\pgfsetdash{}{0pt}%
\pgfpathmoveto{\pgfqpoint{0.556415in}{2.786604in}}%
\pgfpathlineto{\pgfqpoint{0.644151in}{2.786604in}}%
\pgfpathlineto{\pgfqpoint{0.644151in}{2.698868in}}%
\pgfpathlineto{\pgfqpoint{0.556415in}{2.698868in}}%
\pgfpathlineto{\pgfqpoint{0.556415in}{2.786604in}}%
\pgfusepath{stroke,fill}%
\end{pgfscope}%
\begin{pgfscope}%
\pgfpathrectangle{\pgfqpoint{0.380943in}{2.260189in}}{\pgfqpoint{4.650000in}{0.614151in}}%
\pgfusepath{clip}%
\pgfsetbuttcap%
\pgfsetroundjoin%
\definecolor{currentfill}{rgb}{0.964937,0.908651,0.713110}%
\pgfsetfillcolor{currentfill}%
\pgfsetlinewidth{0.250937pt}%
\definecolor{currentstroke}{rgb}{1.000000,1.000000,1.000000}%
\pgfsetstrokecolor{currentstroke}%
\pgfsetdash{}{0pt}%
\pgfpathmoveto{\pgfqpoint{0.644151in}{2.786604in}}%
\pgfpathlineto{\pgfqpoint{0.731886in}{2.786604in}}%
\pgfpathlineto{\pgfqpoint{0.731886in}{2.698868in}}%
\pgfpathlineto{\pgfqpoint{0.644151in}{2.698868in}}%
\pgfpathlineto{\pgfqpoint{0.644151in}{2.786604in}}%
\pgfusepath{stroke,fill}%
\end{pgfscope}%
\begin{pgfscope}%
\pgfpathrectangle{\pgfqpoint{0.380943in}{2.260189in}}{\pgfqpoint{4.650000in}{0.614151in}}%
\pgfusepath{clip}%
\pgfsetbuttcap%
\pgfsetroundjoin%
\definecolor{currentfill}{rgb}{0.963260,0.918478,0.719508}%
\pgfsetfillcolor{currentfill}%
\pgfsetlinewidth{0.250937pt}%
\definecolor{currentstroke}{rgb}{1.000000,1.000000,1.000000}%
\pgfsetstrokecolor{currentstroke}%
\pgfsetdash{}{0pt}%
\pgfpathmoveto{\pgfqpoint{0.731886in}{2.786604in}}%
\pgfpathlineto{\pgfqpoint{0.819622in}{2.786604in}}%
\pgfpathlineto{\pgfqpoint{0.819622in}{2.698868in}}%
\pgfpathlineto{\pgfqpoint{0.731886in}{2.698868in}}%
\pgfpathlineto{\pgfqpoint{0.731886in}{2.786604in}}%
\pgfusepath{stroke,fill}%
\end{pgfscope}%
\begin{pgfscope}%
\pgfpathrectangle{\pgfqpoint{0.380943in}{2.260189in}}{\pgfqpoint{4.650000in}{0.614151in}}%
\pgfusepath{clip}%
\pgfsetbuttcap%
\pgfsetroundjoin%
\definecolor{currentfill}{rgb}{0.964937,0.908651,0.713110}%
\pgfsetfillcolor{currentfill}%
\pgfsetlinewidth{0.250937pt}%
\definecolor{currentstroke}{rgb}{1.000000,1.000000,1.000000}%
\pgfsetstrokecolor{currentstroke}%
\pgfsetdash{}{0pt}%
\pgfpathmoveto{\pgfqpoint{0.819622in}{2.786604in}}%
\pgfpathlineto{\pgfqpoint{0.907358in}{2.786604in}}%
\pgfpathlineto{\pgfqpoint{0.907358in}{2.698868in}}%
\pgfpathlineto{\pgfqpoint{0.819622in}{2.698868in}}%
\pgfpathlineto{\pgfqpoint{0.819622in}{2.786604in}}%
\pgfusepath{stroke,fill}%
\end{pgfscope}%
\begin{pgfscope}%
\pgfpathrectangle{\pgfqpoint{0.380943in}{2.260189in}}{\pgfqpoint{4.650000in}{0.614151in}}%
\pgfusepath{clip}%
\pgfsetbuttcap%
\pgfsetroundjoin%
\definecolor{currentfill}{rgb}{0.996401,0.724937,0.591557}%
\pgfsetfillcolor{currentfill}%
\pgfsetlinewidth{0.250937pt}%
\definecolor{currentstroke}{rgb}{1.000000,1.000000,1.000000}%
\pgfsetstrokecolor{currentstroke}%
\pgfsetdash{}{0pt}%
\pgfpathmoveto{\pgfqpoint{0.907358in}{2.786604in}}%
\pgfpathlineto{\pgfqpoint{0.995094in}{2.786604in}}%
\pgfpathlineto{\pgfqpoint{0.995094in}{2.698868in}}%
\pgfpathlineto{\pgfqpoint{0.907358in}{2.698868in}}%
\pgfpathlineto{\pgfqpoint{0.907358in}{2.786604in}}%
\pgfusepath{stroke,fill}%
\end{pgfscope}%
\begin{pgfscope}%
\pgfpathrectangle{\pgfqpoint{0.380943in}{2.260189in}}{\pgfqpoint{4.650000in}{0.614151in}}%
\pgfusepath{clip}%
\pgfsetbuttcap%
\pgfsetroundjoin%
\definecolor{currentfill}{rgb}{0.987266,0.804198,0.639170}%
\pgfsetfillcolor{currentfill}%
\pgfsetlinewidth{0.250937pt}%
\definecolor{currentstroke}{rgb}{1.000000,1.000000,1.000000}%
\pgfsetstrokecolor{currentstroke}%
\pgfsetdash{}{0pt}%
\pgfpathmoveto{\pgfqpoint{0.995094in}{2.786604in}}%
\pgfpathlineto{\pgfqpoint{1.082830in}{2.786604in}}%
\pgfpathlineto{\pgfqpoint{1.082830in}{2.698868in}}%
\pgfpathlineto{\pgfqpoint{0.995094in}{2.698868in}}%
\pgfpathlineto{\pgfqpoint{0.995094in}{2.786604in}}%
\pgfusepath{stroke,fill}%
\end{pgfscope}%
\begin{pgfscope}%
\pgfpathrectangle{\pgfqpoint{0.380943in}{2.260189in}}{\pgfqpoint{4.650000in}{0.614151in}}%
\pgfusepath{clip}%
\pgfsetbuttcap%
\pgfsetroundjoin%
\definecolor{currentfill}{rgb}{0.978639,0.841584,0.673679}%
\pgfsetfillcolor{currentfill}%
\pgfsetlinewidth{0.250937pt}%
\definecolor{currentstroke}{rgb}{1.000000,1.000000,1.000000}%
\pgfsetstrokecolor{currentstroke}%
\pgfsetdash{}{0pt}%
\pgfpathmoveto{\pgfqpoint{1.082830in}{2.786604in}}%
\pgfpathlineto{\pgfqpoint{1.170566in}{2.786604in}}%
\pgfpathlineto{\pgfqpoint{1.170566in}{2.698868in}}%
\pgfpathlineto{\pgfqpoint{1.082830in}{2.698868in}}%
\pgfpathlineto{\pgfqpoint{1.082830in}{2.786604in}}%
\pgfusepath{stroke,fill}%
\end{pgfscope}%
\begin{pgfscope}%
\pgfpathrectangle{\pgfqpoint{0.380943in}{2.260189in}}{\pgfqpoint{4.650000in}{0.614151in}}%
\pgfusepath{clip}%
\pgfsetbuttcap%
\pgfsetroundjoin%
\definecolor{currentfill}{rgb}{0.974072,0.862976,0.688750}%
\pgfsetfillcolor{currentfill}%
\pgfsetlinewidth{0.250937pt}%
\definecolor{currentstroke}{rgb}{1.000000,1.000000,1.000000}%
\pgfsetstrokecolor{currentstroke}%
\pgfsetdash{}{0pt}%
\pgfpathmoveto{\pgfqpoint{1.170566in}{2.786604in}}%
\pgfpathlineto{\pgfqpoint{1.258302in}{2.786604in}}%
\pgfpathlineto{\pgfqpoint{1.258302in}{2.698868in}}%
\pgfpathlineto{\pgfqpoint{1.170566in}{2.698868in}}%
\pgfpathlineto{\pgfqpoint{1.170566in}{2.786604in}}%
\pgfusepath{stroke,fill}%
\end{pgfscope}%
\begin{pgfscope}%
\pgfpathrectangle{\pgfqpoint{0.380943in}{2.260189in}}{\pgfqpoint{4.650000in}{0.614151in}}%
\pgfusepath{clip}%
\pgfsetbuttcap%
\pgfsetroundjoin%
\definecolor{currentfill}{rgb}{0.964937,0.908651,0.713110}%
\pgfsetfillcolor{currentfill}%
\pgfsetlinewidth{0.250937pt}%
\definecolor{currentstroke}{rgb}{1.000000,1.000000,1.000000}%
\pgfsetstrokecolor{currentstroke}%
\pgfsetdash{}{0pt}%
\pgfpathmoveto{\pgfqpoint{1.258302in}{2.786604in}}%
\pgfpathlineto{\pgfqpoint{1.346037in}{2.786604in}}%
\pgfpathlineto{\pgfqpoint{1.346037in}{2.698868in}}%
\pgfpathlineto{\pgfqpoint{1.258302in}{2.698868in}}%
\pgfpathlineto{\pgfqpoint{1.258302in}{2.786604in}}%
\pgfusepath{stroke,fill}%
\end{pgfscope}%
\begin{pgfscope}%
\pgfpathrectangle{\pgfqpoint{0.380943in}{2.260189in}}{\pgfqpoint{4.650000in}{0.614151in}}%
\pgfusepath{clip}%
\pgfsetbuttcap%
\pgfsetroundjoin%
\definecolor{currentfill}{rgb}{1.000000,1.000000,0.857516}%
\pgfsetfillcolor{currentfill}%
\pgfsetlinewidth{0.250937pt}%
\definecolor{currentstroke}{rgb}{1.000000,1.000000,1.000000}%
\pgfsetstrokecolor{currentstroke}%
\pgfsetdash{}{0pt}%
\pgfpathmoveto{\pgfqpoint{1.346037in}{2.786604in}}%
\pgfpathlineto{\pgfqpoint{1.433773in}{2.786604in}}%
\pgfpathlineto{\pgfqpoint{1.433773in}{2.698868in}}%
\pgfpathlineto{\pgfqpoint{1.346037in}{2.698868in}}%
\pgfpathlineto{\pgfqpoint{1.346037in}{2.786604in}}%
\pgfusepath{stroke,fill}%
\end{pgfscope}%
\begin{pgfscope}%
\pgfpathrectangle{\pgfqpoint{0.380943in}{2.260189in}}{\pgfqpoint{4.650000in}{0.614151in}}%
\pgfusepath{clip}%
\pgfsetbuttcap%
\pgfsetroundjoin%
\definecolor{currentfill}{rgb}{0.964783,0.940131,0.739808}%
\pgfsetfillcolor{currentfill}%
\pgfsetlinewidth{0.250937pt}%
\definecolor{currentstroke}{rgb}{1.000000,1.000000,1.000000}%
\pgfsetstrokecolor{currentstroke}%
\pgfsetdash{}{0pt}%
\pgfpathmoveto{\pgfqpoint{1.433773in}{2.786604in}}%
\pgfpathlineto{\pgfqpoint{1.521509in}{2.786604in}}%
\pgfpathlineto{\pgfqpoint{1.521509in}{2.698868in}}%
\pgfpathlineto{\pgfqpoint{1.433773in}{2.698868in}}%
\pgfpathlineto{\pgfqpoint{1.433773in}{2.786604in}}%
\pgfusepath{stroke,fill}%
\end{pgfscope}%
\begin{pgfscope}%
\pgfpathrectangle{\pgfqpoint{0.380943in}{2.260189in}}{\pgfqpoint{4.650000in}{0.614151in}}%
\pgfusepath{clip}%
\pgfsetbuttcap%
\pgfsetroundjoin%
\definecolor{currentfill}{rgb}{0.980008,0.966013,0.779393}%
\pgfsetfillcolor{currentfill}%
\pgfsetlinewidth{0.250937pt}%
\definecolor{currentstroke}{rgb}{1.000000,1.000000,1.000000}%
\pgfsetstrokecolor{currentstroke}%
\pgfsetdash{}{0pt}%
\pgfpathmoveto{\pgfqpoint{1.521509in}{2.786604in}}%
\pgfpathlineto{\pgfqpoint{1.609245in}{2.786604in}}%
\pgfpathlineto{\pgfqpoint{1.609245in}{2.698868in}}%
\pgfpathlineto{\pgfqpoint{1.521509in}{2.698868in}}%
\pgfpathlineto{\pgfqpoint{1.521509in}{2.786604in}}%
\pgfusepath{stroke,fill}%
\end{pgfscope}%
\begin{pgfscope}%
\pgfpathrectangle{\pgfqpoint{0.380943in}{2.260189in}}{\pgfqpoint{4.650000in}{0.614151in}}%
\pgfusepath{clip}%
\pgfsetbuttcap%
\pgfsetroundjoin%
\definecolor{currentfill}{rgb}{0.964783,0.940131,0.739808}%
\pgfsetfillcolor{currentfill}%
\pgfsetlinewidth{0.250937pt}%
\definecolor{currentstroke}{rgb}{1.000000,1.000000,1.000000}%
\pgfsetstrokecolor{currentstroke}%
\pgfsetdash{}{0pt}%
\pgfpathmoveto{\pgfqpoint{1.609245in}{2.786604in}}%
\pgfpathlineto{\pgfqpoint{1.696981in}{2.786604in}}%
\pgfpathlineto{\pgfqpoint{1.696981in}{2.698868in}}%
\pgfpathlineto{\pgfqpoint{1.609245in}{2.698868in}}%
\pgfpathlineto{\pgfqpoint{1.609245in}{2.786604in}}%
\pgfusepath{stroke,fill}%
\end{pgfscope}%
\begin{pgfscope}%
\pgfpathrectangle{\pgfqpoint{0.380943in}{2.260189in}}{\pgfqpoint{4.650000in}{0.614151in}}%
\pgfusepath{clip}%
\pgfsetbuttcap%
\pgfsetroundjoin%
\definecolor{currentfill}{rgb}{0.995233,0.991895,0.818977}%
\pgfsetfillcolor{currentfill}%
\pgfsetlinewidth{0.250937pt}%
\definecolor{currentstroke}{rgb}{1.000000,1.000000,1.000000}%
\pgfsetstrokecolor{currentstroke}%
\pgfsetdash{}{0pt}%
\pgfpathmoveto{\pgfqpoint{1.696981in}{2.786604in}}%
\pgfpathlineto{\pgfqpoint{1.784717in}{2.786604in}}%
\pgfpathlineto{\pgfqpoint{1.784717in}{2.698868in}}%
\pgfpathlineto{\pgfqpoint{1.696981in}{2.698868in}}%
\pgfpathlineto{\pgfqpoint{1.696981in}{2.786604in}}%
\pgfusepath{stroke,fill}%
\end{pgfscope}%
\begin{pgfscope}%
\pgfpathrectangle{\pgfqpoint{0.380943in}{2.260189in}}{\pgfqpoint{4.650000in}{0.614151in}}%
\pgfusepath{clip}%
\pgfsetbuttcap%
\pgfsetroundjoin%
\definecolor{currentfill}{rgb}{0.964783,0.940131,0.739808}%
\pgfsetfillcolor{currentfill}%
\pgfsetlinewidth{0.250937pt}%
\definecolor{currentstroke}{rgb}{1.000000,1.000000,1.000000}%
\pgfsetstrokecolor{currentstroke}%
\pgfsetdash{}{0pt}%
\pgfpathmoveto{\pgfqpoint{1.784717in}{2.786604in}}%
\pgfpathlineto{\pgfqpoint{1.872452in}{2.786604in}}%
\pgfpathlineto{\pgfqpoint{1.872452in}{2.698868in}}%
\pgfpathlineto{\pgfqpoint{1.784717in}{2.698868in}}%
\pgfpathlineto{\pgfqpoint{1.784717in}{2.786604in}}%
\pgfusepath{stroke,fill}%
\end{pgfscope}%
\begin{pgfscope}%
\pgfpathrectangle{\pgfqpoint{0.380943in}{2.260189in}}{\pgfqpoint{4.650000in}{0.614151in}}%
\pgfusepath{clip}%
\pgfsetbuttcap%
\pgfsetroundjoin%
\definecolor{currentfill}{rgb}{0.964937,0.908651,0.713110}%
\pgfsetfillcolor{currentfill}%
\pgfsetlinewidth{0.250937pt}%
\definecolor{currentstroke}{rgb}{1.000000,1.000000,1.000000}%
\pgfsetstrokecolor{currentstroke}%
\pgfsetdash{}{0pt}%
\pgfpathmoveto{\pgfqpoint{1.872452in}{2.786604in}}%
\pgfpathlineto{\pgfqpoint{1.960188in}{2.786604in}}%
\pgfpathlineto{\pgfqpoint{1.960188in}{2.698868in}}%
\pgfpathlineto{\pgfqpoint{1.872452in}{2.698868in}}%
\pgfpathlineto{\pgfqpoint{1.872452in}{2.786604in}}%
\pgfusepath{stroke,fill}%
\end{pgfscope}%
\begin{pgfscope}%
\pgfpathrectangle{\pgfqpoint{0.380943in}{2.260189in}}{\pgfqpoint{4.650000in}{0.614151in}}%
\pgfusepath{clip}%
\pgfsetbuttcap%
\pgfsetroundjoin%
\definecolor{currentfill}{rgb}{0.963260,0.918478,0.719508}%
\pgfsetfillcolor{currentfill}%
\pgfsetlinewidth{0.250937pt}%
\definecolor{currentstroke}{rgb}{1.000000,1.000000,1.000000}%
\pgfsetstrokecolor{currentstroke}%
\pgfsetdash{}{0pt}%
\pgfpathmoveto{\pgfqpoint{1.960188in}{2.786604in}}%
\pgfpathlineto{\pgfqpoint{2.047924in}{2.786604in}}%
\pgfpathlineto{\pgfqpoint{2.047924in}{2.698868in}}%
\pgfpathlineto{\pgfqpoint{1.960188in}{2.698868in}}%
\pgfpathlineto{\pgfqpoint{1.960188in}{2.786604in}}%
\pgfusepath{stroke,fill}%
\end{pgfscope}%
\begin{pgfscope}%
\pgfpathrectangle{\pgfqpoint{0.380943in}{2.260189in}}{\pgfqpoint{4.650000in}{0.614151in}}%
\pgfusepath{clip}%
\pgfsetbuttcap%
\pgfsetroundjoin%
\definecolor{currentfill}{rgb}{0.961738,0.927612,0.725598}%
\pgfsetfillcolor{currentfill}%
\pgfsetlinewidth{0.250937pt}%
\definecolor{currentstroke}{rgb}{1.000000,1.000000,1.000000}%
\pgfsetstrokecolor{currentstroke}%
\pgfsetdash{}{0pt}%
\pgfpathmoveto{\pgfqpoint{2.047924in}{2.786604in}}%
\pgfpathlineto{\pgfqpoint{2.135660in}{2.786604in}}%
\pgfpathlineto{\pgfqpoint{2.135660in}{2.698868in}}%
\pgfpathlineto{\pgfqpoint{2.047924in}{2.698868in}}%
\pgfpathlineto{\pgfqpoint{2.047924in}{2.786604in}}%
\pgfusepath{stroke,fill}%
\end{pgfscope}%
\begin{pgfscope}%
\pgfpathrectangle{\pgfqpoint{0.380943in}{2.260189in}}{\pgfqpoint{4.650000in}{0.614151in}}%
\pgfusepath{clip}%
\pgfsetbuttcap%
\pgfsetroundjoin%
\definecolor{currentfill}{rgb}{0.978639,0.841584,0.673679}%
\pgfsetfillcolor{currentfill}%
\pgfsetlinewidth{0.250937pt}%
\definecolor{currentstroke}{rgb}{1.000000,1.000000,1.000000}%
\pgfsetstrokecolor{currentstroke}%
\pgfsetdash{}{0pt}%
\pgfpathmoveto{\pgfqpoint{2.135660in}{2.786604in}}%
\pgfpathlineto{\pgfqpoint{2.223396in}{2.786604in}}%
\pgfpathlineto{\pgfqpoint{2.223396in}{2.698868in}}%
\pgfpathlineto{\pgfqpoint{2.135660in}{2.698868in}}%
\pgfpathlineto{\pgfqpoint{2.135660in}{2.786604in}}%
\pgfusepath{stroke,fill}%
\end{pgfscope}%
\begin{pgfscope}%
\pgfpathrectangle{\pgfqpoint{0.380943in}{2.260189in}}{\pgfqpoint{4.650000in}{0.614151in}}%
\pgfusepath{clip}%
\pgfsetbuttcap%
\pgfsetroundjoin%
\definecolor{currentfill}{rgb}{0.963260,0.918478,0.719508}%
\pgfsetfillcolor{currentfill}%
\pgfsetlinewidth{0.250937pt}%
\definecolor{currentstroke}{rgb}{1.000000,1.000000,1.000000}%
\pgfsetstrokecolor{currentstroke}%
\pgfsetdash{}{0pt}%
\pgfpathmoveto{\pgfqpoint{2.223396in}{2.786604in}}%
\pgfpathlineto{\pgfqpoint{2.311132in}{2.786604in}}%
\pgfpathlineto{\pgfqpoint{2.311132in}{2.698868in}}%
\pgfpathlineto{\pgfqpoint{2.223396in}{2.698868in}}%
\pgfpathlineto{\pgfqpoint{2.223396in}{2.786604in}}%
\pgfusepath{stroke,fill}%
\end{pgfscope}%
\begin{pgfscope}%
\pgfpathrectangle{\pgfqpoint{0.380943in}{2.260189in}}{\pgfqpoint{4.650000in}{0.614151in}}%
\pgfusepath{clip}%
\pgfsetbuttcap%
\pgfsetroundjoin%
\definecolor{currentfill}{rgb}{0.974072,0.862976,0.688750}%
\pgfsetfillcolor{currentfill}%
\pgfsetlinewidth{0.250937pt}%
\definecolor{currentstroke}{rgb}{1.000000,1.000000,1.000000}%
\pgfsetstrokecolor{currentstroke}%
\pgfsetdash{}{0pt}%
\pgfpathmoveto{\pgfqpoint{2.311132in}{2.786604in}}%
\pgfpathlineto{\pgfqpoint{2.398868in}{2.786604in}}%
\pgfpathlineto{\pgfqpoint{2.398868in}{2.698868in}}%
\pgfpathlineto{\pgfqpoint{2.311132in}{2.698868in}}%
\pgfpathlineto{\pgfqpoint{2.311132in}{2.786604in}}%
\pgfusepath{stroke,fill}%
\end{pgfscope}%
\begin{pgfscope}%
\pgfpathrectangle{\pgfqpoint{0.380943in}{2.260189in}}{\pgfqpoint{4.650000in}{0.614151in}}%
\pgfusepath{clip}%
\pgfsetbuttcap%
\pgfsetroundjoin%
\definecolor{currentfill}{rgb}{0.969504,0.885813,0.700930}%
\pgfsetfillcolor{currentfill}%
\pgfsetlinewidth{0.250937pt}%
\definecolor{currentstroke}{rgb}{1.000000,1.000000,1.000000}%
\pgfsetstrokecolor{currentstroke}%
\pgfsetdash{}{0pt}%
\pgfpathmoveto{\pgfqpoint{2.398868in}{2.786604in}}%
\pgfpathlineto{\pgfqpoint{2.486603in}{2.786604in}}%
\pgfpathlineto{\pgfqpoint{2.486603in}{2.698868in}}%
\pgfpathlineto{\pgfqpoint{2.398868in}{2.698868in}}%
\pgfpathlineto{\pgfqpoint{2.398868in}{2.786604in}}%
\pgfusepath{stroke,fill}%
\end{pgfscope}%
\begin{pgfscope}%
\pgfpathrectangle{\pgfqpoint{0.380943in}{2.260189in}}{\pgfqpoint{4.650000in}{0.614151in}}%
\pgfusepath{clip}%
\pgfsetbuttcap%
\pgfsetroundjoin%
\definecolor{currentfill}{rgb}{0.990004,0.468435,0.468435}%
\pgfsetfillcolor{currentfill}%
\pgfsetlinewidth{0.250937pt}%
\definecolor{currentstroke}{rgb}{1.000000,1.000000,1.000000}%
\pgfsetstrokecolor{currentstroke}%
\pgfsetdash{}{0pt}%
\pgfpathmoveto{\pgfqpoint{2.486603in}{2.786604in}}%
\pgfpathlineto{\pgfqpoint{2.574339in}{2.786604in}}%
\pgfpathlineto{\pgfqpoint{2.574339in}{2.698868in}}%
\pgfpathlineto{\pgfqpoint{2.486603in}{2.698868in}}%
\pgfpathlineto{\pgfqpoint{2.486603in}{2.786604in}}%
\pgfusepath{stroke,fill}%
\end{pgfscope}%
\begin{pgfscope}%
\pgfpathrectangle{\pgfqpoint{0.380943in}{2.260189in}}{\pgfqpoint{4.650000in}{0.614151in}}%
\pgfusepath{clip}%
\pgfsetbuttcap%
\pgfsetroundjoin%
\definecolor{currentfill}{rgb}{1.000000,0.615379,0.534779}%
\pgfsetfillcolor{currentfill}%
\pgfsetlinewidth{0.250937pt}%
\definecolor{currentstroke}{rgb}{1.000000,1.000000,1.000000}%
\pgfsetstrokecolor{currentstroke}%
\pgfsetdash{}{0pt}%
\pgfpathmoveto{\pgfqpoint{2.574339in}{2.786604in}}%
\pgfpathlineto{\pgfqpoint{2.662075in}{2.786604in}}%
\pgfpathlineto{\pgfqpoint{2.662075in}{2.698868in}}%
\pgfpathlineto{\pgfqpoint{2.574339in}{2.698868in}}%
\pgfpathlineto{\pgfqpoint{2.574339in}{2.786604in}}%
\pgfusepath{stroke,fill}%
\end{pgfscope}%
\begin{pgfscope}%
\pgfpathrectangle{\pgfqpoint{0.380943in}{2.260189in}}{\pgfqpoint{4.650000in}{0.614151in}}%
\pgfusepath{clip}%
\pgfsetbuttcap%
\pgfsetroundjoin%
\definecolor{currentfill}{rgb}{1.000000,0.584929,0.522599}%
\pgfsetfillcolor{currentfill}%
\pgfsetlinewidth{0.250937pt}%
\definecolor{currentstroke}{rgb}{1.000000,1.000000,1.000000}%
\pgfsetstrokecolor{currentstroke}%
\pgfsetdash{}{0pt}%
\pgfpathmoveto{\pgfqpoint{2.662075in}{2.786604in}}%
\pgfpathlineto{\pgfqpoint{2.749811in}{2.786604in}}%
\pgfpathlineto{\pgfqpoint{2.749811in}{2.698868in}}%
\pgfpathlineto{\pgfqpoint{2.662075in}{2.698868in}}%
\pgfpathlineto{\pgfqpoint{2.662075in}{2.786604in}}%
\pgfusepath{stroke,fill}%
\end{pgfscope}%
\begin{pgfscope}%
\pgfpathrectangle{\pgfqpoint{0.380943in}{2.260189in}}{\pgfqpoint{4.650000in}{0.614151in}}%
\pgfusepath{clip}%
\pgfsetbuttcap%
\pgfsetroundjoin%
\definecolor{currentfill}{rgb}{0.987266,0.804198,0.639170}%
\pgfsetfillcolor{currentfill}%
\pgfsetlinewidth{0.250937pt}%
\definecolor{currentstroke}{rgb}{1.000000,1.000000,1.000000}%
\pgfsetstrokecolor{currentstroke}%
\pgfsetdash{}{0pt}%
\pgfpathmoveto{\pgfqpoint{2.749811in}{2.786604in}}%
\pgfpathlineto{\pgfqpoint{2.837547in}{2.786604in}}%
\pgfpathlineto{\pgfqpoint{2.837547in}{2.698868in}}%
\pgfpathlineto{\pgfqpoint{2.749811in}{2.698868in}}%
\pgfpathlineto{\pgfqpoint{2.749811in}{2.786604in}}%
\pgfusepath{stroke,fill}%
\end{pgfscope}%
\begin{pgfscope}%
\pgfpathrectangle{\pgfqpoint{0.380943in}{2.260189in}}{\pgfqpoint{4.650000in}{0.614151in}}%
\pgfusepath{clip}%
\pgfsetbuttcap%
\pgfsetroundjoin%
\definecolor{currentfill}{rgb}{0.987266,0.804198,0.639170}%
\pgfsetfillcolor{currentfill}%
\pgfsetlinewidth{0.250937pt}%
\definecolor{currentstroke}{rgb}{1.000000,1.000000,1.000000}%
\pgfsetstrokecolor{currentstroke}%
\pgfsetdash{}{0pt}%
\pgfpathmoveto{\pgfqpoint{2.837547in}{2.786604in}}%
\pgfpathlineto{\pgfqpoint{2.925283in}{2.786604in}}%
\pgfpathlineto{\pgfqpoint{2.925283in}{2.698868in}}%
\pgfpathlineto{\pgfqpoint{2.837547in}{2.698868in}}%
\pgfpathlineto{\pgfqpoint{2.837547in}{2.786604in}}%
\pgfusepath{stroke,fill}%
\end{pgfscope}%
\begin{pgfscope}%
\pgfpathrectangle{\pgfqpoint{0.380943in}{2.260189in}}{\pgfqpoint{4.650000in}{0.614151in}}%
\pgfusepath{clip}%
\pgfsetbuttcap%
\pgfsetroundjoin%
\definecolor{currentfill}{rgb}{0.996401,0.724937,0.591557}%
\pgfsetfillcolor{currentfill}%
\pgfsetlinewidth{0.250937pt}%
\definecolor{currentstroke}{rgb}{1.000000,1.000000,1.000000}%
\pgfsetstrokecolor{currentstroke}%
\pgfsetdash{}{0pt}%
\pgfpathmoveto{\pgfqpoint{2.925283in}{2.786604in}}%
\pgfpathlineto{\pgfqpoint{3.013019in}{2.786604in}}%
\pgfpathlineto{\pgfqpoint{3.013019in}{2.698868in}}%
\pgfpathlineto{\pgfqpoint{2.925283in}{2.698868in}}%
\pgfpathlineto{\pgfqpoint{2.925283in}{2.786604in}}%
\pgfusepath{stroke,fill}%
\end{pgfscope}%
\begin{pgfscope}%
\pgfpathrectangle{\pgfqpoint{0.380943in}{2.260189in}}{\pgfqpoint{4.650000in}{0.614151in}}%
\pgfusepath{clip}%
\pgfsetbuttcap%
\pgfsetroundjoin%
\definecolor{currentfill}{rgb}{0.993679,0.753725,0.608074}%
\pgfsetfillcolor{currentfill}%
\pgfsetlinewidth{0.250937pt}%
\definecolor{currentstroke}{rgb}{1.000000,1.000000,1.000000}%
\pgfsetstrokecolor{currentstroke}%
\pgfsetdash{}{0pt}%
\pgfpathmoveto{\pgfqpoint{3.013019in}{2.786604in}}%
\pgfpathlineto{\pgfqpoint{3.100754in}{2.786604in}}%
\pgfpathlineto{\pgfqpoint{3.100754in}{2.698868in}}%
\pgfpathlineto{\pgfqpoint{3.013019in}{2.698868in}}%
\pgfpathlineto{\pgfqpoint{3.013019in}{2.786604in}}%
\pgfusepath{stroke,fill}%
\end{pgfscope}%
\begin{pgfscope}%
\pgfpathrectangle{\pgfqpoint{0.380943in}{2.260189in}}{\pgfqpoint{4.650000in}{0.614151in}}%
\pgfusepath{clip}%
\pgfsetbuttcap%
\pgfsetroundjoin%
\definecolor{currentfill}{rgb}{0.961738,0.927612,0.725598}%
\pgfsetfillcolor{currentfill}%
\pgfsetlinewidth{0.250937pt}%
\definecolor{currentstroke}{rgb}{1.000000,1.000000,1.000000}%
\pgfsetstrokecolor{currentstroke}%
\pgfsetdash{}{0pt}%
\pgfpathmoveto{\pgfqpoint{3.100754in}{2.786604in}}%
\pgfpathlineto{\pgfqpoint{3.188490in}{2.786604in}}%
\pgfpathlineto{\pgfqpoint{3.188490in}{2.698868in}}%
\pgfpathlineto{\pgfqpoint{3.100754in}{2.698868in}}%
\pgfpathlineto{\pgfqpoint{3.100754in}{2.786604in}}%
\pgfusepath{stroke,fill}%
\end{pgfscope}%
\begin{pgfscope}%
\pgfpathrectangle{\pgfqpoint{0.380943in}{2.260189in}}{\pgfqpoint{4.650000in}{0.614151in}}%
\pgfusepath{clip}%
\pgfsetbuttcap%
\pgfsetroundjoin%
\definecolor{currentfill}{rgb}{0.987266,0.804198,0.639170}%
\pgfsetfillcolor{currentfill}%
\pgfsetlinewidth{0.250937pt}%
\definecolor{currentstroke}{rgb}{1.000000,1.000000,1.000000}%
\pgfsetstrokecolor{currentstroke}%
\pgfsetdash{}{0pt}%
\pgfpathmoveto{\pgfqpoint{3.188490in}{2.786604in}}%
\pgfpathlineto{\pgfqpoint{3.276226in}{2.786604in}}%
\pgfpathlineto{\pgfqpoint{3.276226in}{2.698868in}}%
\pgfpathlineto{\pgfqpoint{3.188490in}{2.698868in}}%
\pgfpathlineto{\pgfqpoint{3.188490in}{2.786604in}}%
\pgfusepath{stroke,fill}%
\end{pgfscope}%
\begin{pgfscope}%
\pgfpathrectangle{\pgfqpoint{0.380943in}{2.260189in}}{\pgfqpoint{4.650000in}{0.614151in}}%
\pgfusepath{clip}%
\pgfsetbuttcap%
\pgfsetroundjoin%
\definecolor{currentfill}{rgb}{0.969504,0.885813,0.700930}%
\pgfsetfillcolor{currentfill}%
\pgfsetlinewidth{0.250937pt}%
\definecolor{currentstroke}{rgb}{1.000000,1.000000,1.000000}%
\pgfsetstrokecolor{currentstroke}%
\pgfsetdash{}{0pt}%
\pgfpathmoveto{\pgfqpoint{3.276226in}{2.786604in}}%
\pgfpathlineto{\pgfqpoint{3.363962in}{2.786604in}}%
\pgfpathlineto{\pgfqpoint{3.363962in}{2.698868in}}%
\pgfpathlineto{\pgfqpoint{3.276226in}{2.698868in}}%
\pgfpathlineto{\pgfqpoint{3.276226in}{2.786604in}}%
\pgfusepath{stroke,fill}%
\end{pgfscope}%
\begin{pgfscope}%
\pgfpathrectangle{\pgfqpoint{0.380943in}{2.260189in}}{\pgfqpoint{4.650000in}{0.614151in}}%
\pgfusepath{clip}%
\pgfsetbuttcap%
\pgfsetroundjoin%
\definecolor{currentfill}{rgb}{0.999277,0.650165,0.551296}%
\pgfsetfillcolor{currentfill}%
\pgfsetlinewidth{0.250937pt}%
\definecolor{currentstroke}{rgb}{1.000000,1.000000,1.000000}%
\pgfsetstrokecolor{currentstroke}%
\pgfsetdash{}{0pt}%
\pgfpathmoveto{\pgfqpoint{3.363962in}{2.786604in}}%
\pgfpathlineto{\pgfqpoint{3.451698in}{2.786604in}}%
\pgfpathlineto{\pgfqpoint{3.451698in}{2.698868in}}%
\pgfpathlineto{\pgfqpoint{3.363962in}{2.698868in}}%
\pgfpathlineto{\pgfqpoint{3.363962in}{2.786604in}}%
\pgfusepath{stroke,fill}%
\end{pgfscope}%
\begin{pgfscope}%
\pgfpathrectangle{\pgfqpoint{0.380943in}{2.260189in}}{\pgfqpoint{4.650000in}{0.614151in}}%
\pgfusepath{clip}%
\pgfsetbuttcap%
\pgfsetroundjoin%
\definecolor{currentfill}{rgb}{0.974072,0.862976,0.688750}%
\pgfsetfillcolor{currentfill}%
\pgfsetlinewidth{0.250937pt}%
\definecolor{currentstroke}{rgb}{1.000000,1.000000,1.000000}%
\pgfsetstrokecolor{currentstroke}%
\pgfsetdash{}{0pt}%
\pgfpathmoveto{\pgfqpoint{3.451698in}{2.786604in}}%
\pgfpathlineto{\pgfqpoint{3.539434in}{2.786604in}}%
\pgfpathlineto{\pgfqpoint{3.539434in}{2.698868in}}%
\pgfpathlineto{\pgfqpoint{3.451698in}{2.698868in}}%
\pgfpathlineto{\pgfqpoint{3.451698in}{2.786604in}}%
\pgfusepath{stroke,fill}%
\end{pgfscope}%
\begin{pgfscope}%
\pgfpathrectangle{\pgfqpoint{0.380943in}{2.260189in}}{\pgfqpoint{4.650000in}{0.614151in}}%
\pgfusepath{clip}%
\pgfsetbuttcap%
\pgfsetroundjoin%
\definecolor{currentfill}{rgb}{0.982699,0.823991,0.657439}%
\pgfsetfillcolor{currentfill}%
\pgfsetlinewidth{0.250937pt}%
\definecolor{currentstroke}{rgb}{1.000000,1.000000,1.000000}%
\pgfsetstrokecolor{currentstroke}%
\pgfsetdash{}{0pt}%
\pgfpathmoveto{\pgfqpoint{3.539434in}{2.786604in}}%
\pgfpathlineto{\pgfqpoint{3.627169in}{2.786604in}}%
\pgfpathlineto{\pgfqpoint{3.627169in}{2.698868in}}%
\pgfpathlineto{\pgfqpoint{3.539434in}{2.698868in}}%
\pgfpathlineto{\pgfqpoint{3.539434in}{2.786604in}}%
\pgfusepath{stroke,fill}%
\end{pgfscope}%
\begin{pgfscope}%
\pgfpathrectangle{\pgfqpoint{0.380943in}{2.260189in}}{\pgfqpoint{4.650000in}{0.614151in}}%
\pgfusepath{clip}%
\pgfsetbuttcap%
\pgfsetroundjoin%
\definecolor{currentfill}{rgb}{0.990634,0.779608,0.623299}%
\pgfsetfillcolor{currentfill}%
\pgfsetlinewidth{0.250937pt}%
\definecolor{currentstroke}{rgb}{1.000000,1.000000,1.000000}%
\pgfsetstrokecolor{currentstroke}%
\pgfsetdash{}{0pt}%
\pgfpathmoveto{\pgfqpoint{3.627169in}{2.786604in}}%
\pgfpathlineto{\pgfqpoint{3.714905in}{2.786604in}}%
\pgfpathlineto{\pgfqpoint{3.714905in}{2.698868in}}%
\pgfpathlineto{\pgfqpoint{3.627169in}{2.698868in}}%
\pgfpathlineto{\pgfqpoint{3.627169in}{2.786604in}}%
\pgfusepath{stroke,fill}%
\end{pgfscope}%
\begin{pgfscope}%
\pgfpathrectangle{\pgfqpoint{0.380943in}{2.260189in}}{\pgfqpoint{4.650000in}{0.614151in}}%
\pgfusepath{clip}%
\pgfsetbuttcap%
\pgfsetroundjoin%
\definecolor{currentfill}{rgb}{0.978639,0.841584,0.673679}%
\pgfsetfillcolor{currentfill}%
\pgfsetlinewidth{0.250937pt}%
\definecolor{currentstroke}{rgb}{1.000000,1.000000,1.000000}%
\pgfsetstrokecolor{currentstroke}%
\pgfsetdash{}{0pt}%
\pgfpathmoveto{\pgfqpoint{3.714905in}{2.786604in}}%
\pgfpathlineto{\pgfqpoint{3.802641in}{2.786604in}}%
\pgfpathlineto{\pgfqpoint{3.802641in}{2.698868in}}%
\pgfpathlineto{\pgfqpoint{3.714905in}{2.698868in}}%
\pgfpathlineto{\pgfqpoint{3.714905in}{2.786604in}}%
\pgfusepath{stroke,fill}%
\end{pgfscope}%
\begin{pgfscope}%
\pgfpathrectangle{\pgfqpoint{0.380943in}{2.260189in}}{\pgfqpoint{4.650000in}{0.614151in}}%
\pgfusepath{clip}%
\pgfsetbuttcap%
\pgfsetroundjoin%
\definecolor{currentfill}{rgb}{1.000000,0.615379,0.534779}%
\pgfsetfillcolor{currentfill}%
\pgfsetlinewidth{0.250937pt}%
\definecolor{currentstroke}{rgb}{1.000000,1.000000,1.000000}%
\pgfsetstrokecolor{currentstroke}%
\pgfsetdash{}{0pt}%
\pgfpathmoveto{\pgfqpoint{3.802641in}{2.786604in}}%
\pgfpathlineto{\pgfqpoint{3.890377in}{2.786604in}}%
\pgfpathlineto{\pgfqpoint{3.890377in}{2.698868in}}%
\pgfpathlineto{\pgfqpoint{3.802641in}{2.698868in}}%
\pgfpathlineto{\pgfqpoint{3.802641in}{2.786604in}}%
\pgfusepath{stroke,fill}%
\end{pgfscope}%
\begin{pgfscope}%
\pgfpathrectangle{\pgfqpoint{0.380943in}{2.260189in}}{\pgfqpoint{4.650000in}{0.614151in}}%
\pgfusepath{clip}%
\pgfsetbuttcap%
\pgfsetroundjoin%
\definecolor{currentfill}{rgb}{0.987266,0.804198,0.639170}%
\pgfsetfillcolor{currentfill}%
\pgfsetlinewidth{0.250937pt}%
\definecolor{currentstroke}{rgb}{1.000000,1.000000,1.000000}%
\pgfsetstrokecolor{currentstroke}%
\pgfsetdash{}{0pt}%
\pgfpathmoveto{\pgfqpoint{3.890377in}{2.786604in}}%
\pgfpathlineto{\pgfqpoint{3.978113in}{2.786604in}}%
\pgfpathlineto{\pgfqpoint{3.978113in}{2.698868in}}%
\pgfpathlineto{\pgfqpoint{3.890377in}{2.698868in}}%
\pgfpathlineto{\pgfqpoint{3.890377in}{2.786604in}}%
\pgfusepath{stroke,fill}%
\end{pgfscope}%
\begin{pgfscope}%
\pgfpathrectangle{\pgfqpoint{0.380943in}{2.260189in}}{\pgfqpoint{4.650000in}{0.614151in}}%
\pgfusepath{clip}%
\pgfsetbuttcap%
\pgfsetroundjoin%
\definecolor{currentfill}{rgb}{0.990634,0.779608,0.623299}%
\pgfsetfillcolor{currentfill}%
\pgfsetlinewidth{0.250937pt}%
\definecolor{currentstroke}{rgb}{1.000000,1.000000,1.000000}%
\pgfsetstrokecolor{currentstroke}%
\pgfsetdash{}{0pt}%
\pgfpathmoveto{\pgfqpoint{3.978113in}{2.786604in}}%
\pgfpathlineto{\pgfqpoint{4.065849in}{2.786604in}}%
\pgfpathlineto{\pgfqpoint{4.065849in}{2.698868in}}%
\pgfpathlineto{\pgfqpoint{3.978113in}{2.698868in}}%
\pgfpathlineto{\pgfqpoint{3.978113in}{2.786604in}}%
\pgfusepath{stroke,fill}%
\end{pgfscope}%
\begin{pgfscope}%
\pgfpathrectangle{\pgfqpoint{0.380943in}{2.260189in}}{\pgfqpoint{4.650000in}{0.614151in}}%
\pgfusepath{clip}%
\pgfsetbuttcap%
\pgfsetroundjoin%
\definecolor{currentfill}{rgb}{0.997924,0.685352,0.570242}%
\pgfsetfillcolor{currentfill}%
\pgfsetlinewidth{0.250937pt}%
\definecolor{currentstroke}{rgb}{1.000000,1.000000,1.000000}%
\pgfsetstrokecolor{currentstroke}%
\pgfsetdash{}{0pt}%
\pgfpathmoveto{\pgfqpoint{4.065849in}{2.786604in}}%
\pgfpathlineto{\pgfqpoint{4.153585in}{2.786604in}}%
\pgfpathlineto{\pgfqpoint{4.153585in}{2.698868in}}%
\pgfpathlineto{\pgfqpoint{4.065849in}{2.698868in}}%
\pgfpathlineto{\pgfqpoint{4.065849in}{2.786604in}}%
\pgfusepath{stroke,fill}%
\end{pgfscope}%
\begin{pgfscope}%
\pgfpathrectangle{\pgfqpoint{0.380943in}{2.260189in}}{\pgfqpoint{4.650000in}{0.614151in}}%
\pgfusepath{clip}%
\pgfsetbuttcap%
\pgfsetroundjoin%
\definecolor{currentfill}{rgb}{0.999277,0.650165,0.551296}%
\pgfsetfillcolor{currentfill}%
\pgfsetlinewidth{0.250937pt}%
\definecolor{currentstroke}{rgb}{1.000000,1.000000,1.000000}%
\pgfsetstrokecolor{currentstroke}%
\pgfsetdash{}{0pt}%
\pgfpathmoveto{\pgfqpoint{4.153585in}{2.786604in}}%
\pgfpathlineto{\pgfqpoint{4.241320in}{2.786604in}}%
\pgfpathlineto{\pgfqpoint{4.241320in}{2.698868in}}%
\pgfpathlineto{\pgfqpoint{4.153585in}{2.698868in}}%
\pgfpathlineto{\pgfqpoint{4.153585in}{2.786604in}}%
\pgfusepath{stroke,fill}%
\end{pgfscope}%
\begin{pgfscope}%
\pgfpathrectangle{\pgfqpoint{0.380943in}{2.260189in}}{\pgfqpoint{4.650000in}{0.614151in}}%
\pgfusepath{clip}%
\pgfsetbuttcap%
\pgfsetroundjoin%
\definecolor{currentfill}{rgb}{0.997924,0.685352,0.570242}%
\pgfsetfillcolor{currentfill}%
\pgfsetlinewidth{0.250937pt}%
\definecolor{currentstroke}{rgb}{1.000000,1.000000,1.000000}%
\pgfsetstrokecolor{currentstroke}%
\pgfsetdash{}{0pt}%
\pgfpathmoveto{\pgfqpoint{4.241320in}{2.786604in}}%
\pgfpathlineto{\pgfqpoint{4.329056in}{2.786604in}}%
\pgfpathlineto{\pgfqpoint{4.329056in}{2.698868in}}%
\pgfpathlineto{\pgfqpoint{4.241320in}{2.698868in}}%
\pgfpathlineto{\pgfqpoint{4.241320in}{2.786604in}}%
\pgfusepath{stroke,fill}%
\end{pgfscope}%
\begin{pgfscope}%
\pgfpathrectangle{\pgfqpoint{0.380943in}{2.260189in}}{\pgfqpoint{4.650000in}{0.614151in}}%
\pgfusepath{clip}%
\pgfsetbuttcap%
\pgfsetroundjoin%
\definecolor{currentfill}{rgb}{0.990634,0.779608,0.623299}%
\pgfsetfillcolor{currentfill}%
\pgfsetlinewidth{0.250937pt}%
\definecolor{currentstroke}{rgb}{1.000000,1.000000,1.000000}%
\pgfsetstrokecolor{currentstroke}%
\pgfsetdash{}{0pt}%
\pgfpathmoveto{\pgfqpoint{4.329056in}{2.786604in}}%
\pgfpathlineto{\pgfqpoint{4.416792in}{2.786604in}}%
\pgfpathlineto{\pgfqpoint{4.416792in}{2.698868in}}%
\pgfpathlineto{\pgfqpoint{4.329056in}{2.698868in}}%
\pgfpathlineto{\pgfqpoint{4.329056in}{2.786604in}}%
\pgfusepath{stroke,fill}%
\end{pgfscope}%
\begin{pgfscope}%
\pgfpathrectangle{\pgfqpoint{0.380943in}{2.260189in}}{\pgfqpoint{4.650000in}{0.614151in}}%
\pgfusepath{clip}%
\pgfsetbuttcap%
\pgfsetroundjoin%
\definecolor{currentfill}{rgb}{0.982699,0.823991,0.657439}%
\pgfsetfillcolor{currentfill}%
\pgfsetlinewidth{0.250937pt}%
\definecolor{currentstroke}{rgb}{1.000000,1.000000,1.000000}%
\pgfsetstrokecolor{currentstroke}%
\pgfsetdash{}{0pt}%
\pgfpathmoveto{\pgfqpoint{4.416792in}{2.786604in}}%
\pgfpathlineto{\pgfqpoint{4.504528in}{2.786604in}}%
\pgfpathlineto{\pgfqpoint{4.504528in}{2.698868in}}%
\pgfpathlineto{\pgfqpoint{4.416792in}{2.698868in}}%
\pgfpathlineto{\pgfqpoint{4.416792in}{2.786604in}}%
\pgfusepath{stroke,fill}%
\end{pgfscope}%
\begin{pgfscope}%
\pgfpathrectangle{\pgfqpoint{0.380943in}{2.260189in}}{\pgfqpoint{4.650000in}{0.614151in}}%
\pgfusepath{clip}%
\pgfsetbuttcap%
\pgfsetroundjoin%
\definecolor{currentfill}{rgb}{0.974072,0.862976,0.688750}%
\pgfsetfillcolor{currentfill}%
\pgfsetlinewidth{0.250937pt}%
\definecolor{currentstroke}{rgb}{1.000000,1.000000,1.000000}%
\pgfsetstrokecolor{currentstroke}%
\pgfsetdash{}{0pt}%
\pgfpathmoveto{\pgfqpoint{4.504528in}{2.786604in}}%
\pgfpathlineto{\pgfqpoint{4.592264in}{2.786604in}}%
\pgfpathlineto{\pgfqpoint{4.592264in}{2.698868in}}%
\pgfpathlineto{\pgfqpoint{4.504528in}{2.698868in}}%
\pgfpathlineto{\pgfqpoint{4.504528in}{2.786604in}}%
\pgfusepath{stroke,fill}%
\end{pgfscope}%
\begin{pgfscope}%
\pgfpathrectangle{\pgfqpoint{0.380943in}{2.260189in}}{\pgfqpoint{4.650000in}{0.614151in}}%
\pgfusepath{clip}%
\pgfsetbuttcap%
\pgfsetroundjoin%
\definecolor{currentfill}{rgb}{0.963260,0.918478,0.719508}%
\pgfsetfillcolor{currentfill}%
\pgfsetlinewidth{0.250937pt}%
\definecolor{currentstroke}{rgb}{1.000000,1.000000,1.000000}%
\pgfsetstrokecolor{currentstroke}%
\pgfsetdash{}{0pt}%
\pgfpathmoveto{\pgfqpoint{4.592264in}{2.786604in}}%
\pgfpathlineto{\pgfqpoint{4.680000in}{2.786604in}}%
\pgfpathlineto{\pgfqpoint{4.680000in}{2.698868in}}%
\pgfpathlineto{\pgfqpoint{4.592264in}{2.698868in}}%
\pgfpathlineto{\pgfqpoint{4.592264in}{2.786604in}}%
\pgfusepath{stroke,fill}%
\end{pgfscope}%
\begin{pgfscope}%
\pgfpathrectangle{\pgfqpoint{0.380943in}{2.260189in}}{\pgfqpoint{4.650000in}{0.614151in}}%
\pgfusepath{clip}%
\pgfsetbuttcap%
\pgfsetroundjoin%
\definecolor{currentfill}{rgb}{0.997924,0.685352,0.570242}%
\pgfsetfillcolor{currentfill}%
\pgfsetlinewidth{0.250937pt}%
\definecolor{currentstroke}{rgb}{1.000000,1.000000,1.000000}%
\pgfsetstrokecolor{currentstroke}%
\pgfsetdash{}{0pt}%
\pgfpathmoveto{\pgfqpoint{4.680000in}{2.786604in}}%
\pgfpathlineto{\pgfqpoint{4.767736in}{2.786604in}}%
\pgfpathlineto{\pgfqpoint{4.767736in}{2.698868in}}%
\pgfpathlineto{\pgfqpoint{4.680000in}{2.698868in}}%
\pgfpathlineto{\pgfqpoint{4.680000in}{2.786604in}}%
\pgfusepath{stroke,fill}%
\end{pgfscope}%
\begin{pgfscope}%
\pgfpathrectangle{\pgfqpoint{0.380943in}{2.260189in}}{\pgfqpoint{4.650000in}{0.614151in}}%
\pgfusepath{clip}%
\pgfsetbuttcap%
\pgfsetroundjoin%
\definecolor{currentfill}{rgb}{0.978639,0.841584,0.673679}%
\pgfsetfillcolor{currentfill}%
\pgfsetlinewidth{0.250937pt}%
\definecolor{currentstroke}{rgb}{1.000000,1.000000,1.000000}%
\pgfsetstrokecolor{currentstroke}%
\pgfsetdash{}{0pt}%
\pgfpathmoveto{\pgfqpoint{4.767736in}{2.786604in}}%
\pgfpathlineto{\pgfqpoint{4.855471in}{2.786604in}}%
\pgfpathlineto{\pgfqpoint{4.855471in}{2.698868in}}%
\pgfpathlineto{\pgfqpoint{4.767736in}{2.698868in}}%
\pgfpathlineto{\pgfqpoint{4.767736in}{2.786604in}}%
\pgfusepath{stroke,fill}%
\end{pgfscope}%
\begin{pgfscope}%
\pgfpathrectangle{\pgfqpoint{0.380943in}{2.260189in}}{\pgfqpoint{4.650000in}{0.614151in}}%
\pgfusepath{clip}%
\pgfsetbuttcap%
\pgfsetroundjoin%
\definecolor{currentfill}{rgb}{0.974072,0.862976,0.688750}%
\pgfsetfillcolor{currentfill}%
\pgfsetlinewidth{0.250937pt}%
\definecolor{currentstroke}{rgb}{1.000000,1.000000,1.000000}%
\pgfsetstrokecolor{currentstroke}%
\pgfsetdash{}{0pt}%
\pgfpathmoveto{\pgfqpoint{4.855471in}{2.786604in}}%
\pgfpathlineto{\pgfqpoint{4.943207in}{2.786604in}}%
\pgfpathlineto{\pgfqpoint{4.943207in}{2.698868in}}%
\pgfpathlineto{\pgfqpoint{4.855471in}{2.698868in}}%
\pgfpathlineto{\pgfqpoint{4.855471in}{2.786604in}}%
\pgfusepath{stroke,fill}%
\end{pgfscope}%
\begin{pgfscope}%
\pgfpathrectangle{\pgfqpoint{0.380943in}{2.260189in}}{\pgfqpoint{4.650000in}{0.614151in}}%
\pgfusepath{clip}%
\pgfsetbuttcap%
\pgfsetroundjoin%
\definecolor{currentfill}{rgb}{0.990634,0.779608,0.623299}%
\pgfsetfillcolor{currentfill}%
\pgfsetlinewidth{0.250937pt}%
\definecolor{currentstroke}{rgb}{1.000000,1.000000,1.000000}%
\pgfsetstrokecolor{currentstroke}%
\pgfsetdash{}{0pt}%
\pgfpathmoveto{\pgfqpoint{4.943207in}{2.786604in}}%
\pgfpathlineto{\pgfqpoint{5.030943in}{2.786604in}}%
\pgfpathlineto{\pgfqpoint{5.030943in}{2.698868in}}%
\pgfpathlineto{\pgfqpoint{4.943207in}{2.698868in}}%
\pgfpathlineto{\pgfqpoint{4.943207in}{2.786604in}}%
\pgfusepath{stroke,fill}%
\end{pgfscope}%
\begin{pgfscope}%
\pgfpathrectangle{\pgfqpoint{0.380943in}{2.260189in}}{\pgfqpoint{4.650000in}{0.614151in}}%
\pgfusepath{clip}%
\pgfsetbuttcap%
\pgfsetroundjoin%
\definecolor{currentfill}{rgb}{0.995233,0.991895,0.818977}%
\pgfsetfillcolor{currentfill}%
\pgfsetlinewidth{0.250937pt}%
\definecolor{currentstroke}{rgb}{1.000000,1.000000,1.000000}%
\pgfsetstrokecolor{currentstroke}%
\pgfsetdash{}{0pt}%
\pgfpathmoveto{\pgfqpoint{0.380943in}{2.698868in}}%
\pgfpathlineto{\pgfqpoint{0.468679in}{2.698868in}}%
\pgfpathlineto{\pgfqpoint{0.468679in}{2.611132in}}%
\pgfpathlineto{\pgfqpoint{0.380943in}{2.611132in}}%
\pgfpathlineto{\pgfqpoint{0.380943in}{2.698868in}}%
\pgfusepath{stroke,fill}%
\end{pgfscope}%
\begin{pgfscope}%
\pgfpathrectangle{\pgfqpoint{0.380943in}{2.260189in}}{\pgfqpoint{4.650000in}{0.614151in}}%
\pgfusepath{clip}%
\pgfsetbuttcap%
\pgfsetroundjoin%
\definecolor{currentfill}{rgb}{0.990634,0.779608,0.623299}%
\pgfsetfillcolor{currentfill}%
\pgfsetlinewidth{0.250937pt}%
\definecolor{currentstroke}{rgb}{1.000000,1.000000,1.000000}%
\pgfsetstrokecolor{currentstroke}%
\pgfsetdash{}{0pt}%
\pgfpathmoveto{\pgfqpoint{0.468679in}{2.698868in}}%
\pgfpathlineto{\pgfqpoint{0.556415in}{2.698868in}}%
\pgfpathlineto{\pgfqpoint{0.556415in}{2.611132in}}%
\pgfpathlineto{\pgfqpoint{0.468679in}{2.611132in}}%
\pgfpathlineto{\pgfqpoint{0.468679in}{2.698868in}}%
\pgfusepath{stroke,fill}%
\end{pgfscope}%
\begin{pgfscope}%
\pgfpathrectangle{\pgfqpoint{0.380943in}{2.260189in}}{\pgfqpoint{4.650000in}{0.614151in}}%
\pgfusepath{clip}%
\pgfsetbuttcap%
\pgfsetroundjoin%
\definecolor{currentfill}{rgb}{0.969504,0.885813,0.700930}%
\pgfsetfillcolor{currentfill}%
\pgfsetlinewidth{0.250937pt}%
\definecolor{currentstroke}{rgb}{1.000000,1.000000,1.000000}%
\pgfsetstrokecolor{currentstroke}%
\pgfsetdash{}{0pt}%
\pgfpathmoveto{\pgfqpoint{0.556415in}{2.698868in}}%
\pgfpathlineto{\pgfqpoint{0.644151in}{2.698868in}}%
\pgfpathlineto{\pgfqpoint{0.644151in}{2.611132in}}%
\pgfpathlineto{\pgfqpoint{0.556415in}{2.611132in}}%
\pgfpathlineto{\pgfqpoint{0.556415in}{2.698868in}}%
\pgfusepath{stroke,fill}%
\end{pgfscope}%
\begin{pgfscope}%
\pgfpathrectangle{\pgfqpoint{0.380943in}{2.260189in}}{\pgfqpoint{4.650000in}{0.614151in}}%
\pgfusepath{clip}%
\pgfsetbuttcap%
\pgfsetroundjoin%
\definecolor{currentfill}{rgb}{0.969504,0.885813,0.700930}%
\pgfsetfillcolor{currentfill}%
\pgfsetlinewidth{0.250937pt}%
\definecolor{currentstroke}{rgb}{1.000000,1.000000,1.000000}%
\pgfsetstrokecolor{currentstroke}%
\pgfsetdash{}{0pt}%
\pgfpathmoveto{\pgfqpoint{0.644151in}{2.698868in}}%
\pgfpathlineto{\pgfqpoint{0.731886in}{2.698868in}}%
\pgfpathlineto{\pgfqpoint{0.731886in}{2.611132in}}%
\pgfpathlineto{\pgfqpoint{0.644151in}{2.611132in}}%
\pgfpathlineto{\pgfqpoint{0.644151in}{2.698868in}}%
\pgfusepath{stroke,fill}%
\end{pgfscope}%
\begin{pgfscope}%
\pgfpathrectangle{\pgfqpoint{0.380943in}{2.260189in}}{\pgfqpoint{4.650000in}{0.614151in}}%
\pgfusepath{clip}%
\pgfsetbuttcap%
\pgfsetroundjoin%
\definecolor{currentfill}{rgb}{0.964783,0.940131,0.739808}%
\pgfsetfillcolor{currentfill}%
\pgfsetlinewidth{0.250937pt}%
\definecolor{currentstroke}{rgb}{1.000000,1.000000,1.000000}%
\pgfsetstrokecolor{currentstroke}%
\pgfsetdash{}{0pt}%
\pgfpathmoveto{\pgfqpoint{0.731886in}{2.698868in}}%
\pgfpathlineto{\pgfqpoint{0.819622in}{2.698868in}}%
\pgfpathlineto{\pgfqpoint{0.819622in}{2.611132in}}%
\pgfpathlineto{\pgfqpoint{0.731886in}{2.611132in}}%
\pgfpathlineto{\pgfqpoint{0.731886in}{2.698868in}}%
\pgfusepath{stroke,fill}%
\end{pgfscope}%
\begin{pgfscope}%
\pgfpathrectangle{\pgfqpoint{0.380943in}{2.260189in}}{\pgfqpoint{4.650000in}{0.614151in}}%
\pgfusepath{clip}%
\pgfsetbuttcap%
\pgfsetroundjoin%
\definecolor{currentfill}{rgb}{0.974072,0.862976,0.688750}%
\pgfsetfillcolor{currentfill}%
\pgfsetlinewidth{0.250937pt}%
\definecolor{currentstroke}{rgb}{1.000000,1.000000,1.000000}%
\pgfsetstrokecolor{currentstroke}%
\pgfsetdash{}{0pt}%
\pgfpathmoveto{\pgfqpoint{0.819622in}{2.698868in}}%
\pgfpathlineto{\pgfqpoint{0.907358in}{2.698868in}}%
\pgfpathlineto{\pgfqpoint{0.907358in}{2.611132in}}%
\pgfpathlineto{\pgfqpoint{0.819622in}{2.611132in}}%
\pgfpathlineto{\pgfqpoint{0.819622in}{2.698868in}}%
\pgfusepath{stroke,fill}%
\end{pgfscope}%
\begin{pgfscope}%
\pgfpathrectangle{\pgfqpoint{0.380943in}{2.260189in}}{\pgfqpoint{4.650000in}{0.614151in}}%
\pgfusepath{clip}%
\pgfsetbuttcap%
\pgfsetroundjoin%
\definecolor{currentfill}{rgb}{0.963260,0.918478,0.719508}%
\pgfsetfillcolor{currentfill}%
\pgfsetlinewidth{0.250937pt}%
\definecolor{currentstroke}{rgb}{1.000000,1.000000,1.000000}%
\pgfsetstrokecolor{currentstroke}%
\pgfsetdash{}{0pt}%
\pgfpathmoveto{\pgfqpoint{0.907358in}{2.698868in}}%
\pgfpathlineto{\pgfqpoint{0.995094in}{2.698868in}}%
\pgfpathlineto{\pgfqpoint{0.995094in}{2.611132in}}%
\pgfpathlineto{\pgfqpoint{0.907358in}{2.611132in}}%
\pgfpathlineto{\pgfqpoint{0.907358in}{2.698868in}}%
\pgfusepath{stroke,fill}%
\end{pgfscope}%
\begin{pgfscope}%
\pgfpathrectangle{\pgfqpoint{0.380943in}{2.260189in}}{\pgfqpoint{4.650000in}{0.614151in}}%
\pgfusepath{clip}%
\pgfsetbuttcap%
\pgfsetroundjoin%
\definecolor{currentfill}{rgb}{0.993679,0.753725,0.608074}%
\pgfsetfillcolor{currentfill}%
\pgfsetlinewidth{0.250937pt}%
\definecolor{currentstroke}{rgb}{1.000000,1.000000,1.000000}%
\pgfsetstrokecolor{currentstroke}%
\pgfsetdash{}{0pt}%
\pgfpathmoveto{\pgfqpoint{0.995094in}{2.698868in}}%
\pgfpathlineto{\pgfqpoint{1.082830in}{2.698868in}}%
\pgfpathlineto{\pgfqpoint{1.082830in}{2.611132in}}%
\pgfpathlineto{\pgfqpoint{0.995094in}{2.611132in}}%
\pgfpathlineto{\pgfqpoint{0.995094in}{2.698868in}}%
\pgfusepath{stroke,fill}%
\end{pgfscope}%
\begin{pgfscope}%
\pgfpathrectangle{\pgfqpoint{0.380943in}{2.260189in}}{\pgfqpoint{4.650000in}{0.614151in}}%
\pgfusepath{clip}%
\pgfsetbuttcap%
\pgfsetroundjoin%
\definecolor{currentfill}{rgb}{0.969504,0.885813,0.700930}%
\pgfsetfillcolor{currentfill}%
\pgfsetlinewidth{0.250937pt}%
\definecolor{currentstroke}{rgb}{1.000000,1.000000,1.000000}%
\pgfsetstrokecolor{currentstroke}%
\pgfsetdash{}{0pt}%
\pgfpathmoveto{\pgfqpoint{1.082830in}{2.698868in}}%
\pgfpathlineto{\pgfqpoint{1.170566in}{2.698868in}}%
\pgfpathlineto{\pgfqpoint{1.170566in}{2.611132in}}%
\pgfpathlineto{\pgfqpoint{1.082830in}{2.611132in}}%
\pgfpathlineto{\pgfqpoint{1.082830in}{2.698868in}}%
\pgfusepath{stroke,fill}%
\end{pgfscope}%
\begin{pgfscope}%
\pgfpathrectangle{\pgfqpoint{0.380943in}{2.260189in}}{\pgfqpoint{4.650000in}{0.614151in}}%
\pgfusepath{clip}%
\pgfsetbuttcap%
\pgfsetroundjoin%
\definecolor{currentfill}{rgb}{0.978639,0.841584,0.673679}%
\pgfsetfillcolor{currentfill}%
\pgfsetlinewidth{0.250937pt}%
\definecolor{currentstroke}{rgb}{1.000000,1.000000,1.000000}%
\pgfsetstrokecolor{currentstroke}%
\pgfsetdash{}{0pt}%
\pgfpathmoveto{\pgfqpoint{1.170566in}{2.698868in}}%
\pgfpathlineto{\pgfqpoint{1.258302in}{2.698868in}}%
\pgfpathlineto{\pgfqpoint{1.258302in}{2.611132in}}%
\pgfpathlineto{\pgfqpoint{1.170566in}{2.611132in}}%
\pgfpathlineto{\pgfqpoint{1.170566in}{2.698868in}}%
\pgfusepath{stroke,fill}%
\end{pgfscope}%
\begin{pgfscope}%
\pgfpathrectangle{\pgfqpoint{0.380943in}{2.260189in}}{\pgfqpoint{4.650000in}{0.614151in}}%
\pgfusepath{clip}%
\pgfsetbuttcap%
\pgfsetroundjoin%
\definecolor{currentfill}{rgb}{0.964937,0.908651,0.713110}%
\pgfsetfillcolor{currentfill}%
\pgfsetlinewidth{0.250937pt}%
\definecolor{currentstroke}{rgb}{1.000000,1.000000,1.000000}%
\pgfsetstrokecolor{currentstroke}%
\pgfsetdash{}{0pt}%
\pgfpathmoveto{\pgfqpoint{1.258302in}{2.698868in}}%
\pgfpathlineto{\pgfqpoint{1.346037in}{2.698868in}}%
\pgfpathlineto{\pgfqpoint{1.346037in}{2.611132in}}%
\pgfpathlineto{\pgfqpoint{1.258302in}{2.611132in}}%
\pgfpathlineto{\pgfqpoint{1.258302in}{2.698868in}}%
\pgfusepath{stroke,fill}%
\end{pgfscope}%
\begin{pgfscope}%
\pgfpathrectangle{\pgfqpoint{0.380943in}{2.260189in}}{\pgfqpoint{4.650000in}{0.614151in}}%
\pgfusepath{clip}%
\pgfsetbuttcap%
\pgfsetroundjoin%
\definecolor{currentfill}{rgb}{0.969504,0.885813,0.700930}%
\pgfsetfillcolor{currentfill}%
\pgfsetlinewidth{0.250937pt}%
\definecolor{currentstroke}{rgb}{1.000000,1.000000,1.000000}%
\pgfsetstrokecolor{currentstroke}%
\pgfsetdash{}{0pt}%
\pgfpathmoveto{\pgfqpoint{1.346037in}{2.698868in}}%
\pgfpathlineto{\pgfqpoint{1.433773in}{2.698868in}}%
\pgfpathlineto{\pgfqpoint{1.433773in}{2.611132in}}%
\pgfpathlineto{\pgfqpoint{1.346037in}{2.611132in}}%
\pgfpathlineto{\pgfqpoint{1.346037in}{2.698868in}}%
\pgfusepath{stroke,fill}%
\end{pgfscope}%
\begin{pgfscope}%
\pgfpathrectangle{\pgfqpoint{0.380943in}{2.260189in}}{\pgfqpoint{4.650000in}{0.614151in}}%
\pgfusepath{clip}%
\pgfsetbuttcap%
\pgfsetroundjoin%
\definecolor{currentfill}{rgb}{0.995233,0.991895,0.818977}%
\pgfsetfillcolor{currentfill}%
\pgfsetlinewidth{0.250937pt}%
\definecolor{currentstroke}{rgb}{1.000000,1.000000,1.000000}%
\pgfsetstrokecolor{currentstroke}%
\pgfsetdash{}{0pt}%
\pgfpathmoveto{\pgfqpoint{1.433773in}{2.698868in}}%
\pgfpathlineto{\pgfqpoint{1.521509in}{2.698868in}}%
\pgfpathlineto{\pgfqpoint{1.521509in}{2.611132in}}%
\pgfpathlineto{\pgfqpoint{1.433773in}{2.611132in}}%
\pgfpathlineto{\pgfqpoint{1.433773in}{2.698868in}}%
\pgfusepath{stroke,fill}%
\end{pgfscope}%
\begin{pgfscope}%
\pgfpathrectangle{\pgfqpoint{0.380943in}{2.260189in}}{\pgfqpoint{4.650000in}{0.614151in}}%
\pgfusepath{clip}%
\pgfsetbuttcap%
\pgfsetroundjoin%
\definecolor{currentfill}{rgb}{0.995233,0.991895,0.818977}%
\pgfsetfillcolor{currentfill}%
\pgfsetlinewidth{0.250937pt}%
\definecolor{currentstroke}{rgb}{1.000000,1.000000,1.000000}%
\pgfsetstrokecolor{currentstroke}%
\pgfsetdash{}{0pt}%
\pgfpathmoveto{\pgfqpoint{1.521509in}{2.698868in}}%
\pgfpathlineto{\pgfqpoint{1.609245in}{2.698868in}}%
\pgfpathlineto{\pgfqpoint{1.609245in}{2.611132in}}%
\pgfpathlineto{\pgfqpoint{1.521509in}{2.611132in}}%
\pgfpathlineto{\pgfqpoint{1.521509in}{2.698868in}}%
\pgfusepath{stroke,fill}%
\end{pgfscope}%
\begin{pgfscope}%
\pgfpathrectangle{\pgfqpoint{0.380943in}{2.260189in}}{\pgfqpoint{4.650000in}{0.614151in}}%
\pgfusepath{clip}%
\pgfsetbuttcap%
\pgfsetroundjoin%
\definecolor{currentfill}{rgb}{0.964783,0.940131,0.739808}%
\pgfsetfillcolor{currentfill}%
\pgfsetlinewidth{0.250937pt}%
\definecolor{currentstroke}{rgb}{1.000000,1.000000,1.000000}%
\pgfsetstrokecolor{currentstroke}%
\pgfsetdash{}{0pt}%
\pgfpathmoveto{\pgfqpoint{1.609245in}{2.698868in}}%
\pgfpathlineto{\pgfqpoint{1.696981in}{2.698868in}}%
\pgfpathlineto{\pgfqpoint{1.696981in}{2.611132in}}%
\pgfpathlineto{\pgfqpoint{1.609245in}{2.611132in}}%
\pgfpathlineto{\pgfqpoint{1.609245in}{2.698868in}}%
\pgfusepath{stroke,fill}%
\end{pgfscope}%
\begin{pgfscope}%
\pgfpathrectangle{\pgfqpoint{0.380943in}{2.260189in}}{\pgfqpoint{4.650000in}{0.614151in}}%
\pgfusepath{clip}%
\pgfsetbuttcap%
\pgfsetroundjoin%
\definecolor{currentfill}{rgb}{0.961738,0.927612,0.725598}%
\pgfsetfillcolor{currentfill}%
\pgfsetlinewidth{0.250937pt}%
\definecolor{currentstroke}{rgb}{1.000000,1.000000,1.000000}%
\pgfsetstrokecolor{currentstroke}%
\pgfsetdash{}{0pt}%
\pgfpathmoveto{\pgfqpoint{1.696981in}{2.698868in}}%
\pgfpathlineto{\pgfqpoint{1.784717in}{2.698868in}}%
\pgfpathlineto{\pgfqpoint{1.784717in}{2.611132in}}%
\pgfpathlineto{\pgfqpoint{1.696981in}{2.611132in}}%
\pgfpathlineto{\pgfqpoint{1.696981in}{2.698868in}}%
\pgfusepath{stroke,fill}%
\end{pgfscope}%
\begin{pgfscope}%
\pgfpathrectangle{\pgfqpoint{0.380943in}{2.260189in}}{\pgfqpoint{4.650000in}{0.614151in}}%
\pgfusepath{clip}%
\pgfsetbuttcap%
\pgfsetroundjoin%
\definecolor{currentfill}{rgb}{0.963260,0.918478,0.719508}%
\pgfsetfillcolor{currentfill}%
\pgfsetlinewidth{0.250937pt}%
\definecolor{currentstroke}{rgb}{1.000000,1.000000,1.000000}%
\pgfsetstrokecolor{currentstroke}%
\pgfsetdash{}{0pt}%
\pgfpathmoveto{\pgfqpoint{1.784717in}{2.698868in}}%
\pgfpathlineto{\pgfqpoint{1.872452in}{2.698868in}}%
\pgfpathlineto{\pgfqpoint{1.872452in}{2.611132in}}%
\pgfpathlineto{\pgfqpoint{1.784717in}{2.611132in}}%
\pgfpathlineto{\pgfqpoint{1.784717in}{2.698868in}}%
\pgfusepath{stroke,fill}%
\end{pgfscope}%
\begin{pgfscope}%
\pgfpathrectangle{\pgfqpoint{0.380943in}{2.260189in}}{\pgfqpoint{4.650000in}{0.614151in}}%
\pgfusepath{clip}%
\pgfsetbuttcap%
\pgfsetroundjoin%
\definecolor{currentfill}{rgb}{0.963260,0.918478,0.719508}%
\pgfsetfillcolor{currentfill}%
\pgfsetlinewidth{0.250937pt}%
\definecolor{currentstroke}{rgb}{1.000000,1.000000,1.000000}%
\pgfsetstrokecolor{currentstroke}%
\pgfsetdash{}{0pt}%
\pgfpathmoveto{\pgfqpoint{1.872452in}{2.698868in}}%
\pgfpathlineto{\pgfqpoint{1.960188in}{2.698868in}}%
\pgfpathlineto{\pgfqpoint{1.960188in}{2.611132in}}%
\pgfpathlineto{\pgfqpoint{1.872452in}{2.611132in}}%
\pgfpathlineto{\pgfqpoint{1.872452in}{2.698868in}}%
\pgfusepath{stroke,fill}%
\end{pgfscope}%
\begin{pgfscope}%
\pgfpathrectangle{\pgfqpoint{0.380943in}{2.260189in}}{\pgfqpoint{4.650000in}{0.614151in}}%
\pgfusepath{clip}%
\pgfsetbuttcap%
\pgfsetroundjoin%
\definecolor{currentfill}{rgb}{0.974072,0.862976,0.688750}%
\pgfsetfillcolor{currentfill}%
\pgfsetlinewidth{0.250937pt}%
\definecolor{currentstroke}{rgb}{1.000000,1.000000,1.000000}%
\pgfsetstrokecolor{currentstroke}%
\pgfsetdash{}{0pt}%
\pgfpathmoveto{\pgfqpoint{1.960188in}{2.698868in}}%
\pgfpathlineto{\pgfqpoint{2.047924in}{2.698868in}}%
\pgfpathlineto{\pgfqpoint{2.047924in}{2.611132in}}%
\pgfpathlineto{\pgfqpoint{1.960188in}{2.611132in}}%
\pgfpathlineto{\pgfqpoint{1.960188in}{2.698868in}}%
\pgfusepath{stroke,fill}%
\end{pgfscope}%
\begin{pgfscope}%
\pgfpathrectangle{\pgfqpoint{0.380943in}{2.260189in}}{\pgfqpoint{4.650000in}{0.614151in}}%
\pgfusepath{clip}%
\pgfsetbuttcap%
\pgfsetroundjoin%
\definecolor{currentfill}{rgb}{0.995233,0.991895,0.818977}%
\pgfsetfillcolor{currentfill}%
\pgfsetlinewidth{0.250937pt}%
\definecolor{currentstroke}{rgb}{1.000000,1.000000,1.000000}%
\pgfsetstrokecolor{currentstroke}%
\pgfsetdash{}{0pt}%
\pgfpathmoveto{\pgfqpoint{2.047924in}{2.698868in}}%
\pgfpathlineto{\pgfqpoint{2.135660in}{2.698868in}}%
\pgfpathlineto{\pgfqpoint{2.135660in}{2.611132in}}%
\pgfpathlineto{\pgfqpoint{2.047924in}{2.611132in}}%
\pgfpathlineto{\pgfqpoint{2.047924in}{2.698868in}}%
\pgfusepath{stroke,fill}%
\end{pgfscope}%
\begin{pgfscope}%
\pgfpathrectangle{\pgfqpoint{0.380943in}{2.260189in}}{\pgfqpoint{4.650000in}{0.614151in}}%
\pgfusepath{clip}%
\pgfsetbuttcap%
\pgfsetroundjoin%
\definecolor{currentfill}{rgb}{0.974072,0.862976,0.688750}%
\pgfsetfillcolor{currentfill}%
\pgfsetlinewidth{0.250937pt}%
\definecolor{currentstroke}{rgb}{1.000000,1.000000,1.000000}%
\pgfsetstrokecolor{currentstroke}%
\pgfsetdash{}{0pt}%
\pgfpathmoveto{\pgfqpoint{2.135660in}{2.698868in}}%
\pgfpathlineto{\pgfqpoint{2.223396in}{2.698868in}}%
\pgfpathlineto{\pgfqpoint{2.223396in}{2.611132in}}%
\pgfpathlineto{\pgfqpoint{2.135660in}{2.611132in}}%
\pgfpathlineto{\pgfqpoint{2.135660in}{2.698868in}}%
\pgfusepath{stroke,fill}%
\end{pgfscope}%
\begin{pgfscope}%
\pgfpathrectangle{\pgfqpoint{0.380943in}{2.260189in}}{\pgfqpoint{4.650000in}{0.614151in}}%
\pgfusepath{clip}%
\pgfsetbuttcap%
\pgfsetroundjoin%
\definecolor{currentfill}{rgb}{0.978639,0.841584,0.673679}%
\pgfsetfillcolor{currentfill}%
\pgfsetlinewidth{0.250937pt}%
\definecolor{currentstroke}{rgb}{1.000000,1.000000,1.000000}%
\pgfsetstrokecolor{currentstroke}%
\pgfsetdash{}{0pt}%
\pgfpathmoveto{\pgfqpoint{2.223396in}{2.698868in}}%
\pgfpathlineto{\pgfqpoint{2.311132in}{2.698868in}}%
\pgfpathlineto{\pgfqpoint{2.311132in}{2.611132in}}%
\pgfpathlineto{\pgfqpoint{2.223396in}{2.611132in}}%
\pgfpathlineto{\pgfqpoint{2.223396in}{2.698868in}}%
\pgfusepath{stroke,fill}%
\end{pgfscope}%
\begin{pgfscope}%
\pgfpathrectangle{\pgfqpoint{0.380943in}{2.260189in}}{\pgfqpoint{4.650000in}{0.614151in}}%
\pgfusepath{clip}%
\pgfsetbuttcap%
\pgfsetroundjoin%
\definecolor{currentfill}{rgb}{0.969504,0.885813,0.700930}%
\pgfsetfillcolor{currentfill}%
\pgfsetlinewidth{0.250937pt}%
\definecolor{currentstroke}{rgb}{1.000000,1.000000,1.000000}%
\pgfsetstrokecolor{currentstroke}%
\pgfsetdash{}{0pt}%
\pgfpathmoveto{\pgfqpoint{2.311132in}{2.698868in}}%
\pgfpathlineto{\pgfqpoint{2.398868in}{2.698868in}}%
\pgfpathlineto{\pgfqpoint{2.398868in}{2.611132in}}%
\pgfpathlineto{\pgfqpoint{2.311132in}{2.611132in}}%
\pgfpathlineto{\pgfqpoint{2.311132in}{2.698868in}}%
\pgfusepath{stroke,fill}%
\end{pgfscope}%
\begin{pgfscope}%
\pgfpathrectangle{\pgfqpoint{0.380943in}{2.260189in}}{\pgfqpoint{4.650000in}{0.614151in}}%
\pgfusepath{clip}%
\pgfsetbuttcap%
\pgfsetroundjoin%
\definecolor{currentfill}{rgb}{0.978639,0.841584,0.673679}%
\pgfsetfillcolor{currentfill}%
\pgfsetlinewidth{0.250937pt}%
\definecolor{currentstroke}{rgb}{1.000000,1.000000,1.000000}%
\pgfsetstrokecolor{currentstroke}%
\pgfsetdash{}{0pt}%
\pgfpathmoveto{\pgfqpoint{2.398868in}{2.698868in}}%
\pgfpathlineto{\pgfqpoint{2.486603in}{2.698868in}}%
\pgfpathlineto{\pgfqpoint{2.486603in}{2.611132in}}%
\pgfpathlineto{\pgfqpoint{2.398868in}{2.611132in}}%
\pgfpathlineto{\pgfqpoint{2.398868in}{2.698868in}}%
\pgfusepath{stroke,fill}%
\end{pgfscope}%
\begin{pgfscope}%
\pgfpathrectangle{\pgfqpoint{0.380943in}{2.260189in}}{\pgfqpoint{4.650000in}{0.614151in}}%
\pgfusepath{clip}%
\pgfsetbuttcap%
\pgfsetroundjoin%
\definecolor{currentfill}{rgb}{0.913879,0.392311,0.392311}%
\pgfsetfillcolor{currentfill}%
\pgfsetlinewidth{0.250937pt}%
\definecolor{currentstroke}{rgb}{1.000000,1.000000,1.000000}%
\pgfsetstrokecolor{currentstroke}%
\pgfsetdash{}{0pt}%
\pgfpathmoveto{\pgfqpoint{2.486603in}{2.698868in}}%
\pgfpathlineto{\pgfqpoint{2.574339in}{2.698868in}}%
\pgfpathlineto{\pgfqpoint{2.574339in}{2.611132in}}%
\pgfpathlineto{\pgfqpoint{2.486603in}{2.611132in}}%
\pgfpathlineto{\pgfqpoint{2.486603in}{2.698868in}}%
\pgfusepath{stroke,fill}%
\end{pgfscope}%
\begin{pgfscope}%
\pgfpathrectangle{\pgfqpoint{0.380943in}{2.260189in}}{\pgfqpoint{4.650000in}{0.614151in}}%
\pgfusepath{clip}%
\pgfsetbuttcap%
\pgfsetroundjoin%
\definecolor{currentfill}{rgb}{0.996401,0.724937,0.591557}%
\pgfsetfillcolor{currentfill}%
\pgfsetlinewidth{0.250937pt}%
\definecolor{currentstroke}{rgb}{1.000000,1.000000,1.000000}%
\pgfsetstrokecolor{currentstroke}%
\pgfsetdash{}{0pt}%
\pgfpathmoveto{\pgfqpoint{2.574339in}{2.698868in}}%
\pgfpathlineto{\pgfqpoint{2.662075in}{2.698868in}}%
\pgfpathlineto{\pgfqpoint{2.662075in}{2.611132in}}%
\pgfpathlineto{\pgfqpoint{2.574339in}{2.611132in}}%
\pgfpathlineto{\pgfqpoint{2.574339in}{2.698868in}}%
\pgfusepath{stroke,fill}%
\end{pgfscope}%
\begin{pgfscope}%
\pgfpathrectangle{\pgfqpoint{0.380943in}{2.260189in}}{\pgfqpoint{4.650000in}{0.614151in}}%
\pgfusepath{clip}%
\pgfsetbuttcap%
\pgfsetroundjoin%
\definecolor{currentfill}{rgb}{0.987266,0.804198,0.639170}%
\pgfsetfillcolor{currentfill}%
\pgfsetlinewidth{0.250937pt}%
\definecolor{currentstroke}{rgb}{1.000000,1.000000,1.000000}%
\pgfsetstrokecolor{currentstroke}%
\pgfsetdash{}{0pt}%
\pgfpathmoveto{\pgfqpoint{2.662075in}{2.698868in}}%
\pgfpathlineto{\pgfqpoint{2.749811in}{2.698868in}}%
\pgfpathlineto{\pgfqpoint{2.749811in}{2.611132in}}%
\pgfpathlineto{\pgfqpoint{2.662075in}{2.611132in}}%
\pgfpathlineto{\pgfqpoint{2.662075in}{2.698868in}}%
\pgfusepath{stroke,fill}%
\end{pgfscope}%
\begin{pgfscope}%
\pgfpathrectangle{\pgfqpoint{0.380943in}{2.260189in}}{\pgfqpoint{4.650000in}{0.614151in}}%
\pgfusepath{clip}%
\pgfsetbuttcap%
\pgfsetroundjoin%
\definecolor{currentfill}{rgb}{0.987266,0.804198,0.639170}%
\pgfsetfillcolor{currentfill}%
\pgfsetlinewidth{0.250937pt}%
\definecolor{currentstroke}{rgb}{1.000000,1.000000,1.000000}%
\pgfsetstrokecolor{currentstroke}%
\pgfsetdash{}{0pt}%
\pgfpathmoveto{\pgfqpoint{2.749811in}{2.698868in}}%
\pgfpathlineto{\pgfqpoint{2.837547in}{2.698868in}}%
\pgfpathlineto{\pgfqpoint{2.837547in}{2.611132in}}%
\pgfpathlineto{\pgfqpoint{2.749811in}{2.611132in}}%
\pgfpathlineto{\pgfqpoint{2.749811in}{2.698868in}}%
\pgfusepath{stroke,fill}%
\end{pgfscope}%
\begin{pgfscope}%
\pgfpathrectangle{\pgfqpoint{0.380943in}{2.260189in}}{\pgfqpoint{4.650000in}{0.614151in}}%
\pgfusepath{clip}%
\pgfsetbuttcap%
\pgfsetroundjoin%
\definecolor{currentfill}{rgb}{0.997924,0.685352,0.570242}%
\pgfsetfillcolor{currentfill}%
\pgfsetlinewidth{0.250937pt}%
\definecolor{currentstroke}{rgb}{1.000000,1.000000,1.000000}%
\pgfsetstrokecolor{currentstroke}%
\pgfsetdash{}{0pt}%
\pgfpathmoveto{\pgfqpoint{2.837547in}{2.698868in}}%
\pgfpathlineto{\pgfqpoint{2.925283in}{2.698868in}}%
\pgfpathlineto{\pgfqpoint{2.925283in}{2.611132in}}%
\pgfpathlineto{\pgfqpoint{2.837547in}{2.611132in}}%
\pgfpathlineto{\pgfqpoint{2.837547in}{2.698868in}}%
\pgfusepath{stroke,fill}%
\end{pgfscope}%
\begin{pgfscope}%
\pgfpathrectangle{\pgfqpoint{0.380943in}{2.260189in}}{\pgfqpoint{4.650000in}{0.614151in}}%
\pgfusepath{clip}%
\pgfsetbuttcap%
\pgfsetroundjoin%
\definecolor{currentfill}{rgb}{0.996401,0.724937,0.591557}%
\pgfsetfillcolor{currentfill}%
\pgfsetlinewidth{0.250937pt}%
\definecolor{currentstroke}{rgb}{1.000000,1.000000,1.000000}%
\pgfsetstrokecolor{currentstroke}%
\pgfsetdash{}{0pt}%
\pgfpathmoveto{\pgfqpoint{2.925283in}{2.698868in}}%
\pgfpathlineto{\pgfqpoint{3.013019in}{2.698868in}}%
\pgfpathlineto{\pgfqpoint{3.013019in}{2.611132in}}%
\pgfpathlineto{\pgfqpoint{2.925283in}{2.611132in}}%
\pgfpathlineto{\pgfqpoint{2.925283in}{2.698868in}}%
\pgfusepath{stroke,fill}%
\end{pgfscope}%
\begin{pgfscope}%
\pgfpathrectangle{\pgfqpoint{0.380943in}{2.260189in}}{\pgfqpoint{4.650000in}{0.614151in}}%
\pgfusepath{clip}%
\pgfsetbuttcap%
\pgfsetroundjoin%
\definecolor{currentfill}{rgb}{0.961738,0.927612,0.725598}%
\pgfsetfillcolor{currentfill}%
\pgfsetlinewidth{0.250937pt}%
\definecolor{currentstroke}{rgb}{1.000000,1.000000,1.000000}%
\pgfsetstrokecolor{currentstroke}%
\pgfsetdash{}{0pt}%
\pgfpathmoveto{\pgfqpoint{3.013019in}{2.698868in}}%
\pgfpathlineto{\pgfqpoint{3.100754in}{2.698868in}}%
\pgfpathlineto{\pgfqpoint{3.100754in}{2.611132in}}%
\pgfpathlineto{\pgfqpoint{3.013019in}{2.611132in}}%
\pgfpathlineto{\pgfqpoint{3.013019in}{2.698868in}}%
\pgfusepath{stroke,fill}%
\end{pgfscope}%
\begin{pgfscope}%
\pgfpathrectangle{\pgfqpoint{0.380943in}{2.260189in}}{\pgfqpoint{4.650000in}{0.614151in}}%
\pgfusepath{clip}%
\pgfsetbuttcap%
\pgfsetroundjoin%
\definecolor{currentfill}{rgb}{0.982699,0.823991,0.657439}%
\pgfsetfillcolor{currentfill}%
\pgfsetlinewidth{0.250937pt}%
\definecolor{currentstroke}{rgb}{1.000000,1.000000,1.000000}%
\pgfsetstrokecolor{currentstroke}%
\pgfsetdash{}{0pt}%
\pgfpathmoveto{\pgfqpoint{3.100754in}{2.698868in}}%
\pgfpathlineto{\pgfqpoint{3.188490in}{2.698868in}}%
\pgfpathlineto{\pgfqpoint{3.188490in}{2.611132in}}%
\pgfpathlineto{\pgfqpoint{3.100754in}{2.611132in}}%
\pgfpathlineto{\pgfqpoint{3.100754in}{2.698868in}}%
\pgfusepath{stroke,fill}%
\end{pgfscope}%
\begin{pgfscope}%
\pgfpathrectangle{\pgfqpoint{0.380943in}{2.260189in}}{\pgfqpoint{4.650000in}{0.614151in}}%
\pgfusepath{clip}%
\pgfsetbuttcap%
\pgfsetroundjoin%
\definecolor{currentfill}{rgb}{0.978639,0.841584,0.673679}%
\pgfsetfillcolor{currentfill}%
\pgfsetlinewidth{0.250937pt}%
\definecolor{currentstroke}{rgb}{1.000000,1.000000,1.000000}%
\pgfsetstrokecolor{currentstroke}%
\pgfsetdash{}{0pt}%
\pgfpathmoveto{\pgfqpoint{3.188490in}{2.698868in}}%
\pgfpathlineto{\pgfqpoint{3.276226in}{2.698868in}}%
\pgfpathlineto{\pgfqpoint{3.276226in}{2.611132in}}%
\pgfpathlineto{\pgfqpoint{3.188490in}{2.611132in}}%
\pgfpathlineto{\pgfqpoint{3.188490in}{2.698868in}}%
\pgfusepath{stroke,fill}%
\end{pgfscope}%
\begin{pgfscope}%
\pgfpathrectangle{\pgfqpoint{0.380943in}{2.260189in}}{\pgfqpoint{4.650000in}{0.614151in}}%
\pgfusepath{clip}%
\pgfsetbuttcap%
\pgfsetroundjoin%
\definecolor{currentfill}{rgb}{0.996401,0.724937,0.591557}%
\pgfsetfillcolor{currentfill}%
\pgfsetlinewidth{0.250937pt}%
\definecolor{currentstroke}{rgb}{1.000000,1.000000,1.000000}%
\pgfsetstrokecolor{currentstroke}%
\pgfsetdash{}{0pt}%
\pgfpathmoveto{\pgfqpoint{3.276226in}{2.698868in}}%
\pgfpathlineto{\pgfqpoint{3.363962in}{2.698868in}}%
\pgfpathlineto{\pgfqpoint{3.363962in}{2.611132in}}%
\pgfpathlineto{\pgfqpoint{3.276226in}{2.611132in}}%
\pgfpathlineto{\pgfqpoint{3.276226in}{2.698868in}}%
\pgfusepath{stroke,fill}%
\end{pgfscope}%
\begin{pgfscope}%
\pgfpathrectangle{\pgfqpoint{0.380943in}{2.260189in}}{\pgfqpoint{4.650000in}{0.614151in}}%
\pgfusepath{clip}%
\pgfsetbuttcap%
\pgfsetroundjoin%
\definecolor{currentfill}{rgb}{0.964937,0.908651,0.713110}%
\pgfsetfillcolor{currentfill}%
\pgfsetlinewidth{0.250937pt}%
\definecolor{currentstroke}{rgb}{1.000000,1.000000,1.000000}%
\pgfsetstrokecolor{currentstroke}%
\pgfsetdash{}{0pt}%
\pgfpathmoveto{\pgfqpoint{3.363962in}{2.698868in}}%
\pgfpathlineto{\pgfqpoint{3.451698in}{2.698868in}}%
\pgfpathlineto{\pgfqpoint{3.451698in}{2.611132in}}%
\pgfpathlineto{\pgfqpoint{3.363962in}{2.611132in}}%
\pgfpathlineto{\pgfqpoint{3.363962in}{2.698868in}}%
\pgfusepath{stroke,fill}%
\end{pgfscope}%
\begin{pgfscope}%
\pgfpathrectangle{\pgfqpoint{0.380943in}{2.260189in}}{\pgfqpoint{4.650000in}{0.614151in}}%
\pgfusepath{clip}%
\pgfsetbuttcap%
\pgfsetroundjoin%
\definecolor{currentfill}{rgb}{0.999277,0.650165,0.551296}%
\pgfsetfillcolor{currentfill}%
\pgfsetlinewidth{0.250937pt}%
\definecolor{currentstroke}{rgb}{1.000000,1.000000,1.000000}%
\pgfsetstrokecolor{currentstroke}%
\pgfsetdash{}{0pt}%
\pgfpathmoveto{\pgfqpoint{3.451698in}{2.698868in}}%
\pgfpathlineto{\pgfqpoint{3.539434in}{2.698868in}}%
\pgfpathlineto{\pgfqpoint{3.539434in}{2.611132in}}%
\pgfpathlineto{\pgfqpoint{3.451698in}{2.611132in}}%
\pgfpathlineto{\pgfqpoint{3.451698in}{2.698868in}}%
\pgfusepath{stroke,fill}%
\end{pgfscope}%
\begin{pgfscope}%
\pgfpathrectangle{\pgfqpoint{0.380943in}{2.260189in}}{\pgfqpoint{4.650000in}{0.614151in}}%
\pgfusepath{clip}%
\pgfsetbuttcap%
\pgfsetroundjoin%
\definecolor{currentfill}{rgb}{0.982699,0.823991,0.657439}%
\pgfsetfillcolor{currentfill}%
\pgfsetlinewidth{0.250937pt}%
\definecolor{currentstroke}{rgb}{1.000000,1.000000,1.000000}%
\pgfsetstrokecolor{currentstroke}%
\pgfsetdash{}{0pt}%
\pgfpathmoveto{\pgfqpoint{3.539434in}{2.698868in}}%
\pgfpathlineto{\pgfqpoint{3.627169in}{2.698868in}}%
\pgfpathlineto{\pgfqpoint{3.627169in}{2.611132in}}%
\pgfpathlineto{\pgfqpoint{3.539434in}{2.611132in}}%
\pgfpathlineto{\pgfqpoint{3.539434in}{2.698868in}}%
\pgfusepath{stroke,fill}%
\end{pgfscope}%
\begin{pgfscope}%
\pgfpathrectangle{\pgfqpoint{0.380943in}{2.260189in}}{\pgfqpoint{4.650000in}{0.614151in}}%
\pgfusepath{clip}%
\pgfsetbuttcap%
\pgfsetroundjoin%
\definecolor{currentfill}{rgb}{0.990634,0.779608,0.623299}%
\pgfsetfillcolor{currentfill}%
\pgfsetlinewidth{0.250937pt}%
\definecolor{currentstroke}{rgb}{1.000000,1.000000,1.000000}%
\pgfsetstrokecolor{currentstroke}%
\pgfsetdash{}{0pt}%
\pgfpathmoveto{\pgfqpoint{3.627169in}{2.698868in}}%
\pgfpathlineto{\pgfqpoint{3.714905in}{2.698868in}}%
\pgfpathlineto{\pgfqpoint{3.714905in}{2.611132in}}%
\pgfpathlineto{\pgfqpoint{3.627169in}{2.611132in}}%
\pgfpathlineto{\pgfqpoint{3.627169in}{2.698868in}}%
\pgfusepath{stroke,fill}%
\end{pgfscope}%
\begin{pgfscope}%
\pgfpathrectangle{\pgfqpoint{0.380943in}{2.260189in}}{\pgfqpoint{4.650000in}{0.614151in}}%
\pgfusepath{clip}%
\pgfsetbuttcap%
\pgfsetroundjoin%
\definecolor{currentfill}{rgb}{0.996401,0.724937,0.591557}%
\pgfsetfillcolor{currentfill}%
\pgfsetlinewidth{0.250937pt}%
\definecolor{currentstroke}{rgb}{1.000000,1.000000,1.000000}%
\pgfsetstrokecolor{currentstroke}%
\pgfsetdash{}{0pt}%
\pgfpathmoveto{\pgfqpoint{3.714905in}{2.698868in}}%
\pgfpathlineto{\pgfqpoint{3.802641in}{2.698868in}}%
\pgfpathlineto{\pgfqpoint{3.802641in}{2.611132in}}%
\pgfpathlineto{\pgfqpoint{3.714905in}{2.611132in}}%
\pgfpathlineto{\pgfqpoint{3.714905in}{2.698868in}}%
\pgfusepath{stroke,fill}%
\end{pgfscope}%
\begin{pgfscope}%
\pgfpathrectangle{\pgfqpoint{0.380943in}{2.260189in}}{\pgfqpoint{4.650000in}{0.614151in}}%
\pgfusepath{clip}%
\pgfsetbuttcap%
\pgfsetroundjoin%
\definecolor{currentfill}{rgb}{1.000000,0.525475,0.498239}%
\pgfsetfillcolor{currentfill}%
\pgfsetlinewidth{0.250937pt}%
\definecolor{currentstroke}{rgb}{1.000000,1.000000,1.000000}%
\pgfsetstrokecolor{currentstroke}%
\pgfsetdash{}{0pt}%
\pgfpathmoveto{\pgfqpoint{3.802641in}{2.698868in}}%
\pgfpathlineto{\pgfqpoint{3.890377in}{2.698868in}}%
\pgfpathlineto{\pgfqpoint{3.890377in}{2.611132in}}%
\pgfpathlineto{\pgfqpoint{3.802641in}{2.611132in}}%
\pgfpathlineto{\pgfqpoint{3.802641in}{2.698868in}}%
\pgfusepath{stroke,fill}%
\end{pgfscope}%
\begin{pgfscope}%
\pgfpathrectangle{\pgfqpoint{0.380943in}{2.260189in}}{\pgfqpoint{4.650000in}{0.614151in}}%
\pgfusepath{clip}%
\pgfsetbuttcap%
\pgfsetroundjoin%
\definecolor{currentfill}{rgb}{0.982699,0.823991,0.657439}%
\pgfsetfillcolor{currentfill}%
\pgfsetlinewidth{0.250937pt}%
\definecolor{currentstroke}{rgb}{1.000000,1.000000,1.000000}%
\pgfsetstrokecolor{currentstroke}%
\pgfsetdash{}{0pt}%
\pgfpathmoveto{\pgfqpoint{3.890377in}{2.698868in}}%
\pgfpathlineto{\pgfqpoint{3.978113in}{2.698868in}}%
\pgfpathlineto{\pgfqpoint{3.978113in}{2.611132in}}%
\pgfpathlineto{\pgfqpoint{3.890377in}{2.611132in}}%
\pgfpathlineto{\pgfqpoint{3.890377in}{2.698868in}}%
\pgfusepath{stroke,fill}%
\end{pgfscope}%
\begin{pgfscope}%
\pgfpathrectangle{\pgfqpoint{0.380943in}{2.260189in}}{\pgfqpoint{4.650000in}{0.614151in}}%
\pgfusepath{clip}%
\pgfsetbuttcap%
\pgfsetroundjoin%
\definecolor{currentfill}{rgb}{0.999277,0.650165,0.551296}%
\pgfsetfillcolor{currentfill}%
\pgfsetlinewidth{0.250937pt}%
\definecolor{currentstroke}{rgb}{1.000000,1.000000,1.000000}%
\pgfsetstrokecolor{currentstroke}%
\pgfsetdash{}{0pt}%
\pgfpathmoveto{\pgfqpoint{3.978113in}{2.698868in}}%
\pgfpathlineto{\pgfqpoint{4.065849in}{2.698868in}}%
\pgfpathlineto{\pgfqpoint{4.065849in}{2.611132in}}%
\pgfpathlineto{\pgfqpoint{3.978113in}{2.611132in}}%
\pgfpathlineto{\pgfqpoint{3.978113in}{2.698868in}}%
\pgfusepath{stroke,fill}%
\end{pgfscope}%
\begin{pgfscope}%
\pgfpathrectangle{\pgfqpoint{0.380943in}{2.260189in}}{\pgfqpoint{4.650000in}{0.614151in}}%
\pgfusepath{clip}%
\pgfsetbuttcap%
\pgfsetroundjoin%
\definecolor{currentfill}{rgb}{1.000000,0.554479,0.510419}%
\pgfsetfillcolor{currentfill}%
\pgfsetlinewidth{0.250937pt}%
\definecolor{currentstroke}{rgb}{1.000000,1.000000,1.000000}%
\pgfsetstrokecolor{currentstroke}%
\pgfsetdash{}{0pt}%
\pgfpathmoveto{\pgfqpoint{4.065849in}{2.698868in}}%
\pgfpathlineto{\pgfqpoint{4.153585in}{2.698868in}}%
\pgfpathlineto{\pgfqpoint{4.153585in}{2.611132in}}%
\pgfpathlineto{\pgfqpoint{4.065849in}{2.611132in}}%
\pgfpathlineto{\pgfqpoint{4.065849in}{2.698868in}}%
\pgfusepath{stroke,fill}%
\end{pgfscope}%
\begin{pgfscope}%
\pgfpathrectangle{\pgfqpoint{0.380943in}{2.260189in}}{\pgfqpoint{4.650000in}{0.614151in}}%
\pgfusepath{clip}%
\pgfsetbuttcap%
\pgfsetroundjoin%
\definecolor{currentfill}{rgb}{0.961738,0.927612,0.725598}%
\pgfsetfillcolor{currentfill}%
\pgfsetlinewidth{0.250937pt}%
\definecolor{currentstroke}{rgb}{1.000000,1.000000,1.000000}%
\pgfsetstrokecolor{currentstroke}%
\pgfsetdash{}{0pt}%
\pgfpathmoveto{\pgfqpoint{4.153585in}{2.698868in}}%
\pgfpathlineto{\pgfqpoint{4.241320in}{2.698868in}}%
\pgfpathlineto{\pgfqpoint{4.241320in}{2.611132in}}%
\pgfpathlineto{\pgfqpoint{4.153585in}{2.611132in}}%
\pgfpathlineto{\pgfqpoint{4.153585in}{2.698868in}}%
\pgfusepath{stroke,fill}%
\end{pgfscope}%
\begin{pgfscope}%
\pgfpathrectangle{\pgfqpoint{0.380943in}{2.260189in}}{\pgfqpoint{4.650000in}{0.614151in}}%
\pgfusepath{clip}%
\pgfsetbuttcap%
\pgfsetroundjoin%
\definecolor{currentfill}{rgb}{0.969504,0.885813,0.700930}%
\pgfsetfillcolor{currentfill}%
\pgfsetlinewidth{0.250937pt}%
\definecolor{currentstroke}{rgb}{1.000000,1.000000,1.000000}%
\pgfsetstrokecolor{currentstroke}%
\pgfsetdash{}{0pt}%
\pgfpathmoveto{\pgfqpoint{4.241320in}{2.698868in}}%
\pgfpathlineto{\pgfqpoint{4.329056in}{2.698868in}}%
\pgfpathlineto{\pgfqpoint{4.329056in}{2.611132in}}%
\pgfpathlineto{\pgfqpoint{4.241320in}{2.611132in}}%
\pgfpathlineto{\pgfqpoint{4.241320in}{2.698868in}}%
\pgfusepath{stroke,fill}%
\end{pgfscope}%
\begin{pgfscope}%
\pgfpathrectangle{\pgfqpoint{0.380943in}{2.260189in}}{\pgfqpoint{4.650000in}{0.614151in}}%
\pgfusepath{clip}%
\pgfsetbuttcap%
\pgfsetroundjoin%
\definecolor{currentfill}{rgb}{0.963260,0.918478,0.719508}%
\pgfsetfillcolor{currentfill}%
\pgfsetlinewidth{0.250937pt}%
\definecolor{currentstroke}{rgb}{1.000000,1.000000,1.000000}%
\pgfsetstrokecolor{currentstroke}%
\pgfsetdash{}{0pt}%
\pgfpathmoveto{\pgfqpoint{4.329056in}{2.698868in}}%
\pgfpathlineto{\pgfqpoint{4.416792in}{2.698868in}}%
\pgfpathlineto{\pgfqpoint{4.416792in}{2.611132in}}%
\pgfpathlineto{\pgfqpoint{4.329056in}{2.611132in}}%
\pgfpathlineto{\pgfqpoint{4.329056in}{2.698868in}}%
\pgfusepath{stroke,fill}%
\end{pgfscope}%
\begin{pgfscope}%
\pgfpathrectangle{\pgfqpoint{0.380943in}{2.260189in}}{\pgfqpoint{4.650000in}{0.614151in}}%
\pgfusepath{clip}%
\pgfsetbuttcap%
\pgfsetroundjoin%
\definecolor{currentfill}{rgb}{0.987266,0.804198,0.639170}%
\pgfsetfillcolor{currentfill}%
\pgfsetlinewidth{0.250937pt}%
\definecolor{currentstroke}{rgb}{1.000000,1.000000,1.000000}%
\pgfsetstrokecolor{currentstroke}%
\pgfsetdash{}{0pt}%
\pgfpathmoveto{\pgfqpoint{4.416792in}{2.698868in}}%
\pgfpathlineto{\pgfqpoint{4.504528in}{2.698868in}}%
\pgfpathlineto{\pgfqpoint{4.504528in}{2.611132in}}%
\pgfpathlineto{\pgfqpoint{4.416792in}{2.611132in}}%
\pgfpathlineto{\pgfqpoint{4.416792in}{2.698868in}}%
\pgfusepath{stroke,fill}%
\end{pgfscope}%
\begin{pgfscope}%
\pgfpathrectangle{\pgfqpoint{0.380943in}{2.260189in}}{\pgfqpoint{4.650000in}{0.614151in}}%
\pgfusepath{clip}%
\pgfsetbuttcap%
\pgfsetroundjoin%
\definecolor{currentfill}{rgb}{0.978639,0.841584,0.673679}%
\pgfsetfillcolor{currentfill}%
\pgfsetlinewidth{0.250937pt}%
\definecolor{currentstroke}{rgb}{1.000000,1.000000,1.000000}%
\pgfsetstrokecolor{currentstroke}%
\pgfsetdash{}{0pt}%
\pgfpathmoveto{\pgfqpoint{4.504528in}{2.698868in}}%
\pgfpathlineto{\pgfqpoint{4.592264in}{2.698868in}}%
\pgfpathlineto{\pgfqpoint{4.592264in}{2.611132in}}%
\pgfpathlineto{\pgfqpoint{4.504528in}{2.611132in}}%
\pgfpathlineto{\pgfqpoint{4.504528in}{2.698868in}}%
\pgfusepath{stroke,fill}%
\end{pgfscope}%
\begin{pgfscope}%
\pgfpathrectangle{\pgfqpoint{0.380943in}{2.260189in}}{\pgfqpoint{4.650000in}{0.614151in}}%
\pgfusepath{clip}%
\pgfsetbuttcap%
\pgfsetroundjoin%
\definecolor{currentfill}{rgb}{0.997924,0.685352,0.570242}%
\pgfsetfillcolor{currentfill}%
\pgfsetlinewidth{0.250937pt}%
\definecolor{currentstroke}{rgb}{1.000000,1.000000,1.000000}%
\pgfsetstrokecolor{currentstroke}%
\pgfsetdash{}{0pt}%
\pgfpathmoveto{\pgfqpoint{4.592264in}{2.698868in}}%
\pgfpathlineto{\pgfqpoint{4.680000in}{2.698868in}}%
\pgfpathlineto{\pgfqpoint{4.680000in}{2.611132in}}%
\pgfpathlineto{\pgfqpoint{4.592264in}{2.611132in}}%
\pgfpathlineto{\pgfqpoint{4.592264in}{2.698868in}}%
\pgfusepath{stroke,fill}%
\end{pgfscope}%
\begin{pgfscope}%
\pgfpathrectangle{\pgfqpoint{0.380943in}{2.260189in}}{\pgfqpoint{4.650000in}{0.614151in}}%
\pgfusepath{clip}%
\pgfsetbuttcap%
\pgfsetroundjoin%
\definecolor{currentfill}{rgb}{0.997924,0.685352,0.570242}%
\pgfsetfillcolor{currentfill}%
\pgfsetlinewidth{0.250937pt}%
\definecolor{currentstroke}{rgb}{1.000000,1.000000,1.000000}%
\pgfsetstrokecolor{currentstroke}%
\pgfsetdash{}{0pt}%
\pgfpathmoveto{\pgfqpoint{4.680000in}{2.698868in}}%
\pgfpathlineto{\pgfqpoint{4.767736in}{2.698868in}}%
\pgfpathlineto{\pgfqpoint{4.767736in}{2.611132in}}%
\pgfpathlineto{\pgfqpoint{4.680000in}{2.611132in}}%
\pgfpathlineto{\pgfqpoint{4.680000in}{2.698868in}}%
\pgfusepath{stroke,fill}%
\end{pgfscope}%
\begin{pgfscope}%
\pgfpathrectangle{\pgfqpoint{0.380943in}{2.260189in}}{\pgfqpoint{4.650000in}{0.614151in}}%
\pgfusepath{clip}%
\pgfsetbuttcap%
\pgfsetroundjoin%
\definecolor{currentfill}{rgb}{0.993679,0.753725,0.608074}%
\pgfsetfillcolor{currentfill}%
\pgfsetlinewidth{0.250937pt}%
\definecolor{currentstroke}{rgb}{1.000000,1.000000,1.000000}%
\pgfsetstrokecolor{currentstroke}%
\pgfsetdash{}{0pt}%
\pgfpathmoveto{\pgfqpoint{4.767736in}{2.698868in}}%
\pgfpathlineto{\pgfqpoint{4.855471in}{2.698868in}}%
\pgfpathlineto{\pgfqpoint{4.855471in}{2.611132in}}%
\pgfpathlineto{\pgfqpoint{4.767736in}{2.611132in}}%
\pgfpathlineto{\pgfqpoint{4.767736in}{2.698868in}}%
\pgfusepath{stroke,fill}%
\end{pgfscope}%
\begin{pgfscope}%
\pgfpathrectangle{\pgfqpoint{0.380943in}{2.260189in}}{\pgfqpoint{4.650000in}{0.614151in}}%
\pgfusepath{clip}%
\pgfsetbuttcap%
\pgfsetroundjoin%
\definecolor{currentfill}{rgb}{0.990634,0.779608,0.623299}%
\pgfsetfillcolor{currentfill}%
\pgfsetlinewidth{0.250937pt}%
\definecolor{currentstroke}{rgb}{1.000000,1.000000,1.000000}%
\pgfsetstrokecolor{currentstroke}%
\pgfsetdash{}{0pt}%
\pgfpathmoveto{\pgfqpoint{4.855471in}{2.698868in}}%
\pgfpathlineto{\pgfqpoint{4.943207in}{2.698868in}}%
\pgfpathlineto{\pgfqpoint{4.943207in}{2.611132in}}%
\pgfpathlineto{\pgfqpoint{4.855471in}{2.611132in}}%
\pgfpathlineto{\pgfqpoint{4.855471in}{2.698868in}}%
\pgfusepath{stroke,fill}%
\end{pgfscope}%
\begin{pgfscope}%
\pgfpathrectangle{\pgfqpoint{0.380943in}{2.260189in}}{\pgfqpoint{4.650000in}{0.614151in}}%
\pgfusepath{clip}%
\pgfsetbuttcap%
\pgfsetroundjoin%
\definecolor{currentfill}{rgb}{0.987266,0.804198,0.639170}%
\pgfsetfillcolor{currentfill}%
\pgfsetlinewidth{0.250937pt}%
\definecolor{currentstroke}{rgb}{1.000000,1.000000,1.000000}%
\pgfsetstrokecolor{currentstroke}%
\pgfsetdash{}{0pt}%
\pgfpathmoveto{\pgfqpoint{4.943207in}{2.698868in}}%
\pgfpathlineto{\pgfqpoint{5.030943in}{2.698868in}}%
\pgfpathlineto{\pgfqpoint{5.030943in}{2.611132in}}%
\pgfpathlineto{\pgfqpoint{4.943207in}{2.611132in}}%
\pgfpathlineto{\pgfqpoint{4.943207in}{2.698868in}}%
\pgfusepath{stroke,fill}%
\end{pgfscope}%
\begin{pgfscope}%
\pgfpathrectangle{\pgfqpoint{0.380943in}{2.260189in}}{\pgfqpoint{4.650000in}{0.614151in}}%
\pgfusepath{clip}%
\pgfsetbuttcap%
\pgfsetroundjoin%
\definecolor{currentfill}{rgb}{0.961738,0.927612,0.725598}%
\pgfsetfillcolor{currentfill}%
\pgfsetlinewidth{0.250937pt}%
\definecolor{currentstroke}{rgb}{1.000000,1.000000,1.000000}%
\pgfsetstrokecolor{currentstroke}%
\pgfsetdash{}{0pt}%
\pgfpathmoveto{\pgfqpoint{0.380943in}{2.611132in}}%
\pgfpathlineto{\pgfqpoint{0.468679in}{2.611132in}}%
\pgfpathlineto{\pgfqpoint{0.468679in}{2.523396in}}%
\pgfpathlineto{\pgfqpoint{0.380943in}{2.523396in}}%
\pgfpathlineto{\pgfqpoint{0.380943in}{2.611132in}}%
\pgfusepath{stroke,fill}%
\end{pgfscope}%
\begin{pgfscope}%
\pgfpathrectangle{\pgfqpoint{0.380943in}{2.260189in}}{\pgfqpoint{4.650000in}{0.614151in}}%
\pgfusepath{clip}%
\pgfsetbuttcap%
\pgfsetroundjoin%
\definecolor{currentfill}{rgb}{0.987266,0.804198,0.639170}%
\pgfsetfillcolor{currentfill}%
\pgfsetlinewidth{0.250937pt}%
\definecolor{currentstroke}{rgb}{1.000000,1.000000,1.000000}%
\pgfsetstrokecolor{currentstroke}%
\pgfsetdash{}{0pt}%
\pgfpathmoveto{\pgfqpoint{0.468679in}{2.611132in}}%
\pgfpathlineto{\pgfqpoint{0.556415in}{2.611132in}}%
\pgfpathlineto{\pgfqpoint{0.556415in}{2.523396in}}%
\pgfpathlineto{\pgfqpoint{0.468679in}{2.523396in}}%
\pgfpathlineto{\pgfqpoint{0.468679in}{2.611132in}}%
\pgfusepath{stroke,fill}%
\end{pgfscope}%
\begin{pgfscope}%
\pgfpathrectangle{\pgfqpoint{0.380943in}{2.260189in}}{\pgfqpoint{4.650000in}{0.614151in}}%
\pgfusepath{clip}%
\pgfsetbuttcap%
\pgfsetroundjoin%
\definecolor{currentfill}{rgb}{0.974072,0.862976,0.688750}%
\pgfsetfillcolor{currentfill}%
\pgfsetlinewidth{0.250937pt}%
\definecolor{currentstroke}{rgb}{1.000000,1.000000,1.000000}%
\pgfsetstrokecolor{currentstroke}%
\pgfsetdash{}{0pt}%
\pgfpathmoveto{\pgfqpoint{0.556415in}{2.611132in}}%
\pgfpathlineto{\pgfqpoint{0.644151in}{2.611132in}}%
\pgfpathlineto{\pgfqpoint{0.644151in}{2.523396in}}%
\pgfpathlineto{\pgfqpoint{0.556415in}{2.523396in}}%
\pgfpathlineto{\pgfqpoint{0.556415in}{2.611132in}}%
\pgfusepath{stroke,fill}%
\end{pgfscope}%
\begin{pgfscope}%
\pgfpathrectangle{\pgfqpoint{0.380943in}{2.260189in}}{\pgfqpoint{4.650000in}{0.614151in}}%
\pgfusepath{clip}%
\pgfsetbuttcap%
\pgfsetroundjoin%
\definecolor{currentfill}{rgb}{0.987266,0.804198,0.639170}%
\pgfsetfillcolor{currentfill}%
\pgfsetlinewidth{0.250937pt}%
\definecolor{currentstroke}{rgb}{1.000000,1.000000,1.000000}%
\pgfsetstrokecolor{currentstroke}%
\pgfsetdash{}{0pt}%
\pgfpathmoveto{\pgfqpoint{0.644151in}{2.611132in}}%
\pgfpathlineto{\pgfqpoint{0.731886in}{2.611132in}}%
\pgfpathlineto{\pgfqpoint{0.731886in}{2.523396in}}%
\pgfpathlineto{\pgfqpoint{0.644151in}{2.523396in}}%
\pgfpathlineto{\pgfqpoint{0.644151in}{2.611132in}}%
\pgfusepath{stroke,fill}%
\end{pgfscope}%
\begin{pgfscope}%
\pgfpathrectangle{\pgfqpoint{0.380943in}{2.260189in}}{\pgfqpoint{4.650000in}{0.614151in}}%
\pgfusepath{clip}%
\pgfsetbuttcap%
\pgfsetroundjoin%
\definecolor{currentfill}{rgb}{0.982699,0.823991,0.657439}%
\pgfsetfillcolor{currentfill}%
\pgfsetlinewidth{0.250937pt}%
\definecolor{currentstroke}{rgb}{1.000000,1.000000,1.000000}%
\pgfsetstrokecolor{currentstroke}%
\pgfsetdash{}{0pt}%
\pgfpathmoveto{\pgfqpoint{0.731886in}{2.611132in}}%
\pgfpathlineto{\pgfqpoint{0.819622in}{2.611132in}}%
\pgfpathlineto{\pgfqpoint{0.819622in}{2.523396in}}%
\pgfpathlineto{\pgfqpoint{0.731886in}{2.523396in}}%
\pgfpathlineto{\pgfqpoint{0.731886in}{2.611132in}}%
\pgfusepath{stroke,fill}%
\end{pgfscope}%
\begin{pgfscope}%
\pgfpathrectangle{\pgfqpoint{0.380943in}{2.260189in}}{\pgfqpoint{4.650000in}{0.614151in}}%
\pgfusepath{clip}%
\pgfsetbuttcap%
\pgfsetroundjoin%
\definecolor{currentfill}{rgb}{0.987266,0.804198,0.639170}%
\pgfsetfillcolor{currentfill}%
\pgfsetlinewidth{0.250937pt}%
\definecolor{currentstroke}{rgb}{1.000000,1.000000,1.000000}%
\pgfsetstrokecolor{currentstroke}%
\pgfsetdash{}{0pt}%
\pgfpathmoveto{\pgfqpoint{0.819622in}{2.611132in}}%
\pgfpathlineto{\pgfqpoint{0.907358in}{2.611132in}}%
\pgfpathlineto{\pgfqpoint{0.907358in}{2.523396in}}%
\pgfpathlineto{\pgfqpoint{0.819622in}{2.523396in}}%
\pgfpathlineto{\pgfqpoint{0.819622in}{2.611132in}}%
\pgfusepath{stroke,fill}%
\end{pgfscope}%
\begin{pgfscope}%
\pgfpathrectangle{\pgfqpoint{0.380943in}{2.260189in}}{\pgfqpoint{4.650000in}{0.614151in}}%
\pgfusepath{clip}%
\pgfsetbuttcap%
\pgfsetroundjoin%
\definecolor{currentfill}{rgb}{0.987266,0.804198,0.639170}%
\pgfsetfillcolor{currentfill}%
\pgfsetlinewidth{0.250937pt}%
\definecolor{currentstroke}{rgb}{1.000000,1.000000,1.000000}%
\pgfsetstrokecolor{currentstroke}%
\pgfsetdash{}{0pt}%
\pgfpathmoveto{\pgfqpoint{0.907358in}{2.611132in}}%
\pgfpathlineto{\pgfqpoint{0.995094in}{2.611132in}}%
\pgfpathlineto{\pgfqpoint{0.995094in}{2.523396in}}%
\pgfpathlineto{\pgfqpoint{0.907358in}{2.523396in}}%
\pgfpathlineto{\pgfqpoint{0.907358in}{2.611132in}}%
\pgfusepath{stroke,fill}%
\end{pgfscope}%
\begin{pgfscope}%
\pgfpathrectangle{\pgfqpoint{0.380943in}{2.260189in}}{\pgfqpoint{4.650000in}{0.614151in}}%
\pgfusepath{clip}%
\pgfsetbuttcap%
\pgfsetroundjoin%
\definecolor{currentfill}{rgb}{0.982699,0.823991,0.657439}%
\pgfsetfillcolor{currentfill}%
\pgfsetlinewidth{0.250937pt}%
\definecolor{currentstroke}{rgb}{1.000000,1.000000,1.000000}%
\pgfsetstrokecolor{currentstroke}%
\pgfsetdash{}{0pt}%
\pgfpathmoveto{\pgfqpoint{0.995094in}{2.611132in}}%
\pgfpathlineto{\pgfqpoint{1.082830in}{2.611132in}}%
\pgfpathlineto{\pgfqpoint{1.082830in}{2.523396in}}%
\pgfpathlineto{\pgfqpoint{0.995094in}{2.523396in}}%
\pgfpathlineto{\pgfqpoint{0.995094in}{2.611132in}}%
\pgfusepath{stroke,fill}%
\end{pgfscope}%
\begin{pgfscope}%
\pgfpathrectangle{\pgfqpoint{0.380943in}{2.260189in}}{\pgfqpoint{4.650000in}{0.614151in}}%
\pgfusepath{clip}%
\pgfsetbuttcap%
\pgfsetroundjoin%
\definecolor{currentfill}{rgb}{0.978639,0.841584,0.673679}%
\pgfsetfillcolor{currentfill}%
\pgfsetlinewidth{0.250937pt}%
\definecolor{currentstroke}{rgb}{1.000000,1.000000,1.000000}%
\pgfsetstrokecolor{currentstroke}%
\pgfsetdash{}{0pt}%
\pgfpathmoveto{\pgfqpoint{1.082830in}{2.611132in}}%
\pgfpathlineto{\pgfqpoint{1.170566in}{2.611132in}}%
\pgfpathlineto{\pgfqpoint{1.170566in}{2.523396in}}%
\pgfpathlineto{\pgfqpoint{1.082830in}{2.523396in}}%
\pgfpathlineto{\pgfqpoint{1.082830in}{2.611132in}}%
\pgfusepath{stroke,fill}%
\end{pgfscope}%
\begin{pgfscope}%
\pgfpathrectangle{\pgfqpoint{0.380943in}{2.260189in}}{\pgfqpoint{4.650000in}{0.614151in}}%
\pgfusepath{clip}%
\pgfsetbuttcap%
\pgfsetroundjoin%
\definecolor{currentfill}{rgb}{0.961738,0.927612,0.725598}%
\pgfsetfillcolor{currentfill}%
\pgfsetlinewidth{0.250937pt}%
\definecolor{currentstroke}{rgb}{1.000000,1.000000,1.000000}%
\pgfsetstrokecolor{currentstroke}%
\pgfsetdash{}{0pt}%
\pgfpathmoveto{\pgfqpoint{1.170566in}{2.611132in}}%
\pgfpathlineto{\pgfqpoint{1.258302in}{2.611132in}}%
\pgfpathlineto{\pgfqpoint{1.258302in}{2.523396in}}%
\pgfpathlineto{\pgfqpoint{1.170566in}{2.523396in}}%
\pgfpathlineto{\pgfqpoint{1.170566in}{2.611132in}}%
\pgfusepath{stroke,fill}%
\end{pgfscope}%
\begin{pgfscope}%
\pgfpathrectangle{\pgfqpoint{0.380943in}{2.260189in}}{\pgfqpoint{4.650000in}{0.614151in}}%
\pgfusepath{clip}%
\pgfsetbuttcap%
\pgfsetroundjoin%
\definecolor{currentfill}{rgb}{1.000000,1.000000,0.857516}%
\pgfsetfillcolor{currentfill}%
\pgfsetlinewidth{0.250937pt}%
\definecolor{currentstroke}{rgb}{1.000000,1.000000,1.000000}%
\pgfsetstrokecolor{currentstroke}%
\pgfsetdash{}{0pt}%
\pgfpathmoveto{\pgfqpoint{1.258302in}{2.611132in}}%
\pgfpathlineto{\pgfqpoint{1.346037in}{2.611132in}}%
\pgfpathlineto{\pgfqpoint{1.346037in}{2.523396in}}%
\pgfpathlineto{\pgfqpoint{1.258302in}{2.523396in}}%
\pgfpathlineto{\pgfqpoint{1.258302in}{2.611132in}}%
\pgfusepath{stroke,fill}%
\end{pgfscope}%
\begin{pgfscope}%
\pgfpathrectangle{\pgfqpoint{0.380943in}{2.260189in}}{\pgfqpoint{4.650000in}{0.614151in}}%
\pgfusepath{clip}%
\pgfsetbuttcap%
\pgfsetroundjoin%
\definecolor{currentfill}{rgb}{0.980008,0.966013,0.779393}%
\pgfsetfillcolor{currentfill}%
\pgfsetlinewidth{0.250937pt}%
\definecolor{currentstroke}{rgb}{1.000000,1.000000,1.000000}%
\pgfsetstrokecolor{currentstroke}%
\pgfsetdash{}{0pt}%
\pgfpathmoveto{\pgfqpoint{1.346037in}{2.611132in}}%
\pgfpathlineto{\pgfqpoint{1.433773in}{2.611132in}}%
\pgfpathlineto{\pgfqpoint{1.433773in}{2.523396in}}%
\pgfpathlineto{\pgfqpoint{1.346037in}{2.523396in}}%
\pgfpathlineto{\pgfqpoint{1.346037in}{2.611132in}}%
\pgfusepath{stroke,fill}%
\end{pgfscope}%
\begin{pgfscope}%
\pgfpathrectangle{\pgfqpoint{0.380943in}{2.260189in}}{\pgfqpoint{4.650000in}{0.614151in}}%
\pgfusepath{clip}%
\pgfsetbuttcap%
\pgfsetroundjoin%
\definecolor{currentfill}{rgb}{1.000000,1.000000,0.929412}%
\pgfsetfillcolor{currentfill}%
\pgfsetlinewidth{0.250937pt}%
\definecolor{currentstroke}{rgb}{1.000000,1.000000,1.000000}%
\pgfsetstrokecolor{currentstroke}%
\pgfsetdash{}{0pt}%
\pgfpathmoveto{\pgfqpoint{1.433773in}{2.611132in}}%
\pgfpathlineto{\pgfqpoint{1.521509in}{2.611132in}}%
\pgfpathlineto{\pgfqpoint{1.521509in}{2.523396in}}%
\pgfpathlineto{\pgfqpoint{1.433773in}{2.523396in}}%
\pgfpathlineto{\pgfqpoint{1.433773in}{2.611132in}}%
\pgfusepath{stroke,fill}%
\end{pgfscope}%
\begin{pgfscope}%
\pgfpathrectangle{\pgfqpoint{0.380943in}{2.260189in}}{\pgfqpoint{4.650000in}{0.614151in}}%
\pgfusepath{clip}%
\pgfsetbuttcap%
\pgfsetroundjoin%
\definecolor{currentfill}{rgb}{0.961738,0.927612,0.725598}%
\pgfsetfillcolor{currentfill}%
\pgfsetlinewidth{0.250937pt}%
\definecolor{currentstroke}{rgb}{1.000000,1.000000,1.000000}%
\pgfsetstrokecolor{currentstroke}%
\pgfsetdash{}{0pt}%
\pgfpathmoveto{\pgfqpoint{1.521509in}{2.611132in}}%
\pgfpathlineto{\pgfqpoint{1.609245in}{2.611132in}}%
\pgfpathlineto{\pgfqpoint{1.609245in}{2.523396in}}%
\pgfpathlineto{\pgfqpoint{1.521509in}{2.523396in}}%
\pgfpathlineto{\pgfqpoint{1.521509in}{2.611132in}}%
\pgfusepath{stroke,fill}%
\end{pgfscope}%
\begin{pgfscope}%
\pgfpathrectangle{\pgfqpoint{0.380943in}{2.260189in}}{\pgfqpoint{4.650000in}{0.614151in}}%
\pgfusepath{clip}%
\pgfsetbuttcap%
\pgfsetroundjoin%
\definecolor{currentfill}{rgb}{0.964783,0.940131,0.739808}%
\pgfsetfillcolor{currentfill}%
\pgfsetlinewidth{0.250937pt}%
\definecolor{currentstroke}{rgb}{1.000000,1.000000,1.000000}%
\pgfsetstrokecolor{currentstroke}%
\pgfsetdash{}{0pt}%
\pgfpathmoveto{\pgfqpoint{1.609245in}{2.611132in}}%
\pgfpathlineto{\pgfqpoint{1.696981in}{2.611132in}}%
\pgfpathlineto{\pgfqpoint{1.696981in}{2.523396in}}%
\pgfpathlineto{\pgfqpoint{1.609245in}{2.523396in}}%
\pgfpathlineto{\pgfqpoint{1.609245in}{2.611132in}}%
\pgfusepath{stroke,fill}%
\end{pgfscope}%
\begin{pgfscope}%
\pgfpathrectangle{\pgfqpoint{0.380943in}{2.260189in}}{\pgfqpoint{4.650000in}{0.614151in}}%
\pgfusepath{clip}%
\pgfsetbuttcap%
\pgfsetroundjoin%
\definecolor{currentfill}{rgb}{1.000000,1.000000,0.895579}%
\pgfsetfillcolor{currentfill}%
\pgfsetlinewidth{0.250937pt}%
\definecolor{currentstroke}{rgb}{1.000000,1.000000,1.000000}%
\pgfsetstrokecolor{currentstroke}%
\pgfsetdash{}{0pt}%
\pgfpathmoveto{\pgfqpoint{1.696981in}{2.611132in}}%
\pgfpathlineto{\pgfqpoint{1.784717in}{2.611132in}}%
\pgfpathlineto{\pgfqpoint{1.784717in}{2.523396in}}%
\pgfpathlineto{\pgfqpoint{1.696981in}{2.523396in}}%
\pgfpathlineto{\pgfqpoint{1.696981in}{2.611132in}}%
\pgfusepath{stroke,fill}%
\end{pgfscope}%
\begin{pgfscope}%
\pgfpathrectangle{\pgfqpoint{0.380943in}{2.260189in}}{\pgfqpoint{4.650000in}{0.614151in}}%
\pgfusepath{clip}%
\pgfsetbuttcap%
\pgfsetroundjoin%
\definecolor{currentfill}{rgb}{0.964937,0.908651,0.713110}%
\pgfsetfillcolor{currentfill}%
\pgfsetlinewidth{0.250937pt}%
\definecolor{currentstroke}{rgb}{1.000000,1.000000,1.000000}%
\pgfsetstrokecolor{currentstroke}%
\pgfsetdash{}{0pt}%
\pgfpathmoveto{\pgfqpoint{1.784717in}{2.611132in}}%
\pgfpathlineto{\pgfqpoint{1.872452in}{2.611132in}}%
\pgfpathlineto{\pgfqpoint{1.872452in}{2.523396in}}%
\pgfpathlineto{\pgfqpoint{1.784717in}{2.523396in}}%
\pgfpathlineto{\pgfqpoint{1.784717in}{2.611132in}}%
\pgfusepath{stroke,fill}%
\end{pgfscope}%
\begin{pgfscope}%
\pgfpathrectangle{\pgfqpoint{0.380943in}{2.260189in}}{\pgfqpoint{4.650000in}{0.614151in}}%
\pgfusepath{clip}%
\pgfsetbuttcap%
\pgfsetroundjoin%
\definecolor{currentfill}{rgb}{0.995233,0.991895,0.818977}%
\pgfsetfillcolor{currentfill}%
\pgfsetlinewidth{0.250937pt}%
\definecolor{currentstroke}{rgb}{1.000000,1.000000,1.000000}%
\pgfsetstrokecolor{currentstroke}%
\pgfsetdash{}{0pt}%
\pgfpathmoveto{\pgfqpoint{1.872452in}{2.611132in}}%
\pgfpathlineto{\pgfqpoint{1.960188in}{2.611132in}}%
\pgfpathlineto{\pgfqpoint{1.960188in}{2.523396in}}%
\pgfpathlineto{\pgfqpoint{1.872452in}{2.523396in}}%
\pgfpathlineto{\pgfqpoint{1.872452in}{2.611132in}}%
\pgfusepath{stroke,fill}%
\end{pgfscope}%
\begin{pgfscope}%
\pgfpathrectangle{\pgfqpoint{0.380943in}{2.260189in}}{\pgfqpoint{4.650000in}{0.614151in}}%
\pgfusepath{clip}%
\pgfsetbuttcap%
\pgfsetroundjoin%
\definecolor{currentfill}{rgb}{0.961738,0.927612,0.725598}%
\pgfsetfillcolor{currentfill}%
\pgfsetlinewidth{0.250937pt}%
\definecolor{currentstroke}{rgb}{1.000000,1.000000,1.000000}%
\pgfsetstrokecolor{currentstroke}%
\pgfsetdash{}{0pt}%
\pgfpathmoveto{\pgfqpoint{1.960188in}{2.611132in}}%
\pgfpathlineto{\pgfqpoint{2.047924in}{2.611132in}}%
\pgfpathlineto{\pgfqpoint{2.047924in}{2.523396in}}%
\pgfpathlineto{\pgfqpoint{1.960188in}{2.523396in}}%
\pgfpathlineto{\pgfqpoint{1.960188in}{2.611132in}}%
\pgfusepath{stroke,fill}%
\end{pgfscope}%
\begin{pgfscope}%
\pgfpathrectangle{\pgfqpoint{0.380943in}{2.260189in}}{\pgfqpoint{4.650000in}{0.614151in}}%
\pgfusepath{clip}%
\pgfsetbuttcap%
\pgfsetroundjoin%
\definecolor{currentfill}{rgb}{0.995233,0.991895,0.818977}%
\pgfsetfillcolor{currentfill}%
\pgfsetlinewidth{0.250937pt}%
\definecolor{currentstroke}{rgb}{1.000000,1.000000,1.000000}%
\pgfsetstrokecolor{currentstroke}%
\pgfsetdash{}{0pt}%
\pgfpathmoveto{\pgfqpoint{2.047924in}{2.611132in}}%
\pgfpathlineto{\pgfqpoint{2.135660in}{2.611132in}}%
\pgfpathlineto{\pgfqpoint{2.135660in}{2.523396in}}%
\pgfpathlineto{\pgfqpoint{2.047924in}{2.523396in}}%
\pgfpathlineto{\pgfqpoint{2.047924in}{2.611132in}}%
\pgfusepath{stroke,fill}%
\end{pgfscope}%
\begin{pgfscope}%
\pgfpathrectangle{\pgfqpoint{0.380943in}{2.260189in}}{\pgfqpoint{4.650000in}{0.614151in}}%
\pgfusepath{clip}%
\pgfsetbuttcap%
\pgfsetroundjoin%
\definecolor{currentfill}{rgb}{0.963260,0.918478,0.719508}%
\pgfsetfillcolor{currentfill}%
\pgfsetlinewidth{0.250937pt}%
\definecolor{currentstroke}{rgb}{1.000000,1.000000,1.000000}%
\pgfsetstrokecolor{currentstroke}%
\pgfsetdash{}{0pt}%
\pgfpathmoveto{\pgfqpoint{2.135660in}{2.611132in}}%
\pgfpathlineto{\pgfqpoint{2.223396in}{2.611132in}}%
\pgfpathlineto{\pgfqpoint{2.223396in}{2.523396in}}%
\pgfpathlineto{\pgfqpoint{2.135660in}{2.523396in}}%
\pgfpathlineto{\pgfqpoint{2.135660in}{2.611132in}}%
\pgfusepath{stroke,fill}%
\end{pgfscope}%
\begin{pgfscope}%
\pgfpathrectangle{\pgfqpoint{0.380943in}{2.260189in}}{\pgfqpoint{4.650000in}{0.614151in}}%
\pgfusepath{clip}%
\pgfsetbuttcap%
\pgfsetroundjoin%
\definecolor{currentfill}{rgb}{0.963260,0.918478,0.719508}%
\pgfsetfillcolor{currentfill}%
\pgfsetlinewidth{0.250937pt}%
\definecolor{currentstroke}{rgb}{1.000000,1.000000,1.000000}%
\pgfsetstrokecolor{currentstroke}%
\pgfsetdash{}{0pt}%
\pgfpathmoveto{\pgfqpoint{2.223396in}{2.611132in}}%
\pgfpathlineto{\pgfqpoint{2.311132in}{2.611132in}}%
\pgfpathlineto{\pgfqpoint{2.311132in}{2.523396in}}%
\pgfpathlineto{\pgfqpoint{2.223396in}{2.523396in}}%
\pgfpathlineto{\pgfqpoint{2.223396in}{2.611132in}}%
\pgfusepath{stroke,fill}%
\end{pgfscope}%
\begin{pgfscope}%
\pgfpathrectangle{\pgfqpoint{0.380943in}{2.260189in}}{\pgfqpoint{4.650000in}{0.614151in}}%
\pgfusepath{clip}%
\pgfsetbuttcap%
\pgfsetroundjoin%
\definecolor{currentfill}{rgb}{0.978639,0.841584,0.673679}%
\pgfsetfillcolor{currentfill}%
\pgfsetlinewidth{0.250937pt}%
\definecolor{currentstroke}{rgb}{1.000000,1.000000,1.000000}%
\pgfsetstrokecolor{currentstroke}%
\pgfsetdash{}{0pt}%
\pgfpathmoveto{\pgfqpoint{2.311132in}{2.611132in}}%
\pgfpathlineto{\pgfqpoint{2.398868in}{2.611132in}}%
\pgfpathlineto{\pgfqpoint{2.398868in}{2.523396in}}%
\pgfpathlineto{\pgfqpoint{2.311132in}{2.523396in}}%
\pgfpathlineto{\pgfqpoint{2.311132in}{2.611132in}}%
\pgfusepath{stroke,fill}%
\end{pgfscope}%
\begin{pgfscope}%
\pgfpathrectangle{\pgfqpoint{0.380943in}{2.260189in}}{\pgfqpoint{4.650000in}{0.614151in}}%
\pgfusepath{clip}%
\pgfsetbuttcap%
\pgfsetroundjoin%
\definecolor{currentfill}{rgb}{0.974072,0.862976,0.688750}%
\pgfsetfillcolor{currentfill}%
\pgfsetlinewidth{0.250937pt}%
\definecolor{currentstroke}{rgb}{1.000000,1.000000,1.000000}%
\pgfsetstrokecolor{currentstroke}%
\pgfsetdash{}{0pt}%
\pgfpathmoveto{\pgfqpoint{2.398868in}{2.611132in}}%
\pgfpathlineto{\pgfqpoint{2.486603in}{2.611132in}}%
\pgfpathlineto{\pgfqpoint{2.486603in}{2.523396in}}%
\pgfpathlineto{\pgfqpoint{2.398868in}{2.523396in}}%
\pgfpathlineto{\pgfqpoint{2.398868in}{2.611132in}}%
\pgfusepath{stroke,fill}%
\end{pgfscope}%
\begin{pgfscope}%
\pgfpathrectangle{\pgfqpoint{0.380943in}{2.260189in}}{\pgfqpoint{4.650000in}{0.614151in}}%
\pgfusepath{clip}%
\pgfsetbuttcap%
\pgfsetroundjoin%
\definecolor{currentfill}{rgb}{0.996401,0.724937,0.591557}%
\pgfsetfillcolor{currentfill}%
\pgfsetlinewidth{0.250937pt}%
\definecolor{currentstroke}{rgb}{1.000000,1.000000,1.000000}%
\pgfsetstrokecolor{currentstroke}%
\pgfsetdash{}{0pt}%
\pgfpathmoveto{\pgfqpoint{2.486603in}{2.611132in}}%
\pgfpathlineto{\pgfqpoint{2.574339in}{2.611132in}}%
\pgfpathlineto{\pgfqpoint{2.574339in}{2.523396in}}%
\pgfpathlineto{\pgfqpoint{2.486603in}{2.523396in}}%
\pgfpathlineto{\pgfqpoint{2.486603in}{2.611132in}}%
\pgfusepath{stroke,fill}%
\end{pgfscope}%
\begin{pgfscope}%
\pgfpathrectangle{\pgfqpoint{0.380943in}{2.260189in}}{\pgfqpoint{4.650000in}{0.614151in}}%
\pgfusepath{clip}%
\pgfsetbuttcap%
\pgfsetroundjoin%
\definecolor{currentfill}{rgb}{0.997924,0.685352,0.570242}%
\pgfsetfillcolor{currentfill}%
\pgfsetlinewidth{0.250937pt}%
\definecolor{currentstroke}{rgb}{1.000000,1.000000,1.000000}%
\pgfsetstrokecolor{currentstroke}%
\pgfsetdash{}{0pt}%
\pgfpathmoveto{\pgfqpoint{2.574339in}{2.611132in}}%
\pgfpathlineto{\pgfqpoint{2.662075in}{2.611132in}}%
\pgfpathlineto{\pgfqpoint{2.662075in}{2.523396in}}%
\pgfpathlineto{\pgfqpoint{2.574339in}{2.523396in}}%
\pgfpathlineto{\pgfqpoint{2.574339in}{2.611132in}}%
\pgfusepath{stroke,fill}%
\end{pgfscope}%
\begin{pgfscope}%
\pgfpathrectangle{\pgfqpoint{0.380943in}{2.260189in}}{\pgfqpoint{4.650000in}{0.614151in}}%
\pgfusepath{clip}%
\pgfsetbuttcap%
\pgfsetroundjoin%
\definecolor{currentfill}{rgb}{0.987266,0.804198,0.639170}%
\pgfsetfillcolor{currentfill}%
\pgfsetlinewidth{0.250937pt}%
\definecolor{currentstroke}{rgb}{1.000000,1.000000,1.000000}%
\pgfsetstrokecolor{currentstroke}%
\pgfsetdash{}{0pt}%
\pgfpathmoveto{\pgfqpoint{2.662075in}{2.611132in}}%
\pgfpathlineto{\pgfqpoint{2.749811in}{2.611132in}}%
\pgfpathlineto{\pgfqpoint{2.749811in}{2.523396in}}%
\pgfpathlineto{\pgfqpoint{2.662075in}{2.523396in}}%
\pgfpathlineto{\pgfqpoint{2.662075in}{2.611132in}}%
\pgfusepath{stroke,fill}%
\end{pgfscope}%
\begin{pgfscope}%
\pgfpathrectangle{\pgfqpoint{0.380943in}{2.260189in}}{\pgfqpoint{4.650000in}{0.614151in}}%
\pgfusepath{clip}%
\pgfsetbuttcap%
\pgfsetroundjoin%
\definecolor{currentfill}{rgb}{0.990634,0.779608,0.623299}%
\pgfsetfillcolor{currentfill}%
\pgfsetlinewidth{0.250937pt}%
\definecolor{currentstroke}{rgb}{1.000000,1.000000,1.000000}%
\pgfsetstrokecolor{currentstroke}%
\pgfsetdash{}{0pt}%
\pgfpathmoveto{\pgfqpoint{2.749811in}{2.611132in}}%
\pgfpathlineto{\pgfqpoint{2.837547in}{2.611132in}}%
\pgfpathlineto{\pgfqpoint{2.837547in}{2.523396in}}%
\pgfpathlineto{\pgfqpoint{2.749811in}{2.523396in}}%
\pgfpathlineto{\pgfqpoint{2.749811in}{2.611132in}}%
\pgfusepath{stroke,fill}%
\end{pgfscope}%
\begin{pgfscope}%
\pgfpathrectangle{\pgfqpoint{0.380943in}{2.260189in}}{\pgfqpoint{4.650000in}{0.614151in}}%
\pgfusepath{clip}%
\pgfsetbuttcap%
\pgfsetroundjoin%
\definecolor{currentfill}{rgb}{0.993679,0.753725,0.608074}%
\pgfsetfillcolor{currentfill}%
\pgfsetlinewidth{0.250937pt}%
\definecolor{currentstroke}{rgb}{1.000000,1.000000,1.000000}%
\pgfsetstrokecolor{currentstroke}%
\pgfsetdash{}{0pt}%
\pgfpathmoveto{\pgfqpoint{2.837547in}{2.611132in}}%
\pgfpathlineto{\pgfqpoint{2.925283in}{2.611132in}}%
\pgfpathlineto{\pgfqpoint{2.925283in}{2.523396in}}%
\pgfpathlineto{\pgfqpoint{2.837547in}{2.523396in}}%
\pgfpathlineto{\pgfqpoint{2.837547in}{2.611132in}}%
\pgfusepath{stroke,fill}%
\end{pgfscope}%
\begin{pgfscope}%
\pgfpathrectangle{\pgfqpoint{0.380943in}{2.260189in}}{\pgfqpoint{4.650000in}{0.614151in}}%
\pgfusepath{clip}%
\pgfsetbuttcap%
\pgfsetroundjoin%
\definecolor{currentfill}{rgb}{0.999277,0.650165,0.551296}%
\pgfsetfillcolor{currentfill}%
\pgfsetlinewidth{0.250937pt}%
\definecolor{currentstroke}{rgb}{1.000000,1.000000,1.000000}%
\pgfsetstrokecolor{currentstroke}%
\pgfsetdash{}{0pt}%
\pgfpathmoveto{\pgfqpoint{2.925283in}{2.611132in}}%
\pgfpathlineto{\pgfqpoint{3.013019in}{2.611132in}}%
\pgfpathlineto{\pgfqpoint{3.013019in}{2.523396in}}%
\pgfpathlineto{\pgfqpoint{2.925283in}{2.523396in}}%
\pgfpathlineto{\pgfqpoint{2.925283in}{2.611132in}}%
\pgfusepath{stroke,fill}%
\end{pgfscope}%
\begin{pgfscope}%
\pgfpathrectangle{\pgfqpoint{0.380943in}{2.260189in}}{\pgfqpoint{4.650000in}{0.614151in}}%
\pgfusepath{clip}%
\pgfsetbuttcap%
\pgfsetroundjoin%
\definecolor{currentfill}{rgb}{1.000000,0.615379,0.534779}%
\pgfsetfillcolor{currentfill}%
\pgfsetlinewidth{0.250937pt}%
\definecolor{currentstroke}{rgb}{1.000000,1.000000,1.000000}%
\pgfsetstrokecolor{currentstroke}%
\pgfsetdash{}{0pt}%
\pgfpathmoveto{\pgfqpoint{3.013019in}{2.611132in}}%
\pgfpathlineto{\pgfqpoint{3.100754in}{2.611132in}}%
\pgfpathlineto{\pgfqpoint{3.100754in}{2.523396in}}%
\pgfpathlineto{\pgfqpoint{3.013019in}{2.523396in}}%
\pgfpathlineto{\pgfqpoint{3.013019in}{2.611132in}}%
\pgfusepath{stroke,fill}%
\end{pgfscope}%
\begin{pgfscope}%
\pgfpathrectangle{\pgfqpoint{0.380943in}{2.260189in}}{\pgfqpoint{4.650000in}{0.614151in}}%
\pgfusepath{clip}%
\pgfsetbuttcap%
\pgfsetroundjoin%
\definecolor{currentfill}{rgb}{0.993679,0.753725,0.608074}%
\pgfsetfillcolor{currentfill}%
\pgfsetlinewidth{0.250937pt}%
\definecolor{currentstroke}{rgb}{1.000000,1.000000,1.000000}%
\pgfsetstrokecolor{currentstroke}%
\pgfsetdash{}{0pt}%
\pgfpathmoveto{\pgfqpoint{3.100754in}{2.611132in}}%
\pgfpathlineto{\pgfqpoint{3.188490in}{2.611132in}}%
\pgfpathlineto{\pgfqpoint{3.188490in}{2.523396in}}%
\pgfpathlineto{\pgfqpoint{3.100754in}{2.523396in}}%
\pgfpathlineto{\pgfqpoint{3.100754in}{2.611132in}}%
\pgfusepath{stroke,fill}%
\end{pgfscope}%
\begin{pgfscope}%
\pgfpathrectangle{\pgfqpoint{0.380943in}{2.260189in}}{\pgfqpoint{4.650000in}{0.614151in}}%
\pgfusepath{clip}%
\pgfsetbuttcap%
\pgfsetroundjoin%
\definecolor{currentfill}{rgb}{0.990634,0.779608,0.623299}%
\pgfsetfillcolor{currentfill}%
\pgfsetlinewidth{0.250937pt}%
\definecolor{currentstroke}{rgb}{1.000000,1.000000,1.000000}%
\pgfsetstrokecolor{currentstroke}%
\pgfsetdash{}{0pt}%
\pgfpathmoveto{\pgfqpoint{3.188490in}{2.611132in}}%
\pgfpathlineto{\pgfqpoint{3.276226in}{2.611132in}}%
\pgfpathlineto{\pgfqpoint{3.276226in}{2.523396in}}%
\pgfpathlineto{\pgfqpoint{3.188490in}{2.523396in}}%
\pgfpathlineto{\pgfqpoint{3.188490in}{2.611132in}}%
\pgfusepath{stroke,fill}%
\end{pgfscope}%
\begin{pgfscope}%
\pgfpathrectangle{\pgfqpoint{0.380943in}{2.260189in}}{\pgfqpoint{4.650000in}{0.614151in}}%
\pgfusepath{clip}%
\pgfsetbuttcap%
\pgfsetroundjoin%
\definecolor{currentfill}{rgb}{0.990634,0.779608,0.623299}%
\pgfsetfillcolor{currentfill}%
\pgfsetlinewidth{0.250937pt}%
\definecolor{currentstroke}{rgb}{1.000000,1.000000,1.000000}%
\pgfsetstrokecolor{currentstroke}%
\pgfsetdash{}{0pt}%
\pgfpathmoveto{\pgfqpoint{3.276226in}{2.611132in}}%
\pgfpathlineto{\pgfqpoint{3.363962in}{2.611132in}}%
\pgfpathlineto{\pgfqpoint{3.363962in}{2.523396in}}%
\pgfpathlineto{\pgfqpoint{3.276226in}{2.523396in}}%
\pgfpathlineto{\pgfqpoint{3.276226in}{2.611132in}}%
\pgfusepath{stroke,fill}%
\end{pgfscope}%
\begin{pgfscope}%
\pgfpathrectangle{\pgfqpoint{0.380943in}{2.260189in}}{\pgfqpoint{4.650000in}{0.614151in}}%
\pgfusepath{clip}%
\pgfsetbuttcap%
\pgfsetroundjoin%
\definecolor{currentfill}{rgb}{0.978639,0.841584,0.673679}%
\pgfsetfillcolor{currentfill}%
\pgfsetlinewidth{0.250937pt}%
\definecolor{currentstroke}{rgb}{1.000000,1.000000,1.000000}%
\pgfsetstrokecolor{currentstroke}%
\pgfsetdash{}{0pt}%
\pgfpathmoveto{\pgfqpoint{3.363962in}{2.611132in}}%
\pgfpathlineto{\pgfqpoint{3.451698in}{2.611132in}}%
\pgfpathlineto{\pgfqpoint{3.451698in}{2.523396in}}%
\pgfpathlineto{\pgfqpoint{3.363962in}{2.523396in}}%
\pgfpathlineto{\pgfqpoint{3.363962in}{2.611132in}}%
\pgfusepath{stroke,fill}%
\end{pgfscope}%
\begin{pgfscope}%
\pgfpathrectangle{\pgfqpoint{0.380943in}{2.260189in}}{\pgfqpoint{4.650000in}{0.614151in}}%
\pgfusepath{clip}%
\pgfsetbuttcap%
\pgfsetroundjoin%
\definecolor{currentfill}{rgb}{0.978639,0.841584,0.673679}%
\pgfsetfillcolor{currentfill}%
\pgfsetlinewidth{0.250937pt}%
\definecolor{currentstroke}{rgb}{1.000000,1.000000,1.000000}%
\pgfsetstrokecolor{currentstroke}%
\pgfsetdash{}{0pt}%
\pgfpathmoveto{\pgfqpoint{3.451698in}{2.611132in}}%
\pgfpathlineto{\pgfqpoint{3.539434in}{2.611132in}}%
\pgfpathlineto{\pgfqpoint{3.539434in}{2.523396in}}%
\pgfpathlineto{\pgfqpoint{3.451698in}{2.523396in}}%
\pgfpathlineto{\pgfqpoint{3.451698in}{2.611132in}}%
\pgfusepath{stroke,fill}%
\end{pgfscope}%
\begin{pgfscope}%
\pgfpathrectangle{\pgfqpoint{0.380943in}{2.260189in}}{\pgfqpoint{4.650000in}{0.614151in}}%
\pgfusepath{clip}%
\pgfsetbuttcap%
\pgfsetroundjoin%
\definecolor{currentfill}{rgb}{0.996401,0.724937,0.591557}%
\pgfsetfillcolor{currentfill}%
\pgfsetlinewidth{0.250937pt}%
\definecolor{currentstroke}{rgb}{1.000000,1.000000,1.000000}%
\pgfsetstrokecolor{currentstroke}%
\pgfsetdash{}{0pt}%
\pgfpathmoveto{\pgfqpoint{3.539434in}{2.611132in}}%
\pgfpathlineto{\pgfqpoint{3.627169in}{2.611132in}}%
\pgfpathlineto{\pgfqpoint{3.627169in}{2.523396in}}%
\pgfpathlineto{\pgfqpoint{3.539434in}{2.523396in}}%
\pgfpathlineto{\pgfqpoint{3.539434in}{2.611132in}}%
\pgfusepath{stroke,fill}%
\end{pgfscope}%
\begin{pgfscope}%
\pgfpathrectangle{\pgfqpoint{0.380943in}{2.260189in}}{\pgfqpoint{4.650000in}{0.614151in}}%
\pgfusepath{clip}%
\pgfsetbuttcap%
\pgfsetroundjoin%
\definecolor{currentfill}{rgb}{0.987266,0.804198,0.639170}%
\pgfsetfillcolor{currentfill}%
\pgfsetlinewidth{0.250937pt}%
\definecolor{currentstroke}{rgb}{1.000000,1.000000,1.000000}%
\pgfsetstrokecolor{currentstroke}%
\pgfsetdash{}{0pt}%
\pgfpathmoveto{\pgfqpoint{3.627169in}{2.611132in}}%
\pgfpathlineto{\pgfqpoint{3.714905in}{2.611132in}}%
\pgfpathlineto{\pgfqpoint{3.714905in}{2.523396in}}%
\pgfpathlineto{\pgfqpoint{3.627169in}{2.523396in}}%
\pgfpathlineto{\pgfqpoint{3.627169in}{2.611132in}}%
\pgfusepath{stroke,fill}%
\end{pgfscope}%
\begin{pgfscope}%
\pgfpathrectangle{\pgfqpoint{0.380943in}{2.260189in}}{\pgfqpoint{4.650000in}{0.614151in}}%
\pgfusepath{clip}%
\pgfsetbuttcap%
\pgfsetroundjoin%
\definecolor{currentfill}{rgb}{1.000000,0.525475,0.498239}%
\pgfsetfillcolor{currentfill}%
\pgfsetlinewidth{0.250937pt}%
\definecolor{currentstroke}{rgb}{1.000000,1.000000,1.000000}%
\pgfsetstrokecolor{currentstroke}%
\pgfsetdash{}{0pt}%
\pgfpathmoveto{\pgfqpoint{3.714905in}{2.611132in}}%
\pgfpathlineto{\pgfqpoint{3.802641in}{2.611132in}}%
\pgfpathlineto{\pgfqpoint{3.802641in}{2.523396in}}%
\pgfpathlineto{\pgfqpoint{3.714905in}{2.523396in}}%
\pgfpathlineto{\pgfqpoint{3.714905in}{2.611132in}}%
\pgfusepath{stroke,fill}%
\end{pgfscope}%
\begin{pgfscope}%
\pgfpathrectangle{\pgfqpoint{0.380943in}{2.260189in}}{\pgfqpoint{4.650000in}{0.614151in}}%
\pgfusepath{clip}%
\pgfsetbuttcap%
\pgfsetroundjoin%
\definecolor{currentfill}{rgb}{1.000000,0.525475,0.498239}%
\pgfsetfillcolor{currentfill}%
\pgfsetlinewidth{0.250937pt}%
\definecolor{currentstroke}{rgb}{1.000000,1.000000,1.000000}%
\pgfsetstrokecolor{currentstroke}%
\pgfsetdash{}{0pt}%
\pgfpathmoveto{\pgfqpoint{3.802641in}{2.611132in}}%
\pgfpathlineto{\pgfqpoint{3.890377in}{2.611132in}}%
\pgfpathlineto{\pgfqpoint{3.890377in}{2.523396in}}%
\pgfpathlineto{\pgfqpoint{3.802641in}{2.523396in}}%
\pgfpathlineto{\pgfqpoint{3.802641in}{2.611132in}}%
\pgfusepath{stroke,fill}%
\end{pgfscope}%
\begin{pgfscope}%
\pgfpathrectangle{\pgfqpoint{0.380943in}{2.260189in}}{\pgfqpoint{4.650000in}{0.614151in}}%
\pgfusepath{clip}%
\pgfsetbuttcap%
\pgfsetroundjoin%
\definecolor{currentfill}{rgb}{0.978639,0.841584,0.673679}%
\pgfsetfillcolor{currentfill}%
\pgfsetlinewidth{0.250937pt}%
\definecolor{currentstroke}{rgb}{1.000000,1.000000,1.000000}%
\pgfsetstrokecolor{currentstroke}%
\pgfsetdash{}{0pt}%
\pgfpathmoveto{\pgfqpoint{3.890377in}{2.611132in}}%
\pgfpathlineto{\pgfqpoint{3.978113in}{2.611132in}}%
\pgfpathlineto{\pgfqpoint{3.978113in}{2.523396in}}%
\pgfpathlineto{\pgfqpoint{3.890377in}{2.523396in}}%
\pgfpathlineto{\pgfqpoint{3.890377in}{2.611132in}}%
\pgfusepath{stroke,fill}%
\end{pgfscope}%
\begin{pgfscope}%
\pgfpathrectangle{\pgfqpoint{0.380943in}{2.260189in}}{\pgfqpoint{4.650000in}{0.614151in}}%
\pgfusepath{clip}%
\pgfsetbuttcap%
\pgfsetroundjoin%
\definecolor{currentfill}{rgb}{0.978639,0.841584,0.673679}%
\pgfsetfillcolor{currentfill}%
\pgfsetlinewidth{0.250937pt}%
\definecolor{currentstroke}{rgb}{1.000000,1.000000,1.000000}%
\pgfsetstrokecolor{currentstroke}%
\pgfsetdash{}{0pt}%
\pgfpathmoveto{\pgfqpoint{3.978113in}{2.611132in}}%
\pgfpathlineto{\pgfqpoint{4.065849in}{2.611132in}}%
\pgfpathlineto{\pgfqpoint{4.065849in}{2.523396in}}%
\pgfpathlineto{\pgfqpoint{3.978113in}{2.523396in}}%
\pgfpathlineto{\pgfqpoint{3.978113in}{2.611132in}}%
\pgfusepath{stroke,fill}%
\end{pgfscope}%
\begin{pgfscope}%
\pgfpathrectangle{\pgfqpoint{0.380943in}{2.260189in}}{\pgfqpoint{4.650000in}{0.614151in}}%
\pgfusepath{clip}%
\pgfsetbuttcap%
\pgfsetroundjoin%
\definecolor{currentfill}{rgb}{0.913879,0.392311,0.392311}%
\pgfsetfillcolor{currentfill}%
\pgfsetlinewidth{0.250937pt}%
\definecolor{currentstroke}{rgb}{1.000000,1.000000,1.000000}%
\pgfsetstrokecolor{currentstroke}%
\pgfsetdash{}{0pt}%
\pgfpathmoveto{\pgfqpoint{4.065849in}{2.611132in}}%
\pgfpathlineto{\pgfqpoint{4.153585in}{2.611132in}}%
\pgfpathlineto{\pgfqpoint{4.153585in}{2.523396in}}%
\pgfpathlineto{\pgfqpoint{4.065849in}{2.523396in}}%
\pgfpathlineto{\pgfqpoint{4.065849in}{2.611132in}}%
\pgfusepath{stroke,fill}%
\end{pgfscope}%
\begin{pgfscope}%
\pgfpathrectangle{\pgfqpoint{0.380943in}{2.260189in}}{\pgfqpoint{4.650000in}{0.614151in}}%
\pgfusepath{clip}%
\pgfsetbuttcap%
\pgfsetroundjoin%
\definecolor{currentfill}{rgb}{0.990634,0.779608,0.623299}%
\pgfsetfillcolor{currentfill}%
\pgfsetlinewidth{0.250937pt}%
\definecolor{currentstroke}{rgb}{1.000000,1.000000,1.000000}%
\pgfsetstrokecolor{currentstroke}%
\pgfsetdash{}{0pt}%
\pgfpathmoveto{\pgfqpoint{4.153585in}{2.611132in}}%
\pgfpathlineto{\pgfqpoint{4.241320in}{2.611132in}}%
\pgfpathlineto{\pgfqpoint{4.241320in}{2.523396in}}%
\pgfpathlineto{\pgfqpoint{4.153585in}{2.523396in}}%
\pgfpathlineto{\pgfqpoint{4.153585in}{2.611132in}}%
\pgfusepath{stroke,fill}%
\end{pgfscope}%
\begin{pgfscope}%
\pgfpathrectangle{\pgfqpoint{0.380943in}{2.260189in}}{\pgfqpoint{4.650000in}{0.614151in}}%
\pgfusepath{clip}%
\pgfsetbuttcap%
\pgfsetroundjoin%
\definecolor{currentfill}{rgb}{0.996401,0.724937,0.591557}%
\pgfsetfillcolor{currentfill}%
\pgfsetlinewidth{0.250937pt}%
\definecolor{currentstroke}{rgb}{1.000000,1.000000,1.000000}%
\pgfsetstrokecolor{currentstroke}%
\pgfsetdash{}{0pt}%
\pgfpathmoveto{\pgfqpoint{4.241320in}{2.611132in}}%
\pgfpathlineto{\pgfqpoint{4.329056in}{2.611132in}}%
\pgfpathlineto{\pgfqpoint{4.329056in}{2.523396in}}%
\pgfpathlineto{\pgfqpoint{4.241320in}{2.523396in}}%
\pgfpathlineto{\pgfqpoint{4.241320in}{2.611132in}}%
\pgfusepath{stroke,fill}%
\end{pgfscope}%
\begin{pgfscope}%
\pgfpathrectangle{\pgfqpoint{0.380943in}{2.260189in}}{\pgfqpoint{4.650000in}{0.614151in}}%
\pgfusepath{clip}%
\pgfsetbuttcap%
\pgfsetroundjoin%
\definecolor{currentfill}{rgb}{0.974072,0.862976,0.688750}%
\pgfsetfillcolor{currentfill}%
\pgfsetlinewidth{0.250937pt}%
\definecolor{currentstroke}{rgb}{1.000000,1.000000,1.000000}%
\pgfsetstrokecolor{currentstroke}%
\pgfsetdash{}{0pt}%
\pgfpathmoveto{\pgfqpoint{4.329056in}{2.611132in}}%
\pgfpathlineto{\pgfqpoint{4.416792in}{2.611132in}}%
\pgfpathlineto{\pgfqpoint{4.416792in}{2.523396in}}%
\pgfpathlineto{\pgfqpoint{4.329056in}{2.523396in}}%
\pgfpathlineto{\pgfqpoint{4.329056in}{2.611132in}}%
\pgfusepath{stroke,fill}%
\end{pgfscope}%
\begin{pgfscope}%
\pgfpathrectangle{\pgfqpoint{0.380943in}{2.260189in}}{\pgfqpoint{4.650000in}{0.614151in}}%
\pgfusepath{clip}%
\pgfsetbuttcap%
\pgfsetroundjoin%
\definecolor{currentfill}{rgb}{0.996401,0.724937,0.591557}%
\pgfsetfillcolor{currentfill}%
\pgfsetlinewidth{0.250937pt}%
\definecolor{currentstroke}{rgb}{1.000000,1.000000,1.000000}%
\pgfsetstrokecolor{currentstroke}%
\pgfsetdash{}{0pt}%
\pgfpathmoveto{\pgfqpoint{4.416792in}{2.611132in}}%
\pgfpathlineto{\pgfqpoint{4.504528in}{2.611132in}}%
\pgfpathlineto{\pgfqpoint{4.504528in}{2.523396in}}%
\pgfpathlineto{\pgfqpoint{4.416792in}{2.523396in}}%
\pgfpathlineto{\pgfqpoint{4.416792in}{2.611132in}}%
\pgfusepath{stroke,fill}%
\end{pgfscope}%
\begin{pgfscope}%
\pgfpathrectangle{\pgfqpoint{0.380943in}{2.260189in}}{\pgfqpoint{4.650000in}{0.614151in}}%
\pgfusepath{clip}%
\pgfsetbuttcap%
\pgfsetroundjoin%
\definecolor{currentfill}{rgb}{0.993679,0.753725,0.608074}%
\pgfsetfillcolor{currentfill}%
\pgfsetlinewidth{0.250937pt}%
\definecolor{currentstroke}{rgb}{1.000000,1.000000,1.000000}%
\pgfsetstrokecolor{currentstroke}%
\pgfsetdash{}{0pt}%
\pgfpathmoveto{\pgfqpoint{4.504528in}{2.611132in}}%
\pgfpathlineto{\pgfqpoint{4.592264in}{2.611132in}}%
\pgfpathlineto{\pgfqpoint{4.592264in}{2.523396in}}%
\pgfpathlineto{\pgfqpoint{4.504528in}{2.523396in}}%
\pgfpathlineto{\pgfqpoint{4.504528in}{2.611132in}}%
\pgfusepath{stroke,fill}%
\end{pgfscope}%
\begin{pgfscope}%
\pgfpathrectangle{\pgfqpoint{0.380943in}{2.260189in}}{\pgfqpoint{4.650000in}{0.614151in}}%
\pgfusepath{clip}%
\pgfsetbuttcap%
\pgfsetroundjoin%
\definecolor{currentfill}{rgb}{0.993679,0.753725,0.608074}%
\pgfsetfillcolor{currentfill}%
\pgfsetlinewidth{0.250937pt}%
\definecolor{currentstroke}{rgb}{1.000000,1.000000,1.000000}%
\pgfsetstrokecolor{currentstroke}%
\pgfsetdash{}{0pt}%
\pgfpathmoveto{\pgfqpoint{4.592264in}{2.611132in}}%
\pgfpathlineto{\pgfqpoint{4.680000in}{2.611132in}}%
\pgfpathlineto{\pgfqpoint{4.680000in}{2.523396in}}%
\pgfpathlineto{\pgfqpoint{4.592264in}{2.523396in}}%
\pgfpathlineto{\pgfqpoint{4.592264in}{2.611132in}}%
\pgfusepath{stroke,fill}%
\end{pgfscope}%
\begin{pgfscope}%
\pgfpathrectangle{\pgfqpoint{0.380943in}{2.260189in}}{\pgfqpoint{4.650000in}{0.614151in}}%
\pgfusepath{clip}%
\pgfsetbuttcap%
\pgfsetroundjoin%
\definecolor{currentfill}{rgb}{1.000000,0.584929,0.522599}%
\pgfsetfillcolor{currentfill}%
\pgfsetlinewidth{0.250937pt}%
\definecolor{currentstroke}{rgb}{1.000000,1.000000,1.000000}%
\pgfsetstrokecolor{currentstroke}%
\pgfsetdash{}{0pt}%
\pgfpathmoveto{\pgfqpoint{4.680000in}{2.611132in}}%
\pgfpathlineto{\pgfqpoint{4.767736in}{2.611132in}}%
\pgfpathlineto{\pgfqpoint{4.767736in}{2.523396in}}%
\pgfpathlineto{\pgfqpoint{4.680000in}{2.523396in}}%
\pgfpathlineto{\pgfqpoint{4.680000in}{2.611132in}}%
\pgfusepath{stroke,fill}%
\end{pgfscope}%
\begin{pgfscope}%
\pgfpathrectangle{\pgfqpoint{0.380943in}{2.260189in}}{\pgfqpoint{4.650000in}{0.614151in}}%
\pgfusepath{clip}%
\pgfsetbuttcap%
\pgfsetroundjoin%
\definecolor{currentfill}{rgb}{0.997924,0.685352,0.570242}%
\pgfsetfillcolor{currentfill}%
\pgfsetlinewidth{0.250937pt}%
\definecolor{currentstroke}{rgb}{1.000000,1.000000,1.000000}%
\pgfsetstrokecolor{currentstroke}%
\pgfsetdash{}{0pt}%
\pgfpathmoveto{\pgfqpoint{4.767736in}{2.611132in}}%
\pgfpathlineto{\pgfqpoint{4.855471in}{2.611132in}}%
\pgfpathlineto{\pgfqpoint{4.855471in}{2.523396in}}%
\pgfpathlineto{\pgfqpoint{4.767736in}{2.523396in}}%
\pgfpathlineto{\pgfqpoint{4.767736in}{2.611132in}}%
\pgfusepath{stroke,fill}%
\end{pgfscope}%
\begin{pgfscope}%
\pgfpathrectangle{\pgfqpoint{0.380943in}{2.260189in}}{\pgfqpoint{4.650000in}{0.614151in}}%
\pgfusepath{clip}%
\pgfsetbuttcap%
\pgfsetroundjoin%
\definecolor{currentfill}{rgb}{0.963260,0.918478,0.719508}%
\pgfsetfillcolor{currentfill}%
\pgfsetlinewidth{0.250937pt}%
\definecolor{currentstroke}{rgb}{1.000000,1.000000,1.000000}%
\pgfsetstrokecolor{currentstroke}%
\pgfsetdash{}{0pt}%
\pgfpathmoveto{\pgfqpoint{4.855471in}{2.611132in}}%
\pgfpathlineto{\pgfqpoint{4.943207in}{2.611132in}}%
\pgfpathlineto{\pgfqpoint{4.943207in}{2.523396in}}%
\pgfpathlineto{\pgfqpoint{4.855471in}{2.523396in}}%
\pgfpathlineto{\pgfqpoint{4.855471in}{2.611132in}}%
\pgfusepath{stroke,fill}%
\end{pgfscope}%
\begin{pgfscope}%
\pgfpathrectangle{\pgfqpoint{0.380943in}{2.260189in}}{\pgfqpoint{4.650000in}{0.614151in}}%
\pgfusepath{clip}%
\pgfsetbuttcap%
\pgfsetroundjoin%
\definecolor{currentfill}{rgb}{0.982699,0.823991,0.657439}%
\pgfsetfillcolor{currentfill}%
\pgfsetlinewidth{0.250937pt}%
\definecolor{currentstroke}{rgb}{1.000000,1.000000,1.000000}%
\pgfsetstrokecolor{currentstroke}%
\pgfsetdash{}{0pt}%
\pgfpathmoveto{\pgfqpoint{4.943207in}{2.611132in}}%
\pgfpathlineto{\pgfqpoint{5.030943in}{2.611132in}}%
\pgfpathlineto{\pgfqpoint{5.030943in}{2.523396in}}%
\pgfpathlineto{\pgfqpoint{4.943207in}{2.523396in}}%
\pgfpathlineto{\pgfqpoint{4.943207in}{2.611132in}}%
\pgfusepath{stroke,fill}%
\end{pgfscope}%
\begin{pgfscope}%
\pgfpathrectangle{\pgfqpoint{0.380943in}{2.260189in}}{\pgfqpoint{4.650000in}{0.614151in}}%
\pgfusepath{clip}%
\pgfsetbuttcap%
\pgfsetroundjoin%
\definecolor{currentfill}{rgb}{0.990634,0.779608,0.623299}%
\pgfsetfillcolor{currentfill}%
\pgfsetlinewidth{0.250937pt}%
\definecolor{currentstroke}{rgb}{1.000000,1.000000,1.000000}%
\pgfsetstrokecolor{currentstroke}%
\pgfsetdash{}{0pt}%
\pgfpathmoveto{\pgfqpoint{0.380943in}{2.523396in}}%
\pgfpathlineto{\pgfqpoint{0.468679in}{2.523396in}}%
\pgfpathlineto{\pgfqpoint{0.468679in}{2.435661in}}%
\pgfpathlineto{\pgfqpoint{0.380943in}{2.435661in}}%
\pgfpathlineto{\pgfqpoint{0.380943in}{2.523396in}}%
\pgfusepath{stroke,fill}%
\end{pgfscope}%
\begin{pgfscope}%
\pgfpathrectangle{\pgfqpoint{0.380943in}{2.260189in}}{\pgfqpoint{4.650000in}{0.614151in}}%
\pgfusepath{clip}%
\pgfsetbuttcap%
\pgfsetroundjoin%
\definecolor{currentfill}{rgb}{0.990634,0.779608,0.623299}%
\pgfsetfillcolor{currentfill}%
\pgfsetlinewidth{0.250937pt}%
\definecolor{currentstroke}{rgb}{1.000000,1.000000,1.000000}%
\pgfsetstrokecolor{currentstroke}%
\pgfsetdash{}{0pt}%
\pgfpathmoveto{\pgfqpoint{0.468679in}{2.523396in}}%
\pgfpathlineto{\pgfqpoint{0.556415in}{2.523396in}}%
\pgfpathlineto{\pgfqpoint{0.556415in}{2.435661in}}%
\pgfpathlineto{\pgfqpoint{0.468679in}{2.435661in}}%
\pgfpathlineto{\pgfqpoint{0.468679in}{2.523396in}}%
\pgfusepath{stroke,fill}%
\end{pgfscope}%
\begin{pgfscope}%
\pgfpathrectangle{\pgfqpoint{0.380943in}{2.260189in}}{\pgfqpoint{4.650000in}{0.614151in}}%
\pgfusepath{clip}%
\pgfsetbuttcap%
\pgfsetroundjoin%
\definecolor{currentfill}{rgb}{0.964937,0.908651,0.713110}%
\pgfsetfillcolor{currentfill}%
\pgfsetlinewidth{0.250937pt}%
\definecolor{currentstroke}{rgb}{1.000000,1.000000,1.000000}%
\pgfsetstrokecolor{currentstroke}%
\pgfsetdash{}{0pt}%
\pgfpathmoveto{\pgfqpoint{0.556415in}{2.523396in}}%
\pgfpathlineto{\pgfqpoint{0.644151in}{2.523396in}}%
\pgfpathlineto{\pgfqpoint{0.644151in}{2.435661in}}%
\pgfpathlineto{\pgfqpoint{0.556415in}{2.435661in}}%
\pgfpathlineto{\pgfqpoint{0.556415in}{2.523396in}}%
\pgfusepath{stroke,fill}%
\end{pgfscope}%
\begin{pgfscope}%
\pgfpathrectangle{\pgfqpoint{0.380943in}{2.260189in}}{\pgfqpoint{4.650000in}{0.614151in}}%
\pgfusepath{clip}%
\pgfsetbuttcap%
\pgfsetroundjoin%
\definecolor{currentfill}{rgb}{0.987266,0.804198,0.639170}%
\pgfsetfillcolor{currentfill}%
\pgfsetlinewidth{0.250937pt}%
\definecolor{currentstroke}{rgb}{1.000000,1.000000,1.000000}%
\pgfsetstrokecolor{currentstroke}%
\pgfsetdash{}{0pt}%
\pgfpathmoveto{\pgfqpoint{0.644151in}{2.523396in}}%
\pgfpathlineto{\pgfqpoint{0.731886in}{2.523396in}}%
\pgfpathlineto{\pgfqpoint{0.731886in}{2.435661in}}%
\pgfpathlineto{\pgfqpoint{0.644151in}{2.435661in}}%
\pgfpathlineto{\pgfqpoint{0.644151in}{2.523396in}}%
\pgfusepath{stroke,fill}%
\end{pgfscope}%
\begin{pgfscope}%
\pgfpathrectangle{\pgfqpoint{0.380943in}{2.260189in}}{\pgfqpoint{4.650000in}{0.614151in}}%
\pgfusepath{clip}%
\pgfsetbuttcap%
\pgfsetroundjoin%
\definecolor{currentfill}{rgb}{0.964783,0.940131,0.739808}%
\pgfsetfillcolor{currentfill}%
\pgfsetlinewidth{0.250937pt}%
\definecolor{currentstroke}{rgb}{1.000000,1.000000,1.000000}%
\pgfsetstrokecolor{currentstroke}%
\pgfsetdash{}{0pt}%
\pgfpathmoveto{\pgfqpoint{0.731886in}{2.523396in}}%
\pgfpathlineto{\pgfqpoint{0.819622in}{2.523396in}}%
\pgfpathlineto{\pgfqpoint{0.819622in}{2.435661in}}%
\pgfpathlineto{\pgfqpoint{0.731886in}{2.435661in}}%
\pgfpathlineto{\pgfqpoint{0.731886in}{2.523396in}}%
\pgfusepath{stroke,fill}%
\end{pgfscope}%
\begin{pgfscope}%
\pgfpathrectangle{\pgfqpoint{0.380943in}{2.260189in}}{\pgfqpoint{4.650000in}{0.614151in}}%
\pgfusepath{clip}%
\pgfsetbuttcap%
\pgfsetroundjoin%
\definecolor{currentfill}{rgb}{0.963260,0.918478,0.719508}%
\pgfsetfillcolor{currentfill}%
\pgfsetlinewidth{0.250937pt}%
\definecolor{currentstroke}{rgb}{1.000000,1.000000,1.000000}%
\pgfsetstrokecolor{currentstroke}%
\pgfsetdash{}{0pt}%
\pgfpathmoveto{\pgfqpoint{0.819622in}{2.523396in}}%
\pgfpathlineto{\pgfqpoint{0.907358in}{2.523396in}}%
\pgfpathlineto{\pgfqpoint{0.907358in}{2.435661in}}%
\pgfpathlineto{\pgfqpoint{0.819622in}{2.435661in}}%
\pgfpathlineto{\pgfqpoint{0.819622in}{2.523396in}}%
\pgfusepath{stroke,fill}%
\end{pgfscope}%
\begin{pgfscope}%
\pgfpathrectangle{\pgfqpoint{0.380943in}{2.260189in}}{\pgfqpoint{4.650000in}{0.614151in}}%
\pgfusepath{clip}%
\pgfsetbuttcap%
\pgfsetroundjoin%
\definecolor{currentfill}{rgb}{0.997924,0.685352,0.570242}%
\pgfsetfillcolor{currentfill}%
\pgfsetlinewidth{0.250937pt}%
\definecolor{currentstroke}{rgb}{1.000000,1.000000,1.000000}%
\pgfsetstrokecolor{currentstroke}%
\pgfsetdash{}{0pt}%
\pgfpathmoveto{\pgfqpoint{0.907358in}{2.523396in}}%
\pgfpathlineto{\pgfqpoint{0.995094in}{2.523396in}}%
\pgfpathlineto{\pgfqpoint{0.995094in}{2.435661in}}%
\pgfpathlineto{\pgfqpoint{0.907358in}{2.435661in}}%
\pgfpathlineto{\pgfqpoint{0.907358in}{2.523396in}}%
\pgfusepath{stroke,fill}%
\end{pgfscope}%
\begin{pgfscope}%
\pgfpathrectangle{\pgfqpoint{0.380943in}{2.260189in}}{\pgfqpoint{4.650000in}{0.614151in}}%
\pgfusepath{clip}%
\pgfsetbuttcap%
\pgfsetroundjoin%
\definecolor{currentfill}{rgb}{0.978639,0.841584,0.673679}%
\pgfsetfillcolor{currentfill}%
\pgfsetlinewidth{0.250937pt}%
\definecolor{currentstroke}{rgb}{1.000000,1.000000,1.000000}%
\pgfsetstrokecolor{currentstroke}%
\pgfsetdash{}{0pt}%
\pgfpathmoveto{\pgfqpoint{0.995094in}{2.523396in}}%
\pgfpathlineto{\pgfqpoint{1.082830in}{2.523396in}}%
\pgfpathlineto{\pgfqpoint{1.082830in}{2.435661in}}%
\pgfpathlineto{\pgfqpoint{0.995094in}{2.435661in}}%
\pgfpathlineto{\pgfqpoint{0.995094in}{2.523396in}}%
\pgfusepath{stroke,fill}%
\end{pgfscope}%
\begin{pgfscope}%
\pgfpathrectangle{\pgfqpoint{0.380943in}{2.260189in}}{\pgfqpoint{4.650000in}{0.614151in}}%
\pgfusepath{clip}%
\pgfsetbuttcap%
\pgfsetroundjoin%
\definecolor{currentfill}{rgb}{0.969504,0.885813,0.700930}%
\pgfsetfillcolor{currentfill}%
\pgfsetlinewidth{0.250937pt}%
\definecolor{currentstroke}{rgb}{1.000000,1.000000,1.000000}%
\pgfsetstrokecolor{currentstroke}%
\pgfsetdash{}{0pt}%
\pgfpathmoveto{\pgfqpoint{1.082830in}{2.523396in}}%
\pgfpathlineto{\pgfqpoint{1.170566in}{2.523396in}}%
\pgfpathlineto{\pgfqpoint{1.170566in}{2.435661in}}%
\pgfpathlineto{\pgfqpoint{1.082830in}{2.435661in}}%
\pgfpathlineto{\pgfqpoint{1.082830in}{2.523396in}}%
\pgfusepath{stroke,fill}%
\end{pgfscope}%
\begin{pgfscope}%
\pgfpathrectangle{\pgfqpoint{0.380943in}{2.260189in}}{\pgfqpoint{4.650000in}{0.614151in}}%
\pgfusepath{clip}%
\pgfsetbuttcap%
\pgfsetroundjoin%
\definecolor{currentfill}{rgb}{0.978639,0.841584,0.673679}%
\pgfsetfillcolor{currentfill}%
\pgfsetlinewidth{0.250937pt}%
\definecolor{currentstroke}{rgb}{1.000000,1.000000,1.000000}%
\pgfsetstrokecolor{currentstroke}%
\pgfsetdash{}{0pt}%
\pgfpathmoveto{\pgfqpoint{1.170566in}{2.523396in}}%
\pgfpathlineto{\pgfqpoint{1.258302in}{2.523396in}}%
\pgfpathlineto{\pgfqpoint{1.258302in}{2.435661in}}%
\pgfpathlineto{\pgfqpoint{1.170566in}{2.435661in}}%
\pgfpathlineto{\pgfqpoint{1.170566in}{2.523396in}}%
\pgfusepath{stroke,fill}%
\end{pgfscope}%
\begin{pgfscope}%
\pgfpathrectangle{\pgfqpoint{0.380943in}{2.260189in}}{\pgfqpoint{4.650000in}{0.614151in}}%
\pgfusepath{clip}%
\pgfsetbuttcap%
\pgfsetroundjoin%
\definecolor{currentfill}{rgb}{0.974072,0.862976,0.688750}%
\pgfsetfillcolor{currentfill}%
\pgfsetlinewidth{0.250937pt}%
\definecolor{currentstroke}{rgb}{1.000000,1.000000,1.000000}%
\pgfsetstrokecolor{currentstroke}%
\pgfsetdash{}{0pt}%
\pgfpathmoveto{\pgfqpoint{1.258302in}{2.523396in}}%
\pgfpathlineto{\pgfqpoint{1.346037in}{2.523396in}}%
\pgfpathlineto{\pgfqpoint{1.346037in}{2.435661in}}%
\pgfpathlineto{\pgfqpoint{1.258302in}{2.435661in}}%
\pgfpathlineto{\pgfqpoint{1.258302in}{2.523396in}}%
\pgfusepath{stroke,fill}%
\end{pgfscope}%
\begin{pgfscope}%
\pgfpathrectangle{\pgfqpoint{0.380943in}{2.260189in}}{\pgfqpoint{4.650000in}{0.614151in}}%
\pgfusepath{clip}%
\pgfsetbuttcap%
\pgfsetroundjoin%
\definecolor{currentfill}{rgb}{0.964783,0.940131,0.739808}%
\pgfsetfillcolor{currentfill}%
\pgfsetlinewidth{0.250937pt}%
\definecolor{currentstroke}{rgb}{1.000000,1.000000,1.000000}%
\pgfsetstrokecolor{currentstroke}%
\pgfsetdash{}{0pt}%
\pgfpathmoveto{\pgfqpoint{1.346037in}{2.523396in}}%
\pgfpathlineto{\pgfqpoint{1.433773in}{2.523396in}}%
\pgfpathlineto{\pgfqpoint{1.433773in}{2.435661in}}%
\pgfpathlineto{\pgfqpoint{1.346037in}{2.435661in}}%
\pgfpathlineto{\pgfqpoint{1.346037in}{2.523396in}}%
\pgfusepath{stroke,fill}%
\end{pgfscope}%
\begin{pgfscope}%
\pgfpathrectangle{\pgfqpoint{0.380943in}{2.260189in}}{\pgfqpoint{4.650000in}{0.614151in}}%
\pgfusepath{clip}%
\pgfsetbuttcap%
\pgfsetroundjoin%
\definecolor{currentfill}{rgb}{1.000000,1.000000,0.929412}%
\pgfsetfillcolor{currentfill}%
\pgfsetlinewidth{0.250937pt}%
\definecolor{currentstroke}{rgb}{1.000000,1.000000,1.000000}%
\pgfsetstrokecolor{currentstroke}%
\pgfsetdash{}{0pt}%
\pgfpathmoveto{\pgfqpoint{1.433773in}{2.523396in}}%
\pgfpathlineto{\pgfqpoint{1.521509in}{2.523396in}}%
\pgfpathlineto{\pgfqpoint{1.521509in}{2.435661in}}%
\pgfpathlineto{\pgfqpoint{1.433773in}{2.435661in}}%
\pgfpathlineto{\pgfqpoint{1.433773in}{2.523396in}}%
\pgfusepath{stroke,fill}%
\end{pgfscope}%
\begin{pgfscope}%
\pgfpathrectangle{\pgfqpoint{0.380943in}{2.260189in}}{\pgfqpoint{4.650000in}{0.614151in}}%
\pgfusepath{clip}%
\pgfsetbuttcap%
\pgfsetroundjoin%
\definecolor{currentfill}{rgb}{0.995233,0.991895,0.818977}%
\pgfsetfillcolor{currentfill}%
\pgfsetlinewidth{0.250937pt}%
\definecolor{currentstroke}{rgb}{1.000000,1.000000,1.000000}%
\pgfsetstrokecolor{currentstroke}%
\pgfsetdash{}{0pt}%
\pgfpathmoveto{\pgfqpoint{1.521509in}{2.523396in}}%
\pgfpathlineto{\pgfqpoint{1.609245in}{2.523396in}}%
\pgfpathlineto{\pgfqpoint{1.609245in}{2.435661in}}%
\pgfpathlineto{\pgfqpoint{1.521509in}{2.435661in}}%
\pgfpathlineto{\pgfqpoint{1.521509in}{2.523396in}}%
\pgfusepath{stroke,fill}%
\end{pgfscope}%
\begin{pgfscope}%
\pgfpathrectangle{\pgfqpoint{0.380943in}{2.260189in}}{\pgfqpoint{4.650000in}{0.614151in}}%
\pgfusepath{clip}%
\pgfsetbuttcap%
\pgfsetroundjoin%
\definecolor{currentfill}{rgb}{0.969504,0.885813,0.700930}%
\pgfsetfillcolor{currentfill}%
\pgfsetlinewidth{0.250937pt}%
\definecolor{currentstroke}{rgb}{1.000000,1.000000,1.000000}%
\pgfsetstrokecolor{currentstroke}%
\pgfsetdash{}{0pt}%
\pgfpathmoveto{\pgfqpoint{1.609245in}{2.523396in}}%
\pgfpathlineto{\pgfqpoint{1.696981in}{2.523396in}}%
\pgfpathlineto{\pgfqpoint{1.696981in}{2.435661in}}%
\pgfpathlineto{\pgfqpoint{1.609245in}{2.435661in}}%
\pgfpathlineto{\pgfqpoint{1.609245in}{2.523396in}}%
\pgfusepath{stroke,fill}%
\end{pgfscope}%
\begin{pgfscope}%
\pgfpathrectangle{\pgfqpoint{0.380943in}{2.260189in}}{\pgfqpoint{4.650000in}{0.614151in}}%
\pgfusepath{clip}%
\pgfsetbuttcap%
\pgfsetroundjoin%
\definecolor{currentfill}{rgb}{0.969504,0.885813,0.700930}%
\pgfsetfillcolor{currentfill}%
\pgfsetlinewidth{0.250937pt}%
\definecolor{currentstroke}{rgb}{1.000000,1.000000,1.000000}%
\pgfsetstrokecolor{currentstroke}%
\pgfsetdash{}{0pt}%
\pgfpathmoveto{\pgfqpoint{1.696981in}{2.523396in}}%
\pgfpathlineto{\pgfqpoint{1.784717in}{2.523396in}}%
\pgfpathlineto{\pgfqpoint{1.784717in}{2.435661in}}%
\pgfpathlineto{\pgfqpoint{1.696981in}{2.435661in}}%
\pgfpathlineto{\pgfqpoint{1.696981in}{2.523396in}}%
\pgfusepath{stroke,fill}%
\end{pgfscope}%
\begin{pgfscope}%
\pgfpathrectangle{\pgfqpoint{0.380943in}{2.260189in}}{\pgfqpoint{4.650000in}{0.614151in}}%
\pgfusepath{clip}%
\pgfsetbuttcap%
\pgfsetroundjoin%
\definecolor{currentfill}{rgb}{0.974072,0.862976,0.688750}%
\pgfsetfillcolor{currentfill}%
\pgfsetlinewidth{0.250937pt}%
\definecolor{currentstroke}{rgb}{1.000000,1.000000,1.000000}%
\pgfsetstrokecolor{currentstroke}%
\pgfsetdash{}{0pt}%
\pgfpathmoveto{\pgfqpoint{1.784717in}{2.523396in}}%
\pgfpathlineto{\pgfqpoint{1.872452in}{2.523396in}}%
\pgfpathlineto{\pgfqpoint{1.872452in}{2.435661in}}%
\pgfpathlineto{\pgfqpoint{1.784717in}{2.435661in}}%
\pgfpathlineto{\pgfqpoint{1.784717in}{2.523396in}}%
\pgfusepath{stroke,fill}%
\end{pgfscope}%
\begin{pgfscope}%
\pgfpathrectangle{\pgfqpoint{0.380943in}{2.260189in}}{\pgfqpoint{4.650000in}{0.614151in}}%
\pgfusepath{clip}%
\pgfsetbuttcap%
\pgfsetroundjoin%
\definecolor{currentfill}{rgb}{0.995233,0.991895,0.818977}%
\pgfsetfillcolor{currentfill}%
\pgfsetlinewidth{0.250937pt}%
\definecolor{currentstroke}{rgb}{1.000000,1.000000,1.000000}%
\pgfsetstrokecolor{currentstroke}%
\pgfsetdash{}{0pt}%
\pgfpathmoveto{\pgfqpoint{1.872452in}{2.523396in}}%
\pgfpathlineto{\pgfqpoint{1.960188in}{2.523396in}}%
\pgfpathlineto{\pgfqpoint{1.960188in}{2.435661in}}%
\pgfpathlineto{\pgfqpoint{1.872452in}{2.435661in}}%
\pgfpathlineto{\pgfqpoint{1.872452in}{2.523396in}}%
\pgfusepath{stroke,fill}%
\end{pgfscope}%
\begin{pgfscope}%
\pgfpathrectangle{\pgfqpoint{0.380943in}{2.260189in}}{\pgfqpoint{4.650000in}{0.614151in}}%
\pgfusepath{clip}%
\pgfsetbuttcap%
\pgfsetroundjoin%
\definecolor{currentfill}{rgb}{0.980008,0.966013,0.779393}%
\pgfsetfillcolor{currentfill}%
\pgfsetlinewidth{0.250937pt}%
\definecolor{currentstroke}{rgb}{1.000000,1.000000,1.000000}%
\pgfsetstrokecolor{currentstroke}%
\pgfsetdash{}{0pt}%
\pgfpathmoveto{\pgfqpoint{1.960188in}{2.523396in}}%
\pgfpathlineto{\pgfqpoint{2.047924in}{2.523396in}}%
\pgfpathlineto{\pgfqpoint{2.047924in}{2.435661in}}%
\pgfpathlineto{\pgfqpoint{1.960188in}{2.435661in}}%
\pgfpathlineto{\pgfqpoint{1.960188in}{2.523396in}}%
\pgfusepath{stroke,fill}%
\end{pgfscope}%
\begin{pgfscope}%
\pgfpathrectangle{\pgfqpoint{0.380943in}{2.260189in}}{\pgfqpoint{4.650000in}{0.614151in}}%
\pgfusepath{clip}%
\pgfsetbuttcap%
\pgfsetroundjoin%
\definecolor{currentfill}{rgb}{0.980008,0.966013,0.779393}%
\pgfsetfillcolor{currentfill}%
\pgfsetlinewidth{0.250937pt}%
\definecolor{currentstroke}{rgb}{1.000000,1.000000,1.000000}%
\pgfsetstrokecolor{currentstroke}%
\pgfsetdash{}{0pt}%
\pgfpathmoveto{\pgfqpoint{2.047924in}{2.523396in}}%
\pgfpathlineto{\pgfqpoint{2.135660in}{2.523396in}}%
\pgfpathlineto{\pgfqpoint{2.135660in}{2.435661in}}%
\pgfpathlineto{\pgfqpoint{2.047924in}{2.435661in}}%
\pgfpathlineto{\pgfqpoint{2.047924in}{2.523396in}}%
\pgfusepath{stroke,fill}%
\end{pgfscope}%
\begin{pgfscope}%
\pgfpathrectangle{\pgfqpoint{0.380943in}{2.260189in}}{\pgfqpoint{4.650000in}{0.614151in}}%
\pgfusepath{clip}%
\pgfsetbuttcap%
\pgfsetroundjoin%
\definecolor{currentfill}{rgb}{0.980008,0.966013,0.779393}%
\pgfsetfillcolor{currentfill}%
\pgfsetlinewidth{0.250937pt}%
\definecolor{currentstroke}{rgb}{1.000000,1.000000,1.000000}%
\pgfsetstrokecolor{currentstroke}%
\pgfsetdash{}{0pt}%
\pgfpathmoveto{\pgfqpoint{2.135660in}{2.523396in}}%
\pgfpathlineto{\pgfqpoint{2.223396in}{2.523396in}}%
\pgfpathlineto{\pgfqpoint{2.223396in}{2.435661in}}%
\pgfpathlineto{\pgfqpoint{2.135660in}{2.435661in}}%
\pgfpathlineto{\pgfqpoint{2.135660in}{2.523396in}}%
\pgfusepath{stroke,fill}%
\end{pgfscope}%
\begin{pgfscope}%
\pgfpathrectangle{\pgfqpoint{0.380943in}{2.260189in}}{\pgfqpoint{4.650000in}{0.614151in}}%
\pgfusepath{clip}%
\pgfsetbuttcap%
\pgfsetroundjoin%
\definecolor{currentfill}{rgb}{0.978639,0.841584,0.673679}%
\pgfsetfillcolor{currentfill}%
\pgfsetlinewidth{0.250937pt}%
\definecolor{currentstroke}{rgb}{1.000000,1.000000,1.000000}%
\pgfsetstrokecolor{currentstroke}%
\pgfsetdash{}{0pt}%
\pgfpathmoveto{\pgfqpoint{2.223396in}{2.523396in}}%
\pgfpathlineto{\pgfqpoint{2.311132in}{2.523396in}}%
\pgfpathlineto{\pgfqpoint{2.311132in}{2.435661in}}%
\pgfpathlineto{\pgfqpoint{2.223396in}{2.435661in}}%
\pgfpathlineto{\pgfqpoint{2.223396in}{2.523396in}}%
\pgfusepath{stroke,fill}%
\end{pgfscope}%
\begin{pgfscope}%
\pgfpathrectangle{\pgfqpoint{0.380943in}{2.260189in}}{\pgfqpoint{4.650000in}{0.614151in}}%
\pgfusepath{clip}%
\pgfsetbuttcap%
\pgfsetroundjoin%
\definecolor{currentfill}{rgb}{0.978639,0.841584,0.673679}%
\pgfsetfillcolor{currentfill}%
\pgfsetlinewidth{0.250937pt}%
\definecolor{currentstroke}{rgb}{1.000000,1.000000,1.000000}%
\pgfsetstrokecolor{currentstroke}%
\pgfsetdash{}{0pt}%
\pgfpathmoveto{\pgfqpoint{2.311132in}{2.523396in}}%
\pgfpathlineto{\pgfqpoint{2.398868in}{2.523396in}}%
\pgfpathlineto{\pgfqpoint{2.398868in}{2.435661in}}%
\pgfpathlineto{\pgfqpoint{2.311132in}{2.435661in}}%
\pgfpathlineto{\pgfqpoint{2.311132in}{2.523396in}}%
\pgfusepath{stroke,fill}%
\end{pgfscope}%
\begin{pgfscope}%
\pgfpathrectangle{\pgfqpoint{0.380943in}{2.260189in}}{\pgfqpoint{4.650000in}{0.614151in}}%
\pgfusepath{clip}%
\pgfsetbuttcap%
\pgfsetroundjoin%
\definecolor{currentfill}{rgb}{0.982699,0.823991,0.657439}%
\pgfsetfillcolor{currentfill}%
\pgfsetlinewidth{0.250937pt}%
\definecolor{currentstroke}{rgb}{1.000000,1.000000,1.000000}%
\pgfsetstrokecolor{currentstroke}%
\pgfsetdash{}{0pt}%
\pgfpathmoveto{\pgfqpoint{2.398868in}{2.523396in}}%
\pgfpathlineto{\pgfqpoint{2.486603in}{2.523396in}}%
\pgfpathlineto{\pgfqpoint{2.486603in}{2.435661in}}%
\pgfpathlineto{\pgfqpoint{2.398868in}{2.435661in}}%
\pgfpathlineto{\pgfqpoint{2.398868in}{2.523396in}}%
\pgfusepath{stroke,fill}%
\end{pgfscope}%
\begin{pgfscope}%
\pgfpathrectangle{\pgfqpoint{0.380943in}{2.260189in}}{\pgfqpoint{4.650000in}{0.614151in}}%
\pgfusepath{clip}%
\pgfsetbuttcap%
\pgfsetroundjoin%
\definecolor{currentfill}{rgb}{0.990634,0.779608,0.623299}%
\pgfsetfillcolor{currentfill}%
\pgfsetlinewidth{0.250937pt}%
\definecolor{currentstroke}{rgb}{1.000000,1.000000,1.000000}%
\pgfsetstrokecolor{currentstroke}%
\pgfsetdash{}{0pt}%
\pgfpathmoveto{\pgfqpoint{2.486603in}{2.523396in}}%
\pgfpathlineto{\pgfqpoint{2.574339in}{2.523396in}}%
\pgfpathlineto{\pgfqpoint{2.574339in}{2.435661in}}%
\pgfpathlineto{\pgfqpoint{2.486603in}{2.435661in}}%
\pgfpathlineto{\pgfqpoint{2.486603in}{2.523396in}}%
\pgfusepath{stroke,fill}%
\end{pgfscope}%
\begin{pgfscope}%
\pgfpathrectangle{\pgfqpoint{0.380943in}{2.260189in}}{\pgfqpoint{4.650000in}{0.614151in}}%
\pgfusepath{clip}%
\pgfsetbuttcap%
\pgfsetroundjoin%
\definecolor{currentfill}{rgb}{0.993679,0.753725,0.608074}%
\pgfsetfillcolor{currentfill}%
\pgfsetlinewidth{0.250937pt}%
\definecolor{currentstroke}{rgb}{1.000000,1.000000,1.000000}%
\pgfsetstrokecolor{currentstroke}%
\pgfsetdash{}{0pt}%
\pgfpathmoveto{\pgfqpoint{2.574339in}{2.523396in}}%
\pgfpathlineto{\pgfqpoint{2.662075in}{2.523396in}}%
\pgfpathlineto{\pgfqpoint{2.662075in}{2.435661in}}%
\pgfpathlineto{\pgfqpoint{2.574339in}{2.435661in}}%
\pgfpathlineto{\pgfqpoint{2.574339in}{2.523396in}}%
\pgfusepath{stroke,fill}%
\end{pgfscope}%
\begin{pgfscope}%
\pgfpathrectangle{\pgfqpoint{0.380943in}{2.260189in}}{\pgfqpoint{4.650000in}{0.614151in}}%
\pgfusepath{clip}%
\pgfsetbuttcap%
\pgfsetroundjoin%
\definecolor{currentfill}{rgb}{0.982699,0.823991,0.657439}%
\pgfsetfillcolor{currentfill}%
\pgfsetlinewidth{0.250937pt}%
\definecolor{currentstroke}{rgb}{1.000000,1.000000,1.000000}%
\pgfsetstrokecolor{currentstroke}%
\pgfsetdash{}{0pt}%
\pgfpathmoveto{\pgfqpoint{2.662075in}{2.523396in}}%
\pgfpathlineto{\pgfqpoint{2.749811in}{2.523396in}}%
\pgfpathlineto{\pgfqpoint{2.749811in}{2.435661in}}%
\pgfpathlineto{\pgfqpoint{2.662075in}{2.435661in}}%
\pgfpathlineto{\pgfqpoint{2.662075in}{2.523396in}}%
\pgfusepath{stroke,fill}%
\end{pgfscope}%
\begin{pgfscope}%
\pgfpathrectangle{\pgfqpoint{0.380943in}{2.260189in}}{\pgfqpoint{4.650000in}{0.614151in}}%
\pgfusepath{clip}%
\pgfsetbuttcap%
\pgfsetroundjoin%
\definecolor{currentfill}{rgb}{0.993679,0.753725,0.608074}%
\pgfsetfillcolor{currentfill}%
\pgfsetlinewidth{0.250937pt}%
\definecolor{currentstroke}{rgb}{1.000000,1.000000,1.000000}%
\pgfsetstrokecolor{currentstroke}%
\pgfsetdash{}{0pt}%
\pgfpathmoveto{\pgfqpoint{2.749811in}{2.523396in}}%
\pgfpathlineto{\pgfqpoint{2.837547in}{2.523396in}}%
\pgfpathlineto{\pgfqpoint{2.837547in}{2.435661in}}%
\pgfpathlineto{\pgfqpoint{2.749811in}{2.435661in}}%
\pgfpathlineto{\pgfqpoint{2.749811in}{2.523396in}}%
\pgfusepath{stroke,fill}%
\end{pgfscope}%
\begin{pgfscope}%
\pgfpathrectangle{\pgfqpoint{0.380943in}{2.260189in}}{\pgfqpoint{4.650000in}{0.614151in}}%
\pgfusepath{clip}%
\pgfsetbuttcap%
\pgfsetroundjoin%
\definecolor{currentfill}{rgb}{0.800000,0.278431,0.278431}%
\pgfsetfillcolor{currentfill}%
\pgfsetlinewidth{0.250937pt}%
\definecolor{currentstroke}{rgb}{1.000000,1.000000,1.000000}%
\pgfsetstrokecolor{currentstroke}%
\pgfsetdash{}{0pt}%
\pgfpathmoveto{\pgfqpoint{2.837547in}{2.523396in}}%
\pgfpathlineto{\pgfqpoint{2.925283in}{2.523396in}}%
\pgfpathlineto{\pgfqpoint{2.925283in}{2.435661in}}%
\pgfpathlineto{\pgfqpoint{2.837547in}{2.435661in}}%
\pgfpathlineto{\pgfqpoint{2.837547in}{2.523396in}}%
\pgfusepath{stroke,fill}%
\end{pgfscope}%
\begin{pgfscope}%
\pgfpathrectangle{\pgfqpoint{0.380943in}{2.260189in}}{\pgfqpoint{4.650000in}{0.614151in}}%
\pgfusepath{clip}%
\pgfsetbuttcap%
\pgfsetroundjoin%
\definecolor{currentfill}{rgb}{0.964937,0.908651,0.713110}%
\pgfsetfillcolor{currentfill}%
\pgfsetlinewidth{0.250937pt}%
\definecolor{currentstroke}{rgb}{1.000000,1.000000,1.000000}%
\pgfsetstrokecolor{currentstroke}%
\pgfsetdash{}{0pt}%
\pgfpathmoveto{\pgfqpoint{2.925283in}{2.523396in}}%
\pgfpathlineto{\pgfqpoint{3.013019in}{2.523396in}}%
\pgfpathlineto{\pgfqpoint{3.013019in}{2.435661in}}%
\pgfpathlineto{\pgfqpoint{2.925283in}{2.435661in}}%
\pgfpathlineto{\pgfqpoint{2.925283in}{2.523396in}}%
\pgfusepath{stroke,fill}%
\end{pgfscope}%
\begin{pgfscope}%
\pgfpathrectangle{\pgfqpoint{0.380943in}{2.260189in}}{\pgfqpoint{4.650000in}{0.614151in}}%
\pgfusepath{clip}%
\pgfsetbuttcap%
\pgfsetroundjoin%
\definecolor{currentfill}{rgb}{0.969504,0.885813,0.700930}%
\pgfsetfillcolor{currentfill}%
\pgfsetlinewidth{0.250937pt}%
\definecolor{currentstroke}{rgb}{1.000000,1.000000,1.000000}%
\pgfsetstrokecolor{currentstroke}%
\pgfsetdash{}{0pt}%
\pgfpathmoveto{\pgfqpoint{3.013019in}{2.523396in}}%
\pgfpathlineto{\pgfqpoint{3.100754in}{2.523396in}}%
\pgfpathlineto{\pgfqpoint{3.100754in}{2.435661in}}%
\pgfpathlineto{\pgfqpoint{3.013019in}{2.435661in}}%
\pgfpathlineto{\pgfqpoint{3.013019in}{2.523396in}}%
\pgfusepath{stroke,fill}%
\end{pgfscope}%
\begin{pgfscope}%
\pgfpathrectangle{\pgfqpoint{0.380943in}{2.260189in}}{\pgfqpoint{4.650000in}{0.614151in}}%
\pgfusepath{clip}%
\pgfsetbuttcap%
\pgfsetroundjoin%
\definecolor{currentfill}{rgb}{0.993679,0.753725,0.608074}%
\pgfsetfillcolor{currentfill}%
\pgfsetlinewidth{0.250937pt}%
\definecolor{currentstroke}{rgb}{1.000000,1.000000,1.000000}%
\pgfsetstrokecolor{currentstroke}%
\pgfsetdash{}{0pt}%
\pgfpathmoveto{\pgfqpoint{3.100754in}{2.523396in}}%
\pgfpathlineto{\pgfqpoint{3.188490in}{2.523396in}}%
\pgfpathlineto{\pgfqpoint{3.188490in}{2.435661in}}%
\pgfpathlineto{\pgfqpoint{3.100754in}{2.435661in}}%
\pgfpathlineto{\pgfqpoint{3.100754in}{2.523396in}}%
\pgfusepath{stroke,fill}%
\end{pgfscope}%
\begin{pgfscope}%
\pgfpathrectangle{\pgfqpoint{0.380943in}{2.260189in}}{\pgfqpoint{4.650000in}{0.614151in}}%
\pgfusepath{clip}%
\pgfsetbuttcap%
\pgfsetroundjoin%
\definecolor{currentfill}{rgb}{0.974072,0.862976,0.688750}%
\pgfsetfillcolor{currentfill}%
\pgfsetlinewidth{0.250937pt}%
\definecolor{currentstroke}{rgb}{1.000000,1.000000,1.000000}%
\pgfsetstrokecolor{currentstroke}%
\pgfsetdash{}{0pt}%
\pgfpathmoveto{\pgfqpoint{3.188490in}{2.523396in}}%
\pgfpathlineto{\pgfqpoint{3.276226in}{2.523396in}}%
\pgfpathlineto{\pgfqpoint{3.276226in}{2.435661in}}%
\pgfpathlineto{\pgfqpoint{3.188490in}{2.435661in}}%
\pgfpathlineto{\pgfqpoint{3.188490in}{2.523396in}}%
\pgfusepath{stroke,fill}%
\end{pgfscope}%
\begin{pgfscope}%
\pgfpathrectangle{\pgfqpoint{0.380943in}{2.260189in}}{\pgfqpoint{4.650000in}{0.614151in}}%
\pgfusepath{clip}%
\pgfsetbuttcap%
\pgfsetroundjoin%
\definecolor{currentfill}{rgb}{0.995233,0.991895,0.818977}%
\pgfsetfillcolor{currentfill}%
\pgfsetlinewidth{0.250937pt}%
\definecolor{currentstroke}{rgb}{1.000000,1.000000,1.000000}%
\pgfsetstrokecolor{currentstroke}%
\pgfsetdash{}{0pt}%
\pgfpathmoveto{\pgfqpoint{3.276226in}{2.523396in}}%
\pgfpathlineto{\pgfqpoint{3.363962in}{2.523396in}}%
\pgfpathlineto{\pgfqpoint{3.363962in}{2.435661in}}%
\pgfpathlineto{\pgfqpoint{3.276226in}{2.435661in}}%
\pgfpathlineto{\pgfqpoint{3.276226in}{2.523396in}}%
\pgfusepath{stroke,fill}%
\end{pgfscope}%
\begin{pgfscope}%
\pgfpathrectangle{\pgfqpoint{0.380943in}{2.260189in}}{\pgfqpoint{4.650000in}{0.614151in}}%
\pgfusepath{clip}%
\pgfsetbuttcap%
\pgfsetroundjoin%
\definecolor{currentfill}{rgb}{0.969504,0.885813,0.700930}%
\pgfsetfillcolor{currentfill}%
\pgfsetlinewidth{0.250937pt}%
\definecolor{currentstroke}{rgb}{1.000000,1.000000,1.000000}%
\pgfsetstrokecolor{currentstroke}%
\pgfsetdash{}{0pt}%
\pgfpathmoveto{\pgfqpoint{3.363962in}{2.523396in}}%
\pgfpathlineto{\pgfqpoint{3.451698in}{2.523396in}}%
\pgfpathlineto{\pgfqpoint{3.451698in}{2.435661in}}%
\pgfpathlineto{\pgfqpoint{3.363962in}{2.435661in}}%
\pgfpathlineto{\pgfqpoint{3.363962in}{2.523396in}}%
\pgfusepath{stroke,fill}%
\end{pgfscope}%
\begin{pgfscope}%
\pgfpathrectangle{\pgfqpoint{0.380943in}{2.260189in}}{\pgfqpoint{4.650000in}{0.614151in}}%
\pgfusepath{clip}%
\pgfsetbuttcap%
\pgfsetroundjoin%
\definecolor{currentfill}{rgb}{0.961738,0.927612,0.725598}%
\pgfsetfillcolor{currentfill}%
\pgfsetlinewidth{0.250937pt}%
\definecolor{currentstroke}{rgb}{1.000000,1.000000,1.000000}%
\pgfsetstrokecolor{currentstroke}%
\pgfsetdash{}{0pt}%
\pgfpathmoveto{\pgfqpoint{3.451698in}{2.523396in}}%
\pgfpathlineto{\pgfqpoint{3.539434in}{2.523396in}}%
\pgfpathlineto{\pgfqpoint{3.539434in}{2.435661in}}%
\pgfpathlineto{\pgfqpoint{3.451698in}{2.435661in}}%
\pgfpathlineto{\pgfqpoint{3.451698in}{2.523396in}}%
\pgfusepath{stroke,fill}%
\end{pgfscope}%
\begin{pgfscope}%
\pgfpathrectangle{\pgfqpoint{0.380943in}{2.260189in}}{\pgfqpoint{4.650000in}{0.614151in}}%
\pgfusepath{clip}%
\pgfsetbuttcap%
\pgfsetroundjoin%
\definecolor{currentfill}{rgb}{0.978639,0.841584,0.673679}%
\pgfsetfillcolor{currentfill}%
\pgfsetlinewidth{0.250937pt}%
\definecolor{currentstroke}{rgb}{1.000000,1.000000,1.000000}%
\pgfsetstrokecolor{currentstroke}%
\pgfsetdash{}{0pt}%
\pgfpathmoveto{\pgfqpoint{3.539434in}{2.523396in}}%
\pgfpathlineto{\pgfqpoint{3.627169in}{2.523396in}}%
\pgfpathlineto{\pgfqpoint{3.627169in}{2.435661in}}%
\pgfpathlineto{\pgfqpoint{3.539434in}{2.435661in}}%
\pgfpathlineto{\pgfqpoint{3.539434in}{2.523396in}}%
\pgfusepath{stroke,fill}%
\end{pgfscope}%
\begin{pgfscope}%
\pgfpathrectangle{\pgfqpoint{0.380943in}{2.260189in}}{\pgfqpoint{4.650000in}{0.614151in}}%
\pgfusepath{clip}%
\pgfsetbuttcap%
\pgfsetroundjoin%
\definecolor{currentfill}{rgb}{0.996401,0.724937,0.591557}%
\pgfsetfillcolor{currentfill}%
\pgfsetlinewidth{0.250937pt}%
\definecolor{currentstroke}{rgb}{1.000000,1.000000,1.000000}%
\pgfsetstrokecolor{currentstroke}%
\pgfsetdash{}{0pt}%
\pgfpathmoveto{\pgfqpoint{3.627169in}{2.523396in}}%
\pgfpathlineto{\pgfqpoint{3.714905in}{2.523396in}}%
\pgfpathlineto{\pgfqpoint{3.714905in}{2.435661in}}%
\pgfpathlineto{\pgfqpoint{3.627169in}{2.435661in}}%
\pgfpathlineto{\pgfqpoint{3.627169in}{2.523396in}}%
\pgfusepath{stroke,fill}%
\end{pgfscope}%
\begin{pgfscope}%
\pgfpathrectangle{\pgfqpoint{0.380943in}{2.260189in}}{\pgfqpoint{4.650000in}{0.614151in}}%
\pgfusepath{clip}%
\pgfsetbuttcap%
\pgfsetroundjoin%
\definecolor{currentfill}{rgb}{0.982699,0.823991,0.657439}%
\pgfsetfillcolor{currentfill}%
\pgfsetlinewidth{0.250937pt}%
\definecolor{currentstroke}{rgb}{1.000000,1.000000,1.000000}%
\pgfsetstrokecolor{currentstroke}%
\pgfsetdash{}{0pt}%
\pgfpathmoveto{\pgfqpoint{3.714905in}{2.523396in}}%
\pgfpathlineto{\pgfqpoint{3.802641in}{2.523396in}}%
\pgfpathlineto{\pgfqpoint{3.802641in}{2.435661in}}%
\pgfpathlineto{\pgfqpoint{3.714905in}{2.435661in}}%
\pgfpathlineto{\pgfqpoint{3.714905in}{2.523396in}}%
\pgfusepath{stroke,fill}%
\end{pgfscope}%
\begin{pgfscope}%
\pgfpathrectangle{\pgfqpoint{0.380943in}{2.260189in}}{\pgfqpoint{4.650000in}{0.614151in}}%
\pgfusepath{clip}%
\pgfsetbuttcap%
\pgfsetroundjoin%
\definecolor{currentfill}{rgb}{0.993679,0.753725,0.608074}%
\pgfsetfillcolor{currentfill}%
\pgfsetlinewidth{0.250937pt}%
\definecolor{currentstroke}{rgb}{1.000000,1.000000,1.000000}%
\pgfsetstrokecolor{currentstroke}%
\pgfsetdash{}{0pt}%
\pgfpathmoveto{\pgfqpoint{3.802641in}{2.523396in}}%
\pgfpathlineto{\pgfqpoint{3.890377in}{2.523396in}}%
\pgfpathlineto{\pgfqpoint{3.890377in}{2.435661in}}%
\pgfpathlineto{\pgfqpoint{3.802641in}{2.435661in}}%
\pgfpathlineto{\pgfqpoint{3.802641in}{2.523396in}}%
\pgfusepath{stroke,fill}%
\end{pgfscope}%
\begin{pgfscope}%
\pgfpathrectangle{\pgfqpoint{0.380943in}{2.260189in}}{\pgfqpoint{4.650000in}{0.614151in}}%
\pgfusepath{clip}%
\pgfsetbuttcap%
\pgfsetroundjoin%
\definecolor{currentfill}{rgb}{1.000000,0.615379,0.534779}%
\pgfsetfillcolor{currentfill}%
\pgfsetlinewidth{0.250937pt}%
\definecolor{currentstroke}{rgb}{1.000000,1.000000,1.000000}%
\pgfsetstrokecolor{currentstroke}%
\pgfsetdash{}{0pt}%
\pgfpathmoveto{\pgfqpoint{3.890377in}{2.523396in}}%
\pgfpathlineto{\pgfqpoint{3.978113in}{2.523396in}}%
\pgfpathlineto{\pgfqpoint{3.978113in}{2.435661in}}%
\pgfpathlineto{\pgfqpoint{3.890377in}{2.435661in}}%
\pgfpathlineto{\pgfqpoint{3.890377in}{2.523396in}}%
\pgfusepath{stroke,fill}%
\end{pgfscope}%
\begin{pgfscope}%
\pgfpathrectangle{\pgfqpoint{0.380943in}{2.260189in}}{\pgfqpoint{4.650000in}{0.614151in}}%
\pgfusepath{clip}%
\pgfsetbuttcap%
\pgfsetroundjoin%
\definecolor{currentfill}{rgb}{0.964937,0.908651,0.713110}%
\pgfsetfillcolor{currentfill}%
\pgfsetlinewidth{0.250937pt}%
\definecolor{currentstroke}{rgb}{1.000000,1.000000,1.000000}%
\pgfsetstrokecolor{currentstroke}%
\pgfsetdash{}{0pt}%
\pgfpathmoveto{\pgfqpoint{3.978113in}{2.523396in}}%
\pgfpathlineto{\pgfqpoint{4.065849in}{2.523396in}}%
\pgfpathlineto{\pgfqpoint{4.065849in}{2.435661in}}%
\pgfpathlineto{\pgfqpoint{3.978113in}{2.435661in}}%
\pgfpathlineto{\pgfqpoint{3.978113in}{2.523396in}}%
\pgfusepath{stroke,fill}%
\end{pgfscope}%
\begin{pgfscope}%
\pgfpathrectangle{\pgfqpoint{0.380943in}{2.260189in}}{\pgfqpoint{4.650000in}{0.614151in}}%
\pgfusepath{clip}%
\pgfsetbuttcap%
\pgfsetroundjoin%
\definecolor{currentfill}{rgb}{0.997924,0.685352,0.570242}%
\pgfsetfillcolor{currentfill}%
\pgfsetlinewidth{0.250937pt}%
\definecolor{currentstroke}{rgb}{1.000000,1.000000,1.000000}%
\pgfsetstrokecolor{currentstroke}%
\pgfsetdash{}{0pt}%
\pgfpathmoveto{\pgfqpoint{4.065849in}{2.523396in}}%
\pgfpathlineto{\pgfqpoint{4.153585in}{2.523396in}}%
\pgfpathlineto{\pgfqpoint{4.153585in}{2.435661in}}%
\pgfpathlineto{\pgfqpoint{4.065849in}{2.435661in}}%
\pgfpathlineto{\pgfqpoint{4.065849in}{2.523396in}}%
\pgfusepath{stroke,fill}%
\end{pgfscope}%
\begin{pgfscope}%
\pgfpathrectangle{\pgfqpoint{0.380943in}{2.260189in}}{\pgfqpoint{4.650000in}{0.614151in}}%
\pgfusepath{clip}%
\pgfsetbuttcap%
\pgfsetroundjoin%
\definecolor{currentfill}{rgb}{1.000000,0.584929,0.522599}%
\pgfsetfillcolor{currentfill}%
\pgfsetlinewidth{0.250937pt}%
\definecolor{currentstroke}{rgb}{1.000000,1.000000,1.000000}%
\pgfsetstrokecolor{currentstroke}%
\pgfsetdash{}{0pt}%
\pgfpathmoveto{\pgfqpoint{4.153585in}{2.523396in}}%
\pgfpathlineto{\pgfqpoint{4.241320in}{2.523396in}}%
\pgfpathlineto{\pgfqpoint{4.241320in}{2.435661in}}%
\pgfpathlineto{\pgfqpoint{4.153585in}{2.435661in}}%
\pgfpathlineto{\pgfqpoint{4.153585in}{2.523396in}}%
\pgfusepath{stroke,fill}%
\end{pgfscope}%
\begin{pgfscope}%
\pgfpathrectangle{\pgfqpoint{0.380943in}{2.260189in}}{\pgfqpoint{4.650000in}{0.614151in}}%
\pgfusepath{clip}%
\pgfsetbuttcap%
\pgfsetroundjoin%
\definecolor{currentfill}{rgb}{0.974072,0.862976,0.688750}%
\pgfsetfillcolor{currentfill}%
\pgfsetlinewidth{0.250937pt}%
\definecolor{currentstroke}{rgb}{1.000000,1.000000,1.000000}%
\pgfsetstrokecolor{currentstroke}%
\pgfsetdash{}{0pt}%
\pgfpathmoveto{\pgfqpoint{4.241320in}{2.523396in}}%
\pgfpathlineto{\pgfqpoint{4.329056in}{2.523396in}}%
\pgfpathlineto{\pgfqpoint{4.329056in}{2.435661in}}%
\pgfpathlineto{\pgfqpoint{4.241320in}{2.435661in}}%
\pgfpathlineto{\pgfqpoint{4.241320in}{2.523396in}}%
\pgfusepath{stroke,fill}%
\end{pgfscope}%
\begin{pgfscope}%
\pgfpathrectangle{\pgfqpoint{0.380943in}{2.260189in}}{\pgfqpoint{4.650000in}{0.614151in}}%
\pgfusepath{clip}%
\pgfsetbuttcap%
\pgfsetroundjoin%
\definecolor{currentfill}{rgb}{1.000000,0.496547,0.486059}%
\pgfsetfillcolor{currentfill}%
\pgfsetlinewidth{0.250937pt}%
\definecolor{currentstroke}{rgb}{1.000000,1.000000,1.000000}%
\pgfsetstrokecolor{currentstroke}%
\pgfsetdash{}{0pt}%
\pgfpathmoveto{\pgfqpoint{4.329056in}{2.523396in}}%
\pgfpathlineto{\pgfqpoint{4.416792in}{2.523396in}}%
\pgfpathlineto{\pgfqpoint{4.416792in}{2.435661in}}%
\pgfpathlineto{\pgfqpoint{4.329056in}{2.435661in}}%
\pgfpathlineto{\pgfqpoint{4.329056in}{2.523396in}}%
\pgfusepath{stroke,fill}%
\end{pgfscope}%
\begin{pgfscope}%
\pgfpathrectangle{\pgfqpoint{0.380943in}{2.260189in}}{\pgfqpoint{4.650000in}{0.614151in}}%
\pgfusepath{clip}%
\pgfsetbuttcap%
\pgfsetroundjoin%
\definecolor{currentfill}{rgb}{0.997924,0.685352,0.570242}%
\pgfsetfillcolor{currentfill}%
\pgfsetlinewidth{0.250937pt}%
\definecolor{currentstroke}{rgb}{1.000000,1.000000,1.000000}%
\pgfsetstrokecolor{currentstroke}%
\pgfsetdash{}{0pt}%
\pgfpathmoveto{\pgfqpoint{4.416792in}{2.523396in}}%
\pgfpathlineto{\pgfqpoint{4.504528in}{2.523396in}}%
\pgfpathlineto{\pgfqpoint{4.504528in}{2.435661in}}%
\pgfpathlineto{\pgfqpoint{4.416792in}{2.435661in}}%
\pgfpathlineto{\pgfqpoint{4.416792in}{2.523396in}}%
\pgfusepath{stroke,fill}%
\end{pgfscope}%
\begin{pgfscope}%
\pgfpathrectangle{\pgfqpoint{0.380943in}{2.260189in}}{\pgfqpoint{4.650000in}{0.614151in}}%
\pgfusepath{clip}%
\pgfsetbuttcap%
\pgfsetroundjoin%
\definecolor{currentfill}{rgb}{0.982699,0.823991,0.657439}%
\pgfsetfillcolor{currentfill}%
\pgfsetlinewidth{0.250937pt}%
\definecolor{currentstroke}{rgb}{1.000000,1.000000,1.000000}%
\pgfsetstrokecolor{currentstroke}%
\pgfsetdash{}{0pt}%
\pgfpathmoveto{\pgfqpoint{4.504528in}{2.523396in}}%
\pgfpathlineto{\pgfqpoint{4.592264in}{2.523396in}}%
\pgfpathlineto{\pgfqpoint{4.592264in}{2.435661in}}%
\pgfpathlineto{\pgfqpoint{4.504528in}{2.435661in}}%
\pgfpathlineto{\pgfqpoint{4.504528in}{2.523396in}}%
\pgfusepath{stroke,fill}%
\end{pgfscope}%
\begin{pgfscope}%
\pgfpathrectangle{\pgfqpoint{0.380943in}{2.260189in}}{\pgfqpoint{4.650000in}{0.614151in}}%
\pgfusepath{clip}%
\pgfsetbuttcap%
\pgfsetroundjoin%
\definecolor{currentfill}{rgb}{0.969504,0.885813,0.700930}%
\pgfsetfillcolor{currentfill}%
\pgfsetlinewidth{0.250937pt}%
\definecolor{currentstroke}{rgb}{1.000000,1.000000,1.000000}%
\pgfsetstrokecolor{currentstroke}%
\pgfsetdash{}{0pt}%
\pgfpathmoveto{\pgfqpoint{4.592264in}{2.523396in}}%
\pgfpathlineto{\pgfqpoint{4.680000in}{2.523396in}}%
\pgfpathlineto{\pgfqpoint{4.680000in}{2.435661in}}%
\pgfpathlineto{\pgfqpoint{4.592264in}{2.435661in}}%
\pgfpathlineto{\pgfqpoint{4.592264in}{2.523396in}}%
\pgfusepath{stroke,fill}%
\end{pgfscope}%
\begin{pgfscope}%
\pgfpathrectangle{\pgfqpoint{0.380943in}{2.260189in}}{\pgfqpoint{4.650000in}{0.614151in}}%
\pgfusepath{clip}%
\pgfsetbuttcap%
\pgfsetroundjoin%
\definecolor{currentfill}{rgb}{1.000000,0.554479,0.510419}%
\pgfsetfillcolor{currentfill}%
\pgfsetlinewidth{0.250937pt}%
\definecolor{currentstroke}{rgb}{1.000000,1.000000,1.000000}%
\pgfsetstrokecolor{currentstroke}%
\pgfsetdash{}{0pt}%
\pgfpathmoveto{\pgfqpoint{4.680000in}{2.523396in}}%
\pgfpathlineto{\pgfqpoint{4.767736in}{2.523396in}}%
\pgfpathlineto{\pgfqpoint{4.767736in}{2.435661in}}%
\pgfpathlineto{\pgfqpoint{4.680000in}{2.435661in}}%
\pgfpathlineto{\pgfqpoint{4.680000in}{2.523396in}}%
\pgfusepath{stroke,fill}%
\end{pgfscope}%
\begin{pgfscope}%
\pgfpathrectangle{\pgfqpoint{0.380943in}{2.260189in}}{\pgfqpoint{4.650000in}{0.614151in}}%
\pgfusepath{clip}%
\pgfsetbuttcap%
\pgfsetroundjoin%
\definecolor{currentfill}{rgb}{0.990634,0.779608,0.623299}%
\pgfsetfillcolor{currentfill}%
\pgfsetlinewidth{0.250937pt}%
\definecolor{currentstroke}{rgb}{1.000000,1.000000,1.000000}%
\pgfsetstrokecolor{currentstroke}%
\pgfsetdash{}{0pt}%
\pgfpathmoveto{\pgfqpoint{4.767736in}{2.523396in}}%
\pgfpathlineto{\pgfqpoint{4.855471in}{2.523396in}}%
\pgfpathlineto{\pgfqpoint{4.855471in}{2.435661in}}%
\pgfpathlineto{\pgfqpoint{4.767736in}{2.435661in}}%
\pgfpathlineto{\pgfqpoint{4.767736in}{2.523396in}}%
\pgfusepath{stroke,fill}%
\end{pgfscope}%
\begin{pgfscope}%
\pgfpathrectangle{\pgfqpoint{0.380943in}{2.260189in}}{\pgfqpoint{4.650000in}{0.614151in}}%
\pgfusepath{clip}%
\pgfsetbuttcap%
\pgfsetroundjoin%
\definecolor{currentfill}{rgb}{0.964783,0.940131,0.739808}%
\pgfsetfillcolor{currentfill}%
\pgfsetlinewidth{0.250937pt}%
\definecolor{currentstroke}{rgb}{1.000000,1.000000,1.000000}%
\pgfsetstrokecolor{currentstroke}%
\pgfsetdash{}{0pt}%
\pgfpathmoveto{\pgfqpoint{4.855471in}{2.523396in}}%
\pgfpathlineto{\pgfqpoint{4.943207in}{2.523396in}}%
\pgfpathlineto{\pgfqpoint{4.943207in}{2.435661in}}%
\pgfpathlineto{\pgfqpoint{4.855471in}{2.435661in}}%
\pgfpathlineto{\pgfqpoint{4.855471in}{2.523396in}}%
\pgfusepath{stroke,fill}%
\end{pgfscope}%
\begin{pgfscope}%
\pgfpathrectangle{\pgfqpoint{0.380943in}{2.260189in}}{\pgfqpoint{4.650000in}{0.614151in}}%
\pgfusepath{clip}%
\pgfsetbuttcap%
\pgfsetroundjoin%
\pgfsetlinewidth{0.250937pt}%
\definecolor{currentstroke}{rgb}{1.000000,1.000000,1.000000}%
\pgfsetstrokecolor{currentstroke}%
\pgfsetdash{}{0pt}%
\pgfpathmoveto{\pgfqpoint{4.943207in}{2.523396in}}%
\pgfpathlineto{\pgfqpoint{5.030943in}{2.523396in}}%
\pgfpathlineto{\pgfqpoint{5.030943in}{2.435661in}}%
\pgfpathlineto{\pgfqpoint{4.943207in}{2.435661in}}%
\pgfpathlineto{\pgfqpoint{4.943207in}{2.523396in}}%
\pgfusepath{stroke}%
\end{pgfscope}%
\begin{pgfscope}%
\pgfpathrectangle{\pgfqpoint{0.380943in}{2.260189in}}{\pgfqpoint{4.650000in}{0.614151in}}%
\pgfusepath{clip}%
\pgfsetbuttcap%
\pgfsetroundjoin%
\definecolor{currentfill}{rgb}{0.980008,0.966013,0.779393}%
\pgfsetfillcolor{currentfill}%
\pgfsetlinewidth{0.250937pt}%
\definecolor{currentstroke}{rgb}{1.000000,1.000000,1.000000}%
\pgfsetstrokecolor{currentstroke}%
\pgfsetdash{}{0pt}%
\pgfpathmoveto{\pgfqpoint{0.380943in}{2.435661in}}%
\pgfpathlineto{\pgfqpoint{0.468679in}{2.435661in}}%
\pgfpathlineto{\pgfqpoint{0.468679in}{2.347925in}}%
\pgfpathlineto{\pgfqpoint{0.380943in}{2.347925in}}%
\pgfpathlineto{\pgfqpoint{0.380943in}{2.435661in}}%
\pgfusepath{stroke,fill}%
\end{pgfscope}%
\begin{pgfscope}%
\pgfpathrectangle{\pgfqpoint{0.380943in}{2.260189in}}{\pgfqpoint{4.650000in}{0.614151in}}%
\pgfusepath{clip}%
\pgfsetbuttcap%
\pgfsetroundjoin%
\definecolor{currentfill}{rgb}{0.961738,0.927612,0.725598}%
\pgfsetfillcolor{currentfill}%
\pgfsetlinewidth{0.250937pt}%
\definecolor{currentstroke}{rgb}{1.000000,1.000000,1.000000}%
\pgfsetstrokecolor{currentstroke}%
\pgfsetdash{}{0pt}%
\pgfpathmoveto{\pgfqpoint{0.468679in}{2.435661in}}%
\pgfpathlineto{\pgfqpoint{0.556415in}{2.435661in}}%
\pgfpathlineto{\pgfqpoint{0.556415in}{2.347925in}}%
\pgfpathlineto{\pgfqpoint{0.468679in}{2.347925in}}%
\pgfpathlineto{\pgfqpoint{0.468679in}{2.435661in}}%
\pgfusepath{stroke,fill}%
\end{pgfscope}%
\begin{pgfscope}%
\pgfpathrectangle{\pgfqpoint{0.380943in}{2.260189in}}{\pgfqpoint{4.650000in}{0.614151in}}%
\pgfusepath{clip}%
\pgfsetbuttcap%
\pgfsetroundjoin%
\definecolor{currentfill}{rgb}{0.995233,0.991895,0.818977}%
\pgfsetfillcolor{currentfill}%
\pgfsetlinewidth{0.250937pt}%
\definecolor{currentstroke}{rgb}{1.000000,1.000000,1.000000}%
\pgfsetstrokecolor{currentstroke}%
\pgfsetdash{}{0pt}%
\pgfpathmoveto{\pgfqpoint{0.556415in}{2.435661in}}%
\pgfpathlineto{\pgfqpoint{0.644151in}{2.435661in}}%
\pgfpathlineto{\pgfqpoint{0.644151in}{2.347925in}}%
\pgfpathlineto{\pgfqpoint{0.556415in}{2.347925in}}%
\pgfpathlineto{\pgfqpoint{0.556415in}{2.435661in}}%
\pgfusepath{stroke,fill}%
\end{pgfscope}%
\begin{pgfscope}%
\pgfpathrectangle{\pgfqpoint{0.380943in}{2.260189in}}{\pgfqpoint{4.650000in}{0.614151in}}%
\pgfusepath{clip}%
\pgfsetbuttcap%
\pgfsetroundjoin%
\definecolor{currentfill}{rgb}{0.961738,0.927612,0.725598}%
\pgfsetfillcolor{currentfill}%
\pgfsetlinewidth{0.250937pt}%
\definecolor{currentstroke}{rgb}{1.000000,1.000000,1.000000}%
\pgfsetstrokecolor{currentstroke}%
\pgfsetdash{}{0pt}%
\pgfpathmoveto{\pgfqpoint{0.644151in}{2.435661in}}%
\pgfpathlineto{\pgfqpoint{0.731886in}{2.435661in}}%
\pgfpathlineto{\pgfqpoint{0.731886in}{2.347925in}}%
\pgfpathlineto{\pgfqpoint{0.644151in}{2.347925in}}%
\pgfpathlineto{\pgfqpoint{0.644151in}{2.435661in}}%
\pgfusepath{stroke,fill}%
\end{pgfscope}%
\begin{pgfscope}%
\pgfpathrectangle{\pgfqpoint{0.380943in}{2.260189in}}{\pgfqpoint{4.650000in}{0.614151in}}%
\pgfusepath{clip}%
\pgfsetbuttcap%
\pgfsetroundjoin%
\definecolor{currentfill}{rgb}{0.980008,0.966013,0.779393}%
\pgfsetfillcolor{currentfill}%
\pgfsetlinewidth{0.250937pt}%
\definecolor{currentstroke}{rgb}{1.000000,1.000000,1.000000}%
\pgfsetstrokecolor{currentstroke}%
\pgfsetdash{}{0pt}%
\pgfpathmoveto{\pgfqpoint{0.731886in}{2.435661in}}%
\pgfpathlineto{\pgfqpoint{0.819622in}{2.435661in}}%
\pgfpathlineto{\pgfqpoint{0.819622in}{2.347925in}}%
\pgfpathlineto{\pgfqpoint{0.731886in}{2.347925in}}%
\pgfpathlineto{\pgfqpoint{0.731886in}{2.435661in}}%
\pgfusepath{stroke,fill}%
\end{pgfscope}%
\begin{pgfscope}%
\pgfpathrectangle{\pgfqpoint{0.380943in}{2.260189in}}{\pgfqpoint{4.650000in}{0.614151in}}%
\pgfusepath{clip}%
\pgfsetbuttcap%
\pgfsetroundjoin%
\definecolor{currentfill}{rgb}{0.978639,0.841584,0.673679}%
\pgfsetfillcolor{currentfill}%
\pgfsetlinewidth{0.250937pt}%
\definecolor{currentstroke}{rgb}{1.000000,1.000000,1.000000}%
\pgfsetstrokecolor{currentstroke}%
\pgfsetdash{}{0pt}%
\pgfpathmoveto{\pgfqpoint{0.819622in}{2.435661in}}%
\pgfpathlineto{\pgfqpoint{0.907358in}{2.435661in}}%
\pgfpathlineto{\pgfqpoint{0.907358in}{2.347925in}}%
\pgfpathlineto{\pgfqpoint{0.819622in}{2.347925in}}%
\pgfpathlineto{\pgfqpoint{0.819622in}{2.435661in}}%
\pgfusepath{stroke,fill}%
\end{pgfscope}%
\begin{pgfscope}%
\pgfpathrectangle{\pgfqpoint{0.380943in}{2.260189in}}{\pgfqpoint{4.650000in}{0.614151in}}%
\pgfusepath{clip}%
\pgfsetbuttcap%
\pgfsetroundjoin%
\definecolor{currentfill}{rgb}{0.980008,0.966013,0.779393}%
\pgfsetfillcolor{currentfill}%
\pgfsetlinewidth{0.250937pt}%
\definecolor{currentstroke}{rgb}{1.000000,1.000000,1.000000}%
\pgfsetstrokecolor{currentstroke}%
\pgfsetdash{}{0pt}%
\pgfpathmoveto{\pgfqpoint{0.907358in}{2.435661in}}%
\pgfpathlineto{\pgfqpoint{0.995094in}{2.435661in}}%
\pgfpathlineto{\pgfqpoint{0.995094in}{2.347925in}}%
\pgfpathlineto{\pgfqpoint{0.907358in}{2.347925in}}%
\pgfpathlineto{\pgfqpoint{0.907358in}{2.435661in}}%
\pgfusepath{stroke,fill}%
\end{pgfscope}%
\begin{pgfscope}%
\pgfpathrectangle{\pgfqpoint{0.380943in}{2.260189in}}{\pgfqpoint{4.650000in}{0.614151in}}%
\pgfusepath{clip}%
\pgfsetbuttcap%
\pgfsetroundjoin%
\definecolor{currentfill}{rgb}{0.980008,0.966013,0.779393}%
\pgfsetfillcolor{currentfill}%
\pgfsetlinewidth{0.250937pt}%
\definecolor{currentstroke}{rgb}{1.000000,1.000000,1.000000}%
\pgfsetstrokecolor{currentstroke}%
\pgfsetdash{}{0pt}%
\pgfpathmoveto{\pgfqpoint{0.995094in}{2.435661in}}%
\pgfpathlineto{\pgfqpoint{1.082830in}{2.435661in}}%
\pgfpathlineto{\pgfqpoint{1.082830in}{2.347925in}}%
\pgfpathlineto{\pgfqpoint{0.995094in}{2.347925in}}%
\pgfpathlineto{\pgfqpoint{0.995094in}{2.435661in}}%
\pgfusepath{stroke,fill}%
\end{pgfscope}%
\begin{pgfscope}%
\pgfpathrectangle{\pgfqpoint{0.380943in}{2.260189in}}{\pgfqpoint{4.650000in}{0.614151in}}%
\pgfusepath{clip}%
\pgfsetbuttcap%
\pgfsetroundjoin%
\definecolor{currentfill}{rgb}{0.964783,0.940131,0.739808}%
\pgfsetfillcolor{currentfill}%
\pgfsetlinewidth{0.250937pt}%
\definecolor{currentstroke}{rgb}{1.000000,1.000000,1.000000}%
\pgfsetstrokecolor{currentstroke}%
\pgfsetdash{}{0pt}%
\pgfpathmoveto{\pgfqpoint{1.082830in}{2.435661in}}%
\pgfpathlineto{\pgfqpoint{1.170566in}{2.435661in}}%
\pgfpathlineto{\pgfqpoint{1.170566in}{2.347925in}}%
\pgfpathlineto{\pgfqpoint{1.082830in}{2.347925in}}%
\pgfpathlineto{\pgfqpoint{1.082830in}{2.435661in}}%
\pgfusepath{stroke,fill}%
\end{pgfscope}%
\begin{pgfscope}%
\pgfpathrectangle{\pgfqpoint{0.380943in}{2.260189in}}{\pgfqpoint{4.650000in}{0.614151in}}%
\pgfusepath{clip}%
\pgfsetbuttcap%
\pgfsetroundjoin%
\definecolor{currentfill}{rgb}{0.964783,0.940131,0.739808}%
\pgfsetfillcolor{currentfill}%
\pgfsetlinewidth{0.250937pt}%
\definecolor{currentstroke}{rgb}{1.000000,1.000000,1.000000}%
\pgfsetstrokecolor{currentstroke}%
\pgfsetdash{}{0pt}%
\pgfpathmoveto{\pgfqpoint{1.170566in}{2.435661in}}%
\pgfpathlineto{\pgfqpoint{1.258302in}{2.435661in}}%
\pgfpathlineto{\pgfqpoint{1.258302in}{2.347925in}}%
\pgfpathlineto{\pgfqpoint{1.170566in}{2.347925in}}%
\pgfpathlineto{\pgfqpoint{1.170566in}{2.435661in}}%
\pgfusepath{stroke,fill}%
\end{pgfscope}%
\begin{pgfscope}%
\pgfpathrectangle{\pgfqpoint{0.380943in}{2.260189in}}{\pgfqpoint{4.650000in}{0.614151in}}%
\pgfusepath{clip}%
\pgfsetbuttcap%
\pgfsetroundjoin%
\definecolor{currentfill}{rgb}{0.980008,0.966013,0.779393}%
\pgfsetfillcolor{currentfill}%
\pgfsetlinewidth{0.250937pt}%
\definecolor{currentstroke}{rgb}{1.000000,1.000000,1.000000}%
\pgfsetstrokecolor{currentstroke}%
\pgfsetdash{}{0pt}%
\pgfpathmoveto{\pgfqpoint{1.258302in}{2.435661in}}%
\pgfpathlineto{\pgfqpoint{1.346037in}{2.435661in}}%
\pgfpathlineto{\pgfqpoint{1.346037in}{2.347925in}}%
\pgfpathlineto{\pgfqpoint{1.258302in}{2.347925in}}%
\pgfpathlineto{\pgfqpoint{1.258302in}{2.435661in}}%
\pgfusepath{stroke,fill}%
\end{pgfscope}%
\begin{pgfscope}%
\pgfpathrectangle{\pgfqpoint{0.380943in}{2.260189in}}{\pgfqpoint{4.650000in}{0.614151in}}%
\pgfusepath{clip}%
\pgfsetbuttcap%
\pgfsetroundjoin%
\definecolor{currentfill}{rgb}{0.995233,0.991895,0.818977}%
\pgfsetfillcolor{currentfill}%
\pgfsetlinewidth{0.250937pt}%
\definecolor{currentstroke}{rgb}{1.000000,1.000000,1.000000}%
\pgfsetstrokecolor{currentstroke}%
\pgfsetdash{}{0pt}%
\pgfpathmoveto{\pgfqpoint{1.346037in}{2.435661in}}%
\pgfpathlineto{\pgfqpoint{1.433773in}{2.435661in}}%
\pgfpathlineto{\pgfqpoint{1.433773in}{2.347925in}}%
\pgfpathlineto{\pgfqpoint{1.346037in}{2.347925in}}%
\pgfpathlineto{\pgfqpoint{1.346037in}{2.435661in}}%
\pgfusepath{stroke,fill}%
\end{pgfscope}%
\begin{pgfscope}%
\pgfpathrectangle{\pgfqpoint{0.380943in}{2.260189in}}{\pgfqpoint{4.650000in}{0.614151in}}%
\pgfusepath{clip}%
\pgfsetbuttcap%
\pgfsetroundjoin%
\definecolor{currentfill}{rgb}{1.000000,1.000000,0.895579}%
\pgfsetfillcolor{currentfill}%
\pgfsetlinewidth{0.250937pt}%
\definecolor{currentstroke}{rgb}{1.000000,1.000000,1.000000}%
\pgfsetstrokecolor{currentstroke}%
\pgfsetdash{}{0pt}%
\pgfpathmoveto{\pgfqpoint{1.433773in}{2.435661in}}%
\pgfpathlineto{\pgfqpoint{1.521509in}{2.435661in}}%
\pgfpathlineto{\pgfqpoint{1.521509in}{2.347925in}}%
\pgfpathlineto{\pgfqpoint{1.433773in}{2.347925in}}%
\pgfpathlineto{\pgfqpoint{1.433773in}{2.435661in}}%
\pgfusepath{stroke,fill}%
\end{pgfscope}%
\begin{pgfscope}%
\pgfpathrectangle{\pgfqpoint{0.380943in}{2.260189in}}{\pgfqpoint{4.650000in}{0.614151in}}%
\pgfusepath{clip}%
\pgfsetbuttcap%
\pgfsetroundjoin%
\definecolor{currentfill}{rgb}{1.000000,1.000000,0.857516}%
\pgfsetfillcolor{currentfill}%
\pgfsetlinewidth{0.250937pt}%
\definecolor{currentstroke}{rgb}{1.000000,1.000000,1.000000}%
\pgfsetstrokecolor{currentstroke}%
\pgfsetdash{}{0pt}%
\pgfpathmoveto{\pgfqpoint{1.521509in}{2.435661in}}%
\pgfpathlineto{\pgfqpoint{1.609245in}{2.435661in}}%
\pgfpathlineto{\pgfqpoint{1.609245in}{2.347925in}}%
\pgfpathlineto{\pgfqpoint{1.521509in}{2.347925in}}%
\pgfpathlineto{\pgfqpoint{1.521509in}{2.435661in}}%
\pgfusepath{stroke,fill}%
\end{pgfscope}%
\begin{pgfscope}%
\pgfpathrectangle{\pgfqpoint{0.380943in}{2.260189in}}{\pgfqpoint{4.650000in}{0.614151in}}%
\pgfusepath{clip}%
\pgfsetbuttcap%
\pgfsetroundjoin%
\definecolor{currentfill}{rgb}{1.000000,1.000000,0.929412}%
\pgfsetfillcolor{currentfill}%
\pgfsetlinewidth{0.250937pt}%
\definecolor{currentstroke}{rgb}{1.000000,1.000000,1.000000}%
\pgfsetstrokecolor{currentstroke}%
\pgfsetdash{}{0pt}%
\pgfpathmoveto{\pgfqpoint{1.609245in}{2.435661in}}%
\pgfpathlineto{\pgfqpoint{1.696981in}{2.435661in}}%
\pgfpathlineto{\pgfqpoint{1.696981in}{2.347925in}}%
\pgfpathlineto{\pgfqpoint{1.609245in}{2.347925in}}%
\pgfpathlineto{\pgfqpoint{1.609245in}{2.435661in}}%
\pgfusepath{stroke,fill}%
\end{pgfscope}%
\begin{pgfscope}%
\pgfpathrectangle{\pgfqpoint{0.380943in}{2.260189in}}{\pgfqpoint{4.650000in}{0.614151in}}%
\pgfusepath{clip}%
\pgfsetbuttcap%
\pgfsetroundjoin%
\definecolor{currentfill}{rgb}{0.980008,0.966013,0.779393}%
\pgfsetfillcolor{currentfill}%
\pgfsetlinewidth{0.250937pt}%
\definecolor{currentstroke}{rgb}{1.000000,1.000000,1.000000}%
\pgfsetstrokecolor{currentstroke}%
\pgfsetdash{}{0pt}%
\pgfpathmoveto{\pgfqpoint{1.696981in}{2.435661in}}%
\pgfpathlineto{\pgfqpoint{1.784717in}{2.435661in}}%
\pgfpathlineto{\pgfqpoint{1.784717in}{2.347925in}}%
\pgfpathlineto{\pgfqpoint{1.696981in}{2.347925in}}%
\pgfpathlineto{\pgfqpoint{1.696981in}{2.435661in}}%
\pgfusepath{stroke,fill}%
\end{pgfscope}%
\begin{pgfscope}%
\pgfpathrectangle{\pgfqpoint{0.380943in}{2.260189in}}{\pgfqpoint{4.650000in}{0.614151in}}%
\pgfusepath{clip}%
\pgfsetbuttcap%
\pgfsetroundjoin%
\definecolor{currentfill}{rgb}{1.000000,1.000000,0.857516}%
\pgfsetfillcolor{currentfill}%
\pgfsetlinewidth{0.250937pt}%
\definecolor{currentstroke}{rgb}{1.000000,1.000000,1.000000}%
\pgfsetstrokecolor{currentstroke}%
\pgfsetdash{}{0pt}%
\pgfpathmoveto{\pgfqpoint{1.784717in}{2.435661in}}%
\pgfpathlineto{\pgfqpoint{1.872452in}{2.435661in}}%
\pgfpathlineto{\pgfqpoint{1.872452in}{2.347925in}}%
\pgfpathlineto{\pgfqpoint{1.784717in}{2.347925in}}%
\pgfpathlineto{\pgfqpoint{1.784717in}{2.435661in}}%
\pgfusepath{stroke,fill}%
\end{pgfscope}%
\begin{pgfscope}%
\pgfpathrectangle{\pgfqpoint{0.380943in}{2.260189in}}{\pgfqpoint{4.650000in}{0.614151in}}%
\pgfusepath{clip}%
\pgfsetbuttcap%
\pgfsetroundjoin%
\definecolor{currentfill}{rgb}{0.995233,0.991895,0.818977}%
\pgfsetfillcolor{currentfill}%
\pgfsetlinewidth{0.250937pt}%
\definecolor{currentstroke}{rgb}{1.000000,1.000000,1.000000}%
\pgfsetstrokecolor{currentstroke}%
\pgfsetdash{}{0pt}%
\pgfpathmoveto{\pgfqpoint{1.872452in}{2.435661in}}%
\pgfpathlineto{\pgfqpoint{1.960188in}{2.435661in}}%
\pgfpathlineto{\pgfqpoint{1.960188in}{2.347925in}}%
\pgfpathlineto{\pgfqpoint{1.872452in}{2.347925in}}%
\pgfpathlineto{\pgfqpoint{1.872452in}{2.435661in}}%
\pgfusepath{stroke,fill}%
\end{pgfscope}%
\begin{pgfscope}%
\pgfpathrectangle{\pgfqpoint{0.380943in}{2.260189in}}{\pgfqpoint{4.650000in}{0.614151in}}%
\pgfusepath{clip}%
\pgfsetbuttcap%
\pgfsetroundjoin%
\definecolor{currentfill}{rgb}{0.964783,0.940131,0.739808}%
\pgfsetfillcolor{currentfill}%
\pgfsetlinewidth{0.250937pt}%
\definecolor{currentstroke}{rgb}{1.000000,1.000000,1.000000}%
\pgfsetstrokecolor{currentstroke}%
\pgfsetdash{}{0pt}%
\pgfpathmoveto{\pgfqpoint{1.960188in}{2.435661in}}%
\pgfpathlineto{\pgfqpoint{2.047924in}{2.435661in}}%
\pgfpathlineto{\pgfqpoint{2.047924in}{2.347925in}}%
\pgfpathlineto{\pgfqpoint{1.960188in}{2.347925in}}%
\pgfpathlineto{\pgfqpoint{1.960188in}{2.435661in}}%
\pgfusepath{stroke,fill}%
\end{pgfscope}%
\begin{pgfscope}%
\pgfpathrectangle{\pgfqpoint{0.380943in}{2.260189in}}{\pgfqpoint{4.650000in}{0.614151in}}%
\pgfusepath{clip}%
\pgfsetbuttcap%
\pgfsetroundjoin%
\definecolor{currentfill}{rgb}{1.000000,1.000000,0.895579}%
\pgfsetfillcolor{currentfill}%
\pgfsetlinewidth{0.250937pt}%
\definecolor{currentstroke}{rgb}{1.000000,1.000000,1.000000}%
\pgfsetstrokecolor{currentstroke}%
\pgfsetdash{}{0pt}%
\pgfpathmoveto{\pgfqpoint{2.047924in}{2.435661in}}%
\pgfpathlineto{\pgfqpoint{2.135660in}{2.435661in}}%
\pgfpathlineto{\pgfqpoint{2.135660in}{2.347925in}}%
\pgfpathlineto{\pgfqpoint{2.047924in}{2.347925in}}%
\pgfpathlineto{\pgfqpoint{2.047924in}{2.435661in}}%
\pgfusepath{stroke,fill}%
\end{pgfscope}%
\begin{pgfscope}%
\pgfpathrectangle{\pgfqpoint{0.380943in}{2.260189in}}{\pgfqpoint{4.650000in}{0.614151in}}%
\pgfusepath{clip}%
\pgfsetbuttcap%
\pgfsetroundjoin%
\definecolor{currentfill}{rgb}{0.961738,0.927612,0.725598}%
\pgfsetfillcolor{currentfill}%
\pgfsetlinewidth{0.250937pt}%
\definecolor{currentstroke}{rgb}{1.000000,1.000000,1.000000}%
\pgfsetstrokecolor{currentstroke}%
\pgfsetdash{}{0pt}%
\pgfpathmoveto{\pgfqpoint{2.135660in}{2.435661in}}%
\pgfpathlineto{\pgfqpoint{2.223396in}{2.435661in}}%
\pgfpathlineto{\pgfqpoint{2.223396in}{2.347925in}}%
\pgfpathlineto{\pgfqpoint{2.135660in}{2.347925in}}%
\pgfpathlineto{\pgfqpoint{2.135660in}{2.435661in}}%
\pgfusepath{stroke,fill}%
\end{pgfscope}%
\begin{pgfscope}%
\pgfpathrectangle{\pgfqpoint{0.380943in}{2.260189in}}{\pgfqpoint{4.650000in}{0.614151in}}%
\pgfusepath{clip}%
\pgfsetbuttcap%
\pgfsetroundjoin%
\definecolor{currentfill}{rgb}{0.964783,0.940131,0.739808}%
\pgfsetfillcolor{currentfill}%
\pgfsetlinewidth{0.250937pt}%
\definecolor{currentstroke}{rgb}{1.000000,1.000000,1.000000}%
\pgfsetstrokecolor{currentstroke}%
\pgfsetdash{}{0pt}%
\pgfpathmoveto{\pgfqpoint{2.223396in}{2.435661in}}%
\pgfpathlineto{\pgfqpoint{2.311132in}{2.435661in}}%
\pgfpathlineto{\pgfqpoint{2.311132in}{2.347925in}}%
\pgfpathlineto{\pgfqpoint{2.223396in}{2.347925in}}%
\pgfpathlineto{\pgfqpoint{2.223396in}{2.435661in}}%
\pgfusepath{stroke,fill}%
\end{pgfscope}%
\begin{pgfscope}%
\pgfpathrectangle{\pgfqpoint{0.380943in}{2.260189in}}{\pgfqpoint{4.650000in}{0.614151in}}%
\pgfusepath{clip}%
\pgfsetbuttcap%
\pgfsetroundjoin%
\definecolor{currentfill}{rgb}{0.980008,0.966013,0.779393}%
\pgfsetfillcolor{currentfill}%
\pgfsetlinewidth{0.250937pt}%
\definecolor{currentstroke}{rgb}{1.000000,1.000000,1.000000}%
\pgfsetstrokecolor{currentstroke}%
\pgfsetdash{}{0pt}%
\pgfpathmoveto{\pgfqpoint{2.311132in}{2.435661in}}%
\pgfpathlineto{\pgfqpoint{2.398868in}{2.435661in}}%
\pgfpathlineto{\pgfqpoint{2.398868in}{2.347925in}}%
\pgfpathlineto{\pgfqpoint{2.311132in}{2.347925in}}%
\pgfpathlineto{\pgfqpoint{2.311132in}{2.435661in}}%
\pgfusepath{stroke,fill}%
\end{pgfscope}%
\begin{pgfscope}%
\pgfpathrectangle{\pgfqpoint{0.380943in}{2.260189in}}{\pgfqpoint{4.650000in}{0.614151in}}%
\pgfusepath{clip}%
\pgfsetbuttcap%
\pgfsetroundjoin%
\definecolor{currentfill}{rgb}{0.990634,0.779608,0.623299}%
\pgfsetfillcolor{currentfill}%
\pgfsetlinewidth{0.250937pt}%
\definecolor{currentstroke}{rgb}{1.000000,1.000000,1.000000}%
\pgfsetstrokecolor{currentstroke}%
\pgfsetdash{}{0pt}%
\pgfpathmoveto{\pgfqpoint{2.398868in}{2.435661in}}%
\pgfpathlineto{\pgfqpoint{2.486603in}{2.435661in}}%
\pgfpathlineto{\pgfqpoint{2.486603in}{2.347925in}}%
\pgfpathlineto{\pgfqpoint{2.398868in}{2.347925in}}%
\pgfpathlineto{\pgfqpoint{2.398868in}{2.435661in}}%
\pgfusepath{stroke,fill}%
\end{pgfscope}%
\begin{pgfscope}%
\pgfpathrectangle{\pgfqpoint{0.380943in}{2.260189in}}{\pgfqpoint{4.650000in}{0.614151in}}%
\pgfusepath{clip}%
\pgfsetbuttcap%
\pgfsetroundjoin%
\definecolor{currentfill}{rgb}{0.990634,0.779608,0.623299}%
\pgfsetfillcolor{currentfill}%
\pgfsetlinewidth{0.250937pt}%
\definecolor{currentstroke}{rgb}{1.000000,1.000000,1.000000}%
\pgfsetstrokecolor{currentstroke}%
\pgfsetdash{}{0pt}%
\pgfpathmoveto{\pgfqpoint{2.486603in}{2.435661in}}%
\pgfpathlineto{\pgfqpoint{2.574339in}{2.435661in}}%
\pgfpathlineto{\pgfqpoint{2.574339in}{2.347925in}}%
\pgfpathlineto{\pgfqpoint{2.486603in}{2.347925in}}%
\pgfpathlineto{\pgfqpoint{2.486603in}{2.435661in}}%
\pgfusepath{stroke,fill}%
\end{pgfscope}%
\begin{pgfscope}%
\pgfpathrectangle{\pgfqpoint{0.380943in}{2.260189in}}{\pgfqpoint{4.650000in}{0.614151in}}%
\pgfusepath{clip}%
\pgfsetbuttcap%
\pgfsetroundjoin%
\definecolor{currentfill}{rgb}{0.987266,0.804198,0.639170}%
\pgfsetfillcolor{currentfill}%
\pgfsetlinewidth{0.250937pt}%
\definecolor{currentstroke}{rgb}{1.000000,1.000000,1.000000}%
\pgfsetstrokecolor{currentstroke}%
\pgfsetdash{}{0pt}%
\pgfpathmoveto{\pgfqpoint{2.574339in}{2.435661in}}%
\pgfpathlineto{\pgfqpoint{2.662075in}{2.435661in}}%
\pgfpathlineto{\pgfqpoint{2.662075in}{2.347925in}}%
\pgfpathlineto{\pgfqpoint{2.574339in}{2.347925in}}%
\pgfpathlineto{\pgfqpoint{2.574339in}{2.435661in}}%
\pgfusepath{stroke,fill}%
\end{pgfscope}%
\begin{pgfscope}%
\pgfpathrectangle{\pgfqpoint{0.380943in}{2.260189in}}{\pgfqpoint{4.650000in}{0.614151in}}%
\pgfusepath{clip}%
\pgfsetbuttcap%
\pgfsetroundjoin%
\definecolor{currentfill}{rgb}{0.980008,0.966013,0.779393}%
\pgfsetfillcolor{currentfill}%
\pgfsetlinewidth{0.250937pt}%
\definecolor{currentstroke}{rgb}{1.000000,1.000000,1.000000}%
\pgfsetstrokecolor{currentstroke}%
\pgfsetdash{}{0pt}%
\pgfpathmoveto{\pgfqpoint{2.662075in}{2.435661in}}%
\pgfpathlineto{\pgfqpoint{2.749811in}{2.435661in}}%
\pgfpathlineto{\pgfqpoint{2.749811in}{2.347925in}}%
\pgfpathlineto{\pgfqpoint{2.662075in}{2.347925in}}%
\pgfpathlineto{\pgfqpoint{2.662075in}{2.435661in}}%
\pgfusepath{stroke,fill}%
\end{pgfscope}%
\begin{pgfscope}%
\pgfpathrectangle{\pgfqpoint{0.380943in}{2.260189in}}{\pgfqpoint{4.650000in}{0.614151in}}%
\pgfusepath{clip}%
\pgfsetbuttcap%
\pgfsetroundjoin%
\definecolor{currentfill}{rgb}{0.963260,0.918478,0.719508}%
\pgfsetfillcolor{currentfill}%
\pgfsetlinewidth{0.250937pt}%
\definecolor{currentstroke}{rgb}{1.000000,1.000000,1.000000}%
\pgfsetstrokecolor{currentstroke}%
\pgfsetdash{}{0pt}%
\pgfpathmoveto{\pgfqpoint{2.749811in}{2.435661in}}%
\pgfpathlineto{\pgfqpoint{2.837547in}{2.435661in}}%
\pgfpathlineto{\pgfqpoint{2.837547in}{2.347925in}}%
\pgfpathlineto{\pgfqpoint{2.749811in}{2.347925in}}%
\pgfpathlineto{\pgfqpoint{2.749811in}{2.435661in}}%
\pgfusepath{stroke,fill}%
\end{pgfscope}%
\begin{pgfscope}%
\pgfpathrectangle{\pgfqpoint{0.380943in}{2.260189in}}{\pgfqpoint{4.650000in}{0.614151in}}%
\pgfusepath{clip}%
\pgfsetbuttcap%
\pgfsetroundjoin%
\definecolor{currentfill}{rgb}{0.964937,0.908651,0.713110}%
\pgfsetfillcolor{currentfill}%
\pgfsetlinewidth{0.250937pt}%
\definecolor{currentstroke}{rgb}{1.000000,1.000000,1.000000}%
\pgfsetstrokecolor{currentstroke}%
\pgfsetdash{}{0pt}%
\pgfpathmoveto{\pgfqpoint{2.837547in}{2.435661in}}%
\pgfpathlineto{\pgfqpoint{2.925283in}{2.435661in}}%
\pgfpathlineto{\pgfqpoint{2.925283in}{2.347925in}}%
\pgfpathlineto{\pgfqpoint{2.837547in}{2.347925in}}%
\pgfpathlineto{\pgfqpoint{2.837547in}{2.435661in}}%
\pgfusepath{stroke,fill}%
\end{pgfscope}%
\begin{pgfscope}%
\pgfpathrectangle{\pgfqpoint{0.380943in}{2.260189in}}{\pgfqpoint{4.650000in}{0.614151in}}%
\pgfusepath{clip}%
\pgfsetbuttcap%
\pgfsetroundjoin%
\definecolor{currentfill}{rgb}{0.995233,0.991895,0.818977}%
\pgfsetfillcolor{currentfill}%
\pgfsetlinewidth{0.250937pt}%
\definecolor{currentstroke}{rgb}{1.000000,1.000000,1.000000}%
\pgfsetstrokecolor{currentstroke}%
\pgfsetdash{}{0pt}%
\pgfpathmoveto{\pgfqpoint{2.925283in}{2.435661in}}%
\pgfpathlineto{\pgfqpoint{3.013019in}{2.435661in}}%
\pgfpathlineto{\pgfqpoint{3.013019in}{2.347925in}}%
\pgfpathlineto{\pgfqpoint{2.925283in}{2.347925in}}%
\pgfpathlineto{\pgfqpoint{2.925283in}{2.435661in}}%
\pgfusepath{stroke,fill}%
\end{pgfscope}%
\begin{pgfscope}%
\pgfpathrectangle{\pgfqpoint{0.380943in}{2.260189in}}{\pgfqpoint{4.650000in}{0.614151in}}%
\pgfusepath{clip}%
\pgfsetbuttcap%
\pgfsetroundjoin%
\definecolor{currentfill}{rgb}{0.961738,0.927612,0.725598}%
\pgfsetfillcolor{currentfill}%
\pgfsetlinewidth{0.250937pt}%
\definecolor{currentstroke}{rgb}{1.000000,1.000000,1.000000}%
\pgfsetstrokecolor{currentstroke}%
\pgfsetdash{}{0pt}%
\pgfpathmoveto{\pgfqpoint{3.013019in}{2.435661in}}%
\pgfpathlineto{\pgfqpoint{3.100754in}{2.435661in}}%
\pgfpathlineto{\pgfqpoint{3.100754in}{2.347925in}}%
\pgfpathlineto{\pgfqpoint{3.013019in}{2.347925in}}%
\pgfpathlineto{\pgfqpoint{3.013019in}{2.435661in}}%
\pgfusepath{stroke,fill}%
\end{pgfscope}%
\begin{pgfscope}%
\pgfpathrectangle{\pgfqpoint{0.380943in}{2.260189in}}{\pgfqpoint{4.650000in}{0.614151in}}%
\pgfusepath{clip}%
\pgfsetbuttcap%
\pgfsetroundjoin%
\definecolor{currentfill}{rgb}{0.961738,0.927612,0.725598}%
\pgfsetfillcolor{currentfill}%
\pgfsetlinewidth{0.250937pt}%
\definecolor{currentstroke}{rgb}{1.000000,1.000000,1.000000}%
\pgfsetstrokecolor{currentstroke}%
\pgfsetdash{}{0pt}%
\pgfpathmoveto{\pgfqpoint{3.100754in}{2.435661in}}%
\pgfpathlineto{\pgfqpoint{3.188490in}{2.435661in}}%
\pgfpathlineto{\pgfqpoint{3.188490in}{2.347925in}}%
\pgfpathlineto{\pgfqpoint{3.100754in}{2.347925in}}%
\pgfpathlineto{\pgfqpoint{3.100754in}{2.435661in}}%
\pgfusepath{stroke,fill}%
\end{pgfscope}%
\begin{pgfscope}%
\pgfpathrectangle{\pgfqpoint{0.380943in}{2.260189in}}{\pgfqpoint{4.650000in}{0.614151in}}%
\pgfusepath{clip}%
\pgfsetbuttcap%
\pgfsetroundjoin%
\definecolor{currentfill}{rgb}{0.980008,0.966013,0.779393}%
\pgfsetfillcolor{currentfill}%
\pgfsetlinewidth{0.250937pt}%
\definecolor{currentstroke}{rgb}{1.000000,1.000000,1.000000}%
\pgfsetstrokecolor{currentstroke}%
\pgfsetdash{}{0pt}%
\pgfpathmoveto{\pgfqpoint{3.188490in}{2.435661in}}%
\pgfpathlineto{\pgfqpoint{3.276226in}{2.435661in}}%
\pgfpathlineto{\pgfqpoint{3.276226in}{2.347925in}}%
\pgfpathlineto{\pgfqpoint{3.188490in}{2.347925in}}%
\pgfpathlineto{\pgfqpoint{3.188490in}{2.435661in}}%
\pgfusepath{stroke,fill}%
\end{pgfscope}%
\begin{pgfscope}%
\pgfpathrectangle{\pgfqpoint{0.380943in}{2.260189in}}{\pgfqpoint{4.650000in}{0.614151in}}%
\pgfusepath{clip}%
\pgfsetbuttcap%
\pgfsetroundjoin%
\definecolor{currentfill}{rgb}{0.964783,0.940131,0.739808}%
\pgfsetfillcolor{currentfill}%
\pgfsetlinewidth{0.250937pt}%
\definecolor{currentstroke}{rgb}{1.000000,1.000000,1.000000}%
\pgfsetstrokecolor{currentstroke}%
\pgfsetdash{}{0pt}%
\pgfpathmoveto{\pgfqpoint{3.276226in}{2.435661in}}%
\pgfpathlineto{\pgfqpoint{3.363962in}{2.435661in}}%
\pgfpathlineto{\pgfqpoint{3.363962in}{2.347925in}}%
\pgfpathlineto{\pgfqpoint{3.276226in}{2.347925in}}%
\pgfpathlineto{\pgfqpoint{3.276226in}{2.435661in}}%
\pgfusepath{stroke,fill}%
\end{pgfscope}%
\begin{pgfscope}%
\pgfpathrectangle{\pgfqpoint{0.380943in}{2.260189in}}{\pgfqpoint{4.650000in}{0.614151in}}%
\pgfusepath{clip}%
\pgfsetbuttcap%
\pgfsetroundjoin%
\definecolor{currentfill}{rgb}{0.963260,0.918478,0.719508}%
\pgfsetfillcolor{currentfill}%
\pgfsetlinewidth{0.250937pt}%
\definecolor{currentstroke}{rgb}{1.000000,1.000000,1.000000}%
\pgfsetstrokecolor{currentstroke}%
\pgfsetdash{}{0pt}%
\pgfpathmoveto{\pgfqpoint{3.363962in}{2.435661in}}%
\pgfpathlineto{\pgfqpoint{3.451698in}{2.435661in}}%
\pgfpathlineto{\pgfqpoint{3.451698in}{2.347925in}}%
\pgfpathlineto{\pgfqpoint{3.363962in}{2.347925in}}%
\pgfpathlineto{\pgfqpoint{3.363962in}{2.435661in}}%
\pgfusepath{stroke,fill}%
\end{pgfscope}%
\begin{pgfscope}%
\pgfpathrectangle{\pgfqpoint{0.380943in}{2.260189in}}{\pgfqpoint{4.650000in}{0.614151in}}%
\pgfusepath{clip}%
\pgfsetbuttcap%
\pgfsetroundjoin%
\definecolor{currentfill}{rgb}{1.000000,1.000000,0.857516}%
\pgfsetfillcolor{currentfill}%
\pgfsetlinewidth{0.250937pt}%
\definecolor{currentstroke}{rgb}{1.000000,1.000000,1.000000}%
\pgfsetstrokecolor{currentstroke}%
\pgfsetdash{}{0pt}%
\pgfpathmoveto{\pgfqpoint{3.451698in}{2.435661in}}%
\pgfpathlineto{\pgfqpoint{3.539434in}{2.435661in}}%
\pgfpathlineto{\pgfqpoint{3.539434in}{2.347925in}}%
\pgfpathlineto{\pgfqpoint{3.451698in}{2.347925in}}%
\pgfpathlineto{\pgfqpoint{3.451698in}{2.435661in}}%
\pgfusepath{stroke,fill}%
\end{pgfscope}%
\begin{pgfscope}%
\pgfpathrectangle{\pgfqpoint{0.380943in}{2.260189in}}{\pgfqpoint{4.650000in}{0.614151in}}%
\pgfusepath{clip}%
\pgfsetbuttcap%
\pgfsetroundjoin%
\definecolor{currentfill}{rgb}{0.963260,0.918478,0.719508}%
\pgfsetfillcolor{currentfill}%
\pgfsetlinewidth{0.250937pt}%
\definecolor{currentstroke}{rgb}{1.000000,1.000000,1.000000}%
\pgfsetstrokecolor{currentstroke}%
\pgfsetdash{}{0pt}%
\pgfpathmoveto{\pgfqpoint{3.539434in}{2.435661in}}%
\pgfpathlineto{\pgfqpoint{3.627169in}{2.435661in}}%
\pgfpathlineto{\pgfqpoint{3.627169in}{2.347925in}}%
\pgfpathlineto{\pgfqpoint{3.539434in}{2.347925in}}%
\pgfpathlineto{\pgfqpoint{3.539434in}{2.435661in}}%
\pgfusepath{stroke,fill}%
\end{pgfscope}%
\begin{pgfscope}%
\pgfpathrectangle{\pgfqpoint{0.380943in}{2.260189in}}{\pgfqpoint{4.650000in}{0.614151in}}%
\pgfusepath{clip}%
\pgfsetbuttcap%
\pgfsetroundjoin%
\definecolor{currentfill}{rgb}{0.969504,0.885813,0.700930}%
\pgfsetfillcolor{currentfill}%
\pgfsetlinewidth{0.250937pt}%
\definecolor{currentstroke}{rgb}{1.000000,1.000000,1.000000}%
\pgfsetstrokecolor{currentstroke}%
\pgfsetdash{}{0pt}%
\pgfpathmoveto{\pgfqpoint{3.627169in}{2.435661in}}%
\pgfpathlineto{\pgfqpoint{3.714905in}{2.435661in}}%
\pgfpathlineto{\pgfqpoint{3.714905in}{2.347925in}}%
\pgfpathlineto{\pgfqpoint{3.627169in}{2.347925in}}%
\pgfpathlineto{\pgfqpoint{3.627169in}{2.435661in}}%
\pgfusepath{stroke,fill}%
\end{pgfscope}%
\begin{pgfscope}%
\pgfpathrectangle{\pgfqpoint{0.380943in}{2.260189in}}{\pgfqpoint{4.650000in}{0.614151in}}%
\pgfusepath{clip}%
\pgfsetbuttcap%
\pgfsetroundjoin%
\definecolor{currentfill}{rgb}{0.964937,0.908651,0.713110}%
\pgfsetfillcolor{currentfill}%
\pgfsetlinewidth{0.250937pt}%
\definecolor{currentstroke}{rgb}{1.000000,1.000000,1.000000}%
\pgfsetstrokecolor{currentstroke}%
\pgfsetdash{}{0pt}%
\pgfpathmoveto{\pgfqpoint{3.714905in}{2.435661in}}%
\pgfpathlineto{\pgfqpoint{3.802641in}{2.435661in}}%
\pgfpathlineto{\pgfqpoint{3.802641in}{2.347925in}}%
\pgfpathlineto{\pgfqpoint{3.714905in}{2.347925in}}%
\pgfpathlineto{\pgfqpoint{3.714905in}{2.435661in}}%
\pgfusepath{stroke,fill}%
\end{pgfscope}%
\begin{pgfscope}%
\pgfpathrectangle{\pgfqpoint{0.380943in}{2.260189in}}{\pgfqpoint{4.650000in}{0.614151in}}%
\pgfusepath{clip}%
\pgfsetbuttcap%
\pgfsetroundjoin%
\definecolor{currentfill}{rgb}{0.964783,0.940131,0.739808}%
\pgfsetfillcolor{currentfill}%
\pgfsetlinewidth{0.250937pt}%
\definecolor{currentstroke}{rgb}{1.000000,1.000000,1.000000}%
\pgfsetstrokecolor{currentstroke}%
\pgfsetdash{}{0pt}%
\pgfpathmoveto{\pgfqpoint{3.802641in}{2.435661in}}%
\pgfpathlineto{\pgfqpoint{3.890377in}{2.435661in}}%
\pgfpathlineto{\pgfqpoint{3.890377in}{2.347925in}}%
\pgfpathlineto{\pgfqpoint{3.802641in}{2.347925in}}%
\pgfpathlineto{\pgfqpoint{3.802641in}{2.435661in}}%
\pgfusepath{stroke,fill}%
\end{pgfscope}%
\begin{pgfscope}%
\pgfpathrectangle{\pgfqpoint{0.380943in}{2.260189in}}{\pgfqpoint{4.650000in}{0.614151in}}%
\pgfusepath{clip}%
\pgfsetbuttcap%
\pgfsetroundjoin%
\definecolor{currentfill}{rgb}{0.987266,0.804198,0.639170}%
\pgfsetfillcolor{currentfill}%
\pgfsetlinewidth{0.250937pt}%
\definecolor{currentstroke}{rgb}{1.000000,1.000000,1.000000}%
\pgfsetstrokecolor{currentstroke}%
\pgfsetdash{}{0pt}%
\pgfpathmoveto{\pgfqpoint{3.890377in}{2.435661in}}%
\pgfpathlineto{\pgfqpoint{3.978113in}{2.435661in}}%
\pgfpathlineto{\pgfqpoint{3.978113in}{2.347925in}}%
\pgfpathlineto{\pgfqpoint{3.890377in}{2.347925in}}%
\pgfpathlineto{\pgfqpoint{3.890377in}{2.435661in}}%
\pgfusepath{stroke,fill}%
\end{pgfscope}%
\begin{pgfscope}%
\pgfpathrectangle{\pgfqpoint{0.380943in}{2.260189in}}{\pgfqpoint{4.650000in}{0.614151in}}%
\pgfusepath{clip}%
\pgfsetbuttcap%
\pgfsetroundjoin%
\definecolor{currentfill}{rgb}{1.000000,1.000000,0.857516}%
\pgfsetfillcolor{currentfill}%
\pgfsetlinewidth{0.250937pt}%
\definecolor{currentstroke}{rgb}{1.000000,1.000000,1.000000}%
\pgfsetstrokecolor{currentstroke}%
\pgfsetdash{}{0pt}%
\pgfpathmoveto{\pgfqpoint{3.978113in}{2.435661in}}%
\pgfpathlineto{\pgfqpoint{4.065849in}{2.435661in}}%
\pgfpathlineto{\pgfqpoint{4.065849in}{2.347925in}}%
\pgfpathlineto{\pgfqpoint{3.978113in}{2.347925in}}%
\pgfpathlineto{\pgfqpoint{3.978113in}{2.435661in}}%
\pgfusepath{stroke,fill}%
\end{pgfscope}%
\begin{pgfscope}%
\pgfpathrectangle{\pgfqpoint{0.380943in}{2.260189in}}{\pgfqpoint{4.650000in}{0.614151in}}%
\pgfusepath{clip}%
\pgfsetbuttcap%
\pgfsetroundjoin%
\definecolor{currentfill}{rgb}{0.978639,0.841584,0.673679}%
\pgfsetfillcolor{currentfill}%
\pgfsetlinewidth{0.250937pt}%
\definecolor{currentstroke}{rgb}{1.000000,1.000000,1.000000}%
\pgfsetstrokecolor{currentstroke}%
\pgfsetdash{}{0pt}%
\pgfpathmoveto{\pgfqpoint{4.065849in}{2.435661in}}%
\pgfpathlineto{\pgfqpoint{4.153585in}{2.435661in}}%
\pgfpathlineto{\pgfqpoint{4.153585in}{2.347925in}}%
\pgfpathlineto{\pgfqpoint{4.065849in}{2.347925in}}%
\pgfpathlineto{\pgfqpoint{4.065849in}{2.435661in}}%
\pgfusepath{stroke,fill}%
\end{pgfscope}%
\begin{pgfscope}%
\pgfpathrectangle{\pgfqpoint{0.380943in}{2.260189in}}{\pgfqpoint{4.650000in}{0.614151in}}%
\pgfusepath{clip}%
\pgfsetbuttcap%
\pgfsetroundjoin%
\definecolor{currentfill}{rgb}{0.969504,0.885813,0.700930}%
\pgfsetfillcolor{currentfill}%
\pgfsetlinewidth{0.250937pt}%
\definecolor{currentstroke}{rgb}{1.000000,1.000000,1.000000}%
\pgfsetstrokecolor{currentstroke}%
\pgfsetdash{}{0pt}%
\pgfpathmoveto{\pgfqpoint{4.153585in}{2.435661in}}%
\pgfpathlineto{\pgfqpoint{4.241320in}{2.435661in}}%
\pgfpathlineto{\pgfqpoint{4.241320in}{2.347925in}}%
\pgfpathlineto{\pgfqpoint{4.153585in}{2.347925in}}%
\pgfpathlineto{\pgfqpoint{4.153585in}{2.435661in}}%
\pgfusepath{stroke,fill}%
\end{pgfscope}%
\begin{pgfscope}%
\pgfpathrectangle{\pgfqpoint{0.380943in}{2.260189in}}{\pgfqpoint{4.650000in}{0.614151in}}%
\pgfusepath{clip}%
\pgfsetbuttcap%
\pgfsetroundjoin%
\definecolor{currentfill}{rgb}{0.982699,0.823991,0.657439}%
\pgfsetfillcolor{currentfill}%
\pgfsetlinewidth{0.250937pt}%
\definecolor{currentstroke}{rgb}{1.000000,1.000000,1.000000}%
\pgfsetstrokecolor{currentstroke}%
\pgfsetdash{}{0pt}%
\pgfpathmoveto{\pgfqpoint{4.241320in}{2.435661in}}%
\pgfpathlineto{\pgfqpoint{4.329056in}{2.435661in}}%
\pgfpathlineto{\pgfqpoint{4.329056in}{2.347925in}}%
\pgfpathlineto{\pgfqpoint{4.241320in}{2.347925in}}%
\pgfpathlineto{\pgfqpoint{4.241320in}{2.435661in}}%
\pgfusepath{stroke,fill}%
\end{pgfscope}%
\begin{pgfscope}%
\pgfpathrectangle{\pgfqpoint{0.380943in}{2.260189in}}{\pgfqpoint{4.650000in}{0.614151in}}%
\pgfusepath{clip}%
\pgfsetbuttcap%
\pgfsetroundjoin%
\definecolor{currentfill}{rgb}{0.980008,0.966013,0.779393}%
\pgfsetfillcolor{currentfill}%
\pgfsetlinewidth{0.250937pt}%
\definecolor{currentstroke}{rgb}{1.000000,1.000000,1.000000}%
\pgfsetstrokecolor{currentstroke}%
\pgfsetdash{}{0pt}%
\pgfpathmoveto{\pgfqpoint{4.329056in}{2.435661in}}%
\pgfpathlineto{\pgfqpoint{4.416792in}{2.435661in}}%
\pgfpathlineto{\pgfqpoint{4.416792in}{2.347925in}}%
\pgfpathlineto{\pgfqpoint{4.329056in}{2.347925in}}%
\pgfpathlineto{\pgfqpoint{4.329056in}{2.435661in}}%
\pgfusepath{stroke,fill}%
\end{pgfscope}%
\begin{pgfscope}%
\pgfpathrectangle{\pgfqpoint{0.380943in}{2.260189in}}{\pgfqpoint{4.650000in}{0.614151in}}%
\pgfusepath{clip}%
\pgfsetbuttcap%
\pgfsetroundjoin%
\definecolor{currentfill}{rgb}{0.969504,0.885813,0.700930}%
\pgfsetfillcolor{currentfill}%
\pgfsetlinewidth{0.250937pt}%
\definecolor{currentstroke}{rgb}{1.000000,1.000000,1.000000}%
\pgfsetstrokecolor{currentstroke}%
\pgfsetdash{}{0pt}%
\pgfpathmoveto{\pgfqpoint{4.416792in}{2.435661in}}%
\pgfpathlineto{\pgfqpoint{4.504528in}{2.435661in}}%
\pgfpathlineto{\pgfqpoint{4.504528in}{2.347925in}}%
\pgfpathlineto{\pgfqpoint{4.416792in}{2.347925in}}%
\pgfpathlineto{\pgfqpoint{4.416792in}{2.435661in}}%
\pgfusepath{stroke,fill}%
\end{pgfscope}%
\begin{pgfscope}%
\pgfpathrectangle{\pgfqpoint{0.380943in}{2.260189in}}{\pgfqpoint{4.650000in}{0.614151in}}%
\pgfusepath{clip}%
\pgfsetbuttcap%
\pgfsetroundjoin%
\definecolor{currentfill}{rgb}{0.964783,0.940131,0.739808}%
\pgfsetfillcolor{currentfill}%
\pgfsetlinewidth{0.250937pt}%
\definecolor{currentstroke}{rgb}{1.000000,1.000000,1.000000}%
\pgfsetstrokecolor{currentstroke}%
\pgfsetdash{}{0pt}%
\pgfpathmoveto{\pgfqpoint{4.504528in}{2.435661in}}%
\pgfpathlineto{\pgfqpoint{4.592264in}{2.435661in}}%
\pgfpathlineto{\pgfqpoint{4.592264in}{2.347925in}}%
\pgfpathlineto{\pgfqpoint{4.504528in}{2.347925in}}%
\pgfpathlineto{\pgfqpoint{4.504528in}{2.435661in}}%
\pgfusepath{stroke,fill}%
\end{pgfscope}%
\begin{pgfscope}%
\pgfpathrectangle{\pgfqpoint{0.380943in}{2.260189in}}{\pgfqpoint{4.650000in}{0.614151in}}%
\pgfusepath{clip}%
\pgfsetbuttcap%
\pgfsetroundjoin%
\definecolor{currentfill}{rgb}{0.963260,0.918478,0.719508}%
\pgfsetfillcolor{currentfill}%
\pgfsetlinewidth{0.250937pt}%
\definecolor{currentstroke}{rgb}{1.000000,1.000000,1.000000}%
\pgfsetstrokecolor{currentstroke}%
\pgfsetdash{}{0pt}%
\pgfpathmoveto{\pgfqpoint{4.592264in}{2.435661in}}%
\pgfpathlineto{\pgfqpoint{4.680000in}{2.435661in}}%
\pgfpathlineto{\pgfqpoint{4.680000in}{2.347925in}}%
\pgfpathlineto{\pgfqpoint{4.592264in}{2.347925in}}%
\pgfpathlineto{\pgfqpoint{4.592264in}{2.435661in}}%
\pgfusepath{stroke,fill}%
\end{pgfscope}%
\begin{pgfscope}%
\pgfpathrectangle{\pgfqpoint{0.380943in}{2.260189in}}{\pgfqpoint{4.650000in}{0.614151in}}%
\pgfusepath{clip}%
\pgfsetbuttcap%
\pgfsetroundjoin%
\definecolor{currentfill}{rgb}{0.978639,0.841584,0.673679}%
\pgfsetfillcolor{currentfill}%
\pgfsetlinewidth{0.250937pt}%
\definecolor{currentstroke}{rgb}{1.000000,1.000000,1.000000}%
\pgfsetstrokecolor{currentstroke}%
\pgfsetdash{}{0pt}%
\pgfpathmoveto{\pgfqpoint{4.680000in}{2.435661in}}%
\pgfpathlineto{\pgfqpoint{4.767736in}{2.435661in}}%
\pgfpathlineto{\pgfqpoint{4.767736in}{2.347925in}}%
\pgfpathlineto{\pgfqpoint{4.680000in}{2.347925in}}%
\pgfpathlineto{\pgfqpoint{4.680000in}{2.435661in}}%
\pgfusepath{stroke,fill}%
\end{pgfscope}%
\begin{pgfscope}%
\pgfpathrectangle{\pgfqpoint{0.380943in}{2.260189in}}{\pgfqpoint{4.650000in}{0.614151in}}%
\pgfusepath{clip}%
\pgfsetbuttcap%
\pgfsetroundjoin%
\definecolor{currentfill}{rgb}{0.964937,0.908651,0.713110}%
\pgfsetfillcolor{currentfill}%
\pgfsetlinewidth{0.250937pt}%
\definecolor{currentstroke}{rgb}{1.000000,1.000000,1.000000}%
\pgfsetstrokecolor{currentstroke}%
\pgfsetdash{}{0pt}%
\pgfpathmoveto{\pgfqpoint{4.767736in}{2.435661in}}%
\pgfpathlineto{\pgfqpoint{4.855471in}{2.435661in}}%
\pgfpathlineto{\pgfqpoint{4.855471in}{2.347925in}}%
\pgfpathlineto{\pgfqpoint{4.767736in}{2.347925in}}%
\pgfpathlineto{\pgfqpoint{4.767736in}{2.435661in}}%
\pgfusepath{stroke,fill}%
\end{pgfscope}%
\begin{pgfscope}%
\pgfpathrectangle{\pgfqpoint{0.380943in}{2.260189in}}{\pgfqpoint{4.650000in}{0.614151in}}%
\pgfusepath{clip}%
\pgfsetbuttcap%
\pgfsetroundjoin%
\definecolor{currentfill}{rgb}{0.995233,0.991895,0.818977}%
\pgfsetfillcolor{currentfill}%
\pgfsetlinewidth{0.250937pt}%
\definecolor{currentstroke}{rgb}{1.000000,1.000000,1.000000}%
\pgfsetstrokecolor{currentstroke}%
\pgfsetdash{}{0pt}%
\pgfpathmoveto{\pgfqpoint{4.855471in}{2.435661in}}%
\pgfpathlineto{\pgfqpoint{4.943207in}{2.435661in}}%
\pgfpathlineto{\pgfqpoint{4.943207in}{2.347925in}}%
\pgfpathlineto{\pgfqpoint{4.855471in}{2.347925in}}%
\pgfpathlineto{\pgfqpoint{4.855471in}{2.435661in}}%
\pgfusepath{stroke,fill}%
\end{pgfscope}%
\begin{pgfscope}%
\pgfpathrectangle{\pgfqpoint{0.380943in}{2.260189in}}{\pgfqpoint{4.650000in}{0.614151in}}%
\pgfusepath{clip}%
\pgfsetbuttcap%
\pgfsetroundjoin%
\pgfsetlinewidth{0.250937pt}%
\definecolor{currentstroke}{rgb}{1.000000,1.000000,1.000000}%
\pgfsetstrokecolor{currentstroke}%
\pgfsetdash{}{0pt}%
\pgfpathmoveto{\pgfqpoint{4.943207in}{2.435661in}}%
\pgfpathlineto{\pgfqpoint{5.030943in}{2.435661in}}%
\pgfpathlineto{\pgfqpoint{5.030943in}{2.347925in}}%
\pgfpathlineto{\pgfqpoint{4.943207in}{2.347925in}}%
\pgfpathlineto{\pgfqpoint{4.943207in}{2.435661in}}%
\pgfusepath{stroke}%
\end{pgfscope}%
\begin{pgfscope}%
\pgfpathrectangle{\pgfqpoint{0.380943in}{2.260189in}}{\pgfqpoint{4.650000in}{0.614151in}}%
\pgfusepath{clip}%
\pgfsetbuttcap%
\pgfsetroundjoin%
\definecolor{currentfill}{rgb}{0.964783,0.940131,0.739808}%
\pgfsetfillcolor{currentfill}%
\pgfsetlinewidth{0.250937pt}%
\definecolor{currentstroke}{rgb}{1.000000,1.000000,1.000000}%
\pgfsetstrokecolor{currentstroke}%
\pgfsetdash{}{0pt}%
\pgfpathmoveto{\pgfqpoint{0.380943in}{2.347925in}}%
\pgfpathlineto{\pgfqpoint{0.468679in}{2.347925in}}%
\pgfpathlineto{\pgfqpoint{0.468679in}{2.260189in}}%
\pgfpathlineto{\pgfqpoint{0.380943in}{2.260189in}}%
\pgfpathlineto{\pgfqpoint{0.380943in}{2.347925in}}%
\pgfusepath{stroke,fill}%
\end{pgfscope}%
\begin{pgfscope}%
\pgfpathrectangle{\pgfqpoint{0.380943in}{2.260189in}}{\pgfqpoint{4.650000in}{0.614151in}}%
\pgfusepath{clip}%
\pgfsetbuttcap%
\pgfsetroundjoin%
\definecolor{currentfill}{rgb}{0.995233,0.991895,0.818977}%
\pgfsetfillcolor{currentfill}%
\pgfsetlinewidth{0.250937pt}%
\definecolor{currentstroke}{rgb}{1.000000,1.000000,1.000000}%
\pgfsetstrokecolor{currentstroke}%
\pgfsetdash{}{0pt}%
\pgfpathmoveto{\pgfqpoint{0.468679in}{2.347925in}}%
\pgfpathlineto{\pgfqpoint{0.556415in}{2.347925in}}%
\pgfpathlineto{\pgfqpoint{0.556415in}{2.260189in}}%
\pgfpathlineto{\pgfqpoint{0.468679in}{2.260189in}}%
\pgfpathlineto{\pgfqpoint{0.468679in}{2.347925in}}%
\pgfusepath{stroke,fill}%
\end{pgfscope}%
\begin{pgfscope}%
\pgfpathrectangle{\pgfqpoint{0.380943in}{2.260189in}}{\pgfqpoint{4.650000in}{0.614151in}}%
\pgfusepath{clip}%
\pgfsetbuttcap%
\pgfsetroundjoin%
\definecolor{currentfill}{rgb}{1.000000,1.000000,0.857516}%
\pgfsetfillcolor{currentfill}%
\pgfsetlinewidth{0.250937pt}%
\definecolor{currentstroke}{rgb}{1.000000,1.000000,1.000000}%
\pgfsetstrokecolor{currentstroke}%
\pgfsetdash{}{0pt}%
\pgfpathmoveto{\pgfqpoint{0.556415in}{2.347925in}}%
\pgfpathlineto{\pgfqpoint{0.644151in}{2.347925in}}%
\pgfpathlineto{\pgfqpoint{0.644151in}{2.260189in}}%
\pgfpathlineto{\pgfqpoint{0.556415in}{2.260189in}}%
\pgfpathlineto{\pgfqpoint{0.556415in}{2.347925in}}%
\pgfusepath{stroke,fill}%
\end{pgfscope}%
\begin{pgfscope}%
\pgfpathrectangle{\pgfqpoint{0.380943in}{2.260189in}}{\pgfqpoint{4.650000in}{0.614151in}}%
\pgfusepath{clip}%
\pgfsetbuttcap%
\pgfsetroundjoin%
\definecolor{currentfill}{rgb}{1.000000,1.000000,0.857516}%
\pgfsetfillcolor{currentfill}%
\pgfsetlinewidth{0.250937pt}%
\definecolor{currentstroke}{rgb}{1.000000,1.000000,1.000000}%
\pgfsetstrokecolor{currentstroke}%
\pgfsetdash{}{0pt}%
\pgfpathmoveto{\pgfqpoint{0.644151in}{2.347925in}}%
\pgfpathlineto{\pgfqpoint{0.731886in}{2.347925in}}%
\pgfpathlineto{\pgfqpoint{0.731886in}{2.260189in}}%
\pgfpathlineto{\pgfqpoint{0.644151in}{2.260189in}}%
\pgfpathlineto{\pgfqpoint{0.644151in}{2.347925in}}%
\pgfusepath{stroke,fill}%
\end{pgfscope}%
\begin{pgfscope}%
\pgfpathrectangle{\pgfqpoint{0.380943in}{2.260189in}}{\pgfqpoint{4.650000in}{0.614151in}}%
\pgfusepath{clip}%
\pgfsetbuttcap%
\pgfsetroundjoin%
\definecolor{currentfill}{rgb}{1.000000,1.000000,0.857516}%
\pgfsetfillcolor{currentfill}%
\pgfsetlinewidth{0.250937pt}%
\definecolor{currentstroke}{rgb}{1.000000,1.000000,1.000000}%
\pgfsetstrokecolor{currentstroke}%
\pgfsetdash{}{0pt}%
\pgfpathmoveto{\pgfqpoint{0.731886in}{2.347925in}}%
\pgfpathlineto{\pgfqpoint{0.819622in}{2.347925in}}%
\pgfpathlineto{\pgfqpoint{0.819622in}{2.260189in}}%
\pgfpathlineto{\pgfqpoint{0.731886in}{2.260189in}}%
\pgfpathlineto{\pgfqpoint{0.731886in}{2.347925in}}%
\pgfusepath{stroke,fill}%
\end{pgfscope}%
\begin{pgfscope}%
\pgfpathrectangle{\pgfqpoint{0.380943in}{2.260189in}}{\pgfqpoint{4.650000in}{0.614151in}}%
\pgfusepath{clip}%
\pgfsetbuttcap%
\pgfsetroundjoin%
\definecolor{currentfill}{rgb}{0.980008,0.966013,0.779393}%
\pgfsetfillcolor{currentfill}%
\pgfsetlinewidth{0.250937pt}%
\definecolor{currentstroke}{rgb}{1.000000,1.000000,1.000000}%
\pgfsetstrokecolor{currentstroke}%
\pgfsetdash{}{0pt}%
\pgfpathmoveto{\pgfqpoint{0.819622in}{2.347925in}}%
\pgfpathlineto{\pgfqpoint{0.907358in}{2.347925in}}%
\pgfpathlineto{\pgfqpoint{0.907358in}{2.260189in}}%
\pgfpathlineto{\pgfqpoint{0.819622in}{2.260189in}}%
\pgfpathlineto{\pgfqpoint{0.819622in}{2.347925in}}%
\pgfusepath{stroke,fill}%
\end{pgfscope}%
\begin{pgfscope}%
\pgfpathrectangle{\pgfqpoint{0.380943in}{2.260189in}}{\pgfqpoint{4.650000in}{0.614151in}}%
\pgfusepath{clip}%
\pgfsetbuttcap%
\pgfsetroundjoin%
\definecolor{currentfill}{rgb}{1.000000,1.000000,0.857516}%
\pgfsetfillcolor{currentfill}%
\pgfsetlinewidth{0.250937pt}%
\definecolor{currentstroke}{rgb}{1.000000,1.000000,1.000000}%
\pgfsetstrokecolor{currentstroke}%
\pgfsetdash{}{0pt}%
\pgfpathmoveto{\pgfqpoint{0.907358in}{2.347925in}}%
\pgfpathlineto{\pgfqpoint{0.995094in}{2.347925in}}%
\pgfpathlineto{\pgfqpoint{0.995094in}{2.260189in}}%
\pgfpathlineto{\pgfqpoint{0.907358in}{2.260189in}}%
\pgfpathlineto{\pgfqpoint{0.907358in}{2.347925in}}%
\pgfusepath{stroke,fill}%
\end{pgfscope}%
\begin{pgfscope}%
\pgfpathrectangle{\pgfqpoint{0.380943in}{2.260189in}}{\pgfqpoint{4.650000in}{0.614151in}}%
\pgfusepath{clip}%
\pgfsetbuttcap%
\pgfsetroundjoin%
\definecolor{currentfill}{rgb}{0.995233,0.991895,0.818977}%
\pgfsetfillcolor{currentfill}%
\pgfsetlinewidth{0.250937pt}%
\definecolor{currentstroke}{rgb}{1.000000,1.000000,1.000000}%
\pgfsetstrokecolor{currentstroke}%
\pgfsetdash{}{0pt}%
\pgfpathmoveto{\pgfqpoint{0.995094in}{2.347925in}}%
\pgfpathlineto{\pgfqpoint{1.082830in}{2.347925in}}%
\pgfpathlineto{\pgfqpoint{1.082830in}{2.260189in}}%
\pgfpathlineto{\pgfqpoint{0.995094in}{2.260189in}}%
\pgfpathlineto{\pgfqpoint{0.995094in}{2.347925in}}%
\pgfusepath{stroke,fill}%
\end{pgfscope}%
\begin{pgfscope}%
\pgfpathrectangle{\pgfqpoint{0.380943in}{2.260189in}}{\pgfqpoint{4.650000in}{0.614151in}}%
\pgfusepath{clip}%
\pgfsetbuttcap%
\pgfsetroundjoin%
\definecolor{currentfill}{rgb}{0.961738,0.927612,0.725598}%
\pgfsetfillcolor{currentfill}%
\pgfsetlinewidth{0.250937pt}%
\definecolor{currentstroke}{rgb}{1.000000,1.000000,1.000000}%
\pgfsetstrokecolor{currentstroke}%
\pgfsetdash{}{0pt}%
\pgfpathmoveto{\pgfqpoint{1.082830in}{2.347925in}}%
\pgfpathlineto{\pgfqpoint{1.170566in}{2.347925in}}%
\pgfpathlineto{\pgfqpoint{1.170566in}{2.260189in}}%
\pgfpathlineto{\pgfqpoint{1.082830in}{2.260189in}}%
\pgfpathlineto{\pgfqpoint{1.082830in}{2.347925in}}%
\pgfusepath{stroke,fill}%
\end{pgfscope}%
\begin{pgfscope}%
\pgfpathrectangle{\pgfqpoint{0.380943in}{2.260189in}}{\pgfqpoint{4.650000in}{0.614151in}}%
\pgfusepath{clip}%
\pgfsetbuttcap%
\pgfsetroundjoin%
\definecolor{currentfill}{rgb}{0.995233,0.991895,0.818977}%
\pgfsetfillcolor{currentfill}%
\pgfsetlinewidth{0.250937pt}%
\definecolor{currentstroke}{rgb}{1.000000,1.000000,1.000000}%
\pgfsetstrokecolor{currentstroke}%
\pgfsetdash{}{0pt}%
\pgfpathmoveto{\pgfqpoint{1.170566in}{2.347925in}}%
\pgfpathlineto{\pgfqpoint{1.258302in}{2.347925in}}%
\pgfpathlineto{\pgfqpoint{1.258302in}{2.260189in}}%
\pgfpathlineto{\pgfqpoint{1.170566in}{2.260189in}}%
\pgfpathlineto{\pgfqpoint{1.170566in}{2.347925in}}%
\pgfusepath{stroke,fill}%
\end{pgfscope}%
\begin{pgfscope}%
\pgfpathrectangle{\pgfqpoint{0.380943in}{2.260189in}}{\pgfqpoint{4.650000in}{0.614151in}}%
\pgfusepath{clip}%
\pgfsetbuttcap%
\pgfsetroundjoin%
\definecolor{currentfill}{rgb}{1.000000,1.000000,0.929412}%
\pgfsetfillcolor{currentfill}%
\pgfsetlinewidth{0.250937pt}%
\definecolor{currentstroke}{rgb}{1.000000,1.000000,1.000000}%
\pgfsetstrokecolor{currentstroke}%
\pgfsetdash{}{0pt}%
\pgfpathmoveto{\pgfqpoint{1.258302in}{2.347925in}}%
\pgfpathlineto{\pgfqpoint{1.346037in}{2.347925in}}%
\pgfpathlineto{\pgfqpoint{1.346037in}{2.260189in}}%
\pgfpathlineto{\pgfqpoint{1.258302in}{2.260189in}}%
\pgfpathlineto{\pgfqpoint{1.258302in}{2.347925in}}%
\pgfusepath{stroke,fill}%
\end{pgfscope}%
\begin{pgfscope}%
\pgfpathrectangle{\pgfqpoint{0.380943in}{2.260189in}}{\pgfqpoint{4.650000in}{0.614151in}}%
\pgfusepath{clip}%
\pgfsetbuttcap%
\pgfsetroundjoin%
\definecolor{currentfill}{rgb}{1.000000,1.000000,0.895579}%
\pgfsetfillcolor{currentfill}%
\pgfsetlinewidth{0.250937pt}%
\definecolor{currentstroke}{rgb}{1.000000,1.000000,1.000000}%
\pgfsetstrokecolor{currentstroke}%
\pgfsetdash{}{0pt}%
\pgfpathmoveto{\pgfqpoint{1.346037in}{2.347925in}}%
\pgfpathlineto{\pgfqpoint{1.433773in}{2.347925in}}%
\pgfpathlineto{\pgfqpoint{1.433773in}{2.260189in}}%
\pgfpathlineto{\pgfqpoint{1.346037in}{2.260189in}}%
\pgfpathlineto{\pgfqpoint{1.346037in}{2.347925in}}%
\pgfusepath{stroke,fill}%
\end{pgfscope}%
\begin{pgfscope}%
\pgfpathrectangle{\pgfqpoint{0.380943in}{2.260189in}}{\pgfqpoint{4.650000in}{0.614151in}}%
\pgfusepath{clip}%
\pgfsetbuttcap%
\pgfsetroundjoin%
\definecolor{currentfill}{rgb}{1.000000,1.000000,0.929412}%
\pgfsetfillcolor{currentfill}%
\pgfsetlinewidth{0.250937pt}%
\definecolor{currentstroke}{rgb}{1.000000,1.000000,1.000000}%
\pgfsetstrokecolor{currentstroke}%
\pgfsetdash{}{0pt}%
\pgfpathmoveto{\pgfqpoint{1.433773in}{2.347925in}}%
\pgfpathlineto{\pgfqpoint{1.521509in}{2.347925in}}%
\pgfpathlineto{\pgfqpoint{1.521509in}{2.260189in}}%
\pgfpathlineto{\pgfqpoint{1.433773in}{2.260189in}}%
\pgfpathlineto{\pgfqpoint{1.433773in}{2.347925in}}%
\pgfusepath{stroke,fill}%
\end{pgfscope}%
\begin{pgfscope}%
\pgfpathrectangle{\pgfqpoint{0.380943in}{2.260189in}}{\pgfqpoint{4.650000in}{0.614151in}}%
\pgfusepath{clip}%
\pgfsetbuttcap%
\pgfsetroundjoin%
\definecolor{currentfill}{rgb}{1.000000,1.000000,0.929412}%
\pgfsetfillcolor{currentfill}%
\pgfsetlinewidth{0.250937pt}%
\definecolor{currentstroke}{rgb}{1.000000,1.000000,1.000000}%
\pgfsetstrokecolor{currentstroke}%
\pgfsetdash{}{0pt}%
\pgfpathmoveto{\pgfqpoint{1.521509in}{2.347925in}}%
\pgfpathlineto{\pgfqpoint{1.609245in}{2.347925in}}%
\pgfpathlineto{\pgfqpoint{1.609245in}{2.260189in}}%
\pgfpathlineto{\pgfqpoint{1.521509in}{2.260189in}}%
\pgfpathlineto{\pgfqpoint{1.521509in}{2.347925in}}%
\pgfusepath{stroke,fill}%
\end{pgfscope}%
\begin{pgfscope}%
\pgfpathrectangle{\pgfqpoint{0.380943in}{2.260189in}}{\pgfqpoint{4.650000in}{0.614151in}}%
\pgfusepath{clip}%
\pgfsetbuttcap%
\pgfsetroundjoin%
\definecolor{currentfill}{rgb}{1.000000,1.000000,0.857516}%
\pgfsetfillcolor{currentfill}%
\pgfsetlinewidth{0.250937pt}%
\definecolor{currentstroke}{rgb}{1.000000,1.000000,1.000000}%
\pgfsetstrokecolor{currentstroke}%
\pgfsetdash{}{0pt}%
\pgfpathmoveto{\pgfqpoint{1.609245in}{2.347925in}}%
\pgfpathlineto{\pgfqpoint{1.696981in}{2.347925in}}%
\pgfpathlineto{\pgfqpoint{1.696981in}{2.260189in}}%
\pgfpathlineto{\pgfqpoint{1.609245in}{2.260189in}}%
\pgfpathlineto{\pgfqpoint{1.609245in}{2.347925in}}%
\pgfusepath{stroke,fill}%
\end{pgfscope}%
\begin{pgfscope}%
\pgfpathrectangle{\pgfqpoint{0.380943in}{2.260189in}}{\pgfqpoint{4.650000in}{0.614151in}}%
\pgfusepath{clip}%
\pgfsetbuttcap%
\pgfsetroundjoin%
\definecolor{currentfill}{rgb}{0.995233,0.991895,0.818977}%
\pgfsetfillcolor{currentfill}%
\pgfsetlinewidth{0.250937pt}%
\definecolor{currentstroke}{rgb}{1.000000,1.000000,1.000000}%
\pgfsetstrokecolor{currentstroke}%
\pgfsetdash{}{0pt}%
\pgfpathmoveto{\pgfqpoint{1.696981in}{2.347925in}}%
\pgfpathlineto{\pgfqpoint{1.784717in}{2.347925in}}%
\pgfpathlineto{\pgfqpoint{1.784717in}{2.260189in}}%
\pgfpathlineto{\pgfqpoint{1.696981in}{2.260189in}}%
\pgfpathlineto{\pgfqpoint{1.696981in}{2.347925in}}%
\pgfusepath{stroke,fill}%
\end{pgfscope}%
\begin{pgfscope}%
\pgfpathrectangle{\pgfqpoint{0.380943in}{2.260189in}}{\pgfqpoint{4.650000in}{0.614151in}}%
\pgfusepath{clip}%
\pgfsetbuttcap%
\pgfsetroundjoin%
\definecolor{currentfill}{rgb}{0.980008,0.966013,0.779393}%
\pgfsetfillcolor{currentfill}%
\pgfsetlinewidth{0.250937pt}%
\definecolor{currentstroke}{rgb}{1.000000,1.000000,1.000000}%
\pgfsetstrokecolor{currentstroke}%
\pgfsetdash{}{0pt}%
\pgfpathmoveto{\pgfqpoint{1.784717in}{2.347925in}}%
\pgfpathlineto{\pgfqpoint{1.872452in}{2.347925in}}%
\pgfpathlineto{\pgfqpoint{1.872452in}{2.260189in}}%
\pgfpathlineto{\pgfqpoint{1.784717in}{2.260189in}}%
\pgfpathlineto{\pgfqpoint{1.784717in}{2.347925in}}%
\pgfusepath{stroke,fill}%
\end{pgfscope}%
\begin{pgfscope}%
\pgfpathrectangle{\pgfqpoint{0.380943in}{2.260189in}}{\pgfqpoint{4.650000in}{0.614151in}}%
\pgfusepath{clip}%
\pgfsetbuttcap%
\pgfsetroundjoin%
\definecolor{currentfill}{rgb}{0.980008,0.966013,0.779393}%
\pgfsetfillcolor{currentfill}%
\pgfsetlinewidth{0.250937pt}%
\definecolor{currentstroke}{rgb}{1.000000,1.000000,1.000000}%
\pgfsetstrokecolor{currentstroke}%
\pgfsetdash{}{0pt}%
\pgfpathmoveto{\pgfqpoint{1.872452in}{2.347925in}}%
\pgfpathlineto{\pgfqpoint{1.960188in}{2.347925in}}%
\pgfpathlineto{\pgfqpoint{1.960188in}{2.260189in}}%
\pgfpathlineto{\pgfqpoint{1.872452in}{2.260189in}}%
\pgfpathlineto{\pgfqpoint{1.872452in}{2.347925in}}%
\pgfusepath{stroke,fill}%
\end{pgfscope}%
\begin{pgfscope}%
\pgfpathrectangle{\pgfqpoint{0.380943in}{2.260189in}}{\pgfqpoint{4.650000in}{0.614151in}}%
\pgfusepath{clip}%
\pgfsetbuttcap%
\pgfsetroundjoin%
\definecolor{currentfill}{rgb}{0.980008,0.966013,0.779393}%
\pgfsetfillcolor{currentfill}%
\pgfsetlinewidth{0.250937pt}%
\definecolor{currentstroke}{rgb}{1.000000,1.000000,1.000000}%
\pgfsetstrokecolor{currentstroke}%
\pgfsetdash{}{0pt}%
\pgfpathmoveto{\pgfqpoint{1.960188in}{2.347925in}}%
\pgfpathlineto{\pgfqpoint{2.047924in}{2.347925in}}%
\pgfpathlineto{\pgfqpoint{2.047924in}{2.260189in}}%
\pgfpathlineto{\pgfqpoint{1.960188in}{2.260189in}}%
\pgfpathlineto{\pgfqpoint{1.960188in}{2.347925in}}%
\pgfusepath{stroke,fill}%
\end{pgfscope}%
\begin{pgfscope}%
\pgfpathrectangle{\pgfqpoint{0.380943in}{2.260189in}}{\pgfqpoint{4.650000in}{0.614151in}}%
\pgfusepath{clip}%
\pgfsetbuttcap%
\pgfsetroundjoin%
\definecolor{currentfill}{rgb}{1.000000,1.000000,0.895579}%
\pgfsetfillcolor{currentfill}%
\pgfsetlinewidth{0.250937pt}%
\definecolor{currentstroke}{rgb}{1.000000,1.000000,1.000000}%
\pgfsetstrokecolor{currentstroke}%
\pgfsetdash{}{0pt}%
\pgfpathmoveto{\pgfqpoint{2.047924in}{2.347925in}}%
\pgfpathlineto{\pgfqpoint{2.135660in}{2.347925in}}%
\pgfpathlineto{\pgfqpoint{2.135660in}{2.260189in}}%
\pgfpathlineto{\pgfqpoint{2.047924in}{2.260189in}}%
\pgfpathlineto{\pgfqpoint{2.047924in}{2.347925in}}%
\pgfusepath{stroke,fill}%
\end{pgfscope}%
\begin{pgfscope}%
\pgfpathrectangle{\pgfqpoint{0.380943in}{2.260189in}}{\pgfqpoint{4.650000in}{0.614151in}}%
\pgfusepath{clip}%
\pgfsetbuttcap%
\pgfsetroundjoin%
\definecolor{currentfill}{rgb}{0.995233,0.991895,0.818977}%
\pgfsetfillcolor{currentfill}%
\pgfsetlinewidth{0.250937pt}%
\definecolor{currentstroke}{rgb}{1.000000,1.000000,1.000000}%
\pgfsetstrokecolor{currentstroke}%
\pgfsetdash{}{0pt}%
\pgfpathmoveto{\pgfqpoint{2.135660in}{2.347925in}}%
\pgfpathlineto{\pgfqpoint{2.223396in}{2.347925in}}%
\pgfpathlineto{\pgfqpoint{2.223396in}{2.260189in}}%
\pgfpathlineto{\pgfqpoint{2.135660in}{2.260189in}}%
\pgfpathlineto{\pgfqpoint{2.135660in}{2.347925in}}%
\pgfusepath{stroke,fill}%
\end{pgfscope}%
\begin{pgfscope}%
\pgfpathrectangle{\pgfqpoint{0.380943in}{2.260189in}}{\pgfqpoint{4.650000in}{0.614151in}}%
\pgfusepath{clip}%
\pgfsetbuttcap%
\pgfsetroundjoin%
\definecolor{currentfill}{rgb}{0.995233,0.991895,0.818977}%
\pgfsetfillcolor{currentfill}%
\pgfsetlinewidth{0.250937pt}%
\definecolor{currentstroke}{rgb}{1.000000,1.000000,1.000000}%
\pgfsetstrokecolor{currentstroke}%
\pgfsetdash{}{0pt}%
\pgfpathmoveto{\pgfqpoint{2.223396in}{2.347925in}}%
\pgfpathlineto{\pgfqpoint{2.311132in}{2.347925in}}%
\pgfpathlineto{\pgfqpoint{2.311132in}{2.260189in}}%
\pgfpathlineto{\pgfqpoint{2.223396in}{2.260189in}}%
\pgfpathlineto{\pgfqpoint{2.223396in}{2.347925in}}%
\pgfusepath{stroke,fill}%
\end{pgfscope}%
\begin{pgfscope}%
\pgfpathrectangle{\pgfqpoint{0.380943in}{2.260189in}}{\pgfqpoint{4.650000in}{0.614151in}}%
\pgfusepath{clip}%
\pgfsetbuttcap%
\pgfsetroundjoin%
\definecolor{currentfill}{rgb}{0.980008,0.966013,0.779393}%
\pgfsetfillcolor{currentfill}%
\pgfsetlinewidth{0.250937pt}%
\definecolor{currentstroke}{rgb}{1.000000,1.000000,1.000000}%
\pgfsetstrokecolor{currentstroke}%
\pgfsetdash{}{0pt}%
\pgfpathmoveto{\pgfqpoint{2.311132in}{2.347925in}}%
\pgfpathlineto{\pgfqpoint{2.398868in}{2.347925in}}%
\pgfpathlineto{\pgfqpoint{2.398868in}{2.260189in}}%
\pgfpathlineto{\pgfqpoint{2.311132in}{2.260189in}}%
\pgfpathlineto{\pgfqpoint{2.311132in}{2.347925in}}%
\pgfusepath{stroke,fill}%
\end{pgfscope}%
\begin{pgfscope}%
\pgfpathrectangle{\pgfqpoint{0.380943in}{2.260189in}}{\pgfqpoint{4.650000in}{0.614151in}}%
\pgfusepath{clip}%
\pgfsetbuttcap%
\pgfsetroundjoin%
\definecolor{currentfill}{rgb}{0.974072,0.862976,0.688750}%
\pgfsetfillcolor{currentfill}%
\pgfsetlinewidth{0.250937pt}%
\definecolor{currentstroke}{rgb}{1.000000,1.000000,1.000000}%
\pgfsetstrokecolor{currentstroke}%
\pgfsetdash{}{0pt}%
\pgfpathmoveto{\pgfqpoint{2.398868in}{2.347925in}}%
\pgfpathlineto{\pgfqpoint{2.486603in}{2.347925in}}%
\pgfpathlineto{\pgfqpoint{2.486603in}{2.260189in}}%
\pgfpathlineto{\pgfqpoint{2.398868in}{2.260189in}}%
\pgfpathlineto{\pgfqpoint{2.398868in}{2.347925in}}%
\pgfusepath{stroke,fill}%
\end{pgfscope}%
\begin{pgfscope}%
\pgfpathrectangle{\pgfqpoint{0.380943in}{2.260189in}}{\pgfqpoint{4.650000in}{0.614151in}}%
\pgfusepath{clip}%
\pgfsetbuttcap%
\pgfsetroundjoin%
\definecolor{currentfill}{rgb}{0.964937,0.908651,0.713110}%
\pgfsetfillcolor{currentfill}%
\pgfsetlinewidth{0.250937pt}%
\definecolor{currentstroke}{rgb}{1.000000,1.000000,1.000000}%
\pgfsetstrokecolor{currentstroke}%
\pgfsetdash{}{0pt}%
\pgfpathmoveto{\pgfqpoint{2.486603in}{2.347925in}}%
\pgfpathlineto{\pgfqpoint{2.574339in}{2.347925in}}%
\pgfpathlineto{\pgfqpoint{2.574339in}{2.260189in}}%
\pgfpathlineto{\pgfqpoint{2.486603in}{2.260189in}}%
\pgfpathlineto{\pgfqpoint{2.486603in}{2.347925in}}%
\pgfusepath{stroke,fill}%
\end{pgfscope}%
\begin{pgfscope}%
\pgfpathrectangle{\pgfqpoint{0.380943in}{2.260189in}}{\pgfqpoint{4.650000in}{0.614151in}}%
\pgfusepath{clip}%
\pgfsetbuttcap%
\pgfsetroundjoin%
\definecolor{currentfill}{rgb}{0.963260,0.918478,0.719508}%
\pgfsetfillcolor{currentfill}%
\pgfsetlinewidth{0.250937pt}%
\definecolor{currentstroke}{rgb}{1.000000,1.000000,1.000000}%
\pgfsetstrokecolor{currentstroke}%
\pgfsetdash{}{0pt}%
\pgfpathmoveto{\pgfqpoint{2.574339in}{2.347925in}}%
\pgfpathlineto{\pgfqpoint{2.662075in}{2.347925in}}%
\pgfpathlineto{\pgfqpoint{2.662075in}{2.260189in}}%
\pgfpathlineto{\pgfqpoint{2.574339in}{2.260189in}}%
\pgfpathlineto{\pgfqpoint{2.574339in}{2.347925in}}%
\pgfusepath{stroke,fill}%
\end{pgfscope}%
\begin{pgfscope}%
\pgfpathrectangle{\pgfqpoint{0.380943in}{2.260189in}}{\pgfqpoint{4.650000in}{0.614151in}}%
\pgfusepath{clip}%
\pgfsetbuttcap%
\pgfsetroundjoin%
\definecolor{currentfill}{rgb}{1.000000,1.000000,0.857516}%
\pgfsetfillcolor{currentfill}%
\pgfsetlinewidth{0.250937pt}%
\definecolor{currentstroke}{rgb}{1.000000,1.000000,1.000000}%
\pgfsetstrokecolor{currentstroke}%
\pgfsetdash{}{0pt}%
\pgfpathmoveto{\pgfqpoint{2.662075in}{2.347925in}}%
\pgfpathlineto{\pgfqpoint{2.749811in}{2.347925in}}%
\pgfpathlineto{\pgfqpoint{2.749811in}{2.260189in}}%
\pgfpathlineto{\pgfqpoint{2.662075in}{2.260189in}}%
\pgfpathlineto{\pgfqpoint{2.662075in}{2.347925in}}%
\pgfusepath{stroke,fill}%
\end{pgfscope}%
\begin{pgfscope}%
\pgfpathrectangle{\pgfqpoint{0.380943in}{2.260189in}}{\pgfqpoint{4.650000in}{0.614151in}}%
\pgfusepath{clip}%
\pgfsetbuttcap%
\pgfsetroundjoin%
\definecolor{currentfill}{rgb}{0.964937,0.908651,0.713110}%
\pgfsetfillcolor{currentfill}%
\pgfsetlinewidth{0.250937pt}%
\definecolor{currentstroke}{rgb}{1.000000,1.000000,1.000000}%
\pgfsetstrokecolor{currentstroke}%
\pgfsetdash{}{0pt}%
\pgfpathmoveto{\pgfqpoint{2.749811in}{2.347925in}}%
\pgfpathlineto{\pgfqpoint{2.837547in}{2.347925in}}%
\pgfpathlineto{\pgfqpoint{2.837547in}{2.260189in}}%
\pgfpathlineto{\pgfqpoint{2.749811in}{2.260189in}}%
\pgfpathlineto{\pgfqpoint{2.749811in}{2.347925in}}%
\pgfusepath{stroke,fill}%
\end{pgfscope}%
\begin{pgfscope}%
\pgfpathrectangle{\pgfqpoint{0.380943in}{2.260189in}}{\pgfqpoint{4.650000in}{0.614151in}}%
\pgfusepath{clip}%
\pgfsetbuttcap%
\pgfsetroundjoin%
\definecolor{currentfill}{rgb}{0.963260,0.918478,0.719508}%
\pgfsetfillcolor{currentfill}%
\pgfsetlinewidth{0.250937pt}%
\definecolor{currentstroke}{rgb}{1.000000,1.000000,1.000000}%
\pgfsetstrokecolor{currentstroke}%
\pgfsetdash{}{0pt}%
\pgfpathmoveto{\pgfqpoint{2.837547in}{2.347925in}}%
\pgfpathlineto{\pgfqpoint{2.925283in}{2.347925in}}%
\pgfpathlineto{\pgfqpoint{2.925283in}{2.260189in}}%
\pgfpathlineto{\pgfqpoint{2.837547in}{2.260189in}}%
\pgfpathlineto{\pgfqpoint{2.837547in}{2.347925in}}%
\pgfusepath{stroke,fill}%
\end{pgfscope}%
\begin{pgfscope}%
\pgfpathrectangle{\pgfqpoint{0.380943in}{2.260189in}}{\pgfqpoint{4.650000in}{0.614151in}}%
\pgfusepath{clip}%
\pgfsetbuttcap%
\pgfsetroundjoin%
\definecolor{currentfill}{rgb}{0.969504,0.885813,0.700930}%
\pgfsetfillcolor{currentfill}%
\pgfsetlinewidth{0.250937pt}%
\definecolor{currentstroke}{rgb}{1.000000,1.000000,1.000000}%
\pgfsetstrokecolor{currentstroke}%
\pgfsetdash{}{0pt}%
\pgfpathmoveto{\pgfqpoint{2.925283in}{2.347925in}}%
\pgfpathlineto{\pgfqpoint{3.013019in}{2.347925in}}%
\pgfpathlineto{\pgfqpoint{3.013019in}{2.260189in}}%
\pgfpathlineto{\pgfqpoint{2.925283in}{2.260189in}}%
\pgfpathlineto{\pgfqpoint{2.925283in}{2.347925in}}%
\pgfusepath{stroke,fill}%
\end{pgfscope}%
\begin{pgfscope}%
\pgfpathrectangle{\pgfqpoint{0.380943in}{2.260189in}}{\pgfqpoint{4.650000in}{0.614151in}}%
\pgfusepath{clip}%
\pgfsetbuttcap%
\pgfsetroundjoin%
\definecolor{currentfill}{rgb}{0.964783,0.940131,0.739808}%
\pgfsetfillcolor{currentfill}%
\pgfsetlinewidth{0.250937pt}%
\definecolor{currentstroke}{rgb}{1.000000,1.000000,1.000000}%
\pgfsetstrokecolor{currentstroke}%
\pgfsetdash{}{0pt}%
\pgfpathmoveto{\pgfqpoint{3.013019in}{2.347925in}}%
\pgfpathlineto{\pgfqpoint{3.100754in}{2.347925in}}%
\pgfpathlineto{\pgfqpoint{3.100754in}{2.260189in}}%
\pgfpathlineto{\pgfqpoint{3.013019in}{2.260189in}}%
\pgfpathlineto{\pgfqpoint{3.013019in}{2.347925in}}%
\pgfusepath{stroke,fill}%
\end{pgfscope}%
\begin{pgfscope}%
\pgfpathrectangle{\pgfqpoint{0.380943in}{2.260189in}}{\pgfqpoint{4.650000in}{0.614151in}}%
\pgfusepath{clip}%
\pgfsetbuttcap%
\pgfsetroundjoin%
\definecolor{currentfill}{rgb}{0.964937,0.908651,0.713110}%
\pgfsetfillcolor{currentfill}%
\pgfsetlinewidth{0.250937pt}%
\definecolor{currentstroke}{rgb}{1.000000,1.000000,1.000000}%
\pgfsetstrokecolor{currentstroke}%
\pgfsetdash{}{0pt}%
\pgfpathmoveto{\pgfqpoint{3.100754in}{2.347925in}}%
\pgfpathlineto{\pgfqpoint{3.188490in}{2.347925in}}%
\pgfpathlineto{\pgfqpoint{3.188490in}{2.260189in}}%
\pgfpathlineto{\pgfqpoint{3.100754in}{2.260189in}}%
\pgfpathlineto{\pgfqpoint{3.100754in}{2.347925in}}%
\pgfusepath{stroke,fill}%
\end{pgfscope}%
\begin{pgfscope}%
\pgfpathrectangle{\pgfqpoint{0.380943in}{2.260189in}}{\pgfqpoint{4.650000in}{0.614151in}}%
\pgfusepath{clip}%
\pgfsetbuttcap%
\pgfsetroundjoin%
\definecolor{currentfill}{rgb}{0.995233,0.991895,0.818977}%
\pgfsetfillcolor{currentfill}%
\pgfsetlinewidth{0.250937pt}%
\definecolor{currentstroke}{rgb}{1.000000,1.000000,1.000000}%
\pgfsetstrokecolor{currentstroke}%
\pgfsetdash{}{0pt}%
\pgfpathmoveto{\pgfqpoint{3.188490in}{2.347925in}}%
\pgfpathlineto{\pgfqpoint{3.276226in}{2.347925in}}%
\pgfpathlineto{\pgfqpoint{3.276226in}{2.260189in}}%
\pgfpathlineto{\pgfqpoint{3.188490in}{2.260189in}}%
\pgfpathlineto{\pgfqpoint{3.188490in}{2.347925in}}%
\pgfusepath{stroke,fill}%
\end{pgfscope}%
\begin{pgfscope}%
\pgfpathrectangle{\pgfqpoint{0.380943in}{2.260189in}}{\pgfqpoint{4.650000in}{0.614151in}}%
\pgfusepath{clip}%
\pgfsetbuttcap%
\pgfsetroundjoin%
\definecolor{currentfill}{rgb}{0.961738,0.927612,0.725598}%
\pgfsetfillcolor{currentfill}%
\pgfsetlinewidth{0.250937pt}%
\definecolor{currentstroke}{rgb}{1.000000,1.000000,1.000000}%
\pgfsetstrokecolor{currentstroke}%
\pgfsetdash{}{0pt}%
\pgfpathmoveto{\pgfqpoint{3.276226in}{2.347925in}}%
\pgfpathlineto{\pgfqpoint{3.363962in}{2.347925in}}%
\pgfpathlineto{\pgfqpoint{3.363962in}{2.260189in}}%
\pgfpathlineto{\pgfqpoint{3.276226in}{2.260189in}}%
\pgfpathlineto{\pgfqpoint{3.276226in}{2.347925in}}%
\pgfusepath{stroke,fill}%
\end{pgfscope}%
\begin{pgfscope}%
\pgfpathrectangle{\pgfqpoint{0.380943in}{2.260189in}}{\pgfqpoint{4.650000in}{0.614151in}}%
\pgfusepath{clip}%
\pgfsetbuttcap%
\pgfsetroundjoin%
\definecolor{currentfill}{rgb}{0.961738,0.927612,0.725598}%
\pgfsetfillcolor{currentfill}%
\pgfsetlinewidth{0.250937pt}%
\definecolor{currentstroke}{rgb}{1.000000,1.000000,1.000000}%
\pgfsetstrokecolor{currentstroke}%
\pgfsetdash{}{0pt}%
\pgfpathmoveto{\pgfqpoint{3.363962in}{2.347925in}}%
\pgfpathlineto{\pgfqpoint{3.451698in}{2.347925in}}%
\pgfpathlineto{\pgfqpoint{3.451698in}{2.260189in}}%
\pgfpathlineto{\pgfqpoint{3.363962in}{2.260189in}}%
\pgfpathlineto{\pgfqpoint{3.363962in}{2.347925in}}%
\pgfusepath{stroke,fill}%
\end{pgfscope}%
\begin{pgfscope}%
\pgfpathrectangle{\pgfqpoint{0.380943in}{2.260189in}}{\pgfqpoint{4.650000in}{0.614151in}}%
\pgfusepath{clip}%
\pgfsetbuttcap%
\pgfsetroundjoin%
\definecolor{currentfill}{rgb}{0.995233,0.991895,0.818977}%
\pgfsetfillcolor{currentfill}%
\pgfsetlinewidth{0.250937pt}%
\definecolor{currentstroke}{rgb}{1.000000,1.000000,1.000000}%
\pgfsetstrokecolor{currentstroke}%
\pgfsetdash{}{0pt}%
\pgfpathmoveto{\pgfqpoint{3.451698in}{2.347925in}}%
\pgfpathlineto{\pgfqpoint{3.539434in}{2.347925in}}%
\pgfpathlineto{\pgfqpoint{3.539434in}{2.260189in}}%
\pgfpathlineto{\pgfqpoint{3.451698in}{2.260189in}}%
\pgfpathlineto{\pgfqpoint{3.451698in}{2.347925in}}%
\pgfusepath{stroke,fill}%
\end{pgfscope}%
\begin{pgfscope}%
\pgfpathrectangle{\pgfqpoint{0.380943in}{2.260189in}}{\pgfqpoint{4.650000in}{0.614151in}}%
\pgfusepath{clip}%
\pgfsetbuttcap%
\pgfsetroundjoin%
\definecolor{currentfill}{rgb}{1.000000,1.000000,0.857516}%
\pgfsetfillcolor{currentfill}%
\pgfsetlinewidth{0.250937pt}%
\definecolor{currentstroke}{rgb}{1.000000,1.000000,1.000000}%
\pgfsetstrokecolor{currentstroke}%
\pgfsetdash{}{0pt}%
\pgfpathmoveto{\pgfqpoint{3.539434in}{2.347925in}}%
\pgfpathlineto{\pgfqpoint{3.627169in}{2.347925in}}%
\pgfpathlineto{\pgfqpoint{3.627169in}{2.260189in}}%
\pgfpathlineto{\pgfqpoint{3.539434in}{2.260189in}}%
\pgfpathlineto{\pgfqpoint{3.539434in}{2.347925in}}%
\pgfusepath{stroke,fill}%
\end{pgfscope}%
\begin{pgfscope}%
\pgfpathrectangle{\pgfqpoint{0.380943in}{2.260189in}}{\pgfqpoint{4.650000in}{0.614151in}}%
\pgfusepath{clip}%
\pgfsetbuttcap%
\pgfsetroundjoin%
\definecolor{currentfill}{rgb}{1.000000,1.000000,0.895579}%
\pgfsetfillcolor{currentfill}%
\pgfsetlinewidth{0.250937pt}%
\definecolor{currentstroke}{rgb}{1.000000,1.000000,1.000000}%
\pgfsetstrokecolor{currentstroke}%
\pgfsetdash{}{0pt}%
\pgfpathmoveto{\pgfqpoint{3.627169in}{2.347925in}}%
\pgfpathlineto{\pgfqpoint{3.714905in}{2.347925in}}%
\pgfpathlineto{\pgfqpoint{3.714905in}{2.260189in}}%
\pgfpathlineto{\pgfqpoint{3.627169in}{2.260189in}}%
\pgfpathlineto{\pgfqpoint{3.627169in}{2.347925in}}%
\pgfusepath{stroke,fill}%
\end{pgfscope}%
\begin{pgfscope}%
\pgfpathrectangle{\pgfqpoint{0.380943in}{2.260189in}}{\pgfqpoint{4.650000in}{0.614151in}}%
\pgfusepath{clip}%
\pgfsetbuttcap%
\pgfsetroundjoin%
\definecolor{currentfill}{rgb}{0.961738,0.927612,0.725598}%
\pgfsetfillcolor{currentfill}%
\pgfsetlinewidth{0.250937pt}%
\definecolor{currentstroke}{rgb}{1.000000,1.000000,1.000000}%
\pgfsetstrokecolor{currentstroke}%
\pgfsetdash{}{0pt}%
\pgfpathmoveto{\pgfqpoint{3.714905in}{2.347925in}}%
\pgfpathlineto{\pgfqpoint{3.802641in}{2.347925in}}%
\pgfpathlineto{\pgfqpoint{3.802641in}{2.260189in}}%
\pgfpathlineto{\pgfqpoint{3.714905in}{2.260189in}}%
\pgfpathlineto{\pgfqpoint{3.714905in}{2.347925in}}%
\pgfusepath{stroke,fill}%
\end{pgfscope}%
\begin{pgfscope}%
\pgfpathrectangle{\pgfqpoint{0.380943in}{2.260189in}}{\pgfqpoint{4.650000in}{0.614151in}}%
\pgfusepath{clip}%
\pgfsetbuttcap%
\pgfsetroundjoin%
\definecolor{currentfill}{rgb}{0.963260,0.918478,0.719508}%
\pgfsetfillcolor{currentfill}%
\pgfsetlinewidth{0.250937pt}%
\definecolor{currentstroke}{rgb}{1.000000,1.000000,1.000000}%
\pgfsetstrokecolor{currentstroke}%
\pgfsetdash{}{0pt}%
\pgfpathmoveto{\pgfqpoint{3.802641in}{2.347925in}}%
\pgfpathlineto{\pgfqpoint{3.890377in}{2.347925in}}%
\pgfpathlineto{\pgfqpoint{3.890377in}{2.260189in}}%
\pgfpathlineto{\pgfqpoint{3.802641in}{2.260189in}}%
\pgfpathlineto{\pgfqpoint{3.802641in}{2.347925in}}%
\pgfusepath{stroke,fill}%
\end{pgfscope}%
\begin{pgfscope}%
\pgfpathrectangle{\pgfqpoint{0.380943in}{2.260189in}}{\pgfqpoint{4.650000in}{0.614151in}}%
\pgfusepath{clip}%
\pgfsetbuttcap%
\pgfsetroundjoin%
\definecolor{currentfill}{rgb}{1.000000,1.000000,0.857516}%
\pgfsetfillcolor{currentfill}%
\pgfsetlinewidth{0.250937pt}%
\definecolor{currentstroke}{rgb}{1.000000,1.000000,1.000000}%
\pgfsetstrokecolor{currentstroke}%
\pgfsetdash{}{0pt}%
\pgfpathmoveto{\pgfqpoint{3.890377in}{2.347925in}}%
\pgfpathlineto{\pgfqpoint{3.978113in}{2.347925in}}%
\pgfpathlineto{\pgfqpoint{3.978113in}{2.260189in}}%
\pgfpathlineto{\pgfqpoint{3.890377in}{2.260189in}}%
\pgfpathlineto{\pgfqpoint{3.890377in}{2.347925in}}%
\pgfusepath{stroke,fill}%
\end{pgfscope}%
\begin{pgfscope}%
\pgfpathrectangle{\pgfqpoint{0.380943in}{2.260189in}}{\pgfqpoint{4.650000in}{0.614151in}}%
\pgfusepath{clip}%
\pgfsetbuttcap%
\pgfsetroundjoin%
\definecolor{currentfill}{rgb}{0.964783,0.940131,0.739808}%
\pgfsetfillcolor{currentfill}%
\pgfsetlinewidth{0.250937pt}%
\definecolor{currentstroke}{rgb}{1.000000,1.000000,1.000000}%
\pgfsetstrokecolor{currentstroke}%
\pgfsetdash{}{0pt}%
\pgfpathmoveto{\pgfqpoint{3.978113in}{2.347925in}}%
\pgfpathlineto{\pgfqpoint{4.065849in}{2.347925in}}%
\pgfpathlineto{\pgfqpoint{4.065849in}{2.260189in}}%
\pgfpathlineto{\pgfqpoint{3.978113in}{2.260189in}}%
\pgfpathlineto{\pgfqpoint{3.978113in}{2.347925in}}%
\pgfusepath{stroke,fill}%
\end{pgfscope}%
\begin{pgfscope}%
\pgfpathrectangle{\pgfqpoint{0.380943in}{2.260189in}}{\pgfqpoint{4.650000in}{0.614151in}}%
\pgfusepath{clip}%
\pgfsetbuttcap%
\pgfsetroundjoin%
\definecolor{currentfill}{rgb}{0.982699,0.823991,0.657439}%
\pgfsetfillcolor{currentfill}%
\pgfsetlinewidth{0.250937pt}%
\definecolor{currentstroke}{rgb}{1.000000,1.000000,1.000000}%
\pgfsetstrokecolor{currentstroke}%
\pgfsetdash{}{0pt}%
\pgfpathmoveto{\pgfqpoint{4.065849in}{2.347925in}}%
\pgfpathlineto{\pgfqpoint{4.153585in}{2.347925in}}%
\pgfpathlineto{\pgfqpoint{4.153585in}{2.260189in}}%
\pgfpathlineto{\pgfqpoint{4.065849in}{2.260189in}}%
\pgfpathlineto{\pgfqpoint{4.065849in}{2.347925in}}%
\pgfusepath{stroke,fill}%
\end{pgfscope}%
\begin{pgfscope}%
\pgfpathrectangle{\pgfqpoint{0.380943in}{2.260189in}}{\pgfqpoint{4.650000in}{0.614151in}}%
\pgfusepath{clip}%
\pgfsetbuttcap%
\pgfsetroundjoin%
\definecolor{currentfill}{rgb}{0.969504,0.885813,0.700930}%
\pgfsetfillcolor{currentfill}%
\pgfsetlinewidth{0.250937pt}%
\definecolor{currentstroke}{rgb}{1.000000,1.000000,1.000000}%
\pgfsetstrokecolor{currentstroke}%
\pgfsetdash{}{0pt}%
\pgfpathmoveto{\pgfqpoint{4.153585in}{2.347925in}}%
\pgfpathlineto{\pgfqpoint{4.241320in}{2.347925in}}%
\pgfpathlineto{\pgfqpoint{4.241320in}{2.260189in}}%
\pgfpathlineto{\pgfqpoint{4.153585in}{2.260189in}}%
\pgfpathlineto{\pgfqpoint{4.153585in}{2.347925in}}%
\pgfusepath{stroke,fill}%
\end{pgfscope}%
\begin{pgfscope}%
\pgfpathrectangle{\pgfqpoint{0.380943in}{2.260189in}}{\pgfqpoint{4.650000in}{0.614151in}}%
\pgfusepath{clip}%
\pgfsetbuttcap%
\pgfsetroundjoin%
\definecolor{currentfill}{rgb}{0.964937,0.908651,0.713110}%
\pgfsetfillcolor{currentfill}%
\pgfsetlinewidth{0.250937pt}%
\definecolor{currentstroke}{rgb}{1.000000,1.000000,1.000000}%
\pgfsetstrokecolor{currentstroke}%
\pgfsetdash{}{0pt}%
\pgfpathmoveto{\pgfqpoint{4.241320in}{2.347925in}}%
\pgfpathlineto{\pgfqpoint{4.329056in}{2.347925in}}%
\pgfpathlineto{\pgfqpoint{4.329056in}{2.260189in}}%
\pgfpathlineto{\pgfqpoint{4.241320in}{2.260189in}}%
\pgfpathlineto{\pgfqpoint{4.241320in}{2.347925in}}%
\pgfusepath{stroke,fill}%
\end{pgfscope}%
\begin{pgfscope}%
\pgfpathrectangle{\pgfqpoint{0.380943in}{2.260189in}}{\pgfqpoint{4.650000in}{0.614151in}}%
\pgfusepath{clip}%
\pgfsetbuttcap%
\pgfsetroundjoin%
\definecolor{currentfill}{rgb}{0.961738,0.927612,0.725598}%
\pgfsetfillcolor{currentfill}%
\pgfsetlinewidth{0.250937pt}%
\definecolor{currentstroke}{rgb}{1.000000,1.000000,1.000000}%
\pgfsetstrokecolor{currentstroke}%
\pgfsetdash{}{0pt}%
\pgfpathmoveto{\pgfqpoint{4.329056in}{2.347925in}}%
\pgfpathlineto{\pgfqpoint{4.416792in}{2.347925in}}%
\pgfpathlineto{\pgfqpoint{4.416792in}{2.260189in}}%
\pgfpathlineto{\pgfqpoint{4.329056in}{2.260189in}}%
\pgfpathlineto{\pgfqpoint{4.329056in}{2.347925in}}%
\pgfusepath{stroke,fill}%
\end{pgfscope}%
\begin{pgfscope}%
\pgfpathrectangle{\pgfqpoint{0.380943in}{2.260189in}}{\pgfqpoint{4.650000in}{0.614151in}}%
\pgfusepath{clip}%
\pgfsetbuttcap%
\pgfsetroundjoin%
\definecolor{currentfill}{rgb}{0.961738,0.927612,0.725598}%
\pgfsetfillcolor{currentfill}%
\pgfsetlinewidth{0.250937pt}%
\definecolor{currentstroke}{rgb}{1.000000,1.000000,1.000000}%
\pgfsetstrokecolor{currentstroke}%
\pgfsetdash{}{0pt}%
\pgfpathmoveto{\pgfqpoint{4.416792in}{2.347925in}}%
\pgfpathlineto{\pgfqpoint{4.504528in}{2.347925in}}%
\pgfpathlineto{\pgfqpoint{4.504528in}{2.260189in}}%
\pgfpathlineto{\pgfqpoint{4.416792in}{2.260189in}}%
\pgfpathlineto{\pgfqpoint{4.416792in}{2.347925in}}%
\pgfusepath{stroke,fill}%
\end{pgfscope}%
\begin{pgfscope}%
\pgfpathrectangle{\pgfqpoint{0.380943in}{2.260189in}}{\pgfqpoint{4.650000in}{0.614151in}}%
\pgfusepath{clip}%
\pgfsetbuttcap%
\pgfsetroundjoin%
\definecolor{currentfill}{rgb}{0.969504,0.885813,0.700930}%
\pgfsetfillcolor{currentfill}%
\pgfsetlinewidth{0.250937pt}%
\definecolor{currentstroke}{rgb}{1.000000,1.000000,1.000000}%
\pgfsetstrokecolor{currentstroke}%
\pgfsetdash{}{0pt}%
\pgfpathmoveto{\pgfqpoint{4.504528in}{2.347925in}}%
\pgfpathlineto{\pgfqpoint{4.592264in}{2.347925in}}%
\pgfpathlineto{\pgfqpoint{4.592264in}{2.260189in}}%
\pgfpathlineto{\pgfqpoint{4.504528in}{2.260189in}}%
\pgfpathlineto{\pgfqpoint{4.504528in}{2.347925in}}%
\pgfusepath{stroke,fill}%
\end{pgfscope}%
\begin{pgfscope}%
\pgfpathrectangle{\pgfqpoint{0.380943in}{2.260189in}}{\pgfqpoint{4.650000in}{0.614151in}}%
\pgfusepath{clip}%
\pgfsetbuttcap%
\pgfsetroundjoin%
\definecolor{currentfill}{rgb}{1.000000,1.000000,0.857516}%
\pgfsetfillcolor{currentfill}%
\pgfsetlinewidth{0.250937pt}%
\definecolor{currentstroke}{rgb}{1.000000,1.000000,1.000000}%
\pgfsetstrokecolor{currentstroke}%
\pgfsetdash{}{0pt}%
\pgfpathmoveto{\pgfqpoint{4.592264in}{2.347925in}}%
\pgfpathlineto{\pgfqpoint{4.680000in}{2.347925in}}%
\pgfpathlineto{\pgfqpoint{4.680000in}{2.260189in}}%
\pgfpathlineto{\pgfqpoint{4.592264in}{2.260189in}}%
\pgfpathlineto{\pgfqpoint{4.592264in}{2.347925in}}%
\pgfusepath{stroke,fill}%
\end{pgfscope}%
\begin{pgfscope}%
\pgfpathrectangle{\pgfqpoint{0.380943in}{2.260189in}}{\pgfqpoint{4.650000in}{0.614151in}}%
\pgfusepath{clip}%
\pgfsetbuttcap%
\pgfsetroundjoin%
\definecolor{currentfill}{rgb}{0.964783,0.940131,0.739808}%
\pgfsetfillcolor{currentfill}%
\pgfsetlinewidth{0.250937pt}%
\definecolor{currentstroke}{rgb}{1.000000,1.000000,1.000000}%
\pgfsetstrokecolor{currentstroke}%
\pgfsetdash{}{0pt}%
\pgfpathmoveto{\pgfqpoint{4.680000in}{2.347925in}}%
\pgfpathlineto{\pgfqpoint{4.767736in}{2.347925in}}%
\pgfpathlineto{\pgfqpoint{4.767736in}{2.260189in}}%
\pgfpathlineto{\pgfqpoint{4.680000in}{2.260189in}}%
\pgfpathlineto{\pgfqpoint{4.680000in}{2.347925in}}%
\pgfusepath{stroke,fill}%
\end{pgfscope}%
\begin{pgfscope}%
\pgfpathrectangle{\pgfqpoint{0.380943in}{2.260189in}}{\pgfqpoint{4.650000in}{0.614151in}}%
\pgfusepath{clip}%
\pgfsetbuttcap%
\pgfsetroundjoin%
\definecolor{currentfill}{rgb}{0.969504,0.885813,0.700930}%
\pgfsetfillcolor{currentfill}%
\pgfsetlinewidth{0.250937pt}%
\definecolor{currentstroke}{rgb}{1.000000,1.000000,1.000000}%
\pgfsetstrokecolor{currentstroke}%
\pgfsetdash{}{0pt}%
\pgfpathmoveto{\pgfqpoint{4.767736in}{2.347925in}}%
\pgfpathlineto{\pgfqpoint{4.855471in}{2.347925in}}%
\pgfpathlineto{\pgfqpoint{4.855471in}{2.260189in}}%
\pgfpathlineto{\pgfqpoint{4.767736in}{2.260189in}}%
\pgfpathlineto{\pgfqpoint{4.767736in}{2.347925in}}%
\pgfusepath{stroke,fill}%
\end{pgfscope}%
\begin{pgfscope}%
\pgfpathrectangle{\pgfqpoint{0.380943in}{2.260189in}}{\pgfqpoint{4.650000in}{0.614151in}}%
\pgfusepath{clip}%
\pgfsetbuttcap%
\pgfsetroundjoin%
\definecolor{currentfill}{rgb}{0.980008,0.966013,0.779393}%
\pgfsetfillcolor{currentfill}%
\pgfsetlinewidth{0.250937pt}%
\definecolor{currentstroke}{rgb}{1.000000,1.000000,1.000000}%
\pgfsetstrokecolor{currentstroke}%
\pgfsetdash{}{0pt}%
\pgfpathmoveto{\pgfqpoint{4.855471in}{2.347925in}}%
\pgfpathlineto{\pgfqpoint{4.943207in}{2.347925in}}%
\pgfpathlineto{\pgfqpoint{4.943207in}{2.260189in}}%
\pgfpathlineto{\pgfqpoint{4.855471in}{2.260189in}}%
\pgfpathlineto{\pgfqpoint{4.855471in}{2.347925in}}%
\pgfusepath{stroke,fill}%
\end{pgfscope}%
\begin{pgfscope}%
\pgfpathrectangle{\pgfqpoint{0.380943in}{2.260189in}}{\pgfqpoint{4.650000in}{0.614151in}}%
\pgfusepath{clip}%
\pgfsetbuttcap%
\pgfsetroundjoin%
\pgfsetlinewidth{0.250937pt}%
\definecolor{currentstroke}{rgb}{1.000000,1.000000,1.000000}%
\pgfsetstrokecolor{currentstroke}%
\pgfsetdash{}{0pt}%
\pgfpathmoveto{\pgfqpoint{4.943207in}{2.347925in}}%
\pgfpathlineto{\pgfqpoint{5.030943in}{2.347925in}}%
\pgfpathlineto{\pgfqpoint{5.030943in}{2.260189in}}%
\pgfpathlineto{\pgfqpoint{4.943207in}{2.260189in}}%
\pgfpathlineto{\pgfqpoint{4.943207in}{2.347925in}}%
\pgfusepath{stroke}%
\end{pgfscope}%
\begin{pgfscope}%
\pgfsetbuttcap%
\pgfsetroundjoin%
\definecolor{currentfill}{rgb}{0.000000,0.000000,0.000000}%
\pgfsetfillcolor{currentfill}%
\pgfsetlinewidth{0.803000pt}%
\definecolor{currentstroke}{rgb}{0.000000,0.000000,0.000000}%
\pgfsetstrokecolor{currentstroke}%
\pgfsetdash{}{0pt}%
\pgfsys@defobject{currentmarker}{\pgfqpoint{0.000000in}{-0.048611in}}{\pgfqpoint{0.000000in}{0.000000in}}{%
\pgfpathmoveto{\pgfqpoint{0.000000in}{0.000000in}}%
\pgfpathlineto{\pgfqpoint{0.000000in}{-0.048611in}}%
\pgfusepath{stroke,fill}%
}%
\begin{pgfscope}%
\pgfsys@transformshift{0.600283in}{2.260189in}%
\pgfsys@useobject{currentmarker}{}%
\end{pgfscope}%
\end{pgfscope}%
\begin{pgfscope}%
\definecolor{textcolor}{rgb}{0.000000,0.000000,0.000000}%
\pgfsetstrokecolor{textcolor}%
\pgfsetfillcolor{textcolor}%
\pgftext[x=0.600283in,y=2.162967in,,top]{\color{textcolor}\rmfamily\fontsize{8.000000}{9.600000}\selectfont Jan}%
\end{pgfscope}%
\begin{pgfscope}%
\pgfsetbuttcap%
\pgfsetroundjoin%
\definecolor{currentfill}{rgb}{0.000000,0.000000,0.000000}%
\pgfsetfillcolor{currentfill}%
\pgfsetlinewidth{0.803000pt}%
\definecolor{currentstroke}{rgb}{0.000000,0.000000,0.000000}%
\pgfsetstrokecolor{currentstroke}%
\pgfsetdash{}{0pt}%
\pgfsys@defobject{currentmarker}{\pgfqpoint{0.000000in}{-0.048611in}}{\pgfqpoint{0.000000in}{0.000000in}}{%
\pgfpathmoveto{\pgfqpoint{0.000000in}{0.000000in}}%
\pgfpathlineto{\pgfqpoint{0.000000in}{-0.048611in}}%
\pgfusepath{stroke,fill}%
}%
\begin{pgfscope}%
\pgfsys@transformshift{0.951226in}{2.260189in}%
\pgfsys@useobject{currentmarker}{}%
\end{pgfscope}%
\end{pgfscope}%
\begin{pgfscope}%
\definecolor{textcolor}{rgb}{0.000000,0.000000,0.000000}%
\pgfsetstrokecolor{textcolor}%
\pgfsetfillcolor{textcolor}%
\pgftext[x=0.951226in,y=2.162967in,,top]{\color{textcolor}\rmfamily\fontsize{8.000000}{9.600000}\selectfont Feb}%
\end{pgfscope}%
\begin{pgfscope}%
\pgfsetbuttcap%
\pgfsetroundjoin%
\definecolor{currentfill}{rgb}{0.000000,0.000000,0.000000}%
\pgfsetfillcolor{currentfill}%
\pgfsetlinewidth{0.803000pt}%
\definecolor{currentstroke}{rgb}{0.000000,0.000000,0.000000}%
\pgfsetstrokecolor{currentstroke}%
\pgfsetdash{}{0pt}%
\pgfsys@defobject{currentmarker}{\pgfqpoint{0.000000in}{-0.048611in}}{\pgfqpoint{0.000000in}{0.000000in}}{%
\pgfpathmoveto{\pgfqpoint{0.000000in}{0.000000in}}%
\pgfpathlineto{\pgfqpoint{0.000000in}{-0.048611in}}%
\pgfusepath{stroke,fill}%
}%
\begin{pgfscope}%
\pgfsys@transformshift{1.346037in}{2.260189in}%
\pgfsys@useobject{currentmarker}{}%
\end{pgfscope}%
\end{pgfscope}%
\begin{pgfscope}%
\definecolor{textcolor}{rgb}{0.000000,0.000000,0.000000}%
\pgfsetstrokecolor{textcolor}%
\pgfsetfillcolor{textcolor}%
\pgftext[x=1.346037in,y=2.162967in,,top]{\color{textcolor}\rmfamily\fontsize{8.000000}{9.600000}\selectfont Mar}%
\end{pgfscope}%
\begin{pgfscope}%
\pgfsetbuttcap%
\pgfsetroundjoin%
\definecolor{currentfill}{rgb}{0.000000,0.000000,0.000000}%
\pgfsetfillcolor{currentfill}%
\pgfsetlinewidth{0.803000pt}%
\definecolor{currentstroke}{rgb}{0.000000,0.000000,0.000000}%
\pgfsetstrokecolor{currentstroke}%
\pgfsetdash{}{0pt}%
\pgfsys@defobject{currentmarker}{\pgfqpoint{0.000000in}{-0.048611in}}{\pgfqpoint{0.000000in}{0.000000in}}{%
\pgfpathmoveto{\pgfqpoint{0.000000in}{0.000000in}}%
\pgfpathlineto{\pgfqpoint{0.000000in}{-0.048611in}}%
\pgfusepath{stroke,fill}%
}%
\begin{pgfscope}%
\pgfsys@transformshift{1.740849in}{2.260189in}%
\pgfsys@useobject{currentmarker}{}%
\end{pgfscope}%
\end{pgfscope}%
\begin{pgfscope}%
\definecolor{textcolor}{rgb}{0.000000,0.000000,0.000000}%
\pgfsetstrokecolor{textcolor}%
\pgfsetfillcolor{textcolor}%
\pgftext[x=1.740849in,y=2.162967in,,top]{\color{textcolor}\rmfamily\fontsize{8.000000}{9.600000}\selectfont Apr}%
\end{pgfscope}%
\begin{pgfscope}%
\pgfsetbuttcap%
\pgfsetroundjoin%
\definecolor{currentfill}{rgb}{0.000000,0.000000,0.000000}%
\pgfsetfillcolor{currentfill}%
\pgfsetlinewidth{0.803000pt}%
\definecolor{currentstroke}{rgb}{0.000000,0.000000,0.000000}%
\pgfsetstrokecolor{currentstroke}%
\pgfsetdash{}{0pt}%
\pgfsys@defobject{currentmarker}{\pgfqpoint{0.000000in}{-0.048611in}}{\pgfqpoint{0.000000in}{0.000000in}}{%
\pgfpathmoveto{\pgfqpoint{0.000000in}{0.000000in}}%
\pgfpathlineto{\pgfqpoint{0.000000in}{-0.048611in}}%
\pgfusepath{stroke,fill}%
}%
\begin{pgfscope}%
\pgfsys@transformshift{2.091792in}{2.260189in}%
\pgfsys@useobject{currentmarker}{}%
\end{pgfscope}%
\end{pgfscope}%
\begin{pgfscope}%
\definecolor{textcolor}{rgb}{0.000000,0.000000,0.000000}%
\pgfsetstrokecolor{textcolor}%
\pgfsetfillcolor{textcolor}%
\pgftext[x=2.091792in,y=2.162967in,,top]{\color{textcolor}\rmfamily\fontsize{8.000000}{9.600000}\selectfont May}%
\end{pgfscope}%
\begin{pgfscope}%
\pgfsetbuttcap%
\pgfsetroundjoin%
\definecolor{currentfill}{rgb}{0.000000,0.000000,0.000000}%
\pgfsetfillcolor{currentfill}%
\pgfsetlinewidth{0.803000pt}%
\definecolor{currentstroke}{rgb}{0.000000,0.000000,0.000000}%
\pgfsetstrokecolor{currentstroke}%
\pgfsetdash{}{0pt}%
\pgfsys@defobject{currentmarker}{\pgfqpoint{0.000000in}{-0.048611in}}{\pgfqpoint{0.000000in}{0.000000in}}{%
\pgfpathmoveto{\pgfqpoint{0.000000in}{0.000000in}}%
\pgfpathlineto{\pgfqpoint{0.000000in}{-0.048611in}}%
\pgfusepath{stroke,fill}%
}%
\begin{pgfscope}%
\pgfsys@transformshift{2.530471in}{2.260189in}%
\pgfsys@useobject{currentmarker}{}%
\end{pgfscope}%
\end{pgfscope}%
\begin{pgfscope}%
\definecolor{textcolor}{rgb}{0.000000,0.000000,0.000000}%
\pgfsetstrokecolor{textcolor}%
\pgfsetfillcolor{textcolor}%
\pgftext[x=2.530471in,y=2.162967in,,top]{\color{textcolor}\rmfamily\fontsize{8.000000}{9.600000}\selectfont Jun}%
\end{pgfscope}%
\begin{pgfscope}%
\pgfsetbuttcap%
\pgfsetroundjoin%
\definecolor{currentfill}{rgb}{0.000000,0.000000,0.000000}%
\pgfsetfillcolor{currentfill}%
\pgfsetlinewidth{0.803000pt}%
\definecolor{currentstroke}{rgb}{0.000000,0.000000,0.000000}%
\pgfsetstrokecolor{currentstroke}%
\pgfsetdash{}{0pt}%
\pgfsys@defobject{currentmarker}{\pgfqpoint{0.000000in}{-0.048611in}}{\pgfqpoint{0.000000in}{0.000000in}}{%
\pgfpathmoveto{\pgfqpoint{0.000000in}{0.000000in}}%
\pgfpathlineto{\pgfqpoint{0.000000in}{-0.048611in}}%
\pgfusepath{stroke,fill}%
}%
\begin{pgfscope}%
\pgfsys@transformshift{2.881415in}{2.260189in}%
\pgfsys@useobject{currentmarker}{}%
\end{pgfscope}%
\end{pgfscope}%
\begin{pgfscope}%
\definecolor{textcolor}{rgb}{0.000000,0.000000,0.000000}%
\pgfsetstrokecolor{textcolor}%
\pgfsetfillcolor{textcolor}%
\pgftext[x=2.881415in,y=2.162967in,,top]{\color{textcolor}\rmfamily\fontsize{8.000000}{9.600000}\selectfont Jul}%
\end{pgfscope}%
\begin{pgfscope}%
\pgfsetbuttcap%
\pgfsetroundjoin%
\definecolor{currentfill}{rgb}{0.000000,0.000000,0.000000}%
\pgfsetfillcolor{currentfill}%
\pgfsetlinewidth{0.803000pt}%
\definecolor{currentstroke}{rgb}{0.000000,0.000000,0.000000}%
\pgfsetstrokecolor{currentstroke}%
\pgfsetdash{}{0pt}%
\pgfsys@defobject{currentmarker}{\pgfqpoint{0.000000in}{-0.048611in}}{\pgfqpoint{0.000000in}{0.000000in}}{%
\pgfpathmoveto{\pgfqpoint{0.000000in}{0.000000in}}%
\pgfpathlineto{\pgfqpoint{0.000000in}{-0.048611in}}%
\pgfusepath{stroke,fill}%
}%
\begin{pgfscope}%
\pgfsys@transformshift{3.276226in}{2.260189in}%
\pgfsys@useobject{currentmarker}{}%
\end{pgfscope}%
\end{pgfscope}%
\begin{pgfscope}%
\definecolor{textcolor}{rgb}{0.000000,0.000000,0.000000}%
\pgfsetstrokecolor{textcolor}%
\pgfsetfillcolor{textcolor}%
\pgftext[x=3.276226in,y=2.162967in,,top]{\color{textcolor}\rmfamily\fontsize{8.000000}{9.600000}\selectfont Aug}%
\end{pgfscope}%
\begin{pgfscope}%
\pgfsetbuttcap%
\pgfsetroundjoin%
\definecolor{currentfill}{rgb}{0.000000,0.000000,0.000000}%
\pgfsetfillcolor{currentfill}%
\pgfsetlinewidth{0.803000pt}%
\definecolor{currentstroke}{rgb}{0.000000,0.000000,0.000000}%
\pgfsetstrokecolor{currentstroke}%
\pgfsetdash{}{0pt}%
\pgfsys@defobject{currentmarker}{\pgfqpoint{0.000000in}{-0.048611in}}{\pgfqpoint{0.000000in}{0.000000in}}{%
\pgfpathmoveto{\pgfqpoint{0.000000in}{0.000000in}}%
\pgfpathlineto{\pgfqpoint{0.000000in}{-0.048611in}}%
\pgfusepath{stroke,fill}%
}%
\begin{pgfscope}%
\pgfsys@transformshift{3.671037in}{2.260189in}%
\pgfsys@useobject{currentmarker}{}%
\end{pgfscope}%
\end{pgfscope}%
\begin{pgfscope}%
\definecolor{textcolor}{rgb}{0.000000,0.000000,0.000000}%
\pgfsetstrokecolor{textcolor}%
\pgfsetfillcolor{textcolor}%
\pgftext[x=3.671037in,y=2.162967in,,top]{\color{textcolor}\rmfamily\fontsize{8.000000}{9.600000}\selectfont Sep}%
\end{pgfscope}%
\begin{pgfscope}%
\pgfsetbuttcap%
\pgfsetroundjoin%
\definecolor{currentfill}{rgb}{0.000000,0.000000,0.000000}%
\pgfsetfillcolor{currentfill}%
\pgfsetlinewidth{0.803000pt}%
\definecolor{currentstroke}{rgb}{0.000000,0.000000,0.000000}%
\pgfsetstrokecolor{currentstroke}%
\pgfsetdash{}{0pt}%
\pgfsys@defobject{currentmarker}{\pgfqpoint{0.000000in}{-0.048611in}}{\pgfqpoint{0.000000in}{0.000000in}}{%
\pgfpathmoveto{\pgfqpoint{0.000000in}{0.000000in}}%
\pgfpathlineto{\pgfqpoint{0.000000in}{-0.048611in}}%
\pgfusepath{stroke,fill}%
}%
\begin{pgfscope}%
\pgfsys@transformshift{4.021981in}{2.260189in}%
\pgfsys@useobject{currentmarker}{}%
\end{pgfscope}%
\end{pgfscope}%
\begin{pgfscope}%
\definecolor{textcolor}{rgb}{0.000000,0.000000,0.000000}%
\pgfsetstrokecolor{textcolor}%
\pgfsetfillcolor{textcolor}%
\pgftext[x=4.021981in,y=2.162967in,,top]{\color{textcolor}\rmfamily\fontsize{8.000000}{9.600000}\selectfont Oct}%
\end{pgfscope}%
\begin{pgfscope}%
\pgfsetbuttcap%
\pgfsetroundjoin%
\definecolor{currentfill}{rgb}{0.000000,0.000000,0.000000}%
\pgfsetfillcolor{currentfill}%
\pgfsetlinewidth{0.803000pt}%
\definecolor{currentstroke}{rgb}{0.000000,0.000000,0.000000}%
\pgfsetstrokecolor{currentstroke}%
\pgfsetdash{}{0pt}%
\pgfsys@defobject{currentmarker}{\pgfqpoint{0.000000in}{-0.048611in}}{\pgfqpoint{0.000000in}{0.000000in}}{%
\pgfpathmoveto{\pgfqpoint{0.000000in}{0.000000in}}%
\pgfpathlineto{\pgfqpoint{0.000000in}{-0.048611in}}%
\pgfusepath{stroke,fill}%
}%
\begin{pgfscope}%
\pgfsys@transformshift{4.416792in}{2.260189in}%
\pgfsys@useobject{currentmarker}{}%
\end{pgfscope}%
\end{pgfscope}%
\begin{pgfscope}%
\definecolor{textcolor}{rgb}{0.000000,0.000000,0.000000}%
\pgfsetstrokecolor{textcolor}%
\pgfsetfillcolor{textcolor}%
\pgftext[x=4.416792in,y=2.162967in,,top]{\color{textcolor}\rmfamily\fontsize{8.000000}{9.600000}\selectfont Nov}%
\end{pgfscope}%
\begin{pgfscope}%
\pgfsetbuttcap%
\pgfsetroundjoin%
\definecolor{currentfill}{rgb}{0.000000,0.000000,0.000000}%
\pgfsetfillcolor{currentfill}%
\pgfsetlinewidth{0.803000pt}%
\definecolor{currentstroke}{rgb}{0.000000,0.000000,0.000000}%
\pgfsetstrokecolor{currentstroke}%
\pgfsetdash{}{0pt}%
\pgfsys@defobject{currentmarker}{\pgfqpoint{0.000000in}{-0.048611in}}{\pgfqpoint{0.000000in}{0.000000in}}{%
\pgfpathmoveto{\pgfqpoint{0.000000in}{0.000000in}}%
\pgfpathlineto{\pgfqpoint{0.000000in}{-0.048611in}}%
\pgfusepath{stroke,fill}%
}%
\begin{pgfscope}%
\pgfsys@transformshift{4.811603in}{2.260189in}%
\pgfsys@useobject{currentmarker}{}%
\end{pgfscope}%
\end{pgfscope}%
\begin{pgfscope}%
\definecolor{textcolor}{rgb}{0.000000,0.000000,0.000000}%
\pgfsetstrokecolor{textcolor}%
\pgfsetfillcolor{textcolor}%
\pgftext[x=4.811603in,y=2.162967in,,top]{\color{textcolor}\rmfamily\fontsize{8.000000}{9.600000}\selectfont Dec}%
\end{pgfscope}%
\begin{pgfscope}%
\pgfsetbuttcap%
\pgfsetroundjoin%
\definecolor{currentfill}{rgb}{0.000000,0.000000,0.000000}%
\pgfsetfillcolor{currentfill}%
\pgfsetlinewidth{0.803000pt}%
\definecolor{currentstroke}{rgb}{0.000000,0.000000,0.000000}%
\pgfsetstrokecolor{currentstroke}%
\pgfsetdash{}{0pt}%
\pgfsys@defobject{currentmarker}{\pgfqpoint{-0.048611in}{0.000000in}}{\pgfqpoint{-0.000000in}{0.000000in}}{%
\pgfpathmoveto{\pgfqpoint{-0.000000in}{0.000000in}}%
\pgfpathlineto{\pgfqpoint{-0.048611in}{0.000000in}}%
\pgfusepath{stroke,fill}%
}%
\begin{pgfscope}%
\pgfsys@transformshift{0.380943in}{2.830472in}%
\pgfsys@useobject{currentmarker}{}%
\end{pgfscope}%
\end{pgfscope}%
\begin{pgfscope}%
\definecolor{textcolor}{rgb}{0.000000,0.000000,0.000000}%
\pgfsetstrokecolor{textcolor}%
\pgfsetfillcolor{textcolor}%
\pgftext[x=0.113117in, y=2.791892in, left, base]{\color{textcolor}\rmfamily\fontsize{8.000000}{9.600000}\selectfont M}%
\end{pgfscope}%
\begin{pgfscope}%
\pgfsetbuttcap%
\pgfsetroundjoin%
\definecolor{currentfill}{rgb}{0.000000,0.000000,0.000000}%
\pgfsetfillcolor{currentfill}%
\pgfsetlinewidth{0.803000pt}%
\definecolor{currentstroke}{rgb}{0.000000,0.000000,0.000000}%
\pgfsetstrokecolor{currentstroke}%
\pgfsetdash{}{0pt}%
\pgfsys@defobject{currentmarker}{\pgfqpoint{-0.048611in}{0.000000in}}{\pgfqpoint{-0.000000in}{0.000000in}}{%
\pgfpathmoveto{\pgfqpoint{-0.000000in}{0.000000in}}%
\pgfpathlineto{\pgfqpoint{-0.048611in}{0.000000in}}%
\pgfusepath{stroke,fill}%
}%
\begin{pgfscope}%
\pgfsys@transformshift{0.380943in}{2.742736in}%
\pgfsys@useobject{currentmarker}{}%
\end{pgfscope}%
\end{pgfscope}%
\begin{pgfscope}%
\definecolor{textcolor}{rgb}{0.000000,0.000000,0.000000}%
\pgfsetstrokecolor{textcolor}%
\pgfsetfillcolor{textcolor}%
\pgftext[x=0.135957in, y=2.704156in, left, base]{\color{textcolor}\rmfamily\fontsize{8.000000}{9.600000}\selectfont T}%
\end{pgfscope}%
\begin{pgfscope}%
\pgfsetbuttcap%
\pgfsetroundjoin%
\definecolor{currentfill}{rgb}{0.000000,0.000000,0.000000}%
\pgfsetfillcolor{currentfill}%
\pgfsetlinewidth{0.803000pt}%
\definecolor{currentstroke}{rgb}{0.000000,0.000000,0.000000}%
\pgfsetstrokecolor{currentstroke}%
\pgfsetdash{}{0pt}%
\pgfsys@defobject{currentmarker}{\pgfqpoint{-0.048611in}{0.000000in}}{\pgfqpoint{-0.000000in}{0.000000in}}{%
\pgfpathmoveto{\pgfqpoint{-0.000000in}{0.000000in}}%
\pgfpathlineto{\pgfqpoint{-0.048611in}{0.000000in}}%
\pgfusepath{stroke,fill}%
}%
\begin{pgfscope}%
\pgfsys@transformshift{0.380943in}{2.655000in}%
\pgfsys@useobject{currentmarker}{}%
\end{pgfscope}%
\end{pgfscope}%
\begin{pgfscope}%
\definecolor{textcolor}{rgb}{0.000000,0.000000,0.000000}%
\pgfsetstrokecolor{textcolor}%
\pgfsetfillcolor{textcolor}%
\pgftext[x=0.100000in, y=2.616420in, left, base]{\color{textcolor}\rmfamily\fontsize{8.000000}{9.600000}\selectfont W}%
\end{pgfscope}%
\begin{pgfscope}%
\pgfsetbuttcap%
\pgfsetroundjoin%
\definecolor{currentfill}{rgb}{0.000000,0.000000,0.000000}%
\pgfsetfillcolor{currentfill}%
\pgfsetlinewidth{0.803000pt}%
\definecolor{currentstroke}{rgb}{0.000000,0.000000,0.000000}%
\pgfsetstrokecolor{currentstroke}%
\pgfsetdash{}{0pt}%
\pgfsys@defobject{currentmarker}{\pgfqpoint{-0.048611in}{0.000000in}}{\pgfqpoint{-0.000000in}{0.000000in}}{%
\pgfpathmoveto{\pgfqpoint{-0.000000in}{0.000000in}}%
\pgfpathlineto{\pgfqpoint{-0.048611in}{0.000000in}}%
\pgfusepath{stroke,fill}%
}%
\begin{pgfscope}%
\pgfsys@transformshift{0.380943in}{2.567264in}%
\pgfsys@useobject{currentmarker}{}%
\end{pgfscope}%
\end{pgfscope}%
\begin{pgfscope}%
\definecolor{textcolor}{rgb}{0.000000,0.000000,0.000000}%
\pgfsetstrokecolor{textcolor}%
\pgfsetfillcolor{textcolor}%
\pgftext[x=0.135957in, y=2.528684in, left, base]{\color{textcolor}\rmfamily\fontsize{8.000000}{9.600000}\selectfont T}%
\end{pgfscope}%
\begin{pgfscope}%
\pgfsetbuttcap%
\pgfsetroundjoin%
\definecolor{currentfill}{rgb}{0.000000,0.000000,0.000000}%
\pgfsetfillcolor{currentfill}%
\pgfsetlinewidth{0.803000pt}%
\definecolor{currentstroke}{rgb}{0.000000,0.000000,0.000000}%
\pgfsetstrokecolor{currentstroke}%
\pgfsetdash{}{0pt}%
\pgfsys@defobject{currentmarker}{\pgfqpoint{-0.048611in}{0.000000in}}{\pgfqpoint{-0.000000in}{0.000000in}}{%
\pgfpathmoveto{\pgfqpoint{-0.000000in}{0.000000in}}%
\pgfpathlineto{\pgfqpoint{-0.048611in}{0.000000in}}%
\pgfusepath{stroke,fill}%
}%
\begin{pgfscope}%
\pgfsys@transformshift{0.380943in}{2.479529in}%
\pgfsys@useobject{currentmarker}{}%
\end{pgfscope}%
\end{pgfscope}%
\begin{pgfscope}%
\definecolor{textcolor}{rgb}{0.000000,0.000000,0.000000}%
\pgfsetstrokecolor{textcolor}%
\pgfsetfillcolor{textcolor}%
\pgftext[x=0.144213in, y=2.440948in, left, base]{\color{textcolor}\rmfamily\fontsize{8.000000}{9.600000}\selectfont F}%
\end{pgfscope}%
\begin{pgfscope}%
\pgfsetbuttcap%
\pgfsetroundjoin%
\definecolor{currentfill}{rgb}{0.000000,0.000000,0.000000}%
\pgfsetfillcolor{currentfill}%
\pgfsetlinewidth{0.803000pt}%
\definecolor{currentstroke}{rgb}{0.000000,0.000000,0.000000}%
\pgfsetstrokecolor{currentstroke}%
\pgfsetdash{}{0pt}%
\pgfsys@defobject{currentmarker}{\pgfqpoint{-0.048611in}{0.000000in}}{\pgfqpoint{-0.000000in}{0.000000in}}{%
\pgfpathmoveto{\pgfqpoint{-0.000000in}{0.000000in}}%
\pgfpathlineto{\pgfqpoint{-0.048611in}{0.000000in}}%
\pgfusepath{stroke,fill}%
}%
\begin{pgfscope}%
\pgfsys@transformshift{0.380943in}{2.391793in}%
\pgfsys@useobject{currentmarker}{}%
\end{pgfscope}%
\end{pgfscope}%
\begin{pgfscope}%
\definecolor{textcolor}{rgb}{0.000000,0.000000,0.000000}%
\pgfsetstrokecolor{textcolor}%
\pgfsetfillcolor{textcolor}%
\pgftext[x=0.155633in, y=2.353212in, left, base]{\color{textcolor}\rmfamily\fontsize{8.000000}{9.600000}\selectfont S}%
\end{pgfscope}%
\begin{pgfscope}%
\pgfsetbuttcap%
\pgfsetroundjoin%
\definecolor{currentfill}{rgb}{0.000000,0.000000,0.000000}%
\pgfsetfillcolor{currentfill}%
\pgfsetlinewidth{0.803000pt}%
\definecolor{currentstroke}{rgb}{0.000000,0.000000,0.000000}%
\pgfsetstrokecolor{currentstroke}%
\pgfsetdash{}{0pt}%
\pgfsys@defobject{currentmarker}{\pgfqpoint{-0.048611in}{0.000000in}}{\pgfqpoint{-0.000000in}{0.000000in}}{%
\pgfpathmoveto{\pgfqpoint{-0.000000in}{0.000000in}}%
\pgfpathlineto{\pgfqpoint{-0.048611in}{0.000000in}}%
\pgfusepath{stroke,fill}%
}%
\begin{pgfscope}%
\pgfsys@transformshift{0.380943in}{2.304057in}%
\pgfsys@useobject{currentmarker}{}%
\end{pgfscope}%
\end{pgfscope}%
\begin{pgfscope}%
\definecolor{textcolor}{rgb}{0.000000,0.000000,0.000000}%
\pgfsetstrokecolor{textcolor}%
\pgfsetfillcolor{textcolor}%
\pgftext[x=0.155633in, y=2.265477in, left, base]{\color{textcolor}\rmfamily\fontsize{8.000000}{9.600000}\selectfont S}%
\end{pgfscope}%
\begin{pgfscope}%
\definecolor{textcolor}{rgb}{0.000000,0.000000,0.000000}%
\pgfsetstrokecolor{textcolor}%
\pgfsetfillcolor{textcolor}%
\pgftext[x=2.705943in,y=3.041007in,,]{\color{textcolor}\ttfamily\fontsize{14.400000}{17.280000}\selectfont 2020}%
\end{pgfscope}%
\begin{pgfscope}%
\pgfpathrectangle{\pgfqpoint{0.380943in}{0.295988in}}{\pgfqpoint{4.650000in}{0.692553in}}%
\pgfusepath{clip}%
\pgfsetbuttcap%
\pgfsetroundjoin%
\pgfsetlinewidth{0.250937pt}%
\definecolor{currentstroke}{rgb}{1.000000,1.000000,1.000000}%
\pgfsetstrokecolor{currentstroke}%
\pgfsetdash{}{0pt}%
\pgfpathmoveto{\pgfqpoint{0.380943in}{0.988541in}}%
\pgfpathlineto{\pgfqpoint{0.479879in}{0.988541in}}%
\pgfpathlineto{\pgfqpoint{0.479879in}{0.889605in}}%
\pgfpathlineto{\pgfqpoint{0.380943in}{0.889605in}}%
\pgfpathlineto{\pgfqpoint{0.380943in}{0.988541in}}%
\pgfusepath{stroke}%
\end{pgfscope}%
\begin{pgfscope}%
\pgfpathrectangle{\pgfqpoint{0.380943in}{0.295988in}}{\pgfqpoint{4.650000in}{0.692553in}}%
\pgfusepath{clip}%
\pgfsetbuttcap%
\pgfsetroundjoin%
\definecolor{currentfill}{rgb}{0.991849,0.986144,0.810181}%
\pgfsetfillcolor{currentfill}%
\pgfsetlinewidth{0.250937pt}%
\definecolor{currentstroke}{rgb}{1.000000,1.000000,1.000000}%
\pgfsetstrokecolor{currentstroke}%
\pgfsetdash{}{0pt}%
\pgfpathmoveto{\pgfqpoint{0.479879in}{0.988541in}}%
\pgfpathlineto{\pgfqpoint{0.578815in}{0.988541in}}%
\pgfpathlineto{\pgfqpoint{0.578815in}{0.889605in}}%
\pgfpathlineto{\pgfqpoint{0.479879in}{0.889605in}}%
\pgfpathlineto{\pgfqpoint{0.479879in}{0.988541in}}%
\pgfusepath{stroke,fill}%
\end{pgfscope}%
\begin{pgfscope}%
\pgfpathrectangle{\pgfqpoint{0.380943in}{0.295988in}}{\pgfqpoint{4.650000in}{0.692553in}}%
\pgfusepath{clip}%
\pgfsetbuttcap%
\pgfsetroundjoin%
\definecolor{currentfill}{rgb}{1.000000,1.000000,0.844829}%
\pgfsetfillcolor{currentfill}%
\pgfsetlinewidth{0.250937pt}%
\definecolor{currentstroke}{rgb}{1.000000,1.000000,1.000000}%
\pgfsetstrokecolor{currentstroke}%
\pgfsetdash{}{0pt}%
\pgfpathmoveto{\pgfqpoint{0.578815in}{0.988541in}}%
\pgfpathlineto{\pgfqpoint{0.677752in}{0.988541in}}%
\pgfpathlineto{\pgfqpoint{0.677752in}{0.889605in}}%
\pgfpathlineto{\pgfqpoint{0.578815in}{0.889605in}}%
\pgfpathlineto{\pgfqpoint{0.578815in}{0.988541in}}%
\pgfusepath{stroke,fill}%
\end{pgfscope}%
\begin{pgfscope}%
\pgfpathrectangle{\pgfqpoint{0.380943in}{0.295988in}}{\pgfqpoint{4.650000in}{0.692553in}}%
\pgfusepath{clip}%
\pgfsetbuttcap%
\pgfsetroundjoin%
\definecolor{currentfill}{rgb}{0.973241,0.954510,0.761799}%
\pgfsetfillcolor{currentfill}%
\pgfsetlinewidth{0.250937pt}%
\definecolor{currentstroke}{rgb}{1.000000,1.000000,1.000000}%
\pgfsetstrokecolor{currentstroke}%
\pgfsetdash{}{0pt}%
\pgfpathmoveto{\pgfqpoint{0.677752in}{0.988541in}}%
\pgfpathlineto{\pgfqpoint{0.776688in}{0.988541in}}%
\pgfpathlineto{\pgfqpoint{0.776688in}{0.889605in}}%
\pgfpathlineto{\pgfqpoint{0.677752in}{0.889605in}}%
\pgfpathlineto{\pgfqpoint{0.677752in}{0.988541in}}%
\pgfusepath{stroke,fill}%
\end{pgfscope}%
\begin{pgfscope}%
\pgfpathrectangle{\pgfqpoint{0.380943in}{0.295988in}}{\pgfqpoint{4.650000in}{0.692553in}}%
\pgfusepath{clip}%
\pgfsetbuttcap%
\pgfsetroundjoin%
\definecolor{currentfill}{rgb}{0.983391,0.971765,0.788189}%
\pgfsetfillcolor{currentfill}%
\pgfsetlinewidth{0.250937pt}%
\definecolor{currentstroke}{rgb}{1.000000,1.000000,1.000000}%
\pgfsetstrokecolor{currentstroke}%
\pgfsetdash{}{0pt}%
\pgfpathmoveto{\pgfqpoint{0.776688in}{0.988541in}}%
\pgfpathlineto{\pgfqpoint{0.875624in}{0.988541in}}%
\pgfpathlineto{\pgfqpoint{0.875624in}{0.889605in}}%
\pgfpathlineto{\pgfqpoint{0.776688in}{0.889605in}}%
\pgfpathlineto{\pgfqpoint{0.776688in}{0.988541in}}%
\pgfusepath{stroke,fill}%
\end{pgfscope}%
\begin{pgfscope}%
\pgfpathrectangle{\pgfqpoint{0.380943in}{0.295988in}}{\pgfqpoint{4.650000in}{0.692553in}}%
\pgfusepath{clip}%
\pgfsetbuttcap%
\pgfsetroundjoin%
\definecolor{currentfill}{rgb}{1.000000,1.000000,0.844829}%
\pgfsetfillcolor{currentfill}%
\pgfsetlinewidth{0.250937pt}%
\definecolor{currentstroke}{rgb}{1.000000,1.000000,1.000000}%
\pgfsetstrokecolor{currentstroke}%
\pgfsetdash{}{0pt}%
\pgfpathmoveto{\pgfqpoint{0.875624in}{0.988541in}}%
\pgfpathlineto{\pgfqpoint{0.974560in}{0.988541in}}%
\pgfpathlineto{\pgfqpoint{0.974560in}{0.889605in}}%
\pgfpathlineto{\pgfqpoint{0.875624in}{0.889605in}}%
\pgfpathlineto{\pgfqpoint{0.875624in}{0.988541in}}%
\pgfusepath{stroke,fill}%
\end{pgfscope}%
\begin{pgfscope}%
\pgfpathrectangle{\pgfqpoint{0.380943in}{0.295988in}}{\pgfqpoint{4.650000in}{0.692553in}}%
\pgfusepath{clip}%
\pgfsetbuttcap%
\pgfsetroundjoin%
\definecolor{currentfill}{rgb}{1.000000,1.000000,0.887120}%
\pgfsetfillcolor{currentfill}%
\pgfsetlinewidth{0.250937pt}%
\definecolor{currentstroke}{rgb}{1.000000,1.000000,1.000000}%
\pgfsetstrokecolor{currentstroke}%
\pgfsetdash{}{0pt}%
\pgfpathmoveto{\pgfqpoint{0.974560in}{0.988541in}}%
\pgfpathlineto{\pgfqpoint{1.073496in}{0.988541in}}%
\pgfpathlineto{\pgfqpoint{1.073496in}{0.889605in}}%
\pgfpathlineto{\pgfqpoint{0.974560in}{0.889605in}}%
\pgfpathlineto{\pgfqpoint{0.974560in}{0.988541in}}%
\pgfusepath{stroke,fill}%
\end{pgfscope}%
\begin{pgfscope}%
\pgfpathrectangle{\pgfqpoint{0.380943in}{0.295988in}}{\pgfqpoint{4.650000in}{0.692553in}}%
\pgfusepath{clip}%
\pgfsetbuttcap%
\pgfsetroundjoin%
\definecolor{currentfill}{rgb}{0.960892,0.932687,0.728981}%
\pgfsetfillcolor{currentfill}%
\pgfsetlinewidth{0.250937pt}%
\definecolor{currentstroke}{rgb}{1.000000,1.000000,1.000000}%
\pgfsetstrokecolor{currentstroke}%
\pgfsetdash{}{0pt}%
\pgfpathmoveto{\pgfqpoint{1.073496in}{0.988541in}}%
\pgfpathlineto{\pgfqpoint{1.172432in}{0.988541in}}%
\pgfpathlineto{\pgfqpoint{1.172432in}{0.889605in}}%
\pgfpathlineto{\pgfqpoint{1.073496in}{0.889605in}}%
\pgfpathlineto{\pgfqpoint{1.073496in}{0.988541in}}%
\pgfusepath{stroke,fill}%
\end{pgfscope}%
\begin{pgfscope}%
\pgfpathrectangle{\pgfqpoint{0.380943in}{0.295988in}}{\pgfqpoint{4.650000in}{0.692553in}}%
\pgfusepath{clip}%
\pgfsetbuttcap%
\pgfsetroundjoin%
\definecolor{currentfill}{rgb}{0.991849,0.986144,0.810181}%
\pgfsetfillcolor{currentfill}%
\pgfsetlinewidth{0.250937pt}%
\definecolor{currentstroke}{rgb}{1.000000,1.000000,1.000000}%
\pgfsetstrokecolor{currentstroke}%
\pgfsetdash{}{0pt}%
\pgfpathmoveto{\pgfqpoint{1.172432in}{0.988541in}}%
\pgfpathlineto{\pgfqpoint{1.271369in}{0.988541in}}%
\pgfpathlineto{\pgfqpoint{1.271369in}{0.889605in}}%
\pgfpathlineto{\pgfqpoint{1.172432in}{0.889605in}}%
\pgfpathlineto{\pgfqpoint{1.172432in}{0.988541in}}%
\pgfusepath{stroke,fill}%
\end{pgfscope}%
\begin{pgfscope}%
\pgfpathrectangle{\pgfqpoint{0.380943in}{0.295988in}}{\pgfqpoint{4.650000in}{0.692553in}}%
\pgfusepath{clip}%
\pgfsetbuttcap%
\pgfsetroundjoin%
\definecolor{currentfill}{rgb}{1.000000,0.564629,0.514479}%
\pgfsetfillcolor{currentfill}%
\pgfsetlinewidth{0.250937pt}%
\definecolor{currentstroke}{rgb}{1.000000,1.000000,1.000000}%
\pgfsetstrokecolor{currentstroke}%
\pgfsetdash{}{0pt}%
\pgfpathmoveto{\pgfqpoint{1.271369in}{0.988541in}}%
\pgfpathlineto{\pgfqpoint{1.370305in}{0.988541in}}%
\pgfpathlineto{\pgfqpoint{1.370305in}{0.889605in}}%
\pgfpathlineto{\pgfqpoint{1.271369in}{0.889605in}}%
\pgfpathlineto{\pgfqpoint{1.271369in}{0.988541in}}%
\pgfusepath{stroke,fill}%
\end{pgfscope}%
\begin{pgfscope}%
\pgfpathrectangle{\pgfqpoint{0.380943in}{0.295988in}}{\pgfqpoint{4.650000in}{0.692553in}}%
\pgfusepath{clip}%
\pgfsetbuttcap%
\pgfsetroundjoin%
\definecolor{currentfill}{rgb}{0.998939,0.658962,0.556032}%
\pgfsetfillcolor{currentfill}%
\pgfsetlinewidth{0.250937pt}%
\definecolor{currentstroke}{rgb}{1.000000,1.000000,1.000000}%
\pgfsetstrokecolor{currentstroke}%
\pgfsetdash{}{0pt}%
\pgfpathmoveto{\pgfqpoint{1.370305in}{0.988541in}}%
\pgfpathlineto{\pgfqpoint{1.469241in}{0.988541in}}%
\pgfpathlineto{\pgfqpoint{1.469241in}{0.889605in}}%
\pgfpathlineto{\pgfqpoint{1.370305in}{0.889605in}}%
\pgfpathlineto{\pgfqpoint{1.370305in}{0.988541in}}%
\pgfusepath{stroke,fill}%
\end{pgfscope}%
\begin{pgfscope}%
\pgfpathrectangle{\pgfqpoint{0.380943in}{0.295988in}}{\pgfqpoint{4.650000in}{0.692553in}}%
\pgfusepath{clip}%
\pgfsetbuttcap%
\pgfsetroundjoin%
\definecolor{currentfill}{rgb}{0.997247,0.702945,0.579715}%
\pgfsetfillcolor{currentfill}%
\pgfsetlinewidth{0.250937pt}%
\definecolor{currentstroke}{rgb}{1.000000,1.000000,1.000000}%
\pgfsetstrokecolor{currentstroke}%
\pgfsetdash{}{0pt}%
\pgfpathmoveto{\pgfqpoint{1.469241in}{0.988541in}}%
\pgfpathlineto{\pgfqpoint{1.568177in}{0.988541in}}%
\pgfpathlineto{\pgfqpoint{1.568177in}{0.889605in}}%
\pgfpathlineto{\pgfqpoint{1.469241in}{0.889605in}}%
\pgfpathlineto{\pgfqpoint{1.469241in}{0.988541in}}%
\pgfusepath{stroke,fill}%
\end{pgfscope}%
\begin{pgfscope}%
\pgfpathrectangle{\pgfqpoint{0.380943in}{0.295988in}}{\pgfqpoint{4.650000in}{0.692553in}}%
\pgfusepath{clip}%
\pgfsetbuttcap%
\pgfsetroundjoin%
\definecolor{currentfill}{rgb}{0.913879,0.392311,0.392311}%
\pgfsetfillcolor{currentfill}%
\pgfsetlinewidth{0.250937pt}%
\definecolor{currentstroke}{rgb}{1.000000,1.000000,1.000000}%
\pgfsetstrokecolor{currentstroke}%
\pgfsetdash{}{0pt}%
\pgfpathmoveto{\pgfqpoint{1.568177in}{0.988541in}}%
\pgfpathlineto{\pgfqpoint{1.667113in}{0.988541in}}%
\pgfpathlineto{\pgfqpoint{1.667113in}{0.889605in}}%
\pgfpathlineto{\pgfqpoint{1.568177in}{0.889605in}}%
\pgfpathlineto{\pgfqpoint{1.568177in}{0.988541in}}%
\pgfusepath{stroke,fill}%
\end{pgfscope}%
\begin{pgfscope}%
\pgfpathrectangle{\pgfqpoint{0.380943in}{0.295988in}}{\pgfqpoint{4.650000in}{0.692553in}}%
\pgfusepath{clip}%
\pgfsetbuttcap%
\pgfsetroundjoin%
\definecolor{currentfill}{rgb}{0.994018,0.750850,0.606382}%
\pgfsetfillcolor{currentfill}%
\pgfsetlinewidth{0.250937pt}%
\definecolor{currentstroke}{rgb}{1.000000,1.000000,1.000000}%
\pgfsetstrokecolor{currentstroke}%
\pgfsetdash{}{0pt}%
\pgfpathmoveto{\pgfqpoint{1.667113in}{0.988541in}}%
\pgfpathlineto{\pgfqpoint{1.766049in}{0.988541in}}%
\pgfpathlineto{\pgfqpoint{1.766049in}{0.889605in}}%
\pgfpathlineto{\pgfqpoint{1.667113in}{0.889605in}}%
\pgfpathlineto{\pgfqpoint{1.667113in}{0.988541in}}%
\pgfusepath{stroke,fill}%
\end{pgfscope}%
\begin{pgfscope}%
\pgfpathrectangle{\pgfqpoint{0.380943in}{0.295988in}}{\pgfqpoint{4.650000in}{0.692553in}}%
\pgfusepath{clip}%
\pgfsetbuttcap%
\pgfsetroundjoin%
\definecolor{currentfill}{rgb}{0.974072,0.862976,0.688750}%
\pgfsetfillcolor{currentfill}%
\pgfsetlinewidth{0.250937pt}%
\definecolor{currentstroke}{rgb}{1.000000,1.000000,1.000000}%
\pgfsetstrokecolor{currentstroke}%
\pgfsetdash{}{0pt}%
\pgfpathmoveto{\pgfqpoint{1.766049in}{0.988541in}}%
\pgfpathlineto{\pgfqpoint{1.864986in}{0.988541in}}%
\pgfpathlineto{\pgfqpoint{1.864986in}{0.889605in}}%
\pgfpathlineto{\pgfqpoint{1.766049in}{0.889605in}}%
\pgfpathlineto{\pgfqpoint{1.766049in}{0.988541in}}%
\pgfusepath{stroke,fill}%
\end{pgfscope}%
\begin{pgfscope}%
\pgfpathrectangle{\pgfqpoint{0.380943in}{0.295988in}}{\pgfqpoint{4.650000in}{0.692553in}}%
\pgfusepath{clip}%
\pgfsetbuttcap%
\pgfsetroundjoin%
\definecolor{currentfill}{rgb}{0.999785,0.636970,0.544191}%
\pgfsetfillcolor{currentfill}%
\pgfsetlinewidth{0.250937pt}%
\definecolor{currentstroke}{rgb}{1.000000,1.000000,1.000000}%
\pgfsetstrokecolor{currentstroke}%
\pgfsetdash{}{0pt}%
\pgfpathmoveto{\pgfqpoint{1.864986in}{0.988541in}}%
\pgfpathlineto{\pgfqpoint{1.963922in}{0.988541in}}%
\pgfpathlineto{\pgfqpoint{1.963922in}{0.889605in}}%
\pgfpathlineto{\pgfqpoint{1.864986in}{0.889605in}}%
\pgfpathlineto{\pgfqpoint{1.864986in}{0.988541in}}%
\pgfusepath{stroke,fill}%
\end{pgfscope}%
\begin{pgfscope}%
\pgfpathrectangle{\pgfqpoint{0.380943in}{0.295988in}}{\pgfqpoint{4.650000in}{0.692553in}}%
\pgfusepath{clip}%
\pgfsetbuttcap%
\pgfsetroundjoin%
\definecolor{currentfill}{rgb}{0.994694,0.745098,0.602999}%
\pgfsetfillcolor{currentfill}%
\pgfsetlinewidth{0.250937pt}%
\definecolor{currentstroke}{rgb}{1.000000,1.000000,1.000000}%
\pgfsetstrokecolor{currentstroke}%
\pgfsetdash{}{0pt}%
\pgfpathmoveto{\pgfqpoint{1.963922in}{0.988541in}}%
\pgfpathlineto{\pgfqpoint{2.062858in}{0.988541in}}%
\pgfpathlineto{\pgfqpoint{2.062858in}{0.889605in}}%
\pgfpathlineto{\pgfqpoint{1.963922in}{0.889605in}}%
\pgfpathlineto{\pgfqpoint{1.963922in}{0.988541in}}%
\pgfusepath{stroke,fill}%
\end{pgfscope}%
\begin{pgfscope}%
\pgfpathrectangle{\pgfqpoint{0.380943in}{0.295988in}}{\pgfqpoint{4.650000in}{0.692553in}}%
\pgfusepath{clip}%
\pgfsetbuttcap%
\pgfsetroundjoin%
\definecolor{currentfill}{rgb}{0.996401,0.724937,0.591557}%
\pgfsetfillcolor{currentfill}%
\pgfsetlinewidth{0.250937pt}%
\definecolor{currentstroke}{rgb}{1.000000,1.000000,1.000000}%
\pgfsetstrokecolor{currentstroke}%
\pgfsetdash{}{0pt}%
\pgfpathmoveto{\pgfqpoint{2.062858in}{0.988541in}}%
\pgfpathlineto{\pgfqpoint{2.161794in}{0.988541in}}%
\pgfpathlineto{\pgfqpoint{2.161794in}{0.889605in}}%
\pgfpathlineto{\pgfqpoint{2.062858in}{0.889605in}}%
\pgfpathlineto{\pgfqpoint{2.062858in}{0.988541in}}%
\pgfusepath{stroke,fill}%
\end{pgfscope}%
\begin{pgfscope}%
\pgfpathrectangle{\pgfqpoint{0.380943in}{0.295988in}}{\pgfqpoint{4.650000in}{0.692553in}}%
\pgfusepath{clip}%
\pgfsetbuttcap%
\pgfsetroundjoin%
\definecolor{currentfill}{rgb}{0.989619,0.788235,0.628374}%
\pgfsetfillcolor{currentfill}%
\pgfsetlinewidth{0.250937pt}%
\definecolor{currentstroke}{rgb}{1.000000,1.000000,1.000000}%
\pgfsetstrokecolor{currentstroke}%
\pgfsetdash{}{0pt}%
\pgfpathmoveto{\pgfqpoint{2.161794in}{0.988541in}}%
\pgfpathlineto{\pgfqpoint{2.260730in}{0.988541in}}%
\pgfpathlineto{\pgfqpoint{2.260730in}{0.889605in}}%
\pgfpathlineto{\pgfqpoint{2.161794in}{0.889605in}}%
\pgfpathlineto{\pgfqpoint{2.161794in}{0.988541in}}%
\pgfusepath{stroke,fill}%
\end{pgfscope}%
\begin{pgfscope}%
\pgfpathrectangle{\pgfqpoint{0.380943in}{0.295988in}}{\pgfqpoint{4.650000in}{0.692553in}}%
\pgfusepath{clip}%
\pgfsetbuttcap%
\pgfsetroundjoin%
\definecolor{currentfill}{rgb}{0.996401,0.724937,0.591557}%
\pgfsetfillcolor{currentfill}%
\pgfsetlinewidth{0.250937pt}%
\definecolor{currentstroke}{rgb}{1.000000,1.000000,1.000000}%
\pgfsetstrokecolor{currentstroke}%
\pgfsetdash{}{0pt}%
\pgfpathmoveto{\pgfqpoint{2.260730in}{0.988541in}}%
\pgfpathlineto{\pgfqpoint{2.359666in}{0.988541in}}%
\pgfpathlineto{\pgfqpoint{2.359666in}{0.889605in}}%
\pgfpathlineto{\pgfqpoint{2.260730in}{0.889605in}}%
\pgfpathlineto{\pgfqpoint{2.260730in}{0.988541in}}%
\pgfusepath{stroke,fill}%
\end{pgfscope}%
\begin{pgfscope}%
\pgfpathrectangle{\pgfqpoint{0.380943in}{0.295988in}}{\pgfqpoint{4.650000in}{0.692553in}}%
\pgfusepath{clip}%
\pgfsetbuttcap%
\pgfsetroundjoin%
\definecolor{currentfill}{rgb}{0.999446,0.645767,0.548927}%
\pgfsetfillcolor{currentfill}%
\pgfsetlinewidth{0.250937pt}%
\definecolor{currentstroke}{rgb}{1.000000,1.000000,1.000000}%
\pgfsetstrokecolor{currentstroke}%
\pgfsetdash{}{0pt}%
\pgfpathmoveto{\pgfqpoint{2.359666in}{0.988541in}}%
\pgfpathlineto{\pgfqpoint{2.458603in}{0.988541in}}%
\pgfpathlineto{\pgfqpoint{2.458603in}{0.889605in}}%
\pgfpathlineto{\pgfqpoint{2.359666in}{0.889605in}}%
\pgfpathlineto{\pgfqpoint{2.359666in}{0.988541in}}%
\pgfusepath{stroke,fill}%
\end{pgfscope}%
\begin{pgfscope}%
\pgfpathrectangle{\pgfqpoint{0.380943in}{0.295988in}}{\pgfqpoint{4.650000in}{0.692553in}}%
\pgfusepath{clip}%
\pgfsetbuttcap%
\pgfsetroundjoin%
\definecolor{currentfill}{rgb}{0.964783,0.940131,0.739808}%
\pgfsetfillcolor{currentfill}%
\pgfsetlinewidth{0.250937pt}%
\definecolor{currentstroke}{rgb}{1.000000,1.000000,1.000000}%
\pgfsetstrokecolor{currentstroke}%
\pgfsetdash{}{0pt}%
\pgfpathmoveto{\pgfqpoint{2.458603in}{0.988541in}}%
\pgfpathlineto{\pgfqpoint{2.557539in}{0.988541in}}%
\pgfpathlineto{\pgfqpoint{2.557539in}{0.889605in}}%
\pgfpathlineto{\pgfqpoint{2.458603in}{0.889605in}}%
\pgfpathlineto{\pgfqpoint{2.458603in}{0.988541in}}%
\pgfusepath{stroke,fill}%
\end{pgfscope}%
\begin{pgfscope}%
\pgfpathrectangle{\pgfqpoint{0.380943in}{0.295988in}}{\pgfqpoint{4.650000in}{0.692553in}}%
\pgfusepath{clip}%
\pgfsetbuttcap%
\pgfsetroundjoin%
\definecolor{currentfill}{rgb}{0.998939,0.658962,0.556032}%
\pgfsetfillcolor{currentfill}%
\pgfsetlinewidth{0.250937pt}%
\definecolor{currentstroke}{rgb}{1.000000,1.000000,1.000000}%
\pgfsetstrokecolor{currentstroke}%
\pgfsetdash{}{0pt}%
\pgfpathmoveto{\pgfqpoint{2.557539in}{0.988541in}}%
\pgfpathlineto{\pgfqpoint{2.656475in}{0.988541in}}%
\pgfpathlineto{\pgfqpoint{2.656475in}{0.889605in}}%
\pgfpathlineto{\pgfqpoint{2.557539in}{0.889605in}}%
\pgfpathlineto{\pgfqpoint{2.557539in}{0.988541in}}%
\pgfusepath{stroke,fill}%
\end{pgfscope}%
\begin{pgfscope}%
\pgfpathrectangle{\pgfqpoint{0.380943in}{0.295988in}}{\pgfqpoint{4.650000in}{0.692553in}}%
\pgfusepath{clip}%
\pgfsetbuttcap%
\pgfsetroundjoin%
\definecolor{currentfill}{rgb}{0.999446,0.645767,0.548927}%
\pgfsetfillcolor{currentfill}%
\pgfsetlinewidth{0.250937pt}%
\definecolor{currentstroke}{rgb}{1.000000,1.000000,1.000000}%
\pgfsetstrokecolor{currentstroke}%
\pgfsetdash{}{0pt}%
\pgfpathmoveto{\pgfqpoint{2.656475in}{0.988541in}}%
\pgfpathlineto{\pgfqpoint{2.755411in}{0.988541in}}%
\pgfpathlineto{\pgfqpoint{2.755411in}{0.889605in}}%
\pgfpathlineto{\pgfqpoint{2.656475in}{0.889605in}}%
\pgfpathlineto{\pgfqpoint{2.656475in}{0.988541in}}%
\pgfusepath{stroke,fill}%
\end{pgfscope}%
\begin{pgfscope}%
\pgfpathrectangle{\pgfqpoint{0.380943in}{0.295988in}}{\pgfqpoint{4.650000in}{0.692553in}}%
\pgfusepath{clip}%
\pgfsetbuttcap%
\pgfsetroundjoin%
\definecolor{currentfill}{rgb}{0.993003,0.759477,0.611457}%
\pgfsetfillcolor{currentfill}%
\pgfsetlinewidth{0.250937pt}%
\definecolor{currentstroke}{rgb}{1.000000,1.000000,1.000000}%
\pgfsetstrokecolor{currentstroke}%
\pgfsetdash{}{0pt}%
\pgfpathmoveto{\pgfqpoint{2.755411in}{0.988541in}}%
\pgfpathlineto{\pgfqpoint{2.854347in}{0.988541in}}%
\pgfpathlineto{\pgfqpoint{2.854347in}{0.889605in}}%
\pgfpathlineto{\pgfqpoint{2.755411in}{0.889605in}}%
\pgfpathlineto{\pgfqpoint{2.755411in}{0.988541in}}%
\pgfusepath{stroke,fill}%
\end{pgfscope}%
\begin{pgfscope}%
\pgfpathrectangle{\pgfqpoint{0.380943in}{0.295988in}}{\pgfqpoint{4.650000in}{0.692553in}}%
\pgfusepath{clip}%
\pgfsetbuttcap%
\pgfsetroundjoin%
\definecolor{currentfill}{rgb}{0.990296,0.782484,0.624990}%
\pgfsetfillcolor{currentfill}%
\pgfsetlinewidth{0.250937pt}%
\definecolor{currentstroke}{rgb}{1.000000,1.000000,1.000000}%
\pgfsetstrokecolor{currentstroke}%
\pgfsetdash{}{0pt}%
\pgfpathmoveto{\pgfqpoint{2.854347in}{0.988541in}}%
\pgfpathlineto{\pgfqpoint{2.953283in}{0.988541in}}%
\pgfpathlineto{\pgfqpoint{2.953283in}{0.889605in}}%
\pgfpathlineto{\pgfqpoint{2.854347in}{0.889605in}}%
\pgfpathlineto{\pgfqpoint{2.854347in}{0.988541in}}%
\pgfusepath{stroke,fill}%
\end{pgfscope}%
\begin{pgfscope}%
\pgfpathrectangle{\pgfqpoint{0.380943in}{0.295988in}}{\pgfqpoint{4.650000in}{0.692553in}}%
\pgfusepath{clip}%
\pgfsetbuttcap%
\pgfsetroundjoin%
\definecolor{currentfill}{rgb}{0.980669,0.832787,0.665559}%
\pgfsetfillcolor{currentfill}%
\pgfsetlinewidth{0.250937pt}%
\definecolor{currentstroke}{rgb}{1.000000,1.000000,1.000000}%
\pgfsetstrokecolor{currentstroke}%
\pgfsetdash{}{0pt}%
\pgfpathmoveto{\pgfqpoint{2.953283in}{0.988541in}}%
\pgfpathlineto{\pgfqpoint{3.052220in}{0.988541in}}%
\pgfpathlineto{\pgfqpoint{3.052220in}{0.889605in}}%
\pgfpathlineto{\pgfqpoint{2.953283in}{0.889605in}}%
\pgfpathlineto{\pgfqpoint{2.953283in}{0.988541in}}%
\pgfusepath{stroke,fill}%
\end{pgfscope}%
\begin{pgfscope}%
\pgfpathrectangle{\pgfqpoint{0.380943in}{0.295988in}}{\pgfqpoint{4.650000in}{0.692553in}}%
\pgfusepath{clip}%
\pgfsetbuttcap%
\pgfsetroundjoin%
\definecolor{currentfill}{rgb}{0.991311,0.773856,0.619915}%
\pgfsetfillcolor{currentfill}%
\pgfsetlinewidth{0.250937pt}%
\definecolor{currentstroke}{rgb}{1.000000,1.000000,1.000000}%
\pgfsetstrokecolor{currentstroke}%
\pgfsetdash{}{0pt}%
\pgfpathmoveto{\pgfqpoint{3.052220in}{0.988541in}}%
\pgfpathlineto{\pgfqpoint{3.151156in}{0.988541in}}%
\pgfpathlineto{\pgfqpoint{3.151156in}{0.889605in}}%
\pgfpathlineto{\pgfqpoint{3.052220in}{0.889605in}}%
\pgfpathlineto{\pgfqpoint{3.052220in}{0.988541in}}%
\pgfusepath{stroke,fill}%
\end{pgfscope}%
\begin{pgfscope}%
\pgfpathrectangle{\pgfqpoint{0.380943in}{0.295988in}}{\pgfqpoint{4.650000in}{0.692553in}}%
\pgfusepath{clip}%
\pgfsetbuttcap%
\pgfsetroundjoin%
\definecolor{currentfill}{rgb}{0.988604,0.796863,0.633449}%
\pgfsetfillcolor{currentfill}%
\pgfsetlinewidth{0.250937pt}%
\definecolor{currentstroke}{rgb}{1.000000,1.000000,1.000000}%
\pgfsetstrokecolor{currentstroke}%
\pgfsetdash{}{0pt}%
\pgfpathmoveto{\pgfqpoint{3.151156in}{0.988541in}}%
\pgfpathlineto{\pgfqpoint{3.250092in}{0.988541in}}%
\pgfpathlineto{\pgfqpoint{3.250092in}{0.889605in}}%
\pgfpathlineto{\pgfqpoint{3.151156in}{0.889605in}}%
\pgfpathlineto{\pgfqpoint{3.151156in}{0.988541in}}%
\pgfusepath{stroke,fill}%
\end{pgfscope}%
\begin{pgfscope}%
\pgfpathrectangle{\pgfqpoint{0.380943in}{0.295988in}}{\pgfqpoint{4.650000in}{0.692553in}}%
\pgfusepath{clip}%
\pgfsetbuttcap%
\pgfsetroundjoin%
\definecolor{currentfill}{rgb}{0.990296,0.782484,0.624990}%
\pgfsetfillcolor{currentfill}%
\pgfsetlinewidth{0.250937pt}%
\definecolor{currentstroke}{rgb}{1.000000,1.000000,1.000000}%
\pgfsetstrokecolor{currentstroke}%
\pgfsetdash{}{0pt}%
\pgfpathmoveto{\pgfqpoint{3.250092in}{0.988541in}}%
\pgfpathlineto{\pgfqpoint{3.349028in}{0.988541in}}%
\pgfpathlineto{\pgfqpoint{3.349028in}{0.889605in}}%
\pgfpathlineto{\pgfqpoint{3.250092in}{0.889605in}}%
\pgfpathlineto{\pgfqpoint{3.250092in}{0.988541in}}%
\pgfusepath{stroke,fill}%
\end{pgfscope}%
\begin{pgfscope}%
\pgfpathrectangle{\pgfqpoint{0.380943in}{0.295988in}}{\pgfqpoint{4.650000in}{0.692553in}}%
\pgfusepath{clip}%
\pgfsetbuttcap%
\pgfsetroundjoin%
\definecolor{currentfill}{rgb}{0.887966,0.366398,0.366398}%
\pgfsetfillcolor{currentfill}%
\pgfsetlinewidth{0.250937pt}%
\definecolor{currentstroke}{rgb}{1.000000,1.000000,1.000000}%
\pgfsetstrokecolor{currentstroke}%
\pgfsetdash{}{0pt}%
\pgfpathmoveto{\pgfqpoint{3.349028in}{0.988541in}}%
\pgfpathlineto{\pgfqpoint{3.447964in}{0.988541in}}%
\pgfpathlineto{\pgfqpoint{3.447964in}{0.889605in}}%
\pgfpathlineto{\pgfqpoint{3.349028in}{0.889605in}}%
\pgfpathlineto{\pgfqpoint{3.349028in}{0.988541in}}%
\pgfusepath{stroke,fill}%
\end{pgfscope}%
\begin{pgfscope}%
\pgfpathrectangle{\pgfqpoint{0.380943in}{0.295988in}}{\pgfqpoint{4.650000in}{0.692553in}}%
\pgfusepath{clip}%
\pgfsetbuttcap%
\pgfsetroundjoin%
\definecolor{currentfill}{rgb}{1.000000,0.625529,0.538839}%
\pgfsetfillcolor{currentfill}%
\pgfsetlinewidth{0.250937pt}%
\definecolor{currentstroke}{rgb}{1.000000,1.000000,1.000000}%
\pgfsetstrokecolor{currentstroke}%
\pgfsetdash{}{0pt}%
\pgfpathmoveto{\pgfqpoint{3.447964in}{0.988541in}}%
\pgfpathlineto{\pgfqpoint{3.546901in}{0.988541in}}%
\pgfpathlineto{\pgfqpoint{3.546901in}{0.889605in}}%
\pgfpathlineto{\pgfqpoint{3.447964in}{0.889605in}}%
\pgfpathlineto{\pgfqpoint{3.447964in}{0.988541in}}%
\pgfusepath{stroke,fill}%
\end{pgfscope}%
\begin{pgfscope}%
\pgfpathrectangle{\pgfqpoint{0.380943in}{0.295988in}}{\pgfqpoint{4.650000in}{0.692553in}}%
\pgfusepath{clip}%
\pgfsetbuttcap%
\pgfsetroundjoin%
\definecolor{currentfill}{rgb}{0.998093,0.680953,0.567874}%
\pgfsetfillcolor{currentfill}%
\pgfsetlinewidth{0.250937pt}%
\definecolor{currentstroke}{rgb}{1.000000,1.000000,1.000000}%
\pgfsetstrokecolor{currentstroke}%
\pgfsetdash{}{0pt}%
\pgfpathmoveto{\pgfqpoint{3.546901in}{0.988541in}}%
\pgfpathlineto{\pgfqpoint{3.645837in}{0.988541in}}%
\pgfpathlineto{\pgfqpoint{3.645837in}{0.889605in}}%
\pgfpathlineto{\pgfqpoint{3.546901in}{0.889605in}}%
\pgfpathlineto{\pgfqpoint{3.546901in}{0.988541in}}%
\pgfusepath{stroke,fill}%
\end{pgfscope}%
\begin{pgfscope}%
\pgfpathrectangle{\pgfqpoint{0.380943in}{0.295988in}}{\pgfqpoint{4.650000in}{0.692553in}}%
\pgfusepath{clip}%
\pgfsetbuttcap%
\pgfsetroundjoin%
\definecolor{currentfill}{rgb}{0.998939,0.658962,0.556032}%
\pgfsetfillcolor{currentfill}%
\pgfsetlinewidth{0.250937pt}%
\definecolor{currentstroke}{rgb}{1.000000,1.000000,1.000000}%
\pgfsetstrokecolor{currentstroke}%
\pgfsetdash{}{0pt}%
\pgfpathmoveto{\pgfqpoint{3.645837in}{0.988541in}}%
\pgfpathlineto{\pgfqpoint{3.744773in}{0.988541in}}%
\pgfpathlineto{\pgfqpoint{3.744773in}{0.889605in}}%
\pgfpathlineto{\pgfqpoint{3.645837in}{0.889605in}}%
\pgfpathlineto{\pgfqpoint{3.645837in}{0.988541in}}%
\pgfusepath{stroke,fill}%
\end{pgfscope}%
\begin{pgfscope}%
\pgfpathrectangle{\pgfqpoint{0.380943in}{0.295988in}}{\pgfqpoint{4.650000in}{0.692553in}}%
\pgfusepath{clip}%
\pgfsetbuttcap%
\pgfsetroundjoin%
\definecolor{currentfill}{rgb}{0.997247,0.702945,0.579715}%
\pgfsetfillcolor{currentfill}%
\pgfsetlinewidth{0.250937pt}%
\definecolor{currentstroke}{rgb}{1.000000,1.000000,1.000000}%
\pgfsetstrokecolor{currentstroke}%
\pgfsetdash{}{0pt}%
\pgfpathmoveto{\pgfqpoint{3.744773in}{0.988541in}}%
\pgfpathlineto{\pgfqpoint{3.843709in}{0.988541in}}%
\pgfpathlineto{\pgfqpoint{3.843709in}{0.889605in}}%
\pgfpathlineto{\pgfqpoint{3.744773in}{0.889605in}}%
\pgfpathlineto{\pgfqpoint{3.744773in}{0.988541in}}%
\pgfusepath{stroke,fill}%
\end{pgfscope}%
\begin{pgfscope}%
\pgfpathrectangle{\pgfqpoint{0.380943in}{0.295988in}}{\pgfqpoint{4.650000in}{0.692553in}}%
\pgfusepath{clip}%
\pgfsetbuttcap%
\pgfsetroundjoin%
\definecolor{currentfill}{rgb}{1.000000,0.538331,0.503652}%
\pgfsetfillcolor{currentfill}%
\pgfsetlinewidth{0.250937pt}%
\definecolor{currentstroke}{rgb}{1.000000,1.000000,1.000000}%
\pgfsetstrokecolor{currentstroke}%
\pgfsetdash{}{0pt}%
\pgfpathmoveto{\pgfqpoint{3.843709in}{0.988541in}}%
\pgfpathlineto{\pgfqpoint{3.942645in}{0.988541in}}%
\pgfpathlineto{\pgfqpoint{3.942645in}{0.889605in}}%
\pgfpathlineto{\pgfqpoint{3.843709in}{0.889605in}}%
\pgfpathlineto{\pgfqpoint{3.843709in}{0.988541in}}%
\pgfusepath{stroke,fill}%
\end{pgfscope}%
\begin{pgfscope}%
\pgfpathrectangle{\pgfqpoint{0.380943in}{0.295988in}}{\pgfqpoint{4.650000in}{0.692553in}}%
\pgfusepath{clip}%
\pgfsetbuttcap%
\pgfsetroundjoin%
\definecolor{currentfill}{rgb}{0.998939,0.658962,0.556032}%
\pgfsetfillcolor{currentfill}%
\pgfsetlinewidth{0.250937pt}%
\definecolor{currentstroke}{rgb}{1.000000,1.000000,1.000000}%
\pgfsetstrokecolor{currentstroke}%
\pgfsetdash{}{0pt}%
\pgfpathmoveto{\pgfqpoint{3.942645in}{0.988541in}}%
\pgfpathlineto{\pgfqpoint{4.041581in}{0.988541in}}%
\pgfpathlineto{\pgfqpoint{4.041581in}{0.889605in}}%
\pgfpathlineto{\pgfqpoint{3.942645in}{0.889605in}}%
\pgfpathlineto{\pgfqpoint{3.942645in}{0.988541in}}%
\pgfusepath{stroke,fill}%
\end{pgfscope}%
\begin{pgfscope}%
\pgfpathrectangle{\pgfqpoint{0.380943in}{0.295988in}}{\pgfqpoint{4.650000in}{0.692553in}}%
\pgfusepath{clip}%
\pgfsetbuttcap%
\pgfsetroundjoin%
\definecolor{currentfill}{rgb}{1.000000,0.564629,0.514479}%
\pgfsetfillcolor{currentfill}%
\pgfsetlinewidth{0.250937pt}%
\definecolor{currentstroke}{rgb}{1.000000,1.000000,1.000000}%
\pgfsetstrokecolor{currentstroke}%
\pgfsetdash{}{0pt}%
\pgfpathmoveto{\pgfqpoint{4.041581in}{0.988541in}}%
\pgfpathlineto{\pgfqpoint{4.140518in}{0.988541in}}%
\pgfpathlineto{\pgfqpoint{4.140518in}{0.889605in}}%
\pgfpathlineto{\pgfqpoint{4.041581in}{0.889605in}}%
\pgfpathlineto{\pgfqpoint{4.041581in}{0.988541in}}%
\pgfusepath{stroke,fill}%
\end{pgfscope}%
\begin{pgfscope}%
\pgfpathrectangle{\pgfqpoint{0.380943in}{0.295988in}}{\pgfqpoint{4.650000in}{0.692553in}}%
\pgfusepath{clip}%
\pgfsetbuttcap%
\pgfsetroundjoin%
\definecolor{currentfill}{rgb}{0.996401,0.724937,0.591557}%
\pgfsetfillcolor{currentfill}%
\pgfsetlinewidth{0.250937pt}%
\definecolor{currentstroke}{rgb}{1.000000,1.000000,1.000000}%
\pgfsetstrokecolor{currentstroke}%
\pgfsetdash{}{0pt}%
\pgfpathmoveto{\pgfqpoint{4.140518in}{0.988541in}}%
\pgfpathlineto{\pgfqpoint{4.239454in}{0.988541in}}%
\pgfpathlineto{\pgfqpoint{4.239454in}{0.889605in}}%
\pgfpathlineto{\pgfqpoint{4.140518in}{0.889605in}}%
\pgfpathlineto{\pgfqpoint{4.140518in}{0.988541in}}%
\pgfusepath{stroke,fill}%
\end{pgfscope}%
\begin{pgfscope}%
\pgfpathrectangle{\pgfqpoint{0.380943in}{0.295988in}}{\pgfqpoint{4.650000in}{0.692553in}}%
\pgfusepath{clip}%
\pgfsetbuttcap%
\pgfsetroundjoin%
\definecolor{currentfill}{rgb}{0.998093,0.680953,0.567874}%
\pgfsetfillcolor{currentfill}%
\pgfsetlinewidth{0.250937pt}%
\definecolor{currentstroke}{rgb}{1.000000,1.000000,1.000000}%
\pgfsetstrokecolor{currentstroke}%
\pgfsetdash{}{0pt}%
\pgfpathmoveto{\pgfqpoint{4.239454in}{0.988541in}}%
\pgfpathlineto{\pgfqpoint{4.338390in}{0.988541in}}%
\pgfpathlineto{\pgfqpoint{4.338390in}{0.889605in}}%
\pgfpathlineto{\pgfqpoint{4.239454in}{0.889605in}}%
\pgfpathlineto{\pgfqpoint{4.239454in}{0.988541in}}%
\pgfusepath{stroke,fill}%
\end{pgfscope}%
\begin{pgfscope}%
\pgfpathrectangle{\pgfqpoint{0.380943in}{0.295988in}}{\pgfqpoint{4.650000in}{0.692553in}}%
\pgfusepath{clip}%
\pgfsetbuttcap%
\pgfsetroundjoin%
\definecolor{currentfill}{rgb}{0.964275,0.912388,0.715448}%
\pgfsetfillcolor{currentfill}%
\pgfsetlinewidth{0.250937pt}%
\definecolor{currentstroke}{rgb}{1.000000,1.000000,1.000000}%
\pgfsetstrokecolor{currentstroke}%
\pgfsetdash{}{0pt}%
\pgfpathmoveto{\pgfqpoint{4.338390in}{0.988541in}}%
\pgfpathlineto{\pgfqpoint{4.437326in}{0.988541in}}%
\pgfpathlineto{\pgfqpoint{4.437326in}{0.889605in}}%
\pgfpathlineto{\pgfqpoint{4.338390in}{0.889605in}}%
\pgfpathlineto{\pgfqpoint{4.338390in}{0.988541in}}%
\pgfusepath{stroke,fill}%
\end{pgfscope}%
\begin{pgfscope}%
\pgfpathrectangle{\pgfqpoint{0.380943in}{0.295988in}}{\pgfqpoint{4.650000in}{0.692553in}}%
\pgfusepath{clip}%
\pgfsetbuttcap%
\pgfsetroundjoin%
\definecolor{currentfill}{rgb}{0.966459,0.901038,0.709050}%
\pgfsetfillcolor{currentfill}%
\pgfsetlinewidth{0.250937pt}%
\definecolor{currentstroke}{rgb}{1.000000,1.000000,1.000000}%
\pgfsetstrokecolor{currentstroke}%
\pgfsetdash{}{0pt}%
\pgfpathmoveto{\pgfqpoint{4.437326in}{0.988541in}}%
\pgfpathlineto{\pgfqpoint{4.536262in}{0.988541in}}%
\pgfpathlineto{\pgfqpoint{4.536262in}{0.889605in}}%
\pgfpathlineto{\pgfqpoint{4.437326in}{0.889605in}}%
\pgfpathlineto{\pgfqpoint{4.437326in}{0.988541in}}%
\pgfusepath{stroke,fill}%
\end{pgfscope}%
\begin{pgfscope}%
\pgfpathrectangle{\pgfqpoint{0.380943in}{0.295988in}}{\pgfqpoint{4.650000in}{0.692553in}}%
\pgfusepath{clip}%
\pgfsetbuttcap%
\pgfsetroundjoin%
\definecolor{currentfill}{rgb}{0.978131,0.843783,0.675709}%
\pgfsetfillcolor{currentfill}%
\pgfsetlinewidth{0.250937pt}%
\definecolor{currentstroke}{rgb}{1.000000,1.000000,1.000000}%
\pgfsetstrokecolor{currentstroke}%
\pgfsetdash{}{0pt}%
\pgfpathmoveto{\pgfqpoint{4.536262in}{0.988541in}}%
\pgfpathlineto{\pgfqpoint{4.635198in}{0.988541in}}%
\pgfpathlineto{\pgfqpoint{4.635198in}{0.889605in}}%
\pgfpathlineto{\pgfqpoint{4.536262in}{0.889605in}}%
\pgfpathlineto{\pgfqpoint{4.536262in}{0.988541in}}%
\pgfusepath{stroke,fill}%
\end{pgfscope}%
\begin{pgfscope}%
\pgfpathrectangle{\pgfqpoint{0.380943in}{0.295988in}}{\pgfqpoint{4.650000in}{0.692553in}}%
\pgfusepath{clip}%
\pgfsetbuttcap%
\pgfsetroundjoin%
\definecolor{currentfill}{rgb}{0.961738,0.927612,0.725598}%
\pgfsetfillcolor{currentfill}%
\pgfsetlinewidth{0.250937pt}%
\definecolor{currentstroke}{rgb}{1.000000,1.000000,1.000000}%
\pgfsetstrokecolor{currentstroke}%
\pgfsetdash{}{0pt}%
\pgfpathmoveto{\pgfqpoint{4.635198in}{0.988541in}}%
\pgfpathlineto{\pgfqpoint{4.734135in}{0.988541in}}%
\pgfpathlineto{\pgfqpoint{4.734135in}{0.889605in}}%
\pgfpathlineto{\pgfqpoint{4.635198in}{0.889605in}}%
\pgfpathlineto{\pgfqpoint{4.635198in}{0.988541in}}%
\pgfusepath{stroke,fill}%
\end{pgfscope}%
\begin{pgfscope}%
\pgfpathrectangle{\pgfqpoint{0.380943in}{0.295988in}}{\pgfqpoint{4.650000in}{0.692553in}}%
\pgfusepath{clip}%
\pgfsetbuttcap%
\pgfsetroundjoin%
\definecolor{currentfill}{rgb}{1.000000,1.000000,0.899808}%
\pgfsetfillcolor{currentfill}%
\pgfsetlinewidth{0.250937pt}%
\definecolor{currentstroke}{rgb}{1.000000,1.000000,1.000000}%
\pgfsetstrokecolor{currentstroke}%
\pgfsetdash{}{0pt}%
\pgfpathmoveto{\pgfqpoint{4.734135in}{0.988541in}}%
\pgfpathlineto{\pgfqpoint{4.833071in}{0.988541in}}%
\pgfpathlineto{\pgfqpoint{4.833071in}{0.889605in}}%
\pgfpathlineto{\pgfqpoint{4.734135in}{0.889605in}}%
\pgfpathlineto{\pgfqpoint{4.734135in}{0.988541in}}%
\pgfusepath{stroke,fill}%
\end{pgfscope}%
\begin{pgfscope}%
\pgfpathrectangle{\pgfqpoint{0.380943in}{0.295988in}}{\pgfqpoint{4.650000in}{0.692553in}}%
\pgfusepath{clip}%
\pgfsetbuttcap%
\pgfsetroundjoin%
\definecolor{currentfill}{rgb}{0.963429,0.917463,0.718831}%
\pgfsetfillcolor{currentfill}%
\pgfsetlinewidth{0.250937pt}%
\definecolor{currentstroke}{rgb}{1.000000,1.000000,1.000000}%
\pgfsetstrokecolor{currentstroke}%
\pgfsetdash{}{0pt}%
\pgfpathmoveto{\pgfqpoint{4.833071in}{0.988541in}}%
\pgfpathlineto{\pgfqpoint{4.932007in}{0.988541in}}%
\pgfpathlineto{\pgfqpoint{4.932007in}{0.889605in}}%
\pgfpathlineto{\pgfqpoint{4.833071in}{0.889605in}}%
\pgfpathlineto{\pgfqpoint{4.833071in}{0.988541in}}%
\pgfusepath{stroke,fill}%
\end{pgfscope}%
\begin{pgfscope}%
\pgfpathrectangle{\pgfqpoint{0.380943in}{0.295988in}}{\pgfqpoint{4.650000in}{0.692553in}}%
\pgfusepath{clip}%
\pgfsetbuttcap%
\pgfsetroundjoin%
\definecolor{currentfill}{rgb}{0.974072,0.862976,0.688750}%
\pgfsetfillcolor{currentfill}%
\pgfsetlinewidth{0.250937pt}%
\definecolor{currentstroke}{rgb}{1.000000,1.000000,1.000000}%
\pgfsetstrokecolor{currentstroke}%
\pgfsetdash{}{0pt}%
\pgfpathmoveto{\pgfqpoint{4.932007in}{0.988541in}}%
\pgfpathlineto{\pgfqpoint{5.030943in}{0.988541in}}%
\pgfpathlineto{\pgfqpoint{5.030943in}{0.889605in}}%
\pgfpathlineto{\pgfqpoint{4.932007in}{0.889605in}}%
\pgfpathlineto{\pgfqpoint{4.932007in}{0.988541in}}%
\pgfusepath{stroke,fill}%
\end{pgfscope}%
\begin{pgfscope}%
\pgfpathrectangle{\pgfqpoint{0.380943in}{0.295988in}}{\pgfqpoint{4.650000in}{0.692553in}}%
\pgfusepath{clip}%
\pgfsetbuttcap%
\pgfsetroundjoin%
\pgfsetlinewidth{0.250937pt}%
\definecolor{currentstroke}{rgb}{1.000000,1.000000,1.000000}%
\pgfsetstrokecolor{currentstroke}%
\pgfsetdash{}{0pt}%
\pgfpathmoveto{\pgfqpoint{0.380943in}{0.889605in}}%
\pgfpathlineto{\pgfqpoint{0.479879in}{0.889605in}}%
\pgfpathlineto{\pgfqpoint{0.479879in}{0.790669in}}%
\pgfpathlineto{\pgfqpoint{0.380943in}{0.790669in}}%
\pgfpathlineto{\pgfqpoint{0.380943in}{0.889605in}}%
\pgfusepath{stroke}%
\end{pgfscope}%
\begin{pgfscope}%
\pgfpathrectangle{\pgfqpoint{0.380943in}{0.295988in}}{\pgfqpoint{4.650000in}{0.692553in}}%
\pgfusepath{clip}%
\pgfsetbuttcap%
\pgfsetroundjoin%
\definecolor{currentfill}{rgb}{0.986774,0.977516,0.796986}%
\pgfsetfillcolor{currentfill}%
\pgfsetlinewidth{0.250937pt}%
\definecolor{currentstroke}{rgb}{1.000000,1.000000,1.000000}%
\pgfsetstrokecolor{currentstroke}%
\pgfsetdash{}{0pt}%
\pgfpathmoveto{\pgfqpoint{0.479879in}{0.889605in}}%
\pgfpathlineto{\pgfqpoint{0.578815in}{0.889605in}}%
\pgfpathlineto{\pgfqpoint{0.578815in}{0.790669in}}%
\pgfpathlineto{\pgfqpoint{0.479879in}{0.790669in}}%
\pgfpathlineto{\pgfqpoint{0.479879in}{0.889605in}}%
\pgfusepath{stroke,fill}%
\end{pgfscope}%
\begin{pgfscope}%
\pgfpathrectangle{\pgfqpoint{0.380943in}{0.295988in}}{\pgfqpoint{4.650000in}{0.692553in}}%
\pgfusepath{clip}%
\pgfsetbuttcap%
\pgfsetroundjoin%
\definecolor{currentfill}{rgb}{0.964275,0.912388,0.715448}%
\pgfsetfillcolor{currentfill}%
\pgfsetlinewidth{0.250937pt}%
\definecolor{currentstroke}{rgb}{1.000000,1.000000,1.000000}%
\pgfsetstrokecolor{currentstroke}%
\pgfsetdash{}{0pt}%
\pgfpathmoveto{\pgfqpoint{0.578815in}{0.889605in}}%
\pgfpathlineto{\pgfqpoint{0.677752in}{0.889605in}}%
\pgfpathlineto{\pgfqpoint{0.677752in}{0.790669in}}%
\pgfpathlineto{\pgfqpoint{0.578815in}{0.790669in}}%
\pgfpathlineto{\pgfqpoint{0.578815in}{0.889605in}}%
\pgfusepath{stroke,fill}%
\end{pgfscope}%
\begin{pgfscope}%
\pgfpathrectangle{\pgfqpoint{0.380943in}{0.295988in}}{\pgfqpoint{4.650000in}{0.692553in}}%
\pgfusepath{clip}%
\pgfsetbuttcap%
\pgfsetroundjoin%
\definecolor{currentfill}{rgb}{0.991849,0.986144,0.810181}%
\pgfsetfillcolor{currentfill}%
\pgfsetlinewidth{0.250937pt}%
\definecolor{currentstroke}{rgb}{1.000000,1.000000,1.000000}%
\pgfsetstrokecolor{currentstroke}%
\pgfsetdash{}{0pt}%
\pgfpathmoveto{\pgfqpoint{0.677752in}{0.889605in}}%
\pgfpathlineto{\pgfqpoint{0.776688in}{0.889605in}}%
\pgfpathlineto{\pgfqpoint{0.776688in}{0.790669in}}%
\pgfpathlineto{\pgfqpoint{0.677752in}{0.790669in}}%
\pgfpathlineto{\pgfqpoint{0.677752in}{0.889605in}}%
\pgfusepath{stroke,fill}%
\end{pgfscope}%
\begin{pgfscope}%
\pgfpathrectangle{\pgfqpoint{0.380943in}{0.295988in}}{\pgfqpoint{4.650000in}{0.692553in}}%
\pgfusepath{clip}%
\pgfsetbuttcap%
\pgfsetroundjoin%
\definecolor{currentfill}{rgb}{0.963937,0.914418,0.716801}%
\pgfsetfillcolor{currentfill}%
\pgfsetlinewidth{0.250937pt}%
\definecolor{currentstroke}{rgb}{1.000000,1.000000,1.000000}%
\pgfsetstrokecolor{currentstroke}%
\pgfsetdash{}{0pt}%
\pgfpathmoveto{\pgfqpoint{0.776688in}{0.889605in}}%
\pgfpathlineto{\pgfqpoint{0.875624in}{0.889605in}}%
\pgfpathlineto{\pgfqpoint{0.875624in}{0.790669in}}%
\pgfpathlineto{\pgfqpoint{0.776688in}{0.790669in}}%
\pgfpathlineto{\pgfqpoint{0.776688in}{0.889605in}}%
\pgfusepath{stroke,fill}%
\end{pgfscope}%
\begin{pgfscope}%
\pgfpathrectangle{\pgfqpoint{0.380943in}{0.295988in}}{\pgfqpoint{4.650000in}{0.692553in}}%
\pgfusepath{clip}%
\pgfsetbuttcap%
\pgfsetroundjoin%
\definecolor{currentfill}{rgb}{1.000000,1.000000,0.887120}%
\pgfsetfillcolor{currentfill}%
\pgfsetlinewidth{0.250937pt}%
\definecolor{currentstroke}{rgb}{1.000000,1.000000,1.000000}%
\pgfsetstrokecolor{currentstroke}%
\pgfsetdash{}{0pt}%
\pgfpathmoveto{\pgfqpoint{0.875624in}{0.889605in}}%
\pgfpathlineto{\pgfqpoint{0.974560in}{0.889605in}}%
\pgfpathlineto{\pgfqpoint{0.974560in}{0.790669in}}%
\pgfpathlineto{\pgfqpoint{0.875624in}{0.790669in}}%
\pgfpathlineto{\pgfqpoint{0.875624in}{0.889605in}}%
\pgfusepath{stroke,fill}%
\end{pgfscope}%
\begin{pgfscope}%
\pgfpathrectangle{\pgfqpoint{0.380943in}{0.295988in}}{\pgfqpoint{4.650000in}{0.692553in}}%
\pgfusepath{clip}%
\pgfsetbuttcap%
\pgfsetroundjoin%
\definecolor{currentfill}{rgb}{0.960892,0.932687,0.728981}%
\pgfsetfillcolor{currentfill}%
\pgfsetlinewidth{0.250937pt}%
\definecolor{currentstroke}{rgb}{1.000000,1.000000,1.000000}%
\pgfsetstrokecolor{currentstroke}%
\pgfsetdash{}{0pt}%
\pgfpathmoveto{\pgfqpoint{0.974560in}{0.889605in}}%
\pgfpathlineto{\pgfqpoint{1.073496in}{0.889605in}}%
\pgfpathlineto{\pgfqpoint{1.073496in}{0.790669in}}%
\pgfpathlineto{\pgfqpoint{0.974560in}{0.790669in}}%
\pgfpathlineto{\pgfqpoint{0.974560in}{0.889605in}}%
\pgfusepath{stroke,fill}%
\end{pgfscope}%
\begin{pgfscope}%
\pgfpathrectangle{\pgfqpoint{0.380943in}{0.295988in}}{\pgfqpoint{4.650000in}{0.692553in}}%
\pgfusepath{clip}%
\pgfsetbuttcap%
\pgfsetroundjoin%
\definecolor{currentfill}{rgb}{0.975594,0.855363,0.684691}%
\pgfsetfillcolor{currentfill}%
\pgfsetlinewidth{0.250937pt}%
\definecolor{currentstroke}{rgb}{1.000000,1.000000,1.000000}%
\pgfsetstrokecolor{currentstroke}%
\pgfsetdash{}{0pt}%
\pgfpathmoveto{\pgfqpoint{1.073496in}{0.889605in}}%
\pgfpathlineto{\pgfqpoint{1.172432in}{0.889605in}}%
\pgfpathlineto{\pgfqpoint{1.172432in}{0.790669in}}%
\pgfpathlineto{\pgfqpoint{1.073496in}{0.790669in}}%
\pgfpathlineto{\pgfqpoint{1.073496in}{0.889605in}}%
\pgfusepath{stroke,fill}%
\end{pgfscope}%
\begin{pgfscope}%
\pgfpathrectangle{\pgfqpoint{0.380943in}{0.295988in}}{\pgfqpoint{4.650000in}{0.692553in}}%
\pgfusepath{clip}%
\pgfsetbuttcap%
\pgfsetroundjoin%
\definecolor{currentfill}{rgb}{0.978316,0.963137,0.774994}%
\pgfsetfillcolor{currentfill}%
\pgfsetlinewidth{0.250937pt}%
\definecolor{currentstroke}{rgb}{1.000000,1.000000,1.000000}%
\pgfsetstrokecolor{currentstroke}%
\pgfsetdash{}{0pt}%
\pgfpathmoveto{\pgfqpoint{1.172432in}{0.889605in}}%
\pgfpathlineto{\pgfqpoint{1.271369in}{0.889605in}}%
\pgfpathlineto{\pgfqpoint{1.271369in}{0.790669in}}%
\pgfpathlineto{\pgfqpoint{1.172432in}{0.790669in}}%
\pgfpathlineto{\pgfqpoint{1.172432in}{0.889605in}}%
\pgfusepath{stroke,fill}%
\end{pgfscope}%
\begin{pgfscope}%
\pgfpathrectangle{\pgfqpoint{0.380943in}{0.295988in}}{\pgfqpoint{4.650000in}{0.692553in}}%
\pgfusepath{clip}%
\pgfsetbuttcap%
\pgfsetroundjoin%
\definecolor{currentfill}{rgb}{1.000000,0.512618,0.492826}%
\pgfsetfillcolor{currentfill}%
\pgfsetlinewidth{0.250937pt}%
\definecolor{currentstroke}{rgb}{1.000000,1.000000,1.000000}%
\pgfsetstrokecolor{currentstroke}%
\pgfsetdash{}{0pt}%
\pgfpathmoveto{\pgfqpoint{1.271369in}{0.889605in}}%
\pgfpathlineto{\pgfqpoint{1.370305in}{0.889605in}}%
\pgfpathlineto{\pgfqpoint{1.370305in}{0.790669in}}%
\pgfpathlineto{\pgfqpoint{1.271369in}{0.790669in}}%
\pgfpathlineto{\pgfqpoint{1.271369in}{0.889605in}}%
\pgfusepath{stroke,fill}%
\end{pgfscope}%
\begin{pgfscope}%
\pgfpathrectangle{\pgfqpoint{0.380943in}{0.295988in}}{\pgfqpoint{4.650000in}{0.692553in}}%
\pgfusepath{clip}%
\pgfsetbuttcap%
\pgfsetroundjoin%
\definecolor{currentfill}{rgb}{0.988604,0.796863,0.633449}%
\pgfsetfillcolor{currentfill}%
\pgfsetlinewidth{0.250937pt}%
\definecolor{currentstroke}{rgb}{1.000000,1.000000,1.000000}%
\pgfsetstrokecolor{currentstroke}%
\pgfsetdash{}{0pt}%
\pgfpathmoveto{\pgfqpoint{1.370305in}{0.889605in}}%
\pgfpathlineto{\pgfqpoint{1.469241in}{0.889605in}}%
\pgfpathlineto{\pgfqpoint{1.469241in}{0.790669in}}%
\pgfpathlineto{\pgfqpoint{1.370305in}{0.790669in}}%
\pgfpathlineto{\pgfqpoint{1.370305in}{0.889605in}}%
\pgfusepath{stroke,fill}%
\end{pgfscope}%
\begin{pgfscope}%
\pgfpathrectangle{\pgfqpoint{0.380943in}{0.295988in}}{\pgfqpoint{4.650000in}{0.692553in}}%
\pgfusepath{clip}%
\pgfsetbuttcap%
\pgfsetroundjoin%
\definecolor{currentfill}{rgb}{0.988604,0.796863,0.633449}%
\pgfsetfillcolor{currentfill}%
\pgfsetlinewidth{0.250937pt}%
\definecolor{currentstroke}{rgb}{1.000000,1.000000,1.000000}%
\pgfsetstrokecolor{currentstroke}%
\pgfsetdash{}{0pt}%
\pgfpathmoveto{\pgfqpoint{1.469241in}{0.889605in}}%
\pgfpathlineto{\pgfqpoint{1.568177in}{0.889605in}}%
\pgfpathlineto{\pgfqpoint{1.568177in}{0.790669in}}%
\pgfpathlineto{\pgfqpoint{1.469241in}{0.790669in}}%
\pgfpathlineto{\pgfqpoint{1.469241in}{0.889605in}}%
\pgfusepath{stroke,fill}%
\end{pgfscope}%
\begin{pgfscope}%
\pgfpathrectangle{\pgfqpoint{0.380943in}{0.295988in}}{\pgfqpoint{4.650000in}{0.692553in}}%
\pgfusepath{clip}%
\pgfsetbuttcap%
\pgfsetroundjoin%
\definecolor{currentfill}{rgb}{1.000000,0.588312,0.523952}%
\pgfsetfillcolor{currentfill}%
\pgfsetlinewidth{0.250937pt}%
\definecolor{currentstroke}{rgb}{1.000000,1.000000,1.000000}%
\pgfsetstrokecolor{currentstroke}%
\pgfsetdash{}{0pt}%
\pgfpathmoveto{\pgfqpoint{1.568177in}{0.889605in}}%
\pgfpathlineto{\pgfqpoint{1.667113in}{0.889605in}}%
\pgfpathlineto{\pgfqpoint{1.667113in}{0.790669in}}%
\pgfpathlineto{\pgfqpoint{1.568177in}{0.790669in}}%
\pgfpathlineto{\pgfqpoint{1.568177in}{0.889605in}}%
\pgfusepath{stroke,fill}%
\end{pgfscope}%
\begin{pgfscope}%
\pgfpathrectangle{\pgfqpoint{0.380943in}{0.295988in}}{\pgfqpoint{4.650000in}{0.692553in}}%
\pgfusepath{clip}%
\pgfsetbuttcap%
\pgfsetroundjoin%
\definecolor{currentfill}{rgb}{0.999446,0.645767,0.548927}%
\pgfsetfillcolor{currentfill}%
\pgfsetlinewidth{0.250937pt}%
\definecolor{currentstroke}{rgb}{1.000000,1.000000,1.000000}%
\pgfsetstrokecolor{currentstroke}%
\pgfsetdash{}{0pt}%
\pgfpathmoveto{\pgfqpoint{1.667113in}{0.889605in}}%
\pgfpathlineto{\pgfqpoint{1.766049in}{0.889605in}}%
\pgfpathlineto{\pgfqpoint{1.766049in}{0.790669in}}%
\pgfpathlineto{\pgfqpoint{1.667113in}{0.790669in}}%
\pgfpathlineto{\pgfqpoint{1.667113in}{0.889605in}}%
\pgfusepath{stroke,fill}%
\end{pgfscope}%
\begin{pgfscope}%
\pgfpathrectangle{\pgfqpoint{0.380943in}{0.295988in}}{\pgfqpoint{4.650000in}{0.692553in}}%
\pgfusepath{clip}%
\pgfsetbuttcap%
\pgfsetroundjoin%
\definecolor{currentfill}{rgb}{0.998601,0.667759,0.560769}%
\pgfsetfillcolor{currentfill}%
\pgfsetlinewidth{0.250937pt}%
\definecolor{currentstroke}{rgb}{1.000000,1.000000,1.000000}%
\pgfsetstrokecolor{currentstroke}%
\pgfsetdash{}{0pt}%
\pgfpathmoveto{\pgfqpoint{1.766049in}{0.889605in}}%
\pgfpathlineto{\pgfqpoint{1.864986in}{0.889605in}}%
\pgfpathlineto{\pgfqpoint{1.864986in}{0.790669in}}%
\pgfpathlineto{\pgfqpoint{1.766049in}{0.790669in}}%
\pgfpathlineto{\pgfqpoint{1.766049in}{0.889605in}}%
\pgfusepath{stroke,fill}%
\end{pgfscope}%
\begin{pgfscope}%
\pgfpathrectangle{\pgfqpoint{0.380943in}{0.295988in}}{\pgfqpoint{4.650000in}{0.692553in}}%
\pgfusepath{clip}%
\pgfsetbuttcap%
\pgfsetroundjoin%
\definecolor{currentfill}{rgb}{0.996401,0.724937,0.591557}%
\pgfsetfillcolor{currentfill}%
\pgfsetlinewidth{0.250937pt}%
\definecolor{currentstroke}{rgb}{1.000000,1.000000,1.000000}%
\pgfsetstrokecolor{currentstroke}%
\pgfsetdash{}{0pt}%
\pgfpathmoveto{\pgfqpoint{1.864986in}{0.889605in}}%
\pgfpathlineto{\pgfqpoint{1.963922in}{0.889605in}}%
\pgfpathlineto{\pgfqpoint{1.963922in}{0.790669in}}%
\pgfpathlineto{\pgfqpoint{1.864986in}{0.790669in}}%
\pgfpathlineto{\pgfqpoint{1.864986in}{0.889605in}}%
\pgfusepath{stroke,fill}%
\end{pgfscope}%
\begin{pgfscope}%
\pgfpathrectangle{\pgfqpoint{0.380943in}{0.295988in}}{\pgfqpoint{4.650000in}{0.692553in}}%
\pgfusepath{clip}%
\pgfsetbuttcap%
\pgfsetroundjoin%
\definecolor{currentfill}{rgb}{0.989619,0.788235,0.628374}%
\pgfsetfillcolor{currentfill}%
\pgfsetlinewidth{0.250937pt}%
\definecolor{currentstroke}{rgb}{1.000000,1.000000,1.000000}%
\pgfsetstrokecolor{currentstroke}%
\pgfsetdash{}{0pt}%
\pgfpathmoveto{\pgfqpoint{1.963922in}{0.889605in}}%
\pgfpathlineto{\pgfqpoint{2.062858in}{0.889605in}}%
\pgfpathlineto{\pgfqpoint{2.062858in}{0.790669in}}%
\pgfpathlineto{\pgfqpoint{1.963922in}{0.790669in}}%
\pgfpathlineto{\pgfqpoint{1.963922in}{0.889605in}}%
\pgfusepath{stroke,fill}%
\end{pgfscope}%
\begin{pgfscope}%
\pgfpathrectangle{\pgfqpoint{0.380943in}{0.295988in}}{\pgfqpoint{4.650000in}{0.692553in}}%
\pgfusepath{clip}%
\pgfsetbuttcap%
\pgfsetroundjoin%
\definecolor{currentfill}{rgb}{0.998939,0.658962,0.556032}%
\pgfsetfillcolor{currentfill}%
\pgfsetlinewidth{0.250937pt}%
\definecolor{currentstroke}{rgb}{1.000000,1.000000,1.000000}%
\pgfsetstrokecolor{currentstroke}%
\pgfsetdash{}{0pt}%
\pgfpathmoveto{\pgfqpoint{2.062858in}{0.889605in}}%
\pgfpathlineto{\pgfqpoint{2.161794in}{0.889605in}}%
\pgfpathlineto{\pgfqpoint{2.161794in}{0.790669in}}%
\pgfpathlineto{\pgfqpoint{2.062858in}{0.790669in}}%
\pgfpathlineto{\pgfqpoint{2.062858in}{0.889605in}}%
\pgfusepath{stroke,fill}%
\end{pgfscope}%
\begin{pgfscope}%
\pgfpathrectangle{\pgfqpoint{0.380943in}{0.295988in}}{\pgfqpoint{4.650000in}{0.692553in}}%
\pgfusepath{clip}%
\pgfsetbuttcap%
\pgfsetroundjoin%
\definecolor{currentfill}{rgb}{0.999785,0.636970,0.544191}%
\pgfsetfillcolor{currentfill}%
\pgfsetlinewidth{0.250937pt}%
\definecolor{currentstroke}{rgb}{1.000000,1.000000,1.000000}%
\pgfsetstrokecolor{currentstroke}%
\pgfsetdash{}{0pt}%
\pgfpathmoveto{\pgfqpoint{2.161794in}{0.889605in}}%
\pgfpathlineto{\pgfqpoint{2.260730in}{0.889605in}}%
\pgfpathlineto{\pgfqpoint{2.260730in}{0.790669in}}%
\pgfpathlineto{\pgfqpoint{2.161794in}{0.790669in}}%
\pgfpathlineto{\pgfqpoint{2.161794in}{0.889605in}}%
\pgfusepath{stroke,fill}%
\end{pgfscope}%
\begin{pgfscope}%
\pgfpathrectangle{\pgfqpoint{0.380943in}{0.295988in}}{\pgfqpoint{4.650000in}{0.692553in}}%
\pgfusepath{clip}%
\pgfsetbuttcap%
\pgfsetroundjoin%
\definecolor{currentfill}{rgb}{0.995709,0.736471,0.597924}%
\pgfsetfillcolor{currentfill}%
\pgfsetlinewidth{0.250937pt}%
\definecolor{currentstroke}{rgb}{1.000000,1.000000,1.000000}%
\pgfsetstrokecolor{currentstroke}%
\pgfsetdash{}{0pt}%
\pgfpathmoveto{\pgfqpoint{2.260730in}{0.889605in}}%
\pgfpathlineto{\pgfqpoint{2.359666in}{0.889605in}}%
\pgfpathlineto{\pgfqpoint{2.359666in}{0.790669in}}%
\pgfpathlineto{\pgfqpoint{2.260730in}{0.790669in}}%
\pgfpathlineto{\pgfqpoint{2.260730in}{0.889605in}}%
\pgfusepath{stroke,fill}%
\end{pgfscope}%
\begin{pgfscope}%
\pgfpathrectangle{\pgfqpoint{0.380943in}{0.295988in}}{\pgfqpoint{4.650000in}{0.692553in}}%
\pgfusepath{clip}%
\pgfsetbuttcap%
\pgfsetroundjoin%
\definecolor{currentfill}{rgb}{0.999785,0.636970,0.544191}%
\pgfsetfillcolor{currentfill}%
\pgfsetlinewidth{0.250937pt}%
\definecolor{currentstroke}{rgb}{1.000000,1.000000,1.000000}%
\pgfsetstrokecolor{currentstroke}%
\pgfsetdash{}{0pt}%
\pgfpathmoveto{\pgfqpoint{2.359666in}{0.889605in}}%
\pgfpathlineto{\pgfqpoint{2.458603in}{0.889605in}}%
\pgfpathlineto{\pgfqpoint{2.458603in}{0.790669in}}%
\pgfpathlineto{\pgfqpoint{2.359666in}{0.790669in}}%
\pgfpathlineto{\pgfqpoint{2.359666in}{0.889605in}}%
\pgfusepath{stroke,fill}%
\end{pgfscope}%
\begin{pgfscope}%
\pgfpathrectangle{\pgfqpoint{0.380943in}{0.295988in}}{\pgfqpoint{4.650000in}{0.692553in}}%
\pgfusepath{clip}%
\pgfsetbuttcap%
\pgfsetroundjoin%
\definecolor{currentfill}{rgb}{0.998093,0.680953,0.567874}%
\pgfsetfillcolor{currentfill}%
\pgfsetlinewidth{0.250937pt}%
\definecolor{currentstroke}{rgb}{1.000000,1.000000,1.000000}%
\pgfsetstrokecolor{currentstroke}%
\pgfsetdash{}{0pt}%
\pgfpathmoveto{\pgfqpoint{2.458603in}{0.889605in}}%
\pgfpathlineto{\pgfqpoint{2.557539in}{0.889605in}}%
\pgfpathlineto{\pgfqpoint{2.557539in}{0.790669in}}%
\pgfpathlineto{\pgfqpoint{2.458603in}{0.790669in}}%
\pgfpathlineto{\pgfqpoint{2.458603in}{0.889605in}}%
\pgfusepath{stroke,fill}%
\end{pgfscope}%
\begin{pgfscope}%
\pgfpathrectangle{\pgfqpoint{0.380943in}{0.295988in}}{\pgfqpoint{4.650000in}{0.692553in}}%
\pgfusepath{clip}%
\pgfsetbuttcap%
\pgfsetroundjoin%
\definecolor{currentfill}{rgb}{1.000000,0.581546,0.521246}%
\pgfsetfillcolor{currentfill}%
\pgfsetlinewidth{0.250937pt}%
\definecolor{currentstroke}{rgb}{1.000000,1.000000,1.000000}%
\pgfsetstrokecolor{currentstroke}%
\pgfsetdash{}{0pt}%
\pgfpathmoveto{\pgfqpoint{2.557539in}{0.889605in}}%
\pgfpathlineto{\pgfqpoint{2.656475in}{0.889605in}}%
\pgfpathlineto{\pgfqpoint{2.656475in}{0.790669in}}%
\pgfpathlineto{\pgfqpoint{2.557539in}{0.790669in}}%
\pgfpathlineto{\pgfqpoint{2.557539in}{0.889605in}}%
\pgfusepath{stroke,fill}%
\end{pgfscope}%
\begin{pgfscope}%
\pgfpathrectangle{\pgfqpoint{0.380943in}{0.295988in}}{\pgfqpoint{4.650000in}{0.692553in}}%
\pgfusepath{clip}%
\pgfsetbuttcap%
\pgfsetroundjoin%
\definecolor{currentfill}{rgb}{0.991311,0.773856,0.619915}%
\pgfsetfillcolor{currentfill}%
\pgfsetlinewidth{0.250937pt}%
\definecolor{currentstroke}{rgb}{1.000000,1.000000,1.000000}%
\pgfsetstrokecolor{currentstroke}%
\pgfsetdash{}{0pt}%
\pgfpathmoveto{\pgfqpoint{2.656475in}{0.889605in}}%
\pgfpathlineto{\pgfqpoint{2.755411in}{0.889605in}}%
\pgfpathlineto{\pgfqpoint{2.755411in}{0.790669in}}%
\pgfpathlineto{\pgfqpoint{2.656475in}{0.790669in}}%
\pgfpathlineto{\pgfqpoint{2.656475in}{0.889605in}}%
\pgfusepath{stroke,fill}%
\end{pgfscope}%
\begin{pgfscope}%
\pgfpathrectangle{\pgfqpoint{0.380943in}{0.295988in}}{\pgfqpoint{4.650000in}{0.692553in}}%
\pgfusepath{clip}%
\pgfsetbuttcap%
\pgfsetroundjoin%
\definecolor{currentfill}{rgb}{0.988604,0.796863,0.633449}%
\pgfsetfillcolor{currentfill}%
\pgfsetlinewidth{0.250937pt}%
\definecolor{currentstroke}{rgb}{1.000000,1.000000,1.000000}%
\pgfsetstrokecolor{currentstroke}%
\pgfsetdash{}{0pt}%
\pgfpathmoveto{\pgfqpoint{2.755411in}{0.889605in}}%
\pgfpathlineto{\pgfqpoint{2.854347in}{0.889605in}}%
\pgfpathlineto{\pgfqpoint{2.854347in}{0.790669in}}%
\pgfpathlineto{\pgfqpoint{2.755411in}{0.790669in}}%
\pgfpathlineto{\pgfqpoint{2.755411in}{0.889605in}}%
\pgfusepath{stroke,fill}%
\end{pgfscope}%
\begin{pgfscope}%
\pgfpathrectangle{\pgfqpoint{0.380943in}{0.295988in}}{\pgfqpoint{4.650000in}{0.692553in}}%
\pgfusepath{clip}%
\pgfsetbuttcap%
\pgfsetroundjoin%
\definecolor{currentfill}{rgb}{0.994694,0.745098,0.602999}%
\pgfsetfillcolor{currentfill}%
\pgfsetlinewidth{0.250937pt}%
\definecolor{currentstroke}{rgb}{1.000000,1.000000,1.000000}%
\pgfsetstrokecolor{currentstroke}%
\pgfsetdash{}{0pt}%
\pgfpathmoveto{\pgfqpoint{2.854347in}{0.889605in}}%
\pgfpathlineto{\pgfqpoint{2.953283in}{0.889605in}}%
\pgfpathlineto{\pgfqpoint{2.953283in}{0.790669in}}%
\pgfpathlineto{\pgfqpoint{2.854347in}{0.790669in}}%
\pgfpathlineto{\pgfqpoint{2.854347in}{0.889605in}}%
\pgfusepath{stroke,fill}%
\end{pgfscope}%
\begin{pgfscope}%
\pgfpathrectangle{\pgfqpoint{0.380943in}{0.295988in}}{\pgfqpoint{4.650000in}{0.692553in}}%
\pgfusepath{clip}%
\pgfsetbuttcap%
\pgfsetroundjoin%
\definecolor{currentfill}{rgb}{0.982191,0.826190,0.659469}%
\pgfsetfillcolor{currentfill}%
\pgfsetlinewidth{0.250937pt}%
\definecolor{currentstroke}{rgb}{1.000000,1.000000,1.000000}%
\pgfsetstrokecolor{currentstroke}%
\pgfsetdash{}{0pt}%
\pgfpathmoveto{\pgfqpoint{2.953283in}{0.889605in}}%
\pgfpathlineto{\pgfqpoint{3.052220in}{0.889605in}}%
\pgfpathlineto{\pgfqpoint{3.052220in}{0.790669in}}%
\pgfpathlineto{\pgfqpoint{2.953283in}{0.790669in}}%
\pgfpathlineto{\pgfqpoint{2.953283in}{0.889605in}}%
\pgfusepath{stroke,fill}%
\end{pgfscope}%
\begin{pgfscope}%
\pgfpathrectangle{\pgfqpoint{0.380943in}{0.295988in}}{\pgfqpoint{4.650000in}{0.692553in}}%
\pgfusepath{clip}%
\pgfsetbuttcap%
\pgfsetroundjoin%
\definecolor{currentfill}{rgb}{0.999446,0.645767,0.548927}%
\pgfsetfillcolor{currentfill}%
\pgfsetlinewidth{0.250937pt}%
\definecolor{currentstroke}{rgb}{1.000000,1.000000,1.000000}%
\pgfsetstrokecolor{currentstroke}%
\pgfsetdash{}{0pt}%
\pgfpathmoveto{\pgfqpoint{3.052220in}{0.889605in}}%
\pgfpathlineto{\pgfqpoint{3.151156in}{0.889605in}}%
\pgfpathlineto{\pgfqpoint{3.151156in}{0.790669in}}%
\pgfpathlineto{\pgfqpoint{3.052220in}{0.790669in}}%
\pgfpathlineto{\pgfqpoint{3.052220in}{0.889605in}}%
\pgfusepath{stroke,fill}%
\end{pgfscope}%
\begin{pgfscope}%
\pgfpathrectangle{\pgfqpoint{0.380943in}{0.295988in}}{\pgfqpoint{4.650000in}{0.692553in}}%
\pgfusepath{clip}%
\pgfsetbuttcap%
\pgfsetroundjoin%
\definecolor{currentfill}{rgb}{0.994694,0.745098,0.602999}%
\pgfsetfillcolor{currentfill}%
\pgfsetlinewidth{0.250937pt}%
\definecolor{currentstroke}{rgb}{1.000000,1.000000,1.000000}%
\pgfsetstrokecolor{currentstroke}%
\pgfsetdash{}{0pt}%
\pgfpathmoveto{\pgfqpoint{3.151156in}{0.889605in}}%
\pgfpathlineto{\pgfqpoint{3.250092in}{0.889605in}}%
\pgfpathlineto{\pgfqpoint{3.250092in}{0.790669in}}%
\pgfpathlineto{\pgfqpoint{3.151156in}{0.790669in}}%
\pgfpathlineto{\pgfqpoint{3.151156in}{0.889605in}}%
\pgfusepath{stroke,fill}%
\end{pgfscope}%
\begin{pgfscope}%
\pgfpathrectangle{\pgfqpoint{0.380943in}{0.295988in}}{\pgfqpoint{4.650000in}{0.692553in}}%
\pgfusepath{clip}%
\pgfsetbuttcap%
\pgfsetroundjoin%
\definecolor{currentfill}{rgb}{0.996401,0.724937,0.591557}%
\pgfsetfillcolor{currentfill}%
\pgfsetlinewidth{0.250937pt}%
\definecolor{currentstroke}{rgb}{1.000000,1.000000,1.000000}%
\pgfsetstrokecolor{currentstroke}%
\pgfsetdash{}{0pt}%
\pgfpathmoveto{\pgfqpoint{3.250092in}{0.889605in}}%
\pgfpathlineto{\pgfqpoint{3.349028in}{0.889605in}}%
\pgfpathlineto{\pgfqpoint{3.349028in}{0.790669in}}%
\pgfpathlineto{\pgfqpoint{3.250092in}{0.790669in}}%
\pgfpathlineto{\pgfqpoint{3.250092in}{0.889605in}}%
\pgfusepath{stroke,fill}%
\end{pgfscope}%
\begin{pgfscope}%
\pgfpathrectangle{\pgfqpoint{0.380943in}{0.295988in}}{\pgfqpoint{4.650000in}{0.692553in}}%
\pgfusepath{clip}%
\pgfsetbuttcap%
\pgfsetroundjoin%
\definecolor{currentfill}{rgb}{0.956171,0.434602,0.434602}%
\pgfsetfillcolor{currentfill}%
\pgfsetlinewidth{0.250937pt}%
\definecolor{currentstroke}{rgb}{1.000000,1.000000,1.000000}%
\pgfsetstrokecolor{currentstroke}%
\pgfsetdash{}{0pt}%
\pgfpathmoveto{\pgfqpoint{3.349028in}{0.889605in}}%
\pgfpathlineto{\pgfqpoint{3.447964in}{0.889605in}}%
\pgfpathlineto{\pgfqpoint{3.447964in}{0.790669in}}%
\pgfpathlineto{\pgfqpoint{3.349028in}{0.790669in}}%
\pgfpathlineto{\pgfqpoint{3.349028in}{0.889605in}}%
\pgfusepath{stroke,fill}%
\end{pgfscope}%
\begin{pgfscope}%
\pgfpathrectangle{\pgfqpoint{0.380943in}{0.295988in}}{\pgfqpoint{4.650000in}{0.692553in}}%
\pgfusepath{clip}%
\pgfsetbuttcap%
\pgfsetroundjoin%
\definecolor{currentfill}{rgb}{1.000000,0.528689,0.499592}%
\pgfsetfillcolor{currentfill}%
\pgfsetlinewidth{0.250937pt}%
\definecolor{currentstroke}{rgb}{1.000000,1.000000,1.000000}%
\pgfsetstrokecolor{currentstroke}%
\pgfsetdash{}{0pt}%
\pgfpathmoveto{\pgfqpoint{3.447964in}{0.889605in}}%
\pgfpathlineto{\pgfqpoint{3.546901in}{0.889605in}}%
\pgfpathlineto{\pgfqpoint{3.546901in}{0.790669in}}%
\pgfpathlineto{\pgfqpoint{3.447964in}{0.790669in}}%
\pgfpathlineto{\pgfqpoint{3.447964in}{0.889605in}}%
\pgfusepath{stroke,fill}%
\end{pgfscope}%
\begin{pgfscope}%
\pgfpathrectangle{\pgfqpoint{0.380943in}{0.295988in}}{\pgfqpoint{4.650000in}{0.692553in}}%
\pgfusepath{clip}%
\pgfsetbuttcap%
\pgfsetroundjoin%
\definecolor{currentfill}{rgb}{0.997247,0.702945,0.579715}%
\pgfsetfillcolor{currentfill}%
\pgfsetlinewidth{0.250937pt}%
\definecolor{currentstroke}{rgb}{1.000000,1.000000,1.000000}%
\pgfsetstrokecolor{currentstroke}%
\pgfsetdash{}{0pt}%
\pgfpathmoveto{\pgfqpoint{3.546901in}{0.889605in}}%
\pgfpathlineto{\pgfqpoint{3.645837in}{0.889605in}}%
\pgfpathlineto{\pgfqpoint{3.645837in}{0.790669in}}%
\pgfpathlineto{\pgfqpoint{3.546901in}{0.790669in}}%
\pgfpathlineto{\pgfqpoint{3.546901in}{0.889605in}}%
\pgfusepath{stroke,fill}%
\end{pgfscope}%
\begin{pgfscope}%
\pgfpathrectangle{\pgfqpoint{0.380943in}{0.295988in}}{\pgfqpoint{4.650000in}{0.692553in}}%
\pgfusepath{clip}%
\pgfsetbuttcap%
\pgfsetroundjoin%
\definecolor{currentfill}{rgb}{0.995709,0.736471,0.597924}%
\pgfsetfillcolor{currentfill}%
\pgfsetlinewidth{0.250937pt}%
\definecolor{currentstroke}{rgb}{1.000000,1.000000,1.000000}%
\pgfsetstrokecolor{currentstroke}%
\pgfsetdash{}{0pt}%
\pgfpathmoveto{\pgfqpoint{3.645837in}{0.889605in}}%
\pgfpathlineto{\pgfqpoint{3.744773in}{0.889605in}}%
\pgfpathlineto{\pgfqpoint{3.744773in}{0.790669in}}%
\pgfpathlineto{\pgfqpoint{3.645837in}{0.790669in}}%
\pgfpathlineto{\pgfqpoint{3.645837in}{0.889605in}}%
\pgfusepath{stroke,fill}%
\end{pgfscope}%
\begin{pgfscope}%
\pgfpathrectangle{\pgfqpoint{0.380943in}{0.295988in}}{\pgfqpoint{4.650000in}{0.692553in}}%
\pgfusepath{clip}%
\pgfsetbuttcap%
\pgfsetroundjoin%
\definecolor{currentfill}{rgb}{0.998939,0.658962,0.556032}%
\pgfsetfillcolor{currentfill}%
\pgfsetlinewidth{0.250937pt}%
\definecolor{currentstroke}{rgb}{1.000000,1.000000,1.000000}%
\pgfsetstrokecolor{currentstroke}%
\pgfsetdash{}{0pt}%
\pgfpathmoveto{\pgfqpoint{3.744773in}{0.889605in}}%
\pgfpathlineto{\pgfqpoint{3.843709in}{0.889605in}}%
\pgfpathlineto{\pgfqpoint{3.843709in}{0.790669in}}%
\pgfpathlineto{\pgfqpoint{3.744773in}{0.790669in}}%
\pgfpathlineto{\pgfqpoint{3.744773in}{0.889605in}}%
\pgfusepath{stroke,fill}%
\end{pgfscope}%
\begin{pgfscope}%
\pgfpathrectangle{\pgfqpoint{0.380943in}{0.295988in}}{\pgfqpoint{4.650000in}{0.692553in}}%
\pgfusepath{clip}%
\pgfsetbuttcap%
\pgfsetroundjoin%
\definecolor{currentfill}{rgb}{1.000000,0.522261,0.496886}%
\pgfsetfillcolor{currentfill}%
\pgfsetlinewidth{0.250937pt}%
\definecolor{currentstroke}{rgb}{1.000000,1.000000,1.000000}%
\pgfsetstrokecolor{currentstroke}%
\pgfsetdash{}{0pt}%
\pgfpathmoveto{\pgfqpoint{3.843709in}{0.889605in}}%
\pgfpathlineto{\pgfqpoint{3.942645in}{0.889605in}}%
\pgfpathlineto{\pgfqpoint{3.942645in}{0.790669in}}%
\pgfpathlineto{\pgfqpoint{3.843709in}{0.790669in}}%
\pgfpathlineto{\pgfqpoint{3.843709in}{0.889605in}}%
\pgfusepath{stroke,fill}%
\end{pgfscope}%
\begin{pgfscope}%
\pgfpathrectangle{\pgfqpoint{0.380943in}{0.295988in}}{\pgfqpoint{4.650000in}{0.692553in}}%
\pgfusepath{clip}%
\pgfsetbuttcap%
\pgfsetroundjoin%
\definecolor{currentfill}{rgb}{0.998939,0.658962,0.556032}%
\pgfsetfillcolor{currentfill}%
\pgfsetlinewidth{0.250937pt}%
\definecolor{currentstroke}{rgb}{1.000000,1.000000,1.000000}%
\pgfsetstrokecolor{currentstroke}%
\pgfsetdash{}{0pt}%
\pgfpathmoveto{\pgfqpoint{3.942645in}{0.889605in}}%
\pgfpathlineto{\pgfqpoint{4.041581in}{0.889605in}}%
\pgfpathlineto{\pgfqpoint{4.041581in}{0.790669in}}%
\pgfpathlineto{\pgfqpoint{3.942645in}{0.790669in}}%
\pgfpathlineto{\pgfqpoint{3.942645in}{0.889605in}}%
\pgfusepath{stroke,fill}%
\end{pgfscope}%
\begin{pgfscope}%
\pgfpathrectangle{\pgfqpoint{0.380943in}{0.295988in}}{\pgfqpoint{4.650000in}{0.692553in}}%
\pgfusepath{clip}%
\pgfsetbuttcap%
\pgfsetroundjoin%
\definecolor{currentfill}{rgb}{0.995709,0.736471,0.597924}%
\pgfsetfillcolor{currentfill}%
\pgfsetlinewidth{0.250937pt}%
\definecolor{currentstroke}{rgb}{1.000000,1.000000,1.000000}%
\pgfsetstrokecolor{currentstroke}%
\pgfsetdash{}{0pt}%
\pgfpathmoveto{\pgfqpoint{4.041581in}{0.889605in}}%
\pgfpathlineto{\pgfqpoint{4.140518in}{0.889605in}}%
\pgfpathlineto{\pgfqpoint{4.140518in}{0.790669in}}%
\pgfpathlineto{\pgfqpoint{4.041581in}{0.790669in}}%
\pgfpathlineto{\pgfqpoint{4.041581in}{0.889605in}}%
\pgfusepath{stroke,fill}%
\end{pgfscope}%
\begin{pgfscope}%
\pgfpathrectangle{\pgfqpoint{0.380943in}{0.295988in}}{\pgfqpoint{4.650000in}{0.692553in}}%
\pgfusepath{clip}%
\pgfsetbuttcap%
\pgfsetroundjoin%
\definecolor{currentfill}{rgb}{0.997586,0.694148,0.574979}%
\pgfsetfillcolor{currentfill}%
\pgfsetlinewidth{0.250937pt}%
\definecolor{currentstroke}{rgb}{1.000000,1.000000,1.000000}%
\pgfsetstrokecolor{currentstroke}%
\pgfsetdash{}{0pt}%
\pgfpathmoveto{\pgfqpoint{4.140518in}{0.889605in}}%
\pgfpathlineto{\pgfqpoint{4.239454in}{0.889605in}}%
\pgfpathlineto{\pgfqpoint{4.239454in}{0.790669in}}%
\pgfpathlineto{\pgfqpoint{4.140518in}{0.790669in}}%
\pgfpathlineto{\pgfqpoint{4.140518in}{0.889605in}}%
\pgfusepath{stroke,fill}%
\end{pgfscope}%
\begin{pgfscope}%
\pgfpathrectangle{\pgfqpoint{0.380943in}{0.295988in}}{\pgfqpoint{4.650000in}{0.692553in}}%
\pgfusepath{clip}%
\pgfsetbuttcap%
\pgfsetroundjoin%
\definecolor{currentfill}{rgb}{1.000000,0.571396,0.517186}%
\pgfsetfillcolor{currentfill}%
\pgfsetlinewidth{0.250937pt}%
\definecolor{currentstroke}{rgb}{1.000000,1.000000,1.000000}%
\pgfsetstrokecolor{currentstroke}%
\pgfsetdash{}{0pt}%
\pgfpathmoveto{\pgfqpoint{4.239454in}{0.889605in}}%
\pgfpathlineto{\pgfqpoint{4.338390in}{0.889605in}}%
\pgfpathlineto{\pgfqpoint{4.338390in}{0.790669in}}%
\pgfpathlineto{\pgfqpoint{4.239454in}{0.790669in}}%
\pgfpathlineto{\pgfqpoint{4.239454in}{0.889605in}}%
\pgfusepath{stroke,fill}%
\end{pgfscope}%
\begin{pgfscope}%
\pgfpathrectangle{\pgfqpoint{0.380943in}{0.295988in}}{\pgfqpoint{4.650000in}{0.692553in}}%
\pgfusepath{clip}%
\pgfsetbuttcap%
\pgfsetroundjoin%
\definecolor{currentfill}{rgb}{0.973057,0.868051,0.691457}%
\pgfsetfillcolor{currentfill}%
\pgfsetlinewidth{0.250937pt}%
\definecolor{currentstroke}{rgb}{1.000000,1.000000,1.000000}%
\pgfsetstrokecolor{currentstroke}%
\pgfsetdash{}{0pt}%
\pgfpathmoveto{\pgfqpoint{4.338390in}{0.889605in}}%
\pgfpathlineto{\pgfqpoint{4.437326in}{0.889605in}}%
\pgfpathlineto{\pgfqpoint{4.437326in}{0.790669in}}%
\pgfpathlineto{\pgfqpoint{4.338390in}{0.790669in}}%
\pgfpathlineto{\pgfqpoint{4.338390in}{0.889605in}}%
\pgfusepath{stroke,fill}%
\end{pgfscope}%
\begin{pgfscope}%
\pgfpathrectangle{\pgfqpoint{0.380943in}{0.295988in}}{\pgfqpoint{4.650000in}{0.692553in}}%
\pgfusepath{clip}%
\pgfsetbuttcap%
\pgfsetroundjoin%
\definecolor{currentfill}{rgb}{0.970012,0.883276,0.699577}%
\pgfsetfillcolor{currentfill}%
\pgfsetlinewidth{0.250937pt}%
\definecolor{currentstroke}{rgb}{1.000000,1.000000,1.000000}%
\pgfsetstrokecolor{currentstroke}%
\pgfsetdash{}{0pt}%
\pgfpathmoveto{\pgfqpoint{4.437326in}{0.889605in}}%
\pgfpathlineto{\pgfqpoint{4.536262in}{0.889605in}}%
\pgfpathlineto{\pgfqpoint{4.536262in}{0.790669in}}%
\pgfpathlineto{\pgfqpoint{4.437326in}{0.790669in}}%
\pgfpathlineto{\pgfqpoint{4.437326in}{0.889605in}}%
\pgfusepath{stroke,fill}%
\end{pgfscope}%
\begin{pgfscope}%
\pgfpathrectangle{\pgfqpoint{0.380943in}{0.295988in}}{\pgfqpoint{4.650000in}{0.692553in}}%
\pgfusepath{clip}%
\pgfsetbuttcap%
\pgfsetroundjoin%
\definecolor{currentfill}{rgb}{0.973057,0.868051,0.691457}%
\pgfsetfillcolor{currentfill}%
\pgfsetlinewidth{0.250937pt}%
\definecolor{currentstroke}{rgb}{1.000000,1.000000,1.000000}%
\pgfsetstrokecolor{currentstroke}%
\pgfsetdash{}{0pt}%
\pgfpathmoveto{\pgfqpoint{4.536262in}{0.889605in}}%
\pgfpathlineto{\pgfqpoint{4.635198in}{0.889605in}}%
\pgfpathlineto{\pgfqpoint{4.635198in}{0.790669in}}%
\pgfpathlineto{\pgfqpoint{4.536262in}{0.790669in}}%
\pgfpathlineto{\pgfqpoint{4.536262in}{0.889605in}}%
\pgfusepath{stroke,fill}%
\end{pgfscope}%
\begin{pgfscope}%
\pgfpathrectangle{\pgfqpoint{0.380943in}{0.295988in}}{\pgfqpoint{4.650000in}{0.692553in}}%
\pgfusepath{clip}%
\pgfsetbuttcap%
\pgfsetroundjoin%
\definecolor{currentfill}{rgb}{0.977116,0.848181,0.679769}%
\pgfsetfillcolor{currentfill}%
\pgfsetlinewidth{0.250937pt}%
\definecolor{currentstroke}{rgb}{1.000000,1.000000,1.000000}%
\pgfsetstrokecolor{currentstroke}%
\pgfsetdash{}{0pt}%
\pgfpathmoveto{\pgfqpoint{4.635198in}{0.889605in}}%
\pgfpathlineto{\pgfqpoint{4.734135in}{0.889605in}}%
\pgfpathlineto{\pgfqpoint{4.734135in}{0.790669in}}%
\pgfpathlineto{\pgfqpoint{4.635198in}{0.790669in}}%
\pgfpathlineto{\pgfqpoint{4.635198in}{0.889605in}}%
\pgfusepath{stroke,fill}%
\end{pgfscope}%
\begin{pgfscope}%
\pgfpathrectangle{\pgfqpoint{0.380943in}{0.295988in}}{\pgfqpoint{4.650000in}{0.692553in}}%
\pgfusepath{clip}%
\pgfsetbuttcap%
\pgfsetroundjoin%
\definecolor{currentfill}{rgb}{0.963091,0.919493,0.720185}%
\pgfsetfillcolor{currentfill}%
\pgfsetlinewidth{0.250937pt}%
\definecolor{currentstroke}{rgb}{1.000000,1.000000,1.000000}%
\pgfsetstrokecolor{currentstroke}%
\pgfsetdash{}{0pt}%
\pgfpathmoveto{\pgfqpoint{4.734135in}{0.889605in}}%
\pgfpathlineto{\pgfqpoint{4.833071in}{0.889605in}}%
\pgfpathlineto{\pgfqpoint{4.833071in}{0.790669in}}%
\pgfpathlineto{\pgfqpoint{4.734135in}{0.790669in}}%
\pgfpathlineto{\pgfqpoint{4.734135in}{0.889605in}}%
\pgfusepath{stroke,fill}%
\end{pgfscope}%
\begin{pgfscope}%
\pgfpathrectangle{\pgfqpoint{0.380943in}{0.295988in}}{\pgfqpoint{4.650000in}{0.692553in}}%
\pgfusepath{clip}%
\pgfsetbuttcap%
\pgfsetroundjoin%
\definecolor{currentfill}{rgb}{0.964275,0.912388,0.715448}%
\pgfsetfillcolor{currentfill}%
\pgfsetlinewidth{0.250937pt}%
\definecolor{currentstroke}{rgb}{1.000000,1.000000,1.000000}%
\pgfsetstrokecolor{currentstroke}%
\pgfsetdash{}{0pt}%
\pgfpathmoveto{\pgfqpoint{4.833071in}{0.889605in}}%
\pgfpathlineto{\pgfqpoint{4.932007in}{0.889605in}}%
\pgfpathlineto{\pgfqpoint{4.932007in}{0.790669in}}%
\pgfpathlineto{\pgfqpoint{4.833071in}{0.790669in}}%
\pgfpathlineto{\pgfqpoint{4.833071in}{0.889605in}}%
\pgfusepath{stroke,fill}%
\end{pgfscope}%
\begin{pgfscope}%
\pgfpathrectangle{\pgfqpoint{0.380943in}{0.295988in}}{\pgfqpoint{4.650000in}{0.692553in}}%
\pgfusepath{clip}%
\pgfsetbuttcap%
\pgfsetroundjoin%
\definecolor{currentfill}{rgb}{0.983391,0.971765,0.788189}%
\pgfsetfillcolor{currentfill}%
\pgfsetlinewidth{0.250937pt}%
\definecolor{currentstroke}{rgb}{1.000000,1.000000,1.000000}%
\pgfsetstrokecolor{currentstroke}%
\pgfsetdash{}{0pt}%
\pgfpathmoveto{\pgfqpoint{4.932007in}{0.889605in}}%
\pgfpathlineto{\pgfqpoint{5.030943in}{0.889605in}}%
\pgfpathlineto{\pgfqpoint{5.030943in}{0.790669in}}%
\pgfpathlineto{\pgfqpoint{4.932007in}{0.790669in}}%
\pgfpathlineto{\pgfqpoint{4.932007in}{0.889605in}}%
\pgfusepath{stroke,fill}%
\end{pgfscope}%
\begin{pgfscope}%
\pgfpathrectangle{\pgfqpoint{0.380943in}{0.295988in}}{\pgfqpoint{4.650000in}{0.692553in}}%
\pgfusepath{clip}%
\pgfsetbuttcap%
\pgfsetroundjoin%
\pgfsetlinewidth{0.250937pt}%
\definecolor{currentstroke}{rgb}{1.000000,1.000000,1.000000}%
\pgfsetstrokecolor{currentstroke}%
\pgfsetdash{}{0pt}%
\pgfpathmoveto{\pgfqpoint{0.380943in}{0.790669in}}%
\pgfpathlineto{\pgfqpoint{0.479879in}{0.790669in}}%
\pgfpathlineto{\pgfqpoint{0.479879in}{0.691732in}}%
\pgfpathlineto{\pgfqpoint{0.380943in}{0.691732in}}%
\pgfpathlineto{\pgfqpoint{0.380943in}{0.790669in}}%
\pgfusepath{stroke}%
\end{pgfscope}%
\begin{pgfscope}%
\pgfpathrectangle{\pgfqpoint{0.380943in}{0.295988in}}{\pgfqpoint{4.650000in}{0.692553in}}%
\pgfusepath{clip}%
\pgfsetbuttcap%
\pgfsetroundjoin%
\definecolor{currentfill}{rgb}{0.986774,0.977516,0.796986}%
\pgfsetfillcolor{currentfill}%
\pgfsetlinewidth{0.250937pt}%
\definecolor{currentstroke}{rgb}{1.000000,1.000000,1.000000}%
\pgfsetstrokecolor{currentstroke}%
\pgfsetdash{}{0pt}%
\pgfpathmoveto{\pgfqpoint{0.479879in}{0.790669in}}%
\pgfpathlineto{\pgfqpoint{0.578815in}{0.790669in}}%
\pgfpathlineto{\pgfqpoint{0.578815in}{0.691732in}}%
\pgfpathlineto{\pgfqpoint{0.479879in}{0.691732in}}%
\pgfpathlineto{\pgfqpoint{0.479879in}{0.790669in}}%
\pgfusepath{stroke,fill}%
\end{pgfscope}%
\begin{pgfscope}%
\pgfpathrectangle{\pgfqpoint{0.380943in}{0.295988in}}{\pgfqpoint{4.650000in}{0.692553in}}%
\pgfusepath{clip}%
\pgfsetbuttcap%
\pgfsetroundjoin%
\definecolor{currentfill}{rgb}{0.971534,0.875663,0.695517}%
\pgfsetfillcolor{currentfill}%
\pgfsetlinewidth{0.250937pt}%
\definecolor{currentstroke}{rgb}{1.000000,1.000000,1.000000}%
\pgfsetstrokecolor{currentstroke}%
\pgfsetdash{}{0pt}%
\pgfpathmoveto{\pgfqpoint{0.578815in}{0.790669in}}%
\pgfpathlineto{\pgfqpoint{0.677752in}{0.790669in}}%
\pgfpathlineto{\pgfqpoint{0.677752in}{0.691732in}}%
\pgfpathlineto{\pgfqpoint{0.578815in}{0.691732in}}%
\pgfpathlineto{\pgfqpoint{0.578815in}{0.790669in}}%
\pgfusepath{stroke,fill}%
\end{pgfscope}%
\begin{pgfscope}%
\pgfpathrectangle{\pgfqpoint{0.380943in}{0.295988in}}{\pgfqpoint{4.650000in}{0.692553in}}%
\pgfusepath{clip}%
\pgfsetbuttcap%
\pgfsetroundjoin%
\definecolor{currentfill}{rgb}{0.963937,0.914418,0.716801}%
\pgfsetfillcolor{currentfill}%
\pgfsetlinewidth{0.250937pt}%
\definecolor{currentstroke}{rgb}{1.000000,1.000000,1.000000}%
\pgfsetstrokecolor{currentstroke}%
\pgfsetdash{}{0pt}%
\pgfpathmoveto{\pgfqpoint{0.677752in}{0.790669in}}%
\pgfpathlineto{\pgfqpoint{0.776688in}{0.790669in}}%
\pgfpathlineto{\pgfqpoint{0.776688in}{0.691732in}}%
\pgfpathlineto{\pgfqpoint{0.677752in}{0.691732in}}%
\pgfpathlineto{\pgfqpoint{0.677752in}{0.790669in}}%
\pgfusepath{stroke,fill}%
\end{pgfscope}%
\begin{pgfscope}%
\pgfpathrectangle{\pgfqpoint{0.380943in}{0.295988in}}{\pgfqpoint{4.650000in}{0.692553in}}%
\pgfusepath{clip}%
\pgfsetbuttcap%
\pgfsetroundjoin%
\definecolor{currentfill}{rgb}{0.961738,0.927612,0.725598}%
\pgfsetfillcolor{currentfill}%
\pgfsetlinewidth{0.250937pt}%
\definecolor{currentstroke}{rgb}{1.000000,1.000000,1.000000}%
\pgfsetstrokecolor{currentstroke}%
\pgfsetdash{}{0pt}%
\pgfpathmoveto{\pgfqpoint{0.776688in}{0.790669in}}%
\pgfpathlineto{\pgfqpoint{0.875624in}{0.790669in}}%
\pgfpathlineto{\pgfqpoint{0.875624in}{0.691732in}}%
\pgfpathlineto{\pgfqpoint{0.776688in}{0.691732in}}%
\pgfpathlineto{\pgfqpoint{0.776688in}{0.790669in}}%
\pgfusepath{stroke,fill}%
\end{pgfscope}%
\begin{pgfscope}%
\pgfpathrectangle{\pgfqpoint{0.380943in}{0.295988in}}{\pgfqpoint{4.650000in}{0.692553in}}%
\pgfusepath{clip}%
\pgfsetbuttcap%
\pgfsetroundjoin%
\definecolor{currentfill}{rgb}{0.960892,0.932687,0.728981}%
\pgfsetfillcolor{currentfill}%
\pgfsetlinewidth{0.250937pt}%
\definecolor{currentstroke}{rgb}{1.000000,1.000000,1.000000}%
\pgfsetstrokecolor{currentstroke}%
\pgfsetdash{}{0pt}%
\pgfpathmoveto{\pgfqpoint{0.875624in}{0.790669in}}%
\pgfpathlineto{\pgfqpoint{0.974560in}{0.790669in}}%
\pgfpathlineto{\pgfqpoint{0.974560in}{0.691732in}}%
\pgfpathlineto{\pgfqpoint{0.875624in}{0.691732in}}%
\pgfpathlineto{\pgfqpoint{0.875624in}{0.790669in}}%
\pgfusepath{stroke,fill}%
\end{pgfscope}%
\begin{pgfscope}%
\pgfpathrectangle{\pgfqpoint{0.380943in}{0.295988in}}{\pgfqpoint{4.650000in}{0.692553in}}%
\pgfusepath{clip}%
\pgfsetbuttcap%
\pgfsetroundjoin%
\definecolor{currentfill}{rgb}{0.962076,0.925582,0.724245}%
\pgfsetfillcolor{currentfill}%
\pgfsetlinewidth{0.250937pt}%
\definecolor{currentstroke}{rgb}{1.000000,1.000000,1.000000}%
\pgfsetstrokecolor{currentstroke}%
\pgfsetdash{}{0pt}%
\pgfpathmoveto{\pgfqpoint{0.974560in}{0.790669in}}%
\pgfpathlineto{\pgfqpoint{1.073496in}{0.790669in}}%
\pgfpathlineto{\pgfqpoint{1.073496in}{0.691732in}}%
\pgfpathlineto{\pgfqpoint{0.974560in}{0.691732in}}%
\pgfpathlineto{\pgfqpoint{0.974560in}{0.790669in}}%
\pgfusepath{stroke,fill}%
\end{pgfscope}%
\begin{pgfscope}%
\pgfpathrectangle{\pgfqpoint{0.380943in}{0.295988in}}{\pgfqpoint{4.650000in}{0.692553in}}%
\pgfusepath{clip}%
\pgfsetbuttcap%
\pgfsetroundjoin%
\definecolor{currentfill}{rgb}{0.968997,0.888351,0.702284}%
\pgfsetfillcolor{currentfill}%
\pgfsetlinewidth{0.250937pt}%
\definecolor{currentstroke}{rgb}{1.000000,1.000000,1.000000}%
\pgfsetstrokecolor{currentstroke}%
\pgfsetdash{}{0pt}%
\pgfpathmoveto{\pgfqpoint{1.073496in}{0.790669in}}%
\pgfpathlineto{\pgfqpoint{1.172432in}{0.790669in}}%
\pgfpathlineto{\pgfqpoint{1.172432in}{0.691732in}}%
\pgfpathlineto{\pgfqpoint{1.073496in}{0.691732in}}%
\pgfpathlineto{\pgfqpoint{1.073496in}{0.790669in}}%
\pgfusepath{stroke,fill}%
\end{pgfscope}%
\begin{pgfscope}%
\pgfpathrectangle{\pgfqpoint{0.380943in}{0.295988in}}{\pgfqpoint{4.650000in}{0.692553in}}%
\pgfusepath{clip}%
\pgfsetbuttcap%
\pgfsetroundjoin%
\definecolor{currentfill}{rgb}{0.963091,0.919493,0.720185}%
\pgfsetfillcolor{currentfill}%
\pgfsetlinewidth{0.250937pt}%
\definecolor{currentstroke}{rgb}{1.000000,1.000000,1.000000}%
\pgfsetstrokecolor{currentstroke}%
\pgfsetdash{}{0pt}%
\pgfpathmoveto{\pgfqpoint{1.172432in}{0.790669in}}%
\pgfpathlineto{\pgfqpoint{1.271369in}{0.790669in}}%
\pgfpathlineto{\pgfqpoint{1.271369in}{0.691732in}}%
\pgfpathlineto{\pgfqpoint{1.172432in}{0.691732in}}%
\pgfpathlineto{\pgfqpoint{1.172432in}{0.790669in}}%
\pgfusepath{stroke,fill}%
\end{pgfscope}%
\begin{pgfscope}%
\pgfpathrectangle{\pgfqpoint{0.380943in}{0.295988in}}{\pgfqpoint{4.650000in}{0.692553in}}%
\pgfusepath{clip}%
\pgfsetbuttcap%
\pgfsetroundjoin%
\definecolor{currentfill}{rgb}{0.922338,0.400769,0.400769}%
\pgfsetfillcolor{currentfill}%
\pgfsetlinewidth{0.250937pt}%
\definecolor{currentstroke}{rgb}{1.000000,1.000000,1.000000}%
\pgfsetstrokecolor{currentstroke}%
\pgfsetdash{}{0pt}%
\pgfpathmoveto{\pgfqpoint{1.271369in}{0.790669in}}%
\pgfpathlineto{\pgfqpoint{1.370305in}{0.790669in}}%
\pgfpathlineto{\pgfqpoint{1.370305in}{0.691732in}}%
\pgfpathlineto{\pgfqpoint{1.271369in}{0.691732in}}%
\pgfpathlineto{\pgfqpoint{1.271369in}{0.790669in}}%
\pgfusepath{stroke,fill}%
\end{pgfscope}%
\begin{pgfscope}%
\pgfpathrectangle{\pgfqpoint{0.380943in}{0.295988in}}{\pgfqpoint{4.650000in}{0.692553in}}%
\pgfusepath{clip}%
\pgfsetbuttcap%
\pgfsetroundjoin%
\definecolor{currentfill}{rgb}{0.998939,0.658962,0.556032}%
\pgfsetfillcolor{currentfill}%
\pgfsetlinewidth{0.250937pt}%
\definecolor{currentstroke}{rgb}{1.000000,1.000000,1.000000}%
\pgfsetstrokecolor{currentstroke}%
\pgfsetdash{}{0pt}%
\pgfpathmoveto{\pgfqpoint{1.370305in}{0.790669in}}%
\pgfpathlineto{\pgfqpoint{1.469241in}{0.790669in}}%
\pgfpathlineto{\pgfqpoint{1.469241in}{0.691732in}}%
\pgfpathlineto{\pgfqpoint{1.370305in}{0.691732in}}%
\pgfpathlineto{\pgfqpoint{1.370305in}{0.790669in}}%
\pgfusepath{stroke,fill}%
\end{pgfscope}%
\begin{pgfscope}%
\pgfpathrectangle{\pgfqpoint{0.380943in}{0.295988in}}{\pgfqpoint{4.650000in}{0.692553in}}%
\pgfusepath{clip}%
\pgfsetbuttcap%
\pgfsetroundjoin%
\definecolor{currentfill}{rgb}{0.989619,0.788235,0.628374}%
\pgfsetfillcolor{currentfill}%
\pgfsetlinewidth{0.250937pt}%
\definecolor{currentstroke}{rgb}{1.000000,1.000000,1.000000}%
\pgfsetstrokecolor{currentstroke}%
\pgfsetdash{}{0pt}%
\pgfpathmoveto{\pgfqpoint{1.469241in}{0.790669in}}%
\pgfpathlineto{\pgfqpoint{1.568177in}{0.790669in}}%
\pgfpathlineto{\pgfqpoint{1.568177in}{0.691732in}}%
\pgfpathlineto{\pgfqpoint{1.469241in}{0.691732in}}%
\pgfpathlineto{\pgfqpoint{1.469241in}{0.790669in}}%
\pgfusepath{stroke,fill}%
\end{pgfscope}%
\begin{pgfscope}%
\pgfpathrectangle{\pgfqpoint{0.380943in}{0.295988in}}{\pgfqpoint{4.650000in}{0.692553in}}%
\pgfusepath{clip}%
\pgfsetbuttcap%
\pgfsetroundjoin%
\definecolor{currentfill}{rgb}{0.843983,0.322414,0.322414}%
\pgfsetfillcolor{currentfill}%
\pgfsetlinewidth{0.250937pt}%
\definecolor{currentstroke}{rgb}{1.000000,1.000000,1.000000}%
\pgfsetstrokecolor{currentstroke}%
\pgfsetdash{}{0pt}%
\pgfpathmoveto{\pgfqpoint{1.568177in}{0.790669in}}%
\pgfpathlineto{\pgfqpoint{1.667113in}{0.790669in}}%
\pgfpathlineto{\pgfqpoint{1.667113in}{0.691732in}}%
\pgfpathlineto{\pgfqpoint{1.568177in}{0.691732in}}%
\pgfpathlineto{\pgfqpoint{1.568177in}{0.790669in}}%
\pgfusepath{stroke,fill}%
\end{pgfscope}%
\begin{pgfscope}%
\pgfpathrectangle{\pgfqpoint{0.380943in}{0.295988in}}{\pgfqpoint{4.650000in}{0.692553in}}%
\pgfusepath{clip}%
\pgfsetbuttcap%
\pgfsetroundjoin%
\definecolor{currentfill}{rgb}{1.000000,0.608612,0.532072}%
\pgfsetfillcolor{currentfill}%
\pgfsetlinewidth{0.250937pt}%
\definecolor{currentstroke}{rgb}{1.000000,1.000000,1.000000}%
\pgfsetstrokecolor{currentstroke}%
\pgfsetdash{}{0pt}%
\pgfpathmoveto{\pgfqpoint{1.667113in}{0.790669in}}%
\pgfpathlineto{\pgfqpoint{1.766049in}{0.790669in}}%
\pgfpathlineto{\pgfqpoint{1.766049in}{0.691732in}}%
\pgfpathlineto{\pgfqpoint{1.667113in}{0.691732in}}%
\pgfpathlineto{\pgfqpoint{1.667113in}{0.790669in}}%
\pgfusepath{stroke,fill}%
\end{pgfscope}%
\begin{pgfscope}%
\pgfpathrectangle{\pgfqpoint{0.380943in}{0.295988in}}{\pgfqpoint{4.650000in}{0.692553in}}%
\pgfusepath{clip}%
\pgfsetbuttcap%
\pgfsetroundjoin%
\definecolor{currentfill}{rgb}{1.000000,0.608612,0.532072}%
\pgfsetfillcolor{currentfill}%
\pgfsetlinewidth{0.250937pt}%
\definecolor{currentstroke}{rgb}{1.000000,1.000000,1.000000}%
\pgfsetstrokecolor{currentstroke}%
\pgfsetdash{}{0pt}%
\pgfpathmoveto{\pgfqpoint{1.766049in}{0.790669in}}%
\pgfpathlineto{\pgfqpoint{1.864986in}{0.790669in}}%
\pgfpathlineto{\pgfqpoint{1.864986in}{0.691732in}}%
\pgfpathlineto{\pgfqpoint{1.766049in}{0.691732in}}%
\pgfpathlineto{\pgfqpoint{1.766049in}{0.790669in}}%
\pgfusepath{stroke,fill}%
\end{pgfscope}%
\begin{pgfscope}%
\pgfpathrectangle{\pgfqpoint{0.380943in}{0.295988in}}{\pgfqpoint{4.650000in}{0.692553in}}%
\pgfusepath{clip}%
\pgfsetbuttcap%
\pgfsetroundjoin%
\definecolor{currentfill}{rgb}{0.990296,0.782484,0.624990}%
\pgfsetfillcolor{currentfill}%
\pgfsetlinewidth{0.250937pt}%
\definecolor{currentstroke}{rgb}{1.000000,1.000000,1.000000}%
\pgfsetstrokecolor{currentstroke}%
\pgfsetdash{}{0pt}%
\pgfpathmoveto{\pgfqpoint{1.864986in}{0.790669in}}%
\pgfpathlineto{\pgfqpoint{1.963922in}{0.790669in}}%
\pgfpathlineto{\pgfqpoint{1.963922in}{0.691732in}}%
\pgfpathlineto{\pgfqpoint{1.864986in}{0.691732in}}%
\pgfpathlineto{\pgfqpoint{1.864986in}{0.790669in}}%
\pgfusepath{stroke,fill}%
\end{pgfscope}%
\begin{pgfscope}%
\pgfpathrectangle{\pgfqpoint{0.380943in}{0.295988in}}{\pgfqpoint{4.650000in}{0.692553in}}%
\pgfusepath{clip}%
\pgfsetbuttcap%
\pgfsetroundjoin%
\definecolor{currentfill}{rgb}{0.993003,0.759477,0.611457}%
\pgfsetfillcolor{currentfill}%
\pgfsetlinewidth{0.250937pt}%
\definecolor{currentstroke}{rgb}{1.000000,1.000000,1.000000}%
\pgfsetstrokecolor{currentstroke}%
\pgfsetdash{}{0pt}%
\pgfpathmoveto{\pgfqpoint{1.963922in}{0.790669in}}%
\pgfpathlineto{\pgfqpoint{2.062858in}{0.790669in}}%
\pgfpathlineto{\pgfqpoint{2.062858in}{0.691732in}}%
\pgfpathlineto{\pgfqpoint{1.963922in}{0.691732in}}%
\pgfpathlineto{\pgfqpoint{1.963922in}{0.790669in}}%
\pgfusepath{stroke,fill}%
\end{pgfscope}%
\begin{pgfscope}%
\pgfpathrectangle{\pgfqpoint{0.380943in}{0.295988in}}{\pgfqpoint{4.650000in}{0.692553in}}%
\pgfusepath{clip}%
\pgfsetbuttcap%
\pgfsetroundjoin%
\definecolor{currentfill}{rgb}{0.994018,0.750850,0.606382}%
\pgfsetfillcolor{currentfill}%
\pgfsetlinewidth{0.250937pt}%
\definecolor{currentstroke}{rgb}{1.000000,1.000000,1.000000}%
\pgfsetstrokecolor{currentstroke}%
\pgfsetdash{}{0pt}%
\pgfpathmoveto{\pgfqpoint{2.062858in}{0.790669in}}%
\pgfpathlineto{\pgfqpoint{2.161794in}{0.790669in}}%
\pgfpathlineto{\pgfqpoint{2.161794in}{0.691732in}}%
\pgfpathlineto{\pgfqpoint{2.062858in}{0.691732in}}%
\pgfpathlineto{\pgfqpoint{2.062858in}{0.790669in}}%
\pgfusepath{stroke,fill}%
\end{pgfscope}%
\begin{pgfscope}%
\pgfpathrectangle{\pgfqpoint{0.380943in}{0.295988in}}{\pgfqpoint{4.650000in}{0.692553in}}%
\pgfusepath{clip}%
\pgfsetbuttcap%
\pgfsetroundjoin%
\definecolor{currentfill}{rgb}{0.998093,0.680953,0.567874}%
\pgfsetfillcolor{currentfill}%
\pgfsetlinewidth{0.250937pt}%
\definecolor{currentstroke}{rgb}{1.000000,1.000000,1.000000}%
\pgfsetstrokecolor{currentstroke}%
\pgfsetdash{}{0pt}%
\pgfpathmoveto{\pgfqpoint{2.161794in}{0.790669in}}%
\pgfpathlineto{\pgfqpoint{2.260730in}{0.790669in}}%
\pgfpathlineto{\pgfqpoint{2.260730in}{0.691732in}}%
\pgfpathlineto{\pgfqpoint{2.161794in}{0.691732in}}%
\pgfpathlineto{\pgfqpoint{2.161794in}{0.790669in}}%
\pgfusepath{stroke,fill}%
\end{pgfscope}%
\begin{pgfscope}%
\pgfpathrectangle{\pgfqpoint{0.380943in}{0.295988in}}{\pgfqpoint{4.650000in}{0.692553in}}%
\pgfusepath{clip}%
\pgfsetbuttcap%
\pgfsetroundjoin%
\definecolor{currentfill}{rgb}{0.988604,0.796863,0.633449}%
\pgfsetfillcolor{currentfill}%
\pgfsetlinewidth{0.250937pt}%
\definecolor{currentstroke}{rgb}{1.000000,1.000000,1.000000}%
\pgfsetstrokecolor{currentstroke}%
\pgfsetdash{}{0pt}%
\pgfpathmoveto{\pgfqpoint{2.260730in}{0.790669in}}%
\pgfpathlineto{\pgfqpoint{2.359666in}{0.790669in}}%
\pgfpathlineto{\pgfqpoint{2.359666in}{0.691732in}}%
\pgfpathlineto{\pgfqpoint{2.260730in}{0.691732in}}%
\pgfpathlineto{\pgfqpoint{2.260730in}{0.790669in}}%
\pgfusepath{stroke,fill}%
\end{pgfscope}%
\begin{pgfscope}%
\pgfpathrectangle{\pgfqpoint{0.380943in}{0.295988in}}{\pgfqpoint{4.650000in}{0.692553in}}%
\pgfusepath{clip}%
\pgfsetbuttcap%
\pgfsetroundjoin%
\definecolor{currentfill}{rgb}{1.000000,0.588312,0.523952}%
\pgfsetfillcolor{currentfill}%
\pgfsetlinewidth{0.250937pt}%
\definecolor{currentstroke}{rgb}{1.000000,1.000000,1.000000}%
\pgfsetstrokecolor{currentstroke}%
\pgfsetdash{}{0pt}%
\pgfpathmoveto{\pgfqpoint{2.359666in}{0.790669in}}%
\pgfpathlineto{\pgfqpoint{2.458603in}{0.790669in}}%
\pgfpathlineto{\pgfqpoint{2.458603in}{0.691732in}}%
\pgfpathlineto{\pgfqpoint{2.359666in}{0.691732in}}%
\pgfpathlineto{\pgfqpoint{2.359666in}{0.790669in}}%
\pgfusepath{stroke,fill}%
\end{pgfscope}%
\begin{pgfscope}%
\pgfpathrectangle{\pgfqpoint{0.380943in}{0.295988in}}{\pgfqpoint{4.650000in}{0.692553in}}%
\pgfusepath{clip}%
\pgfsetbuttcap%
\pgfsetroundjoin%
\definecolor{currentfill}{rgb}{1.000000,0.554479,0.510419}%
\pgfsetfillcolor{currentfill}%
\pgfsetlinewidth{0.250937pt}%
\definecolor{currentstroke}{rgb}{1.000000,1.000000,1.000000}%
\pgfsetstrokecolor{currentstroke}%
\pgfsetdash{}{0pt}%
\pgfpathmoveto{\pgfqpoint{2.458603in}{0.790669in}}%
\pgfpathlineto{\pgfqpoint{2.557539in}{0.790669in}}%
\pgfpathlineto{\pgfqpoint{2.557539in}{0.691732in}}%
\pgfpathlineto{\pgfqpoint{2.458603in}{0.691732in}}%
\pgfpathlineto{\pgfqpoint{2.458603in}{0.790669in}}%
\pgfusepath{stroke,fill}%
\end{pgfscope}%
\begin{pgfscope}%
\pgfpathrectangle{\pgfqpoint{0.380943in}{0.295988in}}{\pgfqpoint{4.650000in}{0.692553in}}%
\pgfusepath{clip}%
\pgfsetbuttcap%
\pgfsetroundjoin%
\definecolor{currentfill}{rgb}{0.999785,0.636970,0.544191}%
\pgfsetfillcolor{currentfill}%
\pgfsetlinewidth{0.250937pt}%
\definecolor{currentstroke}{rgb}{1.000000,1.000000,1.000000}%
\pgfsetstrokecolor{currentstroke}%
\pgfsetdash{}{0pt}%
\pgfpathmoveto{\pgfqpoint{2.557539in}{0.790669in}}%
\pgfpathlineto{\pgfqpoint{2.656475in}{0.790669in}}%
\pgfpathlineto{\pgfqpoint{2.656475in}{0.691732in}}%
\pgfpathlineto{\pgfqpoint{2.557539in}{0.691732in}}%
\pgfpathlineto{\pgfqpoint{2.557539in}{0.790669in}}%
\pgfusepath{stroke,fill}%
\end{pgfscope}%
\begin{pgfscope}%
\pgfpathrectangle{\pgfqpoint{0.380943in}{0.295988in}}{\pgfqpoint{4.650000in}{0.692553in}}%
\pgfusepath{clip}%
\pgfsetbuttcap%
\pgfsetroundjoin%
\definecolor{currentfill}{rgb}{1.000000,0.571396,0.517186}%
\pgfsetfillcolor{currentfill}%
\pgfsetlinewidth{0.250937pt}%
\definecolor{currentstroke}{rgb}{1.000000,1.000000,1.000000}%
\pgfsetstrokecolor{currentstroke}%
\pgfsetdash{}{0pt}%
\pgfpathmoveto{\pgfqpoint{2.656475in}{0.790669in}}%
\pgfpathlineto{\pgfqpoint{2.755411in}{0.790669in}}%
\pgfpathlineto{\pgfqpoint{2.755411in}{0.691732in}}%
\pgfpathlineto{\pgfqpoint{2.656475in}{0.691732in}}%
\pgfpathlineto{\pgfqpoint{2.656475in}{0.790669in}}%
\pgfusepath{stroke,fill}%
\end{pgfscope}%
\begin{pgfscope}%
\pgfpathrectangle{\pgfqpoint{0.380943in}{0.295988in}}{\pgfqpoint{4.650000in}{0.692553in}}%
\pgfusepath{clip}%
\pgfsetbuttcap%
\pgfsetroundjoin%
\definecolor{currentfill}{rgb}{0.995709,0.736471,0.597924}%
\pgfsetfillcolor{currentfill}%
\pgfsetlinewidth{0.250937pt}%
\definecolor{currentstroke}{rgb}{1.000000,1.000000,1.000000}%
\pgfsetstrokecolor{currentstroke}%
\pgfsetdash{}{0pt}%
\pgfpathmoveto{\pgfqpoint{2.755411in}{0.790669in}}%
\pgfpathlineto{\pgfqpoint{2.854347in}{0.790669in}}%
\pgfpathlineto{\pgfqpoint{2.854347in}{0.691732in}}%
\pgfpathlineto{\pgfqpoint{2.755411in}{0.691732in}}%
\pgfpathlineto{\pgfqpoint{2.755411in}{0.790669in}}%
\pgfusepath{stroke,fill}%
\end{pgfscope}%
\begin{pgfscope}%
\pgfpathrectangle{\pgfqpoint{0.380943in}{0.295988in}}{\pgfqpoint{4.650000in}{0.692553in}}%
\pgfusepath{clip}%
\pgfsetbuttcap%
\pgfsetroundjoin%
\definecolor{currentfill}{rgb}{0.990296,0.782484,0.624990}%
\pgfsetfillcolor{currentfill}%
\pgfsetlinewidth{0.250937pt}%
\definecolor{currentstroke}{rgb}{1.000000,1.000000,1.000000}%
\pgfsetstrokecolor{currentstroke}%
\pgfsetdash{}{0pt}%
\pgfpathmoveto{\pgfqpoint{2.854347in}{0.790669in}}%
\pgfpathlineto{\pgfqpoint{2.953283in}{0.790669in}}%
\pgfpathlineto{\pgfqpoint{2.953283in}{0.691732in}}%
\pgfpathlineto{\pgfqpoint{2.854347in}{0.691732in}}%
\pgfpathlineto{\pgfqpoint{2.854347in}{0.790669in}}%
\pgfusepath{stroke,fill}%
\end{pgfscope}%
\begin{pgfscope}%
\pgfpathrectangle{\pgfqpoint{0.380943in}{0.295988in}}{\pgfqpoint{4.650000in}{0.692553in}}%
\pgfusepath{clip}%
\pgfsetbuttcap%
\pgfsetroundjoin%
\definecolor{currentfill}{rgb}{0.994018,0.750850,0.606382}%
\pgfsetfillcolor{currentfill}%
\pgfsetlinewidth{0.250937pt}%
\definecolor{currentstroke}{rgb}{1.000000,1.000000,1.000000}%
\pgfsetstrokecolor{currentstroke}%
\pgfsetdash{}{0pt}%
\pgfpathmoveto{\pgfqpoint{2.953283in}{0.790669in}}%
\pgfpathlineto{\pgfqpoint{3.052220in}{0.790669in}}%
\pgfpathlineto{\pgfqpoint{3.052220in}{0.691732in}}%
\pgfpathlineto{\pgfqpoint{2.953283in}{0.691732in}}%
\pgfpathlineto{\pgfqpoint{2.953283in}{0.790669in}}%
\pgfusepath{stroke,fill}%
\end{pgfscope}%
\begin{pgfscope}%
\pgfpathrectangle{\pgfqpoint{0.380943in}{0.295988in}}{\pgfqpoint{4.650000in}{0.692553in}}%
\pgfusepath{clip}%
\pgfsetbuttcap%
\pgfsetroundjoin%
\definecolor{currentfill}{rgb}{0.997586,0.694148,0.574979}%
\pgfsetfillcolor{currentfill}%
\pgfsetlinewidth{0.250937pt}%
\definecolor{currentstroke}{rgb}{1.000000,1.000000,1.000000}%
\pgfsetstrokecolor{currentstroke}%
\pgfsetdash{}{0pt}%
\pgfpathmoveto{\pgfqpoint{3.052220in}{0.790669in}}%
\pgfpathlineto{\pgfqpoint{3.151156in}{0.790669in}}%
\pgfpathlineto{\pgfqpoint{3.151156in}{0.691732in}}%
\pgfpathlineto{\pgfqpoint{3.052220in}{0.691732in}}%
\pgfpathlineto{\pgfqpoint{3.052220in}{0.790669in}}%
\pgfusepath{stroke,fill}%
\end{pgfscope}%
\begin{pgfscope}%
\pgfpathrectangle{\pgfqpoint{0.380943in}{0.295988in}}{\pgfqpoint{4.650000in}{0.692553in}}%
\pgfusepath{clip}%
\pgfsetbuttcap%
\pgfsetroundjoin%
\definecolor{currentfill}{rgb}{0.964937,0.908651,0.713110}%
\pgfsetfillcolor{currentfill}%
\pgfsetlinewidth{0.250937pt}%
\definecolor{currentstroke}{rgb}{1.000000,1.000000,1.000000}%
\pgfsetstrokecolor{currentstroke}%
\pgfsetdash{}{0pt}%
\pgfpathmoveto{\pgfqpoint{3.151156in}{0.790669in}}%
\pgfpathlineto{\pgfqpoint{3.250092in}{0.790669in}}%
\pgfpathlineto{\pgfqpoint{3.250092in}{0.691732in}}%
\pgfpathlineto{\pgfqpoint{3.151156in}{0.691732in}}%
\pgfpathlineto{\pgfqpoint{3.151156in}{0.790669in}}%
\pgfusepath{stroke,fill}%
\end{pgfscope}%
\begin{pgfscope}%
\pgfpathrectangle{\pgfqpoint{0.380943in}{0.295988in}}{\pgfqpoint{4.650000in}{0.692553in}}%
\pgfusepath{clip}%
\pgfsetbuttcap%
\pgfsetroundjoin%
\definecolor{currentfill}{rgb}{0.993003,0.759477,0.611457}%
\pgfsetfillcolor{currentfill}%
\pgfsetlinewidth{0.250937pt}%
\definecolor{currentstroke}{rgb}{1.000000,1.000000,1.000000}%
\pgfsetstrokecolor{currentstroke}%
\pgfsetdash{}{0pt}%
\pgfpathmoveto{\pgfqpoint{3.250092in}{0.790669in}}%
\pgfpathlineto{\pgfqpoint{3.349028in}{0.790669in}}%
\pgfpathlineto{\pgfqpoint{3.349028in}{0.691732in}}%
\pgfpathlineto{\pgfqpoint{3.250092in}{0.691732in}}%
\pgfpathlineto{\pgfqpoint{3.250092in}{0.790669in}}%
\pgfusepath{stroke,fill}%
\end{pgfscope}%
\begin{pgfscope}%
\pgfpathrectangle{\pgfqpoint{0.380943in}{0.295988in}}{\pgfqpoint{4.650000in}{0.692553in}}%
\pgfusepath{clip}%
\pgfsetbuttcap%
\pgfsetroundjoin%
\definecolor{currentfill}{rgb}{1.000000,0.538331,0.503652}%
\pgfsetfillcolor{currentfill}%
\pgfsetlinewidth{0.250937pt}%
\definecolor{currentstroke}{rgb}{1.000000,1.000000,1.000000}%
\pgfsetstrokecolor{currentstroke}%
\pgfsetdash{}{0pt}%
\pgfpathmoveto{\pgfqpoint{3.349028in}{0.790669in}}%
\pgfpathlineto{\pgfqpoint{3.447964in}{0.790669in}}%
\pgfpathlineto{\pgfqpoint{3.447964in}{0.691732in}}%
\pgfpathlineto{\pgfqpoint{3.349028in}{0.691732in}}%
\pgfpathlineto{\pgfqpoint{3.349028in}{0.790669in}}%
\pgfusepath{stroke,fill}%
\end{pgfscope}%
\begin{pgfscope}%
\pgfpathrectangle{\pgfqpoint{0.380943in}{0.295988in}}{\pgfqpoint{4.650000in}{0.692553in}}%
\pgfusepath{clip}%
\pgfsetbuttcap%
\pgfsetroundjoin%
\definecolor{currentfill}{rgb}{0.977316,0.455748,0.455748}%
\pgfsetfillcolor{currentfill}%
\pgfsetlinewidth{0.250937pt}%
\definecolor{currentstroke}{rgb}{1.000000,1.000000,1.000000}%
\pgfsetstrokecolor{currentstroke}%
\pgfsetdash{}{0pt}%
\pgfpathmoveto{\pgfqpoint{3.447964in}{0.790669in}}%
\pgfpathlineto{\pgfqpoint{3.546901in}{0.790669in}}%
\pgfpathlineto{\pgfqpoint{3.546901in}{0.691732in}}%
\pgfpathlineto{\pgfqpoint{3.447964in}{0.691732in}}%
\pgfpathlineto{\pgfqpoint{3.447964in}{0.790669in}}%
\pgfusepath{stroke,fill}%
\end{pgfscope}%
\begin{pgfscope}%
\pgfpathrectangle{\pgfqpoint{0.380943in}{0.295988in}}{\pgfqpoint{4.650000in}{0.692553in}}%
\pgfusepath{clip}%
\pgfsetbuttcap%
\pgfsetroundjoin%
\definecolor{currentfill}{rgb}{1.000000,0.615379,0.534779}%
\pgfsetfillcolor{currentfill}%
\pgfsetlinewidth{0.250937pt}%
\definecolor{currentstroke}{rgb}{1.000000,1.000000,1.000000}%
\pgfsetstrokecolor{currentstroke}%
\pgfsetdash{}{0pt}%
\pgfpathmoveto{\pgfqpoint{3.546901in}{0.790669in}}%
\pgfpathlineto{\pgfqpoint{3.645837in}{0.790669in}}%
\pgfpathlineto{\pgfqpoint{3.645837in}{0.691732in}}%
\pgfpathlineto{\pgfqpoint{3.546901in}{0.691732in}}%
\pgfpathlineto{\pgfqpoint{3.546901in}{0.790669in}}%
\pgfusepath{stroke,fill}%
\end{pgfscope}%
\begin{pgfscope}%
\pgfpathrectangle{\pgfqpoint{0.380943in}{0.295988in}}{\pgfqpoint{4.650000in}{0.692553in}}%
\pgfusepath{clip}%
\pgfsetbuttcap%
\pgfsetroundjoin%
\definecolor{currentfill}{rgb}{0.999785,0.636970,0.544191}%
\pgfsetfillcolor{currentfill}%
\pgfsetlinewidth{0.250937pt}%
\definecolor{currentstroke}{rgb}{1.000000,1.000000,1.000000}%
\pgfsetstrokecolor{currentstroke}%
\pgfsetdash{}{0pt}%
\pgfpathmoveto{\pgfqpoint{3.645837in}{0.790669in}}%
\pgfpathlineto{\pgfqpoint{3.744773in}{0.790669in}}%
\pgfpathlineto{\pgfqpoint{3.744773in}{0.691732in}}%
\pgfpathlineto{\pgfqpoint{3.645837in}{0.691732in}}%
\pgfpathlineto{\pgfqpoint{3.645837in}{0.790669in}}%
\pgfusepath{stroke,fill}%
\end{pgfscope}%
\begin{pgfscope}%
\pgfpathrectangle{\pgfqpoint{0.380943in}{0.295988in}}{\pgfqpoint{4.650000in}{0.692553in}}%
\pgfusepath{clip}%
\pgfsetbuttcap%
\pgfsetroundjoin%
\definecolor{currentfill}{rgb}{0.995709,0.736471,0.597924}%
\pgfsetfillcolor{currentfill}%
\pgfsetlinewidth{0.250937pt}%
\definecolor{currentstroke}{rgb}{1.000000,1.000000,1.000000}%
\pgfsetstrokecolor{currentstroke}%
\pgfsetdash{}{0pt}%
\pgfpathmoveto{\pgfqpoint{3.744773in}{0.790669in}}%
\pgfpathlineto{\pgfqpoint{3.843709in}{0.790669in}}%
\pgfpathlineto{\pgfqpoint{3.843709in}{0.691732in}}%
\pgfpathlineto{\pgfqpoint{3.744773in}{0.691732in}}%
\pgfpathlineto{\pgfqpoint{3.744773in}{0.790669in}}%
\pgfusepath{stroke,fill}%
\end{pgfscope}%
\begin{pgfscope}%
\pgfpathrectangle{\pgfqpoint{0.380943in}{0.295988in}}{\pgfqpoint{4.650000in}{0.692553in}}%
\pgfusepath{clip}%
\pgfsetbuttcap%
\pgfsetroundjoin%
\definecolor{currentfill}{rgb}{1.000000,0.480477,0.479293}%
\pgfsetfillcolor{currentfill}%
\pgfsetlinewidth{0.250937pt}%
\definecolor{currentstroke}{rgb}{1.000000,1.000000,1.000000}%
\pgfsetstrokecolor{currentstroke}%
\pgfsetdash{}{0pt}%
\pgfpathmoveto{\pgfqpoint{3.843709in}{0.790669in}}%
\pgfpathlineto{\pgfqpoint{3.942645in}{0.790669in}}%
\pgfpathlineto{\pgfqpoint{3.942645in}{0.691732in}}%
\pgfpathlineto{\pgfqpoint{3.843709in}{0.691732in}}%
\pgfpathlineto{\pgfqpoint{3.843709in}{0.790669in}}%
\pgfusepath{stroke,fill}%
\end{pgfscope}%
\begin{pgfscope}%
\pgfpathrectangle{\pgfqpoint{0.380943in}{0.295988in}}{\pgfqpoint{4.650000in}{0.692553in}}%
\pgfusepath{clip}%
\pgfsetbuttcap%
\pgfsetroundjoin%
\definecolor{currentfill}{rgb}{1.000000,0.522261,0.496886}%
\pgfsetfillcolor{currentfill}%
\pgfsetlinewidth{0.250937pt}%
\definecolor{currentstroke}{rgb}{1.000000,1.000000,1.000000}%
\pgfsetstrokecolor{currentstroke}%
\pgfsetdash{}{0pt}%
\pgfpathmoveto{\pgfqpoint{3.942645in}{0.790669in}}%
\pgfpathlineto{\pgfqpoint{4.041581in}{0.790669in}}%
\pgfpathlineto{\pgfqpoint{4.041581in}{0.691732in}}%
\pgfpathlineto{\pgfqpoint{3.942645in}{0.691732in}}%
\pgfpathlineto{\pgfqpoint{3.942645in}{0.790669in}}%
\pgfusepath{stroke,fill}%
\end{pgfscope}%
\begin{pgfscope}%
\pgfpathrectangle{\pgfqpoint{0.380943in}{0.295988in}}{\pgfqpoint{4.650000in}{0.692553in}}%
\pgfusepath{clip}%
\pgfsetbuttcap%
\pgfsetroundjoin%
\definecolor{currentfill}{rgb}{0.995709,0.736471,0.597924}%
\pgfsetfillcolor{currentfill}%
\pgfsetlinewidth{0.250937pt}%
\definecolor{currentstroke}{rgb}{1.000000,1.000000,1.000000}%
\pgfsetstrokecolor{currentstroke}%
\pgfsetdash{}{0pt}%
\pgfpathmoveto{\pgfqpoint{4.041581in}{0.790669in}}%
\pgfpathlineto{\pgfqpoint{4.140518in}{0.790669in}}%
\pgfpathlineto{\pgfqpoint{4.140518in}{0.691732in}}%
\pgfpathlineto{\pgfqpoint{4.041581in}{0.691732in}}%
\pgfpathlineto{\pgfqpoint{4.041581in}{0.790669in}}%
\pgfusepath{stroke,fill}%
\end{pgfscope}%
\begin{pgfscope}%
\pgfpathrectangle{\pgfqpoint{0.380943in}{0.295988in}}{\pgfqpoint{4.650000in}{0.692553in}}%
\pgfusepath{clip}%
\pgfsetbuttcap%
\pgfsetroundjoin%
\definecolor{currentfill}{rgb}{1.000000,0.554479,0.510419}%
\pgfsetfillcolor{currentfill}%
\pgfsetlinewidth{0.250937pt}%
\definecolor{currentstroke}{rgb}{1.000000,1.000000,1.000000}%
\pgfsetstrokecolor{currentstroke}%
\pgfsetdash{}{0pt}%
\pgfpathmoveto{\pgfqpoint{4.140518in}{0.790669in}}%
\pgfpathlineto{\pgfqpoint{4.239454in}{0.790669in}}%
\pgfpathlineto{\pgfqpoint{4.239454in}{0.691732in}}%
\pgfpathlineto{\pgfqpoint{4.140518in}{0.691732in}}%
\pgfpathlineto{\pgfqpoint{4.140518in}{0.790669in}}%
\pgfusepath{stroke,fill}%
\end{pgfscope}%
\begin{pgfscope}%
\pgfpathrectangle{\pgfqpoint{0.380943in}{0.295988in}}{\pgfqpoint{4.650000in}{0.692553in}}%
\pgfusepath{clip}%
\pgfsetbuttcap%
\pgfsetroundjoin%
\definecolor{currentfill}{rgb}{0.922338,0.400769,0.400769}%
\pgfsetfillcolor{currentfill}%
\pgfsetlinewidth{0.250937pt}%
\definecolor{currentstroke}{rgb}{1.000000,1.000000,1.000000}%
\pgfsetstrokecolor{currentstroke}%
\pgfsetdash{}{0pt}%
\pgfpathmoveto{\pgfqpoint{4.239454in}{0.790669in}}%
\pgfpathlineto{\pgfqpoint{4.338390in}{0.790669in}}%
\pgfpathlineto{\pgfqpoint{4.338390in}{0.691732in}}%
\pgfpathlineto{\pgfqpoint{4.239454in}{0.691732in}}%
\pgfpathlineto{\pgfqpoint{4.239454in}{0.790669in}}%
\pgfusepath{stroke,fill}%
\end{pgfscope}%
\begin{pgfscope}%
\pgfpathrectangle{\pgfqpoint{0.380943in}{0.295988in}}{\pgfqpoint{4.650000in}{0.692553in}}%
\pgfusepath{clip}%
\pgfsetbuttcap%
\pgfsetroundjoin%
\definecolor{currentfill}{rgb}{0.967474,0.895963,0.706344}%
\pgfsetfillcolor{currentfill}%
\pgfsetlinewidth{0.250937pt}%
\definecolor{currentstroke}{rgb}{1.000000,1.000000,1.000000}%
\pgfsetstrokecolor{currentstroke}%
\pgfsetdash{}{0pt}%
\pgfpathmoveto{\pgfqpoint{4.338390in}{0.790669in}}%
\pgfpathlineto{\pgfqpoint{4.437326in}{0.790669in}}%
\pgfpathlineto{\pgfqpoint{4.437326in}{0.691732in}}%
\pgfpathlineto{\pgfqpoint{4.338390in}{0.691732in}}%
\pgfpathlineto{\pgfqpoint{4.338390in}{0.790669in}}%
\pgfusepath{stroke,fill}%
\end{pgfscope}%
\begin{pgfscope}%
\pgfpathrectangle{\pgfqpoint{0.380943in}{0.295988in}}{\pgfqpoint{4.650000in}{0.692553in}}%
\pgfusepath{clip}%
\pgfsetbuttcap%
\pgfsetroundjoin%
\definecolor{currentfill}{rgb}{0.980669,0.832787,0.665559}%
\pgfsetfillcolor{currentfill}%
\pgfsetlinewidth{0.250937pt}%
\definecolor{currentstroke}{rgb}{1.000000,1.000000,1.000000}%
\pgfsetstrokecolor{currentstroke}%
\pgfsetdash{}{0pt}%
\pgfpathmoveto{\pgfqpoint{4.437326in}{0.790669in}}%
\pgfpathlineto{\pgfqpoint{4.536262in}{0.790669in}}%
\pgfpathlineto{\pgfqpoint{4.536262in}{0.691732in}}%
\pgfpathlineto{\pgfqpoint{4.437326in}{0.691732in}}%
\pgfpathlineto{\pgfqpoint{4.437326in}{0.790669in}}%
\pgfusepath{stroke,fill}%
\end{pgfscope}%
\begin{pgfscope}%
\pgfpathrectangle{\pgfqpoint{0.380943in}{0.295988in}}{\pgfqpoint{4.650000in}{0.692553in}}%
\pgfusepath{clip}%
\pgfsetbuttcap%
\pgfsetroundjoin%
\definecolor{currentfill}{rgb}{0.963429,0.917463,0.718831}%
\pgfsetfillcolor{currentfill}%
\pgfsetlinewidth{0.250937pt}%
\definecolor{currentstroke}{rgb}{1.000000,1.000000,1.000000}%
\pgfsetstrokecolor{currentstroke}%
\pgfsetdash{}{0pt}%
\pgfpathmoveto{\pgfqpoint{4.536262in}{0.790669in}}%
\pgfpathlineto{\pgfqpoint{4.635198in}{0.790669in}}%
\pgfpathlineto{\pgfqpoint{4.635198in}{0.691732in}}%
\pgfpathlineto{\pgfqpoint{4.536262in}{0.691732in}}%
\pgfpathlineto{\pgfqpoint{4.536262in}{0.790669in}}%
\pgfusepath{stroke,fill}%
\end{pgfscope}%
\begin{pgfscope}%
\pgfpathrectangle{\pgfqpoint{0.380943in}{0.295988in}}{\pgfqpoint{4.650000in}{0.692553in}}%
\pgfusepath{clip}%
\pgfsetbuttcap%
\pgfsetroundjoin%
\definecolor{currentfill}{rgb}{0.963429,0.917463,0.718831}%
\pgfsetfillcolor{currentfill}%
\pgfsetlinewidth{0.250937pt}%
\definecolor{currentstroke}{rgb}{1.000000,1.000000,1.000000}%
\pgfsetstrokecolor{currentstroke}%
\pgfsetdash{}{0pt}%
\pgfpathmoveto{\pgfqpoint{4.635198in}{0.790669in}}%
\pgfpathlineto{\pgfqpoint{4.734135in}{0.790669in}}%
\pgfpathlineto{\pgfqpoint{4.734135in}{0.691732in}}%
\pgfpathlineto{\pgfqpoint{4.635198in}{0.691732in}}%
\pgfpathlineto{\pgfqpoint{4.635198in}{0.790669in}}%
\pgfusepath{stroke,fill}%
\end{pgfscope}%
\begin{pgfscope}%
\pgfpathrectangle{\pgfqpoint{0.380943in}{0.295988in}}{\pgfqpoint{4.650000in}{0.692553in}}%
\pgfusepath{clip}%
\pgfsetbuttcap%
\pgfsetroundjoin%
\definecolor{currentfill}{rgb}{0.963429,0.917463,0.718831}%
\pgfsetfillcolor{currentfill}%
\pgfsetlinewidth{0.250937pt}%
\definecolor{currentstroke}{rgb}{1.000000,1.000000,1.000000}%
\pgfsetstrokecolor{currentstroke}%
\pgfsetdash{}{0pt}%
\pgfpathmoveto{\pgfqpoint{4.734135in}{0.790669in}}%
\pgfpathlineto{\pgfqpoint{4.833071in}{0.790669in}}%
\pgfpathlineto{\pgfqpoint{4.833071in}{0.691732in}}%
\pgfpathlineto{\pgfqpoint{4.734135in}{0.691732in}}%
\pgfpathlineto{\pgfqpoint{4.734135in}{0.790669in}}%
\pgfusepath{stroke,fill}%
\end{pgfscope}%
\begin{pgfscope}%
\pgfpathrectangle{\pgfqpoint{0.380943in}{0.295988in}}{\pgfqpoint{4.650000in}{0.692553in}}%
\pgfusepath{clip}%
\pgfsetbuttcap%
\pgfsetroundjoin%
\definecolor{currentfill}{rgb}{0.963091,0.919493,0.720185}%
\pgfsetfillcolor{currentfill}%
\pgfsetlinewidth{0.250937pt}%
\definecolor{currentstroke}{rgb}{1.000000,1.000000,1.000000}%
\pgfsetstrokecolor{currentstroke}%
\pgfsetdash{}{0pt}%
\pgfpathmoveto{\pgfqpoint{4.833071in}{0.790669in}}%
\pgfpathlineto{\pgfqpoint{4.932007in}{0.790669in}}%
\pgfpathlineto{\pgfqpoint{4.932007in}{0.691732in}}%
\pgfpathlineto{\pgfqpoint{4.833071in}{0.691732in}}%
\pgfpathlineto{\pgfqpoint{4.833071in}{0.790669in}}%
\pgfusepath{stroke,fill}%
\end{pgfscope}%
\begin{pgfscope}%
\pgfpathrectangle{\pgfqpoint{0.380943in}{0.295988in}}{\pgfqpoint{4.650000in}{0.692553in}}%
\pgfusepath{clip}%
\pgfsetbuttcap%
\pgfsetroundjoin%
\pgfsetlinewidth{0.250937pt}%
\definecolor{currentstroke}{rgb}{1.000000,1.000000,1.000000}%
\pgfsetstrokecolor{currentstroke}%
\pgfsetdash{}{0pt}%
\pgfpathmoveto{\pgfqpoint{4.932007in}{0.790669in}}%
\pgfpathlineto{\pgfqpoint{5.030943in}{0.790669in}}%
\pgfpathlineto{\pgfqpoint{5.030943in}{0.691732in}}%
\pgfpathlineto{\pgfqpoint{4.932007in}{0.691732in}}%
\pgfpathlineto{\pgfqpoint{4.932007in}{0.790669in}}%
\pgfusepath{stroke}%
\end{pgfscope}%
\begin{pgfscope}%
\pgfpathrectangle{\pgfqpoint{0.380943in}{0.295988in}}{\pgfqpoint{4.650000in}{0.692553in}}%
\pgfusepath{clip}%
\pgfsetbuttcap%
\pgfsetroundjoin%
\pgfsetlinewidth{0.250937pt}%
\definecolor{currentstroke}{rgb}{1.000000,1.000000,1.000000}%
\pgfsetstrokecolor{currentstroke}%
\pgfsetdash{}{0pt}%
\pgfpathmoveto{\pgfqpoint{0.380943in}{0.691732in}}%
\pgfpathlineto{\pgfqpoint{0.479879in}{0.691732in}}%
\pgfpathlineto{\pgfqpoint{0.479879in}{0.592796in}}%
\pgfpathlineto{\pgfqpoint{0.380943in}{0.592796in}}%
\pgfpathlineto{\pgfqpoint{0.380943in}{0.691732in}}%
\pgfusepath{stroke}%
\end{pgfscope}%
\begin{pgfscope}%
\pgfpathrectangle{\pgfqpoint{0.380943in}{0.295988in}}{\pgfqpoint{4.650000in}{0.692553in}}%
\pgfusepath{clip}%
\pgfsetbuttcap%
\pgfsetroundjoin%
\definecolor{currentfill}{rgb}{1.000000,1.000000,0.853287}%
\pgfsetfillcolor{currentfill}%
\pgfsetlinewidth{0.250937pt}%
\definecolor{currentstroke}{rgb}{1.000000,1.000000,1.000000}%
\pgfsetstrokecolor{currentstroke}%
\pgfsetdash{}{0pt}%
\pgfpathmoveto{\pgfqpoint{0.479879in}{0.691732in}}%
\pgfpathlineto{\pgfqpoint{0.578815in}{0.691732in}}%
\pgfpathlineto{\pgfqpoint{0.578815in}{0.592796in}}%
\pgfpathlineto{\pgfqpoint{0.479879in}{0.592796in}}%
\pgfpathlineto{\pgfqpoint{0.479879in}{0.691732in}}%
\pgfusepath{stroke,fill}%
\end{pgfscope}%
\begin{pgfscope}%
\pgfpathrectangle{\pgfqpoint{0.380943in}{0.295988in}}{\pgfqpoint{4.650000in}{0.692553in}}%
\pgfusepath{clip}%
\pgfsetbuttcap%
\pgfsetroundjoin%
\definecolor{currentfill}{rgb}{0.969858,0.948758,0.753003}%
\pgfsetfillcolor{currentfill}%
\pgfsetlinewidth{0.250937pt}%
\definecolor{currentstroke}{rgb}{1.000000,1.000000,1.000000}%
\pgfsetstrokecolor{currentstroke}%
\pgfsetdash{}{0pt}%
\pgfpathmoveto{\pgfqpoint{0.578815in}{0.691732in}}%
\pgfpathlineto{\pgfqpoint{0.677752in}{0.691732in}}%
\pgfpathlineto{\pgfqpoint{0.677752in}{0.592796in}}%
\pgfpathlineto{\pgfqpoint{0.578815in}{0.592796in}}%
\pgfpathlineto{\pgfqpoint{0.578815in}{0.691732in}}%
\pgfusepath{stroke,fill}%
\end{pgfscope}%
\begin{pgfscope}%
\pgfpathrectangle{\pgfqpoint{0.380943in}{0.295988in}}{\pgfqpoint{4.650000in}{0.692553in}}%
\pgfusepath{clip}%
\pgfsetbuttcap%
\pgfsetroundjoin%
\definecolor{currentfill}{rgb}{0.960892,0.932687,0.728981}%
\pgfsetfillcolor{currentfill}%
\pgfsetlinewidth{0.250937pt}%
\definecolor{currentstroke}{rgb}{1.000000,1.000000,1.000000}%
\pgfsetstrokecolor{currentstroke}%
\pgfsetdash{}{0pt}%
\pgfpathmoveto{\pgfqpoint{0.677752in}{0.691732in}}%
\pgfpathlineto{\pgfqpoint{0.776688in}{0.691732in}}%
\pgfpathlineto{\pgfqpoint{0.776688in}{0.592796in}}%
\pgfpathlineto{\pgfqpoint{0.677752in}{0.592796in}}%
\pgfpathlineto{\pgfqpoint{0.677752in}{0.691732in}}%
\pgfusepath{stroke,fill}%
\end{pgfscope}%
\begin{pgfscope}%
\pgfpathrectangle{\pgfqpoint{0.380943in}{0.295988in}}{\pgfqpoint{4.650000in}{0.692553in}}%
\pgfusepath{clip}%
\pgfsetbuttcap%
\pgfsetroundjoin%
\definecolor{currentfill}{rgb}{0.986774,0.977516,0.796986}%
\pgfsetfillcolor{currentfill}%
\pgfsetlinewidth{0.250937pt}%
\definecolor{currentstroke}{rgb}{1.000000,1.000000,1.000000}%
\pgfsetstrokecolor{currentstroke}%
\pgfsetdash{}{0pt}%
\pgfpathmoveto{\pgfqpoint{0.776688in}{0.691732in}}%
\pgfpathlineto{\pgfqpoint{0.875624in}{0.691732in}}%
\pgfpathlineto{\pgfqpoint{0.875624in}{0.592796in}}%
\pgfpathlineto{\pgfqpoint{0.776688in}{0.592796in}}%
\pgfpathlineto{\pgfqpoint{0.776688in}{0.691732in}}%
\pgfusepath{stroke,fill}%
\end{pgfscope}%
\begin{pgfscope}%
\pgfpathrectangle{\pgfqpoint{0.380943in}{0.295988in}}{\pgfqpoint{4.650000in}{0.692553in}}%
\pgfusepath{clip}%
\pgfsetbuttcap%
\pgfsetroundjoin%
\definecolor{currentfill}{rgb}{0.960892,0.932687,0.728981}%
\pgfsetfillcolor{currentfill}%
\pgfsetlinewidth{0.250937pt}%
\definecolor{currentstroke}{rgb}{1.000000,1.000000,1.000000}%
\pgfsetstrokecolor{currentstroke}%
\pgfsetdash{}{0pt}%
\pgfpathmoveto{\pgfqpoint{0.875624in}{0.691732in}}%
\pgfpathlineto{\pgfqpoint{0.974560in}{0.691732in}}%
\pgfpathlineto{\pgfqpoint{0.974560in}{0.592796in}}%
\pgfpathlineto{\pgfqpoint{0.875624in}{0.592796in}}%
\pgfpathlineto{\pgfqpoint{0.875624in}{0.691732in}}%
\pgfusepath{stroke,fill}%
\end{pgfscope}%
\begin{pgfscope}%
\pgfpathrectangle{\pgfqpoint{0.380943in}{0.295988in}}{\pgfqpoint{4.650000in}{0.692553in}}%
\pgfusepath{clip}%
\pgfsetbuttcap%
\pgfsetroundjoin%
\definecolor{currentfill}{rgb}{0.961230,0.930657,0.727628}%
\pgfsetfillcolor{currentfill}%
\pgfsetlinewidth{0.250937pt}%
\definecolor{currentstroke}{rgb}{1.000000,1.000000,1.000000}%
\pgfsetstrokecolor{currentstroke}%
\pgfsetdash{}{0pt}%
\pgfpathmoveto{\pgfqpoint{0.974560in}{0.691732in}}%
\pgfpathlineto{\pgfqpoint{1.073496in}{0.691732in}}%
\pgfpathlineto{\pgfqpoint{1.073496in}{0.592796in}}%
\pgfpathlineto{\pgfqpoint{0.974560in}{0.592796in}}%
\pgfpathlineto{\pgfqpoint{0.974560in}{0.691732in}}%
\pgfusepath{stroke,fill}%
\end{pgfscope}%
\begin{pgfscope}%
\pgfpathrectangle{\pgfqpoint{0.380943in}{0.295988in}}{\pgfqpoint{4.650000in}{0.692553in}}%
\pgfusepath{clip}%
\pgfsetbuttcap%
\pgfsetroundjoin%
\definecolor{currentfill}{rgb}{0.961738,0.927612,0.725598}%
\pgfsetfillcolor{currentfill}%
\pgfsetlinewidth{0.250937pt}%
\definecolor{currentstroke}{rgb}{1.000000,1.000000,1.000000}%
\pgfsetstrokecolor{currentstroke}%
\pgfsetdash{}{0pt}%
\pgfpathmoveto{\pgfqpoint{1.073496in}{0.691732in}}%
\pgfpathlineto{\pgfqpoint{1.172432in}{0.691732in}}%
\pgfpathlineto{\pgfqpoint{1.172432in}{0.592796in}}%
\pgfpathlineto{\pgfqpoint{1.073496in}{0.592796in}}%
\pgfpathlineto{\pgfqpoint{1.073496in}{0.691732in}}%
\pgfusepath{stroke,fill}%
\end{pgfscope}%
\begin{pgfscope}%
\pgfpathrectangle{\pgfqpoint{0.380943in}{0.295988in}}{\pgfqpoint{4.650000in}{0.692553in}}%
\pgfusepath{clip}%
\pgfsetbuttcap%
\pgfsetroundjoin%
\definecolor{currentfill}{rgb}{0.995233,0.991895,0.818977}%
\pgfsetfillcolor{currentfill}%
\pgfsetlinewidth{0.250937pt}%
\definecolor{currentstroke}{rgb}{1.000000,1.000000,1.000000}%
\pgfsetstrokecolor{currentstroke}%
\pgfsetdash{}{0pt}%
\pgfpathmoveto{\pgfqpoint{1.172432in}{0.691732in}}%
\pgfpathlineto{\pgfqpoint{1.271369in}{0.691732in}}%
\pgfpathlineto{\pgfqpoint{1.271369in}{0.592796in}}%
\pgfpathlineto{\pgfqpoint{1.172432in}{0.592796in}}%
\pgfpathlineto{\pgfqpoint{1.172432in}{0.691732in}}%
\pgfusepath{stroke,fill}%
\end{pgfscope}%
\begin{pgfscope}%
\pgfpathrectangle{\pgfqpoint{0.380943in}{0.295988in}}{\pgfqpoint{4.650000in}{0.692553in}}%
\pgfusepath{clip}%
\pgfsetbuttcap%
\pgfsetroundjoin%
\definecolor{currentfill}{rgb}{1.000000,0.598462,0.528012}%
\pgfsetfillcolor{currentfill}%
\pgfsetlinewidth{0.250937pt}%
\definecolor{currentstroke}{rgb}{1.000000,1.000000,1.000000}%
\pgfsetstrokecolor{currentstroke}%
\pgfsetdash{}{0pt}%
\pgfpathmoveto{\pgfqpoint{1.271369in}{0.691732in}}%
\pgfpathlineto{\pgfqpoint{1.370305in}{0.691732in}}%
\pgfpathlineto{\pgfqpoint{1.370305in}{0.592796in}}%
\pgfpathlineto{\pgfqpoint{1.271369in}{0.592796in}}%
\pgfpathlineto{\pgfqpoint{1.271369in}{0.691732in}}%
\pgfusepath{stroke,fill}%
\end{pgfscope}%
\begin{pgfscope}%
\pgfpathrectangle{\pgfqpoint{0.380943in}{0.295988in}}{\pgfqpoint{4.650000in}{0.692553in}}%
\pgfusepath{clip}%
\pgfsetbuttcap%
\pgfsetroundjoin%
\definecolor{currentfill}{rgb}{0.989619,0.788235,0.628374}%
\pgfsetfillcolor{currentfill}%
\pgfsetlinewidth{0.250937pt}%
\definecolor{currentstroke}{rgb}{1.000000,1.000000,1.000000}%
\pgfsetstrokecolor{currentstroke}%
\pgfsetdash{}{0pt}%
\pgfpathmoveto{\pgfqpoint{1.370305in}{0.691732in}}%
\pgfpathlineto{\pgfqpoint{1.469241in}{0.691732in}}%
\pgfpathlineto{\pgfqpoint{1.469241in}{0.592796in}}%
\pgfpathlineto{\pgfqpoint{1.370305in}{0.592796in}}%
\pgfpathlineto{\pgfqpoint{1.370305in}{0.691732in}}%
\pgfusepath{stroke,fill}%
\end{pgfscope}%
\begin{pgfscope}%
\pgfpathrectangle{\pgfqpoint{0.380943in}{0.295988in}}{\pgfqpoint{4.650000in}{0.692553in}}%
\pgfusepath{clip}%
\pgfsetbuttcap%
\pgfsetroundjoin%
\definecolor{currentfill}{rgb}{1.000000,0.598462,0.528012}%
\pgfsetfillcolor{currentfill}%
\pgfsetlinewidth{0.250937pt}%
\definecolor{currentstroke}{rgb}{1.000000,1.000000,1.000000}%
\pgfsetstrokecolor{currentstroke}%
\pgfsetdash{}{0pt}%
\pgfpathmoveto{\pgfqpoint{1.469241in}{0.691732in}}%
\pgfpathlineto{\pgfqpoint{1.568177in}{0.691732in}}%
\pgfpathlineto{\pgfqpoint{1.568177in}{0.592796in}}%
\pgfpathlineto{\pgfqpoint{1.469241in}{0.592796in}}%
\pgfpathlineto{\pgfqpoint{1.469241in}{0.691732in}}%
\pgfusepath{stroke,fill}%
\end{pgfscope}%
\begin{pgfscope}%
\pgfpathrectangle{\pgfqpoint{0.380943in}{0.295988in}}{\pgfqpoint{4.650000in}{0.692553in}}%
\pgfusepath{clip}%
\pgfsetbuttcap%
\pgfsetroundjoin%
\definecolor{currentfill}{rgb}{1.000000,0.581546,0.521246}%
\pgfsetfillcolor{currentfill}%
\pgfsetlinewidth{0.250937pt}%
\definecolor{currentstroke}{rgb}{1.000000,1.000000,1.000000}%
\pgfsetstrokecolor{currentstroke}%
\pgfsetdash{}{0pt}%
\pgfpathmoveto{\pgfqpoint{1.568177in}{0.691732in}}%
\pgfpathlineto{\pgfqpoint{1.667113in}{0.691732in}}%
\pgfpathlineto{\pgfqpoint{1.667113in}{0.592796in}}%
\pgfpathlineto{\pgfqpoint{1.568177in}{0.592796in}}%
\pgfpathlineto{\pgfqpoint{1.568177in}{0.691732in}}%
\pgfusepath{stroke,fill}%
\end{pgfscope}%
\begin{pgfscope}%
\pgfpathrectangle{\pgfqpoint{0.380943in}{0.295988in}}{\pgfqpoint{4.650000in}{0.692553in}}%
\pgfusepath{clip}%
\pgfsetbuttcap%
\pgfsetroundjoin%
\definecolor{currentfill}{rgb}{0.998601,0.667759,0.560769}%
\pgfsetfillcolor{currentfill}%
\pgfsetlinewidth{0.250937pt}%
\definecolor{currentstroke}{rgb}{1.000000,1.000000,1.000000}%
\pgfsetstrokecolor{currentstroke}%
\pgfsetdash{}{0pt}%
\pgfpathmoveto{\pgfqpoint{1.667113in}{0.691732in}}%
\pgfpathlineto{\pgfqpoint{1.766049in}{0.691732in}}%
\pgfpathlineto{\pgfqpoint{1.766049in}{0.592796in}}%
\pgfpathlineto{\pgfqpoint{1.667113in}{0.592796in}}%
\pgfpathlineto{\pgfqpoint{1.667113in}{0.691732in}}%
\pgfusepath{stroke,fill}%
\end{pgfscope}%
\begin{pgfscope}%
\pgfpathrectangle{\pgfqpoint{0.380943in}{0.295988in}}{\pgfqpoint{4.650000in}{0.692553in}}%
\pgfusepath{clip}%
\pgfsetbuttcap%
\pgfsetroundjoin%
\definecolor{currentfill}{rgb}{0.800000,0.278431,0.278431}%
\pgfsetfillcolor{currentfill}%
\pgfsetlinewidth{0.250937pt}%
\definecolor{currentstroke}{rgb}{1.000000,1.000000,1.000000}%
\pgfsetstrokecolor{currentstroke}%
\pgfsetdash{}{0pt}%
\pgfpathmoveto{\pgfqpoint{1.766049in}{0.691732in}}%
\pgfpathlineto{\pgfqpoint{1.864986in}{0.691732in}}%
\pgfpathlineto{\pgfqpoint{1.864986in}{0.592796in}}%
\pgfpathlineto{\pgfqpoint{1.766049in}{0.592796in}}%
\pgfpathlineto{\pgfqpoint{1.766049in}{0.691732in}}%
\pgfusepath{stroke,fill}%
\end{pgfscope}%
\begin{pgfscope}%
\pgfpathrectangle{\pgfqpoint{0.380943in}{0.295988in}}{\pgfqpoint{4.650000in}{0.692553in}}%
\pgfusepath{clip}%
\pgfsetbuttcap%
\pgfsetroundjoin%
\definecolor{currentfill}{rgb}{0.994694,0.745098,0.602999}%
\pgfsetfillcolor{currentfill}%
\pgfsetlinewidth{0.250937pt}%
\definecolor{currentstroke}{rgb}{1.000000,1.000000,1.000000}%
\pgfsetstrokecolor{currentstroke}%
\pgfsetdash{}{0pt}%
\pgfpathmoveto{\pgfqpoint{1.864986in}{0.691732in}}%
\pgfpathlineto{\pgfqpoint{1.963922in}{0.691732in}}%
\pgfpathlineto{\pgfqpoint{1.963922in}{0.592796in}}%
\pgfpathlineto{\pgfqpoint{1.864986in}{0.592796in}}%
\pgfpathlineto{\pgfqpoint{1.864986in}{0.691732in}}%
\pgfusepath{stroke,fill}%
\end{pgfscope}%
\begin{pgfscope}%
\pgfpathrectangle{\pgfqpoint{0.380943in}{0.295988in}}{\pgfqpoint{4.650000in}{0.692553in}}%
\pgfusepath{clip}%
\pgfsetbuttcap%
\pgfsetroundjoin%
\definecolor{currentfill}{rgb}{0.997586,0.694148,0.574979}%
\pgfsetfillcolor{currentfill}%
\pgfsetlinewidth{0.250937pt}%
\definecolor{currentstroke}{rgb}{1.000000,1.000000,1.000000}%
\pgfsetstrokecolor{currentstroke}%
\pgfsetdash{}{0pt}%
\pgfpathmoveto{\pgfqpoint{1.963922in}{0.691732in}}%
\pgfpathlineto{\pgfqpoint{2.062858in}{0.691732in}}%
\pgfpathlineto{\pgfqpoint{2.062858in}{0.592796in}}%
\pgfpathlineto{\pgfqpoint{1.963922in}{0.592796in}}%
\pgfpathlineto{\pgfqpoint{1.963922in}{0.691732in}}%
\pgfusepath{stroke,fill}%
\end{pgfscope}%
\begin{pgfscope}%
\pgfpathrectangle{\pgfqpoint{0.380943in}{0.295988in}}{\pgfqpoint{4.650000in}{0.692553in}}%
\pgfusepath{clip}%
\pgfsetbuttcap%
\pgfsetroundjoin%
\definecolor{currentfill}{rgb}{0.994018,0.750850,0.606382}%
\pgfsetfillcolor{currentfill}%
\pgfsetlinewidth{0.250937pt}%
\definecolor{currentstroke}{rgb}{1.000000,1.000000,1.000000}%
\pgfsetstrokecolor{currentstroke}%
\pgfsetdash{}{0pt}%
\pgfpathmoveto{\pgfqpoint{2.062858in}{0.691732in}}%
\pgfpathlineto{\pgfqpoint{2.161794in}{0.691732in}}%
\pgfpathlineto{\pgfqpoint{2.161794in}{0.592796in}}%
\pgfpathlineto{\pgfqpoint{2.062858in}{0.592796in}}%
\pgfpathlineto{\pgfqpoint{2.062858in}{0.691732in}}%
\pgfusepath{stroke,fill}%
\end{pgfscope}%
\begin{pgfscope}%
\pgfpathrectangle{\pgfqpoint{0.380943in}{0.295988in}}{\pgfqpoint{4.650000in}{0.692553in}}%
\pgfusepath{clip}%
\pgfsetbuttcap%
\pgfsetroundjoin%
\definecolor{currentfill}{rgb}{0.996401,0.724937,0.591557}%
\pgfsetfillcolor{currentfill}%
\pgfsetlinewidth{0.250937pt}%
\definecolor{currentstroke}{rgb}{1.000000,1.000000,1.000000}%
\pgfsetstrokecolor{currentstroke}%
\pgfsetdash{}{0pt}%
\pgfpathmoveto{\pgfqpoint{2.161794in}{0.691732in}}%
\pgfpathlineto{\pgfqpoint{2.260730in}{0.691732in}}%
\pgfpathlineto{\pgfqpoint{2.260730in}{0.592796in}}%
\pgfpathlineto{\pgfqpoint{2.161794in}{0.592796in}}%
\pgfpathlineto{\pgfqpoint{2.161794in}{0.691732in}}%
\pgfusepath{stroke,fill}%
\end{pgfscope}%
\begin{pgfscope}%
\pgfpathrectangle{\pgfqpoint{0.380943in}{0.295988in}}{\pgfqpoint{4.650000in}{0.692553in}}%
\pgfusepath{clip}%
\pgfsetbuttcap%
\pgfsetroundjoin%
\definecolor{currentfill}{rgb}{0.964937,0.908651,0.713110}%
\pgfsetfillcolor{currentfill}%
\pgfsetlinewidth{0.250937pt}%
\definecolor{currentstroke}{rgb}{1.000000,1.000000,1.000000}%
\pgfsetstrokecolor{currentstroke}%
\pgfsetdash{}{0pt}%
\pgfpathmoveto{\pgfqpoint{2.260730in}{0.691732in}}%
\pgfpathlineto{\pgfqpoint{2.359666in}{0.691732in}}%
\pgfpathlineto{\pgfqpoint{2.359666in}{0.592796in}}%
\pgfpathlineto{\pgfqpoint{2.260730in}{0.592796in}}%
\pgfpathlineto{\pgfqpoint{2.260730in}{0.691732in}}%
\pgfusepath{stroke,fill}%
\end{pgfscope}%
\begin{pgfscope}%
\pgfpathrectangle{\pgfqpoint{0.380943in}{0.295988in}}{\pgfqpoint{4.650000in}{0.692553in}}%
\pgfusepath{clip}%
\pgfsetbuttcap%
\pgfsetroundjoin%
\definecolor{currentfill}{rgb}{0.977116,0.848181,0.679769}%
\pgfsetfillcolor{currentfill}%
\pgfsetlinewidth{0.250937pt}%
\definecolor{currentstroke}{rgb}{1.000000,1.000000,1.000000}%
\pgfsetstrokecolor{currentstroke}%
\pgfsetdash{}{0pt}%
\pgfpathmoveto{\pgfqpoint{2.359666in}{0.691732in}}%
\pgfpathlineto{\pgfqpoint{2.458603in}{0.691732in}}%
\pgfpathlineto{\pgfqpoint{2.458603in}{0.592796in}}%
\pgfpathlineto{\pgfqpoint{2.359666in}{0.592796in}}%
\pgfpathlineto{\pgfqpoint{2.359666in}{0.691732in}}%
\pgfusepath{stroke,fill}%
\end{pgfscope}%
\begin{pgfscope}%
\pgfpathrectangle{\pgfqpoint{0.380943in}{0.295988in}}{\pgfqpoint{4.650000in}{0.692553in}}%
\pgfusepath{clip}%
\pgfsetbuttcap%
\pgfsetroundjoin%
\definecolor{currentfill}{rgb}{0.991311,0.773856,0.619915}%
\pgfsetfillcolor{currentfill}%
\pgfsetlinewidth{0.250937pt}%
\definecolor{currentstroke}{rgb}{1.000000,1.000000,1.000000}%
\pgfsetstrokecolor{currentstroke}%
\pgfsetdash{}{0pt}%
\pgfpathmoveto{\pgfqpoint{2.458603in}{0.691732in}}%
\pgfpathlineto{\pgfqpoint{2.557539in}{0.691732in}}%
\pgfpathlineto{\pgfqpoint{2.557539in}{0.592796in}}%
\pgfpathlineto{\pgfqpoint{2.458603in}{0.592796in}}%
\pgfpathlineto{\pgfqpoint{2.458603in}{0.691732in}}%
\pgfusepath{stroke,fill}%
\end{pgfscope}%
\begin{pgfscope}%
\pgfpathrectangle{\pgfqpoint{0.380943in}{0.295988in}}{\pgfqpoint{4.650000in}{0.692553in}}%
\pgfusepath{clip}%
\pgfsetbuttcap%
\pgfsetroundjoin%
\definecolor{currentfill}{rgb}{0.994018,0.750850,0.606382}%
\pgfsetfillcolor{currentfill}%
\pgfsetlinewidth{0.250937pt}%
\definecolor{currentstroke}{rgb}{1.000000,1.000000,1.000000}%
\pgfsetstrokecolor{currentstroke}%
\pgfsetdash{}{0pt}%
\pgfpathmoveto{\pgfqpoint{2.557539in}{0.691732in}}%
\pgfpathlineto{\pgfqpoint{2.656475in}{0.691732in}}%
\pgfpathlineto{\pgfqpoint{2.656475in}{0.592796in}}%
\pgfpathlineto{\pgfqpoint{2.557539in}{0.592796in}}%
\pgfpathlineto{\pgfqpoint{2.557539in}{0.691732in}}%
\pgfusepath{stroke,fill}%
\end{pgfscope}%
\begin{pgfscope}%
\pgfpathrectangle{\pgfqpoint{0.380943in}{0.295988in}}{\pgfqpoint{4.650000in}{0.692553in}}%
\pgfusepath{clip}%
\pgfsetbuttcap%
\pgfsetroundjoin%
\definecolor{currentfill}{rgb}{1.000000,0.598462,0.528012}%
\pgfsetfillcolor{currentfill}%
\pgfsetlinewidth{0.250937pt}%
\definecolor{currentstroke}{rgb}{1.000000,1.000000,1.000000}%
\pgfsetstrokecolor{currentstroke}%
\pgfsetdash{}{0pt}%
\pgfpathmoveto{\pgfqpoint{2.656475in}{0.691732in}}%
\pgfpathlineto{\pgfqpoint{2.755411in}{0.691732in}}%
\pgfpathlineto{\pgfqpoint{2.755411in}{0.592796in}}%
\pgfpathlineto{\pgfqpoint{2.656475in}{0.592796in}}%
\pgfpathlineto{\pgfqpoint{2.656475in}{0.691732in}}%
\pgfusepath{stroke,fill}%
\end{pgfscope}%
\begin{pgfscope}%
\pgfpathrectangle{\pgfqpoint{0.380943in}{0.295988in}}{\pgfqpoint{4.650000in}{0.692553in}}%
\pgfusepath{clip}%
\pgfsetbuttcap%
\pgfsetroundjoin%
\definecolor{currentfill}{rgb}{0.989619,0.788235,0.628374}%
\pgfsetfillcolor{currentfill}%
\pgfsetlinewidth{0.250937pt}%
\definecolor{currentstroke}{rgb}{1.000000,1.000000,1.000000}%
\pgfsetstrokecolor{currentstroke}%
\pgfsetdash{}{0pt}%
\pgfpathmoveto{\pgfqpoint{2.755411in}{0.691732in}}%
\pgfpathlineto{\pgfqpoint{2.854347in}{0.691732in}}%
\pgfpathlineto{\pgfqpoint{2.854347in}{0.592796in}}%
\pgfpathlineto{\pgfqpoint{2.755411in}{0.592796in}}%
\pgfpathlineto{\pgfqpoint{2.755411in}{0.691732in}}%
\pgfusepath{stroke,fill}%
\end{pgfscope}%
\begin{pgfscope}%
\pgfpathrectangle{\pgfqpoint{0.380943in}{0.295988in}}{\pgfqpoint{4.650000in}{0.692553in}}%
\pgfusepath{clip}%
\pgfsetbuttcap%
\pgfsetroundjoin%
\definecolor{currentfill}{rgb}{0.975594,0.855363,0.684691}%
\pgfsetfillcolor{currentfill}%
\pgfsetlinewidth{0.250937pt}%
\definecolor{currentstroke}{rgb}{1.000000,1.000000,1.000000}%
\pgfsetstrokecolor{currentstroke}%
\pgfsetdash{}{0pt}%
\pgfpathmoveto{\pgfqpoint{2.854347in}{0.691732in}}%
\pgfpathlineto{\pgfqpoint{2.953283in}{0.691732in}}%
\pgfpathlineto{\pgfqpoint{2.953283in}{0.592796in}}%
\pgfpathlineto{\pgfqpoint{2.854347in}{0.592796in}}%
\pgfpathlineto{\pgfqpoint{2.854347in}{0.691732in}}%
\pgfusepath{stroke,fill}%
\end{pgfscope}%
\begin{pgfscope}%
\pgfpathrectangle{\pgfqpoint{0.380943in}{0.295988in}}{\pgfqpoint{4.650000in}{0.692553in}}%
\pgfusepath{clip}%
\pgfsetbuttcap%
\pgfsetroundjoin%
\definecolor{currentfill}{rgb}{1.000000,0.554479,0.510419}%
\pgfsetfillcolor{currentfill}%
\pgfsetlinewidth{0.250937pt}%
\definecolor{currentstroke}{rgb}{1.000000,1.000000,1.000000}%
\pgfsetstrokecolor{currentstroke}%
\pgfsetdash{}{0pt}%
\pgfpathmoveto{\pgfqpoint{2.953283in}{0.691732in}}%
\pgfpathlineto{\pgfqpoint{3.052220in}{0.691732in}}%
\pgfpathlineto{\pgfqpoint{3.052220in}{0.592796in}}%
\pgfpathlineto{\pgfqpoint{2.953283in}{0.592796in}}%
\pgfpathlineto{\pgfqpoint{2.953283in}{0.691732in}}%
\pgfusepath{stroke,fill}%
\end{pgfscope}%
\begin{pgfscope}%
\pgfpathrectangle{\pgfqpoint{0.380943in}{0.295988in}}{\pgfqpoint{4.650000in}{0.692553in}}%
\pgfusepath{clip}%
\pgfsetbuttcap%
\pgfsetroundjoin%
\definecolor{currentfill}{rgb}{0.980669,0.832787,0.665559}%
\pgfsetfillcolor{currentfill}%
\pgfsetlinewidth{0.250937pt}%
\definecolor{currentstroke}{rgb}{1.000000,1.000000,1.000000}%
\pgfsetstrokecolor{currentstroke}%
\pgfsetdash{}{0pt}%
\pgfpathmoveto{\pgfqpoint{3.052220in}{0.691732in}}%
\pgfpathlineto{\pgfqpoint{3.151156in}{0.691732in}}%
\pgfpathlineto{\pgfqpoint{3.151156in}{0.592796in}}%
\pgfpathlineto{\pgfqpoint{3.052220in}{0.592796in}}%
\pgfpathlineto{\pgfqpoint{3.052220in}{0.691732in}}%
\pgfusepath{stroke,fill}%
\end{pgfscope}%
\begin{pgfscope}%
\pgfpathrectangle{\pgfqpoint{0.380943in}{0.295988in}}{\pgfqpoint{4.650000in}{0.692553in}}%
\pgfusepath{clip}%
\pgfsetbuttcap%
\pgfsetroundjoin%
\definecolor{currentfill}{rgb}{0.984729,0.815194,0.649319}%
\pgfsetfillcolor{currentfill}%
\pgfsetlinewidth{0.250937pt}%
\definecolor{currentstroke}{rgb}{1.000000,1.000000,1.000000}%
\pgfsetstrokecolor{currentstroke}%
\pgfsetdash{}{0pt}%
\pgfpathmoveto{\pgfqpoint{3.151156in}{0.691732in}}%
\pgfpathlineto{\pgfqpoint{3.250092in}{0.691732in}}%
\pgfpathlineto{\pgfqpoint{3.250092in}{0.592796in}}%
\pgfpathlineto{\pgfqpoint{3.151156in}{0.592796in}}%
\pgfpathlineto{\pgfqpoint{3.151156in}{0.691732in}}%
\pgfusepath{stroke,fill}%
\end{pgfscope}%
\begin{pgfscope}%
\pgfpathrectangle{\pgfqpoint{0.380943in}{0.295988in}}{\pgfqpoint{4.650000in}{0.692553in}}%
\pgfusepath{clip}%
\pgfsetbuttcap%
\pgfsetroundjoin%
\definecolor{currentfill}{rgb}{0.998601,0.667759,0.560769}%
\pgfsetfillcolor{currentfill}%
\pgfsetlinewidth{0.250937pt}%
\definecolor{currentstroke}{rgb}{1.000000,1.000000,1.000000}%
\pgfsetstrokecolor{currentstroke}%
\pgfsetdash{}{0pt}%
\pgfpathmoveto{\pgfqpoint{3.250092in}{0.691732in}}%
\pgfpathlineto{\pgfqpoint{3.349028in}{0.691732in}}%
\pgfpathlineto{\pgfqpoint{3.349028in}{0.592796in}}%
\pgfpathlineto{\pgfqpoint{3.250092in}{0.592796in}}%
\pgfpathlineto{\pgfqpoint{3.250092in}{0.691732in}}%
\pgfusepath{stroke,fill}%
\end{pgfscope}%
\begin{pgfscope}%
\pgfpathrectangle{\pgfqpoint{0.380943in}{0.295988in}}{\pgfqpoint{4.650000in}{0.692553in}}%
\pgfusepath{clip}%
\pgfsetbuttcap%
\pgfsetroundjoin%
\definecolor{currentfill}{rgb}{0.998093,0.680953,0.567874}%
\pgfsetfillcolor{currentfill}%
\pgfsetlinewidth{0.250937pt}%
\definecolor{currentstroke}{rgb}{1.000000,1.000000,1.000000}%
\pgfsetstrokecolor{currentstroke}%
\pgfsetdash{}{0pt}%
\pgfpathmoveto{\pgfqpoint{3.349028in}{0.691732in}}%
\pgfpathlineto{\pgfqpoint{3.447964in}{0.691732in}}%
\pgfpathlineto{\pgfqpoint{3.447964in}{0.592796in}}%
\pgfpathlineto{\pgfqpoint{3.349028in}{0.592796in}}%
\pgfpathlineto{\pgfqpoint{3.349028in}{0.691732in}}%
\pgfusepath{stroke,fill}%
\end{pgfscope}%
\begin{pgfscope}%
\pgfpathrectangle{\pgfqpoint{0.380943in}{0.295988in}}{\pgfqpoint{4.650000in}{0.692553in}}%
\pgfusepath{clip}%
\pgfsetbuttcap%
\pgfsetroundjoin%
\definecolor{currentfill}{rgb}{1.000000,0.581546,0.521246}%
\pgfsetfillcolor{currentfill}%
\pgfsetlinewidth{0.250937pt}%
\definecolor{currentstroke}{rgb}{1.000000,1.000000,1.000000}%
\pgfsetstrokecolor{currentstroke}%
\pgfsetdash{}{0pt}%
\pgfpathmoveto{\pgfqpoint{3.447964in}{0.691732in}}%
\pgfpathlineto{\pgfqpoint{3.546901in}{0.691732in}}%
\pgfpathlineto{\pgfqpoint{3.546901in}{0.592796in}}%
\pgfpathlineto{\pgfqpoint{3.447964in}{0.592796in}}%
\pgfpathlineto{\pgfqpoint{3.447964in}{0.691732in}}%
\pgfusepath{stroke,fill}%
\end{pgfscope}%
\begin{pgfscope}%
\pgfpathrectangle{\pgfqpoint{0.380943in}{0.295988in}}{\pgfqpoint{4.650000in}{0.692553in}}%
\pgfusepath{clip}%
\pgfsetbuttcap%
\pgfsetroundjoin%
\definecolor{currentfill}{rgb}{0.995709,0.736471,0.597924}%
\pgfsetfillcolor{currentfill}%
\pgfsetlinewidth{0.250937pt}%
\definecolor{currentstroke}{rgb}{1.000000,1.000000,1.000000}%
\pgfsetstrokecolor{currentstroke}%
\pgfsetdash{}{0pt}%
\pgfpathmoveto{\pgfqpoint{3.546901in}{0.691732in}}%
\pgfpathlineto{\pgfqpoint{3.645837in}{0.691732in}}%
\pgfpathlineto{\pgfqpoint{3.645837in}{0.592796in}}%
\pgfpathlineto{\pgfqpoint{3.546901in}{0.592796in}}%
\pgfpathlineto{\pgfqpoint{3.546901in}{0.691732in}}%
\pgfusepath{stroke,fill}%
\end{pgfscope}%
\begin{pgfscope}%
\pgfpathrectangle{\pgfqpoint{0.380943in}{0.295988in}}{\pgfqpoint{4.650000in}{0.692553in}}%
\pgfusepath{clip}%
\pgfsetbuttcap%
\pgfsetroundjoin%
\definecolor{currentfill}{rgb}{0.998939,0.658962,0.556032}%
\pgfsetfillcolor{currentfill}%
\pgfsetlinewidth{0.250937pt}%
\definecolor{currentstroke}{rgb}{1.000000,1.000000,1.000000}%
\pgfsetstrokecolor{currentstroke}%
\pgfsetdash{}{0pt}%
\pgfpathmoveto{\pgfqpoint{3.645837in}{0.691732in}}%
\pgfpathlineto{\pgfqpoint{3.744773in}{0.691732in}}%
\pgfpathlineto{\pgfqpoint{3.744773in}{0.592796in}}%
\pgfpathlineto{\pgfqpoint{3.645837in}{0.592796in}}%
\pgfpathlineto{\pgfqpoint{3.645837in}{0.691732in}}%
\pgfusepath{stroke,fill}%
\end{pgfscope}%
\begin{pgfscope}%
\pgfpathrectangle{\pgfqpoint{0.380943in}{0.295988in}}{\pgfqpoint{4.650000in}{0.692553in}}%
\pgfusepath{clip}%
\pgfsetbuttcap%
\pgfsetroundjoin%
\definecolor{currentfill}{rgb}{0.998939,0.658962,0.556032}%
\pgfsetfillcolor{currentfill}%
\pgfsetlinewidth{0.250937pt}%
\definecolor{currentstroke}{rgb}{1.000000,1.000000,1.000000}%
\pgfsetstrokecolor{currentstroke}%
\pgfsetdash{}{0pt}%
\pgfpathmoveto{\pgfqpoint{3.744773in}{0.691732in}}%
\pgfpathlineto{\pgfqpoint{3.843709in}{0.691732in}}%
\pgfpathlineto{\pgfqpoint{3.843709in}{0.592796in}}%
\pgfpathlineto{\pgfqpoint{3.744773in}{0.592796in}}%
\pgfpathlineto{\pgfqpoint{3.744773in}{0.691732in}}%
\pgfusepath{stroke,fill}%
\end{pgfscope}%
\begin{pgfscope}%
\pgfpathrectangle{\pgfqpoint{0.380943in}{0.295988in}}{\pgfqpoint{4.650000in}{0.692553in}}%
\pgfusepath{clip}%
\pgfsetbuttcap%
\pgfsetroundjoin%
\definecolor{currentfill}{rgb}{0.808797,0.287228,0.287228}%
\pgfsetfillcolor{currentfill}%
\pgfsetlinewidth{0.250937pt}%
\definecolor{currentstroke}{rgb}{1.000000,1.000000,1.000000}%
\pgfsetstrokecolor{currentstroke}%
\pgfsetdash{}{0pt}%
\pgfpathmoveto{\pgfqpoint{3.843709in}{0.691732in}}%
\pgfpathlineto{\pgfqpoint{3.942645in}{0.691732in}}%
\pgfpathlineto{\pgfqpoint{3.942645in}{0.592796in}}%
\pgfpathlineto{\pgfqpoint{3.843709in}{0.592796in}}%
\pgfpathlineto{\pgfqpoint{3.843709in}{0.691732in}}%
\pgfusepath{stroke,fill}%
\end{pgfscope}%
\begin{pgfscope}%
\pgfpathrectangle{\pgfqpoint{0.380943in}{0.295988in}}{\pgfqpoint{4.650000in}{0.692553in}}%
\pgfusepath{clip}%
\pgfsetbuttcap%
\pgfsetroundjoin%
\definecolor{currentfill}{rgb}{1.000000,0.615379,0.534779}%
\pgfsetfillcolor{currentfill}%
\pgfsetlinewidth{0.250937pt}%
\definecolor{currentstroke}{rgb}{1.000000,1.000000,1.000000}%
\pgfsetstrokecolor{currentstroke}%
\pgfsetdash{}{0pt}%
\pgfpathmoveto{\pgfqpoint{3.942645in}{0.691732in}}%
\pgfpathlineto{\pgfqpoint{4.041581in}{0.691732in}}%
\pgfpathlineto{\pgfqpoint{4.041581in}{0.592796in}}%
\pgfpathlineto{\pgfqpoint{3.942645in}{0.592796in}}%
\pgfpathlineto{\pgfqpoint{3.942645in}{0.691732in}}%
\pgfusepath{stroke,fill}%
\end{pgfscope}%
\begin{pgfscope}%
\pgfpathrectangle{\pgfqpoint{0.380943in}{0.295988in}}{\pgfqpoint{4.650000in}{0.692553in}}%
\pgfusepath{clip}%
\pgfsetbuttcap%
\pgfsetroundjoin%
\definecolor{currentfill}{rgb}{0.970012,0.883276,0.699577}%
\pgfsetfillcolor{currentfill}%
\pgfsetlinewidth{0.250937pt}%
\definecolor{currentstroke}{rgb}{1.000000,1.000000,1.000000}%
\pgfsetstrokecolor{currentstroke}%
\pgfsetdash{}{0pt}%
\pgfpathmoveto{\pgfqpoint{4.041581in}{0.691732in}}%
\pgfpathlineto{\pgfqpoint{4.140518in}{0.691732in}}%
\pgfpathlineto{\pgfqpoint{4.140518in}{0.592796in}}%
\pgfpathlineto{\pgfqpoint{4.041581in}{0.592796in}}%
\pgfpathlineto{\pgfqpoint{4.041581in}{0.691732in}}%
\pgfusepath{stroke,fill}%
\end{pgfscope}%
\begin{pgfscope}%
\pgfpathrectangle{\pgfqpoint{0.380943in}{0.295988in}}{\pgfqpoint{4.650000in}{0.692553in}}%
\pgfusepath{clip}%
\pgfsetbuttcap%
\pgfsetroundjoin%
\definecolor{currentfill}{rgb}{0.995709,0.736471,0.597924}%
\pgfsetfillcolor{currentfill}%
\pgfsetlinewidth{0.250937pt}%
\definecolor{currentstroke}{rgb}{1.000000,1.000000,1.000000}%
\pgfsetstrokecolor{currentstroke}%
\pgfsetdash{}{0pt}%
\pgfpathmoveto{\pgfqpoint{4.140518in}{0.691732in}}%
\pgfpathlineto{\pgfqpoint{4.239454in}{0.691732in}}%
\pgfpathlineto{\pgfqpoint{4.239454in}{0.592796in}}%
\pgfpathlineto{\pgfqpoint{4.140518in}{0.592796in}}%
\pgfpathlineto{\pgfqpoint{4.140518in}{0.691732in}}%
\pgfusepath{stroke,fill}%
\end{pgfscope}%
\begin{pgfscope}%
\pgfpathrectangle{\pgfqpoint{0.380943in}{0.295988in}}{\pgfqpoint{4.650000in}{0.692553in}}%
\pgfusepath{clip}%
\pgfsetbuttcap%
\pgfsetroundjoin%
\definecolor{currentfill}{rgb}{1.000000,0.554479,0.510419}%
\pgfsetfillcolor{currentfill}%
\pgfsetlinewidth{0.250937pt}%
\definecolor{currentstroke}{rgb}{1.000000,1.000000,1.000000}%
\pgfsetstrokecolor{currentstroke}%
\pgfsetdash{}{0pt}%
\pgfpathmoveto{\pgfqpoint{4.239454in}{0.691732in}}%
\pgfpathlineto{\pgfqpoint{4.338390in}{0.691732in}}%
\pgfpathlineto{\pgfqpoint{4.338390in}{0.592796in}}%
\pgfpathlineto{\pgfqpoint{4.239454in}{0.592796in}}%
\pgfpathlineto{\pgfqpoint{4.239454in}{0.691732in}}%
\pgfusepath{stroke,fill}%
\end{pgfscope}%
\begin{pgfscope}%
\pgfpathrectangle{\pgfqpoint{0.380943in}{0.295988in}}{\pgfqpoint{4.650000in}{0.692553in}}%
\pgfusepath{clip}%
\pgfsetbuttcap%
\pgfsetroundjoin%
\definecolor{currentfill}{rgb}{0.974072,0.862976,0.688750}%
\pgfsetfillcolor{currentfill}%
\pgfsetlinewidth{0.250937pt}%
\definecolor{currentstroke}{rgb}{1.000000,1.000000,1.000000}%
\pgfsetstrokecolor{currentstroke}%
\pgfsetdash{}{0pt}%
\pgfpathmoveto{\pgfqpoint{4.338390in}{0.691732in}}%
\pgfpathlineto{\pgfqpoint{4.437326in}{0.691732in}}%
\pgfpathlineto{\pgfqpoint{4.437326in}{0.592796in}}%
\pgfpathlineto{\pgfqpoint{4.338390in}{0.592796in}}%
\pgfpathlineto{\pgfqpoint{4.338390in}{0.691732in}}%
\pgfusepath{stroke,fill}%
\end{pgfscope}%
\begin{pgfscope}%
\pgfpathrectangle{\pgfqpoint{0.380943in}{0.295988in}}{\pgfqpoint{4.650000in}{0.692553in}}%
\pgfusepath{clip}%
\pgfsetbuttcap%
\pgfsetroundjoin%
\definecolor{currentfill}{rgb}{0.975594,0.855363,0.684691}%
\pgfsetfillcolor{currentfill}%
\pgfsetlinewidth{0.250937pt}%
\definecolor{currentstroke}{rgb}{1.000000,1.000000,1.000000}%
\pgfsetstrokecolor{currentstroke}%
\pgfsetdash{}{0pt}%
\pgfpathmoveto{\pgfqpoint{4.437326in}{0.691732in}}%
\pgfpathlineto{\pgfqpoint{4.536262in}{0.691732in}}%
\pgfpathlineto{\pgfqpoint{4.536262in}{0.592796in}}%
\pgfpathlineto{\pgfqpoint{4.437326in}{0.592796in}}%
\pgfpathlineto{\pgfqpoint{4.437326in}{0.691732in}}%
\pgfusepath{stroke,fill}%
\end{pgfscope}%
\begin{pgfscope}%
\pgfpathrectangle{\pgfqpoint{0.380943in}{0.295988in}}{\pgfqpoint{4.650000in}{0.692553in}}%
\pgfusepath{clip}%
\pgfsetbuttcap%
\pgfsetroundjoin%
\definecolor{currentfill}{rgb}{0.983714,0.819592,0.653379}%
\pgfsetfillcolor{currentfill}%
\pgfsetlinewidth{0.250937pt}%
\definecolor{currentstroke}{rgb}{1.000000,1.000000,1.000000}%
\pgfsetstrokecolor{currentstroke}%
\pgfsetdash{}{0pt}%
\pgfpathmoveto{\pgfqpoint{4.536262in}{0.691732in}}%
\pgfpathlineto{\pgfqpoint{4.635198in}{0.691732in}}%
\pgfpathlineto{\pgfqpoint{4.635198in}{0.592796in}}%
\pgfpathlineto{\pgfqpoint{4.536262in}{0.592796in}}%
\pgfpathlineto{\pgfqpoint{4.536262in}{0.691732in}}%
\pgfusepath{stroke,fill}%
\end{pgfscope}%
\begin{pgfscope}%
\pgfpathrectangle{\pgfqpoint{0.380943in}{0.295988in}}{\pgfqpoint{4.650000in}{0.692553in}}%
\pgfusepath{clip}%
\pgfsetbuttcap%
\pgfsetroundjoin%
\definecolor{currentfill}{rgb}{0.966459,0.901038,0.709050}%
\pgfsetfillcolor{currentfill}%
\pgfsetlinewidth{0.250937pt}%
\definecolor{currentstroke}{rgb}{1.000000,1.000000,1.000000}%
\pgfsetstrokecolor{currentstroke}%
\pgfsetdash{}{0pt}%
\pgfpathmoveto{\pgfqpoint{4.635198in}{0.691732in}}%
\pgfpathlineto{\pgfqpoint{4.734135in}{0.691732in}}%
\pgfpathlineto{\pgfqpoint{4.734135in}{0.592796in}}%
\pgfpathlineto{\pgfqpoint{4.635198in}{0.592796in}}%
\pgfpathlineto{\pgfqpoint{4.635198in}{0.691732in}}%
\pgfusepath{stroke,fill}%
\end{pgfscope}%
\begin{pgfscope}%
\pgfpathrectangle{\pgfqpoint{0.380943in}{0.295988in}}{\pgfqpoint{4.650000in}{0.692553in}}%
\pgfusepath{clip}%
\pgfsetbuttcap%
\pgfsetroundjoin%
\definecolor{currentfill}{rgb}{0.973057,0.868051,0.691457}%
\pgfsetfillcolor{currentfill}%
\pgfsetlinewidth{0.250937pt}%
\definecolor{currentstroke}{rgb}{1.000000,1.000000,1.000000}%
\pgfsetstrokecolor{currentstroke}%
\pgfsetdash{}{0pt}%
\pgfpathmoveto{\pgfqpoint{4.734135in}{0.691732in}}%
\pgfpathlineto{\pgfqpoint{4.833071in}{0.691732in}}%
\pgfpathlineto{\pgfqpoint{4.833071in}{0.592796in}}%
\pgfpathlineto{\pgfqpoint{4.734135in}{0.592796in}}%
\pgfpathlineto{\pgfqpoint{4.734135in}{0.691732in}}%
\pgfusepath{stroke,fill}%
\end{pgfscope}%
\begin{pgfscope}%
\pgfpathrectangle{\pgfqpoint{0.380943in}{0.295988in}}{\pgfqpoint{4.650000in}{0.692553in}}%
\pgfusepath{clip}%
\pgfsetbuttcap%
\pgfsetroundjoin%
\definecolor{currentfill}{rgb}{0.983391,0.971765,0.788189}%
\pgfsetfillcolor{currentfill}%
\pgfsetlinewidth{0.250937pt}%
\definecolor{currentstroke}{rgb}{1.000000,1.000000,1.000000}%
\pgfsetstrokecolor{currentstroke}%
\pgfsetdash{}{0pt}%
\pgfpathmoveto{\pgfqpoint{4.833071in}{0.691732in}}%
\pgfpathlineto{\pgfqpoint{4.932007in}{0.691732in}}%
\pgfpathlineto{\pgfqpoint{4.932007in}{0.592796in}}%
\pgfpathlineto{\pgfqpoint{4.833071in}{0.592796in}}%
\pgfpathlineto{\pgfqpoint{4.833071in}{0.691732in}}%
\pgfusepath{stroke,fill}%
\end{pgfscope}%
\begin{pgfscope}%
\pgfpathrectangle{\pgfqpoint{0.380943in}{0.295988in}}{\pgfqpoint{4.650000in}{0.692553in}}%
\pgfusepath{clip}%
\pgfsetbuttcap%
\pgfsetroundjoin%
\pgfsetlinewidth{0.250937pt}%
\definecolor{currentstroke}{rgb}{1.000000,1.000000,1.000000}%
\pgfsetstrokecolor{currentstroke}%
\pgfsetdash{}{0pt}%
\pgfpathmoveto{\pgfqpoint{4.932007in}{0.691732in}}%
\pgfpathlineto{\pgfqpoint{5.030943in}{0.691732in}}%
\pgfpathlineto{\pgfqpoint{5.030943in}{0.592796in}}%
\pgfpathlineto{\pgfqpoint{4.932007in}{0.592796in}}%
\pgfpathlineto{\pgfqpoint{4.932007in}{0.691732in}}%
\pgfusepath{stroke}%
\end{pgfscope}%
\begin{pgfscope}%
\pgfpathrectangle{\pgfqpoint{0.380943in}{0.295988in}}{\pgfqpoint{4.650000in}{0.692553in}}%
\pgfusepath{clip}%
\pgfsetbuttcap%
\pgfsetroundjoin%
\definecolor{currentfill}{rgb}{1.000000,1.000000,0.908266}%
\pgfsetfillcolor{currentfill}%
\pgfsetlinewidth{0.250937pt}%
\definecolor{currentstroke}{rgb}{1.000000,1.000000,1.000000}%
\pgfsetstrokecolor{currentstroke}%
\pgfsetdash{}{0pt}%
\pgfpathmoveto{\pgfqpoint{0.380943in}{0.592796in}}%
\pgfpathlineto{\pgfqpoint{0.479879in}{0.592796in}}%
\pgfpathlineto{\pgfqpoint{0.479879in}{0.493860in}}%
\pgfpathlineto{\pgfqpoint{0.380943in}{0.493860in}}%
\pgfpathlineto{\pgfqpoint{0.380943in}{0.592796in}}%
\pgfusepath{stroke,fill}%
\end{pgfscope}%
\begin{pgfscope}%
\pgfpathrectangle{\pgfqpoint{0.380943in}{0.295988in}}{\pgfqpoint{4.650000in}{0.692553in}}%
\pgfusepath{clip}%
\pgfsetbuttcap%
\pgfsetroundjoin%
\definecolor{currentfill}{rgb}{1.000000,1.000000,0.865975}%
\pgfsetfillcolor{currentfill}%
\pgfsetlinewidth{0.250937pt}%
\definecolor{currentstroke}{rgb}{1.000000,1.000000,1.000000}%
\pgfsetstrokecolor{currentstroke}%
\pgfsetdash{}{0pt}%
\pgfpathmoveto{\pgfqpoint{0.479879in}{0.592796in}}%
\pgfpathlineto{\pgfqpoint{0.578815in}{0.592796in}}%
\pgfpathlineto{\pgfqpoint{0.578815in}{0.493860in}}%
\pgfpathlineto{\pgfqpoint{0.479879in}{0.493860in}}%
\pgfpathlineto{\pgfqpoint{0.479879in}{0.592796in}}%
\pgfusepath{stroke,fill}%
\end{pgfscope}%
\begin{pgfscope}%
\pgfpathrectangle{\pgfqpoint{0.380943in}{0.295988in}}{\pgfqpoint{4.650000in}{0.692553in}}%
\pgfusepath{clip}%
\pgfsetbuttcap%
\pgfsetroundjoin%
\definecolor{currentfill}{rgb}{0.973241,0.954510,0.761799}%
\pgfsetfillcolor{currentfill}%
\pgfsetlinewidth{0.250937pt}%
\definecolor{currentstroke}{rgb}{1.000000,1.000000,1.000000}%
\pgfsetstrokecolor{currentstroke}%
\pgfsetdash{}{0pt}%
\pgfpathmoveto{\pgfqpoint{0.578815in}{0.592796in}}%
\pgfpathlineto{\pgfqpoint{0.677752in}{0.592796in}}%
\pgfpathlineto{\pgfqpoint{0.677752in}{0.493860in}}%
\pgfpathlineto{\pgfqpoint{0.578815in}{0.493860in}}%
\pgfpathlineto{\pgfqpoint{0.578815in}{0.592796in}}%
\pgfusepath{stroke,fill}%
\end{pgfscope}%
\begin{pgfscope}%
\pgfpathrectangle{\pgfqpoint{0.380943in}{0.295988in}}{\pgfqpoint{4.650000in}{0.692553in}}%
\pgfusepath{clip}%
\pgfsetbuttcap%
\pgfsetroundjoin%
\definecolor{currentfill}{rgb}{0.978316,0.963137,0.774994}%
\pgfsetfillcolor{currentfill}%
\pgfsetlinewidth{0.250937pt}%
\definecolor{currentstroke}{rgb}{1.000000,1.000000,1.000000}%
\pgfsetstrokecolor{currentstroke}%
\pgfsetdash{}{0pt}%
\pgfpathmoveto{\pgfqpoint{0.677752in}{0.592796in}}%
\pgfpathlineto{\pgfqpoint{0.776688in}{0.592796in}}%
\pgfpathlineto{\pgfqpoint{0.776688in}{0.493860in}}%
\pgfpathlineto{\pgfqpoint{0.677752in}{0.493860in}}%
\pgfpathlineto{\pgfqpoint{0.677752in}{0.592796in}}%
\pgfusepath{stroke,fill}%
\end{pgfscope}%
\begin{pgfscope}%
\pgfpathrectangle{\pgfqpoint{0.380943in}{0.295988in}}{\pgfqpoint{4.650000in}{0.692553in}}%
\pgfusepath{clip}%
\pgfsetbuttcap%
\pgfsetroundjoin%
\definecolor{currentfill}{rgb}{0.986774,0.977516,0.796986}%
\pgfsetfillcolor{currentfill}%
\pgfsetlinewidth{0.250937pt}%
\definecolor{currentstroke}{rgb}{1.000000,1.000000,1.000000}%
\pgfsetstrokecolor{currentstroke}%
\pgfsetdash{}{0pt}%
\pgfpathmoveto{\pgfqpoint{0.776688in}{0.592796in}}%
\pgfpathlineto{\pgfqpoint{0.875624in}{0.592796in}}%
\pgfpathlineto{\pgfqpoint{0.875624in}{0.493860in}}%
\pgfpathlineto{\pgfqpoint{0.776688in}{0.493860in}}%
\pgfpathlineto{\pgfqpoint{0.776688in}{0.592796in}}%
\pgfusepath{stroke,fill}%
\end{pgfscope}%
\begin{pgfscope}%
\pgfpathrectangle{\pgfqpoint{0.380943in}{0.295988in}}{\pgfqpoint{4.650000in}{0.692553in}}%
\pgfusepath{clip}%
\pgfsetbuttcap%
\pgfsetroundjoin%
\definecolor{currentfill}{rgb}{0.986774,0.977516,0.796986}%
\pgfsetfillcolor{currentfill}%
\pgfsetlinewidth{0.250937pt}%
\definecolor{currentstroke}{rgb}{1.000000,1.000000,1.000000}%
\pgfsetstrokecolor{currentstroke}%
\pgfsetdash{}{0pt}%
\pgfpathmoveto{\pgfqpoint{0.875624in}{0.592796in}}%
\pgfpathlineto{\pgfqpoint{0.974560in}{0.592796in}}%
\pgfpathlineto{\pgfqpoint{0.974560in}{0.493860in}}%
\pgfpathlineto{\pgfqpoint{0.875624in}{0.493860in}}%
\pgfpathlineto{\pgfqpoint{0.875624in}{0.592796in}}%
\pgfusepath{stroke,fill}%
\end{pgfscope}%
\begin{pgfscope}%
\pgfpathrectangle{\pgfqpoint{0.380943in}{0.295988in}}{\pgfqpoint{4.650000in}{0.692553in}}%
\pgfusepath{clip}%
\pgfsetbuttcap%
\pgfsetroundjoin%
\definecolor{currentfill}{rgb}{0.986774,0.977516,0.796986}%
\pgfsetfillcolor{currentfill}%
\pgfsetlinewidth{0.250937pt}%
\definecolor{currentstroke}{rgb}{1.000000,1.000000,1.000000}%
\pgfsetstrokecolor{currentstroke}%
\pgfsetdash{}{0pt}%
\pgfpathmoveto{\pgfqpoint{0.974560in}{0.592796in}}%
\pgfpathlineto{\pgfqpoint{1.073496in}{0.592796in}}%
\pgfpathlineto{\pgfqpoint{1.073496in}{0.493860in}}%
\pgfpathlineto{\pgfqpoint{0.974560in}{0.493860in}}%
\pgfpathlineto{\pgfqpoint{0.974560in}{0.592796in}}%
\pgfusepath{stroke,fill}%
\end{pgfscope}%
\begin{pgfscope}%
\pgfpathrectangle{\pgfqpoint{0.380943in}{0.295988in}}{\pgfqpoint{4.650000in}{0.692553in}}%
\pgfusepath{clip}%
\pgfsetbuttcap%
\pgfsetroundjoin%
\definecolor{currentfill}{rgb}{0.973241,0.954510,0.761799}%
\pgfsetfillcolor{currentfill}%
\pgfsetlinewidth{0.250937pt}%
\definecolor{currentstroke}{rgb}{1.000000,1.000000,1.000000}%
\pgfsetstrokecolor{currentstroke}%
\pgfsetdash{}{0pt}%
\pgfpathmoveto{\pgfqpoint{1.073496in}{0.592796in}}%
\pgfpathlineto{\pgfqpoint{1.172432in}{0.592796in}}%
\pgfpathlineto{\pgfqpoint{1.172432in}{0.493860in}}%
\pgfpathlineto{\pgfqpoint{1.073496in}{0.493860in}}%
\pgfpathlineto{\pgfqpoint{1.073496in}{0.592796in}}%
\pgfusepath{stroke,fill}%
\end{pgfscope}%
\begin{pgfscope}%
\pgfpathrectangle{\pgfqpoint{0.380943in}{0.295988in}}{\pgfqpoint{4.650000in}{0.692553in}}%
\pgfusepath{clip}%
\pgfsetbuttcap%
\pgfsetroundjoin%
\definecolor{currentfill}{rgb}{0.978316,0.963137,0.774994}%
\pgfsetfillcolor{currentfill}%
\pgfsetlinewidth{0.250937pt}%
\definecolor{currentstroke}{rgb}{1.000000,1.000000,1.000000}%
\pgfsetstrokecolor{currentstroke}%
\pgfsetdash{}{0pt}%
\pgfpathmoveto{\pgfqpoint{1.172432in}{0.592796in}}%
\pgfpathlineto{\pgfqpoint{1.271369in}{0.592796in}}%
\pgfpathlineto{\pgfqpoint{1.271369in}{0.493860in}}%
\pgfpathlineto{\pgfqpoint{1.172432in}{0.493860in}}%
\pgfpathlineto{\pgfqpoint{1.172432in}{0.592796in}}%
\pgfusepath{stroke,fill}%
\end{pgfscope}%
\begin{pgfscope}%
\pgfpathrectangle{\pgfqpoint{0.380943in}{0.295988in}}{\pgfqpoint{4.650000in}{0.692553in}}%
\pgfusepath{clip}%
\pgfsetbuttcap%
\pgfsetroundjoin%
\definecolor{currentfill}{rgb}{0.996740,0.716140,0.586820}%
\pgfsetfillcolor{currentfill}%
\pgfsetlinewidth{0.250937pt}%
\definecolor{currentstroke}{rgb}{1.000000,1.000000,1.000000}%
\pgfsetstrokecolor{currentstroke}%
\pgfsetdash{}{0pt}%
\pgfpathmoveto{\pgfqpoint{1.271369in}{0.592796in}}%
\pgfpathlineto{\pgfqpoint{1.370305in}{0.592796in}}%
\pgfpathlineto{\pgfqpoint{1.370305in}{0.493860in}}%
\pgfpathlineto{\pgfqpoint{1.271369in}{0.493860in}}%
\pgfpathlineto{\pgfqpoint{1.271369in}{0.592796in}}%
\pgfusepath{stroke,fill}%
\end{pgfscope}%
\begin{pgfscope}%
\pgfpathrectangle{\pgfqpoint{0.380943in}{0.295988in}}{\pgfqpoint{4.650000in}{0.692553in}}%
\pgfusepath{clip}%
\pgfsetbuttcap%
\pgfsetroundjoin%
\definecolor{currentfill}{rgb}{0.986251,0.808597,0.643230}%
\pgfsetfillcolor{currentfill}%
\pgfsetlinewidth{0.250937pt}%
\definecolor{currentstroke}{rgb}{1.000000,1.000000,1.000000}%
\pgfsetstrokecolor{currentstroke}%
\pgfsetdash{}{0pt}%
\pgfpathmoveto{\pgfqpoint{1.370305in}{0.592796in}}%
\pgfpathlineto{\pgfqpoint{1.469241in}{0.592796in}}%
\pgfpathlineto{\pgfqpoint{1.469241in}{0.493860in}}%
\pgfpathlineto{\pgfqpoint{1.370305in}{0.493860in}}%
\pgfpathlineto{\pgfqpoint{1.370305in}{0.592796in}}%
\pgfusepath{stroke,fill}%
\end{pgfscope}%
\begin{pgfscope}%
\pgfpathrectangle{\pgfqpoint{0.380943in}{0.295988in}}{\pgfqpoint{4.650000in}{0.692553in}}%
\pgfusepath{clip}%
\pgfsetbuttcap%
\pgfsetroundjoin%
\definecolor{currentfill}{rgb}{0.997247,0.702945,0.579715}%
\pgfsetfillcolor{currentfill}%
\pgfsetlinewidth{0.250937pt}%
\definecolor{currentstroke}{rgb}{1.000000,1.000000,1.000000}%
\pgfsetstrokecolor{currentstroke}%
\pgfsetdash{}{0pt}%
\pgfpathmoveto{\pgfqpoint{1.469241in}{0.592796in}}%
\pgfpathlineto{\pgfqpoint{1.568177in}{0.592796in}}%
\pgfpathlineto{\pgfqpoint{1.568177in}{0.493860in}}%
\pgfpathlineto{\pgfqpoint{1.469241in}{0.493860in}}%
\pgfpathlineto{\pgfqpoint{1.469241in}{0.592796in}}%
\pgfusepath{stroke,fill}%
\end{pgfscope}%
\begin{pgfscope}%
\pgfpathrectangle{\pgfqpoint{0.380943in}{0.295988in}}{\pgfqpoint{4.650000in}{0.692553in}}%
\pgfusepath{clip}%
\pgfsetbuttcap%
\pgfsetroundjoin%
\definecolor{currentfill}{rgb}{0.999446,0.645767,0.548927}%
\pgfsetfillcolor{currentfill}%
\pgfsetlinewidth{0.250937pt}%
\definecolor{currentstroke}{rgb}{1.000000,1.000000,1.000000}%
\pgfsetstrokecolor{currentstroke}%
\pgfsetdash{}{0pt}%
\pgfpathmoveto{\pgfqpoint{1.568177in}{0.592796in}}%
\pgfpathlineto{\pgfqpoint{1.667113in}{0.592796in}}%
\pgfpathlineto{\pgfqpoint{1.667113in}{0.493860in}}%
\pgfpathlineto{\pgfqpoint{1.568177in}{0.493860in}}%
\pgfpathlineto{\pgfqpoint{1.568177in}{0.592796in}}%
\pgfusepath{stroke,fill}%
\end{pgfscope}%
\begin{pgfscope}%
\pgfpathrectangle{\pgfqpoint{0.380943in}{0.295988in}}{\pgfqpoint{4.650000in}{0.692553in}}%
\pgfusepath{clip}%
\pgfsetbuttcap%
\pgfsetroundjoin%
\definecolor{currentfill}{rgb}{0.997247,0.702945,0.579715}%
\pgfsetfillcolor{currentfill}%
\pgfsetlinewidth{0.250937pt}%
\definecolor{currentstroke}{rgb}{1.000000,1.000000,1.000000}%
\pgfsetstrokecolor{currentstroke}%
\pgfsetdash{}{0pt}%
\pgfpathmoveto{\pgfqpoint{1.667113in}{0.592796in}}%
\pgfpathlineto{\pgfqpoint{1.766049in}{0.592796in}}%
\pgfpathlineto{\pgfqpoint{1.766049in}{0.493860in}}%
\pgfpathlineto{\pgfqpoint{1.667113in}{0.493860in}}%
\pgfpathlineto{\pgfqpoint{1.667113in}{0.592796in}}%
\pgfusepath{stroke,fill}%
\end{pgfscope}%
\begin{pgfscope}%
\pgfpathrectangle{\pgfqpoint{0.380943in}{0.295988in}}{\pgfqpoint{4.650000in}{0.692553in}}%
\pgfusepath{clip}%
\pgfsetbuttcap%
\pgfsetroundjoin%
\definecolor{currentfill}{rgb}{0.994694,0.745098,0.602999}%
\pgfsetfillcolor{currentfill}%
\pgfsetlinewidth{0.250937pt}%
\definecolor{currentstroke}{rgb}{1.000000,1.000000,1.000000}%
\pgfsetstrokecolor{currentstroke}%
\pgfsetdash{}{0pt}%
\pgfpathmoveto{\pgfqpoint{1.766049in}{0.592796in}}%
\pgfpathlineto{\pgfqpoint{1.864986in}{0.592796in}}%
\pgfpathlineto{\pgfqpoint{1.864986in}{0.493860in}}%
\pgfpathlineto{\pgfqpoint{1.766049in}{0.493860in}}%
\pgfpathlineto{\pgfqpoint{1.766049in}{0.592796in}}%
\pgfusepath{stroke,fill}%
\end{pgfscope}%
\begin{pgfscope}%
\pgfpathrectangle{\pgfqpoint{0.380943in}{0.295988in}}{\pgfqpoint{4.650000in}{0.692553in}}%
\pgfusepath{clip}%
\pgfsetbuttcap%
\pgfsetroundjoin%
\definecolor{currentfill}{rgb}{0.998939,0.658962,0.556032}%
\pgfsetfillcolor{currentfill}%
\pgfsetlinewidth{0.250937pt}%
\definecolor{currentstroke}{rgb}{1.000000,1.000000,1.000000}%
\pgfsetstrokecolor{currentstroke}%
\pgfsetdash{}{0pt}%
\pgfpathmoveto{\pgfqpoint{1.864986in}{0.592796in}}%
\pgfpathlineto{\pgfqpoint{1.963922in}{0.592796in}}%
\pgfpathlineto{\pgfqpoint{1.963922in}{0.493860in}}%
\pgfpathlineto{\pgfqpoint{1.864986in}{0.493860in}}%
\pgfpathlineto{\pgfqpoint{1.864986in}{0.592796in}}%
\pgfusepath{stroke,fill}%
\end{pgfscope}%
\begin{pgfscope}%
\pgfpathrectangle{\pgfqpoint{0.380943in}{0.295988in}}{\pgfqpoint{4.650000in}{0.692553in}}%
\pgfusepath{clip}%
\pgfsetbuttcap%
\pgfsetroundjoin%
\definecolor{currentfill}{rgb}{0.995709,0.736471,0.597924}%
\pgfsetfillcolor{currentfill}%
\pgfsetlinewidth{0.250937pt}%
\definecolor{currentstroke}{rgb}{1.000000,1.000000,1.000000}%
\pgfsetstrokecolor{currentstroke}%
\pgfsetdash{}{0pt}%
\pgfpathmoveto{\pgfqpoint{1.963922in}{0.592796in}}%
\pgfpathlineto{\pgfqpoint{2.062858in}{0.592796in}}%
\pgfpathlineto{\pgfqpoint{2.062858in}{0.493860in}}%
\pgfpathlineto{\pgfqpoint{1.963922in}{0.493860in}}%
\pgfpathlineto{\pgfqpoint{1.963922in}{0.592796in}}%
\pgfusepath{stroke,fill}%
\end{pgfscope}%
\begin{pgfscope}%
\pgfpathrectangle{\pgfqpoint{0.380943in}{0.295988in}}{\pgfqpoint{4.650000in}{0.692553in}}%
\pgfusepath{clip}%
\pgfsetbuttcap%
\pgfsetroundjoin%
\definecolor{currentfill}{rgb}{0.988604,0.796863,0.633449}%
\pgfsetfillcolor{currentfill}%
\pgfsetlinewidth{0.250937pt}%
\definecolor{currentstroke}{rgb}{1.000000,1.000000,1.000000}%
\pgfsetstrokecolor{currentstroke}%
\pgfsetdash{}{0pt}%
\pgfpathmoveto{\pgfqpoint{2.062858in}{0.592796in}}%
\pgfpathlineto{\pgfqpoint{2.161794in}{0.592796in}}%
\pgfpathlineto{\pgfqpoint{2.161794in}{0.493860in}}%
\pgfpathlineto{\pgfqpoint{2.062858in}{0.493860in}}%
\pgfpathlineto{\pgfqpoint{2.062858in}{0.592796in}}%
\pgfusepath{stroke,fill}%
\end{pgfscope}%
\begin{pgfscope}%
\pgfpathrectangle{\pgfqpoint{0.380943in}{0.295988in}}{\pgfqpoint{4.650000in}{0.692553in}}%
\pgfusepath{clip}%
\pgfsetbuttcap%
\pgfsetroundjoin%
\definecolor{currentfill}{rgb}{0.983714,0.819592,0.653379}%
\pgfsetfillcolor{currentfill}%
\pgfsetlinewidth{0.250937pt}%
\definecolor{currentstroke}{rgb}{1.000000,1.000000,1.000000}%
\pgfsetstrokecolor{currentstroke}%
\pgfsetdash{}{0pt}%
\pgfpathmoveto{\pgfqpoint{2.161794in}{0.592796in}}%
\pgfpathlineto{\pgfqpoint{2.260730in}{0.592796in}}%
\pgfpathlineto{\pgfqpoint{2.260730in}{0.493860in}}%
\pgfpathlineto{\pgfqpoint{2.161794in}{0.493860in}}%
\pgfpathlineto{\pgfqpoint{2.161794in}{0.592796in}}%
\pgfusepath{stroke,fill}%
\end{pgfscope}%
\begin{pgfscope}%
\pgfpathrectangle{\pgfqpoint{0.380943in}{0.295988in}}{\pgfqpoint{4.650000in}{0.692553in}}%
\pgfusepath{clip}%
\pgfsetbuttcap%
\pgfsetroundjoin%
\definecolor{currentfill}{rgb}{0.982191,0.826190,0.659469}%
\pgfsetfillcolor{currentfill}%
\pgfsetlinewidth{0.250937pt}%
\definecolor{currentstroke}{rgb}{1.000000,1.000000,1.000000}%
\pgfsetstrokecolor{currentstroke}%
\pgfsetdash{}{0pt}%
\pgfpathmoveto{\pgfqpoint{2.260730in}{0.592796in}}%
\pgfpathlineto{\pgfqpoint{2.359666in}{0.592796in}}%
\pgfpathlineto{\pgfqpoint{2.359666in}{0.493860in}}%
\pgfpathlineto{\pgfqpoint{2.260730in}{0.493860in}}%
\pgfpathlineto{\pgfqpoint{2.260730in}{0.592796in}}%
\pgfusepath{stroke,fill}%
\end{pgfscope}%
\begin{pgfscope}%
\pgfpathrectangle{\pgfqpoint{0.380943in}{0.295988in}}{\pgfqpoint{4.650000in}{0.692553in}}%
\pgfusepath{clip}%
\pgfsetbuttcap%
\pgfsetroundjoin%
\definecolor{currentfill}{rgb}{0.975594,0.855363,0.684691}%
\pgfsetfillcolor{currentfill}%
\pgfsetlinewidth{0.250937pt}%
\definecolor{currentstroke}{rgb}{1.000000,1.000000,1.000000}%
\pgfsetstrokecolor{currentstroke}%
\pgfsetdash{}{0pt}%
\pgfpathmoveto{\pgfqpoint{2.359666in}{0.592796in}}%
\pgfpathlineto{\pgfqpoint{2.458603in}{0.592796in}}%
\pgfpathlineto{\pgfqpoint{2.458603in}{0.493860in}}%
\pgfpathlineto{\pgfqpoint{2.359666in}{0.493860in}}%
\pgfpathlineto{\pgfqpoint{2.359666in}{0.592796in}}%
\pgfusepath{stroke,fill}%
\end{pgfscope}%
\begin{pgfscope}%
\pgfpathrectangle{\pgfqpoint{0.380943in}{0.295988in}}{\pgfqpoint{4.650000in}{0.692553in}}%
\pgfusepath{clip}%
\pgfsetbuttcap%
\pgfsetroundjoin%
\definecolor{currentfill}{rgb}{0.996740,0.716140,0.586820}%
\pgfsetfillcolor{currentfill}%
\pgfsetlinewidth{0.250937pt}%
\definecolor{currentstroke}{rgb}{1.000000,1.000000,1.000000}%
\pgfsetstrokecolor{currentstroke}%
\pgfsetdash{}{0pt}%
\pgfpathmoveto{\pgfqpoint{2.458603in}{0.592796in}}%
\pgfpathlineto{\pgfqpoint{2.557539in}{0.592796in}}%
\pgfpathlineto{\pgfqpoint{2.557539in}{0.493860in}}%
\pgfpathlineto{\pgfqpoint{2.458603in}{0.493860in}}%
\pgfpathlineto{\pgfqpoint{2.458603in}{0.592796in}}%
\pgfusepath{stroke,fill}%
\end{pgfscope}%
\begin{pgfscope}%
\pgfpathrectangle{\pgfqpoint{0.380943in}{0.295988in}}{\pgfqpoint{4.650000in}{0.692553in}}%
\pgfusepath{clip}%
\pgfsetbuttcap%
\pgfsetroundjoin%
\definecolor{currentfill}{rgb}{1.000000,0.554479,0.510419}%
\pgfsetfillcolor{currentfill}%
\pgfsetlinewidth{0.250937pt}%
\definecolor{currentstroke}{rgb}{1.000000,1.000000,1.000000}%
\pgfsetstrokecolor{currentstroke}%
\pgfsetdash{}{0pt}%
\pgfpathmoveto{\pgfqpoint{2.557539in}{0.592796in}}%
\pgfpathlineto{\pgfqpoint{2.656475in}{0.592796in}}%
\pgfpathlineto{\pgfqpoint{2.656475in}{0.493860in}}%
\pgfpathlineto{\pgfqpoint{2.557539in}{0.493860in}}%
\pgfpathlineto{\pgfqpoint{2.557539in}{0.592796in}}%
\pgfusepath{stroke,fill}%
\end{pgfscope}%
\begin{pgfscope}%
\pgfpathrectangle{\pgfqpoint{0.380943in}{0.295988in}}{\pgfqpoint{4.650000in}{0.692553in}}%
\pgfusepath{clip}%
\pgfsetbuttcap%
\pgfsetroundjoin%
\definecolor{currentfill}{rgb}{0.989619,0.788235,0.628374}%
\pgfsetfillcolor{currentfill}%
\pgfsetlinewidth{0.250937pt}%
\definecolor{currentstroke}{rgb}{1.000000,1.000000,1.000000}%
\pgfsetstrokecolor{currentstroke}%
\pgfsetdash{}{0pt}%
\pgfpathmoveto{\pgfqpoint{2.656475in}{0.592796in}}%
\pgfpathlineto{\pgfqpoint{2.755411in}{0.592796in}}%
\pgfpathlineto{\pgfqpoint{2.755411in}{0.493860in}}%
\pgfpathlineto{\pgfqpoint{2.656475in}{0.493860in}}%
\pgfpathlineto{\pgfqpoint{2.656475in}{0.592796in}}%
\pgfusepath{stroke,fill}%
\end{pgfscope}%
\begin{pgfscope}%
\pgfpathrectangle{\pgfqpoint{0.380943in}{0.295988in}}{\pgfqpoint{4.650000in}{0.692553in}}%
\pgfusepath{clip}%
\pgfsetbuttcap%
\pgfsetroundjoin%
\definecolor{currentfill}{rgb}{0.995709,0.736471,0.597924}%
\pgfsetfillcolor{currentfill}%
\pgfsetlinewidth{0.250937pt}%
\definecolor{currentstroke}{rgb}{1.000000,1.000000,1.000000}%
\pgfsetstrokecolor{currentstroke}%
\pgfsetdash{}{0pt}%
\pgfpathmoveto{\pgfqpoint{2.755411in}{0.592796in}}%
\pgfpathlineto{\pgfqpoint{2.854347in}{0.592796in}}%
\pgfpathlineto{\pgfqpoint{2.854347in}{0.493860in}}%
\pgfpathlineto{\pgfqpoint{2.755411in}{0.493860in}}%
\pgfpathlineto{\pgfqpoint{2.755411in}{0.592796in}}%
\pgfusepath{stroke,fill}%
\end{pgfscope}%
\begin{pgfscope}%
\pgfpathrectangle{\pgfqpoint{0.380943in}{0.295988in}}{\pgfqpoint{4.650000in}{0.692553in}}%
\pgfusepath{clip}%
\pgfsetbuttcap%
\pgfsetroundjoin%
\definecolor{currentfill}{rgb}{0.993003,0.759477,0.611457}%
\pgfsetfillcolor{currentfill}%
\pgfsetlinewidth{0.250937pt}%
\definecolor{currentstroke}{rgb}{1.000000,1.000000,1.000000}%
\pgfsetstrokecolor{currentstroke}%
\pgfsetdash{}{0pt}%
\pgfpathmoveto{\pgfqpoint{2.854347in}{0.592796in}}%
\pgfpathlineto{\pgfqpoint{2.953283in}{0.592796in}}%
\pgfpathlineto{\pgfqpoint{2.953283in}{0.493860in}}%
\pgfpathlineto{\pgfqpoint{2.854347in}{0.493860in}}%
\pgfpathlineto{\pgfqpoint{2.854347in}{0.592796in}}%
\pgfusepath{stroke,fill}%
\end{pgfscope}%
\begin{pgfscope}%
\pgfpathrectangle{\pgfqpoint{0.380943in}{0.295988in}}{\pgfqpoint{4.650000in}{0.692553in}}%
\pgfusepath{clip}%
\pgfsetbuttcap%
\pgfsetroundjoin%
\definecolor{currentfill}{rgb}{0.994018,0.750850,0.606382}%
\pgfsetfillcolor{currentfill}%
\pgfsetlinewidth{0.250937pt}%
\definecolor{currentstroke}{rgb}{1.000000,1.000000,1.000000}%
\pgfsetstrokecolor{currentstroke}%
\pgfsetdash{}{0pt}%
\pgfpathmoveto{\pgfqpoint{2.953283in}{0.592796in}}%
\pgfpathlineto{\pgfqpoint{3.052220in}{0.592796in}}%
\pgfpathlineto{\pgfqpoint{3.052220in}{0.493860in}}%
\pgfpathlineto{\pgfqpoint{2.953283in}{0.493860in}}%
\pgfpathlineto{\pgfqpoint{2.953283in}{0.592796in}}%
\pgfusepath{stroke,fill}%
\end{pgfscope}%
\begin{pgfscope}%
\pgfpathrectangle{\pgfqpoint{0.380943in}{0.295988in}}{\pgfqpoint{4.650000in}{0.692553in}}%
\pgfusepath{clip}%
\pgfsetbuttcap%
\pgfsetroundjoin%
\definecolor{currentfill}{rgb}{0.990296,0.782484,0.624990}%
\pgfsetfillcolor{currentfill}%
\pgfsetlinewidth{0.250937pt}%
\definecolor{currentstroke}{rgb}{1.000000,1.000000,1.000000}%
\pgfsetstrokecolor{currentstroke}%
\pgfsetdash{}{0pt}%
\pgfpathmoveto{\pgfqpoint{3.052220in}{0.592796in}}%
\pgfpathlineto{\pgfqpoint{3.151156in}{0.592796in}}%
\pgfpathlineto{\pgfqpoint{3.151156in}{0.493860in}}%
\pgfpathlineto{\pgfqpoint{3.052220in}{0.493860in}}%
\pgfpathlineto{\pgfqpoint{3.052220in}{0.592796in}}%
\pgfusepath{stroke,fill}%
\end{pgfscope}%
\begin{pgfscope}%
\pgfpathrectangle{\pgfqpoint{0.380943in}{0.295988in}}{\pgfqpoint{4.650000in}{0.692553in}}%
\pgfusepath{clip}%
\pgfsetbuttcap%
\pgfsetroundjoin%
\definecolor{currentfill}{rgb}{0.990296,0.782484,0.624990}%
\pgfsetfillcolor{currentfill}%
\pgfsetlinewidth{0.250937pt}%
\definecolor{currentstroke}{rgb}{1.000000,1.000000,1.000000}%
\pgfsetstrokecolor{currentstroke}%
\pgfsetdash{}{0pt}%
\pgfpathmoveto{\pgfqpoint{3.151156in}{0.592796in}}%
\pgfpathlineto{\pgfqpoint{3.250092in}{0.592796in}}%
\pgfpathlineto{\pgfqpoint{3.250092in}{0.493860in}}%
\pgfpathlineto{\pgfqpoint{3.151156in}{0.493860in}}%
\pgfpathlineto{\pgfqpoint{3.151156in}{0.592796in}}%
\pgfusepath{stroke,fill}%
\end{pgfscope}%
\begin{pgfscope}%
\pgfpathrectangle{\pgfqpoint{0.380943in}{0.295988in}}{\pgfqpoint{4.650000in}{0.692553in}}%
\pgfusepath{clip}%
\pgfsetbuttcap%
\pgfsetroundjoin%
\definecolor{currentfill}{rgb}{1.000000,0.571396,0.517186}%
\pgfsetfillcolor{currentfill}%
\pgfsetlinewidth{0.250937pt}%
\definecolor{currentstroke}{rgb}{1.000000,1.000000,1.000000}%
\pgfsetstrokecolor{currentstroke}%
\pgfsetdash{}{0pt}%
\pgfpathmoveto{\pgfqpoint{3.250092in}{0.592796in}}%
\pgfpathlineto{\pgfqpoint{3.349028in}{0.592796in}}%
\pgfpathlineto{\pgfqpoint{3.349028in}{0.493860in}}%
\pgfpathlineto{\pgfqpoint{3.250092in}{0.493860in}}%
\pgfpathlineto{\pgfqpoint{3.250092in}{0.592796in}}%
\pgfusepath{stroke,fill}%
\end{pgfscope}%
\begin{pgfscope}%
\pgfpathrectangle{\pgfqpoint{0.380943in}{0.295988in}}{\pgfqpoint{4.650000in}{0.692553in}}%
\pgfusepath{clip}%
\pgfsetbuttcap%
\pgfsetroundjoin%
\definecolor{currentfill}{rgb}{1.000000,0.554479,0.510419}%
\pgfsetfillcolor{currentfill}%
\pgfsetlinewidth{0.250937pt}%
\definecolor{currentstroke}{rgb}{1.000000,1.000000,1.000000}%
\pgfsetstrokecolor{currentstroke}%
\pgfsetdash{}{0pt}%
\pgfpathmoveto{\pgfqpoint{3.349028in}{0.592796in}}%
\pgfpathlineto{\pgfqpoint{3.447964in}{0.592796in}}%
\pgfpathlineto{\pgfqpoint{3.447964in}{0.493860in}}%
\pgfpathlineto{\pgfqpoint{3.349028in}{0.493860in}}%
\pgfpathlineto{\pgfqpoint{3.349028in}{0.592796in}}%
\pgfusepath{stroke,fill}%
\end{pgfscope}%
\begin{pgfscope}%
\pgfpathrectangle{\pgfqpoint{0.380943in}{0.295988in}}{\pgfqpoint{4.650000in}{0.692553in}}%
\pgfusepath{clip}%
\pgfsetbuttcap%
\pgfsetroundjoin%
\definecolor{currentfill}{rgb}{1.000000,0.512618,0.492826}%
\pgfsetfillcolor{currentfill}%
\pgfsetlinewidth{0.250937pt}%
\definecolor{currentstroke}{rgb}{1.000000,1.000000,1.000000}%
\pgfsetstrokecolor{currentstroke}%
\pgfsetdash{}{0pt}%
\pgfpathmoveto{\pgfqpoint{3.447964in}{0.592796in}}%
\pgfpathlineto{\pgfqpoint{3.546901in}{0.592796in}}%
\pgfpathlineto{\pgfqpoint{3.546901in}{0.493860in}}%
\pgfpathlineto{\pgfqpoint{3.447964in}{0.493860in}}%
\pgfpathlineto{\pgfqpoint{3.447964in}{0.592796in}}%
\pgfusepath{stroke,fill}%
\end{pgfscope}%
\begin{pgfscope}%
\pgfpathrectangle{\pgfqpoint{0.380943in}{0.295988in}}{\pgfqpoint{4.650000in}{0.692553in}}%
\pgfusepath{clip}%
\pgfsetbuttcap%
\pgfsetroundjoin%
\definecolor{currentfill}{rgb}{0.989619,0.788235,0.628374}%
\pgfsetfillcolor{currentfill}%
\pgfsetlinewidth{0.250937pt}%
\definecolor{currentstroke}{rgb}{1.000000,1.000000,1.000000}%
\pgfsetstrokecolor{currentstroke}%
\pgfsetdash{}{0pt}%
\pgfpathmoveto{\pgfqpoint{3.546901in}{0.592796in}}%
\pgfpathlineto{\pgfqpoint{3.645837in}{0.592796in}}%
\pgfpathlineto{\pgfqpoint{3.645837in}{0.493860in}}%
\pgfpathlineto{\pgfqpoint{3.546901in}{0.493860in}}%
\pgfpathlineto{\pgfqpoint{3.546901in}{0.592796in}}%
\pgfusepath{stroke,fill}%
\end{pgfscope}%
\begin{pgfscope}%
\pgfpathrectangle{\pgfqpoint{0.380943in}{0.295988in}}{\pgfqpoint{4.650000in}{0.692553in}}%
\pgfusepath{clip}%
\pgfsetbuttcap%
\pgfsetroundjoin%
\definecolor{currentfill}{rgb}{0.999785,0.636970,0.544191}%
\pgfsetfillcolor{currentfill}%
\pgfsetlinewidth{0.250937pt}%
\definecolor{currentstroke}{rgb}{1.000000,1.000000,1.000000}%
\pgfsetstrokecolor{currentstroke}%
\pgfsetdash{}{0pt}%
\pgfpathmoveto{\pgfqpoint{3.645837in}{0.592796in}}%
\pgfpathlineto{\pgfqpoint{3.744773in}{0.592796in}}%
\pgfpathlineto{\pgfqpoint{3.744773in}{0.493860in}}%
\pgfpathlineto{\pgfqpoint{3.645837in}{0.493860in}}%
\pgfpathlineto{\pgfqpoint{3.645837in}{0.592796in}}%
\pgfusepath{stroke,fill}%
\end{pgfscope}%
\begin{pgfscope}%
\pgfpathrectangle{\pgfqpoint{0.380943in}{0.295988in}}{\pgfqpoint{4.650000in}{0.692553in}}%
\pgfusepath{clip}%
\pgfsetbuttcap%
\pgfsetroundjoin%
\definecolor{currentfill}{rgb}{1.000000,0.625529,0.538839}%
\pgfsetfillcolor{currentfill}%
\pgfsetlinewidth{0.250937pt}%
\definecolor{currentstroke}{rgb}{1.000000,1.000000,1.000000}%
\pgfsetstrokecolor{currentstroke}%
\pgfsetdash{}{0pt}%
\pgfpathmoveto{\pgfqpoint{3.744773in}{0.592796in}}%
\pgfpathlineto{\pgfqpoint{3.843709in}{0.592796in}}%
\pgfpathlineto{\pgfqpoint{3.843709in}{0.493860in}}%
\pgfpathlineto{\pgfqpoint{3.744773in}{0.493860in}}%
\pgfpathlineto{\pgfqpoint{3.744773in}{0.592796in}}%
\pgfusepath{stroke,fill}%
\end{pgfscope}%
\begin{pgfscope}%
\pgfpathrectangle{\pgfqpoint{0.380943in}{0.295988in}}{\pgfqpoint{4.650000in}{0.692553in}}%
\pgfusepath{clip}%
\pgfsetbuttcap%
\pgfsetroundjoin%
\definecolor{currentfill}{rgb}{0.956171,0.434602,0.434602}%
\pgfsetfillcolor{currentfill}%
\pgfsetlinewidth{0.250937pt}%
\definecolor{currentstroke}{rgb}{1.000000,1.000000,1.000000}%
\pgfsetstrokecolor{currentstroke}%
\pgfsetdash{}{0pt}%
\pgfpathmoveto{\pgfqpoint{3.843709in}{0.592796in}}%
\pgfpathlineto{\pgfqpoint{3.942645in}{0.592796in}}%
\pgfpathlineto{\pgfqpoint{3.942645in}{0.493860in}}%
\pgfpathlineto{\pgfqpoint{3.843709in}{0.493860in}}%
\pgfpathlineto{\pgfqpoint{3.843709in}{0.592796in}}%
\pgfusepath{stroke,fill}%
\end{pgfscope}%
\begin{pgfscope}%
\pgfpathrectangle{\pgfqpoint{0.380943in}{0.295988in}}{\pgfqpoint{4.650000in}{0.692553in}}%
\pgfusepath{clip}%
\pgfsetbuttcap%
\pgfsetroundjoin%
\definecolor{currentfill}{rgb}{1.000000,0.598462,0.528012}%
\pgfsetfillcolor{currentfill}%
\pgfsetlinewidth{0.250937pt}%
\definecolor{currentstroke}{rgb}{1.000000,1.000000,1.000000}%
\pgfsetstrokecolor{currentstroke}%
\pgfsetdash{}{0pt}%
\pgfpathmoveto{\pgfqpoint{3.942645in}{0.592796in}}%
\pgfpathlineto{\pgfqpoint{4.041581in}{0.592796in}}%
\pgfpathlineto{\pgfqpoint{4.041581in}{0.493860in}}%
\pgfpathlineto{\pgfqpoint{3.942645in}{0.493860in}}%
\pgfpathlineto{\pgfqpoint{3.942645in}{0.592796in}}%
\pgfusepath{stroke,fill}%
\end{pgfscope}%
\begin{pgfscope}%
\pgfpathrectangle{\pgfqpoint{0.380943in}{0.295988in}}{\pgfqpoint{4.650000in}{0.692553in}}%
\pgfusepath{clip}%
\pgfsetbuttcap%
\pgfsetroundjoin%
\definecolor{currentfill}{rgb}{1.000000,0.522261,0.496886}%
\pgfsetfillcolor{currentfill}%
\pgfsetlinewidth{0.250937pt}%
\definecolor{currentstroke}{rgb}{1.000000,1.000000,1.000000}%
\pgfsetstrokecolor{currentstroke}%
\pgfsetdash{}{0pt}%
\pgfpathmoveto{\pgfqpoint{4.041581in}{0.592796in}}%
\pgfpathlineto{\pgfqpoint{4.140518in}{0.592796in}}%
\pgfpathlineto{\pgfqpoint{4.140518in}{0.493860in}}%
\pgfpathlineto{\pgfqpoint{4.041581in}{0.493860in}}%
\pgfpathlineto{\pgfqpoint{4.041581in}{0.592796in}}%
\pgfusepath{stroke,fill}%
\end{pgfscope}%
\begin{pgfscope}%
\pgfpathrectangle{\pgfqpoint{0.380943in}{0.295988in}}{\pgfqpoint{4.650000in}{0.692553in}}%
\pgfusepath{clip}%
\pgfsetbuttcap%
\pgfsetroundjoin%
\definecolor{currentfill}{rgb}{0.999446,0.645767,0.548927}%
\pgfsetfillcolor{currentfill}%
\pgfsetlinewidth{0.250937pt}%
\definecolor{currentstroke}{rgb}{1.000000,1.000000,1.000000}%
\pgfsetstrokecolor{currentstroke}%
\pgfsetdash{}{0pt}%
\pgfpathmoveto{\pgfqpoint{4.140518in}{0.592796in}}%
\pgfpathlineto{\pgfqpoint{4.239454in}{0.592796in}}%
\pgfpathlineto{\pgfqpoint{4.239454in}{0.493860in}}%
\pgfpathlineto{\pgfqpoint{4.140518in}{0.493860in}}%
\pgfpathlineto{\pgfqpoint{4.140518in}{0.592796in}}%
\pgfusepath{stroke,fill}%
\end{pgfscope}%
\begin{pgfscope}%
\pgfpathrectangle{\pgfqpoint{0.380943in}{0.295988in}}{\pgfqpoint{4.650000in}{0.692553in}}%
\pgfusepath{clip}%
\pgfsetbuttcap%
\pgfsetroundjoin%
\definecolor{currentfill}{rgb}{0.995709,0.736471,0.597924}%
\pgfsetfillcolor{currentfill}%
\pgfsetlinewidth{0.250937pt}%
\definecolor{currentstroke}{rgb}{1.000000,1.000000,1.000000}%
\pgfsetstrokecolor{currentstroke}%
\pgfsetdash{}{0pt}%
\pgfpathmoveto{\pgfqpoint{4.239454in}{0.592796in}}%
\pgfpathlineto{\pgfqpoint{4.338390in}{0.592796in}}%
\pgfpathlineto{\pgfqpoint{4.338390in}{0.493860in}}%
\pgfpathlineto{\pgfqpoint{4.239454in}{0.493860in}}%
\pgfpathlineto{\pgfqpoint{4.239454in}{0.592796in}}%
\pgfusepath{stroke,fill}%
\end{pgfscope}%
\begin{pgfscope}%
\pgfpathrectangle{\pgfqpoint{0.380943in}{0.295988in}}{\pgfqpoint{4.650000in}{0.692553in}}%
\pgfusepath{clip}%
\pgfsetbuttcap%
\pgfsetroundjoin%
\definecolor{currentfill}{rgb}{0.963937,0.914418,0.716801}%
\pgfsetfillcolor{currentfill}%
\pgfsetlinewidth{0.250937pt}%
\definecolor{currentstroke}{rgb}{1.000000,1.000000,1.000000}%
\pgfsetstrokecolor{currentstroke}%
\pgfsetdash{}{0pt}%
\pgfpathmoveto{\pgfqpoint{4.338390in}{0.592796in}}%
\pgfpathlineto{\pgfqpoint{4.437326in}{0.592796in}}%
\pgfpathlineto{\pgfqpoint{4.437326in}{0.493860in}}%
\pgfpathlineto{\pgfqpoint{4.338390in}{0.493860in}}%
\pgfpathlineto{\pgfqpoint{4.338390in}{0.592796in}}%
\pgfusepath{stroke,fill}%
\end{pgfscope}%
\begin{pgfscope}%
\pgfpathrectangle{\pgfqpoint{0.380943in}{0.295988in}}{\pgfqpoint{4.650000in}{0.692553in}}%
\pgfusepath{clip}%
\pgfsetbuttcap%
\pgfsetroundjoin%
\definecolor{currentfill}{rgb}{0.963091,0.919493,0.720185}%
\pgfsetfillcolor{currentfill}%
\pgfsetlinewidth{0.250937pt}%
\definecolor{currentstroke}{rgb}{1.000000,1.000000,1.000000}%
\pgfsetstrokecolor{currentstroke}%
\pgfsetdash{}{0pt}%
\pgfpathmoveto{\pgfqpoint{4.437326in}{0.592796in}}%
\pgfpathlineto{\pgfqpoint{4.536262in}{0.592796in}}%
\pgfpathlineto{\pgfqpoint{4.536262in}{0.493860in}}%
\pgfpathlineto{\pgfqpoint{4.437326in}{0.493860in}}%
\pgfpathlineto{\pgfqpoint{4.437326in}{0.592796in}}%
\pgfusepath{stroke,fill}%
\end{pgfscope}%
\begin{pgfscope}%
\pgfpathrectangle{\pgfqpoint{0.380943in}{0.295988in}}{\pgfqpoint{4.650000in}{0.692553in}}%
\pgfusepath{clip}%
\pgfsetbuttcap%
\pgfsetroundjoin%
\definecolor{currentfill}{rgb}{0.961230,0.930657,0.727628}%
\pgfsetfillcolor{currentfill}%
\pgfsetlinewidth{0.250937pt}%
\definecolor{currentstroke}{rgb}{1.000000,1.000000,1.000000}%
\pgfsetstrokecolor{currentstroke}%
\pgfsetdash{}{0pt}%
\pgfpathmoveto{\pgfqpoint{4.536262in}{0.592796in}}%
\pgfpathlineto{\pgfqpoint{4.635198in}{0.592796in}}%
\pgfpathlineto{\pgfqpoint{4.635198in}{0.493860in}}%
\pgfpathlineto{\pgfqpoint{4.536262in}{0.493860in}}%
\pgfpathlineto{\pgfqpoint{4.536262in}{0.592796in}}%
\pgfusepath{stroke,fill}%
\end{pgfscope}%
\begin{pgfscope}%
\pgfpathrectangle{\pgfqpoint{0.380943in}{0.295988in}}{\pgfqpoint{4.650000in}{0.692553in}}%
\pgfusepath{clip}%
\pgfsetbuttcap%
\pgfsetroundjoin%
\definecolor{currentfill}{rgb}{0.962584,0.922537,0.722215}%
\pgfsetfillcolor{currentfill}%
\pgfsetlinewidth{0.250937pt}%
\definecolor{currentstroke}{rgb}{1.000000,1.000000,1.000000}%
\pgfsetstrokecolor{currentstroke}%
\pgfsetdash{}{0pt}%
\pgfpathmoveto{\pgfqpoint{4.635198in}{0.592796in}}%
\pgfpathlineto{\pgfqpoint{4.734135in}{0.592796in}}%
\pgfpathlineto{\pgfqpoint{4.734135in}{0.493860in}}%
\pgfpathlineto{\pgfqpoint{4.635198in}{0.493860in}}%
\pgfpathlineto{\pgfqpoint{4.635198in}{0.592796in}}%
\pgfusepath{stroke,fill}%
\end{pgfscope}%
\begin{pgfscope}%
\pgfpathrectangle{\pgfqpoint{0.380943in}{0.295988in}}{\pgfqpoint{4.650000in}{0.692553in}}%
\pgfusepath{clip}%
\pgfsetbuttcap%
\pgfsetroundjoin%
\definecolor{currentfill}{rgb}{0.964275,0.912388,0.715448}%
\pgfsetfillcolor{currentfill}%
\pgfsetlinewidth{0.250937pt}%
\definecolor{currentstroke}{rgb}{1.000000,1.000000,1.000000}%
\pgfsetstrokecolor{currentstroke}%
\pgfsetdash{}{0pt}%
\pgfpathmoveto{\pgfqpoint{4.734135in}{0.592796in}}%
\pgfpathlineto{\pgfqpoint{4.833071in}{0.592796in}}%
\pgfpathlineto{\pgfqpoint{4.833071in}{0.493860in}}%
\pgfpathlineto{\pgfqpoint{4.734135in}{0.493860in}}%
\pgfpathlineto{\pgfqpoint{4.734135in}{0.592796in}}%
\pgfusepath{stroke,fill}%
\end{pgfscope}%
\begin{pgfscope}%
\pgfpathrectangle{\pgfqpoint{0.380943in}{0.295988in}}{\pgfqpoint{4.650000in}{0.692553in}}%
\pgfusepath{clip}%
\pgfsetbuttcap%
\pgfsetroundjoin%
\definecolor{currentfill}{rgb}{0.964783,0.940131,0.739808}%
\pgfsetfillcolor{currentfill}%
\pgfsetlinewidth{0.250937pt}%
\definecolor{currentstroke}{rgb}{1.000000,1.000000,1.000000}%
\pgfsetstrokecolor{currentstroke}%
\pgfsetdash{}{0pt}%
\pgfpathmoveto{\pgfqpoint{4.833071in}{0.592796in}}%
\pgfpathlineto{\pgfqpoint{4.932007in}{0.592796in}}%
\pgfpathlineto{\pgfqpoint{4.932007in}{0.493860in}}%
\pgfpathlineto{\pgfqpoint{4.833071in}{0.493860in}}%
\pgfpathlineto{\pgfqpoint{4.833071in}{0.592796in}}%
\pgfusepath{stroke,fill}%
\end{pgfscope}%
\begin{pgfscope}%
\pgfpathrectangle{\pgfqpoint{0.380943in}{0.295988in}}{\pgfqpoint{4.650000in}{0.692553in}}%
\pgfusepath{clip}%
\pgfsetbuttcap%
\pgfsetroundjoin%
\pgfsetlinewidth{0.250937pt}%
\definecolor{currentstroke}{rgb}{1.000000,1.000000,1.000000}%
\pgfsetstrokecolor{currentstroke}%
\pgfsetdash{}{0pt}%
\pgfpathmoveto{\pgfqpoint{4.932007in}{0.592796in}}%
\pgfpathlineto{\pgfqpoint{5.030943in}{0.592796in}}%
\pgfpathlineto{\pgfqpoint{5.030943in}{0.493860in}}%
\pgfpathlineto{\pgfqpoint{4.932007in}{0.493860in}}%
\pgfpathlineto{\pgfqpoint{4.932007in}{0.592796in}}%
\pgfusepath{stroke}%
\end{pgfscope}%
\begin{pgfscope}%
\pgfpathrectangle{\pgfqpoint{0.380943in}{0.295988in}}{\pgfqpoint{4.650000in}{0.692553in}}%
\pgfusepath{clip}%
\pgfsetbuttcap%
\pgfsetroundjoin%
\definecolor{currentfill}{rgb}{1.000000,1.000000,0.899808}%
\pgfsetfillcolor{currentfill}%
\pgfsetlinewidth{0.250937pt}%
\definecolor{currentstroke}{rgb}{1.000000,1.000000,1.000000}%
\pgfsetstrokecolor{currentstroke}%
\pgfsetdash{}{0pt}%
\pgfpathmoveto{\pgfqpoint{0.380943in}{0.493860in}}%
\pgfpathlineto{\pgfqpoint{0.479879in}{0.493860in}}%
\pgfpathlineto{\pgfqpoint{0.479879in}{0.394924in}}%
\pgfpathlineto{\pgfqpoint{0.380943in}{0.394924in}}%
\pgfpathlineto{\pgfqpoint{0.380943in}{0.493860in}}%
\pgfusepath{stroke,fill}%
\end{pgfscope}%
\begin{pgfscope}%
\pgfpathrectangle{\pgfqpoint{0.380943in}{0.295988in}}{\pgfqpoint{4.650000in}{0.692553in}}%
\pgfusepath{clip}%
\pgfsetbuttcap%
\pgfsetroundjoin%
\definecolor{currentfill}{rgb}{1.000000,1.000000,0.908266}%
\pgfsetfillcolor{currentfill}%
\pgfsetlinewidth{0.250937pt}%
\definecolor{currentstroke}{rgb}{1.000000,1.000000,1.000000}%
\pgfsetstrokecolor{currentstroke}%
\pgfsetdash{}{0pt}%
\pgfpathmoveto{\pgfqpoint{0.479879in}{0.493860in}}%
\pgfpathlineto{\pgfqpoint{0.578815in}{0.493860in}}%
\pgfpathlineto{\pgfqpoint{0.578815in}{0.394924in}}%
\pgfpathlineto{\pgfqpoint{0.479879in}{0.394924in}}%
\pgfpathlineto{\pgfqpoint{0.479879in}{0.493860in}}%
\pgfusepath{stroke,fill}%
\end{pgfscope}%
\begin{pgfscope}%
\pgfpathrectangle{\pgfqpoint{0.380943in}{0.295988in}}{\pgfqpoint{4.650000in}{0.692553in}}%
\pgfusepath{clip}%
\pgfsetbuttcap%
\pgfsetroundjoin%
\definecolor{currentfill}{rgb}{1.000000,1.000000,0.920953}%
\pgfsetfillcolor{currentfill}%
\pgfsetlinewidth{0.250937pt}%
\definecolor{currentstroke}{rgb}{1.000000,1.000000,1.000000}%
\pgfsetstrokecolor{currentstroke}%
\pgfsetdash{}{0pt}%
\pgfpathmoveto{\pgfqpoint{0.578815in}{0.493860in}}%
\pgfpathlineto{\pgfqpoint{0.677752in}{0.493860in}}%
\pgfpathlineto{\pgfqpoint{0.677752in}{0.394924in}}%
\pgfpathlineto{\pgfqpoint{0.578815in}{0.394924in}}%
\pgfpathlineto{\pgfqpoint{0.578815in}{0.493860in}}%
\pgfusepath{stroke,fill}%
\end{pgfscope}%
\begin{pgfscope}%
\pgfpathrectangle{\pgfqpoint{0.380943in}{0.295988in}}{\pgfqpoint{4.650000in}{0.692553in}}%
\pgfusepath{clip}%
\pgfsetbuttcap%
\pgfsetroundjoin%
\definecolor{currentfill}{rgb}{1.000000,1.000000,0.899808}%
\pgfsetfillcolor{currentfill}%
\pgfsetlinewidth{0.250937pt}%
\definecolor{currentstroke}{rgb}{1.000000,1.000000,1.000000}%
\pgfsetstrokecolor{currentstroke}%
\pgfsetdash{}{0pt}%
\pgfpathmoveto{\pgfqpoint{0.677752in}{0.493860in}}%
\pgfpathlineto{\pgfqpoint{0.776688in}{0.493860in}}%
\pgfpathlineto{\pgfqpoint{0.776688in}{0.394924in}}%
\pgfpathlineto{\pgfqpoint{0.677752in}{0.394924in}}%
\pgfpathlineto{\pgfqpoint{0.677752in}{0.493860in}}%
\pgfusepath{stroke,fill}%
\end{pgfscope}%
\begin{pgfscope}%
\pgfpathrectangle{\pgfqpoint{0.380943in}{0.295988in}}{\pgfqpoint{4.650000in}{0.692553in}}%
\pgfusepath{clip}%
\pgfsetbuttcap%
\pgfsetroundjoin%
\definecolor{currentfill}{rgb}{1.000000,1.000000,0.899808}%
\pgfsetfillcolor{currentfill}%
\pgfsetlinewidth{0.250937pt}%
\definecolor{currentstroke}{rgb}{1.000000,1.000000,1.000000}%
\pgfsetstrokecolor{currentstroke}%
\pgfsetdash{}{0pt}%
\pgfpathmoveto{\pgfqpoint{0.776688in}{0.493860in}}%
\pgfpathlineto{\pgfqpoint{0.875624in}{0.493860in}}%
\pgfpathlineto{\pgfqpoint{0.875624in}{0.394924in}}%
\pgfpathlineto{\pgfqpoint{0.776688in}{0.394924in}}%
\pgfpathlineto{\pgfqpoint{0.776688in}{0.493860in}}%
\pgfusepath{stroke,fill}%
\end{pgfscope}%
\begin{pgfscope}%
\pgfpathrectangle{\pgfqpoint{0.380943in}{0.295988in}}{\pgfqpoint{4.650000in}{0.692553in}}%
\pgfusepath{clip}%
\pgfsetbuttcap%
\pgfsetroundjoin%
\definecolor{currentfill}{rgb}{1.000000,1.000000,0.865975}%
\pgfsetfillcolor{currentfill}%
\pgfsetlinewidth{0.250937pt}%
\definecolor{currentstroke}{rgb}{1.000000,1.000000,1.000000}%
\pgfsetstrokecolor{currentstroke}%
\pgfsetdash{}{0pt}%
\pgfpathmoveto{\pgfqpoint{0.875624in}{0.493860in}}%
\pgfpathlineto{\pgfqpoint{0.974560in}{0.493860in}}%
\pgfpathlineto{\pgfqpoint{0.974560in}{0.394924in}}%
\pgfpathlineto{\pgfqpoint{0.875624in}{0.394924in}}%
\pgfpathlineto{\pgfqpoint{0.875624in}{0.493860in}}%
\pgfusepath{stroke,fill}%
\end{pgfscope}%
\begin{pgfscope}%
\pgfpathrectangle{\pgfqpoint{0.380943in}{0.295988in}}{\pgfqpoint{4.650000in}{0.692553in}}%
\pgfusepath{clip}%
\pgfsetbuttcap%
\pgfsetroundjoin%
\definecolor{currentfill}{rgb}{1.000000,1.000000,0.832141}%
\pgfsetfillcolor{currentfill}%
\pgfsetlinewidth{0.250937pt}%
\definecolor{currentstroke}{rgb}{1.000000,1.000000,1.000000}%
\pgfsetstrokecolor{currentstroke}%
\pgfsetdash{}{0pt}%
\pgfpathmoveto{\pgfqpoint{0.974560in}{0.493860in}}%
\pgfpathlineto{\pgfqpoint{1.073496in}{0.493860in}}%
\pgfpathlineto{\pgfqpoint{1.073496in}{0.394924in}}%
\pgfpathlineto{\pgfqpoint{0.974560in}{0.394924in}}%
\pgfpathlineto{\pgfqpoint{0.974560in}{0.493860in}}%
\pgfusepath{stroke,fill}%
\end{pgfscope}%
\begin{pgfscope}%
\pgfpathrectangle{\pgfqpoint{0.380943in}{0.295988in}}{\pgfqpoint{4.650000in}{0.692553in}}%
\pgfusepath{clip}%
\pgfsetbuttcap%
\pgfsetroundjoin%
\definecolor{currentfill}{rgb}{1.000000,1.000000,0.920953}%
\pgfsetfillcolor{currentfill}%
\pgfsetlinewidth{0.250937pt}%
\definecolor{currentstroke}{rgb}{1.000000,1.000000,1.000000}%
\pgfsetstrokecolor{currentstroke}%
\pgfsetdash{}{0pt}%
\pgfpathmoveto{\pgfqpoint{1.073496in}{0.493860in}}%
\pgfpathlineto{\pgfqpoint{1.172432in}{0.493860in}}%
\pgfpathlineto{\pgfqpoint{1.172432in}{0.394924in}}%
\pgfpathlineto{\pgfqpoint{1.073496in}{0.394924in}}%
\pgfpathlineto{\pgfqpoint{1.073496in}{0.493860in}}%
\pgfusepath{stroke,fill}%
\end{pgfscope}%
\begin{pgfscope}%
\pgfpathrectangle{\pgfqpoint{0.380943in}{0.295988in}}{\pgfqpoint{4.650000in}{0.692553in}}%
\pgfusepath{clip}%
\pgfsetbuttcap%
\pgfsetroundjoin%
\definecolor{currentfill}{rgb}{1.000000,1.000000,0.887120}%
\pgfsetfillcolor{currentfill}%
\pgfsetlinewidth{0.250937pt}%
\definecolor{currentstroke}{rgb}{1.000000,1.000000,1.000000}%
\pgfsetstrokecolor{currentstroke}%
\pgfsetdash{}{0pt}%
\pgfpathmoveto{\pgfqpoint{1.172432in}{0.493860in}}%
\pgfpathlineto{\pgfqpoint{1.271369in}{0.493860in}}%
\pgfpathlineto{\pgfqpoint{1.271369in}{0.394924in}}%
\pgfpathlineto{\pgfqpoint{1.172432in}{0.394924in}}%
\pgfpathlineto{\pgfqpoint{1.172432in}{0.493860in}}%
\pgfusepath{stroke,fill}%
\end{pgfscope}%
\begin{pgfscope}%
\pgfpathrectangle{\pgfqpoint{0.380943in}{0.295988in}}{\pgfqpoint{4.650000in}{0.692553in}}%
\pgfusepath{clip}%
\pgfsetbuttcap%
\pgfsetroundjoin%
\definecolor{currentfill}{rgb}{0.973057,0.868051,0.691457}%
\pgfsetfillcolor{currentfill}%
\pgfsetlinewidth{0.250937pt}%
\definecolor{currentstroke}{rgb}{1.000000,1.000000,1.000000}%
\pgfsetstrokecolor{currentstroke}%
\pgfsetdash{}{0pt}%
\pgfpathmoveto{\pgfqpoint{1.271369in}{0.493860in}}%
\pgfpathlineto{\pgfqpoint{1.370305in}{0.493860in}}%
\pgfpathlineto{\pgfqpoint{1.370305in}{0.394924in}}%
\pgfpathlineto{\pgfqpoint{1.271369in}{0.394924in}}%
\pgfpathlineto{\pgfqpoint{1.271369in}{0.493860in}}%
\pgfusepath{stroke,fill}%
\end{pgfscope}%
\begin{pgfscope}%
\pgfpathrectangle{\pgfqpoint{0.380943in}{0.295988in}}{\pgfqpoint{4.650000in}{0.692553in}}%
\pgfusepath{clip}%
\pgfsetbuttcap%
\pgfsetroundjoin%
\definecolor{currentfill}{rgb}{0.964275,0.912388,0.715448}%
\pgfsetfillcolor{currentfill}%
\pgfsetlinewidth{0.250937pt}%
\definecolor{currentstroke}{rgb}{1.000000,1.000000,1.000000}%
\pgfsetstrokecolor{currentstroke}%
\pgfsetdash{}{0pt}%
\pgfpathmoveto{\pgfqpoint{1.370305in}{0.493860in}}%
\pgfpathlineto{\pgfqpoint{1.469241in}{0.493860in}}%
\pgfpathlineto{\pgfqpoint{1.469241in}{0.394924in}}%
\pgfpathlineto{\pgfqpoint{1.370305in}{0.394924in}}%
\pgfpathlineto{\pgfqpoint{1.370305in}{0.493860in}}%
\pgfusepath{stroke,fill}%
\end{pgfscope}%
\begin{pgfscope}%
\pgfpathrectangle{\pgfqpoint{0.380943in}{0.295988in}}{\pgfqpoint{4.650000in}{0.692553in}}%
\pgfusepath{clip}%
\pgfsetbuttcap%
\pgfsetroundjoin%
\definecolor{currentfill}{rgb}{0.978131,0.843783,0.675709}%
\pgfsetfillcolor{currentfill}%
\pgfsetlinewidth{0.250937pt}%
\definecolor{currentstroke}{rgb}{1.000000,1.000000,1.000000}%
\pgfsetstrokecolor{currentstroke}%
\pgfsetdash{}{0pt}%
\pgfpathmoveto{\pgfqpoint{1.469241in}{0.493860in}}%
\pgfpathlineto{\pgfqpoint{1.568177in}{0.493860in}}%
\pgfpathlineto{\pgfqpoint{1.568177in}{0.394924in}}%
\pgfpathlineto{\pgfqpoint{1.469241in}{0.394924in}}%
\pgfpathlineto{\pgfqpoint{1.469241in}{0.493860in}}%
\pgfusepath{stroke,fill}%
\end{pgfscope}%
\begin{pgfscope}%
\pgfpathrectangle{\pgfqpoint{0.380943in}{0.295988in}}{\pgfqpoint{4.650000in}{0.692553in}}%
\pgfusepath{clip}%
\pgfsetbuttcap%
\pgfsetroundjoin%
\definecolor{currentfill}{rgb}{0.975594,0.855363,0.684691}%
\pgfsetfillcolor{currentfill}%
\pgfsetlinewidth{0.250937pt}%
\definecolor{currentstroke}{rgb}{1.000000,1.000000,1.000000}%
\pgfsetstrokecolor{currentstroke}%
\pgfsetdash{}{0pt}%
\pgfpathmoveto{\pgfqpoint{1.568177in}{0.493860in}}%
\pgfpathlineto{\pgfqpoint{1.667113in}{0.493860in}}%
\pgfpathlineto{\pgfqpoint{1.667113in}{0.394924in}}%
\pgfpathlineto{\pgfqpoint{1.568177in}{0.394924in}}%
\pgfpathlineto{\pgfqpoint{1.568177in}{0.493860in}}%
\pgfusepath{stroke,fill}%
\end{pgfscope}%
\begin{pgfscope}%
\pgfpathrectangle{\pgfqpoint{0.380943in}{0.295988in}}{\pgfqpoint{4.650000in}{0.692553in}}%
\pgfusepath{clip}%
\pgfsetbuttcap%
\pgfsetroundjoin%
\definecolor{currentfill}{rgb}{0.974072,0.862976,0.688750}%
\pgfsetfillcolor{currentfill}%
\pgfsetlinewidth{0.250937pt}%
\definecolor{currentstroke}{rgb}{1.000000,1.000000,1.000000}%
\pgfsetstrokecolor{currentstroke}%
\pgfsetdash{}{0pt}%
\pgfpathmoveto{\pgfqpoint{1.667113in}{0.493860in}}%
\pgfpathlineto{\pgfqpoint{1.766049in}{0.493860in}}%
\pgfpathlineto{\pgfqpoint{1.766049in}{0.394924in}}%
\pgfpathlineto{\pgfqpoint{1.667113in}{0.394924in}}%
\pgfpathlineto{\pgfqpoint{1.667113in}{0.493860in}}%
\pgfusepath{stroke,fill}%
\end{pgfscope}%
\begin{pgfscope}%
\pgfpathrectangle{\pgfqpoint{0.380943in}{0.295988in}}{\pgfqpoint{4.650000in}{0.692553in}}%
\pgfusepath{clip}%
\pgfsetbuttcap%
\pgfsetroundjoin%
\definecolor{currentfill}{rgb}{0.978131,0.843783,0.675709}%
\pgfsetfillcolor{currentfill}%
\pgfsetlinewidth{0.250937pt}%
\definecolor{currentstroke}{rgb}{1.000000,1.000000,1.000000}%
\pgfsetstrokecolor{currentstroke}%
\pgfsetdash{}{0pt}%
\pgfpathmoveto{\pgfqpoint{1.766049in}{0.493860in}}%
\pgfpathlineto{\pgfqpoint{1.864986in}{0.493860in}}%
\pgfpathlineto{\pgfqpoint{1.864986in}{0.394924in}}%
\pgfpathlineto{\pgfqpoint{1.766049in}{0.394924in}}%
\pgfpathlineto{\pgfqpoint{1.766049in}{0.493860in}}%
\pgfusepath{stroke,fill}%
\end{pgfscope}%
\begin{pgfscope}%
\pgfpathrectangle{\pgfqpoint{0.380943in}{0.295988in}}{\pgfqpoint{4.650000in}{0.692553in}}%
\pgfusepath{clip}%
\pgfsetbuttcap%
\pgfsetroundjoin%
\definecolor{currentfill}{rgb}{0.968997,0.888351,0.702284}%
\pgfsetfillcolor{currentfill}%
\pgfsetlinewidth{0.250937pt}%
\definecolor{currentstroke}{rgb}{1.000000,1.000000,1.000000}%
\pgfsetstrokecolor{currentstroke}%
\pgfsetdash{}{0pt}%
\pgfpathmoveto{\pgfqpoint{1.864986in}{0.493860in}}%
\pgfpathlineto{\pgfqpoint{1.963922in}{0.493860in}}%
\pgfpathlineto{\pgfqpoint{1.963922in}{0.394924in}}%
\pgfpathlineto{\pgfqpoint{1.864986in}{0.394924in}}%
\pgfpathlineto{\pgfqpoint{1.864986in}{0.493860in}}%
\pgfusepath{stroke,fill}%
\end{pgfscope}%
\begin{pgfscope}%
\pgfpathrectangle{\pgfqpoint{0.380943in}{0.295988in}}{\pgfqpoint{4.650000in}{0.692553in}}%
\pgfusepath{clip}%
\pgfsetbuttcap%
\pgfsetroundjoin%
\definecolor{currentfill}{rgb}{0.961738,0.927612,0.725598}%
\pgfsetfillcolor{currentfill}%
\pgfsetlinewidth{0.250937pt}%
\definecolor{currentstroke}{rgb}{1.000000,1.000000,1.000000}%
\pgfsetstrokecolor{currentstroke}%
\pgfsetdash{}{0pt}%
\pgfpathmoveto{\pgfqpoint{1.963922in}{0.493860in}}%
\pgfpathlineto{\pgfqpoint{2.062858in}{0.493860in}}%
\pgfpathlineto{\pgfqpoint{2.062858in}{0.394924in}}%
\pgfpathlineto{\pgfqpoint{1.963922in}{0.394924in}}%
\pgfpathlineto{\pgfqpoint{1.963922in}{0.493860in}}%
\pgfusepath{stroke,fill}%
\end{pgfscope}%
\begin{pgfscope}%
\pgfpathrectangle{\pgfqpoint{0.380943in}{0.295988in}}{\pgfqpoint{4.650000in}{0.692553in}}%
\pgfusepath{clip}%
\pgfsetbuttcap%
\pgfsetroundjoin%
\definecolor{currentfill}{rgb}{0.971534,0.875663,0.695517}%
\pgfsetfillcolor{currentfill}%
\pgfsetlinewidth{0.250937pt}%
\definecolor{currentstroke}{rgb}{1.000000,1.000000,1.000000}%
\pgfsetstrokecolor{currentstroke}%
\pgfsetdash{}{0pt}%
\pgfpathmoveto{\pgfqpoint{2.062858in}{0.493860in}}%
\pgfpathlineto{\pgfqpoint{2.161794in}{0.493860in}}%
\pgfpathlineto{\pgfqpoint{2.161794in}{0.394924in}}%
\pgfpathlineto{\pgfqpoint{2.062858in}{0.394924in}}%
\pgfpathlineto{\pgfqpoint{2.062858in}{0.493860in}}%
\pgfusepath{stroke,fill}%
\end{pgfscope}%
\begin{pgfscope}%
\pgfpathrectangle{\pgfqpoint{0.380943in}{0.295988in}}{\pgfqpoint{4.650000in}{0.692553in}}%
\pgfusepath{clip}%
\pgfsetbuttcap%
\pgfsetroundjoin%
\definecolor{currentfill}{rgb}{0.962584,0.922537,0.722215}%
\pgfsetfillcolor{currentfill}%
\pgfsetlinewidth{0.250937pt}%
\definecolor{currentstroke}{rgb}{1.000000,1.000000,1.000000}%
\pgfsetstrokecolor{currentstroke}%
\pgfsetdash{}{0pt}%
\pgfpathmoveto{\pgfqpoint{2.161794in}{0.493860in}}%
\pgfpathlineto{\pgfqpoint{2.260730in}{0.493860in}}%
\pgfpathlineto{\pgfqpoint{2.260730in}{0.394924in}}%
\pgfpathlineto{\pgfqpoint{2.161794in}{0.394924in}}%
\pgfpathlineto{\pgfqpoint{2.161794in}{0.493860in}}%
\pgfusepath{stroke,fill}%
\end{pgfscope}%
\begin{pgfscope}%
\pgfpathrectangle{\pgfqpoint{0.380943in}{0.295988in}}{\pgfqpoint{4.650000in}{0.692553in}}%
\pgfusepath{clip}%
\pgfsetbuttcap%
\pgfsetroundjoin%
\definecolor{currentfill}{rgb}{0.963937,0.914418,0.716801}%
\pgfsetfillcolor{currentfill}%
\pgfsetlinewidth{0.250937pt}%
\definecolor{currentstroke}{rgb}{1.000000,1.000000,1.000000}%
\pgfsetstrokecolor{currentstroke}%
\pgfsetdash{}{0pt}%
\pgfpathmoveto{\pgfqpoint{2.260730in}{0.493860in}}%
\pgfpathlineto{\pgfqpoint{2.359666in}{0.493860in}}%
\pgfpathlineto{\pgfqpoint{2.359666in}{0.394924in}}%
\pgfpathlineto{\pgfqpoint{2.260730in}{0.394924in}}%
\pgfpathlineto{\pgfqpoint{2.260730in}{0.493860in}}%
\pgfusepath{stroke,fill}%
\end{pgfscope}%
\begin{pgfscope}%
\pgfpathrectangle{\pgfqpoint{0.380943in}{0.295988in}}{\pgfqpoint{4.650000in}{0.692553in}}%
\pgfusepath{clip}%
\pgfsetbuttcap%
\pgfsetroundjoin%
\definecolor{currentfill}{rgb}{0.970012,0.883276,0.699577}%
\pgfsetfillcolor{currentfill}%
\pgfsetlinewidth{0.250937pt}%
\definecolor{currentstroke}{rgb}{1.000000,1.000000,1.000000}%
\pgfsetstrokecolor{currentstroke}%
\pgfsetdash{}{0pt}%
\pgfpathmoveto{\pgfqpoint{2.359666in}{0.493860in}}%
\pgfpathlineto{\pgfqpoint{2.458603in}{0.493860in}}%
\pgfpathlineto{\pgfqpoint{2.458603in}{0.394924in}}%
\pgfpathlineto{\pgfqpoint{2.359666in}{0.394924in}}%
\pgfpathlineto{\pgfqpoint{2.359666in}{0.493860in}}%
\pgfusepath{stroke,fill}%
\end{pgfscope}%
\begin{pgfscope}%
\pgfpathrectangle{\pgfqpoint{0.380943in}{0.295988in}}{\pgfqpoint{4.650000in}{0.692553in}}%
\pgfusepath{clip}%
\pgfsetbuttcap%
\pgfsetroundjoin%
\definecolor{currentfill}{rgb}{0.970012,0.883276,0.699577}%
\pgfsetfillcolor{currentfill}%
\pgfsetlinewidth{0.250937pt}%
\definecolor{currentstroke}{rgb}{1.000000,1.000000,1.000000}%
\pgfsetstrokecolor{currentstroke}%
\pgfsetdash{}{0pt}%
\pgfpathmoveto{\pgfqpoint{2.458603in}{0.493860in}}%
\pgfpathlineto{\pgfqpoint{2.557539in}{0.493860in}}%
\pgfpathlineto{\pgfqpoint{2.557539in}{0.394924in}}%
\pgfpathlineto{\pgfqpoint{2.458603in}{0.394924in}}%
\pgfpathlineto{\pgfqpoint{2.458603in}{0.493860in}}%
\pgfusepath{stroke,fill}%
\end{pgfscope}%
\begin{pgfscope}%
\pgfpathrectangle{\pgfqpoint{0.380943in}{0.295988in}}{\pgfqpoint{4.650000in}{0.692553in}}%
\pgfusepath{clip}%
\pgfsetbuttcap%
\pgfsetroundjoin%
\definecolor{currentfill}{rgb}{0.968997,0.888351,0.702284}%
\pgfsetfillcolor{currentfill}%
\pgfsetlinewidth{0.250937pt}%
\definecolor{currentstroke}{rgb}{1.000000,1.000000,1.000000}%
\pgfsetstrokecolor{currentstroke}%
\pgfsetdash{}{0pt}%
\pgfpathmoveto{\pgfqpoint{2.557539in}{0.493860in}}%
\pgfpathlineto{\pgfqpoint{2.656475in}{0.493860in}}%
\pgfpathlineto{\pgfqpoint{2.656475in}{0.394924in}}%
\pgfpathlineto{\pgfqpoint{2.557539in}{0.394924in}}%
\pgfpathlineto{\pgfqpoint{2.557539in}{0.493860in}}%
\pgfusepath{stroke,fill}%
\end{pgfscope}%
\begin{pgfscope}%
\pgfpathrectangle{\pgfqpoint{0.380943in}{0.295988in}}{\pgfqpoint{4.650000in}{0.692553in}}%
\pgfusepath{clip}%
\pgfsetbuttcap%
\pgfsetroundjoin%
\definecolor{currentfill}{rgb}{0.962584,0.922537,0.722215}%
\pgfsetfillcolor{currentfill}%
\pgfsetlinewidth{0.250937pt}%
\definecolor{currentstroke}{rgb}{1.000000,1.000000,1.000000}%
\pgfsetstrokecolor{currentstroke}%
\pgfsetdash{}{0pt}%
\pgfpathmoveto{\pgfqpoint{2.656475in}{0.493860in}}%
\pgfpathlineto{\pgfqpoint{2.755411in}{0.493860in}}%
\pgfpathlineto{\pgfqpoint{2.755411in}{0.394924in}}%
\pgfpathlineto{\pgfqpoint{2.656475in}{0.394924in}}%
\pgfpathlineto{\pgfqpoint{2.656475in}{0.493860in}}%
\pgfusepath{stroke,fill}%
\end{pgfscope}%
\begin{pgfscope}%
\pgfpathrectangle{\pgfqpoint{0.380943in}{0.295988in}}{\pgfqpoint{4.650000in}{0.692553in}}%
\pgfusepath{clip}%
\pgfsetbuttcap%
\pgfsetroundjoin%
\definecolor{currentfill}{rgb}{0.962584,0.922537,0.722215}%
\pgfsetfillcolor{currentfill}%
\pgfsetlinewidth{0.250937pt}%
\definecolor{currentstroke}{rgb}{1.000000,1.000000,1.000000}%
\pgfsetstrokecolor{currentstroke}%
\pgfsetdash{}{0pt}%
\pgfpathmoveto{\pgfqpoint{2.755411in}{0.493860in}}%
\pgfpathlineto{\pgfqpoint{2.854347in}{0.493860in}}%
\pgfpathlineto{\pgfqpoint{2.854347in}{0.394924in}}%
\pgfpathlineto{\pgfqpoint{2.755411in}{0.394924in}}%
\pgfpathlineto{\pgfqpoint{2.755411in}{0.493860in}}%
\pgfusepath{stroke,fill}%
\end{pgfscope}%
\begin{pgfscope}%
\pgfpathrectangle{\pgfqpoint{0.380943in}{0.295988in}}{\pgfqpoint{4.650000in}{0.692553in}}%
\pgfusepath{clip}%
\pgfsetbuttcap%
\pgfsetroundjoin%
\definecolor{currentfill}{rgb}{0.962584,0.922537,0.722215}%
\pgfsetfillcolor{currentfill}%
\pgfsetlinewidth{0.250937pt}%
\definecolor{currentstroke}{rgb}{1.000000,1.000000,1.000000}%
\pgfsetstrokecolor{currentstroke}%
\pgfsetdash{}{0pt}%
\pgfpathmoveto{\pgfqpoint{2.854347in}{0.493860in}}%
\pgfpathlineto{\pgfqpoint{2.953283in}{0.493860in}}%
\pgfpathlineto{\pgfqpoint{2.953283in}{0.394924in}}%
\pgfpathlineto{\pgfqpoint{2.854347in}{0.394924in}}%
\pgfpathlineto{\pgfqpoint{2.854347in}{0.493860in}}%
\pgfusepath{stroke,fill}%
\end{pgfscope}%
\begin{pgfscope}%
\pgfpathrectangle{\pgfqpoint{0.380943in}{0.295988in}}{\pgfqpoint{4.650000in}{0.692553in}}%
\pgfusepath{clip}%
\pgfsetbuttcap%
\pgfsetroundjoin%
\definecolor{currentfill}{rgb}{0.963937,0.914418,0.716801}%
\pgfsetfillcolor{currentfill}%
\pgfsetlinewidth{0.250937pt}%
\definecolor{currentstroke}{rgb}{1.000000,1.000000,1.000000}%
\pgfsetstrokecolor{currentstroke}%
\pgfsetdash{}{0pt}%
\pgfpathmoveto{\pgfqpoint{2.953283in}{0.493860in}}%
\pgfpathlineto{\pgfqpoint{3.052220in}{0.493860in}}%
\pgfpathlineto{\pgfqpoint{3.052220in}{0.394924in}}%
\pgfpathlineto{\pgfqpoint{2.953283in}{0.394924in}}%
\pgfpathlineto{\pgfqpoint{2.953283in}{0.493860in}}%
\pgfusepath{stroke,fill}%
\end{pgfscope}%
\begin{pgfscope}%
\pgfpathrectangle{\pgfqpoint{0.380943in}{0.295988in}}{\pgfqpoint{4.650000in}{0.692553in}}%
\pgfusepath{clip}%
\pgfsetbuttcap%
\pgfsetroundjoin%
\definecolor{currentfill}{rgb}{0.971534,0.875663,0.695517}%
\pgfsetfillcolor{currentfill}%
\pgfsetlinewidth{0.250937pt}%
\definecolor{currentstroke}{rgb}{1.000000,1.000000,1.000000}%
\pgfsetstrokecolor{currentstroke}%
\pgfsetdash{}{0pt}%
\pgfpathmoveto{\pgfqpoint{3.052220in}{0.493860in}}%
\pgfpathlineto{\pgfqpoint{3.151156in}{0.493860in}}%
\pgfpathlineto{\pgfqpoint{3.151156in}{0.394924in}}%
\pgfpathlineto{\pgfqpoint{3.052220in}{0.394924in}}%
\pgfpathlineto{\pgfqpoint{3.052220in}{0.493860in}}%
\pgfusepath{stroke,fill}%
\end{pgfscope}%
\begin{pgfscope}%
\pgfpathrectangle{\pgfqpoint{0.380943in}{0.295988in}}{\pgfqpoint{4.650000in}{0.692553in}}%
\pgfusepath{clip}%
\pgfsetbuttcap%
\pgfsetroundjoin%
\definecolor{currentfill}{rgb}{0.962584,0.922537,0.722215}%
\pgfsetfillcolor{currentfill}%
\pgfsetlinewidth{0.250937pt}%
\definecolor{currentstroke}{rgb}{1.000000,1.000000,1.000000}%
\pgfsetstrokecolor{currentstroke}%
\pgfsetdash{}{0pt}%
\pgfpathmoveto{\pgfqpoint{3.151156in}{0.493860in}}%
\pgfpathlineto{\pgfqpoint{3.250092in}{0.493860in}}%
\pgfpathlineto{\pgfqpoint{3.250092in}{0.394924in}}%
\pgfpathlineto{\pgfqpoint{3.151156in}{0.394924in}}%
\pgfpathlineto{\pgfqpoint{3.151156in}{0.493860in}}%
\pgfusepath{stroke,fill}%
\end{pgfscope}%
\begin{pgfscope}%
\pgfpathrectangle{\pgfqpoint{0.380943in}{0.295988in}}{\pgfqpoint{4.650000in}{0.692553in}}%
\pgfusepath{clip}%
\pgfsetbuttcap%
\pgfsetroundjoin%
\definecolor{currentfill}{rgb}{0.993003,0.759477,0.611457}%
\pgfsetfillcolor{currentfill}%
\pgfsetlinewidth{0.250937pt}%
\definecolor{currentstroke}{rgb}{1.000000,1.000000,1.000000}%
\pgfsetstrokecolor{currentstroke}%
\pgfsetdash{}{0pt}%
\pgfpathmoveto{\pgfqpoint{3.250092in}{0.493860in}}%
\pgfpathlineto{\pgfqpoint{3.349028in}{0.493860in}}%
\pgfpathlineto{\pgfqpoint{3.349028in}{0.394924in}}%
\pgfpathlineto{\pgfqpoint{3.250092in}{0.394924in}}%
\pgfpathlineto{\pgfqpoint{3.250092in}{0.493860in}}%
\pgfusepath{stroke,fill}%
\end{pgfscope}%
\begin{pgfscope}%
\pgfpathrectangle{\pgfqpoint{0.380943in}{0.295988in}}{\pgfqpoint{4.650000in}{0.692553in}}%
\pgfusepath{clip}%
\pgfsetbuttcap%
\pgfsetroundjoin%
\definecolor{currentfill}{rgb}{0.977116,0.848181,0.679769}%
\pgfsetfillcolor{currentfill}%
\pgfsetlinewidth{0.250937pt}%
\definecolor{currentstroke}{rgb}{1.000000,1.000000,1.000000}%
\pgfsetstrokecolor{currentstroke}%
\pgfsetdash{}{0pt}%
\pgfpathmoveto{\pgfqpoint{3.349028in}{0.493860in}}%
\pgfpathlineto{\pgfqpoint{3.447964in}{0.493860in}}%
\pgfpathlineto{\pgfqpoint{3.447964in}{0.394924in}}%
\pgfpathlineto{\pgfqpoint{3.349028in}{0.394924in}}%
\pgfpathlineto{\pgfqpoint{3.349028in}{0.493860in}}%
\pgfusepath{stroke,fill}%
\end{pgfscope}%
\begin{pgfscope}%
\pgfpathrectangle{\pgfqpoint{0.380943in}{0.295988in}}{\pgfqpoint{4.650000in}{0.692553in}}%
\pgfusepath{clip}%
\pgfsetbuttcap%
\pgfsetroundjoin%
\definecolor{currentfill}{rgb}{0.980669,0.832787,0.665559}%
\pgfsetfillcolor{currentfill}%
\pgfsetlinewidth{0.250937pt}%
\definecolor{currentstroke}{rgb}{1.000000,1.000000,1.000000}%
\pgfsetstrokecolor{currentstroke}%
\pgfsetdash{}{0pt}%
\pgfpathmoveto{\pgfqpoint{3.447964in}{0.493860in}}%
\pgfpathlineto{\pgfqpoint{3.546901in}{0.493860in}}%
\pgfpathlineto{\pgfqpoint{3.546901in}{0.394924in}}%
\pgfpathlineto{\pgfqpoint{3.447964in}{0.394924in}}%
\pgfpathlineto{\pgfqpoint{3.447964in}{0.493860in}}%
\pgfusepath{stroke,fill}%
\end{pgfscope}%
\begin{pgfscope}%
\pgfpathrectangle{\pgfqpoint{0.380943in}{0.295988in}}{\pgfqpoint{4.650000in}{0.692553in}}%
\pgfusepath{clip}%
\pgfsetbuttcap%
\pgfsetroundjoin%
\definecolor{currentfill}{rgb}{0.978131,0.843783,0.675709}%
\pgfsetfillcolor{currentfill}%
\pgfsetlinewidth{0.250937pt}%
\definecolor{currentstroke}{rgb}{1.000000,1.000000,1.000000}%
\pgfsetstrokecolor{currentstroke}%
\pgfsetdash{}{0pt}%
\pgfpathmoveto{\pgfqpoint{3.546901in}{0.493860in}}%
\pgfpathlineto{\pgfqpoint{3.645837in}{0.493860in}}%
\pgfpathlineto{\pgfqpoint{3.645837in}{0.394924in}}%
\pgfpathlineto{\pgfqpoint{3.546901in}{0.394924in}}%
\pgfpathlineto{\pgfqpoint{3.546901in}{0.493860in}}%
\pgfusepath{stroke,fill}%
\end{pgfscope}%
\begin{pgfscope}%
\pgfpathrectangle{\pgfqpoint{0.380943in}{0.295988in}}{\pgfqpoint{4.650000in}{0.692553in}}%
\pgfusepath{clip}%
\pgfsetbuttcap%
\pgfsetroundjoin%
\definecolor{currentfill}{rgb}{0.967474,0.895963,0.706344}%
\pgfsetfillcolor{currentfill}%
\pgfsetlinewidth{0.250937pt}%
\definecolor{currentstroke}{rgb}{1.000000,1.000000,1.000000}%
\pgfsetstrokecolor{currentstroke}%
\pgfsetdash{}{0pt}%
\pgfpathmoveto{\pgfqpoint{3.645837in}{0.493860in}}%
\pgfpathlineto{\pgfqpoint{3.744773in}{0.493860in}}%
\pgfpathlineto{\pgfqpoint{3.744773in}{0.394924in}}%
\pgfpathlineto{\pgfqpoint{3.645837in}{0.394924in}}%
\pgfpathlineto{\pgfqpoint{3.645837in}{0.493860in}}%
\pgfusepath{stroke,fill}%
\end{pgfscope}%
\begin{pgfscope}%
\pgfpathrectangle{\pgfqpoint{0.380943in}{0.295988in}}{\pgfqpoint{4.650000in}{0.692553in}}%
\pgfusepath{clip}%
\pgfsetbuttcap%
\pgfsetroundjoin%
\definecolor{currentfill}{rgb}{0.986251,0.808597,0.643230}%
\pgfsetfillcolor{currentfill}%
\pgfsetlinewidth{0.250937pt}%
\definecolor{currentstroke}{rgb}{1.000000,1.000000,1.000000}%
\pgfsetstrokecolor{currentstroke}%
\pgfsetdash{}{0pt}%
\pgfpathmoveto{\pgfqpoint{3.744773in}{0.493860in}}%
\pgfpathlineto{\pgfqpoint{3.843709in}{0.493860in}}%
\pgfpathlineto{\pgfqpoint{3.843709in}{0.394924in}}%
\pgfpathlineto{\pgfqpoint{3.744773in}{0.394924in}}%
\pgfpathlineto{\pgfqpoint{3.744773in}{0.493860in}}%
\pgfusepath{stroke,fill}%
\end{pgfscope}%
\begin{pgfscope}%
\pgfpathrectangle{\pgfqpoint{0.380943in}{0.295988in}}{\pgfqpoint{4.650000in}{0.692553in}}%
\pgfusepath{clip}%
\pgfsetbuttcap%
\pgfsetroundjoin%
\definecolor{currentfill}{rgb}{0.999446,0.645767,0.548927}%
\pgfsetfillcolor{currentfill}%
\pgfsetlinewidth{0.250937pt}%
\definecolor{currentstroke}{rgb}{1.000000,1.000000,1.000000}%
\pgfsetstrokecolor{currentstroke}%
\pgfsetdash{}{0pt}%
\pgfpathmoveto{\pgfqpoint{3.843709in}{0.493860in}}%
\pgfpathlineto{\pgfqpoint{3.942645in}{0.493860in}}%
\pgfpathlineto{\pgfqpoint{3.942645in}{0.394924in}}%
\pgfpathlineto{\pgfqpoint{3.843709in}{0.394924in}}%
\pgfpathlineto{\pgfqpoint{3.843709in}{0.493860in}}%
\pgfusepath{stroke,fill}%
\end{pgfscope}%
\begin{pgfscope}%
\pgfpathrectangle{\pgfqpoint{0.380943in}{0.295988in}}{\pgfqpoint{4.650000in}{0.692553in}}%
\pgfusepath{clip}%
\pgfsetbuttcap%
\pgfsetroundjoin%
\definecolor{currentfill}{rgb}{0.980669,0.832787,0.665559}%
\pgfsetfillcolor{currentfill}%
\pgfsetlinewidth{0.250937pt}%
\definecolor{currentstroke}{rgb}{1.000000,1.000000,1.000000}%
\pgfsetstrokecolor{currentstroke}%
\pgfsetdash{}{0pt}%
\pgfpathmoveto{\pgfqpoint{3.942645in}{0.493860in}}%
\pgfpathlineto{\pgfqpoint{4.041581in}{0.493860in}}%
\pgfpathlineto{\pgfqpoint{4.041581in}{0.394924in}}%
\pgfpathlineto{\pgfqpoint{3.942645in}{0.394924in}}%
\pgfpathlineto{\pgfqpoint{3.942645in}{0.493860in}}%
\pgfusepath{stroke,fill}%
\end{pgfscope}%
\begin{pgfscope}%
\pgfpathrectangle{\pgfqpoint{0.380943in}{0.295988in}}{\pgfqpoint{4.650000in}{0.692553in}}%
\pgfusepath{clip}%
\pgfsetbuttcap%
\pgfsetroundjoin%
\definecolor{currentfill}{rgb}{0.997586,0.694148,0.574979}%
\pgfsetfillcolor{currentfill}%
\pgfsetlinewidth{0.250937pt}%
\definecolor{currentstroke}{rgb}{1.000000,1.000000,1.000000}%
\pgfsetstrokecolor{currentstroke}%
\pgfsetdash{}{0pt}%
\pgfpathmoveto{\pgfqpoint{4.041581in}{0.493860in}}%
\pgfpathlineto{\pgfqpoint{4.140518in}{0.493860in}}%
\pgfpathlineto{\pgfqpoint{4.140518in}{0.394924in}}%
\pgfpathlineto{\pgfqpoint{4.041581in}{0.394924in}}%
\pgfpathlineto{\pgfqpoint{4.041581in}{0.493860in}}%
\pgfusepath{stroke,fill}%
\end{pgfscope}%
\begin{pgfscope}%
\pgfpathrectangle{\pgfqpoint{0.380943in}{0.295988in}}{\pgfqpoint{4.650000in}{0.692553in}}%
\pgfusepath{clip}%
\pgfsetbuttcap%
\pgfsetroundjoin%
\definecolor{currentfill}{rgb}{0.991311,0.773856,0.619915}%
\pgfsetfillcolor{currentfill}%
\pgfsetlinewidth{0.250937pt}%
\definecolor{currentstroke}{rgb}{1.000000,1.000000,1.000000}%
\pgfsetstrokecolor{currentstroke}%
\pgfsetdash{}{0pt}%
\pgfpathmoveto{\pgfqpoint{4.140518in}{0.493860in}}%
\pgfpathlineto{\pgfqpoint{4.239454in}{0.493860in}}%
\pgfpathlineto{\pgfqpoint{4.239454in}{0.394924in}}%
\pgfpathlineto{\pgfqpoint{4.140518in}{0.394924in}}%
\pgfpathlineto{\pgfqpoint{4.140518in}{0.493860in}}%
\pgfusepath{stroke,fill}%
\end{pgfscope}%
\begin{pgfscope}%
\pgfpathrectangle{\pgfqpoint{0.380943in}{0.295988in}}{\pgfqpoint{4.650000in}{0.692553in}}%
\pgfusepath{clip}%
\pgfsetbuttcap%
\pgfsetroundjoin%
\definecolor{currentfill}{rgb}{0.963429,0.917463,0.718831}%
\pgfsetfillcolor{currentfill}%
\pgfsetlinewidth{0.250937pt}%
\definecolor{currentstroke}{rgb}{1.000000,1.000000,1.000000}%
\pgfsetstrokecolor{currentstroke}%
\pgfsetdash{}{0pt}%
\pgfpathmoveto{\pgfqpoint{4.239454in}{0.493860in}}%
\pgfpathlineto{\pgfqpoint{4.338390in}{0.493860in}}%
\pgfpathlineto{\pgfqpoint{4.338390in}{0.394924in}}%
\pgfpathlineto{\pgfqpoint{4.239454in}{0.394924in}}%
\pgfpathlineto{\pgfqpoint{4.239454in}{0.493860in}}%
\pgfusepath{stroke,fill}%
\end{pgfscope}%
\begin{pgfscope}%
\pgfpathrectangle{\pgfqpoint{0.380943in}{0.295988in}}{\pgfqpoint{4.650000in}{0.692553in}}%
\pgfusepath{clip}%
\pgfsetbuttcap%
\pgfsetroundjoin%
\definecolor{currentfill}{rgb}{0.960892,0.932687,0.728981}%
\pgfsetfillcolor{currentfill}%
\pgfsetlinewidth{0.250937pt}%
\definecolor{currentstroke}{rgb}{1.000000,1.000000,1.000000}%
\pgfsetstrokecolor{currentstroke}%
\pgfsetdash{}{0pt}%
\pgfpathmoveto{\pgfqpoint{4.338390in}{0.493860in}}%
\pgfpathlineto{\pgfqpoint{4.437326in}{0.493860in}}%
\pgfpathlineto{\pgfqpoint{4.437326in}{0.394924in}}%
\pgfpathlineto{\pgfqpoint{4.338390in}{0.394924in}}%
\pgfpathlineto{\pgfqpoint{4.338390in}{0.493860in}}%
\pgfusepath{stroke,fill}%
\end{pgfscope}%
\begin{pgfscope}%
\pgfpathrectangle{\pgfqpoint{0.380943in}{0.295988in}}{\pgfqpoint{4.650000in}{0.692553in}}%
\pgfusepath{clip}%
\pgfsetbuttcap%
\pgfsetroundjoin%
\definecolor{currentfill}{rgb}{0.964783,0.940131,0.739808}%
\pgfsetfillcolor{currentfill}%
\pgfsetlinewidth{0.250937pt}%
\definecolor{currentstroke}{rgb}{1.000000,1.000000,1.000000}%
\pgfsetstrokecolor{currentstroke}%
\pgfsetdash{}{0pt}%
\pgfpathmoveto{\pgfqpoint{4.437326in}{0.493860in}}%
\pgfpathlineto{\pgfqpoint{4.536262in}{0.493860in}}%
\pgfpathlineto{\pgfqpoint{4.536262in}{0.394924in}}%
\pgfpathlineto{\pgfqpoint{4.437326in}{0.394924in}}%
\pgfpathlineto{\pgfqpoint{4.437326in}{0.493860in}}%
\pgfusepath{stroke,fill}%
\end{pgfscope}%
\begin{pgfscope}%
\pgfpathrectangle{\pgfqpoint{0.380943in}{0.295988in}}{\pgfqpoint{4.650000in}{0.692553in}}%
\pgfusepath{clip}%
\pgfsetbuttcap%
\pgfsetroundjoin%
\definecolor{currentfill}{rgb}{0.962584,0.922537,0.722215}%
\pgfsetfillcolor{currentfill}%
\pgfsetlinewidth{0.250937pt}%
\definecolor{currentstroke}{rgb}{1.000000,1.000000,1.000000}%
\pgfsetstrokecolor{currentstroke}%
\pgfsetdash{}{0pt}%
\pgfpathmoveto{\pgfqpoint{4.536262in}{0.493860in}}%
\pgfpathlineto{\pgfqpoint{4.635198in}{0.493860in}}%
\pgfpathlineto{\pgfqpoint{4.635198in}{0.394924in}}%
\pgfpathlineto{\pgfqpoint{4.536262in}{0.394924in}}%
\pgfpathlineto{\pgfqpoint{4.536262in}{0.493860in}}%
\pgfusepath{stroke,fill}%
\end{pgfscope}%
\begin{pgfscope}%
\pgfpathrectangle{\pgfqpoint{0.380943in}{0.295988in}}{\pgfqpoint{4.650000in}{0.692553in}}%
\pgfusepath{clip}%
\pgfsetbuttcap%
\pgfsetroundjoin%
\definecolor{currentfill}{rgb}{0.964783,0.940131,0.739808}%
\pgfsetfillcolor{currentfill}%
\pgfsetlinewidth{0.250937pt}%
\definecolor{currentstroke}{rgb}{1.000000,1.000000,1.000000}%
\pgfsetstrokecolor{currentstroke}%
\pgfsetdash{}{0pt}%
\pgfpathmoveto{\pgfqpoint{4.635198in}{0.493860in}}%
\pgfpathlineto{\pgfqpoint{4.734135in}{0.493860in}}%
\pgfpathlineto{\pgfqpoint{4.734135in}{0.394924in}}%
\pgfpathlineto{\pgfqpoint{4.635198in}{0.394924in}}%
\pgfpathlineto{\pgfqpoint{4.635198in}{0.493860in}}%
\pgfusepath{stroke,fill}%
\end{pgfscope}%
\begin{pgfscope}%
\pgfpathrectangle{\pgfqpoint{0.380943in}{0.295988in}}{\pgfqpoint{4.650000in}{0.692553in}}%
\pgfusepath{clip}%
\pgfsetbuttcap%
\pgfsetroundjoin%
\definecolor{currentfill}{rgb}{0.960892,0.932687,0.728981}%
\pgfsetfillcolor{currentfill}%
\pgfsetlinewidth{0.250937pt}%
\definecolor{currentstroke}{rgb}{1.000000,1.000000,1.000000}%
\pgfsetstrokecolor{currentstroke}%
\pgfsetdash{}{0pt}%
\pgfpathmoveto{\pgfqpoint{4.734135in}{0.493860in}}%
\pgfpathlineto{\pgfqpoint{4.833071in}{0.493860in}}%
\pgfpathlineto{\pgfqpoint{4.833071in}{0.394924in}}%
\pgfpathlineto{\pgfqpoint{4.734135in}{0.394924in}}%
\pgfpathlineto{\pgfqpoint{4.734135in}{0.493860in}}%
\pgfusepath{stroke,fill}%
\end{pgfscope}%
\begin{pgfscope}%
\pgfpathrectangle{\pgfqpoint{0.380943in}{0.295988in}}{\pgfqpoint{4.650000in}{0.692553in}}%
\pgfusepath{clip}%
\pgfsetbuttcap%
\pgfsetroundjoin%
\definecolor{currentfill}{rgb}{0.960892,0.932687,0.728981}%
\pgfsetfillcolor{currentfill}%
\pgfsetlinewidth{0.250937pt}%
\definecolor{currentstroke}{rgb}{1.000000,1.000000,1.000000}%
\pgfsetstrokecolor{currentstroke}%
\pgfsetdash{}{0pt}%
\pgfpathmoveto{\pgfqpoint{4.833071in}{0.493860in}}%
\pgfpathlineto{\pgfqpoint{4.932007in}{0.493860in}}%
\pgfpathlineto{\pgfqpoint{4.932007in}{0.394924in}}%
\pgfpathlineto{\pgfqpoint{4.833071in}{0.394924in}}%
\pgfpathlineto{\pgfqpoint{4.833071in}{0.493860in}}%
\pgfusepath{stroke,fill}%
\end{pgfscope}%
\begin{pgfscope}%
\pgfpathrectangle{\pgfqpoint{0.380943in}{0.295988in}}{\pgfqpoint{4.650000in}{0.692553in}}%
\pgfusepath{clip}%
\pgfsetbuttcap%
\pgfsetroundjoin%
\pgfsetlinewidth{0.250937pt}%
\definecolor{currentstroke}{rgb}{1.000000,1.000000,1.000000}%
\pgfsetstrokecolor{currentstroke}%
\pgfsetdash{}{0pt}%
\pgfpathmoveto{\pgfqpoint{4.932007in}{0.493860in}}%
\pgfpathlineto{\pgfqpoint{5.030943in}{0.493860in}}%
\pgfpathlineto{\pgfqpoint{5.030943in}{0.394924in}}%
\pgfpathlineto{\pgfqpoint{4.932007in}{0.394924in}}%
\pgfpathlineto{\pgfqpoint{4.932007in}{0.493860in}}%
\pgfusepath{stroke}%
\end{pgfscope}%
\begin{pgfscope}%
\pgfpathrectangle{\pgfqpoint{0.380943in}{0.295988in}}{\pgfqpoint{4.650000in}{0.692553in}}%
\pgfusepath{clip}%
\pgfsetbuttcap%
\pgfsetroundjoin%
\definecolor{currentfill}{rgb}{1.000000,1.000000,0.929412}%
\pgfsetfillcolor{currentfill}%
\pgfsetlinewidth{0.250937pt}%
\definecolor{currentstroke}{rgb}{1.000000,1.000000,1.000000}%
\pgfsetstrokecolor{currentstroke}%
\pgfsetdash{}{0pt}%
\pgfpathmoveto{\pgfqpoint{0.380943in}{0.394924in}}%
\pgfpathlineto{\pgfqpoint{0.479879in}{0.394924in}}%
\pgfpathlineto{\pgfqpoint{0.479879in}{0.295988in}}%
\pgfpathlineto{\pgfqpoint{0.380943in}{0.295988in}}%
\pgfpathlineto{\pgfqpoint{0.380943in}{0.394924in}}%
\pgfusepath{stroke,fill}%
\end{pgfscope}%
\begin{pgfscope}%
\pgfpathrectangle{\pgfqpoint{0.380943in}{0.295988in}}{\pgfqpoint{4.650000in}{0.692553in}}%
\pgfusepath{clip}%
\pgfsetbuttcap%
\pgfsetroundjoin%
\definecolor{currentfill}{rgb}{1.000000,1.000000,0.874433}%
\pgfsetfillcolor{currentfill}%
\pgfsetlinewidth{0.250937pt}%
\definecolor{currentstroke}{rgb}{1.000000,1.000000,1.000000}%
\pgfsetstrokecolor{currentstroke}%
\pgfsetdash{}{0pt}%
\pgfpathmoveto{\pgfqpoint{0.479879in}{0.394924in}}%
\pgfpathlineto{\pgfqpoint{0.578815in}{0.394924in}}%
\pgfpathlineto{\pgfqpoint{0.578815in}{0.295988in}}%
\pgfpathlineto{\pgfqpoint{0.479879in}{0.295988in}}%
\pgfpathlineto{\pgfqpoint{0.479879in}{0.394924in}}%
\pgfusepath{stroke,fill}%
\end{pgfscope}%
\begin{pgfscope}%
\pgfpathrectangle{\pgfqpoint{0.380943in}{0.295988in}}{\pgfqpoint{4.650000in}{0.692553in}}%
\pgfusepath{clip}%
\pgfsetbuttcap%
\pgfsetroundjoin%
\definecolor{currentfill}{rgb}{1.000000,1.000000,0.865975}%
\pgfsetfillcolor{currentfill}%
\pgfsetlinewidth{0.250937pt}%
\definecolor{currentstroke}{rgb}{1.000000,1.000000,1.000000}%
\pgfsetstrokecolor{currentstroke}%
\pgfsetdash{}{0pt}%
\pgfpathmoveto{\pgfqpoint{0.578815in}{0.394924in}}%
\pgfpathlineto{\pgfqpoint{0.677752in}{0.394924in}}%
\pgfpathlineto{\pgfqpoint{0.677752in}{0.295988in}}%
\pgfpathlineto{\pgfqpoint{0.578815in}{0.295988in}}%
\pgfpathlineto{\pgfqpoint{0.578815in}{0.394924in}}%
\pgfusepath{stroke,fill}%
\end{pgfscope}%
\begin{pgfscope}%
\pgfpathrectangle{\pgfqpoint{0.380943in}{0.295988in}}{\pgfqpoint{4.650000in}{0.692553in}}%
\pgfusepath{clip}%
\pgfsetbuttcap%
\pgfsetroundjoin%
\definecolor{currentfill}{rgb}{1.000000,1.000000,0.920953}%
\pgfsetfillcolor{currentfill}%
\pgfsetlinewidth{0.250937pt}%
\definecolor{currentstroke}{rgb}{1.000000,1.000000,1.000000}%
\pgfsetstrokecolor{currentstroke}%
\pgfsetdash{}{0pt}%
\pgfpathmoveto{\pgfqpoint{0.677752in}{0.394924in}}%
\pgfpathlineto{\pgfqpoint{0.776688in}{0.394924in}}%
\pgfpathlineto{\pgfqpoint{0.776688in}{0.295988in}}%
\pgfpathlineto{\pgfqpoint{0.677752in}{0.295988in}}%
\pgfpathlineto{\pgfqpoint{0.677752in}{0.394924in}}%
\pgfusepath{stroke,fill}%
\end{pgfscope}%
\begin{pgfscope}%
\pgfpathrectangle{\pgfqpoint{0.380943in}{0.295988in}}{\pgfqpoint{4.650000in}{0.692553in}}%
\pgfusepath{clip}%
\pgfsetbuttcap%
\pgfsetroundjoin%
\definecolor{currentfill}{rgb}{1.000000,1.000000,0.865975}%
\pgfsetfillcolor{currentfill}%
\pgfsetlinewidth{0.250937pt}%
\definecolor{currentstroke}{rgb}{1.000000,1.000000,1.000000}%
\pgfsetstrokecolor{currentstroke}%
\pgfsetdash{}{0pt}%
\pgfpathmoveto{\pgfqpoint{0.776688in}{0.394924in}}%
\pgfpathlineto{\pgfqpoint{0.875624in}{0.394924in}}%
\pgfpathlineto{\pgfqpoint{0.875624in}{0.295988in}}%
\pgfpathlineto{\pgfqpoint{0.776688in}{0.295988in}}%
\pgfpathlineto{\pgfqpoint{0.776688in}{0.394924in}}%
\pgfusepath{stroke,fill}%
\end{pgfscope}%
\begin{pgfscope}%
\pgfpathrectangle{\pgfqpoint{0.380943in}{0.295988in}}{\pgfqpoint{4.650000in}{0.692553in}}%
\pgfusepath{clip}%
\pgfsetbuttcap%
\pgfsetroundjoin%
\definecolor{currentfill}{rgb}{1.000000,1.000000,0.920953}%
\pgfsetfillcolor{currentfill}%
\pgfsetlinewidth{0.250937pt}%
\definecolor{currentstroke}{rgb}{1.000000,1.000000,1.000000}%
\pgfsetstrokecolor{currentstroke}%
\pgfsetdash{}{0pt}%
\pgfpathmoveto{\pgfqpoint{0.875624in}{0.394924in}}%
\pgfpathlineto{\pgfqpoint{0.974560in}{0.394924in}}%
\pgfpathlineto{\pgfqpoint{0.974560in}{0.295988in}}%
\pgfpathlineto{\pgfqpoint{0.875624in}{0.295988in}}%
\pgfpathlineto{\pgfqpoint{0.875624in}{0.394924in}}%
\pgfusepath{stroke,fill}%
\end{pgfscope}%
\begin{pgfscope}%
\pgfpathrectangle{\pgfqpoint{0.380943in}{0.295988in}}{\pgfqpoint{4.650000in}{0.692553in}}%
\pgfusepath{clip}%
\pgfsetbuttcap%
\pgfsetroundjoin%
\definecolor{currentfill}{rgb}{1.000000,1.000000,0.908266}%
\pgfsetfillcolor{currentfill}%
\pgfsetlinewidth{0.250937pt}%
\definecolor{currentstroke}{rgb}{1.000000,1.000000,1.000000}%
\pgfsetstrokecolor{currentstroke}%
\pgfsetdash{}{0pt}%
\pgfpathmoveto{\pgfqpoint{0.974560in}{0.394924in}}%
\pgfpathlineto{\pgfqpoint{1.073496in}{0.394924in}}%
\pgfpathlineto{\pgfqpoint{1.073496in}{0.295988in}}%
\pgfpathlineto{\pgfqpoint{0.974560in}{0.295988in}}%
\pgfpathlineto{\pgfqpoint{0.974560in}{0.394924in}}%
\pgfusepath{stroke,fill}%
\end{pgfscope}%
\begin{pgfscope}%
\pgfpathrectangle{\pgfqpoint{0.380943in}{0.295988in}}{\pgfqpoint{4.650000in}{0.692553in}}%
\pgfusepath{clip}%
\pgfsetbuttcap%
\pgfsetroundjoin%
\definecolor{currentfill}{rgb}{1.000000,1.000000,0.887120}%
\pgfsetfillcolor{currentfill}%
\pgfsetlinewidth{0.250937pt}%
\definecolor{currentstroke}{rgb}{1.000000,1.000000,1.000000}%
\pgfsetstrokecolor{currentstroke}%
\pgfsetdash{}{0pt}%
\pgfpathmoveto{\pgfqpoint{1.073496in}{0.394924in}}%
\pgfpathlineto{\pgfqpoint{1.172432in}{0.394924in}}%
\pgfpathlineto{\pgfqpoint{1.172432in}{0.295988in}}%
\pgfpathlineto{\pgfqpoint{1.073496in}{0.295988in}}%
\pgfpathlineto{\pgfqpoint{1.073496in}{0.394924in}}%
\pgfusepath{stroke,fill}%
\end{pgfscope}%
\begin{pgfscope}%
\pgfpathrectangle{\pgfqpoint{0.380943in}{0.295988in}}{\pgfqpoint{4.650000in}{0.692553in}}%
\pgfusepath{clip}%
\pgfsetbuttcap%
\pgfsetroundjoin%
\definecolor{currentfill}{rgb}{1.000000,1.000000,0.908266}%
\pgfsetfillcolor{currentfill}%
\pgfsetlinewidth{0.250937pt}%
\definecolor{currentstroke}{rgb}{1.000000,1.000000,1.000000}%
\pgfsetstrokecolor{currentstroke}%
\pgfsetdash{}{0pt}%
\pgfpathmoveto{\pgfqpoint{1.172432in}{0.394924in}}%
\pgfpathlineto{\pgfqpoint{1.271369in}{0.394924in}}%
\pgfpathlineto{\pgfqpoint{1.271369in}{0.295988in}}%
\pgfpathlineto{\pgfqpoint{1.172432in}{0.295988in}}%
\pgfpathlineto{\pgfqpoint{1.172432in}{0.394924in}}%
\pgfusepath{stroke,fill}%
\end{pgfscope}%
\begin{pgfscope}%
\pgfpathrectangle{\pgfqpoint{0.380943in}{0.295988in}}{\pgfqpoint{4.650000in}{0.692553in}}%
\pgfusepath{clip}%
\pgfsetbuttcap%
\pgfsetroundjoin%
\definecolor{currentfill}{rgb}{0.966459,0.901038,0.709050}%
\pgfsetfillcolor{currentfill}%
\pgfsetlinewidth{0.250937pt}%
\definecolor{currentstroke}{rgb}{1.000000,1.000000,1.000000}%
\pgfsetstrokecolor{currentstroke}%
\pgfsetdash{}{0pt}%
\pgfpathmoveto{\pgfqpoint{1.271369in}{0.394924in}}%
\pgfpathlineto{\pgfqpoint{1.370305in}{0.394924in}}%
\pgfpathlineto{\pgfqpoint{1.370305in}{0.295988in}}%
\pgfpathlineto{\pgfqpoint{1.271369in}{0.295988in}}%
\pgfpathlineto{\pgfqpoint{1.271369in}{0.394924in}}%
\pgfusepath{stroke,fill}%
\end{pgfscope}%
\begin{pgfscope}%
\pgfpathrectangle{\pgfqpoint{0.380943in}{0.295988in}}{\pgfqpoint{4.650000in}{0.692553in}}%
\pgfusepath{clip}%
\pgfsetbuttcap%
\pgfsetroundjoin%
\definecolor{currentfill}{rgb}{0.963429,0.917463,0.718831}%
\pgfsetfillcolor{currentfill}%
\pgfsetlinewidth{0.250937pt}%
\definecolor{currentstroke}{rgb}{1.000000,1.000000,1.000000}%
\pgfsetstrokecolor{currentstroke}%
\pgfsetdash{}{0pt}%
\pgfpathmoveto{\pgfqpoint{1.370305in}{0.394924in}}%
\pgfpathlineto{\pgfqpoint{1.469241in}{0.394924in}}%
\pgfpathlineto{\pgfqpoint{1.469241in}{0.295988in}}%
\pgfpathlineto{\pgfqpoint{1.370305in}{0.295988in}}%
\pgfpathlineto{\pgfqpoint{1.370305in}{0.394924in}}%
\pgfusepath{stroke,fill}%
\end{pgfscope}%
\begin{pgfscope}%
\pgfpathrectangle{\pgfqpoint{0.380943in}{0.295988in}}{\pgfqpoint{4.650000in}{0.692553in}}%
\pgfusepath{clip}%
\pgfsetbuttcap%
\pgfsetroundjoin%
\definecolor{currentfill}{rgb}{0.964937,0.908651,0.713110}%
\pgfsetfillcolor{currentfill}%
\pgfsetlinewidth{0.250937pt}%
\definecolor{currentstroke}{rgb}{1.000000,1.000000,1.000000}%
\pgfsetstrokecolor{currentstroke}%
\pgfsetdash{}{0pt}%
\pgfpathmoveto{\pgfqpoint{1.469241in}{0.394924in}}%
\pgfpathlineto{\pgfqpoint{1.568177in}{0.394924in}}%
\pgfpathlineto{\pgfqpoint{1.568177in}{0.295988in}}%
\pgfpathlineto{\pgfqpoint{1.469241in}{0.295988in}}%
\pgfpathlineto{\pgfqpoint{1.469241in}{0.394924in}}%
\pgfusepath{stroke,fill}%
\end{pgfscope}%
\begin{pgfscope}%
\pgfpathrectangle{\pgfqpoint{0.380943in}{0.295988in}}{\pgfqpoint{4.650000in}{0.692553in}}%
\pgfusepath{clip}%
\pgfsetbuttcap%
\pgfsetroundjoin%
\definecolor{currentfill}{rgb}{0.961738,0.927612,0.725598}%
\pgfsetfillcolor{currentfill}%
\pgfsetlinewidth{0.250937pt}%
\definecolor{currentstroke}{rgb}{1.000000,1.000000,1.000000}%
\pgfsetstrokecolor{currentstroke}%
\pgfsetdash{}{0pt}%
\pgfpathmoveto{\pgfqpoint{1.568177in}{0.394924in}}%
\pgfpathlineto{\pgfqpoint{1.667113in}{0.394924in}}%
\pgfpathlineto{\pgfqpoint{1.667113in}{0.295988in}}%
\pgfpathlineto{\pgfqpoint{1.568177in}{0.295988in}}%
\pgfpathlineto{\pgfqpoint{1.568177in}{0.394924in}}%
\pgfusepath{stroke,fill}%
\end{pgfscope}%
\begin{pgfscope}%
\pgfpathrectangle{\pgfqpoint{0.380943in}{0.295988in}}{\pgfqpoint{4.650000in}{0.692553in}}%
\pgfusepath{clip}%
\pgfsetbuttcap%
\pgfsetroundjoin%
\definecolor{currentfill}{rgb}{0.961230,0.930657,0.727628}%
\pgfsetfillcolor{currentfill}%
\pgfsetlinewidth{0.250937pt}%
\definecolor{currentstroke}{rgb}{1.000000,1.000000,1.000000}%
\pgfsetstrokecolor{currentstroke}%
\pgfsetdash{}{0pt}%
\pgfpathmoveto{\pgfqpoint{1.667113in}{0.394924in}}%
\pgfpathlineto{\pgfqpoint{1.766049in}{0.394924in}}%
\pgfpathlineto{\pgfqpoint{1.766049in}{0.295988in}}%
\pgfpathlineto{\pgfqpoint{1.667113in}{0.295988in}}%
\pgfpathlineto{\pgfqpoint{1.667113in}{0.394924in}}%
\pgfusepath{stroke,fill}%
\end{pgfscope}%
\begin{pgfscope}%
\pgfpathrectangle{\pgfqpoint{0.380943in}{0.295988in}}{\pgfqpoint{4.650000in}{0.692553in}}%
\pgfusepath{clip}%
\pgfsetbuttcap%
\pgfsetroundjoin%
\definecolor{currentfill}{rgb}{0.967474,0.895963,0.706344}%
\pgfsetfillcolor{currentfill}%
\pgfsetlinewidth{0.250937pt}%
\definecolor{currentstroke}{rgb}{1.000000,1.000000,1.000000}%
\pgfsetstrokecolor{currentstroke}%
\pgfsetdash{}{0pt}%
\pgfpathmoveto{\pgfqpoint{1.766049in}{0.394924in}}%
\pgfpathlineto{\pgfqpoint{1.864986in}{0.394924in}}%
\pgfpathlineto{\pgfqpoint{1.864986in}{0.295988in}}%
\pgfpathlineto{\pgfqpoint{1.766049in}{0.295988in}}%
\pgfpathlineto{\pgfqpoint{1.766049in}{0.394924in}}%
\pgfusepath{stroke,fill}%
\end{pgfscope}%
\begin{pgfscope}%
\pgfpathrectangle{\pgfqpoint{0.380943in}{0.295988in}}{\pgfqpoint{4.650000in}{0.692553in}}%
\pgfusepath{clip}%
\pgfsetbuttcap%
\pgfsetroundjoin%
\definecolor{currentfill}{rgb}{0.964783,0.940131,0.739808}%
\pgfsetfillcolor{currentfill}%
\pgfsetlinewidth{0.250937pt}%
\definecolor{currentstroke}{rgb}{1.000000,1.000000,1.000000}%
\pgfsetstrokecolor{currentstroke}%
\pgfsetdash{}{0pt}%
\pgfpathmoveto{\pgfqpoint{1.864986in}{0.394924in}}%
\pgfpathlineto{\pgfqpoint{1.963922in}{0.394924in}}%
\pgfpathlineto{\pgfqpoint{1.963922in}{0.295988in}}%
\pgfpathlineto{\pgfqpoint{1.864986in}{0.295988in}}%
\pgfpathlineto{\pgfqpoint{1.864986in}{0.394924in}}%
\pgfusepath{stroke,fill}%
\end{pgfscope}%
\begin{pgfscope}%
\pgfpathrectangle{\pgfqpoint{0.380943in}{0.295988in}}{\pgfqpoint{4.650000in}{0.692553in}}%
\pgfusepath{clip}%
\pgfsetbuttcap%
\pgfsetroundjoin%
\definecolor{currentfill}{rgb}{0.978316,0.963137,0.774994}%
\pgfsetfillcolor{currentfill}%
\pgfsetlinewidth{0.250937pt}%
\definecolor{currentstroke}{rgb}{1.000000,1.000000,1.000000}%
\pgfsetstrokecolor{currentstroke}%
\pgfsetdash{}{0pt}%
\pgfpathmoveto{\pgfqpoint{1.963922in}{0.394924in}}%
\pgfpathlineto{\pgfqpoint{2.062858in}{0.394924in}}%
\pgfpathlineto{\pgfqpoint{2.062858in}{0.295988in}}%
\pgfpathlineto{\pgfqpoint{1.963922in}{0.295988in}}%
\pgfpathlineto{\pgfqpoint{1.963922in}{0.394924in}}%
\pgfusepath{stroke,fill}%
\end{pgfscope}%
\begin{pgfscope}%
\pgfpathrectangle{\pgfqpoint{0.380943in}{0.295988in}}{\pgfqpoint{4.650000in}{0.692553in}}%
\pgfusepath{clip}%
\pgfsetbuttcap%
\pgfsetroundjoin%
\definecolor{currentfill}{rgb}{0.963091,0.919493,0.720185}%
\pgfsetfillcolor{currentfill}%
\pgfsetlinewidth{0.250937pt}%
\definecolor{currentstroke}{rgb}{1.000000,1.000000,1.000000}%
\pgfsetstrokecolor{currentstroke}%
\pgfsetdash{}{0pt}%
\pgfpathmoveto{\pgfqpoint{2.062858in}{0.394924in}}%
\pgfpathlineto{\pgfqpoint{2.161794in}{0.394924in}}%
\pgfpathlineto{\pgfqpoint{2.161794in}{0.295988in}}%
\pgfpathlineto{\pgfqpoint{2.062858in}{0.295988in}}%
\pgfpathlineto{\pgfqpoint{2.062858in}{0.394924in}}%
\pgfusepath{stroke,fill}%
\end{pgfscope}%
\begin{pgfscope}%
\pgfpathrectangle{\pgfqpoint{0.380943in}{0.295988in}}{\pgfqpoint{4.650000in}{0.692553in}}%
\pgfusepath{clip}%
\pgfsetbuttcap%
\pgfsetroundjoin%
\definecolor{currentfill}{rgb}{0.964275,0.912388,0.715448}%
\pgfsetfillcolor{currentfill}%
\pgfsetlinewidth{0.250937pt}%
\definecolor{currentstroke}{rgb}{1.000000,1.000000,1.000000}%
\pgfsetstrokecolor{currentstroke}%
\pgfsetdash{}{0pt}%
\pgfpathmoveto{\pgfqpoint{2.161794in}{0.394924in}}%
\pgfpathlineto{\pgfqpoint{2.260730in}{0.394924in}}%
\pgfpathlineto{\pgfqpoint{2.260730in}{0.295988in}}%
\pgfpathlineto{\pgfqpoint{2.161794in}{0.295988in}}%
\pgfpathlineto{\pgfqpoint{2.161794in}{0.394924in}}%
\pgfusepath{stroke,fill}%
\end{pgfscope}%
\begin{pgfscope}%
\pgfpathrectangle{\pgfqpoint{0.380943in}{0.295988in}}{\pgfqpoint{4.650000in}{0.692553in}}%
\pgfusepath{clip}%
\pgfsetbuttcap%
\pgfsetroundjoin%
\definecolor{currentfill}{rgb}{0.961230,0.930657,0.727628}%
\pgfsetfillcolor{currentfill}%
\pgfsetlinewidth{0.250937pt}%
\definecolor{currentstroke}{rgb}{1.000000,1.000000,1.000000}%
\pgfsetstrokecolor{currentstroke}%
\pgfsetdash{}{0pt}%
\pgfpathmoveto{\pgfqpoint{2.260730in}{0.394924in}}%
\pgfpathlineto{\pgfqpoint{2.359666in}{0.394924in}}%
\pgfpathlineto{\pgfqpoint{2.359666in}{0.295988in}}%
\pgfpathlineto{\pgfqpoint{2.260730in}{0.295988in}}%
\pgfpathlineto{\pgfqpoint{2.260730in}{0.394924in}}%
\pgfusepath{stroke,fill}%
\end{pgfscope}%
\begin{pgfscope}%
\pgfpathrectangle{\pgfqpoint{0.380943in}{0.295988in}}{\pgfqpoint{4.650000in}{0.692553in}}%
\pgfusepath{clip}%
\pgfsetbuttcap%
\pgfsetroundjoin%
\definecolor{currentfill}{rgb}{0.983391,0.971765,0.788189}%
\pgfsetfillcolor{currentfill}%
\pgfsetlinewidth{0.250937pt}%
\definecolor{currentstroke}{rgb}{1.000000,1.000000,1.000000}%
\pgfsetstrokecolor{currentstroke}%
\pgfsetdash{}{0pt}%
\pgfpathmoveto{\pgfqpoint{2.359666in}{0.394924in}}%
\pgfpathlineto{\pgfqpoint{2.458603in}{0.394924in}}%
\pgfpathlineto{\pgfqpoint{2.458603in}{0.295988in}}%
\pgfpathlineto{\pgfqpoint{2.359666in}{0.295988in}}%
\pgfpathlineto{\pgfqpoint{2.359666in}{0.394924in}}%
\pgfusepath{stroke,fill}%
\end{pgfscope}%
\begin{pgfscope}%
\pgfpathrectangle{\pgfqpoint{0.380943in}{0.295988in}}{\pgfqpoint{4.650000in}{0.692553in}}%
\pgfusepath{clip}%
\pgfsetbuttcap%
\pgfsetroundjoin%
\definecolor{currentfill}{rgb}{0.973241,0.954510,0.761799}%
\pgfsetfillcolor{currentfill}%
\pgfsetlinewidth{0.250937pt}%
\definecolor{currentstroke}{rgb}{1.000000,1.000000,1.000000}%
\pgfsetstrokecolor{currentstroke}%
\pgfsetdash{}{0pt}%
\pgfpathmoveto{\pgfqpoint{2.458603in}{0.394924in}}%
\pgfpathlineto{\pgfqpoint{2.557539in}{0.394924in}}%
\pgfpathlineto{\pgfqpoint{2.557539in}{0.295988in}}%
\pgfpathlineto{\pgfqpoint{2.458603in}{0.295988in}}%
\pgfpathlineto{\pgfqpoint{2.458603in}{0.394924in}}%
\pgfusepath{stroke,fill}%
\end{pgfscope}%
\begin{pgfscope}%
\pgfpathrectangle{\pgfqpoint{0.380943in}{0.295988in}}{\pgfqpoint{4.650000in}{0.692553in}}%
\pgfusepath{clip}%
\pgfsetbuttcap%
\pgfsetroundjoin%
\definecolor{currentfill}{rgb}{0.962584,0.922537,0.722215}%
\pgfsetfillcolor{currentfill}%
\pgfsetlinewidth{0.250937pt}%
\definecolor{currentstroke}{rgb}{1.000000,1.000000,1.000000}%
\pgfsetstrokecolor{currentstroke}%
\pgfsetdash{}{0pt}%
\pgfpathmoveto{\pgfqpoint{2.557539in}{0.394924in}}%
\pgfpathlineto{\pgfqpoint{2.656475in}{0.394924in}}%
\pgfpathlineto{\pgfqpoint{2.656475in}{0.295988in}}%
\pgfpathlineto{\pgfqpoint{2.557539in}{0.295988in}}%
\pgfpathlineto{\pgfqpoint{2.557539in}{0.394924in}}%
\pgfusepath{stroke,fill}%
\end{pgfscope}%
\begin{pgfscope}%
\pgfpathrectangle{\pgfqpoint{0.380943in}{0.295988in}}{\pgfqpoint{4.650000in}{0.692553in}}%
\pgfusepath{clip}%
\pgfsetbuttcap%
\pgfsetroundjoin%
\definecolor{currentfill}{rgb}{0.973241,0.954510,0.761799}%
\pgfsetfillcolor{currentfill}%
\pgfsetlinewidth{0.250937pt}%
\definecolor{currentstroke}{rgb}{1.000000,1.000000,1.000000}%
\pgfsetstrokecolor{currentstroke}%
\pgfsetdash{}{0pt}%
\pgfpathmoveto{\pgfqpoint{2.656475in}{0.394924in}}%
\pgfpathlineto{\pgfqpoint{2.755411in}{0.394924in}}%
\pgfpathlineto{\pgfqpoint{2.755411in}{0.295988in}}%
\pgfpathlineto{\pgfqpoint{2.656475in}{0.295988in}}%
\pgfpathlineto{\pgfqpoint{2.656475in}{0.394924in}}%
\pgfusepath{stroke,fill}%
\end{pgfscope}%
\begin{pgfscope}%
\pgfpathrectangle{\pgfqpoint{0.380943in}{0.295988in}}{\pgfqpoint{4.650000in}{0.692553in}}%
\pgfusepath{clip}%
\pgfsetbuttcap%
\pgfsetroundjoin%
\definecolor{currentfill}{rgb}{0.978316,0.963137,0.774994}%
\pgfsetfillcolor{currentfill}%
\pgfsetlinewidth{0.250937pt}%
\definecolor{currentstroke}{rgb}{1.000000,1.000000,1.000000}%
\pgfsetstrokecolor{currentstroke}%
\pgfsetdash{}{0pt}%
\pgfpathmoveto{\pgfqpoint{2.755411in}{0.394924in}}%
\pgfpathlineto{\pgfqpoint{2.854347in}{0.394924in}}%
\pgfpathlineto{\pgfqpoint{2.854347in}{0.295988in}}%
\pgfpathlineto{\pgfqpoint{2.755411in}{0.295988in}}%
\pgfpathlineto{\pgfqpoint{2.755411in}{0.394924in}}%
\pgfusepath{stroke,fill}%
\end{pgfscope}%
\begin{pgfscope}%
\pgfpathrectangle{\pgfqpoint{0.380943in}{0.295988in}}{\pgfqpoint{4.650000in}{0.692553in}}%
\pgfusepath{clip}%
\pgfsetbuttcap%
\pgfsetroundjoin%
\definecolor{currentfill}{rgb}{1.000000,1.000000,0.832141}%
\pgfsetfillcolor{currentfill}%
\pgfsetlinewidth{0.250937pt}%
\definecolor{currentstroke}{rgb}{1.000000,1.000000,1.000000}%
\pgfsetstrokecolor{currentstroke}%
\pgfsetdash{}{0pt}%
\pgfpathmoveto{\pgfqpoint{2.854347in}{0.394924in}}%
\pgfpathlineto{\pgfqpoint{2.953283in}{0.394924in}}%
\pgfpathlineto{\pgfqpoint{2.953283in}{0.295988in}}%
\pgfpathlineto{\pgfqpoint{2.854347in}{0.295988in}}%
\pgfpathlineto{\pgfqpoint{2.854347in}{0.394924in}}%
\pgfusepath{stroke,fill}%
\end{pgfscope}%
\begin{pgfscope}%
\pgfpathrectangle{\pgfqpoint{0.380943in}{0.295988in}}{\pgfqpoint{4.650000in}{0.692553in}}%
\pgfusepath{clip}%
\pgfsetbuttcap%
\pgfsetroundjoin%
\definecolor{currentfill}{rgb}{0.966459,0.901038,0.709050}%
\pgfsetfillcolor{currentfill}%
\pgfsetlinewidth{0.250937pt}%
\definecolor{currentstroke}{rgb}{1.000000,1.000000,1.000000}%
\pgfsetstrokecolor{currentstroke}%
\pgfsetdash{}{0pt}%
\pgfpathmoveto{\pgfqpoint{2.953283in}{0.394924in}}%
\pgfpathlineto{\pgfqpoint{3.052220in}{0.394924in}}%
\pgfpathlineto{\pgfqpoint{3.052220in}{0.295988in}}%
\pgfpathlineto{\pgfqpoint{2.953283in}{0.295988in}}%
\pgfpathlineto{\pgfqpoint{2.953283in}{0.394924in}}%
\pgfusepath{stroke,fill}%
\end{pgfscope}%
\begin{pgfscope}%
\pgfpathrectangle{\pgfqpoint{0.380943in}{0.295988in}}{\pgfqpoint{4.650000in}{0.692553in}}%
\pgfusepath{clip}%
\pgfsetbuttcap%
\pgfsetroundjoin%
\definecolor{currentfill}{rgb}{0.964783,0.940131,0.739808}%
\pgfsetfillcolor{currentfill}%
\pgfsetlinewidth{0.250937pt}%
\definecolor{currentstroke}{rgb}{1.000000,1.000000,1.000000}%
\pgfsetstrokecolor{currentstroke}%
\pgfsetdash{}{0pt}%
\pgfpathmoveto{\pgfqpoint{3.052220in}{0.394924in}}%
\pgfpathlineto{\pgfqpoint{3.151156in}{0.394924in}}%
\pgfpathlineto{\pgfqpoint{3.151156in}{0.295988in}}%
\pgfpathlineto{\pgfqpoint{3.052220in}{0.295988in}}%
\pgfpathlineto{\pgfqpoint{3.052220in}{0.394924in}}%
\pgfusepath{stroke,fill}%
\end{pgfscope}%
\begin{pgfscope}%
\pgfpathrectangle{\pgfqpoint{0.380943in}{0.295988in}}{\pgfqpoint{4.650000in}{0.692553in}}%
\pgfusepath{clip}%
\pgfsetbuttcap%
\pgfsetroundjoin%
\definecolor{currentfill}{rgb}{0.967474,0.895963,0.706344}%
\pgfsetfillcolor{currentfill}%
\pgfsetlinewidth{0.250937pt}%
\definecolor{currentstroke}{rgb}{1.000000,1.000000,1.000000}%
\pgfsetstrokecolor{currentstroke}%
\pgfsetdash{}{0pt}%
\pgfpathmoveto{\pgfqpoint{3.151156in}{0.394924in}}%
\pgfpathlineto{\pgfqpoint{3.250092in}{0.394924in}}%
\pgfpathlineto{\pgfqpoint{3.250092in}{0.295988in}}%
\pgfpathlineto{\pgfqpoint{3.151156in}{0.295988in}}%
\pgfpathlineto{\pgfqpoint{3.151156in}{0.394924in}}%
\pgfusepath{stroke,fill}%
\end{pgfscope}%
\begin{pgfscope}%
\pgfpathrectangle{\pgfqpoint{0.380943in}{0.295988in}}{\pgfqpoint{4.650000in}{0.692553in}}%
\pgfusepath{clip}%
\pgfsetbuttcap%
\pgfsetroundjoin%
\definecolor{currentfill}{rgb}{0.974072,0.862976,0.688750}%
\pgfsetfillcolor{currentfill}%
\pgfsetlinewidth{0.250937pt}%
\definecolor{currentstroke}{rgb}{1.000000,1.000000,1.000000}%
\pgfsetstrokecolor{currentstroke}%
\pgfsetdash{}{0pt}%
\pgfpathmoveto{\pgfqpoint{3.250092in}{0.394924in}}%
\pgfpathlineto{\pgfqpoint{3.349028in}{0.394924in}}%
\pgfpathlineto{\pgfqpoint{3.349028in}{0.295988in}}%
\pgfpathlineto{\pgfqpoint{3.250092in}{0.295988in}}%
\pgfpathlineto{\pgfqpoint{3.250092in}{0.394924in}}%
\pgfusepath{stroke,fill}%
\end{pgfscope}%
\begin{pgfscope}%
\pgfpathrectangle{\pgfqpoint{0.380943in}{0.295988in}}{\pgfqpoint{4.650000in}{0.692553in}}%
\pgfusepath{clip}%
\pgfsetbuttcap%
\pgfsetroundjoin%
\definecolor{currentfill}{rgb}{0.983714,0.819592,0.653379}%
\pgfsetfillcolor{currentfill}%
\pgfsetlinewidth{0.250937pt}%
\definecolor{currentstroke}{rgb}{1.000000,1.000000,1.000000}%
\pgfsetstrokecolor{currentstroke}%
\pgfsetdash{}{0pt}%
\pgfpathmoveto{\pgfqpoint{3.349028in}{0.394924in}}%
\pgfpathlineto{\pgfqpoint{3.447964in}{0.394924in}}%
\pgfpathlineto{\pgfqpoint{3.447964in}{0.295988in}}%
\pgfpathlineto{\pgfqpoint{3.349028in}{0.295988in}}%
\pgfpathlineto{\pgfqpoint{3.349028in}{0.394924in}}%
\pgfusepath{stroke,fill}%
\end{pgfscope}%
\begin{pgfscope}%
\pgfpathrectangle{\pgfqpoint{0.380943in}{0.295988in}}{\pgfqpoint{4.650000in}{0.692553in}}%
\pgfusepath{clip}%
\pgfsetbuttcap%
\pgfsetroundjoin%
\definecolor{currentfill}{rgb}{0.964275,0.912388,0.715448}%
\pgfsetfillcolor{currentfill}%
\pgfsetlinewidth{0.250937pt}%
\definecolor{currentstroke}{rgb}{1.000000,1.000000,1.000000}%
\pgfsetstrokecolor{currentstroke}%
\pgfsetdash{}{0pt}%
\pgfpathmoveto{\pgfqpoint{3.447964in}{0.394924in}}%
\pgfpathlineto{\pgfqpoint{3.546901in}{0.394924in}}%
\pgfpathlineto{\pgfqpoint{3.546901in}{0.295988in}}%
\pgfpathlineto{\pgfqpoint{3.447964in}{0.295988in}}%
\pgfpathlineto{\pgfqpoint{3.447964in}{0.394924in}}%
\pgfusepath{stroke,fill}%
\end{pgfscope}%
\begin{pgfscope}%
\pgfpathrectangle{\pgfqpoint{0.380943in}{0.295988in}}{\pgfqpoint{4.650000in}{0.692553in}}%
\pgfusepath{clip}%
\pgfsetbuttcap%
\pgfsetroundjoin%
\definecolor{currentfill}{rgb}{0.964783,0.940131,0.739808}%
\pgfsetfillcolor{currentfill}%
\pgfsetlinewidth{0.250937pt}%
\definecolor{currentstroke}{rgb}{1.000000,1.000000,1.000000}%
\pgfsetstrokecolor{currentstroke}%
\pgfsetdash{}{0pt}%
\pgfpathmoveto{\pgfqpoint{3.546901in}{0.394924in}}%
\pgfpathlineto{\pgfqpoint{3.645837in}{0.394924in}}%
\pgfpathlineto{\pgfqpoint{3.645837in}{0.295988in}}%
\pgfpathlineto{\pgfqpoint{3.546901in}{0.295988in}}%
\pgfpathlineto{\pgfqpoint{3.546901in}{0.394924in}}%
\pgfusepath{stroke,fill}%
\end{pgfscope}%
\begin{pgfscope}%
\pgfpathrectangle{\pgfqpoint{0.380943in}{0.295988in}}{\pgfqpoint{4.650000in}{0.692553in}}%
\pgfusepath{clip}%
\pgfsetbuttcap%
\pgfsetroundjoin%
\definecolor{currentfill}{rgb}{0.969858,0.948758,0.753003}%
\pgfsetfillcolor{currentfill}%
\pgfsetlinewidth{0.250937pt}%
\definecolor{currentstroke}{rgb}{1.000000,1.000000,1.000000}%
\pgfsetstrokecolor{currentstroke}%
\pgfsetdash{}{0pt}%
\pgfpathmoveto{\pgfqpoint{3.645837in}{0.394924in}}%
\pgfpathlineto{\pgfqpoint{3.744773in}{0.394924in}}%
\pgfpathlineto{\pgfqpoint{3.744773in}{0.295988in}}%
\pgfpathlineto{\pgfqpoint{3.645837in}{0.295988in}}%
\pgfpathlineto{\pgfqpoint{3.645837in}{0.394924in}}%
\pgfusepath{stroke,fill}%
\end{pgfscope}%
\begin{pgfscope}%
\pgfpathrectangle{\pgfqpoint{0.380943in}{0.295988in}}{\pgfqpoint{4.650000in}{0.692553in}}%
\pgfusepath{clip}%
\pgfsetbuttcap%
\pgfsetroundjoin%
\definecolor{currentfill}{rgb}{0.988604,0.796863,0.633449}%
\pgfsetfillcolor{currentfill}%
\pgfsetlinewidth{0.250937pt}%
\definecolor{currentstroke}{rgb}{1.000000,1.000000,1.000000}%
\pgfsetstrokecolor{currentstroke}%
\pgfsetdash{}{0pt}%
\pgfpathmoveto{\pgfqpoint{3.744773in}{0.394924in}}%
\pgfpathlineto{\pgfqpoint{3.843709in}{0.394924in}}%
\pgfpathlineto{\pgfqpoint{3.843709in}{0.295988in}}%
\pgfpathlineto{\pgfqpoint{3.744773in}{0.295988in}}%
\pgfpathlineto{\pgfqpoint{3.744773in}{0.394924in}}%
\pgfusepath{stroke,fill}%
\end{pgfscope}%
\begin{pgfscope}%
\pgfpathrectangle{\pgfqpoint{0.380943in}{0.295988in}}{\pgfqpoint{4.650000in}{0.692553in}}%
\pgfusepath{clip}%
\pgfsetbuttcap%
\pgfsetroundjoin%
\definecolor{currentfill}{rgb}{0.980669,0.832787,0.665559}%
\pgfsetfillcolor{currentfill}%
\pgfsetlinewidth{0.250937pt}%
\definecolor{currentstroke}{rgb}{1.000000,1.000000,1.000000}%
\pgfsetstrokecolor{currentstroke}%
\pgfsetdash{}{0pt}%
\pgfpathmoveto{\pgfqpoint{3.843709in}{0.394924in}}%
\pgfpathlineto{\pgfqpoint{3.942645in}{0.394924in}}%
\pgfpathlineto{\pgfqpoint{3.942645in}{0.295988in}}%
\pgfpathlineto{\pgfqpoint{3.843709in}{0.295988in}}%
\pgfpathlineto{\pgfqpoint{3.843709in}{0.394924in}}%
\pgfusepath{stroke,fill}%
\end{pgfscope}%
\begin{pgfscope}%
\pgfpathrectangle{\pgfqpoint{0.380943in}{0.295988in}}{\pgfqpoint{4.650000in}{0.692553in}}%
\pgfusepath{clip}%
\pgfsetbuttcap%
\pgfsetroundjoin%
\definecolor{currentfill}{rgb}{0.979654,0.837186,0.669619}%
\pgfsetfillcolor{currentfill}%
\pgfsetlinewidth{0.250937pt}%
\definecolor{currentstroke}{rgb}{1.000000,1.000000,1.000000}%
\pgfsetstrokecolor{currentstroke}%
\pgfsetdash{}{0pt}%
\pgfpathmoveto{\pgfqpoint{3.942645in}{0.394924in}}%
\pgfpathlineto{\pgfqpoint{4.041581in}{0.394924in}}%
\pgfpathlineto{\pgfqpoint{4.041581in}{0.295988in}}%
\pgfpathlineto{\pgfqpoint{3.942645in}{0.295988in}}%
\pgfpathlineto{\pgfqpoint{3.942645in}{0.394924in}}%
\pgfusepath{stroke,fill}%
\end{pgfscope}%
\begin{pgfscope}%
\pgfpathrectangle{\pgfqpoint{0.380943in}{0.295988in}}{\pgfqpoint{4.650000in}{0.692553in}}%
\pgfusepath{clip}%
\pgfsetbuttcap%
\pgfsetroundjoin%
\definecolor{currentfill}{rgb}{0.963937,0.914418,0.716801}%
\pgfsetfillcolor{currentfill}%
\pgfsetlinewidth{0.250937pt}%
\definecolor{currentstroke}{rgb}{1.000000,1.000000,1.000000}%
\pgfsetstrokecolor{currentstroke}%
\pgfsetdash{}{0pt}%
\pgfpathmoveto{\pgfqpoint{4.041581in}{0.394924in}}%
\pgfpathlineto{\pgfqpoint{4.140518in}{0.394924in}}%
\pgfpathlineto{\pgfqpoint{4.140518in}{0.295988in}}%
\pgfpathlineto{\pgfqpoint{4.041581in}{0.295988in}}%
\pgfpathlineto{\pgfqpoint{4.041581in}{0.394924in}}%
\pgfusepath{stroke,fill}%
\end{pgfscope}%
\begin{pgfscope}%
\pgfpathrectangle{\pgfqpoint{0.380943in}{0.295988in}}{\pgfqpoint{4.650000in}{0.692553in}}%
\pgfusepath{clip}%
\pgfsetbuttcap%
\pgfsetroundjoin%
\definecolor{currentfill}{rgb}{0.963429,0.917463,0.718831}%
\pgfsetfillcolor{currentfill}%
\pgfsetlinewidth{0.250937pt}%
\definecolor{currentstroke}{rgb}{1.000000,1.000000,1.000000}%
\pgfsetstrokecolor{currentstroke}%
\pgfsetdash{}{0pt}%
\pgfpathmoveto{\pgfqpoint{4.140518in}{0.394924in}}%
\pgfpathlineto{\pgfqpoint{4.239454in}{0.394924in}}%
\pgfpathlineto{\pgfqpoint{4.239454in}{0.295988in}}%
\pgfpathlineto{\pgfqpoint{4.140518in}{0.295988in}}%
\pgfpathlineto{\pgfqpoint{4.140518in}{0.394924in}}%
\pgfusepath{stroke,fill}%
\end{pgfscope}%
\begin{pgfscope}%
\pgfpathrectangle{\pgfqpoint{0.380943in}{0.295988in}}{\pgfqpoint{4.650000in}{0.692553in}}%
\pgfusepath{clip}%
\pgfsetbuttcap%
\pgfsetroundjoin%
\definecolor{currentfill}{rgb}{0.978316,0.963137,0.774994}%
\pgfsetfillcolor{currentfill}%
\pgfsetlinewidth{0.250937pt}%
\definecolor{currentstroke}{rgb}{1.000000,1.000000,1.000000}%
\pgfsetstrokecolor{currentstroke}%
\pgfsetdash{}{0pt}%
\pgfpathmoveto{\pgfqpoint{4.239454in}{0.394924in}}%
\pgfpathlineto{\pgfqpoint{4.338390in}{0.394924in}}%
\pgfpathlineto{\pgfqpoint{4.338390in}{0.295988in}}%
\pgfpathlineto{\pgfqpoint{4.239454in}{0.295988in}}%
\pgfpathlineto{\pgfqpoint{4.239454in}{0.394924in}}%
\pgfusepath{stroke,fill}%
\end{pgfscope}%
\begin{pgfscope}%
\pgfpathrectangle{\pgfqpoint{0.380943in}{0.295988in}}{\pgfqpoint{4.650000in}{0.692553in}}%
\pgfusepath{clip}%
\pgfsetbuttcap%
\pgfsetroundjoin%
\definecolor{currentfill}{rgb}{0.991849,0.986144,0.810181}%
\pgfsetfillcolor{currentfill}%
\pgfsetlinewidth{0.250937pt}%
\definecolor{currentstroke}{rgb}{1.000000,1.000000,1.000000}%
\pgfsetstrokecolor{currentstroke}%
\pgfsetdash{}{0pt}%
\pgfpathmoveto{\pgfqpoint{4.338390in}{0.394924in}}%
\pgfpathlineto{\pgfqpoint{4.437326in}{0.394924in}}%
\pgfpathlineto{\pgfqpoint{4.437326in}{0.295988in}}%
\pgfpathlineto{\pgfqpoint{4.338390in}{0.295988in}}%
\pgfpathlineto{\pgfqpoint{4.338390in}{0.394924in}}%
\pgfusepath{stroke,fill}%
\end{pgfscope}%
\begin{pgfscope}%
\pgfpathrectangle{\pgfqpoint{0.380943in}{0.295988in}}{\pgfqpoint{4.650000in}{0.692553in}}%
\pgfusepath{clip}%
\pgfsetbuttcap%
\pgfsetroundjoin%
\definecolor{currentfill}{rgb}{1.000000,1.000000,0.865975}%
\pgfsetfillcolor{currentfill}%
\pgfsetlinewidth{0.250937pt}%
\definecolor{currentstroke}{rgb}{1.000000,1.000000,1.000000}%
\pgfsetstrokecolor{currentstroke}%
\pgfsetdash{}{0pt}%
\pgfpathmoveto{\pgfqpoint{4.437326in}{0.394924in}}%
\pgfpathlineto{\pgfqpoint{4.536262in}{0.394924in}}%
\pgfpathlineto{\pgfqpoint{4.536262in}{0.295988in}}%
\pgfpathlineto{\pgfqpoint{4.437326in}{0.295988in}}%
\pgfpathlineto{\pgfqpoint{4.437326in}{0.394924in}}%
\pgfusepath{stroke,fill}%
\end{pgfscope}%
\begin{pgfscope}%
\pgfpathrectangle{\pgfqpoint{0.380943in}{0.295988in}}{\pgfqpoint{4.650000in}{0.692553in}}%
\pgfusepath{clip}%
\pgfsetbuttcap%
\pgfsetroundjoin%
\definecolor{currentfill}{rgb}{1.000000,1.000000,0.844829}%
\pgfsetfillcolor{currentfill}%
\pgfsetlinewidth{0.250937pt}%
\definecolor{currentstroke}{rgb}{1.000000,1.000000,1.000000}%
\pgfsetstrokecolor{currentstroke}%
\pgfsetdash{}{0pt}%
\pgfpathmoveto{\pgfqpoint{4.536262in}{0.394924in}}%
\pgfpathlineto{\pgfqpoint{4.635198in}{0.394924in}}%
\pgfpathlineto{\pgfqpoint{4.635198in}{0.295988in}}%
\pgfpathlineto{\pgfqpoint{4.536262in}{0.295988in}}%
\pgfpathlineto{\pgfqpoint{4.536262in}{0.394924in}}%
\pgfusepath{stroke,fill}%
\end{pgfscope}%
\begin{pgfscope}%
\pgfpathrectangle{\pgfqpoint{0.380943in}{0.295988in}}{\pgfqpoint{4.650000in}{0.692553in}}%
\pgfusepath{clip}%
\pgfsetbuttcap%
\pgfsetroundjoin%
\definecolor{currentfill}{rgb}{1.000000,1.000000,0.853287}%
\pgfsetfillcolor{currentfill}%
\pgfsetlinewidth{0.250937pt}%
\definecolor{currentstroke}{rgb}{1.000000,1.000000,1.000000}%
\pgfsetstrokecolor{currentstroke}%
\pgfsetdash{}{0pt}%
\pgfpathmoveto{\pgfqpoint{4.635198in}{0.394924in}}%
\pgfpathlineto{\pgfqpoint{4.734135in}{0.394924in}}%
\pgfpathlineto{\pgfqpoint{4.734135in}{0.295988in}}%
\pgfpathlineto{\pgfqpoint{4.635198in}{0.295988in}}%
\pgfpathlineto{\pgfqpoint{4.635198in}{0.394924in}}%
\pgfusepath{stroke,fill}%
\end{pgfscope}%
\begin{pgfscope}%
\pgfpathrectangle{\pgfqpoint{0.380943in}{0.295988in}}{\pgfqpoint{4.650000in}{0.692553in}}%
\pgfusepath{clip}%
\pgfsetbuttcap%
\pgfsetroundjoin%
\definecolor{currentfill}{rgb}{1.000000,1.000000,0.920953}%
\pgfsetfillcolor{currentfill}%
\pgfsetlinewidth{0.250937pt}%
\definecolor{currentstroke}{rgb}{1.000000,1.000000,1.000000}%
\pgfsetstrokecolor{currentstroke}%
\pgfsetdash{}{0pt}%
\pgfpathmoveto{\pgfqpoint{4.734135in}{0.394924in}}%
\pgfpathlineto{\pgfqpoint{4.833071in}{0.394924in}}%
\pgfpathlineto{\pgfqpoint{4.833071in}{0.295988in}}%
\pgfpathlineto{\pgfqpoint{4.734135in}{0.295988in}}%
\pgfpathlineto{\pgfqpoint{4.734135in}{0.394924in}}%
\pgfusepath{stroke,fill}%
\end{pgfscope}%
\begin{pgfscope}%
\pgfpathrectangle{\pgfqpoint{0.380943in}{0.295988in}}{\pgfqpoint{4.650000in}{0.692553in}}%
\pgfusepath{clip}%
\pgfsetbuttcap%
\pgfsetroundjoin%
\definecolor{currentfill}{rgb}{0.991849,0.986144,0.810181}%
\pgfsetfillcolor{currentfill}%
\pgfsetlinewidth{0.250937pt}%
\definecolor{currentstroke}{rgb}{1.000000,1.000000,1.000000}%
\pgfsetstrokecolor{currentstroke}%
\pgfsetdash{}{0pt}%
\pgfpathmoveto{\pgfqpoint{4.833071in}{0.394924in}}%
\pgfpathlineto{\pgfqpoint{4.932007in}{0.394924in}}%
\pgfpathlineto{\pgfqpoint{4.932007in}{0.295988in}}%
\pgfpathlineto{\pgfqpoint{4.833071in}{0.295988in}}%
\pgfpathlineto{\pgfqpoint{4.833071in}{0.394924in}}%
\pgfusepath{stroke,fill}%
\end{pgfscope}%
\begin{pgfscope}%
\pgfpathrectangle{\pgfqpoint{0.380943in}{0.295988in}}{\pgfqpoint{4.650000in}{0.692553in}}%
\pgfusepath{clip}%
\pgfsetbuttcap%
\pgfsetroundjoin%
\pgfsetlinewidth{0.250937pt}%
\definecolor{currentstroke}{rgb}{1.000000,1.000000,1.000000}%
\pgfsetstrokecolor{currentstroke}%
\pgfsetdash{}{0pt}%
\pgfpathmoveto{\pgfqpoint{4.932007in}{0.394924in}}%
\pgfpathlineto{\pgfqpoint{5.030943in}{0.394924in}}%
\pgfpathlineto{\pgfqpoint{5.030943in}{0.295988in}}%
\pgfpathlineto{\pgfqpoint{4.932007in}{0.295988in}}%
\pgfpathlineto{\pgfqpoint{4.932007in}{0.394924in}}%
\pgfusepath{stroke}%
\end{pgfscope}%
\begin{pgfscope}%
\pgfsetbuttcap%
\pgfsetroundjoin%
\definecolor{currentfill}{rgb}{0.000000,0.000000,0.000000}%
\pgfsetfillcolor{currentfill}%
\pgfsetlinewidth{0.803000pt}%
\definecolor{currentstroke}{rgb}{0.000000,0.000000,0.000000}%
\pgfsetstrokecolor{currentstroke}%
\pgfsetdash{}{0pt}%
\pgfsys@defobject{currentmarker}{\pgfqpoint{0.000000in}{-0.048611in}}{\pgfqpoint{0.000000in}{0.000000in}}{%
\pgfpathmoveto{\pgfqpoint{0.000000in}{0.000000in}}%
\pgfpathlineto{\pgfqpoint{0.000000in}{-0.048611in}}%
\pgfusepath{stroke,fill}%
}%
\begin{pgfscope}%
\pgfsys@transformshift{0.628283in}{0.295988in}%
\pgfsys@useobject{currentmarker}{}%
\end{pgfscope}%
\end{pgfscope}%
\begin{pgfscope}%
\definecolor{textcolor}{rgb}{0.000000,0.000000,0.000000}%
\pgfsetstrokecolor{textcolor}%
\pgfsetfillcolor{textcolor}%
\pgftext[x=0.628283in,y=0.198766in,,top]{\color{textcolor}\rmfamily\fontsize{8.000000}{9.600000}\selectfont Jan}%
\end{pgfscope}%
\begin{pgfscope}%
\pgfsetbuttcap%
\pgfsetroundjoin%
\definecolor{currentfill}{rgb}{0.000000,0.000000,0.000000}%
\pgfsetfillcolor{currentfill}%
\pgfsetlinewidth{0.803000pt}%
\definecolor{currentstroke}{rgb}{0.000000,0.000000,0.000000}%
\pgfsetstrokecolor{currentstroke}%
\pgfsetdash{}{0pt}%
\pgfsys@defobject{currentmarker}{\pgfqpoint{0.000000in}{-0.048611in}}{\pgfqpoint{0.000000in}{0.000000in}}{%
\pgfpathmoveto{\pgfqpoint{0.000000in}{0.000000in}}%
\pgfpathlineto{\pgfqpoint{0.000000in}{-0.048611in}}%
\pgfusepath{stroke,fill}%
}%
\begin{pgfscope}%
\pgfsys@transformshift{1.073496in}{0.295988in}%
\pgfsys@useobject{currentmarker}{}%
\end{pgfscope}%
\end{pgfscope}%
\begin{pgfscope}%
\definecolor{textcolor}{rgb}{0.000000,0.000000,0.000000}%
\pgfsetstrokecolor{textcolor}%
\pgfsetfillcolor{textcolor}%
\pgftext[x=1.073496in,y=0.198766in,,top]{\color{textcolor}\rmfamily\fontsize{8.000000}{9.600000}\selectfont Feb}%
\end{pgfscope}%
\begin{pgfscope}%
\pgfsetbuttcap%
\pgfsetroundjoin%
\definecolor{currentfill}{rgb}{0.000000,0.000000,0.000000}%
\pgfsetfillcolor{currentfill}%
\pgfsetlinewidth{0.803000pt}%
\definecolor{currentstroke}{rgb}{0.000000,0.000000,0.000000}%
\pgfsetstrokecolor{currentstroke}%
\pgfsetdash{}{0pt}%
\pgfsys@defobject{currentmarker}{\pgfqpoint{0.000000in}{-0.048611in}}{\pgfqpoint{0.000000in}{0.000000in}}{%
\pgfpathmoveto{\pgfqpoint{0.000000in}{0.000000in}}%
\pgfpathlineto{\pgfqpoint{0.000000in}{-0.048611in}}%
\pgfusepath{stroke,fill}%
}%
\begin{pgfscope}%
\pgfsys@transformshift{1.518709in}{0.295988in}%
\pgfsys@useobject{currentmarker}{}%
\end{pgfscope}%
\end{pgfscope}%
\begin{pgfscope}%
\definecolor{textcolor}{rgb}{0.000000,0.000000,0.000000}%
\pgfsetstrokecolor{textcolor}%
\pgfsetfillcolor{textcolor}%
\pgftext[x=1.518709in,y=0.198766in,,top]{\color{textcolor}\rmfamily\fontsize{8.000000}{9.600000}\selectfont Mar}%
\end{pgfscope}%
\begin{pgfscope}%
\pgfsetbuttcap%
\pgfsetroundjoin%
\definecolor{currentfill}{rgb}{0.000000,0.000000,0.000000}%
\pgfsetfillcolor{currentfill}%
\pgfsetlinewidth{0.803000pt}%
\definecolor{currentstroke}{rgb}{0.000000,0.000000,0.000000}%
\pgfsetstrokecolor{currentstroke}%
\pgfsetdash{}{0pt}%
\pgfsys@defobject{currentmarker}{\pgfqpoint{0.000000in}{-0.048611in}}{\pgfqpoint{0.000000in}{0.000000in}}{%
\pgfpathmoveto{\pgfqpoint{0.000000in}{0.000000in}}%
\pgfpathlineto{\pgfqpoint{0.000000in}{-0.048611in}}%
\pgfusepath{stroke,fill}%
}%
\begin{pgfscope}%
\pgfsys@transformshift{1.914454in}{0.295988in}%
\pgfsys@useobject{currentmarker}{}%
\end{pgfscope}%
\end{pgfscope}%
\begin{pgfscope}%
\definecolor{textcolor}{rgb}{0.000000,0.000000,0.000000}%
\pgfsetstrokecolor{textcolor}%
\pgfsetfillcolor{textcolor}%
\pgftext[x=1.914454in,y=0.198766in,,top]{\color{textcolor}\rmfamily\fontsize{8.000000}{9.600000}\selectfont Apr}%
\end{pgfscope}%
\begin{pgfscope}%
\pgfsetbuttcap%
\pgfsetroundjoin%
\definecolor{currentfill}{rgb}{0.000000,0.000000,0.000000}%
\pgfsetfillcolor{currentfill}%
\pgfsetlinewidth{0.803000pt}%
\definecolor{currentstroke}{rgb}{0.000000,0.000000,0.000000}%
\pgfsetstrokecolor{currentstroke}%
\pgfsetdash{}{0pt}%
\pgfsys@defobject{currentmarker}{\pgfqpoint{0.000000in}{-0.048611in}}{\pgfqpoint{0.000000in}{0.000000in}}{%
\pgfpathmoveto{\pgfqpoint{0.000000in}{0.000000in}}%
\pgfpathlineto{\pgfqpoint{0.000000in}{-0.048611in}}%
\pgfusepath{stroke,fill}%
}%
\begin{pgfscope}%
\pgfsys@transformshift{2.359666in}{0.295988in}%
\pgfsys@useobject{currentmarker}{}%
\end{pgfscope}%
\end{pgfscope}%
\begin{pgfscope}%
\definecolor{textcolor}{rgb}{0.000000,0.000000,0.000000}%
\pgfsetstrokecolor{textcolor}%
\pgfsetfillcolor{textcolor}%
\pgftext[x=2.359666in,y=0.198766in,,top]{\color{textcolor}\rmfamily\fontsize{8.000000}{9.600000}\selectfont May}%
\end{pgfscope}%
\begin{pgfscope}%
\pgfsetbuttcap%
\pgfsetroundjoin%
\definecolor{currentfill}{rgb}{0.000000,0.000000,0.000000}%
\pgfsetfillcolor{currentfill}%
\pgfsetlinewidth{0.803000pt}%
\definecolor{currentstroke}{rgb}{0.000000,0.000000,0.000000}%
\pgfsetstrokecolor{currentstroke}%
\pgfsetdash{}{0pt}%
\pgfsys@defobject{currentmarker}{\pgfqpoint{0.000000in}{-0.048611in}}{\pgfqpoint{0.000000in}{0.000000in}}{%
\pgfpathmoveto{\pgfqpoint{0.000000in}{0.000000in}}%
\pgfpathlineto{\pgfqpoint{0.000000in}{-0.048611in}}%
\pgfusepath{stroke,fill}%
}%
\begin{pgfscope}%
\pgfsys@transformshift{2.804879in}{0.295988in}%
\pgfsys@useobject{currentmarker}{}%
\end{pgfscope}%
\end{pgfscope}%
\begin{pgfscope}%
\definecolor{textcolor}{rgb}{0.000000,0.000000,0.000000}%
\pgfsetstrokecolor{textcolor}%
\pgfsetfillcolor{textcolor}%
\pgftext[x=2.804879in,y=0.198766in,,top]{\color{textcolor}\rmfamily\fontsize{8.000000}{9.600000}\selectfont Jun}%
\end{pgfscope}%
\begin{pgfscope}%
\pgfsetbuttcap%
\pgfsetroundjoin%
\definecolor{currentfill}{rgb}{0.000000,0.000000,0.000000}%
\pgfsetfillcolor{currentfill}%
\pgfsetlinewidth{0.803000pt}%
\definecolor{currentstroke}{rgb}{0.000000,0.000000,0.000000}%
\pgfsetstrokecolor{currentstroke}%
\pgfsetdash{}{0pt}%
\pgfsys@defobject{currentmarker}{\pgfqpoint{0.000000in}{-0.048611in}}{\pgfqpoint{0.000000in}{0.000000in}}{%
\pgfpathmoveto{\pgfqpoint{0.000000in}{0.000000in}}%
\pgfpathlineto{\pgfqpoint{0.000000in}{-0.048611in}}%
\pgfusepath{stroke,fill}%
}%
\begin{pgfscope}%
\pgfsys@transformshift{3.200624in}{0.295988in}%
\pgfsys@useobject{currentmarker}{}%
\end{pgfscope}%
\end{pgfscope}%
\begin{pgfscope}%
\definecolor{textcolor}{rgb}{0.000000,0.000000,0.000000}%
\pgfsetstrokecolor{textcolor}%
\pgfsetfillcolor{textcolor}%
\pgftext[x=3.200624in,y=0.198766in,,top]{\color{textcolor}\rmfamily\fontsize{8.000000}{9.600000}\selectfont Jul}%
\end{pgfscope}%
\begin{pgfscope}%
\pgfsetbuttcap%
\pgfsetroundjoin%
\definecolor{currentfill}{rgb}{0.000000,0.000000,0.000000}%
\pgfsetfillcolor{currentfill}%
\pgfsetlinewidth{0.803000pt}%
\definecolor{currentstroke}{rgb}{0.000000,0.000000,0.000000}%
\pgfsetstrokecolor{currentstroke}%
\pgfsetdash{}{0pt}%
\pgfsys@defobject{currentmarker}{\pgfqpoint{0.000000in}{-0.048611in}}{\pgfqpoint{0.000000in}{0.000000in}}{%
\pgfpathmoveto{\pgfqpoint{0.000000in}{0.000000in}}%
\pgfpathlineto{\pgfqpoint{0.000000in}{-0.048611in}}%
\pgfusepath{stroke,fill}%
}%
\begin{pgfscope}%
\pgfsys@transformshift{3.645837in}{0.295988in}%
\pgfsys@useobject{currentmarker}{}%
\end{pgfscope}%
\end{pgfscope}%
\begin{pgfscope}%
\definecolor{textcolor}{rgb}{0.000000,0.000000,0.000000}%
\pgfsetstrokecolor{textcolor}%
\pgfsetfillcolor{textcolor}%
\pgftext[x=3.645837in,y=0.198766in,,top]{\color{textcolor}\rmfamily\fontsize{8.000000}{9.600000}\selectfont Aug}%
\end{pgfscope}%
\begin{pgfscope}%
\pgfsetbuttcap%
\pgfsetroundjoin%
\definecolor{currentfill}{rgb}{0.000000,0.000000,0.000000}%
\pgfsetfillcolor{currentfill}%
\pgfsetlinewidth{0.803000pt}%
\definecolor{currentstroke}{rgb}{0.000000,0.000000,0.000000}%
\pgfsetstrokecolor{currentstroke}%
\pgfsetdash{}{0pt}%
\pgfsys@defobject{currentmarker}{\pgfqpoint{0.000000in}{-0.048611in}}{\pgfqpoint{0.000000in}{0.000000in}}{%
\pgfpathmoveto{\pgfqpoint{0.000000in}{0.000000in}}%
\pgfpathlineto{\pgfqpoint{0.000000in}{-0.048611in}}%
\pgfusepath{stroke,fill}%
}%
\begin{pgfscope}%
\pgfsys@transformshift{4.091049in}{0.295988in}%
\pgfsys@useobject{currentmarker}{}%
\end{pgfscope}%
\end{pgfscope}%
\begin{pgfscope}%
\definecolor{textcolor}{rgb}{0.000000,0.000000,0.000000}%
\pgfsetstrokecolor{textcolor}%
\pgfsetfillcolor{textcolor}%
\pgftext[x=4.091049in,y=0.198766in,,top]{\color{textcolor}\rmfamily\fontsize{8.000000}{9.600000}\selectfont Sep}%
\end{pgfscope}%
\begin{pgfscope}%
\pgfsetbuttcap%
\pgfsetroundjoin%
\definecolor{currentfill}{rgb}{0.000000,0.000000,0.000000}%
\pgfsetfillcolor{currentfill}%
\pgfsetlinewidth{0.803000pt}%
\definecolor{currentstroke}{rgb}{0.000000,0.000000,0.000000}%
\pgfsetstrokecolor{currentstroke}%
\pgfsetdash{}{0pt}%
\pgfsys@defobject{currentmarker}{\pgfqpoint{0.000000in}{-0.048611in}}{\pgfqpoint{0.000000in}{0.000000in}}{%
\pgfpathmoveto{\pgfqpoint{0.000000in}{0.000000in}}%
\pgfpathlineto{\pgfqpoint{0.000000in}{-0.048611in}}%
\pgfusepath{stroke,fill}%
}%
\begin{pgfscope}%
\pgfsys@transformshift{4.486794in}{0.295988in}%
\pgfsys@useobject{currentmarker}{}%
\end{pgfscope}%
\end{pgfscope}%
\begin{pgfscope}%
\definecolor{textcolor}{rgb}{0.000000,0.000000,0.000000}%
\pgfsetstrokecolor{textcolor}%
\pgfsetfillcolor{textcolor}%
\pgftext[x=4.486794in,y=0.198766in,,top]{\color{textcolor}\rmfamily\fontsize{8.000000}{9.600000}\selectfont Oct}%
\end{pgfscope}%
\begin{pgfscope}%
\pgfsetbuttcap%
\pgfsetroundjoin%
\definecolor{currentfill}{rgb}{0.000000,0.000000,0.000000}%
\pgfsetfillcolor{currentfill}%
\pgfsetlinewidth{0.803000pt}%
\definecolor{currentstroke}{rgb}{0.000000,0.000000,0.000000}%
\pgfsetstrokecolor{currentstroke}%
\pgfsetdash{}{0pt}%
\pgfsys@defobject{currentmarker}{\pgfqpoint{0.000000in}{-0.048611in}}{\pgfqpoint{0.000000in}{0.000000in}}{%
\pgfpathmoveto{\pgfqpoint{0.000000in}{0.000000in}}%
\pgfpathlineto{\pgfqpoint{0.000000in}{-0.048611in}}%
\pgfusepath{stroke,fill}%
}%
\begin{pgfscope}%
\pgfsys@transformshift{4.882539in}{0.295988in}%
\pgfsys@useobject{currentmarker}{}%
\end{pgfscope}%
\end{pgfscope}%
\begin{pgfscope}%
\definecolor{textcolor}{rgb}{0.000000,0.000000,0.000000}%
\pgfsetstrokecolor{textcolor}%
\pgfsetfillcolor{textcolor}%
\pgftext[x=4.882539in,y=0.198766in,,top]{\color{textcolor}\rmfamily\fontsize{8.000000}{9.600000}\selectfont Nov}%
\end{pgfscope}%
\begin{pgfscope}%
\pgfsetbuttcap%
\pgfsetroundjoin%
\definecolor{currentfill}{rgb}{0.000000,0.000000,0.000000}%
\pgfsetfillcolor{currentfill}%
\pgfsetlinewidth{0.803000pt}%
\definecolor{currentstroke}{rgb}{0.000000,0.000000,0.000000}%
\pgfsetstrokecolor{currentstroke}%
\pgfsetdash{}{0pt}%
\pgfsys@defobject{currentmarker}{\pgfqpoint{-0.048611in}{0.000000in}}{\pgfqpoint{-0.000000in}{0.000000in}}{%
\pgfpathmoveto{\pgfqpoint{-0.000000in}{0.000000in}}%
\pgfpathlineto{\pgfqpoint{-0.048611in}{0.000000in}}%
\pgfusepath{stroke,fill}%
}%
\begin{pgfscope}%
\pgfsys@transformshift{0.380943in}{0.939073in}%
\pgfsys@useobject{currentmarker}{}%
\end{pgfscope}%
\end{pgfscope}%
\begin{pgfscope}%
\definecolor{textcolor}{rgb}{0.000000,0.000000,0.000000}%
\pgfsetstrokecolor{textcolor}%
\pgfsetfillcolor{textcolor}%
\pgftext[x=0.113117in, y=0.900493in, left, base]{\color{textcolor}\rmfamily\fontsize{8.000000}{9.600000}\selectfont M}%
\end{pgfscope}%
\begin{pgfscope}%
\pgfsetbuttcap%
\pgfsetroundjoin%
\definecolor{currentfill}{rgb}{0.000000,0.000000,0.000000}%
\pgfsetfillcolor{currentfill}%
\pgfsetlinewidth{0.803000pt}%
\definecolor{currentstroke}{rgb}{0.000000,0.000000,0.000000}%
\pgfsetstrokecolor{currentstroke}%
\pgfsetdash{}{0pt}%
\pgfsys@defobject{currentmarker}{\pgfqpoint{-0.048611in}{0.000000in}}{\pgfqpoint{-0.000000in}{0.000000in}}{%
\pgfpathmoveto{\pgfqpoint{-0.000000in}{0.000000in}}%
\pgfpathlineto{\pgfqpoint{-0.048611in}{0.000000in}}%
\pgfusepath{stroke,fill}%
}%
\begin{pgfscope}%
\pgfsys@transformshift{0.380943in}{0.840137in}%
\pgfsys@useobject{currentmarker}{}%
\end{pgfscope}%
\end{pgfscope}%
\begin{pgfscope}%
\definecolor{textcolor}{rgb}{0.000000,0.000000,0.000000}%
\pgfsetstrokecolor{textcolor}%
\pgfsetfillcolor{textcolor}%
\pgftext[x=0.135957in, y=0.801556in, left, base]{\color{textcolor}\rmfamily\fontsize{8.000000}{9.600000}\selectfont T}%
\end{pgfscope}%
\begin{pgfscope}%
\pgfsetbuttcap%
\pgfsetroundjoin%
\definecolor{currentfill}{rgb}{0.000000,0.000000,0.000000}%
\pgfsetfillcolor{currentfill}%
\pgfsetlinewidth{0.803000pt}%
\definecolor{currentstroke}{rgb}{0.000000,0.000000,0.000000}%
\pgfsetstrokecolor{currentstroke}%
\pgfsetdash{}{0pt}%
\pgfsys@defobject{currentmarker}{\pgfqpoint{-0.048611in}{0.000000in}}{\pgfqpoint{-0.000000in}{0.000000in}}{%
\pgfpathmoveto{\pgfqpoint{-0.000000in}{0.000000in}}%
\pgfpathlineto{\pgfqpoint{-0.048611in}{0.000000in}}%
\pgfusepath{stroke,fill}%
}%
\begin{pgfscope}%
\pgfsys@transformshift{0.380943in}{0.741201in}%
\pgfsys@useobject{currentmarker}{}%
\end{pgfscope}%
\end{pgfscope}%
\begin{pgfscope}%
\definecolor{textcolor}{rgb}{0.000000,0.000000,0.000000}%
\pgfsetstrokecolor{textcolor}%
\pgfsetfillcolor{textcolor}%
\pgftext[x=0.100000in, y=0.702620in, left, base]{\color{textcolor}\rmfamily\fontsize{8.000000}{9.600000}\selectfont W}%
\end{pgfscope}%
\begin{pgfscope}%
\pgfsetbuttcap%
\pgfsetroundjoin%
\definecolor{currentfill}{rgb}{0.000000,0.000000,0.000000}%
\pgfsetfillcolor{currentfill}%
\pgfsetlinewidth{0.803000pt}%
\definecolor{currentstroke}{rgb}{0.000000,0.000000,0.000000}%
\pgfsetstrokecolor{currentstroke}%
\pgfsetdash{}{0pt}%
\pgfsys@defobject{currentmarker}{\pgfqpoint{-0.048611in}{0.000000in}}{\pgfqpoint{-0.000000in}{0.000000in}}{%
\pgfpathmoveto{\pgfqpoint{-0.000000in}{0.000000in}}%
\pgfpathlineto{\pgfqpoint{-0.048611in}{0.000000in}}%
\pgfusepath{stroke,fill}%
}%
\begin{pgfscope}%
\pgfsys@transformshift{0.380943in}{0.642264in}%
\pgfsys@useobject{currentmarker}{}%
\end{pgfscope}%
\end{pgfscope}%
\begin{pgfscope}%
\definecolor{textcolor}{rgb}{0.000000,0.000000,0.000000}%
\pgfsetstrokecolor{textcolor}%
\pgfsetfillcolor{textcolor}%
\pgftext[x=0.135957in, y=0.603684in, left, base]{\color{textcolor}\rmfamily\fontsize{8.000000}{9.600000}\selectfont T}%
\end{pgfscope}%
\begin{pgfscope}%
\pgfsetbuttcap%
\pgfsetroundjoin%
\definecolor{currentfill}{rgb}{0.000000,0.000000,0.000000}%
\pgfsetfillcolor{currentfill}%
\pgfsetlinewidth{0.803000pt}%
\definecolor{currentstroke}{rgb}{0.000000,0.000000,0.000000}%
\pgfsetstrokecolor{currentstroke}%
\pgfsetdash{}{0pt}%
\pgfsys@defobject{currentmarker}{\pgfqpoint{-0.048611in}{0.000000in}}{\pgfqpoint{-0.000000in}{0.000000in}}{%
\pgfpathmoveto{\pgfqpoint{-0.000000in}{0.000000in}}%
\pgfpathlineto{\pgfqpoint{-0.048611in}{0.000000in}}%
\pgfusepath{stroke,fill}%
}%
\begin{pgfscope}%
\pgfsys@transformshift{0.380943in}{0.543328in}%
\pgfsys@useobject{currentmarker}{}%
\end{pgfscope}%
\end{pgfscope}%
\begin{pgfscope}%
\definecolor{textcolor}{rgb}{0.000000,0.000000,0.000000}%
\pgfsetstrokecolor{textcolor}%
\pgfsetfillcolor{textcolor}%
\pgftext[x=0.144213in, y=0.504748in, left, base]{\color{textcolor}\rmfamily\fontsize{8.000000}{9.600000}\selectfont F}%
\end{pgfscope}%
\begin{pgfscope}%
\pgfsetbuttcap%
\pgfsetroundjoin%
\definecolor{currentfill}{rgb}{0.000000,0.000000,0.000000}%
\pgfsetfillcolor{currentfill}%
\pgfsetlinewidth{0.803000pt}%
\definecolor{currentstroke}{rgb}{0.000000,0.000000,0.000000}%
\pgfsetstrokecolor{currentstroke}%
\pgfsetdash{}{0pt}%
\pgfsys@defobject{currentmarker}{\pgfqpoint{-0.048611in}{0.000000in}}{\pgfqpoint{-0.000000in}{0.000000in}}{%
\pgfpathmoveto{\pgfqpoint{-0.000000in}{0.000000in}}%
\pgfpathlineto{\pgfqpoint{-0.048611in}{0.000000in}}%
\pgfusepath{stroke,fill}%
}%
\begin{pgfscope}%
\pgfsys@transformshift{0.380943in}{0.444392in}%
\pgfsys@useobject{currentmarker}{}%
\end{pgfscope}%
\end{pgfscope}%
\begin{pgfscope}%
\definecolor{textcolor}{rgb}{0.000000,0.000000,0.000000}%
\pgfsetstrokecolor{textcolor}%
\pgfsetfillcolor{textcolor}%
\pgftext[x=0.155633in, y=0.405812in, left, base]{\color{textcolor}\rmfamily\fontsize{8.000000}{9.600000}\selectfont S}%
\end{pgfscope}%
\begin{pgfscope}%
\pgfsetbuttcap%
\pgfsetroundjoin%
\definecolor{currentfill}{rgb}{0.000000,0.000000,0.000000}%
\pgfsetfillcolor{currentfill}%
\pgfsetlinewidth{0.803000pt}%
\definecolor{currentstroke}{rgb}{0.000000,0.000000,0.000000}%
\pgfsetstrokecolor{currentstroke}%
\pgfsetdash{}{0pt}%
\pgfsys@defobject{currentmarker}{\pgfqpoint{-0.048611in}{0.000000in}}{\pgfqpoint{-0.000000in}{0.000000in}}{%
\pgfpathmoveto{\pgfqpoint{-0.000000in}{0.000000in}}%
\pgfpathlineto{\pgfqpoint{-0.048611in}{0.000000in}}%
\pgfusepath{stroke,fill}%
}%
\begin{pgfscope}%
\pgfsys@transformshift{0.380943in}{0.345456in}%
\pgfsys@useobject{currentmarker}{}%
\end{pgfscope}%
\end{pgfscope}%
\begin{pgfscope}%
\definecolor{textcolor}{rgb}{0.000000,0.000000,0.000000}%
\pgfsetstrokecolor{textcolor}%
\pgfsetfillcolor{textcolor}%
\pgftext[x=0.155633in, y=0.306876in, left, base]{\color{textcolor}\rmfamily\fontsize{8.000000}{9.600000}\selectfont S}%
\end{pgfscope}%
\begin{pgfscope}%
\definecolor{textcolor}{rgb}{0.000000,0.000000,0.000000}%
\pgfsetstrokecolor{textcolor}%
\pgfsetfillcolor{textcolor}%
\pgftext[x=2.705943in,y=1.155208in,,]{\color{textcolor}\ttfamily\fontsize{14.400000}{17.280000}\selectfont 2021}%
\end{pgfscope}%
\end{pgfpicture}%
\makeatother%
\endgroup%

    \caption{Number of review per day}
    \label{fig:count_calendar}
\end{figure}
\section{Model}




\end{document}
