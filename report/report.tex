\documentclass[a4paper, 12pt, one column]{article}

\usepackage[english]{babel}
\usepackage[utf8x]{inputenc}
\usepackage[T1]{fontenc}
\usepackage{tikz}
\usepackage{xcolor}
\usepackage{subfig}
\usepackage{caption}
\usepackage{float}
\usepackage{float}
\usepackage[top=1.3cm, bottom=2.0cm, outer=2.5cm, inner=2.5cm, heightrounded,
marginparwidth=1.5cm, marginparsep=0.4cm, margin=2.5cm]{geometry}
\usepackage{graphicx} 
\usepackage{hyperref} 
\usepackage{amsmath} 
\usepackage{amsfonts}
\usepackage{amssymb} 
\usepackage{multirow}
\usepackage{layouts}
\usepackage[nameinlink]{cleveref}
%\usepackage{listings}
\usepackage{listingsutf8}
\crefdefaultlabelformat{#2#1#3}
\graphicspath{{images/}}

\renewcommand{\partname}{}
\renewcommand{\thepart}{}

\lstset{basicstyle=\ttfamily, keywordstyle=\bfseries, inputencoding=utf8/latin1}

\begin{document}
\input{titlepage}
\newpage


\section{Data}

\subsection{Cleaning}

The first column to clean was the \lstinline{date} column that we converted into \lstinline{Datetime} object.
All the dates had the same format, for example the first row had the following string for the \lstinline{date} column: \textit{"06 septembre 2021 suite à une expérience en septembre 2021"}.
To convert it into \lstinline[language=python]{Datetime} object we first replace the french months by their respective two digit format (\lstinline{janvier: "01", fevrier: "02", etc}), then we strip the begining trailing whitespaces and only keep the first 10 characters. Then we use the \lstinline{pandas}' function \lstinline{to_datetime()} to convert the column.

We also substracted 1 to all the \lstinline{note} column so that the target starts at 0.



\subsection{Exploration}

First we looked into the stars distribution for all the assureur (\cref{fig:distrib}).


\begin{figure}[H]
    \centering
    %% Creator: Matplotlib, PGF backend
%%
%% To include the figure in your LaTeX document, write
%%   \input{<filename>.pgf}
%%
%% Make sure the required packages are loaded in your preamble
%%   \usepackage{pgf}
%%
%% Also ensure that all the required font packages are loaded; for instance,
%% the lmodern package is sometimes necessary when using math font.
%%   \usepackage{lmodern}
%%
%% Figures using additional raster images can only be included by \input if
%% they are in the same directory as the main LaTeX file. For loading figures
%% from other directories you can use the `import` package
%%   \usepackage{import}
%%
%% and then include the figures with
%%   \import{<path to file>}{<filename>.pgf}
%%
%% Matplotlib used the following preamble
%%
\begingroup%
\makeatletter%
\begin{pgfpicture}%
\pgfpathrectangle{\pgfpointorigin}{\pgfqpoint{5.636862in}{4.449691in}}%
\pgfusepath{use as bounding box, clip}%
\begin{pgfscope}%
\pgfsetbuttcap%
\pgfsetmiterjoin%
\definecolor{currentfill}{rgb}{1.000000,1.000000,1.000000}%
\pgfsetfillcolor{currentfill}%
\pgfsetlinewidth{0.000000pt}%
\definecolor{currentstroke}{rgb}{1.000000,1.000000,1.000000}%
\pgfsetstrokecolor{currentstroke}%
\pgfsetdash{}{0pt}%
\pgfpathmoveto{\pgfqpoint{0.000000in}{0.000000in}}%
\pgfpathlineto{\pgfqpoint{5.636862in}{0.000000in}}%
\pgfpathlineto{\pgfqpoint{5.636862in}{4.449691in}}%
\pgfpathlineto{\pgfqpoint{0.000000in}{4.449691in}}%
\pgfpathlineto{\pgfqpoint{0.000000in}{0.000000in}}%
\pgfpathclose%
\pgfusepath{fill}%
\end{pgfscope}%
\begin{pgfscope}%
\pgfsetbuttcap%
\pgfsetmiterjoin%
\definecolor{currentfill}{rgb}{1.000000,1.000000,1.000000}%
\pgfsetfillcolor{currentfill}%
\pgfsetlinewidth{0.000000pt}%
\definecolor{currentstroke}{rgb}{0.000000,0.000000,0.000000}%
\pgfsetstrokecolor{currentstroke}%
\pgfsetstrokeopacity{0.000000}%
\pgfsetdash{}{0pt}%
\pgfpathmoveto{\pgfqpoint{0.654013in}{0.499691in}}%
\pgfpathlineto{\pgfqpoint{5.536862in}{0.499691in}}%
\pgfpathlineto{\pgfqpoint{5.536862in}{4.349691in}}%
\pgfpathlineto{\pgfqpoint{0.654013in}{4.349691in}}%
\pgfpathlineto{\pgfqpoint{0.654013in}{0.499691in}}%
\pgfpathclose%
\pgfusepath{fill}%
\end{pgfscope}%
\begin{pgfscope}%
\pgfpathrectangle{\pgfqpoint{0.654013in}{0.499691in}}{\pgfqpoint{4.882849in}{3.850000in}}%
\pgfusepath{clip}%
\pgfsetbuttcap%
\pgfsetmiterjoin%
\definecolor{currentfill}{rgb}{0.121569,0.466667,0.705882}%
\pgfsetfillcolor{currentfill}%
\pgfsetfillopacity{0.750000}%
\pgfsetlinewidth{1.003750pt}%
\definecolor{currentstroke}{rgb}{0.000000,0.000000,0.000000}%
\pgfsetstrokecolor{currentstroke}%
\pgfsetdash{}{0pt}%
\pgfpathmoveto{\pgfqpoint{0.875961in}{0.499691in}}%
\pgfpathlineto{\pgfqpoint{1.763751in}{0.499691in}}%
\pgfpathlineto{\pgfqpoint{1.763751in}{4.166358in}}%
\pgfpathlineto{\pgfqpoint{0.875961in}{4.166358in}}%
\pgfpathlineto{\pgfqpoint{0.875961in}{0.499691in}}%
\pgfpathclose%
\pgfusepath{stroke,fill}%
\end{pgfscope}%
\begin{pgfscope}%
\pgfpathrectangle{\pgfqpoint{0.654013in}{0.499691in}}{\pgfqpoint{4.882849in}{3.850000in}}%
\pgfusepath{clip}%
\pgfsetbuttcap%
\pgfsetmiterjoin%
\definecolor{currentfill}{rgb}{0.121569,0.466667,0.705882}%
\pgfsetfillcolor{currentfill}%
\pgfsetfillopacity{0.750000}%
\pgfsetlinewidth{1.003750pt}%
\definecolor{currentstroke}{rgb}{0.000000,0.000000,0.000000}%
\pgfsetstrokecolor{currentstroke}%
\pgfsetdash{}{0pt}%
\pgfpathmoveto{\pgfqpoint{1.763751in}{0.499691in}}%
\pgfpathlineto{\pgfqpoint{2.651542in}{0.499691in}}%
\pgfpathlineto{\pgfqpoint{2.651542in}{2.374124in}}%
\pgfpathlineto{\pgfqpoint{1.763751in}{2.374124in}}%
\pgfpathlineto{\pgfqpoint{1.763751in}{0.499691in}}%
\pgfpathclose%
\pgfusepath{stroke,fill}%
\end{pgfscope}%
\begin{pgfscope}%
\pgfpathrectangle{\pgfqpoint{0.654013in}{0.499691in}}{\pgfqpoint{4.882849in}{3.850000in}}%
\pgfusepath{clip}%
\pgfsetbuttcap%
\pgfsetmiterjoin%
\definecolor{currentfill}{rgb}{0.121569,0.466667,0.705882}%
\pgfsetfillcolor{currentfill}%
\pgfsetfillopacity{0.750000}%
\pgfsetlinewidth{1.003750pt}%
\definecolor{currentstroke}{rgb}{0.000000,0.000000,0.000000}%
\pgfsetstrokecolor{currentstroke}%
\pgfsetdash{}{0pt}%
\pgfpathmoveto{\pgfqpoint{2.651542in}{0.499691in}}%
\pgfpathlineto{\pgfqpoint{3.539333in}{0.499691in}}%
\pgfpathlineto{\pgfqpoint{3.539333in}{2.205188in}}%
\pgfpathlineto{\pgfqpoint{2.651542in}{2.205188in}}%
\pgfpathlineto{\pgfqpoint{2.651542in}{0.499691in}}%
\pgfpathclose%
\pgfusepath{stroke,fill}%
\end{pgfscope}%
\begin{pgfscope}%
\pgfpathrectangle{\pgfqpoint{0.654013in}{0.499691in}}{\pgfqpoint{4.882849in}{3.850000in}}%
\pgfusepath{clip}%
\pgfsetbuttcap%
\pgfsetmiterjoin%
\definecolor{currentfill}{rgb}{0.121569,0.466667,0.705882}%
\pgfsetfillcolor{currentfill}%
\pgfsetfillopacity{0.750000}%
\pgfsetlinewidth{1.003750pt}%
\definecolor{currentstroke}{rgb}{0.000000,0.000000,0.000000}%
\pgfsetstrokecolor{currentstroke}%
\pgfsetdash{}{0pt}%
\pgfpathmoveto{\pgfqpoint{3.539333in}{0.499691in}}%
\pgfpathlineto{\pgfqpoint{4.427123in}{0.499691in}}%
\pgfpathlineto{\pgfqpoint{4.427123in}{2.963130in}}%
\pgfpathlineto{\pgfqpoint{3.539333in}{2.963130in}}%
\pgfpathlineto{\pgfqpoint{3.539333in}{0.499691in}}%
\pgfpathclose%
\pgfusepath{stroke,fill}%
\end{pgfscope}%
\begin{pgfscope}%
\pgfpathrectangle{\pgfqpoint{0.654013in}{0.499691in}}{\pgfqpoint{4.882849in}{3.850000in}}%
\pgfusepath{clip}%
\pgfsetbuttcap%
\pgfsetmiterjoin%
\definecolor{currentfill}{rgb}{0.121569,0.466667,0.705882}%
\pgfsetfillcolor{currentfill}%
\pgfsetfillopacity{0.750000}%
\pgfsetlinewidth{1.003750pt}%
\definecolor{currentstroke}{rgb}{0.000000,0.000000,0.000000}%
\pgfsetstrokecolor{currentstroke}%
\pgfsetdash{}{0pt}%
\pgfpathmoveto{\pgfqpoint{4.427123in}{0.499691in}}%
\pgfpathlineto{\pgfqpoint{5.314914in}{0.499691in}}%
\pgfpathlineto{\pgfqpoint{5.314914in}{2.945480in}}%
\pgfpathlineto{\pgfqpoint{4.427123in}{2.945480in}}%
\pgfpathlineto{\pgfqpoint{4.427123in}{0.499691in}}%
\pgfpathclose%
\pgfusepath{stroke,fill}%
\end{pgfscope}%
\begin{pgfscope}%
\pgfsetbuttcap%
\pgfsetroundjoin%
\definecolor{currentfill}{rgb}{0.000000,0.000000,0.000000}%
\pgfsetfillcolor{currentfill}%
\pgfsetlinewidth{0.803000pt}%
\definecolor{currentstroke}{rgb}{0.000000,0.000000,0.000000}%
\pgfsetstrokecolor{currentstroke}%
\pgfsetdash{}{0pt}%
\pgfsys@defobject{currentmarker}{\pgfqpoint{0.000000in}{-0.048611in}}{\pgfqpoint{0.000000in}{0.000000in}}{%
\pgfpathmoveto{\pgfqpoint{0.000000in}{0.000000in}}%
\pgfpathlineto{\pgfqpoint{0.000000in}{-0.048611in}}%
\pgfusepath{stroke,fill}%
}%
\begin{pgfscope}%
\pgfsys@transformshift{1.319856in}{0.499691in}%
\pgfsys@useobject{currentmarker}{}%
\end{pgfscope}%
\end{pgfscope}%
\begin{pgfscope}%
\definecolor{textcolor}{rgb}{0.000000,0.000000,0.000000}%
\pgfsetstrokecolor{textcolor}%
\pgfsetfillcolor{textcolor}%
\pgftext[x=1.319856in,y=0.402469in,,top]{\color{textcolor}\rmfamily\fontsize{10.000000}{12.000000}\selectfont \(\displaystyle {1}\)}%
\end{pgfscope}%
\begin{pgfscope}%
\pgfsetbuttcap%
\pgfsetroundjoin%
\definecolor{currentfill}{rgb}{0.000000,0.000000,0.000000}%
\pgfsetfillcolor{currentfill}%
\pgfsetlinewidth{0.803000pt}%
\definecolor{currentstroke}{rgb}{0.000000,0.000000,0.000000}%
\pgfsetstrokecolor{currentstroke}%
\pgfsetdash{}{0pt}%
\pgfsys@defobject{currentmarker}{\pgfqpoint{0.000000in}{-0.048611in}}{\pgfqpoint{0.000000in}{0.000000in}}{%
\pgfpathmoveto{\pgfqpoint{0.000000in}{0.000000in}}%
\pgfpathlineto{\pgfqpoint{0.000000in}{-0.048611in}}%
\pgfusepath{stroke,fill}%
}%
\begin{pgfscope}%
\pgfsys@transformshift{2.207647in}{0.499691in}%
\pgfsys@useobject{currentmarker}{}%
\end{pgfscope}%
\end{pgfscope}%
\begin{pgfscope}%
\definecolor{textcolor}{rgb}{0.000000,0.000000,0.000000}%
\pgfsetstrokecolor{textcolor}%
\pgfsetfillcolor{textcolor}%
\pgftext[x=2.207647in,y=0.402469in,,top]{\color{textcolor}\rmfamily\fontsize{10.000000}{12.000000}\selectfont \(\displaystyle {2}\)}%
\end{pgfscope}%
\begin{pgfscope}%
\pgfsetbuttcap%
\pgfsetroundjoin%
\definecolor{currentfill}{rgb}{0.000000,0.000000,0.000000}%
\pgfsetfillcolor{currentfill}%
\pgfsetlinewidth{0.803000pt}%
\definecolor{currentstroke}{rgb}{0.000000,0.000000,0.000000}%
\pgfsetstrokecolor{currentstroke}%
\pgfsetdash{}{0pt}%
\pgfsys@defobject{currentmarker}{\pgfqpoint{0.000000in}{-0.048611in}}{\pgfqpoint{0.000000in}{0.000000in}}{%
\pgfpathmoveto{\pgfqpoint{0.000000in}{0.000000in}}%
\pgfpathlineto{\pgfqpoint{0.000000in}{-0.048611in}}%
\pgfusepath{stroke,fill}%
}%
\begin{pgfscope}%
\pgfsys@transformshift{3.095437in}{0.499691in}%
\pgfsys@useobject{currentmarker}{}%
\end{pgfscope}%
\end{pgfscope}%
\begin{pgfscope}%
\definecolor{textcolor}{rgb}{0.000000,0.000000,0.000000}%
\pgfsetstrokecolor{textcolor}%
\pgfsetfillcolor{textcolor}%
\pgftext[x=3.095437in,y=0.402469in,,top]{\color{textcolor}\rmfamily\fontsize{10.000000}{12.000000}\selectfont \(\displaystyle {3}\)}%
\end{pgfscope}%
\begin{pgfscope}%
\pgfsetbuttcap%
\pgfsetroundjoin%
\definecolor{currentfill}{rgb}{0.000000,0.000000,0.000000}%
\pgfsetfillcolor{currentfill}%
\pgfsetlinewidth{0.803000pt}%
\definecolor{currentstroke}{rgb}{0.000000,0.000000,0.000000}%
\pgfsetstrokecolor{currentstroke}%
\pgfsetdash{}{0pt}%
\pgfsys@defobject{currentmarker}{\pgfqpoint{0.000000in}{-0.048611in}}{\pgfqpoint{0.000000in}{0.000000in}}{%
\pgfpathmoveto{\pgfqpoint{0.000000in}{0.000000in}}%
\pgfpathlineto{\pgfqpoint{0.000000in}{-0.048611in}}%
\pgfusepath{stroke,fill}%
}%
\begin{pgfscope}%
\pgfsys@transformshift{3.983228in}{0.499691in}%
\pgfsys@useobject{currentmarker}{}%
\end{pgfscope}%
\end{pgfscope}%
\begin{pgfscope}%
\definecolor{textcolor}{rgb}{0.000000,0.000000,0.000000}%
\pgfsetstrokecolor{textcolor}%
\pgfsetfillcolor{textcolor}%
\pgftext[x=3.983228in,y=0.402469in,,top]{\color{textcolor}\rmfamily\fontsize{10.000000}{12.000000}\selectfont \(\displaystyle {4}\)}%
\end{pgfscope}%
\begin{pgfscope}%
\pgfsetbuttcap%
\pgfsetroundjoin%
\definecolor{currentfill}{rgb}{0.000000,0.000000,0.000000}%
\pgfsetfillcolor{currentfill}%
\pgfsetlinewidth{0.803000pt}%
\definecolor{currentstroke}{rgb}{0.000000,0.000000,0.000000}%
\pgfsetstrokecolor{currentstroke}%
\pgfsetdash{}{0pt}%
\pgfsys@defobject{currentmarker}{\pgfqpoint{0.000000in}{-0.048611in}}{\pgfqpoint{0.000000in}{0.000000in}}{%
\pgfpathmoveto{\pgfqpoint{0.000000in}{0.000000in}}%
\pgfpathlineto{\pgfqpoint{0.000000in}{-0.048611in}}%
\pgfusepath{stroke,fill}%
}%
\begin{pgfscope}%
\pgfsys@transformshift{4.871019in}{0.499691in}%
\pgfsys@useobject{currentmarker}{}%
\end{pgfscope}%
\end{pgfscope}%
\begin{pgfscope}%
\definecolor{textcolor}{rgb}{0.000000,0.000000,0.000000}%
\pgfsetstrokecolor{textcolor}%
\pgfsetfillcolor{textcolor}%
\pgftext[x=4.871019in,y=0.402469in,,top]{\color{textcolor}\rmfamily\fontsize{10.000000}{12.000000}\selectfont \(\displaystyle {5}\)}%
\end{pgfscope}%
\begin{pgfscope}%
\definecolor{textcolor}{rgb}{0.000000,0.000000,0.000000}%
\pgfsetstrokecolor{textcolor}%
\pgfsetfillcolor{textcolor}%
\pgftext[x=3.095437in,y=0.223457in,,top]{\color{textcolor}\rmfamily\fontsize{10.000000}{12.000000}\selectfont note}%
\end{pgfscope}%
\begin{pgfscope}%
\pgfsetbuttcap%
\pgfsetroundjoin%
\definecolor{currentfill}{rgb}{0.000000,0.000000,0.000000}%
\pgfsetfillcolor{currentfill}%
\pgfsetlinewidth{0.803000pt}%
\definecolor{currentstroke}{rgb}{0.000000,0.000000,0.000000}%
\pgfsetstrokecolor{currentstroke}%
\pgfsetdash{}{0pt}%
\pgfsys@defobject{currentmarker}{\pgfqpoint{-0.048611in}{0.000000in}}{\pgfqpoint{-0.000000in}{0.000000in}}{%
\pgfpathmoveto{\pgfqpoint{-0.000000in}{0.000000in}}%
\pgfpathlineto{\pgfqpoint{-0.048611in}{0.000000in}}%
\pgfusepath{stroke,fill}%
}%
\begin{pgfscope}%
\pgfsys@transformshift{0.654013in}{0.499691in}%
\pgfsys@useobject{currentmarker}{}%
\end{pgfscope}%
\end{pgfscope}%
\begin{pgfscope}%
\definecolor{textcolor}{rgb}{0.000000,0.000000,0.000000}%
\pgfsetstrokecolor{textcolor}%
\pgfsetfillcolor{textcolor}%
\pgftext[x=0.487346in, y=0.451466in, left, base]{\color{textcolor}\rmfamily\fontsize{10.000000}{12.000000}\selectfont \(\displaystyle {0}\)}%
\end{pgfscope}%
\begin{pgfscope}%
\pgfsetbuttcap%
\pgfsetroundjoin%
\definecolor{currentfill}{rgb}{0.000000,0.000000,0.000000}%
\pgfsetfillcolor{currentfill}%
\pgfsetlinewidth{0.803000pt}%
\definecolor{currentstroke}{rgb}{0.000000,0.000000,0.000000}%
\pgfsetstrokecolor{currentstroke}%
\pgfsetdash{}{0pt}%
\pgfsys@defobject{currentmarker}{\pgfqpoint{-0.048611in}{0.000000in}}{\pgfqpoint{-0.000000in}{0.000000in}}{%
\pgfpathmoveto{\pgfqpoint{-0.000000in}{0.000000in}}%
\pgfpathlineto{\pgfqpoint{-0.048611in}{0.000000in}}%
\pgfusepath{stroke,fill}%
}%
\begin{pgfscope}%
\pgfsys@transformshift{0.654013in}{1.003978in}%
\pgfsys@useobject{currentmarker}{}%
\end{pgfscope}%
\end{pgfscope}%
\begin{pgfscope}%
\definecolor{textcolor}{rgb}{0.000000,0.000000,0.000000}%
\pgfsetstrokecolor{textcolor}%
\pgfsetfillcolor{textcolor}%
\pgftext[x=0.279012in, y=0.955752in, left, base]{\color{textcolor}\rmfamily\fontsize{10.000000}{12.000000}\selectfont \(\displaystyle {1000}\)}%
\end{pgfscope}%
\begin{pgfscope}%
\pgfsetbuttcap%
\pgfsetroundjoin%
\definecolor{currentfill}{rgb}{0.000000,0.000000,0.000000}%
\pgfsetfillcolor{currentfill}%
\pgfsetlinewidth{0.803000pt}%
\definecolor{currentstroke}{rgb}{0.000000,0.000000,0.000000}%
\pgfsetstrokecolor{currentstroke}%
\pgfsetdash{}{0pt}%
\pgfsys@defobject{currentmarker}{\pgfqpoint{-0.048611in}{0.000000in}}{\pgfqpoint{-0.000000in}{0.000000in}}{%
\pgfpathmoveto{\pgfqpoint{-0.000000in}{0.000000in}}%
\pgfpathlineto{\pgfqpoint{-0.048611in}{0.000000in}}%
\pgfusepath{stroke,fill}%
}%
\begin{pgfscope}%
\pgfsys@transformshift{0.654013in}{1.508264in}%
\pgfsys@useobject{currentmarker}{}%
\end{pgfscope}%
\end{pgfscope}%
\begin{pgfscope}%
\definecolor{textcolor}{rgb}{0.000000,0.000000,0.000000}%
\pgfsetstrokecolor{textcolor}%
\pgfsetfillcolor{textcolor}%
\pgftext[x=0.279012in, y=1.460039in, left, base]{\color{textcolor}\rmfamily\fontsize{10.000000}{12.000000}\selectfont \(\displaystyle {2000}\)}%
\end{pgfscope}%
\begin{pgfscope}%
\pgfsetbuttcap%
\pgfsetroundjoin%
\definecolor{currentfill}{rgb}{0.000000,0.000000,0.000000}%
\pgfsetfillcolor{currentfill}%
\pgfsetlinewidth{0.803000pt}%
\definecolor{currentstroke}{rgb}{0.000000,0.000000,0.000000}%
\pgfsetstrokecolor{currentstroke}%
\pgfsetdash{}{0pt}%
\pgfsys@defobject{currentmarker}{\pgfqpoint{-0.048611in}{0.000000in}}{\pgfqpoint{-0.000000in}{0.000000in}}{%
\pgfpathmoveto{\pgfqpoint{-0.000000in}{0.000000in}}%
\pgfpathlineto{\pgfqpoint{-0.048611in}{0.000000in}}%
\pgfusepath{stroke,fill}%
}%
\begin{pgfscope}%
\pgfsys@transformshift{0.654013in}{2.012550in}%
\pgfsys@useobject{currentmarker}{}%
\end{pgfscope}%
\end{pgfscope}%
\begin{pgfscope}%
\definecolor{textcolor}{rgb}{0.000000,0.000000,0.000000}%
\pgfsetstrokecolor{textcolor}%
\pgfsetfillcolor{textcolor}%
\pgftext[x=0.279012in, y=1.964325in, left, base]{\color{textcolor}\rmfamily\fontsize{10.000000}{12.000000}\selectfont \(\displaystyle {3000}\)}%
\end{pgfscope}%
\begin{pgfscope}%
\pgfsetbuttcap%
\pgfsetroundjoin%
\definecolor{currentfill}{rgb}{0.000000,0.000000,0.000000}%
\pgfsetfillcolor{currentfill}%
\pgfsetlinewidth{0.803000pt}%
\definecolor{currentstroke}{rgb}{0.000000,0.000000,0.000000}%
\pgfsetstrokecolor{currentstroke}%
\pgfsetdash{}{0pt}%
\pgfsys@defobject{currentmarker}{\pgfqpoint{-0.048611in}{0.000000in}}{\pgfqpoint{-0.000000in}{0.000000in}}{%
\pgfpathmoveto{\pgfqpoint{-0.000000in}{0.000000in}}%
\pgfpathlineto{\pgfqpoint{-0.048611in}{0.000000in}}%
\pgfusepath{stroke,fill}%
}%
\begin{pgfscope}%
\pgfsys@transformshift{0.654013in}{2.516837in}%
\pgfsys@useobject{currentmarker}{}%
\end{pgfscope}%
\end{pgfscope}%
\begin{pgfscope}%
\definecolor{textcolor}{rgb}{0.000000,0.000000,0.000000}%
\pgfsetstrokecolor{textcolor}%
\pgfsetfillcolor{textcolor}%
\pgftext[x=0.279012in, y=2.468612in, left, base]{\color{textcolor}\rmfamily\fontsize{10.000000}{12.000000}\selectfont \(\displaystyle {4000}\)}%
\end{pgfscope}%
\begin{pgfscope}%
\pgfsetbuttcap%
\pgfsetroundjoin%
\definecolor{currentfill}{rgb}{0.000000,0.000000,0.000000}%
\pgfsetfillcolor{currentfill}%
\pgfsetlinewidth{0.803000pt}%
\definecolor{currentstroke}{rgb}{0.000000,0.000000,0.000000}%
\pgfsetstrokecolor{currentstroke}%
\pgfsetdash{}{0pt}%
\pgfsys@defobject{currentmarker}{\pgfqpoint{-0.048611in}{0.000000in}}{\pgfqpoint{-0.000000in}{0.000000in}}{%
\pgfpathmoveto{\pgfqpoint{-0.000000in}{0.000000in}}%
\pgfpathlineto{\pgfqpoint{-0.048611in}{0.000000in}}%
\pgfusepath{stroke,fill}%
}%
\begin{pgfscope}%
\pgfsys@transformshift{0.654013in}{3.021123in}%
\pgfsys@useobject{currentmarker}{}%
\end{pgfscope}%
\end{pgfscope}%
\begin{pgfscope}%
\definecolor{textcolor}{rgb}{0.000000,0.000000,0.000000}%
\pgfsetstrokecolor{textcolor}%
\pgfsetfillcolor{textcolor}%
\pgftext[x=0.279012in, y=2.972898in, left, base]{\color{textcolor}\rmfamily\fontsize{10.000000}{12.000000}\selectfont \(\displaystyle {5000}\)}%
\end{pgfscope}%
\begin{pgfscope}%
\pgfsetbuttcap%
\pgfsetroundjoin%
\definecolor{currentfill}{rgb}{0.000000,0.000000,0.000000}%
\pgfsetfillcolor{currentfill}%
\pgfsetlinewidth{0.803000pt}%
\definecolor{currentstroke}{rgb}{0.000000,0.000000,0.000000}%
\pgfsetstrokecolor{currentstroke}%
\pgfsetdash{}{0pt}%
\pgfsys@defobject{currentmarker}{\pgfqpoint{-0.048611in}{0.000000in}}{\pgfqpoint{-0.000000in}{0.000000in}}{%
\pgfpathmoveto{\pgfqpoint{-0.000000in}{0.000000in}}%
\pgfpathlineto{\pgfqpoint{-0.048611in}{0.000000in}}%
\pgfusepath{stroke,fill}%
}%
\begin{pgfscope}%
\pgfsys@transformshift{0.654013in}{3.525410in}%
\pgfsys@useobject{currentmarker}{}%
\end{pgfscope}%
\end{pgfscope}%
\begin{pgfscope}%
\definecolor{textcolor}{rgb}{0.000000,0.000000,0.000000}%
\pgfsetstrokecolor{textcolor}%
\pgfsetfillcolor{textcolor}%
\pgftext[x=0.279012in, y=3.477184in, left, base]{\color{textcolor}\rmfamily\fontsize{10.000000}{12.000000}\selectfont \(\displaystyle {6000}\)}%
\end{pgfscope}%
\begin{pgfscope}%
\pgfsetbuttcap%
\pgfsetroundjoin%
\definecolor{currentfill}{rgb}{0.000000,0.000000,0.000000}%
\pgfsetfillcolor{currentfill}%
\pgfsetlinewidth{0.803000pt}%
\definecolor{currentstroke}{rgb}{0.000000,0.000000,0.000000}%
\pgfsetstrokecolor{currentstroke}%
\pgfsetdash{}{0pt}%
\pgfsys@defobject{currentmarker}{\pgfqpoint{-0.048611in}{0.000000in}}{\pgfqpoint{-0.000000in}{0.000000in}}{%
\pgfpathmoveto{\pgfqpoint{-0.000000in}{0.000000in}}%
\pgfpathlineto{\pgfqpoint{-0.048611in}{0.000000in}}%
\pgfusepath{stroke,fill}%
}%
\begin{pgfscope}%
\pgfsys@transformshift{0.654013in}{4.029696in}%
\pgfsys@useobject{currentmarker}{}%
\end{pgfscope}%
\end{pgfscope}%
\begin{pgfscope}%
\definecolor{textcolor}{rgb}{0.000000,0.000000,0.000000}%
\pgfsetstrokecolor{textcolor}%
\pgfsetfillcolor{textcolor}%
\pgftext[x=0.279012in, y=3.981471in, left, base]{\color{textcolor}\rmfamily\fontsize{10.000000}{12.000000}\selectfont \(\displaystyle {7000}\)}%
\end{pgfscope}%
\begin{pgfscope}%
\definecolor{textcolor}{rgb}{0.000000,0.000000,0.000000}%
\pgfsetstrokecolor{textcolor}%
\pgfsetfillcolor{textcolor}%
\pgftext[x=0.223457in,y=2.424691in,,bottom,rotate=90.000000]{\color{textcolor}\rmfamily\fontsize{10.000000}{12.000000}\selectfont Count}%
\end{pgfscope}%
\begin{pgfscope}%
\pgfsetrectcap%
\pgfsetmiterjoin%
\pgfsetlinewidth{0.803000pt}%
\definecolor{currentstroke}{rgb}{0.000000,0.000000,0.000000}%
\pgfsetstrokecolor{currentstroke}%
\pgfsetdash{}{0pt}%
\pgfpathmoveto{\pgfqpoint{0.654013in}{0.499691in}}%
\pgfpathlineto{\pgfqpoint{0.654013in}{4.349691in}}%
\pgfusepath{stroke}%
\end{pgfscope}%
\begin{pgfscope}%
\pgfsetrectcap%
\pgfsetmiterjoin%
\pgfsetlinewidth{0.803000pt}%
\definecolor{currentstroke}{rgb}{0.000000,0.000000,0.000000}%
\pgfsetstrokecolor{currentstroke}%
\pgfsetdash{}{0pt}%
\pgfpathmoveto{\pgfqpoint{5.536862in}{0.499691in}}%
\pgfpathlineto{\pgfqpoint{5.536862in}{4.349691in}}%
\pgfusepath{stroke}%
\end{pgfscope}%
\begin{pgfscope}%
\pgfsetrectcap%
\pgfsetmiterjoin%
\pgfsetlinewidth{0.803000pt}%
\definecolor{currentstroke}{rgb}{0.000000,0.000000,0.000000}%
\pgfsetstrokecolor{currentstroke}%
\pgfsetdash{}{0pt}%
\pgfpathmoveto{\pgfqpoint{0.654013in}{0.499691in}}%
\pgfpathlineto{\pgfqpoint{5.536862in}{0.499691in}}%
\pgfusepath{stroke}%
\end{pgfscope}%
\begin{pgfscope}%
\pgfsetrectcap%
\pgfsetmiterjoin%
\pgfsetlinewidth{0.803000pt}%
\definecolor{currentstroke}{rgb}{0.000000,0.000000,0.000000}%
\pgfsetstrokecolor{currentstroke}%
\pgfsetdash{}{0pt}%
\pgfpathmoveto{\pgfqpoint{0.654013in}{4.349691in}}%
\pgfpathlineto{\pgfqpoint{5.536862in}{4.349691in}}%
\pgfusepath{stroke}%
\end{pgfscope}%
\end{pgfpicture}%
\makeatother%
\endgroup%

    \caption{Stars distribution}
    \label{fig:distrib}
\end{figure}

Then we looked into the stars distribution per assureur without and without scaling the y axis (\cref{fig:distrib_split_noscale} and \cref{fig:distrib_split_scale}).

\newgeometry{top=1cm, bottom=0cm}
\begin{figure}[H]
    \advance\leftskip-3cm
    %% Creator: Matplotlib, PGF backend
%%
%% To include the figure in your LaTeX document, write
%%   \input{<filename>.pgf}
%%
%% Make sure the required packages are loaded in your preamble
%%   \usepackage{pgf}
%%
%% Also ensure that all the required font packages are loaded; for instance,
%% the lmodern package is sometimes necessary when using math font.
%%   \usepackage{lmodern}
%%
%% Figures using additional raster images can only be included by \input if
%% they are in the same directory as the main LaTeX file. For loading figures
%% from other directories you can use the `import` package
%%   \usepackage{import}
%%
%% and then include the figures with
%%   \import{<path to file>}{<filename>.pgf}
%%
%% Matplotlib used the following preamble
%%
\begingroup%
\makeatletter%
\begin{pgfpicture}%
\pgfpathrectangle{\pgfpointorigin}{\pgfqpoint{7.998611in}{5.057558in}}%
\pgfusepath{use as bounding box, clip}%
\begin{pgfscope}%
\pgfsetbuttcap%
\pgfsetmiterjoin%
\definecolor{currentfill}{rgb}{1.000000,1.000000,1.000000}%
\pgfsetfillcolor{currentfill}%
\pgfsetlinewidth{0.000000pt}%
\definecolor{currentstroke}{rgb}{1.000000,1.000000,1.000000}%
\pgfsetstrokecolor{currentstroke}%
\pgfsetdash{}{0pt}%
\pgfpathmoveto{\pgfqpoint{0.000000in}{0.000000in}}%
\pgfpathlineto{\pgfqpoint{7.998611in}{0.000000in}}%
\pgfpathlineto{\pgfqpoint{7.998611in}{5.057558in}}%
\pgfpathlineto{\pgfqpoint{0.000000in}{5.057558in}}%
\pgfpathlineto{\pgfqpoint{0.000000in}{0.000000in}}%
\pgfpathclose%
\pgfusepath{fill}%
\end{pgfscope}%
\begin{pgfscope}%
\pgfsetbuttcap%
\pgfsetmiterjoin%
\definecolor{currentfill}{rgb}{1.000000,1.000000,1.000000}%
\pgfsetfillcolor{currentfill}%
\pgfsetlinewidth{0.000000pt}%
\definecolor{currentstroke}{rgb}{0.000000,0.000000,0.000000}%
\pgfsetstrokecolor{currentstroke}%
\pgfsetstrokeopacity{0.000000}%
\pgfsetdash{}{0pt}%
\pgfpathmoveto{\pgfqpoint{0.148611in}{4.306611in}}%
\pgfpathlineto{\pgfqpoint{0.973079in}{4.306611in}}%
\pgfpathlineto{\pgfqpoint{0.973079in}{4.768611in}}%
\pgfpathlineto{\pgfqpoint{0.148611in}{4.768611in}}%
\pgfpathlineto{\pgfqpoint{0.148611in}{4.306611in}}%
\pgfpathclose%
\pgfusepath{fill}%
\end{pgfscope}%
\begin{pgfscope}%
\pgfpathrectangle{\pgfqpoint{0.148611in}{4.306611in}}{\pgfqpoint{0.824468in}{0.462000in}}%
\pgfusepath{clip}%
\pgfsetbuttcap%
\pgfsetmiterjoin%
\definecolor{currentfill}{rgb}{0.121569,0.466667,0.705882}%
\pgfsetfillcolor{currentfill}%
\pgfsetfillopacity{0.500000}%
\pgfsetlinewidth{1.003750pt}%
\definecolor{currentstroke}{rgb}{0.000000,0.000000,0.000000}%
\pgfsetstrokecolor{currentstroke}%
\pgfsetdash{}{0pt}%
\pgfpathmoveto{\pgfqpoint{0.186087in}{4.306611in}}%
\pgfpathlineto{\pgfqpoint{0.335990in}{4.306611in}}%
\pgfpathlineto{\pgfqpoint{0.335990in}{4.574182in}}%
\pgfpathlineto{\pgfqpoint{0.186087in}{4.574182in}}%
\pgfpathlineto{\pgfqpoint{0.186087in}{4.306611in}}%
\pgfpathclose%
\pgfusepath{stroke,fill}%
\end{pgfscope}%
\begin{pgfscope}%
\pgfpathrectangle{\pgfqpoint{0.148611in}{4.306611in}}{\pgfqpoint{0.824468in}{0.462000in}}%
\pgfusepath{clip}%
\pgfsetbuttcap%
\pgfsetmiterjoin%
\definecolor{currentfill}{rgb}{0.121569,0.466667,0.705882}%
\pgfsetfillcolor{currentfill}%
\pgfsetfillopacity{0.500000}%
\pgfsetlinewidth{1.003750pt}%
\definecolor{currentstroke}{rgb}{0.000000,0.000000,0.000000}%
\pgfsetstrokecolor{currentstroke}%
\pgfsetdash{}{0pt}%
\pgfpathmoveto{\pgfqpoint{0.335990in}{4.306611in}}%
\pgfpathlineto{\pgfqpoint{0.485894in}{4.306611in}}%
\pgfpathlineto{\pgfqpoint{0.485894in}{4.520612in}}%
\pgfpathlineto{\pgfqpoint{0.335990in}{4.520612in}}%
\pgfpathlineto{\pgfqpoint{0.335990in}{4.306611in}}%
\pgfpathclose%
\pgfusepath{stroke,fill}%
\end{pgfscope}%
\begin{pgfscope}%
\pgfpathrectangle{\pgfqpoint{0.148611in}{4.306611in}}{\pgfqpoint{0.824468in}{0.462000in}}%
\pgfusepath{clip}%
\pgfsetbuttcap%
\pgfsetmiterjoin%
\definecolor{currentfill}{rgb}{0.121569,0.466667,0.705882}%
\pgfsetfillcolor{currentfill}%
\pgfsetfillopacity{0.500000}%
\pgfsetlinewidth{1.003750pt}%
\definecolor{currentstroke}{rgb}{0.000000,0.000000,0.000000}%
\pgfsetstrokecolor{currentstroke}%
\pgfsetdash{}{0pt}%
\pgfpathmoveto{\pgfqpoint{0.485894in}{4.306611in}}%
\pgfpathlineto{\pgfqpoint{0.635797in}{4.306611in}}%
\pgfpathlineto{\pgfqpoint{0.635797in}{4.616592in}}%
\pgfpathlineto{\pgfqpoint{0.485894in}{4.616592in}}%
\pgfpathlineto{\pgfqpoint{0.485894in}{4.306611in}}%
\pgfpathclose%
\pgfusepath{stroke,fill}%
\end{pgfscope}%
\begin{pgfscope}%
\pgfpathrectangle{\pgfqpoint{0.148611in}{4.306611in}}{\pgfqpoint{0.824468in}{0.462000in}}%
\pgfusepath{clip}%
\pgfsetbuttcap%
\pgfsetmiterjoin%
\definecolor{currentfill}{rgb}{0.121569,0.466667,0.705882}%
\pgfsetfillcolor{currentfill}%
\pgfsetfillopacity{0.500000}%
\pgfsetlinewidth{1.003750pt}%
\definecolor{currentstroke}{rgb}{0.000000,0.000000,0.000000}%
\pgfsetstrokecolor{currentstroke}%
\pgfsetdash{}{0pt}%
\pgfpathmoveto{\pgfqpoint{0.635797in}{4.306611in}}%
\pgfpathlineto{\pgfqpoint{0.785700in}{4.306611in}}%
\pgfpathlineto{\pgfqpoint{0.785700in}{4.746611in}}%
\pgfpathlineto{\pgfqpoint{0.635797in}{4.746611in}}%
\pgfpathlineto{\pgfqpoint{0.635797in}{4.306611in}}%
\pgfpathclose%
\pgfusepath{stroke,fill}%
\end{pgfscope}%
\begin{pgfscope}%
\pgfpathrectangle{\pgfqpoint{0.148611in}{4.306611in}}{\pgfqpoint{0.824468in}{0.462000in}}%
\pgfusepath{clip}%
\pgfsetbuttcap%
\pgfsetmiterjoin%
\definecolor{currentfill}{rgb}{0.121569,0.466667,0.705882}%
\pgfsetfillcolor{currentfill}%
\pgfsetfillopacity{0.500000}%
\pgfsetlinewidth{1.003750pt}%
\definecolor{currentstroke}{rgb}{0.000000,0.000000,0.000000}%
\pgfsetstrokecolor{currentstroke}%
\pgfsetdash{}{0pt}%
\pgfpathmoveto{\pgfqpoint{0.785700in}{4.306611in}}%
\pgfpathlineto{\pgfqpoint{0.935603in}{4.306611in}}%
\pgfpathlineto{\pgfqpoint{0.935603in}{4.720105in}}%
\pgfpathlineto{\pgfqpoint{0.785700in}{4.720105in}}%
\pgfpathlineto{\pgfqpoint{0.785700in}{4.306611in}}%
\pgfpathclose%
\pgfusepath{stroke,fill}%
\end{pgfscope}%
\begin{pgfscope}%
\pgfsetrectcap%
\pgfsetmiterjoin%
\pgfsetlinewidth{0.803000pt}%
\definecolor{currentstroke}{rgb}{0.000000,0.000000,0.000000}%
\pgfsetstrokecolor{currentstroke}%
\pgfsetdash{}{0pt}%
\pgfpathmoveto{\pgfqpoint{0.148611in}{4.306611in}}%
\pgfpathlineto{\pgfqpoint{0.148611in}{4.768611in}}%
\pgfusepath{stroke}%
\end{pgfscope}%
\begin{pgfscope}%
\pgfsetrectcap%
\pgfsetmiterjoin%
\pgfsetlinewidth{0.803000pt}%
\definecolor{currentstroke}{rgb}{0.000000,0.000000,0.000000}%
\pgfsetstrokecolor{currentstroke}%
\pgfsetdash{}{0pt}%
\pgfpathmoveto{\pgfqpoint{0.973079in}{4.306611in}}%
\pgfpathlineto{\pgfqpoint{0.973079in}{4.768611in}}%
\pgfusepath{stroke}%
\end{pgfscope}%
\begin{pgfscope}%
\pgfsetrectcap%
\pgfsetmiterjoin%
\pgfsetlinewidth{0.803000pt}%
\definecolor{currentstroke}{rgb}{0.000000,0.000000,0.000000}%
\pgfsetstrokecolor{currentstroke}%
\pgfsetdash{}{0pt}%
\pgfpathmoveto{\pgfqpoint{0.148611in}{4.306611in}}%
\pgfpathlineto{\pgfqpoint{0.973079in}{4.306611in}}%
\pgfusepath{stroke}%
\end{pgfscope}%
\begin{pgfscope}%
\pgfsetrectcap%
\pgfsetmiterjoin%
\pgfsetlinewidth{0.803000pt}%
\definecolor{currentstroke}{rgb}{0.000000,0.000000,0.000000}%
\pgfsetstrokecolor{currentstroke}%
\pgfsetdash{}{0pt}%
\pgfpathmoveto{\pgfqpoint{0.148611in}{4.768611in}}%
\pgfpathlineto{\pgfqpoint{0.973079in}{4.768611in}}%
\pgfusepath{stroke}%
\end{pgfscope}%
\begin{pgfscope}%
\definecolor{textcolor}{rgb}{0.000000,0.000000,0.000000}%
\pgfsetstrokecolor{textcolor}%
\pgfsetfillcolor{textcolor}%
\pgftext[x=0.560845in,y=4.851944in,,base]{\color{textcolor}\rmfamily\fontsize{11.000000}{13.200000}\selectfont Direct...}%
\end{pgfscope}%
\begin{pgfscope}%
\pgfsetbuttcap%
\pgfsetmiterjoin%
\definecolor{currentfill}{rgb}{1.000000,1.000000,1.000000}%
\pgfsetfillcolor{currentfill}%
\pgfsetlinewidth{0.000000pt}%
\definecolor{currentstroke}{rgb}{0.000000,0.000000,0.000000}%
\pgfsetstrokecolor{currentstroke}%
\pgfsetstrokeopacity{0.000000}%
\pgfsetdash{}{0pt}%
\pgfpathmoveto{\pgfqpoint{1.137973in}{4.306611in}}%
\pgfpathlineto{\pgfqpoint{1.962441in}{4.306611in}}%
\pgfpathlineto{\pgfqpoint{1.962441in}{4.768611in}}%
\pgfpathlineto{\pgfqpoint{1.137973in}{4.768611in}}%
\pgfpathlineto{\pgfqpoint{1.137973in}{4.306611in}}%
\pgfpathclose%
\pgfusepath{fill}%
\end{pgfscope}%
\begin{pgfscope}%
\pgfpathrectangle{\pgfqpoint{1.137973in}{4.306611in}}{\pgfqpoint{0.824468in}{0.462000in}}%
\pgfusepath{clip}%
\pgfsetbuttcap%
\pgfsetmiterjoin%
\definecolor{currentfill}{rgb}{0.121569,0.466667,0.705882}%
\pgfsetfillcolor{currentfill}%
\pgfsetfillopacity{0.500000}%
\pgfsetlinewidth{1.003750pt}%
\definecolor{currentstroke}{rgb}{0.000000,0.000000,0.000000}%
\pgfsetstrokecolor{currentstroke}%
\pgfsetdash{}{0pt}%
\pgfpathmoveto{\pgfqpoint{1.175449in}{4.306611in}}%
\pgfpathlineto{\pgfqpoint{1.325352in}{4.306611in}}%
\pgfpathlineto{\pgfqpoint{1.325352in}{4.411087in}}%
\pgfpathlineto{\pgfqpoint{1.175449in}{4.411087in}}%
\pgfpathlineto{\pgfqpoint{1.175449in}{4.306611in}}%
\pgfpathclose%
\pgfusepath{stroke,fill}%
\end{pgfscope}%
\begin{pgfscope}%
\pgfpathrectangle{\pgfqpoint{1.137973in}{4.306611in}}{\pgfqpoint{0.824468in}{0.462000in}}%
\pgfusepath{clip}%
\pgfsetbuttcap%
\pgfsetmiterjoin%
\definecolor{currentfill}{rgb}{0.121569,0.466667,0.705882}%
\pgfsetfillcolor{currentfill}%
\pgfsetfillopacity{0.500000}%
\pgfsetlinewidth{1.003750pt}%
\definecolor{currentstroke}{rgb}{0.000000,0.000000,0.000000}%
\pgfsetstrokecolor{currentstroke}%
\pgfsetdash{}{0pt}%
\pgfpathmoveto{\pgfqpoint{1.325352in}{4.306611in}}%
\pgfpathlineto{\pgfqpoint{1.475255in}{4.306611in}}%
\pgfpathlineto{\pgfqpoint{1.475255in}{4.397175in}}%
\pgfpathlineto{\pgfqpoint{1.325352in}{4.397175in}}%
\pgfpathlineto{\pgfqpoint{1.325352in}{4.306611in}}%
\pgfpathclose%
\pgfusepath{stroke,fill}%
\end{pgfscope}%
\begin{pgfscope}%
\pgfpathrectangle{\pgfqpoint{1.137973in}{4.306611in}}{\pgfqpoint{0.824468in}{0.462000in}}%
\pgfusepath{clip}%
\pgfsetbuttcap%
\pgfsetmiterjoin%
\definecolor{currentfill}{rgb}{0.121569,0.466667,0.705882}%
\pgfsetfillcolor{currentfill}%
\pgfsetfillopacity{0.500000}%
\pgfsetlinewidth{1.003750pt}%
\definecolor{currentstroke}{rgb}{0.000000,0.000000,0.000000}%
\pgfsetstrokecolor{currentstroke}%
\pgfsetdash{}{0pt}%
\pgfpathmoveto{\pgfqpoint{1.475255in}{4.306611in}}%
\pgfpathlineto{\pgfqpoint{1.625158in}{4.306611in}}%
\pgfpathlineto{\pgfqpoint{1.625158in}{4.451186in}}%
\pgfpathlineto{\pgfqpoint{1.475255in}{4.451186in}}%
\pgfpathlineto{\pgfqpoint{1.475255in}{4.306611in}}%
\pgfpathclose%
\pgfusepath{stroke,fill}%
\end{pgfscope}%
\begin{pgfscope}%
\pgfpathrectangle{\pgfqpoint{1.137973in}{4.306611in}}{\pgfqpoint{0.824468in}{0.462000in}}%
\pgfusepath{clip}%
\pgfsetbuttcap%
\pgfsetmiterjoin%
\definecolor{currentfill}{rgb}{0.121569,0.466667,0.705882}%
\pgfsetfillcolor{currentfill}%
\pgfsetfillopacity{0.500000}%
\pgfsetlinewidth{1.003750pt}%
\definecolor{currentstroke}{rgb}{0.000000,0.000000,0.000000}%
\pgfsetstrokecolor{currentstroke}%
\pgfsetdash{}{0pt}%
\pgfpathmoveto{\pgfqpoint{1.625158in}{4.306611in}}%
\pgfpathlineto{\pgfqpoint{1.775062in}{4.306611in}}%
\pgfpathlineto{\pgfqpoint{1.775062in}{4.696692in}}%
\pgfpathlineto{\pgfqpoint{1.625158in}{4.696692in}}%
\pgfpathlineto{\pgfqpoint{1.625158in}{4.306611in}}%
\pgfpathclose%
\pgfusepath{stroke,fill}%
\end{pgfscope}%
\begin{pgfscope}%
\pgfpathrectangle{\pgfqpoint{1.137973in}{4.306611in}}{\pgfqpoint{0.824468in}{0.462000in}}%
\pgfusepath{clip}%
\pgfsetbuttcap%
\pgfsetmiterjoin%
\definecolor{currentfill}{rgb}{0.121569,0.466667,0.705882}%
\pgfsetfillcolor{currentfill}%
\pgfsetfillopacity{0.500000}%
\pgfsetlinewidth{1.003750pt}%
\definecolor{currentstroke}{rgb}{0.000000,0.000000,0.000000}%
\pgfsetstrokecolor{currentstroke}%
\pgfsetdash{}{0pt}%
\pgfpathmoveto{\pgfqpoint{1.775062in}{4.306611in}}%
\pgfpathlineto{\pgfqpoint{1.924965in}{4.306611in}}%
\pgfpathlineto{\pgfqpoint{1.924965in}{4.746611in}}%
\pgfpathlineto{\pgfqpoint{1.775062in}{4.746611in}}%
\pgfpathlineto{\pgfqpoint{1.775062in}{4.306611in}}%
\pgfpathclose%
\pgfusepath{stroke,fill}%
\end{pgfscope}%
\begin{pgfscope}%
\pgfsetrectcap%
\pgfsetmiterjoin%
\pgfsetlinewidth{0.803000pt}%
\definecolor{currentstroke}{rgb}{0.000000,0.000000,0.000000}%
\pgfsetstrokecolor{currentstroke}%
\pgfsetdash{}{0pt}%
\pgfpathmoveto{\pgfqpoint{1.137973in}{4.306611in}}%
\pgfpathlineto{\pgfqpoint{1.137973in}{4.768611in}}%
\pgfusepath{stroke}%
\end{pgfscope}%
\begin{pgfscope}%
\pgfsetrectcap%
\pgfsetmiterjoin%
\pgfsetlinewidth{0.803000pt}%
\definecolor{currentstroke}{rgb}{0.000000,0.000000,0.000000}%
\pgfsetstrokecolor{currentstroke}%
\pgfsetdash{}{0pt}%
\pgfpathmoveto{\pgfqpoint{1.962441in}{4.306611in}}%
\pgfpathlineto{\pgfqpoint{1.962441in}{4.768611in}}%
\pgfusepath{stroke}%
\end{pgfscope}%
\begin{pgfscope}%
\pgfsetrectcap%
\pgfsetmiterjoin%
\pgfsetlinewidth{0.803000pt}%
\definecolor{currentstroke}{rgb}{0.000000,0.000000,0.000000}%
\pgfsetstrokecolor{currentstroke}%
\pgfsetdash{}{0pt}%
\pgfpathmoveto{\pgfqpoint{1.137973in}{4.306611in}}%
\pgfpathlineto{\pgfqpoint{1.962441in}{4.306611in}}%
\pgfusepath{stroke}%
\end{pgfscope}%
\begin{pgfscope}%
\pgfsetrectcap%
\pgfsetmiterjoin%
\pgfsetlinewidth{0.803000pt}%
\definecolor{currentstroke}{rgb}{0.000000,0.000000,0.000000}%
\pgfsetstrokecolor{currentstroke}%
\pgfsetdash{}{0pt}%
\pgfpathmoveto{\pgfqpoint{1.137973in}{4.768611in}}%
\pgfpathlineto{\pgfqpoint{1.962441in}{4.768611in}}%
\pgfusepath{stroke}%
\end{pgfscope}%
\begin{pgfscope}%
\definecolor{textcolor}{rgb}{0.000000,0.000000,0.000000}%
\pgfsetstrokecolor{textcolor}%
\pgfsetfillcolor{textcolor}%
\pgftext[x=1.550207in,y=4.851944in,,base]{\color{textcolor}\rmfamily\fontsize{11.000000}{13.200000}\selectfont L'oliv...}%
\end{pgfscope}%
\begin{pgfscope}%
\pgfsetbuttcap%
\pgfsetmiterjoin%
\definecolor{currentfill}{rgb}{1.000000,1.000000,1.000000}%
\pgfsetfillcolor{currentfill}%
\pgfsetlinewidth{0.000000pt}%
\definecolor{currentstroke}{rgb}{0.000000,0.000000,0.000000}%
\pgfsetstrokecolor{currentstroke}%
\pgfsetstrokeopacity{0.000000}%
\pgfsetdash{}{0pt}%
\pgfpathmoveto{\pgfqpoint{2.127335in}{4.306611in}}%
\pgfpathlineto{\pgfqpoint{2.951803in}{4.306611in}}%
\pgfpathlineto{\pgfqpoint{2.951803in}{4.768611in}}%
\pgfpathlineto{\pgfqpoint{2.127335in}{4.768611in}}%
\pgfpathlineto{\pgfqpoint{2.127335in}{4.306611in}}%
\pgfpathclose%
\pgfusepath{fill}%
\end{pgfscope}%
\begin{pgfscope}%
\pgfpathrectangle{\pgfqpoint{2.127335in}{4.306611in}}{\pgfqpoint{0.824468in}{0.462000in}}%
\pgfusepath{clip}%
\pgfsetbuttcap%
\pgfsetmiterjoin%
\definecolor{currentfill}{rgb}{0.121569,0.466667,0.705882}%
\pgfsetfillcolor{currentfill}%
\pgfsetfillopacity{0.500000}%
\pgfsetlinewidth{1.003750pt}%
\definecolor{currentstroke}{rgb}{0.000000,0.000000,0.000000}%
\pgfsetstrokecolor{currentstroke}%
\pgfsetdash{}{0pt}%
\pgfpathmoveto{\pgfqpoint{2.164810in}{4.306611in}}%
\pgfpathlineto{\pgfqpoint{2.314714in}{4.306611in}}%
\pgfpathlineto{\pgfqpoint{2.314714in}{4.746611in}}%
\pgfpathlineto{\pgfqpoint{2.164810in}{4.746611in}}%
\pgfpathlineto{\pgfqpoint{2.164810in}{4.306611in}}%
\pgfpathclose%
\pgfusepath{stroke,fill}%
\end{pgfscope}%
\begin{pgfscope}%
\pgfpathrectangle{\pgfqpoint{2.127335in}{4.306611in}}{\pgfqpoint{0.824468in}{0.462000in}}%
\pgfusepath{clip}%
\pgfsetbuttcap%
\pgfsetmiterjoin%
\definecolor{currentfill}{rgb}{0.121569,0.466667,0.705882}%
\pgfsetfillcolor{currentfill}%
\pgfsetfillopacity{0.500000}%
\pgfsetlinewidth{1.003750pt}%
\definecolor{currentstroke}{rgb}{0.000000,0.000000,0.000000}%
\pgfsetstrokecolor{currentstroke}%
\pgfsetdash{}{0pt}%
\pgfpathmoveto{\pgfqpoint{2.314714in}{4.306611in}}%
\pgfpathlineto{\pgfqpoint{2.464617in}{4.306611in}}%
\pgfpathlineto{\pgfqpoint{2.464617in}{4.581862in}}%
\pgfpathlineto{\pgfqpoint{2.314714in}{4.581862in}}%
\pgfpathlineto{\pgfqpoint{2.314714in}{4.306611in}}%
\pgfpathclose%
\pgfusepath{stroke,fill}%
\end{pgfscope}%
\begin{pgfscope}%
\pgfpathrectangle{\pgfqpoint{2.127335in}{4.306611in}}{\pgfqpoint{0.824468in}{0.462000in}}%
\pgfusepath{clip}%
\pgfsetbuttcap%
\pgfsetmiterjoin%
\definecolor{currentfill}{rgb}{0.121569,0.466667,0.705882}%
\pgfsetfillcolor{currentfill}%
\pgfsetfillopacity{0.500000}%
\pgfsetlinewidth{1.003750pt}%
\definecolor{currentstroke}{rgb}{0.000000,0.000000,0.000000}%
\pgfsetstrokecolor{currentstroke}%
\pgfsetdash{}{0pt}%
\pgfpathmoveto{\pgfqpoint{2.464617in}{4.306611in}}%
\pgfpathlineto{\pgfqpoint{2.614520in}{4.306611in}}%
\pgfpathlineto{\pgfqpoint{2.614520in}{4.403049in}}%
\pgfpathlineto{\pgfqpoint{2.464617in}{4.403049in}}%
\pgfpathlineto{\pgfqpoint{2.464617in}{4.306611in}}%
\pgfpathclose%
\pgfusepath{stroke,fill}%
\end{pgfscope}%
\begin{pgfscope}%
\pgfpathrectangle{\pgfqpoint{2.127335in}{4.306611in}}{\pgfqpoint{0.824468in}{0.462000in}}%
\pgfusepath{clip}%
\pgfsetbuttcap%
\pgfsetmiterjoin%
\definecolor{currentfill}{rgb}{0.121569,0.466667,0.705882}%
\pgfsetfillcolor{currentfill}%
\pgfsetfillopacity{0.500000}%
\pgfsetlinewidth{1.003750pt}%
\definecolor{currentstroke}{rgb}{0.000000,0.000000,0.000000}%
\pgfsetstrokecolor{currentstroke}%
\pgfsetdash{}{0pt}%
\pgfpathmoveto{\pgfqpoint{2.614520in}{4.306611in}}%
\pgfpathlineto{\pgfqpoint{2.764423in}{4.306611in}}%
\pgfpathlineto{\pgfqpoint{2.764423in}{4.376931in}}%
\pgfpathlineto{\pgfqpoint{2.614520in}{4.376931in}}%
\pgfpathlineto{\pgfqpoint{2.614520in}{4.306611in}}%
\pgfpathclose%
\pgfusepath{stroke,fill}%
\end{pgfscope}%
\begin{pgfscope}%
\pgfpathrectangle{\pgfqpoint{2.127335in}{4.306611in}}{\pgfqpoint{0.824468in}{0.462000in}}%
\pgfusepath{clip}%
\pgfsetbuttcap%
\pgfsetmiterjoin%
\definecolor{currentfill}{rgb}{0.121569,0.466667,0.705882}%
\pgfsetfillcolor{currentfill}%
\pgfsetfillopacity{0.500000}%
\pgfsetlinewidth{1.003750pt}%
\definecolor{currentstroke}{rgb}{0.000000,0.000000,0.000000}%
\pgfsetstrokecolor{currentstroke}%
\pgfsetdash{}{0pt}%
\pgfpathmoveto{\pgfqpoint{2.764423in}{4.306611in}}%
\pgfpathlineto{\pgfqpoint{2.914327in}{4.306611in}}%
\pgfpathlineto{\pgfqpoint{2.914327in}{4.376931in}}%
\pgfpathlineto{\pgfqpoint{2.764423in}{4.376931in}}%
\pgfpathlineto{\pgfqpoint{2.764423in}{4.306611in}}%
\pgfpathclose%
\pgfusepath{stroke,fill}%
\end{pgfscope}%
\begin{pgfscope}%
\pgfsetrectcap%
\pgfsetmiterjoin%
\pgfsetlinewidth{0.803000pt}%
\definecolor{currentstroke}{rgb}{0.000000,0.000000,0.000000}%
\pgfsetstrokecolor{currentstroke}%
\pgfsetdash{}{0pt}%
\pgfpathmoveto{\pgfqpoint{2.127335in}{4.306611in}}%
\pgfpathlineto{\pgfqpoint{2.127335in}{4.768611in}}%
\pgfusepath{stroke}%
\end{pgfscope}%
\begin{pgfscope}%
\pgfsetrectcap%
\pgfsetmiterjoin%
\pgfsetlinewidth{0.803000pt}%
\definecolor{currentstroke}{rgb}{0.000000,0.000000,0.000000}%
\pgfsetstrokecolor{currentstroke}%
\pgfsetdash{}{0pt}%
\pgfpathmoveto{\pgfqpoint{2.951803in}{4.306611in}}%
\pgfpathlineto{\pgfqpoint{2.951803in}{4.768611in}}%
\pgfusepath{stroke}%
\end{pgfscope}%
\begin{pgfscope}%
\pgfsetrectcap%
\pgfsetmiterjoin%
\pgfsetlinewidth{0.803000pt}%
\definecolor{currentstroke}{rgb}{0.000000,0.000000,0.000000}%
\pgfsetstrokecolor{currentstroke}%
\pgfsetdash{}{0pt}%
\pgfpathmoveto{\pgfqpoint{2.127335in}{4.306611in}}%
\pgfpathlineto{\pgfqpoint{2.951803in}{4.306611in}}%
\pgfusepath{stroke}%
\end{pgfscope}%
\begin{pgfscope}%
\pgfsetrectcap%
\pgfsetmiterjoin%
\pgfsetlinewidth{0.803000pt}%
\definecolor{currentstroke}{rgb}{0.000000,0.000000,0.000000}%
\pgfsetstrokecolor{currentstroke}%
\pgfsetdash{}{0pt}%
\pgfpathmoveto{\pgfqpoint{2.127335in}{4.768611in}}%
\pgfpathlineto{\pgfqpoint{2.951803in}{4.768611in}}%
\pgfusepath{stroke}%
\end{pgfscope}%
\begin{pgfscope}%
\definecolor{textcolor}{rgb}{0.000000,0.000000,0.000000}%
\pgfsetstrokecolor{textcolor}%
\pgfsetfillcolor{textcolor}%
\pgftext[x=2.539569in,y=4.851944in,,base]{\color{textcolor}\rmfamily\fontsize{11.000000}{13.200000}\selectfont Matmut}%
\end{pgfscope}%
\begin{pgfscope}%
\pgfsetbuttcap%
\pgfsetmiterjoin%
\definecolor{currentfill}{rgb}{1.000000,1.000000,1.000000}%
\pgfsetfillcolor{currentfill}%
\pgfsetlinewidth{0.000000pt}%
\definecolor{currentstroke}{rgb}{0.000000,0.000000,0.000000}%
\pgfsetstrokecolor{currentstroke}%
\pgfsetstrokeopacity{0.000000}%
\pgfsetdash{}{0pt}%
\pgfpathmoveto{\pgfqpoint{3.116696in}{4.306611in}}%
\pgfpathlineto{\pgfqpoint{3.941164in}{4.306611in}}%
\pgfpathlineto{\pgfqpoint{3.941164in}{4.768611in}}%
\pgfpathlineto{\pgfqpoint{3.116696in}{4.768611in}}%
\pgfpathlineto{\pgfqpoint{3.116696in}{4.306611in}}%
\pgfpathclose%
\pgfusepath{fill}%
\end{pgfscope}%
\begin{pgfscope}%
\pgfpathrectangle{\pgfqpoint{3.116696in}{4.306611in}}{\pgfqpoint{0.824468in}{0.462000in}}%
\pgfusepath{clip}%
\pgfsetbuttcap%
\pgfsetmiterjoin%
\definecolor{currentfill}{rgb}{0.121569,0.466667,0.705882}%
\pgfsetfillcolor{currentfill}%
\pgfsetfillopacity{0.500000}%
\pgfsetlinewidth{1.003750pt}%
\definecolor{currentstroke}{rgb}{0.000000,0.000000,0.000000}%
\pgfsetstrokecolor{currentstroke}%
\pgfsetdash{}{0pt}%
\pgfpathmoveto{\pgfqpoint{3.154172in}{4.306611in}}%
\pgfpathlineto{\pgfqpoint{3.304075in}{4.306611in}}%
\pgfpathlineto{\pgfqpoint{3.304075in}{4.746611in}}%
\pgfpathlineto{\pgfqpoint{3.154172in}{4.746611in}}%
\pgfpathlineto{\pgfqpoint{3.154172in}{4.306611in}}%
\pgfpathclose%
\pgfusepath{stroke,fill}%
\end{pgfscope}%
\begin{pgfscope}%
\pgfpathrectangle{\pgfqpoint{3.116696in}{4.306611in}}{\pgfqpoint{0.824468in}{0.462000in}}%
\pgfusepath{clip}%
\pgfsetbuttcap%
\pgfsetmiterjoin%
\definecolor{currentfill}{rgb}{0.121569,0.466667,0.705882}%
\pgfsetfillcolor{currentfill}%
\pgfsetfillopacity{0.500000}%
\pgfsetlinewidth{1.003750pt}%
\definecolor{currentstroke}{rgb}{0.000000,0.000000,0.000000}%
\pgfsetstrokecolor{currentstroke}%
\pgfsetdash{}{0pt}%
\pgfpathmoveto{\pgfqpoint{3.304075in}{4.306611in}}%
\pgfpathlineto{\pgfqpoint{3.453979in}{4.306611in}}%
\pgfpathlineto{\pgfqpoint{3.453979in}{4.481293in}}%
\pgfpathlineto{\pgfqpoint{3.304075in}{4.481293in}}%
\pgfpathlineto{\pgfqpoint{3.304075in}{4.306611in}}%
\pgfpathclose%
\pgfusepath{stroke,fill}%
\end{pgfscope}%
\begin{pgfscope}%
\pgfpathrectangle{\pgfqpoint{3.116696in}{4.306611in}}{\pgfqpoint{0.824468in}{0.462000in}}%
\pgfusepath{clip}%
\pgfsetbuttcap%
\pgfsetmiterjoin%
\definecolor{currentfill}{rgb}{0.121569,0.466667,0.705882}%
\pgfsetfillcolor{currentfill}%
\pgfsetfillopacity{0.500000}%
\pgfsetlinewidth{1.003750pt}%
\definecolor{currentstroke}{rgb}{0.000000,0.000000,0.000000}%
\pgfsetstrokecolor{currentstroke}%
\pgfsetdash{}{0pt}%
\pgfpathmoveto{\pgfqpoint{3.453979in}{4.306611in}}%
\pgfpathlineto{\pgfqpoint{3.603882in}{4.306611in}}%
\pgfpathlineto{\pgfqpoint{3.603882in}{4.580169in}}%
\pgfpathlineto{\pgfqpoint{3.453979in}{4.580169in}}%
\pgfpathlineto{\pgfqpoint{3.453979in}{4.306611in}}%
\pgfpathclose%
\pgfusepath{stroke,fill}%
\end{pgfscope}%
\begin{pgfscope}%
\pgfpathrectangle{\pgfqpoint{3.116696in}{4.306611in}}{\pgfqpoint{0.824468in}{0.462000in}}%
\pgfusepath{clip}%
\pgfsetbuttcap%
\pgfsetmiterjoin%
\definecolor{currentfill}{rgb}{0.121569,0.466667,0.705882}%
\pgfsetfillcolor{currentfill}%
\pgfsetfillopacity{0.500000}%
\pgfsetlinewidth{1.003750pt}%
\definecolor{currentstroke}{rgb}{0.000000,0.000000,0.000000}%
\pgfsetstrokecolor{currentstroke}%
\pgfsetdash{}{0pt}%
\pgfpathmoveto{\pgfqpoint{3.603882in}{4.306611in}}%
\pgfpathlineto{\pgfqpoint{3.753785in}{4.306611in}}%
\pgfpathlineto{\pgfqpoint{3.753785in}{4.603240in}}%
\pgfpathlineto{\pgfqpoint{3.603882in}{4.603240in}}%
\pgfpathlineto{\pgfqpoint{3.603882in}{4.306611in}}%
\pgfpathclose%
\pgfusepath{stroke,fill}%
\end{pgfscope}%
\begin{pgfscope}%
\pgfpathrectangle{\pgfqpoint{3.116696in}{4.306611in}}{\pgfqpoint{0.824468in}{0.462000in}}%
\pgfusepath{clip}%
\pgfsetbuttcap%
\pgfsetmiterjoin%
\definecolor{currentfill}{rgb}{0.121569,0.466667,0.705882}%
\pgfsetfillcolor{currentfill}%
\pgfsetfillopacity{0.500000}%
\pgfsetlinewidth{1.003750pt}%
\definecolor{currentstroke}{rgb}{0.000000,0.000000,0.000000}%
\pgfsetstrokecolor{currentstroke}%
\pgfsetdash{}{0pt}%
\pgfpathmoveto{\pgfqpoint{3.753785in}{4.306611in}}%
\pgfpathlineto{\pgfqpoint{3.903688in}{4.306611in}}%
\pgfpathlineto{\pgfqpoint{3.903688in}{4.540619in}}%
\pgfpathlineto{\pgfqpoint{3.753785in}{4.540619in}}%
\pgfpathlineto{\pgfqpoint{3.753785in}{4.306611in}}%
\pgfpathclose%
\pgfusepath{stroke,fill}%
\end{pgfscope}%
\begin{pgfscope}%
\pgfsetrectcap%
\pgfsetmiterjoin%
\pgfsetlinewidth{0.803000pt}%
\definecolor{currentstroke}{rgb}{0.000000,0.000000,0.000000}%
\pgfsetstrokecolor{currentstroke}%
\pgfsetdash{}{0pt}%
\pgfpathmoveto{\pgfqpoint{3.116696in}{4.306611in}}%
\pgfpathlineto{\pgfqpoint{3.116696in}{4.768611in}}%
\pgfusepath{stroke}%
\end{pgfscope}%
\begin{pgfscope}%
\pgfsetrectcap%
\pgfsetmiterjoin%
\pgfsetlinewidth{0.803000pt}%
\definecolor{currentstroke}{rgb}{0.000000,0.000000,0.000000}%
\pgfsetstrokecolor{currentstroke}%
\pgfsetdash{}{0pt}%
\pgfpathmoveto{\pgfqpoint{3.941164in}{4.306611in}}%
\pgfpathlineto{\pgfqpoint{3.941164in}{4.768611in}}%
\pgfusepath{stroke}%
\end{pgfscope}%
\begin{pgfscope}%
\pgfsetrectcap%
\pgfsetmiterjoin%
\pgfsetlinewidth{0.803000pt}%
\definecolor{currentstroke}{rgb}{0.000000,0.000000,0.000000}%
\pgfsetstrokecolor{currentstroke}%
\pgfsetdash{}{0pt}%
\pgfpathmoveto{\pgfqpoint{3.116696in}{4.306611in}}%
\pgfpathlineto{\pgfqpoint{3.941164in}{4.306611in}}%
\pgfusepath{stroke}%
\end{pgfscope}%
\begin{pgfscope}%
\pgfsetrectcap%
\pgfsetmiterjoin%
\pgfsetlinewidth{0.803000pt}%
\definecolor{currentstroke}{rgb}{0.000000,0.000000,0.000000}%
\pgfsetstrokecolor{currentstroke}%
\pgfsetdash{}{0pt}%
\pgfpathmoveto{\pgfqpoint{3.116696in}{4.768611in}}%
\pgfpathlineto{\pgfqpoint{3.941164in}{4.768611in}}%
\pgfusepath{stroke}%
\end{pgfscope}%
\begin{pgfscope}%
\definecolor{textcolor}{rgb}{0.000000,0.000000,0.000000}%
\pgfsetstrokecolor{textcolor}%
\pgfsetfillcolor{textcolor}%
\pgftext[x=3.528930in,y=4.851944in,,base]{\color{textcolor}\rmfamily\fontsize{11.000000}{13.200000}\selectfont Néolia...}%
\end{pgfscope}%
\begin{pgfscope}%
\pgfsetbuttcap%
\pgfsetmiterjoin%
\definecolor{currentfill}{rgb}{1.000000,1.000000,1.000000}%
\pgfsetfillcolor{currentfill}%
\pgfsetlinewidth{0.000000pt}%
\definecolor{currentstroke}{rgb}{0.000000,0.000000,0.000000}%
\pgfsetstrokecolor{currentstroke}%
\pgfsetstrokeopacity{0.000000}%
\pgfsetdash{}{0pt}%
\pgfpathmoveto{\pgfqpoint{4.106058in}{4.306611in}}%
\pgfpathlineto{\pgfqpoint{4.930526in}{4.306611in}}%
\pgfpathlineto{\pgfqpoint{4.930526in}{4.768611in}}%
\pgfpathlineto{\pgfqpoint{4.106058in}{4.768611in}}%
\pgfpathlineto{\pgfqpoint{4.106058in}{4.306611in}}%
\pgfpathclose%
\pgfusepath{fill}%
\end{pgfscope}%
\begin{pgfscope}%
\pgfpathrectangle{\pgfqpoint{4.106058in}{4.306611in}}{\pgfqpoint{0.824468in}{0.462000in}}%
\pgfusepath{clip}%
\pgfsetbuttcap%
\pgfsetmiterjoin%
\definecolor{currentfill}{rgb}{0.121569,0.466667,0.705882}%
\pgfsetfillcolor{currentfill}%
\pgfsetfillopacity{0.500000}%
\pgfsetlinewidth{1.003750pt}%
\definecolor{currentstroke}{rgb}{0.000000,0.000000,0.000000}%
\pgfsetstrokecolor{currentstroke}%
\pgfsetdash{}{0pt}%
\pgfpathmoveto{\pgfqpoint{4.143534in}{4.306611in}}%
\pgfpathlineto{\pgfqpoint{4.293437in}{4.306611in}}%
\pgfpathlineto{\pgfqpoint{4.293437in}{4.746611in}}%
\pgfpathlineto{\pgfqpoint{4.143534in}{4.746611in}}%
\pgfpathlineto{\pgfqpoint{4.143534in}{4.306611in}}%
\pgfpathclose%
\pgfusepath{stroke,fill}%
\end{pgfscope}%
\begin{pgfscope}%
\pgfpathrectangle{\pgfqpoint{4.106058in}{4.306611in}}{\pgfqpoint{0.824468in}{0.462000in}}%
\pgfusepath{clip}%
\pgfsetbuttcap%
\pgfsetmiterjoin%
\definecolor{currentfill}{rgb}{0.121569,0.466667,0.705882}%
\pgfsetfillcolor{currentfill}%
\pgfsetfillopacity{0.500000}%
\pgfsetlinewidth{1.003750pt}%
\definecolor{currentstroke}{rgb}{0.000000,0.000000,0.000000}%
\pgfsetstrokecolor{currentstroke}%
\pgfsetdash{}{0pt}%
\pgfpathmoveto{\pgfqpoint{4.293437in}{4.306611in}}%
\pgfpathlineto{\pgfqpoint{4.443340in}{4.306611in}}%
\pgfpathlineto{\pgfqpoint{4.443340in}{4.484155in}}%
\pgfpathlineto{\pgfqpoint{4.293437in}{4.484155in}}%
\pgfpathlineto{\pgfqpoint{4.293437in}{4.306611in}}%
\pgfpathclose%
\pgfusepath{stroke,fill}%
\end{pgfscope}%
\begin{pgfscope}%
\pgfpathrectangle{\pgfqpoint{4.106058in}{4.306611in}}{\pgfqpoint{0.824468in}{0.462000in}}%
\pgfusepath{clip}%
\pgfsetbuttcap%
\pgfsetmiterjoin%
\definecolor{currentfill}{rgb}{0.121569,0.466667,0.705882}%
\pgfsetfillcolor{currentfill}%
\pgfsetfillopacity{0.500000}%
\pgfsetlinewidth{1.003750pt}%
\definecolor{currentstroke}{rgb}{0.000000,0.000000,0.000000}%
\pgfsetstrokecolor{currentstroke}%
\pgfsetdash{}{0pt}%
\pgfpathmoveto{\pgfqpoint{4.443340in}{4.306611in}}%
\pgfpathlineto{\pgfqpoint{4.593244in}{4.306611in}}%
\pgfpathlineto{\pgfqpoint{4.593244in}{4.484155in}}%
\pgfpathlineto{\pgfqpoint{4.443340in}{4.484155in}}%
\pgfpathlineto{\pgfqpoint{4.443340in}{4.306611in}}%
\pgfpathclose%
\pgfusepath{stroke,fill}%
\end{pgfscope}%
\begin{pgfscope}%
\pgfpathrectangle{\pgfqpoint{4.106058in}{4.306611in}}{\pgfqpoint{0.824468in}{0.462000in}}%
\pgfusepath{clip}%
\pgfsetbuttcap%
\pgfsetmiterjoin%
\definecolor{currentfill}{rgb}{0.121569,0.466667,0.705882}%
\pgfsetfillcolor{currentfill}%
\pgfsetfillopacity{0.500000}%
\pgfsetlinewidth{1.003750pt}%
\definecolor{currentstroke}{rgb}{0.000000,0.000000,0.000000}%
\pgfsetstrokecolor{currentstroke}%
\pgfsetdash{}{0pt}%
\pgfpathmoveto{\pgfqpoint{4.593244in}{4.306611in}}%
\pgfpathlineto{\pgfqpoint{4.743147in}{4.306611in}}%
\pgfpathlineto{\pgfqpoint{4.743147in}{4.476436in}}%
\pgfpathlineto{\pgfqpoint{4.593244in}{4.476436in}}%
\pgfpathlineto{\pgfqpoint{4.593244in}{4.306611in}}%
\pgfpathclose%
\pgfusepath{stroke,fill}%
\end{pgfscope}%
\begin{pgfscope}%
\pgfpathrectangle{\pgfqpoint{4.106058in}{4.306611in}}{\pgfqpoint{0.824468in}{0.462000in}}%
\pgfusepath{clip}%
\pgfsetbuttcap%
\pgfsetmiterjoin%
\definecolor{currentfill}{rgb}{0.121569,0.466667,0.705882}%
\pgfsetfillcolor{currentfill}%
\pgfsetfillopacity{0.500000}%
\pgfsetlinewidth{1.003750pt}%
\definecolor{currentstroke}{rgb}{0.000000,0.000000,0.000000}%
\pgfsetstrokecolor{currentstroke}%
\pgfsetdash{}{0pt}%
\pgfpathmoveto{\pgfqpoint{4.743147in}{4.306611in}}%
\pgfpathlineto{\pgfqpoint{4.893050in}{4.306611in}}%
\pgfpathlineto{\pgfqpoint{4.893050in}{4.433980in}}%
\pgfpathlineto{\pgfqpoint{4.743147in}{4.433980in}}%
\pgfpathlineto{\pgfqpoint{4.743147in}{4.306611in}}%
\pgfpathclose%
\pgfusepath{stroke,fill}%
\end{pgfscope}%
\begin{pgfscope}%
\pgfsetrectcap%
\pgfsetmiterjoin%
\pgfsetlinewidth{0.803000pt}%
\definecolor{currentstroke}{rgb}{0.000000,0.000000,0.000000}%
\pgfsetstrokecolor{currentstroke}%
\pgfsetdash{}{0pt}%
\pgfpathmoveto{\pgfqpoint{4.106058in}{4.306611in}}%
\pgfpathlineto{\pgfqpoint{4.106058in}{4.768611in}}%
\pgfusepath{stroke}%
\end{pgfscope}%
\begin{pgfscope}%
\pgfsetrectcap%
\pgfsetmiterjoin%
\pgfsetlinewidth{0.803000pt}%
\definecolor{currentstroke}{rgb}{0.000000,0.000000,0.000000}%
\pgfsetstrokecolor{currentstroke}%
\pgfsetdash{}{0pt}%
\pgfpathmoveto{\pgfqpoint{4.930526in}{4.306611in}}%
\pgfpathlineto{\pgfqpoint{4.930526in}{4.768611in}}%
\pgfusepath{stroke}%
\end{pgfscope}%
\begin{pgfscope}%
\pgfsetrectcap%
\pgfsetmiterjoin%
\pgfsetlinewidth{0.803000pt}%
\definecolor{currentstroke}{rgb}{0.000000,0.000000,0.000000}%
\pgfsetstrokecolor{currentstroke}%
\pgfsetdash{}{0pt}%
\pgfpathmoveto{\pgfqpoint{4.106058in}{4.306611in}}%
\pgfpathlineto{\pgfqpoint{4.930526in}{4.306611in}}%
\pgfusepath{stroke}%
\end{pgfscope}%
\begin{pgfscope}%
\pgfsetrectcap%
\pgfsetmiterjoin%
\pgfsetlinewidth{0.803000pt}%
\definecolor{currentstroke}{rgb}{0.000000,0.000000,0.000000}%
\pgfsetstrokecolor{currentstroke}%
\pgfsetdash{}{0pt}%
\pgfpathmoveto{\pgfqpoint{4.106058in}{4.768611in}}%
\pgfpathlineto{\pgfqpoint{4.930526in}{4.768611in}}%
\pgfusepath{stroke}%
\end{pgfscope}%
\begin{pgfscope}%
\definecolor{textcolor}{rgb}{0.000000,0.000000,0.000000}%
\pgfsetstrokecolor{textcolor}%
\pgfsetfillcolor{textcolor}%
\pgftext[x=4.518292in,y=4.851944in,,base]{\color{textcolor}\rmfamily\fontsize{11.000000}{13.200000}\selectfont APRIL}%
\end{pgfscope}%
\begin{pgfscope}%
\pgfsetbuttcap%
\pgfsetmiterjoin%
\definecolor{currentfill}{rgb}{1.000000,1.000000,1.000000}%
\pgfsetfillcolor{currentfill}%
\pgfsetlinewidth{0.000000pt}%
\definecolor{currentstroke}{rgb}{0.000000,0.000000,0.000000}%
\pgfsetstrokecolor{currentstroke}%
\pgfsetstrokeopacity{0.000000}%
\pgfsetdash{}{0pt}%
\pgfpathmoveto{\pgfqpoint{5.095420in}{4.306611in}}%
\pgfpathlineto{\pgfqpoint{5.919888in}{4.306611in}}%
\pgfpathlineto{\pgfqpoint{5.919888in}{4.768611in}}%
\pgfpathlineto{\pgfqpoint{5.095420in}{4.768611in}}%
\pgfpathlineto{\pgfqpoint{5.095420in}{4.306611in}}%
\pgfpathclose%
\pgfusepath{fill}%
\end{pgfscope}%
\begin{pgfscope}%
\pgfpathrectangle{\pgfqpoint{5.095420in}{4.306611in}}{\pgfqpoint{0.824468in}{0.462000in}}%
\pgfusepath{clip}%
\pgfsetbuttcap%
\pgfsetmiterjoin%
\definecolor{currentfill}{rgb}{0.121569,0.466667,0.705882}%
\pgfsetfillcolor{currentfill}%
\pgfsetfillopacity{0.500000}%
\pgfsetlinewidth{1.003750pt}%
\definecolor{currentstroke}{rgb}{0.000000,0.000000,0.000000}%
\pgfsetstrokecolor{currentstroke}%
\pgfsetdash{}{0pt}%
\pgfpathmoveto{\pgfqpoint{5.132895in}{4.306611in}}%
\pgfpathlineto{\pgfqpoint{5.282799in}{4.306611in}}%
\pgfpathlineto{\pgfqpoint{5.282799in}{4.746611in}}%
\pgfpathlineto{\pgfqpoint{5.132895in}{4.746611in}}%
\pgfpathlineto{\pgfqpoint{5.132895in}{4.306611in}}%
\pgfpathclose%
\pgfusepath{stroke,fill}%
\end{pgfscope}%
\begin{pgfscope}%
\pgfpathrectangle{\pgfqpoint{5.095420in}{4.306611in}}{\pgfqpoint{0.824468in}{0.462000in}}%
\pgfusepath{clip}%
\pgfsetbuttcap%
\pgfsetmiterjoin%
\definecolor{currentfill}{rgb}{0.121569,0.466667,0.705882}%
\pgfsetfillcolor{currentfill}%
\pgfsetfillopacity{0.500000}%
\pgfsetlinewidth{1.003750pt}%
\definecolor{currentstroke}{rgb}{0.000000,0.000000,0.000000}%
\pgfsetstrokecolor{currentstroke}%
\pgfsetdash{}{0pt}%
\pgfpathmoveto{\pgfqpoint{5.282799in}{4.306611in}}%
\pgfpathlineto{\pgfqpoint{5.432702in}{4.306611in}}%
\pgfpathlineto{\pgfqpoint{5.432702in}{4.405232in}}%
\pgfpathlineto{\pgfqpoint{5.282799in}{4.405232in}}%
\pgfpathlineto{\pgfqpoint{5.282799in}{4.306611in}}%
\pgfpathclose%
\pgfusepath{stroke,fill}%
\end{pgfscope}%
\begin{pgfscope}%
\pgfpathrectangle{\pgfqpoint{5.095420in}{4.306611in}}{\pgfqpoint{0.824468in}{0.462000in}}%
\pgfusepath{clip}%
\pgfsetbuttcap%
\pgfsetmiterjoin%
\definecolor{currentfill}{rgb}{0.121569,0.466667,0.705882}%
\pgfsetfillcolor{currentfill}%
\pgfsetfillopacity{0.500000}%
\pgfsetlinewidth{1.003750pt}%
\definecolor{currentstroke}{rgb}{0.000000,0.000000,0.000000}%
\pgfsetstrokecolor{currentstroke}%
\pgfsetdash{}{0pt}%
\pgfpathmoveto{\pgfqpoint{5.432702in}{4.306611in}}%
\pgfpathlineto{\pgfqpoint{5.582605in}{4.306611in}}%
\pgfpathlineto{\pgfqpoint{5.582605in}{4.359715in}}%
\pgfpathlineto{\pgfqpoint{5.432702in}{4.359715in}}%
\pgfpathlineto{\pgfqpoint{5.432702in}{4.306611in}}%
\pgfpathclose%
\pgfusepath{stroke,fill}%
\end{pgfscope}%
\begin{pgfscope}%
\pgfpathrectangle{\pgfqpoint{5.095420in}{4.306611in}}{\pgfqpoint{0.824468in}{0.462000in}}%
\pgfusepath{clip}%
\pgfsetbuttcap%
\pgfsetmiterjoin%
\definecolor{currentfill}{rgb}{0.121569,0.466667,0.705882}%
\pgfsetfillcolor{currentfill}%
\pgfsetfillopacity{0.500000}%
\pgfsetlinewidth{1.003750pt}%
\definecolor{currentstroke}{rgb}{0.000000,0.000000,0.000000}%
\pgfsetstrokecolor{currentstroke}%
\pgfsetdash{}{0pt}%
\pgfpathmoveto{\pgfqpoint{5.582605in}{4.306611in}}%
\pgfpathlineto{\pgfqpoint{5.732509in}{4.306611in}}%
\pgfpathlineto{\pgfqpoint{5.732509in}{4.390059in}}%
\pgfpathlineto{\pgfqpoint{5.582605in}{4.390059in}}%
\pgfpathlineto{\pgfqpoint{5.582605in}{4.306611in}}%
\pgfpathclose%
\pgfusepath{stroke,fill}%
\end{pgfscope}%
\begin{pgfscope}%
\pgfpathrectangle{\pgfqpoint{5.095420in}{4.306611in}}{\pgfqpoint{0.824468in}{0.462000in}}%
\pgfusepath{clip}%
\pgfsetbuttcap%
\pgfsetmiterjoin%
\definecolor{currentfill}{rgb}{0.121569,0.466667,0.705882}%
\pgfsetfillcolor{currentfill}%
\pgfsetfillopacity{0.500000}%
\pgfsetlinewidth{1.003750pt}%
\definecolor{currentstroke}{rgb}{0.000000,0.000000,0.000000}%
\pgfsetstrokecolor{currentstroke}%
\pgfsetdash{}{0pt}%
\pgfpathmoveto{\pgfqpoint{5.732509in}{4.306611in}}%
\pgfpathlineto{\pgfqpoint{5.882412in}{4.306611in}}%
\pgfpathlineto{\pgfqpoint{5.882412in}{4.359715in}}%
\pgfpathlineto{\pgfqpoint{5.732509in}{4.359715in}}%
\pgfpathlineto{\pgfqpoint{5.732509in}{4.306611in}}%
\pgfpathclose%
\pgfusepath{stroke,fill}%
\end{pgfscope}%
\begin{pgfscope}%
\pgfsetrectcap%
\pgfsetmiterjoin%
\pgfsetlinewidth{0.803000pt}%
\definecolor{currentstroke}{rgb}{0.000000,0.000000,0.000000}%
\pgfsetstrokecolor{currentstroke}%
\pgfsetdash{}{0pt}%
\pgfpathmoveto{\pgfqpoint{5.095420in}{4.306611in}}%
\pgfpathlineto{\pgfqpoint{5.095420in}{4.768611in}}%
\pgfusepath{stroke}%
\end{pgfscope}%
\begin{pgfscope}%
\pgfsetrectcap%
\pgfsetmiterjoin%
\pgfsetlinewidth{0.803000pt}%
\definecolor{currentstroke}{rgb}{0.000000,0.000000,0.000000}%
\pgfsetstrokecolor{currentstroke}%
\pgfsetdash{}{0pt}%
\pgfpathmoveto{\pgfqpoint{5.919888in}{4.306611in}}%
\pgfpathlineto{\pgfqpoint{5.919888in}{4.768611in}}%
\pgfusepath{stroke}%
\end{pgfscope}%
\begin{pgfscope}%
\pgfsetrectcap%
\pgfsetmiterjoin%
\pgfsetlinewidth{0.803000pt}%
\definecolor{currentstroke}{rgb}{0.000000,0.000000,0.000000}%
\pgfsetstrokecolor{currentstroke}%
\pgfsetdash{}{0pt}%
\pgfpathmoveto{\pgfqpoint{5.095420in}{4.306611in}}%
\pgfpathlineto{\pgfqpoint{5.919888in}{4.306611in}}%
\pgfusepath{stroke}%
\end{pgfscope}%
\begin{pgfscope}%
\pgfsetrectcap%
\pgfsetmiterjoin%
\pgfsetlinewidth{0.803000pt}%
\definecolor{currentstroke}{rgb}{0.000000,0.000000,0.000000}%
\pgfsetstrokecolor{currentstroke}%
\pgfsetdash{}{0pt}%
\pgfpathmoveto{\pgfqpoint{5.095420in}{4.768611in}}%
\pgfpathlineto{\pgfqpoint{5.919888in}{4.768611in}}%
\pgfusepath{stroke}%
\end{pgfscope}%
\begin{pgfscope}%
\definecolor{textcolor}{rgb}{0.000000,0.000000,0.000000}%
\pgfsetstrokecolor{textcolor}%
\pgfsetfillcolor{textcolor}%
\pgftext[x=5.507654in,y=4.851944in,,base]{\color{textcolor}\rmfamily\fontsize{11.000000}{13.200000}\selectfont SantéVet}%
\end{pgfscope}%
\begin{pgfscope}%
\pgfsetbuttcap%
\pgfsetmiterjoin%
\definecolor{currentfill}{rgb}{1.000000,1.000000,1.000000}%
\pgfsetfillcolor{currentfill}%
\pgfsetlinewidth{0.000000pt}%
\definecolor{currentstroke}{rgb}{0.000000,0.000000,0.000000}%
\pgfsetstrokecolor{currentstroke}%
\pgfsetstrokeopacity{0.000000}%
\pgfsetdash{}{0pt}%
\pgfpathmoveto{\pgfqpoint{6.084781in}{4.306611in}}%
\pgfpathlineto{\pgfqpoint{6.909249in}{4.306611in}}%
\pgfpathlineto{\pgfqpoint{6.909249in}{4.768611in}}%
\pgfpathlineto{\pgfqpoint{6.084781in}{4.768611in}}%
\pgfpathlineto{\pgfqpoint{6.084781in}{4.306611in}}%
\pgfpathclose%
\pgfusepath{fill}%
\end{pgfscope}%
\begin{pgfscope}%
\pgfpathrectangle{\pgfqpoint{6.084781in}{4.306611in}}{\pgfqpoint{0.824468in}{0.462000in}}%
\pgfusepath{clip}%
\pgfsetbuttcap%
\pgfsetmiterjoin%
\definecolor{currentfill}{rgb}{0.121569,0.466667,0.705882}%
\pgfsetfillcolor{currentfill}%
\pgfsetfillopacity{0.500000}%
\pgfsetlinewidth{1.003750pt}%
\definecolor{currentstroke}{rgb}{0.000000,0.000000,0.000000}%
\pgfsetstrokecolor{currentstroke}%
\pgfsetdash{}{0pt}%
\pgfpathmoveto{\pgfqpoint{6.122257in}{4.306611in}}%
\pgfpathlineto{\pgfqpoint{6.272160in}{4.306611in}}%
\pgfpathlineto{\pgfqpoint{6.272160in}{4.746611in}}%
\pgfpathlineto{\pgfqpoint{6.122257in}{4.746611in}}%
\pgfpathlineto{\pgfqpoint{6.122257in}{4.306611in}}%
\pgfpathclose%
\pgfusepath{stroke,fill}%
\end{pgfscope}%
\begin{pgfscope}%
\pgfpathrectangle{\pgfqpoint{6.084781in}{4.306611in}}{\pgfqpoint{0.824468in}{0.462000in}}%
\pgfusepath{clip}%
\pgfsetbuttcap%
\pgfsetmiterjoin%
\definecolor{currentfill}{rgb}{0.121569,0.466667,0.705882}%
\pgfsetfillcolor{currentfill}%
\pgfsetfillopacity{0.500000}%
\pgfsetlinewidth{1.003750pt}%
\definecolor{currentstroke}{rgb}{0.000000,0.000000,0.000000}%
\pgfsetstrokecolor{currentstroke}%
\pgfsetdash{}{0pt}%
\pgfpathmoveto{\pgfqpoint{6.272160in}{4.306611in}}%
\pgfpathlineto{\pgfqpoint{6.422064in}{4.306611in}}%
\pgfpathlineto{\pgfqpoint{6.422064in}{4.404389in}}%
\pgfpathlineto{\pgfqpoint{6.272160in}{4.404389in}}%
\pgfpathlineto{\pgfqpoint{6.272160in}{4.306611in}}%
\pgfpathclose%
\pgfusepath{stroke,fill}%
\end{pgfscope}%
\begin{pgfscope}%
\pgfpathrectangle{\pgfqpoint{6.084781in}{4.306611in}}{\pgfqpoint{0.824468in}{0.462000in}}%
\pgfusepath{clip}%
\pgfsetbuttcap%
\pgfsetmiterjoin%
\definecolor{currentfill}{rgb}{0.121569,0.466667,0.705882}%
\pgfsetfillcolor{currentfill}%
\pgfsetfillopacity{0.500000}%
\pgfsetlinewidth{1.003750pt}%
\definecolor{currentstroke}{rgb}{0.000000,0.000000,0.000000}%
\pgfsetstrokecolor{currentstroke}%
\pgfsetdash{}{0pt}%
\pgfpathmoveto{\pgfqpoint{6.422064in}{4.306611in}}%
\pgfpathlineto{\pgfqpoint{6.571967in}{4.306611in}}%
\pgfpathlineto{\pgfqpoint{6.571967in}{4.327867in}}%
\pgfpathlineto{\pgfqpoint{6.422064in}{4.327867in}}%
\pgfpathlineto{\pgfqpoint{6.422064in}{4.306611in}}%
\pgfpathclose%
\pgfusepath{stroke,fill}%
\end{pgfscope}%
\begin{pgfscope}%
\pgfpathrectangle{\pgfqpoint{6.084781in}{4.306611in}}{\pgfqpoint{0.824468in}{0.462000in}}%
\pgfusepath{clip}%
\pgfsetbuttcap%
\pgfsetmiterjoin%
\definecolor{currentfill}{rgb}{0.121569,0.466667,0.705882}%
\pgfsetfillcolor{currentfill}%
\pgfsetfillopacity{0.500000}%
\pgfsetlinewidth{1.003750pt}%
\definecolor{currentstroke}{rgb}{0.000000,0.000000,0.000000}%
\pgfsetstrokecolor{currentstroke}%
\pgfsetdash{}{0pt}%
\pgfpathmoveto{\pgfqpoint{6.571967in}{4.306611in}}%
\pgfpathlineto{\pgfqpoint{6.721870in}{4.306611in}}%
\pgfpathlineto{\pgfqpoint{6.721870in}{4.312988in}}%
\pgfpathlineto{\pgfqpoint{6.571967in}{4.312988in}}%
\pgfpathlineto{\pgfqpoint{6.571967in}{4.306611in}}%
\pgfpathclose%
\pgfusepath{stroke,fill}%
\end{pgfscope}%
\begin{pgfscope}%
\pgfpathrectangle{\pgfqpoint{6.084781in}{4.306611in}}{\pgfqpoint{0.824468in}{0.462000in}}%
\pgfusepath{clip}%
\pgfsetbuttcap%
\pgfsetmiterjoin%
\definecolor{currentfill}{rgb}{0.121569,0.466667,0.705882}%
\pgfsetfillcolor{currentfill}%
\pgfsetfillopacity{0.500000}%
\pgfsetlinewidth{1.003750pt}%
\definecolor{currentstroke}{rgb}{0.000000,0.000000,0.000000}%
\pgfsetstrokecolor{currentstroke}%
\pgfsetdash{}{0pt}%
\pgfpathmoveto{\pgfqpoint{6.721870in}{4.306611in}}%
\pgfpathlineto{\pgfqpoint{6.871774in}{4.306611in}}%
\pgfpathlineto{\pgfqpoint{6.871774in}{4.308737in}}%
\pgfpathlineto{\pgfqpoint{6.721870in}{4.308737in}}%
\pgfpathlineto{\pgfqpoint{6.721870in}{4.306611in}}%
\pgfpathclose%
\pgfusepath{stroke,fill}%
\end{pgfscope}%
\begin{pgfscope}%
\pgfsetrectcap%
\pgfsetmiterjoin%
\pgfsetlinewidth{0.803000pt}%
\definecolor{currentstroke}{rgb}{0.000000,0.000000,0.000000}%
\pgfsetstrokecolor{currentstroke}%
\pgfsetdash{}{0pt}%
\pgfpathmoveto{\pgfqpoint{6.084781in}{4.306611in}}%
\pgfpathlineto{\pgfqpoint{6.084781in}{4.768611in}}%
\pgfusepath{stroke}%
\end{pgfscope}%
\begin{pgfscope}%
\pgfsetrectcap%
\pgfsetmiterjoin%
\pgfsetlinewidth{0.803000pt}%
\definecolor{currentstroke}{rgb}{0.000000,0.000000,0.000000}%
\pgfsetstrokecolor{currentstroke}%
\pgfsetdash{}{0pt}%
\pgfpathmoveto{\pgfqpoint{6.909249in}{4.306611in}}%
\pgfpathlineto{\pgfqpoint{6.909249in}{4.768611in}}%
\pgfusepath{stroke}%
\end{pgfscope}%
\begin{pgfscope}%
\pgfsetrectcap%
\pgfsetmiterjoin%
\pgfsetlinewidth{0.803000pt}%
\definecolor{currentstroke}{rgb}{0.000000,0.000000,0.000000}%
\pgfsetstrokecolor{currentstroke}%
\pgfsetdash{}{0pt}%
\pgfpathmoveto{\pgfqpoint{6.084781in}{4.306611in}}%
\pgfpathlineto{\pgfqpoint{6.909249in}{4.306611in}}%
\pgfusepath{stroke}%
\end{pgfscope}%
\begin{pgfscope}%
\pgfsetrectcap%
\pgfsetmiterjoin%
\pgfsetlinewidth{0.803000pt}%
\definecolor{currentstroke}{rgb}{0.000000,0.000000,0.000000}%
\pgfsetstrokecolor{currentstroke}%
\pgfsetdash{}{0pt}%
\pgfpathmoveto{\pgfqpoint{6.084781in}{4.768611in}}%
\pgfpathlineto{\pgfqpoint{6.909249in}{4.768611in}}%
\pgfusepath{stroke}%
\end{pgfscope}%
\begin{pgfscope}%
\definecolor{textcolor}{rgb}{0.000000,0.000000,0.000000}%
\pgfsetstrokecolor{textcolor}%
\pgfsetfillcolor{textcolor}%
\pgftext[x=6.497015in,y=4.851944in,,base]{\color{textcolor}\rmfamily\fontsize{11.000000}{13.200000}\selectfont Mercer}%
\end{pgfscope}%
\begin{pgfscope}%
\pgfsetbuttcap%
\pgfsetmiterjoin%
\definecolor{currentfill}{rgb}{1.000000,1.000000,1.000000}%
\pgfsetfillcolor{currentfill}%
\pgfsetlinewidth{0.000000pt}%
\definecolor{currentstroke}{rgb}{0.000000,0.000000,0.000000}%
\pgfsetstrokecolor{currentstroke}%
\pgfsetstrokeopacity{0.000000}%
\pgfsetdash{}{0pt}%
\pgfpathmoveto{\pgfqpoint{7.074143in}{4.306611in}}%
\pgfpathlineto{\pgfqpoint{7.898611in}{4.306611in}}%
\pgfpathlineto{\pgfqpoint{7.898611in}{4.768611in}}%
\pgfpathlineto{\pgfqpoint{7.074143in}{4.768611in}}%
\pgfpathlineto{\pgfqpoint{7.074143in}{4.306611in}}%
\pgfpathclose%
\pgfusepath{fill}%
\end{pgfscope}%
\begin{pgfscope}%
\pgfpathrectangle{\pgfqpoint{7.074143in}{4.306611in}}{\pgfqpoint{0.824468in}{0.462000in}}%
\pgfusepath{clip}%
\pgfsetbuttcap%
\pgfsetmiterjoin%
\definecolor{currentfill}{rgb}{0.121569,0.466667,0.705882}%
\pgfsetfillcolor{currentfill}%
\pgfsetfillopacity{0.500000}%
\pgfsetlinewidth{1.003750pt}%
\definecolor{currentstroke}{rgb}{0.000000,0.000000,0.000000}%
\pgfsetstrokecolor{currentstroke}%
\pgfsetdash{}{0pt}%
\pgfpathmoveto{\pgfqpoint{7.111619in}{4.306611in}}%
\pgfpathlineto{\pgfqpoint{7.261522in}{4.306611in}}%
\pgfpathlineto{\pgfqpoint{7.261522in}{4.746611in}}%
\pgfpathlineto{\pgfqpoint{7.111619in}{4.746611in}}%
\pgfpathlineto{\pgfqpoint{7.111619in}{4.306611in}}%
\pgfpathclose%
\pgfusepath{stroke,fill}%
\end{pgfscope}%
\begin{pgfscope}%
\pgfpathrectangle{\pgfqpoint{7.074143in}{4.306611in}}{\pgfqpoint{0.824468in}{0.462000in}}%
\pgfusepath{clip}%
\pgfsetbuttcap%
\pgfsetmiterjoin%
\definecolor{currentfill}{rgb}{0.121569,0.466667,0.705882}%
\pgfsetfillcolor{currentfill}%
\pgfsetfillopacity{0.500000}%
\pgfsetlinewidth{1.003750pt}%
\definecolor{currentstroke}{rgb}{0.000000,0.000000,0.000000}%
\pgfsetstrokecolor{currentstroke}%
\pgfsetdash{}{0pt}%
\pgfpathmoveto{\pgfqpoint{7.261522in}{4.306611in}}%
\pgfpathlineto{\pgfqpoint{7.411425in}{4.306611in}}%
\pgfpathlineto{\pgfqpoint{7.411425in}{4.374303in}}%
\pgfpathlineto{\pgfqpoint{7.261522in}{4.374303in}}%
\pgfpathlineto{\pgfqpoint{7.261522in}{4.306611in}}%
\pgfpathclose%
\pgfusepath{stroke,fill}%
\end{pgfscope}%
\begin{pgfscope}%
\pgfpathrectangle{\pgfqpoint{7.074143in}{4.306611in}}{\pgfqpoint{0.824468in}{0.462000in}}%
\pgfusepath{clip}%
\pgfsetbuttcap%
\pgfsetmiterjoin%
\definecolor{currentfill}{rgb}{0.121569,0.466667,0.705882}%
\pgfsetfillcolor{currentfill}%
\pgfsetfillopacity{0.500000}%
\pgfsetlinewidth{1.003750pt}%
\definecolor{currentstroke}{rgb}{0.000000,0.000000,0.000000}%
\pgfsetstrokecolor{currentstroke}%
\pgfsetdash{}{0pt}%
\pgfpathmoveto{\pgfqpoint{7.411425in}{4.306611in}}%
\pgfpathlineto{\pgfqpoint{7.561329in}{4.306611in}}%
\pgfpathlineto{\pgfqpoint{7.561329in}{4.393644in}}%
\pgfpathlineto{\pgfqpoint{7.411425in}{4.393644in}}%
\pgfpathlineto{\pgfqpoint{7.411425in}{4.306611in}}%
\pgfpathclose%
\pgfusepath{stroke,fill}%
\end{pgfscope}%
\begin{pgfscope}%
\pgfpathrectangle{\pgfqpoint{7.074143in}{4.306611in}}{\pgfqpoint{0.824468in}{0.462000in}}%
\pgfusepath{clip}%
\pgfsetbuttcap%
\pgfsetmiterjoin%
\definecolor{currentfill}{rgb}{0.121569,0.466667,0.705882}%
\pgfsetfillcolor{currentfill}%
\pgfsetfillopacity{0.500000}%
\pgfsetlinewidth{1.003750pt}%
\definecolor{currentstroke}{rgb}{0.000000,0.000000,0.000000}%
\pgfsetstrokecolor{currentstroke}%
\pgfsetdash{}{0pt}%
\pgfpathmoveto{\pgfqpoint{7.561329in}{4.306611in}}%
\pgfpathlineto{\pgfqpoint{7.711232in}{4.306611in}}%
\pgfpathlineto{\pgfqpoint{7.711232in}{4.311446in}}%
\pgfpathlineto{\pgfqpoint{7.561329in}{4.311446in}}%
\pgfpathlineto{\pgfqpoint{7.561329in}{4.306611in}}%
\pgfpathclose%
\pgfusepath{stroke,fill}%
\end{pgfscope}%
\begin{pgfscope}%
\pgfpathrectangle{\pgfqpoint{7.074143in}{4.306611in}}{\pgfqpoint{0.824468in}{0.462000in}}%
\pgfusepath{clip}%
\pgfsetbuttcap%
\pgfsetmiterjoin%
\definecolor{currentfill}{rgb}{0.121569,0.466667,0.705882}%
\pgfsetfillcolor{currentfill}%
\pgfsetfillopacity{0.500000}%
\pgfsetlinewidth{1.003750pt}%
\definecolor{currentstroke}{rgb}{0.000000,0.000000,0.000000}%
\pgfsetstrokecolor{currentstroke}%
\pgfsetdash{}{0pt}%
\pgfpathmoveto{\pgfqpoint{7.711232in}{4.306611in}}%
\pgfpathlineto{\pgfqpoint{7.861135in}{4.306611in}}%
\pgfpathlineto{\pgfqpoint{7.861135in}{4.325952in}}%
\pgfpathlineto{\pgfqpoint{7.711232in}{4.325952in}}%
\pgfpathlineto{\pgfqpoint{7.711232in}{4.306611in}}%
\pgfpathclose%
\pgfusepath{stroke,fill}%
\end{pgfscope}%
\begin{pgfscope}%
\pgfsetrectcap%
\pgfsetmiterjoin%
\pgfsetlinewidth{0.803000pt}%
\definecolor{currentstroke}{rgb}{0.000000,0.000000,0.000000}%
\pgfsetstrokecolor{currentstroke}%
\pgfsetdash{}{0pt}%
\pgfpathmoveto{\pgfqpoint{7.074143in}{4.306611in}}%
\pgfpathlineto{\pgfqpoint{7.074143in}{4.768611in}}%
\pgfusepath{stroke}%
\end{pgfscope}%
\begin{pgfscope}%
\pgfsetrectcap%
\pgfsetmiterjoin%
\pgfsetlinewidth{0.803000pt}%
\definecolor{currentstroke}{rgb}{0.000000,0.000000,0.000000}%
\pgfsetstrokecolor{currentstroke}%
\pgfsetdash{}{0pt}%
\pgfpathmoveto{\pgfqpoint{7.898611in}{4.306611in}}%
\pgfpathlineto{\pgfqpoint{7.898611in}{4.768611in}}%
\pgfusepath{stroke}%
\end{pgfscope}%
\begin{pgfscope}%
\pgfsetrectcap%
\pgfsetmiterjoin%
\pgfsetlinewidth{0.803000pt}%
\definecolor{currentstroke}{rgb}{0.000000,0.000000,0.000000}%
\pgfsetstrokecolor{currentstroke}%
\pgfsetdash{}{0pt}%
\pgfpathmoveto{\pgfqpoint{7.074143in}{4.306611in}}%
\pgfpathlineto{\pgfqpoint{7.898611in}{4.306611in}}%
\pgfusepath{stroke}%
\end{pgfscope}%
\begin{pgfscope}%
\pgfsetrectcap%
\pgfsetmiterjoin%
\pgfsetlinewidth{0.803000pt}%
\definecolor{currentstroke}{rgb}{0.000000,0.000000,0.000000}%
\pgfsetstrokecolor{currentstroke}%
\pgfsetdash{}{0pt}%
\pgfpathmoveto{\pgfqpoint{7.074143in}{4.768611in}}%
\pgfpathlineto{\pgfqpoint{7.898611in}{4.768611in}}%
\pgfusepath{stroke}%
\end{pgfscope}%
\begin{pgfscope}%
\definecolor{textcolor}{rgb}{0.000000,0.000000,0.000000}%
\pgfsetstrokecolor{textcolor}%
\pgfsetfillcolor{textcolor}%
\pgftext[x=7.486377in,y=4.851944in,,base]{\color{textcolor}\rmfamily\fontsize{11.000000}{13.200000}\selectfont Generali}%
\end{pgfscope}%
\begin{pgfscope}%
\pgfsetbuttcap%
\pgfsetmiterjoin%
\definecolor{currentfill}{rgb}{1.000000,1.000000,1.000000}%
\pgfsetfillcolor{currentfill}%
\pgfsetlinewidth{0.000000pt}%
\definecolor{currentstroke}{rgb}{0.000000,0.000000,0.000000}%
\pgfsetstrokecolor{currentstroke}%
\pgfsetstrokeopacity{0.000000}%
\pgfsetdash{}{0pt}%
\pgfpathmoveto{\pgfqpoint{0.148611in}{3.613611in}}%
\pgfpathlineto{\pgfqpoint{0.973079in}{3.613611in}}%
\pgfpathlineto{\pgfqpoint{0.973079in}{4.075611in}}%
\pgfpathlineto{\pgfqpoint{0.148611in}{4.075611in}}%
\pgfpathlineto{\pgfqpoint{0.148611in}{3.613611in}}%
\pgfpathclose%
\pgfusepath{fill}%
\end{pgfscope}%
\begin{pgfscope}%
\pgfpathrectangle{\pgfqpoint{0.148611in}{3.613611in}}{\pgfqpoint{0.824468in}{0.462000in}}%
\pgfusepath{clip}%
\pgfsetbuttcap%
\pgfsetmiterjoin%
\definecolor{currentfill}{rgb}{0.121569,0.466667,0.705882}%
\pgfsetfillcolor{currentfill}%
\pgfsetfillopacity{0.500000}%
\pgfsetlinewidth{1.003750pt}%
\definecolor{currentstroke}{rgb}{0.000000,0.000000,0.000000}%
\pgfsetstrokecolor{currentstroke}%
\pgfsetdash{}{0pt}%
\pgfpathmoveto{\pgfqpoint{0.186087in}{3.613611in}}%
\pgfpathlineto{\pgfqpoint{0.335990in}{3.613611in}}%
\pgfpathlineto{\pgfqpoint{0.335990in}{4.053611in}}%
\pgfpathlineto{\pgfqpoint{0.186087in}{4.053611in}}%
\pgfpathlineto{\pgfqpoint{0.186087in}{3.613611in}}%
\pgfpathclose%
\pgfusepath{stroke,fill}%
\end{pgfscope}%
\begin{pgfscope}%
\pgfpathrectangle{\pgfqpoint{0.148611in}{3.613611in}}{\pgfqpoint{0.824468in}{0.462000in}}%
\pgfusepath{clip}%
\pgfsetbuttcap%
\pgfsetmiterjoin%
\definecolor{currentfill}{rgb}{0.121569,0.466667,0.705882}%
\pgfsetfillcolor{currentfill}%
\pgfsetfillopacity{0.500000}%
\pgfsetlinewidth{1.003750pt}%
\definecolor{currentstroke}{rgb}{0.000000,0.000000,0.000000}%
\pgfsetstrokecolor{currentstroke}%
\pgfsetdash{}{0pt}%
\pgfpathmoveto{\pgfqpoint{0.335990in}{3.613611in}}%
\pgfpathlineto{\pgfqpoint{0.485894in}{3.613611in}}%
\pgfpathlineto{\pgfqpoint{0.485894in}{3.742280in}}%
\pgfpathlineto{\pgfqpoint{0.335990in}{3.742280in}}%
\pgfpathlineto{\pgfqpoint{0.335990in}{3.613611in}}%
\pgfpathclose%
\pgfusepath{stroke,fill}%
\end{pgfscope}%
\begin{pgfscope}%
\pgfpathrectangle{\pgfqpoint{0.148611in}{3.613611in}}{\pgfqpoint{0.824468in}{0.462000in}}%
\pgfusepath{clip}%
\pgfsetbuttcap%
\pgfsetmiterjoin%
\definecolor{currentfill}{rgb}{0.121569,0.466667,0.705882}%
\pgfsetfillcolor{currentfill}%
\pgfsetfillopacity{0.500000}%
\pgfsetlinewidth{1.003750pt}%
\definecolor{currentstroke}{rgb}{0.000000,0.000000,0.000000}%
\pgfsetstrokecolor{currentstroke}%
\pgfsetdash{}{0pt}%
\pgfpathmoveto{\pgfqpoint{0.485894in}{3.613611in}}%
\pgfpathlineto{\pgfqpoint{0.635797in}{3.613611in}}%
\pgfpathlineto{\pgfqpoint{0.635797in}{3.672201in}}%
\pgfpathlineto{\pgfqpoint{0.485894in}{3.672201in}}%
\pgfpathlineto{\pgfqpoint{0.485894in}{3.613611in}}%
\pgfpathclose%
\pgfusepath{stroke,fill}%
\end{pgfscope}%
\begin{pgfscope}%
\pgfpathrectangle{\pgfqpoint{0.148611in}{3.613611in}}{\pgfqpoint{0.824468in}{0.462000in}}%
\pgfusepath{clip}%
\pgfsetbuttcap%
\pgfsetmiterjoin%
\definecolor{currentfill}{rgb}{0.121569,0.466667,0.705882}%
\pgfsetfillcolor{currentfill}%
\pgfsetfillopacity{0.500000}%
\pgfsetlinewidth{1.003750pt}%
\definecolor{currentstroke}{rgb}{0.000000,0.000000,0.000000}%
\pgfsetstrokecolor{currentstroke}%
\pgfsetdash{}{0pt}%
\pgfpathmoveto{\pgfqpoint{0.635797in}{3.613611in}}%
\pgfpathlineto{\pgfqpoint{0.785700in}{3.613611in}}%
\pgfpathlineto{\pgfqpoint{0.785700in}{3.634290in}}%
\pgfpathlineto{\pgfqpoint{0.635797in}{3.634290in}}%
\pgfpathlineto{\pgfqpoint{0.635797in}{3.613611in}}%
\pgfpathclose%
\pgfusepath{stroke,fill}%
\end{pgfscope}%
\begin{pgfscope}%
\pgfpathrectangle{\pgfqpoint{0.148611in}{3.613611in}}{\pgfqpoint{0.824468in}{0.462000in}}%
\pgfusepath{clip}%
\pgfsetbuttcap%
\pgfsetmiterjoin%
\definecolor{currentfill}{rgb}{0.121569,0.466667,0.705882}%
\pgfsetfillcolor{currentfill}%
\pgfsetfillopacity{0.500000}%
\pgfsetlinewidth{1.003750pt}%
\definecolor{currentstroke}{rgb}{0.000000,0.000000,0.000000}%
\pgfsetstrokecolor{currentstroke}%
\pgfsetdash{}{0pt}%
\pgfpathmoveto{\pgfqpoint{0.785700in}{3.613611in}}%
\pgfpathlineto{\pgfqpoint{0.935603in}{3.613611in}}%
\pgfpathlineto{\pgfqpoint{0.935603in}{3.625099in}}%
\pgfpathlineto{\pgfqpoint{0.785700in}{3.625099in}}%
\pgfpathlineto{\pgfqpoint{0.785700in}{3.613611in}}%
\pgfpathclose%
\pgfusepath{stroke,fill}%
\end{pgfscope}%
\begin{pgfscope}%
\pgfsetrectcap%
\pgfsetmiterjoin%
\pgfsetlinewidth{0.803000pt}%
\definecolor{currentstroke}{rgb}{0.000000,0.000000,0.000000}%
\pgfsetstrokecolor{currentstroke}%
\pgfsetdash{}{0pt}%
\pgfpathmoveto{\pgfqpoint{0.148611in}{3.613611in}}%
\pgfpathlineto{\pgfqpoint{0.148611in}{4.075611in}}%
\pgfusepath{stroke}%
\end{pgfscope}%
\begin{pgfscope}%
\pgfsetrectcap%
\pgfsetmiterjoin%
\pgfsetlinewidth{0.803000pt}%
\definecolor{currentstroke}{rgb}{0.000000,0.000000,0.000000}%
\pgfsetstrokecolor{currentstroke}%
\pgfsetdash{}{0pt}%
\pgfpathmoveto{\pgfqpoint{0.973079in}{3.613611in}}%
\pgfpathlineto{\pgfqpoint{0.973079in}{4.075611in}}%
\pgfusepath{stroke}%
\end{pgfscope}%
\begin{pgfscope}%
\pgfsetrectcap%
\pgfsetmiterjoin%
\pgfsetlinewidth{0.803000pt}%
\definecolor{currentstroke}{rgb}{0.000000,0.000000,0.000000}%
\pgfsetstrokecolor{currentstroke}%
\pgfsetdash{}{0pt}%
\pgfpathmoveto{\pgfqpoint{0.148611in}{3.613611in}}%
\pgfpathlineto{\pgfqpoint{0.973079in}{3.613611in}}%
\pgfusepath{stroke}%
\end{pgfscope}%
\begin{pgfscope}%
\pgfsetrectcap%
\pgfsetmiterjoin%
\pgfsetlinewidth{0.803000pt}%
\definecolor{currentstroke}{rgb}{0.000000,0.000000,0.000000}%
\pgfsetstrokecolor{currentstroke}%
\pgfsetdash{}{0pt}%
\pgfpathmoveto{\pgfqpoint{0.148611in}{4.075611in}}%
\pgfpathlineto{\pgfqpoint{0.973079in}{4.075611in}}%
\pgfusepath{stroke}%
\end{pgfscope}%
\begin{pgfscope}%
\definecolor{textcolor}{rgb}{0.000000,0.000000,0.000000}%
\pgfsetstrokecolor{textcolor}%
\pgfsetfillcolor{textcolor}%
\pgftext[x=0.560845in,y=4.158944in,,base]{\color{textcolor}\rmfamily\fontsize{11.000000}{13.200000}\selectfont Allianz}%
\end{pgfscope}%
\begin{pgfscope}%
\pgfsetbuttcap%
\pgfsetmiterjoin%
\definecolor{currentfill}{rgb}{1.000000,1.000000,1.000000}%
\pgfsetfillcolor{currentfill}%
\pgfsetlinewidth{0.000000pt}%
\definecolor{currentstroke}{rgb}{0.000000,0.000000,0.000000}%
\pgfsetstrokecolor{currentstroke}%
\pgfsetstrokeopacity{0.000000}%
\pgfsetdash{}{0pt}%
\pgfpathmoveto{\pgfqpoint{1.137973in}{3.613611in}}%
\pgfpathlineto{\pgfqpoint{1.962441in}{3.613611in}}%
\pgfpathlineto{\pgfqpoint{1.962441in}{4.075611in}}%
\pgfpathlineto{\pgfqpoint{1.137973in}{4.075611in}}%
\pgfpathlineto{\pgfqpoint{1.137973in}{3.613611in}}%
\pgfpathclose%
\pgfusepath{fill}%
\end{pgfscope}%
\begin{pgfscope}%
\pgfpathrectangle{\pgfqpoint{1.137973in}{3.613611in}}{\pgfqpoint{0.824468in}{0.462000in}}%
\pgfusepath{clip}%
\pgfsetbuttcap%
\pgfsetmiterjoin%
\definecolor{currentfill}{rgb}{0.121569,0.466667,0.705882}%
\pgfsetfillcolor{currentfill}%
\pgfsetfillopacity{0.500000}%
\pgfsetlinewidth{1.003750pt}%
\definecolor{currentstroke}{rgb}{0.000000,0.000000,0.000000}%
\pgfsetstrokecolor{currentstroke}%
\pgfsetdash{}{0pt}%
\pgfpathmoveto{\pgfqpoint{1.175449in}{3.613611in}}%
\pgfpathlineto{\pgfqpoint{1.325352in}{3.613611in}}%
\pgfpathlineto{\pgfqpoint{1.325352in}{3.674451in}}%
\pgfpathlineto{\pgfqpoint{1.175449in}{3.674451in}}%
\pgfpathlineto{\pgfqpoint{1.175449in}{3.613611in}}%
\pgfpathclose%
\pgfusepath{stroke,fill}%
\end{pgfscope}%
\begin{pgfscope}%
\pgfpathrectangle{\pgfqpoint{1.137973in}{3.613611in}}{\pgfqpoint{0.824468in}{0.462000in}}%
\pgfusepath{clip}%
\pgfsetbuttcap%
\pgfsetmiterjoin%
\definecolor{currentfill}{rgb}{0.121569,0.466667,0.705882}%
\pgfsetfillcolor{currentfill}%
\pgfsetfillopacity{0.500000}%
\pgfsetlinewidth{1.003750pt}%
\definecolor{currentstroke}{rgb}{0.000000,0.000000,0.000000}%
\pgfsetstrokecolor{currentstroke}%
\pgfsetdash{}{0pt}%
\pgfpathmoveto{\pgfqpoint{1.325352in}{3.613611in}}%
\pgfpathlineto{\pgfqpoint{1.475255in}{3.613611in}}%
\pgfpathlineto{\pgfqpoint{1.475255in}{3.689660in}}%
\pgfpathlineto{\pgfqpoint{1.325352in}{3.689660in}}%
\pgfpathlineto{\pgfqpoint{1.325352in}{3.613611in}}%
\pgfpathclose%
\pgfusepath{stroke,fill}%
\end{pgfscope}%
\begin{pgfscope}%
\pgfpathrectangle{\pgfqpoint{1.137973in}{3.613611in}}{\pgfqpoint{0.824468in}{0.462000in}}%
\pgfusepath{clip}%
\pgfsetbuttcap%
\pgfsetmiterjoin%
\definecolor{currentfill}{rgb}{0.121569,0.466667,0.705882}%
\pgfsetfillcolor{currentfill}%
\pgfsetfillopacity{0.500000}%
\pgfsetlinewidth{1.003750pt}%
\definecolor{currentstroke}{rgb}{0.000000,0.000000,0.000000}%
\pgfsetstrokecolor{currentstroke}%
\pgfsetdash{}{0pt}%
\pgfpathmoveto{\pgfqpoint{1.475255in}{3.613611in}}%
\pgfpathlineto{\pgfqpoint{1.625158in}{3.613611in}}%
\pgfpathlineto{\pgfqpoint{1.625158in}{3.778747in}}%
\pgfpathlineto{\pgfqpoint{1.475255in}{3.778747in}}%
\pgfpathlineto{\pgfqpoint{1.475255in}{3.613611in}}%
\pgfpathclose%
\pgfusepath{stroke,fill}%
\end{pgfscope}%
\begin{pgfscope}%
\pgfpathrectangle{\pgfqpoint{1.137973in}{3.613611in}}{\pgfqpoint{0.824468in}{0.462000in}}%
\pgfusepath{clip}%
\pgfsetbuttcap%
\pgfsetmiterjoin%
\definecolor{currentfill}{rgb}{0.121569,0.466667,0.705882}%
\pgfsetfillcolor{currentfill}%
\pgfsetfillopacity{0.500000}%
\pgfsetlinewidth{1.003750pt}%
\definecolor{currentstroke}{rgb}{0.000000,0.000000,0.000000}%
\pgfsetstrokecolor{currentstroke}%
\pgfsetdash{}{0pt}%
\pgfpathmoveto{\pgfqpoint{1.625158in}{3.613611in}}%
\pgfpathlineto{\pgfqpoint{1.775062in}{3.613611in}}%
\pgfpathlineto{\pgfqpoint{1.775062in}{3.982994in}}%
\pgfpathlineto{\pgfqpoint{1.625158in}{3.982994in}}%
\pgfpathlineto{\pgfqpoint{1.625158in}{3.613611in}}%
\pgfpathclose%
\pgfusepath{stroke,fill}%
\end{pgfscope}%
\begin{pgfscope}%
\pgfpathrectangle{\pgfqpoint{1.137973in}{3.613611in}}{\pgfqpoint{0.824468in}{0.462000in}}%
\pgfusepath{clip}%
\pgfsetbuttcap%
\pgfsetmiterjoin%
\definecolor{currentfill}{rgb}{0.121569,0.466667,0.705882}%
\pgfsetfillcolor{currentfill}%
\pgfsetfillopacity{0.500000}%
\pgfsetlinewidth{1.003750pt}%
\definecolor{currentstroke}{rgb}{0.000000,0.000000,0.000000}%
\pgfsetstrokecolor{currentstroke}%
\pgfsetdash{}{0pt}%
\pgfpathmoveto{\pgfqpoint{1.775062in}{3.613611in}}%
\pgfpathlineto{\pgfqpoint{1.924965in}{3.613611in}}%
\pgfpathlineto{\pgfqpoint{1.924965in}{4.053611in}}%
\pgfpathlineto{\pgfqpoint{1.775062in}{4.053611in}}%
\pgfpathlineto{\pgfqpoint{1.775062in}{3.613611in}}%
\pgfpathclose%
\pgfusepath{stroke,fill}%
\end{pgfscope}%
\begin{pgfscope}%
\pgfsetrectcap%
\pgfsetmiterjoin%
\pgfsetlinewidth{0.803000pt}%
\definecolor{currentstroke}{rgb}{0.000000,0.000000,0.000000}%
\pgfsetstrokecolor{currentstroke}%
\pgfsetdash{}{0pt}%
\pgfpathmoveto{\pgfqpoint{1.137973in}{3.613611in}}%
\pgfpathlineto{\pgfqpoint{1.137973in}{4.075611in}}%
\pgfusepath{stroke}%
\end{pgfscope}%
\begin{pgfscope}%
\pgfsetrectcap%
\pgfsetmiterjoin%
\pgfsetlinewidth{0.803000pt}%
\definecolor{currentstroke}{rgb}{0.000000,0.000000,0.000000}%
\pgfsetstrokecolor{currentstroke}%
\pgfsetdash{}{0pt}%
\pgfpathmoveto{\pgfqpoint{1.962441in}{3.613611in}}%
\pgfpathlineto{\pgfqpoint{1.962441in}{4.075611in}}%
\pgfusepath{stroke}%
\end{pgfscope}%
\begin{pgfscope}%
\pgfsetrectcap%
\pgfsetmiterjoin%
\pgfsetlinewidth{0.803000pt}%
\definecolor{currentstroke}{rgb}{0.000000,0.000000,0.000000}%
\pgfsetstrokecolor{currentstroke}%
\pgfsetdash{}{0pt}%
\pgfpathmoveto{\pgfqpoint{1.137973in}{3.613611in}}%
\pgfpathlineto{\pgfqpoint{1.962441in}{3.613611in}}%
\pgfusepath{stroke}%
\end{pgfscope}%
\begin{pgfscope}%
\pgfsetrectcap%
\pgfsetmiterjoin%
\pgfsetlinewidth{0.803000pt}%
\definecolor{currentstroke}{rgb}{0.000000,0.000000,0.000000}%
\pgfsetstrokecolor{currentstroke}%
\pgfsetdash{}{0pt}%
\pgfpathmoveto{\pgfqpoint{1.137973in}{4.075611in}}%
\pgfpathlineto{\pgfqpoint{1.962441in}{4.075611in}}%
\pgfusepath{stroke}%
\end{pgfscope}%
\begin{pgfscope}%
\definecolor{textcolor}{rgb}{0.000000,0.000000,0.000000}%
\pgfsetstrokecolor{textcolor}%
\pgfsetfillcolor{textcolor}%
\pgftext[x=1.550207in,y=4.158944in,,base]{\color{textcolor}\rmfamily\fontsize{11.000000}{13.200000}\selectfont APRIL ...}%
\end{pgfscope}%
\begin{pgfscope}%
\pgfsetbuttcap%
\pgfsetmiterjoin%
\definecolor{currentfill}{rgb}{1.000000,1.000000,1.000000}%
\pgfsetfillcolor{currentfill}%
\pgfsetlinewidth{0.000000pt}%
\definecolor{currentstroke}{rgb}{0.000000,0.000000,0.000000}%
\pgfsetstrokecolor{currentstroke}%
\pgfsetstrokeopacity{0.000000}%
\pgfsetdash{}{0pt}%
\pgfpathmoveto{\pgfqpoint{2.127335in}{3.613611in}}%
\pgfpathlineto{\pgfqpoint{2.951803in}{3.613611in}}%
\pgfpathlineto{\pgfqpoint{2.951803in}{4.075611in}}%
\pgfpathlineto{\pgfqpoint{2.127335in}{4.075611in}}%
\pgfpathlineto{\pgfqpoint{2.127335in}{3.613611in}}%
\pgfpathclose%
\pgfusepath{fill}%
\end{pgfscope}%
\begin{pgfscope}%
\pgfpathrectangle{\pgfqpoint{2.127335in}{3.613611in}}{\pgfqpoint{0.824468in}{0.462000in}}%
\pgfusepath{clip}%
\pgfsetbuttcap%
\pgfsetmiterjoin%
\definecolor{currentfill}{rgb}{0.121569,0.466667,0.705882}%
\pgfsetfillcolor{currentfill}%
\pgfsetfillopacity{0.500000}%
\pgfsetlinewidth{1.003750pt}%
\definecolor{currentstroke}{rgb}{0.000000,0.000000,0.000000}%
\pgfsetstrokecolor{currentstroke}%
\pgfsetdash{}{0pt}%
\pgfpathmoveto{\pgfqpoint{2.164810in}{3.613611in}}%
\pgfpathlineto{\pgfqpoint{2.314714in}{3.613611in}}%
\pgfpathlineto{\pgfqpoint{2.314714in}{4.053611in}}%
\pgfpathlineto{\pgfqpoint{2.164810in}{4.053611in}}%
\pgfpathlineto{\pgfqpoint{2.164810in}{3.613611in}}%
\pgfpathclose%
\pgfusepath{stroke,fill}%
\end{pgfscope}%
\begin{pgfscope}%
\pgfpathrectangle{\pgfqpoint{2.127335in}{3.613611in}}{\pgfqpoint{0.824468in}{0.462000in}}%
\pgfusepath{clip}%
\pgfsetbuttcap%
\pgfsetmiterjoin%
\definecolor{currentfill}{rgb}{0.121569,0.466667,0.705882}%
\pgfsetfillcolor{currentfill}%
\pgfsetfillopacity{0.500000}%
\pgfsetlinewidth{1.003750pt}%
\definecolor{currentstroke}{rgb}{0.000000,0.000000,0.000000}%
\pgfsetstrokecolor{currentstroke}%
\pgfsetdash{}{0pt}%
\pgfpathmoveto{\pgfqpoint{2.314714in}{3.613611in}}%
\pgfpathlineto{\pgfqpoint{2.464617in}{3.613611in}}%
\pgfpathlineto{\pgfqpoint{2.464617in}{3.743544in}}%
\pgfpathlineto{\pgfqpoint{2.314714in}{3.743544in}}%
\pgfpathlineto{\pgfqpoint{2.314714in}{3.613611in}}%
\pgfpathclose%
\pgfusepath{stroke,fill}%
\end{pgfscope}%
\begin{pgfscope}%
\pgfpathrectangle{\pgfqpoint{2.127335in}{3.613611in}}{\pgfqpoint{0.824468in}{0.462000in}}%
\pgfusepath{clip}%
\pgfsetbuttcap%
\pgfsetmiterjoin%
\definecolor{currentfill}{rgb}{0.121569,0.466667,0.705882}%
\pgfsetfillcolor{currentfill}%
\pgfsetfillopacity{0.500000}%
\pgfsetlinewidth{1.003750pt}%
\definecolor{currentstroke}{rgb}{0.000000,0.000000,0.000000}%
\pgfsetstrokecolor{currentstroke}%
\pgfsetdash{}{0pt}%
\pgfpathmoveto{\pgfqpoint{2.464617in}{3.613611in}}%
\pgfpathlineto{\pgfqpoint{2.614520in}{3.613611in}}%
\pgfpathlineto{\pgfqpoint{2.614520in}{3.634282in}}%
\pgfpathlineto{\pgfqpoint{2.464617in}{3.634282in}}%
\pgfpathlineto{\pgfqpoint{2.464617in}{3.613611in}}%
\pgfpathclose%
\pgfusepath{stroke,fill}%
\end{pgfscope}%
\begin{pgfscope}%
\pgfpathrectangle{\pgfqpoint{2.127335in}{3.613611in}}{\pgfqpoint{0.824468in}{0.462000in}}%
\pgfusepath{clip}%
\pgfsetbuttcap%
\pgfsetmiterjoin%
\definecolor{currentfill}{rgb}{0.121569,0.466667,0.705882}%
\pgfsetfillcolor{currentfill}%
\pgfsetfillopacity{0.500000}%
\pgfsetlinewidth{1.003750pt}%
\definecolor{currentstroke}{rgb}{0.000000,0.000000,0.000000}%
\pgfsetstrokecolor{currentstroke}%
\pgfsetdash{}{0pt}%
\pgfpathmoveto{\pgfqpoint{2.614520in}{3.613611in}}%
\pgfpathlineto{\pgfqpoint{2.764423in}{3.613611in}}%
\pgfpathlineto{\pgfqpoint{2.764423in}{3.637235in}}%
\pgfpathlineto{\pgfqpoint{2.614520in}{3.637235in}}%
\pgfpathlineto{\pgfqpoint{2.614520in}{3.613611in}}%
\pgfpathclose%
\pgfusepath{stroke,fill}%
\end{pgfscope}%
\begin{pgfscope}%
\pgfpathrectangle{\pgfqpoint{2.127335in}{3.613611in}}{\pgfqpoint{0.824468in}{0.462000in}}%
\pgfusepath{clip}%
\pgfsetbuttcap%
\pgfsetmiterjoin%
\definecolor{currentfill}{rgb}{0.121569,0.466667,0.705882}%
\pgfsetfillcolor{currentfill}%
\pgfsetfillopacity{0.500000}%
\pgfsetlinewidth{1.003750pt}%
\definecolor{currentstroke}{rgb}{0.000000,0.000000,0.000000}%
\pgfsetstrokecolor{currentstroke}%
\pgfsetdash{}{0pt}%
\pgfpathmoveto{\pgfqpoint{2.764423in}{3.613611in}}%
\pgfpathlineto{\pgfqpoint{2.914327in}{3.613611in}}%
\pgfpathlineto{\pgfqpoint{2.914327in}{3.631329in}}%
\pgfpathlineto{\pgfqpoint{2.764423in}{3.631329in}}%
\pgfpathlineto{\pgfqpoint{2.764423in}{3.613611in}}%
\pgfpathclose%
\pgfusepath{stroke,fill}%
\end{pgfscope}%
\begin{pgfscope}%
\pgfsetrectcap%
\pgfsetmiterjoin%
\pgfsetlinewidth{0.803000pt}%
\definecolor{currentstroke}{rgb}{0.000000,0.000000,0.000000}%
\pgfsetstrokecolor{currentstroke}%
\pgfsetdash{}{0pt}%
\pgfpathmoveto{\pgfqpoint{2.127335in}{3.613611in}}%
\pgfpathlineto{\pgfqpoint{2.127335in}{4.075611in}}%
\pgfusepath{stroke}%
\end{pgfscope}%
\begin{pgfscope}%
\pgfsetrectcap%
\pgfsetmiterjoin%
\pgfsetlinewidth{0.803000pt}%
\definecolor{currentstroke}{rgb}{0.000000,0.000000,0.000000}%
\pgfsetstrokecolor{currentstroke}%
\pgfsetdash{}{0pt}%
\pgfpathmoveto{\pgfqpoint{2.951803in}{3.613611in}}%
\pgfpathlineto{\pgfqpoint{2.951803in}{4.075611in}}%
\pgfusepath{stroke}%
\end{pgfscope}%
\begin{pgfscope}%
\pgfsetrectcap%
\pgfsetmiterjoin%
\pgfsetlinewidth{0.803000pt}%
\definecolor{currentstroke}{rgb}{0.000000,0.000000,0.000000}%
\pgfsetstrokecolor{currentstroke}%
\pgfsetdash{}{0pt}%
\pgfpathmoveto{\pgfqpoint{2.127335in}{3.613611in}}%
\pgfpathlineto{\pgfqpoint{2.951803in}{3.613611in}}%
\pgfusepath{stroke}%
\end{pgfscope}%
\begin{pgfscope}%
\pgfsetrectcap%
\pgfsetmiterjoin%
\pgfsetlinewidth{0.803000pt}%
\definecolor{currentstroke}{rgb}{0.000000,0.000000,0.000000}%
\pgfsetstrokecolor{currentstroke}%
\pgfsetdash{}{0pt}%
\pgfpathmoveto{\pgfqpoint{2.127335in}{4.075611in}}%
\pgfpathlineto{\pgfqpoint{2.951803in}{4.075611in}}%
\pgfusepath{stroke}%
\end{pgfscope}%
\begin{pgfscope}%
\definecolor{textcolor}{rgb}{0.000000,0.000000,0.000000}%
\pgfsetstrokecolor{textcolor}%
\pgfsetfillcolor{textcolor}%
\pgftext[x=2.539569in,y=4.158944in,,base]{\color{textcolor}\rmfamily\fontsize{11.000000}{13.200000}\selectfont Cegema...}%
\end{pgfscope}%
\begin{pgfscope}%
\pgfsetbuttcap%
\pgfsetmiterjoin%
\definecolor{currentfill}{rgb}{1.000000,1.000000,1.000000}%
\pgfsetfillcolor{currentfill}%
\pgfsetlinewidth{0.000000pt}%
\definecolor{currentstroke}{rgb}{0.000000,0.000000,0.000000}%
\pgfsetstrokecolor{currentstroke}%
\pgfsetstrokeopacity{0.000000}%
\pgfsetdash{}{0pt}%
\pgfpathmoveto{\pgfqpoint{3.116696in}{3.613611in}}%
\pgfpathlineto{\pgfqpoint{3.941164in}{3.613611in}}%
\pgfpathlineto{\pgfqpoint{3.941164in}{4.075611in}}%
\pgfpathlineto{\pgfqpoint{3.116696in}{4.075611in}}%
\pgfpathlineto{\pgfqpoint{3.116696in}{3.613611in}}%
\pgfpathclose%
\pgfusepath{fill}%
\end{pgfscope}%
\begin{pgfscope}%
\pgfpathrectangle{\pgfqpoint{3.116696in}{3.613611in}}{\pgfqpoint{0.824468in}{0.462000in}}%
\pgfusepath{clip}%
\pgfsetbuttcap%
\pgfsetmiterjoin%
\definecolor{currentfill}{rgb}{0.121569,0.466667,0.705882}%
\pgfsetfillcolor{currentfill}%
\pgfsetfillopacity{0.500000}%
\pgfsetlinewidth{1.003750pt}%
\definecolor{currentstroke}{rgb}{0.000000,0.000000,0.000000}%
\pgfsetstrokecolor{currentstroke}%
\pgfsetdash{}{0pt}%
\pgfpathmoveto{\pgfqpoint{3.154172in}{3.613611in}}%
\pgfpathlineto{\pgfqpoint{3.304075in}{3.613611in}}%
\pgfpathlineto{\pgfqpoint{3.304075in}{4.053611in}}%
\pgfpathlineto{\pgfqpoint{3.154172in}{4.053611in}}%
\pgfpathlineto{\pgfqpoint{3.154172in}{3.613611in}}%
\pgfpathclose%
\pgfusepath{stroke,fill}%
\end{pgfscope}%
\begin{pgfscope}%
\pgfpathrectangle{\pgfqpoint{3.116696in}{3.613611in}}{\pgfqpoint{0.824468in}{0.462000in}}%
\pgfusepath{clip}%
\pgfsetbuttcap%
\pgfsetmiterjoin%
\definecolor{currentfill}{rgb}{0.121569,0.466667,0.705882}%
\pgfsetfillcolor{currentfill}%
\pgfsetfillopacity{0.500000}%
\pgfsetlinewidth{1.003750pt}%
\definecolor{currentstroke}{rgb}{0.000000,0.000000,0.000000}%
\pgfsetstrokecolor{currentstroke}%
\pgfsetdash{}{0pt}%
\pgfpathmoveto{\pgfqpoint{3.304075in}{3.613611in}}%
\pgfpathlineto{\pgfqpoint{3.453979in}{3.613611in}}%
\pgfpathlineto{\pgfqpoint{3.453979in}{3.642944in}}%
\pgfpathlineto{\pgfqpoint{3.304075in}{3.642944in}}%
\pgfpathlineto{\pgfqpoint{3.304075in}{3.613611in}}%
\pgfpathclose%
\pgfusepath{stroke,fill}%
\end{pgfscope}%
\begin{pgfscope}%
\pgfpathrectangle{\pgfqpoint{3.116696in}{3.613611in}}{\pgfqpoint{0.824468in}{0.462000in}}%
\pgfusepath{clip}%
\pgfsetbuttcap%
\pgfsetmiterjoin%
\definecolor{currentfill}{rgb}{0.121569,0.466667,0.705882}%
\pgfsetfillcolor{currentfill}%
\pgfsetfillopacity{0.500000}%
\pgfsetlinewidth{1.003750pt}%
\definecolor{currentstroke}{rgb}{0.000000,0.000000,0.000000}%
\pgfsetstrokecolor{currentstroke}%
\pgfsetdash{}{0pt}%
\pgfpathmoveto{\pgfqpoint{3.453979in}{3.613611in}}%
\pgfpathlineto{\pgfqpoint{3.603882in}{3.613611in}}%
\pgfpathlineto{\pgfqpoint{3.603882in}{3.672278in}}%
\pgfpathlineto{\pgfqpoint{3.453979in}{3.672278in}}%
\pgfpathlineto{\pgfqpoint{3.453979in}{3.613611in}}%
\pgfpathclose%
\pgfusepath{stroke,fill}%
\end{pgfscope}%
\begin{pgfscope}%
\pgfpathrectangle{\pgfqpoint{3.116696in}{3.613611in}}{\pgfqpoint{0.824468in}{0.462000in}}%
\pgfusepath{clip}%
\pgfsetbuttcap%
\pgfsetmiterjoin%
\definecolor{currentfill}{rgb}{0.121569,0.466667,0.705882}%
\pgfsetfillcolor{currentfill}%
\pgfsetfillopacity{0.500000}%
\pgfsetlinewidth{1.003750pt}%
\definecolor{currentstroke}{rgb}{0.000000,0.000000,0.000000}%
\pgfsetstrokecolor{currentstroke}%
\pgfsetdash{}{0pt}%
\pgfpathmoveto{\pgfqpoint{3.603882in}{3.613611in}}%
\pgfpathlineto{\pgfqpoint{3.753785in}{3.613611in}}%
\pgfpathlineto{\pgfqpoint{3.753785in}{3.613611in}}%
\pgfpathlineto{\pgfqpoint{3.603882in}{3.613611in}}%
\pgfpathlineto{\pgfqpoint{3.603882in}{3.613611in}}%
\pgfpathclose%
\pgfusepath{stroke,fill}%
\end{pgfscope}%
\begin{pgfscope}%
\pgfpathrectangle{\pgfqpoint{3.116696in}{3.613611in}}{\pgfqpoint{0.824468in}{0.462000in}}%
\pgfusepath{clip}%
\pgfsetbuttcap%
\pgfsetmiterjoin%
\definecolor{currentfill}{rgb}{0.121569,0.466667,0.705882}%
\pgfsetfillcolor{currentfill}%
\pgfsetfillopacity{0.500000}%
\pgfsetlinewidth{1.003750pt}%
\definecolor{currentstroke}{rgb}{0.000000,0.000000,0.000000}%
\pgfsetstrokecolor{currentstroke}%
\pgfsetdash{}{0pt}%
\pgfpathmoveto{\pgfqpoint{3.753785in}{3.613611in}}%
\pgfpathlineto{\pgfqpoint{3.903688in}{3.613611in}}%
\pgfpathlineto{\pgfqpoint{3.903688in}{3.613611in}}%
\pgfpathlineto{\pgfqpoint{3.753785in}{3.613611in}}%
\pgfpathlineto{\pgfqpoint{3.753785in}{3.613611in}}%
\pgfpathclose%
\pgfusepath{stroke,fill}%
\end{pgfscope}%
\begin{pgfscope}%
\pgfsetrectcap%
\pgfsetmiterjoin%
\pgfsetlinewidth{0.803000pt}%
\definecolor{currentstroke}{rgb}{0.000000,0.000000,0.000000}%
\pgfsetstrokecolor{currentstroke}%
\pgfsetdash{}{0pt}%
\pgfpathmoveto{\pgfqpoint{3.116696in}{3.613611in}}%
\pgfpathlineto{\pgfqpoint{3.116696in}{4.075611in}}%
\pgfusepath{stroke}%
\end{pgfscope}%
\begin{pgfscope}%
\pgfsetrectcap%
\pgfsetmiterjoin%
\pgfsetlinewidth{0.803000pt}%
\definecolor{currentstroke}{rgb}{0.000000,0.000000,0.000000}%
\pgfsetstrokecolor{currentstroke}%
\pgfsetdash{}{0pt}%
\pgfpathmoveto{\pgfqpoint{3.941164in}{3.613611in}}%
\pgfpathlineto{\pgfqpoint{3.941164in}{4.075611in}}%
\pgfusepath{stroke}%
\end{pgfscope}%
\begin{pgfscope}%
\pgfsetrectcap%
\pgfsetmiterjoin%
\pgfsetlinewidth{0.803000pt}%
\definecolor{currentstroke}{rgb}{0.000000,0.000000,0.000000}%
\pgfsetstrokecolor{currentstroke}%
\pgfsetdash{}{0pt}%
\pgfpathmoveto{\pgfqpoint{3.116696in}{3.613611in}}%
\pgfpathlineto{\pgfqpoint{3.941164in}{3.613611in}}%
\pgfusepath{stroke}%
\end{pgfscope}%
\begin{pgfscope}%
\pgfsetrectcap%
\pgfsetmiterjoin%
\pgfsetlinewidth{0.803000pt}%
\definecolor{currentstroke}{rgb}{0.000000,0.000000,0.000000}%
\pgfsetstrokecolor{currentstroke}%
\pgfsetdash{}{0pt}%
\pgfpathmoveto{\pgfqpoint{3.116696in}{4.075611in}}%
\pgfpathlineto{\pgfqpoint{3.941164in}{4.075611in}}%
\pgfusepath{stroke}%
\end{pgfscope}%
\begin{pgfscope}%
\definecolor{textcolor}{rgb}{0.000000,0.000000,0.000000}%
\pgfsetstrokecolor{textcolor}%
\pgfsetfillcolor{textcolor}%
\pgftext[x=3.528930in,y=4.158944in,,base]{\color{textcolor}\rmfamily\fontsize{11.000000}{13.200000}\selectfont LCL}%
\end{pgfscope}%
\begin{pgfscope}%
\pgfsetbuttcap%
\pgfsetmiterjoin%
\definecolor{currentfill}{rgb}{1.000000,1.000000,1.000000}%
\pgfsetfillcolor{currentfill}%
\pgfsetlinewidth{0.000000pt}%
\definecolor{currentstroke}{rgb}{0.000000,0.000000,0.000000}%
\pgfsetstrokecolor{currentstroke}%
\pgfsetstrokeopacity{0.000000}%
\pgfsetdash{}{0pt}%
\pgfpathmoveto{\pgfqpoint{4.106058in}{3.613611in}}%
\pgfpathlineto{\pgfqpoint{4.930526in}{3.613611in}}%
\pgfpathlineto{\pgfqpoint{4.930526in}{4.075611in}}%
\pgfpathlineto{\pgfqpoint{4.106058in}{4.075611in}}%
\pgfpathlineto{\pgfqpoint{4.106058in}{3.613611in}}%
\pgfpathclose%
\pgfusepath{fill}%
\end{pgfscope}%
\begin{pgfscope}%
\pgfpathrectangle{\pgfqpoint{4.106058in}{3.613611in}}{\pgfqpoint{0.824468in}{0.462000in}}%
\pgfusepath{clip}%
\pgfsetbuttcap%
\pgfsetmiterjoin%
\definecolor{currentfill}{rgb}{0.121569,0.466667,0.705882}%
\pgfsetfillcolor{currentfill}%
\pgfsetfillopacity{0.500000}%
\pgfsetlinewidth{1.003750pt}%
\definecolor{currentstroke}{rgb}{0.000000,0.000000,0.000000}%
\pgfsetstrokecolor{currentstroke}%
\pgfsetdash{}{0pt}%
\pgfpathmoveto{\pgfqpoint{4.143534in}{3.613611in}}%
\pgfpathlineto{\pgfqpoint{4.293437in}{3.613611in}}%
\pgfpathlineto{\pgfqpoint{4.293437in}{4.053611in}}%
\pgfpathlineto{\pgfqpoint{4.143534in}{4.053611in}}%
\pgfpathlineto{\pgfqpoint{4.143534in}{3.613611in}}%
\pgfpathclose%
\pgfusepath{stroke,fill}%
\end{pgfscope}%
\begin{pgfscope}%
\pgfpathrectangle{\pgfqpoint{4.106058in}{3.613611in}}{\pgfqpoint{0.824468in}{0.462000in}}%
\pgfusepath{clip}%
\pgfsetbuttcap%
\pgfsetmiterjoin%
\definecolor{currentfill}{rgb}{0.121569,0.466667,0.705882}%
\pgfsetfillcolor{currentfill}%
\pgfsetfillopacity{0.500000}%
\pgfsetlinewidth{1.003750pt}%
\definecolor{currentstroke}{rgb}{0.000000,0.000000,0.000000}%
\pgfsetstrokecolor{currentstroke}%
\pgfsetdash{}{0pt}%
\pgfpathmoveto{\pgfqpoint{4.293437in}{3.613611in}}%
\pgfpathlineto{\pgfqpoint{4.443340in}{3.613611in}}%
\pgfpathlineto{\pgfqpoint{4.443340in}{3.887125in}}%
\pgfpathlineto{\pgfqpoint{4.293437in}{3.887125in}}%
\pgfpathlineto{\pgfqpoint{4.293437in}{3.613611in}}%
\pgfpathclose%
\pgfusepath{stroke,fill}%
\end{pgfscope}%
\begin{pgfscope}%
\pgfpathrectangle{\pgfqpoint{4.106058in}{3.613611in}}{\pgfqpoint{0.824468in}{0.462000in}}%
\pgfusepath{clip}%
\pgfsetbuttcap%
\pgfsetmiterjoin%
\definecolor{currentfill}{rgb}{0.121569,0.466667,0.705882}%
\pgfsetfillcolor{currentfill}%
\pgfsetfillopacity{0.500000}%
\pgfsetlinewidth{1.003750pt}%
\definecolor{currentstroke}{rgb}{0.000000,0.000000,0.000000}%
\pgfsetstrokecolor{currentstroke}%
\pgfsetdash{}{0pt}%
\pgfpathmoveto{\pgfqpoint{4.443340in}{3.613611in}}%
\pgfpathlineto{\pgfqpoint{4.593244in}{3.613611in}}%
\pgfpathlineto{\pgfqpoint{4.593244in}{3.684962in}}%
\pgfpathlineto{\pgfqpoint{4.443340in}{3.684962in}}%
\pgfpathlineto{\pgfqpoint{4.443340in}{3.613611in}}%
\pgfpathclose%
\pgfusepath{stroke,fill}%
\end{pgfscope}%
\begin{pgfscope}%
\pgfpathrectangle{\pgfqpoint{4.106058in}{3.613611in}}{\pgfqpoint{0.824468in}{0.462000in}}%
\pgfusepath{clip}%
\pgfsetbuttcap%
\pgfsetmiterjoin%
\definecolor{currentfill}{rgb}{0.121569,0.466667,0.705882}%
\pgfsetfillcolor{currentfill}%
\pgfsetfillopacity{0.500000}%
\pgfsetlinewidth{1.003750pt}%
\definecolor{currentstroke}{rgb}{0.000000,0.000000,0.000000}%
\pgfsetstrokecolor{currentstroke}%
\pgfsetdash{}{0pt}%
\pgfpathmoveto{\pgfqpoint{4.593244in}{3.613611in}}%
\pgfpathlineto{\pgfqpoint{4.743147in}{3.613611in}}%
\pgfpathlineto{\pgfqpoint{4.743147in}{3.661179in}}%
\pgfpathlineto{\pgfqpoint{4.593244in}{3.661179in}}%
\pgfpathlineto{\pgfqpoint{4.593244in}{3.613611in}}%
\pgfpathclose%
\pgfusepath{stroke,fill}%
\end{pgfscope}%
\begin{pgfscope}%
\pgfpathrectangle{\pgfqpoint{4.106058in}{3.613611in}}{\pgfqpoint{0.824468in}{0.462000in}}%
\pgfusepath{clip}%
\pgfsetbuttcap%
\pgfsetmiterjoin%
\definecolor{currentfill}{rgb}{0.121569,0.466667,0.705882}%
\pgfsetfillcolor{currentfill}%
\pgfsetfillopacity{0.500000}%
\pgfsetlinewidth{1.003750pt}%
\definecolor{currentstroke}{rgb}{0.000000,0.000000,0.000000}%
\pgfsetstrokecolor{currentstroke}%
\pgfsetdash{}{0pt}%
\pgfpathmoveto{\pgfqpoint{4.743147in}{3.613611in}}%
\pgfpathlineto{\pgfqpoint{4.893050in}{3.613611in}}%
\pgfpathlineto{\pgfqpoint{4.893050in}{3.637395in}}%
\pgfpathlineto{\pgfqpoint{4.743147in}{3.637395in}}%
\pgfpathlineto{\pgfqpoint{4.743147in}{3.613611in}}%
\pgfpathclose%
\pgfusepath{stroke,fill}%
\end{pgfscope}%
\begin{pgfscope}%
\pgfsetrectcap%
\pgfsetmiterjoin%
\pgfsetlinewidth{0.803000pt}%
\definecolor{currentstroke}{rgb}{0.000000,0.000000,0.000000}%
\pgfsetstrokecolor{currentstroke}%
\pgfsetdash{}{0pt}%
\pgfpathmoveto{\pgfqpoint{4.106058in}{3.613611in}}%
\pgfpathlineto{\pgfqpoint{4.106058in}{4.075611in}}%
\pgfusepath{stroke}%
\end{pgfscope}%
\begin{pgfscope}%
\pgfsetrectcap%
\pgfsetmiterjoin%
\pgfsetlinewidth{0.803000pt}%
\definecolor{currentstroke}{rgb}{0.000000,0.000000,0.000000}%
\pgfsetstrokecolor{currentstroke}%
\pgfsetdash{}{0pt}%
\pgfpathmoveto{\pgfqpoint{4.930526in}{3.613611in}}%
\pgfpathlineto{\pgfqpoint{4.930526in}{4.075611in}}%
\pgfusepath{stroke}%
\end{pgfscope}%
\begin{pgfscope}%
\pgfsetrectcap%
\pgfsetmiterjoin%
\pgfsetlinewidth{0.803000pt}%
\definecolor{currentstroke}{rgb}{0.000000,0.000000,0.000000}%
\pgfsetstrokecolor{currentstroke}%
\pgfsetdash{}{0pt}%
\pgfpathmoveto{\pgfqpoint{4.106058in}{3.613611in}}%
\pgfpathlineto{\pgfqpoint{4.930526in}{3.613611in}}%
\pgfusepath{stroke}%
\end{pgfscope}%
\begin{pgfscope}%
\pgfsetrectcap%
\pgfsetmiterjoin%
\pgfsetlinewidth{0.803000pt}%
\definecolor{currentstroke}{rgb}{0.000000,0.000000,0.000000}%
\pgfsetstrokecolor{currentstroke}%
\pgfsetdash{}{0pt}%
\pgfpathmoveto{\pgfqpoint{4.106058in}{4.075611in}}%
\pgfpathlineto{\pgfqpoint{4.930526in}{4.075611in}}%
\pgfusepath{stroke}%
\end{pgfscope}%
\begin{pgfscope}%
\definecolor{textcolor}{rgb}{0.000000,0.000000,0.000000}%
\pgfsetstrokecolor{textcolor}%
\pgfsetfillcolor{textcolor}%
\pgftext[x=4.518292in,y=4.158944in,,base]{\color{textcolor}\rmfamily\fontsize{11.000000}{13.200000}\selectfont Afer}%
\end{pgfscope}%
\begin{pgfscope}%
\pgfsetbuttcap%
\pgfsetmiterjoin%
\definecolor{currentfill}{rgb}{1.000000,1.000000,1.000000}%
\pgfsetfillcolor{currentfill}%
\pgfsetlinewidth{0.000000pt}%
\definecolor{currentstroke}{rgb}{0.000000,0.000000,0.000000}%
\pgfsetstrokecolor{currentstroke}%
\pgfsetstrokeopacity{0.000000}%
\pgfsetdash{}{0pt}%
\pgfpathmoveto{\pgfqpoint{5.095420in}{3.613611in}}%
\pgfpathlineto{\pgfqpoint{5.919888in}{3.613611in}}%
\pgfpathlineto{\pgfqpoint{5.919888in}{4.075611in}}%
\pgfpathlineto{\pgfqpoint{5.095420in}{4.075611in}}%
\pgfpathlineto{\pgfqpoint{5.095420in}{3.613611in}}%
\pgfpathclose%
\pgfusepath{fill}%
\end{pgfscope}%
\begin{pgfscope}%
\pgfpathrectangle{\pgfqpoint{5.095420in}{3.613611in}}{\pgfqpoint{0.824468in}{0.462000in}}%
\pgfusepath{clip}%
\pgfsetbuttcap%
\pgfsetmiterjoin%
\definecolor{currentfill}{rgb}{0.121569,0.466667,0.705882}%
\pgfsetfillcolor{currentfill}%
\pgfsetfillopacity{0.500000}%
\pgfsetlinewidth{1.003750pt}%
\definecolor{currentstroke}{rgb}{0.000000,0.000000,0.000000}%
\pgfsetstrokecolor{currentstroke}%
\pgfsetdash{}{0pt}%
\pgfpathmoveto{\pgfqpoint{5.132895in}{3.613611in}}%
\pgfpathlineto{\pgfqpoint{5.282799in}{3.613611in}}%
\pgfpathlineto{\pgfqpoint{5.282799in}{4.053611in}}%
\pgfpathlineto{\pgfqpoint{5.132895in}{4.053611in}}%
\pgfpathlineto{\pgfqpoint{5.132895in}{3.613611in}}%
\pgfpathclose%
\pgfusepath{stroke,fill}%
\end{pgfscope}%
\begin{pgfscope}%
\pgfpathrectangle{\pgfqpoint{5.095420in}{3.613611in}}{\pgfqpoint{0.824468in}{0.462000in}}%
\pgfusepath{clip}%
\pgfsetbuttcap%
\pgfsetmiterjoin%
\definecolor{currentfill}{rgb}{0.121569,0.466667,0.705882}%
\pgfsetfillcolor{currentfill}%
\pgfsetfillopacity{0.500000}%
\pgfsetlinewidth{1.003750pt}%
\definecolor{currentstroke}{rgb}{0.000000,0.000000,0.000000}%
\pgfsetstrokecolor{currentstroke}%
\pgfsetdash{}{0pt}%
\pgfpathmoveto{\pgfqpoint{5.282799in}{3.613611in}}%
\pgfpathlineto{\pgfqpoint{5.432702in}{3.613611in}}%
\pgfpathlineto{\pgfqpoint{5.432702in}{3.799355in}}%
\pgfpathlineto{\pgfqpoint{5.282799in}{3.799355in}}%
\pgfpathlineto{\pgfqpoint{5.282799in}{3.613611in}}%
\pgfpathclose%
\pgfusepath{stroke,fill}%
\end{pgfscope}%
\begin{pgfscope}%
\pgfpathrectangle{\pgfqpoint{5.095420in}{3.613611in}}{\pgfqpoint{0.824468in}{0.462000in}}%
\pgfusepath{clip}%
\pgfsetbuttcap%
\pgfsetmiterjoin%
\definecolor{currentfill}{rgb}{0.121569,0.466667,0.705882}%
\pgfsetfillcolor{currentfill}%
\pgfsetfillopacity{0.500000}%
\pgfsetlinewidth{1.003750pt}%
\definecolor{currentstroke}{rgb}{0.000000,0.000000,0.000000}%
\pgfsetstrokecolor{currentstroke}%
\pgfsetdash{}{0pt}%
\pgfpathmoveto{\pgfqpoint{5.432702in}{3.613611in}}%
\pgfpathlineto{\pgfqpoint{5.582605in}{3.613611in}}%
\pgfpathlineto{\pgfqpoint{5.582605in}{3.698871in}}%
\pgfpathlineto{\pgfqpoint{5.432702in}{3.698871in}}%
\pgfpathlineto{\pgfqpoint{5.432702in}{3.613611in}}%
\pgfpathclose%
\pgfusepath{stroke,fill}%
\end{pgfscope}%
\begin{pgfscope}%
\pgfpathrectangle{\pgfqpoint{5.095420in}{3.613611in}}{\pgfqpoint{0.824468in}{0.462000in}}%
\pgfusepath{clip}%
\pgfsetbuttcap%
\pgfsetmiterjoin%
\definecolor{currentfill}{rgb}{0.121569,0.466667,0.705882}%
\pgfsetfillcolor{currentfill}%
\pgfsetfillopacity{0.500000}%
\pgfsetlinewidth{1.003750pt}%
\definecolor{currentstroke}{rgb}{0.000000,0.000000,0.000000}%
\pgfsetstrokecolor{currentstroke}%
\pgfsetdash{}{0pt}%
\pgfpathmoveto{\pgfqpoint{5.582605in}{3.613611in}}%
\pgfpathlineto{\pgfqpoint{5.732509in}{3.613611in}}%
\pgfpathlineto{\pgfqpoint{5.732509in}{3.709528in}}%
\pgfpathlineto{\pgfqpoint{5.582605in}{3.709528in}}%
\pgfpathlineto{\pgfqpoint{5.582605in}{3.613611in}}%
\pgfpathclose%
\pgfusepath{stroke,fill}%
\end{pgfscope}%
\begin{pgfscope}%
\pgfpathrectangle{\pgfqpoint{5.095420in}{3.613611in}}{\pgfqpoint{0.824468in}{0.462000in}}%
\pgfusepath{clip}%
\pgfsetbuttcap%
\pgfsetmiterjoin%
\definecolor{currentfill}{rgb}{0.121569,0.466667,0.705882}%
\pgfsetfillcolor{currentfill}%
\pgfsetfillopacity{0.500000}%
\pgfsetlinewidth{1.003750pt}%
\definecolor{currentstroke}{rgb}{0.000000,0.000000,0.000000}%
\pgfsetstrokecolor{currentstroke}%
\pgfsetdash{}{0pt}%
\pgfpathmoveto{\pgfqpoint{5.732509in}{3.613611in}}%
\pgfpathlineto{\pgfqpoint{5.882412in}{3.613611in}}%
\pgfpathlineto{\pgfqpoint{5.882412in}{3.648628in}}%
\pgfpathlineto{\pgfqpoint{5.732509in}{3.648628in}}%
\pgfpathlineto{\pgfqpoint{5.732509in}{3.613611in}}%
\pgfpathclose%
\pgfusepath{stroke,fill}%
\end{pgfscope}%
\begin{pgfscope}%
\pgfsetrectcap%
\pgfsetmiterjoin%
\pgfsetlinewidth{0.803000pt}%
\definecolor{currentstroke}{rgb}{0.000000,0.000000,0.000000}%
\pgfsetstrokecolor{currentstroke}%
\pgfsetdash{}{0pt}%
\pgfpathmoveto{\pgfqpoint{5.095420in}{3.613611in}}%
\pgfpathlineto{\pgfqpoint{5.095420in}{4.075611in}}%
\pgfusepath{stroke}%
\end{pgfscope}%
\begin{pgfscope}%
\pgfsetrectcap%
\pgfsetmiterjoin%
\pgfsetlinewidth{0.803000pt}%
\definecolor{currentstroke}{rgb}{0.000000,0.000000,0.000000}%
\pgfsetstrokecolor{currentstroke}%
\pgfsetdash{}{0pt}%
\pgfpathmoveto{\pgfqpoint{5.919888in}{3.613611in}}%
\pgfpathlineto{\pgfqpoint{5.919888in}{4.075611in}}%
\pgfusepath{stroke}%
\end{pgfscope}%
\begin{pgfscope}%
\pgfsetrectcap%
\pgfsetmiterjoin%
\pgfsetlinewidth{0.803000pt}%
\definecolor{currentstroke}{rgb}{0.000000,0.000000,0.000000}%
\pgfsetstrokecolor{currentstroke}%
\pgfsetdash{}{0pt}%
\pgfpathmoveto{\pgfqpoint{5.095420in}{3.613611in}}%
\pgfpathlineto{\pgfqpoint{5.919888in}{3.613611in}}%
\pgfusepath{stroke}%
\end{pgfscope}%
\begin{pgfscope}%
\pgfsetrectcap%
\pgfsetmiterjoin%
\pgfsetlinewidth{0.803000pt}%
\definecolor{currentstroke}{rgb}{0.000000,0.000000,0.000000}%
\pgfsetstrokecolor{currentstroke}%
\pgfsetdash{}{0pt}%
\pgfpathmoveto{\pgfqpoint{5.095420in}{4.075611in}}%
\pgfpathlineto{\pgfqpoint{5.919888in}{4.075611in}}%
\pgfusepath{stroke}%
\end{pgfscope}%
\begin{pgfscope}%
\definecolor{textcolor}{rgb}{0.000000,0.000000,0.000000}%
\pgfsetstrokecolor{textcolor}%
\pgfsetfillcolor{textcolor}%
\pgftext[x=5.507654in,y=4.158944in,,base]{\color{textcolor}\rmfamily\fontsize{11.000000}{13.200000}\selectfont Pacifica}%
\end{pgfscope}%
\begin{pgfscope}%
\pgfsetbuttcap%
\pgfsetmiterjoin%
\definecolor{currentfill}{rgb}{1.000000,1.000000,1.000000}%
\pgfsetfillcolor{currentfill}%
\pgfsetlinewidth{0.000000pt}%
\definecolor{currentstroke}{rgb}{0.000000,0.000000,0.000000}%
\pgfsetstrokecolor{currentstroke}%
\pgfsetstrokeopacity{0.000000}%
\pgfsetdash{}{0pt}%
\pgfpathmoveto{\pgfqpoint{6.084781in}{3.613611in}}%
\pgfpathlineto{\pgfqpoint{6.909249in}{3.613611in}}%
\pgfpathlineto{\pgfqpoint{6.909249in}{4.075611in}}%
\pgfpathlineto{\pgfqpoint{6.084781in}{4.075611in}}%
\pgfpathlineto{\pgfqpoint{6.084781in}{3.613611in}}%
\pgfpathclose%
\pgfusepath{fill}%
\end{pgfscope}%
\begin{pgfscope}%
\pgfpathrectangle{\pgfqpoint{6.084781in}{3.613611in}}{\pgfqpoint{0.824468in}{0.462000in}}%
\pgfusepath{clip}%
\pgfsetbuttcap%
\pgfsetmiterjoin%
\definecolor{currentfill}{rgb}{0.121569,0.466667,0.705882}%
\pgfsetfillcolor{currentfill}%
\pgfsetfillopacity{0.500000}%
\pgfsetlinewidth{1.003750pt}%
\definecolor{currentstroke}{rgb}{0.000000,0.000000,0.000000}%
\pgfsetstrokecolor{currentstroke}%
\pgfsetdash{}{0pt}%
\pgfpathmoveto{\pgfqpoint{6.122257in}{3.613611in}}%
\pgfpathlineto{\pgfqpoint{6.272160in}{3.613611in}}%
\pgfpathlineto{\pgfqpoint{6.272160in}{4.053611in}}%
\pgfpathlineto{\pgfqpoint{6.122257in}{4.053611in}}%
\pgfpathlineto{\pgfqpoint{6.122257in}{3.613611in}}%
\pgfpathclose%
\pgfusepath{stroke,fill}%
\end{pgfscope}%
\begin{pgfscope}%
\pgfpathrectangle{\pgfqpoint{6.084781in}{3.613611in}}{\pgfqpoint{0.824468in}{0.462000in}}%
\pgfusepath{clip}%
\pgfsetbuttcap%
\pgfsetmiterjoin%
\definecolor{currentfill}{rgb}{0.121569,0.466667,0.705882}%
\pgfsetfillcolor{currentfill}%
\pgfsetfillopacity{0.500000}%
\pgfsetlinewidth{1.003750pt}%
\definecolor{currentstroke}{rgb}{0.000000,0.000000,0.000000}%
\pgfsetstrokecolor{currentstroke}%
\pgfsetdash{}{0pt}%
\pgfpathmoveto{\pgfqpoint{6.272160in}{3.613611in}}%
\pgfpathlineto{\pgfqpoint{6.422064in}{3.613611in}}%
\pgfpathlineto{\pgfqpoint{6.422064in}{3.701611in}}%
\pgfpathlineto{\pgfqpoint{6.272160in}{3.701611in}}%
\pgfpathlineto{\pgfqpoint{6.272160in}{3.613611in}}%
\pgfpathclose%
\pgfusepath{stroke,fill}%
\end{pgfscope}%
\begin{pgfscope}%
\pgfpathrectangle{\pgfqpoint{6.084781in}{3.613611in}}{\pgfqpoint{0.824468in}{0.462000in}}%
\pgfusepath{clip}%
\pgfsetbuttcap%
\pgfsetmiterjoin%
\definecolor{currentfill}{rgb}{0.121569,0.466667,0.705882}%
\pgfsetfillcolor{currentfill}%
\pgfsetfillopacity{0.500000}%
\pgfsetlinewidth{1.003750pt}%
\definecolor{currentstroke}{rgb}{0.000000,0.000000,0.000000}%
\pgfsetstrokecolor{currentstroke}%
\pgfsetdash{}{0pt}%
\pgfpathmoveto{\pgfqpoint{6.422064in}{3.613611in}}%
\pgfpathlineto{\pgfqpoint{6.571967in}{3.613611in}}%
\pgfpathlineto{\pgfqpoint{6.571967in}{3.692811in}}%
\pgfpathlineto{\pgfqpoint{6.422064in}{3.692811in}}%
\pgfpathlineto{\pgfqpoint{6.422064in}{3.613611in}}%
\pgfpathclose%
\pgfusepath{stroke,fill}%
\end{pgfscope}%
\begin{pgfscope}%
\pgfpathrectangle{\pgfqpoint{6.084781in}{3.613611in}}{\pgfqpoint{0.824468in}{0.462000in}}%
\pgfusepath{clip}%
\pgfsetbuttcap%
\pgfsetmiterjoin%
\definecolor{currentfill}{rgb}{0.121569,0.466667,0.705882}%
\pgfsetfillcolor{currentfill}%
\pgfsetfillopacity{0.500000}%
\pgfsetlinewidth{1.003750pt}%
\definecolor{currentstroke}{rgb}{0.000000,0.000000,0.000000}%
\pgfsetstrokecolor{currentstroke}%
\pgfsetdash{}{0pt}%
\pgfpathmoveto{\pgfqpoint{6.571967in}{3.613611in}}%
\pgfpathlineto{\pgfqpoint{6.721870in}{3.613611in}}%
\pgfpathlineto{\pgfqpoint{6.721870in}{3.613611in}}%
\pgfpathlineto{\pgfqpoint{6.571967in}{3.613611in}}%
\pgfpathlineto{\pgfqpoint{6.571967in}{3.613611in}}%
\pgfpathclose%
\pgfusepath{stroke,fill}%
\end{pgfscope}%
\begin{pgfscope}%
\pgfpathrectangle{\pgfqpoint{6.084781in}{3.613611in}}{\pgfqpoint{0.824468in}{0.462000in}}%
\pgfusepath{clip}%
\pgfsetbuttcap%
\pgfsetmiterjoin%
\definecolor{currentfill}{rgb}{0.121569,0.466667,0.705882}%
\pgfsetfillcolor{currentfill}%
\pgfsetfillopacity{0.500000}%
\pgfsetlinewidth{1.003750pt}%
\definecolor{currentstroke}{rgb}{0.000000,0.000000,0.000000}%
\pgfsetstrokecolor{currentstroke}%
\pgfsetdash{}{0pt}%
\pgfpathmoveto{\pgfqpoint{6.721870in}{3.613611in}}%
\pgfpathlineto{\pgfqpoint{6.871774in}{3.613611in}}%
\pgfpathlineto{\pgfqpoint{6.871774in}{3.613611in}}%
\pgfpathlineto{\pgfqpoint{6.721870in}{3.613611in}}%
\pgfpathlineto{\pgfqpoint{6.721870in}{3.613611in}}%
\pgfpathclose%
\pgfusepath{stroke,fill}%
\end{pgfscope}%
\begin{pgfscope}%
\pgfsetrectcap%
\pgfsetmiterjoin%
\pgfsetlinewidth{0.803000pt}%
\definecolor{currentstroke}{rgb}{0.000000,0.000000,0.000000}%
\pgfsetstrokecolor{currentstroke}%
\pgfsetdash{}{0pt}%
\pgfpathmoveto{\pgfqpoint{6.084781in}{3.613611in}}%
\pgfpathlineto{\pgfqpoint{6.084781in}{4.075611in}}%
\pgfusepath{stroke}%
\end{pgfscope}%
\begin{pgfscope}%
\pgfsetrectcap%
\pgfsetmiterjoin%
\pgfsetlinewidth{0.803000pt}%
\definecolor{currentstroke}{rgb}{0.000000,0.000000,0.000000}%
\pgfsetstrokecolor{currentstroke}%
\pgfsetdash{}{0pt}%
\pgfpathmoveto{\pgfqpoint{6.909249in}{3.613611in}}%
\pgfpathlineto{\pgfqpoint{6.909249in}{4.075611in}}%
\pgfusepath{stroke}%
\end{pgfscope}%
\begin{pgfscope}%
\pgfsetrectcap%
\pgfsetmiterjoin%
\pgfsetlinewidth{0.803000pt}%
\definecolor{currentstroke}{rgb}{0.000000,0.000000,0.000000}%
\pgfsetstrokecolor{currentstroke}%
\pgfsetdash{}{0pt}%
\pgfpathmoveto{\pgfqpoint{6.084781in}{3.613611in}}%
\pgfpathlineto{\pgfqpoint{6.909249in}{3.613611in}}%
\pgfusepath{stroke}%
\end{pgfscope}%
\begin{pgfscope}%
\pgfsetrectcap%
\pgfsetmiterjoin%
\pgfsetlinewidth{0.803000pt}%
\definecolor{currentstroke}{rgb}{0.000000,0.000000,0.000000}%
\pgfsetstrokecolor{currentstroke}%
\pgfsetdash{}{0pt}%
\pgfpathmoveto{\pgfqpoint{6.084781in}{4.075611in}}%
\pgfpathlineto{\pgfqpoint{6.909249in}{4.075611in}}%
\pgfusepath{stroke}%
\end{pgfscope}%
\begin{pgfscope}%
\definecolor{textcolor}{rgb}{0.000000,0.000000,0.000000}%
\pgfsetstrokecolor{textcolor}%
\pgfsetfillcolor{textcolor}%
\pgftext[x=6.497015in,y=4.158944in,,base]{\color{textcolor}\rmfamily\fontsize{11.000000}{13.200000}\selectfont SwissLife}%
\end{pgfscope}%
\begin{pgfscope}%
\pgfsetbuttcap%
\pgfsetmiterjoin%
\definecolor{currentfill}{rgb}{1.000000,1.000000,1.000000}%
\pgfsetfillcolor{currentfill}%
\pgfsetlinewidth{0.000000pt}%
\definecolor{currentstroke}{rgb}{0.000000,0.000000,0.000000}%
\pgfsetstrokecolor{currentstroke}%
\pgfsetstrokeopacity{0.000000}%
\pgfsetdash{}{0pt}%
\pgfpathmoveto{\pgfqpoint{7.074143in}{3.613611in}}%
\pgfpathlineto{\pgfqpoint{7.898611in}{3.613611in}}%
\pgfpathlineto{\pgfqpoint{7.898611in}{4.075611in}}%
\pgfpathlineto{\pgfqpoint{7.074143in}{4.075611in}}%
\pgfpathlineto{\pgfqpoint{7.074143in}{3.613611in}}%
\pgfpathclose%
\pgfusepath{fill}%
\end{pgfscope}%
\begin{pgfscope}%
\pgfpathrectangle{\pgfqpoint{7.074143in}{3.613611in}}{\pgfqpoint{0.824468in}{0.462000in}}%
\pgfusepath{clip}%
\pgfsetbuttcap%
\pgfsetmiterjoin%
\definecolor{currentfill}{rgb}{0.121569,0.466667,0.705882}%
\pgfsetfillcolor{currentfill}%
\pgfsetfillopacity{0.500000}%
\pgfsetlinewidth{1.003750pt}%
\definecolor{currentstroke}{rgb}{0.000000,0.000000,0.000000}%
\pgfsetstrokecolor{currentstroke}%
\pgfsetdash{}{0pt}%
\pgfpathmoveto{\pgfqpoint{7.111619in}{3.613611in}}%
\pgfpathlineto{\pgfqpoint{7.261522in}{3.613611in}}%
\pgfpathlineto{\pgfqpoint{7.261522in}{4.053611in}}%
\pgfpathlineto{\pgfqpoint{7.111619in}{4.053611in}}%
\pgfpathlineto{\pgfqpoint{7.111619in}{3.613611in}}%
\pgfpathclose%
\pgfusepath{stroke,fill}%
\end{pgfscope}%
\begin{pgfscope}%
\pgfpathrectangle{\pgfqpoint{7.074143in}{3.613611in}}{\pgfqpoint{0.824468in}{0.462000in}}%
\pgfusepath{clip}%
\pgfsetbuttcap%
\pgfsetmiterjoin%
\definecolor{currentfill}{rgb}{0.121569,0.466667,0.705882}%
\pgfsetfillcolor{currentfill}%
\pgfsetfillopacity{0.500000}%
\pgfsetlinewidth{1.003750pt}%
\definecolor{currentstroke}{rgb}{0.000000,0.000000,0.000000}%
\pgfsetstrokecolor{currentstroke}%
\pgfsetdash{}{0pt}%
\pgfpathmoveto{\pgfqpoint{7.261522in}{3.613611in}}%
\pgfpathlineto{\pgfqpoint{7.411425in}{3.613611in}}%
\pgfpathlineto{\pgfqpoint{7.411425in}{3.900370in}}%
\pgfpathlineto{\pgfqpoint{7.261522in}{3.900370in}}%
\pgfpathlineto{\pgfqpoint{7.261522in}{3.613611in}}%
\pgfpathclose%
\pgfusepath{stroke,fill}%
\end{pgfscope}%
\begin{pgfscope}%
\pgfpathrectangle{\pgfqpoint{7.074143in}{3.613611in}}{\pgfqpoint{0.824468in}{0.462000in}}%
\pgfusepath{clip}%
\pgfsetbuttcap%
\pgfsetmiterjoin%
\definecolor{currentfill}{rgb}{0.121569,0.466667,0.705882}%
\pgfsetfillcolor{currentfill}%
\pgfsetfillopacity{0.500000}%
\pgfsetlinewidth{1.003750pt}%
\definecolor{currentstroke}{rgb}{0.000000,0.000000,0.000000}%
\pgfsetstrokecolor{currentstroke}%
\pgfsetdash{}{0pt}%
\pgfpathmoveto{\pgfqpoint{7.411425in}{3.613611in}}%
\pgfpathlineto{\pgfqpoint{7.561329in}{3.613611in}}%
\pgfpathlineto{\pgfqpoint{7.561329in}{3.744094in}}%
\pgfpathlineto{\pgfqpoint{7.411425in}{3.744094in}}%
\pgfpathlineto{\pgfqpoint{7.411425in}{3.613611in}}%
\pgfpathclose%
\pgfusepath{stroke,fill}%
\end{pgfscope}%
\begin{pgfscope}%
\pgfpathrectangle{\pgfqpoint{7.074143in}{3.613611in}}{\pgfqpoint{0.824468in}{0.462000in}}%
\pgfusepath{clip}%
\pgfsetbuttcap%
\pgfsetmiterjoin%
\definecolor{currentfill}{rgb}{0.121569,0.466667,0.705882}%
\pgfsetfillcolor{currentfill}%
\pgfsetfillopacity{0.500000}%
\pgfsetlinewidth{1.003750pt}%
\definecolor{currentstroke}{rgb}{0.000000,0.000000,0.000000}%
\pgfsetstrokecolor{currentstroke}%
\pgfsetdash{}{0pt}%
\pgfpathmoveto{\pgfqpoint{7.561329in}{3.613611in}}%
\pgfpathlineto{\pgfqpoint{7.711232in}{3.613611in}}%
\pgfpathlineto{\pgfqpoint{7.711232in}{3.660646in}}%
\pgfpathlineto{\pgfqpoint{7.561329in}{3.660646in}}%
\pgfpathlineto{\pgfqpoint{7.561329in}{3.613611in}}%
\pgfpathclose%
\pgfusepath{stroke,fill}%
\end{pgfscope}%
\begin{pgfscope}%
\pgfpathrectangle{\pgfqpoint{7.074143in}{3.613611in}}{\pgfqpoint{0.824468in}{0.462000in}}%
\pgfusepath{clip}%
\pgfsetbuttcap%
\pgfsetmiterjoin%
\definecolor{currentfill}{rgb}{0.121569,0.466667,0.705882}%
\pgfsetfillcolor{currentfill}%
\pgfsetfillopacity{0.500000}%
\pgfsetlinewidth{1.003750pt}%
\definecolor{currentstroke}{rgb}{0.000000,0.000000,0.000000}%
\pgfsetstrokecolor{currentstroke}%
\pgfsetdash{}{0pt}%
\pgfpathmoveto{\pgfqpoint{7.711232in}{3.613611in}}%
\pgfpathlineto{\pgfqpoint{7.861135in}{3.613611in}}%
\pgfpathlineto{\pgfqpoint{7.861135in}{3.637887in}}%
\pgfpathlineto{\pgfqpoint{7.711232in}{3.637887in}}%
\pgfpathlineto{\pgfqpoint{7.711232in}{3.613611in}}%
\pgfpathclose%
\pgfusepath{stroke,fill}%
\end{pgfscope}%
\begin{pgfscope}%
\pgfsetrectcap%
\pgfsetmiterjoin%
\pgfsetlinewidth{0.803000pt}%
\definecolor{currentstroke}{rgb}{0.000000,0.000000,0.000000}%
\pgfsetstrokecolor{currentstroke}%
\pgfsetdash{}{0pt}%
\pgfpathmoveto{\pgfqpoint{7.074143in}{3.613611in}}%
\pgfpathlineto{\pgfqpoint{7.074143in}{4.075611in}}%
\pgfusepath{stroke}%
\end{pgfscope}%
\begin{pgfscope}%
\pgfsetrectcap%
\pgfsetmiterjoin%
\pgfsetlinewidth{0.803000pt}%
\definecolor{currentstroke}{rgb}{0.000000,0.000000,0.000000}%
\pgfsetstrokecolor{currentstroke}%
\pgfsetdash{}{0pt}%
\pgfpathmoveto{\pgfqpoint{7.898611in}{3.613611in}}%
\pgfpathlineto{\pgfqpoint{7.898611in}{4.075611in}}%
\pgfusepath{stroke}%
\end{pgfscope}%
\begin{pgfscope}%
\pgfsetrectcap%
\pgfsetmiterjoin%
\pgfsetlinewidth{0.803000pt}%
\definecolor{currentstroke}{rgb}{0.000000,0.000000,0.000000}%
\pgfsetstrokecolor{currentstroke}%
\pgfsetdash{}{0pt}%
\pgfpathmoveto{\pgfqpoint{7.074143in}{3.613611in}}%
\pgfpathlineto{\pgfqpoint{7.898611in}{3.613611in}}%
\pgfusepath{stroke}%
\end{pgfscope}%
\begin{pgfscope}%
\pgfsetrectcap%
\pgfsetmiterjoin%
\pgfsetlinewidth{0.803000pt}%
\definecolor{currentstroke}{rgb}{0.000000,0.000000,0.000000}%
\pgfsetstrokecolor{currentstroke}%
\pgfsetdash{}{0pt}%
\pgfpathmoveto{\pgfqpoint{7.074143in}{4.075611in}}%
\pgfpathlineto{\pgfqpoint{7.898611in}{4.075611in}}%
\pgfusepath{stroke}%
\end{pgfscope}%
\begin{pgfscope}%
\definecolor{textcolor}{rgb}{0.000000,0.000000,0.000000}%
\pgfsetstrokecolor{textcolor}%
\pgfsetfillcolor{textcolor}%
\pgftext[x=7.486377in,y=4.158944in,,base]{\color{textcolor}\rmfamily\fontsize{11.000000}{13.200000}\selectfont MAAF}%
\end{pgfscope}%
\begin{pgfscope}%
\pgfsetbuttcap%
\pgfsetmiterjoin%
\definecolor{currentfill}{rgb}{1.000000,1.000000,1.000000}%
\pgfsetfillcolor{currentfill}%
\pgfsetlinewidth{0.000000pt}%
\definecolor{currentstroke}{rgb}{0.000000,0.000000,0.000000}%
\pgfsetstrokecolor{currentstroke}%
\pgfsetstrokeopacity{0.000000}%
\pgfsetdash{}{0pt}%
\pgfpathmoveto{\pgfqpoint{0.148611in}{2.920611in}}%
\pgfpathlineto{\pgfqpoint{0.973079in}{2.920611in}}%
\pgfpathlineto{\pgfqpoint{0.973079in}{3.382611in}}%
\pgfpathlineto{\pgfqpoint{0.148611in}{3.382611in}}%
\pgfpathlineto{\pgfqpoint{0.148611in}{2.920611in}}%
\pgfpathclose%
\pgfusepath{fill}%
\end{pgfscope}%
\begin{pgfscope}%
\pgfpathrectangle{\pgfqpoint{0.148611in}{2.920611in}}{\pgfqpoint{0.824468in}{0.462000in}}%
\pgfusepath{clip}%
\pgfsetbuttcap%
\pgfsetmiterjoin%
\definecolor{currentfill}{rgb}{0.121569,0.466667,0.705882}%
\pgfsetfillcolor{currentfill}%
\pgfsetfillopacity{0.500000}%
\pgfsetlinewidth{1.003750pt}%
\definecolor{currentstroke}{rgb}{0.000000,0.000000,0.000000}%
\pgfsetstrokecolor{currentstroke}%
\pgfsetdash{}{0pt}%
\pgfpathmoveto{\pgfqpoint{0.186087in}{2.920611in}}%
\pgfpathlineto{\pgfqpoint{0.335990in}{2.920611in}}%
\pgfpathlineto{\pgfqpoint{0.335990in}{3.360611in}}%
\pgfpathlineto{\pgfqpoint{0.186087in}{3.360611in}}%
\pgfpathlineto{\pgfqpoint{0.186087in}{2.920611in}}%
\pgfpathclose%
\pgfusepath{stroke,fill}%
\end{pgfscope}%
\begin{pgfscope}%
\pgfpathrectangle{\pgfqpoint{0.148611in}{2.920611in}}{\pgfqpoint{0.824468in}{0.462000in}}%
\pgfusepath{clip}%
\pgfsetbuttcap%
\pgfsetmiterjoin%
\definecolor{currentfill}{rgb}{0.121569,0.466667,0.705882}%
\pgfsetfillcolor{currentfill}%
\pgfsetfillopacity{0.500000}%
\pgfsetlinewidth{1.003750pt}%
\definecolor{currentstroke}{rgb}{0.000000,0.000000,0.000000}%
\pgfsetstrokecolor{currentstroke}%
\pgfsetdash{}{0pt}%
\pgfpathmoveto{\pgfqpoint{0.335990in}{2.920611in}}%
\pgfpathlineto{\pgfqpoint{0.485894in}{2.920611in}}%
\pgfpathlineto{\pgfqpoint{0.485894in}{3.077754in}}%
\pgfpathlineto{\pgfqpoint{0.335990in}{3.077754in}}%
\pgfpathlineto{\pgfqpoint{0.335990in}{2.920611in}}%
\pgfpathclose%
\pgfusepath{stroke,fill}%
\end{pgfscope}%
\begin{pgfscope}%
\pgfpathrectangle{\pgfqpoint{0.148611in}{2.920611in}}{\pgfqpoint{0.824468in}{0.462000in}}%
\pgfusepath{clip}%
\pgfsetbuttcap%
\pgfsetmiterjoin%
\definecolor{currentfill}{rgb}{0.121569,0.466667,0.705882}%
\pgfsetfillcolor{currentfill}%
\pgfsetfillopacity{0.500000}%
\pgfsetlinewidth{1.003750pt}%
\definecolor{currentstroke}{rgb}{0.000000,0.000000,0.000000}%
\pgfsetstrokecolor{currentstroke}%
\pgfsetdash{}{0pt}%
\pgfpathmoveto{\pgfqpoint{0.485894in}{2.920611in}}%
\pgfpathlineto{\pgfqpoint{0.635797in}{2.920611in}}%
\pgfpathlineto{\pgfqpoint{0.635797in}{2.920611in}}%
\pgfpathlineto{\pgfqpoint{0.485894in}{2.920611in}}%
\pgfpathlineto{\pgfqpoint{0.485894in}{2.920611in}}%
\pgfpathclose%
\pgfusepath{stroke,fill}%
\end{pgfscope}%
\begin{pgfscope}%
\pgfpathrectangle{\pgfqpoint{0.148611in}{2.920611in}}{\pgfqpoint{0.824468in}{0.462000in}}%
\pgfusepath{clip}%
\pgfsetbuttcap%
\pgfsetmiterjoin%
\definecolor{currentfill}{rgb}{0.121569,0.466667,0.705882}%
\pgfsetfillcolor{currentfill}%
\pgfsetfillopacity{0.500000}%
\pgfsetlinewidth{1.003750pt}%
\definecolor{currentstroke}{rgb}{0.000000,0.000000,0.000000}%
\pgfsetstrokecolor{currentstroke}%
\pgfsetdash{}{0pt}%
\pgfpathmoveto{\pgfqpoint{0.635797in}{2.920611in}}%
\pgfpathlineto{\pgfqpoint{0.785700in}{2.920611in}}%
\pgfpathlineto{\pgfqpoint{0.785700in}{3.014897in}}%
\pgfpathlineto{\pgfqpoint{0.635797in}{3.014897in}}%
\pgfpathlineto{\pgfqpoint{0.635797in}{2.920611in}}%
\pgfpathclose%
\pgfusepath{stroke,fill}%
\end{pgfscope}%
\begin{pgfscope}%
\pgfpathrectangle{\pgfqpoint{0.148611in}{2.920611in}}{\pgfqpoint{0.824468in}{0.462000in}}%
\pgfusepath{clip}%
\pgfsetbuttcap%
\pgfsetmiterjoin%
\definecolor{currentfill}{rgb}{0.121569,0.466667,0.705882}%
\pgfsetfillcolor{currentfill}%
\pgfsetfillopacity{0.500000}%
\pgfsetlinewidth{1.003750pt}%
\definecolor{currentstroke}{rgb}{0.000000,0.000000,0.000000}%
\pgfsetstrokecolor{currentstroke}%
\pgfsetdash{}{0pt}%
\pgfpathmoveto{\pgfqpoint{0.785700in}{2.920611in}}%
\pgfpathlineto{\pgfqpoint{0.935603in}{2.920611in}}%
\pgfpathlineto{\pgfqpoint{0.935603in}{3.014897in}}%
\pgfpathlineto{\pgfqpoint{0.785700in}{3.014897in}}%
\pgfpathlineto{\pgfqpoint{0.785700in}{2.920611in}}%
\pgfpathclose%
\pgfusepath{stroke,fill}%
\end{pgfscope}%
\begin{pgfscope}%
\pgfsetrectcap%
\pgfsetmiterjoin%
\pgfsetlinewidth{0.803000pt}%
\definecolor{currentstroke}{rgb}{0.000000,0.000000,0.000000}%
\pgfsetstrokecolor{currentstroke}%
\pgfsetdash{}{0pt}%
\pgfpathmoveto{\pgfqpoint{0.148611in}{2.920611in}}%
\pgfpathlineto{\pgfqpoint{0.148611in}{3.382611in}}%
\pgfusepath{stroke}%
\end{pgfscope}%
\begin{pgfscope}%
\pgfsetrectcap%
\pgfsetmiterjoin%
\pgfsetlinewidth{0.803000pt}%
\definecolor{currentstroke}{rgb}{0.000000,0.000000,0.000000}%
\pgfsetstrokecolor{currentstroke}%
\pgfsetdash{}{0pt}%
\pgfpathmoveto{\pgfqpoint{0.973079in}{2.920611in}}%
\pgfpathlineto{\pgfqpoint{0.973079in}{3.382611in}}%
\pgfusepath{stroke}%
\end{pgfscope}%
\begin{pgfscope}%
\pgfsetrectcap%
\pgfsetmiterjoin%
\pgfsetlinewidth{0.803000pt}%
\definecolor{currentstroke}{rgb}{0.000000,0.000000,0.000000}%
\pgfsetstrokecolor{currentstroke}%
\pgfsetdash{}{0pt}%
\pgfpathmoveto{\pgfqpoint{0.148611in}{2.920611in}}%
\pgfpathlineto{\pgfqpoint{0.973079in}{2.920611in}}%
\pgfusepath{stroke}%
\end{pgfscope}%
\begin{pgfscope}%
\pgfsetrectcap%
\pgfsetmiterjoin%
\pgfsetlinewidth{0.803000pt}%
\definecolor{currentstroke}{rgb}{0.000000,0.000000,0.000000}%
\pgfsetstrokecolor{currentstroke}%
\pgfsetdash{}{0pt}%
\pgfpathmoveto{\pgfqpoint{0.148611in}{3.382611in}}%
\pgfpathlineto{\pgfqpoint{0.973079in}{3.382611in}}%
\pgfusepath{stroke}%
\end{pgfscope}%
\begin{pgfscope}%
\definecolor{textcolor}{rgb}{0.000000,0.000000,0.000000}%
\pgfsetstrokecolor{textcolor}%
\pgfsetfillcolor{textcolor}%
\pgftext[x=0.560845in,y=3.465944in,,base]{\color{textcolor}\rmfamily\fontsize{11.000000}{13.200000}\selectfont Solly ...}%
\end{pgfscope}%
\begin{pgfscope}%
\pgfsetbuttcap%
\pgfsetmiterjoin%
\definecolor{currentfill}{rgb}{1.000000,1.000000,1.000000}%
\pgfsetfillcolor{currentfill}%
\pgfsetlinewidth{0.000000pt}%
\definecolor{currentstroke}{rgb}{0.000000,0.000000,0.000000}%
\pgfsetstrokecolor{currentstroke}%
\pgfsetstrokeopacity{0.000000}%
\pgfsetdash{}{0pt}%
\pgfpathmoveto{\pgfqpoint{1.137973in}{2.920611in}}%
\pgfpathlineto{\pgfqpoint{1.962441in}{2.920611in}}%
\pgfpathlineto{\pgfqpoint{1.962441in}{3.382611in}}%
\pgfpathlineto{\pgfqpoint{1.137973in}{3.382611in}}%
\pgfpathlineto{\pgfqpoint{1.137973in}{2.920611in}}%
\pgfpathclose%
\pgfusepath{fill}%
\end{pgfscope}%
\begin{pgfscope}%
\pgfpathrectangle{\pgfqpoint{1.137973in}{2.920611in}}{\pgfqpoint{0.824468in}{0.462000in}}%
\pgfusepath{clip}%
\pgfsetbuttcap%
\pgfsetmiterjoin%
\definecolor{currentfill}{rgb}{0.121569,0.466667,0.705882}%
\pgfsetfillcolor{currentfill}%
\pgfsetfillopacity{0.500000}%
\pgfsetlinewidth{1.003750pt}%
\definecolor{currentstroke}{rgb}{0.000000,0.000000,0.000000}%
\pgfsetstrokecolor{currentstroke}%
\pgfsetdash{}{0pt}%
\pgfpathmoveto{\pgfqpoint{1.175449in}{2.920611in}}%
\pgfpathlineto{\pgfqpoint{1.325352in}{2.920611in}}%
\pgfpathlineto{\pgfqpoint{1.325352in}{3.337351in}}%
\pgfpathlineto{\pgfqpoint{1.175449in}{3.337351in}}%
\pgfpathlineto{\pgfqpoint{1.175449in}{2.920611in}}%
\pgfpathclose%
\pgfusepath{stroke,fill}%
\end{pgfscope}%
\begin{pgfscope}%
\pgfpathrectangle{\pgfqpoint{1.137973in}{2.920611in}}{\pgfqpoint{0.824468in}{0.462000in}}%
\pgfusepath{clip}%
\pgfsetbuttcap%
\pgfsetmiterjoin%
\definecolor{currentfill}{rgb}{0.121569,0.466667,0.705882}%
\pgfsetfillcolor{currentfill}%
\pgfsetfillopacity{0.500000}%
\pgfsetlinewidth{1.003750pt}%
\definecolor{currentstroke}{rgb}{0.000000,0.000000,0.000000}%
\pgfsetstrokecolor{currentstroke}%
\pgfsetdash{}{0pt}%
\pgfpathmoveto{\pgfqpoint{1.325352in}{2.920611in}}%
\pgfpathlineto{\pgfqpoint{1.475255in}{2.920611in}}%
\pgfpathlineto{\pgfqpoint{1.475255in}{3.348981in}}%
\pgfpathlineto{\pgfqpoint{1.325352in}{3.348981in}}%
\pgfpathlineto{\pgfqpoint{1.325352in}{2.920611in}}%
\pgfpathclose%
\pgfusepath{stroke,fill}%
\end{pgfscope}%
\begin{pgfscope}%
\pgfpathrectangle{\pgfqpoint{1.137973in}{2.920611in}}{\pgfqpoint{0.824468in}{0.462000in}}%
\pgfusepath{clip}%
\pgfsetbuttcap%
\pgfsetmiterjoin%
\definecolor{currentfill}{rgb}{0.121569,0.466667,0.705882}%
\pgfsetfillcolor{currentfill}%
\pgfsetfillopacity{0.500000}%
\pgfsetlinewidth{1.003750pt}%
\definecolor{currentstroke}{rgb}{0.000000,0.000000,0.000000}%
\pgfsetstrokecolor{currentstroke}%
\pgfsetdash{}{0pt}%
\pgfpathmoveto{\pgfqpoint{1.475255in}{2.920611in}}%
\pgfpathlineto{\pgfqpoint{1.625158in}{2.920611in}}%
\pgfpathlineto{\pgfqpoint{1.625158in}{3.269510in}}%
\pgfpathlineto{\pgfqpoint{1.475255in}{3.269510in}}%
\pgfpathlineto{\pgfqpoint{1.475255in}{2.920611in}}%
\pgfpathclose%
\pgfusepath{stroke,fill}%
\end{pgfscope}%
\begin{pgfscope}%
\pgfpathrectangle{\pgfqpoint{1.137973in}{2.920611in}}{\pgfqpoint{0.824468in}{0.462000in}}%
\pgfusepath{clip}%
\pgfsetbuttcap%
\pgfsetmiterjoin%
\definecolor{currentfill}{rgb}{0.121569,0.466667,0.705882}%
\pgfsetfillcolor{currentfill}%
\pgfsetfillopacity{0.500000}%
\pgfsetlinewidth{1.003750pt}%
\definecolor{currentstroke}{rgb}{0.000000,0.000000,0.000000}%
\pgfsetstrokecolor{currentstroke}%
\pgfsetdash{}{0pt}%
\pgfpathmoveto{\pgfqpoint{1.625158in}{2.920611in}}%
\pgfpathlineto{\pgfqpoint{1.775062in}{2.920611in}}%
\pgfpathlineto{\pgfqpoint{1.775062in}{3.360611in}}%
\pgfpathlineto{\pgfqpoint{1.625158in}{3.360611in}}%
\pgfpathlineto{\pgfqpoint{1.625158in}{2.920611in}}%
\pgfpathclose%
\pgfusepath{stroke,fill}%
\end{pgfscope}%
\begin{pgfscope}%
\pgfpathrectangle{\pgfqpoint{1.137973in}{2.920611in}}{\pgfqpoint{0.824468in}{0.462000in}}%
\pgfusepath{clip}%
\pgfsetbuttcap%
\pgfsetmiterjoin%
\definecolor{currentfill}{rgb}{0.121569,0.466667,0.705882}%
\pgfsetfillcolor{currentfill}%
\pgfsetfillopacity{0.500000}%
\pgfsetlinewidth{1.003750pt}%
\definecolor{currentstroke}{rgb}{0.000000,0.000000,0.000000}%
\pgfsetstrokecolor{currentstroke}%
\pgfsetdash{}{0pt}%
\pgfpathmoveto{\pgfqpoint{1.775062in}{2.920611in}}%
\pgfpathlineto{\pgfqpoint{1.924965in}{2.920611in}}%
\pgfpathlineto{\pgfqpoint{1.924965in}{3.221052in}}%
\pgfpathlineto{\pgfqpoint{1.775062in}{3.221052in}}%
\pgfpathlineto{\pgfqpoint{1.775062in}{2.920611in}}%
\pgfpathclose%
\pgfusepath{stroke,fill}%
\end{pgfscope}%
\begin{pgfscope}%
\pgfsetrectcap%
\pgfsetmiterjoin%
\pgfsetlinewidth{0.803000pt}%
\definecolor{currentstroke}{rgb}{0.000000,0.000000,0.000000}%
\pgfsetstrokecolor{currentstroke}%
\pgfsetdash{}{0pt}%
\pgfpathmoveto{\pgfqpoint{1.137973in}{2.920611in}}%
\pgfpathlineto{\pgfqpoint{1.137973in}{3.382611in}}%
\pgfusepath{stroke}%
\end{pgfscope}%
\begin{pgfscope}%
\pgfsetrectcap%
\pgfsetmiterjoin%
\pgfsetlinewidth{0.803000pt}%
\definecolor{currentstroke}{rgb}{0.000000,0.000000,0.000000}%
\pgfsetstrokecolor{currentstroke}%
\pgfsetdash{}{0pt}%
\pgfpathmoveto{\pgfqpoint{1.962441in}{2.920611in}}%
\pgfpathlineto{\pgfqpoint{1.962441in}{3.382611in}}%
\pgfusepath{stroke}%
\end{pgfscope}%
\begin{pgfscope}%
\pgfsetrectcap%
\pgfsetmiterjoin%
\pgfsetlinewidth{0.803000pt}%
\definecolor{currentstroke}{rgb}{0.000000,0.000000,0.000000}%
\pgfsetstrokecolor{currentstroke}%
\pgfsetdash{}{0pt}%
\pgfpathmoveto{\pgfqpoint{1.137973in}{2.920611in}}%
\pgfpathlineto{\pgfqpoint{1.962441in}{2.920611in}}%
\pgfusepath{stroke}%
\end{pgfscope}%
\begin{pgfscope}%
\pgfsetrectcap%
\pgfsetmiterjoin%
\pgfsetlinewidth{0.803000pt}%
\definecolor{currentstroke}{rgb}{0.000000,0.000000,0.000000}%
\pgfsetstrokecolor{currentstroke}%
\pgfsetdash{}{0pt}%
\pgfpathmoveto{\pgfqpoint{1.137973in}{3.382611in}}%
\pgfpathlineto{\pgfqpoint{1.962441in}{3.382611in}}%
\pgfusepath{stroke}%
\end{pgfscope}%
\begin{pgfscope}%
\definecolor{textcolor}{rgb}{0.000000,0.000000,0.000000}%
\pgfsetstrokecolor{textcolor}%
\pgfsetfillcolor{textcolor}%
\pgftext[x=1.550207in,y=3.465944in,,base]{\color{textcolor}\rmfamily\fontsize{11.000000}{13.200000}\selectfont GMF}%
\end{pgfscope}%
\begin{pgfscope}%
\pgfsetbuttcap%
\pgfsetmiterjoin%
\definecolor{currentfill}{rgb}{1.000000,1.000000,1.000000}%
\pgfsetfillcolor{currentfill}%
\pgfsetlinewidth{0.000000pt}%
\definecolor{currentstroke}{rgb}{0.000000,0.000000,0.000000}%
\pgfsetstrokecolor{currentstroke}%
\pgfsetstrokeopacity{0.000000}%
\pgfsetdash{}{0pt}%
\pgfpathmoveto{\pgfqpoint{2.127335in}{2.920611in}}%
\pgfpathlineto{\pgfqpoint{2.951803in}{2.920611in}}%
\pgfpathlineto{\pgfqpoint{2.951803in}{3.382611in}}%
\pgfpathlineto{\pgfqpoint{2.127335in}{3.382611in}}%
\pgfpathlineto{\pgfqpoint{2.127335in}{2.920611in}}%
\pgfpathclose%
\pgfusepath{fill}%
\end{pgfscope}%
\begin{pgfscope}%
\pgfpathrectangle{\pgfqpoint{2.127335in}{2.920611in}}{\pgfqpoint{0.824468in}{0.462000in}}%
\pgfusepath{clip}%
\pgfsetbuttcap%
\pgfsetmiterjoin%
\definecolor{currentfill}{rgb}{0.121569,0.466667,0.705882}%
\pgfsetfillcolor{currentfill}%
\pgfsetfillopacity{0.500000}%
\pgfsetlinewidth{1.003750pt}%
\definecolor{currentstroke}{rgb}{0.000000,0.000000,0.000000}%
\pgfsetstrokecolor{currentstroke}%
\pgfsetdash{}{0pt}%
\pgfpathmoveto{\pgfqpoint{2.164810in}{2.920611in}}%
\pgfpathlineto{\pgfqpoint{2.314714in}{2.920611in}}%
\pgfpathlineto{\pgfqpoint{2.314714in}{3.044886in}}%
\pgfpathlineto{\pgfqpoint{2.164810in}{3.044886in}}%
\pgfpathlineto{\pgfqpoint{2.164810in}{2.920611in}}%
\pgfpathclose%
\pgfusepath{stroke,fill}%
\end{pgfscope}%
\begin{pgfscope}%
\pgfpathrectangle{\pgfqpoint{2.127335in}{2.920611in}}{\pgfqpoint{0.824468in}{0.462000in}}%
\pgfusepath{clip}%
\pgfsetbuttcap%
\pgfsetmiterjoin%
\definecolor{currentfill}{rgb}{0.121569,0.466667,0.705882}%
\pgfsetfillcolor{currentfill}%
\pgfsetfillopacity{0.500000}%
\pgfsetlinewidth{1.003750pt}%
\definecolor{currentstroke}{rgb}{0.000000,0.000000,0.000000}%
\pgfsetstrokecolor{currentstroke}%
\pgfsetdash{}{0pt}%
\pgfpathmoveto{\pgfqpoint{2.314714in}{2.920611in}}%
\pgfpathlineto{\pgfqpoint{2.464617in}{2.920611in}}%
\pgfpathlineto{\pgfqpoint{2.464617in}{3.065039in}}%
\pgfpathlineto{\pgfqpoint{2.314714in}{3.065039in}}%
\pgfpathlineto{\pgfqpoint{2.314714in}{2.920611in}}%
\pgfpathclose%
\pgfusepath{stroke,fill}%
\end{pgfscope}%
\begin{pgfscope}%
\pgfpathrectangle{\pgfqpoint{2.127335in}{2.920611in}}{\pgfqpoint{0.824468in}{0.462000in}}%
\pgfusepath{clip}%
\pgfsetbuttcap%
\pgfsetmiterjoin%
\definecolor{currentfill}{rgb}{0.121569,0.466667,0.705882}%
\pgfsetfillcolor{currentfill}%
\pgfsetfillopacity{0.500000}%
\pgfsetlinewidth{1.003750pt}%
\definecolor{currentstroke}{rgb}{0.000000,0.000000,0.000000}%
\pgfsetstrokecolor{currentstroke}%
\pgfsetdash{}{0pt}%
\pgfpathmoveto{\pgfqpoint{2.464617in}{2.920611in}}%
\pgfpathlineto{\pgfqpoint{2.614520in}{2.920611in}}%
\pgfpathlineto{\pgfqpoint{2.614520in}{3.054962in}}%
\pgfpathlineto{\pgfqpoint{2.464617in}{3.054962in}}%
\pgfpathlineto{\pgfqpoint{2.464617in}{2.920611in}}%
\pgfpathclose%
\pgfusepath{stroke,fill}%
\end{pgfscope}%
\begin{pgfscope}%
\pgfpathrectangle{\pgfqpoint{2.127335in}{2.920611in}}{\pgfqpoint{0.824468in}{0.462000in}}%
\pgfusepath{clip}%
\pgfsetbuttcap%
\pgfsetmiterjoin%
\definecolor{currentfill}{rgb}{0.121569,0.466667,0.705882}%
\pgfsetfillcolor{currentfill}%
\pgfsetfillopacity{0.500000}%
\pgfsetlinewidth{1.003750pt}%
\definecolor{currentstroke}{rgb}{0.000000,0.000000,0.000000}%
\pgfsetstrokecolor{currentstroke}%
\pgfsetdash{}{0pt}%
\pgfpathmoveto{\pgfqpoint{2.614520in}{2.920611in}}%
\pgfpathlineto{\pgfqpoint{2.764423in}{2.920611in}}%
\pgfpathlineto{\pgfqpoint{2.764423in}{3.232978in}}%
\pgfpathlineto{\pgfqpoint{2.614520in}{3.232978in}}%
\pgfpathlineto{\pgfqpoint{2.614520in}{2.920611in}}%
\pgfpathclose%
\pgfusepath{stroke,fill}%
\end{pgfscope}%
\begin{pgfscope}%
\pgfpathrectangle{\pgfqpoint{2.127335in}{2.920611in}}{\pgfqpoint{0.824468in}{0.462000in}}%
\pgfusepath{clip}%
\pgfsetbuttcap%
\pgfsetmiterjoin%
\definecolor{currentfill}{rgb}{0.121569,0.466667,0.705882}%
\pgfsetfillcolor{currentfill}%
\pgfsetfillopacity{0.500000}%
\pgfsetlinewidth{1.003750pt}%
\definecolor{currentstroke}{rgb}{0.000000,0.000000,0.000000}%
\pgfsetstrokecolor{currentstroke}%
\pgfsetdash{}{0pt}%
\pgfpathmoveto{\pgfqpoint{2.764423in}{2.920611in}}%
\pgfpathlineto{\pgfqpoint{2.914327in}{2.920611in}}%
\pgfpathlineto{\pgfqpoint{2.914327in}{3.360611in}}%
\pgfpathlineto{\pgfqpoint{2.764423in}{3.360611in}}%
\pgfpathlineto{\pgfqpoint{2.764423in}{2.920611in}}%
\pgfpathclose%
\pgfusepath{stroke,fill}%
\end{pgfscope}%
\begin{pgfscope}%
\pgfsetrectcap%
\pgfsetmiterjoin%
\pgfsetlinewidth{0.803000pt}%
\definecolor{currentstroke}{rgb}{0.000000,0.000000,0.000000}%
\pgfsetstrokecolor{currentstroke}%
\pgfsetdash{}{0pt}%
\pgfpathmoveto{\pgfqpoint{2.127335in}{2.920611in}}%
\pgfpathlineto{\pgfqpoint{2.127335in}{3.382611in}}%
\pgfusepath{stroke}%
\end{pgfscope}%
\begin{pgfscope}%
\pgfsetrectcap%
\pgfsetmiterjoin%
\pgfsetlinewidth{0.803000pt}%
\definecolor{currentstroke}{rgb}{0.000000,0.000000,0.000000}%
\pgfsetstrokecolor{currentstroke}%
\pgfsetdash{}{0pt}%
\pgfpathmoveto{\pgfqpoint{2.951803in}{2.920611in}}%
\pgfpathlineto{\pgfqpoint{2.951803in}{3.382611in}}%
\pgfusepath{stroke}%
\end{pgfscope}%
\begin{pgfscope}%
\pgfsetrectcap%
\pgfsetmiterjoin%
\pgfsetlinewidth{0.803000pt}%
\definecolor{currentstroke}{rgb}{0.000000,0.000000,0.000000}%
\pgfsetstrokecolor{currentstroke}%
\pgfsetdash{}{0pt}%
\pgfpathmoveto{\pgfqpoint{2.127335in}{2.920611in}}%
\pgfpathlineto{\pgfqpoint{2.951803in}{2.920611in}}%
\pgfusepath{stroke}%
\end{pgfscope}%
\begin{pgfscope}%
\pgfsetrectcap%
\pgfsetmiterjoin%
\pgfsetlinewidth{0.803000pt}%
\definecolor{currentstroke}{rgb}{0.000000,0.000000,0.000000}%
\pgfsetstrokecolor{currentstroke}%
\pgfsetdash{}{0pt}%
\pgfpathmoveto{\pgfqpoint{2.127335in}{3.382611in}}%
\pgfpathlineto{\pgfqpoint{2.951803in}{3.382611in}}%
\pgfusepath{stroke}%
\end{pgfscope}%
\begin{pgfscope}%
\definecolor{textcolor}{rgb}{0.000000,0.000000,0.000000}%
\pgfsetstrokecolor{textcolor}%
\pgfsetfillcolor{textcolor}%
\pgftext[x=2.539569in,y=3.465944in,,base]{\color{textcolor}\rmfamily\fontsize{11.000000}{13.200000}\selectfont AMV}%
\end{pgfscope}%
\begin{pgfscope}%
\pgfsetbuttcap%
\pgfsetmiterjoin%
\definecolor{currentfill}{rgb}{1.000000,1.000000,1.000000}%
\pgfsetfillcolor{currentfill}%
\pgfsetlinewidth{0.000000pt}%
\definecolor{currentstroke}{rgb}{0.000000,0.000000,0.000000}%
\pgfsetstrokecolor{currentstroke}%
\pgfsetstrokeopacity{0.000000}%
\pgfsetdash{}{0pt}%
\pgfpathmoveto{\pgfqpoint{3.116696in}{2.920611in}}%
\pgfpathlineto{\pgfqpoint{3.941164in}{2.920611in}}%
\pgfpathlineto{\pgfqpoint{3.941164in}{3.382611in}}%
\pgfpathlineto{\pgfqpoint{3.116696in}{3.382611in}}%
\pgfpathlineto{\pgfqpoint{3.116696in}{2.920611in}}%
\pgfpathclose%
\pgfusepath{fill}%
\end{pgfscope}%
\begin{pgfscope}%
\pgfpathrectangle{\pgfqpoint{3.116696in}{2.920611in}}{\pgfqpoint{0.824468in}{0.462000in}}%
\pgfusepath{clip}%
\pgfsetbuttcap%
\pgfsetmiterjoin%
\definecolor{currentfill}{rgb}{0.121569,0.466667,0.705882}%
\pgfsetfillcolor{currentfill}%
\pgfsetfillopacity{0.500000}%
\pgfsetlinewidth{1.003750pt}%
\definecolor{currentstroke}{rgb}{0.000000,0.000000,0.000000}%
\pgfsetstrokecolor{currentstroke}%
\pgfsetdash{}{0pt}%
\pgfpathmoveto{\pgfqpoint{3.154172in}{2.920611in}}%
\pgfpathlineto{\pgfqpoint{3.304075in}{2.920611in}}%
\pgfpathlineto{\pgfqpoint{3.304075in}{3.360611in}}%
\pgfpathlineto{\pgfqpoint{3.154172in}{3.360611in}}%
\pgfpathlineto{\pgfqpoint{3.154172in}{2.920611in}}%
\pgfpathclose%
\pgfusepath{stroke,fill}%
\end{pgfscope}%
\begin{pgfscope}%
\pgfpathrectangle{\pgfqpoint{3.116696in}{2.920611in}}{\pgfqpoint{0.824468in}{0.462000in}}%
\pgfusepath{clip}%
\pgfsetbuttcap%
\pgfsetmiterjoin%
\definecolor{currentfill}{rgb}{0.121569,0.466667,0.705882}%
\pgfsetfillcolor{currentfill}%
\pgfsetfillopacity{0.500000}%
\pgfsetlinewidth{1.003750pt}%
\definecolor{currentstroke}{rgb}{0.000000,0.000000,0.000000}%
\pgfsetstrokecolor{currentstroke}%
\pgfsetdash{}{0pt}%
\pgfpathmoveto{\pgfqpoint{3.304075in}{2.920611in}}%
\pgfpathlineto{\pgfqpoint{3.453979in}{2.920611in}}%
\pgfpathlineto{\pgfqpoint{3.453979in}{2.987998in}}%
\pgfpathlineto{\pgfqpoint{3.304075in}{2.987998in}}%
\pgfpathlineto{\pgfqpoint{3.304075in}{2.920611in}}%
\pgfpathclose%
\pgfusepath{stroke,fill}%
\end{pgfscope}%
\begin{pgfscope}%
\pgfpathrectangle{\pgfqpoint{3.116696in}{2.920611in}}{\pgfqpoint{0.824468in}{0.462000in}}%
\pgfusepath{clip}%
\pgfsetbuttcap%
\pgfsetmiterjoin%
\definecolor{currentfill}{rgb}{0.121569,0.466667,0.705882}%
\pgfsetfillcolor{currentfill}%
\pgfsetfillopacity{0.500000}%
\pgfsetlinewidth{1.003750pt}%
\definecolor{currentstroke}{rgb}{0.000000,0.000000,0.000000}%
\pgfsetstrokecolor{currentstroke}%
\pgfsetdash{}{0pt}%
\pgfpathmoveto{\pgfqpoint{3.453979in}{2.920611in}}%
\pgfpathlineto{\pgfqpoint{3.603882in}{2.920611in}}%
\pgfpathlineto{\pgfqpoint{3.603882in}{2.980071in}}%
\pgfpathlineto{\pgfqpoint{3.453979in}{2.980071in}}%
\pgfpathlineto{\pgfqpoint{3.453979in}{2.920611in}}%
\pgfpathclose%
\pgfusepath{stroke,fill}%
\end{pgfscope}%
\begin{pgfscope}%
\pgfpathrectangle{\pgfqpoint{3.116696in}{2.920611in}}{\pgfqpoint{0.824468in}{0.462000in}}%
\pgfusepath{clip}%
\pgfsetbuttcap%
\pgfsetmiterjoin%
\definecolor{currentfill}{rgb}{0.121569,0.466667,0.705882}%
\pgfsetfillcolor{currentfill}%
\pgfsetfillopacity{0.500000}%
\pgfsetlinewidth{1.003750pt}%
\definecolor{currentstroke}{rgb}{0.000000,0.000000,0.000000}%
\pgfsetstrokecolor{currentstroke}%
\pgfsetdash{}{0pt}%
\pgfpathmoveto{\pgfqpoint{3.603882in}{2.920611in}}%
\pgfpathlineto{\pgfqpoint{3.753785in}{2.920611in}}%
\pgfpathlineto{\pgfqpoint{3.753785in}{2.920611in}}%
\pgfpathlineto{\pgfqpoint{3.603882in}{2.920611in}}%
\pgfpathlineto{\pgfqpoint{3.603882in}{2.920611in}}%
\pgfpathclose%
\pgfusepath{stroke,fill}%
\end{pgfscope}%
\begin{pgfscope}%
\pgfpathrectangle{\pgfqpoint{3.116696in}{2.920611in}}{\pgfqpoint{0.824468in}{0.462000in}}%
\pgfusepath{clip}%
\pgfsetbuttcap%
\pgfsetmiterjoin%
\definecolor{currentfill}{rgb}{0.121569,0.466667,0.705882}%
\pgfsetfillcolor{currentfill}%
\pgfsetfillopacity{0.500000}%
\pgfsetlinewidth{1.003750pt}%
\definecolor{currentstroke}{rgb}{0.000000,0.000000,0.000000}%
\pgfsetstrokecolor{currentstroke}%
\pgfsetdash{}{0pt}%
\pgfpathmoveto{\pgfqpoint{3.753785in}{2.920611in}}%
\pgfpathlineto{\pgfqpoint{3.903688in}{2.920611in}}%
\pgfpathlineto{\pgfqpoint{3.903688in}{2.936467in}}%
\pgfpathlineto{\pgfqpoint{3.753785in}{2.936467in}}%
\pgfpathlineto{\pgfqpoint{3.753785in}{2.920611in}}%
\pgfpathclose%
\pgfusepath{stroke,fill}%
\end{pgfscope}%
\begin{pgfscope}%
\pgfsetrectcap%
\pgfsetmiterjoin%
\pgfsetlinewidth{0.803000pt}%
\definecolor{currentstroke}{rgb}{0.000000,0.000000,0.000000}%
\pgfsetstrokecolor{currentstroke}%
\pgfsetdash{}{0pt}%
\pgfpathmoveto{\pgfqpoint{3.116696in}{2.920611in}}%
\pgfpathlineto{\pgfqpoint{3.116696in}{3.382611in}}%
\pgfusepath{stroke}%
\end{pgfscope}%
\begin{pgfscope}%
\pgfsetrectcap%
\pgfsetmiterjoin%
\pgfsetlinewidth{0.803000pt}%
\definecolor{currentstroke}{rgb}{0.000000,0.000000,0.000000}%
\pgfsetstrokecolor{currentstroke}%
\pgfsetdash{}{0pt}%
\pgfpathmoveto{\pgfqpoint{3.941164in}{2.920611in}}%
\pgfpathlineto{\pgfqpoint{3.941164in}{3.382611in}}%
\pgfusepath{stroke}%
\end{pgfscope}%
\begin{pgfscope}%
\pgfsetrectcap%
\pgfsetmiterjoin%
\pgfsetlinewidth{0.803000pt}%
\definecolor{currentstroke}{rgb}{0.000000,0.000000,0.000000}%
\pgfsetstrokecolor{currentstroke}%
\pgfsetdash{}{0pt}%
\pgfpathmoveto{\pgfqpoint{3.116696in}{2.920611in}}%
\pgfpathlineto{\pgfqpoint{3.941164in}{2.920611in}}%
\pgfusepath{stroke}%
\end{pgfscope}%
\begin{pgfscope}%
\pgfsetrectcap%
\pgfsetmiterjoin%
\pgfsetlinewidth{0.803000pt}%
\definecolor{currentstroke}{rgb}{0.000000,0.000000,0.000000}%
\pgfsetstrokecolor{currentstroke}%
\pgfsetdash{}{0pt}%
\pgfpathmoveto{\pgfqpoint{3.116696in}{3.382611in}}%
\pgfpathlineto{\pgfqpoint{3.941164in}{3.382611in}}%
\pgfusepath{stroke}%
\end{pgfscope}%
\begin{pgfscope}%
\definecolor{textcolor}{rgb}{0.000000,0.000000,0.000000}%
\pgfsetstrokecolor{textcolor}%
\pgfsetfillcolor{textcolor}%
\pgftext[x=3.528930in,y=3.465944in,,base]{\color{textcolor}\rmfamily\fontsize{11.000000}{13.200000}\selectfont CNP As...}%
\end{pgfscope}%
\begin{pgfscope}%
\pgfsetbuttcap%
\pgfsetmiterjoin%
\definecolor{currentfill}{rgb}{1.000000,1.000000,1.000000}%
\pgfsetfillcolor{currentfill}%
\pgfsetlinewidth{0.000000pt}%
\definecolor{currentstroke}{rgb}{0.000000,0.000000,0.000000}%
\pgfsetstrokecolor{currentstroke}%
\pgfsetstrokeopacity{0.000000}%
\pgfsetdash{}{0pt}%
\pgfpathmoveto{\pgfqpoint{4.106058in}{2.920611in}}%
\pgfpathlineto{\pgfqpoint{4.930526in}{2.920611in}}%
\pgfpathlineto{\pgfqpoint{4.930526in}{3.382611in}}%
\pgfpathlineto{\pgfqpoint{4.106058in}{3.382611in}}%
\pgfpathlineto{\pgfqpoint{4.106058in}{2.920611in}}%
\pgfpathclose%
\pgfusepath{fill}%
\end{pgfscope}%
\begin{pgfscope}%
\pgfpathrectangle{\pgfqpoint{4.106058in}{2.920611in}}{\pgfqpoint{0.824468in}{0.462000in}}%
\pgfusepath{clip}%
\pgfsetbuttcap%
\pgfsetmiterjoin%
\definecolor{currentfill}{rgb}{0.121569,0.466667,0.705882}%
\pgfsetfillcolor{currentfill}%
\pgfsetfillopacity{0.500000}%
\pgfsetlinewidth{1.003750pt}%
\definecolor{currentstroke}{rgb}{0.000000,0.000000,0.000000}%
\pgfsetstrokecolor{currentstroke}%
\pgfsetdash{}{0pt}%
\pgfpathmoveto{\pgfqpoint{4.143534in}{2.920611in}}%
\pgfpathlineto{\pgfqpoint{4.293437in}{2.920611in}}%
\pgfpathlineto{\pgfqpoint{4.293437in}{3.360611in}}%
\pgfpathlineto{\pgfqpoint{4.143534in}{3.360611in}}%
\pgfpathlineto{\pgfqpoint{4.143534in}{2.920611in}}%
\pgfpathclose%
\pgfusepath{stroke,fill}%
\end{pgfscope}%
\begin{pgfscope}%
\pgfpathrectangle{\pgfqpoint{4.106058in}{2.920611in}}{\pgfqpoint{0.824468in}{0.462000in}}%
\pgfusepath{clip}%
\pgfsetbuttcap%
\pgfsetmiterjoin%
\definecolor{currentfill}{rgb}{0.121569,0.466667,0.705882}%
\pgfsetfillcolor{currentfill}%
\pgfsetfillopacity{0.500000}%
\pgfsetlinewidth{1.003750pt}%
\definecolor{currentstroke}{rgb}{0.000000,0.000000,0.000000}%
\pgfsetstrokecolor{currentstroke}%
\pgfsetdash{}{0pt}%
\pgfpathmoveto{\pgfqpoint{4.293437in}{2.920611in}}%
\pgfpathlineto{\pgfqpoint{4.443340in}{2.920611in}}%
\pgfpathlineto{\pgfqpoint{4.443340in}{3.123796in}}%
\pgfpathlineto{\pgfqpoint{4.293437in}{3.123796in}}%
\pgfpathlineto{\pgfqpoint{4.293437in}{2.920611in}}%
\pgfpathclose%
\pgfusepath{stroke,fill}%
\end{pgfscope}%
\begin{pgfscope}%
\pgfpathrectangle{\pgfqpoint{4.106058in}{2.920611in}}{\pgfqpoint{0.824468in}{0.462000in}}%
\pgfusepath{clip}%
\pgfsetbuttcap%
\pgfsetmiterjoin%
\definecolor{currentfill}{rgb}{0.121569,0.466667,0.705882}%
\pgfsetfillcolor{currentfill}%
\pgfsetfillopacity{0.500000}%
\pgfsetlinewidth{1.003750pt}%
\definecolor{currentstroke}{rgb}{0.000000,0.000000,0.000000}%
\pgfsetstrokecolor{currentstroke}%
\pgfsetdash{}{0pt}%
\pgfpathmoveto{\pgfqpoint{4.443340in}{2.920611in}}%
\pgfpathlineto{\pgfqpoint{4.593244in}{2.920611in}}%
\pgfpathlineto{\pgfqpoint{4.593244in}{3.017299in}}%
\pgfpathlineto{\pgfqpoint{4.443340in}{3.017299in}}%
\pgfpathlineto{\pgfqpoint{4.443340in}{2.920611in}}%
\pgfpathclose%
\pgfusepath{stroke,fill}%
\end{pgfscope}%
\begin{pgfscope}%
\pgfpathrectangle{\pgfqpoint{4.106058in}{2.920611in}}{\pgfqpoint{0.824468in}{0.462000in}}%
\pgfusepath{clip}%
\pgfsetbuttcap%
\pgfsetmiterjoin%
\definecolor{currentfill}{rgb}{0.121569,0.466667,0.705882}%
\pgfsetfillcolor{currentfill}%
\pgfsetfillopacity{0.500000}%
\pgfsetlinewidth{1.003750pt}%
\definecolor{currentstroke}{rgb}{0.000000,0.000000,0.000000}%
\pgfsetstrokecolor{currentstroke}%
\pgfsetdash{}{0pt}%
\pgfpathmoveto{\pgfqpoint{4.593244in}{2.920611in}}%
\pgfpathlineto{\pgfqpoint{4.743147in}{2.920611in}}%
\pgfpathlineto{\pgfqpoint{4.743147in}{2.978063in}}%
\pgfpathlineto{\pgfqpoint{4.593244in}{2.978063in}}%
\pgfpathlineto{\pgfqpoint{4.593244in}{2.920611in}}%
\pgfpathclose%
\pgfusepath{stroke,fill}%
\end{pgfscope}%
\begin{pgfscope}%
\pgfpathrectangle{\pgfqpoint{4.106058in}{2.920611in}}{\pgfqpoint{0.824468in}{0.462000in}}%
\pgfusepath{clip}%
\pgfsetbuttcap%
\pgfsetmiterjoin%
\definecolor{currentfill}{rgb}{0.121569,0.466667,0.705882}%
\pgfsetfillcolor{currentfill}%
\pgfsetfillopacity{0.500000}%
\pgfsetlinewidth{1.003750pt}%
\definecolor{currentstroke}{rgb}{0.000000,0.000000,0.000000}%
\pgfsetstrokecolor{currentstroke}%
\pgfsetdash{}{0pt}%
\pgfpathmoveto{\pgfqpoint{4.743147in}{2.920611in}}%
\pgfpathlineto{\pgfqpoint{4.893050in}{2.920611in}}%
\pgfpathlineto{\pgfqpoint{4.893050in}{2.951439in}}%
\pgfpathlineto{\pgfqpoint{4.743147in}{2.951439in}}%
\pgfpathlineto{\pgfqpoint{4.743147in}{2.920611in}}%
\pgfpathclose%
\pgfusepath{stroke,fill}%
\end{pgfscope}%
\begin{pgfscope}%
\pgfsetrectcap%
\pgfsetmiterjoin%
\pgfsetlinewidth{0.803000pt}%
\definecolor{currentstroke}{rgb}{0.000000,0.000000,0.000000}%
\pgfsetstrokecolor{currentstroke}%
\pgfsetdash{}{0pt}%
\pgfpathmoveto{\pgfqpoint{4.106058in}{2.920611in}}%
\pgfpathlineto{\pgfqpoint{4.106058in}{3.382611in}}%
\pgfusepath{stroke}%
\end{pgfscope}%
\begin{pgfscope}%
\pgfsetrectcap%
\pgfsetmiterjoin%
\pgfsetlinewidth{0.803000pt}%
\definecolor{currentstroke}{rgb}{0.000000,0.000000,0.000000}%
\pgfsetstrokecolor{currentstroke}%
\pgfsetdash{}{0pt}%
\pgfpathmoveto{\pgfqpoint{4.930526in}{2.920611in}}%
\pgfpathlineto{\pgfqpoint{4.930526in}{3.382611in}}%
\pgfusepath{stroke}%
\end{pgfscope}%
\begin{pgfscope}%
\pgfsetrectcap%
\pgfsetmiterjoin%
\pgfsetlinewidth{0.803000pt}%
\definecolor{currentstroke}{rgb}{0.000000,0.000000,0.000000}%
\pgfsetstrokecolor{currentstroke}%
\pgfsetdash{}{0pt}%
\pgfpathmoveto{\pgfqpoint{4.106058in}{2.920611in}}%
\pgfpathlineto{\pgfqpoint{4.930526in}{2.920611in}}%
\pgfusepath{stroke}%
\end{pgfscope}%
\begin{pgfscope}%
\pgfsetrectcap%
\pgfsetmiterjoin%
\pgfsetlinewidth{0.803000pt}%
\definecolor{currentstroke}{rgb}{0.000000,0.000000,0.000000}%
\pgfsetstrokecolor{currentstroke}%
\pgfsetdash{}{0pt}%
\pgfpathmoveto{\pgfqpoint{4.106058in}{3.382611in}}%
\pgfpathlineto{\pgfqpoint{4.930526in}{3.382611in}}%
\pgfusepath{stroke}%
\end{pgfscope}%
\begin{pgfscope}%
\definecolor{textcolor}{rgb}{0.000000,0.000000,0.000000}%
\pgfsetstrokecolor{textcolor}%
\pgfsetfillcolor{textcolor}%
\pgftext[x=4.518292in,y=3.465944in,,base]{\color{textcolor}\rmfamily\fontsize{11.000000}{13.200000}\selectfont MAIF}%
\end{pgfscope}%
\begin{pgfscope}%
\pgfsetbuttcap%
\pgfsetmiterjoin%
\definecolor{currentfill}{rgb}{1.000000,1.000000,1.000000}%
\pgfsetfillcolor{currentfill}%
\pgfsetlinewidth{0.000000pt}%
\definecolor{currentstroke}{rgb}{0.000000,0.000000,0.000000}%
\pgfsetstrokecolor{currentstroke}%
\pgfsetstrokeopacity{0.000000}%
\pgfsetdash{}{0pt}%
\pgfpathmoveto{\pgfqpoint{5.095420in}{2.920611in}}%
\pgfpathlineto{\pgfqpoint{5.919888in}{2.920611in}}%
\pgfpathlineto{\pgfqpoint{5.919888in}{3.382611in}}%
\pgfpathlineto{\pgfqpoint{5.095420in}{3.382611in}}%
\pgfpathlineto{\pgfqpoint{5.095420in}{2.920611in}}%
\pgfpathclose%
\pgfusepath{fill}%
\end{pgfscope}%
\begin{pgfscope}%
\pgfpathrectangle{\pgfqpoint{5.095420in}{2.920611in}}{\pgfqpoint{0.824468in}{0.462000in}}%
\pgfusepath{clip}%
\pgfsetbuttcap%
\pgfsetmiterjoin%
\definecolor{currentfill}{rgb}{0.121569,0.466667,0.705882}%
\pgfsetfillcolor{currentfill}%
\pgfsetfillopacity{0.500000}%
\pgfsetlinewidth{1.003750pt}%
\definecolor{currentstroke}{rgb}{0.000000,0.000000,0.000000}%
\pgfsetstrokecolor{currentstroke}%
\pgfsetdash{}{0pt}%
\pgfpathmoveto{\pgfqpoint{5.132895in}{2.920611in}}%
\pgfpathlineto{\pgfqpoint{5.282799in}{2.920611in}}%
\pgfpathlineto{\pgfqpoint{5.282799in}{3.360611in}}%
\pgfpathlineto{\pgfqpoint{5.132895in}{3.360611in}}%
\pgfpathlineto{\pgfqpoint{5.132895in}{2.920611in}}%
\pgfpathclose%
\pgfusepath{stroke,fill}%
\end{pgfscope}%
\begin{pgfscope}%
\pgfpathrectangle{\pgfqpoint{5.095420in}{2.920611in}}{\pgfqpoint{0.824468in}{0.462000in}}%
\pgfusepath{clip}%
\pgfsetbuttcap%
\pgfsetmiterjoin%
\definecolor{currentfill}{rgb}{0.121569,0.466667,0.705882}%
\pgfsetfillcolor{currentfill}%
\pgfsetfillopacity{0.500000}%
\pgfsetlinewidth{1.003750pt}%
\definecolor{currentstroke}{rgb}{0.000000,0.000000,0.000000}%
\pgfsetstrokecolor{currentstroke}%
\pgfsetdash{}{0pt}%
\pgfpathmoveto{\pgfqpoint{5.282799in}{2.920611in}}%
\pgfpathlineto{\pgfqpoint{5.432702in}{2.920611in}}%
\pgfpathlineto{\pgfqpoint{5.432702in}{3.028366in}}%
\pgfpathlineto{\pgfqpoint{5.282799in}{3.028366in}}%
\pgfpathlineto{\pgfqpoint{5.282799in}{2.920611in}}%
\pgfpathclose%
\pgfusepath{stroke,fill}%
\end{pgfscope}%
\begin{pgfscope}%
\pgfpathrectangle{\pgfqpoint{5.095420in}{2.920611in}}{\pgfqpoint{0.824468in}{0.462000in}}%
\pgfusepath{clip}%
\pgfsetbuttcap%
\pgfsetmiterjoin%
\definecolor{currentfill}{rgb}{0.121569,0.466667,0.705882}%
\pgfsetfillcolor{currentfill}%
\pgfsetfillopacity{0.500000}%
\pgfsetlinewidth{1.003750pt}%
\definecolor{currentstroke}{rgb}{0.000000,0.000000,0.000000}%
\pgfsetstrokecolor{currentstroke}%
\pgfsetdash{}{0pt}%
\pgfpathmoveto{\pgfqpoint{5.432702in}{2.920611in}}%
\pgfpathlineto{\pgfqpoint{5.582605in}{2.920611in}}%
\pgfpathlineto{\pgfqpoint{5.582605in}{2.974489in}}%
\pgfpathlineto{\pgfqpoint{5.432702in}{2.974489in}}%
\pgfpathlineto{\pgfqpoint{5.432702in}{2.920611in}}%
\pgfpathclose%
\pgfusepath{stroke,fill}%
\end{pgfscope}%
\begin{pgfscope}%
\pgfpathrectangle{\pgfqpoint{5.095420in}{2.920611in}}{\pgfqpoint{0.824468in}{0.462000in}}%
\pgfusepath{clip}%
\pgfsetbuttcap%
\pgfsetmiterjoin%
\definecolor{currentfill}{rgb}{0.121569,0.466667,0.705882}%
\pgfsetfillcolor{currentfill}%
\pgfsetfillopacity{0.500000}%
\pgfsetlinewidth{1.003750pt}%
\definecolor{currentstroke}{rgb}{0.000000,0.000000,0.000000}%
\pgfsetstrokecolor{currentstroke}%
\pgfsetdash{}{0pt}%
\pgfpathmoveto{\pgfqpoint{5.582605in}{2.920611in}}%
\pgfpathlineto{\pgfqpoint{5.732509in}{2.920611in}}%
\pgfpathlineto{\pgfqpoint{5.732509in}{2.947550in}}%
\pgfpathlineto{\pgfqpoint{5.582605in}{2.947550in}}%
\pgfpathlineto{\pgfqpoint{5.582605in}{2.920611in}}%
\pgfpathclose%
\pgfusepath{stroke,fill}%
\end{pgfscope}%
\begin{pgfscope}%
\pgfpathrectangle{\pgfqpoint{5.095420in}{2.920611in}}{\pgfqpoint{0.824468in}{0.462000in}}%
\pgfusepath{clip}%
\pgfsetbuttcap%
\pgfsetmiterjoin%
\definecolor{currentfill}{rgb}{0.121569,0.466667,0.705882}%
\pgfsetfillcolor{currentfill}%
\pgfsetfillopacity{0.500000}%
\pgfsetlinewidth{1.003750pt}%
\definecolor{currentstroke}{rgb}{0.000000,0.000000,0.000000}%
\pgfsetstrokecolor{currentstroke}%
\pgfsetdash{}{0pt}%
\pgfpathmoveto{\pgfqpoint{5.732509in}{2.920611in}}%
\pgfpathlineto{\pgfqpoint{5.882412in}{2.920611in}}%
\pgfpathlineto{\pgfqpoint{5.882412in}{2.938570in}}%
\pgfpathlineto{\pgfqpoint{5.732509in}{2.938570in}}%
\pgfpathlineto{\pgfqpoint{5.732509in}{2.920611in}}%
\pgfpathclose%
\pgfusepath{stroke,fill}%
\end{pgfscope}%
\begin{pgfscope}%
\pgfsetrectcap%
\pgfsetmiterjoin%
\pgfsetlinewidth{0.803000pt}%
\definecolor{currentstroke}{rgb}{0.000000,0.000000,0.000000}%
\pgfsetstrokecolor{currentstroke}%
\pgfsetdash{}{0pt}%
\pgfpathmoveto{\pgfqpoint{5.095420in}{2.920611in}}%
\pgfpathlineto{\pgfqpoint{5.095420in}{3.382611in}}%
\pgfusepath{stroke}%
\end{pgfscope}%
\begin{pgfscope}%
\pgfsetrectcap%
\pgfsetmiterjoin%
\pgfsetlinewidth{0.803000pt}%
\definecolor{currentstroke}{rgb}{0.000000,0.000000,0.000000}%
\pgfsetstrokecolor{currentstroke}%
\pgfsetdash{}{0pt}%
\pgfpathmoveto{\pgfqpoint{5.919888in}{2.920611in}}%
\pgfpathlineto{\pgfqpoint{5.919888in}{3.382611in}}%
\pgfusepath{stroke}%
\end{pgfscope}%
\begin{pgfscope}%
\pgfsetrectcap%
\pgfsetmiterjoin%
\pgfsetlinewidth{0.803000pt}%
\definecolor{currentstroke}{rgb}{0.000000,0.000000,0.000000}%
\pgfsetstrokecolor{currentstroke}%
\pgfsetdash{}{0pt}%
\pgfpathmoveto{\pgfqpoint{5.095420in}{2.920611in}}%
\pgfpathlineto{\pgfqpoint{5.919888in}{2.920611in}}%
\pgfusepath{stroke}%
\end{pgfscope}%
\begin{pgfscope}%
\pgfsetrectcap%
\pgfsetmiterjoin%
\pgfsetlinewidth{0.803000pt}%
\definecolor{currentstroke}{rgb}{0.000000,0.000000,0.000000}%
\pgfsetstrokecolor{currentstroke}%
\pgfsetdash{}{0pt}%
\pgfpathmoveto{\pgfqpoint{5.095420in}{3.382611in}}%
\pgfpathlineto{\pgfqpoint{5.919888in}{3.382611in}}%
\pgfusepath{stroke}%
\end{pgfscope}%
\begin{pgfscope}%
\definecolor{textcolor}{rgb}{0.000000,0.000000,0.000000}%
\pgfsetstrokecolor{textcolor}%
\pgfsetfillcolor{textcolor}%
\pgftext[x=5.507654in,y=3.465944in,,base]{\color{textcolor}\rmfamily\fontsize{11.000000}{13.200000}\selectfont Sogecap}%
\end{pgfscope}%
\begin{pgfscope}%
\pgfsetbuttcap%
\pgfsetmiterjoin%
\definecolor{currentfill}{rgb}{1.000000,1.000000,1.000000}%
\pgfsetfillcolor{currentfill}%
\pgfsetlinewidth{0.000000pt}%
\definecolor{currentstroke}{rgb}{0.000000,0.000000,0.000000}%
\pgfsetstrokecolor{currentstroke}%
\pgfsetstrokeopacity{0.000000}%
\pgfsetdash{}{0pt}%
\pgfpathmoveto{\pgfqpoint{6.084781in}{2.920611in}}%
\pgfpathlineto{\pgfqpoint{6.909249in}{2.920611in}}%
\pgfpathlineto{\pgfqpoint{6.909249in}{3.382611in}}%
\pgfpathlineto{\pgfqpoint{6.084781in}{3.382611in}}%
\pgfpathlineto{\pgfqpoint{6.084781in}{2.920611in}}%
\pgfpathclose%
\pgfusepath{fill}%
\end{pgfscope}%
\begin{pgfscope}%
\pgfpathrectangle{\pgfqpoint{6.084781in}{2.920611in}}{\pgfqpoint{0.824468in}{0.462000in}}%
\pgfusepath{clip}%
\pgfsetbuttcap%
\pgfsetmiterjoin%
\definecolor{currentfill}{rgb}{0.121569,0.466667,0.705882}%
\pgfsetfillcolor{currentfill}%
\pgfsetfillopacity{0.500000}%
\pgfsetlinewidth{1.003750pt}%
\definecolor{currentstroke}{rgb}{0.000000,0.000000,0.000000}%
\pgfsetstrokecolor{currentstroke}%
\pgfsetdash{}{0pt}%
\pgfpathmoveto{\pgfqpoint{6.122257in}{2.920611in}}%
\pgfpathlineto{\pgfqpoint{6.272160in}{2.920611in}}%
\pgfpathlineto{\pgfqpoint{6.272160in}{3.360611in}}%
\pgfpathlineto{\pgfqpoint{6.122257in}{3.360611in}}%
\pgfpathlineto{\pgfqpoint{6.122257in}{2.920611in}}%
\pgfpathclose%
\pgfusepath{stroke,fill}%
\end{pgfscope}%
\begin{pgfscope}%
\pgfpathrectangle{\pgfqpoint{6.084781in}{2.920611in}}{\pgfqpoint{0.824468in}{0.462000in}}%
\pgfusepath{clip}%
\pgfsetbuttcap%
\pgfsetmiterjoin%
\definecolor{currentfill}{rgb}{0.121569,0.466667,0.705882}%
\pgfsetfillcolor{currentfill}%
\pgfsetfillopacity{0.500000}%
\pgfsetlinewidth{1.003750pt}%
\definecolor{currentstroke}{rgb}{0.000000,0.000000,0.000000}%
\pgfsetstrokecolor{currentstroke}%
\pgfsetdash{}{0pt}%
\pgfpathmoveto{\pgfqpoint{6.272160in}{2.920611in}}%
\pgfpathlineto{\pgfqpoint{6.422064in}{2.920611in}}%
\pgfpathlineto{\pgfqpoint{6.422064in}{3.013143in}}%
\pgfpathlineto{\pgfqpoint{6.272160in}{3.013143in}}%
\pgfpathlineto{\pgfqpoint{6.272160in}{2.920611in}}%
\pgfpathclose%
\pgfusepath{stroke,fill}%
\end{pgfscope}%
\begin{pgfscope}%
\pgfpathrectangle{\pgfqpoint{6.084781in}{2.920611in}}{\pgfqpoint{0.824468in}{0.462000in}}%
\pgfusepath{clip}%
\pgfsetbuttcap%
\pgfsetmiterjoin%
\definecolor{currentfill}{rgb}{0.121569,0.466667,0.705882}%
\pgfsetfillcolor{currentfill}%
\pgfsetfillopacity{0.500000}%
\pgfsetlinewidth{1.003750pt}%
\definecolor{currentstroke}{rgb}{0.000000,0.000000,0.000000}%
\pgfsetstrokecolor{currentstroke}%
\pgfsetdash{}{0pt}%
\pgfpathmoveto{\pgfqpoint{6.422064in}{2.920611in}}%
\pgfpathlineto{\pgfqpoint{6.571967in}{2.920611in}}%
\pgfpathlineto{\pgfqpoint{6.571967in}{2.962156in}}%
\pgfpathlineto{\pgfqpoint{6.422064in}{2.962156in}}%
\pgfpathlineto{\pgfqpoint{6.422064in}{2.920611in}}%
\pgfpathclose%
\pgfusepath{stroke,fill}%
\end{pgfscope}%
\begin{pgfscope}%
\pgfpathrectangle{\pgfqpoint{6.084781in}{2.920611in}}{\pgfqpoint{0.824468in}{0.462000in}}%
\pgfusepath{clip}%
\pgfsetbuttcap%
\pgfsetmiterjoin%
\definecolor{currentfill}{rgb}{0.121569,0.466667,0.705882}%
\pgfsetfillcolor{currentfill}%
\pgfsetfillopacity{0.500000}%
\pgfsetlinewidth{1.003750pt}%
\definecolor{currentstroke}{rgb}{0.000000,0.000000,0.000000}%
\pgfsetstrokecolor{currentstroke}%
\pgfsetdash{}{0pt}%
\pgfpathmoveto{\pgfqpoint{6.571967in}{2.920611in}}%
\pgfpathlineto{\pgfqpoint{6.721870in}{2.920611in}}%
\pgfpathlineto{\pgfqpoint{6.721870in}{2.928165in}}%
\pgfpathlineto{\pgfqpoint{6.571967in}{2.928165in}}%
\pgfpathlineto{\pgfqpoint{6.571967in}{2.920611in}}%
\pgfpathclose%
\pgfusepath{stroke,fill}%
\end{pgfscope}%
\begin{pgfscope}%
\pgfpathrectangle{\pgfqpoint{6.084781in}{2.920611in}}{\pgfqpoint{0.824468in}{0.462000in}}%
\pgfusepath{clip}%
\pgfsetbuttcap%
\pgfsetmiterjoin%
\definecolor{currentfill}{rgb}{0.121569,0.466667,0.705882}%
\pgfsetfillcolor{currentfill}%
\pgfsetfillopacity{0.500000}%
\pgfsetlinewidth{1.003750pt}%
\definecolor{currentstroke}{rgb}{0.000000,0.000000,0.000000}%
\pgfsetstrokecolor{currentstroke}%
\pgfsetdash{}{0pt}%
\pgfpathmoveto{\pgfqpoint{6.721870in}{2.920611in}}%
\pgfpathlineto{\pgfqpoint{6.871774in}{2.920611in}}%
\pgfpathlineto{\pgfqpoint{6.871774in}{2.928165in}}%
\pgfpathlineto{\pgfqpoint{6.721870in}{2.928165in}}%
\pgfpathlineto{\pgfqpoint{6.721870in}{2.920611in}}%
\pgfpathclose%
\pgfusepath{stroke,fill}%
\end{pgfscope}%
\begin{pgfscope}%
\pgfsetrectcap%
\pgfsetmiterjoin%
\pgfsetlinewidth{0.803000pt}%
\definecolor{currentstroke}{rgb}{0.000000,0.000000,0.000000}%
\pgfsetstrokecolor{currentstroke}%
\pgfsetdash{}{0pt}%
\pgfpathmoveto{\pgfqpoint{6.084781in}{2.920611in}}%
\pgfpathlineto{\pgfqpoint{6.084781in}{3.382611in}}%
\pgfusepath{stroke}%
\end{pgfscope}%
\begin{pgfscope}%
\pgfsetrectcap%
\pgfsetmiterjoin%
\pgfsetlinewidth{0.803000pt}%
\definecolor{currentstroke}{rgb}{0.000000,0.000000,0.000000}%
\pgfsetstrokecolor{currentstroke}%
\pgfsetdash{}{0pt}%
\pgfpathmoveto{\pgfqpoint{6.909249in}{2.920611in}}%
\pgfpathlineto{\pgfqpoint{6.909249in}{3.382611in}}%
\pgfusepath{stroke}%
\end{pgfscope}%
\begin{pgfscope}%
\pgfsetrectcap%
\pgfsetmiterjoin%
\pgfsetlinewidth{0.803000pt}%
\definecolor{currentstroke}{rgb}{0.000000,0.000000,0.000000}%
\pgfsetstrokecolor{currentstroke}%
\pgfsetdash{}{0pt}%
\pgfpathmoveto{\pgfqpoint{6.084781in}{2.920611in}}%
\pgfpathlineto{\pgfqpoint{6.909249in}{2.920611in}}%
\pgfusepath{stroke}%
\end{pgfscope}%
\begin{pgfscope}%
\pgfsetrectcap%
\pgfsetmiterjoin%
\pgfsetlinewidth{0.803000pt}%
\definecolor{currentstroke}{rgb}{0.000000,0.000000,0.000000}%
\pgfsetstrokecolor{currentstroke}%
\pgfsetdash{}{0pt}%
\pgfpathmoveto{\pgfqpoint{6.084781in}{3.382611in}}%
\pgfpathlineto{\pgfqpoint{6.909249in}{3.382611in}}%
\pgfusepath{stroke}%
\end{pgfscope}%
\begin{pgfscope}%
\definecolor{textcolor}{rgb}{0.000000,0.000000,0.000000}%
\pgfsetstrokecolor{textcolor}%
\pgfsetfillcolor{textcolor}%
\pgftext[x=6.497015in,y=3.465944in,,base]{\color{textcolor}\rmfamily\fontsize{11.000000}{13.200000}\selectfont Harmon...}%
\end{pgfscope}%
\begin{pgfscope}%
\pgfsetbuttcap%
\pgfsetmiterjoin%
\definecolor{currentfill}{rgb}{1.000000,1.000000,1.000000}%
\pgfsetfillcolor{currentfill}%
\pgfsetlinewidth{0.000000pt}%
\definecolor{currentstroke}{rgb}{0.000000,0.000000,0.000000}%
\pgfsetstrokecolor{currentstroke}%
\pgfsetstrokeopacity{0.000000}%
\pgfsetdash{}{0pt}%
\pgfpathmoveto{\pgfqpoint{7.074143in}{2.920611in}}%
\pgfpathlineto{\pgfqpoint{7.898611in}{2.920611in}}%
\pgfpathlineto{\pgfqpoint{7.898611in}{3.382611in}}%
\pgfpathlineto{\pgfqpoint{7.074143in}{3.382611in}}%
\pgfpathlineto{\pgfqpoint{7.074143in}{2.920611in}}%
\pgfpathclose%
\pgfusepath{fill}%
\end{pgfscope}%
\begin{pgfscope}%
\pgfpathrectangle{\pgfqpoint{7.074143in}{2.920611in}}{\pgfqpoint{0.824468in}{0.462000in}}%
\pgfusepath{clip}%
\pgfsetbuttcap%
\pgfsetmiterjoin%
\definecolor{currentfill}{rgb}{0.121569,0.466667,0.705882}%
\pgfsetfillcolor{currentfill}%
\pgfsetfillopacity{0.500000}%
\pgfsetlinewidth{1.003750pt}%
\definecolor{currentstroke}{rgb}{0.000000,0.000000,0.000000}%
\pgfsetstrokecolor{currentstroke}%
\pgfsetdash{}{0pt}%
\pgfpathmoveto{\pgfqpoint{7.111619in}{2.920611in}}%
\pgfpathlineto{\pgfqpoint{7.261522in}{2.920611in}}%
\pgfpathlineto{\pgfqpoint{7.261522in}{3.360611in}}%
\pgfpathlineto{\pgfqpoint{7.111619in}{3.360611in}}%
\pgfpathlineto{\pgfqpoint{7.111619in}{2.920611in}}%
\pgfpathclose%
\pgfusepath{stroke,fill}%
\end{pgfscope}%
\begin{pgfscope}%
\pgfpathrectangle{\pgfqpoint{7.074143in}{2.920611in}}{\pgfqpoint{0.824468in}{0.462000in}}%
\pgfusepath{clip}%
\pgfsetbuttcap%
\pgfsetmiterjoin%
\definecolor{currentfill}{rgb}{0.121569,0.466667,0.705882}%
\pgfsetfillcolor{currentfill}%
\pgfsetfillopacity{0.500000}%
\pgfsetlinewidth{1.003750pt}%
\definecolor{currentstroke}{rgb}{0.000000,0.000000,0.000000}%
\pgfsetstrokecolor{currentstroke}%
\pgfsetdash{}{0pt}%
\pgfpathmoveto{\pgfqpoint{7.261522in}{2.920611in}}%
\pgfpathlineto{\pgfqpoint{7.411425in}{2.920611in}}%
\pgfpathlineto{\pgfqpoint{7.411425in}{3.044714in}}%
\pgfpathlineto{\pgfqpoint{7.261522in}{3.044714in}}%
\pgfpathlineto{\pgfqpoint{7.261522in}{2.920611in}}%
\pgfpathclose%
\pgfusepath{stroke,fill}%
\end{pgfscope}%
\begin{pgfscope}%
\pgfpathrectangle{\pgfqpoint{7.074143in}{2.920611in}}{\pgfqpoint{0.824468in}{0.462000in}}%
\pgfusepath{clip}%
\pgfsetbuttcap%
\pgfsetmiterjoin%
\definecolor{currentfill}{rgb}{0.121569,0.466667,0.705882}%
\pgfsetfillcolor{currentfill}%
\pgfsetfillopacity{0.500000}%
\pgfsetlinewidth{1.003750pt}%
\definecolor{currentstroke}{rgb}{0.000000,0.000000,0.000000}%
\pgfsetstrokecolor{currentstroke}%
\pgfsetdash{}{0pt}%
\pgfpathmoveto{\pgfqpoint{7.411425in}{2.920611in}}%
\pgfpathlineto{\pgfqpoint{7.561329in}{2.920611in}}%
\pgfpathlineto{\pgfqpoint{7.561329in}{2.977021in}}%
\pgfpathlineto{\pgfqpoint{7.411425in}{2.977021in}}%
\pgfpathlineto{\pgfqpoint{7.411425in}{2.920611in}}%
\pgfpathclose%
\pgfusepath{stroke,fill}%
\end{pgfscope}%
\begin{pgfscope}%
\pgfpathrectangle{\pgfqpoint{7.074143in}{2.920611in}}{\pgfqpoint{0.824468in}{0.462000in}}%
\pgfusepath{clip}%
\pgfsetbuttcap%
\pgfsetmiterjoin%
\definecolor{currentfill}{rgb}{0.121569,0.466667,0.705882}%
\pgfsetfillcolor{currentfill}%
\pgfsetfillopacity{0.500000}%
\pgfsetlinewidth{1.003750pt}%
\definecolor{currentstroke}{rgb}{0.000000,0.000000,0.000000}%
\pgfsetstrokecolor{currentstroke}%
\pgfsetdash{}{0pt}%
\pgfpathmoveto{\pgfqpoint{7.561329in}{2.920611in}}%
\pgfpathlineto{\pgfqpoint{7.711232in}{2.920611in}}%
\pgfpathlineto{\pgfqpoint{7.711232in}{2.982662in}}%
\pgfpathlineto{\pgfqpoint{7.561329in}{2.982662in}}%
\pgfpathlineto{\pgfqpoint{7.561329in}{2.920611in}}%
\pgfpathclose%
\pgfusepath{stroke,fill}%
\end{pgfscope}%
\begin{pgfscope}%
\pgfpathrectangle{\pgfqpoint{7.074143in}{2.920611in}}{\pgfqpoint{0.824468in}{0.462000in}}%
\pgfusepath{clip}%
\pgfsetbuttcap%
\pgfsetmiterjoin%
\definecolor{currentfill}{rgb}{0.121569,0.466667,0.705882}%
\pgfsetfillcolor{currentfill}%
\pgfsetfillopacity{0.500000}%
\pgfsetlinewidth{1.003750pt}%
\definecolor{currentstroke}{rgb}{0.000000,0.000000,0.000000}%
\pgfsetstrokecolor{currentstroke}%
\pgfsetdash{}{0pt}%
\pgfpathmoveto{\pgfqpoint{7.711232in}{2.920611in}}%
\pgfpathlineto{\pgfqpoint{7.861135in}{2.920611in}}%
\pgfpathlineto{\pgfqpoint{7.861135in}{2.960098in}}%
\pgfpathlineto{\pgfqpoint{7.711232in}{2.960098in}}%
\pgfpathlineto{\pgfqpoint{7.711232in}{2.920611in}}%
\pgfpathclose%
\pgfusepath{stroke,fill}%
\end{pgfscope}%
\begin{pgfscope}%
\pgfsetrectcap%
\pgfsetmiterjoin%
\pgfsetlinewidth{0.803000pt}%
\definecolor{currentstroke}{rgb}{0.000000,0.000000,0.000000}%
\pgfsetstrokecolor{currentstroke}%
\pgfsetdash{}{0pt}%
\pgfpathmoveto{\pgfqpoint{7.074143in}{2.920611in}}%
\pgfpathlineto{\pgfqpoint{7.074143in}{3.382611in}}%
\pgfusepath{stroke}%
\end{pgfscope}%
\begin{pgfscope}%
\pgfsetrectcap%
\pgfsetmiterjoin%
\pgfsetlinewidth{0.803000pt}%
\definecolor{currentstroke}{rgb}{0.000000,0.000000,0.000000}%
\pgfsetstrokecolor{currentstroke}%
\pgfsetdash{}{0pt}%
\pgfpathmoveto{\pgfqpoint{7.898611in}{2.920611in}}%
\pgfpathlineto{\pgfqpoint{7.898611in}{3.382611in}}%
\pgfusepath{stroke}%
\end{pgfscope}%
\begin{pgfscope}%
\pgfsetrectcap%
\pgfsetmiterjoin%
\pgfsetlinewidth{0.803000pt}%
\definecolor{currentstroke}{rgb}{0.000000,0.000000,0.000000}%
\pgfsetstrokecolor{currentstroke}%
\pgfsetdash{}{0pt}%
\pgfpathmoveto{\pgfqpoint{7.074143in}{2.920611in}}%
\pgfpathlineto{\pgfqpoint{7.898611in}{2.920611in}}%
\pgfusepath{stroke}%
\end{pgfscope}%
\begin{pgfscope}%
\pgfsetrectcap%
\pgfsetmiterjoin%
\pgfsetlinewidth{0.803000pt}%
\definecolor{currentstroke}{rgb}{0.000000,0.000000,0.000000}%
\pgfsetstrokecolor{currentstroke}%
\pgfsetdash{}{0pt}%
\pgfpathmoveto{\pgfqpoint{7.074143in}{3.382611in}}%
\pgfpathlineto{\pgfqpoint{7.898611in}{3.382611in}}%
\pgfusepath{stroke}%
\end{pgfscope}%
\begin{pgfscope}%
\definecolor{textcolor}{rgb}{0.000000,0.000000,0.000000}%
\pgfsetstrokecolor{textcolor}%
\pgfsetfillcolor{textcolor}%
\pgftext[x=7.486377in,y=3.465944in,,base]{\color{textcolor}\rmfamily\fontsize{11.000000}{13.200000}\selectfont Mutuel...}%
\end{pgfscope}%
\begin{pgfscope}%
\pgfsetbuttcap%
\pgfsetmiterjoin%
\definecolor{currentfill}{rgb}{1.000000,1.000000,1.000000}%
\pgfsetfillcolor{currentfill}%
\pgfsetlinewidth{0.000000pt}%
\definecolor{currentstroke}{rgb}{0.000000,0.000000,0.000000}%
\pgfsetstrokecolor{currentstroke}%
\pgfsetstrokeopacity{0.000000}%
\pgfsetdash{}{0pt}%
\pgfpathmoveto{\pgfqpoint{0.148611in}{2.227611in}}%
\pgfpathlineto{\pgfqpoint{0.973079in}{2.227611in}}%
\pgfpathlineto{\pgfqpoint{0.973079in}{2.689611in}}%
\pgfpathlineto{\pgfqpoint{0.148611in}{2.689611in}}%
\pgfpathlineto{\pgfqpoint{0.148611in}{2.227611in}}%
\pgfpathclose%
\pgfusepath{fill}%
\end{pgfscope}%
\begin{pgfscope}%
\pgfpathrectangle{\pgfqpoint{0.148611in}{2.227611in}}{\pgfqpoint{0.824468in}{0.462000in}}%
\pgfusepath{clip}%
\pgfsetbuttcap%
\pgfsetmiterjoin%
\definecolor{currentfill}{rgb}{0.121569,0.466667,0.705882}%
\pgfsetfillcolor{currentfill}%
\pgfsetfillopacity{0.500000}%
\pgfsetlinewidth{1.003750pt}%
\definecolor{currentstroke}{rgb}{0.000000,0.000000,0.000000}%
\pgfsetstrokecolor{currentstroke}%
\pgfsetdash{}{0pt}%
\pgfpathmoveto{\pgfqpoint{0.186087in}{2.227611in}}%
\pgfpathlineto{\pgfqpoint{0.335990in}{2.227611in}}%
\pgfpathlineto{\pgfqpoint{0.335990in}{2.667611in}}%
\pgfpathlineto{\pgfqpoint{0.186087in}{2.667611in}}%
\pgfpathlineto{\pgfqpoint{0.186087in}{2.227611in}}%
\pgfpathclose%
\pgfusepath{stroke,fill}%
\end{pgfscope}%
\begin{pgfscope}%
\pgfpathrectangle{\pgfqpoint{0.148611in}{2.227611in}}{\pgfqpoint{0.824468in}{0.462000in}}%
\pgfusepath{clip}%
\pgfsetbuttcap%
\pgfsetmiterjoin%
\definecolor{currentfill}{rgb}{0.121569,0.466667,0.705882}%
\pgfsetfillcolor{currentfill}%
\pgfsetfillopacity{0.500000}%
\pgfsetlinewidth{1.003750pt}%
\definecolor{currentstroke}{rgb}{0.000000,0.000000,0.000000}%
\pgfsetstrokecolor{currentstroke}%
\pgfsetdash{}{0pt}%
\pgfpathmoveto{\pgfqpoint{0.335990in}{2.227611in}}%
\pgfpathlineto{\pgfqpoint{0.485894in}{2.227611in}}%
\pgfpathlineto{\pgfqpoint{0.485894in}{2.487658in}}%
\pgfpathlineto{\pgfqpoint{0.335990in}{2.487658in}}%
\pgfpathlineto{\pgfqpoint{0.335990in}{2.227611in}}%
\pgfpathclose%
\pgfusepath{stroke,fill}%
\end{pgfscope}%
\begin{pgfscope}%
\pgfpathrectangle{\pgfqpoint{0.148611in}{2.227611in}}{\pgfqpoint{0.824468in}{0.462000in}}%
\pgfusepath{clip}%
\pgfsetbuttcap%
\pgfsetmiterjoin%
\definecolor{currentfill}{rgb}{0.121569,0.466667,0.705882}%
\pgfsetfillcolor{currentfill}%
\pgfsetfillopacity{0.500000}%
\pgfsetlinewidth{1.003750pt}%
\definecolor{currentstroke}{rgb}{0.000000,0.000000,0.000000}%
\pgfsetstrokecolor{currentstroke}%
\pgfsetdash{}{0pt}%
\pgfpathmoveto{\pgfqpoint{0.485894in}{2.227611in}}%
\pgfpathlineto{\pgfqpoint{0.635797in}{2.227611in}}%
\pgfpathlineto{\pgfqpoint{0.635797in}{2.311866in}}%
\pgfpathlineto{\pgfqpoint{0.485894in}{2.311866in}}%
\pgfpathlineto{\pgfqpoint{0.485894in}{2.227611in}}%
\pgfpathclose%
\pgfusepath{stroke,fill}%
\end{pgfscope}%
\begin{pgfscope}%
\pgfpathrectangle{\pgfqpoint{0.148611in}{2.227611in}}{\pgfqpoint{0.824468in}{0.462000in}}%
\pgfusepath{clip}%
\pgfsetbuttcap%
\pgfsetmiterjoin%
\definecolor{currentfill}{rgb}{0.121569,0.466667,0.705882}%
\pgfsetfillcolor{currentfill}%
\pgfsetfillopacity{0.500000}%
\pgfsetlinewidth{1.003750pt}%
\definecolor{currentstroke}{rgb}{0.000000,0.000000,0.000000}%
\pgfsetstrokecolor{currentstroke}%
\pgfsetdash{}{0pt}%
\pgfpathmoveto{\pgfqpoint{0.635797in}{2.227611in}}%
\pgfpathlineto{\pgfqpoint{0.785700in}{2.227611in}}%
\pgfpathlineto{\pgfqpoint{0.785700in}{2.272339in}}%
\pgfpathlineto{\pgfqpoint{0.635797in}{2.272339in}}%
\pgfpathlineto{\pgfqpoint{0.635797in}{2.227611in}}%
\pgfpathclose%
\pgfusepath{stroke,fill}%
\end{pgfscope}%
\begin{pgfscope}%
\pgfpathrectangle{\pgfqpoint{0.148611in}{2.227611in}}{\pgfqpoint{0.824468in}{0.462000in}}%
\pgfusepath{clip}%
\pgfsetbuttcap%
\pgfsetmiterjoin%
\definecolor{currentfill}{rgb}{0.121569,0.466667,0.705882}%
\pgfsetfillcolor{currentfill}%
\pgfsetfillopacity{0.500000}%
\pgfsetlinewidth{1.003750pt}%
\definecolor{currentstroke}{rgb}{0.000000,0.000000,0.000000}%
\pgfsetstrokecolor{currentstroke}%
\pgfsetdash{}{0pt}%
\pgfpathmoveto{\pgfqpoint{0.785700in}{2.227611in}}%
\pgfpathlineto{\pgfqpoint{0.935603in}{2.227611in}}%
\pgfpathlineto{\pgfqpoint{0.935603in}{2.260897in}}%
\pgfpathlineto{\pgfqpoint{0.785700in}{2.260897in}}%
\pgfpathlineto{\pgfqpoint{0.785700in}{2.227611in}}%
\pgfpathclose%
\pgfusepath{stroke,fill}%
\end{pgfscope}%
\begin{pgfscope}%
\pgfsetrectcap%
\pgfsetmiterjoin%
\pgfsetlinewidth{0.803000pt}%
\definecolor{currentstroke}{rgb}{0.000000,0.000000,0.000000}%
\pgfsetstrokecolor{currentstroke}%
\pgfsetdash{}{0pt}%
\pgfpathmoveto{\pgfqpoint{0.148611in}{2.227611in}}%
\pgfpathlineto{\pgfqpoint{0.148611in}{2.689611in}}%
\pgfusepath{stroke}%
\end{pgfscope}%
\begin{pgfscope}%
\pgfsetrectcap%
\pgfsetmiterjoin%
\pgfsetlinewidth{0.803000pt}%
\definecolor{currentstroke}{rgb}{0.000000,0.000000,0.000000}%
\pgfsetstrokecolor{currentstroke}%
\pgfsetdash{}{0pt}%
\pgfpathmoveto{\pgfqpoint{0.973079in}{2.227611in}}%
\pgfpathlineto{\pgfqpoint{0.973079in}{2.689611in}}%
\pgfusepath{stroke}%
\end{pgfscope}%
\begin{pgfscope}%
\pgfsetrectcap%
\pgfsetmiterjoin%
\pgfsetlinewidth{0.803000pt}%
\definecolor{currentstroke}{rgb}{0.000000,0.000000,0.000000}%
\pgfsetstrokecolor{currentstroke}%
\pgfsetdash{}{0pt}%
\pgfpathmoveto{\pgfqpoint{0.148611in}{2.227611in}}%
\pgfpathlineto{\pgfqpoint{0.973079in}{2.227611in}}%
\pgfusepath{stroke}%
\end{pgfscope}%
\begin{pgfscope}%
\pgfsetrectcap%
\pgfsetmiterjoin%
\pgfsetlinewidth{0.803000pt}%
\definecolor{currentstroke}{rgb}{0.000000,0.000000,0.000000}%
\pgfsetstrokecolor{currentstroke}%
\pgfsetdash{}{0pt}%
\pgfpathmoveto{\pgfqpoint{0.148611in}{2.689611in}}%
\pgfpathlineto{\pgfqpoint{0.973079in}{2.689611in}}%
\pgfusepath{stroke}%
\end{pgfscope}%
\begin{pgfscope}%
\definecolor{textcolor}{rgb}{0.000000,0.000000,0.000000}%
\pgfsetstrokecolor{textcolor}%
\pgfsetfillcolor{textcolor}%
\pgftext[x=0.560845in,y=2.772944in,,base]{\color{textcolor}\rmfamily\fontsize{11.000000}{13.200000}\selectfont MACIF}%
\end{pgfscope}%
\begin{pgfscope}%
\pgfsetbuttcap%
\pgfsetmiterjoin%
\definecolor{currentfill}{rgb}{1.000000,1.000000,1.000000}%
\pgfsetfillcolor{currentfill}%
\pgfsetlinewidth{0.000000pt}%
\definecolor{currentstroke}{rgb}{0.000000,0.000000,0.000000}%
\pgfsetstrokecolor{currentstroke}%
\pgfsetstrokeopacity{0.000000}%
\pgfsetdash{}{0pt}%
\pgfpathmoveto{\pgfqpoint{1.137973in}{2.227611in}}%
\pgfpathlineto{\pgfqpoint{1.962441in}{2.227611in}}%
\pgfpathlineto{\pgfqpoint{1.962441in}{2.689611in}}%
\pgfpathlineto{\pgfqpoint{1.137973in}{2.689611in}}%
\pgfpathlineto{\pgfqpoint{1.137973in}{2.227611in}}%
\pgfpathclose%
\pgfusepath{fill}%
\end{pgfscope}%
\begin{pgfscope}%
\pgfpathrectangle{\pgfqpoint{1.137973in}{2.227611in}}{\pgfqpoint{0.824468in}{0.462000in}}%
\pgfusepath{clip}%
\pgfsetbuttcap%
\pgfsetmiterjoin%
\definecolor{currentfill}{rgb}{0.121569,0.466667,0.705882}%
\pgfsetfillcolor{currentfill}%
\pgfsetfillopacity{0.500000}%
\pgfsetlinewidth{1.003750pt}%
\definecolor{currentstroke}{rgb}{0.000000,0.000000,0.000000}%
\pgfsetstrokecolor{currentstroke}%
\pgfsetdash{}{0pt}%
\pgfpathmoveto{\pgfqpoint{1.175449in}{2.227611in}}%
\pgfpathlineto{\pgfqpoint{1.325352in}{2.227611in}}%
\pgfpathlineto{\pgfqpoint{1.325352in}{2.641286in}}%
\pgfpathlineto{\pgfqpoint{1.175449in}{2.641286in}}%
\pgfpathlineto{\pgfqpoint{1.175449in}{2.227611in}}%
\pgfpathclose%
\pgfusepath{stroke,fill}%
\end{pgfscope}%
\begin{pgfscope}%
\pgfpathrectangle{\pgfqpoint{1.137973in}{2.227611in}}{\pgfqpoint{0.824468in}{0.462000in}}%
\pgfusepath{clip}%
\pgfsetbuttcap%
\pgfsetmiterjoin%
\definecolor{currentfill}{rgb}{0.121569,0.466667,0.705882}%
\pgfsetfillcolor{currentfill}%
\pgfsetfillopacity{0.500000}%
\pgfsetlinewidth{1.003750pt}%
\definecolor{currentstroke}{rgb}{0.000000,0.000000,0.000000}%
\pgfsetstrokecolor{currentstroke}%
\pgfsetdash{}{0pt}%
\pgfpathmoveto{\pgfqpoint{1.325352in}{2.227611in}}%
\pgfpathlineto{\pgfqpoint{1.475255in}{2.227611in}}%
\pgfpathlineto{\pgfqpoint{1.475255in}{2.667611in}}%
\pgfpathlineto{\pgfqpoint{1.325352in}{2.667611in}}%
\pgfpathlineto{\pgfqpoint{1.325352in}{2.227611in}}%
\pgfpathclose%
\pgfusepath{stroke,fill}%
\end{pgfscope}%
\begin{pgfscope}%
\pgfpathrectangle{\pgfqpoint{1.137973in}{2.227611in}}{\pgfqpoint{0.824468in}{0.462000in}}%
\pgfusepath{clip}%
\pgfsetbuttcap%
\pgfsetmiterjoin%
\definecolor{currentfill}{rgb}{0.121569,0.466667,0.705882}%
\pgfsetfillcolor{currentfill}%
\pgfsetfillopacity{0.500000}%
\pgfsetlinewidth{1.003750pt}%
\definecolor{currentstroke}{rgb}{0.000000,0.000000,0.000000}%
\pgfsetstrokecolor{currentstroke}%
\pgfsetdash{}{0pt}%
\pgfpathmoveto{\pgfqpoint{1.475255in}{2.227611in}}%
\pgfpathlineto{\pgfqpoint{1.625158in}{2.227611in}}%
\pgfpathlineto{\pgfqpoint{1.625158in}{2.351714in}}%
\pgfpathlineto{\pgfqpoint{1.475255in}{2.351714in}}%
\pgfpathlineto{\pgfqpoint{1.475255in}{2.227611in}}%
\pgfpathclose%
\pgfusepath{stroke,fill}%
\end{pgfscope}%
\begin{pgfscope}%
\pgfpathrectangle{\pgfqpoint{1.137973in}{2.227611in}}{\pgfqpoint{0.824468in}{0.462000in}}%
\pgfusepath{clip}%
\pgfsetbuttcap%
\pgfsetmiterjoin%
\definecolor{currentfill}{rgb}{0.121569,0.466667,0.705882}%
\pgfsetfillcolor{currentfill}%
\pgfsetfillopacity{0.500000}%
\pgfsetlinewidth{1.003750pt}%
\definecolor{currentstroke}{rgb}{0.000000,0.000000,0.000000}%
\pgfsetstrokecolor{currentstroke}%
\pgfsetdash{}{0pt}%
\pgfpathmoveto{\pgfqpoint{1.625158in}{2.227611in}}%
\pgfpathlineto{\pgfqpoint{1.775062in}{2.227611in}}%
\pgfpathlineto{\pgfqpoint{1.775062in}{2.299064in}}%
\pgfpathlineto{\pgfqpoint{1.625158in}{2.299064in}}%
\pgfpathlineto{\pgfqpoint{1.625158in}{2.227611in}}%
\pgfpathclose%
\pgfusepath{stroke,fill}%
\end{pgfscope}%
\begin{pgfscope}%
\pgfpathrectangle{\pgfqpoint{1.137973in}{2.227611in}}{\pgfqpoint{0.824468in}{0.462000in}}%
\pgfusepath{clip}%
\pgfsetbuttcap%
\pgfsetmiterjoin%
\definecolor{currentfill}{rgb}{0.121569,0.466667,0.705882}%
\pgfsetfillcolor{currentfill}%
\pgfsetfillopacity{0.500000}%
\pgfsetlinewidth{1.003750pt}%
\definecolor{currentstroke}{rgb}{0.000000,0.000000,0.000000}%
\pgfsetstrokecolor{currentstroke}%
\pgfsetdash{}{0pt}%
\pgfpathmoveto{\pgfqpoint{1.775062in}{2.227611in}}%
\pgfpathlineto{\pgfqpoint{1.924965in}{2.227611in}}%
\pgfpathlineto{\pgfqpoint{1.924965in}{2.265218in}}%
\pgfpathlineto{\pgfqpoint{1.775062in}{2.265218in}}%
\pgfpathlineto{\pgfqpoint{1.775062in}{2.227611in}}%
\pgfpathclose%
\pgfusepath{stroke,fill}%
\end{pgfscope}%
\begin{pgfscope}%
\pgfsetrectcap%
\pgfsetmiterjoin%
\pgfsetlinewidth{0.803000pt}%
\definecolor{currentstroke}{rgb}{0.000000,0.000000,0.000000}%
\pgfsetstrokecolor{currentstroke}%
\pgfsetdash{}{0pt}%
\pgfpathmoveto{\pgfqpoint{1.137973in}{2.227611in}}%
\pgfpathlineto{\pgfqpoint{1.137973in}{2.689611in}}%
\pgfusepath{stroke}%
\end{pgfscope}%
\begin{pgfscope}%
\pgfsetrectcap%
\pgfsetmiterjoin%
\pgfsetlinewidth{0.803000pt}%
\definecolor{currentstroke}{rgb}{0.000000,0.000000,0.000000}%
\pgfsetstrokecolor{currentstroke}%
\pgfsetdash{}{0pt}%
\pgfpathmoveto{\pgfqpoint{1.962441in}{2.227611in}}%
\pgfpathlineto{\pgfqpoint{1.962441in}{2.689611in}}%
\pgfusepath{stroke}%
\end{pgfscope}%
\begin{pgfscope}%
\pgfsetrectcap%
\pgfsetmiterjoin%
\pgfsetlinewidth{0.803000pt}%
\definecolor{currentstroke}{rgb}{0.000000,0.000000,0.000000}%
\pgfsetstrokecolor{currentstroke}%
\pgfsetdash{}{0pt}%
\pgfpathmoveto{\pgfqpoint{1.137973in}{2.227611in}}%
\pgfpathlineto{\pgfqpoint{1.962441in}{2.227611in}}%
\pgfusepath{stroke}%
\end{pgfscope}%
\begin{pgfscope}%
\pgfsetrectcap%
\pgfsetmiterjoin%
\pgfsetlinewidth{0.803000pt}%
\definecolor{currentstroke}{rgb}{0.000000,0.000000,0.000000}%
\pgfsetstrokecolor{currentstroke}%
\pgfsetdash{}{0pt}%
\pgfpathmoveto{\pgfqpoint{1.137973in}{2.689611in}}%
\pgfpathlineto{\pgfqpoint{1.962441in}{2.689611in}}%
\pgfusepath{stroke}%
\end{pgfscope}%
\begin{pgfscope}%
\definecolor{textcolor}{rgb}{0.000000,0.000000,0.000000}%
\pgfsetstrokecolor{textcolor}%
\pgfsetfillcolor{textcolor}%
\pgftext[x=1.550207in,y=2.772944in,,base]{\color{textcolor}\rmfamily\fontsize{11.000000}{13.200000}\selectfont Eurofil}%
\end{pgfscope}%
\begin{pgfscope}%
\pgfsetbuttcap%
\pgfsetmiterjoin%
\definecolor{currentfill}{rgb}{1.000000,1.000000,1.000000}%
\pgfsetfillcolor{currentfill}%
\pgfsetlinewidth{0.000000pt}%
\definecolor{currentstroke}{rgb}{0.000000,0.000000,0.000000}%
\pgfsetstrokecolor{currentstroke}%
\pgfsetstrokeopacity{0.000000}%
\pgfsetdash{}{0pt}%
\pgfpathmoveto{\pgfqpoint{2.127335in}{2.227611in}}%
\pgfpathlineto{\pgfqpoint{2.951803in}{2.227611in}}%
\pgfpathlineto{\pgfqpoint{2.951803in}{2.689611in}}%
\pgfpathlineto{\pgfqpoint{2.127335in}{2.689611in}}%
\pgfpathlineto{\pgfqpoint{2.127335in}{2.227611in}}%
\pgfpathclose%
\pgfusepath{fill}%
\end{pgfscope}%
\begin{pgfscope}%
\pgfpathrectangle{\pgfqpoint{2.127335in}{2.227611in}}{\pgfqpoint{0.824468in}{0.462000in}}%
\pgfusepath{clip}%
\pgfsetbuttcap%
\pgfsetmiterjoin%
\definecolor{currentfill}{rgb}{0.121569,0.466667,0.705882}%
\pgfsetfillcolor{currentfill}%
\pgfsetfillopacity{0.500000}%
\pgfsetlinewidth{1.003750pt}%
\definecolor{currentstroke}{rgb}{0.000000,0.000000,0.000000}%
\pgfsetstrokecolor{currentstroke}%
\pgfsetdash{}{0pt}%
\pgfpathmoveto{\pgfqpoint{2.164810in}{2.227611in}}%
\pgfpathlineto{\pgfqpoint{2.314714in}{2.227611in}}%
\pgfpathlineto{\pgfqpoint{2.314714in}{2.667611in}}%
\pgfpathlineto{\pgfqpoint{2.164810in}{2.667611in}}%
\pgfpathlineto{\pgfqpoint{2.164810in}{2.227611in}}%
\pgfpathclose%
\pgfusepath{stroke,fill}%
\end{pgfscope}%
\begin{pgfscope}%
\pgfpathrectangle{\pgfqpoint{2.127335in}{2.227611in}}{\pgfqpoint{0.824468in}{0.462000in}}%
\pgfusepath{clip}%
\pgfsetbuttcap%
\pgfsetmiterjoin%
\definecolor{currentfill}{rgb}{0.121569,0.466667,0.705882}%
\pgfsetfillcolor{currentfill}%
\pgfsetfillopacity{0.500000}%
\pgfsetlinewidth{1.003750pt}%
\definecolor{currentstroke}{rgb}{0.000000,0.000000,0.000000}%
\pgfsetstrokecolor{currentstroke}%
\pgfsetdash{}{0pt}%
\pgfpathmoveto{\pgfqpoint{2.314714in}{2.227611in}}%
\pgfpathlineto{\pgfqpoint{2.464617in}{2.227611in}}%
\pgfpathlineto{\pgfqpoint{2.464617in}{2.419163in}}%
\pgfpathlineto{\pgfqpoint{2.314714in}{2.419163in}}%
\pgfpathlineto{\pgfqpoint{2.314714in}{2.227611in}}%
\pgfpathclose%
\pgfusepath{stroke,fill}%
\end{pgfscope}%
\begin{pgfscope}%
\pgfpathrectangle{\pgfqpoint{2.127335in}{2.227611in}}{\pgfqpoint{0.824468in}{0.462000in}}%
\pgfusepath{clip}%
\pgfsetbuttcap%
\pgfsetmiterjoin%
\definecolor{currentfill}{rgb}{0.121569,0.466667,0.705882}%
\pgfsetfillcolor{currentfill}%
\pgfsetfillopacity{0.500000}%
\pgfsetlinewidth{1.003750pt}%
\definecolor{currentstroke}{rgb}{0.000000,0.000000,0.000000}%
\pgfsetstrokecolor{currentstroke}%
\pgfsetdash{}{0pt}%
\pgfpathmoveto{\pgfqpoint{2.464617in}{2.227611in}}%
\pgfpathlineto{\pgfqpoint{2.614520in}{2.227611in}}%
\pgfpathlineto{\pgfqpoint{2.614520in}{2.307266in}}%
\pgfpathlineto{\pgfqpoint{2.464617in}{2.307266in}}%
\pgfpathlineto{\pgfqpoint{2.464617in}{2.227611in}}%
\pgfpathclose%
\pgfusepath{stroke,fill}%
\end{pgfscope}%
\begin{pgfscope}%
\pgfpathrectangle{\pgfqpoint{2.127335in}{2.227611in}}{\pgfqpoint{0.824468in}{0.462000in}}%
\pgfusepath{clip}%
\pgfsetbuttcap%
\pgfsetmiterjoin%
\definecolor{currentfill}{rgb}{0.121569,0.466667,0.705882}%
\pgfsetfillcolor{currentfill}%
\pgfsetfillopacity{0.500000}%
\pgfsetlinewidth{1.003750pt}%
\definecolor{currentstroke}{rgb}{0.000000,0.000000,0.000000}%
\pgfsetstrokecolor{currentstroke}%
\pgfsetdash{}{0pt}%
\pgfpathmoveto{\pgfqpoint{2.614520in}{2.227611in}}%
\pgfpathlineto{\pgfqpoint{2.764423in}{2.227611in}}%
\pgfpathlineto{\pgfqpoint{2.764423in}{2.246577in}}%
\pgfpathlineto{\pgfqpoint{2.614520in}{2.246577in}}%
\pgfpathlineto{\pgfqpoint{2.614520in}{2.227611in}}%
\pgfpathclose%
\pgfusepath{stroke,fill}%
\end{pgfscope}%
\begin{pgfscope}%
\pgfpathrectangle{\pgfqpoint{2.127335in}{2.227611in}}{\pgfqpoint{0.824468in}{0.462000in}}%
\pgfusepath{clip}%
\pgfsetbuttcap%
\pgfsetmiterjoin%
\definecolor{currentfill}{rgb}{0.121569,0.466667,0.705882}%
\pgfsetfillcolor{currentfill}%
\pgfsetfillopacity{0.500000}%
\pgfsetlinewidth{1.003750pt}%
\definecolor{currentstroke}{rgb}{0.000000,0.000000,0.000000}%
\pgfsetstrokecolor{currentstroke}%
\pgfsetdash{}{0pt}%
\pgfpathmoveto{\pgfqpoint{2.764423in}{2.227611in}}%
\pgfpathlineto{\pgfqpoint{2.914327in}{2.227611in}}%
\pgfpathlineto{\pgfqpoint{2.914327in}{2.261749in}}%
\pgfpathlineto{\pgfqpoint{2.764423in}{2.261749in}}%
\pgfpathlineto{\pgfqpoint{2.764423in}{2.227611in}}%
\pgfpathclose%
\pgfusepath{stroke,fill}%
\end{pgfscope}%
\begin{pgfscope}%
\pgfsetrectcap%
\pgfsetmiterjoin%
\pgfsetlinewidth{0.803000pt}%
\definecolor{currentstroke}{rgb}{0.000000,0.000000,0.000000}%
\pgfsetstrokecolor{currentstroke}%
\pgfsetdash{}{0pt}%
\pgfpathmoveto{\pgfqpoint{2.127335in}{2.227611in}}%
\pgfpathlineto{\pgfqpoint{2.127335in}{2.689611in}}%
\pgfusepath{stroke}%
\end{pgfscope}%
\begin{pgfscope}%
\pgfsetrectcap%
\pgfsetmiterjoin%
\pgfsetlinewidth{0.803000pt}%
\definecolor{currentstroke}{rgb}{0.000000,0.000000,0.000000}%
\pgfsetstrokecolor{currentstroke}%
\pgfsetdash{}{0pt}%
\pgfpathmoveto{\pgfqpoint{2.951803in}{2.227611in}}%
\pgfpathlineto{\pgfqpoint{2.951803in}{2.689611in}}%
\pgfusepath{stroke}%
\end{pgfscope}%
\begin{pgfscope}%
\pgfsetrectcap%
\pgfsetmiterjoin%
\pgfsetlinewidth{0.803000pt}%
\definecolor{currentstroke}{rgb}{0.000000,0.000000,0.000000}%
\pgfsetstrokecolor{currentstroke}%
\pgfsetdash{}{0pt}%
\pgfpathmoveto{\pgfqpoint{2.127335in}{2.227611in}}%
\pgfpathlineto{\pgfqpoint{2.951803in}{2.227611in}}%
\pgfusepath{stroke}%
\end{pgfscope}%
\begin{pgfscope}%
\pgfsetrectcap%
\pgfsetmiterjoin%
\pgfsetlinewidth{0.803000pt}%
\definecolor{currentstroke}{rgb}{0.000000,0.000000,0.000000}%
\pgfsetstrokecolor{currentstroke}%
\pgfsetdash{}{0pt}%
\pgfpathmoveto{\pgfqpoint{2.127335in}{2.689611in}}%
\pgfpathlineto{\pgfqpoint{2.951803in}{2.689611in}}%
\pgfusepath{stroke}%
\end{pgfscope}%
\begin{pgfscope}%
\definecolor{textcolor}{rgb}{0.000000,0.000000,0.000000}%
\pgfsetstrokecolor{textcolor}%
\pgfsetfillcolor{textcolor}%
\pgftext[x=2.539569in,y=2.772944in,,base]{\color{textcolor}\rmfamily\fontsize{11.000000}{13.200000}\selectfont Active...}%
\end{pgfscope}%
\begin{pgfscope}%
\pgfsetbuttcap%
\pgfsetmiterjoin%
\definecolor{currentfill}{rgb}{1.000000,1.000000,1.000000}%
\pgfsetfillcolor{currentfill}%
\pgfsetlinewidth{0.000000pt}%
\definecolor{currentstroke}{rgb}{0.000000,0.000000,0.000000}%
\pgfsetstrokecolor{currentstroke}%
\pgfsetstrokeopacity{0.000000}%
\pgfsetdash{}{0pt}%
\pgfpathmoveto{\pgfqpoint{3.116696in}{2.227611in}}%
\pgfpathlineto{\pgfqpoint{3.941164in}{2.227611in}}%
\pgfpathlineto{\pgfqpoint{3.941164in}{2.689611in}}%
\pgfpathlineto{\pgfqpoint{3.116696in}{2.689611in}}%
\pgfpathlineto{\pgfqpoint{3.116696in}{2.227611in}}%
\pgfpathclose%
\pgfusepath{fill}%
\end{pgfscope}%
\begin{pgfscope}%
\pgfpathrectangle{\pgfqpoint{3.116696in}{2.227611in}}{\pgfqpoint{0.824468in}{0.462000in}}%
\pgfusepath{clip}%
\pgfsetbuttcap%
\pgfsetmiterjoin%
\definecolor{currentfill}{rgb}{0.121569,0.466667,0.705882}%
\pgfsetfillcolor{currentfill}%
\pgfsetfillopacity{0.500000}%
\pgfsetlinewidth{1.003750pt}%
\definecolor{currentstroke}{rgb}{0.000000,0.000000,0.000000}%
\pgfsetstrokecolor{currentstroke}%
\pgfsetdash{}{0pt}%
\pgfpathmoveto{\pgfqpoint{3.154172in}{2.227611in}}%
\pgfpathlineto{\pgfqpoint{3.304075in}{2.227611in}}%
\pgfpathlineto{\pgfqpoint{3.304075in}{2.667611in}}%
\pgfpathlineto{\pgfqpoint{3.154172in}{2.667611in}}%
\pgfpathlineto{\pgfqpoint{3.154172in}{2.227611in}}%
\pgfpathclose%
\pgfusepath{stroke,fill}%
\end{pgfscope}%
\begin{pgfscope}%
\pgfpathrectangle{\pgfqpoint{3.116696in}{2.227611in}}{\pgfqpoint{0.824468in}{0.462000in}}%
\pgfusepath{clip}%
\pgfsetbuttcap%
\pgfsetmiterjoin%
\definecolor{currentfill}{rgb}{0.121569,0.466667,0.705882}%
\pgfsetfillcolor{currentfill}%
\pgfsetfillopacity{0.500000}%
\pgfsetlinewidth{1.003750pt}%
\definecolor{currentstroke}{rgb}{0.000000,0.000000,0.000000}%
\pgfsetstrokecolor{currentstroke}%
\pgfsetdash{}{0pt}%
\pgfpathmoveto{\pgfqpoint{3.304075in}{2.227611in}}%
\pgfpathlineto{\pgfqpoint{3.453979in}{2.227611in}}%
\pgfpathlineto{\pgfqpoint{3.453979in}{2.400944in}}%
\pgfpathlineto{\pgfqpoint{3.304075in}{2.400944in}}%
\pgfpathlineto{\pgfqpoint{3.304075in}{2.227611in}}%
\pgfpathclose%
\pgfusepath{stroke,fill}%
\end{pgfscope}%
\begin{pgfscope}%
\pgfpathrectangle{\pgfqpoint{3.116696in}{2.227611in}}{\pgfqpoint{0.824468in}{0.462000in}}%
\pgfusepath{clip}%
\pgfsetbuttcap%
\pgfsetmiterjoin%
\definecolor{currentfill}{rgb}{0.121569,0.466667,0.705882}%
\pgfsetfillcolor{currentfill}%
\pgfsetfillopacity{0.500000}%
\pgfsetlinewidth{1.003750pt}%
\definecolor{currentstroke}{rgb}{0.000000,0.000000,0.000000}%
\pgfsetstrokecolor{currentstroke}%
\pgfsetdash{}{0pt}%
\pgfpathmoveto{\pgfqpoint{3.453979in}{2.227611in}}%
\pgfpathlineto{\pgfqpoint{3.603882in}{2.227611in}}%
\pgfpathlineto{\pgfqpoint{3.603882in}{2.331854in}}%
\pgfpathlineto{\pgfqpoint{3.453979in}{2.331854in}}%
\pgfpathlineto{\pgfqpoint{3.453979in}{2.227611in}}%
\pgfpathclose%
\pgfusepath{stroke,fill}%
\end{pgfscope}%
\begin{pgfscope}%
\pgfpathrectangle{\pgfqpoint{3.116696in}{2.227611in}}{\pgfqpoint{0.824468in}{0.462000in}}%
\pgfusepath{clip}%
\pgfsetbuttcap%
\pgfsetmiterjoin%
\definecolor{currentfill}{rgb}{0.121569,0.466667,0.705882}%
\pgfsetfillcolor{currentfill}%
\pgfsetfillopacity{0.500000}%
\pgfsetlinewidth{1.003750pt}%
\definecolor{currentstroke}{rgb}{0.000000,0.000000,0.000000}%
\pgfsetstrokecolor{currentstroke}%
\pgfsetdash{}{0pt}%
\pgfpathmoveto{\pgfqpoint{3.603882in}{2.227611in}}%
\pgfpathlineto{\pgfqpoint{3.753785in}{2.227611in}}%
\pgfpathlineto{\pgfqpoint{3.753785in}{2.254278in}}%
\pgfpathlineto{\pgfqpoint{3.603882in}{2.254278in}}%
\pgfpathlineto{\pgfqpoint{3.603882in}{2.227611in}}%
\pgfpathclose%
\pgfusepath{stroke,fill}%
\end{pgfscope}%
\begin{pgfscope}%
\pgfpathrectangle{\pgfqpoint{3.116696in}{2.227611in}}{\pgfqpoint{0.824468in}{0.462000in}}%
\pgfusepath{clip}%
\pgfsetbuttcap%
\pgfsetmiterjoin%
\definecolor{currentfill}{rgb}{0.121569,0.466667,0.705882}%
\pgfsetfillcolor{currentfill}%
\pgfsetfillopacity{0.500000}%
\pgfsetlinewidth{1.003750pt}%
\definecolor{currentstroke}{rgb}{0.000000,0.000000,0.000000}%
\pgfsetstrokecolor{currentstroke}%
\pgfsetdash{}{0pt}%
\pgfpathmoveto{\pgfqpoint{3.753785in}{2.227611in}}%
\pgfpathlineto{\pgfqpoint{3.903688in}{2.227611in}}%
\pgfpathlineto{\pgfqpoint{3.903688in}{2.245793in}}%
\pgfpathlineto{\pgfqpoint{3.753785in}{2.245793in}}%
\pgfpathlineto{\pgfqpoint{3.753785in}{2.227611in}}%
\pgfpathclose%
\pgfusepath{stroke,fill}%
\end{pgfscope}%
\begin{pgfscope}%
\pgfsetrectcap%
\pgfsetmiterjoin%
\pgfsetlinewidth{0.803000pt}%
\definecolor{currentstroke}{rgb}{0.000000,0.000000,0.000000}%
\pgfsetstrokecolor{currentstroke}%
\pgfsetdash{}{0pt}%
\pgfpathmoveto{\pgfqpoint{3.116696in}{2.227611in}}%
\pgfpathlineto{\pgfqpoint{3.116696in}{2.689611in}}%
\pgfusepath{stroke}%
\end{pgfscope}%
\begin{pgfscope}%
\pgfsetrectcap%
\pgfsetmiterjoin%
\pgfsetlinewidth{0.803000pt}%
\definecolor{currentstroke}{rgb}{0.000000,0.000000,0.000000}%
\pgfsetstrokecolor{currentstroke}%
\pgfsetdash{}{0pt}%
\pgfpathmoveto{\pgfqpoint{3.941164in}{2.227611in}}%
\pgfpathlineto{\pgfqpoint{3.941164in}{2.689611in}}%
\pgfusepath{stroke}%
\end{pgfscope}%
\begin{pgfscope}%
\pgfsetrectcap%
\pgfsetmiterjoin%
\pgfsetlinewidth{0.803000pt}%
\definecolor{currentstroke}{rgb}{0.000000,0.000000,0.000000}%
\pgfsetstrokecolor{currentstroke}%
\pgfsetdash{}{0pt}%
\pgfpathmoveto{\pgfqpoint{3.116696in}{2.227611in}}%
\pgfpathlineto{\pgfqpoint{3.941164in}{2.227611in}}%
\pgfusepath{stroke}%
\end{pgfscope}%
\begin{pgfscope}%
\pgfsetrectcap%
\pgfsetmiterjoin%
\pgfsetlinewidth{0.803000pt}%
\definecolor{currentstroke}{rgb}{0.000000,0.000000,0.000000}%
\pgfsetstrokecolor{currentstroke}%
\pgfsetdash{}{0pt}%
\pgfpathmoveto{\pgfqpoint{3.116696in}{2.689611in}}%
\pgfpathlineto{\pgfqpoint{3.941164in}{2.689611in}}%
\pgfusepath{stroke}%
\end{pgfscope}%
\begin{pgfscope}%
\definecolor{textcolor}{rgb}{0.000000,0.000000,0.000000}%
\pgfsetstrokecolor{textcolor}%
\pgfsetfillcolor{textcolor}%
\pgftext[x=3.528930in,y=2.772944in,,base]{\color{textcolor}\rmfamily\fontsize{11.000000}{13.200000}\selectfont AXA}%
\end{pgfscope}%
\begin{pgfscope}%
\pgfsetbuttcap%
\pgfsetmiterjoin%
\definecolor{currentfill}{rgb}{1.000000,1.000000,1.000000}%
\pgfsetfillcolor{currentfill}%
\pgfsetlinewidth{0.000000pt}%
\definecolor{currentstroke}{rgb}{0.000000,0.000000,0.000000}%
\pgfsetstrokecolor{currentstroke}%
\pgfsetstrokeopacity{0.000000}%
\pgfsetdash{}{0pt}%
\pgfpathmoveto{\pgfqpoint{4.106058in}{2.227611in}}%
\pgfpathlineto{\pgfqpoint{4.930526in}{2.227611in}}%
\pgfpathlineto{\pgfqpoint{4.930526in}{2.689611in}}%
\pgfpathlineto{\pgfqpoint{4.106058in}{2.689611in}}%
\pgfpathlineto{\pgfqpoint{4.106058in}{2.227611in}}%
\pgfpathclose%
\pgfusepath{fill}%
\end{pgfscope}%
\begin{pgfscope}%
\pgfpathrectangle{\pgfqpoint{4.106058in}{2.227611in}}{\pgfqpoint{0.824468in}{0.462000in}}%
\pgfusepath{clip}%
\pgfsetbuttcap%
\pgfsetmiterjoin%
\definecolor{currentfill}{rgb}{0.121569,0.466667,0.705882}%
\pgfsetfillcolor{currentfill}%
\pgfsetfillopacity{0.500000}%
\pgfsetlinewidth{1.003750pt}%
\definecolor{currentstroke}{rgb}{0.000000,0.000000,0.000000}%
\pgfsetstrokecolor{currentstroke}%
\pgfsetdash{}{0pt}%
\pgfpathmoveto{\pgfqpoint{4.143534in}{2.227611in}}%
\pgfpathlineto{\pgfqpoint{4.293437in}{2.227611in}}%
\pgfpathlineto{\pgfqpoint{4.293437in}{2.667611in}}%
\pgfpathlineto{\pgfqpoint{4.143534in}{2.667611in}}%
\pgfpathlineto{\pgfqpoint{4.143534in}{2.227611in}}%
\pgfpathclose%
\pgfusepath{stroke,fill}%
\end{pgfscope}%
\begin{pgfscope}%
\pgfpathrectangle{\pgfqpoint{4.106058in}{2.227611in}}{\pgfqpoint{0.824468in}{0.462000in}}%
\pgfusepath{clip}%
\pgfsetbuttcap%
\pgfsetmiterjoin%
\definecolor{currentfill}{rgb}{0.121569,0.466667,0.705882}%
\pgfsetfillcolor{currentfill}%
\pgfsetfillopacity{0.500000}%
\pgfsetlinewidth{1.003750pt}%
\definecolor{currentstroke}{rgb}{0.000000,0.000000,0.000000}%
\pgfsetstrokecolor{currentstroke}%
\pgfsetdash{}{0pt}%
\pgfpathmoveto{\pgfqpoint{4.293437in}{2.227611in}}%
\pgfpathlineto{\pgfqpoint{4.443340in}{2.227611in}}%
\pgfpathlineto{\pgfqpoint{4.443340in}{2.380944in}}%
\pgfpathlineto{\pgfqpoint{4.293437in}{2.380944in}}%
\pgfpathlineto{\pgfqpoint{4.293437in}{2.227611in}}%
\pgfpathclose%
\pgfusepath{stroke,fill}%
\end{pgfscope}%
\begin{pgfscope}%
\pgfpathrectangle{\pgfqpoint{4.106058in}{2.227611in}}{\pgfqpoint{0.824468in}{0.462000in}}%
\pgfusepath{clip}%
\pgfsetbuttcap%
\pgfsetmiterjoin%
\definecolor{currentfill}{rgb}{0.121569,0.466667,0.705882}%
\pgfsetfillcolor{currentfill}%
\pgfsetfillopacity{0.500000}%
\pgfsetlinewidth{1.003750pt}%
\definecolor{currentstroke}{rgb}{0.000000,0.000000,0.000000}%
\pgfsetstrokecolor{currentstroke}%
\pgfsetdash{}{0pt}%
\pgfpathmoveto{\pgfqpoint{4.443340in}{2.227611in}}%
\pgfpathlineto{\pgfqpoint{4.593244in}{2.227611in}}%
\pgfpathlineto{\pgfqpoint{4.593244in}{2.260944in}}%
\pgfpathlineto{\pgfqpoint{4.443340in}{2.260944in}}%
\pgfpathlineto{\pgfqpoint{4.443340in}{2.227611in}}%
\pgfpathclose%
\pgfusepath{stroke,fill}%
\end{pgfscope}%
\begin{pgfscope}%
\pgfpathrectangle{\pgfqpoint{4.106058in}{2.227611in}}{\pgfqpoint{0.824468in}{0.462000in}}%
\pgfusepath{clip}%
\pgfsetbuttcap%
\pgfsetmiterjoin%
\definecolor{currentfill}{rgb}{0.121569,0.466667,0.705882}%
\pgfsetfillcolor{currentfill}%
\pgfsetfillopacity{0.500000}%
\pgfsetlinewidth{1.003750pt}%
\definecolor{currentstroke}{rgb}{0.000000,0.000000,0.000000}%
\pgfsetstrokecolor{currentstroke}%
\pgfsetdash{}{0pt}%
\pgfpathmoveto{\pgfqpoint{4.593244in}{2.227611in}}%
\pgfpathlineto{\pgfqpoint{4.743147in}{2.227611in}}%
\pgfpathlineto{\pgfqpoint{4.743147in}{2.227611in}}%
\pgfpathlineto{\pgfqpoint{4.593244in}{2.227611in}}%
\pgfpathlineto{\pgfqpoint{4.593244in}{2.227611in}}%
\pgfpathclose%
\pgfusepath{stroke,fill}%
\end{pgfscope}%
\begin{pgfscope}%
\pgfpathrectangle{\pgfqpoint{4.106058in}{2.227611in}}{\pgfqpoint{0.824468in}{0.462000in}}%
\pgfusepath{clip}%
\pgfsetbuttcap%
\pgfsetmiterjoin%
\definecolor{currentfill}{rgb}{0.121569,0.466667,0.705882}%
\pgfsetfillcolor{currentfill}%
\pgfsetfillopacity{0.500000}%
\pgfsetlinewidth{1.003750pt}%
\definecolor{currentstroke}{rgb}{0.000000,0.000000,0.000000}%
\pgfsetstrokecolor{currentstroke}%
\pgfsetdash{}{0pt}%
\pgfpathmoveto{\pgfqpoint{4.743147in}{2.227611in}}%
\pgfpathlineto{\pgfqpoint{4.893050in}{2.227611in}}%
\pgfpathlineto{\pgfqpoint{4.893050in}{2.240944in}}%
\pgfpathlineto{\pgfqpoint{4.743147in}{2.240944in}}%
\pgfpathlineto{\pgfqpoint{4.743147in}{2.227611in}}%
\pgfpathclose%
\pgfusepath{stroke,fill}%
\end{pgfscope}%
\begin{pgfscope}%
\pgfsetrectcap%
\pgfsetmiterjoin%
\pgfsetlinewidth{0.803000pt}%
\definecolor{currentstroke}{rgb}{0.000000,0.000000,0.000000}%
\pgfsetstrokecolor{currentstroke}%
\pgfsetdash{}{0pt}%
\pgfpathmoveto{\pgfqpoint{4.106058in}{2.227611in}}%
\pgfpathlineto{\pgfqpoint{4.106058in}{2.689611in}}%
\pgfusepath{stroke}%
\end{pgfscope}%
\begin{pgfscope}%
\pgfsetrectcap%
\pgfsetmiterjoin%
\pgfsetlinewidth{0.803000pt}%
\definecolor{currentstroke}{rgb}{0.000000,0.000000,0.000000}%
\pgfsetstrokecolor{currentstroke}%
\pgfsetdash{}{0pt}%
\pgfpathmoveto{\pgfqpoint{4.930526in}{2.227611in}}%
\pgfpathlineto{\pgfqpoint{4.930526in}{2.689611in}}%
\pgfusepath{stroke}%
\end{pgfscope}%
\begin{pgfscope}%
\pgfsetrectcap%
\pgfsetmiterjoin%
\pgfsetlinewidth{0.803000pt}%
\definecolor{currentstroke}{rgb}{0.000000,0.000000,0.000000}%
\pgfsetstrokecolor{currentstroke}%
\pgfsetdash{}{0pt}%
\pgfpathmoveto{\pgfqpoint{4.106058in}{2.227611in}}%
\pgfpathlineto{\pgfqpoint{4.930526in}{2.227611in}}%
\pgfusepath{stroke}%
\end{pgfscope}%
\begin{pgfscope}%
\pgfsetrectcap%
\pgfsetmiterjoin%
\pgfsetlinewidth{0.803000pt}%
\definecolor{currentstroke}{rgb}{0.000000,0.000000,0.000000}%
\pgfsetstrokecolor{currentstroke}%
\pgfsetdash{}{0pt}%
\pgfpathmoveto{\pgfqpoint{4.106058in}{2.689611in}}%
\pgfpathlineto{\pgfqpoint{4.930526in}{2.689611in}}%
\pgfusepath{stroke}%
\end{pgfscope}%
\begin{pgfscope}%
\definecolor{textcolor}{rgb}{0.000000,0.000000,0.000000}%
\pgfsetstrokecolor{textcolor}%
\pgfsetfillcolor{textcolor}%
\pgftext[x=4.518292in,y=2.772944in,,base]{\color{textcolor}\rmfamily\fontsize{11.000000}{13.200000}\selectfont Sogessur}%
\end{pgfscope}%
\begin{pgfscope}%
\pgfsetbuttcap%
\pgfsetmiterjoin%
\definecolor{currentfill}{rgb}{1.000000,1.000000,1.000000}%
\pgfsetfillcolor{currentfill}%
\pgfsetlinewidth{0.000000pt}%
\definecolor{currentstroke}{rgb}{0.000000,0.000000,0.000000}%
\pgfsetstrokecolor{currentstroke}%
\pgfsetstrokeopacity{0.000000}%
\pgfsetdash{}{0pt}%
\pgfpathmoveto{\pgfqpoint{5.095420in}{2.227611in}}%
\pgfpathlineto{\pgfqpoint{5.919888in}{2.227611in}}%
\pgfpathlineto{\pgfqpoint{5.919888in}{2.689611in}}%
\pgfpathlineto{\pgfqpoint{5.095420in}{2.689611in}}%
\pgfpathlineto{\pgfqpoint{5.095420in}{2.227611in}}%
\pgfpathclose%
\pgfusepath{fill}%
\end{pgfscope}%
\begin{pgfscope}%
\pgfpathrectangle{\pgfqpoint{5.095420in}{2.227611in}}{\pgfqpoint{0.824468in}{0.462000in}}%
\pgfusepath{clip}%
\pgfsetbuttcap%
\pgfsetmiterjoin%
\definecolor{currentfill}{rgb}{0.121569,0.466667,0.705882}%
\pgfsetfillcolor{currentfill}%
\pgfsetfillopacity{0.500000}%
\pgfsetlinewidth{1.003750pt}%
\definecolor{currentstroke}{rgb}{0.000000,0.000000,0.000000}%
\pgfsetstrokecolor{currentstroke}%
\pgfsetdash{}{0pt}%
\pgfpathmoveto{\pgfqpoint{5.132895in}{2.227611in}}%
\pgfpathlineto{\pgfqpoint{5.282799in}{2.227611in}}%
\pgfpathlineto{\pgfqpoint{5.282799in}{2.667611in}}%
\pgfpathlineto{\pgfqpoint{5.132895in}{2.667611in}}%
\pgfpathlineto{\pgfqpoint{5.132895in}{2.227611in}}%
\pgfpathclose%
\pgfusepath{stroke,fill}%
\end{pgfscope}%
\begin{pgfscope}%
\pgfpathrectangle{\pgfqpoint{5.095420in}{2.227611in}}{\pgfqpoint{0.824468in}{0.462000in}}%
\pgfusepath{clip}%
\pgfsetbuttcap%
\pgfsetmiterjoin%
\definecolor{currentfill}{rgb}{0.121569,0.466667,0.705882}%
\pgfsetfillcolor{currentfill}%
\pgfsetfillopacity{0.500000}%
\pgfsetlinewidth{1.003750pt}%
\definecolor{currentstroke}{rgb}{0.000000,0.000000,0.000000}%
\pgfsetstrokecolor{currentstroke}%
\pgfsetdash{}{0pt}%
\pgfpathmoveto{\pgfqpoint{5.282799in}{2.227611in}}%
\pgfpathlineto{\pgfqpoint{5.432702in}{2.227611in}}%
\pgfpathlineto{\pgfqpoint{5.432702in}{2.354603in}}%
\pgfpathlineto{\pgfqpoint{5.282799in}{2.354603in}}%
\pgfpathlineto{\pgfqpoint{5.282799in}{2.227611in}}%
\pgfpathclose%
\pgfusepath{stroke,fill}%
\end{pgfscope}%
\begin{pgfscope}%
\pgfpathrectangle{\pgfqpoint{5.095420in}{2.227611in}}{\pgfqpoint{0.824468in}{0.462000in}}%
\pgfusepath{clip}%
\pgfsetbuttcap%
\pgfsetmiterjoin%
\definecolor{currentfill}{rgb}{0.121569,0.466667,0.705882}%
\pgfsetfillcolor{currentfill}%
\pgfsetfillopacity{0.500000}%
\pgfsetlinewidth{1.003750pt}%
\definecolor{currentstroke}{rgb}{0.000000,0.000000,0.000000}%
\pgfsetstrokecolor{currentstroke}%
\pgfsetdash{}{0pt}%
\pgfpathmoveto{\pgfqpoint{5.432702in}{2.227611in}}%
\pgfpathlineto{\pgfqpoint{5.582605in}{2.227611in}}%
\pgfpathlineto{\pgfqpoint{5.582605in}{2.281270in}}%
\pgfpathlineto{\pgfqpoint{5.432702in}{2.281270in}}%
\pgfpathlineto{\pgfqpoint{5.432702in}{2.227611in}}%
\pgfpathclose%
\pgfusepath{stroke,fill}%
\end{pgfscope}%
\begin{pgfscope}%
\pgfpathrectangle{\pgfqpoint{5.095420in}{2.227611in}}{\pgfqpoint{0.824468in}{0.462000in}}%
\pgfusepath{clip}%
\pgfsetbuttcap%
\pgfsetmiterjoin%
\definecolor{currentfill}{rgb}{0.121569,0.466667,0.705882}%
\pgfsetfillcolor{currentfill}%
\pgfsetfillopacity{0.500000}%
\pgfsetlinewidth{1.003750pt}%
\definecolor{currentstroke}{rgb}{0.000000,0.000000,0.000000}%
\pgfsetstrokecolor{currentstroke}%
\pgfsetdash{}{0pt}%
\pgfpathmoveto{\pgfqpoint{5.582605in}{2.227611in}}%
\pgfpathlineto{\pgfqpoint{5.732509in}{2.227611in}}%
\pgfpathlineto{\pgfqpoint{5.732509in}{2.231188in}}%
\pgfpathlineto{\pgfqpoint{5.582605in}{2.231188in}}%
\pgfpathlineto{\pgfqpoint{5.582605in}{2.227611in}}%
\pgfpathclose%
\pgfusepath{stroke,fill}%
\end{pgfscope}%
\begin{pgfscope}%
\pgfpathrectangle{\pgfqpoint{5.095420in}{2.227611in}}{\pgfqpoint{0.824468in}{0.462000in}}%
\pgfusepath{clip}%
\pgfsetbuttcap%
\pgfsetmiterjoin%
\definecolor{currentfill}{rgb}{0.121569,0.466667,0.705882}%
\pgfsetfillcolor{currentfill}%
\pgfsetfillopacity{0.500000}%
\pgfsetlinewidth{1.003750pt}%
\definecolor{currentstroke}{rgb}{0.000000,0.000000,0.000000}%
\pgfsetstrokecolor{currentstroke}%
\pgfsetdash{}{0pt}%
\pgfpathmoveto{\pgfqpoint{5.732509in}{2.227611in}}%
\pgfpathlineto{\pgfqpoint{5.882412in}{2.227611in}}%
\pgfpathlineto{\pgfqpoint{5.882412in}{2.229400in}}%
\pgfpathlineto{\pgfqpoint{5.732509in}{2.229400in}}%
\pgfpathlineto{\pgfqpoint{5.732509in}{2.227611in}}%
\pgfpathclose%
\pgfusepath{stroke,fill}%
\end{pgfscope}%
\begin{pgfscope}%
\pgfsetrectcap%
\pgfsetmiterjoin%
\pgfsetlinewidth{0.803000pt}%
\definecolor{currentstroke}{rgb}{0.000000,0.000000,0.000000}%
\pgfsetstrokecolor{currentstroke}%
\pgfsetdash{}{0pt}%
\pgfpathmoveto{\pgfqpoint{5.095420in}{2.227611in}}%
\pgfpathlineto{\pgfqpoint{5.095420in}{2.689611in}}%
\pgfusepath{stroke}%
\end{pgfscope}%
\begin{pgfscope}%
\pgfsetrectcap%
\pgfsetmiterjoin%
\pgfsetlinewidth{0.803000pt}%
\definecolor{currentstroke}{rgb}{0.000000,0.000000,0.000000}%
\pgfsetstrokecolor{currentstroke}%
\pgfsetdash{}{0pt}%
\pgfpathmoveto{\pgfqpoint{5.919888in}{2.227611in}}%
\pgfpathlineto{\pgfqpoint{5.919888in}{2.689611in}}%
\pgfusepath{stroke}%
\end{pgfscope}%
\begin{pgfscope}%
\pgfsetrectcap%
\pgfsetmiterjoin%
\pgfsetlinewidth{0.803000pt}%
\definecolor{currentstroke}{rgb}{0.000000,0.000000,0.000000}%
\pgfsetstrokecolor{currentstroke}%
\pgfsetdash{}{0pt}%
\pgfpathmoveto{\pgfqpoint{5.095420in}{2.227611in}}%
\pgfpathlineto{\pgfqpoint{5.919888in}{2.227611in}}%
\pgfusepath{stroke}%
\end{pgfscope}%
\begin{pgfscope}%
\pgfsetrectcap%
\pgfsetmiterjoin%
\pgfsetlinewidth{0.803000pt}%
\definecolor{currentstroke}{rgb}{0.000000,0.000000,0.000000}%
\pgfsetstrokecolor{currentstroke}%
\pgfsetdash{}{0pt}%
\pgfpathmoveto{\pgfqpoint{5.095420in}{2.689611in}}%
\pgfpathlineto{\pgfqpoint{5.919888in}{2.689611in}}%
\pgfusepath{stroke}%
\end{pgfscope}%
\begin{pgfscope}%
\definecolor{textcolor}{rgb}{0.000000,0.000000,0.000000}%
\pgfsetstrokecolor{textcolor}%
\pgfsetfillcolor{textcolor}%
\pgftext[x=5.507654in,y=2.772944in,,base]{\color{textcolor}\rmfamily\fontsize{11.000000}{13.200000}\selectfont Ag2r L...}%
\end{pgfscope}%
\begin{pgfscope}%
\pgfsetbuttcap%
\pgfsetmiterjoin%
\definecolor{currentfill}{rgb}{1.000000,1.000000,1.000000}%
\pgfsetfillcolor{currentfill}%
\pgfsetlinewidth{0.000000pt}%
\definecolor{currentstroke}{rgb}{0.000000,0.000000,0.000000}%
\pgfsetstrokecolor{currentstroke}%
\pgfsetstrokeopacity{0.000000}%
\pgfsetdash{}{0pt}%
\pgfpathmoveto{\pgfqpoint{6.084781in}{2.227611in}}%
\pgfpathlineto{\pgfqpoint{6.909249in}{2.227611in}}%
\pgfpathlineto{\pgfqpoint{6.909249in}{2.689611in}}%
\pgfpathlineto{\pgfqpoint{6.084781in}{2.689611in}}%
\pgfpathlineto{\pgfqpoint{6.084781in}{2.227611in}}%
\pgfpathclose%
\pgfusepath{fill}%
\end{pgfscope}%
\begin{pgfscope}%
\pgfpathrectangle{\pgfqpoint{6.084781in}{2.227611in}}{\pgfqpoint{0.824468in}{0.462000in}}%
\pgfusepath{clip}%
\pgfsetbuttcap%
\pgfsetmiterjoin%
\definecolor{currentfill}{rgb}{0.121569,0.466667,0.705882}%
\pgfsetfillcolor{currentfill}%
\pgfsetfillopacity{0.500000}%
\pgfsetlinewidth{1.003750pt}%
\definecolor{currentstroke}{rgb}{0.000000,0.000000,0.000000}%
\pgfsetstrokecolor{currentstroke}%
\pgfsetdash{}{0pt}%
\pgfpathmoveto{\pgfqpoint{6.122257in}{2.227611in}}%
\pgfpathlineto{\pgfqpoint{6.272160in}{2.227611in}}%
\pgfpathlineto{\pgfqpoint{6.272160in}{2.667611in}}%
\pgfpathlineto{\pgfqpoint{6.122257in}{2.667611in}}%
\pgfpathlineto{\pgfqpoint{6.122257in}{2.227611in}}%
\pgfpathclose%
\pgfusepath{stroke,fill}%
\end{pgfscope}%
\begin{pgfscope}%
\pgfpathrectangle{\pgfqpoint{6.084781in}{2.227611in}}{\pgfqpoint{0.824468in}{0.462000in}}%
\pgfusepath{clip}%
\pgfsetbuttcap%
\pgfsetmiterjoin%
\definecolor{currentfill}{rgb}{0.121569,0.466667,0.705882}%
\pgfsetfillcolor{currentfill}%
\pgfsetfillopacity{0.500000}%
\pgfsetlinewidth{1.003750pt}%
\definecolor{currentstroke}{rgb}{0.000000,0.000000,0.000000}%
\pgfsetstrokecolor{currentstroke}%
\pgfsetdash{}{0pt}%
\pgfpathmoveto{\pgfqpoint{6.272160in}{2.227611in}}%
\pgfpathlineto{\pgfqpoint{6.422064in}{2.227611in}}%
\pgfpathlineto{\pgfqpoint{6.422064in}{2.317407in}}%
\pgfpathlineto{\pgfqpoint{6.272160in}{2.317407in}}%
\pgfpathlineto{\pgfqpoint{6.272160in}{2.227611in}}%
\pgfpathclose%
\pgfusepath{stroke,fill}%
\end{pgfscope}%
\begin{pgfscope}%
\pgfpathrectangle{\pgfqpoint{6.084781in}{2.227611in}}{\pgfqpoint{0.824468in}{0.462000in}}%
\pgfusepath{clip}%
\pgfsetbuttcap%
\pgfsetmiterjoin%
\definecolor{currentfill}{rgb}{0.121569,0.466667,0.705882}%
\pgfsetfillcolor{currentfill}%
\pgfsetfillopacity{0.500000}%
\pgfsetlinewidth{1.003750pt}%
\definecolor{currentstroke}{rgb}{0.000000,0.000000,0.000000}%
\pgfsetstrokecolor{currentstroke}%
\pgfsetdash{}{0pt}%
\pgfpathmoveto{\pgfqpoint{6.422064in}{2.227611in}}%
\pgfpathlineto{\pgfqpoint{6.571967in}{2.227611in}}%
\pgfpathlineto{\pgfqpoint{6.571967in}{2.287475in}}%
\pgfpathlineto{\pgfqpoint{6.422064in}{2.287475in}}%
\pgfpathlineto{\pgfqpoint{6.422064in}{2.227611in}}%
\pgfpathclose%
\pgfusepath{stroke,fill}%
\end{pgfscope}%
\begin{pgfscope}%
\pgfpathrectangle{\pgfqpoint{6.084781in}{2.227611in}}{\pgfqpoint{0.824468in}{0.462000in}}%
\pgfusepath{clip}%
\pgfsetbuttcap%
\pgfsetmiterjoin%
\definecolor{currentfill}{rgb}{0.121569,0.466667,0.705882}%
\pgfsetfillcolor{currentfill}%
\pgfsetfillopacity{0.500000}%
\pgfsetlinewidth{1.003750pt}%
\definecolor{currentstroke}{rgb}{0.000000,0.000000,0.000000}%
\pgfsetstrokecolor{currentstroke}%
\pgfsetdash{}{0pt}%
\pgfpathmoveto{\pgfqpoint{6.571967in}{2.227611in}}%
\pgfpathlineto{\pgfqpoint{6.721870in}{2.227611in}}%
\pgfpathlineto{\pgfqpoint{6.721870in}{2.242577in}}%
\pgfpathlineto{\pgfqpoint{6.571967in}{2.242577in}}%
\pgfpathlineto{\pgfqpoint{6.571967in}{2.227611in}}%
\pgfpathclose%
\pgfusepath{stroke,fill}%
\end{pgfscope}%
\begin{pgfscope}%
\pgfpathrectangle{\pgfqpoint{6.084781in}{2.227611in}}{\pgfqpoint{0.824468in}{0.462000in}}%
\pgfusepath{clip}%
\pgfsetbuttcap%
\pgfsetmiterjoin%
\definecolor{currentfill}{rgb}{0.121569,0.466667,0.705882}%
\pgfsetfillcolor{currentfill}%
\pgfsetfillopacity{0.500000}%
\pgfsetlinewidth{1.003750pt}%
\definecolor{currentstroke}{rgb}{0.000000,0.000000,0.000000}%
\pgfsetstrokecolor{currentstroke}%
\pgfsetdash{}{0pt}%
\pgfpathmoveto{\pgfqpoint{6.721870in}{2.227611in}}%
\pgfpathlineto{\pgfqpoint{6.871774in}{2.227611in}}%
\pgfpathlineto{\pgfqpoint{6.871774in}{2.248563in}}%
\pgfpathlineto{\pgfqpoint{6.721870in}{2.248563in}}%
\pgfpathlineto{\pgfqpoint{6.721870in}{2.227611in}}%
\pgfpathclose%
\pgfusepath{stroke,fill}%
\end{pgfscope}%
\begin{pgfscope}%
\pgfsetrectcap%
\pgfsetmiterjoin%
\pgfsetlinewidth{0.803000pt}%
\definecolor{currentstroke}{rgb}{0.000000,0.000000,0.000000}%
\pgfsetstrokecolor{currentstroke}%
\pgfsetdash{}{0pt}%
\pgfpathmoveto{\pgfqpoint{6.084781in}{2.227611in}}%
\pgfpathlineto{\pgfqpoint{6.084781in}{2.689611in}}%
\pgfusepath{stroke}%
\end{pgfscope}%
\begin{pgfscope}%
\pgfsetrectcap%
\pgfsetmiterjoin%
\pgfsetlinewidth{0.803000pt}%
\definecolor{currentstroke}{rgb}{0.000000,0.000000,0.000000}%
\pgfsetstrokecolor{currentstroke}%
\pgfsetdash{}{0pt}%
\pgfpathmoveto{\pgfqpoint{6.909249in}{2.227611in}}%
\pgfpathlineto{\pgfqpoint{6.909249in}{2.689611in}}%
\pgfusepath{stroke}%
\end{pgfscope}%
\begin{pgfscope}%
\pgfsetrectcap%
\pgfsetmiterjoin%
\pgfsetlinewidth{0.803000pt}%
\definecolor{currentstroke}{rgb}{0.000000,0.000000,0.000000}%
\pgfsetstrokecolor{currentstroke}%
\pgfsetdash{}{0pt}%
\pgfpathmoveto{\pgfqpoint{6.084781in}{2.227611in}}%
\pgfpathlineto{\pgfqpoint{6.909249in}{2.227611in}}%
\pgfusepath{stroke}%
\end{pgfscope}%
\begin{pgfscope}%
\pgfsetrectcap%
\pgfsetmiterjoin%
\pgfsetlinewidth{0.803000pt}%
\definecolor{currentstroke}{rgb}{0.000000,0.000000,0.000000}%
\pgfsetstrokecolor{currentstroke}%
\pgfsetdash{}{0pt}%
\pgfpathmoveto{\pgfqpoint{6.084781in}{2.689611in}}%
\pgfpathlineto{\pgfqpoint{6.909249in}{2.689611in}}%
\pgfusepath{stroke}%
\end{pgfscope}%
\begin{pgfscope}%
\definecolor{textcolor}{rgb}{0.000000,0.000000,0.000000}%
\pgfsetstrokecolor{textcolor}%
\pgfsetfillcolor{textcolor}%
\pgftext[x=6.497015in,y=2.772944in,,base]{\color{textcolor}\rmfamily\fontsize{11.000000}{13.200000}\selectfont Mgen}%
\end{pgfscope}%
\begin{pgfscope}%
\pgfsetbuttcap%
\pgfsetmiterjoin%
\definecolor{currentfill}{rgb}{1.000000,1.000000,1.000000}%
\pgfsetfillcolor{currentfill}%
\pgfsetlinewidth{0.000000pt}%
\definecolor{currentstroke}{rgb}{0.000000,0.000000,0.000000}%
\pgfsetstrokecolor{currentstroke}%
\pgfsetstrokeopacity{0.000000}%
\pgfsetdash{}{0pt}%
\pgfpathmoveto{\pgfqpoint{7.074143in}{2.227611in}}%
\pgfpathlineto{\pgfqpoint{7.898611in}{2.227611in}}%
\pgfpathlineto{\pgfqpoint{7.898611in}{2.689611in}}%
\pgfpathlineto{\pgfqpoint{7.074143in}{2.689611in}}%
\pgfpathlineto{\pgfqpoint{7.074143in}{2.227611in}}%
\pgfpathclose%
\pgfusepath{fill}%
\end{pgfscope}%
\begin{pgfscope}%
\pgfpathrectangle{\pgfqpoint{7.074143in}{2.227611in}}{\pgfqpoint{0.824468in}{0.462000in}}%
\pgfusepath{clip}%
\pgfsetbuttcap%
\pgfsetmiterjoin%
\definecolor{currentfill}{rgb}{0.121569,0.466667,0.705882}%
\pgfsetfillcolor{currentfill}%
\pgfsetfillopacity{0.500000}%
\pgfsetlinewidth{1.003750pt}%
\definecolor{currentstroke}{rgb}{0.000000,0.000000,0.000000}%
\pgfsetstrokecolor{currentstroke}%
\pgfsetdash{}{0pt}%
\pgfpathmoveto{\pgfqpoint{7.111619in}{2.227611in}}%
\pgfpathlineto{\pgfqpoint{7.261522in}{2.227611in}}%
\pgfpathlineto{\pgfqpoint{7.261522in}{2.233517in}}%
\pgfpathlineto{\pgfqpoint{7.111619in}{2.233517in}}%
\pgfpathlineto{\pgfqpoint{7.111619in}{2.227611in}}%
\pgfpathclose%
\pgfusepath{stroke,fill}%
\end{pgfscope}%
\begin{pgfscope}%
\pgfpathrectangle{\pgfqpoint{7.074143in}{2.227611in}}{\pgfqpoint{0.824468in}{0.462000in}}%
\pgfusepath{clip}%
\pgfsetbuttcap%
\pgfsetmiterjoin%
\definecolor{currentfill}{rgb}{0.121569,0.466667,0.705882}%
\pgfsetfillcolor{currentfill}%
\pgfsetfillopacity{0.500000}%
\pgfsetlinewidth{1.003750pt}%
\definecolor{currentstroke}{rgb}{0.000000,0.000000,0.000000}%
\pgfsetstrokecolor{currentstroke}%
\pgfsetdash{}{0pt}%
\pgfpathmoveto{\pgfqpoint{7.261522in}{2.227611in}}%
\pgfpathlineto{\pgfqpoint{7.411425in}{2.227611in}}%
\pgfpathlineto{\pgfqpoint{7.411425in}{2.257141in}}%
\pgfpathlineto{\pgfqpoint{7.261522in}{2.257141in}}%
\pgfpathlineto{\pgfqpoint{7.261522in}{2.227611in}}%
\pgfpathclose%
\pgfusepath{stroke,fill}%
\end{pgfscope}%
\begin{pgfscope}%
\pgfpathrectangle{\pgfqpoint{7.074143in}{2.227611in}}{\pgfqpoint{0.824468in}{0.462000in}}%
\pgfusepath{clip}%
\pgfsetbuttcap%
\pgfsetmiterjoin%
\definecolor{currentfill}{rgb}{0.121569,0.466667,0.705882}%
\pgfsetfillcolor{currentfill}%
\pgfsetfillopacity{0.500000}%
\pgfsetlinewidth{1.003750pt}%
\definecolor{currentstroke}{rgb}{0.000000,0.000000,0.000000}%
\pgfsetstrokecolor{currentstroke}%
\pgfsetdash{}{0pt}%
\pgfpathmoveto{\pgfqpoint{7.411425in}{2.227611in}}%
\pgfpathlineto{\pgfqpoint{7.561329in}{2.227611in}}%
\pgfpathlineto{\pgfqpoint{7.561329in}{2.274859in}}%
\pgfpathlineto{\pgfqpoint{7.411425in}{2.274859in}}%
\pgfpathlineto{\pgfqpoint{7.411425in}{2.227611in}}%
\pgfpathclose%
\pgfusepath{stroke,fill}%
\end{pgfscope}%
\begin{pgfscope}%
\pgfpathrectangle{\pgfqpoint{7.074143in}{2.227611in}}{\pgfqpoint{0.824468in}{0.462000in}}%
\pgfusepath{clip}%
\pgfsetbuttcap%
\pgfsetmiterjoin%
\definecolor{currentfill}{rgb}{0.121569,0.466667,0.705882}%
\pgfsetfillcolor{currentfill}%
\pgfsetfillopacity{0.500000}%
\pgfsetlinewidth{1.003750pt}%
\definecolor{currentstroke}{rgb}{0.000000,0.000000,0.000000}%
\pgfsetstrokecolor{currentstroke}%
\pgfsetdash{}{0pt}%
\pgfpathmoveto{\pgfqpoint{7.561329in}{2.227611in}}%
\pgfpathlineto{\pgfqpoint{7.711232in}{2.227611in}}%
\pgfpathlineto{\pgfqpoint{7.711232in}{2.428416in}}%
\pgfpathlineto{\pgfqpoint{7.561329in}{2.428416in}}%
\pgfpathlineto{\pgfqpoint{7.561329in}{2.227611in}}%
\pgfpathclose%
\pgfusepath{stroke,fill}%
\end{pgfscope}%
\begin{pgfscope}%
\pgfpathrectangle{\pgfqpoint{7.074143in}{2.227611in}}{\pgfqpoint{0.824468in}{0.462000in}}%
\pgfusepath{clip}%
\pgfsetbuttcap%
\pgfsetmiterjoin%
\definecolor{currentfill}{rgb}{0.121569,0.466667,0.705882}%
\pgfsetfillcolor{currentfill}%
\pgfsetfillopacity{0.500000}%
\pgfsetlinewidth{1.003750pt}%
\definecolor{currentstroke}{rgb}{0.000000,0.000000,0.000000}%
\pgfsetstrokecolor{currentstroke}%
\pgfsetdash{}{0pt}%
\pgfpathmoveto{\pgfqpoint{7.711232in}{2.227611in}}%
\pgfpathlineto{\pgfqpoint{7.861135in}{2.227611in}}%
\pgfpathlineto{\pgfqpoint{7.861135in}{2.667611in}}%
\pgfpathlineto{\pgfqpoint{7.711232in}{2.667611in}}%
\pgfpathlineto{\pgfqpoint{7.711232in}{2.227611in}}%
\pgfpathclose%
\pgfusepath{stroke,fill}%
\end{pgfscope}%
\begin{pgfscope}%
\pgfsetrectcap%
\pgfsetmiterjoin%
\pgfsetlinewidth{0.803000pt}%
\definecolor{currentstroke}{rgb}{0.000000,0.000000,0.000000}%
\pgfsetstrokecolor{currentstroke}%
\pgfsetdash{}{0pt}%
\pgfpathmoveto{\pgfqpoint{7.074143in}{2.227611in}}%
\pgfpathlineto{\pgfqpoint{7.074143in}{2.689611in}}%
\pgfusepath{stroke}%
\end{pgfscope}%
\begin{pgfscope}%
\pgfsetrectcap%
\pgfsetmiterjoin%
\pgfsetlinewidth{0.803000pt}%
\definecolor{currentstroke}{rgb}{0.000000,0.000000,0.000000}%
\pgfsetstrokecolor{currentstroke}%
\pgfsetdash{}{0pt}%
\pgfpathmoveto{\pgfqpoint{7.898611in}{2.227611in}}%
\pgfpathlineto{\pgfqpoint{7.898611in}{2.689611in}}%
\pgfusepath{stroke}%
\end{pgfscope}%
\begin{pgfscope}%
\pgfsetrectcap%
\pgfsetmiterjoin%
\pgfsetlinewidth{0.803000pt}%
\definecolor{currentstroke}{rgb}{0.000000,0.000000,0.000000}%
\pgfsetstrokecolor{currentstroke}%
\pgfsetdash{}{0pt}%
\pgfpathmoveto{\pgfqpoint{7.074143in}{2.227611in}}%
\pgfpathlineto{\pgfqpoint{7.898611in}{2.227611in}}%
\pgfusepath{stroke}%
\end{pgfscope}%
\begin{pgfscope}%
\pgfsetrectcap%
\pgfsetmiterjoin%
\pgfsetlinewidth{0.803000pt}%
\definecolor{currentstroke}{rgb}{0.000000,0.000000,0.000000}%
\pgfsetstrokecolor{currentstroke}%
\pgfsetdash{}{0pt}%
\pgfpathmoveto{\pgfqpoint{7.074143in}{2.689611in}}%
\pgfpathlineto{\pgfqpoint{7.898611in}{2.689611in}}%
\pgfusepath{stroke}%
\end{pgfscope}%
\begin{pgfscope}%
\definecolor{textcolor}{rgb}{0.000000,0.000000,0.000000}%
\pgfsetstrokecolor{textcolor}%
\pgfsetfillcolor{textcolor}%
\pgftext[x=7.486377in,y=2.772944in,,base]{\color{textcolor}\rmfamily\fontsize{11.000000}{13.200000}\selectfont Zen'Up}%
\end{pgfscope}%
\begin{pgfscope}%
\pgfsetbuttcap%
\pgfsetmiterjoin%
\definecolor{currentfill}{rgb}{1.000000,1.000000,1.000000}%
\pgfsetfillcolor{currentfill}%
\pgfsetlinewidth{0.000000pt}%
\definecolor{currentstroke}{rgb}{0.000000,0.000000,0.000000}%
\pgfsetstrokecolor{currentstroke}%
\pgfsetstrokeopacity{0.000000}%
\pgfsetdash{}{0pt}%
\pgfpathmoveto{\pgfqpoint{0.148611in}{1.534611in}}%
\pgfpathlineto{\pgfqpoint{0.973079in}{1.534611in}}%
\pgfpathlineto{\pgfqpoint{0.973079in}{1.996611in}}%
\pgfpathlineto{\pgfqpoint{0.148611in}{1.996611in}}%
\pgfpathlineto{\pgfqpoint{0.148611in}{1.534611in}}%
\pgfpathclose%
\pgfusepath{fill}%
\end{pgfscope}%
\begin{pgfscope}%
\pgfpathrectangle{\pgfqpoint{0.148611in}{1.534611in}}{\pgfqpoint{0.824468in}{0.462000in}}%
\pgfusepath{clip}%
\pgfsetbuttcap%
\pgfsetmiterjoin%
\definecolor{currentfill}{rgb}{0.121569,0.466667,0.705882}%
\pgfsetfillcolor{currentfill}%
\pgfsetfillopacity{0.500000}%
\pgfsetlinewidth{1.003750pt}%
\definecolor{currentstroke}{rgb}{0.000000,0.000000,0.000000}%
\pgfsetstrokecolor{currentstroke}%
\pgfsetdash{}{0pt}%
\pgfpathmoveto{\pgfqpoint{0.186087in}{1.534611in}}%
\pgfpathlineto{\pgfqpoint{0.335990in}{1.534611in}}%
\pgfpathlineto{\pgfqpoint{0.335990in}{1.630703in}}%
\pgfpathlineto{\pgfqpoint{0.186087in}{1.630703in}}%
\pgfpathlineto{\pgfqpoint{0.186087in}{1.534611in}}%
\pgfpathclose%
\pgfusepath{stroke,fill}%
\end{pgfscope}%
\begin{pgfscope}%
\pgfpathrectangle{\pgfqpoint{0.148611in}{1.534611in}}{\pgfqpoint{0.824468in}{0.462000in}}%
\pgfusepath{clip}%
\pgfsetbuttcap%
\pgfsetmiterjoin%
\definecolor{currentfill}{rgb}{0.121569,0.466667,0.705882}%
\pgfsetfillcolor{currentfill}%
\pgfsetfillopacity{0.500000}%
\pgfsetlinewidth{1.003750pt}%
\definecolor{currentstroke}{rgb}{0.000000,0.000000,0.000000}%
\pgfsetstrokecolor{currentstroke}%
\pgfsetdash{}{0pt}%
\pgfpathmoveto{\pgfqpoint{0.335990in}{1.534611in}}%
\pgfpathlineto{\pgfqpoint{0.485894in}{1.534611in}}%
\pgfpathlineto{\pgfqpoint{0.485894in}{1.618059in}}%
\pgfpathlineto{\pgfqpoint{0.335990in}{1.618059in}}%
\pgfpathlineto{\pgfqpoint{0.335990in}{1.534611in}}%
\pgfpathclose%
\pgfusepath{stroke,fill}%
\end{pgfscope}%
\begin{pgfscope}%
\pgfpathrectangle{\pgfqpoint{0.148611in}{1.534611in}}{\pgfqpoint{0.824468in}{0.462000in}}%
\pgfusepath{clip}%
\pgfsetbuttcap%
\pgfsetmiterjoin%
\definecolor{currentfill}{rgb}{0.121569,0.466667,0.705882}%
\pgfsetfillcolor{currentfill}%
\pgfsetfillopacity{0.500000}%
\pgfsetlinewidth{1.003750pt}%
\definecolor{currentstroke}{rgb}{0.000000,0.000000,0.000000}%
\pgfsetstrokecolor{currentstroke}%
\pgfsetdash{}{0pt}%
\pgfpathmoveto{\pgfqpoint{0.485894in}{1.534611in}}%
\pgfpathlineto{\pgfqpoint{0.635797in}{1.534611in}}%
\pgfpathlineto{\pgfqpoint{0.635797in}{1.792542in}}%
\pgfpathlineto{\pgfqpoint{0.485894in}{1.792542in}}%
\pgfpathlineto{\pgfqpoint{0.485894in}{1.534611in}}%
\pgfpathclose%
\pgfusepath{stroke,fill}%
\end{pgfscope}%
\begin{pgfscope}%
\pgfpathrectangle{\pgfqpoint{0.148611in}{1.534611in}}{\pgfqpoint{0.824468in}{0.462000in}}%
\pgfusepath{clip}%
\pgfsetbuttcap%
\pgfsetmiterjoin%
\definecolor{currentfill}{rgb}{0.121569,0.466667,0.705882}%
\pgfsetfillcolor{currentfill}%
\pgfsetfillopacity{0.500000}%
\pgfsetlinewidth{1.003750pt}%
\definecolor{currentstroke}{rgb}{0.000000,0.000000,0.000000}%
\pgfsetstrokecolor{currentstroke}%
\pgfsetdash{}{0pt}%
\pgfpathmoveto{\pgfqpoint{0.635797in}{1.534611in}}%
\pgfpathlineto{\pgfqpoint{0.785700in}{1.534611in}}%
\pgfpathlineto{\pgfqpoint{0.785700in}{1.974611in}}%
\pgfpathlineto{\pgfqpoint{0.635797in}{1.974611in}}%
\pgfpathlineto{\pgfqpoint{0.635797in}{1.534611in}}%
\pgfpathclose%
\pgfusepath{stroke,fill}%
\end{pgfscope}%
\begin{pgfscope}%
\pgfpathrectangle{\pgfqpoint{0.148611in}{1.534611in}}{\pgfqpoint{0.824468in}{0.462000in}}%
\pgfusepath{clip}%
\pgfsetbuttcap%
\pgfsetmiterjoin%
\definecolor{currentfill}{rgb}{0.121569,0.466667,0.705882}%
\pgfsetfillcolor{currentfill}%
\pgfsetfillopacity{0.500000}%
\pgfsetlinewidth{1.003750pt}%
\definecolor{currentstroke}{rgb}{0.000000,0.000000,0.000000}%
\pgfsetstrokecolor{currentstroke}%
\pgfsetdash{}{0pt}%
\pgfpathmoveto{\pgfqpoint{0.785700in}{1.534611in}}%
\pgfpathlineto{\pgfqpoint{0.935603in}{1.534611in}}%
\pgfpathlineto{\pgfqpoint{0.935603in}{1.784956in}}%
\pgfpathlineto{\pgfqpoint{0.785700in}{1.784956in}}%
\pgfpathlineto{\pgfqpoint{0.785700in}{1.534611in}}%
\pgfpathclose%
\pgfusepath{stroke,fill}%
\end{pgfscope}%
\begin{pgfscope}%
\pgfsetrectcap%
\pgfsetmiterjoin%
\pgfsetlinewidth{0.803000pt}%
\definecolor{currentstroke}{rgb}{0.000000,0.000000,0.000000}%
\pgfsetstrokecolor{currentstroke}%
\pgfsetdash{}{0pt}%
\pgfpathmoveto{\pgfqpoint{0.148611in}{1.534611in}}%
\pgfpathlineto{\pgfqpoint{0.148611in}{1.996611in}}%
\pgfusepath{stroke}%
\end{pgfscope}%
\begin{pgfscope}%
\pgfsetrectcap%
\pgfsetmiterjoin%
\pgfsetlinewidth{0.803000pt}%
\definecolor{currentstroke}{rgb}{0.000000,0.000000,0.000000}%
\pgfsetstrokecolor{currentstroke}%
\pgfsetdash{}{0pt}%
\pgfpathmoveto{\pgfqpoint{0.973079in}{1.534611in}}%
\pgfpathlineto{\pgfqpoint{0.973079in}{1.996611in}}%
\pgfusepath{stroke}%
\end{pgfscope}%
\begin{pgfscope}%
\pgfsetrectcap%
\pgfsetmiterjoin%
\pgfsetlinewidth{0.803000pt}%
\definecolor{currentstroke}{rgb}{0.000000,0.000000,0.000000}%
\pgfsetstrokecolor{currentstroke}%
\pgfsetdash{}{0pt}%
\pgfpathmoveto{\pgfqpoint{0.148611in}{1.534611in}}%
\pgfpathlineto{\pgfqpoint{0.973079in}{1.534611in}}%
\pgfusepath{stroke}%
\end{pgfscope}%
\begin{pgfscope}%
\pgfsetrectcap%
\pgfsetmiterjoin%
\pgfsetlinewidth{0.803000pt}%
\definecolor{currentstroke}{rgb}{0.000000,0.000000,0.000000}%
\pgfsetstrokecolor{currentstroke}%
\pgfsetdash{}{0pt}%
\pgfpathmoveto{\pgfqpoint{0.148611in}{1.996611in}}%
\pgfpathlineto{\pgfqpoint{0.973079in}{1.996611in}}%
\pgfusepath{stroke}%
\end{pgfscope}%
\begin{pgfscope}%
\definecolor{textcolor}{rgb}{0.000000,0.000000,0.000000}%
\pgfsetstrokecolor{textcolor}%
\pgfsetfillcolor{textcolor}%
\pgftext[x=0.560845in,y=2.079944in,,base]{\color{textcolor}\rmfamily\fontsize{11.000000}{13.200000}\selectfont MGP}%
\end{pgfscope}%
\begin{pgfscope}%
\pgfsetbuttcap%
\pgfsetmiterjoin%
\definecolor{currentfill}{rgb}{1.000000,1.000000,1.000000}%
\pgfsetfillcolor{currentfill}%
\pgfsetlinewidth{0.000000pt}%
\definecolor{currentstroke}{rgb}{0.000000,0.000000,0.000000}%
\pgfsetstrokecolor{currentstroke}%
\pgfsetstrokeopacity{0.000000}%
\pgfsetdash{}{0pt}%
\pgfpathmoveto{\pgfqpoint{1.137973in}{1.534611in}}%
\pgfpathlineto{\pgfqpoint{1.962441in}{1.534611in}}%
\pgfpathlineto{\pgfqpoint{1.962441in}{1.996611in}}%
\pgfpathlineto{\pgfqpoint{1.137973in}{1.996611in}}%
\pgfpathlineto{\pgfqpoint{1.137973in}{1.534611in}}%
\pgfpathclose%
\pgfusepath{fill}%
\end{pgfscope}%
\begin{pgfscope}%
\pgfpathrectangle{\pgfqpoint{1.137973in}{1.534611in}}{\pgfqpoint{0.824468in}{0.462000in}}%
\pgfusepath{clip}%
\pgfsetbuttcap%
\pgfsetmiterjoin%
\definecolor{currentfill}{rgb}{0.121569,0.466667,0.705882}%
\pgfsetfillcolor{currentfill}%
\pgfsetfillopacity{0.500000}%
\pgfsetlinewidth{1.003750pt}%
\definecolor{currentstroke}{rgb}{0.000000,0.000000,0.000000}%
\pgfsetstrokecolor{currentstroke}%
\pgfsetdash{}{0pt}%
\pgfpathmoveto{\pgfqpoint{1.175449in}{1.534611in}}%
\pgfpathlineto{\pgfqpoint{1.325352in}{1.534611in}}%
\pgfpathlineto{\pgfqpoint{1.325352in}{1.974611in}}%
\pgfpathlineto{\pgfqpoint{1.175449in}{1.974611in}}%
\pgfpathlineto{\pgfqpoint{1.175449in}{1.534611in}}%
\pgfpathclose%
\pgfusepath{stroke,fill}%
\end{pgfscope}%
\begin{pgfscope}%
\pgfpathrectangle{\pgfqpoint{1.137973in}{1.534611in}}{\pgfqpoint{0.824468in}{0.462000in}}%
\pgfusepath{clip}%
\pgfsetbuttcap%
\pgfsetmiterjoin%
\definecolor{currentfill}{rgb}{0.121569,0.466667,0.705882}%
\pgfsetfillcolor{currentfill}%
\pgfsetfillopacity{0.500000}%
\pgfsetlinewidth{1.003750pt}%
\definecolor{currentstroke}{rgb}{0.000000,0.000000,0.000000}%
\pgfsetstrokecolor{currentstroke}%
\pgfsetdash{}{0pt}%
\pgfpathmoveto{\pgfqpoint{1.325352in}{1.534611in}}%
\pgfpathlineto{\pgfqpoint{1.475255in}{1.534611in}}%
\pgfpathlineto{\pgfqpoint{1.475255in}{1.669056in}}%
\pgfpathlineto{\pgfqpoint{1.325352in}{1.669056in}}%
\pgfpathlineto{\pgfqpoint{1.325352in}{1.534611in}}%
\pgfpathclose%
\pgfusepath{stroke,fill}%
\end{pgfscope}%
\begin{pgfscope}%
\pgfpathrectangle{\pgfqpoint{1.137973in}{1.534611in}}{\pgfqpoint{0.824468in}{0.462000in}}%
\pgfusepath{clip}%
\pgfsetbuttcap%
\pgfsetmiterjoin%
\definecolor{currentfill}{rgb}{0.121569,0.466667,0.705882}%
\pgfsetfillcolor{currentfill}%
\pgfsetfillopacity{0.500000}%
\pgfsetlinewidth{1.003750pt}%
\definecolor{currentstroke}{rgb}{0.000000,0.000000,0.000000}%
\pgfsetstrokecolor{currentstroke}%
\pgfsetdash{}{0pt}%
\pgfpathmoveto{\pgfqpoint{1.475255in}{1.534611in}}%
\pgfpathlineto{\pgfqpoint{1.625158in}{1.534611in}}%
\pgfpathlineto{\pgfqpoint{1.625158in}{1.656833in}}%
\pgfpathlineto{\pgfqpoint{1.475255in}{1.656833in}}%
\pgfpathlineto{\pgfqpoint{1.475255in}{1.534611in}}%
\pgfpathclose%
\pgfusepath{stroke,fill}%
\end{pgfscope}%
\begin{pgfscope}%
\pgfpathrectangle{\pgfqpoint{1.137973in}{1.534611in}}{\pgfqpoint{0.824468in}{0.462000in}}%
\pgfusepath{clip}%
\pgfsetbuttcap%
\pgfsetmiterjoin%
\definecolor{currentfill}{rgb}{0.121569,0.466667,0.705882}%
\pgfsetfillcolor{currentfill}%
\pgfsetfillopacity{0.500000}%
\pgfsetlinewidth{1.003750pt}%
\definecolor{currentstroke}{rgb}{0.000000,0.000000,0.000000}%
\pgfsetstrokecolor{currentstroke}%
\pgfsetdash{}{0pt}%
\pgfpathmoveto{\pgfqpoint{1.625158in}{1.534611in}}%
\pgfpathlineto{\pgfqpoint{1.775062in}{1.534611in}}%
\pgfpathlineto{\pgfqpoint{1.775062in}{1.595722in}}%
\pgfpathlineto{\pgfqpoint{1.625158in}{1.595722in}}%
\pgfpathlineto{\pgfqpoint{1.625158in}{1.534611in}}%
\pgfpathclose%
\pgfusepath{stroke,fill}%
\end{pgfscope}%
\begin{pgfscope}%
\pgfpathrectangle{\pgfqpoint{1.137973in}{1.534611in}}{\pgfqpoint{0.824468in}{0.462000in}}%
\pgfusepath{clip}%
\pgfsetbuttcap%
\pgfsetmiterjoin%
\definecolor{currentfill}{rgb}{0.121569,0.466667,0.705882}%
\pgfsetfillcolor{currentfill}%
\pgfsetfillopacity{0.500000}%
\pgfsetlinewidth{1.003750pt}%
\definecolor{currentstroke}{rgb}{0.000000,0.000000,0.000000}%
\pgfsetstrokecolor{currentstroke}%
\pgfsetdash{}{0pt}%
\pgfpathmoveto{\pgfqpoint{1.775062in}{1.534611in}}%
\pgfpathlineto{\pgfqpoint{1.924965in}{1.534611in}}%
\pgfpathlineto{\pgfqpoint{1.924965in}{1.546833in}}%
\pgfpathlineto{\pgfqpoint{1.775062in}{1.546833in}}%
\pgfpathlineto{\pgfqpoint{1.775062in}{1.534611in}}%
\pgfpathclose%
\pgfusepath{stroke,fill}%
\end{pgfscope}%
\begin{pgfscope}%
\pgfsetrectcap%
\pgfsetmiterjoin%
\pgfsetlinewidth{0.803000pt}%
\definecolor{currentstroke}{rgb}{0.000000,0.000000,0.000000}%
\pgfsetstrokecolor{currentstroke}%
\pgfsetdash{}{0pt}%
\pgfpathmoveto{\pgfqpoint{1.137973in}{1.534611in}}%
\pgfpathlineto{\pgfqpoint{1.137973in}{1.996611in}}%
\pgfusepath{stroke}%
\end{pgfscope}%
\begin{pgfscope}%
\pgfsetrectcap%
\pgfsetmiterjoin%
\pgfsetlinewidth{0.803000pt}%
\definecolor{currentstroke}{rgb}{0.000000,0.000000,0.000000}%
\pgfsetstrokecolor{currentstroke}%
\pgfsetdash{}{0pt}%
\pgfpathmoveto{\pgfqpoint{1.962441in}{1.534611in}}%
\pgfpathlineto{\pgfqpoint{1.962441in}{1.996611in}}%
\pgfusepath{stroke}%
\end{pgfscope}%
\begin{pgfscope}%
\pgfsetrectcap%
\pgfsetmiterjoin%
\pgfsetlinewidth{0.803000pt}%
\definecolor{currentstroke}{rgb}{0.000000,0.000000,0.000000}%
\pgfsetstrokecolor{currentstroke}%
\pgfsetdash{}{0pt}%
\pgfpathmoveto{\pgfqpoint{1.137973in}{1.534611in}}%
\pgfpathlineto{\pgfqpoint{1.962441in}{1.534611in}}%
\pgfusepath{stroke}%
\end{pgfscope}%
\begin{pgfscope}%
\pgfsetrectcap%
\pgfsetmiterjoin%
\pgfsetlinewidth{0.803000pt}%
\definecolor{currentstroke}{rgb}{0.000000,0.000000,0.000000}%
\pgfsetstrokecolor{currentstroke}%
\pgfsetdash{}{0pt}%
\pgfpathmoveto{\pgfqpoint{1.137973in}{1.996611in}}%
\pgfpathlineto{\pgfqpoint{1.962441in}{1.996611in}}%
\pgfusepath{stroke}%
\end{pgfscope}%
\begin{pgfscope}%
\definecolor{textcolor}{rgb}{0.000000,0.000000,0.000000}%
\pgfsetstrokecolor{textcolor}%
\pgfsetfillcolor{textcolor}%
\pgftext[x=1.550207in,y=2.079944in,,base]{\color{textcolor}\rmfamily\fontsize{11.000000}{13.200000}\selectfont Intériale}%
\end{pgfscope}%
\begin{pgfscope}%
\pgfsetbuttcap%
\pgfsetmiterjoin%
\definecolor{currentfill}{rgb}{1.000000,1.000000,1.000000}%
\pgfsetfillcolor{currentfill}%
\pgfsetlinewidth{0.000000pt}%
\definecolor{currentstroke}{rgb}{0.000000,0.000000,0.000000}%
\pgfsetstrokecolor{currentstroke}%
\pgfsetstrokeopacity{0.000000}%
\pgfsetdash{}{0pt}%
\pgfpathmoveto{\pgfqpoint{2.127335in}{1.534611in}}%
\pgfpathlineto{\pgfqpoint{2.951803in}{1.534611in}}%
\pgfpathlineto{\pgfqpoint{2.951803in}{1.996611in}}%
\pgfpathlineto{\pgfqpoint{2.127335in}{1.996611in}}%
\pgfpathlineto{\pgfqpoint{2.127335in}{1.534611in}}%
\pgfpathclose%
\pgfusepath{fill}%
\end{pgfscope}%
\begin{pgfscope}%
\pgfpathrectangle{\pgfqpoint{2.127335in}{1.534611in}}{\pgfqpoint{0.824468in}{0.462000in}}%
\pgfusepath{clip}%
\pgfsetbuttcap%
\pgfsetmiterjoin%
\definecolor{currentfill}{rgb}{0.121569,0.466667,0.705882}%
\pgfsetfillcolor{currentfill}%
\pgfsetfillopacity{0.500000}%
\pgfsetlinewidth{1.003750pt}%
\definecolor{currentstroke}{rgb}{0.000000,0.000000,0.000000}%
\pgfsetstrokecolor{currentstroke}%
\pgfsetdash{}{0pt}%
\pgfpathmoveto{\pgfqpoint{2.164810in}{1.534611in}}%
\pgfpathlineto{\pgfqpoint{2.314714in}{1.534611in}}%
\pgfpathlineto{\pgfqpoint{2.314714in}{1.974611in}}%
\pgfpathlineto{\pgfqpoint{2.164810in}{1.974611in}}%
\pgfpathlineto{\pgfqpoint{2.164810in}{1.534611in}}%
\pgfpathclose%
\pgfusepath{stroke,fill}%
\end{pgfscope}%
\begin{pgfscope}%
\pgfpathrectangle{\pgfqpoint{2.127335in}{1.534611in}}{\pgfqpoint{0.824468in}{0.462000in}}%
\pgfusepath{clip}%
\pgfsetbuttcap%
\pgfsetmiterjoin%
\definecolor{currentfill}{rgb}{0.121569,0.466667,0.705882}%
\pgfsetfillcolor{currentfill}%
\pgfsetfillopacity{0.500000}%
\pgfsetlinewidth{1.003750pt}%
\definecolor{currentstroke}{rgb}{0.000000,0.000000,0.000000}%
\pgfsetstrokecolor{currentstroke}%
\pgfsetdash{}{0pt}%
\pgfpathmoveto{\pgfqpoint{2.314714in}{1.534611in}}%
\pgfpathlineto{\pgfqpoint{2.464617in}{1.534611in}}%
\pgfpathlineto{\pgfqpoint{2.464617in}{1.637821in}}%
\pgfpathlineto{\pgfqpoint{2.314714in}{1.637821in}}%
\pgfpathlineto{\pgfqpoint{2.314714in}{1.534611in}}%
\pgfpathclose%
\pgfusepath{stroke,fill}%
\end{pgfscope}%
\begin{pgfscope}%
\pgfpathrectangle{\pgfqpoint{2.127335in}{1.534611in}}{\pgfqpoint{0.824468in}{0.462000in}}%
\pgfusepath{clip}%
\pgfsetbuttcap%
\pgfsetmiterjoin%
\definecolor{currentfill}{rgb}{0.121569,0.466667,0.705882}%
\pgfsetfillcolor{currentfill}%
\pgfsetfillopacity{0.500000}%
\pgfsetlinewidth{1.003750pt}%
\definecolor{currentstroke}{rgb}{0.000000,0.000000,0.000000}%
\pgfsetstrokecolor{currentstroke}%
\pgfsetdash{}{0pt}%
\pgfpathmoveto{\pgfqpoint{2.464617in}{1.534611in}}%
\pgfpathlineto{\pgfqpoint{2.614520in}{1.534611in}}%
\pgfpathlineto{\pgfqpoint{2.614520in}{1.659549in}}%
\pgfpathlineto{\pgfqpoint{2.464617in}{1.659549in}}%
\pgfpathlineto{\pgfqpoint{2.464617in}{1.534611in}}%
\pgfpathclose%
\pgfusepath{stroke,fill}%
\end{pgfscope}%
\begin{pgfscope}%
\pgfpathrectangle{\pgfqpoint{2.127335in}{1.534611in}}{\pgfqpoint{0.824468in}{0.462000in}}%
\pgfusepath{clip}%
\pgfsetbuttcap%
\pgfsetmiterjoin%
\definecolor{currentfill}{rgb}{0.121569,0.466667,0.705882}%
\pgfsetfillcolor{currentfill}%
\pgfsetfillopacity{0.500000}%
\pgfsetlinewidth{1.003750pt}%
\definecolor{currentstroke}{rgb}{0.000000,0.000000,0.000000}%
\pgfsetstrokecolor{currentstroke}%
\pgfsetdash{}{0pt}%
\pgfpathmoveto{\pgfqpoint{2.614520in}{1.534611in}}%
\pgfpathlineto{\pgfqpoint{2.764423in}{1.534611in}}%
\pgfpathlineto{\pgfqpoint{2.764423in}{1.806216in}}%
\pgfpathlineto{\pgfqpoint{2.614520in}{1.806216in}}%
\pgfpathlineto{\pgfqpoint{2.614520in}{1.534611in}}%
\pgfpathclose%
\pgfusepath{stroke,fill}%
\end{pgfscope}%
\begin{pgfscope}%
\pgfpathrectangle{\pgfqpoint{2.127335in}{1.534611in}}{\pgfqpoint{0.824468in}{0.462000in}}%
\pgfusepath{clip}%
\pgfsetbuttcap%
\pgfsetmiterjoin%
\definecolor{currentfill}{rgb}{0.121569,0.466667,0.705882}%
\pgfsetfillcolor{currentfill}%
\pgfsetfillopacity{0.500000}%
\pgfsetlinewidth{1.003750pt}%
\definecolor{currentstroke}{rgb}{0.000000,0.000000,0.000000}%
\pgfsetstrokecolor{currentstroke}%
\pgfsetdash{}{0pt}%
\pgfpathmoveto{\pgfqpoint{2.764423in}{1.534611in}}%
\pgfpathlineto{\pgfqpoint{2.914327in}{1.534611in}}%
\pgfpathlineto{\pgfqpoint{2.914327in}{1.746463in}}%
\pgfpathlineto{\pgfqpoint{2.764423in}{1.746463in}}%
\pgfpathlineto{\pgfqpoint{2.764423in}{1.534611in}}%
\pgfpathclose%
\pgfusepath{stroke,fill}%
\end{pgfscope}%
\begin{pgfscope}%
\pgfsetrectcap%
\pgfsetmiterjoin%
\pgfsetlinewidth{0.803000pt}%
\definecolor{currentstroke}{rgb}{0.000000,0.000000,0.000000}%
\pgfsetstrokecolor{currentstroke}%
\pgfsetdash{}{0pt}%
\pgfpathmoveto{\pgfqpoint{2.127335in}{1.534611in}}%
\pgfpathlineto{\pgfqpoint{2.127335in}{1.996611in}}%
\pgfusepath{stroke}%
\end{pgfscope}%
\begin{pgfscope}%
\pgfsetrectcap%
\pgfsetmiterjoin%
\pgfsetlinewidth{0.803000pt}%
\definecolor{currentstroke}{rgb}{0.000000,0.000000,0.000000}%
\pgfsetstrokecolor{currentstroke}%
\pgfsetdash{}{0pt}%
\pgfpathmoveto{\pgfqpoint{2.951803in}{1.534611in}}%
\pgfpathlineto{\pgfqpoint{2.951803in}{1.996611in}}%
\pgfusepath{stroke}%
\end{pgfscope}%
\begin{pgfscope}%
\pgfsetrectcap%
\pgfsetmiterjoin%
\pgfsetlinewidth{0.803000pt}%
\definecolor{currentstroke}{rgb}{0.000000,0.000000,0.000000}%
\pgfsetstrokecolor{currentstroke}%
\pgfsetdash{}{0pt}%
\pgfpathmoveto{\pgfqpoint{2.127335in}{1.534611in}}%
\pgfpathlineto{\pgfqpoint{2.951803in}{1.534611in}}%
\pgfusepath{stroke}%
\end{pgfscope}%
\begin{pgfscope}%
\pgfsetrectcap%
\pgfsetmiterjoin%
\pgfsetlinewidth{0.803000pt}%
\definecolor{currentstroke}{rgb}{0.000000,0.000000,0.000000}%
\pgfsetstrokecolor{currentstroke}%
\pgfsetdash{}{0pt}%
\pgfpathmoveto{\pgfqpoint{2.127335in}{1.996611in}}%
\pgfpathlineto{\pgfqpoint{2.951803in}{1.996611in}}%
\pgfusepath{stroke}%
\end{pgfscope}%
\begin{pgfscope}%
\definecolor{textcolor}{rgb}{0.000000,0.000000,0.000000}%
\pgfsetstrokecolor{textcolor}%
\pgfsetfillcolor{textcolor}%
\pgftext[x=2.539569in,y=2.079944in,,base]{\color{textcolor}\rmfamily\fontsize{11.000000}{13.200000}\selectfont Généra...}%
\end{pgfscope}%
\begin{pgfscope}%
\pgfsetbuttcap%
\pgfsetmiterjoin%
\definecolor{currentfill}{rgb}{1.000000,1.000000,1.000000}%
\pgfsetfillcolor{currentfill}%
\pgfsetlinewidth{0.000000pt}%
\definecolor{currentstroke}{rgb}{0.000000,0.000000,0.000000}%
\pgfsetstrokecolor{currentstroke}%
\pgfsetstrokeopacity{0.000000}%
\pgfsetdash{}{0pt}%
\pgfpathmoveto{\pgfqpoint{3.116696in}{1.534611in}}%
\pgfpathlineto{\pgfqpoint{3.941164in}{1.534611in}}%
\pgfpathlineto{\pgfqpoint{3.941164in}{1.996611in}}%
\pgfpathlineto{\pgfqpoint{3.116696in}{1.996611in}}%
\pgfpathlineto{\pgfqpoint{3.116696in}{1.534611in}}%
\pgfpathclose%
\pgfusepath{fill}%
\end{pgfscope}%
\begin{pgfscope}%
\pgfpathrectangle{\pgfqpoint{3.116696in}{1.534611in}}{\pgfqpoint{0.824468in}{0.462000in}}%
\pgfusepath{clip}%
\pgfsetbuttcap%
\pgfsetmiterjoin%
\definecolor{currentfill}{rgb}{0.121569,0.466667,0.705882}%
\pgfsetfillcolor{currentfill}%
\pgfsetfillopacity{0.500000}%
\pgfsetlinewidth{1.003750pt}%
\definecolor{currentstroke}{rgb}{0.000000,0.000000,0.000000}%
\pgfsetstrokecolor{currentstroke}%
\pgfsetdash{}{0pt}%
\pgfpathmoveto{\pgfqpoint{3.154172in}{1.534611in}}%
\pgfpathlineto{\pgfqpoint{3.304075in}{1.534611in}}%
\pgfpathlineto{\pgfqpoint{3.304075in}{1.974611in}}%
\pgfpathlineto{\pgfqpoint{3.154172in}{1.974611in}}%
\pgfpathlineto{\pgfqpoint{3.154172in}{1.534611in}}%
\pgfpathclose%
\pgfusepath{stroke,fill}%
\end{pgfscope}%
\begin{pgfscope}%
\pgfpathrectangle{\pgfqpoint{3.116696in}{1.534611in}}{\pgfqpoint{0.824468in}{0.462000in}}%
\pgfusepath{clip}%
\pgfsetbuttcap%
\pgfsetmiterjoin%
\definecolor{currentfill}{rgb}{0.121569,0.466667,0.705882}%
\pgfsetfillcolor{currentfill}%
\pgfsetfillopacity{0.500000}%
\pgfsetlinewidth{1.003750pt}%
\definecolor{currentstroke}{rgb}{0.000000,0.000000,0.000000}%
\pgfsetstrokecolor{currentstroke}%
\pgfsetdash{}{0pt}%
\pgfpathmoveto{\pgfqpoint{3.304075in}{1.534611in}}%
\pgfpathlineto{\pgfqpoint{3.453979in}{1.534611in}}%
\pgfpathlineto{\pgfqpoint{3.453979in}{1.593278in}}%
\pgfpathlineto{\pgfqpoint{3.304075in}{1.593278in}}%
\pgfpathlineto{\pgfqpoint{3.304075in}{1.534611in}}%
\pgfpathclose%
\pgfusepath{stroke,fill}%
\end{pgfscope}%
\begin{pgfscope}%
\pgfpathrectangle{\pgfqpoint{3.116696in}{1.534611in}}{\pgfqpoint{0.824468in}{0.462000in}}%
\pgfusepath{clip}%
\pgfsetbuttcap%
\pgfsetmiterjoin%
\definecolor{currentfill}{rgb}{0.121569,0.466667,0.705882}%
\pgfsetfillcolor{currentfill}%
\pgfsetfillopacity{0.500000}%
\pgfsetlinewidth{1.003750pt}%
\definecolor{currentstroke}{rgb}{0.000000,0.000000,0.000000}%
\pgfsetstrokecolor{currentstroke}%
\pgfsetdash{}{0pt}%
\pgfpathmoveto{\pgfqpoint{3.453979in}{1.534611in}}%
\pgfpathlineto{\pgfqpoint{3.603882in}{1.534611in}}%
\pgfpathlineto{\pgfqpoint{3.603882in}{1.584897in}}%
\pgfpathlineto{\pgfqpoint{3.453979in}{1.584897in}}%
\pgfpathlineto{\pgfqpoint{3.453979in}{1.534611in}}%
\pgfpathclose%
\pgfusepath{stroke,fill}%
\end{pgfscope}%
\begin{pgfscope}%
\pgfpathrectangle{\pgfqpoint{3.116696in}{1.534611in}}{\pgfqpoint{0.824468in}{0.462000in}}%
\pgfusepath{clip}%
\pgfsetbuttcap%
\pgfsetmiterjoin%
\definecolor{currentfill}{rgb}{0.121569,0.466667,0.705882}%
\pgfsetfillcolor{currentfill}%
\pgfsetfillopacity{0.500000}%
\pgfsetlinewidth{1.003750pt}%
\definecolor{currentstroke}{rgb}{0.000000,0.000000,0.000000}%
\pgfsetstrokecolor{currentstroke}%
\pgfsetdash{}{0pt}%
\pgfpathmoveto{\pgfqpoint{3.603882in}{1.534611in}}%
\pgfpathlineto{\pgfqpoint{3.753785in}{1.534611in}}%
\pgfpathlineto{\pgfqpoint{3.753785in}{1.540897in}}%
\pgfpathlineto{\pgfqpoint{3.603882in}{1.540897in}}%
\pgfpathlineto{\pgfqpoint{3.603882in}{1.534611in}}%
\pgfpathclose%
\pgfusepath{stroke,fill}%
\end{pgfscope}%
\begin{pgfscope}%
\pgfpathrectangle{\pgfqpoint{3.116696in}{1.534611in}}{\pgfqpoint{0.824468in}{0.462000in}}%
\pgfusepath{clip}%
\pgfsetbuttcap%
\pgfsetmiterjoin%
\definecolor{currentfill}{rgb}{0.121569,0.466667,0.705882}%
\pgfsetfillcolor{currentfill}%
\pgfsetfillopacity{0.500000}%
\pgfsetlinewidth{1.003750pt}%
\definecolor{currentstroke}{rgb}{0.000000,0.000000,0.000000}%
\pgfsetstrokecolor{currentstroke}%
\pgfsetdash{}{0pt}%
\pgfpathmoveto{\pgfqpoint{3.753785in}{1.534611in}}%
\pgfpathlineto{\pgfqpoint{3.903688in}{1.534611in}}%
\pgfpathlineto{\pgfqpoint{3.903688in}{1.534611in}}%
\pgfpathlineto{\pgfqpoint{3.753785in}{1.534611in}}%
\pgfpathlineto{\pgfqpoint{3.753785in}{1.534611in}}%
\pgfpathclose%
\pgfusepath{stroke,fill}%
\end{pgfscope}%
\begin{pgfscope}%
\pgfsetrectcap%
\pgfsetmiterjoin%
\pgfsetlinewidth{0.803000pt}%
\definecolor{currentstroke}{rgb}{0.000000,0.000000,0.000000}%
\pgfsetstrokecolor{currentstroke}%
\pgfsetdash{}{0pt}%
\pgfpathmoveto{\pgfqpoint{3.116696in}{1.534611in}}%
\pgfpathlineto{\pgfqpoint{3.116696in}{1.996611in}}%
\pgfusepath{stroke}%
\end{pgfscope}%
\begin{pgfscope}%
\pgfsetrectcap%
\pgfsetmiterjoin%
\pgfsetlinewidth{0.803000pt}%
\definecolor{currentstroke}{rgb}{0.000000,0.000000,0.000000}%
\pgfsetstrokecolor{currentstroke}%
\pgfsetdash{}{0pt}%
\pgfpathmoveto{\pgfqpoint{3.941164in}{1.534611in}}%
\pgfpathlineto{\pgfqpoint{3.941164in}{1.996611in}}%
\pgfusepath{stroke}%
\end{pgfscope}%
\begin{pgfscope}%
\pgfsetrectcap%
\pgfsetmiterjoin%
\pgfsetlinewidth{0.803000pt}%
\definecolor{currentstroke}{rgb}{0.000000,0.000000,0.000000}%
\pgfsetstrokecolor{currentstroke}%
\pgfsetdash{}{0pt}%
\pgfpathmoveto{\pgfqpoint{3.116696in}{1.534611in}}%
\pgfpathlineto{\pgfqpoint{3.941164in}{1.534611in}}%
\pgfusepath{stroke}%
\end{pgfscope}%
\begin{pgfscope}%
\pgfsetrectcap%
\pgfsetmiterjoin%
\pgfsetlinewidth{0.803000pt}%
\definecolor{currentstroke}{rgb}{0.000000,0.000000,0.000000}%
\pgfsetstrokecolor{currentstroke}%
\pgfsetdash{}{0pt}%
\pgfpathmoveto{\pgfqpoint{3.116696in}{1.996611in}}%
\pgfpathlineto{\pgfqpoint{3.941164in}{1.996611in}}%
\pgfusepath{stroke}%
\end{pgfscope}%
\begin{pgfscope}%
\definecolor{textcolor}{rgb}{0.000000,0.000000,0.000000}%
\pgfsetstrokecolor{textcolor}%
\pgfsetfillcolor{textcolor}%
\pgftext[x=3.528930in,y=2.079944in,,base]{\color{textcolor}\rmfamily\fontsize{11.000000}{13.200000}\selectfont Cardif}%
\end{pgfscope}%
\begin{pgfscope}%
\pgfsetbuttcap%
\pgfsetmiterjoin%
\definecolor{currentfill}{rgb}{1.000000,1.000000,1.000000}%
\pgfsetfillcolor{currentfill}%
\pgfsetlinewidth{0.000000pt}%
\definecolor{currentstroke}{rgb}{0.000000,0.000000,0.000000}%
\pgfsetstrokecolor{currentstroke}%
\pgfsetstrokeopacity{0.000000}%
\pgfsetdash{}{0pt}%
\pgfpathmoveto{\pgfqpoint{4.106058in}{1.534611in}}%
\pgfpathlineto{\pgfqpoint{4.930526in}{1.534611in}}%
\pgfpathlineto{\pgfqpoint{4.930526in}{1.996611in}}%
\pgfpathlineto{\pgfqpoint{4.106058in}{1.996611in}}%
\pgfpathlineto{\pgfqpoint{4.106058in}{1.534611in}}%
\pgfpathclose%
\pgfusepath{fill}%
\end{pgfscope}%
\begin{pgfscope}%
\pgfpathrectangle{\pgfqpoint{4.106058in}{1.534611in}}{\pgfqpoint{0.824468in}{0.462000in}}%
\pgfusepath{clip}%
\pgfsetbuttcap%
\pgfsetmiterjoin%
\definecolor{currentfill}{rgb}{0.121569,0.466667,0.705882}%
\pgfsetfillcolor{currentfill}%
\pgfsetfillopacity{0.500000}%
\pgfsetlinewidth{1.003750pt}%
\definecolor{currentstroke}{rgb}{0.000000,0.000000,0.000000}%
\pgfsetstrokecolor{currentstroke}%
\pgfsetdash{}{0pt}%
\pgfpathmoveto{\pgfqpoint{4.143534in}{1.534611in}}%
\pgfpathlineto{\pgfqpoint{4.293437in}{1.534611in}}%
\pgfpathlineto{\pgfqpoint{4.293437in}{1.706961in}}%
\pgfpathlineto{\pgfqpoint{4.143534in}{1.706961in}}%
\pgfpathlineto{\pgfqpoint{4.143534in}{1.534611in}}%
\pgfpathclose%
\pgfusepath{stroke,fill}%
\end{pgfscope}%
\begin{pgfscope}%
\pgfpathrectangle{\pgfqpoint{4.106058in}{1.534611in}}{\pgfqpoint{0.824468in}{0.462000in}}%
\pgfusepath{clip}%
\pgfsetbuttcap%
\pgfsetmiterjoin%
\definecolor{currentfill}{rgb}{0.121569,0.466667,0.705882}%
\pgfsetfillcolor{currentfill}%
\pgfsetfillopacity{0.500000}%
\pgfsetlinewidth{1.003750pt}%
\definecolor{currentstroke}{rgb}{0.000000,0.000000,0.000000}%
\pgfsetstrokecolor{currentstroke}%
\pgfsetdash{}{0pt}%
\pgfpathmoveto{\pgfqpoint{4.293437in}{1.534611in}}%
\pgfpathlineto{\pgfqpoint{4.443340in}{1.534611in}}%
\pgfpathlineto{\pgfqpoint{4.443340in}{1.650187in}}%
\pgfpathlineto{\pgfqpoint{4.293437in}{1.650187in}}%
\pgfpathlineto{\pgfqpoint{4.293437in}{1.534611in}}%
\pgfpathclose%
\pgfusepath{stroke,fill}%
\end{pgfscope}%
\begin{pgfscope}%
\pgfpathrectangle{\pgfqpoint{4.106058in}{1.534611in}}{\pgfqpoint{0.824468in}{0.462000in}}%
\pgfusepath{clip}%
\pgfsetbuttcap%
\pgfsetmiterjoin%
\definecolor{currentfill}{rgb}{0.121569,0.466667,0.705882}%
\pgfsetfillcolor{currentfill}%
\pgfsetfillopacity{0.500000}%
\pgfsetlinewidth{1.003750pt}%
\definecolor{currentstroke}{rgb}{0.000000,0.000000,0.000000}%
\pgfsetstrokecolor{currentstroke}%
\pgfsetdash{}{0pt}%
\pgfpathmoveto{\pgfqpoint{4.443340in}{1.534611in}}%
\pgfpathlineto{\pgfqpoint{4.593244in}{1.534611in}}%
\pgfpathlineto{\pgfqpoint{4.593244in}{1.818482in}}%
\pgfpathlineto{\pgfqpoint{4.443340in}{1.818482in}}%
\pgfpathlineto{\pgfqpoint{4.443340in}{1.534611in}}%
\pgfpathclose%
\pgfusepath{stroke,fill}%
\end{pgfscope}%
\begin{pgfscope}%
\pgfpathrectangle{\pgfqpoint{4.106058in}{1.534611in}}{\pgfqpoint{0.824468in}{0.462000in}}%
\pgfusepath{clip}%
\pgfsetbuttcap%
\pgfsetmiterjoin%
\definecolor{currentfill}{rgb}{0.121569,0.466667,0.705882}%
\pgfsetfillcolor{currentfill}%
\pgfsetfillopacity{0.500000}%
\pgfsetlinewidth{1.003750pt}%
\definecolor{currentstroke}{rgb}{0.000000,0.000000,0.000000}%
\pgfsetstrokecolor{currentstroke}%
\pgfsetdash{}{0pt}%
\pgfpathmoveto{\pgfqpoint{4.593244in}{1.534611in}}%
\pgfpathlineto{\pgfqpoint{4.743147in}{1.534611in}}%
\pgfpathlineto{\pgfqpoint{4.743147in}{1.974611in}}%
\pgfpathlineto{\pgfqpoint{4.593244in}{1.974611in}}%
\pgfpathlineto{\pgfqpoint{4.593244in}{1.534611in}}%
\pgfpathclose%
\pgfusepath{stroke,fill}%
\end{pgfscope}%
\begin{pgfscope}%
\pgfpathrectangle{\pgfqpoint{4.106058in}{1.534611in}}{\pgfqpoint{0.824468in}{0.462000in}}%
\pgfusepath{clip}%
\pgfsetbuttcap%
\pgfsetmiterjoin%
\definecolor{currentfill}{rgb}{0.121569,0.466667,0.705882}%
\pgfsetfillcolor{currentfill}%
\pgfsetfillopacity{0.500000}%
\pgfsetlinewidth{1.003750pt}%
\definecolor{currentstroke}{rgb}{0.000000,0.000000,0.000000}%
\pgfsetstrokecolor{currentstroke}%
\pgfsetdash{}{0pt}%
\pgfpathmoveto{\pgfqpoint{4.743147in}{1.534611in}}%
\pgfpathlineto{\pgfqpoint{4.893050in}{1.534611in}}%
\pgfpathlineto{\pgfqpoint{4.893050in}{1.857007in}}%
\pgfpathlineto{\pgfqpoint{4.743147in}{1.857007in}}%
\pgfpathlineto{\pgfqpoint{4.743147in}{1.534611in}}%
\pgfpathclose%
\pgfusepath{stroke,fill}%
\end{pgfscope}%
\begin{pgfscope}%
\pgfsetrectcap%
\pgfsetmiterjoin%
\pgfsetlinewidth{0.803000pt}%
\definecolor{currentstroke}{rgb}{0.000000,0.000000,0.000000}%
\pgfsetstrokecolor{currentstroke}%
\pgfsetdash{}{0pt}%
\pgfpathmoveto{\pgfqpoint{4.106058in}{1.534611in}}%
\pgfpathlineto{\pgfqpoint{4.106058in}{1.996611in}}%
\pgfusepath{stroke}%
\end{pgfscope}%
\begin{pgfscope}%
\pgfsetrectcap%
\pgfsetmiterjoin%
\pgfsetlinewidth{0.803000pt}%
\definecolor{currentstroke}{rgb}{0.000000,0.000000,0.000000}%
\pgfsetstrokecolor{currentstroke}%
\pgfsetdash{}{0pt}%
\pgfpathmoveto{\pgfqpoint{4.930526in}{1.534611in}}%
\pgfpathlineto{\pgfqpoint{4.930526in}{1.996611in}}%
\pgfusepath{stroke}%
\end{pgfscope}%
\begin{pgfscope}%
\pgfsetrectcap%
\pgfsetmiterjoin%
\pgfsetlinewidth{0.803000pt}%
\definecolor{currentstroke}{rgb}{0.000000,0.000000,0.000000}%
\pgfsetstrokecolor{currentstroke}%
\pgfsetdash{}{0pt}%
\pgfpathmoveto{\pgfqpoint{4.106058in}{1.534611in}}%
\pgfpathlineto{\pgfqpoint{4.930526in}{1.534611in}}%
\pgfusepath{stroke}%
\end{pgfscope}%
\begin{pgfscope}%
\pgfsetrectcap%
\pgfsetmiterjoin%
\pgfsetlinewidth{0.803000pt}%
\definecolor{currentstroke}{rgb}{0.000000,0.000000,0.000000}%
\pgfsetstrokecolor{currentstroke}%
\pgfsetdash{}{0pt}%
\pgfpathmoveto{\pgfqpoint{4.106058in}{1.996611in}}%
\pgfpathlineto{\pgfqpoint{4.930526in}{1.996611in}}%
\pgfusepath{stroke}%
\end{pgfscope}%
\begin{pgfscope}%
\definecolor{textcolor}{rgb}{0.000000,0.000000,0.000000}%
\pgfsetstrokecolor{textcolor}%
\pgfsetfillcolor{textcolor}%
\pgftext[x=4.518292in,y=2.079944in,,base]{\color{textcolor}\rmfamily\fontsize{11.000000}{13.200000}\selectfont Santiane}%
\end{pgfscope}%
\begin{pgfscope}%
\pgfsetbuttcap%
\pgfsetmiterjoin%
\definecolor{currentfill}{rgb}{1.000000,1.000000,1.000000}%
\pgfsetfillcolor{currentfill}%
\pgfsetlinewidth{0.000000pt}%
\definecolor{currentstroke}{rgb}{0.000000,0.000000,0.000000}%
\pgfsetstrokecolor{currentstroke}%
\pgfsetstrokeopacity{0.000000}%
\pgfsetdash{}{0pt}%
\pgfpathmoveto{\pgfqpoint{5.095420in}{1.534611in}}%
\pgfpathlineto{\pgfqpoint{5.919888in}{1.534611in}}%
\pgfpathlineto{\pgfqpoint{5.919888in}{1.996611in}}%
\pgfpathlineto{\pgfqpoint{5.095420in}{1.996611in}}%
\pgfpathlineto{\pgfqpoint{5.095420in}{1.534611in}}%
\pgfpathclose%
\pgfusepath{fill}%
\end{pgfscope}%
\begin{pgfscope}%
\pgfpathrectangle{\pgfqpoint{5.095420in}{1.534611in}}{\pgfqpoint{0.824468in}{0.462000in}}%
\pgfusepath{clip}%
\pgfsetbuttcap%
\pgfsetmiterjoin%
\definecolor{currentfill}{rgb}{0.121569,0.466667,0.705882}%
\pgfsetfillcolor{currentfill}%
\pgfsetfillopacity{0.500000}%
\pgfsetlinewidth{1.003750pt}%
\definecolor{currentstroke}{rgb}{0.000000,0.000000,0.000000}%
\pgfsetstrokecolor{currentstroke}%
\pgfsetdash{}{0pt}%
\pgfpathmoveto{\pgfqpoint{5.132895in}{1.534611in}}%
\pgfpathlineto{\pgfqpoint{5.282799in}{1.534611in}}%
\pgfpathlineto{\pgfqpoint{5.282799in}{1.974611in}}%
\pgfpathlineto{\pgfqpoint{5.132895in}{1.974611in}}%
\pgfpathlineto{\pgfqpoint{5.132895in}{1.534611in}}%
\pgfpathclose%
\pgfusepath{stroke,fill}%
\end{pgfscope}%
\begin{pgfscope}%
\pgfpathrectangle{\pgfqpoint{5.095420in}{1.534611in}}{\pgfqpoint{0.824468in}{0.462000in}}%
\pgfusepath{clip}%
\pgfsetbuttcap%
\pgfsetmiterjoin%
\definecolor{currentfill}{rgb}{0.121569,0.466667,0.705882}%
\pgfsetfillcolor{currentfill}%
\pgfsetfillopacity{0.500000}%
\pgfsetlinewidth{1.003750pt}%
\definecolor{currentstroke}{rgb}{0.000000,0.000000,0.000000}%
\pgfsetstrokecolor{currentstroke}%
\pgfsetdash{}{0pt}%
\pgfpathmoveto{\pgfqpoint{5.282799in}{1.534611in}}%
\pgfpathlineto{\pgfqpoint{5.432702in}{1.534611in}}%
\pgfpathlineto{\pgfqpoint{5.432702in}{1.661048in}}%
\pgfpathlineto{\pgfqpoint{5.282799in}{1.661048in}}%
\pgfpathlineto{\pgfqpoint{5.282799in}{1.534611in}}%
\pgfpathclose%
\pgfusepath{stroke,fill}%
\end{pgfscope}%
\begin{pgfscope}%
\pgfpathrectangle{\pgfqpoint{5.095420in}{1.534611in}}{\pgfqpoint{0.824468in}{0.462000in}}%
\pgfusepath{clip}%
\pgfsetbuttcap%
\pgfsetmiterjoin%
\definecolor{currentfill}{rgb}{0.121569,0.466667,0.705882}%
\pgfsetfillcolor{currentfill}%
\pgfsetfillopacity{0.500000}%
\pgfsetlinewidth{1.003750pt}%
\definecolor{currentstroke}{rgb}{0.000000,0.000000,0.000000}%
\pgfsetstrokecolor{currentstroke}%
\pgfsetdash{}{0pt}%
\pgfpathmoveto{\pgfqpoint{5.432702in}{1.534611in}}%
\pgfpathlineto{\pgfqpoint{5.582605in}{1.534611in}}%
\pgfpathlineto{\pgfqpoint{5.582605in}{1.585186in}}%
\pgfpathlineto{\pgfqpoint{5.432702in}{1.585186in}}%
\pgfpathlineto{\pgfqpoint{5.432702in}{1.534611in}}%
\pgfpathclose%
\pgfusepath{stroke,fill}%
\end{pgfscope}%
\begin{pgfscope}%
\pgfpathrectangle{\pgfqpoint{5.095420in}{1.534611in}}{\pgfqpoint{0.824468in}{0.462000in}}%
\pgfusepath{clip}%
\pgfsetbuttcap%
\pgfsetmiterjoin%
\definecolor{currentfill}{rgb}{0.121569,0.466667,0.705882}%
\pgfsetfillcolor{currentfill}%
\pgfsetfillopacity{0.500000}%
\pgfsetlinewidth{1.003750pt}%
\definecolor{currentstroke}{rgb}{0.000000,0.000000,0.000000}%
\pgfsetstrokecolor{currentstroke}%
\pgfsetdash{}{0pt}%
\pgfpathmoveto{\pgfqpoint{5.582605in}{1.534611in}}%
\pgfpathlineto{\pgfqpoint{5.732509in}{1.534611in}}%
\pgfpathlineto{\pgfqpoint{5.732509in}{1.544726in}}%
\pgfpathlineto{\pgfqpoint{5.582605in}{1.544726in}}%
\pgfpathlineto{\pgfqpoint{5.582605in}{1.534611in}}%
\pgfpathclose%
\pgfusepath{stroke,fill}%
\end{pgfscope}%
\begin{pgfscope}%
\pgfpathrectangle{\pgfqpoint{5.095420in}{1.534611in}}{\pgfqpoint{0.824468in}{0.462000in}}%
\pgfusepath{clip}%
\pgfsetbuttcap%
\pgfsetmiterjoin%
\definecolor{currentfill}{rgb}{0.121569,0.466667,0.705882}%
\pgfsetfillcolor{currentfill}%
\pgfsetfillopacity{0.500000}%
\pgfsetlinewidth{1.003750pt}%
\definecolor{currentstroke}{rgb}{0.000000,0.000000,0.000000}%
\pgfsetstrokecolor{currentstroke}%
\pgfsetdash{}{0pt}%
\pgfpathmoveto{\pgfqpoint{5.732509in}{1.534611in}}%
\pgfpathlineto{\pgfqpoint{5.882412in}{1.534611in}}%
\pgfpathlineto{\pgfqpoint{5.882412in}{1.575071in}}%
\pgfpathlineto{\pgfqpoint{5.732509in}{1.575071in}}%
\pgfpathlineto{\pgfqpoint{5.732509in}{1.534611in}}%
\pgfpathclose%
\pgfusepath{stroke,fill}%
\end{pgfscope}%
\begin{pgfscope}%
\pgfsetrectcap%
\pgfsetmiterjoin%
\pgfsetlinewidth{0.803000pt}%
\definecolor{currentstroke}{rgb}{0.000000,0.000000,0.000000}%
\pgfsetstrokecolor{currentstroke}%
\pgfsetdash{}{0pt}%
\pgfpathmoveto{\pgfqpoint{5.095420in}{1.534611in}}%
\pgfpathlineto{\pgfqpoint{5.095420in}{1.996611in}}%
\pgfusepath{stroke}%
\end{pgfscope}%
\begin{pgfscope}%
\pgfsetrectcap%
\pgfsetmiterjoin%
\pgfsetlinewidth{0.803000pt}%
\definecolor{currentstroke}{rgb}{0.000000,0.000000,0.000000}%
\pgfsetstrokecolor{currentstroke}%
\pgfsetdash{}{0pt}%
\pgfpathmoveto{\pgfqpoint{5.919888in}{1.534611in}}%
\pgfpathlineto{\pgfqpoint{5.919888in}{1.996611in}}%
\pgfusepath{stroke}%
\end{pgfscope}%
\begin{pgfscope}%
\pgfsetrectcap%
\pgfsetmiterjoin%
\pgfsetlinewidth{0.803000pt}%
\definecolor{currentstroke}{rgb}{0.000000,0.000000,0.000000}%
\pgfsetstrokecolor{currentstroke}%
\pgfsetdash{}{0pt}%
\pgfpathmoveto{\pgfqpoint{5.095420in}{1.534611in}}%
\pgfpathlineto{\pgfqpoint{5.919888in}{1.534611in}}%
\pgfusepath{stroke}%
\end{pgfscope}%
\begin{pgfscope}%
\pgfsetrectcap%
\pgfsetmiterjoin%
\pgfsetlinewidth{0.803000pt}%
\definecolor{currentstroke}{rgb}{0.000000,0.000000,0.000000}%
\pgfsetstrokecolor{currentstroke}%
\pgfsetdash{}{0pt}%
\pgfpathmoveto{\pgfqpoint{5.095420in}{1.996611in}}%
\pgfpathlineto{\pgfqpoint{5.919888in}{1.996611in}}%
\pgfusepath{stroke}%
\end{pgfscope}%
\begin{pgfscope}%
\definecolor{textcolor}{rgb}{0.000000,0.000000,0.000000}%
\pgfsetstrokecolor{textcolor}%
\pgfsetfillcolor{textcolor}%
\pgftext[x=5.507654in,y=2.079944in,,base]{\color{textcolor}\rmfamily\fontsize{11.000000}{13.200000}\selectfont Eca As...}%
\end{pgfscope}%
\begin{pgfscope}%
\pgfsetbuttcap%
\pgfsetmiterjoin%
\definecolor{currentfill}{rgb}{1.000000,1.000000,1.000000}%
\pgfsetfillcolor{currentfill}%
\pgfsetlinewidth{0.000000pt}%
\definecolor{currentstroke}{rgb}{0.000000,0.000000,0.000000}%
\pgfsetstrokecolor{currentstroke}%
\pgfsetstrokeopacity{0.000000}%
\pgfsetdash{}{0pt}%
\pgfpathmoveto{\pgfqpoint{6.084781in}{1.534611in}}%
\pgfpathlineto{\pgfqpoint{6.909249in}{1.534611in}}%
\pgfpathlineto{\pgfqpoint{6.909249in}{1.996611in}}%
\pgfpathlineto{\pgfqpoint{6.084781in}{1.996611in}}%
\pgfpathlineto{\pgfqpoint{6.084781in}{1.534611in}}%
\pgfpathclose%
\pgfusepath{fill}%
\end{pgfscope}%
\begin{pgfscope}%
\pgfpathrectangle{\pgfqpoint{6.084781in}{1.534611in}}{\pgfqpoint{0.824468in}{0.462000in}}%
\pgfusepath{clip}%
\pgfsetbuttcap%
\pgfsetmiterjoin%
\definecolor{currentfill}{rgb}{0.121569,0.466667,0.705882}%
\pgfsetfillcolor{currentfill}%
\pgfsetfillopacity{0.500000}%
\pgfsetlinewidth{1.003750pt}%
\definecolor{currentstroke}{rgb}{0.000000,0.000000,0.000000}%
\pgfsetstrokecolor{currentstroke}%
\pgfsetdash{}{0pt}%
\pgfpathmoveto{\pgfqpoint{6.122257in}{1.534611in}}%
\pgfpathlineto{\pgfqpoint{6.272160in}{1.534611in}}%
\pgfpathlineto{\pgfqpoint{6.272160in}{1.974611in}}%
\pgfpathlineto{\pgfqpoint{6.122257in}{1.974611in}}%
\pgfpathlineto{\pgfqpoint{6.122257in}{1.534611in}}%
\pgfpathclose%
\pgfusepath{stroke,fill}%
\end{pgfscope}%
\begin{pgfscope}%
\pgfpathrectangle{\pgfqpoint{6.084781in}{1.534611in}}{\pgfqpoint{0.824468in}{0.462000in}}%
\pgfusepath{clip}%
\pgfsetbuttcap%
\pgfsetmiterjoin%
\definecolor{currentfill}{rgb}{0.121569,0.466667,0.705882}%
\pgfsetfillcolor{currentfill}%
\pgfsetfillopacity{0.500000}%
\pgfsetlinewidth{1.003750pt}%
\definecolor{currentstroke}{rgb}{0.000000,0.000000,0.000000}%
\pgfsetstrokecolor{currentstroke}%
\pgfsetdash{}{0pt}%
\pgfpathmoveto{\pgfqpoint{6.272160in}{1.534611in}}%
\pgfpathlineto{\pgfqpoint{6.422064in}{1.534611in}}%
\pgfpathlineto{\pgfqpoint{6.422064in}{1.742611in}}%
\pgfpathlineto{\pgfqpoint{6.272160in}{1.742611in}}%
\pgfpathlineto{\pgfqpoint{6.272160in}{1.534611in}}%
\pgfpathclose%
\pgfusepath{stroke,fill}%
\end{pgfscope}%
\begin{pgfscope}%
\pgfpathrectangle{\pgfqpoint{6.084781in}{1.534611in}}{\pgfqpoint{0.824468in}{0.462000in}}%
\pgfusepath{clip}%
\pgfsetbuttcap%
\pgfsetmiterjoin%
\definecolor{currentfill}{rgb}{0.121569,0.466667,0.705882}%
\pgfsetfillcolor{currentfill}%
\pgfsetfillopacity{0.500000}%
\pgfsetlinewidth{1.003750pt}%
\definecolor{currentstroke}{rgb}{0.000000,0.000000,0.000000}%
\pgfsetstrokecolor{currentstroke}%
\pgfsetdash{}{0pt}%
\pgfpathmoveto{\pgfqpoint{6.422064in}{1.534611in}}%
\pgfpathlineto{\pgfqpoint{6.571967in}{1.534611in}}%
\pgfpathlineto{\pgfqpoint{6.571967in}{1.574611in}}%
\pgfpathlineto{\pgfqpoint{6.422064in}{1.574611in}}%
\pgfpathlineto{\pgfqpoint{6.422064in}{1.534611in}}%
\pgfpathclose%
\pgfusepath{stroke,fill}%
\end{pgfscope}%
\begin{pgfscope}%
\pgfpathrectangle{\pgfqpoint{6.084781in}{1.534611in}}{\pgfqpoint{0.824468in}{0.462000in}}%
\pgfusepath{clip}%
\pgfsetbuttcap%
\pgfsetmiterjoin%
\definecolor{currentfill}{rgb}{0.121569,0.466667,0.705882}%
\pgfsetfillcolor{currentfill}%
\pgfsetfillopacity{0.500000}%
\pgfsetlinewidth{1.003750pt}%
\definecolor{currentstroke}{rgb}{0.000000,0.000000,0.000000}%
\pgfsetstrokecolor{currentstroke}%
\pgfsetdash{}{0pt}%
\pgfpathmoveto{\pgfqpoint{6.571967in}{1.534611in}}%
\pgfpathlineto{\pgfqpoint{6.721870in}{1.534611in}}%
\pgfpathlineto{\pgfqpoint{6.721870in}{1.582611in}}%
\pgfpathlineto{\pgfqpoint{6.571967in}{1.582611in}}%
\pgfpathlineto{\pgfqpoint{6.571967in}{1.534611in}}%
\pgfpathclose%
\pgfusepath{stroke,fill}%
\end{pgfscope}%
\begin{pgfscope}%
\pgfpathrectangle{\pgfqpoint{6.084781in}{1.534611in}}{\pgfqpoint{0.824468in}{0.462000in}}%
\pgfusepath{clip}%
\pgfsetbuttcap%
\pgfsetmiterjoin%
\definecolor{currentfill}{rgb}{0.121569,0.466667,0.705882}%
\pgfsetfillcolor{currentfill}%
\pgfsetfillopacity{0.500000}%
\pgfsetlinewidth{1.003750pt}%
\definecolor{currentstroke}{rgb}{0.000000,0.000000,0.000000}%
\pgfsetstrokecolor{currentstroke}%
\pgfsetdash{}{0pt}%
\pgfpathmoveto{\pgfqpoint{6.721870in}{1.534611in}}%
\pgfpathlineto{\pgfqpoint{6.871774in}{1.534611in}}%
\pgfpathlineto{\pgfqpoint{6.871774in}{1.574611in}}%
\pgfpathlineto{\pgfqpoint{6.721870in}{1.574611in}}%
\pgfpathlineto{\pgfqpoint{6.721870in}{1.534611in}}%
\pgfpathclose%
\pgfusepath{stroke,fill}%
\end{pgfscope}%
\begin{pgfscope}%
\pgfsetrectcap%
\pgfsetmiterjoin%
\pgfsetlinewidth{0.803000pt}%
\definecolor{currentstroke}{rgb}{0.000000,0.000000,0.000000}%
\pgfsetstrokecolor{currentstroke}%
\pgfsetdash{}{0pt}%
\pgfpathmoveto{\pgfqpoint{6.084781in}{1.534611in}}%
\pgfpathlineto{\pgfqpoint{6.084781in}{1.996611in}}%
\pgfusepath{stroke}%
\end{pgfscope}%
\begin{pgfscope}%
\pgfsetrectcap%
\pgfsetmiterjoin%
\pgfsetlinewidth{0.803000pt}%
\definecolor{currentstroke}{rgb}{0.000000,0.000000,0.000000}%
\pgfsetstrokecolor{currentstroke}%
\pgfsetdash{}{0pt}%
\pgfpathmoveto{\pgfqpoint{6.909249in}{1.534611in}}%
\pgfpathlineto{\pgfqpoint{6.909249in}{1.996611in}}%
\pgfusepath{stroke}%
\end{pgfscope}%
\begin{pgfscope}%
\pgfsetrectcap%
\pgfsetmiterjoin%
\pgfsetlinewidth{0.803000pt}%
\definecolor{currentstroke}{rgb}{0.000000,0.000000,0.000000}%
\pgfsetstrokecolor{currentstroke}%
\pgfsetdash{}{0pt}%
\pgfpathmoveto{\pgfqpoint{6.084781in}{1.534611in}}%
\pgfpathlineto{\pgfqpoint{6.909249in}{1.534611in}}%
\pgfusepath{stroke}%
\end{pgfscope}%
\begin{pgfscope}%
\pgfsetrectcap%
\pgfsetmiterjoin%
\pgfsetlinewidth{0.803000pt}%
\definecolor{currentstroke}{rgb}{0.000000,0.000000,0.000000}%
\pgfsetstrokecolor{currentstroke}%
\pgfsetdash{}{0pt}%
\pgfpathmoveto{\pgfqpoint{6.084781in}{1.996611in}}%
\pgfpathlineto{\pgfqpoint{6.909249in}{1.996611in}}%
\pgfusepath{stroke}%
\end{pgfscope}%
\begin{pgfscope}%
\definecolor{textcolor}{rgb}{0.000000,0.000000,0.000000}%
\pgfsetstrokecolor{textcolor}%
\pgfsetfillcolor{textcolor}%
\pgftext[x=6.497015in,y=2.079944in,,base]{\color{textcolor}\rmfamily\fontsize{11.000000}{13.200000}\selectfont Groupama}%
\end{pgfscope}%
\begin{pgfscope}%
\pgfsetbuttcap%
\pgfsetmiterjoin%
\definecolor{currentfill}{rgb}{1.000000,1.000000,1.000000}%
\pgfsetfillcolor{currentfill}%
\pgfsetlinewidth{0.000000pt}%
\definecolor{currentstroke}{rgb}{0.000000,0.000000,0.000000}%
\pgfsetstrokecolor{currentstroke}%
\pgfsetstrokeopacity{0.000000}%
\pgfsetdash{}{0pt}%
\pgfpathmoveto{\pgfqpoint{7.074143in}{1.534611in}}%
\pgfpathlineto{\pgfqpoint{7.898611in}{1.534611in}}%
\pgfpathlineto{\pgfqpoint{7.898611in}{1.996611in}}%
\pgfpathlineto{\pgfqpoint{7.074143in}{1.996611in}}%
\pgfpathlineto{\pgfqpoint{7.074143in}{1.534611in}}%
\pgfpathclose%
\pgfusepath{fill}%
\end{pgfscope}%
\begin{pgfscope}%
\pgfpathrectangle{\pgfqpoint{7.074143in}{1.534611in}}{\pgfqpoint{0.824468in}{0.462000in}}%
\pgfusepath{clip}%
\pgfsetbuttcap%
\pgfsetmiterjoin%
\definecolor{currentfill}{rgb}{0.121569,0.466667,0.705882}%
\pgfsetfillcolor{currentfill}%
\pgfsetfillopacity{0.500000}%
\pgfsetlinewidth{1.003750pt}%
\definecolor{currentstroke}{rgb}{0.000000,0.000000,0.000000}%
\pgfsetstrokecolor{currentstroke}%
\pgfsetdash{}{0pt}%
\pgfpathmoveto{\pgfqpoint{7.111619in}{1.534611in}}%
\pgfpathlineto{\pgfqpoint{7.261522in}{1.534611in}}%
\pgfpathlineto{\pgfqpoint{7.261522in}{1.974611in}}%
\pgfpathlineto{\pgfqpoint{7.111619in}{1.974611in}}%
\pgfpathlineto{\pgfqpoint{7.111619in}{1.534611in}}%
\pgfpathclose%
\pgfusepath{stroke,fill}%
\end{pgfscope}%
\begin{pgfscope}%
\pgfpathrectangle{\pgfqpoint{7.074143in}{1.534611in}}{\pgfqpoint{0.824468in}{0.462000in}}%
\pgfusepath{clip}%
\pgfsetbuttcap%
\pgfsetmiterjoin%
\definecolor{currentfill}{rgb}{0.121569,0.466667,0.705882}%
\pgfsetfillcolor{currentfill}%
\pgfsetfillopacity{0.500000}%
\pgfsetlinewidth{1.003750pt}%
\definecolor{currentstroke}{rgb}{0.000000,0.000000,0.000000}%
\pgfsetstrokecolor{currentstroke}%
\pgfsetdash{}{0pt}%
\pgfpathmoveto{\pgfqpoint{7.261522in}{1.534611in}}%
\pgfpathlineto{\pgfqpoint{7.411425in}{1.534611in}}%
\pgfpathlineto{\pgfqpoint{7.411425in}{1.735313in}}%
\pgfpathlineto{\pgfqpoint{7.261522in}{1.735313in}}%
\pgfpathlineto{\pgfqpoint{7.261522in}{1.534611in}}%
\pgfpathclose%
\pgfusepath{stroke,fill}%
\end{pgfscope}%
\begin{pgfscope}%
\pgfpathrectangle{\pgfqpoint{7.074143in}{1.534611in}}{\pgfqpoint{0.824468in}{0.462000in}}%
\pgfusepath{clip}%
\pgfsetbuttcap%
\pgfsetmiterjoin%
\definecolor{currentfill}{rgb}{0.121569,0.466667,0.705882}%
\pgfsetfillcolor{currentfill}%
\pgfsetfillopacity{0.500000}%
\pgfsetlinewidth{1.003750pt}%
\definecolor{currentstroke}{rgb}{0.000000,0.000000,0.000000}%
\pgfsetstrokecolor{currentstroke}%
\pgfsetdash{}{0pt}%
\pgfpathmoveto{\pgfqpoint{7.411425in}{1.534611in}}%
\pgfpathlineto{\pgfqpoint{7.561329in}{1.534611in}}%
\pgfpathlineto{\pgfqpoint{7.561329in}{1.650401in}}%
\pgfpathlineto{\pgfqpoint{7.411425in}{1.650401in}}%
\pgfpathlineto{\pgfqpoint{7.411425in}{1.534611in}}%
\pgfpathclose%
\pgfusepath{stroke,fill}%
\end{pgfscope}%
\begin{pgfscope}%
\pgfpathrectangle{\pgfqpoint{7.074143in}{1.534611in}}{\pgfqpoint{0.824468in}{0.462000in}}%
\pgfusepath{clip}%
\pgfsetbuttcap%
\pgfsetmiterjoin%
\definecolor{currentfill}{rgb}{0.121569,0.466667,0.705882}%
\pgfsetfillcolor{currentfill}%
\pgfsetfillopacity{0.500000}%
\pgfsetlinewidth{1.003750pt}%
\definecolor{currentstroke}{rgb}{0.000000,0.000000,0.000000}%
\pgfsetstrokecolor{currentstroke}%
\pgfsetdash{}{0pt}%
\pgfpathmoveto{\pgfqpoint{7.561329in}{1.534611in}}%
\pgfpathlineto{\pgfqpoint{7.711232in}{1.534611in}}%
\pgfpathlineto{\pgfqpoint{7.711232in}{1.573208in}}%
\pgfpathlineto{\pgfqpoint{7.561329in}{1.573208in}}%
\pgfpathlineto{\pgfqpoint{7.561329in}{1.534611in}}%
\pgfpathclose%
\pgfusepath{stroke,fill}%
\end{pgfscope}%
\begin{pgfscope}%
\pgfpathrectangle{\pgfqpoint{7.074143in}{1.534611in}}{\pgfqpoint{0.824468in}{0.462000in}}%
\pgfusepath{clip}%
\pgfsetbuttcap%
\pgfsetmiterjoin%
\definecolor{currentfill}{rgb}{0.121569,0.466667,0.705882}%
\pgfsetfillcolor{currentfill}%
\pgfsetfillopacity{0.500000}%
\pgfsetlinewidth{1.003750pt}%
\definecolor{currentstroke}{rgb}{0.000000,0.000000,0.000000}%
\pgfsetstrokecolor{currentstroke}%
\pgfsetdash{}{0pt}%
\pgfpathmoveto{\pgfqpoint{7.711232in}{1.534611in}}%
\pgfpathlineto{\pgfqpoint{7.861135in}{1.534611in}}%
\pgfpathlineto{\pgfqpoint{7.861135in}{1.588646in}}%
\pgfpathlineto{\pgfqpoint{7.711232in}{1.588646in}}%
\pgfpathlineto{\pgfqpoint{7.711232in}{1.534611in}}%
\pgfpathclose%
\pgfusepath{stroke,fill}%
\end{pgfscope}%
\begin{pgfscope}%
\pgfsetrectcap%
\pgfsetmiterjoin%
\pgfsetlinewidth{0.803000pt}%
\definecolor{currentstroke}{rgb}{0.000000,0.000000,0.000000}%
\pgfsetstrokecolor{currentstroke}%
\pgfsetdash{}{0pt}%
\pgfpathmoveto{\pgfqpoint{7.074143in}{1.534611in}}%
\pgfpathlineto{\pgfqpoint{7.074143in}{1.996611in}}%
\pgfusepath{stroke}%
\end{pgfscope}%
\begin{pgfscope}%
\pgfsetrectcap%
\pgfsetmiterjoin%
\pgfsetlinewidth{0.803000pt}%
\definecolor{currentstroke}{rgb}{0.000000,0.000000,0.000000}%
\pgfsetstrokecolor{currentstroke}%
\pgfsetdash{}{0pt}%
\pgfpathmoveto{\pgfqpoint{7.898611in}{1.534611in}}%
\pgfpathlineto{\pgfqpoint{7.898611in}{1.996611in}}%
\pgfusepath{stroke}%
\end{pgfscope}%
\begin{pgfscope}%
\pgfsetrectcap%
\pgfsetmiterjoin%
\pgfsetlinewidth{0.803000pt}%
\definecolor{currentstroke}{rgb}{0.000000,0.000000,0.000000}%
\pgfsetstrokecolor{currentstroke}%
\pgfsetdash{}{0pt}%
\pgfpathmoveto{\pgfqpoint{7.074143in}{1.534611in}}%
\pgfpathlineto{\pgfqpoint{7.898611in}{1.534611in}}%
\pgfusepath{stroke}%
\end{pgfscope}%
\begin{pgfscope}%
\pgfsetrectcap%
\pgfsetmiterjoin%
\pgfsetlinewidth{0.803000pt}%
\definecolor{currentstroke}{rgb}{0.000000,0.000000,0.000000}%
\pgfsetstrokecolor{currentstroke}%
\pgfsetdash{}{0pt}%
\pgfpathmoveto{\pgfqpoint{7.074143in}{1.996611in}}%
\pgfpathlineto{\pgfqpoint{7.898611in}{1.996611in}}%
\pgfusepath{stroke}%
\end{pgfscope}%
\begin{pgfscope}%
\definecolor{textcolor}{rgb}{0.000000,0.000000,0.000000}%
\pgfsetstrokecolor{textcolor}%
\pgfsetfillcolor{textcolor}%
\pgftext[x=7.486377in,y=2.079944in,,base]{\color{textcolor}\rmfamily\fontsize{11.000000}{13.200000}\selectfont Assur ...}%
\end{pgfscope}%
\begin{pgfscope}%
\pgfsetbuttcap%
\pgfsetmiterjoin%
\definecolor{currentfill}{rgb}{1.000000,1.000000,1.000000}%
\pgfsetfillcolor{currentfill}%
\pgfsetlinewidth{0.000000pt}%
\definecolor{currentstroke}{rgb}{0.000000,0.000000,0.000000}%
\pgfsetstrokecolor{currentstroke}%
\pgfsetstrokeopacity{0.000000}%
\pgfsetdash{}{0pt}%
\pgfpathmoveto{\pgfqpoint{0.148611in}{0.841611in}}%
\pgfpathlineto{\pgfqpoint{0.973079in}{0.841611in}}%
\pgfpathlineto{\pgfqpoint{0.973079in}{1.303611in}}%
\pgfpathlineto{\pgfqpoint{0.148611in}{1.303611in}}%
\pgfpathlineto{\pgfqpoint{0.148611in}{0.841611in}}%
\pgfpathclose%
\pgfusepath{fill}%
\end{pgfscope}%
\begin{pgfscope}%
\pgfpathrectangle{\pgfqpoint{0.148611in}{0.841611in}}{\pgfqpoint{0.824468in}{0.462000in}}%
\pgfusepath{clip}%
\pgfsetbuttcap%
\pgfsetmiterjoin%
\definecolor{currentfill}{rgb}{0.121569,0.466667,0.705882}%
\pgfsetfillcolor{currentfill}%
\pgfsetfillopacity{0.500000}%
\pgfsetlinewidth{1.003750pt}%
\definecolor{currentstroke}{rgb}{0.000000,0.000000,0.000000}%
\pgfsetstrokecolor{currentstroke}%
\pgfsetdash{}{0pt}%
\pgfpathmoveto{\pgfqpoint{0.186087in}{0.841611in}}%
\pgfpathlineto{\pgfqpoint{0.335990in}{0.841611in}}%
\pgfpathlineto{\pgfqpoint{0.335990in}{1.281611in}}%
\pgfpathlineto{\pgfqpoint{0.186087in}{1.281611in}}%
\pgfpathlineto{\pgfqpoint{0.186087in}{0.841611in}}%
\pgfpathclose%
\pgfusepath{stroke,fill}%
\end{pgfscope}%
\begin{pgfscope}%
\pgfpathrectangle{\pgfqpoint{0.148611in}{0.841611in}}{\pgfqpoint{0.824468in}{0.462000in}}%
\pgfusepath{clip}%
\pgfsetbuttcap%
\pgfsetmiterjoin%
\definecolor{currentfill}{rgb}{0.121569,0.466667,0.705882}%
\pgfsetfillcolor{currentfill}%
\pgfsetfillopacity{0.500000}%
\pgfsetlinewidth{1.003750pt}%
\definecolor{currentstroke}{rgb}{0.000000,0.000000,0.000000}%
\pgfsetstrokecolor{currentstroke}%
\pgfsetdash{}{0pt}%
\pgfpathmoveto{\pgfqpoint{0.335990in}{0.841611in}}%
\pgfpathlineto{\pgfqpoint{0.485894in}{0.841611in}}%
\pgfpathlineto{\pgfqpoint{0.485894in}{0.841611in}}%
\pgfpathlineto{\pgfqpoint{0.335990in}{0.841611in}}%
\pgfpathlineto{\pgfqpoint{0.335990in}{0.841611in}}%
\pgfpathclose%
\pgfusepath{stroke,fill}%
\end{pgfscope}%
\begin{pgfscope}%
\pgfpathrectangle{\pgfqpoint{0.148611in}{0.841611in}}{\pgfqpoint{0.824468in}{0.462000in}}%
\pgfusepath{clip}%
\pgfsetbuttcap%
\pgfsetmiterjoin%
\definecolor{currentfill}{rgb}{0.121569,0.466667,0.705882}%
\pgfsetfillcolor{currentfill}%
\pgfsetfillopacity{0.500000}%
\pgfsetlinewidth{1.003750pt}%
\definecolor{currentstroke}{rgb}{0.000000,0.000000,0.000000}%
\pgfsetstrokecolor{currentstroke}%
\pgfsetdash{}{0pt}%
\pgfpathmoveto{\pgfqpoint{0.485894in}{0.841611in}}%
\pgfpathlineto{\pgfqpoint{0.635797in}{0.841611in}}%
\pgfpathlineto{\pgfqpoint{0.635797in}{0.841611in}}%
\pgfpathlineto{\pgfqpoint{0.485894in}{0.841611in}}%
\pgfpathlineto{\pgfqpoint{0.485894in}{0.841611in}}%
\pgfpathclose%
\pgfusepath{stroke,fill}%
\end{pgfscope}%
\begin{pgfscope}%
\pgfpathrectangle{\pgfqpoint{0.148611in}{0.841611in}}{\pgfqpoint{0.824468in}{0.462000in}}%
\pgfusepath{clip}%
\pgfsetbuttcap%
\pgfsetmiterjoin%
\definecolor{currentfill}{rgb}{0.121569,0.466667,0.705882}%
\pgfsetfillcolor{currentfill}%
\pgfsetfillopacity{0.500000}%
\pgfsetlinewidth{1.003750pt}%
\definecolor{currentstroke}{rgb}{0.000000,0.000000,0.000000}%
\pgfsetstrokecolor{currentstroke}%
\pgfsetdash{}{0pt}%
\pgfpathmoveto{\pgfqpoint{0.635797in}{0.841611in}}%
\pgfpathlineto{\pgfqpoint{0.785700in}{0.841611in}}%
\pgfpathlineto{\pgfqpoint{0.785700in}{0.841611in}}%
\pgfpathlineto{\pgfqpoint{0.635797in}{0.841611in}}%
\pgfpathlineto{\pgfqpoint{0.635797in}{0.841611in}}%
\pgfpathclose%
\pgfusepath{stroke,fill}%
\end{pgfscope}%
\begin{pgfscope}%
\pgfpathrectangle{\pgfqpoint{0.148611in}{0.841611in}}{\pgfqpoint{0.824468in}{0.462000in}}%
\pgfusepath{clip}%
\pgfsetbuttcap%
\pgfsetmiterjoin%
\definecolor{currentfill}{rgb}{0.121569,0.466667,0.705882}%
\pgfsetfillcolor{currentfill}%
\pgfsetfillopacity{0.500000}%
\pgfsetlinewidth{1.003750pt}%
\definecolor{currentstroke}{rgb}{0.000000,0.000000,0.000000}%
\pgfsetstrokecolor{currentstroke}%
\pgfsetdash{}{0pt}%
\pgfpathmoveto{\pgfqpoint{0.785700in}{0.841611in}}%
\pgfpathlineto{\pgfqpoint{0.935603in}{0.841611in}}%
\pgfpathlineto{\pgfqpoint{0.935603in}{0.841611in}}%
\pgfpathlineto{\pgfqpoint{0.785700in}{0.841611in}}%
\pgfpathlineto{\pgfqpoint{0.785700in}{0.841611in}}%
\pgfpathclose%
\pgfusepath{stroke,fill}%
\end{pgfscope}%
\begin{pgfscope}%
\pgfsetrectcap%
\pgfsetmiterjoin%
\pgfsetlinewidth{0.803000pt}%
\definecolor{currentstroke}{rgb}{0.000000,0.000000,0.000000}%
\pgfsetstrokecolor{currentstroke}%
\pgfsetdash{}{0pt}%
\pgfpathmoveto{\pgfqpoint{0.148611in}{0.841611in}}%
\pgfpathlineto{\pgfqpoint{0.148611in}{1.303611in}}%
\pgfusepath{stroke}%
\end{pgfscope}%
\begin{pgfscope}%
\pgfsetrectcap%
\pgfsetmiterjoin%
\pgfsetlinewidth{0.803000pt}%
\definecolor{currentstroke}{rgb}{0.000000,0.000000,0.000000}%
\pgfsetstrokecolor{currentstroke}%
\pgfsetdash{}{0pt}%
\pgfpathmoveto{\pgfqpoint{0.973079in}{0.841611in}}%
\pgfpathlineto{\pgfqpoint{0.973079in}{1.303611in}}%
\pgfusepath{stroke}%
\end{pgfscope}%
\begin{pgfscope}%
\pgfsetrectcap%
\pgfsetmiterjoin%
\pgfsetlinewidth{0.803000pt}%
\definecolor{currentstroke}{rgb}{0.000000,0.000000,0.000000}%
\pgfsetstrokecolor{currentstroke}%
\pgfsetdash{}{0pt}%
\pgfpathmoveto{\pgfqpoint{0.148611in}{0.841611in}}%
\pgfpathlineto{\pgfqpoint{0.973079in}{0.841611in}}%
\pgfusepath{stroke}%
\end{pgfscope}%
\begin{pgfscope}%
\pgfsetrectcap%
\pgfsetmiterjoin%
\pgfsetlinewidth{0.803000pt}%
\definecolor{currentstroke}{rgb}{0.000000,0.000000,0.000000}%
\pgfsetstrokecolor{currentstroke}%
\pgfsetdash{}{0pt}%
\pgfpathmoveto{\pgfqpoint{0.148611in}{1.303611in}}%
\pgfpathlineto{\pgfqpoint{0.973079in}{1.303611in}}%
\pgfusepath{stroke}%
\end{pgfscope}%
\begin{pgfscope}%
\definecolor{textcolor}{rgb}{0.000000,0.000000,0.000000}%
\pgfsetstrokecolor{textcolor}%
\pgfsetfillcolor{textcolor}%
\pgftext[x=0.560845in,y=1.386944in,,base]{\color{textcolor}\rmfamily\fontsize{11.000000}{13.200000}\selectfont MMA}%
\end{pgfscope}%
\begin{pgfscope}%
\pgfsetbuttcap%
\pgfsetmiterjoin%
\definecolor{currentfill}{rgb}{1.000000,1.000000,1.000000}%
\pgfsetfillcolor{currentfill}%
\pgfsetlinewidth{0.000000pt}%
\definecolor{currentstroke}{rgb}{0.000000,0.000000,0.000000}%
\pgfsetstrokecolor{currentstroke}%
\pgfsetstrokeopacity{0.000000}%
\pgfsetdash{}{0pt}%
\pgfpathmoveto{\pgfqpoint{1.137973in}{0.841611in}}%
\pgfpathlineto{\pgfqpoint{1.962441in}{0.841611in}}%
\pgfpathlineto{\pgfqpoint{1.962441in}{1.303611in}}%
\pgfpathlineto{\pgfqpoint{1.137973in}{1.303611in}}%
\pgfpathlineto{\pgfqpoint{1.137973in}{0.841611in}}%
\pgfpathclose%
\pgfusepath{fill}%
\end{pgfscope}%
\begin{pgfscope}%
\pgfpathrectangle{\pgfqpoint{1.137973in}{0.841611in}}{\pgfqpoint{0.824468in}{0.462000in}}%
\pgfusepath{clip}%
\pgfsetbuttcap%
\pgfsetmiterjoin%
\definecolor{currentfill}{rgb}{0.121569,0.466667,0.705882}%
\pgfsetfillcolor{currentfill}%
\pgfsetfillopacity{0.500000}%
\pgfsetlinewidth{1.003750pt}%
\definecolor{currentstroke}{rgb}{0.000000,0.000000,0.000000}%
\pgfsetstrokecolor{currentstroke}%
\pgfsetdash{}{0pt}%
\pgfpathmoveto{\pgfqpoint{1.175449in}{0.841611in}}%
\pgfpathlineto{\pgfqpoint{1.325352in}{0.841611in}}%
\pgfpathlineto{\pgfqpoint{1.325352in}{1.281611in}}%
\pgfpathlineto{\pgfqpoint{1.175449in}{1.281611in}}%
\pgfpathlineto{\pgfqpoint{1.175449in}{0.841611in}}%
\pgfpathclose%
\pgfusepath{stroke,fill}%
\end{pgfscope}%
\begin{pgfscope}%
\pgfpathrectangle{\pgfqpoint{1.137973in}{0.841611in}}{\pgfqpoint{0.824468in}{0.462000in}}%
\pgfusepath{clip}%
\pgfsetbuttcap%
\pgfsetmiterjoin%
\definecolor{currentfill}{rgb}{0.121569,0.466667,0.705882}%
\pgfsetfillcolor{currentfill}%
\pgfsetfillopacity{0.500000}%
\pgfsetlinewidth{1.003750pt}%
\definecolor{currentstroke}{rgb}{0.000000,0.000000,0.000000}%
\pgfsetstrokecolor{currentstroke}%
\pgfsetdash{}{0pt}%
\pgfpathmoveto{\pgfqpoint{1.325352in}{0.841611in}}%
\pgfpathlineto{\pgfqpoint{1.475255in}{0.841611in}}%
\pgfpathlineto{\pgfqpoint{1.475255in}{1.047861in}}%
\pgfpathlineto{\pgfqpoint{1.325352in}{1.047861in}}%
\pgfpathlineto{\pgfqpoint{1.325352in}{0.841611in}}%
\pgfpathclose%
\pgfusepath{stroke,fill}%
\end{pgfscope}%
\begin{pgfscope}%
\pgfpathrectangle{\pgfqpoint{1.137973in}{0.841611in}}{\pgfqpoint{0.824468in}{0.462000in}}%
\pgfusepath{clip}%
\pgfsetbuttcap%
\pgfsetmiterjoin%
\definecolor{currentfill}{rgb}{0.121569,0.466667,0.705882}%
\pgfsetfillcolor{currentfill}%
\pgfsetfillopacity{0.500000}%
\pgfsetlinewidth{1.003750pt}%
\definecolor{currentstroke}{rgb}{0.000000,0.000000,0.000000}%
\pgfsetstrokecolor{currentstroke}%
\pgfsetdash{}{0pt}%
\pgfpathmoveto{\pgfqpoint{1.475255in}{0.841611in}}%
\pgfpathlineto{\pgfqpoint{1.625158in}{0.841611in}}%
\pgfpathlineto{\pgfqpoint{1.625158in}{0.855361in}}%
\pgfpathlineto{\pgfqpoint{1.475255in}{0.855361in}}%
\pgfpathlineto{\pgfqpoint{1.475255in}{0.841611in}}%
\pgfpathclose%
\pgfusepath{stroke,fill}%
\end{pgfscope}%
\begin{pgfscope}%
\pgfpathrectangle{\pgfqpoint{1.137973in}{0.841611in}}{\pgfqpoint{0.824468in}{0.462000in}}%
\pgfusepath{clip}%
\pgfsetbuttcap%
\pgfsetmiterjoin%
\definecolor{currentfill}{rgb}{0.121569,0.466667,0.705882}%
\pgfsetfillcolor{currentfill}%
\pgfsetfillopacity{0.500000}%
\pgfsetlinewidth{1.003750pt}%
\definecolor{currentstroke}{rgb}{0.000000,0.000000,0.000000}%
\pgfsetstrokecolor{currentstroke}%
\pgfsetdash{}{0pt}%
\pgfpathmoveto{\pgfqpoint{1.625158in}{0.841611in}}%
\pgfpathlineto{\pgfqpoint{1.775062in}{0.841611in}}%
\pgfpathlineto{\pgfqpoint{1.775062in}{0.841611in}}%
\pgfpathlineto{\pgfqpoint{1.625158in}{0.841611in}}%
\pgfpathlineto{\pgfqpoint{1.625158in}{0.841611in}}%
\pgfpathclose%
\pgfusepath{stroke,fill}%
\end{pgfscope}%
\begin{pgfscope}%
\pgfpathrectangle{\pgfqpoint{1.137973in}{0.841611in}}{\pgfqpoint{0.824468in}{0.462000in}}%
\pgfusepath{clip}%
\pgfsetbuttcap%
\pgfsetmiterjoin%
\definecolor{currentfill}{rgb}{0.121569,0.466667,0.705882}%
\pgfsetfillcolor{currentfill}%
\pgfsetfillopacity{0.500000}%
\pgfsetlinewidth{1.003750pt}%
\definecolor{currentstroke}{rgb}{0.000000,0.000000,0.000000}%
\pgfsetstrokecolor{currentstroke}%
\pgfsetdash{}{0pt}%
\pgfpathmoveto{\pgfqpoint{1.775062in}{0.841611in}}%
\pgfpathlineto{\pgfqpoint{1.924965in}{0.841611in}}%
\pgfpathlineto{\pgfqpoint{1.924965in}{0.910361in}}%
\pgfpathlineto{\pgfqpoint{1.775062in}{0.910361in}}%
\pgfpathlineto{\pgfqpoint{1.775062in}{0.841611in}}%
\pgfpathclose%
\pgfusepath{stroke,fill}%
\end{pgfscope}%
\begin{pgfscope}%
\pgfsetrectcap%
\pgfsetmiterjoin%
\pgfsetlinewidth{0.803000pt}%
\definecolor{currentstroke}{rgb}{0.000000,0.000000,0.000000}%
\pgfsetstrokecolor{currentstroke}%
\pgfsetdash{}{0pt}%
\pgfpathmoveto{\pgfqpoint{1.137973in}{0.841611in}}%
\pgfpathlineto{\pgfqpoint{1.137973in}{1.303611in}}%
\pgfusepath{stroke}%
\end{pgfscope}%
\begin{pgfscope}%
\pgfsetrectcap%
\pgfsetmiterjoin%
\pgfsetlinewidth{0.803000pt}%
\definecolor{currentstroke}{rgb}{0.000000,0.000000,0.000000}%
\pgfsetstrokecolor{currentstroke}%
\pgfsetdash{}{0pt}%
\pgfpathmoveto{\pgfqpoint{1.962441in}{0.841611in}}%
\pgfpathlineto{\pgfqpoint{1.962441in}{1.303611in}}%
\pgfusepath{stroke}%
\end{pgfscope}%
\begin{pgfscope}%
\pgfsetrectcap%
\pgfsetmiterjoin%
\pgfsetlinewidth{0.803000pt}%
\definecolor{currentstroke}{rgb}{0.000000,0.000000,0.000000}%
\pgfsetstrokecolor{currentstroke}%
\pgfsetdash{}{0pt}%
\pgfpathmoveto{\pgfqpoint{1.137973in}{0.841611in}}%
\pgfpathlineto{\pgfqpoint{1.962441in}{0.841611in}}%
\pgfusepath{stroke}%
\end{pgfscope}%
\begin{pgfscope}%
\pgfsetrectcap%
\pgfsetmiterjoin%
\pgfsetlinewidth{0.803000pt}%
\definecolor{currentstroke}{rgb}{0.000000,0.000000,0.000000}%
\pgfsetstrokecolor{currentstroke}%
\pgfsetdash{}{0pt}%
\pgfpathmoveto{\pgfqpoint{1.137973in}{1.303611in}}%
\pgfpathlineto{\pgfqpoint{1.962441in}{1.303611in}}%
\pgfusepath{stroke}%
\end{pgfscope}%
\begin{pgfscope}%
\definecolor{textcolor}{rgb}{0.000000,0.000000,0.000000}%
\pgfsetstrokecolor{textcolor}%
\pgfsetfillcolor{textcolor}%
\pgftext[x=1.550207in,y=1.386944in,,base]{\color{textcolor}\rmfamily\fontsize{11.000000}{13.200000}\selectfont MetLife}%
\end{pgfscope}%
\begin{pgfscope}%
\pgfsetbuttcap%
\pgfsetmiterjoin%
\definecolor{currentfill}{rgb}{1.000000,1.000000,1.000000}%
\pgfsetfillcolor{currentfill}%
\pgfsetlinewidth{0.000000pt}%
\definecolor{currentstroke}{rgb}{0.000000,0.000000,0.000000}%
\pgfsetstrokecolor{currentstroke}%
\pgfsetstrokeopacity{0.000000}%
\pgfsetdash{}{0pt}%
\pgfpathmoveto{\pgfqpoint{2.127335in}{0.841611in}}%
\pgfpathlineto{\pgfqpoint{2.951803in}{0.841611in}}%
\pgfpathlineto{\pgfqpoint{2.951803in}{1.303611in}}%
\pgfpathlineto{\pgfqpoint{2.127335in}{1.303611in}}%
\pgfpathlineto{\pgfqpoint{2.127335in}{0.841611in}}%
\pgfpathclose%
\pgfusepath{fill}%
\end{pgfscope}%
\begin{pgfscope}%
\pgfpathrectangle{\pgfqpoint{2.127335in}{0.841611in}}{\pgfqpoint{0.824468in}{0.462000in}}%
\pgfusepath{clip}%
\pgfsetbuttcap%
\pgfsetmiterjoin%
\definecolor{currentfill}{rgb}{0.121569,0.466667,0.705882}%
\pgfsetfillcolor{currentfill}%
\pgfsetfillopacity{0.500000}%
\pgfsetlinewidth{1.003750pt}%
\definecolor{currentstroke}{rgb}{0.000000,0.000000,0.000000}%
\pgfsetstrokecolor{currentstroke}%
\pgfsetdash{}{0pt}%
\pgfpathmoveto{\pgfqpoint{2.164810in}{0.841611in}}%
\pgfpathlineto{\pgfqpoint{2.314714in}{0.841611in}}%
\pgfpathlineto{\pgfqpoint{2.314714in}{1.281611in}}%
\pgfpathlineto{\pgfqpoint{2.164810in}{1.281611in}}%
\pgfpathlineto{\pgfqpoint{2.164810in}{0.841611in}}%
\pgfpathclose%
\pgfusepath{stroke,fill}%
\end{pgfscope}%
\begin{pgfscope}%
\pgfpathrectangle{\pgfqpoint{2.127335in}{0.841611in}}{\pgfqpoint{0.824468in}{0.462000in}}%
\pgfusepath{clip}%
\pgfsetbuttcap%
\pgfsetmiterjoin%
\definecolor{currentfill}{rgb}{0.121569,0.466667,0.705882}%
\pgfsetfillcolor{currentfill}%
\pgfsetfillopacity{0.500000}%
\pgfsetlinewidth{1.003750pt}%
\definecolor{currentstroke}{rgb}{0.000000,0.000000,0.000000}%
\pgfsetstrokecolor{currentstroke}%
\pgfsetdash{}{0pt}%
\pgfpathmoveto{\pgfqpoint{2.314714in}{0.841611in}}%
\pgfpathlineto{\pgfqpoint{2.464617in}{0.841611in}}%
\pgfpathlineto{\pgfqpoint{2.464617in}{1.024050in}}%
\pgfpathlineto{\pgfqpoint{2.314714in}{1.024050in}}%
\pgfpathlineto{\pgfqpoint{2.314714in}{0.841611in}}%
\pgfpathclose%
\pgfusepath{stroke,fill}%
\end{pgfscope}%
\begin{pgfscope}%
\pgfpathrectangle{\pgfqpoint{2.127335in}{0.841611in}}{\pgfqpoint{0.824468in}{0.462000in}}%
\pgfusepath{clip}%
\pgfsetbuttcap%
\pgfsetmiterjoin%
\definecolor{currentfill}{rgb}{0.121569,0.466667,0.705882}%
\pgfsetfillcolor{currentfill}%
\pgfsetfillopacity{0.500000}%
\pgfsetlinewidth{1.003750pt}%
\definecolor{currentstroke}{rgb}{0.000000,0.000000,0.000000}%
\pgfsetstrokecolor{currentstroke}%
\pgfsetdash{}{0pt}%
\pgfpathmoveto{\pgfqpoint{2.464617in}{0.841611in}}%
\pgfpathlineto{\pgfqpoint{2.614520in}{0.841611in}}%
\pgfpathlineto{\pgfqpoint{2.614520in}{0.938196in}}%
\pgfpathlineto{\pgfqpoint{2.464617in}{0.938196in}}%
\pgfpathlineto{\pgfqpoint{2.464617in}{0.841611in}}%
\pgfpathclose%
\pgfusepath{stroke,fill}%
\end{pgfscope}%
\begin{pgfscope}%
\pgfpathrectangle{\pgfqpoint{2.127335in}{0.841611in}}{\pgfqpoint{0.824468in}{0.462000in}}%
\pgfusepath{clip}%
\pgfsetbuttcap%
\pgfsetmiterjoin%
\definecolor{currentfill}{rgb}{0.121569,0.466667,0.705882}%
\pgfsetfillcolor{currentfill}%
\pgfsetfillopacity{0.500000}%
\pgfsetlinewidth{1.003750pt}%
\definecolor{currentstroke}{rgb}{0.000000,0.000000,0.000000}%
\pgfsetstrokecolor{currentstroke}%
\pgfsetdash{}{0pt}%
\pgfpathmoveto{\pgfqpoint{2.614520in}{0.841611in}}%
\pgfpathlineto{\pgfqpoint{2.764423in}{0.841611in}}%
\pgfpathlineto{\pgfqpoint{2.764423in}{0.868440in}}%
\pgfpathlineto{\pgfqpoint{2.614520in}{0.868440in}}%
\pgfpathlineto{\pgfqpoint{2.614520in}{0.841611in}}%
\pgfpathclose%
\pgfusepath{stroke,fill}%
\end{pgfscope}%
\begin{pgfscope}%
\pgfpathrectangle{\pgfqpoint{2.127335in}{0.841611in}}{\pgfqpoint{0.824468in}{0.462000in}}%
\pgfusepath{clip}%
\pgfsetbuttcap%
\pgfsetmiterjoin%
\definecolor{currentfill}{rgb}{0.121569,0.466667,0.705882}%
\pgfsetfillcolor{currentfill}%
\pgfsetfillopacity{0.500000}%
\pgfsetlinewidth{1.003750pt}%
\definecolor{currentstroke}{rgb}{0.000000,0.000000,0.000000}%
\pgfsetstrokecolor{currentstroke}%
\pgfsetdash{}{0pt}%
\pgfpathmoveto{\pgfqpoint{2.764423in}{0.841611in}}%
\pgfpathlineto{\pgfqpoint{2.914327in}{0.841611in}}%
\pgfpathlineto{\pgfqpoint{2.914327in}{0.879172in}}%
\pgfpathlineto{\pgfqpoint{2.764423in}{0.879172in}}%
\pgfpathlineto{\pgfqpoint{2.764423in}{0.841611in}}%
\pgfpathclose%
\pgfusepath{stroke,fill}%
\end{pgfscope}%
\begin{pgfscope}%
\pgfsetrectcap%
\pgfsetmiterjoin%
\pgfsetlinewidth{0.803000pt}%
\definecolor{currentstroke}{rgb}{0.000000,0.000000,0.000000}%
\pgfsetstrokecolor{currentstroke}%
\pgfsetdash{}{0pt}%
\pgfpathmoveto{\pgfqpoint{2.127335in}{0.841611in}}%
\pgfpathlineto{\pgfqpoint{2.127335in}{1.303611in}}%
\pgfusepath{stroke}%
\end{pgfscope}%
\begin{pgfscope}%
\pgfsetrectcap%
\pgfsetmiterjoin%
\pgfsetlinewidth{0.803000pt}%
\definecolor{currentstroke}{rgb}{0.000000,0.000000,0.000000}%
\pgfsetstrokecolor{currentstroke}%
\pgfsetdash{}{0pt}%
\pgfpathmoveto{\pgfqpoint{2.951803in}{0.841611in}}%
\pgfpathlineto{\pgfqpoint{2.951803in}{1.303611in}}%
\pgfusepath{stroke}%
\end{pgfscope}%
\begin{pgfscope}%
\pgfsetrectcap%
\pgfsetmiterjoin%
\pgfsetlinewidth{0.803000pt}%
\definecolor{currentstroke}{rgb}{0.000000,0.000000,0.000000}%
\pgfsetstrokecolor{currentstroke}%
\pgfsetdash{}{0pt}%
\pgfpathmoveto{\pgfqpoint{2.127335in}{0.841611in}}%
\pgfpathlineto{\pgfqpoint{2.951803in}{0.841611in}}%
\pgfusepath{stroke}%
\end{pgfscope}%
\begin{pgfscope}%
\pgfsetrectcap%
\pgfsetmiterjoin%
\pgfsetlinewidth{0.803000pt}%
\definecolor{currentstroke}{rgb}{0.000000,0.000000,0.000000}%
\pgfsetstrokecolor{currentstroke}%
\pgfsetdash{}{0pt}%
\pgfpathmoveto{\pgfqpoint{2.127335in}{1.303611in}}%
\pgfpathlineto{\pgfqpoint{2.951803in}{1.303611in}}%
\pgfusepath{stroke}%
\end{pgfscope}%
\begin{pgfscope}%
\definecolor{textcolor}{rgb}{0.000000,0.000000,0.000000}%
\pgfsetstrokecolor{textcolor}%
\pgfsetfillcolor{textcolor}%
\pgftext[x=2.539569in,y=1.386944in,,base]{\color{textcolor}\rmfamily\fontsize{11.000000}{13.200000}\selectfont Crédit...}%
\end{pgfscope}%
\begin{pgfscope}%
\pgfsetbuttcap%
\pgfsetmiterjoin%
\definecolor{currentfill}{rgb}{1.000000,1.000000,1.000000}%
\pgfsetfillcolor{currentfill}%
\pgfsetlinewidth{0.000000pt}%
\definecolor{currentstroke}{rgb}{0.000000,0.000000,0.000000}%
\pgfsetstrokecolor{currentstroke}%
\pgfsetstrokeopacity{0.000000}%
\pgfsetdash{}{0pt}%
\pgfpathmoveto{\pgfqpoint{3.116696in}{0.841611in}}%
\pgfpathlineto{\pgfqpoint{3.941164in}{0.841611in}}%
\pgfpathlineto{\pgfqpoint{3.941164in}{1.303611in}}%
\pgfpathlineto{\pgfqpoint{3.116696in}{1.303611in}}%
\pgfpathlineto{\pgfqpoint{3.116696in}{0.841611in}}%
\pgfpathclose%
\pgfusepath{fill}%
\end{pgfscope}%
\begin{pgfscope}%
\pgfpathrectangle{\pgfqpoint{3.116696in}{0.841611in}}{\pgfqpoint{0.824468in}{0.462000in}}%
\pgfusepath{clip}%
\pgfsetbuttcap%
\pgfsetmiterjoin%
\definecolor{currentfill}{rgb}{0.121569,0.466667,0.705882}%
\pgfsetfillcolor{currentfill}%
\pgfsetfillopacity{0.500000}%
\pgfsetlinewidth{1.003750pt}%
\definecolor{currentstroke}{rgb}{0.000000,0.000000,0.000000}%
\pgfsetstrokecolor{currentstroke}%
\pgfsetdash{}{0pt}%
\pgfpathmoveto{\pgfqpoint{3.154172in}{0.841611in}}%
\pgfpathlineto{\pgfqpoint{3.304075in}{0.841611in}}%
\pgfpathlineto{\pgfqpoint{3.304075in}{1.281611in}}%
\pgfpathlineto{\pgfqpoint{3.154172in}{1.281611in}}%
\pgfpathlineto{\pgfqpoint{3.154172in}{0.841611in}}%
\pgfpathclose%
\pgfusepath{stroke,fill}%
\end{pgfscope}%
\begin{pgfscope}%
\pgfpathrectangle{\pgfqpoint{3.116696in}{0.841611in}}{\pgfqpoint{0.824468in}{0.462000in}}%
\pgfusepath{clip}%
\pgfsetbuttcap%
\pgfsetmiterjoin%
\definecolor{currentfill}{rgb}{0.121569,0.466667,0.705882}%
\pgfsetfillcolor{currentfill}%
\pgfsetfillopacity{0.500000}%
\pgfsetlinewidth{1.003750pt}%
\definecolor{currentstroke}{rgb}{0.000000,0.000000,0.000000}%
\pgfsetstrokecolor{currentstroke}%
\pgfsetdash{}{0pt}%
\pgfpathmoveto{\pgfqpoint{3.304075in}{0.841611in}}%
\pgfpathlineto{\pgfqpoint{3.453979in}{0.841611in}}%
\pgfpathlineto{\pgfqpoint{3.453979in}{0.951611in}}%
\pgfpathlineto{\pgfqpoint{3.304075in}{0.951611in}}%
\pgfpathlineto{\pgfqpoint{3.304075in}{0.841611in}}%
\pgfpathclose%
\pgfusepath{stroke,fill}%
\end{pgfscope}%
\begin{pgfscope}%
\pgfpathrectangle{\pgfqpoint{3.116696in}{0.841611in}}{\pgfqpoint{0.824468in}{0.462000in}}%
\pgfusepath{clip}%
\pgfsetbuttcap%
\pgfsetmiterjoin%
\definecolor{currentfill}{rgb}{0.121569,0.466667,0.705882}%
\pgfsetfillcolor{currentfill}%
\pgfsetfillopacity{0.500000}%
\pgfsetlinewidth{1.003750pt}%
\definecolor{currentstroke}{rgb}{0.000000,0.000000,0.000000}%
\pgfsetstrokecolor{currentstroke}%
\pgfsetdash{}{0pt}%
\pgfpathmoveto{\pgfqpoint{3.453979in}{0.841611in}}%
\pgfpathlineto{\pgfqpoint{3.603882in}{0.841611in}}%
\pgfpathlineto{\pgfqpoint{3.603882in}{0.841611in}}%
\pgfpathlineto{\pgfqpoint{3.453979in}{0.841611in}}%
\pgfpathlineto{\pgfqpoint{3.453979in}{0.841611in}}%
\pgfpathclose%
\pgfusepath{stroke,fill}%
\end{pgfscope}%
\begin{pgfscope}%
\pgfpathrectangle{\pgfqpoint{3.116696in}{0.841611in}}{\pgfqpoint{0.824468in}{0.462000in}}%
\pgfusepath{clip}%
\pgfsetbuttcap%
\pgfsetmiterjoin%
\definecolor{currentfill}{rgb}{0.121569,0.466667,0.705882}%
\pgfsetfillcolor{currentfill}%
\pgfsetfillopacity{0.500000}%
\pgfsetlinewidth{1.003750pt}%
\definecolor{currentstroke}{rgb}{0.000000,0.000000,0.000000}%
\pgfsetstrokecolor{currentstroke}%
\pgfsetdash{}{0pt}%
\pgfpathmoveto{\pgfqpoint{3.603882in}{0.841611in}}%
\pgfpathlineto{\pgfqpoint{3.753785in}{0.841611in}}%
\pgfpathlineto{\pgfqpoint{3.753785in}{0.896611in}}%
\pgfpathlineto{\pgfqpoint{3.603882in}{0.896611in}}%
\pgfpathlineto{\pgfqpoint{3.603882in}{0.841611in}}%
\pgfpathclose%
\pgfusepath{stroke,fill}%
\end{pgfscope}%
\begin{pgfscope}%
\pgfpathrectangle{\pgfqpoint{3.116696in}{0.841611in}}{\pgfqpoint{0.824468in}{0.462000in}}%
\pgfusepath{clip}%
\pgfsetbuttcap%
\pgfsetmiterjoin%
\definecolor{currentfill}{rgb}{0.121569,0.466667,0.705882}%
\pgfsetfillcolor{currentfill}%
\pgfsetfillopacity{0.500000}%
\pgfsetlinewidth{1.003750pt}%
\definecolor{currentstroke}{rgb}{0.000000,0.000000,0.000000}%
\pgfsetstrokecolor{currentstroke}%
\pgfsetdash{}{0pt}%
\pgfpathmoveto{\pgfqpoint{3.753785in}{0.841611in}}%
\pgfpathlineto{\pgfqpoint{3.903688in}{0.841611in}}%
\pgfpathlineto{\pgfqpoint{3.903688in}{0.951611in}}%
\pgfpathlineto{\pgfqpoint{3.753785in}{0.951611in}}%
\pgfpathlineto{\pgfqpoint{3.753785in}{0.841611in}}%
\pgfpathclose%
\pgfusepath{stroke,fill}%
\end{pgfscope}%
\begin{pgfscope}%
\pgfsetrectcap%
\pgfsetmiterjoin%
\pgfsetlinewidth{0.803000pt}%
\definecolor{currentstroke}{rgb}{0.000000,0.000000,0.000000}%
\pgfsetstrokecolor{currentstroke}%
\pgfsetdash{}{0pt}%
\pgfpathmoveto{\pgfqpoint{3.116696in}{0.841611in}}%
\pgfpathlineto{\pgfqpoint{3.116696in}{1.303611in}}%
\pgfusepath{stroke}%
\end{pgfscope}%
\begin{pgfscope}%
\pgfsetrectcap%
\pgfsetmiterjoin%
\pgfsetlinewidth{0.803000pt}%
\definecolor{currentstroke}{rgb}{0.000000,0.000000,0.000000}%
\pgfsetstrokecolor{currentstroke}%
\pgfsetdash{}{0pt}%
\pgfpathmoveto{\pgfqpoint{3.941164in}{0.841611in}}%
\pgfpathlineto{\pgfqpoint{3.941164in}{1.303611in}}%
\pgfusepath{stroke}%
\end{pgfscope}%
\begin{pgfscope}%
\pgfsetrectcap%
\pgfsetmiterjoin%
\pgfsetlinewidth{0.803000pt}%
\definecolor{currentstroke}{rgb}{0.000000,0.000000,0.000000}%
\pgfsetstrokecolor{currentstroke}%
\pgfsetdash{}{0pt}%
\pgfpathmoveto{\pgfqpoint{3.116696in}{0.841611in}}%
\pgfpathlineto{\pgfqpoint{3.941164in}{0.841611in}}%
\pgfusepath{stroke}%
\end{pgfscope}%
\begin{pgfscope}%
\pgfsetrectcap%
\pgfsetmiterjoin%
\pgfsetlinewidth{0.803000pt}%
\definecolor{currentstroke}{rgb}{0.000000,0.000000,0.000000}%
\pgfsetstrokecolor{currentstroke}%
\pgfsetdash{}{0pt}%
\pgfpathmoveto{\pgfqpoint{3.116696in}{1.303611in}}%
\pgfpathlineto{\pgfqpoint{3.941164in}{1.303611in}}%
\pgfusepath{stroke}%
\end{pgfscope}%
\begin{pgfscope}%
\definecolor{textcolor}{rgb}{0.000000,0.000000,0.000000}%
\pgfsetstrokecolor{textcolor}%
\pgfsetfillcolor{textcolor}%
\pgftext[x=3.528930in,y=1.386944in,,base]{\color{textcolor}\rmfamily\fontsize{11.000000}{13.200000}\selectfont Afi Esca}%
\end{pgfscope}%
\begin{pgfscope}%
\pgfsetbuttcap%
\pgfsetmiterjoin%
\definecolor{currentfill}{rgb}{1.000000,1.000000,1.000000}%
\pgfsetfillcolor{currentfill}%
\pgfsetlinewidth{0.000000pt}%
\definecolor{currentstroke}{rgb}{0.000000,0.000000,0.000000}%
\pgfsetstrokecolor{currentstroke}%
\pgfsetstrokeopacity{0.000000}%
\pgfsetdash{}{0pt}%
\pgfpathmoveto{\pgfqpoint{4.106058in}{0.841611in}}%
\pgfpathlineto{\pgfqpoint{4.930526in}{0.841611in}}%
\pgfpathlineto{\pgfqpoint{4.930526in}{1.303611in}}%
\pgfpathlineto{\pgfqpoint{4.106058in}{1.303611in}}%
\pgfpathlineto{\pgfqpoint{4.106058in}{0.841611in}}%
\pgfpathclose%
\pgfusepath{fill}%
\end{pgfscope}%
\begin{pgfscope}%
\pgfpathrectangle{\pgfqpoint{4.106058in}{0.841611in}}{\pgfqpoint{0.824468in}{0.462000in}}%
\pgfusepath{clip}%
\pgfsetbuttcap%
\pgfsetmiterjoin%
\definecolor{currentfill}{rgb}{0.121569,0.466667,0.705882}%
\pgfsetfillcolor{currentfill}%
\pgfsetfillopacity{0.500000}%
\pgfsetlinewidth{1.003750pt}%
\definecolor{currentstroke}{rgb}{0.000000,0.000000,0.000000}%
\pgfsetstrokecolor{currentstroke}%
\pgfsetdash{}{0pt}%
\pgfpathmoveto{\pgfqpoint{4.143534in}{0.841611in}}%
\pgfpathlineto{\pgfqpoint{4.293437in}{0.841611in}}%
\pgfpathlineto{\pgfqpoint{4.293437in}{1.281611in}}%
\pgfpathlineto{\pgfqpoint{4.143534in}{1.281611in}}%
\pgfpathlineto{\pgfqpoint{4.143534in}{0.841611in}}%
\pgfpathclose%
\pgfusepath{stroke,fill}%
\end{pgfscope}%
\begin{pgfscope}%
\pgfpathrectangle{\pgfqpoint{4.106058in}{0.841611in}}{\pgfqpoint{0.824468in}{0.462000in}}%
\pgfusepath{clip}%
\pgfsetbuttcap%
\pgfsetmiterjoin%
\definecolor{currentfill}{rgb}{0.121569,0.466667,0.705882}%
\pgfsetfillcolor{currentfill}%
\pgfsetfillopacity{0.500000}%
\pgfsetlinewidth{1.003750pt}%
\definecolor{currentstroke}{rgb}{0.000000,0.000000,0.000000}%
\pgfsetstrokecolor{currentstroke}%
\pgfsetdash{}{0pt}%
\pgfpathmoveto{\pgfqpoint{4.293437in}{0.841611in}}%
\pgfpathlineto{\pgfqpoint{4.443340in}{0.841611in}}%
\pgfpathlineto{\pgfqpoint{4.443340in}{0.892380in}}%
\pgfpathlineto{\pgfqpoint{4.293437in}{0.892380in}}%
\pgfpathlineto{\pgfqpoint{4.293437in}{0.841611in}}%
\pgfpathclose%
\pgfusepath{stroke,fill}%
\end{pgfscope}%
\begin{pgfscope}%
\pgfpathrectangle{\pgfqpoint{4.106058in}{0.841611in}}{\pgfqpoint{0.824468in}{0.462000in}}%
\pgfusepath{clip}%
\pgfsetbuttcap%
\pgfsetmiterjoin%
\definecolor{currentfill}{rgb}{0.121569,0.466667,0.705882}%
\pgfsetfillcolor{currentfill}%
\pgfsetfillopacity{0.500000}%
\pgfsetlinewidth{1.003750pt}%
\definecolor{currentstroke}{rgb}{0.000000,0.000000,0.000000}%
\pgfsetstrokecolor{currentstroke}%
\pgfsetdash{}{0pt}%
\pgfpathmoveto{\pgfqpoint{4.443340in}{0.841611in}}%
\pgfpathlineto{\pgfqpoint{4.593244in}{0.841611in}}%
\pgfpathlineto{\pgfqpoint{4.593244in}{0.858534in}}%
\pgfpathlineto{\pgfqpoint{4.443340in}{0.858534in}}%
\pgfpathlineto{\pgfqpoint{4.443340in}{0.841611in}}%
\pgfpathclose%
\pgfusepath{stroke,fill}%
\end{pgfscope}%
\begin{pgfscope}%
\pgfpathrectangle{\pgfqpoint{4.106058in}{0.841611in}}{\pgfqpoint{0.824468in}{0.462000in}}%
\pgfusepath{clip}%
\pgfsetbuttcap%
\pgfsetmiterjoin%
\definecolor{currentfill}{rgb}{0.121569,0.466667,0.705882}%
\pgfsetfillcolor{currentfill}%
\pgfsetfillopacity{0.500000}%
\pgfsetlinewidth{1.003750pt}%
\definecolor{currentstroke}{rgb}{0.000000,0.000000,0.000000}%
\pgfsetstrokecolor{currentstroke}%
\pgfsetdash{}{0pt}%
\pgfpathmoveto{\pgfqpoint{4.593244in}{0.841611in}}%
\pgfpathlineto{\pgfqpoint{4.743147in}{0.841611in}}%
\pgfpathlineto{\pgfqpoint{4.743147in}{0.858534in}}%
\pgfpathlineto{\pgfqpoint{4.593244in}{0.858534in}}%
\pgfpathlineto{\pgfqpoint{4.593244in}{0.841611in}}%
\pgfpathclose%
\pgfusepath{stroke,fill}%
\end{pgfscope}%
\begin{pgfscope}%
\pgfpathrectangle{\pgfqpoint{4.106058in}{0.841611in}}{\pgfqpoint{0.824468in}{0.462000in}}%
\pgfusepath{clip}%
\pgfsetbuttcap%
\pgfsetmiterjoin%
\definecolor{currentfill}{rgb}{0.121569,0.466667,0.705882}%
\pgfsetfillcolor{currentfill}%
\pgfsetfillopacity{0.500000}%
\pgfsetlinewidth{1.003750pt}%
\definecolor{currentstroke}{rgb}{0.000000,0.000000,0.000000}%
\pgfsetstrokecolor{currentstroke}%
\pgfsetdash{}{0pt}%
\pgfpathmoveto{\pgfqpoint{4.743147in}{0.841611in}}%
\pgfpathlineto{\pgfqpoint{4.893050in}{0.841611in}}%
\pgfpathlineto{\pgfqpoint{4.893050in}{0.875457in}}%
\pgfpathlineto{\pgfqpoint{4.743147in}{0.875457in}}%
\pgfpathlineto{\pgfqpoint{4.743147in}{0.841611in}}%
\pgfpathclose%
\pgfusepath{stroke,fill}%
\end{pgfscope}%
\begin{pgfscope}%
\pgfsetrectcap%
\pgfsetmiterjoin%
\pgfsetlinewidth{0.803000pt}%
\definecolor{currentstroke}{rgb}{0.000000,0.000000,0.000000}%
\pgfsetstrokecolor{currentstroke}%
\pgfsetdash{}{0pt}%
\pgfpathmoveto{\pgfqpoint{4.106058in}{0.841611in}}%
\pgfpathlineto{\pgfqpoint{4.106058in}{1.303611in}}%
\pgfusepath{stroke}%
\end{pgfscope}%
\begin{pgfscope}%
\pgfsetrectcap%
\pgfsetmiterjoin%
\pgfsetlinewidth{0.803000pt}%
\definecolor{currentstroke}{rgb}{0.000000,0.000000,0.000000}%
\pgfsetstrokecolor{currentstroke}%
\pgfsetdash{}{0pt}%
\pgfpathmoveto{\pgfqpoint{4.930526in}{0.841611in}}%
\pgfpathlineto{\pgfqpoint{4.930526in}{1.303611in}}%
\pgfusepath{stroke}%
\end{pgfscope}%
\begin{pgfscope}%
\pgfsetrectcap%
\pgfsetmiterjoin%
\pgfsetlinewidth{0.803000pt}%
\definecolor{currentstroke}{rgb}{0.000000,0.000000,0.000000}%
\pgfsetstrokecolor{currentstroke}%
\pgfsetdash{}{0pt}%
\pgfpathmoveto{\pgfqpoint{4.106058in}{0.841611in}}%
\pgfpathlineto{\pgfqpoint{4.930526in}{0.841611in}}%
\pgfusepath{stroke}%
\end{pgfscope}%
\begin{pgfscope}%
\pgfsetrectcap%
\pgfsetmiterjoin%
\pgfsetlinewidth{0.803000pt}%
\definecolor{currentstroke}{rgb}{0.000000,0.000000,0.000000}%
\pgfsetstrokecolor{currentstroke}%
\pgfsetdash{}{0pt}%
\pgfpathmoveto{\pgfqpoint{4.106058in}{1.303611in}}%
\pgfpathlineto{\pgfqpoint{4.930526in}{1.303611in}}%
\pgfusepath{stroke}%
\end{pgfscope}%
\begin{pgfscope}%
\definecolor{textcolor}{rgb}{0.000000,0.000000,0.000000}%
\pgfsetstrokecolor{textcolor}%
\pgfsetfillcolor{textcolor}%
\pgftext[x=4.518292in,y=1.386944in,,base]{\color{textcolor}\rmfamily\fontsize{11.000000}{13.200000}\selectfont Gan}%
\end{pgfscope}%
\begin{pgfscope}%
\pgfsetbuttcap%
\pgfsetmiterjoin%
\definecolor{currentfill}{rgb}{1.000000,1.000000,1.000000}%
\pgfsetfillcolor{currentfill}%
\pgfsetlinewidth{0.000000pt}%
\definecolor{currentstroke}{rgb}{0.000000,0.000000,0.000000}%
\pgfsetstrokecolor{currentstroke}%
\pgfsetstrokeopacity{0.000000}%
\pgfsetdash{}{0pt}%
\pgfpathmoveto{\pgfqpoint{5.095420in}{0.841611in}}%
\pgfpathlineto{\pgfqpoint{5.919888in}{0.841611in}}%
\pgfpathlineto{\pgfqpoint{5.919888in}{1.303611in}}%
\pgfpathlineto{\pgfqpoint{5.095420in}{1.303611in}}%
\pgfpathlineto{\pgfqpoint{5.095420in}{0.841611in}}%
\pgfpathclose%
\pgfusepath{fill}%
\end{pgfscope}%
\begin{pgfscope}%
\pgfpathrectangle{\pgfqpoint{5.095420in}{0.841611in}}{\pgfqpoint{0.824468in}{0.462000in}}%
\pgfusepath{clip}%
\pgfsetbuttcap%
\pgfsetmiterjoin%
\definecolor{currentfill}{rgb}{0.121569,0.466667,0.705882}%
\pgfsetfillcolor{currentfill}%
\pgfsetfillopacity{0.500000}%
\pgfsetlinewidth{1.003750pt}%
\definecolor{currentstroke}{rgb}{0.000000,0.000000,0.000000}%
\pgfsetstrokecolor{currentstroke}%
\pgfsetdash{}{0pt}%
\pgfpathmoveto{\pgfqpoint{5.132895in}{0.841611in}}%
\pgfpathlineto{\pgfqpoint{5.282799in}{0.841611in}}%
\pgfpathlineto{\pgfqpoint{5.282799in}{1.171611in}}%
\pgfpathlineto{\pgfqpoint{5.132895in}{1.171611in}}%
\pgfpathlineto{\pgfqpoint{5.132895in}{0.841611in}}%
\pgfpathclose%
\pgfusepath{stroke,fill}%
\end{pgfscope}%
\begin{pgfscope}%
\pgfpathrectangle{\pgfqpoint{5.095420in}{0.841611in}}{\pgfqpoint{0.824468in}{0.462000in}}%
\pgfusepath{clip}%
\pgfsetbuttcap%
\pgfsetmiterjoin%
\definecolor{currentfill}{rgb}{0.121569,0.466667,0.705882}%
\pgfsetfillcolor{currentfill}%
\pgfsetfillopacity{0.500000}%
\pgfsetlinewidth{1.003750pt}%
\definecolor{currentstroke}{rgb}{0.000000,0.000000,0.000000}%
\pgfsetstrokecolor{currentstroke}%
\pgfsetdash{}{0pt}%
\pgfpathmoveto{\pgfqpoint{5.282799in}{0.841611in}}%
\pgfpathlineto{\pgfqpoint{5.432702in}{0.841611in}}%
\pgfpathlineto{\pgfqpoint{5.432702in}{1.281611in}}%
\pgfpathlineto{\pgfqpoint{5.282799in}{1.281611in}}%
\pgfpathlineto{\pgfqpoint{5.282799in}{0.841611in}}%
\pgfpathclose%
\pgfusepath{stroke,fill}%
\end{pgfscope}%
\begin{pgfscope}%
\pgfpathrectangle{\pgfqpoint{5.095420in}{0.841611in}}{\pgfqpoint{0.824468in}{0.462000in}}%
\pgfusepath{clip}%
\pgfsetbuttcap%
\pgfsetmiterjoin%
\definecolor{currentfill}{rgb}{0.121569,0.466667,0.705882}%
\pgfsetfillcolor{currentfill}%
\pgfsetfillopacity{0.500000}%
\pgfsetlinewidth{1.003750pt}%
\definecolor{currentstroke}{rgb}{0.000000,0.000000,0.000000}%
\pgfsetstrokecolor{currentstroke}%
\pgfsetdash{}{0pt}%
\pgfpathmoveto{\pgfqpoint{5.432702in}{0.841611in}}%
\pgfpathlineto{\pgfqpoint{5.582605in}{0.841611in}}%
\pgfpathlineto{\pgfqpoint{5.582605in}{0.841611in}}%
\pgfpathlineto{\pgfqpoint{5.432702in}{0.841611in}}%
\pgfpathlineto{\pgfqpoint{5.432702in}{0.841611in}}%
\pgfpathclose%
\pgfusepath{stroke,fill}%
\end{pgfscope}%
\begin{pgfscope}%
\pgfpathrectangle{\pgfqpoint{5.095420in}{0.841611in}}{\pgfqpoint{0.824468in}{0.462000in}}%
\pgfusepath{clip}%
\pgfsetbuttcap%
\pgfsetmiterjoin%
\definecolor{currentfill}{rgb}{0.121569,0.466667,0.705882}%
\pgfsetfillcolor{currentfill}%
\pgfsetfillopacity{0.500000}%
\pgfsetlinewidth{1.003750pt}%
\definecolor{currentstroke}{rgb}{0.000000,0.000000,0.000000}%
\pgfsetstrokecolor{currentstroke}%
\pgfsetdash{}{0pt}%
\pgfpathmoveto{\pgfqpoint{5.582605in}{0.841611in}}%
\pgfpathlineto{\pgfqpoint{5.732509in}{0.841611in}}%
\pgfpathlineto{\pgfqpoint{5.732509in}{0.896611in}}%
\pgfpathlineto{\pgfqpoint{5.582605in}{0.896611in}}%
\pgfpathlineto{\pgfqpoint{5.582605in}{0.841611in}}%
\pgfpathclose%
\pgfusepath{stroke,fill}%
\end{pgfscope}%
\begin{pgfscope}%
\pgfpathrectangle{\pgfqpoint{5.095420in}{0.841611in}}{\pgfqpoint{0.824468in}{0.462000in}}%
\pgfusepath{clip}%
\pgfsetbuttcap%
\pgfsetmiterjoin%
\definecolor{currentfill}{rgb}{0.121569,0.466667,0.705882}%
\pgfsetfillcolor{currentfill}%
\pgfsetfillopacity{0.500000}%
\pgfsetlinewidth{1.003750pt}%
\definecolor{currentstroke}{rgb}{0.000000,0.000000,0.000000}%
\pgfsetstrokecolor{currentstroke}%
\pgfsetdash{}{0pt}%
\pgfpathmoveto{\pgfqpoint{5.732509in}{0.841611in}}%
\pgfpathlineto{\pgfqpoint{5.882412in}{0.841611in}}%
\pgfpathlineto{\pgfqpoint{5.882412in}{1.061611in}}%
\pgfpathlineto{\pgfqpoint{5.732509in}{1.061611in}}%
\pgfpathlineto{\pgfqpoint{5.732509in}{0.841611in}}%
\pgfpathclose%
\pgfusepath{stroke,fill}%
\end{pgfscope}%
\begin{pgfscope}%
\pgfsetrectcap%
\pgfsetmiterjoin%
\pgfsetlinewidth{0.803000pt}%
\definecolor{currentstroke}{rgb}{0.000000,0.000000,0.000000}%
\pgfsetstrokecolor{currentstroke}%
\pgfsetdash{}{0pt}%
\pgfpathmoveto{\pgfqpoint{5.095420in}{0.841611in}}%
\pgfpathlineto{\pgfqpoint{5.095420in}{1.303611in}}%
\pgfusepath{stroke}%
\end{pgfscope}%
\begin{pgfscope}%
\pgfsetrectcap%
\pgfsetmiterjoin%
\pgfsetlinewidth{0.803000pt}%
\definecolor{currentstroke}{rgb}{0.000000,0.000000,0.000000}%
\pgfsetstrokecolor{currentstroke}%
\pgfsetdash{}{0pt}%
\pgfpathmoveto{\pgfqpoint{5.919888in}{0.841611in}}%
\pgfpathlineto{\pgfqpoint{5.919888in}{1.303611in}}%
\pgfusepath{stroke}%
\end{pgfscope}%
\begin{pgfscope}%
\pgfsetrectcap%
\pgfsetmiterjoin%
\pgfsetlinewidth{0.803000pt}%
\definecolor{currentstroke}{rgb}{0.000000,0.000000,0.000000}%
\pgfsetstrokecolor{currentstroke}%
\pgfsetdash{}{0pt}%
\pgfpathmoveto{\pgfqpoint{5.095420in}{0.841611in}}%
\pgfpathlineto{\pgfqpoint{5.919888in}{0.841611in}}%
\pgfusepath{stroke}%
\end{pgfscope}%
\begin{pgfscope}%
\pgfsetrectcap%
\pgfsetmiterjoin%
\pgfsetlinewidth{0.803000pt}%
\definecolor{currentstroke}{rgb}{0.000000,0.000000,0.000000}%
\pgfsetstrokecolor{currentstroke}%
\pgfsetdash{}{0pt}%
\pgfpathmoveto{\pgfqpoint{5.095420in}{1.303611in}}%
\pgfpathlineto{\pgfqpoint{5.919888in}{1.303611in}}%
\pgfusepath{stroke}%
\end{pgfscope}%
\begin{pgfscope}%
\definecolor{textcolor}{rgb}{0.000000,0.000000,0.000000}%
\pgfsetstrokecolor{textcolor}%
\pgfsetfillcolor{textcolor}%
\pgftext[x=5.507654in,y=1.386944in,,base]{\color{textcolor}\rmfamily\fontsize{11.000000}{13.200000}\selectfont Magnolia}%
\end{pgfscope}%
\begin{pgfscope}%
\pgfsetbuttcap%
\pgfsetmiterjoin%
\definecolor{currentfill}{rgb}{1.000000,1.000000,1.000000}%
\pgfsetfillcolor{currentfill}%
\pgfsetlinewidth{0.000000pt}%
\definecolor{currentstroke}{rgb}{0.000000,0.000000,0.000000}%
\pgfsetstrokecolor{currentstroke}%
\pgfsetstrokeopacity{0.000000}%
\pgfsetdash{}{0pt}%
\pgfpathmoveto{\pgfqpoint{6.084781in}{0.841611in}}%
\pgfpathlineto{\pgfqpoint{6.909249in}{0.841611in}}%
\pgfpathlineto{\pgfqpoint{6.909249in}{1.303611in}}%
\pgfpathlineto{\pgfqpoint{6.084781in}{1.303611in}}%
\pgfpathlineto{\pgfqpoint{6.084781in}{0.841611in}}%
\pgfpathclose%
\pgfusepath{fill}%
\end{pgfscope}%
\begin{pgfscope}%
\pgfpathrectangle{\pgfqpoint{6.084781in}{0.841611in}}{\pgfqpoint{0.824468in}{0.462000in}}%
\pgfusepath{clip}%
\pgfsetbuttcap%
\pgfsetmiterjoin%
\definecolor{currentfill}{rgb}{0.121569,0.466667,0.705882}%
\pgfsetfillcolor{currentfill}%
\pgfsetfillopacity{0.500000}%
\pgfsetlinewidth{1.003750pt}%
\definecolor{currentstroke}{rgb}{0.000000,0.000000,0.000000}%
\pgfsetstrokecolor{currentstroke}%
\pgfsetdash{}{0pt}%
\pgfpathmoveto{\pgfqpoint{6.122257in}{0.841611in}}%
\pgfpathlineto{\pgfqpoint{6.272160in}{0.841611in}}%
\pgfpathlineto{\pgfqpoint{6.272160in}{1.281611in}}%
\pgfpathlineto{\pgfqpoint{6.122257in}{1.281611in}}%
\pgfpathlineto{\pgfqpoint{6.122257in}{0.841611in}}%
\pgfpathclose%
\pgfusepath{stroke,fill}%
\end{pgfscope}%
\begin{pgfscope}%
\pgfpathrectangle{\pgfqpoint{6.084781in}{0.841611in}}{\pgfqpoint{0.824468in}{0.462000in}}%
\pgfusepath{clip}%
\pgfsetbuttcap%
\pgfsetmiterjoin%
\definecolor{currentfill}{rgb}{0.121569,0.466667,0.705882}%
\pgfsetfillcolor{currentfill}%
\pgfsetfillopacity{0.500000}%
\pgfsetlinewidth{1.003750pt}%
\definecolor{currentstroke}{rgb}{0.000000,0.000000,0.000000}%
\pgfsetstrokecolor{currentstroke}%
\pgfsetdash{}{0pt}%
\pgfpathmoveto{\pgfqpoint{6.272160in}{0.841611in}}%
\pgfpathlineto{\pgfqpoint{6.422064in}{0.841611in}}%
\pgfpathlineto{\pgfqpoint{6.422064in}{1.093040in}}%
\pgfpathlineto{\pgfqpoint{6.272160in}{1.093040in}}%
\pgfpathlineto{\pgfqpoint{6.272160in}{0.841611in}}%
\pgfpathclose%
\pgfusepath{stroke,fill}%
\end{pgfscope}%
\begin{pgfscope}%
\pgfpathrectangle{\pgfqpoint{6.084781in}{0.841611in}}{\pgfqpoint{0.824468in}{0.462000in}}%
\pgfusepath{clip}%
\pgfsetbuttcap%
\pgfsetmiterjoin%
\definecolor{currentfill}{rgb}{0.121569,0.466667,0.705882}%
\pgfsetfillcolor{currentfill}%
\pgfsetfillopacity{0.500000}%
\pgfsetlinewidth{1.003750pt}%
\definecolor{currentstroke}{rgb}{0.000000,0.000000,0.000000}%
\pgfsetstrokecolor{currentstroke}%
\pgfsetdash{}{0pt}%
\pgfpathmoveto{\pgfqpoint{6.422064in}{0.841611in}}%
\pgfpathlineto{\pgfqpoint{6.571967in}{0.841611in}}%
\pgfpathlineto{\pgfqpoint{6.571967in}{0.967325in}}%
\pgfpathlineto{\pgfqpoint{6.422064in}{0.967325in}}%
\pgfpathlineto{\pgfqpoint{6.422064in}{0.841611in}}%
\pgfpathclose%
\pgfusepath{stroke,fill}%
\end{pgfscope}%
\begin{pgfscope}%
\pgfpathrectangle{\pgfqpoint{6.084781in}{0.841611in}}{\pgfqpoint{0.824468in}{0.462000in}}%
\pgfusepath{clip}%
\pgfsetbuttcap%
\pgfsetmiterjoin%
\definecolor{currentfill}{rgb}{0.121569,0.466667,0.705882}%
\pgfsetfillcolor{currentfill}%
\pgfsetfillopacity{0.500000}%
\pgfsetlinewidth{1.003750pt}%
\definecolor{currentstroke}{rgb}{0.000000,0.000000,0.000000}%
\pgfsetstrokecolor{currentstroke}%
\pgfsetdash{}{0pt}%
\pgfpathmoveto{\pgfqpoint{6.571967in}{0.841611in}}%
\pgfpathlineto{\pgfqpoint{6.721870in}{0.841611in}}%
\pgfpathlineto{\pgfqpoint{6.721870in}{0.873040in}}%
\pgfpathlineto{\pgfqpoint{6.571967in}{0.873040in}}%
\pgfpathlineto{\pgfqpoint{6.571967in}{0.841611in}}%
\pgfpathclose%
\pgfusepath{stroke,fill}%
\end{pgfscope}%
\begin{pgfscope}%
\pgfpathrectangle{\pgfqpoint{6.084781in}{0.841611in}}{\pgfqpoint{0.824468in}{0.462000in}}%
\pgfusepath{clip}%
\pgfsetbuttcap%
\pgfsetmiterjoin%
\definecolor{currentfill}{rgb}{0.121569,0.466667,0.705882}%
\pgfsetfillcolor{currentfill}%
\pgfsetfillopacity{0.500000}%
\pgfsetlinewidth{1.003750pt}%
\definecolor{currentstroke}{rgb}{0.000000,0.000000,0.000000}%
\pgfsetstrokecolor{currentstroke}%
\pgfsetdash{}{0pt}%
\pgfpathmoveto{\pgfqpoint{6.721870in}{0.841611in}}%
\pgfpathlineto{\pgfqpoint{6.871774in}{0.841611in}}%
\pgfpathlineto{\pgfqpoint{6.871774in}{0.841611in}}%
\pgfpathlineto{\pgfqpoint{6.721870in}{0.841611in}}%
\pgfpathlineto{\pgfqpoint{6.721870in}{0.841611in}}%
\pgfpathclose%
\pgfusepath{stroke,fill}%
\end{pgfscope}%
\begin{pgfscope}%
\pgfsetrectcap%
\pgfsetmiterjoin%
\pgfsetlinewidth{0.803000pt}%
\definecolor{currentstroke}{rgb}{0.000000,0.000000,0.000000}%
\pgfsetstrokecolor{currentstroke}%
\pgfsetdash{}{0pt}%
\pgfpathmoveto{\pgfqpoint{6.084781in}{0.841611in}}%
\pgfpathlineto{\pgfqpoint{6.084781in}{1.303611in}}%
\pgfusepath{stroke}%
\end{pgfscope}%
\begin{pgfscope}%
\pgfsetrectcap%
\pgfsetmiterjoin%
\pgfsetlinewidth{0.803000pt}%
\definecolor{currentstroke}{rgb}{0.000000,0.000000,0.000000}%
\pgfsetstrokecolor{currentstroke}%
\pgfsetdash{}{0pt}%
\pgfpathmoveto{\pgfqpoint{6.909249in}{0.841611in}}%
\pgfpathlineto{\pgfqpoint{6.909249in}{1.303611in}}%
\pgfusepath{stroke}%
\end{pgfscope}%
\begin{pgfscope}%
\pgfsetrectcap%
\pgfsetmiterjoin%
\pgfsetlinewidth{0.803000pt}%
\definecolor{currentstroke}{rgb}{0.000000,0.000000,0.000000}%
\pgfsetstrokecolor{currentstroke}%
\pgfsetdash{}{0pt}%
\pgfpathmoveto{\pgfqpoint{6.084781in}{0.841611in}}%
\pgfpathlineto{\pgfqpoint{6.909249in}{0.841611in}}%
\pgfusepath{stroke}%
\end{pgfscope}%
\begin{pgfscope}%
\pgfsetrectcap%
\pgfsetmiterjoin%
\pgfsetlinewidth{0.803000pt}%
\definecolor{currentstroke}{rgb}{0.000000,0.000000,0.000000}%
\pgfsetstrokecolor{currentstroke}%
\pgfsetdash{}{0pt}%
\pgfpathmoveto{\pgfqpoint{6.084781in}{1.303611in}}%
\pgfpathlineto{\pgfqpoint{6.909249in}{1.303611in}}%
\pgfusepath{stroke}%
\end{pgfscope}%
\begin{pgfscope}%
\definecolor{textcolor}{rgb}{0.000000,0.000000,0.000000}%
\pgfsetstrokecolor{textcolor}%
\pgfsetfillcolor{textcolor}%
\pgftext[x=6.497015in,y=1.386944in,,base]{\color{textcolor}\rmfamily\fontsize{11.000000}{13.200000}\selectfont Suravenir}%
\end{pgfscope}%
\begin{pgfscope}%
\pgfsetbuttcap%
\pgfsetmiterjoin%
\definecolor{currentfill}{rgb}{1.000000,1.000000,1.000000}%
\pgfsetfillcolor{currentfill}%
\pgfsetlinewidth{0.000000pt}%
\definecolor{currentstroke}{rgb}{0.000000,0.000000,0.000000}%
\pgfsetstrokecolor{currentstroke}%
\pgfsetstrokeopacity{0.000000}%
\pgfsetdash{}{0pt}%
\pgfpathmoveto{\pgfqpoint{7.074143in}{0.841611in}}%
\pgfpathlineto{\pgfqpoint{7.898611in}{0.841611in}}%
\pgfpathlineto{\pgfqpoint{7.898611in}{1.303611in}}%
\pgfpathlineto{\pgfqpoint{7.074143in}{1.303611in}}%
\pgfpathlineto{\pgfqpoint{7.074143in}{0.841611in}}%
\pgfpathclose%
\pgfusepath{fill}%
\end{pgfscope}%
\begin{pgfscope}%
\pgfpathrectangle{\pgfqpoint{7.074143in}{0.841611in}}{\pgfqpoint{0.824468in}{0.462000in}}%
\pgfusepath{clip}%
\pgfsetbuttcap%
\pgfsetmiterjoin%
\definecolor{currentfill}{rgb}{0.121569,0.466667,0.705882}%
\pgfsetfillcolor{currentfill}%
\pgfsetfillopacity{0.500000}%
\pgfsetlinewidth{1.003750pt}%
\definecolor{currentstroke}{rgb}{0.000000,0.000000,0.000000}%
\pgfsetstrokecolor{currentstroke}%
\pgfsetdash{}{0pt}%
\pgfpathmoveto{\pgfqpoint{7.111619in}{0.841611in}}%
\pgfpathlineto{\pgfqpoint{7.261522in}{0.841611in}}%
\pgfpathlineto{\pgfqpoint{7.261522in}{1.281611in}}%
\pgfpathlineto{\pgfqpoint{7.111619in}{1.281611in}}%
\pgfpathlineto{\pgfqpoint{7.111619in}{0.841611in}}%
\pgfpathclose%
\pgfusepath{stroke,fill}%
\end{pgfscope}%
\begin{pgfscope}%
\pgfpathrectangle{\pgfqpoint{7.074143in}{0.841611in}}{\pgfqpoint{0.824468in}{0.462000in}}%
\pgfusepath{clip}%
\pgfsetbuttcap%
\pgfsetmiterjoin%
\definecolor{currentfill}{rgb}{0.121569,0.466667,0.705882}%
\pgfsetfillcolor{currentfill}%
\pgfsetfillopacity{0.500000}%
\pgfsetlinewidth{1.003750pt}%
\definecolor{currentstroke}{rgb}{0.000000,0.000000,0.000000}%
\pgfsetstrokecolor{currentstroke}%
\pgfsetdash{}{0pt}%
\pgfpathmoveto{\pgfqpoint{7.261522in}{0.841611in}}%
\pgfpathlineto{\pgfqpoint{7.411425in}{0.841611in}}%
\pgfpathlineto{\pgfqpoint{7.411425in}{1.051135in}}%
\pgfpathlineto{\pgfqpoint{7.261522in}{1.051135in}}%
\pgfpathlineto{\pgfqpoint{7.261522in}{0.841611in}}%
\pgfpathclose%
\pgfusepath{stroke,fill}%
\end{pgfscope}%
\begin{pgfscope}%
\pgfpathrectangle{\pgfqpoint{7.074143in}{0.841611in}}{\pgfqpoint{0.824468in}{0.462000in}}%
\pgfusepath{clip}%
\pgfsetbuttcap%
\pgfsetmiterjoin%
\definecolor{currentfill}{rgb}{0.121569,0.466667,0.705882}%
\pgfsetfillcolor{currentfill}%
\pgfsetfillopacity{0.500000}%
\pgfsetlinewidth{1.003750pt}%
\definecolor{currentstroke}{rgb}{0.000000,0.000000,0.000000}%
\pgfsetstrokecolor{currentstroke}%
\pgfsetdash{}{0pt}%
\pgfpathmoveto{\pgfqpoint{7.411425in}{0.841611in}}%
\pgfpathlineto{\pgfqpoint{7.561329in}{0.841611in}}%
\pgfpathlineto{\pgfqpoint{7.561329in}{0.925421in}}%
\pgfpathlineto{\pgfqpoint{7.411425in}{0.925421in}}%
\pgfpathlineto{\pgfqpoint{7.411425in}{0.841611in}}%
\pgfpathclose%
\pgfusepath{stroke,fill}%
\end{pgfscope}%
\begin{pgfscope}%
\pgfpathrectangle{\pgfqpoint{7.074143in}{0.841611in}}{\pgfqpoint{0.824468in}{0.462000in}}%
\pgfusepath{clip}%
\pgfsetbuttcap%
\pgfsetmiterjoin%
\definecolor{currentfill}{rgb}{0.121569,0.466667,0.705882}%
\pgfsetfillcolor{currentfill}%
\pgfsetfillopacity{0.500000}%
\pgfsetlinewidth{1.003750pt}%
\definecolor{currentstroke}{rgb}{0.000000,0.000000,0.000000}%
\pgfsetstrokecolor{currentstroke}%
\pgfsetdash{}{0pt}%
\pgfpathmoveto{\pgfqpoint{7.561329in}{0.841611in}}%
\pgfpathlineto{\pgfqpoint{7.711232in}{0.841611in}}%
\pgfpathlineto{\pgfqpoint{7.711232in}{0.862563in}}%
\pgfpathlineto{\pgfqpoint{7.561329in}{0.862563in}}%
\pgfpathlineto{\pgfqpoint{7.561329in}{0.841611in}}%
\pgfpathclose%
\pgfusepath{stroke,fill}%
\end{pgfscope}%
\begin{pgfscope}%
\pgfpathrectangle{\pgfqpoint{7.074143in}{0.841611in}}{\pgfqpoint{0.824468in}{0.462000in}}%
\pgfusepath{clip}%
\pgfsetbuttcap%
\pgfsetmiterjoin%
\definecolor{currentfill}{rgb}{0.121569,0.466667,0.705882}%
\pgfsetfillcolor{currentfill}%
\pgfsetfillopacity{0.500000}%
\pgfsetlinewidth{1.003750pt}%
\definecolor{currentstroke}{rgb}{0.000000,0.000000,0.000000}%
\pgfsetstrokecolor{currentstroke}%
\pgfsetdash{}{0pt}%
\pgfpathmoveto{\pgfqpoint{7.711232in}{0.841611in}}%
\pgfpathlineto{\pgfqpoint{7.861135in}{0.841611in}}%
\pgfpathlineto{\pgfqpoint{7.861135in}{1.155897in}}%
\pgfpathlineto{\pgfqpoint{7.711232in}{1.155897in}}%
\pgfpathlineto{\pgfqpoint{7.711232in}{0.841611in}}%
\pgfpathclose%
\pgfusepath{stroke,fill}%
\end{pgfscope}%
\begin{pgfscope}%
\pgfsetrectcap%
\pgfsetmiterjoin%
\pgfsetlinewidth{0.803000pt}%
\definecolor{currentstroke}{rgb}{0.000000,0.000000,0.000000}%
\pgfsetstrokecolor{currentstroke}%
\pgfsetdash{}{0pt}%
\pgfpathmoveto{\pgfqpoint{7.074143in}{0.841611in}}%
\pgfpathlineto{\pgfqpoint{7.074143in}{1.303611in}}%
\pgfusepath{stroke}%
\end{pgfscope}%
\begin{pgfscope}%
\pgfsetrectcap%
\pgfsetmiterjoin%
\pgfsetlinewidth{0.803000pt}%
\definecolor{currentstroke}{rgb}{0.000000,0.000000,0.000000}%
\pgfsetstrokecolor{currentstroke}%
\pgfsetdash{}{0pt}%
\pgfpathmoveto{\pgfqpoint{7.898611in}{0.841611in}}%
\pgfpathlineto{\pgfqpoint{7.898611in}{1.303611in}}%
\pgfusepath{stroke}%
\end{pgfscope}%
\begin{pgfscope}%
\pgfsetrectcap%
\pgfsetmiterjoin%
\pgfsetlinewidth{0.803000pt}%
\definecolor{currentstroke}{rgb}{0.000000,0.000000,0.000000}%
\pgfsetstrokecolor{currentstroke}%
\pgfsetdash{}{0pt}%
\pgfpathmoveto{\pgfqpoint{7.074143in}{0.841611in}}%
\pgfpathlineto{\pgfqpoint{7.898611in}{0.841611in}}%
\pgfusepath{stroke}%
\end{pgfscope}%
\begin{pgfscope}%
\pgfsetrectcap%
\pgfsetmiterjoin%
\pgfsetlinewidth{0.803000pt}%
\definecolor{currentstroke}{rgb}{0.000000,0.000000,0.000000}%
\pgfsetstrokecolor{currentstroke}%
\pgfsetdash{}{0pt}%
\pgfpathmoveto{\pgfqpoint{7.074143in}{1.303611in}}%
\pgfpathlineto{\pgfqpoint{7.898611in}{1.303611in}}%
\pgfusepath{stroke}%
\end{pgfscope}%
\begin{pgfscope}%
\definecolor{textcolor}{rgb}{0.000000,0.000000,0.000000}%
\pgfsetstrokecolor{textcolor}%
\pgfsetfillcolor{textcolor}%
\pgftext[x=7.486377in,y=1.386944in,,base]{\color{textcolor}\rmfamily\fontsize{11.000000}{13.200000}\selectfont Assur ...}%
\end{pgfscope}%
\begin{pgfscope}%
\pgfsetbuttcap%
\pgfsetmiterjoin%
\definecolor{currentfill}{rgb}{1.000000,1.000000,1.000000}%
\pgfsetfillcolor{currentfill}%
\pgfsetlinewidth{0.000000pt}%
\definecolor{currentstroke}{rgb}{0.000000,0.000000,0.000000}%
\pgfsetstrokecolor{currentstroke}%
\pgfsetstrokeopacity{0.000000}%
\pgfsetdash{}{0pt}%
\pgfpathmoveto{\pgfqpoint{0.148611in}{0.148611in}}%
\pgfpathlineto{\pgfqpoint{0.973079in}{0.148611in}}%
\pgfpathlineto{\pgfqpoint{0.973079in}{0.610611in}}%
\pgfpathlineto{\pgfqpoint{0.148611in}{0.610611in}}%
\pgfpathlineto{\pgfqpoint{0.148611in}{0.148611in}}%
\pgfpathclose%
\pgfusepath{fill}%
\end{pgfscope}%
\begin{pgfscope}%
\pgfpathrectangle{\pgfqpoint{0.148611in}{0.148611in}}{\pgfqpoint{0.824468in}{0.462000in}}%
\pgfusepath{clip}%
\pgfsetbuttcap%
\pgfsetmiterjoin%
\definecolor{currentfill}{rgb}{0.121569,0.466667,0.705882}%
\pgfsetfillcolor{currentfill}%
\pgfsetfillopacity{0.500000}%
\pgfsetlinewidth{1.003750pt}%
\definecolor{currentstroke}{rgb}{0.000000,0.000000,0.000000}%
\pgfsetstrokecolor{currentstroke}%
\pgfsetdash{}{0pt}%
\pgfpathmoveto{\pgfqpoint{0.186087in}{0.148611in}}%
\pgfpathlineto{\pgfqpoint{0.335990in}{0.148611in}}%
\pgfpathlineto{\pgfqpoint{0.335990in}{0.588611in}}%
\pgfpathlineto{\pgfqpoint{0.186087in}{0.588611in}}%
\pgfpathlineto{\pgfqpoint{0.186087in}{0.148611in}}%
\pgfpathclose%
\pgfusepath{stroke,fill}%
\end{pgfscope}%
\begin{pgfscope}%
\pgfpathrectangle{\pgfqpoint{0.148611in}{0.148611in}}{\pgfqpoint{0.824468in}{0.462000in}}%
\pgfusepath{clip}%
\pgfsetbuttcap%
\pgfsetmiterjoin%
\definecolor{currentfill}{rgb}{0.121569,0.466667,0.705882}%
\pgfsetfillcolor{currentfill}%
\pgfsetfillopacity{0.500000}%
\pgfsetlinewidth{1.003750pt}%
\definecolor{currentstroke}{rgb}{0.000000,0.000000,0.000000}%
\pgfsetstrokecolor{currentstroke}%
\pgfsetdash{}{0pt}%
\pgfpathmoveto{\pgfqpoint{0.335990in}{0.148611in}}%
\pgfpathlineto{\pgfqpoint{0.485894in}{0.148611in}}%
\pgfpathlineto{\pgfqpoint{0.485894in}{0.412611in}}%
\pgfpathlineto{\pgfqpoint{0.335990in}{0.412611in}}%
\pgfpathlineto{\pgfqpoint{0.335990in}{0.148611in}}%
\pgfpathclose%
\pgfusepath{stroke,fill}%
\end{pgfscope}%
\begin{pgfscope}%
\pgfpathrectangle{\pgfqpoint{0.148611in}{0.148611in}}{\pgfqpoint{0.824468in}{0.462000in}}%
\pgfusepath{clip}%
\pgfsetbuttcap%
\pgfsetmiterjoin%
\definecolor{currentfill}{rgb}{0.121569,0.466667,0.705882}%
\pgfsetfillcolor{currentfill}%
\pgfsetfillopacity{0.500000}%
\pgfsetlinewidth{1.003750pt}%
\definecolor{currentstroke}{rgb}{0.000000,0.000000,0.000000}%
\pgfsetstrokecolor{currentstroke}%
\pgfsetdash{}{0pt}%
\pgfpathmoveto{\pgfqpoint{0.485894in}{0.148611in}}%
\pgfpathlineto{\pgfqpoint{0.635797in}{0.148611in}}%
\pgfpathlineto{\pgfqpoint{0.635797in}{0.177944in}}%
\pgfpathlineto{\pgfqpoint{0.485894in}{0.177944in}}%
\pgfpathlineto{\pgfqpoint{0.485894in}{0.148611in}}%
\pgfpathclose%
\pgfusepath{stroke,fill}%
\end{pgfscope}%
\begin{pgfscope}%
\pgfpathrectangle{\pgfqpoint{0.148611in}{0.148611in}}{\pgfqpoint{0.824468in}{0.462000in}}%
\pgfusepath{clip}%
\pgfsetbuttcap%
\pgfsetmiterjoin%
\definecolor{currentfill}{rgb}{0.121569,0.466667,0.705882}%
\pgfsetfillcolor{currentfill}%
\pgfsetfillopacity{0.500000}%
\pgfsetlinewidth{1.003750pt}%
\definecolor{currentstroke}{rgb}{0.000000,0.000000,0.000000}%
\pgfsetstrokecolor{currentstroke}%
\pgfsetdash{}{0pt}%
\pgfpathmoveto{\pgfqpoint{0.635797in}{0.148611in}}%
\pgfpathlineto{\pgfqpoint{0.785700in}{0.148611in}}%
\pgfpathlineto{\pgfqpoint{0.785700in}{0.148611in}}%
\pgfpathlineto{\pgfqpoint{0.635797in}{0.148611in}}%
\pgfpathlineto{\pgfqpoint{0.635797in}{0.148611in}}%
\pgfpathclose%
\pgfusepath{stroke,fill}%
\end{pgfscope}%
\begin{pgfscope}%
\pgfpathrectangle{\pgfqpoint{0.148611in}{0.148611in}}{\pgfqpoint{0.824468in}{0.462000in}}%
\pgfusepath{clip}%
\pgfsetbuttcap%
\pgfsetmiterjoin%
\definecolor{currentfill}{rgb}{0.121569,0.466667,0.705882}%
\pgfsetfillcolor{currentfill}%
\pgfsetfillopacity{0.500000}%
\pgfsetlinewidth{1.003750pt}%
\definecolor{currentstroke}{rgb}{0.000000,0.000000,0.000000}%
\pgfsetstrokecolor{currentstroke}%
\pgfsetdash{}{0pt}%
\pgfpathmoveto{\pgfqpoint{0.785700in}{0.148611in}}%
\pgfpathlineto{\pgfqpoint{0.935603in}{0.148611in}}%
\pgfpathlineto{\pgfqpoint{0.935603in}{0.148611in}}%
\pgfpathlineto{\pgfqpoint{0.785700in}{0.148611in}}%
\pgfpathlineto{\pgfqpoint{0.785700in}{0.148611in}}%
\pgfpathclose%
\pgfusepath{stroke,fill}%
\end{pgfscope}%
\begin{pgfscope}%
\pgfsetrectcap%
\pgfsetmiterjoin%
\pgfsetlinewidth{0.803000pt}%
\definecolor{currentstroke}{rgb}{0.000000,0.000000,0.000000}%
\pgfsetstrokecolor{currentstroke}%
\pgfsetdash{}{0pt}%
\pgfpathmoveto{\pgfqpoint{0.148611in}{0.148611in}}%
\pgfpathlineto{\pgfqpoint{0.148611in}{0.610611in}}%
\pgfusepath{stroke}%
\end{pgfscope}%
\begin{pgfscope}%
\pgfsetrectcap%
\pgfsetmiterjoin%
\pgfsetlinewidth{0.803000pt}%
\definecolor{currentstroke}{rgb}{0.000000,0.000000,0.000000}%
\pgfsetstrokecolor{currentstroke}%
\pgfsetdash{}{0pt}%
\pgfpathmoveto{\pgfqpoint{0.973079in}{0.148611in}}%
\pgfpathlineto{\pgfqpoint{0.973079in}{0.610611in}}%
\pgfusepath{stroke}%
\end{pgfscope}%
\begin{pgfscope}%
\pgfsetrectcap%
\pgfsetmiterjoin%
\pgfsetlinewidth{0.803000pt}%
\definecolor{currentstroke}{rgb}{0.000000,0.000000,0.000000}%
\pgfsetstrokecolor{currentstroke}%
\pgfsetdash{}{0pt}%
\pgfpathmoveto{\pgfqpoint{0.148611in}{0.148611in}}%
\pgfpathlineto{\pgfqpoint{0.973079in}{0.148611in}}%
\pgfusepath{stroke}%
\end{pgfscope}%
\begin{pgfscope}%
\pgfsetrectcap%
\pgfsetmiterjoin%
\pgfsetlinewidth{0.803000pt}%
\definecolor{currentstroke}{rgb}{0.000000,0.000000,0.000000}%
\pgfsetstrokecolor{currentstroke}%
\pgfsetdash{}{0pt}%
\pgfpathmoveto{\pgfqpoint{0.148611in}{0.610611in}}%
\pgfpathlineto{\pgfqpoint{0.973079in}{0.610611in}}%
\pgfusepath{stroke}%
\end{pgfscope}%
\begin{pgfscope}%
\definecolor{textcolor}{rgb}{0.000000,0.000000,0.000000}%
\pgfsetstrokecolor{textcolor}%
\pgfsetfillcolor{textcolor}%
\pgftext[x=0.560845in,y=0.693944in,,base]{\color{textcolor}\rmfamily\fontsize{11.000000}{13.200000}\selectfont AssurO...}%
\end{pgfscope}%
\begin{pgfscope}%
\pgfsetbuttcap%
\pgfsetmiterjoin%
\definecolor{currentfill}{rgb}{1.000000,1.000000,1.000000}%
\pgfsetfillcolor{currentfill}%
\pgfsetlinewidth{0.000000pt}%
\definecolor{currentstroke}{rgb}{0.000000,0.000000,0.000000}%
\pgfsetstrokecolor{currentstroke}%
\pgfsetstrokeopacity{0.000000}%
\pgfsetdash{}{0pt}%
\pgfpathmoveto{\pgfqpoint{1.137973in}{0.148611in}}%
\pgfpathlineto{\pgfqpoint{1.962441in}{0.148611in}}%
\pgfpathlineto{\pgfqpoint{1.962441in}{0.610611in}}%
\pgfpathlineto{\pgfqpoint{1.137973in}{0.610611in}}%
\pgfpathlineto{\pgfqpoint{1.137973in}{0.148611in}}%
\pgfpathclose%
\pgfusepath{fill}%
\end{pgfscope}%
\begin{pgfscope}%
\pgfpathrectangle{\pgfqpoint{1.137973in}{0.148611in}}{\pgfqpoint{0.824468in}{0.462000in}}%
\pgfusepath{clip}%
\pgfsetbuttcap%
\pgfsetmiterjoin%
\definecolor{currentfill}{rgb}{0.121569,0.466667,0.705882}%
\pgfsetfillcolor{currentfill}%
\pgfsetfillopacity{0.500000}%
\pgfsetlinewidth{1.003750pt}%
\definecolor{currentstroke}{rgb}{0.000000,0.000000,0.000000}%
\pgfsetstrokecolor{currentstroke}%
\pgfsetdash{}{0pt}%
\pgfpathmoveto{\pgfqpoint{1.175449in}{0.148611in}}%
\pgfpathlineto{\pgfqpoint{1.325352in}{0.148611in}}%
\pgfpathlineto{\pgfqpoint{1.325352in}{0.588611in}}%
\pgfpathlineto{\pgfqpoint{1.175449in}{0.588611in}}%
\pgfpathlineto{\pgfqpoint{1.175449in}{0.148611in}}%
\pgfpathclose%
\pgfusepath{stroke,fill}%
\end{pgfscope}%
\begin{pgfscope}%
\pgfpathrectangle{\pgfqpoint{1.137973in}{0.148611in}}{\pgfqpoint{0.824468in}{0.462000in}}%
\pgfusepath{clip}%
\pgfsetbuttcap%
\pgfsetmiterjoin%
\definecolor{currentfill}{rgb}{0.121569,0.466667,0.705882}%
\pgfsetfillcolor{currentfill}%
\pgfsetfillopacity{0.500000}%
\pgfsetlinewidth{1.003750pt}%
\definecolor{currentstroke}{rgb}{0.000000,0.000000,0.000000}%
\pgfsetstrokecolor{currentstroke}%
\pgfsetdash{}{0pt}%
\pgfpathmoveto{\pgfqpoint{1.325352in}{0.148611in}}%
\pgfpathlineto{\pgfqpoint{1.475255in}{0.148611in}}%
\pgfpathlineto{\pgfqpoint{1.475255in}{0.295278in}}%
\pgfpathlineto{\pgfqpoint{1.325352in}{0.295278in}}%
\pgfpathlineto{\pgfqpoint{1.325352in}{0.148611in}}%
\pgfpathclose%
\pgfusepath{stroke,fill}%
\end{pgfscope}%
\begin{pgfscope}%
\pgfpathrectangle{\pgfqpoint{1.137973in}{0.148611in}}{\pgfqpoint{0.824468in}{0.462000in}}%
\pgfusepath{clip}%
\pgfsetbuttcap%
\pgfsetmiterjoin%
\definecolor{currentfill}{rgb}{0.121569,0.466667,0.705882}%
\pgfsetfillcolor{currentfill}%
\pgfsetfillopacity{0.500000}%
\pgfsetlinewidth{1.003750pt}%
\definecolor{currentstroke}{rgb}{0.000000,0.000000,0.000000}%
\pgfsetstrokecolor{currentstroke}%
\pgfsetdash{}{0pt}%
\pgfpathmoveto{\pgfqpoint{1.475255in}{0.148611in}}%
\pgfpathlineto{\pgfqpoint{1.625158in}{0.148611in}}%
\pgfpathlineto{\pgfqpoint{1.625158in}{0.177944in}}%
\pgfpathlineto{\pgfqpoint{1.475255in}{0.177944in}}%
\pgfpathlineto{\pgfqpoint{1.475255in}{0.148611in}}%
\pgfpathclose%
\pgfusepath{stroke,fill}%
\end{pgfscope}%
\begin{pgfscope}%
\pgfpathrectangle{\pgfqpoint{1.137973in}{0.148611in}}{\pgfqpoint{0.824468in}{0.462000in}}%
\pgfusepath{clip}%
\pgfsetbuttcap%
\pgfsetmiterjoin%
\definecolor{currentfill}{rgb}{0.121569,0.466667,0.705882}%
\pgfsetfillcolor{currentfill}%
\pgfsetfillopacity{0.500000}%
\pgfsetlinewidth{1.003750pt}%
\definecolor{currentstroke}{rgb}{0.000000,0.000000,0.000000}%
\pgfsetstrokecolor{currentstroke}%
\pgfsetdash{}{0pt}%
\pgfpathmoveto{\pgfqpoint{1.625158in}{0.148611in}}%
\pgfpathlineto{\pgfqpoint{1.775062in}{0.148611in}}%
\pgfpathlineto{\pgfqpoint{1.775062in}{0.324611in}}%
\pgfpathlineto{\pgfqpoint{1.625158in}{0.324611in}}%
\pgfpathlineto{\pgfqpoint{1.625158in}{0.148611in}}%
\pgfpathclose%
\pgfusepath{stroke,fill}%
\end{pgfscope}%
\begin{pgfscope}%
\pgfpathrectangle{\pgfqpoint{1.137973in}{0.148611in}}{\pgfqpoint{0.824468in}{0.462000in}}%
\pgfusepath{clip}%
\pgfsetbuttcap%
\pgfsetmiterjoin%
\definecolor{currentfill}{rgb}{0.121569,0.466667,0.705882}%
\pgfsetfillcolor{currentfill}%
\pgfsetfillopacity{0.500000}%
\pgfsetlinewidth{1.003750pt}%
\definecolor{currentstroke}{rgb}{0.000000,0.000000,0.000000}%
\pgfsetstrokecolor{currentstroke}%
\pgfsetdash{}{0pt}%
\pgfpathmoveto{\pgfqpoint{1.775062in}{0.148611in}}%
\pgfpathlineto{\pgfqpoint{1.924965in}{0.148611in}}%
\pgfpathlineto{\pgfqpoint{1.924965in}{0.295278in}}%
\pgfpathlineto{\pgfqpoint{1.775062in}{0.295278in}}%
\pgfpathlineto{\pgfqpoint{1.775062in}{0.148611in}}%
\pgfpathclose%
\pgfusepath{stroke,fill}%
\end{pgfscope}%
\begin{pgfscope}%
\pgfsetrectcap%
\pgfsetmiterjoin%
\pgfsetlinewidth{0.803000pt}%
\definecolor{currentstroke}{rgb}{0.000000,0.000000,0.000000}%
\pgfsetstrokecolor{currentstroke}%
\pgfsetdash{}{0pt}%
\pgfpathmoveto{\pgfqpoint{1.137973in}{0.148611in}}%
\pgfpathlineto{\pgfqpoint{1.137973in}{0.610611in}}%
\pgfusepath{stroke}%
\end{pgfscope}%
\begin{pgfscope}%
\pgfsetrectcap%
\pgfsetmiterjoin%
\pgfsetlinewidth{0.803000pt}%
\definecolor{currentstroke}{rgb}{0.000000,0.000000,0.000000}%
\pgfsetstrokecolor{currentstroke}%
\pgfsetdash{}{0pt}%
\pgfpathmoveto{\pgfqpoint{1.962441in}{0.148611in}}%
\pgfpathlineto{\pgfqpoint{1.962441in}{0.610611in}}%
\pgfusepath{stroke}%
\end{pgfscope}%
\begin{pgfscope}%
\pgfsetrectcap%
\pgfsetmiterjoin%
\pgfsetlinewidth{0.803000pt}%
\definecolor{currentstroke}{rgb}{0.000000,0.000000,0.000000}%
\pgfsetstrokecolor{currentstroke}%
\pgfsetdash{}{0pt}%
\pgfpathmoveto{\pgfqpoint{1.137973in}{0.148611in}}%
\pgfpathlineto{\pgfqpoint{1.962441in}{0.148611in}}%
\pgfusepath{stroke}%
\end{pgfscope}%
\begin{pgfscope}%
\pgfsetrectcap%
\pgfsetmiterjoin%
\pgfsetlinewidth{0.803000pt}%
\definecolor{currentstroke}{rgb}{0.000000,0.000000,0.000000}%
\pgfsetstrokecolor{currentstroke}%
\pgfsetdash{}{0pt}%
\pgfpathmoveto{\pgfqpoint{1.137973in}{0.610611in}}%
\pgfpathlineto{\pgfqpoint{1.962441in}{0.610611in}}%
\pgfusepath{stroke}%
\end{pgfscope}%
\begin{pgfscope}%
\definecolor{textcolor}{rgb}{0.000000,0.000000,0.000000}%
\pgfsetstrokecolor{textcolor}%
\pgfsetfillcolor{textcolor}%
\pgftext[x=1.550207in,y=0.693944in,,base]{\color{textcolor}\rmfamily\fontsize{11.000000}{13.200000}\selectfont Carac}%
\end{pgfscope}%
\begin{pgfscope}%
\pgfsetbuttcap%
\pgfsetmiterjoin%
\definecolor{currentfill}{rgb}{1.000000,1.000000,1.000000}%
\pgfsetfillcolor{currentfill}%
\pgfsetlinewidth{0.000000pt}%
\definecolor{currentstroke}{rgb}{0.000000,0.000000,0.000000}%
\pgfsetstrokecolor{currentstroke}%
\pgfsetstrokeopacity{0.000000}%
\pgfsetdash{}{0pt}%
\pgfpathmoveto{\pgfqpoint{2.127335in}{0.148611in}}%
\pgfpathlineto{\pgfqpoint{2.951803in}{0.148611in}}%
\pgfpathlineto{\pgfqpoint{2.951803in}{0.610611in}}%
\pgfpathlineto{\pgfqpoint{2.127335in}{0.610611in}}%
\pgfpathlineto{\pgfqpoint{2.127335in}{0.148611in}}%
\pgfpathclose%
\pgfusepath{fill}%
\end{pgfscope}%
\begin{pgfscope}%
\pgfpathrectangle{\pgfqpoint{2.127335in}{0.148611in}}{\pgfqpoint{0.824468in}{0.462000in}}%
\pgfusepath{clip}%
\pgfsetbuttcap%
\pgfsetmiterjoin%
\definecolor{currentfill}{rgb}{0.121569,0.466667,0.705882}%
\pgfsetfillcolor{currentfill}%
\pgfsetfillopacity{0.500000}%
\pgfsetlinewidth{1.003750pt}%
\definecolor{currentstroke}{rgb}{0.000000,0.000000,0.000000}%
\pgfsetstrokecolor{currentstroke}%
\pgfsetdash{}{0pt}%
\pgfpathmoveto{\pgfqpoint{2.164810in}{0.148611in}}%
\pgfpathlineto{\pgfqpoint{2.314714in}{0.148611in}}%
\pgfpathlineto{\pgfqpoint{2.314714in}{0.211468in}}%
\pgfpathlineto{\pgfqpoint{2.164810in}{0.211468in}}%
\pgfpathlineto{\pgfqpoint{2.164810in}{0.148611in}}%
\pgfpathclose%
\pgfusepath{stroke,fill}%
\end{pgfscope}%
\begin{pgfscope}%
\pgfpathrectangle{\pgfqpoint{2.127335in}{0.148611in}}{\pgfqpoint{0.824468in}{0.462000in}}%
\pgfusepath{clip}%
\pgfsetbuttcap%
\pgfsetmiterjoin%
\definecolor{currentfill}{rgb}{0.121569,0.466667,0.705882}%
\pgfsetfillcolor{currentfill}%
\pgfsetfillopacity{0.500000}%
\pgfsetlinewidth{1.003750pt}%
\definecolor{currentstroke}{rgb}{0.000000,0.000000,0.000000}%
\pgfsetstrokecolor{currentstroke}%
\pgfsetdash{}{0pt}%
\pgfpathmoveto{\pgfqpoint{2.314714in}{0.148611in}}%
\pgfpathlineto{\pgfqpoint{2.464617in}{0.148611in}}%
\pgfpathlineto{\pgfqpoint{2.464617in}{0.148611in}}%
\pgfpathlineto{\pgfqpoint{2.314714in}{0.148611in}}%
\pgfpathlineto{\pgfqpoint{2.314714in}{0.148611in}}%
\pgfpathclose%
\pgfusepath{stroke,fill}%
\end{pgfscope}%
\begin{pgfscope}%
\pgfpathrectangle{\pgfqpoint{2.127335in}{0.148611in}}{\pgfqpoint{0.824468in}{0.462000in}}%
\pgfusepath{clip}%
\pgfsetbuttcap%
\pgfsetmiterjoin%
\definecolor{currentfill}{rgb}{0.121569,0.466667,0.705882}%
\pgfsetfillcolor{currentfill}%
\pgfsetfillopacity{0.500000}%
\pgfsetlinewidth{1.003750pt}%
\definecolor{currentstroke}{rgb}{0.000000,0.000000,0.000000}%
\pgfsetstrokecolor{currentstroke}%
\pgfsetdash{}{0pt}%
\pgfpathmoveto{\pgfqpoint{2.464617in}{0.148611in}}%
\pgfpathlineto{\pgfqpoint{2.614520in}{0.148611in}}%
\pgfpathlineto{\pgfqpoint{2.614520in}{0.148611in}}%
\pgfpathlineto{\pgfqpoint{2.464617in}{0.148611in}}%
\pgfpathlineto{\pgfqpoint{2.464617in}{0.148611in}}%
\pgfpathclose%
\pgfusepath{stroke,fill}%
\end{pgfscope}%
\begin{pgfscope}%
\pgfpathrectangle{\pgfqpoint{2.127335in}{0.148611in}}{\pgfqpoint{0.824468in}{0.462000in}}%
\pgfusepath{clip}%
\pgfsetbuttcap%
\pgfsetmiterjoin%
\definecolor{currentfill}{rgb}{0.121569,0.466667,0.705882}%
\pgfsetfillcolor{currentfill}%
\pgfsetfillopacity{0.500000}%
\pgfsetlinewidth{1.003750pt}%
\definecolor{currentstroke}{rgb}{0.000000,0.000000,0.000000}%
\pgfsetstrokecolor{currentstroke}%
\pgfsetdash{}{0pt}%
\pgfpathmoveto{\pgfqpoint{2.614520in}{0.148611in}}%
\pgfpathlineto{\pgfqpoint{2.764423in}{0.148611in}}%
\pgfpathlineto{\pgfqpoint{2.764423in}{0.588611in}}%
\pgfpathlineto{\pgfqpoint{2.614520in}{0.588611in}}%
\pgfpathlineto{\pgfqpoint{2.614520in}{0.148611in}}%
\pgfpathclose%
\pgfusepath{stroke,fill}%
\end{pgfscope}%
\begin{pgfscope}%
\pgfpathrectangle{\pgfqpoint{2.127335in}{0.148611in}}{\pgfqpoint{0.824468in}{0.462000in}}%
\pgfusepath{clip}%
\pgfsetbuttcap%
\pgfsetmiterjoin%
\definecolor{currentfill}{rgb}{0.121569,0.466667,0.705882}%
\pgfsetfillcolor{currentfill}%
\pgfsetfillopacity{0.500000}%
\pgfsetlinewidth{1.003750pt}%
\definecolor{currentstroke}{rgb}{0.000000,0.000000,0.000000}%
\pgfsetstrokecolor{currentstroke}%
\pgfsetdash{}{0pt}%
\pgfpathmoveto{\pgfqpoint{2.764423in}{0.148611in}}%
\pgfpathlineto{\pgfqpoint{2.914327in}{0.148611in}}%
\pgfpathlineto{\pgfqpoint{2.914327in}{0.274325in}}%
\pgfpathlineto{\pgfqpoint{2.764423in}{0.274325in}}%
\pgfpathlineto{\pgfqpoint{2.764423in}{0.148611in}}%
\pgfpathclose%
\pgfusepath{stroke,fill}%
\end{pgfscope}%
\begin{pgfscope}%
\pgfsetrectcap%
\pgfsetmiterjoin%
\pgfsetlinewidth{0.803000pt}%
\definecolor{currentstroke}{rgb}{0.000000,0.000000,0.000000}%
\pgfsetstrokecolor{currentstroke}%
\pgfsetdash{}{0pt}%
\pgfpathmoveto{\pgfqpoint{2.127335in}{0.148611in}}%
\pgfpathlineto{\pgfqpoint{2.127335in}{0.610611in}}%
\pgfusepath{stroke}%
\end{pgfscope}%
\begin{pgfscope}%
\pgfsetrectcap%
\pgfsetmiterjoin%
\pgfsetlinewidth{0.803000pt}%
\definecolor{currentstroke}{rgb}{0.000000,0.000000,0.000000}%
\pgfsetstrokecolor{currentstroke}%
\pgfsetdash{}{0pt}%
\pgfpathmoveto{\pgfqpoint{2.951803in}{0.148611in}}%
\pgfpathlineto{\pgfqpoint{2.951803in}{0.610611in}}%
\pgfusepath{stroke}%
\end{pgfscope}%
\begin{pgfscope}%
\pgfsetrectcap%
\pgfsetmiterjoin%
\pgfsetlinewidth{0.803000pt}%
\definecolor{currentstroke}{rgb}{0.000000,0.000000,0.000000}%
\pgfsetstrokecolor{currentstroke}%
\pgfsetdash{}{0pt}%
\pgfpathmoveto{\pgfqpoint{2.127335in}{0.148611in}}%
\pgfpathlineto{\pgfqpoint{2.951803in}{0.148611in}}%
\pgfusepath{stroke}%
\end{pgfscope}%
\begin{pgfscope}%
\pgfsetrectcap%
\pgfsetmiterjoin%
\pgfsetlinewidth{0.803000pt}%
\definecolor{currentstroke}{rgb}{0.000000,0.000000,0.000000}%
\pgfsetstrokecolor{currentstroke}%
\pgfsetdash{}{0pt}%
\pgfpathmoveto{\pgfqpoint{2.127335in}{0.610611in}}%
\pgfpathlineto{\pgfqpoint{2.951803in}{0.610611in}}%
\pgfusepath{stroke}%
\end{pgfscope}%
\begin{pgfscope}%
\definecolor{textcolor}{rgb}{0.000000,0.000000,0.000000}%
\pgfsetstrokecolor{textcolor}%
\pgfsetfillcolor{textcolor}%
\pgftext[x=2.539569in,y=0.693944in,,base]{\color{textcolor}\rmfamily\fontsize{11.000000}{13.200000}\selectfont Mapa}%
\end{pgfscope}%
\begin{pgfscope}%
\pgfsetbuttcap%
\pgfsetmiterjoin%
\definecolor{currentfill}{rgb}{1.000000,1.000000,1.000000}%
\pgfsetfillcolor{currentfill}%
\pgfsetlinewidth{0.000000pt}%
\definecolor{currentstroke}{rgb}{0.000000,0.000000,0.000000}%
\pgfsetstrokecolor{currentstroke}%
\pgfsetstrokeopacity{0.000000}%
\pgfsetdash{}{0pt}%
\pgfpathmoveto{\pgfqpoint{3.116696in}{0.148611in}}%
\pgfpathlineto{\pgfqpoint{3.941164in}{0.148611in}}%
\pgfpathlineto{\pgfqpoint{3.941164in}{0.610611in}}%
\pgfpathlineto{\pgfqpoint{3.116696in}{0.610611in}}%
\pgfpathlineto{\pgfqpoint{3.116696in}{0.148611in}}%
\pgfpathclose%
\pgfusepath{fill}%
\end{pgfscope}%
\begin{pgfscope}%
\pgfpathrectangle{\pgfqpoint{3.116696in}{0.148611in}}{\pgfqpoint{0.824468in}{0.462000in}}%
\pgfusepath{clip}%
\pgfsetbuttcap%
\pgfsetmiterjoin%
\definecolor{currentfill}{rgb}{0.121569,0.466667,0.705882}%
\pgfsetfillcolor{currentfill}%
\pgfsetfillopacity{0.500000}%
\pgfsetlinewidth{1.003750pt}%
\definecolor{currentstroke}{rgb}{0.000000,0.000000,0.000000}%
\pgfsetstrokecolor{currentstroke}%
\pgfsetdash{}{0pt}%
\pgfpathmoveto{\pgfqpoint{3.154172in}{0.148611in}}%
\pgfpathlineto{\pgfqpoint{3.304075in}{0.148611in}}%
\pgfpathlineto{\pgfqpoint{3.304075in}{0.588611in}}%
\pgfpathlineto{\pgfqpoint{3.154172in}{0.588611in}}%
\pgfpathlineto{\pgfqpoint{3.154172in}{0.148611in}}%
\pgfpathclose%
\pgfusepath{stroke,fill}%
\end{pgfscope}%
\begin{pgfscope}%
\pgfpathrectangle{\pgfqpoint{3.116696in}{0.148611in}}{\pgfqpoint{0.824468in}{0.462000in}}%
\pgfusepath{clip}%
\pgfsetbuttcap%
\pgfsetmiterjoin%
\definecolor{currentfill}{rgb}{0.121569,0.466667,0.705882}%
\pgfsetfillcolor{currentfill}%
\pgfsetfillopacity{0.500000}%
\pgfsetlinewidth{1.003750pt}%
\definecolor{currentstroke}{rgb}{0.000000,0.000000,0.000000}%
\pgfsetstrokecolor{currentstroke}%
\pgfsetdash{}{0pt}%
\pgfpathmoveto{\pgfqpoint{3.304075in}{0.148611in}}%
\pgfpathlineto{\pgfqpoint{3.453979in}{0.148611in}}%
\pgfpathlineto{\pgfqpoint{3.453979in}{0.242897in}}%
\pgfpathlineto{\pgfqpoint{3.304075in}{0.242897in}}%
\pgfpathlineto{\pgfqpoint{3.304075in}{0.148611in}}%
\pgfpathclose%
\pgfusepath{stroke,fill}%
\end{pgfscope}%
\begin{pgfscope}%
\pgfpathrectangle{\pgfqpoint{3.116696in}{0.148611in}}{\pgfqpoint{0.824468in}{0.462000in}}%
\pgfusepath{clip}%
\pgfsetbuttcap%
\pgfsetmiterjoin%
\definecolor{currentfill}{rgb}{0.121569,0.466667,0.705882}%
\pgfsetfillcolor{currentfill}%
\pgfsetfillopacity{0.500000}%
\pgfsetlinewidth{1.003750pt}%
\definecolor{currentstroke}{rgb}{0.000000,0.000000,0.000000}%
\pgfsetstrokecolor{currentstroke}%
\pgfsetdash{}{0pt}%
\pgfpathmoveto{\pgfqpoint{3.453979in}{0.148611in}}%
\pgfpathlineto{\pgfqpoint{3.603882in}{0.148611in}}%
\pgfpathlineto{\pgfqpoint{3.603882in}{0.211468in}}%
\pgfpathlineto{\pgfqpoint{3.453979in}{0.211468in}}%
\pgfpathlineto{\pgfqpoint{3.453979in}{0.148611in}}%
\pgfpathclose%
\pgfusepath{stroke,fill}%
\end{pgfscope}%
\begin{pgfscope}%
\pgfpathrectangle{\pgfqpoint{3.116696in}{0.148611in}}{\pgfqpoint{0.824468in}{0.462000in}}%
\pgfusepath{clip}%
\pgfsetbuttcap%
\pgfsetmiterjoin%
\definecolor{currentfill}{rgb}{0.121569,0.466667,0.705882}%
\pgfsetfillcolor{currentfill}%
\pgfsetfillopacity{0.500000}%
\pgfsetlinewidth{1.003750pt}%
\definecolor{currentstroke}{rgb}{0.000000,0.000000,0.000000}%
\pgfsetstrokecolor{currentstroke}%
\pgfsetdash{}{0pt}%
\pgfpathmoveto{\pgfqpoint{3.603882in}{0.148611in}}%
\pgfpathlineto{\pgfqpoint{3.753785in}{0.148611in}}%
\pgfpathlineto{\pgfqpoint{3.753785in}{0.148611in}}%
\pgfpathlineto{\pgfqpoint{3.603882in}{0.148611in}}%
\pgfpathlineto{\pgfqpoint{3.603882in}{0.148611in}}%
\pgfpathclose%
\pgfusepath{stroke,fill}%
\end{pgfscope}%
\begin{pgfscope}%
\pgfpathrectangle{\pgfqpoint{3.116696in}{0.148611in}}{\pgfqpoint{0.824468in}{0.462000in}}%
\pgfusepath{clip}%
\pgfsetbuttcap%
\pgfsetmiterjoin%
\definecolor{currentfill}{rgb}{0.121569,0.466667,0.705882}%
\pgfsetfillcolor{currentfill}%
\pgfsetfillopacity{0.500000}%
\pgfsetlinewidth{1.003750pt}%
\definecolor{currentstroke}{rgb}{0.000000,0.000000,0.000000}%
\pgfsetstrokecolor{currentstroke}%
\pgfsetdash{}{0pt}%
\pgfpathmoveto{\pgfqpoint{3.753785in}{0.148611in}}%
\pgfpathlineto{\pgfqpoint{3.903688in}{0.148611in}}%
\pgfpathlineto{\pgfqpoint{3.903688in}{0.148611in}}%
\pgfpathlineto{\pgfqpoint{3.753785in}{0.148611in}}%
\pgfpathlineto{\pgfqpoint{3.753785in}{0.148611in}}%
\pgfpathclose%
\pgfusepath{stroke,fill}%
\end{pgfscope}%
\begin{pgfscope}%
\pgfsetrectcap%
\pgfsetmiterjoin%
\pgfsetlinewidth{0.803000pt}%
\definecolor{currentstroke}{rgb}{0.000000,0.000000,0.000000}%
\pgfsetstrokecolor{currentstroke}%
\pgfsetdash{}{0pt}%
\pgfpathmoveto{\pgfqpoint{3.116696in}{0.148611in}}%
\pgfpathlineto{\pgfqpoint{3.116696in}{0.610611in}}%
\pgfusepath{stroke}%
\end{pgfscope}%
\begin{pgfscope}%
\pgfsetrectcap%
\pgfsetmiterjoin%
\pgfsetlinewidth{0.803000pt}%
\definecolor{currentstroke}{rgb}{0.000000,0.000000,0.000000}%
\pgfsetstrokecolor{currentstroke}%
\pgfsetdash{}{0pt}%
\pgfpathmoveto{\pgfqpoint{3.941164in}{0.148611in}}%
\pgfpathlineto{\pgfqpoint{3.941164in}{0.610611in}}%
\pgfusepath{stroke}%
\end{pgfscope}%
\begin{pgfscope}%
\pgfsetrectcap%
\pgfsetmiterjoin%
\pgfsetlinewidth{0.803000pt}%
\definecolor{currentstroke}{rgb}{0.000000,0.000000,0.000000}%
\pgfsetstrokecolor{currentstroke}%
\pgfsetdash{}{0pt}%
\pgfpathmoveto{\pgfqpoint{3.116696in}{0.148611in}}%
\pgfpathlineto{\pgfqpoint{3.941164in}{0.148611in}}%
\pgfusepath{stroke}%
\end{pgfscope}%
\begin{pgfscope}%
\pgfsetrectcap%
\pgfsetmiterjoin%
\pgfsetlinewidth{0.803000pt}%
\definecolor{currentstroke}{rgb}{0.000000,0.000000,0.000000}%
\pgfsetstrokecolor{currentstroke}%
\pgfsetdash{}{0pt}%
\pgfpathmoveto{\pgfqpoint{3.116696in}{0.610611in}}%
\pgfpathlineto{\pgfqpoint{3.941164in}{0.610611in}}%
\pgfusepath{stroke}%
\end{pgfscope}%
\begin{pgfscope}%
\definecolor{textcolor}{rgb}{0.000000,0.000000,0.000000}%
\pgfsetstrokecolor{textcolor}%
\pgfsetfillcolor{textcolor}%
\pgftext[x=3.528930in,y=0.693944in,,base]{\color{textcolor}\rmfamily\fontsize{11.000000}{13.200000}\selectfont Malako...}%
\end{pgfscope}%
\begin{pgfscope}%
\pgfsetbuttcap%
\pgfsetmiterjoin%
\definecolor{currentfill}{rgb}{1.000000,1.000000,1.000000}%
\pgfsetfillcolor{currentfill}%
\pgfsetlinewidth{0.000000pt}%
\definecolor{currentstroke}{rgb}{0.000000,0.000000,0.000000}%
\pgfsetstrokecolor{currentstroke}%
\pgfsetstrokeopacity{0.000000}%
\pgfsetdash{}{0pt}%
\pgfpathmoveto{\pgfqpoint{4.106058in}{0.148611in}}%
\pgfpathlineto{\pgfqpoint{4.930526in}{0.148611in}}%
\pgfpathlineto{\pgfqpoint{4.930526in}{0.610611in}}%
\pgfpathlineto{\pgfqpoint{4.106058in}{0.610611in}}%
\pgfpathlineto{\pgfqpoint{4.106058in}{0.148611in}}%
\pgfpathclose%
\pgfusepath{fill}%
\end{pgfscope}%
\begin{pgfscope}%
\pgfpathrectangle{\pgfqpoint{4.106058in}{0.148611in}}{\pgfqpoint{0.824468in}{0.462000in}}%
\pgfusepath{clip}%
\pgfsetbuttcap%
\pgfsetmiterjoin%
\definecolor{currentfill}{rgb}{0.121569,0.466667,0.705882}%
\pgfsetfillcolor{currentfill}%
\pgfsetfillopacity{0.500000}%
\pgfsetlinewidth{1.003750pt}%
\definecolor{currentstroke}{rgb}{0.000000,0.000000,0.000000}%
\pgfsetstrokecolor{currentstroke}%
\pgfsetdash{}{0pt}%
\pgfpathmoveto{\pgfqpoint{4.143534in}{0.148611in}}%
\pgfpathlineto{\pgfqpoint{4.293437in}{0.148611in}}%
\pgfpathlineto{\pgfqpoint{4.293437in}{0.588611in}}%
\pgfpathlineto{\pgfqpoint{4.143534in}{0.588611in}}%
\pgfpathlineto{\pgfqpoint{4.143534in}{0.148611in}}%
\pgfpathclose%
\pgfusepath{stroke,fill}%
\end{pgfscope}%
\begin{pgfscope}%
\pgfpathrectangle{\pgfqpoint{4.106058in}{0.148611in}}{\pgfqpoint{0.824468in}{0.462000in}}%
\pgfusepath{clip}%
\pgfsetbuttcap%
\pgfsetmiterjoin%
\definecolor{currentfill}{rgb}{0.121569,0.466667,0.705882}%
\pgfsetfillcolor{currentfill}%
\pgfsetfillopacity{0.500000}%
\pgfsetlinewidth{1.003750pt}%
\definecolor{currentstroke}{rgb}{0.000000,0.000000,0.000000}%
\pgfsetstrokecolor{currentstroke}%
\pgfsetdash{}{0pt}%
\pgfpathmoveto{\pgfqpoint{4.293437in}{0.148611in}}%
\pgfpathlineto{\pgfqpoint{4.443340in}{0.148611in}}%
\pgfpathlineto{\pgfqpoint{4.443340in}{0.468611in}}%
\pgfpathlineto{\pgfqpoint{4.293437in}{0.468611in}}%
\pgfpathlineto{\pgfqpoint{4.293437in}{0.148611in}}%
\pgfpathclose%
\pgfusepath{stroke,fill}%
\end{pgfscope}%
\begin{pgfscope}%
\pgfpathrectangle{\pgfqpoint{4.106058in}{0.148611in}}{\pgfqpoint{0.824468in}{0.462000in}}%
\pgfusepath{clip}%
\pgfsetbuttcap%
\pgfsetmiterjoin%
\definecolor{currentfill}{rgb}{0.121569,0.466667,0.705882}%
\pgfsetfillcolor{currentfill}%
\pgfsetfillopacity{0.500000}%
\pgfsetlinewidth{1.003750pt}%
\definecolor{currentstroke}{rgb}{0.000000,0.000000,0.000000}%
\pgfsetstrokecolor{currentstroke}%
\pgfsetdash{}{0pt}%
\pgfpathmoveto{\pgfqpoint{4.443340in}{0.148611in}}%
\pgfpathlineto{\pgfqpoint{4.593244in}{0.148611in}}%
\pgfpathlineto{\pgfqpoint{4.593244in}{0.228611in}}%
\pgfpathlineto{\pgfqpoint{4.443340in}{0.228611in}}%
\pgfpathlineto{\pgfqpoint{4.443340in}{0.148611in}}%
\pgfpathclose%
\pgfusepath{stroke,fill}%
\end{pgfscope}%
\begin{pgfscope}%
\pgfpathrectangle{\pgfqpoint{4.106058in}{0.148611in}}{\pgfqpoint{0.824468in}{0.462000in}}%
\pgfusepath{clip}%
\pgfsetbuttcap%
\pgfsetmiterjoin%
\definecolor{currentfill}{rgb}{0.121569,0.466667,0.705882}%
\pgfsetfillcolor{currentfill}%
\pgfsetfillopacity{0.500000}%
\pgfsetlinewidth{1.003750pt}%
\definecolor{currentstroke}{rgb}{0.000000,0.000000,0.000000}%
\pgfsetstrokecolor{currentstroke}%
\pgfsetdash{}{0pt}%
\pgfpathmoveto{\pgfqpoint{4.593244in}{0.148611in}}%
\pgfpathlineto{\pgfqpoint{4.743147in}{0.148611in}}%
\pgfpathlineto{\pgfqpoint{4.743147in}{0.168611in}}%
\pgfpathlineto{\pgfqpoint{4.593244in}{0.168611in}}%
\pgfpathlineto{\pgfqpoint{4.593244in}{0.148611in}}%
\pgfpathclose%
\pgfusepath{stroke,fill}%
\end{pgfscope}%
\begin{pgfscope}%
\pgfpathrectangle{\pgfqpoint{4.106058in}{0.148611in}}{\pgfqpoint{0.824468in}{0.462000in}}%
\pgfusepath{clip}%
\pgfsetbuttcap%
\pgfsetmiterjoin%
\definecolor{currentfill}{rgb}{0.121569,0.466667,0.705882}%
\pgfsetfillcolor{currentfill}%
\pgfsetfillopacity{0.500000}%
\pgfsetlinewidth{1.003750pt}%
\definecolor{currentstroke}{rgb}{0.000000,0.000000,0.000000}%
\pgfsetstrokecolor{currentstroke}%
\pgfsetdash{}{0pt}%
\pgfpathmoveto{\pgfqpoint{4.743147in}{0.148611in}}%
\pgfpathlineto{\pgfqpoint{4.893050in}{0.148611in}}%
\pgfpathlineto{\pgfqpoint{4.893050in}{0.148611in}}%
\pgfpathlineto{\pgfqpoint{4.743147in}{0.148611in}}%
\pgfpathlineto{\pgfqpoint{4.743147in}{0.148611in}}%
\pgfpathclose%
\pgfusepath{stroke,fill}%
\end{pgfscope}%
\begin{pgfscope}%
\pgfsetrectcap%
\pgfsetmiterjoin%
\pgfsetlinewidth{0.803000pt}%
\definecolor{currentstroke}{rgb}{0.000000,0.000000,0.000000}%
\pgfsetstrokecolor{currentstroke}%
\pgfsetdash{}{0pt}%
\pgfpathmoveto{\pgfqpoint{4.106058in}{0.148611in}}%
\pgfpathlineto{\pgfqpoint{4.106058in}{0.610611in}}%
\pgfusepath{stroke}%
\end{pgfscope}%
\begin{pgfscope}%
\pgfsetrectcap%
\pgfsetmiterjoin%
\pgfsetlinewidth{0.803000pt}%
\definecolor{currentstroke}{rgb}{0.000000,0.000000,0.000000}%
\pgfsetstrokecolor{currentstroke}%
\pgfsetdash{}{0pt}%
\pgfpathmoveto{\pgfqpoint{4.930526in}{0.148611in}}%
\pgfpathlineto{\pgfqpoint{4.930526in}{0.610611in}}%
\pgfusepath{stroke}%
\end{pgfscope}%
\begin{pgfscope}%
\pgfsetrectcap%
\pgfsetmiterjoin%
\pgfsetlinewidth{0.803000pt}%
\definecolor{currentstroke}{rgb}{0.000000,0.000000,0.000000}%
\pgfsetstrokecolor{currentstroke}%
\pgfsetdash{}{0pt}%
\pgfpathmoveto{\pgfqpoint{4.106058in}{0.148611in}}%
\pgfpathlineto{\pgfqpoint{4.930526in}{0.148611in}}%
\pgfusepath{stroke}%
\end{pgfscope}%
\begin{pgfscope}%
\pgfsetrectcap%
\pgfsetmiterjoin%
\pgfsetlinewidth{0.803000pt}%
\definecolor{currentstroke}{rgb}{0.000000,0.000000,0.000000}%
\pgfsetstrokecolor{currentstroke}%
\pgfsetdash{}{0pt}%
\pgfpathmoveto{\pgfqpoint{4.106058in}{0.610611in}}%
\pgfpathlineto{\pgfqpoint{4.930526in}{0.610611in}}%
\pgfusepath{stroke}%
\end{pgfscope}%
\begin{pgfscope}%
\definecolor{textcolor}{rgb}{0.000000,0.000000,0.000000}%
\pgfsetstrokecolor{textcolor}%
\pgfsetfillcolor{textcolor}%
\pgftext[x=4.518292in,y=0.693944in,,base]{\color{textcolor}\rmfamily\fontsize{11.000000}{13.200000}\selectfont Euro-A...}%
\end{pgfscope}%
\begin{pgfscope}%
\pgfsetbuttcap%
\pgfsetmiterjoin%
\definecolor{currentfill}{rgb}{1.000000,1.000000,1.000000}%
\pgfsetfillcolor{currentfill}%
\pgfsetlinewidth{0.000000pt}%
\definecolor{currentstroke}{rgb}{0.000000,0.000000,0.000000}%
\pgfsetstrokecolor{currentstroke}%
\pgfsetstrokeopacity{0.000000}%
\pgfsetdash{}{0pt}%
\pgfpathmoveto{\pgfqpoint{5.095420in}{0.148611in}}%
\pgfpathlineto{\pgfqpoint{5.919888in}{0.148611in}}%
\pgfpathlineto{\pgfqpoint{5.919888in}{0.610611in}}%
\pgfpathlineto{\pgfqpoint{5.095420in}{0.610611in}}%
\pgfpathlineto{\pgfqpoint{5.095420in}{0.148611in}}%
\pgfpathclose%
\pgfusepath{fill}%
\end{pgfscope}%
\begin{pgfscope}%
\pgfpathrectangle{\pgfqpoint{5.095420in}{0.148611in}}{\pgfqpoint{0.824468in}{0.462000in}}%
\pgfusepath{clip}%
\pgfsetbuttcap%
\pgfsetmiterjoin%
\definecolor{currentfill}{rgb}{0.121569,0.466667,0.705882}%
\pgfsetfillcolor{currentfill}%
\pgfsetfillopacity{0.500000}%
\pgfsetlinewidth{1.003750pt}%
\definecolor{currentstroke}{rgb}{0.000000,0.000000,0.000000}%
\pgfsetstrokecolor{currentstroke}%
\pgfsetdash{}{0pt}%
\pgfpathmoveto{\pgfqpoint{5.132895in}{0.148611in}}%
\pgfpathlineto{\pgfqpoint{5.282799in}{0.148611in}}%
\pgfpathlineto{\pgfqpoint{5.282799in}{0.250150in}}%
\pgfpathlineto{\pgfqpoint{5.132895in}{0.250150in}}%
\pgfpathlineto{\pgfqpoint{5.132895in}{0.148611in}}%
\pgfpathclose%
\pgfusepath{stroke,fill}%
\end{pgfscope}%
\begin{pgfscope}%
\pgfpathrectangle{\pgfqpoint{5.095420in}{0.148611in}}{\pgfqpoint{0.824468in}{0.462000in}}%
\pgfusepath{clip}%
\pgfsetbuttcap%
\pgfsetmiterjoin%
\definecolor{currentfill}{rgb}{0.121569,0.466667,0.705882}%
\pgfsetfillcolor{currentfill}%
\pgfsetfillopacity{0.500000}%
\pgfsetlinewidth{1.003750pt}%
\definecolor{currentstroke}{rgb}{0.000000,0.000000,0.000000}%
\pgfsetstrokecolor{currentstroke}%
\pgfsetdash{}{0pt}%
\pgfpathmoveto{\pgfqpoint{5.282799in}{0.148611in}}%
\pgfpathlineto{\pgfqpoint{5.432702in}{0.148611in}}%
\pgfpathlineto{\pgfqpoint{5.432702in}{0.385534in}}%
\pgfpathlineto{\pgfqpoint{5.282799in}{0.385534in}}%
\pgfpathlineto{\pgfqpoint{5.282799in}{0.148611in}}%
\pgfpathclose%
\pgfusepath{stroke,fill}%
\end{pgfscope}%
\begin{pgfscope}%
\pgfpathrectangle{\pgfqpoint{5.095420in}{0.148611in}}{\pgfqpoint{0.824468in}{0.462000in}}%
\pgfusepath{clip}%
\pgfsetbuttcap%
\pgfsetmiterjoin%
\definecolor{currentfill}{rgb}{0.121569,0.466667,0.705882}%
\pgfsetfillcolor{currentfill}%
\pgfsetfillopacity{0.500000}%
\pgfsetlinewidth{1.003750pt}%
\definecolor{currentstroke}{rgb}{0.000000,0.000000,0.000000}%
\pgfsetstrokecolor{currentstroke}%
\pgfsetdash{}{0pt}%
\pgfpathmoveto{\pgfqpoint{5.432702in}{0.148611in}}%
\pgfpathlineto{\pgfqpoint{5.582605in}{0.148611in}}%
\pgfpathlineto{\pgfqpoint{5.582605in}{0.148611in}}%
\pgfpathlineto{\pgfqpoint{5.432702in}{0.148611in}}%
\pgfpathlineto{\pgfqpoint{5.432702in}{0.148611in}}%
\pgfpathclose%
\pgfusepath{stroke,fill}%
\end{pgfscope}%
\begin{pgfscope}%
\pgfpathrectangle{\pgfqpoint{5.095420in}{0.148611in}}{\pgfqpoint{0.824468in}{0.462000in}}%
\pgfusepath{clip}%
\pgfsetbuttcap%
\pgfsetmiterjoin%
\definecolor{currentfill}{rgb}{0.121569,0.466667,0.705882}%
\pgfsetfillcolor{currentfill}%
\pgfsetfillopacity{0.500000}%
\pgfsetlinewidth{1.003750pt}%
\definecolor{currentstroke}{rgb}{0.000000,0.000000,0.000000}%
\pgfsetstrokecolor{currentstroke}%
\pgfsetdash{}{0pt}%
\pgfpathmoveto{\pgfqpoint{5.582605in}{0.148611in}}%
\pgfpathlineto{\pgfqpoint{5.732509in}{0.148611in}}%
\pgfpathlineto{\pgfqpoint{5.732509in}{0.317842in}}%
\pgfpathlineto{\pgfqpoint{5.582605in}{0.317842in}}%
\pgfpathlineto{\pgfqpoint{5.582605in}{0.148611in}}%
\pgfpathclose%
\pgfusepath{stroke,fill}%
\end{pgfscope}%
\begin{pgfscope}%
\pgfpathrectangle{\pgfqpoint{5.095420in}{0.148611in}}{\pgfqpoint{0.824468in}{0.462000in}}%
\pgfusepath{clip}%
\pgfsetbuttcap%
\pgfsetmiterjoin%
\definecolor{currentfill}{rgb}{0.121569,0.466667,0.705882}%
\pgfsetfillcolor{currentfill}%
\pgfsetfillopacity{0.500000}%
\pgfsetlinewidth{1.003750pt}%
\definecolor{currentstroke}{rgb}{0.000000,0.000000,0.000000}%
\pgfsetstrokecolor{currentstroke}%
\pgfsetdash{}{0pt}%
\pgfpathmoveto{\pgfqpoint{5.732509in}{0.148611in}}%
\pgfpathlineto{\pgfqpoint{5.882412in}{0.148611in}}%
\pgfpathlineto{\pgfqpoint{5.882412in}{0.588611in}}%
\pgfpathlineto{\pgfqpoint{5.732509in}{0.588611in}}%
\pgfpathlineto{\pgfqpoint{5.732509in}{0.148611in}}%
\pgfpathclose%
\pgfusepath{stroke,fill}%
\end{pgfscope}%
\begin{pgfscope}%
\pgfsetrectcap%
\pgfsetmiterjoin%
\pgfsetlinewidth{0.803000pt}%
\definecolor{currentstroke}{rgb}{0.000000,0.000000,0.000000}%
\pgfsetstrokecolor{currentstroke}%
\pgfsetdash{}{0pt}%
\pgfpathmoveto{\pgfqpoint{5.095420in}{0.148611in}}%
\pgfpathlineto{\pgfqpoint{5.095420in}{0.610611in}}%
\pgfusepath{stroke}%
\end{pgfscope}%
\begin{pgfscope}%
\pgfsetrectcap%
\pgfsetmiterjoin%
\pgfsetlinewidth{0.803000pt}%
\definecolor{currentstroke}{rgb}{0.000000,0.000000,0.000000}%
\pgfsetstrokecolor{currentstroke}%
\pgfsetdash{}{0pt}%
\pgfpathmoveto{\pgfqpoint{5.919888in}{0.148611in}}%
\pgfpathlineto{\pgfqpoint{5.919888in}{0.610611in}}%
\pgfusepath{stroke}%
\end{pgfscope}%
\begin{pgfscope}%
\pgfsetrectcap%
\pgfsetmiterjoin%
\pgfsetlinewidth{0.803000pt}%
\definecolor{currentstroke}{rgb}{0.000000,0.000000,0.000000}%
\pgfsetstrokecolor{currentstroke}%
\pgfsetdash{}{0pt}%
\pgfpathmoveto{\pgfqpoint{5.095420in}{0.148611in}}%
\pgfpathlineto{\pgfqpoint{5.919888in}{0.148611in}}%
\pgfusepath{stroke}%
\end{pgfscope}%
\begin{pgfscope}%
\pgfsetrectcap%
\pgfsetmiterjoin%
\pgfsetlinewidth{0.803000pt}%
\definecolor{currentstroke}{rgb}{0.000000,0.000000,0.000000}%
\pgfsetstrokecolor{currentstroke}%
\pgfsetdash{}{0pt}%
\pgfpathmoveto{\pgfqpoint{5.095420in}{0.610611in}}%
\pgfpathlineto{\pgfqpoint{5.919888in}{0.610611in}}%
\pgfusepath{stroke}%
\end{pgfscope}%
\begin{pgfscope}%
\definecolor{textcolor}{rgb}{0.000000,0.000000,0.000000}%
\pgfsetstrokecolor{textcolor}%
\pgfsetfillcolor{textcolor}%
\pgftext[x=5.507654in,y=0.693944in,,base]{\color{textcolor}\rmfamily\fontsize{11.000000}{13.200000}\selectfont Peyrac...}%
\end{pgfscope}%
\begin{pgfscope}%
\pgfsetbuttcap%
\pgfsetmiterjoin%
\definecolor{currentfill}{rgb}{1.000000,1.000000,1.000000}%
\pgfsetfillcolor{currentfill}%
\pgfsetlinewidth{0.000000pt}%
\definecolor{currentstroke}{rgb}{0.000000,0.000000,0.000000}%
\pgfsetstrokecolor{currentstroke}%
\pgfsetstrokeopacity{0.000000}%
\pgfsetdash{}{0pt}%
\pgfpathmoveto{\pgfqpoint{6.084781in}{0.148611in}}%
\pgfpathlineto{\pgfqpoint{6.909249in}{0.148611in}}%
\pgfpathlineto{\pgfqpoint{6.909249in}{0.610611in}}%
\pgfpathlineto{\pgfqpoint{6.084781in}{0.610611in}}%
\pgfpathlineto{\pgfqpoint{6.084781in}{0.148611in}}%
\pgfpathclose%
\pgfusepath{fill}%
\end{pgfscope}%
\begin{pgfscope}%
\pgfpathrectangle{\pgfqpoint{6.084781in}{0.148611in}}{\pgfqpoint{0.824468in}{0.462000in}}%
\pgfusepath{clip}%
\pgfsetbuttcap%
\pgfsetmiterjoin%
\definecolor{currentfill}{rgb}{0.121569,0.466667,0.705882}%
\pgfsetfillcolor{currentfill}%
\pgfsetfillopacity{0.500000}%
\pgfsetlinewidth{1.003750pt}%
\definecolor{currentstroke}{rgb}{0.000000,0.000000,0.000000}%
\pgfsetstrokecolor{currentstroke}%
\pgfsetdash{}{0pt}%
\pgfpathmoveto{\pgfqpoint{6.122257in}{0.148611in}}%
\pgfpathlineto{\pgfqpoint{6.272160in}{0.148611in}}%
\pgfpathlineto{\pgfqpoint{6.272160in}{0.588611in}}%
\pgfpathlineto{\pgfqpoint{6.122257in}{0.588611in}}%
\pgfpathlineto{\pgfqpoint{6.122257in}{0.148611in}}%
\pgfpathclose%
\pgfusepath{stroke,fill}%
\end{pgfscope}%
\begin{pgfscope}%
\pgfpathrectangle{\pgfqpoint{6.084781in}{0.148611in}}{\pgfqpoint{0.824468in}{0.462000in}}%
\pgfusepath{clip}%
\pgfsetbuttcap%
\pgfsetmiterjoin%
\definecolor{currentfill}{rgb}{0.121569,0.466667,0.705882}%
\pgfsetfillcolor{currentfill}%
\pgfsetfillopacity{0.500000}%
\pgfsetlinewidth{1.003750pt}%
\definecolor{currentstroke}{rgb}{0.000000,0.000000,0.000000}%
\pgfsetstrokecolor{currentstroke}%
\pgfsetdash{}{0pt}%
\pgfpathmoveto{\pgfqpoint{6.272160in}{0.148611in}}%
\pgfpathlineto{\pgfqpoint{6.422064in}{0.148611in}}%
\pgfpathlineto{\pgfqpoint{6.422064in}{0.148611in}}%
\pgfpathlineto{\pgfqpoint{6.272160in}{0.148611in}}%
\pgfpathlineto{\pgfqpoint{6.272160in}{0.148611in}}%
\pgfpathclose%
\pgfusepath{stroke,fill}%
\end{pgfscope}%
\begin{pgfscope}%
\pgfpathrectangle{\pgfqpoint{6.084781in}{0.148611in}}{\pgfqpoint{0.824468in}{0.462000in}}%
\pgfusepath{clip}%
\pgfsetbuttcap%
\pgfsetmiterjoin%
\definecolor{currentfill}{rgb}{0.121569,0.466667,0.705882}%
\pgfsetfillcolor{currentfill}%
\pgfsetfillopacity{0.500000}%
\pgfsetlinewidth{1.003750pt}%
\definecolor{currentstroke}{rgb}{0.000000,0.000000,0.000000}%
\pgfsetstrokecolor{currentstroke}%
\pgfsetdash{}{0pt}%
\pgfpathmoveto{\pgfqpoint{6.422064in}{0.148611in}}%
\pgfpathlineto{\pgfqpoint{6.571967in}{0.148611in}}%
\pgfpathlineto{\pgfqpoint{6.571967in}{0.148611in}}%
\pgfpathlineto{\pgfqpoint{6.422064in}{0.148611in}}%
\pgfpathlineto{\pgfqpoint{6.422064in}{0.148611in}}%
\pgfpathclose%
\pgfusepath{stroke,fill}%
\end{pgfscope}%
\begin{pgfscope}%
\pgfpathrectangle{\pgfqpoint{6.084781in}{0.148611in}}{\pgfqpoint{0.824468in}{0.462000in}}%
\pgfusepath{clip}%
\pgfsetbuttcap%
\pgfsetmiterjoin%
\definecolor{currentfill}{rgb}{0.121569,0.466667,0.705882}%
\pgfsetfillcolor{currentfill}%
\pgfsetfillopacity{0.500000}%
\pgfsetlinewidth{1.003750pt}%
\definecolor{currentstroke}{rgb}{0.000000,0.000000,0.000000}%
\pgfsetstrokecolor{currentstroke}%
\pgfsetdash{}{0pt}%
\pgfpathmoveto{\pgfqpoint{6.571967in}{0.148611in}}%
\pgfpathlineto{\pgfqpoint{6.721870in}{0.148611in}}%
\pgfpathlineto{\pgfqpoint{6.721870in}{0.148611in}}%
\pgfpathlineto{\pgfqpoint{6.571967in}{0.148611in}}%
\pgfpathlineto{\pgfqpoint{6.571967in}{0.148611in}}%
\pgfpathclose%
\pgfusepath{stroke,fill}%
\end{pgfscope}%
\begin{pgfscope}%
\pgfpathrectangle{\pgfqpoint{6.084781in}{0.148611in}}{\pgfqpoint{0.824468in}{0.462000in}}%
\pgfusepath{clip}%
\pgfsetbuttcap%
\pgfsetmiterjoin%
\definecolor{currentfill}{rgb}{0.121569,0.466667,0.705882}%
\pgfsetfillcolor{currentfill}%
\pgfsetfillopacity{0.500000}%
\pgfsetlinewidth{1.003750pt}%
\definecolor{currentstroke}{rgb}{0.000000,0.000000,0.000000}%
\pgfsetstrokecolor{currentstroke}%
\pgfsetdash{}{0pt}%
\pgfpathmoveto{\pgfqpoint{6.721870in}{0.148611in}}%
\pgfpathlineto{\pgfqpoint{6.871774in}{0.148611in}}%
\pgfpathlineto{\pgfqpoint{6.871774in}{0.236611in}}%
\pgfpathlineto{\pgfqpoint{6.721870in}{0.236611in}}%
\pgfpathlineto{\pgfqpoint{6.721870in}{0.148611in}}%
\pgfpathclose%
\pgfusepath{stroke,fill}%
\end{pgfscope}%
\begin{pgfscope}%
\pgfsetrectcap%
\pgfsetmiterjoin%
\pgfsetlinewidth{0.803000pt}%
\definecolor{currentstroke}{rgb}{0.000000,0.000000,0.000000}%
\pgfsetstrokecolor{currentstroke}%
\pgfsetdash{}{0pt}%
\pgfpathmoveto{\pgfqpoint{6.084781in}{0.148611in}}%
\pgfpathlineto{\pgfqpoint{6.084781in}{0.610611in}}%
\pgfusepath{stroke}%
\end{pgfscope}%
\begin{pgfscope}%
\pgfsetrectcap%
\pgfsetmiterjoin%
\pgfsetlinewidth{0.803000pt}%
\definecolor{currentstroke}{rgb}{0.000000,0.000000,0.000000}%
\pgfsetstrokecolor{currentstroke}%
\pgfsetdash{}{0pt}%
\pgfpathmoveto{\pgfqpoint{6.909249in}{0.148611in}}%
\pgfpathlineto{\pgfqpoint{6.909249in}{0.610611in}}%
\pgfusepath{stroke}%
\end{pgfscope}%
\begin{pgfscope}%
\pgfsetrectcap%
\pgfsetmiterjoin%
\pgfsetlinewidth{0.803000pt}%
\definecolor{currentstroke}{rgb}{0.000000,0.000000,0.000000}%
\pgfsetstrokecolor{currentstroke}%
\pgfsetdash{}{0pt}%
\pgfpathmoveto{\pgfqpoint{6.084781in}{0.148611in}}%
\pgfpathlineto{\pgfqpoint{6.909249in}{0.148611in}}%
\pgfusepath{stroke}%
\end{pgfscope}%
\begin{pgfscope}%
\pgfsetrectcap%
\pgfsetmiterjoin%
\pgfsetlinewidth{0.803000pt}%
\definecolor{currentstroke}{rgb}{0.000000,0.000000,0.000000}%
\pgfsetstrokecolor{currentstroke}%
\pgfsetdash{}{0pt}%
\pgfpathmoveto{\pgfqpoint{6.084781in}{0.610611in}}%
\pgfpathlineto{\pgfqpoint{6.909249in}{0.610611in}}%
\pgfusepath{stroke}%
\end{pgfscope}%
\begin{pgfscope}%
\definecolor{textcolor}{rgb}{0.000000,0.000000,0.000000}%
\pgfsetstrokecolor{textcolor}%
\pgfsetfillcolor{textcolor}%
\pgftext[x=6.497015in,y=0.693944in,,base]{\color{textcolor}\rmfamily\fontsize{11.000000}{13.200000}\selectfont Sma}%
\end{pgfscope}%
\begin{pgfscope}%
\pgfsetbuttcap%
\pgfsetmiterjoin%
\definecolor{currentfill}{rgb}{1.000000,1.000000,1.000000}%
\pgfsetfillcolor{currentfill}%
\pgfsetlinewidth{0.000000pt}%
\definecolor{currentstroke}{rgb}{0.000000,0.000000,0.000000}%
\pgfsetstrokecolor{currentstroke}%
\pgfsetstrokeopacity{0.000000}%
\pgfsetdash{}{0pt}%
\pgfpathmoveto{\pgfqpoint{7.074143in}{0.148611in}}%
\pgfpathlineto{\pgfqpoint{7.898611in}{0.148611in}}%
\pgfpathlineto{\pgfqpoint{7.898611in}{0.610611in}}%
\pgfpathlineto{\pgfqpoint{7.074143in}{0.610611in}}%
\pgfpathlineto{\pgfqpoint{7.074143in}{0.148611in}}%
\pgfpathclose%
\pgfusepath{fill}%
\end{pgfscope}%
\begin{pgfscope}%
\pgfpathrectangle{\pgfqpoint{7.074143in}{0.148611in}}{\pgfqpoint{0.824468in}{0.462000in}}%
\pgfusepath{clip}%
\pgfsetbuttcap%
\pgfsetmiterjoin%
\definecolor{currentfill}{rgb}{0.121569,0.466667,0.705882}%
\pgfsetfillcolor{currentfill}%
\pgfsetfillopacity{0.500000}%
\pgfsetlinewidth{1.003750pt}%
\definecolor{currentstroke}{rgb}{0.000000,0.000000,0.000000}%
\pgfsetstrokecolor{currentstroke}%
\pgfsetdash{}{0pt}%
\pgfpathmoveto{\pgfqpoint{7.111619in}{0.148611in}}%
\pgfpathlineto{\pgfqpoint{7.261522in}{0.148611in}}%
\pgfpathlineto{\pgfqpoint{7.261522in}{0.588611in}}%
\pgfpathlineto{\pgfqpoint{7.111619in}{0.588611in}}%
\pgfpathlineto{\pgfqpoint{7.111619in}{0.148611in}}%
\pgfpathclose%
\pgfusepath{stroke,fill}%
\end{pgfscope}%
\begin{pgfscope}%
\pgfpathrectangle{\pgfqpoint{7.074143in}{0.148611in}}{\pgfqpoint{0.824468in}{0.462000in}}%
\pgfusepath{clip}%
\pgfsetbuttcap%
\pgfsetmiterjoin%
\definecolor{currentfill}{rgb}{0.121569,0.466667,0.705882}%
\pgfsetfillcolor{currentfill}%
\pgfsetfillopacity{0.500000}%
\pgfsetlinewidth{1.003750pt}%
\definecolor{currentstroke}{rgb}{0.000000,0.000000,0.000000}%
\pgfsetstrokecolor{currentstroke}%
\pgfsetdash{}{0pt}%
\pgfpathmoveto{\pgfqpoint{7.261522in}{0.148611in}}%
\pgfpathlineto{\pgfqpoint{7.411425in}{0.148611in}}%
\pgfpathlineto{\pgfqpoint{7.411425in}{0.148611in}}%
\pgfpathlineto{\pgfqpoint{7.261522in}{0.148611in}}%
\pgfpathlineto{\pgfqpoint{7.261522in}{0.148611in}}%
\pgfpathclose%
\pgfusepath{stroke,fill}%
\end{pgfscope}%
\begin{pgfscope}%
\pgfpathrectangle{\pgfqpoint{7.074143in}{0.148611in}}{\pgfqpoint{0.824468in}{0.462000in}}%
\pgfusepath{clip}%
\pgfsetbuttcap%
\pgfsetmiterjoin%
\definecolor{currentfill}{rgb}{0.121569,0.466667,0.705882}%
\pgfsetfillcolor{currentfill}%
\pgfsetfillopacity{0.500000}%
\pgfsetlinewidth{1.003750pt}%
\definecolor{currentstroke}{rgb}{0.000000,0.000000,0.000000}%
\pgfsetstrokecolor{currentstroke}%
\pgfsetdash{}{0pt}%
\pgfpathmoveto{\pgfqpoint{7.411425in}{0.148611in}}%
\pgfpathlineto{\pgfqpoint{7.561329in}{0.148611in}}%
\pgfpathlineto{\pgfqpoint{7.561329in}{0.148611in}}%
\pgfpathlineto{\pgfqpoint{7.411425in}{0.148611in}}%
\pgfpathlineto{\pgfqpoint{7.411425in}{0.148611in}}%
\pgfpathclose%
\pgfusepath{stroke,fill}%
\end{pgfscope}%
\begin{pgfscope}%
\pgfpathrectangle{\pgfqpoint{7.074143in}{0.148611in}}{\pgfqpoint{0.824468in}{0.462000in}}%
\pgfusepath{clip}%
\pgfsetbuttcap%
\pgfsetmiterjoin%
\definecolor{currentfill}{rgb}{0.121569,0.466667,0.705882}%
\pgfsetfillcolor{currentfill}%
\pgfsetfillopacity{0.500000}%
\pgfsetlinewidth{1.003750pt}%
\definecolor{currentstroke}{rgb}{0.000000,0.000000,0.000000}%
\pgfsetstrokecolor{currentstroke}%
\pgfsetdash{}{0pt}%
\pgfpathmoveto{\pgfqpoint{7.561329in}{0.148611in}}%
\pgfpathlineto{\pgfqpoint{7.711232in}{0.148611in}}%
\pgfpathlineto{\pgfqpoint{7.711232in}{0.148611in}}%
\pgfpathlineto{\pgfqpoint{7.561329in}{0.148611in}}%
\pgfpathlineto{\pgfqpoint{7.561329in}{0.148611in}}%
\pgfpathclose%
\pgfusepath{stroke,fill}%
\end{pgfscope}%
\begin{pgfscope}%
\pgfpathrectangle{\pgfqpoint{7.074143in}{0.148611in}}{\pgfqpoint{0.824468in}{0.462000in}}%
\pgfusepath{clip}%
\pgfsetbuttcap%
\pgfsetmiterjoin%
\definecolor{currentfill}{rgb}{0.121569,0.466667,0.705882}%
\pgfsetfillcolor{currentfill}%
\pgfsetfillopacity{0.500000}%
\pgfsetlinewidth{1.003750pt}%
\definecolor{currentstroke}{rgb}{0.000000,0.000000,0.000000}%
\pgfsetstrokecolor{currentstroke}%
\pgfsetdash{}{0pt}%
\pgfpathmoveto{\pgfqpoint{7.711232in}{0.148611in}}%
\pgfpathlineto{\pgfqpoint{7.861135in}{0.148611in}}%
\pgfpathlineto{\pgfqpoint{7.861135in}{0.148611in}}%
\pgfpathlineto{\pgfqpoint{7.711232in}{0.148611in}}%
\pgfpathlineto{\pgfqpoint{7.711232in}{0.148611in}}%
\pgfpathclose%
\pgfusepath{stroke,fill}%
\end{pgfscope}%
\begin{pgfscope}%
\pgfsetrectcap%
\pgfsetmiterjoin%
\pgfsetlinewidth{0.803000pt}%
\definecolor{currentstroke}{rgb}{0.000000,0.000000,0.000000}%
\pgfsetstrokecolor{currentstroke}%
\pgfsetdash{}{0pt}%
\pgfpathmoveto{\pgfqpoint{7.074143in}{0.148611in}}%
\pgfpathlineto{\pgfqpoint{7.074143in}{0.610611in}}%
\pgfusepath{stroke}%
\end{pgfscope}%
\begin{pgfscope}%
\pgfsetrectcap%
\pgfsetmiterjoin%
\pgfsetlinewidth{0.803000pt}%
\definecolor{currentstroke}{rgb}{0.000000,0.000000,0.000000}%
\pgfsetstrokecolor{currentstroke}%
\pgfsetdash{}{0pt}%
\pgfpathmoveto{\pgfqpoint{7.898611in}{0.148611in}}%
\pgfpathlineto{\pgfqpoint{7.898611in}{0.610611in}}%
\pgfusepath{stroke}%
\end{pgfscope}%
\begin{pgfscope}%
\pgfsetrectcap%
\pgfsetmiterjoin%
\pgfsetlinewidth{0.803000pt}%
\definecolor{currentstroke}{rgb}{0.000000,0.000000,0.000000}%
\pgfsetstrokecolor{currentstroke}%
\pgfsetdash{}{0pt}%
\pgfpathmoveto{\pgfqpoint{7.074143in}{0.148611in}}%
\pgfpathlineto{\pgfqpoint{7.898611in}{0.148611in}}%
\pgfusepath{stroke}%
\end{pgfscope}%
\begin{pgfscope}%
\pgfsetrectcap%
\pgfsetmiterjoin%
\pgfsetlinewidth{0.803000pt}%
\definecolor{currentstroke}{rgb}{0.000000,0.000000,0.000000}%
\pgfsetstrokecolor{currentstroke}%
\pgfsetdash{}{0pt}%
\pgfpathmoveto{\pgfqpoint{7.074143in}{0.610611in}}%
\pgfpathlineto{\pgfqpoint{7.898611in}{0.610611in}}%
\pgfusepath{stroke}%
\end{pgfscope}%
\begin{pgfscope}%
\definecolor{textcolor}{rgb}{0.000000,0.000000,0.000000}%
\pgfsetstrokecolor{textcolor}%
\pgfsetfillcolor{textcolor}%
\pgftext[x=7.486377in,y=0.693944in,,base]{\color{textcolor}\rmfamily\fontsize{11.000000}{13.200000}\selectfont Hiscox}%
\end{pgfscope}%
\end{pgfpicture}%
\makeatother%
\endgroup%

    \caption{Stars distribution per assureur}
    \label{fig:distrib_split_noscale}
\end{figure}


\begin{figure}[H]
    \advance\leftskip-3cm
    %% Creator: Matplotlib, PGF backend
%%
%% To include the figure in your LaTeX document, write
%%   \input{<filename>.pgf}
%%
%% Make sure the required packages are loaded in your preamble
%%   \usepackage{pgf}
%%
%% Also ensure that all the required font packages are loaded; for instance,
%% the lmodern package is sometimes necessary when using math font.
%%   \usepackage{lmodern}
%%
%% Figures using additional raster images can only be included by \input if
%% they are in the same directory as the main LaTeX file. For loading figures
%% from other directories you can use the `import` package
%%   \usepackage{import}
%%
%% and then include the figures with
%%   \import{<path to file>}{<filename>.pgf}
%%
%% Matplotlib used the following preamble
%%
\begingroup%
\makeatletter%
\begin{pgfpicture}%
\pgfpathrectangle{\pgfpointorigin}{\pgfqpoint{7.998611in}{5.057558in}}%
\pgfusepath{use as bounding box, clip}%
\begin{pgfscope}%
\pgfsetbuttcap%
\pgfsetmiterjoin%
\definecolor{currentfill}{rgb}{1.000000,1.000000,1.000000}%
\pgfsetfillcolor{currentfill}%
\pgfsetlinewidth{0.000000pt}%
\definecolor{currentstroke}{rgb}{1.000000,1.000000,1.000000}%
\pgfsetstrokecolor{currentstroke}%
\pgfsetdash{}{0pt}%
\pgfpathmoveto{\pgfqpoint{0.000000in}{0.000000in}}%
\pgfpathlineto{\pgfqpoint{7.998611in}{0.000000in}}%
\pgfpathlineto{\pgfqpoint{7.998611in}{5.057558in}}%
\pgfpathlineto{\pgfqpoint{0.000000in}{5.057558in}}%
\pgfpathlineto{\pgfqpoint{0.000000in}{0.000000in}}%
\pgfpathclose%
\pgfusepath{fill}%
\end{pgfscope}%
\begin{pgfscope}%
\pgfsetbuttcap%
\pgfsetmiterjoin%
\definecolor{currentfill}{rgb}{1.000000,1.000000,1.000000}%
\pgfsetfillcolor{currentfill}%
\pgfsetlinewidth{0.000000pt}%
\definecolor{currentstroke}{rgb}{0.000000,0.000000,0.000000}%
\pgfsetstrokecolor{currentstroke}%
\pgfsetstrokeopacity{0.000000}%
\pgfsetdash{}{0pt}%
\pgfpathmoveto{\pgfqpoint{0.148611in}{4.525453in}}%
\pgfpathlineto{\pgfqpoint{0.973079in}{4.525453in}}%
\pgfpathlineto{\pgfqpoint{0.973079in}{4.768611in}}%
\pgfpathlineto{\pgfqpoint{0.148611in}{4.768611in}}%
\pgfpathlineto{\pgfqpoint{0.148611in}{4.525453in}}%
\pgfpathclose%
\pgfusepath{fill}%
\end{pgfscope}%
\begin{pgfscope}%
\pgfpathrectangle{\pgfqpoint{0.148611in}{4.525453in}}{\pgfqpoint{0.824468in}{0.243158in}}%
\pgfusepath{clip}%
\pgfsetbuttcap%
\pgfsetmiterjoin%
\definecolor{currentfill}{rgb}{0.121569,0.466667,0.705882}%
\pgfsetfillcolor{currentfill}%
\pgfsetfillopacity{0.500000}%
\pgfsetlinewidth{1.003750pt}%
\definecolor{currentstroke}{rgb}{0.000000,0.000000,0.000000}%
\pgfsetstrokecolor{currentstroke}%
\pgfsetdash{}{0pt}%
\pgfpathmoveto{\pgfqpoint{0.186087in}{4.525453in}}%
\pgfpathlineto{\pgfqpoint{0.335990in}{4.525453in}}%
\pgfpathlineto{\pgfqpoint{0.335990in}{4.642047in}}%
\pgfpathlineto{\pgfqpoint{0.186087in}{4.642047in}}%
\pgfpathlineto{\pgfqpoint{0.186087in}{4.525453in}}%
\pgfpathclose%
\pgfusepath{stroke,fill}%
\end{pgfscope}%
\begin{pgfscope}%
\pgfpathrectangle{\pgfqpoint{0.148611in}{4.525453in}}{\pgfqpoint{0.824468in}{0.243158in}}%
\pgfusepath{clip}%
\pgfsetbuttcap%
\pgfsetmiterjoin%
\definecolor{currentfill}{rgb}{0.121569,0.466667,0.705882}%
\pgfsetfillcolor{currentfill}%
\pgfsetfillopacity{0.500000}%
\pgfsetlinewidth{1.003750pt}%
\definecolor{currentstroke}{rgb}{0.000000,0.000000,0.000000}%
\pgfsetstrokecolor{currentstroke}%
\pgfsetdash{}{0pt}%
\pgfpathmoveto{\pgfqpoint{0.335990in}{4.525453in}}%
\pgfpathlineto{\pgfqpoint{0.485894in}{4.525453in}}%
\pgfpathlineto{\pgfqpoint{0.485894in}{4.618704in}}%
\pgfpathlineto{\pgfqpoint{0.335990in}{4.618704in}}%
\pgfpathlineto{\pgfqpoint{0.335990in}{4.525453in}}%
\pgfpathclose%
\pgfusepath{stroke,fill}%
\end{pgfscope}%
\begin{pgfscope}%
\pgfpathrectangle{\pgfqpoint{0.148611in}{4.525453in}}{\pgfqpoint{0.824468in}{0.243158in}}%
\pgfusepath{clip}%
\pgfsetbuttcap%
\pgfsetmiterjoin%
\definecolor{currentfill}{rgb}{0.121569,0.466667,0.705882}%
\pgfsetfillcolor{currentfill}%
\pgfsetfillopacity{0.500000}%
\pgfsetlinewidth{1.003750pt}%
\definecolor{currentstroke}{rgb}{0.000000,0.000000,0.000000}%
\pgfsetstrokecolor{currentstroke}%
\pgfsetdash{}{0pt}%
\pgfpathmoveto{\pgfqpoint{0.485894in}{4.525453in}}%
\pgfpathlineto{\pgfqpoint{0.635797in}{4.525453in}}%
\pgfpathlineto{\pgfqpoint{0.635797in}{4.660527in}}%
\pgfpathlineto{\pgfqpoint{0.485894in}{4.660527in}}%
\pgfpathlineto{\pgfqpoint{0.485894in}{4.525453in}}%
\pgfpathclose%
\pgfusepath{stroke,fill}%
\end{pgfscope}%
\begin{pgfscope}%
\pgfpathrectangle{\pgfqpoint{0.148611in}{4.525453in}}{\pgfqpoint{0.824468in}{0.243158in}}%
\pgfusepath{clip}%
\pgfsetbuttcap%
\pgfsetmiterjoin%
\definecolor{currentfill}{rgb}{0.121569,0.466667,0.705882}%
\pgfsetfillcolor{currentfill}%
\pgfsetfillopacity{0.500000}%
\pgfsetlinewidth{1.003750pt}%
\definecolor{currentstroke}{rgb}{0.000000,0.000000,0.000000}%
\pgfsetstrokecolor{currentstroke}%
\pgfsetdash{}{0pt}%
\pgfpathmoveto{\pgfqpoint{0.635797in}{4.525453in}}%
\pgfpathlineto{\pgfqpoint{0.785700in}{4.525453in}}%
\pgfpathlineto{\pgfqpoint{0.785700in}{4.717183in}}%
\pgfpathlineto{\pgfqpoint{0.635797in}{4.717183in}}%
\pgfpathlineto{\pgfqpoint{0.635797in}{4.525453in}}%
\pgfpathclose%
\pgfusepath{stroke,fill}%
\end{pgfscope}%
\begin{pgfscope}%
\pgfpathrectangle{\pgfqpoint{0.148611in}{4.525453in}}{\pgfqpoint{0.824468in}{0.243158in}}%
\pgfusepath{clip}%
\pgfsetbuttcap%
\pgfsetmiterjoin%
\definecolor{currentfill}{rgb}{0.121569,0.466667,0.705882}%
\pgfsetfillcolor{currentfill}%
\pgfsetfillopacity{0.500000}%
\pgfsetlinewidth{1.003750pt}%
\definecolor{currentstroke}{rgb}{0.000000,0.000000,0.000000}%
\pgfsetstrokecolor{currentstroke}%
\pgfsetdash{}{0pt}%
\pgfpathmoveto{\pgfqpoint{0.785700in}{4.525453in}}%
\pgfpathlineto{\pgfqpoint{0.935603in}{4.525453in}}%
\pgfpathlineto{\pgfqpoint{0.935603in}{4.705633in}}%
\pgfpathlineto{\pgfqpoint{0.785700in}{4.705633in}}%
\pgfpathlineto{\pgfqpoint{0.785700in}{4.525453in}}%
\pgfpathclose%
\pgfusepath{stroke,fill}%
\end{pgfscope}%
\begin{pgfscope}%
\pgfsetrectcap%
\pgfsetmiterjoin%
\pgfsetlinewidth{0.803000pt}%
\definecolor{currentstroke}{rgb}{0.000000,0.000000,0.000000}%
\pgfsetstrokecolor{currentstroke}%
\pgfsetdash{}{0pt}%
\pgfpathmoveto{\pgfqpoint{0.148611in}{4.525453in}}%
\pgfpathlineto{\pgfqpoint{0.148611in}{4.768611in}}%
\pgfusepath{stroke}%
\end{pgfscope}%
\begin{pgfscope}%
\pgfsetrectcap%
\pgfsetmiterjoin%
\pgfsetlinewidth{0.803000pt}%
\definecolor{currentstroke}{rgb}{0.000000,0.000000,0.000000}%
\pgfsetstrokecolor{currentstroke}%
\pgfsetdash{}{0pt}%
\pgfpathmoveto{\pgfqpoint{0.973079in}{4.525453in}}%
\pgfpathlineto{\pgfqpoint{0.973079in}{4.768611in}}%
\pgfusepath{stroke}%
\end{pgfscope}%
\begin{pgfscope}%
\pgfsetrectcap%
\pgfsetmiterjoin%
\pgfsetlinewidth{0.803000pt}%
\definecolor{currentstroke}{rgb}{0.000000,0.000000,0.000000}%
\pgfsetstrokecolor{currentstroke}%
\pgfsetdash{}{0pt}%
\pgfpathmoveto{\pgfqpoint{0.148611in}{4.525453in}}%
\pgfpathlineto{\pgfqpoint{0.973079in}{4.525453in}}%
\pgfusepath{stroke}%
\end{pgfscope}%
\begin{pgfscope}%
\pgfsetrectcap%
\pgfsetmiterjoin%
\pgfsetlinewidth{0.803000pt}%
\definecolor{currentstroke}{rgb}{0.000000,0.000000,0.000000}%
\pgfsetstrokecolor{currentstroke}%
\pgfsetdash{}{0pt}%
\pgfpathmoveto{\pgfqpoint{0.148611in}{4.768611in}}%
\pgfpathlineto{\pgfqpoint{0.973079in}{4.768611in}}%
\pgfusepath{stroke}%
\end{pgfscope}%
\begin{pgfscope}%
\definecolor{textcolor}{rgb}{0.000000,0.000000,0.000000}%
\pgfsetstrokecolor{textcolor}%
\pgfsetfillcolor{textcolor}%
\pgftext[x=0.560845in,y=4.851944in,,base]{\color{textcolor}\rmfamily\fontsize{11.000000}{13.200000}\selectfont Direct...}%
\end{pgfscope}%
\begin{pgfscope}%
\pgfsetbuttcap%
\pgfsetmiterjoin%
\definecolor{currentfill}{rgb}{1.000000,1.000000,1.000000}%
\pgfsetfillcolor{currentfill}%
\pgfsetlinewidth{0.000000pt}%
\definecolor{currentstroke}{rgb}{0.000000,0.000000,0.000000}%
\pgfsetstrokecolor{currentstroke}%
\pgfsetstrokeopacity{0.000000}%
\pgfsetdash{}{0pt}%
\pgfpathmoveto{\pgfqpoint{1.137973in}{4.525453in}}%
\pgfpathlineto{\pgfqpoint{1.962441in}{4.525453in}}%
\pgfpathlineto{\pgfqpoint{1.962441in}{4.768611in}}%
\pgfpathlineto{\pgfqpoint{1.137973in}{4.768611in}}%
\pgfpathlineto{\pgfqpoint{1.137973in}{4.525453in}}%
\pgfpathclose%
\pgfusepath{fill}%
\end{pgfscope}%
\begin{pgfscope}%
\pgfpathrectangle{\pgfqpoint{1.137973in}{4.525453in}}{\pgfqpoint{0.824468in}{0.243158in}}%
\pgfusepath{clip}%
\pgfsetbuttcap%
\pgfsetmiterjoin%
\definecolor{currentfill}{rgb}{0.121569,0.466667,0.705882}%
\pgfsetfillcolor{currentfill}%
\pgfsetfillopacity{0.500000}%
\pgfsetlinewidth{1.003750pt}%
\definecolor{currentstroke}{rgb}{0.000000,0.000000,0.000000}%
\pgfsetstrokecolor{currentstroke}%
\pgfsetdash{}{0pt}%
\pgfpathmoveto{\pgfqpoint{1.175449in}{4.525453in}}%
\pgfpathlineto{\pgfqpoint{1.325352in}{4.525453in}}%
\pgfpathlineto{\pgfqpoint{1.325352in}{4.572018in}}%
\pgfpathlineto{\pgfqpoint{1.175449in}{4.572018in}}%
\pgfpathlineto{\pgfqpoint{1.175449in}{4.525453in}}%
\pgfpathclose%
\pgfusepath{stroke,fill}%
\end{pgfscope}%
\begin{pgfscope}%
\pgfpathrectangle{\pgfqpoint{1.137973in}{4.525453in}}{\pgfqpoint{0.824468in}{0.243158in}}%
\pgfusepath{clip}%
\pgfsetbuttcap%
\pgfsetmiterjoin%
\definecolor{currentfill}{rgb}{0.121569,0.466667,0.705882}%
\pgfsetfillcolor{currentfill}%
\pgfsetfillopacity{0.500000}%
\pgfsetlinewidth{1.003750pt}%
\definecolor{currentstroke}{rgb}{0.000000,0.000000,0.000000}%
\pgfsetstrokecolor{currentstroke}%
\pgfsetdash{}{0pt}%
\pgfpathmoveto{\pgfqpoint{1.325352in}{4.525453in}}%
\pgfpathlineto{\pgfqpoint{1.475255in}{4.525453in}}%
\pgfpathlineto{\pgfqpoint{1.475255in}{4.565817in}}%
\pgfpathlineto{\pgfqpoint{1.325352in}{4.565817in}}%
\pgfpathlineto{\pgfqpoint{1.325352in}{4.525453in}}%
\pgfpathclose%
\pgfusepath{stroke,fill}%
\end{pgfscope}%
\begin{pgfscope}%
\pgfpathrectangle{\pgfqpoint{1.137973in}{4.525453in}}{\pgfqpoint{0.824468in}{0.243158in}}%
\pgfusepath{clip}%
\pgfsetbuttcap%
\pgfsetmiterjoin%
\definecolor{currentfill}{rgb}{0.121569,0.466667,0.705882}%
\pgfsetfillcolor{currentfill}%
\pgfsetfillopacity{0.500000}%
\pgfsetlinewidth{1.003750pt}%
\definecolor{currentstroke}{rgb}{0.000000,0.000000,0.000000}%
\pgfsetstrokecolor{currentstroke}%
\pgfsetdash{}{0pt}%
\pgfpathmoveto{\pgfqpoint{1.475255in}{4.525453in}}%
\pgfpathlineto{\pgfqpoint{1.625158in}{4.525453in}}%
\pgfpathlineto{\pgfqpoint{1.625158in}{4.589890in}}%
\pgfpathlineto{\pgfqpoint{1.475255in}{4.589890in}}%
\pgfpathlineto{\pgfqpoint{1.475255in}{4.525453in}}%
\pgfpathclose%
\pgfusepath{stroke,fill}%
\end{pgfscope}%
\begin{pgfscope}%
\pgfpathrectangle{\pgfqpoint{1.137973in}{4.525453in}}{\pgfqpoint{0.824468in}{0.243158in}}%
\pgfusepath{clip}%
\pgfsetbuttcap%
\pgfsetmiterjoin%
\definecolor{currentfill}{rgb}{0.121569,0.466667,0.705882}%
\pgfsetfillcolor{currentfill}%
\pgfsetfillopacity{0.500000}%
\pgfsetlinewidth{1.003750pt}%
\definecolor{currentstroke}{rgb}{0.000000,0.000000,0.000000}%
\pgfsetstrokecolor{currentstroke}%
\pgfsetdash{}{0pt}%
\pgfpathmoveto{\pgfqpoint{1.625158in}{4.525453in}}%
\pgfpathlineto{\pgfqpoint{1.775062in}{4.525453in}}%
\pgfpathlineto{\pgfqpoint{1.775062in}{4.699311in}}%
\pgfpathlineto{\pgfqpoint{1.625158in}{4.699311in}}%
\pgfpathlineto{\pgfqpoint{1.625158in}{4.525453in}}%
\pgfpathclose%
\pgfusepath{stroke,fill}%
\end{pgfscope}%
\begin{pgfscope}%
\pgfpathrectangle{\pgfqpoint{1.137973in}{4.525453in}}{\pgfqpoint{0.824468in}{0.243158in}}%
\pgfusepath{clip}%
\pgfsetbuttcap%
\pgfsetmiterjoin%
\definecolor{currentfill}{rgb}{0.121569,0.466667,0.705882}%
\pgfsetfillcolor{currentfill}%
\pgfsetfillopacity{0.500000}%
\pgfsetlinewidth{1.003750pt}%
\definecolor{currentstroke}{rgb}{0.000000,0.000000,0.000000}%
\pgfsetstrokecolor{currentstroke}%
\pgfsetdash{}{0pt}%
\pgfpathmoveto{\pgfqpoint{1.775062in}{4.525453in}}%
\pgfpathlineto{\pgfqpoint{1.924965in}{4.525453in}}%
\pgfpathlineto{\pgfqpoint{1.924965in}{4.721560in}}%
\pgfpathlineto{\pgfqpoint{1.775062in}{4.721560in}}%
\pgfpathlineto{\pgfqpoint{1.775062in}{4.525453in}}%
\pgfpathclose%
\pgfusepath{stroke,fill}%
\end{pgfscope}%
\begin{pgfscope}%
\pgfsetrectcap%
\pgfsetmiterjoin%
\pgfsetlinewidth{0.803000pt}%
\definecolor{currentstroke}{rgb}{0.000000,0.000000,0.000000}%
\pgfsetstrokecolor{currentstroke}%
\pgfsetdash{}{0pt}%
\pgfpathmoveto{\pgfqpoint{1.137973in}{4.525453in}}%
\pgfpathlineto{\pgfqpoint{1.137973in}{4.768611in}}%
\pgfusepath{stroke}%
\end{pgfscope}%
\begin{pgfscope}%
\pgfsetrectcap%
\pgfsetmiterjoin%
\pgfsetlinewidth{0.803000pt}%
\definecolor{currentstroke}{rgb}{0.000000,0.000000,0.000000}%
\pgfsetstrokecolor{currentstroke}%
\pgfsetdash{}{0pt}%
\pgfpathmoveto{\pgfqpoint{1.962441in}{4.525453in}}%
\pgfpathlineto{\pgfqpoint{1.962441in}{4.768611in}}%
\pgfusepath{stroke}%
\end{pgfscope}%
\begin{pgfscope}%
\pgfsetrectcap%
\pgfsetmiterjoin%
\pgfsetlinewidth{0.803000pt}%
\definecolor{currentstroke}{rgb}{0.000000,0.000000,0.000000}%
\pgfsetstrokecolor{currentstroke}%
\pgfsetdash{}{0pt}%
\pgfpathmoveto{\pgfqpoint{1.137973in}{4.525453in}}%
\pgfpathlineto{\pgfqpoint{1.962441in}{4.525453in}}%
\pgfusepath{stroke}%
\end{pgfscope}%
\begin{pgfscope}%
\pgfsetrectcap%
\pgfsetmiterjoin%
\pgfsetlinewidth{0.803000pt}%
\definecolor{currentstroke}{rgb}{0.000000,0.000000,0.000000}%
\pgfsetstrokecolor{currentstroke}%
\pgfsetdash{}{0pt}%
\pgfpathmoveto{\pgfqpoint{1.137973in}{4.768611in}}%
\pgfpathlineto{\pgfqpoint{1.962441in}{4.768611in}}%
\pgfusepath{stroke}%
\end{pgfscope}%
\begin{pgfscope}%
\definecolor{textcolor}{rgb}{0.000000,0.000000,0.000000}%
\pgfsetstrokecolor{textcolor}%
\pgfsetfillcolor{textcolor}%
\pgftext[x=1.550207in,y=4.851944in,,base]{\color{textcolor}\rmfamily\fontsize{11.000000}{13.200000}\selectfont L'oliv...}%
\end{pgfscope}%
\begin{pgfscope}%
\pgfsetbuttcap%
\pgfsetmiterjoin%
\definecolor{currentfill}{rgb}{1.000000,1.000000,1.000000}%
\pgfsetfillcolor{currentfill}%
\pgfsetlinewidth{0.000000pt}%
\definecolor{currentstroke}{rgb}{0.000000,0.000000,0.000000}%
\pgfsetstrokecolor{currentstroke}%
\pgfsetstrokeopacity{0.000000}%
\pgfsetdash{}{0pt}%
\pgfpathmoveto{\pgfqpoint{2.127335in}{4.525453in}}%
\pgfpathlineto{\pgfqpoint{2.951803in}{4.525453in}}%
\pgfpathlineto{\pgfqpoint{2.951803in}{4.768611in}}%
\pgfpathlineto{\pgfqpoint{2.127335in}{4.768611in}}%
\pgfpathlineto{\pgfqpoint{2.127335in}{4.525453in}}%
\pgfpathclose%
\pgfusepath{fill}%
\end{pgfscope}%
\begin{pgfscope}%
\pgfpathrectangle{\pgfqpoint{2.127335in}{4.525453in}}{\pgfqpoint{0.824468in}{0.243158in}}%
\pgfusepath{clip}%
\pgfsetbuttcap%
\pgfsetmiterjoin%
\definecolor{currentfill}{rgb}{0.121569,0.466667,0.705882}%
\pgfsetfillcolor{currentfill}%
\pgfsetfillopacity{0.500000}%
\pgfsetlinewidth{1.003750pt}%
\definecolor{currentstroke}{rgb}{0.000000,0.000000,0.000000}%
\pgfsetstrokecolor{currentstroke}%
\pgfsetdash{}{0pt}%
\pgfpathmoveto{\pgfqpoint{2.164810in}{4.525453in}}%
\pgfpathlineto{\pgfqpoint{2.314714in}{4.525453in}}%
\pgfpathlineto{\pgfqpoint{2.314714in}{4.552079in}}%
\pgfpathlineto{\pgfqpoint{2.164810in}{4.552079in}}%
\pgfpathlineto{\pgfqpoint{2.164810in}{4.525453in}}%
\pgfpathclose%
\pgfusepath{stroke,fill}%
\end{pgfscope}%
\begin{pgfscope}%
\pgfpathrectangle{\pgfqpoint{2.127335in}{4.525453in}}{\pgfqpoint{0.824468in}{0.243158in}}%
\pgfusepath{clip}%
\pgfsetbuttcap%
\pgfsetmiterjoin%
\definecolor{currentfill}{rgb}{0.121569,0.466667,0.705882}%
\pgfsetfillcolor{currentfill}%
\pgfsetfillopacity{0.500000}%
\pgfsetlinewidth{1.003750pt}%
\definecolor{currentstroke}{rgb}{0.000000,0.000000,0.000000}%
\pgfsetstrokecolor{currentstroke}%
\pgfsetdash{}{0pt}%
\pgfpathmoveto{\pgfqpoint{2.314714in}{4.525453in}}%
\pgfpathlineto{\pgfqpoint{2.464617in}{4.525453in}}%
\pgfpathlineto{\pgfqpoint{2.464617in}{4.542110in}}%
\pgfpathlineto{\pgfqpoint{2.314714in}{4.542110in}}%
\pgfpathlineto{\pgfqpoint{2.314714in}{4.525453in}}%
\pgfpathclose%
\pgfusepath{stroke,fill}%
\end{pgfscope}%
\begin{pgfscope}%
\pgfpathrectangle{\pgfqpoint{2.127335in}{4.525453in}}{\pgfqpoint{0.824468in}{0.243158in}}%
\pgfusepath{clip}%
\pgfsetbuttcap%
\pgfsetmiterjoin%
\definecolor{currentfill}{rgb}{0.121569,0.466667,0.705882}%
\pgfsetfillcolor{currentfill}%
\pgfsetfillopacity{0.500000}%
\pgfsetlinewidth{1.003750pt}%
\definecolor{currentstroke}{rgb}{0.000000,0.000000,0.000000}%
\pgfsetstrokecolor{currentstroke}%
\pgfsetdash{}{0pt}%
\pgfpathmoveto{\pgfqpoint{2.464617in}{4.525453in}}%
\pgfpathlineto{\pgfqpoint{2.614520in}{4.525453in}}%
\pgfpathlineto{\pgfqpoint{2.614520in}{4.531289in}}%
\pgfpathlineto{\pgfqpoint{2.464617in}{4.531289in}}%
\pgfpathlineto{\pgfqpoint{2.464617in}{4.525453in}}%
\pgfpathclose%
\pgfusepath{stroke,fill}%
\end{pgfscope}%
\begin{pgfscope}%
\pgfpathrectangle{\pgfqpoint{2.127335in}{4.525453in}}{\pgfqpoint{0.824468in}{0.243158in}}%
\pgfusepath{clip}%
\pgfsetbuttcap%
\pgfsetmiterjoin%
\definecolor{currentfill}{rgb}{0.121569,0.466667,0.705882}%
\pgfsetfillcolor{currentfill}%
\pgfsetfillopacity{0.500000}%
\pgfsetlinewidth{1.003750pt}%
\definecolor{currentstroke}{rgb}{0.000000,0.000000,0.000000}%
\pgfsetstrokecolor{currentstroke}%
\pgfsetdash{}{0pt}%
\pgfpathmoveto{\pgfqpoint{2.614520in}{4.525453in}}%
\pgfpathlineto{\pgfqpoint{2.764423in}{4.525453in}}%
\pgfpathlineto{\pgfqpoint{2.764423in}{4.529708in}}%
\pgfpathlineto{\pgfqpoint{2.614520in}{4.529708in}}%
\pgfpathlineto{\pgfqpoint{2.614520in}{4.525453in}}%
\pgfpathclose%
\pgfusepath{stroke,fill}%
\end{pgfscope}%
\begin{pgfscope}%
\pgfpathrectangle{\pgfqpoint{2.127335in}{4.525453in}}{\pgfqpoint{0.824468in}{0.243158in}}%
\pgfusepath{clip}%
\pgfsetbuttcap%
\pgfsetmiterjoin%
\definecolor{currentfill}{rgb}{0.121569,0.466667,0.705882}%
\pgfsetfillcolor{currentfill}%
\pgfsetfillopacity{0.500000}%
\pgfsetlinewidth{1.003750pt}%
\definecolor{currentstroke}{rgb}{0.000000,0.000000,0.000000}%
\pgfsetstrokecolor{currentstroke}%
\pgfsetdash{}{0pt}%
\pgfpathmoveto{\pgfqpoint{2.764423in}{4.525453in}}%
\pgfpathlineto{\pgfqpoint{2.914327in}{4.525453in}}%
\pgfpathlineto{\pgfqpoint{2.914327in}{4.529708in}}%
\pgfpathlineto{\pgfqpoint{2.764423in}{4.529708in}}%
\pgfpathlineto{\pgfqpoint{2.764423in}{4.525453in}}%
\pgfpathclose%
\pgfusepath{stroke,fill}%
\end{pgfscope}%
\begin{pgfscope}%
\pgfsetrectcap%
\pgfsetmiterjoin%
\pgfsetlinewidth{0.803000pt}%
\definecolor{currentstroke}{rgb}{0.000000,0.000000,0.000000}%
\pgfsetstrokecolor{currentstroke}%
\pgfsetdash{}{0pt}%
\pgfpathmoveto{\pgfqpoint{2.127335in}{4.525453in}}%
\pgfpathlineto{\pgfqpoint{2.127335in}{4.768611in}}%
\pgfusepath{stroke}%
\end{pgfscope}%
\begin{pgfscope}%
\pgfsetrectcap%
\pgfsetmiterjoin%
\pgfsetlinewidth{0.803000pt}%
\definecolor{currentstroke}{rgb}{0.000000,0.000000,0.000000}%
\pgfsetstrokecolor{currentstroke}%
\pgfsetdash{}{0pt}%
\pgfpathmoveto{\pgfqpoint{2.951803in}{4.525453in}}%
\pgfpathlineto{\pgfqpoint{2.951803in}{4.768611in}}%
\pgfusepath{stroke}%
\end{pgfscope}%
\begin{pgfscope}%
\pgfsetrectcap%
\pgfsetmiterjoin%
\pgfsetlinewidth{0.803000pt}%
\definecolor{currentstroke}{rgb}{0.000000,0.000000,0.000000}%
\pgfsetstrokecolor{currentstroke}%
\pgfsetdash{}{0pt}%
\pgfpathmoveto{\pgfqpoint{2.127335in}{4.525453in}}%
\pgfpathlineto{\pgfqpoint{2.951803in}{4.525453in}}%
\pgfusepath{stroke}%
\end{pgfscope}%
\begin{pgfscope}%
\pgfsetrectcap%
\pgfsetmiterjoin%
\pgfsetlinewidth{0.803000pt}%
\definecolor{currentstroke}{rgb}{0.000000,0.000000,0.000000}%
\pgfsetstrokecolor{currentstroke}%
\pgfsetdash{}{0pt}%
\pgfpathmoveto{\pgfqpoint{2.127335in}{4.768611in}}%
\pgfpathlineto{\pgfqpoint{2.951803in}{4.768611in}}%
\pgfusepath{stroke}%
\end{pgfscope}%
\begin{pgfscope}%
\definecolor{textcolor}{rgb}{0.000000,0.000000,0.000000}%
\pgfsetstrokecolor{textcolor}%
\pgfsetfillcolor{textcolor}%
\pgftext[x=2.539569in,y=4.851944in,,base]{\color{textcolor}\rmfamily\fontsize{11.000000}{13.200000}\selectfont Matmut}%
\end{pgfscope}%
\begin{pgfscope}%
\pgfsetbuttcap%
\pgfsetmiterjoin%
\definecolor{currentfill}{rgb}{1.000000,1.000000,1.000000}%
\pgfsetfillcolor{currentfill}%
\pgfsetlinewidth{0.000000pt}%
\definecolor{currentstroke}{rgb}{0.000000,0.000000,0.000000}%
\pgfsetstrokecolor{currentstroke}%
\pgfsetstrokeopacity{0.000000}%
\pgfsetdash{}{0pt}%
\pgfpathmoveto{\pgfqpoint{3.116696in}{4.525453in}}%
\pgfpathlineto{\pgfqpoint{3.941164in}{4.525453in}}%
\pgfpathlineto{\pgfqpoint{3.941164in}{4.768611in}}%
\pgfpathlineto{\pgfqpoint{3.116696in}{4.768611in}}%
\pgfpathlineto{\pgfqpoint{3.116696in}{4.525453in}}%
\pgfpathclose%
\pgfusepath{fill}%
\end{pgfscope}%
\begin{pgfscope}%
\pgfpathrectangle{\pgfqpoint{3.116696in}{4.525453in}}{\pgfqpoint{0.824468in}{0.243158in}}%
\pgfusepath{clip}%
\pgfsetbuttcap%
\pgfsetmiterjoin%
\definecolor{currentfill}{rgb}{0.121569,0.466667,0.705882}%
\pgfsetfillcolor{currentfill}%
\pgfsetfillopacity{0.500000}%
\pgfsetlinewidth{1.003750pt}%
\definecolor{currentstroke}{rgb}{0.000000,0.000000,0.000000}%
\pgfsetstrokecolor{currentstroke}%
\pgfsetdash{}{0pt}%
\pgfpathmoveto{\pgfqpoint{3.154172in}{4.525453in}}%
\pgfpathlineto{\pgfqpoint{3.304075in}{4.525453in}}%
\pgfpathlineto{\pgfqpoint{3.304075in}{4.557915in}}%
\pgfpathlineto{\pgfqpoint{3.154172in}{4.557915in}}%
\pgfpathlineto{\pgfqpoint{3.154172in}{4.525453in}}%
\pgfpathclose%
\pgfusepath{stroke,fill}%
\end{pgfscope}%
\begin{pgfscope}%
\pgfpathrectangle{\pgfqpoint{3.116696in}{4.525453in}}{\pgfqpoint{0.824468in}{0.243158in}}%
\pgfusepath{clip}%
\pgfsetbuttcap%
\pgfsetmiterjoin%
\definecolor{currentfill}{rgb}{0.121569,0.466667,0.705882}%
\pgfsetfillcolor{currentfill}%
\pgfsetfillopacity{0.500000}%
\pgfsetlinewidth{1.003750pt}%
\definecolor{currentstroke}{rgb}{0.000000,0.000000,0.000000}%
\pgfsetstrokecolor{currentstroke}%
\pgfsetdash{}{0pt}%
\pgfpathmoveto{\pgfqpoint{3.304075in}{4.525453in}}%
\pgfpathlineto{\pgfqpoint{3.453979in}{4.525453in}}%
\pgfpathlineto{\pgfqpoint{3.453979in}{4.538341in}}%
\pgfpathlineto{\pgfqpoint{3.304075in}{4.538341in}}%
\pgfpathlineto{\pgfqpoint{3.304075in}{4.525453in}}%
\pgfpathclose%
\pgfusepath{stroke,fill}%
\end{pgfscope}%
\begin{pgfscope}%
\pgfpathrectangle{\pgfqpoint{3.116696in}{4.525453in}}{\pgfqpoint{0.824468in}{0.243158in}}%
\pgfusepath{clip}%
\pgfsetbuttcap%
\pgfsetmiterjoin%
\definecolor{currentfill}{rgb}{0.121569,0.466667,0.705882}%
\pgfsetfillcolor{currentfill}%
\pgfsetfillopacity{0.500000}%
\pgfsetlinewidth{1.003750pt}%
\definecolor{currentstroke}{rgb}{0.000000,0.000000,0.000000}%
\pgfsetstrokecolor{currentstroke}%
\pgfsetdash{}{0pt}%
\pgfpathmoveto{\pgfqpoint{3.453979in}{4.525453in}}%
\pgfpathlineto{\pgfqpoint{3.603882in}{4.525453in}}%
\pgfpathlineto{\pgfqpoint{3.603882in}{4.545635in}}%
\pgfpathlineto{\pgfqpoint{3.453979in}{4.545635in}}%
\pgfpathlineto{\pgfqpoint{3.453979in}{4.525453in}}%
\pgfpathclose%
\pgfusepath{stroke,fill}%
\end{pgfscope}%
\begin{pgfscope}%
\pgfpathrectangle{\pgfqpoint{3.116696in}{4.525453in}}{\pgfqpoint{0.824468in}{0.243158in}}%
\pgfusepath{clip}%
\pgfsetbuttcap%
\pgfsetmiterjoin%
\definecolor{currentfill}{rgb}{0.121569,0.466667,0.705882}%
\pgfsetfillcolor{currentfill}%
\pgfsetfillopacity{0.500000}%
\pgfsetlinewidth{1.003750pt}%
\definecolor{currentstroke}{rgb}{0.000000,0.000000,0.000000}%
\pgfsetstrokecolor{currentstroke}%
\pgfsetdash{}{0pt}%
\pgfpathmoveto{\pgfqpoint{3.603882in}{4.525453in}}%
\pgfpathlineto{\pgfqpoint{3.753785in}{4.525453in}}%
\pgfpathlineto{\pgfqpoint{3.753785in}{4.547337in}}%
\pgfpathlineto{\pgfqpoint{3.603882in}{4.547337in}}%
\pgfpathlineto{\pgfqpoint{3.603882in}{4.525453in}}%
\pgfpathclose%
\pgfusepath{stroke,fill}%
\end{pgfscope}%
\begin{pgfscope}%
\pgfpathrectangle{\pgfqpoint{3.116696in}{4.525453in}}{\pgfqpoint{0.824468in}{0.243158in}}%
\pgfusepath{clip}%
\pgfsetbuttcap%
\pgfsetmiterjoin%
\definecolor{currentfill}{rgb}{0.121569,0.466667,0.705882}%
\pgfsetfillcolor{currentfill}%
\pgfsetfillopacity{0.500000}%
\pgfsetlinewidth{1.003750pt}%
\definecolor{currentstroke}{rgb}{0.000000,0.000000,0.000000}%
\pgfsetstrokecolor{currentstroke}%
\pgfsetdash{}{0pt}%
\pgfpathmoveto{\pgfqpoint{3.753785in}{4.525453in}}%
\pgfpathlineto{\pgfqpoint{3.903688in}{4.525453in}}%
\pgfpathlineto{\pgfqpoint{3.903688in}{4.542717in}}%
\pgfpathlineto{\pgfqpoint{3.753785in}{4.542717in}}%
\pgfpathlineto{\pgfqpoint{3.753785in}{4.525453in}}%
\pgfpathclose%
\pgfusepath{stroke,fill}%
\end{pgfscope}%
\begin{pgfscope}%
\pgfsetrectcap%
\pgfsetmiterjoin%
\pgfsetlinewidth{0.803000pt}%
\definecolor{currentstroke}{rgb}{0.000000,0.000000,0.000000}%
\pgfsetstrokecolor{currentstroke}%
\pgfsetdash{}{0pt}%
\pgfpathmoveto{\pgfqpoint{3.116696in}{4.525453in}}%
\pgfpathlineto{\pgfqpoint{3.116696in}{4.768611in}}%
\pgfusepath{stroke}%
\end{pgfscope}%
\begin{pgfscope}%
\pgfsetrectcap%
\pgfsetmiterjoin%
\pgfsetlinewidth{0.803000pt}%
\definecolor{currentstroke}{rgb}{0.000000,0.000000,0.000000}%
\pgfsetstrokecolor{currentstroke}%
\pgfsetdash{}{0pt}%
\pgfpathmoveto{\pgfqpoint{3.941164in}{4.525453in}}%
\pgfpathlineto{\pgfqpoint{3.941164in}{4.768611in}}%
\pgfusepath{stroke}%
\end{pgfscope}%
\begin{pgfscope}%
\pgfsetrectcap%
\pgfsetmiterjoin%
\pgfsetlinewidth{0.803000pt}%
\definecolor{currentstroke}{rgb}{0.000000,0.000000,0.000000}%
\pgfsetstrokecolor{currentstroke}%
\pgfsetdash{}{0pt}%
\pgfpathmoveto{\pgfqpoint{3.116696in}{4.525453in}}%
\pgfpathlineto{\pgfqpoint{3.941164in}{4.525453in}}%
\pgfusepath{stroke}%
\end{pgfscope}%
\begin{pgfscope}%
\pgfsetrectcap%
\pgfsetmiterjoin%
\pgfsetlinewidth{0.803000pt}%
\definecolor{currentstroke}{rgb}{0.000000,0.000000,0.000000}%
\pgfsetstrokecolor{currentstroke}%
\pgfsetdash{}{0pt}%
\pgfpathmoveto{\pgfqpoint{3.116696in}{4.768611in}}%
\pgfpathlineto{\pgfqpoint{3.941164in}{4.768611in}}%
\pgfusepath{stroke}%
\end{pgfscope}%
\begin{pgfscope}%
\definecolor{textcolor}{rgb}{0.000000,0.000000,0.000000}%
\pgfsetstrokecolor{textcolor}%
\pgfsetfillcolor{textcolor}%
\pgftext[x=3.528930in,y=4.851944in,,base]{\color{textcolor}\rmfamily\fontsize{11.000000}{13.200000}\selectfont Néolia...}%
\end{pgfscope}%
\begin{pgfscope}%
\pgfsetbuttcap%
\pgfsetmiterjoin%
\definecolor{currentfill}{rgb}{1.000000,1.000000,1.000000}%
\pgfsetfillcolor{currentfill}%
\pgfsetlinewidth{0.000000pt}%
\definecolor{currentstroke}{rgb}{0.000000,0.000000,0.000000}%
\pgfsetstrokecolor{currentstroke}%
\pgfsetstrokeopacity{0.000000}%
\pgfsetdash{}{0pt}%
\pgfpathmoveto{\pgfqpoint{4.106058in}{4.525453in}}%
\pgfpathlineto{\pgfqpoint{4.930526in}{4.525453in}}%
\pgfpathlineto{\pgfqpoint{4.930526in}{4.768611in}}%
\pgfpathlineto{\pgfqpoint{4.106058in}{4.768611in}}%
\pgfpathlineto{\pgfqpoint{4.106058in}{4.525453in}}%
\pgfpathclose%
\pgfusepath{fill}%
\end{pgfscope}%
\begin{pgfscope}%
\pgfpathrectangle{\pgfqpoint{4.106058in}{4.525453in}}{\pgfqpoint{0.824468in}{0.243158in}}%
\pgfusepath{clip}%
\pgfsetbuttcap%
\pgfsetmiterjoin%
\definecolor{currentfill}{rgb}{0.121569,0.466667,0.705882}%
\pgfsetfillcolor{currentfill}%
\pgfsetfillopacity{0.500000}%
\pgfsetlinewidth{1.003750pt}%
\definecolor{currentstroke}{rgb}{0.000000,0.000000,0.000000}%
\pgfsetstrokecolor{currentstroke}%
\pgfsetdash{}{0pt}%
\pgfpathmoveto{\pgfqpoint{4.143534in}{4.525453in}}%
\pgfpathlineto{\pgfqpoint{4.293437in}{4.525453in}}%
\pgfpathlineto{\pgfqpoint{4.293437in}{4.539313in}}%
\pgfpathlineto{\pgfqpoint{4.143534in}{4.539313in}}%
\pgfpathlineto{\pgfqpoint{4.143534in}{4.525453in}}%
\pgfpathclose%
\pgfusepath{stroke,fill}%
\end{pgfscope}%
\begin{pgfscope}%
\pgfpathrectangle{\pgfqpoint{4.106058in}{4.525453in}}{\pgfqpoint{0.824468in}{0.243158in}}%
\pgfusepath{clip}%
\pgfsetbuttcap%
\pgfsetmiterjoin%
\definecolor{currentfill}{rgb}{0.121569,0.466667,0.705882}%
\pgfsetfillcolor{currentfill}%
\pgfsetfillopacity{0.500000}%
\pgfsetlinewidth{1.003750pt}%
\definecolor{currentstroke}{rgb}{0.000000,0.000000,0.000000}%
\pgfsetstrokecolor{currentstroke}%
\pgfsetdash{}{0pt}%
\pgfpathmoveto{\pgfqpoint{4.293437in}{4.525453in}}%
\pgfpathlineto{\pgfqpoint{4.443340in}{4.525453in}}%
\pgfpathlineto{\pgfqpoint{4.443340in}{4.531046in}}%
\pgfpathlineto{\pgfqpoint{4.293437in}{4.531046in}}%
\pgfpathlineto{\pgfqpoint{4.293437in}{4.525453in}}%
\pgfpathclose%
\pgfusepath{stroke,fill}%
\end{pgfscope}%
\begin{pgfscope}%
\pgfpathrectangle{\pgfqpoint{4.106058in}{4.525453in}}{\pgfqpoint{0.824468in}{0.243158in}}%
\pgfusepath{clip}%
\pgfsetbuttcap%
\pgfsetmiterjoin%
\definecolor{currentfill}{rgb}{0.121569,0.466667,0.705882}%
\pgfsetfillcolor{currentfill}%
\pgfsetfillopacity{0.500000}%
\pgfsetlinewidth{1.003750pt}%
\definecolor{currentstroke}{rgb}{0.000000,0.000000,0.000000}%
\pgfsetstrokecolor{currentstroke}%
\pgfsetdash{}{0pt}%
\pgfpathmoveto{\pgfqpoint{4.443340in}{4.525453in}}%
\pgfpathlineto{\pgfqpoint{4.593244in}{4.525453in}}%
\pgfpathlineto{\pgfqpoint{4.593244in}{4.531046in}}%
\pgfpathlineto{\pgfqpoint{4.443340in}{4.531046in}}%
\pgfpathlineto{\pgfqpoint{4.443340in}{4.525453in}}%
\pgfpathclose%
\pgfusepath{stroke,fill}%
\end{pgfscope}%
\begin{pgfscope}%
\pgfpathrectangle{\pgfqpoint{4.106058in}{4.525453in}}{\pgfqpoint{0.824468in}{0.243158in}}%
\pgfusepath{clip}%
\pgfsetbuttcap%
\pgfsetmiterjoin%
\definecolor{currentfill}{rgb}{0.121569,0.466667,0.705882}%
\pgfsetfillcolor{currentfill}%
\pgfsetfillopacity{0.500000}%
\pgfsetlinewidth{1.003750pt}%
\definecolor{currentstroke}{rgb}{0.000000,0.000000,0.000000}%
\pgfsetstrokecolor{currentstroke}%
\pgfsetdash{}{0pt}%
\pgfpathmoveto{\pgfqpoint{4.593244in}{4.525453in}}%
\pgfpathlineto{\pgfqpoint{4.743147in}{4.525453in}}%
\pgfpathlineto{\pgfqpoint{4.743147in}{4.530803in}}%
\pgfpathlineto{\pgfqpoint{4.593244in}{4.530803in}}%
\pgfpathlineto{\pgfqpoint{4.593244in}{4.525453in}}%
\pgfpathclose%
\pgfusepath{stroke,fill}%
\end{pgfscope}%
\begin{pgfscope}%
\pgfpathrectangle{\pgfqpoint{4.106058in}{4.525453in}}{\pgfqpoint{0.824468in}{0.243158in}}%
\pgfusepath{clip}%
\pgfsetbuttcap%
\pgfsetmiterjoin%
\definecolor{currentfill}{rgb}{0.121569,0.466667,0.705882}%
\pgfsetfillcolor{currentfill}%
\pgfsetfillopacity{0.500000}%
\pgfsetlinewidth{1.003750pt}%
\definecolor{currentstroke}{rgb}{0.000000,0.000000,0.000000}%
\pgfsetstrokecolor{currentstroke}%
\pgfsetdash{}{0pt}%
\pgfpathmoveto{\pgfqpoint{4.743147in}{4.525453in}}%
\pgfpathlineto{\pgfqpoint{4.893050in}{4.525453in}}%
\pgfpathlineto{\pgfqpoint{4.893050in}{4.529465in}}%
\pgfpathlineto{\pgfqpoint{4.743147in}{4.529465in}}%
\pgfpathlineto{\pgfqpoint{4.743147in}{4.525453in}}%
\pgfpathclose%
\pgfusepath{stroke,fill}%
\end{pgfscope}%
\begin{pgfscope}%
\pgfsetrectcap%
\pgfsetmiterjoin%
\pgfsetlinewidth{0.803000pt}%
\definecolor{currentstroke}{rgb}{0.000000,0.000000,0.000000}%
\pgfsetstrokecolor{currentstroke}%
\pgfsetdash{}{0pt}%
\pgfpathmoveto{\pgfqpoint{4.106058in}{4.525453in}}%
\pgfpathlineto{\pgfqpoint{4.106058in}{4.768611in}}%
\pgfusepath{stroke}%
\end{pgfscope}%
\begin{pgfscope}%
\pgfsetrectcap%
\pgfsetmiterjoin%
\pgfsetlinewidth{0.803000pt}%
\definecolor{currentstroke}{rgb}{0.000000,0.000000,0.000000}%
\pgfsetstrokecolor{currentstroke}%
\pgfsetdash{}{0pt}%
\pgfpathmoveto{\pgfqpoint{4.930526in}{4.525453in}}%
\pgfpathlineto{\pgfqpoint{4.930526in}{4.768611in}}%
\pgfusepath{stroke}%
\end{pgfscope}%
\begin{pgfscope}%
\pgfsetrectcap%
\pgfsetmiterjoin%
\pgfsetlinewidth{0.803000pt}%
\definecolor{currentstroke}{rgb}{0.000000,0.000000,0.000000}%
\pgfsetstrokecolor{currentstroke}%
\pgfsetdash{}{0pt}%
\pgfpathmoveto{\pgfqpoint{4.106058in}{4.525453in}}%
\pgfpathlineto{\pgfqpoint{4.930526in}{4.525453in}}%
\pgfusepath{stroke}%
\end{pgfscope}%
\begin{pgfscope}%
\pgfsetrectcap%
\pgfsetmiterjoin%
\pgfsetlinewidth{0.803000pt}%
\definecolor{currentstroke}{rgb}{0.000000,0.000000,0.000000}%
\pgfsetstrokecolor{currentstroke}%
\pgfsetdash{}{0pt}%
\pgfpathmoveto{\pgfqpoint{4.106058in}{4.768611in}}%
\pgfpathlineto{\pgfqpoint{4.930526in}{4.768611in}}%
\pgfusepath{stroke}%
\end{pgfscope}%
\begin{pgfscope}%
\definecolor{textcolor}{rgb}{0.000000,0.000000,0.000000}%
\pgfsetstrokecolor{textcolor}%
\pgfsetfillcolor{textcolor}%
\pgftext[x=4.518292in,y=4.851944in,,base]{\color{textcolor}\rmfamily\fontsize{11.000000}{13.200000}\selectfont APRIL}%
\end{pgfscope}%
\begin{pgfscope}%
\pgfsetbuttcap%
\pgfsetmiterjoin%
\definecolor{currentfill}{rgb}{1.000000,1.000000,1.000000}%
\pgfsetfillcolor{currentfill}%
\pgfsetlinewidth{0.000000pt}%
\definecolor{currentstroke}{rgb}{0.000000,0.000000,0.000000}%
\pgfsetstrokecolor{currentstroke}%
\pgfsetstrokeopacity{0.000000}%
\pgfsetdash{}{0pt}%
\pgfpathmoveto{\pgfqpoint{5.095420in}{4.525453in}}%
\pgfpathlineto{\pgfqpoint{5.919888in}{4.525453in}}%
\pgfpathlineto{\pgfqpoint{5.919888in}{4.768611in}}%
\pgfpathlineto{\pgfqpoint{5.095420in}{4.768611in}}%
\pgfpathlineto{\pgfqpoint{5.095420in}{4.525453in}}%
\pgfpathclose%
\pgfusepath{fill}%
\end{pgfscope}%
\begin{pgfscope}%
\pgfpathrectangle{\pgfqpoint{5.095420in}{4.525453in}}{\pgfqpoint{0.824468in}{0.243158in}}%
\pgfusepath{clip}%
\pgfsetbuttcap%
\pgfsetmiterjoin%
\definecolor{currentfill}{rgb}{0.121569,0.466667,0.705882}%
\pgfsetfillcolor{currentfill}%
\pgfsetfillopacity{0.500000}%
\pgfsetlinewidth{1.003750pt}%
\definecolor{currentstroke}{rgb}{0.000000,0.000000,0.000000}%
\pgfsetstrokecolor{currentstroke}%
\pgfsetdash{}{0pt}%
\pgfpathmoveto{\pgfqpoint{5.132895in}{4.525453in}}%
\pgfpathlineto{\pgfqpoint{5.282799in}{4.525453in}}%
\pgfpathlineto{\pgfqpoint{5.282799in}{4.532505in}}%
\pgfpathlineto{\pgfqpoint{5.132895in}{4.532505in}}%
\pgfpathlineto{\pgfqpoint{5.132895in}{4.525453in}}%
\pgfpathclose%
\pgfusepath{stroke,fill}%
\end{pgfscope}%
\begin{pgfscope}%
\pgfpathrectangle{\pgfqpoint{5.095420in}{4.525453in}}{\pgfqpoint{0.824468in}{0.243158in}}%
\pgfusepath{clip}%
\pgfsetbuttcap%
\pgfsetmiterjoin%
\definecolor{currentfill}{rgb}{0.121569,0.466667,0.705882}%
\pgfsetfillcolor{currentfill}%
\pgfsetfillopacity{0.500000}%
\pgfsetlinewidth{1.003750pt}%
\definecolor{currentstroke}{rgb}{0.000000,0.000000,0.000000}%
\pgfsetstrokecolor{currentstroke}%
\pgfsetdash{}{0pt}%
\pgfpathmoveto{\pgfqpoint{5.282799in}{4.525453in}}%
\pgfpathlineto{\pgfqpoint{5.432702in}{4.525453in}}%
\pgfpathlineto{\pgfqpoint{5.432702in}{4.527034in}}%
\pgfpathlineto{\pgfqpoint{5.282799in}{4.527034in}}%
\pgfpathlineto{\pgfqpoint{5.282799in}{4.525453in}}%
\pgfpathclose%
\pgfusepath{stroke,fill}%
\end{pgfscope}%
\begin{pgfscope}%
\pgfpathrectangle{\pgfqpoint{5.095420in}{4.525453in}}{\pgfqpoint{0.824468in}{0.243158in}}%
\pgfusepath{clip}%
\pgfsetbuttcap%
\pgfsetmiterjoin%
\definecolor{currentfill}{rgb}{0.121569,0.466667,0.705882}%
\pgfsetfillcolor{currentfill}%
\pgfsetfillopacity{0.500000}%
\pgfsetlinewidth{1.003750pt}%
\definecolor{currentstroke}{rgb}{0.000000,0.000000,0.000000}%
\pgfsetstrokecolor{currentstroke}%
\pgfsetdash{}{0pt}%
\pgfpathmoveto{\pgfqpoint{5.432702in}{4.525453in}}%
\pgfpathlineto{\pgfqpoint{5.582605in}{4.525453in}}%
\pgfpathlineto{\pgfqpoint{5.582605in}{4.526304in}}%
\pgfpathlineto{\pgfqpoint{5.432702in}{4.526304in}}%
\pgfpathlineto{\pgfqpoint{5.432702in}{4.525453in}}%
\pgfpathclose%
\pgfusepath{stroke,fill}%
\end{pgfscope}%
\begin{pgfscope}%
\pgfpathrectangle{\pgfqpoint{5.095420in}{4.525453in}}{\pgfqpoint{0.824468in}{0.243158in}}%
\pgfusepath{clip}%
\pgfsetbuttcap%
\pgfsetmiterjoin%
\definecolor{currentfill}{rgb}{0.121569,0.466667,0.705882}%
\pgfsetfillcolor{currentfill}%
\pgfsetfillopacity{0.500000}%
\pgfsetlinewidth{1.003750pt}%
\definecolor{currentstroke}{rgb}{0.000000,0.000000,0.000000}%
\pgfsetstrokecolor{currentstroke}%
\pgfsetdash{}{0pt}%
\pgfpathmoveto{\pgfqpoint{5.582605in}{4.525453in}}%
\pgfpathlineto{\pgfqpoint{5.732509in}{4.525453in}}%
\pgfpathlineto{\pgfqpoint{5.732509in}{4.526791in}}%
\pgfpathlineto{\pgfqpoint{5.582605in}{4.526791in}}%
\pgfpathlineto{\pgfqpoint{5.582605in}{4.525453in}}%
\pgfpathclose%
\pgfusepath{stroke,fill}%
\end{pgfscope}%
\begin{pgfscope}%
\pgfpathrectangle{\pgfqpoint{5.095420in}{4.525453in}}{\pgfqpoint{0.824468in}{0.243158in}}%
\pgfusepath{clip}%
\pgfsetbuttcap%
\pgfsetmiterjoin%
\definecolor{currentfill}{rgb}{0.121569,0.466667,0.705882}%
\pgfsetfillcolor{currentfill}%
\pgfsetfillopacity{0.500000}%
\pgfsetlinewidth{1.003750pt}%
\definecolor{currentstroke}{rgb}{0.000000,0.000000,0.000000}%
\pgfsetstrokecolor{currentstroke}%
\pgfsetdash{}{0pt}%
\pgfpathmoveto{\pgfqpoint{5.732509in}{4.525453in}}%
\pgfpathlineto{\pgfqpoint{5.882412in}{4.525453in}}%
\pgfpathlineto{\pgfqpoint{5.882412in}{4.526304in}}%
\pgfpathlineto{\pgfqpoint{5.732509in}{4.526304in}}%
\pgfpathlineto{\pgfqpoint{5.732509in}{4.525453in}}%
\pgfpathclose%
\pgfusepath{stroke,fill}%
\end{pgfscope}%
\begin{pgfscope}%
\pgfsetrectcap%
\pgfsetmiterjoin%
\pgfsetlinewidth{0.803000pt}%
\definecolor{currentstroke}{rgb}{0.000000,0.000000,0.000000}%
\pgfsetstrokecolor{currentstroke}%
\pgfsetdash{}{0pt}%
\pgfpathmoveto{\pgfqpoint{5.095420in}{4.525453in}}%
\pgfpathlineto{\pgfqpoint{5.095420in}{4.768611in}}%
\pgfusepath{stroke}%
\end{pgfscope}%
\begin{pgfscope}%
\pgfsetrectcap%
\pgfsetmiterjoin%
\pgfsetlinewidth{0.803000pt}%
\definecolor{currentstroke}{rgb}{0.000000,0.000000,0.000000}%
\pgfsetstrokecolor{currentstroke}%
\pgfsetdash{}{0pt}%
\pgfpathmoveto{\pgfqpoint{5.919888in}{4.525453in}}%
\pgfpathlineto{\pgfqpoint{5.919888in}{4.768611in}}%
\pgfusepath{stroke}%
\end{pgfscope}%
\begin{pgfscope}%
\pgfsetrectcap%
\pgfsetmiterjoin%
\pgfsetlinewidth{0.803000pt}%
\definecolor{currentstroke}{rgb}{0.000000,0.000000,0.000000}%
\pgfsetstrokecolor{currentstroke}%
\pgfsetdash{}{0pt}%
\pgfpathmoveto{\pgfqpoint{5.095420in}{4.525453in}}%
\pgfpathlineto{\pgfqpoint{5.919888in}{4.525453in}}%
\pgfusepath{stroke}%
\end{pgfscope}%
\begin{pgfscope}%
\pgfsetrectcap%
\pgfsetmiterjoin%
\pgfsetlinewidth{0.803000pt}%
\definecolor{currentstroke}{rgb}{0.000000,0.000000,0.000000}%
\pgfsetstrokecolor{currentstroke}%
\pgfsetdash{}{0pt}%
\pgfpathmoveto{\pgfqpoint{5.095420in}{4.768611in}}%
\pgfpathlineto{\pgfqpoint{5.919888in}{4.768611in}}%
\pgfusepath{stroke}%
\end{pgfscope}%
\begin{pgfscope}%
\definecolor{textcolor}{rgb}{0.000000,0.000000,0.000000}%
\pgfsetstrokecolor{textcolor}%
\pgfsetfillcolor{textcolor}%
\pgftext[x=5.507654in,y=4.851944in,,base]{\color{textcolor}\rmfamily\fontsize{11.000000}{13.200000}\selectfont SantéVet}%
\end{pgfscope}%
\begin{pgfscope}%
\pgfsetbuttcap%
\pgfsetmiterjoin%
\definecolor{currentfill}{rgb}{1.000000,1.000000,1.000000}%
\pgfsetfillcolor{currentfill}%
\pgfsetlinewidth{0.000000pt}%
\definecolor{currentstroke}{rgb}{0.000000,0.000000,0.000000}%
\pgfsetstrokecolor{currentstroke}%
\pgfsetstrokeopacity{0.000000}%
\pgfsetdash{}{0pt}%
\pgfpathmoveto{\pgfqpoint{6.084781in}{4.525453in}}%
\pgfpathlineto{\pgfqpoint{6.909249in}{4.525453in}}%
\pgfpathlineto{\pgfqpoint{6.909249in}{4.768611in}}%
\pgfpathlineto{\pgfqpoint{6.084781in}{4.768611in}}%
\pgfpathlineto{\pgfqpoint{6.084781in}{4.525453in}}%
\pgfpathclose%
\pgfusepath{fill}%
\end{pgfscope}%
\begin{pgfscope}%
\pgfpathrectangle{\pgfqpoint{6.084781in}{4.525453in}}{\pgfqpoint{0.824468in}{0.243158in}}%
\pgfusepath{clip}%
\pgfsetbuttcap%
\pgfsetmiterjoin%
\definecolor{currentfill}{rgb}{0.121569,0.466667,0.705882}%
\pgfsetfillcolor{currentfill}%
\pgfsetfillopacity{0.500000}%
\pgfsetlinewidth{1.003750pt}%
\definecolor{currentstroke}{rgb}{0.000000,0.000000,0.000000}%
\pgfsetstrokecolor{currentstroke}%
\pgfsetdash{}{0pt}%
\pgfpathmoveto{\pgfqpoint{6.122257in}{4.525453in}}%
\pgfpathlineto{\pgfqpoint{6.272160in}{4.525453in}}%
\pgfpathlineto{\pgfqpoint{6.272160in}{4.550620in}}%
\pgfpathlineto{\pgfqpoint{6.122257in}{4.550620in}}%
\pgfpathlineto{\pgfqpoint{6.122257in}{4.525453in}}%
\pgfpathclose%
\pgfusepath{stroke,fill}%
\end{pgfscope}%
\begin{pgfscope}%
\pgfpathrectangle{\pgfqpoint{6.084781in}{4.525453in}}{\pgfqpoint{0.824468in}{0.243158in}}%
\pgfusepath{clip}%
\pgfsetbuttcap%
\pgfsetmiterjoin%
\definecolor{currentfill}{rgb}{0.121569,0.466667,0.705882}%
\pgfsetfillcolor{currentfill}%
\pgfsetfillopacity{0.500000}%
\pgfsetlinewidth{1.003750pt}%
\definecolor{currentstroke}{rgb}{0.000000,0.000000,0.000000}%
\pgfsetstrokecolor{currentstroke}%
\pgfsetdash{}{0pt}%
\pgfpathmoveto{\pgfqpoint{6.272160in}{4.525453in}}%
\pgfpathlineto{\pgfqpoint{6.422064in}{4.525453in}}%
\pgfpathlineto{\pgfqpoint{6.422064in}{4.531046in}}%
\pgfpathlineto{\pgfqpoint{6.272160in}{4.531046in}}%
\pgfpathlineto{\pgfqpoint{6.272160in}{4.525453in}}%
\pgfpathclose%
\pgfusepath{stroke,fill}%
\end{pgfscope}%
\begin{pgfscope}%
\pgfpathrectangle{\pgfqpoint{6.084781in}{4.525453in}}{\pgfqpoint{0.824468in}{0.243158in}}%
\pgfusepath{clip}%
\pgfsetbuttcap%
\pgfsetmiterjoin%
\definecolor{currentfill}{rgb}{0.121569,0.466667,0.705882}%
\pgfsetfillcolor{currentfill}%
\pgfsetfillopacity{0.500000}%
\pgfsetlinewidth{1.003750pt}%
\definecolor{currentstroke}{rgb}{0.000000,0.000000,0.000000}%
\pgfsetstrokecolor{currentstroke}%
\pgfsetdash{}{0pt}%
\pgfpathmoveto{\pgfqpoint{6.422064in}{4.525453in}}%
\pgfpathlineto{\pgfqpoint{6.571967in}{4.525453in}}%
\pgfpathlineto{\pgfqpoint{6.571967in}{4.526669in}}%
\pgfpathlineto{\pgfqpoint{6.422064in}{4.526669in}}%
\pgfpathlineto{\pgfqpoint{6.422064in}{4.525453in}}%
\pgfpathclose%
\pgfusepath{stroke,fill}%
\end{pgfscope}%
\begin{pgfscope}%
\pgfpathrectangle{\pgfqpoint{6.084781in}{4.525453in}}{\pgfqpoint{0.824468in}{0.243158in}}%
\pgfusepath{clip}%
\pgfsetbuttcap%
\pgfsetmiterjoin%
\definecolor{currentfill}{rgb}{0.121569,0.466667,0.705882}%
\pgfsetfillcolor{currentfill}%
\pgfsetfillopacity{0.500000}%
\pgfsetlinewidth{1.003750pt}%
\definecolor{currentstroke}{rgb}{0.000000,0.000000,0.000000}%
\pgfsetstrokecolor{currentstroke}%
\pgfsetdash{}{0pt}%
\pgfpathmoveto{\pgfqpoint{6.571967in}{4.525453in}}%
\pgfpathlineto{\pgfqpoint{6.721870in}{4.525453in}}%
\pgfpathlineto{\pgfqpoint{6.721870in}{4.525818in}}%
\pgfpathlineto{\pgfqpoint{6.571967in}{4.525818in}}%
\pgfpathlineto{\pgfqpoint{6.571967in}{4.525453in}}%
\pgfpathclose%
\pgfusepath{stroke,fill}%
\end{pgfscope}%
\begin{pgfscope}%
\pgfpathrectangle{\pgfqpoint{6.084781in}{4.525453in}}{\pgfqpoint{0.824468in}{0.243158in}}%
\pgfusepath{clip}%
\pgfsetbuttcap%
\pgfsetmiterjoin%
\definecolor{currentfill}{rgb}{0.121569,0.466667,0.705882}%
\pgfsetfillcolor{currentfill}%
\pgfsetfillopacity{0.500000}%
\pgfsetlinewidth{1.003750pt}%
\definecolor{currentstroke}{rgb}{0.000000,0.000000,0.000000}%
\pgfsetstrokecolor{currentstroke}%
\pgfsetdash{}{0pt}%
\pgfpathmoveto{\pgfqpoint{6.721870in}{4.525453in}}%
\pgfpathlineto{\pgfqpoint{6.871774in}{4.525453in}}%
\pgfpathlineto{\pgfqpoint{6.871774in}{4.525575in}}%
\pgfpathlineto{\pgfqpoint{6.721870in}{4.525575in}}%
\pgfpathlineto{\pgfqpoint{6.721870in}{4.525453in}}%
\pgfpathclose%
\pgfusepath{stroke,fill}%
\end{pgfscope}%
\begin{pgfscope}%
\pgfsetrectcap%
\pgfsetmiterjoin%
\pgfsetlinewidth{0.803000pt}%
\definecolor{currentstroke}{rgb}{0.000000,0.000000,0.000000}%
\pgfsetstrokecolor{currentstroke}%
\pgfsetdash{}{0pt}%
\pgfpathmoveto{\pgfqpoint{6.084781in}{4.525453in}}%
\pgfpathlineto{\pgfqpoint{6.084781in}{4.768611in}}%
\pgfusepath{stroke}%
\end{pgfscope}%
\begin{pgfscope}%
\pgfsetrectcap%
\pgfsetmiterjoin%
\pgfsetlinewidth{0.803000pt}%
\definecolor{currentstroke}{rgb}{0.000000,0.000000,0.000000}%
\pgfsetstrokecolor{currentstroke}%
\pgfsetdash{}{0pt}%
\pgfpathmoveto{\pgfqpoint{6.909249in}{4.525453in}}%
\pgfpathlineto{\pgfqpoint{6.909249in}{4.768611in}}%
\pgfusepath{stroke}%
\end{pgfscope}%
\begin{pgfscope}%
\pgfsetrectcap%
\pgfsetmiterjoin%
\pgfsetlinewidth{0.803000pt}%
\definecolor{currentstroke}{rgb}{0.000000,0.000000,0.000000}%
\pgfsetstrokecolor{currentstroke}%
\pgfsetdash{}{0pt}%
\pgfpathmoveto{\pgfqpoint{6.084781in}{4.525453in}}%
\pgfpathlineto{\pgfqpoint{6.909249in}{4.525453in}}%
\pgfusepath{stroke}%
\end{pgfscope}%
\begin{pgfscope}%
\pgfsetrectcap%
\pgfsetmiterjoin%
\pgfsetlinewidth{0.803000pt}%
\definecolor{currentstroke}{rgb}{0.000000,0.000000,0.000000}%
\pgfsetstrokecolor{currentstroke}%
\pgfsetdash{}{0pt}%
\pgfpathmoveto{\pgfqpoint{6.084781in}{4.768611in}}%
\pgfpathlineto{\pgfqpoint{6.909249in}{4.768611in}}%
\pgfusepath{stroke}%
\end{pgfscope}%
\begin{pgfscope}%
\definecolor{textcolor}{rgb}{0.000000,0.000000,0.000000}%
\pgfsetstrokecolor{textcolor}%
\pgfsetfillcolor{textcolor}%
\pgftext[x=6.497015in,y=4.851944in,,base]{\color{textcolor}\rmfamily\fontsize{11.000000}{13.200000}\selectfont Mercer}%
\end{pgfscope}%
\begin{pgfscope}%
\pgfsetbuttcap%
\pgfsetmiterjoin%
\definecolor{currentfill}{rgb}{1.000000,1.000000,1.000000}%
\pgfsetfillcolor{currentfill}%
\pgfsetlinewidth{0.000000pt}%
\definecolor{currentstroke}{rgb}{0.000000,0.000000,0.000000}%
\pgfsetstrokecolor{currentstroke}%
\pgfsetstrokeopacity{0.000000}%
\pgfsetdash{}{0pt}%
\pgfpathmoveto{\pgfqpoint{7.074143in}{4.525453in}}%
\pgfpathlineto{\pgfqpoint{7.898611in}{4.525453in}}%
\pgfpathlineto{\pgfqpoint{7.898611in}{4.768611in}}%
\pgfpathlineto{\pgfqpoint{7.074143in}{4.768611in}}%
\pgfpathlineto{\pgfqpoint{7.074143in}{4.525453in}}%
\pgfpathclose%
\pgfusepath{fill}%
\end{pgfscope}%
\begin{pgfscope}%
\pgfpathrectangle{\pgfqpoint{7.074143in}{4.525453in}}{\pgfqpoint{0.824468in}{0.243158in}}%
\pgfusepath{clip}%
\pgfsetbuttcap%
\pgfsetmiterjoin%
\definecolor{currentfill}{rgb}{0.121569,0.466667,0.705882}%
\pgfsetfillcolor{currentfill}%
\pgfsetfillopacity{0.500000}%
\pgfsetlinewidth{1.003750pt}%
\definecolor{currentstroke}{rgb}{0.000000,0.000000,0.000000}%
\pgfsetstrokecolor{currentstroke}%
\pgfsetdash{}{0pt}%
\pgfpathmoveto{\pgfqpoint{7.111619in}{4.525453in}}%
\pgfpathlineto{\pgfqpoint{7.261522in}{4.525453in}}%
\pgfpathlineto{\pgfqpoint{7.261522in}{4.536517in}}%
\pgfpathlineto{\pgfqpoint{7.111619in}{4.536517in}}%
\pgfpathlineto{\pgfqpoint{7.111619in}{4.525453in}}%
\pgfpathclose%
\pgfusepath{stroke,fill}%
\end{pgfscope}%
\begin{pgfscope}%
\pgfpathrectangle{\pgfqpoint{7.074143in}{4.525453in}}{\pgfqpoint{0.824468in}{0.243158in}}%
\pgfusepath{clip}%
\pgfsetbuttcap%
\pgfsetmiterjoin%
\definecolor{currentfill}{rgb}{0.121569,0.466667,0.705882}%
\pgfsetfillcolor{currentfill}%
\pgfsetfillopacity{0.500000}%
\pgfsetlinewidth{1.003750pt}%
\definecolor{currentstroke}{rgb}{0.000000,0.000000,0.000000}%
\pgfsetstrokecolor{currentstroke}%
\pgfsetdash{}{0pt}%
\pgfpathmoveto{\pgfqpoint{7.261522in}{4.525453in}}%
\pgfpathlineto{\pgfqpoint{7.411425in}{4.525453in}}%
\pgfpathlineto{\pgfqpoint{7.411425in}{4.527155in}}%
\pgfpathlineto{\pgfqpoint{7.261522in}{4.527155in}}%
\pgfpathlineto{\pgfqpoint{7.261522in}{4.525453in}}%
\pgfpathclose%
\pgfusepath{stroke,fill}%
\end{pgfscope}%
\begin{pgfscope}%
\pgfpathrectangle{\pgfqpoint{7.074143in}{4.525453in}}{\pgfqpoint{0.824468in}{0.243158in}}%
\pgfusepath{clip}%
\pgfsetbuttcap%
\pgfsetmiterjoin%
\definecolor{currentfill}{rgb}{0.121569,0.466667,0.705882}%
\pgfsetfillcolor{currentfill}%
\pgfsetfillopacity{0.500000}%
\pgfsetlinewidth{1.003750pt}%
\definecolor{currentstroke}{rgb}{0.000000,0.000000,0.000000}%
\pgfsetstrokecolor{currentstroke}%
\pgfsetdash{}{0pt}%
\pgfpathmoveto{\pgfqpoint{7.411425in}{4.525453in}}%
\pgfpathlineto{\pgfqpoint{7.561329in}{4.525453in}}%
\pgfpathlineto{\pgfqpoint{7.561329in}{4.527642in}}%
\pgfpathlineto{\pgfqpoint{7.411425in}{4.527642in}}%
\pgfpathlineto{\pgfqpoint{7.411425in}{4.525453in}}%
\pgfpathclose%
\pgfusepath{stroke,fill}%
\end{pgfscope}%
\begin{pgfscope}%
\pgfpathrectangle{\pgfqpoint{7.074143in}{4.525453in}}{\pgfqpoint{0.824468in}{0.243158in}}%
\pgfusepath{clip}%
\pgfsetbuttcap%
\pgfsetmiterjoin%
\definecolor{currentfill}{rgb}{0.121569,0.466667,0.705882}%
\pgfsetfillcolor{currentfill}%
\pgfsetfillopacity{0.500000}%
\pgfsetlinewidth{1.003750pt}%
\definecolor{currentstroke}{rgb}{0.000000,0.000000,0.000000}%
\pgfsetstrokecolor{currentstroke}%
\pgfsetdash{}{0pt}%
\pgfpathmoveto{\pgfqpoint{7.561329in}{4.525453in}}%
\pgfpathlineto{\pgfqpoint{7.711232in}{4.525453in}}%
\pgfpathlineto{\pgfqpoint{7.711232in}{4.525575in}}%
\pgfpathlineto{\pgfqpoint{7.561329in}{4.525575in}}%
\pgfpathlineto{\pgfqpoint{7.561329in}{4.525453in}}%
\pgfpathclose%
\pgfusepath{stroke,fill}%
\end{pgfscope}%
\begin{pgfscope}%
\pgfpathrectangle{\pgfqpoint{7.074143in}{4.525453in}}{\pgfqpoint{0.824468in}{0.243158in}}%
\pgfusepath{clip}%
\pgfsetbuttcap%
\pgfsetmiterjoin%
\definecolor{currentfill}{rgb}{0.121569,0.466667,0.705882}%
\pgfsetfillcolor{currentfill}%
\pgfsetfillopacity{0.500000}%
\pgfsetlinewidth{1.003750pt}%
\definecolor{currentstroke}{rgb}{0.000000,0.000000,0.000000}%
\pgfsetstrokecolor{currentstroke}%
\pgfsetdash{}{0pt}%
\pgfpathmoveto{\pgfqpoint{7.711232in}{4.525453in}}%
\pgfpathlineto{\pgfqpoint{7.861135in}{4.525453in}}%
\pgfpathlineto{\pgfqpoint{7.861135in}{4.525940in}}%
\pgfpathlineto{\pgfqpoint{7.711232in}{4.525940in}}%
\pgfpathlineto{\pgfqpoint{7.711232in}{4.525453in}}%
\pgfpathclose%
\pgfusepath{stroke,fill}%
\end{pgfscope}%
\begin{pgfscope}%
\pgfsetrectcap%
\pgfsetmiterjoin%
\pgfsetlinewidth{0.803000pt}%
\definecolor{currentstroke}{rgb}{0.000000,0.000000,0.000000}%
\pgfsetstrokecolor{currentstroke}%
\pgfsetdash{}{0pt}%
\pgfpathmoveto{\pgfqpoint{7.074143in}{4.525453in}}%
\pgfpathlineto{\pgfqpoint{7.074143in}{4.768611in}}%
\pgfusepath{stroke}%
\end{pgfscope}%
\begin{pgfscope}%
\pgfsetrectcap%
\pgfsetmiterjoin%
\pgfsetlinewidth{0.803000pt}%
\definecolor{currentstroke}{rgb}{0.000000,0.000000,0.000000}%
\pgfsetstrokecolor{currentstroke}%
\pgfsetdash{}{0pt}%
\pgfpathmoveto{\pgfqpoint{7.898611in}{4.525453in}}%
\pgfpathlineto{\pgfqpoint{7.898611in}{4.768611in}}%
\pgfusepath{stroke}%
\end{pgfscope}%
\begin{pgfscope}%
\pgfsetrectcap%
\pgfsetmiterjoin%
\pgfsetlinewidth{0.803000pt}%
\definecolor{currentstroke}{rgb}{0.000000,0.000000,0.000000}%
\pgfsetstrokecolor{currentstroke}%
\pgfsetdash{}{0pt}%
\pgfpathmoveto{\pgfqpoint{7.074143in}{4.525453in}}%
\pgfpathlineto{\pgfqpoint{7.898611in}{4.525453in}}%
\pgfusepath{stroke}%
\end{pgfscope}%
\begin{pgfscope}%
\pgfsetrectcap%
\pgfsetmiterjoin%
\pgfsetlinewidth{0.803000pt}%
\definecolor{currentstroke}{rgb}{0.000000,0.000000,0.000000}%
\pgfsetstrokecolor{currentstroke}%
\pgfsetdash{}{0pt}%
\pgfpathmoveto{\pgfqpoint{7.074143in}{4.768611in}}%
\pgfpathlineto{\pgfqpoint{7.898611in}{4.768611in}}%
\pgfusepath{stroke}%
\end{pgfscope}%
\begin{pgfscope}%
\definecolor{textcolor}{rgb}{0.000000,0.000000,0.000000}%
\pgfsetstrokecolor{textcolor}%
\pgfsetfillcolor{textcolor}%
\pgftext[x=7.486377in,y=4.851944in,,base]{\color{textcolor}\rmfamily\fontsize{11.000000}{13.200000}\selectfont Generali}%
\end{pgfscope}%
\begin{pgfscope}%
\pgfsetbuttcap%
\pgfsetmiterjoin%
\definecolor{currentfill}{rgb}{1.000000,1.000000,1.000000}%
\pgfsetfillcolor{currentfill}%
\pgfsetlinewidth{0.000000pt}%
\definecolor{currentstroke}{rgb}{0.000000,0.000000,0.000000}%
\pgfsetstrokecolor{currentstroke}%
\pgfsetstrokeopacity{0.000000}%
\pgfsetdash{}{0pt}%
\pgfpathmoveto{\pgfqpoint{0.148611in}{3.795980in}}%
\pgfpathlineto{\pgfqpoint{0.973079in}{3.795980in}}%
\pgfpathlineto{\pgfqpoint{0.973079in}{4.039137in}}%
\pgfpathlineto{\pgfqpoint{0.148611in}{4.039137in}}%
\pgfpathlineto{\pgfqpoint{0.148611in}{3.795980in}}%
\pgfpathclose%
\pgfusepath{fill}%
\end{pgfscope}%
\begin{pgfscope}%
\pgfpathrectangle{\pgfqpoint{0.148611in}{3.795980in}}{\pgfqpoint{0.824468in}{0.243158in}}%
\pgfusepath{clip}%
\pgfsetbuttcap%
\pgfsetmiterjoin%
\definecolor{currentfill}{rgb}{0.121569,0.466667,0.705882}%
\pgfsetfillcolor{currentfill}%
\pgfsetfillopacity{0.500000}%
\pgfsetlinewidth{1.003750pt}%
\definecolor{currentstroke}{rgb}{0.000000,0.000000,0.000000}%
\pgfsetstrokecolor{currentstroke}%
\pgfsetdash{}{0pt}%
\pgfpathmoveto{\pgfqpoint{0.186087in}{3.795980in}}%
\pgfpathlineto{\pgfqpoint{0.335990in}{3.795980in}}%
\pgfpathlineto{\pgfqpoint{0.335990in}{3.842544in}}%
\pgfpathlineto{\pgfqpoint{0.186087in}{3.842544in}}%
\pgfpathlineto{\pgfqpoint{0.186087in}{3.795980in}}%
\pgfpathclose%
\pgfusepath{stroke,fill}%
\end{pgfscope}%
\begin{pgfscope}%
\pgfpathrectangle{\pgfqpoint{0.148611in}{3.795980in}}{\pgfqpoint{0.824468in}{0.243158in}}%
\pgfusepath{clip}%
\pgfsetbuttcap%
\pgfsetmiterjoin%
\definecolor{currentfill}{rgb}{0.121569,0.466667,0.705882}%
\pgfsetfillcolor{currentfill}%
\pgfsetfillopacity{0.500000}%
\pgfsetlinewidth{1.003750pt}%
\definecolor{currentstroke}{rgb}{0.000000,0.000000,0.000000}%
\pgfsetstrokecolor{currentstroke}%
\pgfsetdash{}{0pt}%
\pgfpathmoveto{\pgfqpoint{0.335990in}{3.795980in}}%
\pgfpathlineto{\pgfqpoint{0.485894in}{3.795980in}}%
\pgfpathlineto{\pgfqpoint{0.485894in}{3.809596in}}%
\pgfpathlineto{\pgfqpoint{0.335990in}{3.809596in}}%
\pgfpathlineto{\pgfqpoint{0.335990in}{3.795980in}}%
\pgfpathclose%
\pgfusepath{stroke,fill}%
\end{pgfscope}%
\begin{pgfscope}%
\pgfpathrectangle{\pgfqpoint{0.148611in}{3.795980in}}{\pgfqpoint{0.824468in}{0.243158in}}%
\pgfusepath{clip}%
\pgfsetbuttcap%
\pgfsetmiterjoin%
\definecolor{currentfill}{rgb}{0.121569,0.466667,0.705882}%
\pgfsetfillcolor{currentfill}%
\pgfsetfillopacity{0.500000}%
\pgfsetlinewidth{1.003750pt}%
\definecolor{currentstroke}{rgb}{0.000000,0.000000,0.000000}%
\pgfsetstrokecolor{currentstroke}%
\pgfsetdash{}{0pt}%
\pgfpathmoveto{\pgfqpoint{0.485894in}{3.795980in}}%
\pgfpathlineto{\pgfqpoint{0.635797in}{3.795980in}}%
\pgfpathlineto{\pgfqpoint{0.635797in}{3.802180in}}%
\pgfpathlineto{\pgfqpoint{0.485894in}{3.802180in}}%
\pgfpathlineto{\pgfqpoint{0.485894in}{3.795980in}}%
\pgfpathclose%
\pgfusepath{stroke,fill}%
\end{pgfscope}%
\begin{pgfscope}%
\pgfpathrectangle{\pgfqpoint{0.148611in}{3.795980in}}{\pgfqpoint{0.824468in}{0.243158in}}%
\pgfusepath{clip}%
\pgfsetbuttcap%
\pgfsetmiterjoin%
\definecolor{currentfill}{rgb}{0.121569,0.466667,0.705882}%
\pgfsetfillcolor{currentfill}%
\pgfsetfillopacity{0.500000}%
\pgfsetlinewidth{1.003750pt}%
\definecolor{currentstroke}{rgb}{0.000000,0.000000,0.000000}%
\pgfsetstrokecolor{currentstroke}%
\pgfsetdash{}{0pt}%
\pgfpathmoveto{\pgfqpoint{0.635797in}{3.795980in}}%
\pgfpathlineto{\pgfqpoint{0.785700in}{3.795980in}}%
\pgfpathlineto{\pgfqpoint{0.785700in}{3.798168in}}%
\pgfpathlineto{\pgfqpoint{0.635797in}{3.798168in}}%
\pgfpathlineto{\pgfqpoint{0.635797in}{3.795980in}}%
\pgfpathclose%
\pgfusepath{stroke,fill}%
\end{pgfscope}%
\begin{pgfscope}%
\pgfpathrectangle{\pgfqpoint{0.148611in}{3.795980in}}{\pgfqpoint{0.824468in}{0.243158in}}%
\pgfusepath{clip}%
\pgfsetbuttcap%
\pgfsetmiterjoin%
\definecolor{currentfill}{rgb}{0.121569,0.466667,0.705882}%
\pgfsetfillcolor{currentfill}%
\pgfsetfillopacity{0.500000}%
\pgfsetlinewidth{1.003750pt}%
\definecolor{currentstroke}{rgb}{0.000000,0.000000,0.000000}%
\pgfsetstrokecolor{currentstroke}%
\pgfsetdash{}{0pt}%
\pgfpathmoveto{\pgfqpoint{0.785700in}{3.795980in}}%
\pgfpathlineto{\pgfqpoint{0.935603in}{3.795980in}}%
\pgfpathlineto{\pgfqpoint{0.935603in}{3.797195in}}%
\pgfpathlineto{\pgfqpoint{0.785700in}{3.797195in}}%
\pgfpathlineto{\pgfqpoint{0.785700in}{3.795980in}}%
\pgfpathclose%
\pgfusepath{stroke,fill}%
\end{pgfscope}%
\begin{pgfscope}%
\pgfsetrectcap%
\pgfsetmiterjoin%
\pgfsetlinewidth{0.803000pt}%
\definecolor{currentstroke}{rgb}{0.000000,0.000000,0.000000}%
\pgfsetstrokecolor{currentstroke}%
\pgfsetdash{}{0pt}%
\pgfpathmoveto{\pgfqpoint{0.148611in}{3.795980in}}%
\pgfpathlineto{\pgfqpoint{0.148611in}{4.039137in}}%
\pgfusepath{stroke}%
\end{pgfscope}%
\begin{pgfscope}%
\pgfsetrectcap%
\pgfsetmiterjoin%
\pgfsetlinewidth{0.803000pt}%
\definecolor{currentstroke}{rgb}{0.000000,0.000000,0.000000}%
\pgfsetstrokecolor{currentstroke}%
\pgfsetdash{}{0pt}%
\pgfpathmoveto{\pgfqpoint{0.973079in}{3.795980in}}%
\pgfpathlineto{\pgfqpoint{0.973079in}{4.039137in}}%
\pgfusepath{stroke}%
\end{pgfscope}%
\begin{pgfscope}%
\pgfsetrectcap%
\pgfsetmiterjoin%
\pgfsetlinewidth{0.803000pt}%
\definecolor{currentstroke}{rgb}{0.000000,0.000000,0.000000}%
\pgfsetstrokecolor{currentstroke}%
\pgfsetdash{}{0pt}%
\pgfpathmoveto{\pgfqpoint{0.148611in}{3.795980in}}%
\pgfpathlineto{\pgfqpoint{0.973079in}{3.795980in}}%
\pgfusepath{stroke}%
\end{pgfscope}%
\begin{pgfscope}%
\pgfsetrectcap%
\pgfsetmiterjoin%
\pgfsetlinewidth{0.803000pt}%
\definecolor{currentstroke}{rgb}{0.000000,0.000000,0.000000}%
\pgfsetstrokecolor{currentstroke}%
\pgfsetdash{}{0pt}%
\pgfpathmoveto{\pgfqpoint{0.148611in}{4.039137in}}%
\pgfpathlineto{\pgfqpoint{0.973079in}{4.039137in}}%
\pgfusepath{stroke}%
\end{pgfscope}%
\begin{pgfscope}%
\definecolor{textcolor}{rgb}{0.000000,0.000000,0.000000}%
\pgfsetstrokecolor{textcolor}%
\pgfsetfillcolor{textcolor}%
\pgftext[x=0.560845in,y=4.122471in,,base]{\color{textcolor}\rmfamily\fontsize{11.000000}{13.200000}\selectfont Allianz}%
\end{pgfscope}%
\begin{pgfscope}%
\pgfsetbuttcap%
\pgfsetmiterjoin%
\definecolor{currentfill}{rgb}{1.000000,1.000000,1.000000}%
\pgfsetfillcolor{currentfill}%
\pgfsetlinewidth{0.000000pt}%
\definecolor{currentstroke}{rgb}{0.000000,0.000000,0.000000}%
\pgfsetstrokecolor{currentstroke}%
\pgfsetstrokeopacity{0.000000}%
\pgfsetdash{}{0pt}%
\pgfpathmoveto{\pgfqpoint{1.137973in}{3.795980in}}%
\pgfpathlineto{\pgfqpoint{1.962441in}{3.795980in}}%
\pgfpathlineto{\pgfqpoint{1.962441in}{4.039137in}}%
\pgfpathlineto{\pgfqpoint{1.137973in}{4.039137in}}%
\pgfpathlineto{\pgfqpoint{1.137973in}{3.795980in}}%
\pgfpathclose%
\pgfusepath{fill}%
\end{pgfscope}%
\begin{pgfscope}%
\pgfpathrectangle{\pgfqpoint{1.137973in}{3.795980in}}{\pgfqpoint{0.824468in}{0.243158in}}%
\pgfusepath{clip}%
\pgfsetbuttcap%
\pgfsetmiterjoin%
\definecolor{currentfill}{rgb}{0.121569,0.466667,0.705882}%
\pgfsetfillcolor{currentfill}%
\pgfsetfillopacity{0.500000}%
\pgfsetlinewidth{1.003750pt}%
\definecolor{currentstroke}{rgb}{0.000000,0.000000,0.000000}%
\pgfsetstrokecolor{currentstroke}%
\pgfsetdash{}{0pt}%
\pgfpathmoveto{\pgfqpoint{1.175449in}{3.795980in}}%
\pgfpathlineto{\pgfqpoint{1.325352in}{3.795980in}}%
\pgfpathlineto{\pgfqpoint{1.325352in}{3.802788in}}%
\pgfpathlineto{\pgfqpoint{1.175449in}{3.802788in}}%
\pgfpathlineto{\pgfqpoint{1.175449in}{3.795980in}}%
\pgfpathclose%
\pgfusepath{stroke,fill}%
\end{pgfscope}%
\begin{pgfscope}%
\pgfpathrectangle{\pgfqpoint{1.137973in}{3.795980in}}{\pgfqpoint{0.824468in}{0.243158in}}%
\pgfusepath{clip}%
\pgfsetbuttcap%
\pgfsetmiterjoin%
\definecolor{currentfill}{rgb}{0.121569,0.466667,0.705882}%
\pgfsetfillcolor{currentfill}%
\pgfsetfillopacity{0.500000}%
\pgfsetlinewidth{1.003750pt}%
\definecolor{currentstroke}{rgb}{0.000000,0.000000,0.000000}%
\pgfsetstrokecolor{currentstroke}%
\pgfsetdash{}{0pt}%
\pgfpathmoveto{\pgfqpoint{1.325352in}{3.795980in}}%
\pgfpathlineto{\pgfqpoint{1.475255in}{3.795980in}}%
\pgfpathlineto{\pgfqpoint{1.475255in}{3.804490in}}%
\pgfpathlineto{\pgfqpoint{1.325352in}{3.804490in}}%
\pgfpathlineto{\pgfqpoint{1.325352in}{3.795980in}}%
\pgfpathclose%
\pgfusepath{stroke,fill}%
\end{pgfscope}%
\begin{pgfscope}%
\pgfpathrectangle{\pgfqpoint{1.137973in}{3.795980in}}{\pgfqpoint{0.824468in}{0.243158in}}%
\pgfusepath{clip}%
\pgfsetbuttcap%
\pgfsetmiterjoin%
\definecolor{currentfill}{rgb}{0.121569,0.466667,0.705882}%
\pgfsetfillcolor{currentfill}%
\pgfsetfillopacity{0.500000}%
\pgfsetlinewidth{1.003750pt}%
\definecolor{currentstroke}{rgb}{0.000000,0.000000,0.000000}%
\pgfsetstrokecolor{currentstroke}%
\pgfsetdash{}{0pt}%
\pgfpathmoveto{\pgfqpoint{1.475255in}{3.795980in}}%
\pgfpathlineto{\pgfqpoint{1.625158in}{3.795980in}}%
\pgfpathlineto{\pgfqpoint{1.625158in}{3.814460in}}%
\pgfpathlineto{\pgfqpoint{1.475255in}{3.814460in}}%
\pgfpathlineto{\pgfqpoint{1.475255in}{3.795980in}}%
\pgfpathclose%
\pgfusepath{stroke,fill}%
\end{pgfscope}%
\begin{pgfscope}%
\pgfpathrectangle{\pgfqpoint{1.137973in}{3.795980in}}{\pgfqpoint{0.824468in}{0.243158in}}%
\pgfusepath{clip}%
\pgfsetbuttcap%
\pgfsetmiterjoin%
\definecolor{currentfill}{rgb}{0.121569,0.466667,0.705882}%
\pgfsetfillcolor{currentfill}%
\pgfsetfillopacity{0.500000}%
\pgfsetlinewidth{1.003750pt}%
\definecolor{currentstroke}{rgb}{0.000000,0.000000,0.000000}%
\pgfsetstrokecolor{currentstroke}%
\pgfsetdash{}{0pt}%
\pgfpathmoveto{\pgfqpoint{1.625158in}{3.795980in}}%
\pgfpathlineto{\pgfqpoint{1.775062in}{3.795980in}}%
\pgfpathlineto{\pgfqpoint{1.775062in}{3.837316in}}%
\pgfpathlineto{\pgfqpoint{1.625158in}{3.837316in}}%
\pgfpathlineto{\pgfqpoint{1.625158in}{3.795980in}}%
\pgfpathclose%
\pgfusepath{stroke,fill}%
\end{pgfscope}%
\begin{pgfscope}%
\pgfpathrectangle{\pgfqpoint{1.137973in}{3.795980in}}{\pgfqpoint{0.824468in}{0.243158in}}%
\pgfusepath{clip}%
\pgfsetbuttcap%
\pgfsetmiterjoin%
\definecolor{currentfill}{rgb}{0.121569,0.466667,0.705882}%
\pgfsetfillcolor{currentfill}%
\pgfsetfillopacity{0.500000}%
\pgfsetlinewidth{1.003750pt}%
\definecolor{currentstroke}{rgb}{0.000000,0.000000,0.000000}%
\pgfsetstrokecolor{currentstroke}%
\pgfsetdash{}{0pt}%
\pgfpathmoveto{\pgfqpoint{1.775062in}{3.795980in}}%
\pgfpathlineto{\pgfqpoint{1.924965in}{3.795980in}}%
\pgfpathlineto{\pgfqpoint{1.924965in}{3.845219in}}%
\pgfpathlineto{\pgfqpoint{1.775062in}{3.845219in}}%
\pgfpathlineto{\pgfqpoint{1.775062in}{3.795980in}}%
\pgfpathclose%
\pgfusepath{stroke,fill}%
\end{pgfscope}%
\begin{pgfscope}%
\pgfsetrectcap%
\pgfsetmiterjoin%
\pgfsetlinewidth{0.803000pt}%
\definecolor{currentstroke}{rgb}{0.000000,0.000000,0.000000}%
\pgfsetstrokecolor{currentstroke}%
\pgfsetdash{}{0pt}%
\pgfpathmoveto{\pgfqpoint{1.137973in}{3.795980in}}%
\pgfpathlineto{\pgfqpoint{1.137973in}{4.039137in}}%
\pgfusepath{stroke}%
\end{pgfscope}%
\begin{pgfscope}%
\pgfsetrectcap%
\pgfsetmiterjoin%
\pgfsetlinewidth{0.803000pt}%
\definecolor{currentstroke}{rgb}{0.000000,0.000000,0.000000}%
\pgfsetstrokecolor{currentstroke}%
\pgfsetdash{}{0pt}%
\pgfpathmoveto{\pgfqpoint{1.962441in}{3.795980in}}%
\pgfpathlineto{\pgfqpoint{1.962441in}{4.039137in}}%
\pgfusepath{stroke}%
\end{pgfscope}%
\begin{pgfscope}%
\pgfsetrectcap%
\pgfsetmiterjoin%
\pgfsetlinewidth{0.803000pt}%
\definecolor{currentstroke}{rgb}{0.000000,0.000000,0.000000}%
\pgfsetstrokecolor{currentstroke}%
\pgfsetdash{}{0pt}%
\pgfpathmoveto{\pgfqpoint{1.137973in}{3.795980in}}%
\pgfpathlineto{\pgfqpoint{1.962441in}{3.795980in}}%
\pgfusepath{stroke}%
\end{pgfscope}%
\begin{pgfscope}%
\pgfsetrectcap%
\pgfsetmiterjoin%
\pgfsetlinewidth{0.803000pt}%
\definecolor{currentstroke}{rgb}{0.000000,0.000000,0.000000}%
\pgfsetstrokecolor{currentstroke}%
\pgfsetdash{}{0pt}%
\pgfpathmoveto{\pgfqpoint{1.137973in}{4.039137in}}%
\pgfpathlineto{\pgfqpoint{1.962441in}{4.039137in}}%
\pgfusepath{stroke}%
\end{pgfscope}%
\begin{pgfscope}%
\definecolor{textcolor}{rgb}{0.000000,0.000000,0.000000}%
\pgfsetstrokecolor{textcolor}%
\pgfsetfillcolor{textcolor}%
\pgftext[x=1.550207in,y=4.122471in,,base]{\color{textcolor}\rmfamily\fontsize{11.000000}{13.200000}\selectfont APRIL ...}%
\end{pgfscope}%
\begin{pgfscope}%
\pgfsetbuttcap%
\pgfsetmiterjoin%
\definecolor{currentfill}{rgb}{1.000000,1.000000,1.000000}%
\pgfsetfillcolor{currentfill}%
\pgfsetlinewidth{0.000000pt}%
\definecolor{currentstroke}{rgb}{0.000000,0.000000,0.000000}%
\pgfsetstrokecolor{currentstroke}%
\pgfsetstrokeopacity{0.000000}%
\pgfsetdash{}{0pt}%
\pgfpathmoveto{\pgfqpoint{2.127335in}{3.795980in}}%
\pgfpathlineto{\pgfqpoint{2.951803in}{3.795980in}}%
\pgfpathlineto{\pgfqpoint{2.951803in}{4.039137in}}%
\pgfpathlineto{\pgfqpoint{2.127335in}{4.039137in}}%
\pgfpathlineto{\pgfqpoint{2.127335in}{3.795980in}}%
\pgfpathclose%
\pgfusepath{fill}%
\end{pgfscope}%
\begin{pgfscope}%
\pgfpathrectangle{\pgfqpoint{2.127335in}{3.795980in}}{\pgfqpoint{0.824468in}{0.243158in}}%
\pgfusepath{clip}%
\pgfsetbuttcap%
\pgfsetmiterjoin%
\definecolor{currentfill}{rgb}{0.121569,0.466667,0.705882}%
\pgfsetfillcolor{currentfill}%
\pgfsetfillopacity{0.500000}%
\pgfsetlinewidth{1.003750pt}%
\definecolor{currentstroke}{rgb}{0.000000,0.000000,0.000000}%
\pgfsetstrokecolor{currentstroke}%
\pgfsetdash{}{0pt}%
\pgfpathmoveto{\pgfqpoint{2.164810in}{3.795980in}}%
\pgfpathlineto{\pgfqpoint{2.314714in}{3.795980in}}%
\pgfpathlineto{\pgfqpoint{2.314714in}{3.814095in}}%
\pgfpathlineto{\pgfqpoint{2.164810in}{3.814095in}}%
\pgfpathlineto{\pgfqpoint{2.164810in}{3.795980in}}%
\pgfpathclose%
\pgfusepath{stroke,fill}%
\end{pgfscope}%
\begin{pgfscope}%
\pgfpathrectangle{\pgfqpoint{2.127335in}{3.795980in}}{\pgfqpoint{0.824468in}{0.243158in}}%
\pgfusepath{clip}%
\pgfsetbuttcap%
\pgfsetmiterjoin%
\definecolor{currentfill}{rgb}{0.121569,0.466667,0.705882}%
\pgfsetfillcolor{currentfill}%
\pgfsetfillopacity{0.500000}%
\pgfsetlinewidth{1.003750pt}%
\definecolor{currentstroke}{rgb}{0.000000,0.000000,0.000000}%
\pgfsetstrokecolor{currentstroke}%
\pgfsetdash{}{0pt}%
\pgfpathmoveto{\pgfqpoint{2.314714in}{3.795980in}}%
\pgfpathlineto{\pgfqpoint{2.464617in}{3.795980in}}%
\pgfpathlineto{\pgfqpoint{2.464617in}{3.801329in}}%
\pgfpathlineto{\pgfqpoint{2.314714in}{3.801329in}}%
\pgfpathlineto{\pgfqpoint{2.314714in}{3.795980in}}%
\pgfpathclose%
\pgfusepath{stroke,fill}%
\end{pgfscope}%
\begin{pgfscope}%
\pgfpathrectangle{\pgfqpoint{2.127335in}{3.795980in}}{\pgfqpoint{0.824468in}{0.243158in}}%
\pgfusepath{clip}%
\pgfsetbuttcap%
\pgfsetmiterjoin%
\definecolor{currentfill}{rgb}{0.121569,0.466667,0.705882}%
\pgfsetfillcolor{currentfill}%
\pgfsetfillopacity{0.500000}%
\pgfsetlinewidth{1.003750pt}%
\definecolor{currentstroke}{rgb}{0.000000,0.000000,0.000000}%
\pgfsetstrokecolor{currentstroke}%
\pgfsetdash{}{0pt}%
\pgfpathmoveto{\pgfqpoint{2.464617in}{3.795980in}}%
\pgfpathlineto{\pgfqpoint{2.614520in}{3.795980in}}%
\pgfpathlineto{\pgfqpoint{2.614520in}{3.796831in}}%
\pgfpathlineto{\pgfqpoint{2.464617in}{3.796831in}}%
\pgfpathlineto{\pgfqpoint{2.464617in}{3.795980in}}%
\pgfpathclose%
\pgfusepath{stroke,fill}%
\end{pgfscope}%
\begin{pgfscope}%
\pgfpathrectangle{\pgfqpoint{2.127335in}{3.795980in}}{\pgfqpoint{0.824468in}{0.243158in}}%
\pgfusepath{clip}%
\pgfsetbuttcap%
\pgfsetmiterjoin%
\definecolor{currentfill}{rgb}{0.121569,0.466667,0.705882}%
\pgfsetfillcolor{currentfill}%
\pgfsetfillopacity{0.500000}%
\pgfsetlinewidth{1.003750pt}%
\definecolor{currentstroke}{rgb}{0.000000,0.000000,0.000000}%
\pgfsetstrokecolor{currentstroke}%
\pgfsetdash{}{0pt}%
\pgfpathmoveto{\pgfqpoint{2.614520in}{3.795980in}}%
\pgfpathlineto{\pgfqpoint{2.764423in}{3.795980in}}%
\pgfpathlineto{\pgfqpoint{2.764423in}{3.796952in}}%
\pgfpathlineto{\pgfqpoint{2.614520in}{3.796952in}}%
\pgfpathlineto{\pgfqpoint{2.614520in}{3.795980in}}%
\pgfpathclose%
\pgfusepath{stroke,fill}%
\end{pgfscope}%
\begin{pgfscope}%
\pgfpathrectangle{\pgfqpoint{2.127335in}{3.795980in}}{\pgfqpoint{0.824468in}{0.243158in}}%
\pgfusepath{clip}%
\pgfsetbuttcap%
\pgfsetmiterjoin%
\definecolor{currentfill}{rgb}{0.121569,0.466667,0.705882}%
\pgfsetfillcolor{currentfill}%
\pgfsetfillopacity{0.500000}%
\pgfsetlinewidth{1.003750pt}%
\definecolor{currentstroke}{rgb}{0.000000,0.000000,0.000000}%
\pgfsetstrokecolor{currentstroke}%
\pgfsetdash{}{0pt}%
\pgfpathmoveto{\pgfqpoint{2.764423in}{3.795980in}}%
\pgfpathlineto{\pgfqpoint{2.914327in}{3.795980in}}%
\pgfpathlineto{\pgfqpoint{2.914327in}{3.796709in}}%
\pgfpathlineto{\pgfqpoint{2.764423in}{3.796709in}}%
\pgfpathlineto{\pgfqpoint{2.764423in}{3.795980in}}%
\pgfpathclose%
\pgfusepath{stroke,fill}%
\end{pgfscope}%
\begin{pgfscope}%
\pgfsetrectcap%
\pgfsetmiterjoin%
\pgfsetlinewidth{0.803000pt}%
\definecolor{currentstroke}{rgb}{0.000000,0.000000,0.000000}%
\pgfsetstrokecolor{currentstroke}%
\pgfsetdash{}{0pt}%
\pgfpathmoveto{\pgfqpoint{2.127335in}{3.795980in}}%
\pgfpathlineto{\pgfqpoint{2.127335in}{4.039137in}}%
\pgfusepath{stroke}%
\end{pgfscope}%
\begin{pgfscope}%
\pgfsetrectcap%
\pgfsetmiterjoin%
\pgfsetlinewidth{0.803000pt}%
\definecolor{currentstroke}{rgb}{0.000000,0.000000,0.000000}%
\pgfsetstrokecolor{currentstroke}%
\pgfsetdash{}{0pt}%
\pgfpathmoveto{\pgfqpoint{2.951803in}{3.795980in}}%
\pgfpathlineto{\pgfqpoint{2.951803in}{4.039137in}}%
\pgfusepath{stroke}%
\end{pgfscope}%
\begin{pgfscope}%
\pgfsetrectcap%
\pgfsetmiterjoin%
\pgfsetlinewidth{0.803000pt}%
\definecolor{currentstroke}{rgb}{0.000000,0.000000,0.000000}%
\pgfsetstrokecolor{currentstroke}%
\pgfsetdash{}{0pt}%
\pgfpathmoveto{\pgfqpoint{2.127335in}{3.795980in}}%
\pgfpathlineto{\pgfqpoint{2.951803in}{3.795980in}}%
\pgfusepath{stroke}%
\end{pgfscope}%
\begin{pgfscope}%
\pgfsetrectcap%
\pgfsetmiterjoin%
\pgfsetlinewidth{0.803000pt}%
\definecolor{currentstroke}{rgb}{0.000000,0.000000,0.000000}%
\pgfsetstrokecolor{currentstroke}%
\pgfsetdash{}{0pt}%
\pgfpathmoveto{\pgfqpoint{2.127335in}{4.039137in}}%
\pgfpathlineto{\pgfqpoint{2.951803in}{4.039137in}}%
\pgfusepath{stroke}%
\end{pgfscope}%
\begin{pgfscope}%
\definecolor{textcolor}{rgb}{0.000000,0.000000,0.000000}%
\pgfsetstrokecolor{textcolor}%
\pgfsetfillcolor{textcolor}%
\pgftext[x=2.539569in,y=4.122471in,,base]{\color{textcolor}\rmfamily\fontsize{11.000000}{13.200000}\selectfont Cegema...}%
\end{pgfscope}%
\begin{pgfscope}%
\pgfsetbuttcap%
\pgfsetmiterjoin%
\definecolor{currentfill}{rgb}{1.000000,1.000000,1.000000}%
\pgfsetfillcolor{currentfill}%
\pgfsetlinewidth{0.000000pt}%
\definecolor{currentstroke}{rgb}{0.000000,0.000000,0.000000}%
\pgfsetstrokecolor{currentstroke}%
\pgfsetstrokeopacity{0.000000}%
\pgfsetdash{}{0pt}%
\pgfpathmoveto{\pgfqpoint{3.116696in}{3.795980in}}%
\pgfpathlineto{\pgfqpoint{3.941164in}{3.795980in}}%
\pgfpathlineto{\pgfqpoint{3.941164in}{4.039137in}}%
\pgfpathlineto{\pgfqpoint{3.116696in}{4.039137in}}%
\pgfpathlineto{\pgfqpoint{3.116696in}{3.795980in}}%
\pgfpathclose%
\pgfusepath{fill}%
\end{pgfscope}%
\begin{pgfscope}%
\pgfpathrectangle{\pgfqpoint{3.116696in}{3.795980in}}{\pgfqpoint{0.824468in}{0.243158in}}%
\pgfusepath{clip}%
\pgfsetbuttcap%
\pgfsetmiterjoin%
\definecolor{currentfill}{rgb}{0.121569,0.466667,0.705882}%
\pgfsetfillcolor{currentfill}%
\pgfsetfillopacity{0.500000}%
\pgfsetlinewidth{1.003750pt}%
\definecolor{currentstroke}{rgb}{0.000000,0.000000,0.000000}%
\pgfsetstrokecolor{currentstroke}%
\pgfsetdash{}{0pt}%
\pgfpathmoveto{\pgfqpoint{3.154172in}{3.795980in}}%
\pgfpathlineto{\pgfqpoint{3.304075in}{3.795980in}}%
\pgfpathlineto{\pgfqpoint{3.304075in}{3.797803in}}%
\pgfpathlineto{\pgfqpoint{3.154172in}{3.797803in}}%
\pgfpathlineto{\pgfqpoint{3.154172in}{3.795980in}}%
\pgfpathclose%
\pgfusepath{stroke,fill}%
\end{pgfscope}%
\begin{pgfscope}%
\pgfpathrectangle{\pgfqpoint{3.116696in}{3.795980in}}{\pgfqpoint{0.824468in}{0.243158in}}%
\pgfusepath{clip}%
\pgfsetbuttcap%
\pgfsetmiterjoin%
\definecolor{currentfill}{rgb}{0.121569,0.466667,0.705882}%
\pgfsetfillcolor{currentfill}%
\pgfsetfillopacity{0.500000}%
\pgfsetlinewidth{1.003750pt}%
\definecolor{currentstroke}{rgb}{0.000000,0.000000,0.000000}%
\pgfsetstrokecolor{currentstroke}%
\pgfsetdash{}{0pt}%
\pgfpathmoveto{\pgfqpoint{3.304075in}{3.795980in}}%
\pgfpathlineto{\pgfqpoint{3.453979in}{3.795980in}}%
\pgfpathlineto{\pgfqpoint{3.453979in}{3.796101in}}%
\pgfpathlineto{\pgfqpoint{3.304075in}{3.796101in}}%
\pgfpathlineto{\pgfqpoint{3.304075in}{3.795980in}}%
\pgfpathclose%
\pgfusepath{stroke,fill}%
\end{pgfscope}%
\begin{pgfscope}%
\pgfpathrectangle{\pgfqpoint{3.116696in}{3.795980in}}{\pgfqpoint{0.824468in}{0.243158in}}%
\pgfusepath{clip}%
\pgfsetbuttcap%
\pgfsetmiterjoin%
\definecolor{currentfill}{rgb}{0.121569,0.466667,0.705882}%
\pgfsetfillcolor{currentfill}%
\pgfsetfillopacity{0.500000}%
\pgfsetlinewidth{1.003750pt}%
\definecolor{currentstroke}{rgb}{0.000000,0.000000,0.000000}%
\pgfsetstrokecolor{currentstroke}%
\pgfsetdash{}{0pt}%
\pgfpathmoveto{\pgfqpoint{3.453979in}{3.795980in}}%
\pgfpathlineto{\pgfqpoint{3.603882in}{3.795980in}}%
\pgfpathlineto{\pgfqpoint{3.603882in}{3.796223in}}%
\pgfpathlineto{\pgfqpoint{3.453979in}{3.796223in}}%
\pgfpathlineto{\pgfqpoint{3.453979in}{3.795980in}}%
\pgfpathclose%
\pgfusepath{stroke,fill}%
\end{pgfscope}%
\begin{pgfscope}%
\pgfpathrectangle{\pgfqpoint{3.116696in}{3.795980in}}{\pgfqpoint{0.824468in}{0.243158in}}%
\pgfusepath{clip}%
\pgfsetbuttcap%
\pgfsetmiterjoin%
\definecolor{currentfill}{rgb}{0.121569,0.466667,0.705882}%
\pgfsetfillcolor{currentfill}%
\pgfsetfillopacity{0.500000}%
\pgfsetlinewidth{1.003750pt}%
\definecolor{currentstroke}{rgb}{0.000000,0.000000,0.000000}%
\pgfsetstrokecolor{currentstroke}%
\pgfsetdash{}{0pt}%
\pgfpathmoveto{\pgfqpoint{3.603882in}{3.795980in}}%
\pgfpathlineto{\pgfqpoint{3.753785in}{3.795980in}}%
\pgfpathlineto{\pgfqpoint{3.753785in}{3.795980in}}%
\pgfpathlineto{\pgfqpoint{3.603882in}{3.795980in}}%
\pgfpathlineto{\pgfqpoint{3.603882in}{3.795980in}}%
\pgfpathclose%
\pgfusepath{stroke,fill}%
\end{pgfscope}%
\begin{pgfscope}%
\pgfpathrectangle{\pgfqpoint{3.116696in}{3.795980in}}{\pgfqpoint{0.824468in}{0.243158in}}%
\pgfusepath{clip}%
\pgfsetbuttcap%
\pgfsetmiterjoin%
\definecolor{currentfill}{rgb}{0.121569,0.466667,0.705882}%
\pgfsetfillcolor{currentfill}%
\pgfsetfillopacity{0.500000}%
\pgfsetlinewidth{1.003750pt}%
\definecolor{currentstroke}{rgb}{0.000000,0.000000,0.000000}%
\pgfsetstrokecolor{currentstroke}%
\pgfsetdash{}{0pt}%
\pgfpathmoveto{\pgfqpoint{3.753785in}{3.795980in}}%
\pgfpathlineto{\pgfqpoint{3.903688in}{3.795980in}}%
\pgfpathlineto{\pgfqpoint{3.903688in}{3.795980in}}%
\pgfpathlineto{\pgfqpoint{3.753785in}{3.795980in}}%
\pgfpathlineto{\pgfqpoint{3.753785in}{3.795980in}}%
\pgfpathclose%
\pgfusepath{stroke,fill}%
\end{pgfscope}%
\begin{pgfscope}%
\pgfsetrectcap%
\pgfsetmiterjoin%
\pgfsetlinewidth{0.803000pt}%
\definecolor{currentstroke}{rgb}{0.000000,0.000000,0.000000}%
\pgfsetstrokecolor{currentstroke}%
\pgfsetdash{}{0pt}%
\pgfpathmoveto{\pgfqpoint{3.116696in}{3.795980in}}%
\pgfpathlineto{\pgfqpoint{3.116696in}{4.039137in}}%
\pgfusepath{stroke}%
\end{pgfscope}%
\begin{pgfscope}%
\pgfsetrectcap%
\pgfsetmiterjoin%
\pgfsetlinewidth{0.803000pt}%
\definecolor{currentstroke}{rgb}{0.000000,0.000000,0.000000}%
\pgfsetstrokecolor{currentstroke}%
\pgfsetdash{}{0pt}%
\pgfpathmoveto{\pgfqpoint{3.941164in}{3.795980in}}%
\pgfpathlineto{\pgfqpoint{3.941164in}{4.039137in}}%
\pgfusepath{stroke}%
\end{pgfscope}%
\begin{pgfscope}%
\pgfsetrectcap%
\pgfsetmiterjoin%
\pgfsetlinewidth{0.803000pt}%
\definecolor{currentstroke}{rgb}{0.000000,0.000000,0.000000}%
\pgfsetstrokecolor{currentstroke}%
\pgfsetdash{}{0pt}%
\pgfpathmoveto{\pgfqpoint{3.116696in}{3.795980in}}%
\pgfpathlineto{\pgfqpoint{3.941164in}{3.795980in}}%
\pgfusepath{stroke}%
\end{pgfscope}%
\begin{pgfscope}%
\pgfsetrectcap%
\pgfsetmiterjoin%
\pgfsetlinewidth{0.803000pt}%
\definecolor{currentstroke}{rgb}{0.000000,0.000000,0.000000}%
\pgfsetstrokecolor{currentstroke}%
\pgfsetdash{}{0pt}%
\pgfpathmoveto{\pgfqpoint{3.116696in}{4.039137in}}%
\pgfpathlineto{\pgfqpoint{3.941164in}{4.039137in}}%
\pgfusepath{stroke}%
\end{pgfscope}%
\begin{pgfscope}%
\definecolor{textcolor}{rgb}{0.000000,0.000000,0.000000}%
\pgfsetstrokecolor{textcolor}%
\pgfsetfillcolor{textcolor}%
\pgftext[x=3.528930in,y=4.122471in,,base]{\color{textcolor}\rmfamily\fontsize{11.000000}{13.200000}\selectfont LCL}%
\end{pgfscope}%
\begin{pgfscope}%
\pgfsetbuttcap%
\pgfsetmiterjoin%
\definecolor{currentfill}{rgb}{1.000000,1.000000,1.000000}%
\pgfsetfillcolor{currentfill}%
\pgfsetlinewidth{0.000000pt}%
\definecolor{currentstroke}{rgb}{0.000000,0.000000,0.000000}%
\pgfsetstrokecolor{currentstroke}%
\pgfsetstrokeopacity{0.000000}%
\pgfsetdash{}{0pt}%
\pgfpathmoveto{\pgfqpoint{4.106058in}{3.795980in}}%
\pgfpathlineto{\pgfqpoint{4.930526in}{3.795980in}}%
\pgfpathlineto{\pgfqpoint{4.930526in}{4.039137in}}%
\pgfpathlineto{\pgfqpoint{4.106058in}{4.039137in}}%
\pgfpathlineto{\pgfqpoint{4.106058in}{3.795980in}}%
\pgfpathclose%
\pgfusepath{fill}%
\end{pgfscope}%
\begin{pgfscope}%
\pgfpathrectangle{\pgfqpoint{4.106058in}{3.795980in}}{\pgfqpoint{0.824468in}{0.243158in}}%
\pgfusepath{clip}%
\pgfsetbuttcap%
\pgfsetmiterjoin%
\definecolor{currentfill}{rgb}{0.121569,0.466667,0.705882}%
\pgfsetfillcolor{currentfill}%
\pgfsetfillopacity{0.500000}%
\pgfsetlinewidth{1.003750pt}%
\definecolor{currentstroke}{rgb}{0.000000,0.000000,0.000000}%
\pgfsetstrokecolor{currentstroke}%
\pgfsetdash{}{0pt}%
\pgfpathmoveto{\pgfqpoint{4.143534in}{3.795980in}}%
\pgfpathlineto{\pgfqpoint{4.293437in}{3.795980in}}%
\pgfpathlineto{\pgfqpoint{4.293437in}{3.804976in}}%
\pgfpathlineto{\pgfqpoint{4.143534in}{3.804976in}}%
\pgfpathlineto{\pgfqpoint{4.143534in}{3.795980in}}%
\pgfpathclose%
\pgfusepath{stroke,fill}%
\end{pgfscope}%
\begin{pgfscope}%
\pgfpathrectangle{\pgfqpoint{4.106058in}{3.795980in}}{\pgfqpoint{0.824468in}{0.243158in}}%
\pgfusepath{clip}%
\pgfsetbuttcap%
\pgfsetmiterjoin%
\definecolor{currentfill}{rgb}{0.121569,0.466667,0.705882}%
\pgfsetfillcolor{currentfill}%
\pgfsetfillopacity{0.500000}%
\pgfsetlinewidth{1.003750pt}%
\definecolor{currentstroke}{rgb}{0.000000,0.000000,0.000000}%
\pgfsetstrokecolor{currentstroke}%
\pgfsetdash{}{0pt}%
\pgfpathmoveto{\pgfqpoint{4.293437in}{3.795980in}}%
\pgfpathlineto{\pgfqpoint{4.443340in}{3.795980in}}%
\pgfpathlineto{\pgfqpoint{4.443340in}{3.801572in}}%
\pgfpathlineto{\pgfqpoint{4.293437in}{3.801572in}}%
\pgfpathlineto{\pgfqpoint{4.293437in}{3.795980in}}%
\pgfpathclose%
\pgfusepath{stroke,fill}%
\end{pgfscope}%
\begin{pgfscope}%
\pgfpathrectangle{\pgfqpoint{4.106058in}{3.795980in}}{\pgfqpoint{0.824468in}{0.243158in}}%
\pgfusepath{clip}%
\pgfsetbuttcap%
\pgfsetmiterjoin%
\definecolor{currentfill}{rgb}{0.121569,0.466667,0.705882}%
\pgfsetfillcolor{currentfill}%
\pgfsetfillopacity{0.500000}%
\pgfsetlinewidth{1.003750pt}%
\definecolor{currentstroke}{rgb}{0.000000,0.000000,0.000000}%
\pgfsetstrokecolor{currentstroke}%
\pgfsetdash{}{0pt}%
\pgfpathmoveto{\pgfqpoint{4.443340in}{3.795980in}}%
\pgfpathlineto{\pgfqpoint{4.593244in}{3.795980in}}%
\pgfpathlineto{\pgfqpoint{4.593244in}{3.797438in}}%
\pgfpathlineto{\pgfqpoint{4.443340in}{3.797438in}}%
\pgfpathlineto{\pgfqpoint{4.443340in}{3.795980in}}%
\pgfpathclose%
\pgfusepath{stroke,fill}%
\end{pgfscope}%
\begin{pgfscope}%
\pgfpathrectangle{\pgfqpoint{4.106058in}{3.795980in}}{\pgfqpoint{0.824468in}{0.243158in}}%
\pgfusepath{clip}%
\pgfsetbuttcap%
\pgfsetmiterjoin%
\definecolor{currentfill}{rgb}{0.121569,0.466667,0.705882}%
\pgfsetfillcolor{currentfill}%
\pgfsetfillopacity{0.500000}%
\pgfsetlinewidth{1.003750pt}%
\definecolor{currentstroke}{rgb}{0.000000,0.000000,0.000000}%
\pgfsetstrokecolor{currentstroke}%
\pgfsetdash{}{0pt}%
\pgfpathmoveto{\pgfqpoint{4.593244in}{3.795980in}}%
\pgfpathlineto{\pgfqpoint{4.743147in}{3.795980in}}%
\pgfpathlineto{\pgfqpoint{4.743147in}{3.796952in}}%
\pgfpathlineto{\pgfqpoint{4.593244in}{3.796952in}}%
\pgfpathlineto{\pgfqpoint{4.593244in}{3.795980in}}%
\pgfpathclose%
\pgfusepath{stroke,fill}%
\end{pgfscope}%
\begin{pgfscope}%
\pgfpathrectangle{\pgfqpoint{4.106058in}{3.795980in}}{\pgfqpoint{0.824468in}{0.243158in}}%
\pgfusepath{clip}%
\pgfsetbuttcap%
\pgfsetmiterjoin%
\definecolor{currentfill}{rgb}{0.121569,0.466667,0.705882}%
\pgfsetfillcolor{currentfill}%
\pgfsetfillopacity{0.500000}%
\pgfsetlinewidth{1.003750pt}%
\definecolor{currentstroke}{rgb}{0.000000,0.000000,0.000000}%
\pgfsetstrokecolor{currentstroke}%
\pgfsetdash{}{0pt}%
\pgfpathmoveto{\pgfqpoint{4.743147in}{3.795980in}}%
\pgfpathlineto{\pgfqpoint{4.893050in}{3.795980in}}%
\pgfpathlineto{\pgfqpoint{4.893050in}{3.796466in}}%
\pgfpathlineto{\pgfqpoint{4.743147in}{3.796466in}}%
\pgfpathlineto{\pgfqpoint{4.743147in}{3.795980in}}%
\pgfpathclose%
\pgfusepath{stroke,fill}%
\end{pgfscope}%
\begin{pgfscope}%
\pgfsetrectcap%
\pgfsetmiterjoin%
\pgfsetlinewidth{0.803000pt}%
\definecolor{currentstroke}{rgb}{0.000000,0.000000,0.000000}%
\pgfsetstrokecolor{currentstroke}%
\pgfsetdash{}{0pt}%
\pgfpathmoveto{\pgfqpoint{4.106058in}{3.795980in}}%
\pgfpathlineto{\pgfqpoint{4.106058in}{4.039137in}}%
\pgfusepath{stroke}%
\end{pgfscope}%
\begin{pgfscope}%
\pgfsetrectcap%
\pgfsetmiterjoin%
\pgfsetlinewidth{0.803000pt}%
\definecolor{currentstroke}{rgb}{0.000000,0.000000,0.000000}%
\pgfsetstrokecolor{currentstroke}%
\pgfsetdash{}{0pt}%
\pgfpathmoveto{\pgfqpoint{4.930526in}{3.795980in}}%
\pgfpathlineto{\pgfqpoint{4.930526in}{4.039137in}}%
\pgfusepath{stroke}%
\end{pgfscope}%
\begin{pgfscope}%
\pgfsetrectcap%
\pgfsetmiterjoin%
\pgfsetlinewidth{0.803000pt}%
\definecolor{currentstroke}{rgb}{0.000000,0.000000,0.000000}%
\pgfsetstrokecolor{currentstroke}%
\pgfsetdash{}{0pt}%
\pgfpathmoveto{\pgfqpoint{4.106058in}{3.795980in}}%
\pgfpathlineto{\pgfqpoint{4.930526in}{3.795980in}}%
\pgfusepath{stroke}%
\end{pgfscope}%
\begin{pgfscope}%
\pgfsetrectcap%
\pgfsetmiterjoin%
\pgfsetlinewidth{0.803000pt}%
\definecolor{currentstroke}{rgb}{0.000000,0.000000,0.000000}%
\pgfsetstrokecolor{currentstroke}%
\pgfsetdash{}{0pt}%
\pgfpathmoveto{\pgfqpoint{4.106058in}{4.039137in}}%
\pgfpathlineto{\pgfqpoint{4.930526in}{4.039137in}}%
\pgfusepath{stroke}%
\end{pgfscope}%
\begin{pgfscope}%
\definecolor{textcolor}{rgb}{0.000000,0.000000,0.000000}%
\pgfsetstrokecolor{textcolor}%
\pgfsetfillcolor{textcolor}%
\pgftext[x=4.518292in,y=4.122471in,,base]{\color{textcolor}\rmfamily\fontsize{11.000000}{13.200000}\selectfont Afer}%
\end{pgfscope}%
\begin{pgfscope}%
\pgfsetbuttcap%
\pgfsetmiterjoin%
\definecolor{currentfill}{rgb}{1.000000,1.000000,1.000000}%
\pgfsetfillcolor{currentfill}%
\pgfsetlinewidth{0.000000pt}%
\definecolor{currentstroke}{rgb}{0.000000,0.000000,0.000000}%
\pgfsetstrokecolor{currentstroke}%
\pgfsetstrokeopacity{0.000000}%
\pgfsetdash{}{0pt}%
\pgfpathmoveto{\pgfqpoint{5.095420in}{3.795980in}}%
\pgfpathlineto{\pgfqpoint{5.919888in}{3.795980in}}%
\pgfpathlineto{\pgfqpoint{5.919888in}{4.039137in}}%
\pgfpathlineto{\pgfqpoint{5.095420in}{4.039137in}}%
\pgfpathlineto{\pgfqpoint{5.095420in}{3.795980in}}%
\pgfpathclose%
\pgfusepath{fill}%
\end{pgfscope}%
\begin{pgfscope}%
\pgfpathrectangle{\pgfqpoint{5.095420in}{3.795980in}}{\pgfqpoint{0.824468in}{0.243158in}}%
\pgfusepath{clip}%
\pgfsetbuttcap%
\pgfsetmiterjoin%
\definecolor{currentfill}{rgb}{0.121569,0.466667,0.705882}%
\pgfsetfillcolor{currentfill}%
\pgfsetfillopacity{0.500000}%
\pgfsetlinewidth{1.003750pt}%
\definecolor{currentstroke}{rgb}{0.000000,0.000000,0.000000}%
\pgfsetstrokecolor{currentstroke}%
\pgfsetdash{}{0pt}%
\pgfpathmoveto{\pgfqpoint{5.132895in}{3.795980in}}%
\pgfpathlineto{\pgfqpoint{5.282799in}{3.795980in}}%
\pgfpathlineto{\pgfqpoint{5.282799in}{3.831116in}}%
\pgfpathlineto{\pgfqpoint{5.132895in}{3.831116in}}%
\pgfpathlineto{\pgfqpoint{5.132895in}{3.795980in}}%
\pgfpathclose%
\pgfusepath{stroke,fill}%
\end{pgfscope}%
\begin{pgfscope}%
\pgfpathrectangle{\pgfqpoint{5.095420in}{3.795980in}}{\pgfqpoint{0.824468in}{0.243158in}}%
\pgfusepath{clip}%
\pgfsetbuttcap%
\pgfsetmiterjoin%
\definecolor{currentfill}{rgb}{0.121569,0.466667,0.705882}%
\pgfsetfillcolor{currentfill}%
\pgfsetfillopacity{0.500000}%
\pgfsetlinewidth{1.003750pt}%
\definecolor{currentstroke}{rgb}{0.000000,0.000000,0.000000}%
\pgfsetstrokecolor{currentstroke}%
\pgfsetdash{}{0pt}%
\pgfpathmoveto{\pgfqpoint{5.282799in}{3.795980in}}%
\pgfpathlineto{\pgfqpoint{5.432702in}{3.795980in}}%
\pgfpathlineto{\pgfqpoint{5.432702in}{3.810812in}}%
\pgfpathlineto{\pgfqpoint{5.282799in}{3.810812in}}%
\pgfpathlineto{\pgfqpoint{5.282799in}{3.795980in}}%
\pgfpathclose%
\pgfusepath{stroke,fill}%
\end{pgfscope}%
\begin{pgfscope}%
\pgfpathrectangle{\pgfqpoint{5.095420in}{3.795980in}}{\pgfqpoint{0.824468in}{0.243158in}}%
\pgfusepath{clip}%
\pgfsetbuttcap%
\pgfsetmiterjoin%
\definecolor{currentfill}{rgb}{0.121569,0.466667,0.705882}%
\pgfsetfillcolor{currentfill}%
\pgfsetfillopacity{0.500000}%
\pgfsetlinewidth{1.003750pt}%
\definecolor{currentstroke}{rgb}{0.000000,0.000000,0.000000}%
\pgfsetstrokecolor{currentstroke}%
\pgfsetdash{}{0pt}%
\pgfpathmoveto{\pgfqpoint{5.432702in}{3.795980in}}%
\pgfpathlineto{\pgfqpoint{5.582605in}{3.795980in}}%
\pgfpathlineto{\pgfqpoint{5.582605in}{3.802788in}}%
\pgfpathlineto{\pgfqpoint{5.432702in}{3.802788in}}%
\pgfpathlineto{\pgfqpoint{5.432702in}{3.795980in}}%
\pgfpathclose%
\pgfusepath{stroke,fill}%
\end{pgfscope}%
\begin{pgfscope}%
\pgfpathrectangle{\pgfqpoint{5.095420in}{3.795980in}}{\pgfqpoint{0.824468in}{0.243158in}}%
\pgfusepath{clip}%
\pgfsetbuttcap%
\pgfsetmiterjoin%
\definecolor{currentfill}{rgb}{0.121569,0.466667,0.705882}%
\pgfsetfillcolor{currentfill}%
\pgfsetfillopacity{0.500000}%
\pgfsetlinewidth{1.003750pt}%
\definecolor{currentstroke}{rgb}{0.000000,0.000000,0.000000}%
\pgfsetstrokecolor{currentstroke}%
\pgfsetdash{}{0pt}%
\pgfpathmoveto{\pgfqpoint{5.582605in}{3.795980in}}%
\pgfpathlineto{\pgfqpoint{5.732509in}{3.795980in}}%
\pgfpathlineto{\pgfqpoint{5.732509in}{3.803639in}}%
\pgfpathlineto{\pgfqpoint{5.582605in}{3.803639in}}%
\pgfpathlineto{\pgfqpoint{5.582605in}{3.795980in}}%
\pgfpathclose%
\pgfusepath{stroke,fill}%
\end{pgfscope}%
\begin{pgfscope}%
\pgfpathrectangle{\pgfqpoint{5.095420in}{3.795980in}}{\pgfqpoint{0.824468in}{0.243158in}}%
\pgfusepath{clip}%
\pgfsetbuttcap%
\pgfsetmiterjoin%
\definecolor{currentfill}{rgb}{0.121569,0.466667,0.705882}%
\pgfsetfillcolor{currentfill}%
\pgfsetfillopacity{0.500000}%
\pgfsetlinewidth{1.003750pt}%
\definecolor{currentstroke}{rgb}{0.000000,0.000000,0.000000}%
\pgfsetstrokecolor{currentstroke}%
\pgfsetdash{}{0pt}%
\pgfpathmoveto{\pgfqpoint{5.732509in}{3.795980in}}%
\pgfpathlineto{\pgfqpoint{5.882412in}{3.795980in}}%
\pgfpathlineto{\pgfqpoint{5.882412in}{3.798776in}}%
\pgfpathlineto{\pgfqpoint{5.732509in}{3.798776in}}%
\pgfpathlineto{\pgfqpoint{5.732509in}{3.795980in}}%
\pgfpathclose%
\pgfusepath{stroke,fill}%
\end{pgfscope}%
\begin{pgfscope}%
\pgfsetrectcap%
\pgfsetmiterjoin%
\pgfsetlinewidth{0.803000pt}%
\definecolor{currentstroke}{rgb}{0.000000,0.000000,0.000000}%
\pgfsetstrokecolor{currentstroke}%
\pgfsetdash{}{0pt}%
\pgfpathmoveto{\pgfqpoint{5.095420in}{3.795980in}}%
\pgfpathlineto{\pgfqpoint{5.095420in}{4.039137in}}%
\pgfusepath{stroke}%
\end{pgfscope}%
\begin{pgfscope}%
\pgfsetrectcap%
\pgfsetmiterjoin%
\pgfsetlinewidth{0.803000pt}%
\definecolor{currentstroke}{rgb}{0.000000,0.000000,0.000000}%
\pgfsetstrokecolor{currentstroke}%
\pgfsetdash{}{0pt}%
\pgfpathmoveto{\pgfqpoint{5.919888in}{3.795980in}}%
\pgfpathlineto{\pgfqpoint{5.919888in}{4.039137in}}%
\pgfusepath{stroke}%
\end{pgfscope}%
\begin{pgfscope}%
\pgfsetrectcap%
\pgfsetmiterjoin%
\pgfsetlinewidth{0.803000pt}%
\definecolor{currentstroke}{rgb}{0.000000,0.000000,0.000000}%
\pgfsetstrokecolor{currentstroke}%
\pgfsetdash{}{0pt}%
\pgfpathmoveto{\pgfqpoint{5.095420in}{3.795980in}}%
\pgfpathlineto{\pgfqpoint{5.919888in}{3.795980in}}%
\pgfusepath{stroke}%
\end{pgfscope}%
\begin{pgfscope}%
\pgfsetrectcap%
\pgfsetmiterjoin%
\pgfsetlinewidth{0.803000pt}%
\definecolor{currentstroke}{rgb}{0.000000,0.000000,0.000000}%
\pgfsetstrokecolor{currentstroke}%
\pgfsetdash{}{0pt}%
\pgfpathmoveto{\pgfqpoint{5.095420in}{4.039137in}}%
\pgfpathlineto{\pgfqpoint{5.919888in}{4.039137in}}%
\pgfusepath{stroke}%
\end{pgfscope}%
\begin{pgfscope}%
\definecolor{textcolor}{rgb}{0.000000,0.000000,0.000000}%
\pgfsetstrokecolor{textcolor}%
\pgfsetfillcolor{textcolor}%
\pgftext[x=5.507654in,y=4.122471in,,base]{\color{textcolor}\rmfamily\fontsize{11.000000}{13.200000}\selectfont Pacifica}%
\end{pgfscope}%
\begin{pgfscope}%
\pgfsetbuttcap%
\pgfsetmiterjoin%
\definecolor{currentfill}{rgb}{1.000000,1.000000,1.000000}%
\pgfsetfillcolor{currentfill}%
\pgfsetlinewidth{0.000000pt}%
\definecolor{currentstroke}{rgb}{0.000000,0.000000,0.000000}%
\pgfsetstrokecolor{currentstroke}%
\pgfsetstrokeopacity{0.000000}%
\pgfsetdash{}{0pt}%
\pgfpathmoveto{\pgfqpoint{6.084781in}{3.795980in}}%
\pgfpathlineto{\pgfqpoint{6.909249in}{3.795980in}}%
\pgfpathlineto{\pgfqpoint{6.909249in}{4.039137in}}%
\pgfpathlineto{\pgfqpoint{6.084781in}{4.039137in}}%
\pgfpathlineto{\pgfqpoint{6.084781in}{3.795980in}}%
\pgfpathclose%
\pgfusepath{fill}%
\end{pgfscope}%
\begin{pgfscope}%
\pgfpathrectangle{\pgfqpoint{6.084781in}{3.795980in}}{\pgfqpoint{0.824468in}{0.243158in}}%
\pgfusepath{clip}%
\pgfsetbuttcap%
\pgfsetmiterjoin%
\definecolor{currentfill}{rgb}{0.121569,0.466667,0.705882}%
\pgfsetfillcolor{currentfill}%
\pgfsetfillopacity{0.500000}%
\pgfsetlinewidth{1.003750pt}%
\definecolor{currentstroke}{rgb}{0.000000,0.000000,0.000000}%
\pgfsetstrokecolor{currentstroke}%
\pgfsetdash{}{0pt}%
\pgfpathmoveto{\pgfqpoint{6.122257in}{3.795980in}}%
\pgfpathlineto{\pgfqpoint{6.272160in}{3.795980in}}%
\pgfpathlineto{\pgfqpoint{6.272160in}{3.808137in}}%
\pgfpathlineto{\pgfqpoint{6.122257in}{3.808137in}}%
\pgfpathlineto{\pgfqpoint{6.122257in}{3.795980in}}%
\pgfpathclose%
\pgfusepath{stroke,fill}%
\end{pgfscope}%
\begin{pgfscope}%
\pgfpathrectangle{\pgfqpoint{6.084781in}{3.795980in}}{\pgfqpoint{0.824468in}{0.243158in}}%
\pgfusepath{clip}%
\pgfsetbuttcap%
\pgfsetmiterjoin%
\definecolor{currentfill}{rgb}{0.121569,0.466667,0.705882}%
\pgfsetfillcolor{currentfill}%
\pgfsetfillopacity{0.500000}%
\pgfsetlinewidth{1.003750pt}%
\definecolor{currentstroke}{rgb}{0.000000,0.000000,0.000000}%
\pgfsetstrokecolor{currentstroke}%
\pgfsetdash{}{0pt}%
\pgfpathmoveto{\pgfqpoint{6.272160in}{3.795980in}}%
\pgfpathlineto{\pgfqpoint{6.422064in}{3.795980in}}%
\pgfpathlineto{\pgfqpoint{6.422064in}{3.798411in}}%
\pgfpathlineto{\pgfqpoint{6.272160in}{3.798411in}}%
\pgfpathlineto{\pgfqpoint{6.272160in}{3.795980in}}%
\pgfpathclose%
\pgfusepath{stroke,fill}%
\end{pgfscope}%
\begin{pgfscope}%
\pgfpathrectangle{\pgfqpoint{6.084781in}{3.795980in}}{\pgfqpoint{0.824468in}{0.243158in}}%
\pgfusepath{clip}%
\pgfsetbuttcap%
\pgfsetmiterjoin%
\definecolor{currentfill}{rgb}{0.121569,0.466667,0.705882}%
\pgfsetfillcolor{currentfill}%
\pgfsetfillopacity{0.500000}%
\pgfsetlinewidth{1.003750pt}%
\definecolor{currentstroke}{rgb}{0.000000,0.000000,0.000000}%
\pgfsetstrokecolor{currentstroke}%
\pgfsetdash{}{0pt}%
\pgfpathmoveto{\pgfqpoint{6.422064in}{3.795980in}}%
\pgfpathlineto{\pgfqpoint{6.571967in}{3.795980in}}%
\pgfpathlineto{\pgfqpoint{6.571967in}{3.798168in}}%
\pgfpathlineto{\pgfqpoint{6.422064in}{3.798168in}}%
\pgfpathlineto{\pgfqpoint{6.422064in}{3.795980in}}%
\pgfpathclose%
\pgfusepath{stroke,fill}%
\end{pgfscope}%
\begin{pgfscope}%
\pgfpathrectangle{\pgfqpoint{6.084781in}{3.795980in}}{\pgfqpoint{0.824468in}{0.243158in}}%
\pgfusepath{clip}%
\pgfsetbuttcap%
\pgfsetmiterjoin%
\definecolor{currentfill}{rgb}{0.121569,0.466667,0.705882}%
\pgfsetfillcolor{currentfill}%
\pgfsetfillopacity{0.500000}%
\pgfsetlinewidth{1.003750pt}%
\definecolor{currentstroke}{rgb}{0.000000,0.000000,0.000000}%
\pgfsetstrokecolor{currentstroke}%
\pgfsetdash{}{0pt}%
\pgfpathmoveto{\pgfqpoint{6.571967in}{3.795980in}}%
\pgfpathlineto{\pgfqpoint{6.721870in}{3.795980in}}%
\pgfpathlineto{\pgfqpoint{6.721870in}{3.795980in}}%
\pgfpathlineto{\pgfqpoint{6.571967in}{3.795980in}}%
\pgfpathlineto{\pgfqpoint{6.571967in}{3.795980in}}%
\pgfpathclose%
\pgfusepath{stroke,fill}%
\end{pgfscope}%
\begin{pgfscope}%
\pgfpathrectangle{\pgfqpoint{6.084781in}{3.795980in}}{\pgfqpoint{0.824468in}{0.243158in}}%
\pgfusepath{clip}%
\pgfsetbuttcap%
\pgfsetmiterjoin%
\definecolor{currentfill}{rgb}{0.121569,0.466667,0.705882}%
\pgfsetfillcolor{currentfill}%
\pgfsetfillopacity{0.500000}%
\pgfsetlinewidth{1.003750pt}%
\definecolor{currentstroke}{rgb}{0.000000,0.000000,0.000000}%
\pgfsetstrokecolor{currentstroke}%
\pgfsetdash{}{0pt}%
\pgfpathmoveto{\pgfqpoint{6.721870in}{3.795980in}}%
\pgfpathlineto{\pgfqpoint{6.871774in}{3.795980in}}%
\pgfpathlineto{\pgfqpoint{6.871774in}{3.795980in}}%
\pgfpathlineto{\pgfqpoint{6.721870in}{3.795980in}}%
\pgfpathlineto{\pgfqpoint{6.721870in}{3.795980in}}%
\pgfpathclose%
\pgfusepath{stroke,fill}%
\end{pgfscope}%
\begin{pgfscope}%
\pgfsetrectcap%
\pgfsetmiterjoin%
\pgfsetlinewidth{0.803000pt}%
\definecolor{currentstroke}{rgb}{0.000000,0.000000,0.000000}%
\pgfsetstrokecolor{currentstroke}%
\pgfsetdash{}{0pt}%
\pgfpathmoveto{\pgfqpoint{6.084781in}{3.795980in}}%
\pgfpathlineto{\pgfqpoint{6.084781in}{4.039137in}}%
\pgfusepath{stroke}%
\end{pgfscope}%
\begin{pgfscope}%
\pgfsetrectcap%
\pgfsetmiterjoin%
\pgfsetlinewidth{0.803000pt}%
\definecolor{currentstroke}{rgb}{0.000000,0.000000,0.000000}%
\pgfsetstrokecolor{currentstroke}%
\pgfsetdash{}{0pt}%
\pgfpathmoveto{\pgfqpoint{6.909249in}{3.795980in}}%
\pgfpathlineto{\pgfqpoint{6.909249in}{4.039137in}}%
\pgfusepath{stroke}%
\end{pgfscope}%
\begin{pgfscope}%
\pgfsetrectcap%
\pgfsetmiterjoin%
\pgfsetlinewidth{0.803000pt}%
\definecolor{currentstroke}{rgb}{0.000000,0.000000,0.000000}%
\pgfsetstrokecolor{currentstroke}%
\pgfsetdash{}{0pt}%
\pgfpathmoveto{\pgfqpoint{6.084781in}{3.795980in}}%
\pgfpathlineto{\pgfqpoint{6.909249in}{3.795980in}}%
\pgfusepath{stroke}%
\end{pgfscope}%
\begin{pgfscope}%
\pgfsetrectcap%
\pgfsetmiterjoin%
\pgfsetlinewidth{0.803000pt}%
\definecolor{currentstroke}{rgb}{0.000000,0.000000,0.000000}%
\pgfsetstrokecolor{currentstroke}%
\pgfsetdash{}{0pt}%
\pgfpathmoveto{\pgfqpoint{6.084781in}{4.039137in}}%
\pgfpathlineto{\pgfqpoint{6.909249in}{4.039137in}}%
\pgfusepath{stroke}%
\end{pgfscope}%
\begin{pgfscope}%
\definecolor{textcolor}{rgb}{0.000000,0.000000,0.000000}%
\pgfsetstrokecolor{textcolor}%
\pgfsetfillcolor{textcolor}%
\pgftext[x=6.497015in,y=4.122471in,,base]{\color{textcolor}\rmfamily\fontsize{11.000000}{13.200000}\selectfont SwissLife}%
\end{pgfscope}%
\begin{pgfscope}%
\pgfsetbuttcap%
\pgfsetmiterjoin%
\definecolor{currentfill}{rgb}{1.000000,1.000000,1.000000}%
\pgfsetfillcolor{currentfill}%
\pgfsetlinewidth{0.000000pt}%
\definecolor{currentstroke}{rgb}{0.000000,0.000000,0.000000}%
\pgfsetstrokecolor{currentstroke}%
\pgfsetstrokeopacity{0.000000}%
\pgfsetdash{}{0pt}%
\pgfpathmoveto{\pgfqpoint{7.074143in}{3.795980in}}%
\pgfpathlineto{\pgfqpoint{7.898611in}{3.795980in}}%
\pgfpathlineto{\pgfqpoint{7.898611in}{4.039137in}}%
\pgfpathlineto{\pgfqpoint{7.074143in}{4.039137in}}%
\pgfpathlineto{\pgfqpoint{7.074143in}{3.795980in}}%
\pgfpathclose%
\pgfusepath{fill}%
\end{pgfscope}%
\begin{pgfscope}%
\pgfpathrectangle{\pgfqpoint{7.074143in}{3.795980in}}{\pgfqpoint{0.824468in}{0.243158in}}%
\pgfusepath{clip}%
\pgfsetbuttcap%
\pgfsetmiterjoin%
\definecolor{currentfill}{rgb}{0.121569,0.466667,0.705882}%
\pgfsetfillcolor{currentfill}%
\pgfsetfillopacity{0.500000}%
\pgfsetlinewidth{1.003750pt}%
\definecolor{currentstroke}{rgb}{0.000000,0.000000,0.000000}%
\pgfsetstrokecolor{currentstroke}%
\pgfsetdash{}{0pt}%
\pgfpathmoveto{\pgfqpoint{7.111619in}{3.795980in}}%
\pgfpathlineto{\pgfqpoint{7.261522in}{3.795980in}}%
\pgfpathlineto{\pgfqpoint{7.261522in}{3.831237in}}%
\pgfpathlineto{\pgfqpoint{7.111619in}{3.831237in}}%
\pgfpathlineto{\pgfqpoint{7.111619in}{3.795980in}}%
\pgfpathclose%
\pgfusepath{stroke,fill}%
\end{pgfscope}%
\begin{pgfscope}%
\pgfpathrectangle{\pgfqpoint{7.074143in}{3.795980in}}{\pgfqpoint{0.824468in}{0.243158in}}%
\pgfusepath{clip}%
\pgfsetbuttcap%
\pgfsetmiterjoin%
\definecolor{currentfill}{rgb}{0.121569,0.466667,0.705882}%
\pgfsetfillcolor{currentfill}%
\pgfsetfillopacity{0.500000}%
\pgfsetlinewidth{1.003750pt}%
\definecolor{currentstroke}{rgb}{0.000000,0.000000,0.000000}%
\pgfsetstrokecolor{currentstroke}%
\pgfsetdash{}{0pt}%
\pgfpathmoveto{\pgfqpoint{7.261522in}{3.795980in}}%
\pgfpathlineto{\pgfqpoint{7.411425in}{3.795980in}}%
\pgfpathlineto{\pgfqpoint{7.411425in}{3.818958in}}%
\pgfpathlineto{\pgfqpoint{7.261522in}{3.818958in}}%
\pgfpathlineto{\pgfqpoint{7.261522in}{3.795980in}}%
\pgfpathclose%
\pgfusepath{stroke,fill}%
\end{pgfscope}%
\begin{pgfscope}%
\pgfpathrectangle{\pgfqpoint{7.074143in}{3.795980in}}{\pgfqpoint{0.824468in}{0.243158in}}%
\pgfusepath{clip}%
\pgfsetbuttcap%
\pgfsetmiterjoin%
\definecolor{currentfill}{rgb}{0.121569,0.466667,0.705882}%
\pgfsetfillcolor{currentfill}%
\pgfsetfillopacity{0.500000}%
\pgfsetlinewidth{1.003750pt}%
\definecolor{currentstroke}{rgb}{0.000000,0.000000,0.000000}%
\pgfsetstrokecolor{currentstroke}%
\pgfsetdash{}{0pt}%
\pgfpathmoveto{\pgfqpoint{7.411425in}{3.795980in}}%
\pgfpathlineto{\pgfqpoint{7.561329in}{3.795980in}}%
\pgfpathlineto{\pgfqpoint{7.561329in}{3.806435in}}%
\pgfpathlineto{\pgfqpoint{7.411425in}{3.806435in}}%
\pgfpathlineto{\pgfqpoint{7.411425in}{3.795980in}}%
\pgfpathclose%
\pgfusepath{stroke,fill}%
\end{pgfscope}%
\begin{pgfscope}%
\pgfpathrectangle{\pgfqpoint{7.074143in}{3.795980in}}{\pgfqpoint{0.824468in}{0.243158in}}%
\pgfusepath{clip}%
\pgfsetbuttcap%
\pgfsetmiterjoin%
\definecolor{currentfill}{rgb}{0.121569,0.466667,0.705882}%
\pgfsetfillcolor{currentfill}%
\pgfsetfillopacity{0.500000}%
\pgfsetlinewidth{1.003750pt}%
\definecolor{currentstroke}{rgb}{0.000000,0.000000,0.000000}%
\pgfsetstrokecolor{currentstroke}%
\pgfsetdash{}{0pt}%
\pgfpathmoveto{\pgfqpoint{7.561329in}{3.795980in}}%
\pgfpathlineto{\pgfqpoint{7.711232in}{3.795980in}}%
\pgfpathlineto{\pgfqpoint{7.711232in}{3.799748in}}%
\pgfpathlineto{\pgfqpoint{7.561329in}{3.799748in}}%
\pgfpathlineto{\pgfqpoint{7.561329in}{3.795980in}}%
\pgfpathclose%
\pgfusepath{stroke,fill}%
\end{pgfscope}%
\begin{pgfscope}%
\pgfpathrectangle{\pgfqpoint{7.074143in}{3.795980in}}{\pgfqpoint{0.824468in}{0.243158in}}%
\pgfusepath{clip}%
\pgfsetbuttcap%
\pgfsetmiterjoin%
\definecolor{currentfill}{rgb}{0.121569,0.466667,0.705882}%
\pgfsetfillcolor{currentfill}%
\pgfsetfillopacity{0.500000}%
\pgfsetlinewidth{1.003750pt}%
\definecolor{currentstroke}{rgb}{0.000000,0.000000,0.000000}%
\pgfsetstrokecolor{currentstroke}%
\pgfsetdash{}{0pt}%
\pgfpathmoveto{\pgfqpoint{7.711232in}{3.795980in}}%
\pgfpathlineto{\pgfqpoint{7.861135in}{3.795980in}}%
\pgfpathlineto{\pgfqpoint{7.861135in}{3.797925in}}%
\pgfpathlineto{\pgfqpoint{7.711232in}{3.797925in}}%
\pgfpathlineto{\pgfqpoint{7.711232in}{3.795980in}}%
\pgfpathclose%
\pgfusepath{stroke,fill}%
\end{pgfscope}%
\begin{pgfscope}%
\pgfsetrectcap%
\pgfsetmiterjoin%
\pgfsetlinewidth{0.803000pt}%
\definecolor{currentstroke}{rgb}{0.000000,0.000000,0.000000}%
\pgfsetstrokecolor{currentstroke}%
\pgfsetdash{}{0pt}%
\pgfpathmoveto{\pgfqpoint{7.074143in}{3.795980in}}%
\pgfpathlineto{\pgfqpoint{7.074143in}{4.039137in}}%
\pgfusepath{stroke}%
\end{pgfscope}%
\begin{pgfscope}%
\pgfsetrectcap%
\pgfsetmiterjoin%
\pgfsetlinewidth{0.803000pt}%
\definecolor{currentstroke}{rgb}{0.000000,0.000000,0.000000}%
\pgfsetstrokecolor{currentstroke}%
\pgfsetdash{}{0pt}%
\pgfpathmoveto{\pgfqpoint{7.898611in}{3.795980in}}%
\pgfpathlineto{\pgfqpoint{7.898611in}{4.039137in}}%
\pgfusepath{stroke}%
\end{pgfscope}%
\begin{pgfscope}%
\pgfsetrectcap%
\pgfsetmiterjoin%
\pgfsetlinewidth{0.803000pt}%
\definecolor{currentstroke}{rgb}{0.000000,0.000000,0.000000}%
\pgfsetstrokecolor{currentstroke}%
\pgfsetdash{}{0pt}%
\pgfpathmoveto{\pgfqpoint{7.074143in}{3.795980in}}%
\pgfpathlineto{\pgfqpoint{7.898611in}{3.795980in}}%
\pgfusepath{stroke}%
\end{pgfscope}%
\begin{pgfscope}%
\pgfsetrectcap%
\pgfsetmiterjoin%
\pgfsetlinewidth{0.803000pt}%
\definecolor{currentstroke}{rgb}{0.000000,0.000000,0.000000}%
\pgfsetstrokecolor{currentstroke}%
\pgfsetdash{}{0pt}%
\pgfpathmoveto{\pgfqpoint{7.074143in}{4.039137in}}%
\pgfpathlineto{\pgfqpoint{7.898611in}{4.039137in}}%
\pgfusepath{stroke}%
\end{pgfscope}%
\begin{pgfscope}%
\definecolor{textcolor}{rgb}{0.000000,0.000000,0.000000}%
\pgfsetstrokecolor{textcolor}%
\pgfsetfillcolor{textcolor}%
\pgftext[x=7.486377in,y=4.122471in,,base]{\color{textcolor}\rmfamily\fontsize{11.000000}{13.200000}\selectfont MAAF}%
\end{pgfscope}%
\begin{pgfscope}%
\pgfsetbuttcap%
\pgfsetmiterjoin%
\definecolor{currentfill}{rgb}{1.000000,1.000000,1.000000}%
\pgfsetfillcolor{currentfill}%
\pgfsetlinewidth{0.000000pt}%
\definecolor{currentstroke}{rgb}{0.000000,0.000000,0.000000}%
\pgfsetstrokecolor{currentstroke}%
\pgfsetstrokeopacity{0.000000}%
\pgfsetdash{}{0pt}%
\pgfpathmoveto{\pgfqpoint{0.148611in}{3.066506in}}%
\pgfpathlineto{\pgfqpoint{0.973079in}{3.066506in}}%
\pgfpathlineto{\pgfqpoint{0.973079in}{3.309664in}}%
\pgfpathlineto{\pgfqpoint{0.148611in}{3.309664in}}%
\pgfpathlineto{\pgfqpoint{0.148611in}{3.066506in}}%
\pgfpathclose%
\pgfusepath{fill}%
\end{pgfscope}%
\begin{pgfscope}%
\pgfpathrectangle{\pgfqpoint{0.148611in}{3.066506in}}{\pgfqpoint{0.824468in}{0.243158in}}%
\pgfusepath{clip}%
\pgfsetbuttcap%
\pgfsetmiterjoin%
\definecolor{currentfill}{rgb}{0.121569,0.466667,0.705882}%
\pgfsetfillcolor{currentfill}%
\pgfsetfillopacity{0.500000}%
\pgfsetlinewidth{1.003750pt}%
\definecolor{currentstroke}{rgb}{0.000000,0.000000,0.000000}%
\pgfsetstrokecolor{currentstroke}%
\pgfsetdash{}{0pt}%
\pgfpathmoveto{\pgfqpoint{0.186087in}{3.066506in}}%
\pgfpathlineto{\pgfqpoint{0.335990in}{3.066506in}}%
\pgfpathlineto{\pgfqpoint{0.335990in}{3.068208in}}%
\pgfpathlineto{\pgfqpoint{0.186087in}{3.068208in}}%
\pgfpathlineto{\pgfqpoint{0.186087in}{3.066506in}}%
\pgfpathclose%
\pgfusepath{stroke,fill}%
\end{pgfscope}%
\begin{pgfscope}%
\pgfpathrectangle{\pgfqpoint{0.148611in}{3.066506in}}{\pgfqpoint{0.824468in}{0.243158in}}%
\pgfusepath{clip}%
\pgfsetbuttcap%
\pgfsetmiterjoin%
\definecolor{currentfill}{rgb}{0.121569,0.466667,0.705882}%
\pgfsetfillcolor{currentfill}%
\pgfsetfillopacity{0.500000}%
\pgfsetlinewidth{1.003750pt}%
\definecolor{currentstroke}{rgb}{0.000000,0.000000,0.000000}%
\pgfsetstrokecolor{currentstroke}%
\pgfsetdash{}{0pt}%
\pgfpathmoveto{\pgfqpoint{0.335990in}{3.066506in}}%
\pgfpathlineto{\pgfqpoint{0.485894in}{3.066506in}}%
\pgfpathlineto{\pgfqpoint{0.485894in}{3.067114in}}%
\pgfpathlineto{\pgfqpoint{0.335990in}{3.067114in}}%
\pgfpathlineto{\pgfqpoint{0.335990in}{3.066506in}}%
\pgfpathclose%
\pgfusepath{stroke,fill}%
\end{pgfscope}%
\begin{pgfscope}%
\pgfpathrectangle{\pgfqpoint{0.148611in}{3.066506in}}{\pgfqpoint{0.824468in}{0.243158in}}%
\pgfusepath{clip}%
\pgfsetbuttcap%
\pgfsetmiterjoin%
\definecolor{currentfill}{rgb}{0.121569,0.466667,0.705882}%
\pgfsetfillcolor{currentfill}%
\pgfsetfillopacity{0.500000}%
\pgfsetlinewidth{1.003750pt}%
\definecolor{currentstroke}{rgb}{0.000000,0.000000,0.000000}%
\pgfsetstrokecolor{currentstroke}%
\pgfsetdash{}{0pt}%
\pgfpathmoveto{\pgfqpoint{0.485894in}{3.066506in}}%
\pgfpathlineto{\pgfqpoint{0.635797in}{3.066506in}}%
\pgfpathlineto{\pgfqpoint{0.635797in}{3.066506in}}%
\pgfpathlineto{\pgfqpoint{0.485894in}{3.066506in}}%
\pgfpathlineto{\pgfqpoint{0.485894in}{3.066506in}}%
\pgfpathclose%
\pgfusepath{stroke,fill}%
\end{pgfscope}%
\begin{pgfscope}%
\pgfpathrectangle{\pgfqpoint{0.148611in}{3.066506in}}{\pgfqpoint{0.824468in}{0.243158in}}%
\pgfusepath{clip}%
\pgfsetbuttcap%
\pgfsetmiterjoin%
\definecolor{currentfill}{rgb}{0.121569,0.466667,0.705882}%
\pgfsetfillcolor{currentfill}%
\pgfsetfillopacity{0.500000}%
\pgfsetlinewidth{1.003750pt}%
\definecolor{currentstroke}{rgb}{0.000000,0.000000,0.000000}%
\pgfsetstrokecolor{currentstroke}%
\pgfsetdash{}{0pt}%
\pgfpathmoveto{\pgfqpoint{0.635797in}{3.066506in}}%
\pgfpathlineto{\pgfqpoint{0.785700in}{3.066506in}}%
\pgfpathlineto{\pgfqpoint{0.785700in}{3.066871in}}%
\pgfpathlineto{\pgfqpoint{0.635797in}{3.066871in}}%
\pgfpathlineto{\pgfqpoint{0.635797in}{3.066506in}}%
\pgfpathclose%
\pgfusepath{stroke,fill}%
\end{pgfscope}%
\begin{pgfscope}%
\pgfpathrectangle{\pgfqpoint{0.148611in}{3.066506in}}{\pgfqpoint{0.824468in}{0.243158in}}%
\pgfusepath{clip}%
\pgfsetbuttcap%
\pgfsetmiterjoin%
\definecolor{currentfill}{rgb}{0.121569,0.466667,0.705882}%
\pgfsetfillcolor{currentfill}%
\pgfsetfillopacity{0.500000}%
\pgfsetlinewidth{1.003750pt}%
\definecolor{currentstroke}{rgb}{0.000000,0.000000,0.000000}%
\pgfsetstrokecolor{currentstroke}%
\pgfsetdash{}{0pt}%
\pgfpathmoveto{\pgfqpoint{0.785700in}{3.066506in}}%
\pgfpathlineto{\pgfqpoint{0.935603in}{3.066506in}}%
\pgfpathlineto{\pgfqpoint{0.935603in}{3.066871in}}%
\pgfpathlineto{\pgfqpoint{0.785700in}{3.066871in}}%
\pgfpathlineto{\pgfqpoint{0.785700in}{3.066506in}}%
\pgfpathclose%
\pgfusepath{stroke,fill}%
\end{pgfscope}%
\begin{pgfscope}%
\pgfsetrectcap%
\pgfsetmiterjoin%
\pgfsetlinewidth{0.803000pt}%
\definecolor{currentstroke}{rgb}{0.000000,0.000000,0.000000}%
\pgfsetstrokecolor{currentstroke}%
\pgfsetdash{}{0pt}%
\pgfpathmoveto{\pgfqpoint{0.148611in}{3.066506in}}%
\pgfpathlineto{\pgfqpoint{0.148611in}{3.309664in}}%
\pgfusepath{stroke}%
\end{pgfscope}%
\begin{pgfscope}%
\pgfsetrectcap%
\pgfsetmiterjoin%
\pgfsetlinewidth{0.803000pt}%
\definecolor{currentstroke}{rgb}{0.000000,0.000000,0.000000}%
\pgfsetstrokecolor{currentstroke}%
\pgfsetdash{}{0pt}%
\pgfpathmoveto{\pgfqpoint{0.973079in}{3.066506in}}%
\pgfpathlineto{\pgfqpoint{0.973079in}{3.309664in}}%
\pgfusepath{stroke}%
\end{pgfscope}%
\begin{pgfscope}%
\pgfsetrectcap%
\pgfsetmiterjoin%
\pgfsetlinewidth{0.803000pt}%
\definecolor{currentstroke}{rgb}{0.000000,0.000000,0.000000}%
\pgfsetstrokecolor{currentstroke}%
\pgfsetdash{}{0pt}%
\pgfpathmoveto{\pgfqpoint{0.148611in}{3.066506in}}%
\pgfpathlineto{\pgfqpoint{0.973079in}{3.066506in}}%
\pgfusepath{stroke}%
\end{pgfscope}%
\begin{pgfscope}%
\pgfsetrectcap%
\pgfsetmiterjoin%
\pgfsetlinewidth{0.803000pt}%
\definecolor{currentstroke}{rgb}{0.000000,0.000000,0.000000}%
\pgfsetstrokecolor{currentstroke}%
\pgfsetdash{}{0pt}%
\pgfpathmoveto{\pgfqpoint{0.148611in}{3.309664in}}%
\pgfpathlineto{\pgfqpoint{0.973079in}{3.309664in}}%
\pgfusepath{stroke}%
\end{pgfscope}%
\begin{pgfscope}%
\definecolor{textcolor}{rgb}{0.000000,0.000000,0.000000}%
\pgfsetstrokecolor{textcolor}%
\pgfsetfillcolor{textcolor}%
\pgftext[x=0.560845in,y=3.392997in,,base]{\color{textcolor}\rmfamily\fontsize{11.000000}{13.200000}\selectfont Solly ...}%
\end{pgfscope}%
\begin{pgfscope}%
\pgfsetbuttcap%
\pgfsetmiterjoin%
\definecolor{currentfill}{rgb}{1.000000,1.000000,1.000000}%
\pgfsetfillcolor{currentfill}%
\pgfsetlinewidth{0.000000pt}%
\definecolor{currentstroke}{rgb}{0.000000,0.000000,0.000000}%
\pgfsetstrokecolor{currentstroke}%
\pgfsetstrokeopacity{0.000000}%
\pgfsetdash{}{0pt}%
\pgfpathmoveto{\pgfqpoint{1.137973in}{3.066506in}}%
\pgfpathlineto{\pgfqpoint{1.962441in}{3.066506in}}%
\pgfpathlineto{\pgfqpoint{1.962441in}{3.309664in}}%
\pgfpathlineto{\pgfqpoint{1.137973in}{3.309664in}}%
\pgfpathlineto{\pgfqpoint{1.137973in}{3.066506in}}%
\pgfpathclose%
\pgfusepath{fill}%
\end{pgfscope}%
\begin{pgfscope}%
\pgfpathrectangle{\pgfqpoint{1.137973in}{3.066506in}}{\pgfqpoint{0.824468in}{0.243158in}}%
\pgfusepath{clip}%
\pgfsetbuttcap%
\pgfsetmiterjoin%
\definecolor{currentfill}{rgb}{0.121569,0.466667,0.705882}%
\pgfsetfillcolor{currentfill}%
\pgfsetfillopacity{0.500000}%
\pgfsetlinewidth{1.003750pt}%
\definecolor{currentstroke}{rgb}{0.000000,0.000000,0.000000}%
\pgfsetstrokecolor{currentstroke}%
\pgfsetdash{}{0pt}%
\pgfpathmoveto{\pgfqpoint{1.175449in}{3.066506in}}%
\pgfpathlineto{\pgfqpoint{1.325352in}{3.066506in}}%
\pgfpathlineto{\pgfqpoint{1.325352in}{3.092645in}}%
\pgfpathlineto{\pgfqpoint{1.175449in}{3.092645in}}%
\pgfpathlineto{\pgfqpoint{1.175449in}{3.066506in}}%
\pgfpathclose%
\pgfusepath{stroke,fill}%
\end{pgfscope}%
\begin{pgfscope}%
\pgfpathrectangle{\pgfqpoint{1.137973in}{3.066506in}}{\pgfqpoint{0.824468in}{0.243158in}}%
\pgfusepath{clip}%
\pgfsetbuttcap%
\pgfsetmiterjoin%
\definecolor{currentfill}{rgb}{0.121569,0.466667,0.705882}%
\pgfsetfillcolor{currentfill}%
\pgfsetfillopacity{0.500000}%
\pgfsetlinewidth{1.003750pt}%
\definecolor{currentstroke}{rgb}{0.000000,0.000000,0.000000}%
\pgfsetstrokecolor{currentstroke}%
\pgfsetdash{}{0pt}%
\pgfpathmoveto{\pgfqpoint{1.325352in}{3.066506in}}%
\pgfpathlineto{\pgfqpoint{1.475255in}{3.066506in}}%
\pgfpathlineto{\pgfqpoint{1.475255in}{3.093375in}}%
\pgfpathlineto{\pgfqpoint{1.325352in}{3.093375in}}%
\pgfpathlineto{\pgfqpoint{1.325352in}{3.066506in}}%
\pgfpathclose%
\pgfusepath{stroke,fill}%
\end{pgfscope}%
\begin{pgfscope}%
\pgfpathrectangle{\pgfqpoint{1.137973in}{3.066506in}}{\pgfqpoint{0.824468in}{0.243158in}}%
\pgfusepath{clip}%
\pgfsetbuttcap%
\pgfsetmiterjoin%
\definecolor{currentfill}{rgb}{0.121569,0.466667,0.705882}%
\pgfsetfillcolor{currentfill}%
\pgfsetfillopacity{0.500000}%
\pgfsetlinewidth{1.003750pt}%
\definecolor{currentstroke}{rgb}{0.000000,0.000000,0.000000}%
\pgfsetstrokecolor{currentstroke}%
\pgfsetdash{}{0pt}%
\pgfpathmoveto{\pgfqpoint{1.475255in}{3.066506in}}%
\pgfpathlineto{\pgfqpoint{1.625158in}{3.066506in}}%
\pgfpathlineto{\pgfqpoint{1.625158in}{3.088390in}}%
\pgfpathlineto{\pgfqpoint{1.475255in}{3.088390in}}%
\pgfpathlineto{\pgfqpoint{1.475255in}{3.066506in}}%
\pgfpathclose%
\pgfusepath{stroke,fill}%
\end{pgfscope}%
\begin{pgfscope}%
\pgfpathrectangle{\pgfqpoint{1.137973in}{3.066506in}}{\pgfqpoint{0.824468in}{0.243158in}}%
\pgfusepath{clip}%
\pgfsetbuttcap%
\pgfsetmiterjoin%
\definecolor{currentfill}{rgb}{0.121569,0.466667,0.705882}%
\pgfsetfillcolor{currentfill}%
\pgfsetfillopacity{0.500000}%
\pgfsetlinewidth{1.003750pt}%
\definecolor{currentstroke}{rgb}{0.000000,0.000000,0.000000}%
\pgfsetstrokecolor{currentstroke}%
\pgfsetdash{}{0pt}%
\pgfpathmoveto{\pgfqpoint{1.625158in}{3.066506in}}%
\pgfpathlineto{\pgfqpoint{1.775062in}{3.066506in}}%
\pgfpathlineto{\pgfqpoint{1.775062in}{3.094104in}}%
\pgfpathlineto{\pgfqpoint{1.625158in}{3.094104in}}%
\pgfpathlineto{\pgfqpoint{1.625158in}{3.066506in}}%
\pgfpathclose%
\pgfusepath{stroke,fill}%
\end{pgfscope}%
\begin{pgfscope}%
\pgfpathrectangle{\pgfqpoint{1.137973in}{3.066506in}}{\pgfqpoint{0.824468in}{0.243158in}}%
\pgfusepath{clip}%
\pgfsetbuttcap%
\pgfsetmiterjoin%
\definecolor{currentfill}{rgb}{0.121569,0.466667,0.705882}%
\pgfsetfillcolor{currentfill}%
\pgfsetfillopacity{0.500000}%
\pgfsetlinewidth{1.003750pt}%
\definecolor{currentstroke}{rgb}{0.000000,0.000000,0.000000}%
\pgfsetstrokecolor{currentstroke}%
\pgfsetdash{}{0pt}%
\pgfpathmoveto{\pgfqpoint{1.775062in}{3.066506in}}%
\pgfpathlineto{\pgfqpoint{1.924965in}{3.066506in}}%
\pgfpathlineto{\pgfqpoint{1.924965in}{3.085351in}}%
\pgfpathlineto{\pgfqpoint{1.775062in}{3.085351in}}%
\pgfpathlineto{\pgfqpoint{1.775062in}{3.066506in}}%
\pgfpathclose%
\pgfusepath{stroke,fill}%
\end{pgfscope}%
\begin{pgfscope}%
\pgfsetrectcap%
\pgfsetmiterjoin%
\pgfsetlinewidth{0.803000pt}%
\definecolor{currentstroke}{rgb}{0.000000,0.000000,0.000000}%
\pgfsetstrokecolor{currentstroke}%
\pgfsetdash{}{0pt}%
\pgfpathmoveto{\pgfqpoint{1.137973in}{3.066506in}}%
\pgfpathlineto{\pgfqpoint{1.137973in}{3.309664in}}%
\pgfusepath{stroke}%
\end{pgfscope}%
\begin{pgfscope}%
\pgfsetrectcap%
\pgfsetmiterjoin%
\pgfsetlinewidth{0.803000pt}%
\definecolor{currentstroke}{rgb}{0.000000,0.000000,0.000000}%
\pgfsetstrokecolor{currentstroke}%
\pgfsetdash{}{0pt}%
\pgfpathmoveto{\pgfqpoint{1.962441in}{3.066506in}}%
\pgfpathlineto{\pgfqpoint{1.962441in}{3.309664in}}%
\pgfusepath{stroke}%
\end{pgfscope}%
\begin{pgfscope}%
\pgfsetrectcap%
\pgfsetmiterjoin%
\pgfsetlinewidth{0.803000pt}%
\definecolor{currentstroke}{rgb}{0.000000,0.000000,0.000000}%
\pgfsetstrokecolor{currentstroke}%
\pgfsetdash{}{0pt}%
\pgfpathmoveto{\pgfqpoint{1.137973in}{3.066506in}}%
\pgfpathlineto{\pgfqpoint{1.962441in}{3.066506in}}%
\pgfusepath{stroke}%
\end{pgfscope}%
\begin{pgfscope}%
\pgfsetrectcap%
\pgfsetmiterjoin%
\pgfsetlinewidth{0.803000pt}%
\definecolor{currentstroke}{rgb}{0.000000,0.000000,0.000000}%
\pgfsetstrokecolor{currentstroke}%
\pgfsetdash{}{0pt}%
\pgfpathmoveto{\pgfqpoint{1.137973in}{3.309664in}}%
\pgfpathlineto{\pgfqpoint{1.962441in}{3.309664in}}%
\pgfusepath{stroke}%
\end{pgfscope}%
\begin{pgfscope}%
\definecolor{textcolor}{rgb}{0.000000,0.000000,0.000000}%
\pgfsetstrokecolor{textcolor}%
\pgfsetfillcolor{textcolor}%
\pgftext[x=1.550207in,y=3.392997in,,base]{\color{textcolor}\rmfamily\fontsize{11.000000}{13.200000}\selectfont GMF}%
\end{pgfscope}%
\begin{pgfscope}%
\pgfsetbuttcap%
\pgfsetmiterjoin%
\definecolor{currentfill}{rgb}{1.000000,1.000000,1.000000}%
\pgfsetfillcolor{currentfill}%
\pgfsetlinewidth{0.000000pt}%
\definecolor{currentstroke}{rgb}{0.000000,0.000000,0.000000}%
\pgfsetstrokecolor{currentstroke}%
\pgfsetstrokeopacity{0.000000}%
\pgfsetdash{}{0pt}%
\pgfpathmoveto{\pgfqpoint{2.127335in}{3.066506in}}%
\pgfpathlineto{\pgfqpoint{2.951803in}{3.066506in}}%
\pgfpathlineto{\pgfqpoint{2.951803in}{3.309664in}}%
\pgfpathlineto{\pgfqpoint{2.127335in}{3.309664in}}%
\pgfpathlineto{\pgfqpoint{2.127335in}{3.066506in}}%
\pgfpathclose%
\pgfusepath{fill}%
\end{pgfscope}%
\begin{pgfscope}%
\pgfpathrectangle{\pgfqpoint{2.127335in}{3.066506in}}{\pgfqpoint{0.824468in}{0.243158in}}%
\pgfusepath{clip}%
\pgfsetbuttcap%
\pgfsetmiterjoin%
\definecolor{currentfill}{rgb}{0.121569,0.466667,0.705882}%
\pgfsetfillcolor{currentfill}%
\pgfsetfillopacity{0.500000}%
\pgfsetlinewidth{1.003750pt}%
\definecolor{currentstroke}{rgb}{0.000000,0.000000,0.000000}%
\pgfsetstrokecolor{currentstroke}%
\pgfsetdash{}{0pt}%
\pgfpathmoveto{\pgfqpoint{2.164810in}{3.066506in}}%
\pgfpathlineto{\pgfqpoint{2.314714in}{3.066506in}}%
\pgfpathlineto{\pgfqpoint{2.314714in}{3.075503in}}%
\pgfpathlineto{\pgfqpoint{2.164810in}{3.075503in}}%
\pgfpathlineto{\pgfqpoint{2.164810in}{3.066506in}}%
\pgfpathclose%
\pgfusepath{stroke,fill}%
\end{pgfscope}%
\begin{pgfscope}%
\pgfpathrectangle{\pgfqpoint{2.127335in}{3.066506in}}{\pgfqpoint{0.824468in}{0.243158in}}%
\pgfusepath{clip}%
\pgfsetbuttcap%
\pgfsetmiterjoin%
\definecolor{currentfill}{rgb}{0.121569,0.466667,0.705882}%
\pgfsetfillcolor{currentfill}%
\pgfsetfillopacity{0.500000}%
\pgfsetlinewidth{1.003750pt}%
\definecolor{currentstroke}{rgb}{0.000000,0.000000,0.000000}%
\pgfsetstrokecolor{currentstroke}%
\pgfsetdash{}{0pt}%
\pgfpathmoveto{\pgfqpoint{2.314714in}{3.066506in}}%
\pgfpathlineto{\pgfqpoint{2.464617in}{3.066506in}}%
\pgfpathlineto{\pgfqpoint{2.464617in}{3.076962in}}%
\pgfpathlineto{\pgfqpoint{2.314714in}{3.076962in}}%
\pgfpathlineto{\pgfqpoint{2.314714in}{3.066506in}}%
\pgfpathclose%
\pgfusepath{stroke,fill}%
\end{pgfscope}%
\begin{pgfscope}%
\pgfpathrectangle{\pgfqpoint{2.127335in}{3.066506in}}{\pgfqpoint{0.824468in}{0.243158in}}%
\pgfusepath{clip}%
\pgfsetbuttcap%
\pgfsetmiterjoin%
\definecolor{currentfill}{rgb}{0.121569,0.466667,0.705882}%
\pgfsetfillcolor{currentfill}%
\pgfsetfillopacity{0.500000}%
\pgfsetlinewidth{1.003750pt}%
\definecolor{currentstroke}{rgb}{0.000000,0.000000,0.000000}%
\pgfsetstrokecolor{currentstroke}%
\pgfsetdash{}{0pt}%
\pgfpathmoveto{\pgfqpoint{2.464617in}{3.066506in}}%
\pgfpathlineto{\pgfqpoint{2.614520in}{3.066506in}}%
\pgfpathlineto{\pgfqpoint{2.614520in}{3.076232in}}%
\pgfpathlineto{\pgfqpoint{2.464617in}{3.076232in}}%
\pgfpathlineto{\pgfqpoint{2.464617in}{3.066506in}}%
\pgfpathclose%
\pgfusepath{stroke,fill}%
\end{pgfscope}%
\begin{pgfscope}%
\pgfpathrectangle{\pgfqpoint{2.127335in}{3.066506in}}{\pgfqpoint{0.824468in}{0.243158in}}%
\pgfusepath{clip}%
\pgfsetbuttcap%
\pgfsetmiterjoin%
\definecolor{currentfill}{rgb}{0.121569,0.466667,0.705882}%
\pgfsetfillcolor{currentfill}%
\pgfsetfillopacity{0.500000}%
\pgfsetlinewidth{1.003750pt}%
\definecolor{currentstroke}{rgb}{0.000000,0.000000,0.000000}%
\pgfsetstrokecolor{currentstroke}%
\pgfsetdash{}{0pt}%
\pgfpathmoveto{\pgfqpoint{2.614520in}{3.066506in}}%
\pgfpathlineto{\pgfqpoint{2.764423in}{3.066506in}}%
\pgfpathlineto{\pgfqpoint{2.764423in}{3.089120in}}%
\pgfpathlineto{\pgfqpoint{2.614520in}{3.089120in}}%
\pgfpathlineto{\pgfqpoint{2.614520in}{3.066506in}}%
\pgfpathclose%
\pgfusepath{stroke,fill}%
\end{pgfscope}%
\begin{pgfscope}%
\pgfpathrectangle{\pgfqpoint{2.127335in}{3.066506in}}{\pgfqpoint{0.824468in}{0.243158in}}%
\pgfusepath{clip}%
\pgfsetbuttcap%
\pgfsetmiterjoin%
\definecolor{currentfill}{rgb}{0.121569,0.466667,0.705882}%
\pgfsetfillcolor{currentfill}%
\pgfsetfillopacity{0.500000}%
\pgfsetlinewidth{1.003750pt}%
\definecolor{currentstroke}{rgb}{0.000000,0.000000,0.000000}%
\pgfsetstrokecolor{currentstroke}%
\pgfsetdash{}{0pt}%
\pgfpathmoveto{\pgfqpoint{2.764423in}{3.066506in}}%
\pgfpathlineto{\pgfqpoint{2.914327in}{3.066506in}}%
\pgfpathlineto{\pgfqpoint{2.914327in}{3.098360in}}%
\pgfpathlineto{\pgfqpoint{2.764423in}{3.098360in}}%
\pgfpathlineto{\pgfqpoint{2.764423in}{3.066506in}}%
\pgfpathclose%
\pgfusepath{stroke,fill}%
\end{pgfscope}%
\begin{pgfscope}%
\pgfsetrectcap%
\pgfsetmiterjoin%
\pgfsetlinewidth{0.803000pt}%
\definecolor{currentstroke}{rgb}{0.000000,0.000000,0.000000}%
\pgfsetstrokecolor{currentstroke}%
\pgfsetdash{}{0pt}%
\pgfpathmoveto{\pgfqpoint{2.127335in}{3.066506in}}%
\pgfpathlineto{\pgfqpoint{2.127335in}{3.309664in}}%
\pgfusepath{stroke}%
\end{pgfscope}%
\begin{pgfscope}%
\pgfsetrectcap%
\pgfsetmiterjoin%
\pgfsetlinewidth{0.803000pt}%
\definecolor{currentstroke}{rgb}{0.000000,0.000000,0.000000}%
\pgfsetstrokecolor{currentstroke}%
\pgfsetdash{}{0pt}%
\pgfpathmoveto{\pgfqpoint{2.951803in}{3.066506in}}%
\pgfpathlineto{\pgfqpoint{2.951803in}{3.309664in}}%
\pgfusepath{stroke}%
\end{pgfscope}%
\begin{pgfscope}%
\pgfsetrectcap%
\pgfsetmiterjoin%
\pgfsetlinewidth{0.803000pt}%
\definecolor{currentstroke}{rgb}{0.000000,0.000000,0.000000}%
\pgfsetstrokecolor{currentstroke}%
\pgfsetdash{}{0pt}%
\pgfpathmoveto{\pgfqpoint{2.127335in}{3.066506in}}%
\pgfpathlineto{\pgfqpoint{2.951803in}{3.066506in}}%
\pgfusepath{stroke}%
\end{pgfscope}%
\begin{pgfscope}%
\pgfsetrectcap%
\pgfsetmiterjoin%
\pgfsetlinewidth{0.803000pt}%
\definecolor{currentstroke}{rgb}{0.000000,0.000000,0.000000}%
\pgfsetstrokecolor{currentstroke}%
\pgfsetdash{}{0pt}%
\pgfpathmoveto{\pgfqpoint{2.127335in}{3.309664in}}%
\pgfpathlineto{\pgfqpoint{2.951803in}{3.309664in}}%
\pgfusepath{stroke}%
\end{pgfscope}%
\begin{pgfscope}%
\definecolor{textcolor}{rgb}{0.000000,0.000000,0.000000}%
\pgfsetstrokecolor{textcolor}%
\pgfsetfillcolor{textcolor}%
\pgftext[x=2.539569in,y=3.392997in,,base]{\color{textcolor}\rmfamily\fontsize{11.000000}{13.200000}\selectfont AMV}%
\end{pgfscope}%
\begin{pgfscope}%
\pgfsetbuttcap%
\pgfsetmiterjoin%
\definecolor{currentfill}{rgb}{1.000000,1.000000,1.000000}%
\pgfsetfillcolor{currentfill}%
\pgfsetlinewidth{0.000000pt}%
\definecolor{currentstroke}{rgb}{0.000000,0.000000,0.000000}%
\pgfsetstrokecolor{currentstroke}%
\pgfsetstrokeopacity{0.000000}%
\pgfsetdash{}{0pt}%
\pgfpathmoveto{\pgfqpoint{3.116696in}{3.066506in}}%
\pgfpathlineto{\pgfqpoint{3.941164in}{3.066506in}}%
\pgfpathlineto{\pgfqpoint{3.941164in}{3.309664in}}%
\pgfpathlineto{\pgfqpoint{3.116696in}{3.309664in}}%
\pgfpathlineto{\pgfqpoint{3.116696in}{3.066506in}}%
\pgfpathclose%
\pgfusepath{fill}%
\end{pgfscope}%
\begin{pgfscope}%
\pgfpathrectangle{\pgfqpoint{3.116696in}{3.066506in}}{\pgfqpoint{0.824468in}{0.243158in}}%
\pgfusepath{clip}%
\pgfsetbuttcap%
\pgfsetmiterjoin%
\definecolor{currentfill}{rgb}{0.121569,0.466667,0.705882}%
\pgfsetfillcolor{currentfill}%
\pgfsetfillopacity{0.500000}%
\pgfsetlinewidth{1.003750pt}%
\definecolor{currentstroke}{rgb}{0.000000,0.000000,0.000000}%
\pgfsetstrokecolor{currentstroke}%
\pgfsetdash{}{0pt}%
\pgfpathmoveto{\pgfqpoint{3.154172in}{3.066506in}}%
\pgfpathlineto{\pgfqpoint{3.304075in}{3.066506in}}%
\pgfpathlineto{\pgfqpoint{3.304075in}{3.080001in}}%
\pgfpathlineto{\pgfqpoint{3.154172in}{3.080001in}}%
\pgfpathlineto{\pgfqpoint{3.154172in}{3.066506in}}%
\pgfpathclose%
\pgfusepath{stroke,fill}%
\end{pgfscope}%
\begin{pgfscope}%
\pgfpathrectangle{\pgfqpoint{3.116696in}{3.066506in}}{\pgfqpoint{0.824468in}{0.243158in}}%
\pgfusepath{clip}%
\pgfsetbuttcap%
\pgfsetmiterjoin%
\definecolor{currentfill}{rgb}{0.121569,0.466667,0.705882}%
\pgfsetfillcolor{currentfill}%
\pgfsetfillopacity{0.500000}%
\pgfsetlinewidth{1.003750pt}%
\definecolor{currentstroke}{rgb}{0.000000,0.000000,0.000000}%
\pgfsetstrokecolor{currentstroke}%
\pgfsetdash{}{0pt}%
\pgfpathmoveto{\pgfqpoint{3.304075in}{3.066506in}}%
\pgfpathlineto{\pgfqpoint{3.453979in}{3.066506in}}%
\pgfpathlineto{\pgfqpoint{3.453979in}{3.068573in}}%
\pgfpathlineto{\pgfqpoint{3.304075in}{3.068573in}}%
\pgfpathlineto{\pgfqpoint{3.304075in}{3.066506in}}%
\pgfpathclose%
\pgfusepath{stroke,fill}%
\end{pgfscope}%
\begin{pgfscope}%
\pgfpathrectangle{\pgfqpoint{3.116696in}{3.066506in}}{\pgfqpoint{0.824468in}{0.243158in}}%
\pgfusepath{clip}%
\pgfsetbuttcap%
\pgfsetmiterjoin%
\definecolor{currentfill}{rgb}{0.121569,0.466667,0.705882}%
\pgfsetfillcolor{currentfill}%
\pgfsetfillopacity{0.500000}%
\pgfsetlinewidth{1.003750pt}%
\definecolor{currentstroke}{rgb}{0.000000,0.000000,0.000000}%
\pgfsetstrokecolor{currentstroke}%
\pgfsetdash{}{0pt}%
\pgfpathmoveto{\pgfqpoint{3.453979in}{3.066506in}}%
\pgfpathlineto{\pgfqpoint{3.603882in}{3.066506in}}%
\pgfpathlineto{\pgfqpoint{3.603882in}{3.068330in}}%
\pgfpathlineto{\pgfqpoint{3.453979in}{3.068330in}}%
\pgfpathlineto{\pgfqpoint{3.453979in}{3.066506in}}%
\pgfpathclose%
\pgfusepath{stroke,fill}%
\end{pgfscope}%
\begin{pgfscope}%
\pgfpathrectangle{\pgfqpoint{3.116696in}{3.066506in}}{\pgfqpoint{0.824468in}{0.243158in}}%
\pgfusepath{clip}%
\pgfsetbuttcap%
\pgfsetmiterjoin%
\definecolor{currentfill}{rgb}{0.121569,0.466667,0.705882}%
\pgfsetfillcolor{currentfill}%
\pgfsetfillopacity{0.500000}%
\pgfsetlinewidth{1.003750pt}%
\definecolor{currentstroke}{rgb}{0.000000,0.000000,0.000000}%
\pgfsetstrokecolor{currentstroke}%
\pgfsetdash{}{0pt}%
\pgfpathmoveto{\pgfqpoint{3.603882in}{3.066506in}}%
\pgfpathlineto{\pgfqpoint{3.753785in}{3.066506in}}%
\pgfpathlineto{\pgfqpoint{3.753785in}{3.066506in}}%
\pgfpathlineto{\pgfqpoint{3.603882in}{3.066506in}}%
\pgfpathlineto{\pgfqpoint{3.603882in}{3.066506in}}%
\pgfpathclose%
\pgfusepath{stroke,fill}%
\end{pgfscope}%
\begin{pgfscope}%
\pgfpathrectangle{\pgfqpoint{3.116696in}{3.066506in}}{\pgfqpoint{0.824468in}{0.243158in}}%
\pgfusepath{clip}%
\pgfsetbuttcap%
\pgfsetmiterjoin%
\definecolor{currentfill}{rgb}{0.121569,0.466667,0.705882}%
\pgfsetfillcolor{currentfill}%
\pgfsetfillopacity{0.500000}%
\pgfsetlinewidth{1.003750pt}%
\definecolor{currentstroke}{rgb}{0.000000,0.000000,0.000000}%
\pgfsetstrokecolor{currentstroke}%
\pgfsetdash{}{0pt}%
\pgfpathmoveto{\pgfqpoint{3.753785in}{3.066506in}}%
\pgfpathlineto{\pgfqpoint{3.903688in}{3.066506in}}%
\pgfpathlineto{\pgfqpoint{3.903688in}{3.066992in}}%
\pgfpathlineto{\pgfqpoint{3.753785in}{3.066992in}}%
\pgfpathlineto{\pgfqpoint{3.753785in}{3.066506in}}%
\pgfpathclose%
\pgfusepath{stroke,fill}%
\end{pgfscope}%
\begin{pgfscope}%
\pgfsetrectcap%
\pgfsetmiterjoin%
\pgfsetlinewidth{0.803000pt}%
\definecolor{currentstroke}{rgb}{0.000000,0.000000,0.000000}%
\pgfsetstrokecolor{currentstroke}%
\pgfsetdash{}{0pt}%
\pgfpathmoveto{\pgfqpoint{3.116696in}{3.066506in}}%
\pgfpathlineto{\pgfqpoint{3.116696in}{3.309664in}}%
\pgfusepath{stroke}%
\end{pgfscope}%
\begin{pgfscope}%
\pgfsetrectcap%
\pgfsetmiterjoin%
\pgfsetlinewidth{0.803000pt}%
\definecolor{currentstroke}{rgb}{0.000000,0.000000,0.000000}%
\pgfsetstrokecolor{currentstroke}%
\pgfsetdash{}{0pt}%
\pgfpathmoveto{\pgfqpoint{3.941164in}{3.066506in}}%
\pgfpathlineto{\pgfqpoint{3.941164in}{3.309664in}}%
\pgfusepath{stroke}%
\end{pgfscope}%
\begin{pgfscope}%
\pgfsetrectcap%
\pgfsetmiterjoin%
\pgfsetlinewidth{0.803000pt}%
\definecolor{currentstroke}{rgb}{0.000000,0.000000,0.000000}%
\pgfsetstrokecolor{currentstroke}%
\pgfsetdash{}{0pt}%
\pgfpathmoveto{\pgfqpoint{3.116696in}{3.066506in}}%
\pgfpathlineto{\pgfqpoint{3.941164in}{3.066506in}}%
\pgfusepath{stroke}%
\end{pgfscope}%
\begin{pgfscope}%
\pgfsetrectcap%
\pgfsetmiterjoin%
\pgfsetlinewidth{0.803000pt}%
\definecolor{currentstroke}{rgb}{0.000000,0.000000,0.000000}%
\pgfsetstrokecolor{currentstroke}%
\pgfsetdash{}{0pt}%
\pgfpathmoveto{\pgfqpoint{3.116696in}{3.309664in}}%
\pgfpathlineto{\pgfqpoint{3.941164in}{3.309664in}}%
\pgfusepath{stroke}%
\end{pgfscope}%
\begin{pgfscope}%
\definecolor{textcolor}{rgb}{0.000000,0.000000,0.000000}%
\pgfsetstrokecolor{textcolor}%
\pgfsetfillcolor{textcolor}%
\pgftext[x=3.528930in,y=3.392997in,,base]{\color{textcolor}\rmfamily\fontsize{11.000000}{13.200000}\selectfont CNP As...}%
\end{pgfscope}%
\begin{pgfscope}%
\pgfsetbuttcap%
\pgfsetmiterjoin%
\definecolor{currentfill}{rgb}{1.000000,1.000000,1.000000}%
\pgfsetfillcolor{currentfill}%
\pgfsetlinewidth{0.000000pt}%
\definecolor{currentstroke}{rgb}{0.000000,0.000000,0.000000}%
\pgfsetstrokecolor{currentstroke}%
\pgfsetstrokeopacity{0.000000}%
\pgfsetdash{}{0pt}%
\pgfpathmoveto{\pgfqpoint{4.106058in}{3.066506in}}%
\pgfpathlineto{\pgfqpoint{4.930526in}{3.066506in}}%
\pgfpathlineto{\pgfqpoint{4.930526in}{3.309664in}}%
\pgfpathlineto{\pgfqpoint{4.106058in}{3.309664in}}%
\pgfpathlineto{\pgfqpoint{4.106058in}{3.066506in}}%
\pgfpathclose%
\pgfusepath{fill}%
\end{pgfscope}%
\begin{pgfscope}%
\pgfpathrectangle{\pgfqpoint{4.106058in}{3.066506in}}{\pgfqpoint{0.824468in}{0.243158in}}%
\pgfusepath{clip}%
\pgfsetbuttcap%
\pgfsetmiterjoin%
\definecolor{currentfill}{rgb}{0.121569,0.466667,0.705882}%
\pgfsetfillcolor{currentfill}%
\pgfsetfillopacity{0.500000}%
\pgfsetlinewidth{1.003750pt}%
\definecolor{currentstroke}{rgb}{0.000000,0.000000,0.000000}%
\pgfsetstrokecolor{currentstroke}%
\pgfsetdash{}{0pt}%
\pgfpathmoveto{\pgfqpoint{4.143534in}{3.066506in}}%
\pgfpathlineto{\pgfqpoint{4.293437in}{3.066506in}}%
\pgfpathlineto{\pgfqpoint{4.293437in}{3.104682in}}%
\pgfpathlineto{\pgfqpoint{4.143534in}{3.104682in}}%
\pgfpathlineto{\pgfqpoint{4.143534in}{3.066506in}}%
\pgfpathclose%
\pgfusepath{stroke,fill}%
\end{pgfscope}%
\begin{pgfscope}%
\pgfpathrectangle{\pgfqpoint{4.106058in}{3.066506in}}{\pgfqpoint{0.824468in}{0.243158in}}%
\pgfusepath{clip}%
\pgfsetbuttcap%
\pgfsetmiterjoin%
\definecolor{currentfill}{rgb}{0.121569,0.466667,0.705882}%
\pgfsetfillcolor{currentfill}%
\pgfsetfillopacity{0.500000}%
\pgfsetlinewidth{1.003750pt}%
\definecolor{currentstroke}{rgb}{0.000000,0.000000,0.000000}%
\pgfsetstrokecolor{currentstroke}%
\pgfsetdash{}{0pt}%
\pgfpathmoveto{\pgfqpoint{4.293437in}{3.066506in}}%
\pgfpathlineto{\pgfqpoint{4.443340in}{3.066506in}}%
\pgfpathlineto{\pgfqpoint{4.443340in}{3.084135in}}%
\pgfpathlineto{\pgfqpoint{4.293437in}{3.084135in}}%
\pgfpathlineto{\pgfqpoint{4.293437in}{3.066506in}}%
\pgfpathclose%
\pgfusepath{stroke,fill}%
\end{pgfscope}%
\begin{pgfscope}%
\pgfpathrectangle{\pgfqpoint{4.106058in}{3.066506in}}{\pgfqpoint{0.824468in}{0.243158in}}%
\pgfusepath{clip}%
\pgfsetbuttcap%
\pgfsetmiterjoin%
\definecolor{currentfill}{rgb}{0.121569,0.466667,0.705882}%
\pgfsetfillcolor{currentfill}%
\pgfsetfillopacity{0.500000}%
\pgfsetlinewidth{1.003750pt}%
\definecolor{currentstroke}{rgb}{0.000000,0.000000,0.000000}%
\pgfsetstrokecolor{currentstroke}%
\pgfsetdash{}{0pt}%
\pgfpathmoveto{\pgfqpoint{4.443340in}{3.066506in}}%
\pgfpathlineto{\pgfqpoint{4.593244in}{3.066506in}}%
\pgfpathlineto{\pgfqpoint{4.593244in}{3.074895in}}%
\pgfpathlineto{\pgfqpoint{4.443340in}{3.074895in}}%
\pgfpathlineto{\pgfqpoint{4.443340in}{3.066506in}}%
\pgfpathclose%
\pgfusepath{stroke,fill}%
\end{pgfscope}%
\begin{pgfscope}%
\pgfpathrectangle{\pgfqpoint{4.106058in}{3.066506in}}{\pgfqpoint{0.824468in}{0.243158in}}%
\pgfusepath{clip}%
\pgfsetbuttcap%
\pgfsetmiterjoin%
\definecolor{currentfill}{rgb}{0.121569,0.466667,0.705882}%
\pgfsetfillcolor{currentfill}%
\pgfsetfillopacity{0.500000}%
\pgfsetlinewidth{1.003750pt}%
\definecolor{currentstroke}{rgb}{0.000000,0.000000,0.000000}%
\pgfsetstrokecolor{currentstroke}%
\pgfsetdash{}{0pt}%
\pgfpathmoveto{\pgfqpoint{4.593244in}{3.066506in}}%
\pgfpathlineto{\pgfqpoint{4.743147in}{3.066506in}}%
\pgfpathlineto{\pgfqpoint{4.743147in}{3.071491in}}%
\pgfpathlineto{\pgfqpoint{4.593244in}{3.071491in}}%
\pgfpathlineto{\pgfqpoint{4.593244in}{3.066506in}}%
\pgfpathclose%
\pgfusepath{stroke,fill}%
\end{pgfscope}%
\begin{pgfscope}%
\pgfpathrectangle{\pgfqpoint{4.106058in}{3.066506in}}{\pgfqpoint{0.824468in}{0.243158in}}%
\pgfusepath{clip}%
\pgfsetbuttcap%
\pgfsetmiterjoin%
\definecolor{currentfill}{rgb}{0.121569,0.466667,0.705882}%
\pgfsetfillcolor{currentfill}%
\pgfsetfillopacity{0.500000}%
\pgfsetlinewidth{1.003750pt}%
\definecolor{currentstroke}{rgb}{0.000000,0.000000,0.000000}%
\pgfsetstrokecolor{currentstroke}%
\pgfsetdash{}{0pt}%
\pgfpathmoveto{\pgfqpoint{4.743147in}{3.066506in}}%
\pgfpathlineto{\pgfqpoint{4.893050in}{3.066506in}}%
\pgfpathlineto{\pgfqpoint{4.893050in}{3.069181in}}%
\pgfpathlineto{\pgfqpoint{4.743147in}{3.069181in}}%
\pgfpathlineto{\pgfqpoint{4.743147in}{3.066506in}}%
\pgfpathclose%
\pgfusepath{stroke,fill}%
\end{pgfscope}%
\begin{pgfscope}%
\pgfsetrectcap%
\pgfsetmiterjoin%
\pgfsetlinewidth{0.803000pt}%
\definecolor{currentstroke}{rgb}{0.000000,0.000000,0.000000}%
\pgfsetstrokecolor{currentstroke}%
\pgfsetdash{}{0pt}%
\pgfpathmoveto{\pgfqpoint{4.106058in}{3.066506in}}%
\pgfpathlineto{\pgfqpoint{4.106058in}{3.309664in}}%
\pgfusepath{stroke}%
\end{pgfscope}%
\begin{pgfscope}%
\pgfsetrectcap%
\pgfsetmiterjoin%
\pgfsetlinewidth{0.803000pt}%
\definecolor{currentstroke}{rgb}{0.000000,0.000000,0.000000}%
\pgfsetstrokecolor{currentstroke}%
\pgfsetdash{}{0pt}%
\pgfpathmoveto{\pgfqpoint{4.930526in}{3.066506in}}%
\pgfpathlineto{\pgfqpoint{4.930526in}{3.309664in}}%
\pgfusepath{stroke}%
\end{pgfscope}%
\begin{pgfscope}%
\pgfsetrectcap%
\pgfsetmiterjoin%
\pgfsetlinewidth{0.803000pt}%
\definecolor{currentstroke}{rgb}{0.000000,0.000000,0.000000}%
\pgfsetstrokecolor{currentstroke}%
\pgfsetdash{}{0pt}%
\pgfpathmoveto{\pgfqpoint{4.106058in}{3.066506in}}%
\pgfpathlineto{\pgfqpoint{4.930526in}{3.066506in}}%
\pgfusepath{stroke}%
\end{pgfscope}%
\begin{pgfscope}%
\pgfsetrectcap%
\pgfsetmiterjoin%
\pgfsetlinewidth{0.803000pt}%
\definecolor{currentstroke}{rgb}{0.000000,0.000000,0.000000}%
\pgfsetstrokecolor{currentstroke}%
\pgfsetdash{}{0pt}%
\pgfpathmoveto{\pgfqpoint{4.106058in}{3.309664in}}%
\pgfpathlineto{\pgfqpoint{4.930526in}{3.309664in}}%
\pgfusepath{stroke}%
\end{pgfscope}%
\begin{pgfscope}%
\definecolor{textcolor}{rgb}{0.000000,0.000000,0.000000}%
\pgfsetstrokecolor{textcolor}%
\pgfsetfillcolor{textcolor}%
\pgftext[x=4.518292in,y=3.392997in,,base]{\color{textcolor}\rmfamily\fontsize{11.000000}{13.200000}\selectfont MAIF}%
\end{pgfscope}%
\begin{pgfscope}%
\pgfsetbuttcap%
\pgfsetmiterjoin%
\definecolor{currentfill}{rgb}{1.000000,1.000000,1.000000}%
\pgfsetfillcolor{currentfill}%
\pgfsetlinewidth{0.000000pt}%
\definecolor{currentstroke}{rgb}{0.000000,0.000000,0.000000}%
\pgfsetstrokecolor{currentstroke}%
\pgfsetstrokeopacity{0.000000}%
\pgfsetdash{}{0pt}%
\pgfpathmoveto{\pgfqpoint{5.095420in}{3.066506in}}%
\pgfpathlineto{\pgfqpoint{5.919888in}{3.066506in}}%
\pgfpathlineto{\pgfqpoint{5.919888in}{3.309664in}}%
\pgfpathlineto{\pgfqpoint{5.095420in}{3.309664in}}%
\pgfpathlineto{\pgfqpoint{5.095420in}{3.066506in}}%
\pgfpathclose%
\pgfusepath{fill}%
\end{pgfscope}%
\begin{pgfscope}%
\pgfpathrectangle{\pgfqpoint{5.095420in}{3.066506in}}{\pgfqpoint{0.824468in}{0.243158in}}%
\pgfusepath{clip}%
\pgfsetbuttcap%
\pgfsetmiterjoin%
\definecolor{currentfill}{rgb}{0.121569,0.466667,0.705882}%
\pgfsetfillcolor{currentfill}%
\pgfsetfillopacity{0.500000}%
\pgfsetlinewidth{1.003750pt}%
\definecolor{currentstroke}{rgb}{0.000000,0.000000,0.000000}%
\pgfsetstrokecolor{currentstroke}%
\pgfsetdash{}{0pt}%
\pgfpathmoveto{\pgfqpoint{5.132895in}{3.066506in}}%
\pgfpathlineto{\pgfqpoint{5.282799in}{3.066506in}}%
\pgfpathlineto{\pgfqpoint{5.282799in}{3.072463in}}%
\pgfpathlineto{\pgfqpoint{5.132895in}{3.072463in}}%
\pgfpathlineto{\pgfqpoint{5.132895in}{3.066506in}}%
\pgfpathclose%
\pgfusepath{stroke,fill}%
\end{pgfscope}%
\begin{pgfscope}%
\pgfpathrectangle{\pgfqpoint{5.095420in}{3.066506in}}{\pgfqpoint{0.824468in}{0.243158in}}%
\pgfusepath{clip}%
\pgfsetbuttcap%
\pgfsetmiterjoin%
\definecolor{currentfill}{rgb}{0.121569,0.466667,0.705882}%
\pgfsetfillcolor{currentfill}%
\pgfsetfillopacity{0.500000}%
\pgfsetlinewidth{1.003750pt}%
\definecolor{currentstroke}{rgb}{0.000000,0.000000,0.000000}%
\pgfsetstrokecolor{currentstroke}%
\pgfsetdash{}{0pt}%
\pgfpathmoveto{\pgfqpoint{5.282799in}{3.066506in}}%
\pgfpathlineto{\pgfqpoint{5.432702in}{3.066506in}}%
\pgfpathlineto{\pgfqpoint{5.432702in}{3.067965in}}%
\pgfpathlineto{\pgfqpoint{5.282799in}{3.067965in}}%
\pgfpathlineto{\pgfqpoint{5.282799in}{3.066506in}}%
\pgfpathclose%
\pgfusepath{stroke,fill}%
\end{pgfscope}%
\begin{pgfscope}%
\pgfpathrectangle{\pgfqpoint{5.095420in}{3.066506in}}{\pgfqpoint{0.824468in}{0.243158in}}%
\pgfusepath{clip}%
\pgfsetbuttcap%
\pgfsetmiterjoin%
\definecolor{currentfill}{rgb}{0.121569,0.466667,0.705882}%
\pgfsetfillcolor{currentfill}%
\pgfsetfillopacity{0.500000}%
\pgfsetlinewidth{1.003750pt}%
\definecolor{currentstroke}{rgb}{0.000000,0.000000,0.000000}%
\pgfsetstrokecolor{currentstroke}%
\pgfsetdash{}{0pt}%
\pgfpathmoveto{\pgfqpoint{5.432702in}{3.066506in}}%
\pgfpathlineto{\pgfqpoint{5.582605in}{3.066506in}}%
\pgfpathlineto{\pgfqpoint{5.582605in}{3.067235in}}%
\pgfpathlineto{\pgfqpoint{5.432702in}{3.067235in}}%
\pgfpathlineto{\pgfqpoint{5.432702in}{3.066506in}}%
\pgfpathclose%
\pgfusepath{stroke,fill}%
\end{pgfscope}%
\begin{pgfscope}%
\pgfpathrectangle{\pgfqpoint{5.095420in}{3.066506in}}{\pgfqpoint{0.824468in}{0.243158in}}%
\pgfusepath{clip}%
\pgfsetbuttcap%
\pgfsetmiterjoin%
\definecolor{currentfill}{rgb}{0.121569,0.466667,0.705882}%
\pgfsetfillcolor{currentfill}%
\pgfsetfillopacity{0.500000}%
\pgfsetlinewidth{1.003750pt}%
\definecolor{currentstroke}{rgb}{0.000000,0.000000,0.000000}%
\pgfsetstrokecolor{currentstroke}%
\pgfsetdash{}{0pt}%
\pgfpathmoveto{\pgfqpoint{5.582605in}{3.066506in}}%
\pgfpathlineto{\pgfqpoint{5.732509in}{3.066506in}}%
\pgfpathlineto{\pgfqpoint{5.732509in}{3.066871in}}%
\pgfpathlineto{\pgfqpoint{5.582605in}{3.066871in}}%
\pgfpathlineto{\pgfqpoint{5.582605in}{3.066506in}}%
\pgfpathclose%
\pgfusepath{stroke,fill}%
\end{pgfscope}%
\begin{pgfscope}%
\pgfpathrectangle{\pgfqpoint{5.095420in}{3.066506in}}{\pgfqpoint{0.824468in}{0.243158in}}%
\pgfusepath{clip}%
\pgfsetbuttcap%
\pgfsetmiterjoin%
\definecolor{currentfill}{rgb}{0.121569,0.466667,0.705882}%
\pgfsetfillcolor{currentfill}%
\pgfsetfillopacity{0.500000}%
\pgfsetlinewidth{1.003750pt}%
\definecolor{currentstroke}{rgb}{0.000000,0.000000,0.000000}%
\pgfsetstrokecolor{currentstroke}%
\pgfsetdash{}{0pt}%
\pgfpathmoveto{\pgfqpoint{5.732509in}{3.066506in}}%
\pgfpathlineto{\pgfqpoint{5.882412in}{3.066506in}}%
\pgfpathlineto{\pgfqpoint{5.882412in}{3.066749in}}%
\pgfpathlineto{\pgfqpoint{5.732509in}{3.066749in}}%
\pgfpathlineto{\pgfqpoint{5.732509in}{3.066506in}}%
\pgfpathclose%
\pgfusepath{stroke,fill}%
\end{pgfscope}%
\begin{pgfscope}%
\pgfsetrectcap%
\pgfsetmiterjoin%
\pgfsetlinewidth{0.803000pt}%
\definecolor{currentstroke}{rgb}{0.000000,0.000000,0.000000}%
\pgfsetstrokecolor{currentstroke}%
\pgfsetdash{}{0pt}%
\pgfpathmoveto{\pgfqpoint{5.095420in}{3.066506in}}%
\pgfpathlineto{\pgfqpoint{5.095420in}{3.309664in}}%
\pgfusepath{stroke}%
\end{pgfscope}%
\begin{pgfscope}%
\pgfsetrectcap%
\pgfsetmiterjoin%
\pgfsetlinewidth{0.803000pt}%
\definecolor{currentstroke}{rgb}{0.000000,0.000000,0.000000}%
\pgfsetstrokecolor{currentstroke}%
\pgfsetdash{}{0pt}%
\pgfpathmoveto{\pgfqpoint{5.919888in}{3.066506in}}%
\pgfpathlineto{\pgfqpoint{5.919888in}{3.309664in}}%
\pgfusepath{stroke}%
\end{pgfscope}%
\begin{pgfscope}%
\pgfsetrectcap%
\pgfsetmiterjoin%
\pgfsetlinewidth{0.803000pt}%
\definecolor{currentstroke}{rgb}{0.000000,0.000000,0.000000}%
\pgfsetstrokecolor{currentstroke}%
\pgfsetdash{}{0pt}%
\pgfpathmoveto{\pgfqpoint{5.095420in}{3.066506in}}%
\pgfpathlineto{\pgfqpoint{5.919888in}{3.066506in}}%
\pgfusepath{stroke}%
\end{pgfscope}%
\begin{pgfscope}%
\pgfsetrectcap%
\pgfsetmiterjoin%
\pgfsetlinewidth{0.803000pt}%
\definecolor{currentstroke}{rgb}{0.000000,0.000000,0.000000}%
\pgfsetstrokecolor{currentstroke}%
\pgfsetdash{}{0pt}%
\pgfpathmoveto{\pgfqpoint{5.095420in}{3.309664in}}%
\pgfpathlineto{\pgfqpoint{5.919888in}{3.309664in}}%
\pgfusepath{stroke}%
\end{pgfscope}%
\begin{pgfscope}%
\definecolor{textcolor}{rgb}{0.000000,0.000000,0.000000}%
\pgfsetstrokecolor{textcolor}%
\pgfsetfillcolor{textcolor}%
\pgftext[x=5.507654in,y=3.392997in,,base]{\color{textcolor}\rmfamily\fontsize{11.000000}{13.200000}\selectfont Sogecap}%
\end{pgfscope}%
\begin{pgfscope}%
\pgfsetbuttcap%
\pgfsetmiterjoin%
\definecolor{currentfill}{rgb}{1.000000,1.000000,1.000000}%
\pgfsetfillcolor{currentfill}%
\pgfsetlinewidth{0.000000pt}%
\definecolor{currentstroke}{rgb}{0.000000,0.000000,0.000000}%
\pgfsetstrokecolor{currentstroke}%
\pgfsetstrokeopacity{0.000000}%
\pgfsetdash{}{0pt}%
\pgfpathmoveto{\pgfqpoint{6.084781in}{3.066506in}}%
\pgfpathlineto{\pgfqpoint{6.909249in}{3.066506in}}%
\pgfpathlineto{\pgfqpoint{6.909249in}{3.309664in}}%
\pgfpathlineto{\pgfqpoint{6.084781in}{3.309664in}}%
\pgfpathlineto{\pgfqpoint{6.084781in}{3.066506in}}%
\pgfpathclose%
\pgfusepath{fill}%
\end{pgfscope}%
\begin{pgfscope}%
\pgfpathrectangle{\pgfqpoint{6.084781in}{3.066506in}}{\pgfqpoint{0.824468in}{0.243158in}}%
\pgfusepath{clip}%
\pgfsetbuttcap%
\pgfsetmiterjoin%
\definecolor{currentfill}{rgb}{0.121569,0.466667,0.705882}%
\pgfsetfillcolor{currentfill}%
\pgfsetfillopacity{0.500000}%
\pgfsetlinewidth{1.003750pt}%
\definecolor{currentstroke}{rgb}{0.000000,0.000000,0.000000}%
\pgfsetstrokecolor{currentstroke}%
\pgfsetdash{}{0pt}%
\pgfpathmoveto{\pgfqpoint{6.122257in}{3.066506in}}%
\pgfpathlineto{\pgfqpoint{6.272160in}{3.066506in}}%
\pgfpathlineto{\pgfqpoint{6.272160in}{3.094834in}}%
\pgfpathlineto{\pgfqpoint{6.122257in}{3.094834in}}%
\pgfpathlineto{\pgfqpoint{6.122257in}{3.066506in}}%
\pgfpathclose%
\pgfusepath{stroke,fill}%
\end{pgfscope}%
\begin{pgfscope}%
\pgfpathrectangle{\pgfqpoint{6.084781in}{3.066506in}}{\pgfqpoint{0.824468in}{0.243158in}}%
\pgfusepath{clip}%
\pgfsetbuttcap%
\pgfsetmiterjoin%
\definecolor{currentfill}{rgb}{0.121569,0.466667,0.705882}%
\pgfsetfillcolor{currentfill}%
\pgfsetfillopacity{0.500000}%
\pgfsetlinewidth{1.003750pt}%
\definecolor{currentstroke}{rgb}{0.000000,0.000000,0.000000}%
\pgfsetstrokecolor{currentstroke}%
\pgfsetdash{}{0pt}%
\pgfpathmoveto{\pgfqpoint{6.272160in}{3.066506in}}%
\pgfpathlineto{\pgfqpoint{6.422064in}{3.066506in}}%
\pgfpathlineto{\pgfqpoint{6.422064in}{3.072463in}}%
\pgfpathlineto{\pgfqpoint{6.272160in}{3.072463in}}%
\pgfpathlineto{\pgfqpoint{6.272160in}{3.066506in}}%
\pgfpathclose%
\pgfusepath{stroke,fill}%
\end{pgfscope}%
\begin{pgfscope}%
\pgfpathrectangle{\pgfqpoint{6.084781in}{3.066506in}}{\pgfqpoint{0.824468in}{0.243158in}}%
\pgfusepath{clip}%
\pgfsetbuttcap%
\pgfsetmiterjoin%
\definecolor{currentfill}{rgb}{0.121569,0.466667,0.705882}%
\pgfsetfillcolor{currentfill}%
\pgfsetfillopacity{0.500000}%
\pgfsetlinewidth{1.003750pt}%
\definecolor{currentstroke}{rgb}{0.000000,0.000000,0.000000}%
\pgfsetstrokecolor{currentstroke}%
\pgfsetdash{}{0pt}%
\pgfpathmoveto{\pgfqpoint{6.422064in}{3.066506in}}%
\pgfpathlineto{\pgfqpoint{6.571967in}{3.066506in}}%
\pgfpathlineto{\pgfqpoint{6.571967in}{3.069181in}}%
\pgfpathlineto{\pgfqpoint{6.422064in}{3.069181in}}%
\pgfpathlineto{\pgfqpoint{6.422064in}{3.066506in}}%
\pgfpathclose%
\pgfusepath{stroke,fill}%
\end{pgfscope}%
\begin{pgfscope}%
\pgfpathrectangle{\pgfqpoint{6.084781in}{3.066506in}}{\pgfqpoint{0.824468in}{0.243158in}}%
\pgfusepath{clip}%
\pgfsetbuttcap%
\pgfsetmiterjoin%
\definecolor{currentfill}{rgb}{0.121569,0.466667,0.705882}%
\pgfsetfillcolor{currentfill}%
\pgfsetfillopacity{0.500000}%
\pgfsetlinewidth{1.003750pt}%
\definecolor{currentstroke}{rgb}{0.000000,0.000000,0.000000}%
\pgfsetstrokecolor{currentstroke}%
\pgfsetdash{}{0pt}%
\pgfpathmoveto{\pgfqpoint{6.571967in}{3.066506in}}%
\pgfpathlineto{\pgfqpoint{6.721870in}{3.066506in}}%
\pgfpathlineto{\pgfqpoint{6.721870in}{3.066992in}}%
\pgfpathlineto{\pgfqpoint{6.571967in}{3.066992in}}%
\pgfpathlineto{\pgfqpoint{6.571967in}{3.066506in}}%
\pgfpathclose%
\pgfusepath{stroke,fill}%
\end{pgfscope}%
\begin{pgfscope}%
\pgfpathrectangle{\pgfqpoint{6.084781in}{3.066506in}}{\pgfqpoint{0.824468in}{0.243158in}}%
\pgfusepath{clip}%
\pgfsetbuttcap%
\pgfsetmiterjoin%
\definecolor{currentfill}{rgb}{0.121569,0.466667,0.705882}%
\pgfsetfillcolor{currentfill}%
\pgfsetfillopacity{0.500000}%
\pgfsetlinewidth{1.003750pt}%
\definecolor{currentstroke}{rgb}{0.000000,0.000000,0.000000}%
\pgfsetstrokecolor{currentstroke}%
\pgfsetdash{}{0pt}%
\pgfpathmoveto{\pgfqpoint{6.721870in}{3.066506in}}%
\pgfpathlineto{\pgfqpoint{6.871774in}{3.066506in}}%
\pgfpathlineto{\pgfqpoint{6.871774in}{3.066992in}}%
\pgfpathlineto{\pgfqpoint{6.721870in}{3.066992in}}%
\pgfpathlineto{\pgfqpoint{6.721870in}{3.066506in}}%
\pgfpathclose%
\pgfusepath{stroke,fill}%
\end{pgfscope}%
\begin{pgfscope}%
\pgfsetrectcap%
\pgfsetmiterjoin%
\pgfsetlinewidth{0.803000pt}%
\definecolor{currentstroke}{rgb}{0.000000,0.000000,0.000000}%
\pgfsetstrokecolor{currentstroke}%
\pgfsetdash{}{0pt}%
\pgfpathmoveto{\pgfqpoint{6.084781in}{3.066506in}}%
\pgfpathlineto{\pgfqpoint{6.084781in}{3.309664in}}%
\pgfusepath{stroke}%
\end{pgfscope}%
\begin{pgfscope}%
\pgfsetrectcap%
\pgfsetmiterjoin%
\pgfsetlinewidth{0.803000pt}%
\definecolor{currentstroke}{rgb}{0.000000,0.000000,0.000000}%
\pgfsetstrokecolor{currentstroke}%
\pgfsetdash{}{0pt}%
\pgfpathmoveto{\pgfqpoint{6.909249in}{3.066506in}}%
\pgfpathlineto{\pgfqpoint{6.909249in}{3.309664in}}%
\pgfusepath{stroke}%
\end{pgfscope}%
\begin{pgfscope}%
\pgfsetrectcap%
\pgfsetmiterjoin%
\pgfsetlinewidth{0.803000pt}%
\definecolor{currentstroke}{rgb}{0.000000,0.000000,0.000000}%
\pgfsetstrokecolor{currentstroke}%
\pgfsetdash{}{0pt}%
\pgfpathmoveto{\pgfqpoint{6.084781in}{3.066506in}}%
\pgfpathlineto{\pgfqpoint{6.909249in}{3.066506in}}%
\pgfusepath{stroke}%
\end{pgfscope}%
\begin{pgfscope}%
\pgfsetrectcap%
\pgfsetmiterjoin%
\pgfsetlinewidth{0.803000pt}%
\definecolor{currentstroke}{rgb}{0.000000,0.000000,0.000000}%
\pgfsetstrokecolor{currentstroke}%
\pgfsetdash{}{0pt}%
\pgfpathmoveto{\pgfqpoint{6.084781in}{3.309664in}}%
\pgfpathlineto{\pgfqpoint{6.909249in}{3.309664in}}%
\pgfusepath{stroke}%
\end{pgfscope}%
\begin{pgfscope}%
\definecolor{textcolor}{rgb}{0.000000,0.000000,0.000000}%
\pgfsetstrokecolor{textcolor}%
\pgfsetfillcolor{textcolor}%
\pgftext[x=6.497015in,y=3.392997in,,base]{\color{textcolor}\rmfamily\fontsize{11.000000}{13.200000}\selectfont Harmon...}%
\end{pgfscope}%
\begin{pgfscope}%
\pgfsetbuttcap%
\pgfsetmiterjoin%
\definecolor{currentfill}{rgb}{1.000000,1.000000,1.000000}%
\pgfsetfillcolor{currentfill}%
\pgfsetlinewidth{0.000000pt}%
\definecolor{currentstroke}{rgb}{0.000000,0.000000,0.000000}%
\pgfsetstrokecolor{currentstroke}%
\pgfsetstrokeopacity{0.000000}%
\pgfsetdash{}{0pt}%
\pgfpathmoveto{\pgfqpoint{7.074143in}{3.066506in}}%
\pgfpathlineto{\pgfqpoint{7.898611in}{3.066506in}}%
\pgfpathlineto{\pgfqpoint{7.898611in}{3.309664in}}%
\pgfpathlineto{\pgfqpoint{7.074143in}{3.309664in}}%
\pgfpathlineto{\pgfqpoint{7.074143in}{3.066506in}}%
\pgfpathclose%
\pgfusepath{fill}%
\end{pgfscope}%
\begin{pgfscope}%
\pgfpathrectangle{\pgfqpoint{7.074143in}{3.066506in}}{\pgfqpoint{0.824468in}{0.243158in}}%
\pgfusepath{clip}%
\pgfsetbuttcap%
\pgfsetmiterjoin%
\definecolor{currentfill}{rgb}{0.121569,0.466667,0.705882}%
\pgfsetfillcolor{currentfill}%
\pgfsetfillopacity{0.500000}%
\pgfsetlinewidth{1.003750pt}%
\definecolor{currentstroke}{rgb}{0.000000,0.000000,0.000000}%
\pgfsetstrokecolor{currentstroke}%
\pgfsetdash{}{0pt}%
\pgfpathmoveto{\pgfqpoint{7.111619in}{3.066506in}}%
\pgfpathlineto{\pgfqpoint{7.261522in}{3.066506in}}%
\pgfpathlineto{\pgfqpoint{7.261522in}{3.075989in}}%
\pgfpathlineto{\pgfqpoint{7.111619in}{3.075989in}}%
\pgfpathlineto{\pgfqpoint{7.111619in}{3.066506in}}%
\pgfpathclose%
\pgfusepath{stroke,fill}%
\end{pgfscope}%
\begin{pgfscope}%
\pgfpathrectangle{\pgfqpoint{7.074143in}{3.066506in}}{\pgfqpoint{0.824468in}{0.243158in}}%
\pgfusepath{clip}%
\pgfsetbuttcap%
\pgfsetmiterjoin%
\definecolor{currentfill}{rgb}{0.121569,0.466667,0.705882}%
\pgfsetfillcolor{currentfill}%
\pgfsetfillopacity{0.500000}%
\pgfsetlinewidth{1.003750pt}%
\definecolor{currentstroke}{rgb}{0.000000,0.000000,0.000000}%
\pgfsetstrokecolor{currentstroke}%
\pgfsetdash{}{0pt}%
\pgfpathmoveto{\pgfqpoint{7.261522in}{3.066506in}}%
\pgfpathlineto{\pgfqpoint{7.411425in}{3.066506in}}%
\pgfpathlineto{\pgfqpoint{7.411425in}{3.069181in}}%
\pgfpathlineto{\pgfqpoint{7.261522in}{3.069181in}}%
\pgfpathlineto{\pgfqpoint{7.261522in}{3.066506in}}%
\pgfpathclose%
\pgfusepath{stroke,fill}%
\end{pgfscope}%
\begin{pgfscope}%
\pgfpathrectangle{\pgfqpoint{7.074143in}{3.066506in}}{\pgfqpoint{0.824468in}{0.243158in}}%
\pgfusepath{clip}%
\pgfsetbuttcap%
\pgfsetmiterjoin%
\definecolor{currentfill}{rgb}{0.121569,0.466667,0.705882}%
\pgfsetfillcolor{currentfill}%
\pgfsetfillopacity{0.500000}%
\pgfsetlinewidth{1.003750pt}%
\definecolor{currentstroke}{rgb}{0.000000,0.000000,0.000000}%
\pgfsetstrokecolor{currentstroke}%
\pgfsetdash{}{0pt}%
\pgfpathmoveto{\pgfqpoint{7.411425in}{3.066506in}}%
\pgfpathlineto{\pgfqpoint{7.561329in}{3.066506in}}%
\pgfpathlineto{\pgfqpoint{7.561329in}{3.067722in}}%
\pgfpathlineto{\pgfqpoint{7.411425in}{3.067722in}}%
\pgfpathlineto{\pgfqpoint{7.411425in}{3.066506in}}%
\pgfpathclose%
\pgfusepath{stroke,fill}%
\end{pgfscope}%
\begin{pgfscope}%
\pgfpathrectangle{\pgfqpoint{7.074143in}{3.066506in}}{\pgfqpoint{0.824468in}{0.243158in}}%
\pgfusepath{clip}%
\pgfsetbuttcap%
\pgfsetmiterjoin%
\definecolor{currentfill}{rgb}{0.121569,0.466667,0.705882}%
\pgfsetfillcolor{currentfill}%
\pgfsetfillopacity{0.500000}%
\pgfsetlinewidth{1.003750pt}%
\definecolor{currentstroke}{rgb}{0.000000,0.000000,0.000000}%
\pgfsetstrokecolor{currentstroke}%
\pgfsetdash{}{0pt}%
\pgfpathmoveto{\pgfqpoint{7.561329in}{3.066506in}}%
\pgfpathlineto{\pgfqpoint{7.711232in}{3.066506in}}%
\pgfpathlineto{\pgfqpoint{7.711232in}{3.067843in}}%
\pgfpathlineto{\pgfqpoint{7.561329in}{3.067843in}}%
\pgfpathlineto{\pgfqpoint{7.561329in}{3.066506in}}%
\pgfpathclose%
\pgfusepath{stroke,fill}%
\end{pgfscope}%
\begin{pgfscope}%
\pgfpathrectangle{\pgfqpoint{7.074143in}{3.066506in}}{\pgfqpoint{0.824468in}{0.243158in}}%
\pgfusepath{clip}%
\pgfsetbuttcap%
\pgfsetmiterjoin%
\definecolor{currentfill}{rgb}{0.121569,0.466667,0.705882}%
\pgfsetfillcolor{currentfill}%
\pgfsetfillopacity{0.500000}%
\pgfsetlinewidth{1.003750pt}%
\definecolor{currentstroke}{rgb}{0.000000,0.000000,0.000000}%
\pgfsetstrokecolor{currentstroke}%
\pgfsetdash{}{0pt}%
\pgfpathmoveto{\pgfqpoint{7.711232in}{3.066506in}}%
\pgfpathlineto{\pgfqpoint{7.861135in}{3.066506in}}%
\pgfpathlineto{\pgfqpoint{7.861135in}{3.067357in}}%
\pgfpathlineto{\pgfqpoint{7.711232in}{3.067357in}}%
\pgfpathlineto{\pgfqpoint{7.711232in}{3.066506in}}%
\pgfpathclose%
\pgfusepath{stroke,fill}%
\end{pgfscope}%
\begin{pgfscope}%
\pgfsetrectcap%
\pgfsetmiterjoin%
\pgfsetlinewidth{0.803000pt}%
\definecolor{currentstroke}{rgb}{0.000000,0.000000,0.000000}%
\pgfsetstrokecolor{currentstroke}%
\pgfsetdash{}{0pt}%
\pgfpathmoveto{\pgfqpoint{7.074143in}{3.066506in}}%
\pgfpathlineto{\pgfqpoint{7.074143in}{3.309664in}}%
\pgfusepath{stroke}%
\end{pgfscope}%
\begin{pgfscope}%
\pgfsetrectcap%
\pgfsetmiterjoin%
\pgfsetlinewidth{0.803000pt}%
\definecolor{currentstroke}{rgb}{0.000000,0.000000,0.000000}%
\pgfsetstrokecolor{currentstroke}%
\pgfsetdash{}{0pt}%
\pgfpathmoveto{\pgfqpoint{7.898611in}{3.066506in}}%
\pgfpathlineto{\pgfqpoint{7.898611in}{3.309664in}}%
\pgfusepath{stroke}%
\end{pgfscope}%
\begin{pgfscope}%
\pgfsetrectcap%
\pgfsetmiterjoin%
\pgfsetlinewidth{0.803000pt}%
\definecolor{currentstroke}{rgb}{0.000000,0.000000,0.000000}%
\pgfsetstrokecolor{currentstroke}%
\pgfsetdash{}{0pt}%
\pgfpathmoveto{\pgfqpoint{7.074143in}{3.066506in}}%
\pgfpathlineto{\pgfqpoint{7.898611in}{3.066506in}}%
\pgfusepath{stroke}%
\end{pgfscope}%
\begin{pgfscope}%
\pgfsetrectcap%
\pgfsetmiterjoin%
\pgfsetlinewidth{0.803000pt}%
\definecolor{currentstroke}{rgb}{0.000000,0.000000,0.000000}%
\pgfsetstrokecolor{currentstroke}%
\pgfsetdash{}{0pt}%
\pgfpathmoveto{\pgfqpoint{7.074143in}{3.309664in}}%
\pgfpathlineto{\pgfqpoint{7.898611in}{3.309664in}}%
\pgfusepath{stroke}%
\end{pgfscope}%
\begin{pgfscope}%
\definecolor{textcolor}{rgb}{0.000000,0.000000,0.000000}%
\pgfsetstrokecolor{textcolor}%
\pgfsetfillcolor{textcolor}%
\pgftext[x=7.486377in,y=3.392997in,,base]{\color{textcolor}\rmfamily\fontsize{11.000000}{13.200000}\selectfont Mutuel...}%
\end{pgfscope}%
\begin{pgfscope}%
\pgfsetbuttcap%
\pgfsetmiterjoin%
\definecolor{currentfill}{rgb}{1.000000,1.000000,1.000000}%
\pgfsetfillcolor{currentfill}%
\pgfsetlinewidth{0.000000pt}%
\definecolor{currentstroke}{rgb}{0.000000,0.000000,0.000000}%
\pgfsetstrokecolor{currentstroke}%
\pgfsetstrokeopacity{0.000000}%
\pgfsetdash{}{0pt}%
\pgfpathmoveto{\pgfqpoint{0.148611in}{2.337032in}}%
\pgfpathlineto{\pgfqpoint{0.973079in}{2.337032in}}%
\pgfpathlineto{\pgfqpoint{0.973079in}{2.580190in}}%
\pgfpathlineto{\pgfqpoint{0.148611in}{2.580190in}}%
\pgfpathlineto{\pgfqpoint{0.148611in}{2.337032in}}%
\pgfpathclose%
\pgfusepath{fill}%
\end{pgfscope}%
\begin{pgfscope}%
\pgfpathrectangle{\pgfqpoint{0.148611in}{2.337032in}}{\pgfqpoint{0.824468in}{0.243158in}}%
\pgfusepath{clip}%
\pgfsetbuttcap%
\pgfsetmiterjoin%
\definecolor{currentfill}{rgb}{0.121569,0.466667,0.705882}%
\pgfsetfillcolor{currentfill}%
\pgfsetfillopacity{0.500000}%
\pgfsetlinewidth{1.003750pt}%
\definecolor{currentstroke}{rgb}{0.000000,0.000000,0.000000}%
\pgfsetstrokecolor{currentstroke}%
\pgfsetdash{}{0pt}%
\pgfpathmoveto{\pgfqpoint{0.186087in}{2.337032in}}%
\pgfpathlineto{\pgfqpoint{0.335990in}{2.337032in}}%
\pgfpathlineto{\pgfqpoint{0.335990in}{2.388460in}}%
\pgfpathlineto{\pgfqpoint{0.186087in}{2.388460in}}%
\pgfpathlineto{\pgfqpoint{0.186087in}{2.337032in}}%
\pgfpathclose%
\pgfusepath{stroke,fill}%
\end{pgfscope}%
\begin{pgfscope}%
\pgfpathrectangle{\pgfqpoint{0.148611in}{2.337032in}}{\pgfqpoint{0.824468in}{0.243158in}}%
\pgfusepath{clip}%
\pgfsetbuttcap%
\pgfsetmiterjoin%
\definecolor{currentfill}{rgb}{0.121569,0.466667,0.705882}%
\pgfsetfillcolor{currentfill}%
\pgfsetfillopacity{0.500000}%
\pgfsetlinewidth{1.003750pt}%
\definecolor{currentstroke}{rgb}{0.000000,0.000000,0.000000}%
\pgfsetstrokecolor{currentstroke}%
\pgfsetdash{}{0pt}%
\pgfpathmoveto{\pgfqpoint{0.335990in}{2.337032in}}%
\pgfpathlineto{\pgfqpoint{0.485894in}{2.337032in}}%
\pgfpathlineto{\pgfqpoint{0.485894in}{2.367427in}}%
\pgfpathlineto{\pgfqpoint{0.335990in}{2.367427in}}%
\pgfpathlineto{\pgfqpoint{0.335990in}{2.337032in}}%
\pgfpathclose%
\pgfusepath{stroke,fill}%
\end{pgfscope}%
\begin{pgfscope}%
\pgfpathrectangle{\pgfqpoint{0.148611in}{2.337032in}}{\pgfqpoint{0.824468in}{0.243158in}}%
\pgfusepath{clip}%
\pgfsetbuttcap%
\pgfsetmiterjoin%
\definecolor{currentfill}{rgb}{0.121569,0.466667,0.705882}%
\pgfsetfillcolor{currentfill}%
\pgfsetfillopacity{0.500000}%
\pgfsetlinewidth{1.003750pt}%
\definecolor{currentstroke}{rgb}{0.000000,0.000000,0.000000}%
\pgfsetstrokecolor{currentstroke}%
\pgfsetdash{}{0pt}%
\pgfpathmoveto{\pgfqpoint{0.485894in}{2.337032in}}%
\pgfpathlineto{\pgfqpoint{0.635797in}{2.337032in}}%
\pgfpathlineto{\pgfqpoint{0.635797in}{2.346880in}}%
\pgfpathlineto{\pgfqpoint{0.485894in}{2.346880in}}%
\pgfpathlineto{\pgfqpoint{0.485894in}{2.337032in}}%
\pgfpathclose%
\pgfusepath{stroke,fill}%
\end{pgfscope}%
\begin{pgfscope}%
\pgfpathrectangle{\pgfqpoint{0.148611in}{2.337032in}}{\pgfqpoint{0.824468in}{0.243158in}}%
\pgfusepath{clip}%
\pgfsetbuttcap%
\pgfsetmiterjoin%
\definecolor{currentfill}{rgb}{0.121569,0.466667,0.705882}%
\pgfsetfillcolor{currentfill}%
\pgfsetfillopacity{0.500000}%
\pgfsetlinewidth{1.003750pt}%
\definecolor{currentstroke}{rgb}{0.000000,0.000000,0.000000}%
\pgfsetstrokecolor{currentstroke}%
\pgfsetdash{}{0pt}%
\pgfpathmoveto{\pgfqpoint{0.635797in}{2.337032in}}%
\pgfpathlineto{\pgfqpoint{0.785700in}{2.337032in}}%
\pgfpathlineto{\pgfqpoint{0.785700in}{2.342260in}}%
\pgfpathlineto{\pgfqpoint{0.635797in}{2.342260in}}%
\pgfpathlineto{\pgfqpoint{0.635797in}{2.337032in}}%
\pgfpathclose%
\pgfusepath{stroke,fill}%
\end{pgfscope}%
\begin{pgfscope}%
\pgfpathrectangle{\pgfqpoint{0.148611in}{2.337032in}}{\pgfqpoint{0.824468in}{0.243158in}}%
\pgfusepath{clip}%
\pgfsetbuttcap%
\pgfsetmiterjoin%
\definecolor{currentfill}{rgb}{0.121569,0.466667,0.705882}%
\pgfsetfillcolor{currentfill}%
\pgfsetfillopacity{0.500000}%
\pgfsetlinewidth{1.003750pt}%
\definecolor{currentstroke}{rgb}{0.000000,0.000000,0.000000}%
\pgfsetstrokecolor{currentstroke}%
\pgfsetdash{}{0pt}%
\pgfpathmoveto{\pgfqpoint{0.785700in}{2.337032in}}%
\pgfpathlineto{\pgfqpoint{0.935603in}{2.337032in}}%
\pgfpathlineto{\pgfqpoint{0.935603in}{2.340923in}}%
\pgfpathlineto{\pgfqpoint{0.785700in}{2.340923in}}%
\pgfpathlineto{\pgfqpoint{0.785700in}{2.337032in}}%
\pgfpathclose%
\pgfusepath{stroke,fill}%
\end{pgfscope}%
\begin{pgfscope}%
\pgfsetrectcap%
\pgfsetmiterjoin%
\pgfsetlinewidth{0.803000pt}%
\definecolor{currentstroke}{rgb}{0.000000,0.000000,0.000000}%
\pgfsetstrokecolor{currentstroke}%
\pgfsetdash{}{0pt}%
\pgfpathmoveto{\pgfqpoint{0.148611in}{2.337032in}}%
\pgfpathlineto{\pgfqpoint{0.148611in}{2.580190in}}%
\pgfusepath{stroke}%
\end{pgfscope}%
\begin{pgfscope}%
\pgfsetrectcap%
\pgfsetmiterjoin%
\pgfsetlinewidth{0.803000pt}%
\definecolor{currentstroke}{rgb}{0.000000,0.000000,0.000000}%
\pgfsetstrokecolor{currentstroke}%
\pgfsetdash{}{0pt}%
\pgfpathmoveto{\pgfqpoint{0.973079in}{2.337032in}}%
\pgfpathlineto{\pgfqpoint{0.973079in}{2.580190in}}%
\pgfusepath{stroke}%
\end{pgfscope}%
\begin{pgfscope}%
\pgfsetrectcap%
\pgfsetmiterjoin%
\pgfsetlinewidth{0.803000pt}%
\definecolor{currentstroke}{rgb}{0.000000,0.000000,0.000000}%
\pgfsetstrokecolor{currentstroke}%
\pgfsetdash{}{0pt}%
\pgfpathmoveto{\pgfqpoint{0.148611in}{2.337032in}}%
\pgfpathlineto{\pgfqpoint{0.973079in}{2.337032in}}%
\pgfusepath{stroke}%
\end{pgfscope}%
\begin{pgfscope}%
\pgfsetrectcap%
\pgfsetmiterjoin%
\pgfsetlinewidth{0.803000pt}%
\definecolor{currentstroke}{rgb}{0.000000,0.000000,0.000000}%
\pgfsetstrokecolor{currentstroke}%
\pgfsetdash{}{0pt}%
\pgfpathmoveto{\pgfqpoint{0.148611in}{2.580190in}}%
\pgfpathlineto{\pgfqpoint{0.973079in}{2.580190in}}%
\pgfusepath{stroke}%
\end{pgfscope}%
\begin{pgfscope}%
\definecolor{textcolor}{rgb}{0.000000,0.000000,0.000000}%
\pgfsetstrokecolor{textcolor}%
\pgfsetfillcolor{textcolor}%
\pgftext[x=0.560845in,y=2.663523in,,base]{\color{textcolor}\rmfamily\fontsize{11.000000}{13.200000}\selectfont MACIF}%
\end{pgfscope}%
\begin{pgfscope}%
\pgfsetbuttcap%
\pgfsetmiterjoin%
\definecolor{currentfill}{rgb}{1.000000,1.000000,1.000000}%
\pgfsetfillcolor{currentfill}%
\pgfsetlinewidth{0.000000pt}%
\definecolor{currentstroke}{rgb}{0.000000,0.000000,0.000000}%
\pgfsetstrokecolor{currentstroke}%
\pgfsetstrokeopacity{0.000000}%
\pgfsetdash{}{0pt}%
\pgfpathmoveto{\pgfqpoint{1.137973in}{2.337032in}}%
\pgfpathlineto{\pgfqpoint{1.962441in}{2.337032in}}%
\pgfpathlineto{\pgfqpoint{1.962441in}{2.580190in}}%
\pgfpathlineto{\pgfqpoint{1.137973in}{2.580190in}}%
\pgfpathlineto{\pgfqpoint{1.137973in}{2.337032in}}%
\pgfpathclose%
\pgfusepath{fill}%
\end{pgfscope}%
\begin{pgfscope}%
\pgfpathrectangle{\pgfqpoint{1.137973in}{2.337032in}}{\pgfqpoint{0.824468in}{0.243158in}}%
\pgfusepath{clip}%
\pgfsetbuttcap%
\pgfsetmiterjoin%
\definecolor{currentfill}{rgb}{0.121569,0.466667,0.705882}%
\pgfsetfillcolor{currentfill}%
\pgfsetfillopacity{0.500000}%
\pgfsetlinewidth{1.003750pt}%
\definecolor{currentstroke}{rgb}{0.000000,0.000000,0.000000}%
\pgfsetstrokecolor{currentstroke}%
\pgfsetdash{}{0pt}%
\pgfpathmoveto{\pgfqpoint{1.175449in}{2.337032in}}%
\pgfpathlineto{\pgfqpoint{1.325352in}{2.337032in}}%
\pgfpathlineto{\pgfqpoint{1.325352in}{2.350406in}}%
\pgfpathlineto{\pgfqpoint{1.175449in}{2.350406in}}%
\pgfpathlineto{\pgfqpoint{1.175449in}{2.337032in}}%
\pgfpathclose%
\pgfusepath{stroke,fill}%
\end{pgfscope}%
\begin{pgfscope}%
\pgfpathrectangle{\pgfqpoint{1.137973in}{2.337032in}}{\pgfqpoint{0.824468in}{0.243158in}}%
\pgfusepath{clip}%
\pgfsetbuttcap%
\pgfsetmiterjoin%
\definecolor{currentfill}{rgb}{0.121569,0.466667,0.705882}%
\pgfsetfillcolor{currentfill}%
\pgfsetfillopacity{0.500000}%
\pgfsetlinewidth{1.003750pt}%
\definecolor{currentstroke}{rgb}{0.000000,0.000000,0.000000}%
\pgfsetstrokecolor{currentstroke}%
\pgfsetdash{}{0pt}%
\pgfpathmoveto{\pgfqpoint{1.325352in}{2.337032in}}%
\pgfpathlineto{\pgfqpoint{1.475255in}{2.337032in}}%
\pgfpathlineto{\pgfqpoint{1.475255in}{2.351257in}}%
\pgfpathlineto{\pgfqpoint{1.325352in}{2.351257in}}%
\pgfpathlineto{\pgfqpoint{1.325352in}{2.337032in}}%
\pgfpathclose%
\pgfusepath{stroke,fill}%
\end{pgfscope}%
\begin{pgfscope}%
\pgfpathrectangle{\pgfqpoint{1.137973in}{2.337032in}}{\pgfqpoint{0.824468in}{0.243158in}}%
\pgfusepath{clip}%
\pgfsetbuttcap%
\pgfsetmiterjoin%
\definecolor{currentfill}{rgb}{0.121569,0.466667,0.705882}%
\pgfsetfillcolor{currentfill}%
\pgfsetfillopacity{0.500000}%
\pgfsetlinewidth{1.003750pt}%
\definecolor{currentstroke}{rgb}{0.000000,0.000000,0.000000}%
\pgfsetstrokecolor{currentstroke}%
\pgfsetdash{}{0pt}%
\pgfpathmoveto{\pgfqpoint{1.475255in}{2.337032in}}%
\pgfpathlineto{\pgfqpoint{1.625158in}{2.337032in}}%
\pgfpathlineto{\pgfqpoint{1.625158in}{2.341044in}}%
\pgfpathlineto{\pgfqpoint{1.475255in}{2.341044in}}%
\pgfpathlineto{\pgfqpoint{1.475255in}{2.337032in}}%
\pgfpathclose%
\pgfusepath{stroke,fill}%
\end{pgfscope}%
\begin{pgfscope}%
\pgfpathrectangle{\pgfqpoint{1.137973in}{2.337032in}}{\pgfqpoint{0.824468in}{0.243158in}}%
\pgfusepath{clip}%
\pgfsetbuttcap%
\pgfsetmiterjoin%
\definecolor{currentfill}{rgb}{0.121569,0.466667,0.705882}%
\pgfsetfillcolor{currentfill}%
\pgfsetfillopacity{0.500000}%
\pgfsetlinewidth{1.003750pt}%
\definecolor{currentstroke}{rgb}{0.000000,0.000000,0.000000}%
\pgfsetstrokecolor{currentstroke}%
\pgfsetdash{}{0pt}%
\pgfpathmoveto{\pgfqpoint{1.625158in}{2.337032in}}%
\pgfpathlineto{\pgfqpoint{1.775062in}{2.337032in}}%
\pgfpathlineto{\pgfqpoint{1.775062in}{2.339342in}}%
\pgfpathlineto{\pgfqpoint{1.625158in}{2.339342in}}%
\pgfpathlineto{\pgfqpoint{1.625158in}{2.337032in}}%
\pgfpathclose%
\pgfusepath{stroke,fill}%
\end{pgfscope}%
\begin{pgfscope}%
\pgfpathrectangle{\pgfqpoint{1.137973in}{2.337032in}}{\pgfqpoint{0.824468in}{0.243158in}}%
\pgfusepath{clip}%
\pgfsetbuttcap%
\pgfsetmiterjoin%
\definecolor{currentfill}{rgb}{0.121569,0.466667,0.705882}%
\pgfsetfillcolor{currentfill}%
\pgfsetfillopacity{0.500000}%
\pgfsetlinewidth{1.003750pt}%
\definecolor{currentstroke}{rgb}{0.000000,0.000000,0.000000}%
\pgfsetstrokecolor{currentstroke}%
\pgfsetdash{}{0pt}%
\pgfpathmoveto{\pgfqpoint{1.775062in}{2.337032in}}%
\pgfpathlineto{\pgfqpoint{1.924965in}{2.337032in}}%
\pgfpathlineto{\pgfqpoint{1.924965in}{2.338248in}}%
\pgfpathlineto{\pgfqpoint{1.775062in}{2.338248in}}%
\pgfpathlineto{\pgfqpoint{1.775062in}{2.337032in}}%
\pgfpathclose%
\pgfusepath{stroke,fill}%
\end{pgfscope}%
\begin{pgfscope}%
\pgfsetrectcap%
\pgfsetmiterjoin%
\pgfsetlinewidth{0.803000pt}%
\definecolor{currentstroke}{rgb}{0.000000,0.000000,0.000000}%
\pgfsetstrokecolor{currentstroke}%
\pgfsetdash{}{0pt}%
\pgfpathmoveto{\pgfqpoint{1.137973in}{2.337032in}}%
\pgfpathlineto{\pgfqpoint{1.137973in}{2.580190in}}%
\pgfusepath{stroke}%
\end{pgfscope}%
\begin{pgfscope}%
\pgfsetrectcap%
\pgfsetmiterjoin%
\pgfsetlinewidth{0.803000pt}%
\definecolor{currentstroke}{rgb}{0.000000,0.000000,0.000000}%
\pgfsetstrokecolor{currentstroke}%
\pgfsetdash{}{0pt}%
\pgfpathmoveto{\pgfqpoint{1.962441in}{2.337032in}}%
\pgfpathlineto{\pgfqpoint{1.962441in}{2.580190in}}%
\pgfusepath{stroke}%
\end{pgfscope}%
\begin{pgfscope}%
\pgfsetrectcap%
\pgfsetmiterjoin%
\pgfsetlinewidth{0.803000pt}%
\definecolor{currentstroke}{rgb}{0.000000,0.000000,0.000000}%
\pgfsetstrokecolor{currentstroke}%
\pgfsetdash{}{0pt}%
\pgfpathmoveto{\pgfqpoint{1.137973in}{2.337032in}}%
\pgfpathlineto{\pgfqpoint{1.962441in}{2.337032in}}%
\pgfusepath{stroke}%
\end{pgfscope}%
\begin{pgfscope}%
\pgfsetrectcap%
\pgfsetmiterjoin%
\pgfsetlinewidth{0.803000pt}%
\definecolor{currentstroke}{rgb}{0.000000,0.000000,0.000000}%
\pgfsetstrokecolor{currentstroke}%
\pgfsetdash{}{0pt}%
\pgfpathmoveto{\pgfqpoint{1.137973in}{2.580190in}}%
\pgfpathlineto{\pgfqpoint{1.962441in}{2.580190in}}%
\pgfusepath{stroke}%
\end{pgfscope}%
\begin{pgfscope}%
\definecolor{textcolor}{rgb}{0.000000,0.000000,0.000000}%
\pgfsetstrokecolor{textcolor}%
\pgfsetfillcolor{textcolor}%
\pgftext[x=1.550207in,y=2.663523in,,base]{\color{textcolor}\rmfamily\fontsize{11.000000}{13.200000}\selectfont Eurofil}%
\end{pgfscope}%
\begin{pgfscope}%
\pgfsetbuttcap%
\pgfsetmiterjoin%
\definecolor{currentfill}{rgb}{1.000000,1.000000,1.000000}%
\pgfsetfillcolor{currentfill}%
\pgfsetlinewidth{0.000000pt}%
\definecolor{currentstroke}{rgb}{0.000000,0.000000,0.000000}%
\pgfsetstrokecolor{currentstroke}%
\pgfsetstrokeopacity{0.000000}%
\pgfsetdash{}{0pt}%
\pgfpathmoveto{\pgfqpoint{2.127335in}{2.337032in}}%
\pgfpathlineto{\pgfqpoint{2.951803in}{2.337032in}}%
\pgfpathlineto{\pgfqpoint{2.951803in}{2.580190in}}%
\pgfpathlineto{\pgfqpoint{2.127335in}{2.580190in}}%
\pgfpathlineto{\pgfqpoint{2.127335in}{2.337032in}}%
\pgfpathclose%
\pgfusepath{fill}%
\end{pgfscope}%
\begin{pgfscope}%
\pgfpathrectangle{\pgfqpoint{2.127335in}{2.337032in}}{\pgfqpoint{0.824468in}{0.243158in}}%
\pgfusepath{clip}%
\pgfsetbuttcap%
\pgfsetmiterjoin%
\definecolor{currentfill}{rgb}{0.121569,0.466667,0.705882}%
\pgfsetfillcolor{currentfill}%
\pgfsetfillopacity{0.500000}%
\pgfsetlinewidth{1.003750pt}%
\definecolor{currentstroke}{rgb}{0.000000,0.000000,0.000000}%
\pgfsetstrokecolor{currentstroke}%
\pgfsetdash{}{0pt}%
\pgfpathmoveto{\pgfqpoint{2.164810in}{2.337032in}}%
\pgfpathlineto{\pgfqpoint{2.314714in}{2.337032in}}%
\pgfpathlineto{\pgfqpoint{2.314714in}{2.365238in}}%
\pgfpathlineto{\pgfqpoint{2.164810in}{2.365238in}}%
\pgfpathlineto{\pgfqpoint{2.164810in}{2.337032in}}%
\pgfpathclose%
\pgfusepath{stroke,fill}%
\end{pgfscope}%
\begin{pgfscope}%
\pgfpathrectangle{\pgfqpoint{2.127335in}{2.337032in}}{\pgfqpoint{0.824468in}{0.243158in}}%
\pgfusepath{clip}%
\pgfsetbuttcap%
\pgfsetmiterjoin%
\definecolor{currentfill}{rgb}{0.121569,0.466667,0.705882}%
\pgfsetfillcolor{currentfill}%
\pgfsetfillopacity{0.500000}%
\pgfsetlinewidth{1.003750pt}%
\definecolor{currentstroke}{rgb}{0.000000,0.000000,0.000000}%
\pgfsetstrokecolor{currentstroke}%
\pgfsetdash{}{0pt}%
\pgfpathmoveto{\pgfqpoint{2.314714in}{2.337032in}}%
\pgfpathlineto{\pgfqpoint{2.464617in}{2.337032in}}%
\pgfpathlineto{\pgfqpoint{2.464617in}{2.349312in}}%
\pgfpathlineto{\pgfqpoint{2.314714in}{2.349312in}}%
\pgfpathlineto{\pgfqpoint{2.314714in}{2.337032in}}%
\pgfpathclose%
\pgfusepath{stroke,fill}%
\end{pgfscope}%
\begin{pgfscope}%
\pgfpathrectangle{\pgfqpoint{2.127335in}{2.337032in}}{\pgfqpoint{0.824468in}{0.243158in}}%
\pgfusepath{clip}%
\pgfsetbuttcap%
\pgfsetmiterjoin%
\definecolor{currentfill}{rgb}{0.121569,0.466667,0.705882}%
\pgfsetfillcolor{currentfill}%
\pgfsetfillopacity{0.500000}%
\pgfsetlinewidth{1.003750pt}%
\definecolor{currentstroke}{rgb}{0.000000,0.000000,0.000000}%
\pgfsetstrokecolor{currentstroke}%
\pgfsetdash{}{0pt}%
\pgfpathmoveto{\pgfqpoint{2.464617in}{2.337032in}}%
\pgfpathlineto{\pgfqpoint{2.614520in}{2.337032in}}%
\pgfpathlineto{\pgfqpoint{2.614520in}{2.342138in}}%
\pgfpathlineto{\pgfqpoint{2.464617in}{2.342138in}}%
\pgfpathlineto{\pgfqpoint{2.464617in}{2.337032in}}%
\pgfpathclose%
\pgfusepath{stroke,fill}%
\end{pgfscope}%
\begin{pgfscope}%
\pgfpathrectangle{\pgfqpoint{2.127335in}{2.337032in}}{\pgfqpoint{0.824468in}{0.243158in}}%
\pgfusepath{clip}%
\pgfsetbuttcap%
\pgfsetmiterjoin%
\definecolor{currentfill}{rgb}{0.121569,0.466667,0.705882}%
\pgfsetfillcolor{currentfill}%
\pgfsetfillopacity{0.500000}%
\pgfsetlinewidth{1.003750pt}%
\definecolor{currentstroke}{rgb}{0.000000,0.000000,0.000000}%
\pgfsetstrokecolor{currentstroke}%
\pgfsetdash{}{0pt}%
\pgfpathmoveto{\pgfqpoint{2.614520in}{2.337032in}}%
\pgfpathlineto{\pgfqpoint{2.764423in}{2.337032in}}%
\pgfpathlineto{\pgfqpoint{2.764423in}{2.338248in}}%
\pgfpathlineto{\pgfqpoint{2.614520in}{2.338248in}}%
\pgfpathlineto{\pgfqpoint{2.614520in}{2.337032in}}%
\pgfpathclose%
\pgfusepath{stroke,fill}%
\end{pgfscope}%
\begin{pgfscope}%
\pgfpathrectangle{\pgfqpoint{2.127335in}{2.337032in}}{\pgfqpoint{0.824468in}{0.243158in}}%
\pgfusepath{clip}%
\pgfsetbuttcap%
\pgfsetmiterjoin%
\definecolor{currentfill}{rgb}{0.121569,0.466667,0.705882}%
\pgfsetfillcolor{currentfill}%
\pgfsetfillopacity{0.500000}%
\pgfsetlinewidth{1.003750pt}%
\definecolor{currentstroke}{rgb}{0.000000,0.000000,0.000000}%
\pgfsetstrokecolor{currentstroke}%
\pgfsetdash{}{0pt}%
\pgfpathmoveto{\pgfqpoint{2.764423in}{2.337032in}}%
\pgfpathlineto{\pgfqpoint{2.914327in}{2.337032in}}%
\pgfpathlineto{\pgfqpoint{2.914327in}{2.339221in}}%
\pgfpathlineto{\pgfqpoint{2.764423in}{2.339221in}}%
\pgfpathlineto{\pgfqpoint{2.764423in}{2.337032in}}%
\pgfpathclose%
\pgfusepath{stroke,fill}%
\end{pgfscope}%
\begin{pgfscope}%
\pgfsetrectcap%
\pgfsetmiterjoin%
\pgfsetlinewidth{0.803000pt}%
\definecolor{currentstroke}{rgb}{0.000000,0.000000,0.000000}%
\pgfsetstrokecolor{currentstroke}%
\pgfsetdash{}{0pt}%
\pgfpathmoveto{\pgfqpoint{2.127335in}{2.337032in}}%
\pgfpathlineto{\pgfqpoint{2.127335in}{2.580190in}}%
\pgfusepath{stroke}%
\end{pgfscope}%
\begin{pgfscope}%
\pgfsetrectcap%
\pgfsetmiterjoin%
\pgfsetlinewidth{0.803000pt}%
\definecolor{currentstroke}{rgb}{0.000000,0.000000,0.000000}%
\pgfsetstrokecolor{currentstroke}%
\pgfsetdash{}{0pt}%
\pgfpathmoveto{\pgfqpoint{2.951803in}{2.337032in}}%
\pgfpathlineto{\pgfqpoint{2.951803in}{2.580190in}}%
\pgfusepath{stroke}%
\end{pgfscope}%
\begin{pgfscope}%
\pgfsetrectcap%
\pgfsetmiterjoin%
\pgfsetlinewidth{0.803000pt}%
\definecolor{currentstroke}{rgb}{0.000000,0.000000,0.000000}%
\pgfsetstrokecolor{currentstroke}%
\pgfsetdash{}{0pt}%
\pgfpathmoveto{\pgfqpoint{2.127335in}{2.337032in}}%
\pgfpathlineto{\pgfqpoint{2.951803in}{2.337032in}}%
\pgfusepath{stroke}%
\end{pgfscope}%
\begin{pgfscope}%
\pgfsetrectcap%
\pgfsetmiterjoin%
\pgfsetlinewidth{0.803000pt}%
\definecolor{currentstroke}{rgb}{0.000000,0.000000,0.000000}%
\pgfsetstrokecolor{currentstroke}%
\pgfsetdash{}{0pt}%
\pgfpathmoveto{\pgfqpoint{2.127335in}{2.580190in}}%
\pgfpathlineto{\pgfqpoint{2.951803in}{2.580190in}}%
\pgfusepath{stroke}%
\end{pgfscope}%
\begin{pgfscope}%
\definecolor{textcolor}{rgb}{0.000000,0.000000,0.000000}%
\pgfsetstrokecolor{textcolor}%
\pgfsetfillcolor{textcolor}%
\pgftext[x=2.539569in,y=2.663523in,,base]{\color{textcolor}\rmfamily\fontsize{11.000000}{13.200000}\selectfont Active...}%
\end{pgfscope}%
\begin{pgfscope}%
\pgfsetbuttcap%
\pgfsetmiterjoin%
\definecolor{currentfill}{rgb}{1.000000,1.000000,1.000000}%
\pgfsetfillcolor{currentfill}%
\pgfsetlinewidth{0.000000pt}%
\definecolor{currentstroke}{rgb}{0.000000,0.000000,0.000000}%
\pgfsetstrokecolor{currentstroke}%
\pgfsetstrokeopacity{0.000000}%
\pgfsetdash{}{0pt}%
\pgfpathmoveto{\pgfqpoint{3.116696in}{2.337032in}}%
\pgfpathlineto{\pgfqpoint{3.941164in}{2.337032in}}%
\pgfpathlineto{\pgfqpoint{3.941164in}{2.580190in}}%
\pgfpathlineto{\pgfqpoint{3.116696in}{2.580190in}}%
\pgfpathlineto{\pgfqpoint{3.116696in}{2.337032in}}%
\pgfpathclose%
\pgfusepath{fill}%
\end{pgfscope}%
\begin{pgfscope}%
\pgfpathrectangle{\pgfqpoint{3.116696in}{2.337032in}}{\pgfqpoint{0.824468in}{0.243158in}}%
\pgfusepath{clip}%
\pgfsetbuttcap%
\pgfsetmiterjoin%
\definecolor{currentfill}{rgb}{0.121569,0.466667,0.705882}%
\pgfsetfillcolor{currentfill}%
\pgfsetfillopacity{0.500000}%
\pgfsetlinewidth{1.003750pt}%
\definecolor{currentstroke}{rgb}{0.000000,0.000000,0.000000}%
\pgfsetstrokecolor{currentstroke}%
\pgfsetdash{}{0pt}%
\pgfpathmoveto{\pgfqpoint{3.154172in}{2.337032in}}%
\pgfpathlineto{\pgfqpoint{3.304075in}{2.337032in}}%
\pgfpathlineto{\pgfqpoint{3.304075in}{2.381165in}}%
\pgfpathlineto{\pgfqpoint{3.154172in}{2.381165in}}%
\pgfpathlineto{\pgfqpoint{3.154172in}{2.337032in}}%
\pgfpathclose%
\pgfusepath{stroke,fill}%
\end{pgfscope}%
\begin{pgfscope}%
\pgfpathrectangle{\pgfqpoint{3.116696in}{2.337032in}}{\pgfqpoint{0.824468in}{0.243158in}}%
\pgfusepath{clip}%
\pgfsetbuttcap%
\pgfsetmiterjoin%
\definecolor{currentfill}{rgb}{0.121569,0.466667,0.705882}%
\pgfsetfillcolor{currentfill}%
\pgfsetfillopacity{0.500000}%
\pgfsetlinewidth{1.003750pt}%
\definecolor{currentstroke}{rgb}{0.000000,0.000000,0.000000}%
\pgfsetstrokecolor{currentstroke}%
\pgfsetdash{}{0pt}%
\pgfpathmoveto{\pgfqpoint{3.304075in}{2.337032in}}%
\pgfpathlineto{\pgfqpoint{3.453979in}{2.337032in}}%
\pgfpathlineto{\pgfqpoint{3.453979in}{2.354418in}}%
\pgfpathlineto{\pgfqpoint{3.304075in}{2.354418in}}%
\pgfpathlineto{\pgfqpoint{3.304075in}{2.337032in}}%
\pgfpathclose%
\pgfusepath{stroke,fill}%
\end{pgfscope}%
\begin{pgfscope}%
\pgfpathrectangle{\pgfqpoint{3.116696in}{2.337032in}}{\pgfqpoint{0.824468in}{0.243158in}}%
\pgfusepath{clip}%
\pgfsetbuttcap%
\pgfsetmiterjoin%
\definecolor{currentfill}{rgb}{0.121569,0.466667,0.705882}%
\pgfsetfillcolor{currentfill}%
\pgfsetfillopacity{0.500000}%
\pgfsetlinewidth{1.003750pt}%
\definecolor{currentstroke}{rgb}{0.000000,0.000000,0.000000}%
\pgfsetstrokecolor{currentstroke}%
\pgfsetdash{}{0pt}%
\pgfpathmoveto{\pgfqpoint{3.453979in}{2.337032in}}%
\pgfpathlineto{\pgfqpoint{3.603882in}{2.337032in}}%
\pgfpathlineto{\pgfqpoint{3.603882in}{2.347488in}}%
\pgfpathlineto{\pgfqpoint{3.453979in}{2.347488in}}%
\pgfpathlineto{\pgfqpoint{3.453979in}{2.337032in}}%
\pgfpathclose%
\pgfusepath{stroke,fill}%
\end{pgfscope}%
\begin{pgfscope}%
\pgfpathrectangle{\pgfqpoint{3.116696in}{2.337032in}}{\pgfqpoint{0.824468in}{0.243158in}}%
\pgfusepath{clip}%
\pgfsetbuttcap%
\pgfsetmiterjoin%
\definecolor{currentfill}{rgb}{0.121569,0.466667,0.705882}%
\pgfsetfillcolor{currentfill}%
\pgfsetfillopacity{0.500000}%
\pgfsetlinewidth{1.003750pt}%
\definecolor{currentstroke}{rgb}{0.000000,0.000000,0.000000}%
\pgfsetstrokecolor{currentstroke}%
\pgfsetdash{}{0pt}%
\pgfpathmoveto{\pgfqpoint{3.603882in}{2.337032in}}%
\pgfpathlineto{\pgfqpoint{3.753785in}{2.337032in}}%
\pgfpathlineto{\pgfqpoint{3.753785in}{2.339707in}}%
\pgfpathlineto{\pgfqpoint{3.603882in}{2.339707in}}%
\pgfpathlineto{\pgfqpoint{3.603882in}{2.337032in}}%
\pgfpathclose%
\pgfusepath{stroke,fill}%
\end{pgfscope}%
\begin{pgfscope}%
\pgfpathrectangle{\pgfqpoint{3.116696in}{2.337032in}}{\pgfqpoint{0.824468in}{0.243158in}}%
\pgfusepath{clip}%
\pgfsetbuttcap%
\pgfsetmiterjoin%
\definecolor{currentfill}{rgb}{0.121569,0.466667,0.705882}%
\pgfsetfillcolor{currentfill}%
\pgfsetfillopacity{0.500000}%
\pgfsetlinewidth{1.003750pt}%
\definecolor{currentstroke}{rgb}{0.000000,0.000000,0.000000}%
\pgfsetstrokecolor{currentstroke}%
\pgfsetdash{}{0pt}%
\pgfpathmoveto{\pgfqpoint{3.753785in}{2.337032in}}%
\pgfpathlineto{\pgfqpoint{3.903688in}{2.337032in}}%
\pgfpathlineto{\pgfqpoint{3.903688in}{2.338856in}}%
\pgfpathlineto{\pgfqpoint{3.753785in}{2.338856in}}%
\pgfpathlineto{\pgfqpoint{3.753785in}{2.337032in}}%
\pgfpathclose%
\pgfusepath{stroke,fill}%
\end{pgfscope}%
\begin{pgfscope}%
\pgfsetrectcap%
\pgfsetmiterjoin%
\pgfsetlinewidth{0.803000pt}%
\definecolor{currentstroke}{rgb}{0.000000,0.000000,0.000000}%
\pgfsetstrokecolor{currentstroke}%
\pgfsetdash{}{0pt}%
\pgfpathmoveto{\pgfqpoint{3.116696in}{2.337032in}}%
\pgfpathlineto{\pgfqpoint{3.116696in}{2.580190in}}%
\pgfusepath{stroke}%
\end{pgfscope}%
\begin{pgfscope}%
\pgfsetrectcap%
\pgfsetmiterjoin%
\pgfsetlinewidth{0.803000pt}%
\definecolor{currentstroke}{rgb}{0.000000,0.000000,0.000000}%
\pgfsetstrokecolor{currentstroke}%
\pgfsetdash{}{0pt}%
\pgfpathmoveto{\pgfqpoint{3.941164in}{2.337032in}}%
\pgfpathlineto{\pgfqpoint{3.941164in}{2.580190in}}%
\pgfusepath{stroke}%
\end{pgfscope}%
\begin{pgfscope}%
\pgfsetrectcap%
\pgfsetmiterjoin%
\pgfsetlinewidth{0.803000pt}%
\definecolor{currentstroke}{rgb}{0.000000,0.000000,0.000000}%
\pgfsetstrokecolor{currentstroke}%
\pgfsetdash{}{0pt}%
\pgfpathmoveto{\pgfqpoint{3.116696in}{2.337032in}}%
\pgfpathlineto{\pgfqpoint{3.941164in}{2.337032in}}%
\pgfusepath{stroke}%
\end{pgfscope}%
\begin{pgfscope}%
\pgfsetrectcap%
\pgfsetmiterjoin%
\pgfsetlinewidth{0.803000pt}%
\definecolor{currentstroke}{rgb}{0.000000,0.000000,0.000000}%
\pgfsetstrokecolor{currentstroke}%
\pgfsetdash{}{0pt}%
\pgfpathmoveto{\pgfqpoint{3.116696in}{2.580190in}}%
\pgfpathlineto{\pgfqpoint{3.941164in}{2.580190in}}%
\pgfusepath{stroke}%
\end{pgfscope}%
\begin{pgfscope}%
\definecolor{textcolor}{rgb}{0.000000,0.000000,0.000000}%
\pgfsetstrokecolor{textcolor}%
\pgfsetfillcolor{textcolor}%
\pgftext[x=3.528930in,y=2.663523in,,base]{\color{textcolor}\rmfamily\fontsize{11.000000}{13.200000}\selectfont AXA}%
\end{pgfscope}%
\begin{pgfscope}%
\pgfsetbuttcap%
\pgfsetmiterjoin%
\definecolor{currentfill}{rgb}{1.000000,1.000000,1.000000}%
\pgfsetfillcolor{currentfill}%
\pgfsetlinewidth{0.000000pt}%
\definecolor{currentstroke}{rgb}{0.000000,0.000000,0.000000}%
\pgfsetstrokecolor{currentstroke}%
\pgfsetstrokeopacity{0.000000}%
\pgfsetdash{}{0pt}%
\pgfpathmoveto{\pgfqpoint{4.106058in}{2.337032in}}%
\pgfpathlineto{\pgfqpoint{4.930526in}{2.337032in}}%
\pgfpathlineto{\pgfqpoint{4.930526in}{2.580190in}}%
\pgfpathlineto{\pgfqpoint{4.106058in}{2.580190in}}%
\pgfpathlineto{\pgfqpoint{4.106058in}{2.337032in}}%
\pgfpathclose%
\pgfusepath{fill}%
\end{pgfscope}%
\begin{pgfscope}%
\pgfpathrectangle{\pgfqpoint{4.106058in}{2.337032in}}{\pgfqpoint{0.824468in}{0.243158in}}%
\pgfusepath{clip}%
\pgfsetbuttcap%
\pgfsetmiterjoin%
\definecolor{currentfill}{rgb}{0.121569,0.466667,0.705882}%
\pgfsetfillcolor{currentfill}%
\pgfsetfillopacity{0.500000}%
\pgfsetlinewidth{1.003750pt}%
\definecolor{currentstroke}{rgb}{0.000000,0.000000,0.000000}%
\pgfsetstrokecolor{currentstroke}%
\pgfsetdash{}{0pt}%
\pgfpathmoveto{\pgfqpoint{4.143534in}{2.337032in}}%
\pgfpathlineto{\pgfqpoint{4.293437in}{2.337032in}}%
\pgfpathlineto{\pgfqpoint{4.293437in}{2.345056in}}%
\pgfpathlineto{\pgfqpoint{4.143534in}{2.345056in}}%
\pgfpathlineto{\pgfqpoint{4.143534in}{2.337032in}}%
\pgfpathclose%
\pgfusepath{stroke,fill}%
\end{pgfscope}%
\begin{pgfscope}%
\pgfpathrectangle{\pgfqpoint{4.106058in}{2.337032in}}{\pgfqpoint{0.824468in}{0.243158in}}%
\pgfusepath{clip}%
\pgfsetbuttcap%
\pgfsetmiterjoin%
\definecolor{currentfill}{rgb}{0.121569,0.466667,0.705882}%
\pgfsetfillcolor{currentfill}%
\pgfsetfillopacity{0.500000}%
\pgfsetlinewidth{1.003750pt}%
\definecolor{currentstroke}{rgb}{0.000000,0.000000,0.000000}%
\pgfsetstrokecolor{currentstroke}%
\pgfsetdash{}{0pt}%
\pgfpathmoveto{\pgfqpoint{4.293437in}{2.337032in}}%
\pgfpathlineto{\pgfqpoint{4.443340in}{2.337032in}}%
\pgfpathlineto{\pgfqpoint{4.443340in}{2.339828in}}%
\pgfpathlineto{\pgfqpoint{4.293437in}{2.339828in}}%
\pgfpathlineto{\pgfqpoint{4.293437in}{2.337032in}}%
\pgfpathclose%
\pgfusepath{stroke,fill}%
\end{pgfscope}%
\begin{pgfscope}%
\pgfpathrectangle{\pgfqpoint{4.106058in}{2.337032in}}{\pgfqpoint{0.824468in}{0.243158in}}%
\pgfusepath{clip}%
\pgfsetbuttcap%
\pgfsetmiterjoin%
\definecolor{currentfill}{rgb}{0.121569,0.466667,0.705882}%
\pgfsetfillcolor{currentfill}%
\pgfsetfillopacity{0.500000}%
\pgfsetlinewidth{1.003750pt}%
\definecolor{currentstroke}{rgb}{0.000000,0.000000,0.000000}%
\pgfsetstrokecolor{currentstroke}%
\pgfsetdash{}{0pt}%
\pgfpathmoveto{\pgfqpoint{4.443340in}{2.337032in}}%
\pgfpathlineto{\pgfqpoint{4.593244in}{2.337032in}}%
\pgfpathlineto{\pgfqpoint{4.593244in}{2.337640in}}%
\pgfpathlineto{\pgfqpoint{4.443340in}{2.337640in}}%
\pgfpathlineto{\pgfqpoint{4.443340in}{2.337032in}}%
\pgfpathclose%
\pgfusepath{stroke,fill}%
\end{pgfscope}%
\begin{pgfscope}%
\pgfpathrectangle{\pgfqpoint{4.106058in}{2.337032in}}{\pgfqpoint{0.824468in}{0.243158in}}%
\pgfusepath{clip}%
\pgfsetbuttcap%
\pgfsetmiterjoin%
\definecolor{currentfill}{rgb}{0.121569,0.466667,0.705882}%
\pgfsetfillcolor{currentfill}%
\pgfsetfillopacity{0.500000}%
\pgfsetlinewidth{1.003750pt}%
\definecolor{currentstroke}{rgb}{0.000000,0.000000,0.000000}%
\pgfsetstrokecolor{currentstroke}%
\pgfsetdash{}{0pt}%
\pgfpathmoveto{\pgfqpoint{4.593244in}{2.337032in}}%
\pgfpathlineto{\pgfqpoint{4.743147in}{2.337032in}}%
\pgfpathlineto{\pgfqpoint{4.743147in}{2.337032in}}%
\pgfpathlineto{\pgfqpoint{4.593244in}{2.337032in}}%
\pgfpathlineto{\pgfqpoint{4.593244in}{2.337032in}}%
\pgfpathclose%
\pgfusepath{stroke,fill}%
\end{pgfscope}%
\begin{pgfscope}%
\pgfpathrectangle{\pgfqpoint{4.106058in}{2.337032in}}{\pgfqpoint{0.824468in}{0.243158in}}%
\pgfusepath{clip}%
\pgfsetbuttcap%
\pgfsetmiterjoin%
\definecolor{currentfill}{rgb}{0.121569,0.466667,0.705882}%
\pgfsetfillcolor{currentfill}%
\pgfsetfillopacity{0.500000}%
\pgfsetlinewidth{1.003750pt}%
\definecolor{currentstroke}{rgb}{0.000000,0.000000,0.000000}%
\pgfsetstrokecolor{currentstroke}%
\pgfsetdash{}{0pt}%
\pgfpathmoveto{\pgfqpoint{4.743147in}{2.337032in}}%
\pgfpathlineto{\pgfqpoint{4.893050in}{2.337032in}}%
\pgfpathlineto{\pgfqpoint{4.893050in}{2.337275in}}%
\pgfpathlineto{\pgfqpoint{4.743147in}{2.337275in}}%
\pgfpathlineto{\pgfqpoint{4.743147in}{2.337032in}}%
\pgfpathclose%
\pgfusepath{stroke,fill}%
\end{pgfscope}%
\begin{pgfscope}%
\pgfsetrectcap%
\pgfsetmiterjoin%
\pgfsetlinewidth{0.803000pt}%
\definecolor{currentstroke}{rgb}{0.000000,0.000000,0.000000}%
\pgfsetstrokecolor{currentstroke}%
\pgfsetdash{}{0pt}%
\pgfpathmoveto{\pgfqpoint{4.106058in}{2.337032in}}%
\pgfpathlineto{\pgfqpoint{4.106058in}{2.580190in}}%
\pgfusepath{stroke}%
\end{pgfscope}%
\begin{pgfscope}%
\pgfsetrectcap%
\pgfsetmiterjoin%
\pgfsetlinewidth{0.803000pt}%
\definecolor{currentstroke}{rgb}{0.000000,0.000000,0.000000}%
\pgfsetstrokecolor{currentstroke}%
\pgfsetdash{}{0pt}%
\pgfpathmoveto{\pgfqpoint{4.930526in}{2.337032in}}%
\pgfpathlineto{\pgfqpoint{4.930526in}{2.580190in}}%
\pgfusepath{stroke}%
\end{pgfscope}%
\begin{pgfscope}%
\pgfsetrectcap%
\pgfsetmiterjoin%
\pgfsetlinewidth{0.803000pt}%
\definecolor{currentstroke}{rgb}{0.000000,0.000000,0.000000}%
\pgfsetstrokecolor{currentstroke}%
\pgfsetdash{}{0pt}%
\pgfpathmoveto{\pgfqpoint{4.106058in}{2.337032in}}%
\pgfpathlineto{\pgfqpoint{4.930526in}{2.337032in}}%
\pgfusepath{stroke}%
\end{pgfscope}%
\begin{pgfscope}%
\pgfsetrectcap%
\pgfsetmiterjoin%
\pgfsetlinewidth{0.803000pt}%
\definecolor{currentstroke}{rgb}{0.000000,0.000000,0.000000}%
\pgfsetstrokecolor{currentstroke}%
\pgfsetdash{}{0pt}%
\pgfpathmoveto{\pgfqpoint{4.106058in}{2.580190in}}%
\pgfpathlineto{\pgfqpoint{4.930526in}{2.580190in}}%
\pgfusepath{stroke}%
\end{pgfscope}%
\begin{pgfscope}%
\definecolor{textcolor}{rgb}{0.000000,0.000000,0.000000}%
\pgfsetstrokecolor{textcolor}%
\pgfsetfillcolor{textcolor}%
\pgftext[x=4.518292in,y=2.663523in,,base]{\color{textcolor}\rmfamily\fontsize{11.000000}{13.200000}\selectfont Sogessur}%
\end{pgfscope}%
\begin{pgfscope}%
\pgfsetbuttcap%
\pgfsetmiterjoin%
\definecolor{currentfill}{rgb}{1.000000,1.000000,1.000000}%
\pgfsetfillcolor{currentfill}%
\pgfsetlinewidth{0.000000pt}%
\definecolor{currentstroke}{rgb}{0.000000,0.000000,0.000000}%
\pgfsetstrokecolor{currentstroke}%
\pgfsetstrokeopacity{0.000000}%
\pgfsetdash{}{0pt}%
\pgfpathmoveto{\pgfqpoint{5.095420in}{2.337032in}}%
\pgfpathlineto{\pgfqpoint{5.919888in}{2.337032in}}%
\pgfpathlineto{\pgfqpoint{5.919888in}{2.580190in}}%
\pgfpathlineto{\pgfqpoint{5.095420in}{2.580190in}}%
\pgfpathlineto{\pgfqpoint{5.095420in}{2.337032in}}%
\pgfpathclose%
\pgfusepath{fill}%
\end{pgfscope}%
\begin{pgfscope}%
\pgfpathrectangle{\pgfqpoint{5.095420in}{2.337032in}}{\pgfqpoint{0.824468in}{0.243158in}}%
\pgfusepath{clip}%
\pgfsetbuttcap%
\pgfsetmiterjoin%
\definecolor{currentfill}{rgb}{0.121569,0.466667,0.705882}%
\pgfsetfillcolor{currentfill}%
\pgfsetfillopacity{0.500000}%
\pgfsetlinewidth{1.003750pt}%
\definecolor{currentstroke}{rgb}{0.000000,0.000000,0.000000}%
\pgfsetstrokecolor{currentstroke}%
\pgfsetdash{}{0pt}%
\pgfpathmoveto{\pgfqpoint{5.132895in}{2.337032in}}%
\pgfpathlineto{\pgfqpoint{5.282799in}{2.337032in}}%
\pgfpathlineto{\pgfqpoint{5.282799in}{2.366941in}}%
\pgfpathlineto{\pgfqpoint{5.132895in}{2.366941in}}%
\pgfpathlineto{\pgfqpoint{5.132895in}{2.337032in}}%
\pgfpathclose%
\pgfusepath{stroke,fill}%
\end{pgfscope}%
\begin{pgfscope}%
\pgfpathrectangle{\pgfqpoint{5.095420in}{2.337032in}}{\pgfqpoint{0.824468in}{0.243158in}}%
\pgfusepath{clip}%
\pgfsetbuttcap%
\pgfsetmiterjoin%
\definecolor{currentfill}{rgb}{0.121569,0.466667,0.705882}%
\pgfsetfillcolor{currentfill}%
\pgfsetfillopacity{0.500000}%
\pgfsetlinewidth{1.003750pt}%
\definecolor{currentstroke}{rgb}{0.000000,0.000000,0.000000}%
\pgfsetstrokecolor{currentstroke}%
\pgfsetdash{}{0pt}%
\pgfpathmoveto{\pgfqpoint{5.282799in}{2.337032in}}%
\pgfpathlineto{\pgfqpoint{5.432702in}{2.337032in}}%
\pgfpathlineto{\pgfqpoint{5.432702in}{2.345664in}}%
\pgfpathlineto{\pgfqpoint{5.282799in}{2.345664in}}%
\pgfpathlineto{\pgfqpoint{5.282799in}{2.337032in}}%
\pgfpathclose%
\pgfusepath{stroke,fill}%
\end{pgfscope}%
\begin{pgfscope}%
\pgfpathrectangle{\pgfqpoint{5.095420in}{2.337032in}}{\pgfqpoint{0.824468in}{0.243158in}}%
\pgfusepath{clip}%
\pgfsetbuttcap%
\pgfsetmiterjoin%
\definecolor{currentfill}{rgb}{0.121569,0.466667,0.705882}%
\pgfsetfillcolor{currentfill}%
\pgfsetfillopacity{0.500000}%
\pgfsetlinewidth{1.003750pt}%
\definecolor{currentstroke}{rgb}{0.000000,0.000000,0.000000}%
\pgfsetstrokecolor{currentstroke}%
\pgfsetdash{}{0pt}%
\pgfpathmoveto{\pgfqpoint{5.432702in}{2.337032in}}%
\pgfpathlineto{\pgfqpoint{5.582605in}{2.337032in}}%
\pgfpathlineto{\pgfqpoint{5.582605in}{2.340680in}}%
\pgfpathlineto{\pgfqpoint{5.432702in}{2.340680in}}%
\pgfpathlineto{\pgfqpoint{5.432702in}{2.337032in}}%
\pgfpathclose%
\pgfusepath{stroke,fill}%
\end{pgfscope}%
\begin{pgfscope}%
\pgfpathrectangle{\pgfqpoint{5.095420in}{2.337032in}}{\pgfqpoint{0.824468in}{0.243158in}}%
\pgfusepath{clip}%
\pgfsetbuttcap%
\pgfsetmiterjoin%
\definecolor{currentfill}{rgb}{0.121569,0.466667,0.705882}%
\pgfsetfillcolor{currentfill}%
\pgfsetfillopacity{0.500000}%
\pgfsetlinewidth{1.003750pt}%
\definecolor{currentstroke}{rgb}{0.000000,0.000000,0.000000}%
\pgfsetstrokecolor{currentstroke}%
\pgfsetdash{}{0pt}%
\pgfpathmoveto{\pgfqpoint{5.582605in}{2.337032in}}%
\pgfpathlineto{\pgfqpoint{5.732509in}{2.337032in}}%
\pgfpathlineto{\pgfqpoint{5.732509in}{2.337275in}}%
\pgfpathlineto{\pgfqpoint{5.582605in}{2.337275in}}%
\pgfpathlineto{\pgfqpoint{5.582605in}{2.337032in}}%
\pgfpathclose%
\pgfusepath{stroke,fill}%
\end{pgfscope}%
\begin{pgfscope}%
\pgfpathrectangle{\pgfqpoint{5.095420in}{2.337032in}}{\pgfqpoint{0.824468in}{0.243158in}}%
\pgfusepath{clip}%
\pgfsetbuttcap%
\pgfsetmiterjoin%
\definecolor{currentfill}{rgb}{0.121569,0.466667,0.705882}%
\pgfsetfillcolor{currentfill}%
\pgfsetfillopacity{0.500000}%
\pgfsetlinewidth{1.003750pt}%
\definecolor{currentstroke}{rgb}{0.000000,0.000000,0.000000}%
\pgfsetstrokecolor{currentstroke}%
\pgfsetdash{}{0pt}%
\pgfpathmoveto{\pgfqpoint{5.732509in}{2.337032in}}%
\pgfpathlineto{\pgfqpoint{5.882412in}{2.337032in}}%
\pgfpathlineto{\pgfqpoint{5.882412in}{2.337154in}}%
\pgfpathlineto{\pgfqpoint{5.732509in}{2.337154in}}%
\pgfpathlineto{\pgfqpoint{5.732509in}{2.337032in}}%
\pgfpathclose%
\pgfusepath{stroke,fill}%
\end{pgfscope}%
\begin{pgfscope}%
\pgfsetrectcap%
\pgfsetmiterjoin%
\pgfsetlinewidth{0.803000pt}%
\definecolor{currentstroke}{rgb}{0.000000,0.000000,0.000000}%
\pgfsetstrokecolor{currentstroke}%
\pgfsetdash{}{0pt}%
\pgfpathmoveto{\pgfqpoint{5.095420in}{2.337032in}}%
\pgfpathlineto{\pgfqpoint{5.095420in}{2.580190in}}%
\pgfusepath{stroke}%
\end{pgfscope}%
\begin{pgfscope}%
\pgfsetrectcap%
\pgfsetmiterjoin%
\pgfsetlinewidth{0.803000pt}%
\definecolor{currentstroke}{rgb}{0.000000,0.000000,0.000000}%
\pgfsetstrokecolor{currentstroke}%
\pgfsetdash{}{0pt}%
\pgfpathmoveto{\pgfqpoint{5.919888in}{2.337032in}}%
\pgfpathlineto{\pgfqpoint{5.919888in}{2.580190in}}%
\pgfusepath{stroke}%
\end{pgfscope}%
\begin{pgfscope}%
\pgfsetrectcap%
\pgfsetmiterjoin%
\pgfsetlinewidth{0.803000pt}%
\definecolor{currentstroke}{rgb}{0.000000,0.000000,0.000000}%
\pgfsetstrokecolor{currentstroke}%
\pgfsetdash{}{0pt}%
\pgfpathmoveto{\pgfqpoint{5.095420in}{2.337032in}}%
\pgfpathlineto{\pgfqpoint{5.919888in}{2.337032in}}%
\pgfusepath{stroke}%
\end{pgfscope}%
\begin{pgfscope}%
\pgfsetrectcap%
\pgfsetmiterjoin%
\pgfsetlinewidth{0.803000pt}%
\definecolor{currentstroke}{rgb}{0.000000,0.000000,0.000000}%
\pgfsetstrokecolor{currentstroke}%
\pgfsetdash{}{0pt}%
\pgfpathmoveto{\pgfqpoint{5.095420in}{2.580190in}}%
\pgfpathlineto{\pgfqpoint{5.919888in}{2.580190in}}%
\pgfusepath{stroke}%
\end{pgfscope}%
\begin{pgfscope}%
\definecolor{textcolor}{rgb}{0.000000,0.000000,0.000000}%
\pgfsetstrokecolor{textcolor}%
\pgfsetfillcolor{textcolor}%
\pgftext[x=5.507654in,y=2.663523in,,base]{\color{textcolor}\rmfamily\fontsize{11.000000}{13.200000}\selectfont Ag2r L...}%
\end{pgfscope}%
\begin{pgfscope}%
\pgfsetbuttcap%
\pgfsetmiterjoin%
\definecolor{currentfill}{rgb}{1.000000,1.000000,1.000000}%
\pgfsetfillcolor{currentfill}%
\pgfsetlinewidth{0.000000pt}%
\definecolor{currentstroke}{rgb}{0.000000,0.000000,0.000000}%
\pgfsetstrokecolor{currentstroke}%
\pgfsetstrokeopacity{0.000000}%
\pgfsetdash{}{0pt}%
\pgfpathmoveto{\pgfqpoint{6.084781in}{2.337032in}}%
\pgfpathlineto{\pgfqpoint{6.909249in}{2.337032in}}%
\pgfpathlineto{\pgfqpoint{6.909249in}{2.580190in}}%
\pgfpathlineto{\pgfqpoint{6.084781in}{2.580190in}}%
\pgfpathlineto{\pgfqpoint{6.084781in}{2.337032in}}%
\pgfpathclose%
\pgfusepath{fill}%
\end{pgfscope}%
\begin{pgfscope}%
\pgfpathrectangle{\pgfqpoint{6.084781in}{2.337032in}}{\pgfqpoint{0.824468in}{0.243158in}}%
\pgfusepath{clip}%
\pgfsetbuttcap%
\pgfsetmiterjoin%
\definecolor{currentfill}{rgb}{0.121569,0.466667,0.705882}%
\pgfsetfillcolor{currentfill}%
\pgfsetfillopacity{0.500000}%
\pgfsetlinewidth{1.003750pt}%
\definecolor{currentstroke}{rgb}{0.000000,0.000000,0.000000}%
\pgfsetstrokecolor{currentstroke}%
\pgfsetdash{}{0pt}%
\pgfpathmoveto{\pgfqpoint{6.122257in}{2.337032in}}%
\pgfpathlineto{\pgfqpoint{6.272160in}{2.337032in}}%
\pgfpathlineto{\pgfqpoint{6.272160in}{2.354904in}}%
\pgfpathlineto{\pgfqpoint{6.122257in}{2.354904in}}%
\pgfpathlineto{\pgfqpoint{6.122257in}{2.337032in}}%
\pgfpathclose%
\pgfusepath{stroke,fill}%
\end{pgfscope}%
\begin{pgfscope}%
\pgfpathrectangle{\pgfqpoint{6.084781in}{2.337032in}}{\pgfqpoint{0.824468in}{0.243158in}}%
\pgfusepath{clip}%
\pgfsetbuttcap%
\pgfsetmiterjoin%
\definecolor{currentfill}{rgb}{0.121569,0.466667,0.705882}%
\pgfsetfillcolor{currentfill}%
\pgfsetfillopacity{0.500000}%
\pgfsetlinewidth{1.003750pt}%
\definecolor{currentstroke}{rgb}{0.000000,0.000000,0.000000}%
\pgfsetstrokecolor{currentstroke}%
\pgfsetdash{}{0pt}%
\pgfpathmoveto{\pgfqpoint{6.272160in}{2.337032in}}%
\pgfpathlineto{\pgfqpoint{6.422064in}{2.337032in}}%
\pgfpathlineto{\pgfqpoint{6.422064in}{2.340680in}}%
\pgfpathlineto{\pgfqpoint{6.272160in}{2.340680in}}%
\pgfpathlineto{\pgfqpoint{6.272160in}{2.337032in}}%
\pgfpathclose%
\pgfusepath{stroke,fill}%
\end{pgfscope}%
\begin{pgfscope}%
\pgfpathrectangle{\pgfqpoint{6.084781in}{2.337032in}}{\pgfqpoint{0.824468in}{0.243158in}}%
\pgfusepath{clip}%
\pgfsetbuttcap%
\pgfsetmiterjoin%
\definecolor{currentfill}{rgb}{0.121569,0.466667,0.705882}%
\pgfsetfillcolor{currentfill}%
\pgfsetfillopacity{0.500000}%
\pgfsetlinewidth{1.003750pt}%
\definecolor{currentstroke}{rgb}{0.000000,0.000000,0.000000}%
\pgfsetstrokecolor{currentstroke}%
\pgfsetdash{}{0pt}%
\pgfpathmoveto{\pgfqpoint{6.422064in}{2.337032in}}%
\pgfpathlineto{\pgfqpoint{6.571967in}{2.337032in}}%
\pgfpathlineto{\pgfqpoint{6.571967in}{2.339464in}}%
\pgfpathlineto{\pgfqpoint{6.422064in}{2.339464in}}%
\pgfpathlineto{\pgfqpoint{6.422064in}{2.337032in}}%
\pgfpathclose%
\pgfusepath{stroke,fill}%
\end{pgfscope}%
\begin{pgfscope}%
\pgfpathrectangle{\pgfqpoint{6.084781in}{2.337032in}}{\pgfqpoint{0.824468in}{0.243158in}}%
\pgfusepath{clip}%
\pgfsetbuttcap%
\pgfsetmiterjoin%
\definecolor{currentfill}{rgb}{0.121569,0.466667,0.705882}%
\pgfsetfillcolor{currentfill}%
\pgfsetfillopacity{0.500000}%
\pgfsetlinewidth{1.003750pt}%
\definecolor{currentstroke}{rgb}{0.000000,0.000000,0.000000}%
\pgfsetstrokecolor{currentstroke}%
\pgfsetdash{}{0pt}%
\pgfpathmoveto{\pgfqpoint{6.571967in}{2.337032in}}%
\pgfpathlineto{\pgfqpoint{6.721870in}{2.337032in}}%
\pgfpathlineto{\pgfqpoint{6.721870in}{2.337640in}}%
\pgfpathlineto{\pgfqpoint{6.571967in}{2.337640in}}%
\pgfpathlineto{\pgfqpoint{6.571967in}{2.337032in}}%
\pgfpathclose%
\pgfusepath{stroke,fill}%
\end{pgfscope}%
\begin{pgfscope}%
\pgfpathrectangle{\pgfqpoint{6.084781in}{2.337032in}}{\pgfqpoint{0.824468in}{0.243158in}}%
\pgfusepath{clip}%
\pgfsetbuttcap%
\pgfsetmiterjoin%
\definecolor{currentfill}{rgb}{0.121569,0.466667,0.705882}%
\pgfsetfillcolor{currentfill}%
\pgfsetfillopacity{0.500000}%
\pgfsetlinewidth{1.003750pt}%
\definecolor{currentstroke}{rgb}{0.000000,0.000000,0.000000}%
\pgfsetstrokecolor{currentstroke}%
\pgfsetdash{}{0pt}%
\pgfpathmoveto{\pgfqpoint{6.721870in}{2.337032in}}%
\pgfpathlineto{\pgfqpoint{6.871774in}{2.337032in}}%
\pgfpathlineto{\pgfqpoint{6.871774in}{2.337883in}}%
\pgfpathlineto{\pgfqpoint{6.721870in}{2.337883in}}%
\pgfpathlineto{\pgfqpoint{6.721870in}{2.337032in}}%
\pgfpathclose%
\pgfusepath{stroke,fill}%
\end{pgfscope}%
\begin{pgfscope}%
\pgfsetrectcap%
\pgfsetmiterjoin%
\pgfsetlinewidth{0.803000pt}%
\definecolor{currentstroke}{rgb}{0.000000,0.000000,0.000000}%
\pgfsetstrokecolor{currentstroke}%
\pgfsetdash{}{0pt}%
\pgfpathmoveto{\pgfqpoint{6.084781in}{2.337032in}}%
\pgfpathlineto{\pgfqpoint{6.084781in}{2.580190in}}%
\pgfusepath{stroke}%
\end{pgfscope}%
\begin{pgfscope}%
\pgfsetrectcap%
\pgfsetmiterjoin%
\pgfsetlinewidth{0.803000pt}%
\definecolor{currentstroke}{rgb}{0.000000,0.000000,0.000000}%
\pgfsetstrokecolor{currentstroke}%
\pgfsetdash{}{0pt}%
\pgfpathmoveto{\pgfqpoint{6.909249in}{2.337032in}}%
\pgfpathlineto{\pgfqpoint{6.909249in}{2.580190in}}%
\pgfusepath{stroke}%
\end{pgfscope}%
\begin{pgfscope}%
\pgfsetrectcap%
\pgfsetmiterjoin%
\pgfsetlinewidth{0.803000pt}%
\definecolor{currentstroke}{rgb}{0.000000,0.000000,0.000000}%
\pgfsetstrokecolor{currentstroke}%
\pgfsetdash{}{0pt}%
\pgfpathmoveto{\pgfqpoint{6.084781in}{2.337032in}}%
\pgfpathlineto{\pgfqpoint{6.909249in}{2.337032in}}%
\pgfusepath{stroke}%
\end{pgfscope}%
\begin{pgfscope}%
\pgfsetrectcap%
\pgfsetmiterjoin%
\pgfsetlinewidth{0.803000pt}%
\definecolor{currentstroke}{rgb}{0.000000,0.000000,0.000000}%
\pgfsetstrokecolor{currentstroke}%
\pgfsetdash{}{0pt}%
\pgfpathmoveto{\pgfqpoint{6.084781in}{2.580190in}}%
\pgfpathlineto{\pgfqpoint{6.909249in}{2.580190in}}%
\pgfusepath{stroke}%
\end{pgfscope}%
\begin{pgfscope}%
\definecolor{textcolor}{rgb}{0.000000,0.000000,0.000000}%
\pgfsetstrokecolor{textcolor}%
\pgfsetfillcolor{textcolor}%
\pgftext[x=6.497015in,y=2.663523in,,base]{\color{textcolor}\rmfamily\fontsize{11.000000}{13.200000}\selectfont Mgen}%
\end{pgfscope}%
\begin{pgfscope}%
\pgfsetbuttcap%
\pgfsetmiterjoin%
\definecolor{currentfill}{rgb}{1.000000,1.000000,1.000000}%
\pgfsetfillcolor{currentfill}%
\pgfsetlinewidth{0.000000pt}%
\definecolor{currentstroke}{rgb}{0.000000,0.000000,0.000000}%
\pgfsetstrokecolor{currentstroke}%
\pgfsetstrokeopacity{0.000000}%
\pgfsetdash{}{0pt}%
\pgfpathmoveto{\pgfqpoint{7.074143in}{2.337032in}}%
\pgfpathlineto{\pgfqpoint{7.898611in}{2.337032in}}%
\pgfpathlineto{\pgfqpoint{7.898611in}{2.580190in}}%
\pgfpathlineto{\pgfqpoint{7.074143in}{2.580190in}}%
\pgfpathlineto{\pgfqpoint{7.074143in}{2.337032in}}%
\pgfpathclose%
\pgfusepath{fill}%
\end{pgfscope}%
\begin{pgfscope}%
\pgfpathrectangle{\pgfqpoint{7.074143in}{2.337032in}}{\pgfqpoint{0.824468in}{0.243158in}}%
\pgfusepath{clip}%
\pgfsetbuttcap%
\pgfsetmiterjoin%
\definecolor{currentfill}{rgb}{0.121569,0.466667,0.705882}%
\pgfsetfillcolor{currentfill}%
\pgfsetfillopacity{0.500000}%
\pgfsetlinewidth{1.003750pt}%
\definecolor{currentstroke}{rgb}{0.000000,0.000000,0.000000}%
\pgfsetstrokecolor{currentstroke}%
\pgfsetdash{}{0pt}%
\pgfpathmoveto{\pgfqpoint{7.111619in}{2.337032in}}%
\pgfpathlineto{\pgfqpoint{7.261522in}{2.337032in}}%
\pgfpathlineto{\pgfqpoint{7.261522in}{2.337275in}}%
\pgfpathlineto{\pgfqpoint{7.111619in}{2.337275in}}%
\pgfpathlineto{\pgfqpoint{7.111619in}{2.337032in}}%
\pgfpathclose%
\pgfusepath{stroke,fill}%
\end{pgfscope}%
\begin{pgfscope}%
\pgfpathrectangle{\pgfqpoint{7.074143in}{2.337032in}}{\pgfqpoint{0.824468in}{0.243158in}}%
\pgfusepath{clip}%
\pgfsetbuttcap%
\pgfsetmiterjoin%
\definecolor{currentfill}{rgb}{0.121569,0.466667,0.705882}%
\pgfsetfillcolor{currentfill}%
\pgfsetfillopacity{0.500000}%
\pgfsetlinewidth{1.003750pt}%
\definecolor{currentstroke}{rgb}{0.000000,0.000000,0.000000}%
\pgfsetstrokecolor{currentstroke}%
\pgfsetdash{}{0pt}%
\pgfpathmoveto{\pgfqpoint{7.261522in}{2.337032in}}%
\pgfpathlineto{\pgfqpoint{7.411425in}{2.337032in}}%
\pgfpathlineto{\pgfqpoint{7.411425in}{2.338248in}}%
\pgfpathlineto{\pgfqpoint{7.261522in}{2.338248in}}%
\pgfpathlineto{\pgfqpoint{7.261522in}{2.337032in}}%
\pgfpathclose%
\pgfusepath{stroke,fill}%
\end{pgfscope}%
\begin{pgfscope}%
\pgfpathrectangle{\pgfqpoint{7.074143in}{2.337032in}}{\pgfqpoint{0.824468in}{0.243158in}}%
\pgfusepath{clip}%
\pgfsetbuttcap%
\pgfsetmiterjoin%
\definecolor{currentfill}{rgb}{0.121569,0.466667,0.705882}%
\pgfsetfillcolor{currentfill}%
\pgfsetfillopacity{0.500000}%
\pgfsetlinewidth{1.003750pt}%
\definecolor{currentstroke}{rgb}{0.000000,0.000000,0.000000}%
\pgfsetstrokecolor{currentstroke}%
\pgfsetdash{}{0pt}%
\pgfpathmoveto{\pgfqpoint{7.411425in}{2.337032in}}%
\pgfpathlineto{\pgfqpoint{7.561329in}{2.337032in}}%
\pgfpathlineto{\pgfqpoint{7.561329in}{2.338977in}}%
\pgfpathlineto{\pgfqpoint{7.411425in}{2.338977in}}%
\pgfpathlineto{\pgfqpoint{7.411425in}{2.337032in}}%
\pgfpathclose%
\pgfusepath{stroke,fill}%
\end{pgfscope}%
\begin{pgfscope}%
\pgfpathrectangle{\pgfqpoint{7.074143in}{2.337032in}}{\pgfqpoint{0.824468in}{0.243158in}}%
\pgfusepath{clip}%
\pgfsetbuttcap%
\pgfsetmiterjoin%
\definecolor{currentfill}{rgb}{0.121569,0.466667,0.705882}%
\pgfsetfillcolor{currentfill}%
\pgfsetfillopacity{0.500000}%
\pgfsetlinewidth{1.003750pt}%
\definecolor{currentstroke}{rgb}{0.000000,0.000000,0.000000}%
\pgfsetstrokecolor{currentstroke}%
\pgfsetdash{}{0pt}%
\pgfpathmoveto{\pgfqpoint{7.561329in}{2.337032in}}%
\pgfpathlineto{\pgfqpoint{7.711232in}{2.337032in}}%
\pgfpathlineto{\pgfqpoint{7.711232in}{2.345300in}}%
\pgfpathlineto{\pgfqpoint{7.561329in}{2.345300in}}%
\pgfpathlineto{\pgfqpoint{7.561329in}{2.337032in}}%
\pgfpathclose%
\pgfusepath{stroke,fill}%
\end{pgfscope}%
\begin{pgfscope}%
\pgfpathrectangle{\pgfqpoint{7.074143in}{2.337032in}}{\pgfqpoint{0.824468in}{0.243158in}}%
\pgfusepath{clip}%
\pgfsetbuttcap%
\pgfsetmiterjoin%
\definecolor{currentfill}{rgb}{0.121569,0.466667,0.705882}%
\pgfsetfillcolor{currentfill}%
\pgfsetfillopacity{0.500000}%
\pgfsetlinewidth{1.003750pt}%
\definecolor{currentstroke}{rgb}{0.000000,0.000000,0.000000}%
\pgfsetstrokecolor{currentstroke}%
\pgfsetdash{}{0pt}%
\pgfpathmoveto{\pgfqpoint{7.711232in}{2.337032in}}%
\pgfpathlineto{\pgfqpoint{7.861135in}{2.337032in}}%
\pgfpathlineto{\pgfqpoint{7.861135in}{2.355147in}}%
\pgfpathlineto{\pgfqpoint{7.711232in}{2.355147in}}%
\pgfpathlineto{\pgfqpoint{7.711232in}{2.337032in}}%
\pgfpathclose%
\pgfusepath{stroke,fill}%
\end{pgfscope}%
\begin{pgfscope}%
\pgfsetrectcap%
\pgfsetmiterjoin%
\pgfsetlinewidth{0.803000pt}%
\definecolor{currentstroke}{rgb}{0.000000,0.000000,0.000000}%
\pgfsetstrokecolor{currentstroke}%
\pgfsetdash{}{0pt}%
\pgfpathmoveto{\pgfqpoint{7.074143in}{2.337032in}}%
\pgfpathlineto{\pgfqpoint{7.074143in}{2.580190in}}%
\pgfusepath{stroke}%
\end{pgfscope}%
\begin{pgfscope}%
\pgfsetrectcap%
\pgfsetmiterjoin%
\pgfsetlinewidth{0.803000pt}%
\definecolor{currentstroke}{rgb}{0.000000,0.000000,0.000000}%
\pgfsetstrokecolor{currentstroke}%
\pgfsetdash{}{0pt}%
\pgfpathmoveto{\pgfqpoint{7.898611in}{2.337032in}}%
\pgfpathlineto{\pgfqpoint{7.898611in}{2.580190in}}%
\pgfusepath{stroke}%
\end{pgfscope}%
\begin{pgfscope}%
\pgfsetrectcap%
\pgfsetmiterjoin%
\pgfsetlinewidth{0.803000pt}%
\definecolor{currentstroke}{rgb}{0.000000,0.000000,0.000000}%
\pgfsetstrokecolor{currentstroke}%
\pgfsetdash{}{0pt}%
\pgfpathmoveto{\pgfqpoint{7.074143in}{2.337032in}}%
\pgfpathlineto{\pgfqpoint{7.898611in}{2.337032in}}%
\pgfusepath{stroke}%
\end{pgfscope}%
\begin{pgfscope}%
\pgfsetrectcap%
\pgfsetmiterjoin%
\pgfsetlinewidth{0.803000pt}%
\definecolor{currentstroke}{rgb}{0.000000,0.000000,0.000000}%
\pgfsetstrokecolor{currentstroke}%
\pgfsetdash{}{0pt}%
\pgfpathmoveto{\pgfqpoint{7.074143in}{2.580190in}}%
\pgfpathlineto{\pgfqpoint{7.898611in}{2.580190in}}%
\pgfusepath{stroke}%
\end{pgfscope}%
\begin{pgfscope}%
\definecolor{textcolor}{rgb}{0.000000,0.000000,0.000000}%
\pgfsetstrokecolor{textcolor}%
\pgfsetfillcolor{textcolor}%
\pgftext[x=7.486377in,y=2.663523in,,base]{\color{textcolor}\rmfamily\fontsize{11.000000}{13.200000}\selectfont Zen'Up}%
\end{pgfscope}%
\begin{pgfscope}%
\pgfsetbuttcap%
\pgfsetmiterjoin%
\definecolor{currentfill}{rgb}{1.000000,1.000000,1.000000}%
\pgfsetfillcolor{currentfill}%
\pgfsetlinewidth{0.000000pt}%
\definecolor{currentstroke}{rgb}{0.000000,0.000000,0.000000}%
\pgfsetstrokecolor{currentstroke}%
\pgfsetstrokeopacity{0.000000}%
\pgfsetdash{}{0pt}%
\pgfpathmoveto{\pgfqpoint{0.148611in}{1.607558in}}%
\pgfpathlineto{\pgfqpoint{0.973079in}{1.607558in}}%
\pgfpathlineto{\pgfqpoint{0.973079in}{1.850716in}}%
\pgfpathlineto{\pgfqpoint{0.148611in}{1.850716in}}%
\pgfpathlineto{\pgfqpoint{0.148611in}{1.607558in}}%
\pgfpathclose%
\pgfusepath{fill}%
\end{pgfscope}%
\begin{pgfscope}%
\pgfpathrectangle{\pgfqpoint{0.148611in}{1.607558in}}{\pgfqpoint{0.824468in}{0.243158in}}%
\pgfusepath{clip}%
\pgfsetbuttcap%
\pgfsetmiterjoin%
\definecolor{currentfill}{rgb}{0.121569,0.466667,0.705882}%
\pgfsetfillcolor{currentfill}%
\pgfsetfillopacity{0.500000}%
\pgfsetlinewidth{1.003750pt}%
\definecolor{currentstroke}{rgb}{0.000000,0.000000,0.000000}%
\pgfsetstrokecolor{currentstroke}%
\pgfsetdash{}{0pt}%
\pgfpathmoveto{\pgfqpoint{0.186087in}{1.607558in}}%
\pgfpathlineto{\pgfqpoint{0.335990in}{1.607558in}}%
\pgfpathlineto{\pgfqpoint{0.335990in}{1.612178in}}%
\pgfpathlineto{\pgfqpoint{0.186087in}{1.612178in}}%
\pgfpathlineto{\pgfqpoint{0.186087in}{1.607558in}}%
\pgfpathclose%
\pgfusepath{stroke,fill}%
\end{pgfscope}%
\begin{pgfscope}%
\pgfpathrectangle{\pgfqpoint{0.148611in}{1.607558in}}{\pgfqpoint{0.824468in}{0.243158in}}%
\pgfusepath{clip}%
\pgfsetbuttcap%
\pgfsetmiterjoin%
\definecolor{currentfill}{rgb}{0.121569,0.466667,0.705882}%
\pgfsetfillcolor{currentfill}%
\pgfsetfillopacity{0.500000}%
\pgfsetlinewidth{1.003750pt}%
\definecolor{currentstroke}{rgb}{0.000000,0.000000,0.000000}%
\pgfsetstrokecolor{currentstroke}%
\pgfsetdash{}{0pt}%
\pgfpathmoveto{\pgfqpoint{0.335990in}{1.607558in}}%
\pgfpathlineto{\pgfqpoint{0.485894in}{1.607558in}}%
\pgfpathlineto{\pgfqpoint{0.485894in}{1.611571in}}%
\pgfpathlineto{\pgfqpoint{0.335990in}{1.611571in}}%
\pgfpathlineto{\pgfqpoint{0.335990in}{1.607558in}}%
\pgfpathclose%
\pgfusepath{stroke,fill}%
\end{pgfscope}%
\begin{pgfscope}%
\pgfpathrectangle{\pgfqpoint{0.148611in}{1.607558in}}{\pgfqpoint{0.824468in}{0.243158in}}%
\pgfusepath{clip}%
\pgfsetbuttcap%
\pgfsetmiterjoin%
\definecolor{currentfill}{rgb}{0.121569,0.466667,0.705882}%
\pgfsetfillcolor{currentfill}%
\pgfsetfillopacity{0.500000}%
\pgfsetlinewidth{1.003750pt}%
\definecolor{currentstroke}{rgb}{0.000000,0.000000,0.000000}%
\pgfsetstrokecolor{currentstroke}%
\pgfsetdash{}{0pt}%
\pgfpathmoveto{\pgfqpoint{0.485894in}{1.607558in}}%
\pgfpathlineto{\pgfqpoint{0.635797in}{1.607558in}}%
\pgfpathlineto{\pgfqpoint{0.635797in}{1.619960in}}%
\pgfpathlineto{\pgfqpoint{0.485894in}{1.619960in}}%
\pgfpathlineto{\pgfqpoint{0.485894in}{1.607558in}}%
\pgfpathclose%
\pgfusepath{stroke,fill}%
\end{pgfscope}%
\begin{pgfscope}%
\pgfpathrectangle{\pgfqpoint{0.148611in}{1.607558in}}{\pgfqpoint{0.824468in}{0.243158in}}%
\pgfusepath{clip}%
\pgfsetbuttcap%
\pgfsetmiterjoin%
\definecolor{currentfill}{rgb}{0.121569,0.466667,0.705882}%
\pgfsetfillcolor{currentfill}%
\pgfsetfillopacity{0.500000}%
\pgfsetlinewidth{1.003750pt}%
\definecolor{currentstroke}{rgb}{0.000000,0.000000,0.000000}%
\pgfsetstrokecolor{currentstroke}%
\pgfsetdash{}{0pt}%
\pgfpathmoveto{\pgfqpoint{0.635797in}{1.607558in}}%
\pgfpathlineto{\pgfqpoint{0.785700in}{1.607558in}}%
\pgfpathlineto{\pgfqpoint{0.785700in}{1.628713in}}%
\pgfpathlineto{\pgfqpoint{0.635797in}{1.628713in}}%
\pgfpathlineto{\pgfqpoint{0.635797in}{1.607558in}}%
\pgfpathclose%
\pgfusepath{stroke,fill}%
\end{pgfscope}%
\begin{pgfscope}%
\pgfpathrectangle{\pgfqpoint{0.148611in}{1.607558in}}{\pgfqpoint{0.824468in}{0.243158in}}%
\pgfusepath{clip}%
\pgfsetbuttcap%
\pgfsetmiterjoin%
\definecolor{currentfill}{rgb}{0.121569,0.466667,0.705882}%
\pgfsetfillcolor{currentfill}%
\pgfsetfillopacity{0.500000}%
\pgfsetlinewidth{1.003750pt}%
\definecolor{currentstroke}{rgb}{0.000000,0.000000,0.000000}%
\pgfsetstrokecolor{currentstroke}%
\pgfsetdash{}{0pt}%
\pgfpathmoveto{\pgfqpoint{0.785700in}{1.607558in}}%
\pgfpathlineto{\pgfqpoint{0.935603in}{1.607558in}}%
\pgfpathlineto{\pgfqpoint{0.935603in}{1.619595in}}%
\pgfpathlineto{\pgfqpoint{0.785700in}{1.619595in}}%
\pgfpathlineto{\pgfqpoint{0.785700in}{1.607558in}}%
\pgfpathclose%
\pgfusepath{stroke,fill}%
\end{pgfscope}%
\begin{pgfscope}%
\pgfsetrectcap%
\pgfsetmiterjoin%
\pgfsetlinewidth{0.803000pt}%
\definecolor{currentstroke}{rgb}{0.000000,0.000000,0.000000}%
\pgfsetstrokecolor{currentstroke}%
\pgfsetdash{}{0pt}%
\pgfpathmoveto{\pgfqpoint{0.148611in}{1.607558in}}%
\pgfpathlineto{\pgfqpoint{0.148611in}{1.850716in}}%
\pgfusepath{stroke}%
\end{pgfscope}%
\begin{pgfscope}%
\pgfsetrectcap%
\pgfsetmiterjoin%
\pgfsetlinewidth{0.803000pt}%
\definecolor{currentstroke}{rgb}{0.000000,0.000000,0.000000}%
\pgfsetstrokecolor{currentstroke}%
\pgfsetdash{}{0pt}%
\pgfpathmoveto{\pgfqpoint{0.973079in}{1.607558in}}%
\pgfpathlineto{\pgfqpoint{0.973079in}{1.850716in}}%
\pgfusepath{stroke}%
\end{pgfscope}%
\begin{pgfscope}%
\pgfsetrectcap%
\pgfsetmiterjoin%
\pgfsetlinewidth{0.803000pt}%
\definecolor{currentstroke}{rgb}{0.000000,0.000000,0.000000}%
\pgfsetstrokecolor{currentstroke}%
\pgfsetdash{}{0pt}%
\pgfpathmoveto{\pgfqpoint{0.148611in}{1.607558in}}%
\pgfpathlineto{\pgfqpoint{0.973079in}{1.607558in}}%
\pgfusepath{stroke}%
\end{pgfscope}%
\begin{pgfscope}%
\pgfsetrectcap%
\pgfsetmiterjoin%
\pgfsetlinewidth{0.803000pt}%
\definecolor{currentstroke}{rgb}{0.000000,0.000000,0.000000}%
\pgfsetstrokecolor{currentstroke}%
\pgfsetdash{}{0pt}%
\pgfpathmoveto{\pgfqpoint{0.148611in}{1.850716in}}%
\pgfpathlineto{\pgfqpoint{0.973079in}{1.850716in}}%
\pgfusepath{stroke}%
\end{pgfscope}%
\begin{pgfscope}%
\definecolor{textcolor}{rgb}{0.000000,0.000000,0.000000}%
\pgfsetstrokecolor{textcolor}%
\pgfsetfillcolor{textcolor}%
\pgftext[x=0.560845in,y=1.934050in,,base]{\color{textcolor}\rmfamily\fontsize{11.000000}{13.200000}\selectfont MGP}%
\end{pgfscope}%
\begin{pgfscope}%
\pgfsetbuttcap%
\pgfsetmiterjoin%
\definecolor{currentfill}{rgb}{1.000000,1.000000,1.000000}%
\pgfsetfillcolor{currentfill}%
\pgfsetlinewidth{0.000000pt}%
\definecolor{currentstroke}{rgb}{0.000000,0.000000,0.000000}%
\pgfsetstrokecolor{currentstroke}%
\pgfsetstrokeopacity{0.000000}%
\pgfsetdash{}{0pt}%
\pgfpathmoveto{\pgfqpoint{1.137973in}{1.607558in}}%
\pgfpathlineto{\pgfqpoint{1.962441in}{1.607558in}}%
\pgfpathlineto{\pgfqpoint{1.962441in}{1.850716in}}%
\pgfpathlineto{\pgfqpoint{1.137973in}{1.850716in}}%
\pgfpathlineto{\pgfqpoint{1.137973in}{1.607558in}}%
\pgfpathclose%
\pgfusepath{fill}%
\end{pgfscope}%
\begin{pgfscope}%
\pgfpathrectangle{\pgfqpoint{1.137973in}{1.607558in}}{\pgfqpoint{0.824468in}{0.243158in}}%
\pgfusepath{clip}%
\pgfsetbuttcap%
\pgfsetmiterjoin%
\definecolor{currentfill}{rgb}{0.121569,0.466667,0.705882}%
\pgfsetfillcolor{currentfill}%
\pgfsetfillopacity{0.500000}%
\pgfsetlinewidth{1.003750pt}%
\definecolor{currentstroke}{rgb}{0.000000,0.000000,0.000000}%
\pgfsetstrokecolor{currentstroke}%
\pgfsetdash{}{0pt}%
\pgfpathmoveto{\pgfqpoint{1.175449in}{1.607558in}}%
\pgfpathlineto{\pgfqpoint{1.325352in}{1.607558in}}%
\pgfpathlineto{\pgfqpoint{1.325352in}{1.611935in}}%
\pgfpathlineto{\pgfqpoint{1.175449in}{1.611935in}}%
\pgfpathlineto{\pgfqpoint{1.175449in}{1.607558in}}%
\pgfpathclose%
\pgfusepath{stroke,fill}%
\end{pgfscope}%
\begin{pgfscope}%
\pgfpathrectangle{\pgfqpoint{1.137973in}{1.607558in}}{\pgfqpoint{0.824468in}{0.243158in}}%
\pgfusepath{clip}%
\pgfsetbuttcap%
\pgfsetmiterjoin%
\definecolor{currentfill}{rgb}{0.121569,0.466667,0.705882}%
\pgfsetfillcolor{currentfill}%
\pgfsetfillopacity{0.500000}%
\pgfsetlinewidth{1.003750pt}%
\definecolor{currentstroke}{rgb}{0.000000,0.000000,0.000000}%
\pgfsetstrokecolor{currentstroke}%
\pgfsetdash{}{0pt}%
\pgfpathmoveto{\pgfqpoint{1.325352in}{1.607558in}}%
\pgfpathlineto{\pgfqpoint{1.475255in}{1.607558in}}%
\pgfpathlineto{\pgfqpoint{1.475255in}{1.608896in}}%
\pgfpathlineto{\pgfqpoint{1.325352in}{1.608896in}}%
\pgfpathlineto{\pgfqpoint{1.325352in}{1.607558in}}%
\pgfpathclose%
\pgfusepath{stroke,fill}%
\end{pgfscope}%
\begin{pgfscope}%
\pgfpathrectangle{\pgfqpoint{1.137973in}{1.607558in}}{\pgfqpoint{0.824468in}{0.243158in}}%
\pgfusepath{clip}%
\pgfsetbuttcap%
\pgfsetmiterjoin%
\definecolor{currentfill}{rgb}{0.121569,0.466667,0.705882}%
\pgfsetfillcolor{currentfill}%
\pgfsetfillopacity{0.500000}%
\pgfsetlinewidth{1.003750pt}%
\definecolor{currentstroke}{rgb}{0.000000,0.000000,0.000000}%
\pgfsetstrokecolor{currentstroke}%
\pgfsetdash{}{0pt}%
\pgfpathmoveto{\pgfqpoint{1.475255in}{1.607558in}}%
\pgfpathlineto{\pgfqpoint{1.625158in}{1.607558in}}%
\pgfpathlineto{\pgfqpoint{1.625158in}{1.608774in}}%
\pgfpathlineto{\pgfqpoint{1.475255in}{1.608774in}}%
\pgfpathlineto{\pgfqpoint{1.475255in}{1.607558in}}%
\pgfpathclose%
\pgfusepath{stroke,fill}%
\end{pgfscope}%
\begin{pgfscope}%
\pgfpathrectangle{\pgfqpoint{1.137973in}{1.607558in}}{\pgfqpoint{0.824468in}{0.243158in}}%
\pgfusepath{clip}%
\pgfsetbuttcap%
\pgfsetmiterjoin%
\definecolor{currentfill}{rgb}{0.121569,0.466667,0.705882}%
\pgfsetfillcolor{currentfill}%
\pgfsetfillopacity{0.500000}%
\pgfsetlinewidth{1.003750pt}%
\definecolor{currentstroke}{rgb}{0.000000,0.000000,0.000000}%
\pgfsetstrokecolor{currentstroke}%
\pgfsetdash{}{0pt}%
\pgfpathmoveto{\pgfqpoint{1.625158in}{1.607558in}}%
\pgfpathlineto{\pgfqpoint{1.775062in}{1.607558in}}%
\pgfpathlineto{\pgfqpoint{1.775062in}{1.608166in}}%
\pgfpathlineto{\pgfqpoint{1.625158in}{1.608166in}}%
\pgfpathlineto{\pgfqpoint{1.625158in}{1.607558in}}%
\pgfpathclose%
\pgfusepath{stroke,fill}%
\end{pgfscope}%
\begin{pgfscope}%
\pgfpathrectangle{\pgfqpoint{1.137973in}{1.607558in}}{\pgfqpoint{0.824468in}{0.243158in}}%
\pgfusepath{clip}%
\pgfsetbuttcap%
\pgfsetmiterjoin%
\definecolor{currentfill}{rgb}{0.121569,0.466667,0.705882}%
\pgfsetfillcolor{currentfill}%
\pgfsetfillopacity{0.500000}%
\pgfsetlinewidth{1.003750pt}%
\definecolor{currentstroke}{rgb}{0.000000,0.000000,0.000000}%
\pgfsetstrokecolor{currentstroke}%
\pgfsetdash{}{0pt}%
\pgfpathmoveto{\pgfqpoint{1.775062in}{1.607558in}}%
\pgfpathlineto{\pgfqpoint{1.924965in}{1.607558in}}%
\pgfpathlineto{\pgfqpoint{1.924965in}{1.607680in}}%
\pgfpathlineto{\pgfqpoint{1.775062in}{1.607680in}}%
\pgfpathlineto{\pgfqpoint{1.775062in}{1.607558in}}%
\pgfpathclose%
\pgfusepath{stroke,fill}%
\end{pgfscope}%
\begin{pgfscope}%
\pgfsetrectcap%
\pgfsetmiterjoin%
\pgfsetlinewidth{0.803000pt}%
\definecolor{currentstroke}{rgb}{0.000000,0.000000,0.000000}%
\pgfsetstrokecolor{currentstroke}%
\pgfsetdash{}{0pt}%
\pgfpathmoveto{\pgfqpoint{1.137973in}{1.607558in}}%
\pgfpathlineto{\pgfqpoint{1.137973in}{1.850716in}}%
\pgfusepath{stroke}%
\end{pgfscope}%
\begin{pgfscope}%
\pgfsetrectcap%
\pgfsetmiterjoin%
\pgfsetlinewidth{0.803000pt}%
\definecolor{currentstroke}{rgb}{0.000000,0.000000,0.000000}%
\pgfsetstrokecolor{currentstroke}%
\pgfsetdash{}{0pt}%
\pgfpathmoveto{\pgfqpoint{1.962441in}{1.607558in}}%
\pgfpathlineto{\pgfqpoint{1.962441in}{1.850716in}}%
\pgfusepath{stroke}%
\end{pgfscope}%
\begin{pgfscope}%
\pgfsetrectcap%
\pgfsetmiterjoin%
\pgfsetlinewidth{0.803000pt}%
\definecolor{currentstroke}{rgb}{0.000000,0.000000,0.000000}%
\pgfsetstrokecolor{currentstroke}%
\pgfsetdash{}{0pt}%
\pgfpathmoveto{\pgfqpoint{1.137973in}{1.607558in}}%
\pgfpathlineto{\pgfqpoint{1.962441in}{1.607558in}}%
\pgfusepath{stroke}%
\end{pgfscope}%
\begin{pgfscope}%
\pgfsetrectcap%
\pgfsetmiterjoin%
\pgfsetlinewidth{0.803000pt}%
\definecolor{currentstroke}{rgb}{0.000000,0.000000,0.000000}%
\pgfsetstrokecolor{currentstroke}%
\pgfsetdash{}{0pt}%
\pgfpathmoveto{\pgfqpoint{1.137973in}{1.850716in}}%
\pgfpathlineto{\pgfqpoint{1.962441in}{1.850716in}}%
\pgfusepath{stroke}%
\end{pgfscope}%
\begin{pgfscope}%
\definecolor{textcolor}{rgb}{0.000000,0.000000,0.000000}%
\pgfsetstrokecolor{textcolor}%
\pgfsetfillcolor{textcolor}%
\pgftext[x=1.550207in,y=1.934050in,,base]{\color{textcolor}\rmfamily\fontsize{11.000000}{13.200000}\selectfont Intériale}%
\end{pgfscope}%
\begin{pgfscope}%
\pgfsetbuttcap%
\pgfsetmiterjoin%
\definecolor{currentfill}{rgb}{1.000000,1.000000,1.000000}%
\pgfsetfillcolor{currentfill}%
\pgfsetlinewidth{0.000000pt}%
\definecolor{currentstroke}{rgb}{0.000000,0.000000,0.000000}%
\pgfsetstrokecolor{currentstroke}%
\pgfsetstrokeopacity{0.000000}%
\pgfsetdash{}{0pt}%
\pgfpathmoveto{\pgfqpoint{2.127335in}{1.607558in}}%
\pgfpathlineto{\pgfqpoint{2.951803in}{1.607558in}}%
\pgfpathlineto{\pgfqpoint{2.951803in}{1.850716in}}%
\pgfpathlineto{\pgfqpoint{2.127335in}{1.850716in}}%
\pgfpathlineto{\pgfqpoint{2.127335in}{1.607558in}}%
\pgfpathclose%
\pgfusepath{fill}%
\end{pgfscope}%
\begin{pgfscope}%
\pgfpathrectangle{\pgfqpoint{2.127335in}{1.607558in}}{\pgfqpoint{0.824468in}{0.243158in}}%
\pgfusepath{clip}%
\pgfsetbuttcap%
\pgfsetmiterjoin%
\definecolor{currentfill}{rgb}{0.121569,0.466667,0.705882}%
\pgfsetfillcolor{currentfill}%
\pgfsetfillopacity{0.500000}%
\pgfsetlinewidth{1.003750pt}%
\definecolor{currentstroke}{rgb}{0.000000,0.000000,0.000000}%
\pgfsetstrokecolor{currentstroke}%
\pgfsetdash{}{0pt}%
\pgfpathmoveto{\pgfqpoint{2.164810in}{1.607558in}}%
\pgfpathlineto{\pgfqpoint{2.314714in}{1.607558in}}%
\pgfpathlineto{\pgfqpoint{2.314714in}{1.617406in}}%
\pgfpathlineto{\pgfqpoint{2.164810in}{1.617406in}}%
\pgfpathlineto{\pgfqpoint{2.164810in}{1.607558in}}%
\pgfpathclose%
\pgfusepath{stroke,fill}%
\end{pgfscope}%
\begin{pgfscope}%
\pgfpathrectangle{\pgfqpoint{2.127335in}{1.607558in}}{\pgfqpoint{0.824468in}{0.243158in}}%
\pgfusepath{clip}%
\pgfsetbuttcap%
\pgfsetmiterjoin%
\definecolor{currentfill}{rgb}{0.121569,0.466667,0.705882}%
\pgfsetfillcolor{currentfill}%
\pgfsetfillopacity{0.500000}%
\pgfsetlinewidth{1.003750pt}%
\definecolor{currentstroke}{rgb}{0.000000,0.000000,0.000000}%
\pgfsetstrokecolor{currentstroke}%
\pgfsetdash{}{0pt}%
\pgfpathmoveto{\pgfqpoint{2.314714in}{1.607558in}}%
\pgfpathlineto{\pgfqpoint{2.464617in}{1.607558in}}%
\pgfpathlineto{\pgfqpoint{2.464617in}{1.609868in}}%
\pgfpathlineto{\pgfqpoint{2.314714in}{1.609868in}}%
\pgfpathlineto{\pgfqpoint{2.314714in}{1.607558in}}%
\pgfpathclose%
\pgfusepath{stroke,fill}%
\end{pgfscope}%
\begin{pgfscope}%
\pgfpathrectangle{\pgfqpoint{2.127335in}{1.607558in}}{\pgfqpoint{0.824468in}{0.243158in}}%
\pgfusepath{clip}%
\pgfsetbuttcap%
\pgfsetmiterjoin%
\definecolor{currentfill}{rgb}{0.121569,0.466667,0.705882}%
\pgfsetfillcolor{currentfill}%
\pgfsetfillopacity{0.500000}%
\pgfsetlinewidth{1.003750pt}%
\definecolor{currentstroke}{rgb}{0.000000,0.000000,0.000000}%
\pgfsetstrokecolor{currentstroke}%
\pgfsetdash{}{0pt}%
\pgfpathmoveto{\pgfqpoint{2.464617in}{1.607558in}}%
\pgfpathlineto{\pgfqpoint{2.614520in}{1.607558in}}%
\pgfpathlineto{\pgfqpoint{2.614520in}{1.610355in}}%
\pgfpathlineto{\pgfqpoint{2.464617in}{1.610355in}}%
\pgfpathlineto{\pgfqpoint{2.464617in}{1.607558in}}%
\pgfpathclose%
\pgfusepath{stroke,fill}%
\end{pgfscope}%
\begin{pgfscope}%
\pgfpathrectangle{\pgfqpoint{2.127335in}{1.607558in}}{\pgfqpoint{0.824468in}{0.243158in}}%
\pgfusepath{clip}%
\pgfsetbuttcap%
\pgfsetmiterjoin%
\definecolor{currentfill}{rgb}{0.121569,0.466667,0.705882}%
\pgfsetfillcolor{currentfill}%
\pgfsetfillopacity{0.500000}%
\pgfsetlinewidth{1.003750pt}%
\definecolor{currentstroke}{rgb}{0.000000,0.000000,0.000000}%
\pgfsetstrokecolor{currentstroke}%
\pgfsetdash{}{0pt}%
\pgfpathmoveto{\pgfqpoint{2.614520in}{1.607558in}}%
\pgfpathlineto{\pgfqpoint{2.764423in}{1.607558in}}%
\pgfpathlineto{\pgfqpoint{2.764423in}{1.613637in}}%
\pgfpathlineto{\pgfqpoint{2.614520in}{1.613637in}}%
\pgfpathlineto{\pgfqpoint{2.614520in}{1.607558in}}%
\pgfpathclose%
\pgfusepath{stroke,fill}%
\end{pgfscope}%
\begin{pgfscope}%
\pgfpathrectangle{\pgfqpoint{2.127335in}{1.607558in}}{\pgfqpoint{0.824468in}{0.243158in}}%
\pgfusepath{clip}%
\pgfsetbuttcap%
\pgfsetmiterjoin%
\definecolor{currentfill}{rgb}{0.121569,0.466667,0.705882}%
\pgfsetfillcolor{currentfill}%
\pgfsetfillopacity{0.500000}%
\pgfsetlinewidth{1.003750pt}%
\definecolor{currentstroke}{rgb}{0.000000,0.000000,0.000000}%
\pgfsetstrokecolor{currentstroke}%
\pgfsetdash{}{0pt}%
\pgfpathmoveto{\pgfqpoint{2.764423in}{1.607558in}}%
\pgfpathlineto{\pgfqpoint{2.914327in}{1.607558in}}%
\pgfpathlineto{\pgfqpoint{2.914327in}{1.612300in}}%
\pgfpathlineto{\pgfqpoint{2.764423in}{1.612300in}}%
\pgfpathlineto{\pgfqpoint{2.764423in}{1.607558in}}%
\pgfpathclose%
\pgfusepath{stroke,fill}%
\end{pgfscope}%
\begin{pgfscope}%
\pgfsetrectcap%
\pgfsetmiterjoin%
\pgfsetlinewidth{0.803000pt}%
\definecolor{currentstroke}{rgb}{0.000000,0.000000,0.000000}%
\pgfsetstrokecolor{currentstroke}%
\pgfsetdash{}{0pt}%
\pgfpathmoveto{\pgfqpoint{2.127335in}{1.607558in}}%
\pgfpathlineto{\pgfqpoint{2.127335in}{1.850716in}}%
\pgfusepath{stroke}%
\end{pgfscope}%
\begin{pgfscope}%
\pgfsetrectcap%
\pgfsetmiterjoin%
\pgfsetlinewidth{0.803000pt}%
\definecolor{currentstroke}{rgb}{0.000000,0.000000,0.000000}%
\pgfsetstrokecolor{currentstroke}%
\pgfsetdash{}{0pt}%
\pgfpathmoveto{\pgfqpoint{2.951803in}{1.607558in}}%
\pgfpathlineto{\pgfqpoint{2.951803in}{1.850716in}}%
\pgfusepath{stroke}%
\end{pgfscope}%
\begin{pgfscope}%
\pgfsetrectcap%
\pgfsetmiterjoin%
\pgfsetlinewidth{0.803000pt}%
\definecolor{currentstroke}{rgb}{0.000000,0.000000,0.000000}%
\pgfsetstrokecolor{currentstroke}%
\pgfsetdash{}{0pt}%
\pgfpathmoveto{\pgfqpoint{2.127335in}{1.607558in}}%
\pgfpathlineto{\pgfqpoint{2.951803in}{1.607558in}}%
\pgfusepath{stroke}%
\end{pgfscope}%
\begin{pgfscope}%
\pgfsetrectcap%
\pgfsetmiterjoin%
\pgfsetlinewidth{0.803000pt}%
\definecolor{currentstroke}{rgb}{0.000000,0.000000,0.000000}%
\pgfsetstrokecolor{currentstroke}%
\pgfsetdash{}{0pt}%
\pgfpathmoveto{\pgfqpoint{2.127335in}{1.850716in}}%
\pgfpathlineto{\pgfqpoint{2.951803in}{1.850716in}}%
\pgfusepath{stroke}%
\end{pgfscope}%
\begin{pgfscope}%
\definecolor{textcolor}{rgb}{0.000000,0.000000,0.000000}%
\pgfsetstrokecolor{textcolor}%
\pgfsetfillcolor{textcolor}%
\pgftext[x=2.539569in,y=1.934050in,,base]{\color{textcolor}\rmfamily\fontsize{11.000000}{13.200000}\selectfont Généra...}%
\end{pgfscope}%
\begin{pgfscope}%
\pgfsetbuttcap%
\pgfsetmiterjoin%
\definecolor{currentfill}{rgb}{1.000000,1.000000,1.000000}%
\pgfsetfillcolor{currentfill}%
\pgfsetlinewidth{0.000000pt}%
\definecolor{currentstroke}{rgb}{0.000000,0.000000,0.000000}%
\pgfsetstrokecolor{currentstroke}%
\pgfsetstrokeopacity{0.000000}%
\pgfsetdash{}{0pt}%
\pgfpathmoveto{\pgfqpoint{3.116696in}{1.607558in}}%
\pgfpathlineto{\pgfqpoint{3.941164in}{1.607558in}}%
\pgfpathlineto{\pgfqpoint{3.941164in}{1.850716in}}%
\pgfpathlineto{\pgfqpoint{3.116696in}{1.850716in}}%
\pgfpathlineto{\pgfqpoint{3.116696in}{1.607558in}}%
\pgfpathclose%
\pgfusepath{fill}%
\end{pgfscope}%
\begin{pgfscope}%
\pgfpathrectangle{\pgfqpoint{3.116696in}{1.607558in}}{\pgfqpoint{0.824468in}{0.243158in}}%
\pgfusepath{clip}%
\pgfsetbuttcap%
\pgfsetmiterjoin%
\definecolor{currentfill}{rgb}{0.121569,0.466667,0.705882}%
\pgfsetfillcolor{currentfill}%
\pgfsetfillopacity{0.500000}%
\pgfsetlinewidth{1.003750pt}%
\definecolor{currentstroke}{rgb}{0.000000,0.000000,0.000000}%
\pgfsetstrokecolor{currentstroke}%
\pgfsetdash{}{0pt}%
\pgfpathmoveto{\pgfqpoint{3.154172in}{1.607558in}}%
\pgfpathlineto{\pgfqpoint{3.304075in}{1.607558in}}%
\pgfpathlineto{\pgfqpoint{3.304075in}{1.633090in}}%
\pgfpathlineto{\pgfqpoint{3.154172in}{1.633090in}}%
\pgfpathlineto{\pgfqpoint{3.154172in}{1.607558in}}%
\pgfpathclose%
\pgfusepath{stroke,fill}%
\end{pgfscope}%
\begin{pgfscope}%
\pgfpathrectangle{\pgfqpoint{3.116696in}{1.607558in}}{\pgfqpoint{0.824468in}{0.243158in}}%
\pgfusepath{clip}%
\pgfsetbuttcap%
\pgfsetmiterjoin%
\definecolor{currentfill}{rgb}{0.121569,0.466667,0.705882}%
\pgfsetfillcolor{currentfill}%
\pgfsetfillopacity{0.500000}%
\pgfsetlinewidth{1.003750pt}%
\definecolor{currentstroke}{rgb}{0.000000,0.000000,0.000000}%
\pgfsetstrokecolor{currentstroke}%
\pgfsetdash{}{0pt}%
\pgfpathmoveto{\pgfqpoint{3.304075in}{1.607558in}}%
\pgfpathlineto{\pgfqpoint{3.453979in}{1.607558in}}%
\pgfpathlineto{\pgfqpoint{3.453979in}{1.610963in}}%
\pgfpathlineto{\pgfqpoint{3.304075in}{1.610963in}}%
\pgfpathlineto{\pgfqpoint{3.304075in}{1.607558in}}%
\pgfpathclose%
\pgfusepath{stroke,fill}%
\end{pgfscope}%
\begin{pgfscope}%
\pgfpathrectangle{\pgfqpoint{3.116696in}{1.607558in}}{\pgfqpoint{0.824468in}{0.243158in}}%
\pgfusepath{clip}%
\pgfsetbuttcap%
\pgfsetmiterjoin%
\definecolor{currentfill}{rgb}{0.121569,0.466667,0.705882}%
\pgfsetfillcolor{currentfill}%
\pgfsetfillopacity{0.500000}%
\pgfsetlinewidth{1.003750pt}%
\definecolor{currentstroke}{rgb}{0.000000,0.000000,0.000000}%
\pgfsetstrokecolor{currentstroke}%
\pgfsetdash{}{0pt}%
\pgfpathmoveto{\pgfqpoint{3.453979in}{1.607558in}}%
\pgfpathlineto{\pgfqpoint{3.603882in}{1.607558in}}%
\pgfpathlineto{\pgfqpoint{3.603882in}{1.610476in}}%
\pgfpathlineto{\pgfqpoint{3.453979in}{1.610476in}}%
\pgfpathlineto{\pgfqpoint{3.453979in}{1.607558in}}%
\pgfpathclose%
\pgfusepath{stroke,fill}%
\end{pgfscope}%
\begin{pgfscope}%
\pgfpathrectangle{\pgfqpoint{3.116696in}{1.607558in}}{\pgfqpoint{0.824468in}{0.243158in}}%
\pgfusepath{clip}%
\pgfsetbuttcap%
\pgfsetmiterjoin%
\definecolor{currentfill}{rgb}{0.121569,0.466667,0.705882}%
\pgfsetfillcolor{currentfill}%
\pgfsetfillopacity{0.500000}%
\pgfsetlinewidth{1.003750pt}%
\definecolor{currentstroke}{rgb}{0.000000,0.000000,0.000000}%
\pgfsetstrokecolor{currentstroke}%
\pgfsetdash{}{0pt}%
\pgfpathmoveto{\pgfqpoint{3.603882in}{1.607558in}}%
\pgfpathlineto{\pgfqpoint{3.753785in}{1.607558in}}%
\pgfpathlineto{\pgfqpoint{3.753785in}{1.607923in}}%
\pgfpathlineto{\pgfqpoint{3.603882in}{1.607923in}}%
\pgfpathlineto{\pgfqpoint{3.603882in}{1.607558in}}%
\pgfpathclose%
\pgfusepath{stroke,fill}%
\end{pgfscope}%
\begin{pgfscope}%
\pgfpathrectangle{\pgfqpoint{3.116696in}{1.607558in}}{\pgfqpoint{0.824468in}{0.243158in}}%
\pgfusepath{clip}%
\pgfsetbuttcap%
\pgfsetmiterjoin%
\definecolor{currentfill}{rgb}{0.121569,0.466667,0.705882}%
\pgfsetfillcolor{currentfill}%
\pgfsetfillopacity{0.500000}%
\pgfsetlinewidth{1.003750pt}%
\definecolor{currentstroke}{rgb}{0.000000,0.000000,0.000000}%
\pgfsetstrokecolor{currentstroke}%
\pgfsetdash{}{0pt}%
\pgfpathmoveto{\pgfqpoint{3.753785in}{1.607558in}}%
\pgfpathlineto{\pgfqpoint{3.903688in}{1.607558in}}%
\pgfpathlineto{\pgfqpoint{3.903688in}{1.607558in}}%
\pgfpathlineto{\pgfqpoint{3.753785in}{1.607558in}}%
\pgfpathlineto{\pgfqpoint{3.753785in}{1.607558in}}%
\pgfpathclose%
\pgfusepath{stroke,fill}%
\end{pgfscope}%
\begin{pgfscope}%
\pgfsetrectcap%
\pgfsetmiterjoin%
\pgfsetlinewidth{0.803000pt}%
\definecolor{currentstroke}{rgb}{0.000000,0.000000,0.000000}%
\pgfsetstrokecolor{currentstroke}%
\pgfsetdash{}{0pt}%
\pgfpathmoveto{\pgfqpoint{3.116696in}{1.607558in}}%
\pgfpathlineto{\pgfqpoint{3.116696in}{1.850716in}}%
\pgfusepath{stroke}%
\end{pgfscope}%
\begin{pgfscope}%
\pgfsetrectcap%
\pgfsetmiterjoin%
\pgfsetlinewidth{0.803000pt}%
\definecolor{currentstroke}{rgb}{0.000000,0.000000,0.000000}%
\pgfsetstrokecolor{currentstroke}%
\pgfsetdash{}{0pt}%
\pgfpathmoveto{\pgfqpoint{3.941164in}{1.607558in}}%
\pgfpathlineto{\pgfqpoint{3.941164in}{1.850716in}}%
\pgfusepath{stroke}%
\end{pgfscope}%
\begin{pgfscope}%
\pgfsetrectcap%
\pgfsetmiterjoin%
\pgfsetlinewidth{0.803000pt}%
\definecolor{currentstroke}{rgb}{0.000000,0.000000,0.000000}%
\pgfsetstrokecolor{currentstroke}%
\pgfsetdash{}{0pt}%
\pgfpathmoveto{\pgfqpoint{3.116696in}{1.607558in}}%
\pgfpathlineto{\pgfqpoint{3.941164in}{1.607558in}}%
\pgfusepath{stroke}%
\end{pgfscope}%
\begin{pgfscope}%
\pgfsetrectcap%
\pgfsetmiterjoin%
\pgfsetlinewidth{0.803000pt}%
\definecolor{currentstroke}{rgb}{0.000000,0.000000,0.000000}%
\pgfsetstrokecolor{currentstroke}%
\pgfsetdash{}{0pt}%
\pgfpathmoveto{\pgfqpoint{3.116696in}{1.850716in}}%
\pgfpathlineto{\pgfqpoint{3.941164in}{1.850716in}}%
\pgfusepath{stroke}%
\end{pgfscope}%
\begin{pgfscope}%
\definecolor{textcolor}{rgb}{0.000000,0.000000,0.000000}%
\pgfsetstrokecolor{textcolor}%
\pgfsetfillcolor{textcolor}%
\pgftext[x=3.528930in,y=1.934050in,,base]{\color{textcolor}\rmfamily\fontsize{11.000000}{13.200000}\selectfont Cardif}%
\end{pgfscope}%
\begin{pgfscope}%
\pgfsetbuttcap%
\pgfsetmiterjoin%
\definecolor{currentfill}{rgb}{1.000000,1.000000,1.000000}%
\pgfsetfillcolor{currentfill}%
\pgfsetlinewidth{0.000000pt}%
\definecolor{currentstroke}{rgb}{0.000000,0.000000,0.000000}%
\pgfsetstrokecolor{currentstroke}%
\pgfsetstrokeopacity{0.000000}%
\pgfsetdash{}{0pt}%
\pgfpathmoveto{\pgfqpoint{4.106058in}{1.607558in}}%
\pgfpathlineto{\pgfqpoint{4.930526in}{1.607558in}}%
\pgfpathlineto{\pgfqpoint{4.930526in}{1.850716in}}%
\pgfpathlineto{\pgfqpoint{4.106058in}{1.850716in}}%
\pgfpathlineto{\pgfqpoint{4.106058in}{1.607558in}}%
\pgfpathclose%
\pgfusepath{fill}%
\end{pgfscope}%
\begin{pgfscope}%
\pgfpathrectangle{\pgfqpoint{4.106058in}{1.607558in}}{\pgfqpoint{0.824468in}{0.243158in}}%
\pgfusepath{clip}%
\pgfsetbuttcap%
\pgfsetmiterjoin%
\definecolor{currentfill}{rgb}{0.121569,0.466667,0.705882}%
\pgfsetfillcolor{currentfill}%
\pgfsetfillopacity{0.500000}%
\pgfsetlinewidth{1.003750pt}%
\definecolor{currentstroke}{rgb}{0.000000,0.000000,0.000000}%
\pgfsetstrokecolor{currentstroke}%
\pgfsetdash{}{0pt}%
\pgfpathmoveto{\pgfqpoint{4.143534in}{1.607558in}}%
\pgfpathlineto{\pgfqpoint{4.293437in}{1.607558in}}%
\pgfpathlineto{\pgfqpoint{4.293437in}{1.617893in}}%
\pgfpathlineto{\pgfqpoint{4.143534in}{1.617893in}}%
\pgfpathlineto{\pgfqpoint{4.143534in}{1.607558in}}%
\pgfpathclose%
\pgfusepath{stroke,fill}%
\end{pgfscope}%
\begin{pgfscope}%
\pgfpathrectangle{\pgfqpoint{4.106058in}{1.607558in}}{\pgfqpoint{0.824468in}{0.243158in}}%
\pgfusepath{clip}%
\pgfsetbuttcap%
\pgfsetmiterjoin%
\definecolor{currentfill}{rgb}{0.121569,0.466667,0.705882}%
\pgfsetfillcolor{currentfill}%
\pgfsetfillopacity{0.500000}%
\pgfsetlinewidth{1.003750pt}%
\definecolor{currentstroke}{rgb}{0.000000,0.000000,0.000000}%
\pgfsetstrokecolor{currentstroke}%
\pgfsetdash{}{0pt}%
\pgfpathmoveto{\pgfqpoint{4.293437in}{1.607558in}}%
\pgfpathlineto{\pgfqpoint{4.443340in}{1.607558in}}%
\pgfpathlineto{\pgfqpoint{4.443340in}{1.614488in}}%
\pgfpathlineto{\pgfqpoint{4.293437in}{1.614488in}}%
\pgfpathlineto{\pgfqpoint{4.293437in}{1.607558in}}%
\pgfpathclose%
\pgfusepath{stroke,fill}%
\end{pgfscope}%
\begin{pgfscope}%
\pgfpathrectangle{\pgfqpoint{4.106058in}{1.607558in}}{\pgfqpoint{0.824468in}{0.243158in}}%
\pgfusepath{clip}%
\pgfsetbuttcap%
\pgfsetmiterjoin%
\definecolor{currentfill}{rgb}{0.121569,0.466667,0.705882}%
\pgfsetfillcolor{currentfill}%
\pgfsetfillopacity{0.500000}%
\pgfsetlinewidth{1.003750pt}%
\definecolor{currentstroke}{rgb}{0.000000,0.000000,0.000000}%
\pgfsetstrokecolor{currentstroke}%
\pgfsetdash{}{0pt}%
\pgfpathmoveto{\pgfqpoint{4.443340in}{1.607558in}}%
\pgfpathlineto{\pgfqpoint{4.593244in}{1.607558in}}%
\pgfpathlineto{\pgfqpoint{4.593244in}{1.624580in}}%
\pgfpathlineto{\pgfqpoint{4.443340in}{1.624580in}}%
\pgfpathlineto{\pgfqpoint{4.443340in}{1.607558in}}%
\pgfpathclose%
\pgfusepath{stroke,fill}%
\end{pgfscope}%
\begin{pgfscope}%
\pgfpathrectangle{\pgfqpoint{4.106058in}{1.607558in}}{\pgfqpoint{0.824468in}{0.243158in}}%
\pgfusepath{clip}%
\pgfsetbuttcap%
\pgfsetmiterjoin%
\definecolor{currentfill}{rgb}{0.121569,0.466667,0.705882}%
\pgfsetfillcolor{currentfill}%
\pgfsetfillopacity{0.500000}%
\pgfsetlinewidth{1.003750pt}%
\definecolor{currentstroke}{rgb}{0.000000,0.000000,0.000000}%
\pgfsetstrokecolor{currentstroke}%
\pgfsetdash{}{0pt}%
\pgfpathmoveto{\pgfqpoint{4.593244in}{1.607558in}}%
\pgfpathlineto{\pgfqpoint{4.743147in}{1.607558in}}%
\pgfpathlineto{\pgfqpoint{4.743147in}{1.633941in}}%
\pgfpathlineto{\pgfqpoint{4.593244in}{1.633941in}}%
\pgfpathlineto{\pgfqpoint{4.593244in}{1.607558in}}%
\pgfpathclose%
\pgfusepath{stroke,fill}%
\end{pgfscope}%
\begin{pgfscope}%
\pgfpathrectangle{\pgfqpoint{4.106058in}{1.607558in}}{\pgfqpoint{0.824468in}{0.243158in}}%
\pgfusepath{clip}%
\pgfsetbuttcap%
\pgfsetmiterjoin%
\definecolor{currentfill}{rgb}{0.121569,0.466667,0.705882}%
\pgfsetfillcolor{currentfill}%
\pgfsetfillopacity{0.500000}%
\pgfsetlinewidth{1.003750pt}%
\definecolor{currentstroke}{rgb}{0.000000,0.000000,0.000000}%
\pgfsetstrokecolor{currentstroke}%
\pgfsetdash{}{0pt}%
\pgfpathmoveto{\pgfqpoint{4.743147in}{1.607558in}}%
\pgfpathlineto{\pgfqpoint{4.893050in}{1.607558in}}%
\pgfpathlineto{\pgfqpoint{4.893050in}{1.626890in}}%
\pgfpathlineto{\pgfqpoint{4.743147in}{1.626890in}}%
\pgfpathlineto{\pgfqpoint{4.743147in}{1.607558in}}%
\pgfpathclose%
\pgfusepath{stroke,fill}%
\end{pgfscope}%
\begin{pgfscope}%
\pgfsetrectcap%
\pgfsetmiterjoin%
\pgfsetlinewidth{0.803000pt}%
\definecolor{currentstroke}{rgb}{0.000000,0.000000,0.000000}%
\pgfsetstrokecolor{currentstroke}%
\pgfsetdash{}{0pt}%
\pgfpathmoveto{\pgfqpoint{4.106058in}{1.607558in}}%
\pgfpathlineto{\pgfqpoint{4.106058in}{1.850716in}}%
\pgfusepath{stroke}%
\end{pgfscope}%
\begin{pgfscope}%
\pgfsetrectcap%
\pgfsetmiterjoin%
\pgfsetlinewidth{0.803000pt}%
\definecolor{currentstroke}{rgb}{0.000000,0.000000,0.000000}%
\pgfsetstrokecolor{currentstroke}%
\pgfsetdash{}{0pt}%
\pgfpathmoveto{\pgfqpoint{4.930526in}{1.607558in}}%
\pgfpathlineto{\pgfqpoint{4.930526in}{1.850716in}}%
\pgfusepath{stroke}%
\end{pgfscope}%
\begin{pgfscope}%
\pgfsetrectcap%
\pgfsetmiterjoin%
\pgfsetlinewidth{0.803000pt}%
\definecolor{currentstroke}{rgb}{0.000000,0.000000,0.000000}%
\pgfsetstrokecolor{currentstroke}%
\pgfsetdash{}{0pt}%
\pgfpathmoveto{\pgfqpoint{4.106058in}{1.607558in}}%
\pgfpathlineto{\pgfqpoint{4.930526in}{1.607558in}}%
\pgfusepath{stroke}%
\end{pgfscope}%
\begin{pgfscope}%
\pgfsetrectcap%
\pgfsetmiterjoin%
\pgfsetlinewidth{0.803000pt}%
\definecolor{currentstroke}{rgb}{0.000000,0.000000,0.000000}%
\pgfsetstrokecolor{currentstroke}%
\pgfsetdash{}{0pt}%
\pgfpathmoveto{\pgfqpoint{4.106058in}{1.850716in}}%
\pgfpathlineto{\pgfqpoint{4.930526in}{1.850716in}}%
\pgfusepath{stroke}%
\end{pgfscope}%
\begin{pgfscope}%
\definecolor{textcolor}{rgb}{0.000000,0.000000,0.000000}%
\pgfsetstrokecolor{textcolor}%
\pgfsetfillcolor{textcolor}%
\pgftext[x=4.518292in,y=1.934050in,,base]{\color{textcolor}\rmfamily\fontsize{11.000000}{13.200000}\selectfont Santiane}%
\end{pgfscope}%
\begin{pgfscope}%
\pgfsetbuttcap%
\pgfsetmiterjoin%
\definecolor{currentfill}{rgb}{1.000000,1.000000,1.000000}%
\pgfsetfillcolor{currentfill}%
\pgfsetlinewidth{0.000000pt}%
\definecolor{currentstroke}{rgb}{0.000000,0.000000,0.000000}%
\pgfsetstrokecolor{currentstroke}%
\pgfsetstrokeopacity{0.000000}%
\pgfsetdash{}{0pt}%
\pgfpathmoveto{\pgfqpoint{5.095420in}{1.607558in}}%
\pgfpathlineto{\pgfqpoint{5.919888in}{1.607558in}}%
\pgfpathlineto{\pgfqpoint{5.919888in}{1.850716in}}%
\pgfpathlineto{\pgfqpoint{5.095420in}{1.850716in}}%
\pgfpathlineto{\pgfqpoint{5.095420in}{1.607558in}}%
\pgfpathclose%
\pgfusepath{fill}%
\end{pgfscope}%
\begin{pgfscope}%
\pgfpathrectangle{\pgfqpoint{5.095420in}{1.607558in}}{\pgfqpoint{0.824468in}{0.243158in}}%
\pgfusepath{clip}%
\pgfsetbuttcap%
\pgfsetmiterjoin%
\definecolor{currentfill}{rgb}{0.121569,0.466667,0.705882}%
\pgfsetfillcolor{currentfill}%
\pgfsetfillopacity{0.500000}%
\pgfsetlinewidth{1.003750pt}%
\definecolor{currentstroke}{rgb}{0.000000,0.000000,0.000000}%
\pgfsetstrokecolor{currentstroke}%
\pgfsetdash{}{0pt}%
\pgfpathmoveto{\pgfqpoint{5.132895in}{1.607558in}}%
\pgfpathlineto{\pgfqpoint{5.282799in}{1.607558in}}%
\pgfpathlineto{\pgfqpoint{5.282799in}{1.618136in}}%
\pgfpathlineto{\pgfqpoint{5.132895in}{1.618136in}}%
\pgfpathlineto{\pgfqpoint{5.132895in}{1.607558in}}%
\pgfpathclose%
\pgfusepath{stroke,fill}%
\end{pgfscope}%
\begin{pgfscope}%
\pgfpathrectangle{\pgfqpoint{5.095420in}{1.607558in}}{\pgfqpoint{0.824468in}{0.243158in}}%
\pgfusepath{clip}%
\pgfsetbuttcap%
\pgfsetmiterjoin%
\definecolor{currentfill}{rgb}{0.121569,0.466667,0.705882}%
\pgfsetfillcolor{currentfill}%
\pgfsetfillopacity{0.500000}%
\pgfsetlinewidth{1.003750pt}%
\definecolor{currentstroke}{rgb}{0.000000,0.000000,0.000000}%
\pgfsetstrokecolor{currentstroke}%
\pgfsetdash{}{0pt}%
\pgfpathmoveto{\pgfqpoint{5.282799in}{1.607558in}}%
\pgfpathlineto{\pgfqpoint{5.432702in}{1.607558in}}%
\pgfpathlineto{\pgfqpoint{5.432702in}{1.610598in}}%
\pgfpathlineto{\pgfqpoint{5.282799in}{1.610598in}}%
\pgfpathlineto{\pgfqpoint{5.282799in}{1.607558in}}%
\pgfpathclose%
\pgfusepath{stroke,fill}%
\end{pgfscope}%
\begin{pgfscope}%
\pgfpathrectangle{\pgfqpoint{5.095420in}{1.607558in}}{\pgfqpoint{0.824468in}{0.243158in}}%
\pgfusepath{clip}%
\pgfsetbuttcap%
\pgfsetmiterjoin%
\definecolor{currentfill}{rgb}{0.121569,0.466667,0.705882}%
\pgfsetfillcolor{currentfill}%
\pgfsetfillopacity{0.500000}%
\pgfsetlinewidth{1.003750pt}%
\definecolor{currentstroke}{rgb}{0.000000,0.000000,0.000000}%
\pgfsetstrokecolor{currentstroke}%
\pgfsetdash{}{0pt}%
\pgfpathmoveto{\pgfqpoint{5.432702in}{1.607558in}}%
\pgfpathlineto{\pgfqpoint{5.582605in}{1.607558in}}%
\pgfpathlineto{\pgfqpoint{5.582605in}{1.608774in}}%
\pgfpathlineto{\pgfqpoint{5.432702in}{1.608774in}}%
\pgfpathlineto{\pgfqpoint{5.432702in}{1.607558in}}%
\pgfpathclose%
\pgfusepath{stroke,fill}%
\end{pgfscope}%
\begin{pgfscope}%
\pgfpathrectangle{\pgfqpoint{5.095420in}{1.607558in}}{\pgfqpoint{0.824468in}{0.243158in}}%
\pgfusepath{clip}%
\pgfsetbuttcap%
\pgfsetmiterjoin%
\definecolor{currentfill}{rgb}{0.121569,0.466667,0.705882}%
\pgfsetfillcolor{currentfill}%
\pgfsetfillopacity{0.500000}%
\pgfsetlinewidth{1.003750pt}%
\definecolor{currentstroke}{rgb}{0.000000,0.000000,0.000000}%
\pgfsetstrokecolor{currentstroke}%
\pgfsetdash{}{0pt}%
\pgfpathmoveto{\pgfqpoint{5.582605in}{1.607558in}}%
\pgfpathlineto{\pgfqpoint{5.732509in}{1.607558in}}%
\pgfpathlineto{\pgfqpoint{5.732509in}{1.607802in}}%
\pgfpathlineto{\pgfqpoint{5.582605in}{1.607802in}}%
\pgfpathlineto{\pgfqpoint{5.582605in}{1.607558in}}%
\pgfpathclose%
\pgfusepath{stroke,fill}%
\end{pgfscope}%
\begin{pgfscope}%
\pgfpathrectangle{\pgfqpoint{5.095420in}{1.607558in}}{\pgfqpoint{0.824468in}{0.243158in}}%
\pgfusepath{clip}%
\pgfsetbuttcap%
\pgfsetmiterjoin%
\definecolor{currentfill}{rgb}{0.121569,0.466667,0.705882}%
\pgfsetfillcolor{currentfill}%
\pgfsetfillopacity{0.500000}%
\pgfsetlinewidth{1.003750pt}%
\definecolor{currentstroke}{rgb}{0.000000,0.000000,0.000000}%
\pgfsetstrokecolor{currentstroke}%
\pgfsetdash{}{0pt}%
\pgfpathmoveto{\pgfqpoint{5.732509in}{1.607558in}}%
\pgfpathlineto{\pgfqpoint{5.882412in}{1.607558in}}%
\pgfpathlineto{\pgfqpoint{5.882412in}{1.608531in}}%
\pgfpathlineto{\pgfqpoint{5.732509in}{1.608531in}}%
\pgfpathlineto{\pgfqpoint{5.732509in}{1.607558in}}%
\pgfpathclose%
\pgfusepath{stroke,fill}%
\end{pgfscope}%
\begin{pgfscope}%
\pgfsetrectcap%
\pgfsetmiterjoin%
\pgfsetlinewidth{0.803000pt}%
\definecolor{currentstroke}{rgb}{0.000000,0.000000,0.000000}%
\pgfsetstrokecolor{currentstroke}%
\pgfsetdash{}{0pt}%
\pgfpathmoveto{\pgfqpoint{5.095420in}{1.607558in}}%
\pgfpathlineto{\pgfqpoint{5.095420in}{1.850716in}}%
\pgfusepath{stroke}%
\end{pgfscope}%
\begin{pgfscope}%
\pgfsetrectcap%
\pgfsetmiterjoin%
\pgfsetlinewidth{0.803000pt}%
\definecolor{currentstroke}{rgb}{0.000000,0.000000,0.000000}%
\pgfsetstrokecolor{currentstroke}%
\pgfsetdash{}{0pt}%
\pgfpathmoveto{\pgfqpoint{5.919888in}{1.607558in}}%
\pgfpathlineto{\pgfqpoint{5.919888in}{1.850716in}}%
\pgfusepath{stroke}%
\end{pgfscope}%
\begin{pgfscope}%
\pgfsetrectcap%
\pgfsetmiterjoin%
\pgfsetlinewidth{0.803000pt}%
\definecolor{currentstroke}{rgb}{0.000000,0.000000,0.000000}%
\pgfsetstrokecolor{currentstroke}%
\pgfsetdash{}{0pt}%
\pgfpathmoveto{\pgfqpoint{5.095420in}{1.607558in}}%
\pgfpathlineto{\pgfqpoint{5.919888in}{1.607558in}}%
\pgfusepath{stroke}%
\end{pgfscope}%
\begin{pgfscope}%
\pgfsetrectcap%
\pgfsetmiterjoin%
\pgfsetlinewidth{0.803000pt}%
\definecolor{currentstroke}{rgb}{0.000000,0.000000,0.000000}%
\pgfsetstrokecolor{currentstroke}%
\pgfsetdash{}{0pt}%
\pgfpathmoveto{\pgfqpoint{5.095420in}{1.850716in}}%
\pgfpathlineto{\pgfqpoint{5.919888in}{1.850716in}}%
\pgfusepath{stroke}%
\end{pgfscope}%
\begin{pgfscope}%
\definecolor{textcolor}{rgb}{0.000000,0.000000,0.000000}%
\pgfsetstrokecolor{textcolor}%
\pgfsetfillcolor{textcolor}%
\pgftext[x=5.507654in,y=1.934050in,,base]{\color{textcolor}\rmfamily\fontsize{11.000000}{13.200000}\selectfont Eca As...}%
\end{pgfscope}%
\begin{pgfscope}%
\pgfsetbuttcap%
\pgfsetmiterjoin%
\definecolor{currentfill}{rgb}{1.000000,1.000000,1.000000}%
\pgfsetfillcolor{currentfill}%
\pgfsetlinewidth{0.000000pt}%
\definecolor{currentstroke}{rgb}{0.000000,0.000000,0.000000}%
\pgfsetstrokecolor{currentstroke}%
\pgfsetstrokeopacity{0.000000}%
\pgfsetdash{}{0pt}%
\pgfpathmoveto{\pgfqpoint{6.084781in}{1.607558in}}%
\pgfpathlineto{\pgfqpoint{6.909249in}{1.607558in}}%
\pgfpathlineto{\pgfqpoint{6.909249in}{1.850716in}}%
\pgfpathlineto{\pgfqpoint{6.084781in}{1.850716in}}%
\pgfpathlineto{\pgfqpoint{6.084781in}{1.607558in}}%
\pgfpathclose%
\pgfusepath{fill}%
\end{pgfscope}%
\begin{pgfscope}%
\pgfpathrectangle{\pgfqpoint{6.084781in}{1.607558in}}{\pgfqpoint{0.824468in}{0.243158in}}%
\pgfusepath{clip}%
\pgfsetbuttcap%
\pgfsetmiterjoin%
\definecolor{currentfill}{rgb}{0.121569,0.466667,0.705882}%
\pgfsetfillcolor{currentfill}%
\pgfsetfillopacity{0.500000}%
\pgfsetlinewidth{1.003750pt}%
\definecolor{currentstroke}{rgb}{0.000000,0.000000,0.000000}%
\pgfsetstrokecolor{currentstroke}%
\pgfsetdash{}{0pt}%
\pgfpathmoveto{\pgfqpoint{6.122257in}{1.607558in}}%
\pgfpathlineto{\pgfqpoint{6.272160in}{1.607558in}}%
\pgfpathlineto{\pgfqpoint{6.272160in}{1.614245in}}%
\pgfpathlineto{\pgfqpoint{6.122257in}{1.614245in}}%
\pgfpathlineto{\pgfqpoint{6.122257in}{1.607558in}}%
\pgfpathclose%
\pgfusepath{stroke,fill}%
\end{pgfscope}%
\begin{pgfscope}%
\pgfpathrectangle{\pgfqpoint{6.084781in}{1.607558in}}{\pgfqpoint{0.824468in}{0.243158in}}%
\pgfusepath{clip}%
\pgfsetbuttcap%
\pgfsetmiterjoin%
\definecolor{currentfill}{rgb}{0.121569,0.466667,0.705882}%
\pgfsetfillcolor{currentfill}%
\pgfsetfillopacity{0.500000}%
\pgfsetlinewidth{1.003750pt}%
\definecolor{currentstroke}{rgb}{0.000000,0.000000,0.000000}%
\pgfsetstrokecolor{currentstroke}%
\pgfsetdash{}{0pt}%
\pgfpathmoveto{\pgfqpoint{6.272160in}{1.607558in}}%
\pgfpathlineto{\pgfqpoint{6.422064in}{1.607558in}}%
\pgfpathlineto{\pgfqpoint{6.422064in}{1.610720in}}%
\pgfpathlineto{\pgfqpoint{6.272160in}{1.610720in}}%
\pgfpathlineto{\pgfqpoint{6.272160in}{1.607558in}}%
\pgfpathclose%
\pgfusepath{stroke,fill}%
\end{pgfscope}%
\begin{pgfscope}%
\pgfpathrectangle{\pgfqpoint{6.084781in}{1.607558in}}{\pgfqpoint{0.824468in}{0.243158in}}%
\pgfusepath{clip}%
\pgfsetbuttcap%
\pgfsetmiterjoin%
\definecolor{currentfill}{rgb}{0.121569,0.466667,0.705882}%
\pgfsetfillcolor{currentfill}%
\pgfsetfillopacity{0.500000}%
\pgfsetlinewidth{1.003750pt}%
\definecolor{currentstroke}{rgb}{0.000000,0.000000,0.000000}%
\pgfsetstrokecolor{currentstroke}%
\pgfsetdash{}{0pt}%
\pgfpathmoveto{\pgfqpoint{6.422064in}{1.607558in}}%
\pgfpathlineto{\pgfqpoint{6.571967in}{1.607558in}}%
\pgfpathlineto{\pgfqpoint{6.571967in}{1.608166in}}%
\pgfpathlineto{\pgfqpoint{6.422064in}{1.608166in}}%
\pgfpathlineto{\pgfqpoint{6.422064in}{1.607558in}}%
\pgfpathclose%
\pgfusepath{stroke,fill}%
\end{pgfscope}%
\begin{pgfscope}%
\pgfpathrectangle{\pgfqpoint{6.084781in}{1.607558in}}{\pgfqpoint{0.824468in}{0.243158in}}%
\pgfusepath{clip}%
\pgfsetbuttcap%
\pgfsetmiterjoin%
\definecolor{currentfill}{rgb}{0.121569,0.466667,0.705882}%
\pgfsetfillcolor{currentfill}%
\pgfsetfillopacity{0.500000}%
\pgfsetlinewidth{1.003750pt}%
\definecolor{currentstroke}{rgb}{0.000000,0.000000,0.000000}%
\pgfsetstrokecolor{currentstroke}%
\pgfsetdash{}{0pt}%
\pgfpathmoveto{\pgfqpoint{6.571967in}{1.607558in}}%
\pgfpathlineto{\pgfqpoint{6.721870in}{1.607558in}}%
\pgfpathlineto{\pgfqpoint{6.721870in}{1.608288in}}%
\pgfpathlineto{\pgfqpoint{6.571967in}{1.608288in}}%
\pgfpathlineto{\pgfqpoint{6.571967in}{1.607558in}}%
\pgfpathclose%
\pgfusepath{stroke,fill}%
\end{pgfscope}%
\begin{pgfscope}%
\pgfpathrectangle{\pgfqpoint{6.084781in}{1.607558in}}{\pgfqpoint{0.824468in}{0.243158in}}%
\pgfusepath{clip}%
\pgfsetbuttcap%
\pgfsetmiterjoin%
\definecolor{currentfill}{rgb}{0.121569,0.466667,0.705882}%
\pgfsetfillcolor{currentfill}%
\pgfsetfillopacity{0.500000}%
\pgfsetlinewidth{1.003750pt}%
\definecolor{currentstroke}{rgb}{0.000000,0.000000,0.000000}%
\pgfsetstrokecolor{currentstroke}%
\pgfsetdash{}{0pt}%
\pgfpathmoveto{\pgfqpoint{6.721870in}{1.607558in}}%
\pgfpathlineto{\pgfqpoint{6.871774in}{1.607558in}}%
\pgfpathlineto{\pgfqpoint{6.871774in}{1.608166in}}%
\pgfpathlineto{\pgfqpoint{6.721870in}{1.608166in}}%
\pgfpathlineto{\pgfqpoint{6.721870in}{1.607558in}}%
\pgfpathclose%
\pgfusepath{stroke,fill}%
\end{pgfscope}%
\begin{pgfscope}%
\pgfsetrectcap%
\pgfsetmiterjoin%
\pgfsetlinewidth{0.803000pt}%
\definecolor{currentstroke}{rgb}{0.000000,0.000000,0.000000}%
\pgfsetstrokecolor{currentstroke}%
\pgfsetdash{}{0pt}%
\pgfpathmoveto{\pgfqpoint{6.084781in}{1.607558in}}%
\pgfpathlineto{\pgfqpoint{6.084781in}{1.850716in}}%
\pgfusepath{stroke}%
\end{pgfscope}%
\begin{pgfscope}%
\pgfsetrectcap%
\pgfsetmiterjoin%
\pgfsetlinewidth{0.803000pt}%
\definecolor{currentstroke}{rgb}{0.000000,0.000000,0.000000}%
\pgfsetstrokecolor{currentstroke}%
\pgfsetdash{}{0pt}%
\pgfpathmoveto{\pgfqpoint{6.909249in}{1.607558in}}%
\pgfpathlineto{\pgfqpoint{6.909249in}{1.850716in}}%
\pgfusepath{stroke}%
\end{pgfscope}%
\begin{pgfscope}%
\pgfsetrectcap%
\pgfsetmiterjoin%
\pgfsetlinewidth{0.803000pt}%
\definecolor{currentstroke}{rgb}{0.000000,0.000000,0.000000}%
\pgfsetstrokecolor{currentstroke}%
\pgfsetdash{}{0pt}%
\pgfpathmoveto{\pgfqpoint{6.084781in}{1.607558in}}%
\pgfpathlineto{\pgfqpoint{6.909249in}{1.607558in}}%
\pgfusepath{stroke}%
\end{pgfscope}%
\begin{pgfscope}%
\pgfsetrectcap%
\pgfsetmiterjoin%
\pgfsetlinewidth{0.803000pt}%
\definecolor{currentstroke}{rgb}{0.000000,0.000000,0.000000}%
\pgfsetstrokecolor{currentstroke}%
\pgfsetdash{}{0pt}%
\pgfpathmoveto{\pgfqpoint{6.084781in}{1.850716in}}%
\pgfpathlineto{\pgfqpoint{6.909249in}{1.850716in}}%
\pgfusepath{stroke}%
\end{pgfscope}%
\begin{pgfscope}%
\definecolor{textcolor}{rgb}{0.000000,0.000000,0.000000}%
\pgfsetstrokecolor{textcolor}%
\pgfsetfillcolor{textcolor}%
\pgftext[x=6.497015in,y=1.934050in,,base]{\color{textcolor}\rmfamily\fontsize{11.000000}{13.200000}\selectfont Groupama}%
\end{pgfscope}%
\begin{pgfscope}%
\pgfsetbuttcap%
\pgfsetmiterjoin%
\definecolor{currentfill}{rgb}{1.000000,1.000000,1.000000}%
\pgfsetfillcolor{currentfill}%
\pgfsetlinewidth{0.000000pt}%
\definecolor{currentstroke}{rgb}{0.000000,0.000000,0.000000}%
\pgfsetstrokecolor{currentstroke}%
\pgfsetstrokeopacity{0.000000}%
\pgfsetdash{}{0pt}%
\pgfpathmoveto{\pgfqpoint{7.074143in}{1.607558in}}%
\pgfpathlineto{\pgfqpoint{7.898611in}{1.607558in}}%
\pgfpathlineto{\pgfqpoint{7.898611in}{1.850716in}}%
\pgfpathlineto{\pgfqpoint{7.074143in}{1.850716in}}%
\pgfpathlineto{\pgfqpoint{7.074143in}{1.607558in}}%
\pgfpathclose%
\pgfusepath{fill}%
\end{pgfscope}%
\begin{pgfscope}%
\pgfpathrectangle{\pgfqpoint{7.074143in}{1.607558in}}{\pgfqpoint{0.824468in}{0.243158in}}%
\pgfusepath{clip}%
\pgfsetbuttcap%
\pgfsetmiterjoin%
\definecolor{currentfill}{rgb}{0.121569,0.466667,0.705882}%
\pgfsetfillcolor{currentfill}%
\pgfsetfillopacity{0.500000}%
\pgfsetlinewidth{1.003750pt}%
\definecolor{currentstroke}{rgb}{0.000000,0.000000,0.000000}%
\pgfsetstrokecolor{currentstroke}%
\pgfsetdash{}{0pt}%
\pgfpathmoveto{\pgfqpoint{7.111619in}{1.607558in}}%
\pgfpathlineto{\pgfqpoint{7.261522in}{1.607558in}}%
\pgfpathlineto{\pgfqpoint{7.261522in}{1.614488in}}%
\pgfpathlineto{\pgfqpoint{7.111619in}{1.614488in}}%
\pgfpathlineto{\pgfqpoint{7.111619in}{1.607558in}}%
\pgfpathclose%
\pgfusepath{stroke,fill}%
\end{pgfscope}%
\begin{pgfscope}%
\pgfpathrectangle{\pgfqpoint{7.074143in}{1.607558in}}{\pgfqpoint{0.824468in}{0.243158in}}%
\pgfusepath{clip}%
\pgfsetbuttcap%
\pgfsetmiterjoin%
\definecolor{currentfill}{rgb}{0.121569,0.466667,0.705882}%
\pgfsetfillcolor{currentfill}%
\pgfsetfillopacity{0.500000}%
\pgfsetlinewidth{1.003750pt}%
\definecolor{currentstroke}{rgb}{0.000000,0.000000,0.000000}%
\pgfsetstrokecolor{currentstroke}%
\pgfsetdash{}{0pt}%
\pgfpathmoveto{\pgfqpoint{7.261522in}{1.607558in}}%
\pgfpathlineto{\pgfqpoint{7.411425in}{1.607558in}}%
\pgfpathlineto{\pgfqpoint{7.411425in}{1.610720in}}%
\pgfpathlineto{\pgfqpoint{7.261522in}{1.610720in}}%
\pgfpathlineto{\pgfqpoint{7.261522in}{1.607558in}}%
\pgfpathclose%
\pgfusepath{stroke,fill}%
\end{pgfscope}%
\begin{pgfscope}%
\pgfpathrectangle{\pgfqpoint{7.074143in}{1.607558in}}{\pgfqpoint{0.824468in}{0.243158in}}%
\pgfusepath{clip}%
\pgfsetbuttcap%
\pgfsetmiterjoin%
\definecolor{currentfill}{rgb}{0.121569,0.466667,0.705882}%
\pgfsetfillcolor{currentfill}%
\pgfsetfillopacity{0.500000}%
\pgfsetlinewidth{1.003750pt}%
\definecolor{currentstroke}{rgb}{0.000000,0.000000,0.000000}%
\pgfsetstrokecolor{currentstroke}%
\pgfsetdash{}{0pt}%
\pgfpathmoveto{\pgfqpoint{7.411425in}{1.607558in}}%
\pgfpathlineto{\pgfqpoint{7.561329in}{1.607558in}}%
\pgfpathlineto{\pgfqpoint{7.561329in}{1.609382in}}%
\pgfpathlineto{\pgfqpoint{7.411425in}{1.609382in}}%
\pgfpathlineto{\pgfqpoint{7.411425in}{1.607558in}}%
\pgfpathclose%
\pgfusepath{stroke,fill}%
\end{pgfscope}%
\begin{pgfscope}%
\pgfpathrectangle{\pgfqpoint{7.074143in}{1.607558in}}{\pgfqpoint{0.824468in}{0.243158in}}%
\pgfusepath{clip}%
\pgfsetbuttcap%
\pgfsetmiterjoin%
\definecolor{currentfill}{rgb}{0.121569,0.466667,0.705882}%
\pgfsetfillcolor{currentfill}%
\pgfsetfillopacity{0.500000}%
\pgfsetlinewidth{1.003750pt}%
\definecolor{currentstroke}{rgb}{0.000000,0.000000,0.000000}%
\pgfsetstrokecolor{currentstroke}%
\pgfsetdash{}{0pt}%
\pgfpathmoveto{\pgfqpoint{7.561329in}{1.607558in}}%
\pgfpathlineto{\pgfqpoint{7.711232in}{1.607558in}}%
\pgfpathlineto{\pgfqpoint{7.711232in}{1.608166in}}%
\pgfpathlineto{\pgfqpoint{7.561329in}{1.608166in}}%
\pgfpathlineto{\pgfqpoint{7.561329in}{1.607558in}}%
\pgfpathclose%
\pgfusepath{stroke,fill}%
\end{pgfscope}%
\begin{pgfscope}%
\pgfpathrectangle{\pgfqpoint{7.074143in}{1.607558in}}{\pgfqpoint{0.824468in}{0.243158in}}%
\pgfusepath{clip}%
\pgfsetbuttcap%
\pgfsetmiterjoin%
\definecolor{currentfill}{rgb}{0.121569,0.466667,0.705882}%
\pgfsetfillcolor{currentfill}%
\pgfsetfillopacity{0.500000}%
\pgfsetlinewidth{1.003750pt}%
\definecolor{currentstroke}{rgb}{0.000000,0.000000,0.000000}%
\pgfsetstrokecolor{currentstroke}%
\pgfsetdash{}{0pt}%
\pgfpathmoveto{\pgfqpoint{7.711232in}{1.607558in}}%
\pgfpathlineto{\pgfqpoint{7.861135in}{1.607558in}}%
\pgfpathlineto{\pgfqpoint{7.861135in}{1.608410in}}%
\pgfpathlineto{\pgfqpoint{7.711232in}{1.608410in}}%
\pgfpathlineto{\pgfqpoint{7.711232in}{1.607558in}}%
\pgfpathclose%
\pgfusepath{stroke,fill}%
\end{pgfscope}%
\begin{pgfscope}%
\pgfsetrectcap%
\pgfsetmiterjoin%
\pgfsetlinewidth{0.803000pt}%
\definecolor{currentstroke}{rgb}{0.000000,0.000000,0.000000}%
\pgfsetstrokecolor{currentstroke}%
\pgfsetdash{}{0pt}%
\pgfpathmoveto{\pgfqpoint{7.074143in}{1.607558in}}%
\pgfpathlineto{\pgfqpoint{7.074143in}{1.850716in}}%
\pgfusepath{stroke}%
\end{pgfscope}%
\begin{pgfscope}%
\pgfsetrectcap%
\pgfsetmiterjoin%
\pgfsetlinewidth{0.803000pt}%
\definecolor{currentstroke}{rgb}{0.000000,0.000000,0.000000}%
\pgfsetstrokecolor{currentstroke}%
\pgfsetdash{}{0pt}%
\pgfpathmoveto{\pgfqpoint{7.898611in}{1.607558in}}%
\pgfpathlineto{\pgfqpoint{7.898611in}{1.850716in}}%
\pgfusepath{stroke}%
\end{pgfscope}%
\begin{pgfscope}%
\pgfsetrectcap%
\pgfsetmiterjoin%
\pgfsetlinewidth{0.803000pt}%
\definecolor{currentstroke}{rgb}{0.000000,0.000000,0.000000}%
\pgfsetstrokecolor{currentstroke}%
\pgfsetdash{}{0pt}%
\pgfpathmoveto{\pgfqpoint{7.074143in}{1.607558in}}%
\pgfpathlineto{\pgfqpoint{7.898611in}{1.607558in}}%
\pgfusepath{stroke}%
\end{pgfscope}%
\begin{pgfscope}%
\pgfsetrectcap%
\pgfsetmiterjoin%
\pgfsetlinewidth{0.803000pt}%
\definecolor{currentstroke}{rgb}{0.000000,0.000000,0.000000}%
\pgfsetstrokecolor{currentstroke}%
\pgfsetdash{}{0pt}%
\pgfpathmoveto{\pgfqpoint{7.074143in}{1.850716in}}%
\pgfpathlineto{\pgfqpoint{7.898611in}{1.850716in}}%
\pgfusepath{stroke}%
\end{pgfscope}%
\begin{pgfscope}%
\definecolor{textcolor}{rgb}{0.000000,0.000000,0.000000}%
\pgfsetstrokecolor{textcolor}%
\pgfsetfillcolor{textcolor}%
\pgftext[x=7.486377in,y=1.934050in,,base]{\color{textcolor}\rmfamily\fontsize{11.000000}{13.200000}\selectfont Assur ...}%
\end{pgfscope}%
\begin{pgfscope}%
\pgfsetbuttcap%
\pgfsetmiterjoin%
\definecolor{currentfill}{rgb}{1.000000,1.000000,1.000000}%
\pgfsetfillcolor{currentfill}%
\pgfsetlinewidth{0.000000pt}%
\definecolor{currentstroke}{rgb}{0.000000,0.000000,0.000000}%
\pgfsetstrokecolor{currentstroke}%
\pgfsetstrokeopacity{0.000000}%
\pgfsetdash{}{0pt}%
\pgfpathmoveto{\pgfqpoint{0.148611in}{0.878085in}}%
\pgfpathlineto{\pgfqpoint{0.973079in}{0.878085in}}%
\pgfpathlineto{\pgfqpoint{0.973079in}{1.121243in}}%
\pgfpathlineto{\pgfqpoint{0.148611in}{1.121243in}}%
\pgfpathlineto{\pgfqpoint{0.148611in}{0.878085in}}%
\pgfpathclose%
\pgfusepath{fill}%
\end{pgfscope}%
\begin{pgfscope}%
\pgfpathrectangle{\pgfqpoint{0.148611in}{0.878085in}}{\pgfqpoint{0.824468in}{0.243158in}}%
\pgfusepath{clip}%
\pgfsetbuttcap%
\pgfsetmiterjoin%
\definecolor{currentfill}{rgb}{0.121569,0.466667,0.705882}%
\pgfsetfillcolor{currentfill}%
\pgfsetfillopacity{0.500000}%
\pgfsetlinewidth{1.003750pt}%
\definecolor{currentstroke}{rgb}{0.000000,0.000000,0.000000}%
\pgfsetstrokecolor{currentstroke}%
\pgfsetdash{}{0pt}%
\pgfpathmoveto{\pgfqpoint{0.186087in}{0.878085in}}%
\pgfpathlineto{\pgfqpoint{0.335990in}{0.878085in}}%
\pgfpathlineto{\pgfqpoint{0.335990in}{0.878571in}}%
\pgfpathlineto{\pgfqpoint{0.186087in}{0.878571in}}%
\pgfpathlineto{\pgfqpoint{0.186087in}{0.878085in}}%
\pgfpathclose%
\pgfusepath{stroke,fill}%
\end{pgfscope}%
\begin{pgfscope}%
\pgfpathrectangle{\pgfqpoint{0.148611in}{0.878085in}}{\pgfqpoint{0.824468in}{0.243158in}}%
\pgfusepath{clip}%
\pgfsetbuttcap%
\pgfsetmiterjoin%
\definecolor{currentfill}{rgb}{0.121569,0.466667,0.705882}%
\pgfsetfillcolor{currentfill}%
\pgfsetfillopacity{0.500000}%
\pgfsetlinewidth{1.003750pt}%
\definecolor{currentstroke}{rgb}{0.000000,0.000000,0.000000}%
\pgfsetstrokecolor{currentstroke}%
\pgfsetdash{}{0pt}%
\pgfpathmoveto{\pgfqpoint{0.335990in}{0.878085in}}%
\pgfpathlineto{\pgfqpoint{0.485894in}{0.878085in}}%
\pgfpathlineto{\pgfqpoint{0.485894in}{0.878085in}}%
\pgfpathlineto{\pgfqpoint{0.335990in}{0.878085in}}%
\pgfpathlineto{\pgfqpoint{0.335990in}{0.878085in}}%
\pgfpathclose%
\pgfusepath{stroke,fill}%
\end{pgfscope}%
\begin{pgfscope}%
\pgfpathrectangle{\pgfqpoint{0.148611in}{0.878085in}}{\pgfqpoint{0.824468in}{0.243158in}}%
\pgfusepath{clip}%
\pgfsetbuttcap%
\pgfsetmiterjoin%
\definecolor{currentfill}{rgb}{0.121569,0.466667,0.705882}%
\pgfsetfillcolor{currentfill}%
\pgfsetfillopacity{0.500000}%
\pgfsetlinewidth{1.003750pt}%
\definecolor{currentstroke}{rgb}{0.000000,0.000000,0.000000}%
\pgfsetstrokecolor{currentstroke}%
\pgfsetdash{}{0pt}%
\pgfpathmoveto{\pgfqpoint{0.485894in}{0.878085in}}%
\pgfpathlineto{\pgfqpoint{0.635797in}{0.878085in}}%
\pgfpathlineto{\pgfqpoint{0.635797in}{0.878085in}}%
\pgfpathlineto{\pgfqpoint{0.485894in}{0.878085in}}%
\pgfpathlineto{\pgfqpoint{0.485894in}{0.878085in}}%
\pgfpathclose%
\pgfusepath{stroke,fill}%
\end{pgfscope}%
\begin{pgfscope}%
\pgfpathrectangle{\pgfqpoint{0.148611in}{0.878085in}}{\pgfqpoint{0.824468in}{0.243158in}}%
\pgfusepath{clip}%
\pgfsetbuttcap%
\pgfsetmiterjoin%
\definecolor{currentfill}{rgb}{0.121569,0.466667,0.705882}%
\pgfsetfillcolor{currentfill}%
\pgfsetfillopacity{0.500000}%
\pgfsetlinewidth{1.003750pt}%
\definecolor{currentstroke}{rgb}{0.000000,0.000000,0.000000}%
\pgfsetstrokecolor{currentstroke}%
\pgfsetdash{}{0pt}%
\pgfpathmoveto{\pgfqpoint{0.635797in}{0.878085in}}%
\pgfpathlineto{\pgfqpoint{0.785700in}{0.878085in}}%
\pgfpathlineto{\pgfqpoint{0.785700in}{0.878085in}}%
\pgfpathlineto{\pgfqpoint{0.635797in}{0.878085in}}%
\pgfpathlineto{\pgfqpoint{0.635797in}{0.878085in}}%
\pgfpathclose%
\pgfusepath{stroke,fill}%
\end{pgfscope}%
\begin{pgfscope}%
\pgfpathrectangle{\pgfqpoint{0.148611in}{0.878085in}}{\pgfqpoint{0.824468in}{0.243158in}}%
\pgfusepath{clip}%
\pgfsetbuttcap%
\pgfsetmiterjoin%
\definecolor{currentfill}{rgb}{0.121569,0.466667,0.705882}%
\pgfsetfillcolor{currentfill}%
\pgfsetfillopacity{0.500000}%
\pgfsetlinewidth{1.003750pt}%
\definecolor{currentstroke}{rgb}{0.000000,0.000000,0.000000}%
\pgfsetstrokecolor{currentstroke}%
\pgfsetdash{}{0pt}%
\pgfpathmoveto{\pgfqpoint{0.785700in}{0.878085in}}%
\pgfpathlineto{\pgfqpoint{0.935603in}{0.878085in}}%
\pgfpathlineto{\pgfqpoint{0.935603in}{0.878085in}}%
\pgfpathlineto{\pgfqpoint{0.785700in}{0.878085in}}%
\pgfpathlineto{\pgfqpoint{0.785700in}{0.878085in}}%
\pgfpathclose%
\pgfusepath{stroke,fill}%
\end{pgfscope}%
\begin{pgfscope}%
\pgfsetrectcap%
\pgfsetmiterjoin%
\pgfsetlinewidth{0.803000pt}%
\definecolor{currentstroke}{rgb}{0.000000,0.000000,0.000000}%
\pgfsetstrokecolor{currentstroke}%
\pgfsetdash{}{0pt}%
\pgfpathmoveto{\pgfqpoint{0.148611in}{0.878085in}}%
\pgfpathlineto{\pgfqpoint{0.148611in}{1.121243in}}%
\pgfusepath{stroke}%
\end{pgfscope}%
\begin{pgfscope}%
\pgfsetrectcap%
\pgfsetmiterjoin%
\pgfsetlinewidth{0.803000pt}%
\definecolor{currentstroke}{rgb}{0.000000,0.000000,0.000000}%
\pgfsetstrokecolor{currentstroke}%
\pgfsetdash{}{0pt}%
\pgfpathmoveto{\pgfqpoint{0.973079in}{0.878085in}}%
\pgfpathlineto{\pgfqpoint{0.973079in}{1.121243in}}%
\pgfusepath{stroke}%
\end{pgfscope}%
\begin{pgfscope}%
\pgfsetrectcap%
\pgfsetmiterjoin%
\pgfsetlinewidth{0.803000pt}%
\definecolor{currentstroke}{rgb}{0.000000,0.000000,0.000000}%
\pgfsetstrokecolor{currentstroke}%
\pgfsetdash{}{0pt}%
\pgfpathmoveto{\pgfqpoint{0.148611in}{0.878085in}}%
\pgfpathlineto{\pgfqpoint{0.973079in}{0.878085in}}%
\pgfusepath{stroke}%
\end{pgfscope}%
\begin{pgfscope}%
\pgfsetrectcap%
\pgfsetmiterjoin%
\pgfsetlinewidth{0.803000pt}%
\definecolor{currentstroke}{rgb}{0.000000,0.000000,0.000000}%
\pgfsetstrokecolor{currentstroke}%
\pgfsetdash{}{0pt}%
\pgfpathmoveto{\pgfqpoint{0.148611in}{1.121243in}}%
\pgfpathlineto{\pgfqpoint{0.973079in}{1.121243in}}%
\pgfusepath{stroke}%
\end{pgfscope}%
\begin{pgfscope}%
\definecolor{textcolor}{rgb}{0.000000,0.000000,0.000000}%
\pgfsetstrokecolor{textcolor}%
\pgfsetfillcolor{textcolor}%
\pgftext[x=0.560845in,y=1.204576in,,base]{\color{textcolor}\rmfamily\fontsize{11.000000}{13.200000}\selectfont MMA}%
\end{pgfscope}%
\begin{pgfscope}%
\pgfsetbuttcap%
\pgfsetmiterjoin%
\definecolor{currentfill}{rgb}{1.000000,1.000000,1.000000}%
\pgfsetfillcolor{currentfill}%
\pgfsetlinewidth{0.000000pt}%
\definecolor{currentstroke}{rgb}{0.000000,0.000000,0.000000}%
\pgfsetstrokecolor{currentstroke}%
\pgfsetstrokeopacity{0.000000}%
\pgfsetdash{}{0pt}%
\pgfpathmoveto{\pgfqpoint{1.137973in}{0.878085in}}%
\pgfpathlineto{\pgfqpoint{1.962441in}{0.878085in}}%
\pgfpathlineto{\pgfqpoint{1.962441in}{1.121243in}}%
\pgfpathlineto{\pgfqpoint{1.137973in}{1.121243in}}%
\pgfpathlineto{\pgfqpoint{1.137973in}{0.878085in}}%
\pgfpathclose%
\pgfusepath{fill}%
\end{pgfscope}%
\begin{pgfscope}%
\pgfpathrectangle{\pgfqpoint{1.137973in}{0.878085in}}{\pgfqpoint{0.824468in}{0.243158in}}%
\pgfusepath{clip}%
\pgfsetbuttcap%
\pgfsetmiterjoin%
\definecolor{currentfill}{rgb}{0.121569,0.466667,0.705882}%
\pgfsetfillcolor{currentfill}%
\pgfsetfillopacity{0.500000}%
\pgfsetlinewidth{1.003750pt}%
\definecolor{currentstroke}{rgb}{0.000000,0.000000,0.000000}%
\pgfsetstrokecolor{currentstroke}%
\pgfsetdash{}{0pt}%
\pgfpathmoveto{\pgfqpoint{1.175449in}{0.878085in}}%
\pgfpathlineto{\pgfqpoint{1.325352in}{0.878085in}}%
\pgfpathlineto{\pgfqpoint{1.325352in}{0.881975in}}%
\pgfpathlineto{\pgfqpoint{1.175449in}{0.881975in}}%
\pgfpathlineto{\pgfqpoint{1.175449in}{0.878085in}}%
\pgfpathclose%
\pgfusepath{stroke,fill}%
\end{pgfscope}%
\begin{pgfscope}%
\pgfpathrectangle{\pgfqpoint{1.137973in}{0.878085in}}{\pgfqpoint{0.824468in}{0.243158in}}%
\pgfusepath{clip}%
\pgfsetbuttcap%
\pgfsetmiterjoin%
\definecolor{currentfill}{rgb}{0.121569,0.466667,0.705882}%
\pgfsetfillcolor{currentfill}%
\pgfsetfillopacity{0.500000}%
\pgfsetlinewidth{1.003750pt}%
\definecolor{currentstroke}{rgb}{0.000000,0.000000,0.000000}%
\pgfsetstrokecolor{currentstroke}%
\pgfsetdash{}{0pt}%
\pgfpathmoveto{\pgfqpoint{1.325352in}{0.878085in}}%
\pgfpathlineto{\pgfqpoint{1.475255in}{0.878085in}}%
\pgfpathlineto{\pgfqpoint{1.475255in}{0.879908in}}%
\pgfpathlineto{\pgfqpoint{1.325352in}{0.879908in}}%
\pgfpathlineto{\pgfqpoint{1.325352in}{0.878085in}}%
\pgfpathclose%
\pgfusepath{stroke,fill}%
\end{pgfscope}%
\begin{pgfscope}%
\pgfpathrectangle{\pgfqpoint{1.137973in}{0.878085in}}{\pgfqpoint{0.824468in}{0.243158in}}%
\pgfusepath{clip}%
\pgfsetbuttcap%
\pgfsetmiterjoin%
\definecolor{currentfill}{rgb}{0.121569,0.466667,0.705882}%
\pgfsetfillcolor{currentfill}%
\pgfsetfillopacity{0.500000}%
\pgfsetlinewidth{1.003750pt}%
\definecolor{currentstroke}{rgb}{0.000000,0.000000,0.000000}%
\pgfsetstrokecolor{currentstroke}%
\pgfsetdash{}{0pt}%
\pgfpathmoveto{\pgfqpoint{1.475255in}{0.878085in}}%
\pgfpathlineto{\pgfqpoint{1.625158in}{0.878085in}}%
\pgfpathlineto{\pgfqpoint{1.625158in}{0.878206in}}%
\pgfpathlineto{\pgfqpoint{1.475255in}{0.878206in}}%
\pgfpathlineto{\pgfqpoint{1.475255in}{0.878085in}}%
\pgfpathclose%
\pgfusepath{stroke,fill}%
\end{pgfscope}%
\begin{pgfscope}%
\pgfpathrectangle{\pgfqpoint{1.137973in}{0.878085in}}{\pgfqpoint{0.824468in}{0.243158in}}%
\pgfusepath{clip}%
\pgfsetbuttcap%
\pgfsetmiterjoin%
\definecolor{currentfill}{rgb}{0.121569,0.466667,0.705882}%
\pgfsetfillcolor{currentfill}%
\pgfsetfillopacity{0.500000}%
\pgfsetlinewidth{1.003750pt}%
\definecolor{currentstroke}{rgb}{0.000000,0.000000,0.000000}%
\pgfsetstrokecolor{currentstroke}%
\pgfsetdash{}{0pt}%
\pgfpathmoveto{\pgfqpoint{1.625158in}{0.878085in}}%
\pgfpathlineto{\pgfqpoint{1.775062in}{0.878085in}}%
\pgfpathlineto{\pgfqpoint{1.775062in}{0.878085in}}%
\pgfpathlineto{\pgfqpoint{1.625158in}{0.878085in}}%
\pgfpathlineto{\pgfqpoint{1.625158in}{0.878085in}}%
\pgfpathclose%
\pgfusepath{stroke,fill}%
\end{pgfscope}%
\begin{pgfscope}%
\pgfpathrectangle{\pgfqpoint{1.137973in}{0.878085in}}{\pgfqpoint{0.824468in}{0.243158in}}%
\pgfusepath{clip}%
\pgfsetbuttcap%
\pgfsetmiterjoin%
\definecolor{currentfill}{rgb}{0.121569,0.466667,0.705882}%
\pgfsetfillcolor{currentfill}%
\pgfsetfillopacity{0.500000}%
\pgfsetlinewidth{1.003750pt}%
\definecolor{currentstroke}{rgb}{0.000000,0.000000,0.000000}%
\pgfsetstrokecolor{currentstroke}%
\pgfsetdash{}{0pt}%
\pgfpathmoveto{\pgfqpoint{1.775062in}{0.878085in}}%
\pgfpathlineto{\pgfqpoint{1.924965in}{0.878085in}}%
\pgfpathlineto{\pgfqpoint{1.924965in}{0.878693in}}%
\pgfpathlineto{\pgfqpoint{1.775062in}{0.878693in}}%
\pgfpathlineto{\pgfqpoint{1.775062in}{0.878085in}}%
\pgfpathclose%
\pgfusepath{stroke,fill}%
\end{pgfscope}%
\begin{pgfscope}%
\pgfsetrectcap%
\pgfsetmiterjoin%
\pgfsetlinewidth{0.803000pt}%
\definecolor{currentstroke}{rgb}{0.000000,0.000000,0.000000}%
\pgfsetstrokecolor{currentstroke}%
\pgfsetdash{}{0pt}%
\pgfpathmoveto{\pgfqpoint{1.137973in}{0.878085in}}%
\pgfpathlineto{\pgfqpoint{1.137973in}{1.121243in}}%
\pgfusepath{stroke}%
\end{pgfscope}%
\begin{pgfscope}%
\pgfsetrectcap%
\pgfsetmiterjoin%
\pgfsetlinewidth{0.803000pt}%
\definecolor{currentstroke}{rgb}{0.000000,0.000000,0.000000}%
\pgfsetstrokecolor{currentstroke}%
\pgfsetdash{}{0pt}%
\pgfpathmoveto{\pgfqpoint{1.962441in}{0.878085in}}%
\pgfpathlineto{\pgfqpoint{1.962441in}{1.121243in}}%
\pgfusepath{stroke}%
\end{pgfscope}%
\begin{pgfscope}%
\pgfsetrectcap%
\pgfsetmiterjoin%
\pgfsetlinewidth{0.803000pt}%
\definecolor{currentstroke}{rgb}{0.000000,0.000000,0.000000}%
\pgfsetstrokecolor{currentstroke}%
\pgfsetdash{}{0pt}%
\pgfpathmoveto{\pgfqpoint{1.137973in}{0.878085in}}%
\pgfpathlineto{\pgfqpoint{1.962441in}{0.878085in}}%
\pgfusepath{stroke}%
\end{pgfscope}%
\begin{pgfscope}%
\pgfsetrectcap%
\pgfsetmiterjoin%
\pgfsetlinewidth{0.803000pt}%
\definecolor{currentstroke}{rgb}{0.000000,0.000000,0.000000}%
\pgfsetstrokecolor{currentstroke}%
\pgfsetdash{}{0pt}%
\pgfpathmoveto{\pgfqpoint{1.137973in}{1.121243in}}%
\pgfpathlineto{\pgfqpoint{1.962441in}{1.121243in}}%
\pgfusepath{stroke}%
\end{pgfscope}%
\begin{pgfscope}%
\definecolor{textcolor}{rgb}{0.000000,0.000000,0.000000}%
\pgfsetstrokecolor{textcolor}%
\pgfsetfillcolor{textcolor}%
\pgftext[x=1.550207in,y=1.204576in,,base]{\color{textcolor}\rmfamily\fontsize{11.000000}{13.200000}\selectfont MetLife}%
\end{pgfscope}%
\begin{pgfscope}%
\pgfsetbuttcap%
\pgfsetmiterjoin%
\definecolor{currentfill}{rgb}{1.000000,1.000000,1.000000}%
\pgfsetfillcolor{currentfill}%
\pgfsetlinewidth{0.000000pt}%
\definecolor{currentstroke}{rgb}{0.000000,0.000000,0.000000}%
\pgfsetstrokecolor{currentstroke}%
\pgfsetstrokeopacity{0.000000}%
\pgfsetdash{}{0pt}%
\pgfpathmoveto{\pgfqpoint{2.127335in}{0.878085in}}%
\pgfpathlineto{\pgfqpoint{2.951803in}{0.878085in}}%
\pgfpathlineto{\pgfqpoint{2.951803in}{1.121243in}}%
\pgfpathlineto{\pgfqpoint{2.127335in}{1.121243in}}%
\pgfpathlineto{\pgfqpoint{2.127335in}{0.878085in}}%
\pgfpathclose%
\pgfusepath{fill}%
\end{pgfscope}%
\begin{pgfscope}%
\pgfpathrectangle{\pgfqpoint{2.127335in}{0.878085in}}{\pgfqpoint{0.824468in}{0.243158in}}%
\pgfusepath{clip}%
\pgfsetbuttcap%
\pgfsetmiterjoin%
\definecolor{currentfill}{rgb}{0.121569,0.466667,0.705882}%
\pgfsetfillcolor{currentfill}%
\pgfsetfillopacity{0.500000}%
\pgfsetlinewidth{1.003750pt}%
\definecolor{currentstroke}{rgb}{0.000000,0.000000,0.000000}%
\pgfsetstrokecolor{currentstroke}%
\pgfsetdash{}{0pt}%
\pgfpathmoveto{\pgfqpoint{2.164810in}{0.878085in}}%
\pgfpathlineto{\pgfqpoint{2.314714in}{0.878085in}}%
\pgfpathlineto{\pgfqpoint{2.314714in}{0.888054in}}%
\pgfpathlineto{\pgfqpoint{2.164810in}{0.888054in}}%
\pgfpathlineto{\pgfqpoint{2.164810in}{0.878085in}}%
\pgfpathclose%
\pgfusepath{stroke,fill}%
\end{pgfscope}%
\begin{pgfscope}%
\pgfpathrectangle{\pgfqpoint{2.127335in}{0.878085in}}{\pgfqpoint{0.824468in}{0.243158in}}%
\pgfusepath{clip}%
\pgfsetbuttcap%
\pgfsetmiterjoin%
\definecolor{currentfill}{rgb}{0.121569,0.466667,0.705882}%
\pgfsetfillcolor{currentfill}%
\pgfsetfillopacity{0.500000}%
\pgfsetlinewidth{1.003750pt}%
\definecolor{currentstroke}{rgb}{0.000000,0.000000,0.000000}%
\pgfsetstrokecolor{currentstroke}%
\pgfsetdash{}{0pt}%
\pgfpathmoveto{\pgfqpoint{2.314714in}{0.878085in}}%
\pgfpathlineto{\pgfqpoint{2.464617in}{0.878085in}}%
\pgfpathlineto{\pgfqpoint{2.464617in}{0.882218in}}%
\pgfpathlineto{\pgfqpoint{2.314714in}{0.882218in}}%
\pgfpathlineto{\pgfqpoint{2.314714in}{0.878085in}}%
\pgfpathclose%
\pgfusepath{stroke,fill}%
\end{pgfscope}%
\begin{pgfscope}%
\pgfpathrectangle{\pgfqpoint{2.127335in}{0.878085in}}{\pgfqpoint{0.824468in}{0.243158in}}%
\pgfusepath{clip}%
\pgfsetbuttcap%
\pgfsetmiterjoin%
\definecolor{currentfill}{rgb}{0.121569,0.466667,0.705882}%
\pgfsetfillcolor{currentfill}%
\pgfsetfillopacity{0.500000}%
\pgfsetlinewidth{1.003750pt}%
\definecolor{currentstroke}{rgb}{0.000000,0.000000,0.000000}%
\pgfsetstrokecolor{currentstroke}%
\pgfsetdash{}{0pt}%
\pgfpathmoveto{\pgfqpoint{2.464617in}{0.878085in}}%
\pgfpathlineto{\pgfqpoint{2.614520in}{0.878085in}}%
\pgfpathlineto{\pgfqpoint{2.614520in}{0.880273in}}%
\pgfpathlineto{\pgfqpoint{2.464617in}{0.880273in}}%
\pgfpathlineto{\pgfqpoint{2.464617in}{0.878085in}}%
\pgfpathclose%
\pgfusepath{stroke,fill}%
\end{pgfscope}%
\begin{pgfscope}%
\pgfpathrectangle{\pgfqpoint{2.127335in}{0.878085in}}{\pgfqpoint{0.824468in}{0.243158in}}%
\pgfusepath{clip}%
\pgfsetbuttcap%
\pgfsetmiterjoin%
\definecolor{currentfill}{rgb}{0.121569,0.466667,0.705882}%
\pgfsetfillcolor{currentfill}%
\pgfsetfillopacity{0.500000}%
\pgfsetlinewidth{1.003750pt}%
\definecolor{currentstroke}{rgb}{0.000000,0.000000,0.000000}%
\pgfsetstrokecolor{currentstroke}%
\pgfsetdash{}{0pt}%
\pgfpathmoveto{\pgfqpoint{2.614520in}{0.878085in}}%
\pgfpathlineto{\pgfqpoint{2.764423in}{0.878085in}}%
\pgfpathlineto{\pgfqpoint{2.764423in}{0.878693in}}%
\pgfpathlineto{\pgfqpoint{2.614520in}{0.878693in}}%
\pgfpathlineto{\pgfqpoint{2.614520in}{0.878085in}}%
\pgfpathclose%
\pgfusepath{stroke,fill}%
\end{pgfscope}%
\begin{pgfscope}%
\pgfpathrectangle{\pgfqpoint{2.127335in}{0.878085in}}{\pgfqpoint{0.824468in}{0.243158in}}%
\pgfusepath{clip}%
\pgfsetbuttcap%
\pgfsetmiterjoin%
\definecolor{currentfill}{rgb}{0.121569,0.466667,0.705882}%
\pgfsetfillcolor{currentfill}%
\pgfsetfillopacity{0.500000}%
\pgfsetlinewidth{1.003750pt}%
\definecolor{currentstroke}{rgb}{0.000000,0.000000,0.000000}%
\pgfsetstrokecolor{currentstroke}%
\pgfsetdash{}{0pt}%
\pgfpathmoveto{\pgfqpoint{2.764423in}{0.878085in}}%
\pgfpathlineto{\pgfqpoint{2.914327in}{0.878085in}}%
\pgfpathlineto{\pgfqpoint{2.914327in}{0.878936in}}%
\pgfpathlineto{\pgfqpoint{2.764423in}{0.878936in}}%
\pgfpathlineto{\pgfqpoint{2.764423in}{0.878085in}}%
\pgfpathclose%
\pgfusepath{stroke,fill}%
\end{pgfscope}%
\begin{pgfscope}%
\pgfsetrectcap%
\pgfsetmiterjoin%
\pgfsetlinewidth{0.803000pt}%
\definecolor{currentstroke}{rgb}{0.000000,0.000000,0.000000}%
\pgfsetstrokecolor{currentstroke}%
\pgfsetdash{}{0pt}%
\pgfpathmoveto{\pgfqpoint{2.127335in}{0.878085in}}%
\pgfpathlineto{\pgfqpoint{2.127335in}{1.121243in}}%
\pgfusepath{stroke}%
\end{pgfscope}%
\begin{pgfscope}%
\pgfsetrectcap%
\pgfsetmiterjoin%
\pgfsetlinewidth{0.803000pt}%
\definecolor{currentstroke}{rgb}{0.000000,0.000000,0.000000}%
\pgfsetstrokecolor{currentstroke}%
\pgfsetdash{}{0pt}%
\pgfpathmoveto{\pgfqpoint{2.951803in}{0.878085in}}%
\pgfpathlineto{\pgfqpoint{2.951803in}{1.121243in}}%
\pgfusepath{stroke}%
\end{pgfscope}%
\begin{pgfscope}%
\pgfsetrectcap%
\pgfsetmiterjoin%
\pgfsetlinewidth{0.803000pt}%
\definecolor{currentstroke}{rgb}{0.000000,0.000000,0.000000}%
\pgfsetstrokecolor{currentstroke}%
\pgfsetdash{}{0pt}%
\pgfpathmoveto{\pgfqpoint{2.127335in}{0.878085in}}%
\pgfpathlineto{\pgfqpoint{2.951803in}{0.878085in}}%
\pgfusepath{stroke}%
\end{pgfscope}%
\begin{pgfscope}%
\pgfsetrectcap%
\pgfsetmiterjoin%
\pgfsetlinewidth{0.803000pt}%
\definecolor{currentstroke}{rgb}{0.000000,0.000000,0.000000}%
\pgfsetstrokecolor{currentstroke}%
\pgfsetdash{}{0pt}%
\pgfpathmoveto{\pgfqpoint{2.127335in}{1.121243in}}%
\pgfpathlineto{\pgfqpoint{2.951803in}{1.121243in}}%
\pgfusepath{stroke}%
\end{pgfscope}%
\begin{pgfscope}%
\definecolor{textcolor}{rgb}{0.000000,0.000000,0.000000}%
\pgfsetstrokecolor{textcolor}%
\pgfsetfillcolor{textcolor}%
\pgftext[x=2.539569in,y=1.204576in,,base]{\color{textcolor}\rmfamily\fontsize{11.000000}{13.200000}\selectfont Crédit...}%
\end{pgfscope}%
\begin{pgfscope}%
\pgfsetbuttcap%
\pgfsetmiterjoin%
\definecolor{currentfill}{rgb}{1.000000,1.000000,1.000000}%
\pgfsetfillcolor{currentfill}%
\pgfsetlinewidth{0.000000pt}%
\definecolor{currentstroke}{rgb}{0.000000,0.000000,0.000000}%
\pgfsetstrokecolor{currentstroke}%
\pgfsetstrokeopacity{0.000000}%
\pgfsetdash{}{0pt}%
\pgfpathmoveto{\pgfqpoint{3.116696in}{0.878085in}}%
\pgfpathlineto{\pgfqpoint{3.941164in}{0.878085in}}%
\pgfpathlineto{\pgfqpoint{3.941164in}{1.121243in}}%
\pgfpathlineto{\pgfqpoint{3.116696in}{1.121243in}}%
\pgfpathlineto{\pgfqpoint{3.116696in}{0.878085in}}%
\pgfpathclose%
\pgfusepath{fill}%
\end{pgfscope}%
\begin{pgfscope}%
\pgfpathrectangle{\pgfqpoint{3.116696in}{0.878085in}}{\pgfqpoint{0.824468in}{0.243158in}}%
\pgfusepath{clip}%
\pgfsetbuttcap%
\pgfsetmiterjoin%
\definecolor{currentfill}{rgb}{0.121569,0.466667,0.705882}%
\pgfsetfillcolor{currentfill}%
\pgfsetfillopacity{0.500000}%
\pgfsetlinewidth{1.003750pt}%
\definecolor{currentstroke}{rgb}{0.000000,0.000000,0.000000}%
\pgfsetstrokecolor{currentstroke}%
\pgfsetdash{}{0pt}%
\pgfpathmoveto{\pgfqpoint{3.154172in}{0.878085in}}%
\pgfpathlineto{\pgfqpoint{3.304075in}{0.878085in}}%
\pgfpathlineto{\pgfqpoint{3.304075in}{0.880030in}}%
\pgfpathlineto{\pgfqpoint{3.154172in}{0.880030in}}%
\pgfpathlineto{\pgfqpoint{3.154172in}{0.878085in}}%
\pgfpathclose%
\pgfusepath{stroke,fill}%
\end{pgfscope}%
\begin{pgfscope}%
\pgfpathrectangle{\pgfqpoint{3.116696in}{0.878085in}}{\pgfqpoint{0.824468in}{0.243158in}}%
\pgfusepath{clip}%
\pgfsetbuttcap%
\pgfsetmiterjoin%
\definecolor{currentfill}{rgb}{0.121569,0.466667,0.705882}%
\pgfsetfillcolor{currentfill}%
\pgfsetfillopacity{0.500000}%
\pgfsetlinewidth{1.003750pt}%
\definecolor{currentstroke}{rgb}{0.000000,0.000000,0.000000}%
\pgfsetstrokecolor{currentstroke}%
\pgfsetdash{}{0pt}%
\pgfpathmoveto{\pgfqpoint{3.304075in}{0.878085in}}%
\pgfpathlineto{\pgfqpoint{3.453979in}{0.878085in}}%
\pgfpathlineto{\pgfqpoint{3.453979in}{0.878571in}}%
\pgfpathlineto{\pgfqpoint{3.304075in}{0.878571in}}%
\pgfpathlineto{\pgfqpoint{3.304075in}{0.878085in}}%
\pgfpathclose%
\pgfusepath{stroke,fill}%
\end{pgfscope}%
\begin{pgfscope}%
\pgfpathrectangle{\pgfqpoint{3.116696in}{0.878085in}}{\pgfqpoint{0.824468in}{0.243158in}}%
\pgfusepath{clip}%
\pgfsetbuttcap%
\pgfsetmiterjoin%
\definecolor{currentfill}{rgb}{0.121569,0.466667,0.705882}%
\pgfsetfillcolor{currentfill}%
\pgfsetfillopacity{0.500000}%
\pgfsetlinewidth{1.003750pt}%
\definecolor{currentstroke}{rgb}{0.000000,0.000000,0.000000}%
\pgfsetstrokecolor{currentstroke}%
\pgfsetdash{}{0pt}%
\pgfpathmoveto{\pgfqpoint{3.453979in}{0.878085in}}%
\pgfpathlineto{\pgfqpoint{3.603882in}{0.878085in}}%
\pgfpathlineto{\pgfqpoint{3.603882in}{0.878085in}}%
\pgfpathlineto{\pgfqpoint{3.453979in}{0.878085in}}%
\pgfpathlineto{\pgfqpoint{3.453979in}{0.878085in}}%
\pgfpathclose%
\pgfusepath{stroke,fill}%
\end{pgfscope}%
\begin{pgfscope}%
\pgfpathrectangle{\pgfqpoint{3.116696in}{0.878085in}}{\pgfqpoint{0.824468in}{0.243158in}}%
\pgfusepath{clip}%
\pgfsetbuttcap%
\pgfsetmiterjoin%
\definecolor{currentfill}{rgb}{0.121569,0.466667,0.705882}%
\pgfsetfillcolor{currentfill}%
\pgfsetfillopacity{0.500000}%
\pgfsetlinewidth{1.003750pt}%
\definecolor{currentstroke}{rgb}{0.000000,0.000000,0.000000}%
\pgfsetstrokecolor{currentstroke}%
\pgfsetdash{}{0pt}%
\pgfpathmoveto{\pgfqpoint{3.603882in}{0.878085in}}%
\pgfpathlineto{\pgfqpoint{3.753785in}{0.878085in}}%
\pgfpathlineto{\pgfqpoint{3.753785in}{0.878328in}}%
\pgfpathlineto{\pgfqpoint{3.603882in}{0.878328in}}%
\pgfpathlineto{\pgfqpoint{3.603882in}{0.878085in}}%
\pgfpathclose%
\pgfusepath{stroke,fill}%
\end{pgfscope}%
\begin{pgfscope}%
\pgfpathrectangle{\pgfqpoint{3.116696in}{0.878085in}}{\pgfqpoint{0.824468in}{0.243158in}}%
\pgfusepath{clip}%
\pgfsetbuttcap%
\pgfsetmiterjoin%
\definecolor{currentfill}{rgb}{0.121569,0.466667,0.705882}%
\pgfsetfillcolor{currentfill}%
\pgfsetfillopacity{0.500000}%
\pgfsetlinewidth{1.003750pt}%
\definecolor{currentstroke}{rgb}{0.000000,0.000000,0.000000}%
\pgfsetstrokecolor{currentstroke}%
\pgfsetdash{}{0pt}%
\pgfpathmoveto{\pgfqpoint{3.753785in}{0.878085in}}%
\pgfpathlineto{\pgfqpoint{3.903688in}{0.878085in}}%
\pgfpathlineto{\pgfqpoint{3.903688in}{0.878571in}}%
\pgfpathlineto{\pgfqpoint{3.753785in}{0.878571in}}%
\pgfpathlineto{\pgfqpoint{3.753785in}{0.878085in}}%
\pgfpathclose%
\pgfusepath{stroke,fill}%
\end{pgfscope}%
\begin{pgfscope}%
\pgfsetrectcap%
\pgfsetmiterjoin%
\pgfsetlinewidth{0.803000pt}%
\definecolor{currentstroke}{rgb}{0.000000,0.000000,0.000000}%
\pgfsetstrokecolor{currentstroke}%
\pgfsetdash{}{0pt}%
\pgfpathmoveto{\pgfqpoint{3.116696in}{0.878085in}}%
\pgfpathlineto{\pgfqpoint{3.116696in}{1.121243in}}%
\pgfusepath{stroke}%
\end{pgfscope}%
\begin{pgfscope}%
\pgfsetrectcap%
\pgfsetmiterjoin%
\pgfsetlinewidth{0.803000pt}%
\definecolor{currentstroke}{rgb}{0.000000,0.000000,0.000000}%
\pgfsetstrokecolor{currentstroke}%
\pgfsetdash{}{0pt}%
\pgfpathmoveto{\pgfqpoint{3.941164in}{0.878085in}}%
\pgfpathlineto{\pgfqpoint{3.941164in}{1.121243in}}%
\pgfusepath{stroke}%
\end{pgfscope}%
\begin{pgfscope}%
\pgfsetrectcap%
\pgfsetmiterjoin%
\pgfsetlinewidth{0.803000pt}%
\definecolor{currentstroke}{rgb}{0.000000,0.000000,0.000000}%
\pgfsetstrokecolor{currentstroke}%
\pgfsetdash{}{0pt}%
\pgfpathmoveto{\pgfqpoint{3.116696in}{0.878085in}}%
\pgfpathlineto{\pgfqpoint{3.941164in}{0.878085in}}%
\pgfusepath{stroke}%
\end{pgfscope}%
\begin{pgfscope}%
\pgfsetrectcap%
\pgfsetmiterjoin%
\pgfsetlinewidth{0.803000pt}%
\definecolor{currentstroke}{rgb}{0.000000,0.000000,0.000000}%
\pgfsetstrokecolor{currentstroke}%
\pgfsetdash{}{0pt}%
\pgfpathmoveto{\pgfqpoint{3.116696in}{1.121243in}}%
\pgfpathlineto{\pgfqpoint{3.941164in}{1.121243in}}%
\pgfusepath{stroke}%
\end{pgfscope}%
\begin{pgfscope}%
\definecolor{textcolor}{rgb}{0.000000,0.000000,0.000000}%
\pgfsetstrokecolor{textcolor}%
\pgfsetfillcolor{textcolor}%
\pgftext[x=3.528930in,y=1.204576in,,base]{\color{textcolor}\rmfamily\fontsize{11.000000}{13.200000}\selectfont Afi Esca}%
\end{pgfscope}%
\begin{pgfscope}%
\pgfsetbuttcap%
\pgfsetmiterjoin%
\definecolor{currentfill}{rgb}{1.000000,1.000000,1.000000}%
\pgfsetfillcolor{currentfill}%
\pgfsetlinewidth{0.000000pt}%
\definecolor{currentstroke}{rgb}{0.000000,0.000000,0.000000}%
\pgfsetstrokecolor{currentstroke}%
\pgfsetstrokeopacity{0.000000}%
\pgfsetdash{}{0pt}%
\pgfpathmoveto{\pgfqpoint{4.106058in}{0.878085in}}%
\pgfpathlineto{\pgfqpoint{4.930526in}{0.878085in}}%
\pgfpathlineto{\pgfqpoint{4.930526in}{1.121243in}}%
\pgfpathlineto{\pgfqpoint{4.106058in}{1.121243in}}%
\pgfpathlineto{\pgfqpoint{4.106058in}{0.878085in}}%
\pgfpathclose%
\pgfusepath{fill}%
\end{pgfscope}%
\begin{pgfscope}%
\pgfpathrectangle{\pgfqpoint{4.106058in}{0.878085in}}{\pgfqpoint{0.824468in}{0.243158in}}%
\pgfusepath{clip}%
\pgfsetbuttcap%
\pgfsetmiterjoin%
\definecolor{currentfill}{rgb}{0.121569,0.466667,0.705882}%
\pgfsetfillcolor{currentfill}%
\pgfsetfillopacity{0.500000}%
\pgfsetlinewidth{1.003750pt}%
\definecolor{currentstroke}{rgb}{0.000000,0.000000,0.000000}%
\pgfsetstrokecolor{currentstroke}%
\pgfsetdash{}{0pt}%
\pgfpathmoveto{\pgfqpoint{4.143534in}{0.878085in}}%
\pgfpathlineto{\pgfqpoint{4.293437in}{0.878085in}}%
\pgfpathlineto{\pgfqpoint{4.293437in}{0.881246in}}%
\pgfpathlineto{\pgfqpoint{4.143534in}{0.881246in}}%
\pgfpathlineto{\pgfqpoint{4.143534in}{0.878085in}}%
\pgfpathclose%
\pgfusepath{stroke,fill}%
\end{pgfscope}%
\begin{pgfscope}%
\pgfpathrectangle{\pgfqpoint{4.106058in}{0.878085in}}{\pgfqpoint{0.824468in}{0.243158in}}%
\pgfusepath{clip}%
\pgfsetbuttcap%
\pgfsetmiterjoin%
\definecolor{currentfill}{rgb}{0.121569,0.466667,0.705882}%
\pgfsetfillcolor{currentfill}%
\pgfsetfillopacity{0.500000}%
\pgfsetlinewidth{1.003750pt}%
\definecolor{currentstroke}{rgb}{0.000000,0.000000,0.000000}%
\pgfsetstrokecolor{currentstroke}%
\pgfsetdash{}{0pt}%
\pgfpathmoveto{\pgfqpoint{4.293437in}{0.878085in}}%
\pgfpathlineto{\pgfqpoint{4.443340in}{0.878085in}}%
\pgfpathlineto{\pgfqpoint{4.443340in}{0.878450in}}%
\pgfpathlineto{\pgfqpoint{4.293437in}{0.878450in}}%
\pgfpathlineto{\pgfqpoint{4.293437in}{0.878085in}}%
\pgfpathclose%
\pgfusepath{stroke,fill}%
\end{pgfscope}%
\begin{pgfscope}%
\pgfpathrectangle{\pgfqpoint{4.106058in}{0.878085in}}{\pgfqpoint{0.824468in}{0.243158in}}%
\pgfusepath{clip}%
\pgfsetbuttcap%
\pgfsetmiterjoin%
\definecolor{currentfill}{rgb}{0.121569,0.466667,0.705882}%
\pgfsetfillcolor{currentfill}%
\pgfsetfillopacity{0.500000}%
\pgfsetlinewidth{1.003750pt}%
\definecolor{currentstroke}{rgb}{0.000000,0.000000,0.000000}%
\pgfsetstrokecolor{currentstroke}%
\pgfsetdash{}{0pt}%
\pgfpathmoveto{\pgfqpoint{4.443340in}{0.878085in}}%
\pgfpathlineto{\pgfqpoint{4.593244in}{0.878085in}}%
\pgfpathlineto{\pgfqpoint{4.593244in}{0.878206in}}%
\pgfpathlineto{\pgfqpoint{4.443340in}{0.878206in}}%
\pgfpathlineto{\pgfqpoint{4.443340in}{0.878085in}}%
\pgfpathclose%
\pgfusepath{stroke,fill}%
\end{pgfscope}%
\begin{pgfscope}%
\pgfpathrectangle{\pgfqpoint{4.106058in}{0.878085in}}{\pgfqpoint{0.824468in}{0.243158in}}%
\pgfusepath{clip}%
\pgfsetbuttcap%
\pgfsetmiterjoin%
\definecolor{currentfill}{rgb}{0.121569,0.466667,0.705882}%
\pgfsetfillcolor{currentfill}%
\pgfsetfillopacity{0.500000}%
\pgfsetlinewidth{1.003750pt}%
\definecolor{currentstroke}{rgb}{0.000000,0.000000,0.000000}%
\pgfsetstrokecolor{currentstroke}%
\pgfsetdash{}{0pt}%
\pgfpathmoveto{\pgfqpoint{4.593244in}{0.878085in}}%
\pgfpathlineto{\pgfqpoint{4.743147in}{0.878085in}}%
\pgfpathlineto{\pgfqpoint{4.743147in}{0.878206in}}%
\pgfpathlineto{\pgfqpoint{4.593244in}{0.878206in}}%
\pgfpathlineto{\pgfqpoint{4.593244in}{0.878085in}}%
\pgfpathclose%
\pgfusepath{stroke,fill}%
\end{pgfscope}%
\begin{pgfscope}%
\pgfpathrectangle{\pgfqpoint{4.106058in}{0.878085in}}{\pgfqpoint{0.824468in}{0.243158in}}%
\pgfusepath{clip}%
\pgfsetbuttcap%
\pgfsetmiterjoin%
\definecolor{currentfill}{rgb}{0.121569,0.466667,0.705882}%
\pgfsetfillcolor{currentfill}%
\pgfsetfillopacity{0.500000}%
\pgfsetlinewidth{1.003750pt}%
\definecolor{currentstroke}{rgb}{0.000000,0.000000,0.000000}%
\pgfsetstrokecolor{currentstroke}%
\pgfsetdash{}{0pt}%
\pgfpathmoveto{\pgfqpoint{4.743147in}{0.878085in}}%
\pgfpathlineto{\pgfqpoint{4.893050in}{0.878085in}}%
\pgfpathlineto{\pgfqpoint{4.893050in}{0.878328in}}%
\pgfpathlineto{\pgfqpoint{4.743147in}{0.878328in}}%
\pgfpathlineto{\pgfqpoint{4.743147in}{0.878085in}}%
\pgfpathclose%
\pgfusepath{stroke,fill}%
\end{pgfscope}%
\begin{pgfscope}%
\pgfsetrectcap%
\pgfsetmiterjoin%
\pgfsetlinewidth{0.803000pt}%
\definecolor{currentstroke}{rgb}{0.000000,0.000000,0.000000}%
\pgfsetstrokecolor{currentstroke}%
\pgfsetdash{}{0pt}%
\pgfpathmoveto{\pgfqpoint{4.106058in}{0.878085in}}%
\pgfpathlineto{\pgfqpoint{4.106058in}{1.121243in}}%
\pgfusepath{stroke}%
\end{pgfscope}%
\begin{pgfscope}%
\pgfsetrectcap%
\pgfsetmiterjoin%
\pgfsetlinewidth{0.803000pt}%
\definecolor{currentstroke}{rgb}{0.000000,0.000000,0.000000}%
\pgfsetstrokecolor{currentstroke}%
\pgfsetdash{}{0pt}%
\pgfpathmoveto{\pgfqpoint{4.930526in}{0.878085in}}%
\pgfpathlineto{\pgfqpoint{4.930526in}{1.121243in}}%
\pgfusepath{stroke}%
\end{pgfscope}%
\begin{pgfscope}%
\pgfsetrectcap%
\pgfsetmiterjoin%
\pgfsetlinewidth{0.803000pt}%
\definecolor{currentstroke}{rgb}{0.000000,0.000000,0.000000}%
\pgfsetstrokecolor{currentstroke}%
\pgfsetdash{}{0pt}%
\pgfpathmoveto{\pgfqpoint{4.106058in}{0.878085in}}%
\pgfpathlineto{\pgfqpoint{4.930526in}{0.878085in}}%
\pgfusepath{stroke}%
\end{pgfscope}%
\begin{pgfscope}%
\pgfsetrectcap%
\pgfsetmiterjoin%
\pgfsetlinewidth{0.803000pt}%
\definecolor{currentstroke}{rgb}{0.000000,0.000000,0.000000}%
\pgfsetstrokecolor{currentstroke}%
\pgfsetdash{}{0pt}%
\pgfpathmoveto{\pgfqpoint{4.106058in}{1.121243in}}%
\pgfpathlineto{\pgfqpoint{4.930526in}{1.121243in}}%
\pgfusepath{stroke}%
\end{pgfscope}%
\begin{pgfscope}%
\definecolor{textcolor}{rgb}{0.000000,0.000000,0.000000}%
\pgfsetstrokecolor{textcolor}%
\pgfsetfillcolor{textcolor}%
\pgftext[x=4.518292in,y=1.204576in,,base]{\color{textcolor}\rmfamily\fontsize{11.000000}{13.200000}\selectfont Gan}%
\end{pgfscope}%
\begin{pgfscope}%
\pgfsetbuttcap%
\pgfsetmiterjoin%
\definecolor{currentfill}{rgb}{1.000000,1.000000,1.000000}%
\pgfsetfillcolor{currentfill}%
\pgfsetlinewidth{0.000000pt}%
\definecolor{currentstroke}{rgb}{0.000000,0.000000,0.000000}%
\pgfsetstrokecolor{currentstroke}%
\pgfsetstrokeopacity{0.000000}%
\pgfsetdash{}{0pt}%
\pgfpathmoveto{\pgfqpoint{5.095420in}{0.878085in}}%
\pgfpathlineto{\pgfqpoint{5.919888in}{0.878085in}}%
\pgfpathlineto{\pgfqpoint{5.919888in}{1.121243in}}%
\pgfpathlineto{\pgfqpoint{5.095420in}{1.121243in}}%
\pgfpathlineto{\pgfqpoint{5.095420in}{0.878085in}}%
\pgfpathclose%
\pgfusepath{fill}%
\end{pgfscope}%
\begin{pgfscope}%
\pgfpathrectangle{\pgfqpoint{5.095420in}{0.878085in}}{\pgfqpoint{0.824468in}{0.243158in}}%
\pgfusepath{clip}%
\pgfsetbuttcap%
\pgfsetmiterjoin%
\definecolor{currentfill}{rgb}{0.121569,0.466667,0.705882}%
\pgfsetfillcolor{currentfill}%
\pgfsetfillopacity{0.500000}%
\pgfsetlinewidth{1.003750pt}%
\definecolor{currentstroke}{rgb}{0.000000,0.000000,0.000000}%
\pgfsetstrokecolor{currentstroke}%
\pgfsetdash{}{0pt}%
\pgfpathmoveto{\pgfqpoint{5.132895in}{0.878085in}}%
\pgfpathlineto{\pgfqpoint{5.282799in}{0.878085in}}%
\pgfpathlineto{\pgfqpoint{5.282799in}{0.878814in}}%
\pgfpathlineto{\pgfqpoint{5.132895in}{0.878814in}}%
\pgfpathlineto{\pgfqpoint{5.132895in}{0.878085in}}%
\pgfpathclose%
\pgfusepath{stroke,fill}%
\end{pgfscope}%
\begin{pgfscope}%
\pgfpathrectangle{\pgfqpoint{5.095420in}{0.878085in}}{\pgfqpoint{0.824468in}{0.243158in}}%
\pgfusepath{clip}%
\pgfsetbuttcap%
\pgfsetmiterjoin%
\definecolor{currentfill}{rgb}{0.121569,0.466667,0.705882}%
\pgfsetfillcolor{currentfill}%
\pgfsetfillopacity{0.500000}%
\pgfsetlinewidth{1.003750pt}%
\definecolor{currentstroke}{rgb}{0.000000,0.000000,0.000000}%
\pgfsetstrokecolor{currentstroke}%
\pgfsetdash{}{0pt}%
\pgfpathmoveto{\pgfqpoint{5.282799in}{0.878085in}}%
\pgfpathlineto{\pgfqpoint{5.432702in}{0.878085in}}%
\pgfpathlineto{\pgfqpoint{5.432702in}{0.879057in}}%
\pgfpathlineto{\pgfqpoint{5.282799in}{0.879057in}}%
\pgfpathlineto{\pgfqpoint{5.282799in}{0.878085in}}%
\pgfpathclose%
\pgfusepath{stroke,fill}%
\end{pgfscope}%
\begin{pgfscope}%
\pgfpathrectangle{\pgfqpoint{5.095420in}{0.878085in}}{\pgfqpoint{0.824468in}{0.243158in}}%
\pgfusepath{clip}%
\pgfsetbuttcap%
\pgfsetmiterjoin%
\definecolor{currentfill}{rgb}{0.121569,0.466667,0.705882}%
\pgfsetfillcolor{currentfill}%
\pgfsetfillopacity{0.500000}%
\pgfsetlinewidth{1.003750pt}%
\definecolor{currentstroke}{rgb}{0.000000,0.000000,0.000000}%
\pgfsetstrokecolor{currentstroke}%
\pgfsetdash{}{0pt}%
\pgfpathmoveto{\pgfqpoint{5.432702in}{0.878085in}}%
\pgfpathlineto{\pgfqpoint{5.582605in}{0.878085in}}%
\pgfpathlineto{\pgfqpoint{5.582605in}{0.878085in}}%
\pgfpathlineto{\pgfqpoint{5.432702in}{0.878085in}}%
\pgfpathlineto{\pgfqpoint{5.432702in}{0.878085in}}%
\pgfpathclose%
\pgfusepath{stroke,fill}%
\end{pgfscope}%
\begin{pgfscope}%
\pgfpathrectangle{\pgfqpoint{5.095420in}{0.878085in}}{\pgfqpoint{0.824468in}{0.243158in}}%
\pgfusepath{clip}%
\pgfsetbuttcap%
\pgfsetmiterjoin%
\definecolor{currentfill}{rgb}{0.121569,0.466667,0.705882}%
\pgfsetfillcolor{currentfill}%
\pgfsetfillopacity{0.500000}%
\pgfsetlinewidth{1.003750pt}%
\definecolor{currentstroke}{rgb}{0.000000,0.000000,0.000000}%
\pgfsetstrokecolor{currentstroke}%
\pgfsetdash{}{0pt}%
\pgfpathmoveto{\pgfqpoint{5.582605in}{0.878085in}}%
\pgfpathlineto{\pgfqpoint{5.732509in}{0.878085in}}%
\pgfpathlineto{\pgfqpoint{5.732509in}{0.878206in}}%
\pgfpathlineto{\pgfqpoint{5.582605in}{0.878206in}}%
\pgfpathlineto{\pgfqpoint{5.582605in}{0.878085in}}%
\pgfpathclose%
\pgfusepath{stroke,fill}%
\end{pgfscope}%
\begin{pgfscope}%
\pgfpathrectangle{\pgfqpoint{5.095420in}{0.878085in}}{\pgfqpoint{0.824468in}{0.243158in}}%
\pgfusepath{clip}%
\pgfsetbuttcap%
\pgfsetmiterjoin%
\definecolor{currentfill}{rgb}{0.121569,0.466667,0.705882}%
\pgfsetfillcolor{currentfill}%
\pgfsetfillopacity{0.500000}%
\pgfsetlinewidth{1.003750pt}%
\definecolor{currentstroke}{rgb}{0.000000,0.000000,0.000000}%
\pgfsetstrokecolor{currentstroke}%
\pgfsetdash{}{0pt}%
\pgfpathmoveto{\pgfqpoint{5.732509in}{0.878085in}}%
\pgfpathlineto{\pgfqpoint{5.882412in}{0.878085in}}%
\pgfpathlineto{\pgfqpoint{5.882412in}{0.878571in}}%
\pgfpathlineto{\pgfqpoint{5.732509in}{0.878571in}}%
\pgfpathlineto{\pgfqpoint{5.732509in}{0.878085in}}%
\pgfpathclose%
\pgfusepath{stroke,fill}%
\end{pgfscope}%
\begin{pgfscope}%
\pgfsetrectcap%
\pgfsetmiterjoin%
\pgfsetlinewidth{0.803000pt}%
\definecolor{currentstroke}{rgb}{0.000000,0.000000,0.000000}%
\pgfsetstrokecolor{currentstroke}%
\pgfsetdash{}{0pt}%
\pgfpathmoveto{\pgfqpoint{5.095420in}{0.878085in}}%
\pgfpathlineto{\pgfqpoint{5.095420in}{1.121243in}}%
\pgfusepath{stroke}%
\end{pgfscope}%
\begin{pgfscope}%
\pgfsetrectcap%
\pgfsetmiterjoin%
\pgfsetlinewidth{0.803000pt}%
\definecolor{currentstroke}{rgb}{0.000000,0.000000,0.000000}%
\pgfsetstrokecolor{currentstroke}%
\pgfsetdash{}{0pt}%
\pgfpathmoveto{\pgfqpoint{5.919888in}{0.878085in}}%
\pgfpathlineto{\pgfqpoint{5.919888in}{1.121243in}}%
\pgfusepath{stroke}%
\end{pgfscope}%
\begin{pgfscope}%
\pgfsetrectcap%
\pgfsetmiterjoin%
\pgfsetlinewidth{0.803000pt}%
\definecolor{currentstroke}{rgb}{0.000000,0.000000,0.000000}%
\pgfsetstrokecolor{currentstroke}%
\pgfsetdash{}{0pt}%
\pgfpathmoveto{\pgfqpoint{5.095420in}{0.878085in}}%
\pgfpathlineto{\pgfqpoint{5.919888in}{0.878085in}}%
\pgfusepath{stroke}%
\end{pgfscope}%
\begin{pgfscope}%
\pgfsetrectcap%
\pgfsetmiterjoin%
\pgfsetlinewidth{0.803000pt}%
\definecolor{currentstroke}{rgb}{0.000000,0.000000,0.000000}%
\pgfsetstrokecolor{currentstroke}%
\pgfsetdash{}{0pt}%
\pgfpathmoveto{\pgfqpoint{5.095420in}{1.121243in}}%
\pgfpathlineto{\pgfqpoint{5.919888in}{1.121243in}}%
\pgfusepath{stroke}%
\end{pgfscope}%
\begin{pgfscope}%
\definecolor{textcolor}{rgb}{0.000000,0.000000,0.000000}%
\pgfsetstrokecolor{textcolor}%
\pgfsetfillcolor{textcolor}%
\pgftext[x=5.507654in,y=1.204576in,,base]{\color{textcolor}\rmfamily\fontsize{11.000000}{13.200000}\selectfont Magnolia}%
\end{pgfscope}%
\begin{pgfscope}%
\pgfsetbuttcap%
\pgfsetmiterjoin%
\definecolor{currentfill}{rgb}{1.000000,1.000000,1.000000}%
\pgfsetfillcolor{currentfill}%
\pgfsetlinewidth{0.000000pt}%
\definecolor{currentstroke}{rgb}{0.000000,0.000000,0.000000}%
\pgfsetstrokecolor{currentstroke}%
\pgfsetstrokeopacity{0.000000}%
\pgfsetdash{}{0pt}%
\pgfpathmoveto{\pgfqpoint{6.084781in}{0.878085in}}%
\pgfpathlineto{\pgfqpoint{6.909249in}{0.878085in}}%
\pgfpathlineto{\pgfqpoint{6.909249in}{1.121243in}}%
\pgfpathlineto{\pgfqpoint{6.084781in}{1.121243in}}%
\pgfpathlineto{\pgfqpoint{6.084781in}{0.878085in}}%
\pgfpathclose%
\pgfusepath{fill}%
\end{pgfscope}%
\begin{pgfscope}%
\pgfpathrectangle{\pgfqpoint{6.084781in}{0.878085in}}{\pgfqpoint{0.824468in}{0.243158in}}%
\pgfusepath{clip}%
\pgfsetbuttcap%
\pgfsetmiterjoin%
\definecolor{currentfill}{rgb}{0.121569,0.466667,0.705882}%
\pgfsetfillcolor{currentfill}%
\pgfsetfillopacity{0.500000}%
\pgfsetlinewidth{1.003750pt}%
\definecolor{currentstroke}{rgb}{0.000000,0.000000,0.000000}%
\pgfsetstrokecolor{currentstroke}%
\pgfsetdash{}{0pt}%
\pgfpathmoveto{\pgfqpoint{6.122257in}{0.878085in}}%
\pgfpathlineto{\pgfqpoint{6.272160in}{0.878085in}}%
\pgfpathlineto{\pgfqpoint{6.272160in}{0.879787in}}%
\pgfpathlineto{\pgfqpoint{6.122257in}{0.879787in}}%
\pgfpathlineto{\pgfqpoint{6.122257in}{0.878085in}}%
\pgfpathclose%
\pgfusepath{stroke,fill}%
\end{pgfscope}%
\begin{pgfscope}%
\pgfpathrectangle{\pgfqpoint{6.084781in}{0.878085in}}{\pgfqpoint{0.824468in}{0.243158in}}%
\pgfusepath{clip}%
\pgfsetbuttcap%
\pgfsetmiterjoin%
\definecolor{currentfill}{rgb}{0.121569,0.466667,0.705882}%
\pgfsetfillcolor{currentfill}%
\pgfsetfillopacity{0.500000}%
\pgfsetlinewidth{1.003750pt}%
\definecolor{currentstroke}{rgb}{0.000000,0.000000,0.000000}%
\pgfsetstrokecolor{currentstroke}%
\pgfsetdash{}{0pt}%
\pgfpathmoveto{\pgfqpoint{6.272160in}{0.878085in}}%
\pgfpathlineto{\pgfqpoint{6.422064in}{0.878085in}}%
\pgfpathlineto{\pgfqpoint{6.422064in}{0.879057in}}%
\pgfpathlineto{\pgfqpoint{6.272160in}{0.879057in}}%
\pgfpathlineto{\pgfqpoint{6.272160in}{0.878085in}}%
\pgfpathclose%
\pgfusepath{stroke,fill}%
\end{pgfscope}%
\begin{pgfscope}%
\pgfpathrectangle{\pgfqpoint{6.084781in}{0.878085in}}{\pgfqpoint{0.824468in}{0.243158in}}%
\pgfusepath{clip}%
\pgfsetbuttcap%
\pgfsetmiterjoin%
\definecolor{currentfill}{rgb}{0.121569,0.466667,0.705882}%
\pgfsetfillcolor{currentfill}%
\pgfsetfillopacity{0.500000}%
\pgfsetlinewidth{1.003750pt}%
\definecolor{currentstroke}{rgb}{0.000000,0.000000,0.000000}%
\pgfsetstrokecolor{currentstroke}%
\pgfsetdash{}{0pt}%
\pgfpathmoveto{\pgfqpoint{6.422064in}{0.878085in}}%
\pgfpathlineto{\pgfqpoint{6.571967in}{0.878085in}}%
\pgfpathlineto{\pgfqpoint{6.571967in}{0.878571in}}%
\pgfpathlineto{\pgfqpoint{6.422064in}{0.878571in}}%
\pgfpathlineto{\pgfqpoint{6.422064in}{0.878085in}}%
\pgfpathclose%
\pgfusepath{stroke,fill}%
\end{pgfscope}%
\begin{pgfscope}%
\pgfpathrectangle{\pgfqpoint{6.084781in}{0.878085in}}{\pgfqpoint{0.824468in}{0.243158in}}%
\pgfusepath{clip}%
\pgfsetbuttcap%
\pgfsetmiterjoin%
\definecolor{currentfill}{rgb}{0.121569,0.466667,0.705882}%
\pgfsetfillcolor{currentfill}%
\pgfsetfillopacity{0.500000}%
\pgfsetlinewidth{1.003750pt}%
\definecolor{currentstroke}{rgb}{0.000000,0.000000,0.000000}%
\pgfsetstrokecolor{currentstroke}%
\pgfsetdash{}{0pt}%
\pgfpathmoveto{\pgfqpoint{6.571967in}{0.878085in}}%
\pgfpathlineto{\pgfqpoint{6.721870in}{0.878085in}}%
\pgfpathlineto{\pgfqpoint{6.721870in}{0.878206in}}%
\pgfpathlineto{\pgfqpoint{6.571967in}{0.878206in}}%
\pgfpathlineto{\pgfqpoint{6.571967in}{0.878085in}}%
\pgfpathclose%
\pgfusepath{stroke,fill}%
\end{pgfscope}%
\begin{pgfscope}%
\pgfpathrectangle{\pgfqpoint{6.084781in}{0.878085in}}{\pgfqpoint{0.824468in}{0.243158in}}%
\pgfusepath{clip}%
\pgfsetbuttcap%
\pgfsetmiterjoin%
\definecolor{currentfill}{rgb}{0.121569,0.466667,0.705882}%
\pgfsetfillcolor{currentfill}%
\pgfsetfillopacity{0.500000}%
\pgfsetlinewidth{1.003750pt}%
\definecolor{currentstroke}{rgb}{0.000000,0.000000,0.000000}%
\pgfsetstrokecolor{currentstroke}%
\pgfsetdash{}{0pt}%
\pgfpathmoveto{\pgfqpoint{6.721870in}{0.878085in}}%
\pgfpathlineto{\pgfqpoint{6.871774in}{0.878085in}}%
\pgfpathlineto{\pgfqpoint{6.871774in}{0.878085in}}%
\pgfpathlineto{\pgfqpoint{6.721870in}{0.878085in}}%
\pgfpathlineto{\pgfqpoint{6.721870in}{0.878085in}}%
\pgfpathclose%
\pgfusepath{stroke,fill}%
\end{pgfscope}%
\begin{pgfscope}%
\pgfsetrectcap%
\pgfsetmiterjoin%
\pgfsetlinewidth{0.803000pt}%
\definecolor{currentstroke}{rgb}{0.000000,0.000000,0.000000}%
\pgfsetstrokecolor{currentstroke}%
\pgfsetdash{}{0pt}%
\pgfpathmoveto{\pgfqpoint{6.084781in}{0.878085in}}%
\pgfpathlineto{\pgfqpoint{6.084781in}{1.121243in}}%
\pgfusepath{stroke}%
\end{pgfscope}%
\begin{pgfscope}%
\pgfsetrectcap%
\pgfsetmiterjoin%
\pgfsetlinewidth{0.803000pt}%
\definecolor{currentstroke}{rgb}{0.000000,0.000000,0.000000}%
\pgfsetstrokecolor{currentstroke}%
\pgfsetdash{}{0pt}%
\pgfpathmoveto{\pgfqpoint{6.909249in}{0.878085in}}%
\pgfpathlineto{\pgfqpoint{6.909249in}{1.121243in}}%
\pgfusepath{stroke}%
\end{pgfscope}%
\begin{pgfscope}%
\pgfsetrectcap%
\pgfsetmiterjoin%
\pgfsetlinewidth{0.803000pt}%
\definecolor{currentstroke}{rgb}{0.000000,0.000000,0.000000}%
\pgfsetstrokecolor{currentstroke}%
\pgfsetdash{}{0pt}%
\pgfpathmoveto{\pgfqpoint{6.084781in}{0.878085in}}%
\pgfpathlineto{\pgfqpoint{6.909249in}{0.878085in}}%
\pgfusepath{stroke}%
\end{pgfscope}%
\begin{pgfscope}%
\pgfsetrectcap%
\pgfsetmiterjoin%
\pgfsetlinewidth{0.803000pt}%
\definecolor{currentstroke}{rgb}{0.000000,0.000000,0.000000}%
\pgfsetstrokecolor{currentstroke}%
\pgfsetdash{}{0pt}%
\pgfpathmoveto{\pgfqpoint{6.084781in}{1.121243in}}%
\pgfpathlineto{\pgfqpoint{6.909249in}{1.121243in}}%
\pgfusepath{stroke}%
\end{pgfscope}%
\begin{pgfscope}%
\definecolor{textcolor}{rgb}{0.000000,0.000000,0.000000}%
\pgfsetstrokecolor{textcolor}%
\pgfsetfillcolor{textcolor}%
\pgftext[x=6.497015in,y=1.204576in,,base]{\color{textcolor}\rmfamily\fontsize{11.000000}{13.200000}\selectfont Suravenir}%
\end{pgfscope}%
\begin{pgfscope}%
\pgfsetbuttcap%
\pgfsetmiterjoin%
\definecolor{currentfill}{rgb}{1.000000,1.000000,1.000000}%
\pgfsetfillcolor{currentfill}%
\pgfsetlinewidth{0.000000pt}%
\definecolor{currentstroke}{rgb}{0.000000,0.000000,0.000000}%
\pgfsetstrokecolor{currentstroke}%
\pgfsetstrokeopacity{0.000000}%
\pgfsetdash{}{0pt}%
\pgfpathmoveto{\pgfqpoint{7.074143in}{0.878085in}}%
\pgfpathlineto{\pgfqpoint{7.898611in}{0.878085in}}%
\pgfpathlineto{\pgfqpoint{7.898611in}{1.121243in}}%
\pgfpathlineto{\pgfqpoint{7.074143in}{1.121243in}}%
\pgfpathlineto{\pgfqpoint{7.074143in}{0.878085in}}%
\pgfpathclose%
\pgfusepath{fill}%
\end{pgfscope}%
\begin{pgfscope}%
\pgfpathrectangle{\pgfqpoint{7.074143in}{0.878085in}}{\pgfqpoint{0.824468in}{0.243158in}}%
\pgfusepath{clip}%
\pgfsetbuttcap%
\pgfsetmiterjoin%
\definecolor{currentfill}{rgb}{0.121569,0.466667,0.705882}%
\pgfsetfillcolor{currentfill}%
\pgfsetfillopacity{0.500000}%
\pgfsetlinewidth{1.003750pt}%
\definecolor{currentstroke}{rgb}{0.000000,0.000000,0.000000}%
\pgfsetstrokecolor{currentstroke}%
\pgfsetdash{}{0pt}%
\pgfpathmoveto{\pgfqpoint{7.111619in}{0.878085in}}%
\pgfpathlineto{\pgfqpoint{7.261522in}{0.878085in}}%
\pgfpathlineto{\pgfqpoint{7.261522in}{0.880638in}}%
\pgfpathlineto{\pgfqpoint{7.111619in}{0.880638in}}%
\pgfpathlineto{\pgfqpoint{7.111619in}{0.878085in}}%
\pgfpathclose%
\pgfusepath{stroke,fill}%
\end{pgfscope}%
\begin{pgfscope}%
\pgfpathrectangle{\pgfqpoint{7.074143in}{0.878085in}}{\pgfqpoint{0.824468in}{0.243158in}}%
\pgfusepath{clip}%
\pgfsetbuttcap%
\pgfsetmiterjoin%
\definecolor{currentfill}{rgb}{0.121569,0.466667,0.705882}%
\pgfsetfillcolor{currentfill}%
\pgfsetfillopacity{0.500000}%
\pgfsetlinewidth{1.003750pt}%
\definecolor{currentstroke}{rgb}{0.000000,0.000000,0.000000}%
\pgfsetstrokecolor{currentstroke}%
\pgfsetdash{}{0pt}%
\pgfpathmoveto{\pgfqpoint{7.261522in}{0.878085in}}%
\pgfpathlineto{\pgfqpoint{7.411425in}{0.878085in}}%
\pgfpathlineto{\pgfqpoint{7.411425in}{0.879301in}}%
\pgfpathlineto{\pgfqpoint{7.261522in}{0.879301in}}%
\pgfpathlineto{\pgfqpoint{7.261522in}{0.878085in}}%
\pgfpathclose%
\pgfusepath{stroke,fill}%
\end{pgfscope}%
\begin{pgfscope}%
\pgfpathrectangle{\pgfqpoint{7.074143in}{0.878085in}}{\pgfqpoint{0.824468in}{0.243158in}}%
\pgfusepath{clip}%
\pgfsetbuttcap%
\pgfsetmiterjoin%
\definecolor{currentfill}{rgb}{0.121569,0.466667,0.705882}%
\pgfsetfillcolor{currentfill}%
\pgfsetfillopacity{0.500000}%
\pgfsetlinewidth{1.003750pt}%
\definecolor{currentstroke}{rgb}{0.000000,0.000000,0.000000}%
\pgfsetstrokecolor{currentstroke}%
\pgfsetdash{}{0pt}%
\pgfpathmoveto{\pgfqpoint{7.411425in}{0.878085in}}%
\pgfpathlineto{\pgfqpoint{7.561329in}{0.878085in}}%
\pgfpathlineto{\pgfqpoint{7.561329in}{0.878571in}}%
\pgfpathlineto{\pgfqpoint{7.411425in}{0.878571in}}%
\pgfpathlineto{\pgfqpoint{7.411425in}{0.878085in}}%
\pgfpathclose%
\pgfusepath{stroke,fill}%
\end{pgfscope}%
\begin{pgfscope}%
\pgfpathrectangle{\pgfqpoint{7.074143in}{0.878085in}}{\pgfqpoint{0.824468in}{0.243158in}}%
\pgfusepath{clip}%
\pgfsetbuttcap%
\pgfsetmiterjoin%
\definecolor{currentfill}{rgb}{0.121569,0.466667,0.705882}%
\pgfsetfillcolor{currentfill}%
\pgfsetfillopacity{0.500000}%
\pgfsetlinewidth{1.003750pt}%
\definecolor{currentstroke}{rgb}{0.000000,0.000000,0.000000}%
\pgfsetstrokecolor{currentstroke}%
\pgfsetdash{}{0pt}%
\pgfpathmoveto{\pgfqpoint{7.561329in}{0.878085in}}%
\pgfpathlineto{\pgfqpoint{7.711232in}{0.878085in}}%
\pgfpathlineto{\pgfqpoint{7.711232in}{0.878206in}}%
\pgfpathlineto{\pgfqpoint{7.561329in}{0.878206in}}%
\pgfpathlineto{\pgfqpoint{7.561329in}{0.878085in}}%
\pgfpathclose%
\pgfusepath{stroke,fill}%
\end{pgfscope}%
\begin{pgfscope}%
\pgfpathrectangle{\pgfqpoint{7.074143in}{0.878085in}}{\pgfqpoint{0.824468in}{0.243158in}}%
\pgfusepath{clip}%
\pgfsetbuttcap%
\pgfsetmiterjoin%
\definecolor{currentfill}{rgb}{0.121569,0.466667,0.705882}%
\pgfsetfillcolor{currentfill}%
\pgfsetfillopacity{0.500000}%
\pgfsetlinewidth{1.003750pt}%
\definecolor{currentstroke}{rgb}{0.000000,0.000000,0.000000}%
\pgfsetstrokecolor{currentstroke}%
\pgfsetdash{}{0pt}%
\pgfpathmoveto{\pgfqpoint{7.711232in}{0.878085in}}%
\pgfpathlineto{\pgfqpoint{7.861135in}{0.878085in}}%
\pgfpathlineto{\pgfqpoint{7.861135in}{0.879908in}}%
\pgfpathlineto{\pgfqpoint{7.711232in}{0.879908in}}%
\pgfpathlineto{\pgfqpoint{7.711232in}{0.878085in}}%
\pgfpathclose%
\pgfusepath{stroke,fill}%
\end{pgfscope}%
\begin{pgfscope}%
\pgfsetrectcap%
\pgfsetmiterjoin%
\pgfsetlinewidth{0.803000pt}%
\definecolor{currentstroke}{rgb}{0.000000,0.000000,0.000000}%
\pgfsetstrokecolor{currentstroke}%
\pgfsetdash{}{0pt}%
\pgfpathmoveto{\pgfqpoint{7.074143in}{0.878085in}}%
\pgfpathlineto{\pgfqpoint{7.074143in}{1.121243in}}%
\pgfusepath{stroke}%
\end{pgfscope}%
\begin{pgfscope}%
\pgfsetrectcap%
\pgfsetmiterjoin%
\pgfsetlinewidth{0.803000pt}%
\definecolor{currentstroke}{rgb}{0.000000,0.000000,0.000000}%
\pgfsetstrokecolor{currentstroke}%
\pgfsetdash{}{0pt}%
\pgfpathmoveto{\pgfqpoint{7.898611in}{0.878085in}}%
\pgfpathlineto{\pgfqpoint{7.898611in}{1.121243in}}%
\pgfusepath{stroke}%
\end{pgfscope}%
\begin{pgfscope}%
\pgfsetrectcap%
\pgfsetmiterjoin%
\pgfsetlinewidth{0.803000pt}%
\definecolor{currentstroke}{rgb}{0.000000,0.000000,0.000000}%
\pgfsetstrokecolor{currentstroke}%
\pgfsetdash{}{0pt}%
\pgfpathmoveto{\pgfqpoint{7.074143in}{0.878085in}}%
\pgfpathlineto{\pgfqpoint{7.898611in}{0.878085in}}%
\pgfusepath{stroke}%
\end{pgfscope}%
\begin{pgfscope}%
\pgfsetrectcap%
\pgfsetmiterjoin%
\pgfsetlinewidth{0.803000pt}%
\definecolor{currentstroke}{rgb}{0.000000,0.000000,0.000000}%
\pgfsetstrokecolor{currentstroke}%
\pgfsetdash{}{0pt}%
\pgfpathmoveto{\pgfqpoint{7.074143in}{1.121243in}}%
\pgfpathlineto{\pgfqpoint{7.898611in}{1.121243in}}%
\pgfusepath{stroke}%
\end{pgfscope}%
\begin{pgfscope}%
\definecolor{textcolor}{rgb}{0.000000,0.000000,0.000000}%
\pgfsetstrokecolor{textcolor}%
\pgfsetfillcolor{textcolor}%
\pgftext[x=7.486377in,y=1.204576in,,base]{\color{textcolor}\rmfamily\fontsize{11.000000}{13.200000}\selectfont Assur ...}%
\end{pgfscope}%
\begin{pgfscope}%
\pgfsetbuttcap%
\pgfsetmiterjoin%
\definecolor{currentfill}{rgb}{1.000000,1.000000,1.000000}%
\pgfsetfillcolor{currentfill}%
\pgfsetlinewidth{0.000000pt}%
\definecolor{currentstroke}{rgb}{0.000000,0.000000,0.000000}%
\pgfsetstrokecolor{currentstroke}%
\pgfsetstrokeopacity{0.000000}%
\pgfsetdash{}{0pt}%
\pgfpathmoveto{\pgfqpoint{0.148611in}{0.148611in}}%
\pgfpathlineto{\pgfqpoint{0.973079in}{0.148611in}}%
\pgfpathlineto{\pgfqpoint{0.973079in}{0.391769in}}%
\pgfpathlineto{\pgfqpoint{0.148611in}{0.391769in}}%
\pgfpathlineto{\pgfqpoint{0.148611in}{0.148611in}}%
\pgfpathclose%
\pgfusepath{fill}%
\end{pgfscope}%
\begin{pgfscope}%
\pgfpathrectangle{\pgfqpoint{0.148611in}{0.148611in}}{\pgfqpoint{0.824468in}{0.243158in}}%
\pgfusepath{clip}%
\pgfsetbuttcap%
\pgfsetmiterjoin%
\definecolor{currentfill}{rgb}{0.121569,0.466667,0.705882}%
\pgfsetfillcolor{currentfill}%
\pgfsetfillopacity{0.500000}%
\pgfsetlinewidth{1.003750pt}%
\definecolor{currentstroke}{rgb}{0.000000,0.000000,0.000000}%
\pgfsetstrokecolor{currentstroke}%
\pgfsetdash{}{0pt}%
\pgfpathmoveto{\pgfqpoint{0.186087in}{0.148611in}}%
\pgfpathlineto{\pgfqpoint{0.335990in}{0.148611in}}%
\pgfpathlineto{\pgfqpoint{0.335990in}{0.150435in}}%
\pgfpathlineto{\pgfqpoint{0.186087in}{0.150435in}}%
\pgfpathlineto{\pgfqpoint{0.186087in}{0.148611in}}%
\pgfpathclose%
\pgfusepath{stroke,fill}%
\end{pgfscope}%
\begin{pgfscope}%
\pgfpathrectangle{\pgfqpoint{0.148611in}{0.148611in}}{\pgfqpoint{0.824468in}{0.243158in}}%
\pgfusepath{clip}%
\pgfsetbuttcap%
\pgfsetmiterjoin%
\definecolor{currentfill}{rgb}{0.121569,0.466667,0.705882}%
\pgfsetfillcolor{currentfill}%
\pgfsetfillopacity{0.500000}%
\pgfsetlinewidth{1.003750pt}%
\definecolor{currentstroke}{rgb}{0.000000,0.000000,0.000000}%
\pgfsetstrokecolor{currentstroke}%
\pgfsetdash{}{0pt}%
\pgfpathmoveto{\pgfqpoint{0.335990in}{0.148611in}}%
\pgfpathlineto{\pgfqpoint{0.485894in}{0.148611in}}%
\pgfpathlineto{\pgfqpoint{0.485894in}{0.149705in}}%
\pgfpathlineto{\pgfqpoint{0.335990in}{0.149705in}}%
\pgfpathlineto{\pgfqpoint{0.335990in}{0.148611in}}%
\pgfpathclose%
\pgfusepath{stroke,fill}%
\end{pgfscope}%
\begin{pgfscope}%
\pgfpathrectangle{\pgfqpoint{0.148611in}{0.148611in}}{\pgfqpoint{0.824468in}{0.243158in}}%
\pgfusepath{clip}%
\pgfsetbuttcap%
\pgfsetmiterjoin%
\definecolor{currentfill}{rgb}{0.121569,0.466667,0.705882}%
\pgfsetfillcolor{currentfill}%
\pgfsetfillopacity{0.500000}%
\pgfsetlinewidth{1.003750pt}%
\definecolor{currentstroke}{rgb}{0.000000,0.000000,0.000000}%
\pgfsetstrokecolor{currentstroke}%
\pgfsetdash{}{0pt}%
\pgfpathmoveto{\pgfqpoint{0.485894in}{0.148611in}}%
\pgfpathlineto{\pgfqpoint{0.635797in}{0.148611in}}%
\pgfpathlineto{\pgfqpoint{0.635797in}{0.148733in}}%
\pgfpathlineto{\pgfqpoint{0.485894in}{0.148733in}}%
\pgfpathlineto{\pgfqpoint{0.485894in}{0.148611in}}%
\pgfpathclose%
\pgfusepath{stroke,fill}%
\end{pgfscope}%
\begin{pgfscope}%
\pgfpathrectangle{\pgfqpoint{0.148611in}{0.148611in}}{\pgfqpoint{0.824468in}{0.243158in}}%
\pgfusepath{clip}%
\pgfsetbuttcap%
\pgfsetmiterjoin%
\definecolor{currentfill}{rgb}{0.121569,0.466667,0.705882}%
\pgfsetfillcolor{currentfill}%
\pgfsetfillopacity{0.500000}%
\pgfsetlinewidth{1.003750pt}%
\definecolor{currentstroke}{rgb}{0.000000,0.000000,0.000000}%
\pgfsetstrokecolor{currentstroke}%
\pgfsetdash{}{0pt}%
\pgfpathmoveto{\pgfqpoint{0.635797in}{0.148611in}}%
\pgfpathlineto{\pgfqpoint{0.785700in}{0.148611in}}%
\pgfpathlineto{\pgfqpoint{0.785700in}{0.148611in}}%
\pgfpathlineto{\pgfqpoint{0.635797in}{0.148611in}}%
\pgfpathlineto{\pgfqpoint{0.635797in}{0.148611in}}%
\pgfpathclose%
\pgfusepath{stroke,fill}%
\end{pgfscope}%
\begin{pgfscope}%
\pgfpathrectangle{\pgfqpoint{0.148611in}{0.148611in}}{\pgfqpoint{0.824468in}{0.243158in}}%
\pgfusepath{clip}%
\pgfsetbuttcap%
\pgfsetmiterjoin%
\definecolor{currentfill}{rgb}{0.121569,0.466667,0.705882}%
\pgfsetfillcolor{currentfill}%
\pgfsetfillopacity{0.500000}%
\pgfsetlinewidth{1.003750pt}%
\definecolor{currentstroke}{rgb}{0.000000,0.000000,0.000000}%
\pgfsetstrokecolor{currentstroke}%
\pgfsetdash{}{0pt}%
\pgfpathmoveto{\pgfqpoint{0.785700in}{0.148611in}}%
\pgfpathlineto{\pgfqpoint{0.935603in}{0.148611in}}%
\pgfpathlineto{\pgfqpoint{0.935603in}{0.148611in}}%
\pgfpathlineto{\pgfqpoint{0.785700in}{0.148611in}}%
\pgfpathlineto{\pgfqpoint{0.785700in}{0.148611in}}%
\pgfpathclose%
\pgfusepath{stroke,fill}%
\end{pgfscope}%
\begin{pgfscope}%
\pgfsetrectcap%
\pgfsetmiterjoin%
\pgfsetlinewidth{0.803000pt}%
\definecolor{currentstroke}{rgb}{0.000000,0.000000,0.000000}%
\pgfsetstrokecolor{currentstroke}%
\pgfsetdash{}{0pt}%
\pgfpathmoveto{\pgfqpoint{0.148611in}{0.148611in}}%
\pgfpathlineto{\pgfqpoint{0.148611in}{0.391769in}}%
\pgfusepath{stroke}%
\end{pgfscope}%
\begin{pgfscope}%
\pgfsetrectcap%
\pgfsetmiterjoin%
\pgfsetlinewidth{0.803000pt}%
\definecolor{currentstroke}{rgb}{0.000000,0.000000,0.000000}%
\pgfsetstrokecolor{currentstroke}%
\pgfsetdash{}{0pt}%
\pgfpathmoveto{\pgfqpoint{0.973079in}{0.148611in}}%
\pgfpathlineto{\pgfqpoint{0.973079in}{0.391769in}}%
\pgfusepath{stroke}%
\end{pgfscope}%
\begin{pgfscope}%
\pgfsetrectcap%
\pgfsetmiterjoin%
\pgfsetlinewidth{0.803000pt}%
\definecolor{currentstroke}{rgb}{0.000000,0.000000,0.000000}%
\pgfsetstrokecolor{currentstroke}%
\pgfsetdash{}{0pt}%
\pgfpathmoveto{\pgfqpoint{0.148611in}{0.148611in}}%
\pgfpathlineto{\pgfqpoint{0.973079in}{0.148611in}}%
\pgfusepath{stroke}%
\end{pgfscope}%
\begin{pgfscope}%
\pgfsetrectcap%
\pgfsetmiterjoin%
\pgfsetlinewidth{0.803000pt}%
\definecolor{currentstroke}{rgb}{0.000000,0.000000,0.000000}%
\pgfsetstrokecolor{currentstroke}%
\pgfsetdash{}{0pt}%
\pgfpathmoveto{\pgfqpoint{0.148611in}{0.391769in}}%
\pgfpathlineto{\pgfqpoint{0.973079in}{0.391769in}}%
\pgfusepath{stroke}%
\end{pgfscope}%
\begin{pgfscope}%
\definecolor{textcolor}{rgb}{0.000000,0.000000,0.000000}%
\pgfsetstrokecolor{textcolor}%
\pgfsetfillcolor{textcolor}%
\pgftext[x=0.560845in,y=0.475102in,,base]{\color{textcolor}\rmfamily\fontsize{11.000000}{13.200000}\selectfont AssurO...}%
\end{pgfscope}%
\begin{pgfscope}%
\pgfsetbuttcap%
\pgfsetmiterjoin%
\definecolor{currentfill}{rgb}{1.000000,1.000000,1.000000}%
\pgfsetfillcolor{currentfill}%
\pgfsetlinewidth{0.000000pt}%
\definecolor{currentstroke}{rgb}{0.000000,0.000000,0.000000}%
\pgfsetstrokecolor{currentstroke}%
\pgfsetstrokeopacity{0.000000}%
\pgfsetdash{}{0pt}%
\pgfpathmoveto{\pgfqpoint{1.137973in}{0.148611in}}%
\pgfpathlineto{\pgfqpoint{1.962441in}{0.148611in}}%
\pgfpathlineto{\pgfqpoint{1.962441in}{0.391769in}}%
\pgfpathlineto{\pgfqpoint{1.137973in}{0.391769in}}%
\pgfpathlineto{\pgfqpoint{1.137973in}{0.148611in}}%
\pgfpathclose%
\pgfusepath{fill}%
\end{pgfscope}%
\begin{pgfscope}%
\pgfpathrectangle{\pgfqpoint{1.137973in}{0.148611in}}{\pgfqpoint{0.824468in}{0.243158in}}%
\pgfusepath{clip}%
\pgfsetbuttcap%
\pgfsetmiterjoin%
\definecolor{currentfill}{rgb}{0.121569,0.466667,0.705882}%
\pgfsetfillcolor{currentfill}%
\pgfsetfillopacity{0.500000}%
\pgfsetlinewidth{1.003750pt}%
\definecolor{currentstroke}{rgb}{0.000000,0.000000,0.000000}%
\pgfsetstrokecolor{currentstroke}%
\pgfsetdash{}{0pt}%
\pgfpathmoveto{\pgfqpoint{1.175449in}{0.148611in}}%
\pgfpathlineto{\pgfqpoint{1.325352in}{0.148611in}}%
\pgfpathlineto{\pgfqpoint{1.325352in}{0.150435in}}%
\pgfpathlineto{\pgfqpoint{1.175449in}{0.150435in}}%
\pgfpathlineto{\pgfqpoint{1.175449in}{0.148611in}}%
\pgfpathclose%
\pgfusepath{stroke,fill}%
\end{pgfscope}%
\begin{pgfscope}%
\pgfpathrectangle{\pgfqpoint{1.137973in}{0.148611in}}{\pgfqpoint{0.824468in}{0.243158in}}%
\pgfusepath{clip}%
\pgfsetbuttcap%
\pgfsetmiterjoin%
\definecolor{currentfill}{rgb}{0.121569,0.466667,0.705882}%
\pgfsetfillcolor{currentfill}%
\pgfsetfillopacity{0.500000}%
\pgfsetlinewidth{1.003750pt}%
\definecolor{currentstroke}{rgb}{0.000000,0.000000,0.000000}%
\pgfsetstrokecolor{currentstroke}%
\pgfsetdash{}{0pt}%
\pgfpathmoveto{\pgfqpoint{1.325352in}{0.148611in}}%
\pgfpathlineto{\pgfqpoint{1.475255in}{0.148611in}}%
\pgfpathlineto{\pgfqpoint{1.475255in}{0.149219in}}%
\pgfpathlineto{\pgfqpoint{1.325352in}{0.149219in}}%
\pgfpathlineto{\pgfqpoint{1.325352in}{0.148611in}}%
\pgfpathclose%
\pgfusepath{stroke,fill}%
\end{pgfscope}%
\begin{pgfscope}%
\pgfpathrectangle{\pgfqpoint{1.137973in}{0.148611in}}{\pgfqpoint{0.824468in}{0.243158in}}%
\pgfusepath{clip}%
\pgfsetbuttcap%
\pgfsetmiterjoin%
\definecolor{currentfill}{rgb}{0.121569,0.466667,0.705882}%
\pgfsetfillcolor{currentfill}%
\pgfsetfillopacity{0.500000}%
\pgfsetlinewidth{1.003750pt}%
\definecolor{currentstroke}{rgb}{0.000000,0.000000,0.000000}%
\pgfsetstrokecolor{currentstroke}%
\pgfsetdash{}{0pt}%
\pgfpathmoveto{\pgfqpoint{1.475255in}{0.148611in}}%
\pgfpathlineto{\pgfqpoint{1.625158in}{0.148611in}}%
\pgfpathlineto{\pgfqpoint{1.625158in}{0.148733in}}%
\pgfpathlineto{\pgfqpoint{1.475255in}{0.148733in}}%
\pgfpathlineto{\pgfqpoint{1.475255in}{0.148611in}}%
\pgfpathclose%
\pgfusepath{stroke,fill}%
\end{pgfscope}%
\begin{pgfscope}%
\pgfpathrectangle{\pgfqpoint{1.137973in}{0.148611in}}{\pgfqpoint{0.824468in}{0.243158in}}%
\pgfusepath{clip}%
\pgfsetbuttcap%
\pgfsetmiterjoin%
\definecolor{currentfill}{rgb}{0.121569,0.466667,0.705882}%
\pgfsetfillcolor{currentfill}%
\pgfsetfillopacity{0.500000}%
\pgfsetlinewidth{1.003750pt}%
\definecolor{currentstroke}{rgb}{0.000000,0.000000,0.000000}%
\pgfsetstrokecolor{currentstroke}%
\pgfsetdash{}{0pt}%
\pgfpathmoveto{\pgfqpoint{1.625158in}{0.148611in}}%
\pgfpathlineto{\pgfqpoint{1.775062in}{0.148611in}}%
\pgfpathlineto{\pgfqpoint{1.775062in}{0.149341in}}%
\pgfpathlineto{\pgfqpoint{1.625158in}{0.149341in}}%
\pgfpathlineto{\pgfqpoint{1.625158in}{0.148611in}}%
\pgfpathclose%
\pgfusepath{stroke,fill}%
\end{pgfscope}%
\begin{pgfscope}%
\pgfpathrectangle{\pgfqpoint{1.137973in}{0.148611in}}{\pgfqpoint{0.824468in}{0.243158in}}%
\pgfusepath{clip}%
\pgfsetbuttcap%
\pgfsetmiterjoin%
\definecolor{currentfill}{rgb}{0.121569,0.466667,0.705882}%
\pgfsetfillcolor{currentfill}%
\pgfsetfillopacity{0.500000}%
\pgfsetlinewidth{1.003750pt}%
\definecolor{currentstroke}{rgb}{0.000000,0.000000,0.000000}%
\pgfsetstrokecolor{currentstroke}%
\pgfsetdash{}{0pt}%
\pgfpathmoveto{\pgfqpoint{1.775062in}{0.148611in}}%
\pgfpathlineto{\pgfqpoint{1.924965in}{0.148611in}}%
\pgfpathlineto{\pgfqpoint{1.924965in}{0.149219in}}%
\pgfpathlineto{\pgfqpoint{1.775062in}{0.149219in}}%
\pgfpathlineto{\pgfqpoint{1.775062in}{0.148611in}}%
\pgfpathclose%
\pgfusepath{stroke,fill}%
\end{pgfscope}%
\begin{pgfscope}%
\pgfsetrectcap%
\pgfsetmiterjoin%
\pgfsetlinewidth{0.803000pt}%
\definecolor{currentstroke}{rgb}{0.000000,0.000000,0.000000}%
\pgfsetstrokecolor{currentstroke}%
\pgfsetdash{}{0pt}%
\pgfpathmoveto{\pgfqpoint{1.137973in}{0.148611in}}%
\pgfpathlineto{\pgfqpoint{1.137973in}{0.391769in}}%
\pgfusepath{stroke}%
\end{pgfscope}%
\begin{pgfscope}%
\pgfsetrectcap%
\pgfsetmiterjoin%
\pgfsetlinewidth{0.803000pt}%
\definecolor{currentstroke}{rgb}{0.000000,0.000000,0.000000}%
\pgfsetstrokecolor{currentstroke}%
\pgfsetdash{}{0pt}%
\pgfpathmoveto{\pgfqpoint{1.962441in}{0.148611in}}%
\pgfpathlineto{\pgfqpoint{1.962441in}{0.391769in}}%
\pgfusepath{stroke}%
\end{pgfscope}%
\begin{pgfscope}%
\pgfsetrectcap%
\pgfsetmiterjoin%
\pgfsetlinewidth{0.803000pt}%
\definecolor{currentstroke}{rgb}{0.000000,0.000000,0.000000}%
\pgfsetstrokecolor{currentstroke}%
\pgfsetdash{}{0pt}%
\pgfpathmoveto{\pgfqpoint{1.137973in}{0.148611in}}%
\pgfpathlineto{\pgfqpoint{1.962441in}{0.148611in}}%
\pgfusepath{stroke}%
\end{pgfscope}%
\begin{pgfscope}%
\pgfsetrectcap%
\pgfsetmiterjoin%
\pgfsetlinewidth{0.803000pt}%
\definecolor{currentstroke}{rgb}{0.000000,0.000000,0.000000}%
\pgfsetstrokecolor{currentstroke}%
\pgfsetdash{}{0pt}%
\pgfpathmoveto{\pgfqpoint{1.137973in}{0.391769in}}%
\pgfpathlineto{\pgfqpoint{1.962441in}{0.391769in}}%
\pgfusepath{stroke}%
\end{pgfscope}%
\begin{pgfscope}%
\definecolor{textcolor}{rgb}{0.000000,0.000000,0.000000}%
\pgfsetstrokecolor{textcolor}%
\pgfsetfillcolor{textcolor}%
\pgftext[x=1.550207in,y=0.475102in,,base]{\color{textcolor}\rmfamily\fontsize{11.000000}{13.200000}\selectfont Carac}%
\end{pgfscope}%
\begin{pgfscope}%
\pgfsetbuttcap%
\pgfsetmiterjoin%
\definecolor{currentfill}{rgb}{1.000000,1.000000,1.000000}%
\pgfsetfillcolor{currentfill}%
\pgfsetlinewidth{0.000000pt}%
\definecolor{currentstroke}{rgb}{0.000000,0.000000,0.000000}%
\pgfsetstrokecolor{currentstroke}%
\pgfsetstrokeopacity{0.000000}%
\pgfsetdash{}{0pt}%
\pgfpathmoveto{\pgfqpoint{2.127335in}{0.148611in}}%
\pgfpathlineto{\pgfqpoint{2.951803in}{0.148611in}}%
\pgfpathlineto{\pgfqpoint{2.951803in}{0.391769in}}%
\pgfpathlineto{\pgfqpoint{2.127335in}{0.391769in}}%
\pgfpathlineto{\pgfqpoint{2.127335in}{0.148611in}}%
\pgfpathclose%
\pgfusepath{fill}%
\end{pgfscope}%
\begin{pgfscope}%
\pgfpathrectangle{\pgfqpoint{2.127335in}{0.148611in}}{\pgfqpoint{0.824468in}{0.243158in}}%
\pgfusepath{clip}%
\pgfsetbuttcap%
\pgfsetmiterjoin%
\definecolor{currentfill}{rgb}{0.121569,0.466667,0.705882}%
\pgfsetfillcolor{currentfill}%
\pgfsetfillopacity{0.500000}%
\pgfsetlinewidth{1.003750pt}%
\definecolor{currentstroke}{rgb}{0.000000,0.000000,0.000000}%
\pgfsetstrokecolor{currentstroke}%
\pgfsetdash{}{0pt}%
\pgfpathmoveto{\pgfqpoint{2.164810in}{0.148611in}}%
\pgfpathlineto{\pgfqpoint{2.314714in}{0.148611in}}%
\pgfpathlineto{\pgfqpoint{2.314714in}{0.148733in}}%
\pgfpathlineto{\pgfqpoint{2.164810in}{0.148733in}}%
\pgfpathlineto{\pgfqpoint{2.164810in}{0.148611in}}%
\pgfpathclose%
\pgfusepath{stroke,fill}%
\end{pgfscope}%
\begin{pgfscope}%
\pgfpathrectangle{\pgfqpoint{2.127335in}{0.148611in}}{\pgfqpoint{0.824468in}{0.243158in}}%
\pgfusepath{clip}%
\pgfsetbuttcap%
\pgfsetmiterjoin%
\definecolor{currentfill}{rgb}{0.121569,0.466667,0.705882}%
\pgfsetfillcolor{currentfill}%
\pgfsetfillopacity{0.500000}%
\pgfsetlinewidth{1.003750pt}%
\definecolor{currentstroke}{rgb}{0.000000,0.000000,0.000000}%
\pgfsetstrokecolor{currentstroke}%
\pgfsetdash{}{0pt}%
\pgfpathmoveto{\pgfqpoint{2.314714in}{0.148611in}}%
\pgfpathlineto{\pgfqpoint{2.464617in}{0.148611in}}%
\pgfpathlineto{\pgfqpoint{2.464617in}{0.148611in}}%
\pgfpathlineto{\pgfqpoint{2.314714in}{0.148611in}}%
\pgfpathlineto{\pgfqpoint{2.314714in}{0.148611in}}%
\pgfpathclose%
\pgfusepath{stroke,fill}%
\end{pgfscope}%
\begin{pgfscope}%
\pgfpathrectangle{\pgfqpoint{2.127335in}{0.148611in}}{\pgfqpoint{0.824468in}{0.243158in}}%
\pgfusepath{clip}%
\pgfsetbuttcap%
\pgfsetmiterjoin%
\definecolor{currentfill}{rgb}{0.121569,0.466667,0.705882}%
\pgfsetfillcolor{currentfill}%
\pgfsetfillopacity{0.500000}%
\pgfsetlinewidth{1.003750pt}%
\definecolor{currentstroke}{rgb}{0.000000,0.000000,0.000000}%
\pgfsetstrokecolor{currentstroke}%
\pgfsetdash{}{0pt}%
\pgfpathmoveto{\pgfqpoint{2.464617in}{0.148611in}}%
\pgfpathlineto{\pgfqpoint{2.614520in}{0.148611in}}%
\pgfpathlineto{\pgfqpoint{2.614520in}{0.148611in}}%
\pgfpathlineto{\pgfqpoint{2.464617in}{0.148611in}}%
\pgfpathlineto{\pgfqpoint{2.464617in}{0.148611in}}%
\pgfpathclose%
\pgfusepath{stroke,fill}%
\end{pgfscope}%
\begin{pgfscope}%
\pgfpathrectangle{\pgfqpoint{2.127335in}{0.148611in}}{\pgfqpoint{0.824468in}{0.243158in}}%
\pgfusepath{clip}%
\pgfsetbuttcap%
\pgfsetmiterjoin%
\definecolor{currentfill}{rgb}{0.121569,0.466667,0.705882}%
\pgfsetfillcolor{currentfill}%
\pgfsetfillopacity{0.500000}%
\pgfsetlinewidth{1.003750pt}%
\definecolor{currentstroke}{rgb}{0.000000,0.000000,0.000000}%
\pgfsetstrokecolor{currentstroke}%
\pgfsetdash{}{0pt}%
\pgfpathmoveto{\pgfqpoint{2.614520in}{0.148611in}}%
\pgfpathlineto{\pgfqpoint{2.764423in}{0.148611in}}%
\pgfpathlineto{\pgfqpoint{2.764423in}{0.149462in}}%
\pgfpathlineto{\pgfqpoint{2.614520in}{0.149462in}}%
\pgfpathlineto{\pgfqpoint{2.614520in}{0.148611in}}%
\pgfpathclose%
\pgfusepath{stroke,fill}%
\end{pgfscope}%
\begin{pgfscope}%
\pgfpathrectangle{\pgfqpoint{2.127335in}{0.148611in}}{\pgfqpoint{0.824468in}{0.243158in}}%
\pgfusepath{clip}%
\pgfsetbuttcap%
\pgfsetmiterjoin%
\definecolor{currentfill}{rgb}{0.121569,0.466667,0.705882}%
\pgfsetfillcolor{currentfill}%
\pgfsetfillopacity{0.500000}%
\pgfsetlinewidth{1.003750pt}%
\definecolor{currentstroke}{rgb}{0.000000,0.000000,0.000000}%
\pgfsetstrokecolor{currentstroke}%
\pgfsetdash{}{0pt}%
\pgfpathmoveto{\pgfqpoint{2.764423in}{0.148611in}}%
\pgfpathlineto{\pgfqpoint{2.914327in}{0.148611in}}%
\pgfpathlineto{\pgfqpoint{2.914327in}{0.148854in}}%
\pgfpathlineto{\pgfqpoint{2.764423in}{0.148854in}}%
\pgfpathlineto{\pgfqpoint{2.764423in}{0.148611in}}%
\pgfpathclose%
\pgfusepath{stroke,fill}%
\end{pgfscope}%
\begin{pgfscope}%
\pgfsetrectcap%
\pgfsetmiterjoin%
\pgfsetlinewidth{0.803000pt}%
\definecolor{currentstroke}{rgb}{0.000000,0.000000,0.000000}%
\pgfsetstrokecolor{currentstroke}%
\pgfsetdash{}{0pt}%
\pgfpathmoveto{\pgfqpoint{2.127335in}{0.148611in}}%
\pgfpathlineto{\pgfqpoint{2.127335in}{0.391769in}}%
\pgfusepath{stroke}%
\end{pgfscope}%
\begin{pgfscope}%
\pgfsetrectcap%
\pgfsetmiterjoin%
\pgfsetlinewidth{0.803000pt}%
\definecolor{currentstroke}{rgb}{0.000000,0.000000,0.000000}%
\pgfsetstrokecolor{currentstroke}%
\pgfsetdash{}{0pt}%
\pgfpathmoveto{\pgfqpoint{2.951803in}{0.148611in}}%
\pgfpathlineto{\pgfqpoint{2.951803in}{0.391769in}}%
\pgfusepath{stroke}%
\end{pgfscope}%
\begin{pgfscope}%
\pgfsetrectcap%
\pgfsetmiterjoin%
\pgfsetlinewidth{0.803000pt}%
\definecolor{currentstroke}{rgb}{0.000000,0.000000,0.000000}%
\pgfsetstrokecolor{currentstroke}%
\pgfsetdash{}{0pt}%
\pgfpathmoveto{\pgfqpoint{2.127335in}{0.148611in}}%
\pgfpathlineto{\pgfqpoint{2.951803in}{0.148611in}}%
\pgfusepath{stroke}%
\end{pgfscope}%
\begin{pgfscope}%
\pgfsetrectcap%
\pgfsetmiterjoin%
\pgfsetlinewidth{0.803000pt}%
\definecolor{currentstroke}{rgb}{0.000000,0.000000,0.000000}%
\pgfsetstrokecolor{currentstroke}%
\pgfsetdash{}{0pt}%
\pgfpathmoveto{\pgfqpoint{2.127335in}{0.391769in}}%
\pgfpathlineto{\pgfqpoint{2.951803in}{0.391769in}}%
\pgfusepath{stroke}%
\end{pgfscope}%
\begin{pgfscope}%
\definecolor{textcolor}{rgb}{0.000000,0.000000,0.000000}%
\pgfsetstrokecolor{textcolor}%
\pgfsetfillcolor{textcolor}%
\pgftext[x=2.539569in,y=0.475102in,,base]{\color{textcolor}\rmfamily\fontsize{11.000000}{13.200000}\selectfont Mapa}%
\end{pgfscope}%
\begin{pgfscope}%
\pgfsetbuttcap%
\pgfsetmiterjoin%
\definecolor{currentfill}{rgb}{1.000000,1.000000,1.000000}%
\pgfsetfillcolor{currentfill}%
\pgfsetlinewidth{0.000000pt}%
\definecolor{currentstroke}{rgb}{0.000000,0.000000,0.000000}%
\pgfsetstrokecolor{currentstroke}%
\pgfsetstrokeopacity{0.000000}%
\pgfsetdash{}{0pt}%
\pgfpathmoveto{\pgfqpoint{3.116696in}{0.148611in}}%
\pgfpathlineto{\pgfqpoint{3.941164in}{0.148611in}}%
\pgfpathlineto{\pgfqpoint{3.941164in}{0.391769in}}%
\pgfpathlineto{\pgfqpoint{3.116696in}{0.391769in}}%
\pgfpathlineto{\pgfqpoint{3.116696in}{0.148611in}}%
\pgfpathclose%
\pgfusepath{fill}%
\end{pgfscope}%
\begin{pgfscope}%
\pgfpathrectangle{\pgfqpoint{3.116696in}{0.148611in}}{\pgfqpoint{0.824468in}{0.243158in}}%
\pgfusepath{clip}%
\pgfsetbuttcap%
\pgfsetmiterjoin%
\definecolor{currentfill}{rgb}{0.121569,0.466667,0.705882}%
\pgfsetfillcolor{currentfill}%
\pgfsetfillopacity{0.500000}%
\pgfsetlinewidth{1.003750pt}%
\definecolor{currentstroke}{rgb}{0.000000,0.000000,0.000000}%
\pgfsetstrokecolor{currentstroke}%
\pgfsetdash{}{0pt}%
\pgfpathmoveto{\pgfqpoint{3.154172in}{0.148611in}}%
\pgfpathlineto{\pgfqpoint{3.304075in}{0.148611in}}%
\pgfpathlineto{\pgfqpoint{3.304075in}{0.152015in}}%
\pgfpathlineto{\pgfqpoint{3.154172in}{0.152015in}}%
\pgfpathlineto{\pgfqpoint{3.154172in}{0.148611in}}%
\pgfpathclose%
\pgfusepath{stroke,fill}%
\end{pgfscope}%
\begin{pgfscope}%
\pgfpathrectangle{\pgfqpoint{3.116696in}{0.148611in}}{\pgfqpoint{0.824468in}{0.243158in}}%
\pgfusepath{clip}%
\pgfsetbuttcap%
\pgfsetmiterjoin%
\definecolor{currentfill}{rgb}{0.121569,0.466667,0.705882}%
\pgfsetfillcolor{currentfill}%
\pgfsetfillopacity{0.500000}%
\pgfsetlinewidth{1.003750pt}%
\definecolor{currentstroke}{rgb}{0.000000,0.000000,0.000000}%
\pgfsetstrokecolor{currentstroke}%
\pgfsetdash{}{0pt}%
\pgfpathmoveto{\pgfqpoint{3.304075in}{0.148611in}}%
\pgfpathlineto{\pgfqpoint{3.453979in}{0.148611in}}%
\pgfpathlineto{\pgfqpoint{3.453979in}{0.149341in}}%
\pgfpathlineto{\pgfqpoint{3.304075in}{0.149341in}}%
\pgfpathlineto{\pgfqpoint{3.304075in}{0.148611in}}%
\pgfpathclose%
\pgfusepath{stroke,fill}%
\end{pgfscope}%
\begin{pgfscope}%
\pgfpathrectangle{\pgfqpoint{3.116696in}{0.148611in}}{\pgfqpoint{0.824468in}{0.243158in}}%
\pgfusepath{clip}%
\pgfsetbuttcap%
\pgfsetmiterjoin%
\definecolor{currentfill}{rgb}{0.121569,0.466667,0.705882}%
\pgfsetfillcolor{currentfill}%
\pgfsetfillopacity{0.500000}%
\pgfsetlinewidth{1.003750pt}%
\definecolor{currentstroke}{rgb}{0.000000,0.000000,0.000000}%
\pgfsetstrokecolor{currentstroke}%
\pgfsetdash{}{0pt}%
\pgfpathmoveto{\pgfqpoint{3.453979in}{0.148611in}}%
\pgfpathlineto{\pgfqpoint{3.603882in}{0.148611in}}%
\pgfpathlineto{\pgfqpoint{3.603882in}{0.149097in}}%
\pgfpathlineto{\pgfqpoint{3.453979in}{0.149097in}}%
\pgfpathlineto{\pgfqpoint{3.453979in}{0.148611in}}%
\pgfpathclose%
\pgfusepath{stroke,fill}%
\end{pgfscope}%
\begin{pgfscope}%
\pgfpathrectangle{\pgfqpoint{3.116696in}{0.148611in}}{\pgfqpoint{0.824468in}{0.243158in}}%
\pgfusepath{clip}%
\pgfsetbuttcap%
\pgfsetmiterjoin%
\definecolor{currentfill}{rgb}{0.121569,0.466667,0.705882}%
\pgfsetfillcolor{currentfill}%
\pgfsetfillopacity{0.500000}%
\pgfsetlinewidth{1.003750pt}%
\definecolor{currentstroke}{rgb}{0.000000,0.000000,0.000000}%
\pgfsetstrokecolor{currentstroke}%
\pgfsetdash{}{0pt}%
\pgfpathmoveto{\pgfqpoint{3.603882in}{0.148611in}}%
\pgfpathlineto{\pgfqpoint{3.753785in}{0.148611in}}%
\pgfpathlineto{\pgfqpoint{3.753785in}{0.148611in}}%
\pgfpathlineto{\pgfqpoint{3.603882in}{0.148611in}}%
\pgfpathlineto{\pgfqpoint{3.603882in}{0.148611in}}%
\pgfpathclose%
\pgfusepath{stroke,fill}%
\end{pgfscope}%
\begin{pgfscope}%
\pgfpathrectangle{\pgfqpoint{3.116696in}{0.148611in}}{\pgfqpoint{0.824468in}{0.243158in}}%
\pgfusepath{clip}%
\pgfsetbuttcap%
\pgfsetmiterjoin%
\definecolor{currentfill}{rgb}{0.121569,0.466667,0.705882}%
\pgfsetfillcolor{currentfill}%
\pgfsetfillopacity{0.500000}%
\pgfsetlinewidth{1.003750pt}%
\definecolor{currentstroke}{rgb}{0.000000,0.000000,0.000000}%
\pgfsetstrokecolor{currentstroke}%
\pgfsetdash{}{0pt}%
\pgfpathmoveto{\pgfqpoint{3.753785in}{0.148611in}}%
\pgfpathlineto{\pgfqpoint{3.903688in}{0.148611in}}%
\pgfpathlineto{\pgfqpoint{3.903688in}{0.148611in}}%
\pgfpathlineto{\pgfqpoint{3.753785in}{0.148611in}}%
\pgfpathlineto{\pgfqpoint{3.753785in}{0.148611in}}%
\pgfpathclose%
\pgfusepath{stroke,fill}%
\end{pgfscope}%
\begin{pgfscope}%
\pgfsetrectcap%
\pgfsetmiterjoin%
\pgfsetlinewidth{0.803000pt}%
\definecolor{currentstroke}{rgb}{0.000000,0.000000,0.000000}%
\pgfsetstrokecolor{currentstroke}%
\pgfsetdash{}{0pt}%
\pgfpathmoveto{\pgfqpoint{3.116696in}{0.148611in}}%
\pgfpathlineto{\pgfqpoint{3.116696in}{0.391769in}}%
\pgfusepath{stroke}%
\end{pgfscope}%
\begin{pgfscope}%
\pgfsetrectcap%
\pgfsetmiterjoin%
\pgfsetlinewidth{0.803000pt}%
\definecolor{currentstroke}{rgb}{0.000000,0.000000,0.000000}%
\pgfsetstrokecolor{currentstroke}%
\pgfsetdash{}{0pt}%
\pgfpathmoveto{\pgfqpoint{3.941164in}{0.148611in}}%
\pgfpathlineto{\pgfqpoint{3.941164in}{0.391769in}}%
\pgfusepath{stroke}%
\end{pgfscope}%
\begin{pgfscope}%
\pgfsetrectcap%
\pgfsetmiterjoin%
\pgfsetlinewidth{0.803000pt}%
\definecolor{currentstroke}{rgb}{0.000000,0.000000,0.000000}%
\pgfsetstrokecolor{currentstroke}%
\pgfsetdash{}{0pt}%
\pgfpathmoveto{\pgfqpoint{3.116696in}{0.148611in}}%
\pgfpathlineto{\pgfqpoint{3.941164in}{0.148611in}}%
\pgfusepath{stroke}%
\end{pgfscope}%
\begin{pgfscope}%
\pgfsetrectcap%
\pgfsetmiterjoin%
\pgfsetlinewidth{0.803000pt}%
\definecolor{currentstroke}{rgb}{0.000000,0.000000,0.000000}%
\pgfsetstrokecolor{currentstroke}%
\pgfsetdash{}{0pt}%
\pgfpathmoveto{\pgfqpoint{3.116696in}{0.391769in}}%
\pgfpathlineto{\pgfqpoint{3.941164in}{0.391769in}}%
\pgfusepath{stroke}%
\end{pgfscope}%
\begin{pgfscope}%
\definecolor{textcolor}{rgb}{0.000000,0.000000,0.000000}%
\pgfsetstrokecolor{textcolor}%
\pgfsetfillcolor{textcolor}%
\pgftext[x=3.528930in,y=0.475102in,,base]{\color{textcolor}\rmfamily\fontsize{11.000000}{13.200000}\selectfont Malako...}%
\end{pgfscope}%
\begin{pgfscope}%
\pgfsetbuttcap%
\pgfsetmiterjoin%
\definecolor{currentfill}{rgb}{1.000000,1.000000,1.000000}%
\pgfsetfillcolor{currentfill}%
\pgfsetlinewidth{0.000000pt}%
\definecolor{currentstroke}{rgb}{0.000000,0.000000,0.000000}%
\pgfsetstrokecolor{currentstroke}%
\pgfsetstrokeopacity{0.000000}%
\pgfsetdash{}{0pt}%
\pgfpathmoveto{\pgfqpoint{4.106058in}{0.148611in}}%
\pgfpathlineto{\pgfqpoint{4.930526in}{0.148611in}}%
\pgfpathlineto{\pgfqpoint{4.930526in}{0.391769in}}%
\pgfpathlineto{\pgfqpoint{4.106058in}{0.391769in}}%
\pgfpathlineto{\pgfqpoint{4.106058in}{0.148611in}}%
\pgfpathclose%
\pgfusepath{fill}%
\end{pgfscope}%
\begin{pgfscope}%
\pgfpathrectangle{\pgfqpoint{4.106058in}{0.148611in}}{\pgfqpoint{0.824468in}{0.243158in}}%
\pgfusepath{clip}%
\pgfsetbuttcap%
\pgfsetmiterjoin%
\definecolor{currentfill}{rgb}{0.121569,0.466667,0.705882}%
\pgfsetfillcolor{currentfill}%
\pgfsetfillopacity{0.500000}%
\pgfsetlinewidth{1.003750pt}%
\definecolor{currentstroke}{rgb}{0.000000,0.000000,0.000000}%
\pgfsetstrokecolor{currentstroke}%
\pgfsetdash{}{0pt}%
\pgfpathmoveto{\pgfqpoint{4.143534in}{0.148611in}}%
\pgfpathlineto{\pgfqpoint{4.293437in}{0.148611in}}%
\pgfpathlineto{\pgfqpoint{4.293437in}{0.151286in}}%
\pgfpathlineto{\pgfqpoint{4.143534in}{0.151286in}}%
\pgfpathlineto{\pgfqpoint{4.143534in}{0.148611in}}%
\pgfpathclose%
\pgfusepath{stroke,fill}%
\end{pgfscope}%
\begin{pgfscope}%
\pgfpathrectangle{\pgfqpoint{4.106058in}{0.148611in}}{\pgfqpoint{0.824468in}{0.243158in}}%
\pgfusepath{clip}%
\pgfsetbuttcap%
\pgfsetmiterjoin%
\definecolor{currentfill}{rgb}{0.121569,0.466667,0.705882}%
\pgfsetfillcolor{currentfill}%
\pgfsetfillopacity{0.500000}%
\pgfsetlinewidth{1.003750pt}%
\definecolor{currentstroke}{rgb}{0.000000,0.000000,0.000000}%
\pgfsetstrokecolor{currentstroke}%
\pgfsetdash{}{0pt}%
\pgfpathmoveto{\pgfqpoint{4.293437in}{0.148611in}}%
\pgfpathlineto{\pgfqpoint{4.443340in}{0.148611in}}%
\pgfpathlineto{\pgfqpoint{4.443340in}{0.150556in}}%
\pgfpathlineto{\pgfqpoint{4.293437in}{0.150556in}}%
\pgfpathlineto{\pgfqpoint{4.293437in}{0.148611in}}%
\pgfpathclose%
\pgfusepath{stroke,fill}%
\end{pgfscope}%
\begin{pgfscope}%
\pgfpathrectangle{\pgfqpoint{4.106058in}{0.148611in}}{\pgfqpoint{0.824468in}{0.243158in}}%
\pgfusepath{clip}%
\pgfsetbuttcap%
\pgfsetmiterjoin%
\definecolor{currentfill}{rgb}{0.121569,0.466667,0.705882}%
\pgfsetfillcolor{currentfill}%
\pgfsetfillopacity{0.500000}%
\pgfsetlinewidth{1.003750pt}%
\definecolor{currentstroke}{rgb}{0.000000,0.000000,0.000000}%
\pgfsetstrokecolor{currentstroke}%
\pgfsetdash{}{0pt}%
\pgfpathmoveto{\pgfqpoint{4.443340in}{0.148611in}}%
\pgfpathlineto{\pgfqpoint{4.593244in}{0.148611in}}%
\pgfpathlineto{\pgfqpoint{4.593244in}{0.149097in}}%
\pgfpathlineto{\pgfqpoint{4.443340in}{0.149097in}}%
\pgfpathlineto{\pgfqpoint{4.443340in}{0.148611in}}%
\pgfpathclose%
\pgfusepath{stroke,fill}%
\end{pgfscope}%
\begin{pgfscope}%
\pgfpathrectangle{\pgfqpoint{4.106058in}{0.148611in}}{\pgfqpoint{0.824468in}{0.243158in}}%
\pgfusepath{clip}%
\pgfsetbuttcap%
\pgfsetmiterjoin%
\definecolor{currentfill}{rgb}{0.121569,0.466667,0.705882}%
\pgfsetfillcolor{currentfill}%
\pgfsetfillopacity{0.500000}%
\pgfsetlinewidth{1.003750pt}%
\definecolor{currentstroke}{rgb}{0.000000,0.000000,0.000000}%
\pgfsetstrokecolor{currentstroke}%
\pgfsetdash{}{0pt}%
\pgfpathmoveto{\pgfqpoint{4.593244in}{0.148611in}}%
\pgfpathlineto{\pgfqpoint{4.743147in}{0.148611in}}%
\pgfpathlineto{\pgfqpoint{4.743147in}{0.148733in}}%
\pgfpathlineto{\pgfqpoint{4.593244in}{0.148733in}}%
\pgfpathlineto{\pgfqpoint{4.593244in}{0.148611in}}%
\pgfpathclose%
\pgfusepath{stroke,fill}%
\end{pgfscope}%
\begin{pgfscope}%
\pgfpathrectangle{\pgfqpoint{4.106058in}{0.148611in}}{\pgfqpoint{0.824468in}{0.243158in}}%
\pgfusepath{clip}%
\pgfsetbuttcap%
\pgfsetmiterjoin%
\definecolor{currentfill}{rgb}{0.121569,0.466667,0.705882}%
\pgfsetfillcolor{currentfill}%
\pgfsetfillopacity{0.500000}%
\pgfsetlinewidth{1.003750pt}%
\definecolor{currentstroke}{rgb}{0.000000,0.000000,0.000000}%
\pgfsetstrokecolor{currentstroke}%
\pgfsetdash{}{0pt}%
\pgfpathmoveto{\pgfqpoint{4.743147in}{0.148611in}}%
\pgfpathlineto{\pgfqpoint{4.893050in}{0.148611in}}%
\pgfpathlineto{\pgfqpoint{4.893050in}{0.148611in}}%
\pgfpathlineto{\pgfqpoint{4.743147in}{0.148611in}}%
\pgfpathlineto{\pgfqpoint{4.743147in}{0.148611in}}%
\pgfpathclose%
\pgfusepath{stroke,fill}%
\end{pgfscope}%
\begin{pgfscope}%
\pgfsetrectcap%
\pgfsetmiterjoin%
\pgfsetlinewidth{0.803000pt}%
\definecolor{currentstroke}{rgb}{0.000000,0.000000,0.000000}%
\pgfsetstrokecolor{currentstroke}%
\pgfsetdash{}{0pt}%
\pgfpathmoveto{\pgfqpoint{4.106058in}{0.148611in}}%
\pgfpathlineto{\pgfqpoint{4.106058in}{0.391769in}}%
\pgfusepath{stroke}%
\end{pgfscope}%
\begin{pgfscope}%
\pgfsetrectcap%
\pgfsetmiterjoin%
\pgfsetlinewidth{0.803000pt}%
\definecolor{currentstroke}{rgb}{0.000000,0.000000,0.000000}%
\pgfsetstrokecolor{currentstroke}%
\pgfsetdash{}{0pt}%
\pgfpathmoveto{\pgfqpoint{4.930526in}{0.148611in}}%
\pgfpathlineto{\pgfqpoint{4.930526in}{0.391769in}}%
\pgfusepath{stroke}%
\end{pgfscope}%
\begin{pgfscope}%
\pgfsetrectcap%
\pgfsetmiterjoin%
\pgfsetlinewidth{0.803000pt}%
\definecolor{currentstroke}{rgb}{0.000000,0.000000,0.000000}%
\pgfsetstrokecolor{currentstroke}%
\pgfsetdash{}{0pt}%
\pgfpathmoveto{\pgfqpoint{4.106058in}{0.148611in}}%
\pgfpathlineto{\pgfqpoint{4.930526in}{0.148611in}}%
\pgfusepath{stroke}%
\end{pgfscope}%
\begin{pgfscope}%
\pgfsetrectcap%
\pgfsetmiterjoin%
\pgfsetlinewidth{0.803000pt}%
\definecolor{currentstroke}{rgb}{0.000000,0.000000,0.000000}%
\pgfsetstrokecolor{currentstroke}%
\pgfsetdash{}{0pt}%
\pgfpathmoveto{\pgfqpoint{4.106058in}{0.391769in}}%
\pgfpathlineto{\pgfqpoint{4.930526in}{0.391769in}}%
\pgfusepath{stroke}%
\end{pgfscope}%
\begin{pgfscope}%
\definecolor{textcolor}{rgb}{0.000000,0.000000,0.000000}%
\pgfsetstrokecolor{textcolor}%
\pgfsetfillcolor{textcolor}%
\pgftext[x=4.518292in,y=0.475102in,,base]{\color{textcolor}\rmfamily\fontsize{11.000000}{13.200000}\selectfont Euro-A...}%
\end{pgfscope}%
\begin{pgfscope}%
\pgfsetbuttcap%
\pgfsetmiterjoin%
\definecolor{currentfill}{rgb}{1.000000,1.000000,1.000000}%
\pgfsetfillcolor{currentfill}%
\pgfsetlinewidth{0.000000pt}%
\definecolor{currentstroke}{rgb}{0.000000,0.000000,0.000000}%
\pgfsetstrokecolor{currentstroke}%
\pgfsetstrokeopacity{0.000000}%
\pgfsetdash{}{0pt}%
\pgfpathmoveto{\pgfqpoint{5.095420in}{0.148611in}}%
\pgfpathlineto{\pgfqpoint{5.919888in}{0.148611in}}%
\pgfpathlineto{\pgfqpoint{5.919888in}{0.391769in}}%
\pgfpathlineto{\pgfqpoint{5.095420in}{0.391769in}}%
\pgfpathlineto{\pgfqpoint{5.095420in}{0.148611in}}%
\pgfpathclose%
\pgfusepath{fill}%
\end{pgfscope}%
\begin{pgfscope}%
\pgfpathrectangle{\pgfqpoint{5.095420in}{0.148611in}}{\pgfqpoint{0.824468in}{0.243158in}}%
\pgfusepath{clip}%
\pgfsetbuttcap%
\pgfsetmiterjoin%
\definecolor{currentfill}{rgb}{0.121569,0.466667,0.705882}%
\pgfsetfillcolor{currentfill}%
\pgfsetfillopacity{0.500000}%
\pgfsetlinewidth{1.003750pt}%
\definecolor{currentstroke}{rgb}{0.000000,0.000000,0.000000}%
\pgfsetstrokecolor{currentstroke}%
\pgfsetdash{}{0pt}%
\pgfpathmoveto{\pgfqpoint{5.132895in}{0.148611in}}%
\pgfpathlineto{\pgfqpoint{5.282799in}{0.148611in}}%
\pgfpathlineto{\pgfqpoint{5.282799in}{0.148976in}}%
\pgfpathlineto{\pgfqpoint{5.132895in}{0.148976in}}%
\pgfpathlineto{\pgfqpoint{5.132895in}{0.148611in}}%
\pgfpathclose%
\pgfusepath{stroke,fill}%
\end{pgfscope}%
\begin{pgfscope}%
\pgfpathrectangle{\pgfqpoint{5.095420in}{0.148611in}}{\pgfqpoint{0.824468in}{0.243158in}}%
\pgfusepath{clip}%
\pgfsetbuttcap%
\pgfsetmiterjoin%
\definecolor{currentfill}{rgb}{0.121569,0.466667,0.705882}%
\pgfsetfillcolor{currentfill}%
\pgfsetfillopacity{0.500000}%
\pgfsetlinewidth{1.003750pt}%
\definecolor{currentstroke}{rgb}{0.000000,0.000000,0.000000}%
\pgfsetstrokecolor{currentstroke}%
\pgfsetdash{}{0pt}%
\pgfpathmoveto{\pgfqpoint{5.282799in}{0.148611in}}%
\pgfpathlineto{\pgfqpoint{5.432702in}{0.148611in}}%
\pgfpathlineto{\pgfqpoint{5.432702in}{0.149462in}}%
\pgfpathlineto{\pgfqpoint{5.282799in}{0.149462in}}%
\pgfpathlineto{\pgfqpoint{5.282799in}{0.148611in}}%
\pgfpathclose%
\pgfusepath{stroke,fill}%
\end{pgfscope}%
\begin{pgfscope}%
\pgfpathrectangle{\pgfqpoint{5.095420in}{0.148611in}}{\pgfqpoint{0.824468in}{0.243158in}}%
\pgfusepath{clip}%
\pgfsetbuttcap%
\pgfsetmiterjoin%
\definecolor{currentfill}{rgb}{0.121569,0.466667,0.705882}%
\pgfsetfillcolor{currentfill}%
\pgfsetfillopacity{0.500000}%
\pgfsetlinewidth{1.003750pt}%
\definecolor{currentstroke}{rgb}{0.000000,0.000000,0.000000}%
\pgfsetstrokecolor{currentstroke}%
\pgfsetdash{}{0pt}%
\pgfpathmoveto{\pgfqpoint{5.432702in}{0.148611in}}%
\pgfpathlineto{\pgfqpoint{5.582605in}{0.148611in}}%
\pgfpathlineto{\pgfqpoint{5.582605in}{0.148611in}}%
\pgfpathlineto{\pgfqpoint{5.432702in}{0.148611in}}%
\pgfpathlineto{\pgfqpoint{5.432702in}{0.148611in}}%
\pgfpathclose%
\pgfusepath{stroke,fill}%
\end{pgfscope}%
\begin{pgfscope}%
\pgfpathrectangle{\pgfqpoint{5.095420in}{0.148611in}}{\pgfqpoint{0.824468in}{0.243158in}}%
\pgfusepath{clip}%
\pgfsetbuttcap%
\pgfsetmiterjoin%
\definecolor{currentfill}{rgb}{0.121569,0.466667,0.705882}%
\pgfsetfillcolor{currentfill}%
\pgfsetfillopacity{0.500000}%
\pgfsetlinewidth{1.003750pt}%
\definecolor{currentstroke}{rgb}{0.000000,0.000000,0.000000}%
\pgfsetstrokecolor{currentstroke}%
\pgfsetdash{}{0pt}%
\pgfpathmoveto{\pgfqpoint{5.582605in}{0.148611in}}%
\pgfpathlineto{\pgfqpoint{5.732509in}{0.148611in}}%
\pgfpathlineto{\pgfqpoint{5.732509in}{0.149219in}}%
\pgfpathlineto{\pgfqpoint{5.582605in}{0.149219in}}%
\pgfpathlineto{\pgfqpoint{5.582605in}{0.148611in}}%
\pgfpathclose%
\pgfusepath{stroke,fill}%
\end{pgfscope}%
\begin{pgfscope}%
\pgfpathrectangle{\pgfqpoint{5.095420in}{0.148611in}}{\pgfqpoint{0.824468in}{0.243158in}}%
\pgfusepath{clip}%
\pgfsetbuttcap%
\pgfsetmiterjoin%
\definecolor{currentfill}{rgb}{0.121569,0.466667,0.705882}%
\pgfsetfillcolor{currentfill}%
\pgfsetfillopacity{0.500000}%
\pgfsetlinewidth{1.003750pt}%
\definecolor{currentstroke}{rgb}{0.000000,0.000000,0.000000}%
\pgfsetstrokecolor{currentstroke}%
\pgfsetdash{}{0pt}%
\pgfpathmoveto{\pgfqpoint{5.732509in}{0.148611in}}%
\pgfpathlineto{\pgfqpoint{5.882412in}{0.148611in}}%
\pgfpathlineto{\pgfqpoint{5.882412in}{0.150192in}}%
\pgfpathlineto{\pgfqpoint{5.732509in}{0.150192in}}%
\pgfpathlineto{\pgfqpoint{5.732509in}{0.148611in}}%
\pgfpathclose%
\pgfusepath{stroke,fill}%
\end{pgfscope}%
\begin{pgfscope}%
\pgfsetrectcap%
\pgfsetmiterjoin%
\pgfsetlinewidth{0.803000pt}%
\definecolor{currentstroke}{rgb}{0.000000,0.000000,0.000000}%
\pgfsetstrokecolor{currentstroke}%
\pgfsetdash{}{0pt}%
\pgfpathmoveto{\pgfqpoint{5.095420in}{0.148611in}}%
\pgfpathlineto{\pgfqpoint{5.095420in}{0.391769in}}%
\pgfusepath{stroke}%
\end{pgfscope}%
\begin{pgfscope}%
\pgfsetrectcap%
\pgfsetmiterjoin%
\pgfsetlinewidth{0.803000pt}%
\definecolor{currentstroke}{rgb}{0.000000,0.000000,0.000000}%
\pgfsetstrokecolor{currentstroke}%
\pgfsetdash{}{0pt}%
\pgfpathmoveto{\pgfqpoint{5.919888in}{0.148611in}}%
\pgfpathlineto{\pgfqpoint{5.919888in}{0.391769in}}%
\pgfusepath{stroke}%
\end{pgfscope}%
\begin{pgfscope}%
\pgfsetrectcap%
\pgfsetmiterjoin%
\pgfsetlinewidth{0.803000pt}%
\definecolor{currentstroke}{rgb}{0.000000,0.000000,0.000000}%
\pgfsetstrokecolor{currentstroke}%
\pgfsetdash{}{0pt}%
\pgfpathmoveto{\pgfqpoint{5.095420in}{0.148611in}}%
\pgfpathlineto{\pgfqpoint{5.919888in}{0.148611in}}%
\pgfusepath{stroke}%
\end{pgfscope}%
\begin{pgfscope}%
\pgfsetrectcap%
\pgfsetmiterjoin%
\pgfsetlinewidth{0.803000pt}%
\definecolor{currentstroke}{rgb}{0.000000,0.000000,0.000000}%
\pgfsetstrokecolor{currentstroke}%
\pgfsetdash{}{0pt}%
\pgfpathmoveto{\pgfqpoint{5.095420in}{0.391769in}}%
\pgfpathlineto{\pgfqpoint{5.919888in}{0.391769in}}%
\pgfusepath{stroke}%
\end{pgfscope}%
\begin{pgfscope}%
\definecolor{textcolor}{rgb}{0.000000,0.000000,0.000000}%
\pgfsetstrokecolor{textcolor}%
\pgfsetfillcolor{textcolor}%
\pgftext[x=5.507654in,y=0.475102in,,base]{\color{textcolor}\rmfamily\fontsize{11.000000}{13.200000}\selectfont Peyrac...}%
\end{pgfscope}%
\begin{pgfscope}%
\pgfsetbuttcap%
\pgfsetmiterjoin%
\definecolor{currentfill}{rgb}{1.000000,1.000000,1.000000}%
\pgfsetfillcolor{currentfill}%
\pgfsetlinewidth{0.000000pt}%
\definecolor{currentstroke}{rgb}{0.000000,0.000000,0.000000}%
\pgfsetstrokecolor{currentstroke}%
\pgfsetstrokeopacity{0.000000}%
\pgfsetdash{}{0pt}%
\pgfpathmoveto{\pgfqpoint{6.084781in}{0.148611in}}%
\pgfpathlineto{\pgfqpoint{6.909249in}{0.148611in}}%
\pgfpathlineto{\pgfqpoint{6.909249in}{0.391769in}}%
\pgfpathlineto{\pgfqpoint{6.084781in}{0.391769in}}%
\pgfpathlineto{\pgfqpoint{6.084781in}{0.148611in}}%
\pgfpathclose%
\pgfusepath{fill}%
\end{pgfscope}%
\begin{pgfscope}%
\pgfpathrectangle{\pgfqpoint{6.084781in}{0.148611in}}{\pgfqpoint{0.824468in}{0.243158in}}%
\pgfusepath{clip}%
\pgfsetbuttcap%
\pgfsetmiterjoin%
\definecolor{currentfill}{rgb}{0.121569,0.466667,0.705882}%
\pgfsetfillcolor{currentfill}%
\pgfsetfillopacity{0.500000}%
\pgfsetlinewidth{1.003750pt}%
\definecolor{currentstroke}{rgb}{0.000000,0.000000,0.000000}%
\pgfsetstrokecolor{currentstroke}%
\pgfsetdash{}{0pt}%
\pgfpathmoveto{\pgfqpoint{6.122257in}{0.148611in}}%
\pgfpathlineto{\pgfqpoint{6.272160in}{0.148611in}}%
\pgfpathlineto{\pgfqpoint{6.272160in}{0.149219in}}%
\pgfpathlineto{\pgfqpoint{6.122257in}{0.149219in}}%
\pgfpathlineto{\pgfqpoint{6.122257in}{0.148611in}}%
\pgfpathclose%
\pgfusepath{stroke,fill}%
\end{pgfscope}%
\begin{pgfscope}%
\pgfpathrectangle{\pgfqpoint{6.084781in}{0.148611in}}{\pgfqpoint{0.824468in}{0.243158in}}%
\pgfusepath{clip}%
\pgfsetbuttcap%
\pgfsetmiterjoin%
\definecolor{currentfill}{rgb}{0.121569,0.466667,0.705882}%
\pgfsetfillcolor{currentfill}%
\pgfsetfillopacity{0.500000}%
\pgfsetlinewidth{1.003750pt}%
\definecolor{currentstroke}{rgb}{0.000000,0.000000,0.000000}%
\pgfsetstrokecolor{currentstroke}%
\pgfsetdash{}{0pt}%
\pgfpathmoveto{\pgfqpoint{6.272160in}{0.148611in}}%
\pgfpathlineto{\pgfqpoint{6.422064in}{0.148611in}}%
\pgfpathlineto{\pgfqpoint{6.422064in}{0.148611in}}%
\pgfpathlineto{\pgfqpoint{6.272160in}{0.148611in}}%
\pgfpathlineto{\pgfqpoint{6.272160in}{0.148611in}}%
\pgfpathclose%
\pgfusepath{stroke,fill}%
\end{pgfscope}%
\begin{pgfscope}%
\pgfpathrectangle{\pgfqpoint{6.084781in}{0.148611in}}{\pgfqpoint{0.824468in}{0.243158in}}%
\pgfusepath{clip}%
\pgfsetbuttcap%
\pgfsetmiterjoin%
\definecolor{currentfill}{rgb}{0.121569,0.466667,0.705882}%
\pgfsetfillcolor{currentfill}%
\pgfsetfillopacity{0.500000}%
\pgfsetlinewidth{1.003750pt}%
\definecolor{currentstroke}{rgb}{0.000000,0.000000,0.000000}%
\pgfsetstrokecolor{currentstroke}%
\pgfsetdash{}{0pt}%
\pgfpathmoveto{\pgfqpoint{6.422064in}{0.148611in}}%
\pgfpathlineto{\pgfqpoint{6.571967in}{0.148611in}}%
\pgfpathlineto{\pgfqpoint{6.571967in}{0.148611in}}%
\pgfpathlineto{\pgfqpoint{6.422064in}{0.148611in}}%
\pgfpathlineto{\pgfqpoint{6.422064in}{0.148611in}}%
\pgfpathclose%
\pgfusepath{stroke,fill}%
\end{pgfscope}%
\begin{pgfscope}%
\pgfpathrectangle{\pgfqpoint{6.084781in}{0.148611in}}{\pgfqpoint{0.824468in}{0.243158in}}%
\pgfusepath{clip}%
\pgfsetbuttcap%
\pgfsetmiterjoin%
\definecolor{currentfill}{rgb}{0.121569,0.466667,0.705882}%
\pgfsetfillcolor{currentfill}%
\pgfsetfillopacity{0.500000}%
\pgfsetlinewidth{1.003750pt}%
\definecolor{currentstroke}{rgb}{0.000000,0.000000,0.000000}%
\pgfsetstrokecolor{currentstroke}%
\pgfsetdash{}{0pt}%
\pgfpathmoveto{\pgfqpoint{6.571967in}{0.148611in}}%
\pgfpathlineto{\pgfqpoint{6.721870in}{0.148611in}}%
\pgfpathlineto{\pgfqpoint{6.721870in}{0.148611in}}%
\pgfpathlineto{\pgfqpoint{6.571967in}{0.148611in}}%
\pgfpathlineto{\pgfqpoint{6.571967in}{0.148611in}}%
\pgfpathclose%
\pgfusepath{stroke,fill}%
\end{pgfscope}%
\begin{pgfscope}%
\pgfpathrectangle{\pgfqpoint{6.084781in}{0.148611in}}{\pgfqpoint{0.824468in}{0.243158in}}%
\pgfusepath{clip}%
\pgfsetbuttcap%
\pgfsetmiterjoin%
\definecolor{currentfill}{rgb}{0.121569,0.466667,0.705882}%
\pgfsetfillcolor{currentfill}%
\pgfsetfillopacity{0.500000}%
\pgfsetlinewidth{1.003750pt}%
\definecolor{currentstroke}{rgb}{0.000000,0.000000,0.000000}%
\pgfsetstrokecolor{currentstroke}%
\pgfsetdash{}{0pt}%
\pgfpathmoveto{\pgfqpoint{6.721870in}{0.148611in}}%
\pgfpathlineto{\pgfqpoint{6.871774in}{0.148611in}}%
\pgfpathlineto{\pgfqpoint{6.871774in}{0.148733in}}%
\pgfpathlineto{\pgfqpoint{6.721870in}{0.148733in}}%
\pgfpathlineto{\pgfqpoint{6.721870in}{0.148611in}}%
\pgfpathclose%
\pgfusepath{stroke,fill}%
\end{pgfscope}%
\begin{pgfscope}%
\pgfsetrectcap%
\pgfsetmiterjoin%
\pgfsetlinewidth{0.803000pt}%
\definecolor{currentstroke}{rgb}{0.000000,0.000000,0.000000}%
\pgfsetstrokecolor{currentstroke}%
\pgfsetdash{}{0pt}%
\pgfpathmoveto{\pgfqpoint{6.084781in}{0.148611in}}%
\pgfpathlineto{\pgfqpoint{6.084781in}{0.391769in}}%
\pgfusepath{stroke}%
\end{pgfscope}%
\begin{pgfscope}%
\pgfsetrectcap%
\pgfsetmiterjoin%
\pgfsetlinewidth{0.803000pt}%
\definecolor{currentstroke}{rgb}{0.000000,0.000000,0.000000}%
\pgfsetstrokecolor{currentstroke}%
\pgfsetdash{}{0pt}%
\pgfpathmoveto{\pgfqpoint{6.909249in}{0.148611in}}%
\pgfpathlineto{\pgfqpoint{6.909249in}{0.391769in}}%
\pgfusepath{stroke}%
\end{pgfscope}%
\begin{pgfscope}%
\pgfsetrectcap%
\pgfsetmiterjoin%
\pgfsetlinewidth{0.803000pt}%
\definecolor{currentstroke}{rgb}{0.000000,0.000000,0.000000}%
\pgfsetstrokecolor{currentstroke}%
\pgfsetdash{}{0pt}%
\pgfpathmoveto{\pgfqpoint{6.084781in}{0.148611in}}%
\pgfpathlineto{\pgfqpoint{6.909249in}{0.148611in}}%
\pgfusepath{stroke}%
\end{pgfscope}%
\begin{pgfscope}%
\pgfsetrectcap%
\pgfsetmiterjoin%
\pgfsetlinewidth{0.803000pt}%
\definecolor{currentstroke}{rgb}{0.000000,0.000000,0.000000}%
\pgfsetstrokecolor{currentstroke}%
\pgfsetdash{}{0pt}%
\pgfpathmoveto{\pgfqpoint{6.084781in}{0.391769in}}%
\pgfpathlineto{\pgfqpoint{6.909249in}{0.391769in}}%
\pgfusepath{stroke}%
\end{pgfscope}%
\begin{pgfscope}%
\definecolor{textcolor}{rgb}{0.000000,0.000000,0.000000}%
\pgfsetstrokecolor{textcolor}%
\pgfsetfillcolor{textcolor}%
\pgftext[x=6.497015in,y=0.475102in,,base]{\color{textcolor}\rmfamily\fontsize{11.000000}{13.200000}\selectfont Sma}%
\end{pgfscope}%
\begin{pgfscope}%
\pgfsetbuttcap%
\pgfsetmiterjoin%
\definecolor{currentfill}{rgb}{1.000000,1.000000,1.000000}%
\pgfsetfillcolor{currentfill}%
\pgfsetlinewidth{0.000000pt}%
\definecolor{currentstroke}{rgb}{0.000000,0.000000,0.000000}%
\pgfsetstrokecolor{currentstroke}%
\pgfsetstrokeopacity{0.000000}%
\pgfsetdash{}{0pt}%
\pgfpathmoveto{\pgfqpoint{7.074143in}{0.148611in}}%
\pgfpathlineto{\pgfqpoint{7.898611in}{0.148611in}}%
\pgfpathlineto{\pgfqpoint{7.898611in}{0.391769in}}%
\pgfpathlineto{\pgfqpoint{7.074143in}{0.391769in}}%
\pgfpathlineto{\pgfqpoint{7.074143in}{0.148611in}}%
\pgfpathclose%
\pgfusepath{fill}%
\end{pgfscope}%
\begin{pgfscope}%
\pgfpathrectangle{\pgfqpoint{7.074143in}{0.148611in}}{\pgfqpoint{0.824468in}{0.243158in}}%
\pgfusepath{clip}%
\pgfsetbuttcap%
\pgfsetmiterjoin%
\definecolor{currentfill}{rgb}{0.121569,0.466667,0.705882}%
\pgfsetfillcolor{currentfill}%
\pgfsetfillopacity{0.500000}%
\pgfsetlinewidth{1.003750pt}%
\definecolor{currentstroke}{rgb}{0.000000,0.000000,0.000000}%
\pgfsetstrokecolor{currentstroke}%
\pgfsetdash{}{0pt}%
\pgfpathmoveto{\pgfqpoint{7.111619in}{0.148611in}}%
\pgfpathlineto{\pgfqpoint{7.261522in}{0.148611in}}%
\pgfpathlineto{\pgfqpoint{7.261522in}{0.148733in}}%
\pgfpathlineto{\pgfqpoint{7.111619in}{0.148733in}}%
\pgfpathlineto{\pgfqpoint{7.111619in}{0.148611in}}%
\pgfpathclose%
\pgfusepath{stroke,fill}%
\end{pgfscope}%
\begin{pgfscope}%
\pgfpathrectangle{\pgfqpoint{7.074143in}{0.148611in}}{\pgfqpoint{0.824468in}{0.243158in}}%
\pgfusepath{clip}%
\pgfsetbuttcap%
\pgfsetmiterjoin%
\definecolor{currentfill}{rgb}{0.121569,0.466667,0.705882}%
\pgfsetfillcolor{currentfill}%
\pgfsetfillopacity{0.500000}%
\pgfsetlinewidth{1.003750pt}%
\definecolor{currentstroke}{rgb}{0.000000,0.000000,0.000000}%
\pgfsetstrokecolor{currentstroke}%
\pgfsetdash{}{0pt}%
\pgfpathmoveto{\pgfqpoint{7.261522in}{0.148611in}}%
\pgfpathlineto{\pgfqpoint{7.411425in}{0.148611in}}%
\pgfpathlineto{\pgfqpoint{7.411425in}{0.148611in}}%
\pgfpathlineto{\pgfqpoint{7.261522in}{0.148611in}}%
\pgfpathlineto{\pgfqpoint{7.261522in}{0.148611in}}%
\pgfpathclose%
\pgfusepath{stroke,fill}%
\end{pgfscope}%
\begin{pgfscope}%
\pgfpathrectangle{\pgfqpoint{7.074143in}{0.148611in}}{\pgfqpoint{0.824468in}{0.243158in}}%
\pgfusepath{clip}%
\pgfsetbuttcap%
\pgfsetmiterjoin%
\definecolor{currentfill}{rgb}{0.121569,0.466667,0.705882}%
\pgfsetfillcolor{currentfill}%
\pgfsetfillopacity{0.500000}%
\pgfsetlinewidth{1.003750pt}%
\definecolor{currentstroke}{rgb}{0.000000,0.000000,0.000000}%
\pgfsetstrokecolor{currentstroke}%
\pgfsetdash{}{0pt}%
\pgfpathmoveto{\pgfqpoint{7.411425in}{0.148611in}}%
\pgfpathlineto{\pgfqpoint{7.561329in}{0.148611in}}%
\pgfpathlineto{\pgfqpoint{7.561329in}{0.148611in}}%
\pgfpathlineto{\pgfqpoint{7.411425in}{0.148611in}}%
\pgfpathlineto{\pgfqpoint{7.411425in}{0.148611in}}%
\pgfpathclose%
\pgfusepath{stroke,fill}%
\end{pgfscope}%
\begin{pgfscope}%
\pgfpathrectangle{\pgfqpoint{7.074143in}{0.148611in}}{\pgfqpoint{0.824468in}{0.243158in}}%
\pgfusepath{clip}%
\pgfsetbuttcap%
\pgfsetmiterjoin%
\definecolor{currentfill}{rgb}{0.121569,0.466667,0.705882}%
\pgfsetfillcolor{currentfill}%
\pgfsetfillopacity{0.500000}%
\pgfsetlinewidth{1.003750pt}%
\definecolor{currentstroke}{rgb}{0.000000,0.000000,0.000000}%
\pgfsetstrokecolor{currentstroke}%
\pgfsetdash{}{0pt}%
\pgfpathmoveto{\pgfqpoint{7.561329in}{0.148611in}}%
\pgfpathlineto{\pgfqpoint{7.711232in}{0.148611in}}%
\pgfpathlineto{\pgfqpoint{7.711232in}{0.148611in}}%
\pgfpathlineto{\pgfqpoint{7.561329in}{0.148611in}}%
\pgfpathlineto{\pgfqpoint{7.561329in}{0.148611in}}%
\pgfpathclose%
\pgfusepath{stroke,fill}%
\end{pgfscope}%
\begin{pgfscope}%
\pgfpathrectangle{\pgfqpoint{7.074143in}{0.148611in}}{\pgfqpoint{0.824468in}{0.243158in}}%
\pgfusepath{clip}%
\pgfsetbuttcap%
\pgfsetmiterjoin%
\definecolor{currentfill}{rgb}{0.121569,0.466667,0.705882}%
\pgfsetfillcolor{currentfill}%
\pgfsetfillopacity{0.500000}%
\pgfsetlinewidth{1.003750pt}%
\definecolor{currentstroke}{rgb}{0.000000,0.000000,0.000000}%
\pgfsetstrokecolor{currentstroke}%
\pgfsetdash{}{0pt}%
\pgfpathmoveto{\pgfqpoint{7.711232in}{0.148611in}}%
\pgfpathlineto{\pgfqpoint{7.861135in}{0.148611in}}%
\pgfpathlineto{\pgfqpoint{7.861135in}{0.148611in}}%
\pgfpathlineto{\pgfqpoint{7.711232in}{0.148611in}}%
\pgfpathlineto{\pgfqpoint{7.711232in}{0.148611in}}%
\pgfpathclose%
\pgfusepath{stroke,fill}%
\end{pgfscope}%
\begin{pgfscope}%
\pgfsetrectcap%
\pgfsetmiterjoin%
\pgfsetlinewidth{0.803000pt}%
\definecolor{currentstroke}{rgb}{0.000000,0.000000,0.000000}%
\pgfsetstrokecolor{currentstroke}%
\pgfsetdash{}{0pt}%
\pgfpathmoveto{\pgfqpoint{7.074143in}{0.148611in}}%
\pgfpathlineto{\pgfqpoint{7.074143in}{0.391769in}}%
\pgfusepath{stroke}%
\end{pgfscope}%
\begin{pgfscope}%
\pgfsetrectcap%
\pgfsetmiterjoin%
\pgfsetlinewidth{0.803000pt}%
\definecolor{currentstroke}{rgb}{0.000000,0.000000,0.000000}%
\pgfsetstrokecolor{currentstroke}%
\pgfsetdash{}{0pt}%
\pgfpathmoveto{\pgfqpoint{7.898611in}{0.148611in}}%
\pgfpathlineto{\pgfqpoint{7.898611in}{0.391769in}}%
\pgfusepath{stroke}%
\end{pgfscope}%
\begin{pgfscope}%
\pgfsetrectcap%
\pgfsetmiterjoin%
\pgfsetlinewidth{0.803000pt}%
\definecolor{currentstroke}{rgb}{0.000000,0.000000,0.000000}%
\pgfsetstrokecolor{currentstroke}%
\pgfsetdash{}{0pt}%
\pgfpathmoveto{\pgfqpoint{7.074143in}{0.148611in}}%
\pgfpathlineto{\pgfqpoint{7.898611in}{0.148611in}}%
\pgfusepath{stroke}%
\end{pgfscope}%
\begin{pgfscope}%
\pgfsetrectcap%
\pgfsetmiterjoin%
\pgfsetlinewidth{0.803000pt}%
\definecolor{currentstroke}{rgb}{0.000000,0.000000,0.000000}%
\pgfsetstrokecolor{currentstroke}%
\pgfsetdash{}{0pt}%
\pgfpathmoveto{\pgfqpoint{7.074143in}{0.391769in}}%
\pgfpathlineto{\pgfqpoint{7.898611in}{0.391769in}}%
\pgfusepath{stroke}%
\end{pgfscope}%
\begin{pgfscope}%
\definecolor{textcolor}{rgb}{0.000000,0.000000,0.000000}%
\pgfsetstrokecolor{textcolor}%
\pgfsetfillcolor{textcolor}%
\pgftext[x=7.486377in,y=0.475102in,,base]{\color{textcolor}\rmfamily\fontsize{11.000000}{13.200000}\selectfont Hiscox}%
\end{pgfscope}%
\end{pgfpicture}%
\makeatother%
\endgroup%

    \caption{Stars distribution per assureur (y-axis scaled)}
    \label{fig:distrib_split_scale}
\end{figure}

\restoregeometry

\newgeometry{bottom=0cm}
We also looked at the mean note per assureur, we used a gradient color to show the number of rating per assureur, the gradient intensity is defined by the number of ratings in \cref{fig:mean_note_per_assureur} and by the rank of the assureur (ordered by number of rating) in \cref{fig:mean_note_per_assureur_linear}

\begin{figure}[H]
    \advance\leftskip-3cm
    %% Creator: Matplotlib, PGF backend
%%
%% To include the figure in your LaTeX document, write
%%   \input{<filename>.pgf}
%%
%% Make sure the required packages are loaded in your preamble
%%   \usepackage{pgf}
%%
%% Also ensure that all the required font packages are loaded; for instance,
%% the lmodern package is sometimes necessary when using math font.
%%   \usepackage{lmodern}
%%
%% Figures using additional raster images can only be included by \input if
%% they are in the same directory as the main LaTeX file. For loading figures
%% from other directories you can use the `import` package
%%   \usepackage{import}
%%
%% and then include the figures with
%%   \import{<path to file>}{<filename>.pgf}
%%
%% Matplotlib used the following preamble
%%
\begingroup%
\makeatletter%
\begin{pgfpicture}%
\pgfpathrectangle{\pgfpointorigin}{\pgfqpoint{7.944444in}{3.829041in}}%
\pgfusepath{use as bounding box, clip}%
\begin{pgfscope}%
\pgfsetbuttcap%
\pgfsetmiterjoin%
\definecolor{currentfill}{rgb}{1.000000,1.000000,1.000000}%
\pgfsetfillcolor{currentfill}%
\pgfsetlinewidth{0.000000pt}%
\definecolor{currentstroke}{rgb}{1.000000,1.000000,1.000000}%
\pgfsetstrokecolor{currentstroke}%
\pgfsetdash{}{0pt}%
\pgfpathmoveto{\pgfqpoint{0.000000in}{0.000000in}}%
\pgfpathlineto{\pgfqpoint{7.944444in}{0.000000in}}%
\pgfpathlineto{\pgfqpoint{7.944444in}{3.829041in}}%
\pgfpathlineto{\pgfqpoint{0.000000in}{3.829041in}}%
\pgfpathlineto{\pgfqpoint{0.000000in}{0.000000in}}%
\pgfpathclose%
\pgfusepath{fill}%
\end{pgfscope}%
\begin{pgfscope}%
\pgfsetbuttcap%
\pgfsetmiterjoin%
\definecolor{currentfill}{rgb}{1.000000,1.000000,1.000000}%
\pgfsetfillcolor{currentfill}%
\pgfsetlinewidth{0.000000pt}%
\definecolor{currentstroke}{rgb}{0.000000,0.000000,0.000000}%
\pgfsetstrokecolor{currentstroke}%
\pgfsetstrokeopacity{0.000000}%
\pgfsetdash{}{0pt}%
\pgfpathmoveto{\pgfqpoint{0.481944in}{1.034041in}}%
\pgfpathlineto{\pgfqpoint{7.844444in}{1.034041in}}%
\pgfpathlineto{\pgfqpoint{7.844444in}{3.729041in}}%
\pgfpathlineto{\pgfqpoint{0.481944in}{3.729041in}}%
\pgfpathlineto{\pgfqpoint{0.481944in}{1.034041in}}%
\pgfpathclose%
\pgfusepath{fill}%
\end{pgfscope}%
\begin{pgfscope}%
\pgfpathrectangle{\pgfqpoint{0.481944in}{1.034041in}}{\pgfqpoint{7.362500in}{2.695000in}}%
\pgfusepath{clip}%
\pgfsetbuttcap%
\pgfsetmiterjoin%
\definecolor{currentfill}{rgb}{0.548178,0.714944,0.584403}%
\pgfsetfillcolor{currentfill}%
\pgfsetlinewidth{0.000000pt}%
\definecolor{currentstroke}{rgb}{0.000000,0.000000,0.000000}%
\pgfsetstrokecolor{currentstroke}%
\pgfsetstrokeopacity{0.000000}%
\pgfsetdash{}{0pt}%
\pgfpathmoveto{\pgfqpoint{0.495091in}{1.034041in}}%
\pgfpathlineto{\pgfqpoint{0.600270in}{1.034041in}}%
\pgfpathlineto{\pgfqpoint{0.600270in}{3.276583in}}%
\pgfpathlineto{\pgfqpoint{0.495091in}{3.276583in}}%
\pgfpathlineto{\pgfqpoint{0.495091in}{1.034041in}}%
\pgfpathclose%
\pgfusepath{fill}%
\end{pgfscope}%
\begin{pgfscope}%
\pgfpathrectangle{\pgfqpoint{0.481944in}{1.034041in}}{\pgfqpoint{7.362500in}{2.695000in}}%
\pgfusepath{clip}%
\pgfsetbuttcap%
\pgfsetmiterjoin%
\definecolor{currentfill}{rgb}{0.614104,0.751679,0.595029}%
\pgfsetfillcolor{currentfill}%
\pgfsetlinewidth{0.000000pt}%
\definecolor{currentstroke}{rgb}{0.000000,0.000000,0.000000}%
\pgfsetstrokecolor{currentstroke}%
\pgfsetstrokeopacity{0.000000}%
\pgfsetdash{}{0pt}%
\pgfpathmoveto{\pgfqpoint{0.626565in}{1.034041in}}%
\pgfpathlineto{\pgfqpoint{0.731743in}{1.034041in}}%
\pgfpathlineto{\pgfqpoint{0.731743in}{2.504319in}}%
\pgfpathlineto{\pgfqpoint{0.626565in}{2.504319in}}%
\pgfpathlineto{\pgfqpoint{0.626565in}{1.034041in}}%
\pgfpathclose%
\pgfusepath{fill}%
\end{pgfscope}%
\begin{pgfscope}%
\pgfpathrectangle{\pgfqpoint{0.481944in}{1.034041in}}{\pgfqpoint{7.362500in}{2.695000in}}%
\pgfusepath{clip}%
\pgfsetbuttcap%
\pgfsetmiterjoin%
\definecolor{currentfill}{rgb}{0.487377,0.679908,0.570839}%
\pgfsetfillcolor{currentfill}%
\pgfsetlinewidth{0.000000pt}%
\definecolor{currentstroke}{rgb}{0.000000,0.000000,0.000000}%
\pgfsetstrokecolor{currentstroke}%
\pgfsetstrokeopacity{0.000000}%
\pgfsetdash{}{0pt}%
\pgfpathmoveto{\pgfqpoint{0.758038in}{1.034041in}}%
\pgfpathlineto{\pgfqpoint{0.863216in}{1.034041in}}%
\pgfpathlineto{\pgfqpoint{0.863216in}{3.431098in}}%
\pgfpathlineto{\pgfqpoint{0.758038in}{3.431098in}}%
\pgfpathlineto{\pgfqpoint{0.758038in}{1.034041in}}%
\pgfpathclose%
\pgfusepath{fill}%
\end{pgfscope}%
\begin{pgfscope}%
\pgfpathrectangle{\pgfqpoint{0.481944in}{1.034041in}}{\pgfqpoint{7.362500in}{2.695000in}}%
\pgfusepath{clip}%
\pgfsetbuttcap%
\pgfsetmiterjoin%
\definecolor{currentfill}{rgb}{0.556377,0.719609,0.586195}%
\pgfsetfillcolor{currentfill}%
\pgfsetlinewidth{0.000000pt}%
\definecolor{currentstroke}{rgb}{0.000000,0.000000,0.000000}%
\pgfsetstrokecolor{currentstroke}%
\pgfsetstrokeopacity{0.000000}%
\pgfsetdash{}{0pt}%
\pgfpathmoveto{\pgfqpoint{0.889511in}{1.034041in}}%
\pgfpathlineto{\pgfqpoint{0.994690in}{1.034041in}}%
\pgfpathlineto{\pgfqpoint{0.994690in}{2.067345in}}%
\pgfpathlineto{\pgfqpoint{0.889511in}{2.067345in}}%
\pgfpathlineto{\pgfqpoint{0.889511in}{1.034041in}}%
\pgfpathclose%
\pgfusepath{fill}%
\end{pgfscope}%
\begin{pgfscope}%
\pgfpathrectangle{\pgfqpoint{0.481944in}{1.034041in}}{\pgfqpoint{7.362500in}{2.695000in}}%
\pgfusepath{clip}%
\pgfsetbuttcap%
\pgfsetmiterjoin%
\definecolor{currentfill}{rgb}{0.596607,0.742633,0.594077}%
\pgfsetfillcolor{currentfill}%
\pgfsetlinewidth{0.000000pt}%
\definecolor{currentstroke}{rgb}{0.000000,0.000000,0.000000}%
\pgfsetstrokecolor{currentstroke}%
\pgfsetstrokeopacity{0.000000}%
\pgfsetdash{}{0pt}%
\pgfpathmoveto{\pgfqpoint{1.020984in}{1.034041in}}%
\pgfpathlineto{\pgfqpoint{1.126163in}{1.034041in}}%
\pgfpathlineto{\pgfqpoint{1.126163in}{2.074055in}}%
\pgfpathlineto{\pgfqpoint{1.020984in}{2.074055in}}%
\pgfpathlineto{\pgfqpoint{1.020984in}{1.034041in}}%
\pgfpathclose%
\pgfusepath{fill}%
\end{pgfscope}%
\begin{pgfscope}%
\pgfpathrectangle{\pgfqpoint{0.481944in}{1.034041in}}{\pgfqpoint{7.362500in}{2.695000in}}%
\pgfusepath{clip}%
\pgfsetbuttcap%
\pgfsetmiterjoin%
\definecolor{currentfill}{rgb}{0.635089,0.762521,0.595670}%
\pgfsetfillcolor{currentfill}%
\pgfsetlinewidth{0.000000pt}%
\definecolor{currentstroke}{rgb}{0.000000,0.000000,0.000000}%
\pgfsetstrokecolor{currentstroke}%
\pgfsetstrokeopacity{0.000000}%
\pgfsetdash{}{0pt}%
\pgfpathmoveto{\pgfqpoint{1.152458in}{1.034041in}}%
\pgfpathlineto{\pgfqpoint{1.257636in}{1.034041in}}%
\pgfpathlineto{\pgfqpoint{1.257636in}{2.105477in}}%
\pgfpathlineto{\pgfqpoint{1.152458in}{2.105477in}}%
\pgfpathlineto{\pgfqpoint{1.152458in}{1.034041in}}%
\pgfpathclose%
\pgfusepath{fill}%
\end{pgfscope}%
\begin{pgfscope}%
\pgfpathrectangle{\pgfqpoint{0.481944in}{1.034041in}}{\pgfqpoint{7.362500in}{2.695000in}}%
\pgfusepath{clip}%
\pgfsetbuttcap%
\pgfsetmiterjoin%
\definecolor{currentfill}{rgb}{0.652651,0.771509,0.595697}%
\pgfsetfillcolor{currentfill}%
\pgfsetlinewidth{0.000000pt}%
\definecolor{currentstroke}{rgb}{0.000000,0.000000,0.000000}%
\pgfsetstrokecolor{currentstroke}%
\pgfsetstrokeopacity{0.000000}%
\pgfsetdash{}{0pt}%
\pgfpathmoveto{\pgfqpoint{1.283931in}{1.034041in}}%
\pgfpathlineto{\pgfqpoint{1.389109in}{1.034041in}}%
\pgfpathlineto{\pgfqpoint{1.389109in}{2.248898in}}%
\pgfpathlineto{\pgfqpoint{1.283931in}{2.248898in}}%
\pgfpathlineto{\pgfqpoint{1.283931in}{1.034041in}}%
\pgfpathclose%
\pgfusepath{fill}%
\end{pgfscope}%
\begin{pgfscope}%
\pgfpathrectangle{\pgfqpoint{0.481944in}{1.034041in}}{\pgfqpoint{7.362500in}{2.695000in}}%
\pgfusepath{clip}%
\pgfsetbuttcap%
\pgfsetmiterjoin%
\definecolor{currentfill}{rgb}{0.603611,0.746252,0.594501}%
\pgfsetfillcolor{currentfill}%
\pgfsetlinewidth{0.000000pt}%
\definecolor{currentstroke}{rgb}{0.000000,0.000000,0.000000}%
\pgfsetstrokecolor{currentstroke}%
\pgfsetstrokeopacity{0.000000}%
\pgfsetdash{}{0pt}%
\pgfpathmoveto{\pgfqpoint{1.415404in}{1.034041in}}%
\pgfpathlineto{\pgfqpoint{1.520583in}{1.034041in}}%
\pgfpathlineto{\pgfqpoint{1.520583in}{1.886176in}}%
\pgfpathlineto{\pgfqpoint{1.415404in}{1.886176in}}%
\pgfpathlineto{\pgfqpoint{1.415404in}{1.034041in}}%
\pgfpathclose%
\pgfusepath{fill}%
\end{pgfscope}%
\begin{pgfscope}%
\pgfpathrectangle{\pgfqpoint{0.481944in}{1.034041in}}{\pgfqpoint{7.362500in}{2.695000in}}%
\pgfusepath{clip}%
\pgfsetbuttcap%
\pgfsetmiterjoin%
\definecolor{currentfill}{rgb}{0.568623,0.726620,0.588802}%
\pgfsetfillcolor{currentfill}%
\pgfsetlinewidth{0.000000pt}%
\definecolor{currentstroke}{rgb}{0.000000,0.000000,0.000000}%
\pgfsetstrokecolor{currentstroke}%
\pgfsetstrokeopacity{0.000000}%
\pgfsetdash{}{0pt}%
\pgfpathmoveto{\pgfqpoint{1.546877in}{1.034041in}}%
\pgfpathlineto{\pgfqpoint{1.652056in}{1.034041in}}%
\pgfpathlineto{\pgfqpoint{1.652056in}{1.967407in}}%
\pgfpathlineto{\pgfqpoint{1.546877in}{1.967407in}}%
\pgfpathlineto{\pgfqpoint{1.546877in}{1.034041in}}%
\pgfpathclose%
\pgfusepath{fill}%
\end{pgfscope}%
\begin{pgfscope}%
\pgfpathrectangle{\pgfqpoint{0.481944in}{1.034041in}}{\pgfqpoint{7.362500in}{2.695000in}}%
\pgfusepath{clip}%
\pgfsetbuttcap%
\pgfsetmiterjoin%
\definecolor{currentfill}{rgb}{0.649123,0.769720,0.595746}%
\pgfsetfillcolor{currentfill}%
\pgfsetlinewidth{0.000000pt}%
\definecolor{currentstroke}{rgb}{0.000000,0.000000,0.000000}%
\pgfsetstrokecolor{currentstroke}%
\pgfsetstrokeopacity{0.000000}%
\pgfsetdash{}{0pt}%
\pgfpathmoveto{\pgfqpoint{1.678350in}{1.034041in}}%
\pgfpathlineto{\pgfqpoint{1.783529in}{1.034041in}}%
\pgfpathlineto{\pgfqpoint{1.783529in}{2.606209in}}%
\pgfpathlineto{\pgfqpoint{1.678350in}{2.606209in}}%
\pgfpathlineto{\pgfqpoint{1.678350in}{1.034041in}}%
\pgfpathclose%
\pgfusepath{fill}%
\end{pgfscope}%
\begin{pgfscope}%
\pgfpathrectangle{\pgfqpoint{0.481944in}{1.034041in}}{\pgfqpoint{7.362500in}{2.695000in}}%
\pgfusepath{clip}%
\pgfsetbuttcap%
\pgfsetmiterjoin%
\definecolor{currentfill}{rgb}{0.642098,0.766125,0.595747}%
\pgfsetfillcolor{currentfill}%
\pgfsetlinewidth{0.000000pt}%
\definecolor{currentstroke}{rgb}{0.000000,0.000000,0.000000}%
\pgfsetstrokecolor{currentstroke}%
\pgfsetstrokeopacity{0.000000}%
\pgfsetdash{}{0pt}%
\pgfpathmoveto{\pgfqpoint{1.809824in}{1.034041in}}%
\pgfpathlineto{\pgfqpoint{1.915002in}{1.034041in}}%
\pgfpathlineto{\pgfqpoint{1.915002in}{2.188155in}}%
\pgfpathlineto{\pgfqpoint{1.809824in}{2.188155in}}%
\pgfpathlineto{\pgfqpoint{1.809824in}{1.034041in}}%
\pgfpathclose%
\pgfusepath{fill}%
\end{pgfscope}%
\begin{pgfscope}%
\pgfpathrectangle{\pgfqpoint{0.481944in}{1.034041in}}{\pgfqpoint{7.362500in}{2.695000in}}%
\pgfusepath{clip}%
\pgfsetbuttcap%
\pgfsetmiterjoin%
\definecolor{currentfill}{rgb}{0.652651,0.771509,0.595697}%
\pgfsetfillcolor{currentfill}%
\pgfsetlinewidth{0.000000pt}%
\definecolor{currentstroke}{rgb}{0.000000,0.000000,0.000000}%
\pgfsetstrokecolor{currentstroke}%
\pgfsetstrokeopacity{0.000000}%
\pgfsetdash{}{0pt}%
\pgfpathmoveto{\pgfqpoint{1.941297in}{1.034041in}}%
\pgfpathlineto{\pgfqpoint{2.046475in}{1.034041in}}%
\pgfpathlineto{\pgfqpoint{2.046475in}{1.908738in}}%
\pgfpathlineto{\pgfqpoint{1.941297in}{1.908738in}}%
\pgfpathlineto{\pgfqpoint{1.941297in}{1.034041in}}%
\pgfpathclose%
\pgfusepath{fill}%
\end{pgfscope}%
\begin{pgfscope}%
\pgfpathrectangle{\pgfqpoint{0.481944in}{1.034041in}}{\pgfqpoint{7.362500in}{2.695000in}}%
\pgfusepath{clip}%
\pgfsetbuttcap%
\pgfsetmiterjoin%
\definecolor{currentfill}{rgb}{0.635089,0.762521,0.595670}%
\pgfsetfillcolor{currentfill}%
\pgfsetlinewidth{0.000000pt}%
\definecolor{currentstroke}{rgb}{0.000000,0.000000,0.000000}%
\pgfsetstrokecolor{currentstroke}%
\pgfsetstrokeopacity{0.000000}%
\pgfsetdash{}{0pt}%
\pgfpathmoveto{\pgfqpoint{2.072770in}{1.034041in}}%
\pgfpathlineto{\pgfqpoint{2.177949in}{1.034041in}}%
\pgfpathlineto{\pgfqpoint{2.177949in}{1.901796in}}%
\pgfpathlineto{\pgfqpoint{2.072770in}{1.901796in}}%
\pgfpathlineto{\pgfqpoint{2.072770in}{1.034041in}}%
\pgfpathclose%
\pgfusepath{fill}%
\end{pgfscope}%
\begin{pgfscope}%
\pgfpathrectangle{\pgfqpoint{0.481944in}{1.034041in}}{\pgfqpoint{7.362500in}{2.695000in}}%
\pgfusepath{clip}%
\pgfsetbuttcap%
\pgfsetmiterjoin%
\definecolor{currentfill}{rgb}{0.652651,0.771509,0.595697}%
\pgfsetfillcolor{currentfill}%
\pgfsetlinewidth{0.000000pt}%
\definecolor{currentstroke}{rgb}{0.000000,0.000000,0.000000}%
\pgfsetstrokecolor{currentstroke}%
\pgfsetstrokeopacity{0.000000}%
\pgfsetdash{}{0pt}%
\pgfpathmoveto{\pgfqpoint{2.204243in}{1.034041in}}%
\pgfpathlineto{\pgfqpoint{2.309422in}{1.034041in}}%
\pgfpathlineto{\pgfqpoint{2.309422in}{2.495666in}}%
\pgfpathlineto{\pgfqpoint{2.204243in}{2.495666in}}%
\pgfpathlineto{\pgfqpoint{2.204243in}{1.034041in}}%
\pgfpathclose%
\pgfusepath{fill}%
\end{pgfscope}%
\begin{pgfscope}%
\pgfpathrectangle{\pgfqpoint{0.481944in}{1.034041in}}{\pgfqpoint{7.362500in}{2.695000in}}%
\pgfusepath{clip}%
\pgfsetbuttcap%
\pgfsetmiterjoin%
\definecolor{currentfill}{rgb}{0.617599,0.753489,0.595178}%
\pgfsetfillcolor{currentfill}%
\pgfsetlinewidth{0.000000pt}%
\definecolor{currentstroke}{rgb}{0.000000,0.000000,0.000000}%
\pgfsetstrokecolor{currentstroke}%
\pgfsetstrokeopacity{0.000000}%
\pgfsetdash{}{0pt}%
\pgfpathmoveto{\pgfqpoint{2.335716in}{1.034041in}}%
\pgfpathlineto{\pgfqpoint{2.440895in}{1.034041in}}%
\pgfpathlineto{\pgfqpoint{2.440895in}{1.836305in}}%
\pgfpathlineto{\pgfqpoint{2.335716in}{1.836305in}}%
\pgfpathlineto{\pgfqpoint{2.335716in}{1.034041in}}%
\pgfpathclose%
\pgfusepath{fill}%
\end{pgfscope}%
\begin{pgfscope}%
\pgfpathrectangle{\pgfqpoint{0.481944in}{1.034041in}}{\pgfqpoint{7.362500in}{2.695000in}}%
\pgfusepath{clip}%
\pgfsetbuttcap%
\pgfsetmiterjoin%
\definecolor{currentfill}{rgb}{0.624593,0.757104,0.595421}%
\pgfsetfillcolor{currentfill}%
\pgfsetlinewidth{0.000000pt}%
\definecolor{currentstroke}{rgb}{0.000000,0.000000,0.000000}%
\pgfsetstrokecolor{currentstroke}%
\pgfsetstrokeopacity{0.000000}%
\pgfsetdash{}{0pt}%
\pgfpathmoveto{\pgfqpoint{2.467190in}{1.034041in}}%
\pgfpathlineto{\pgfqpoint{2.572368in}{1.034041in}}%
\pgfpathlineto{\pgfqpoint{2.572368in}{1.942345in}}%
\pgfpathlineto{\pgfqpoint{2.467190in}{1.942345in}}%
\pgfpathlineto{\pgfqpoint{2.467190in}{1.034041in}}%
\pgfpathclose%
\pgfusepath{fill}%
\end{pgfscope}%
\begin{pgfscope}%
\pgfpathrectangle{\pgfqpoint{0.481944in}{1.034041in}}{\pgfqpoint{7.362500in}{2.695000in}}%
\pgfusepath{clip}%
\pgfsetbuttcap%
\pgfsetmiterjoin%
\definecolor{currentfill}{rgb}{0.635089,0.762521,0.595670}%
\pgfsetfillcolor{currentfill}%
\pgfsetlinewidth{0.000000pt}%
\definecolor{currentstroke}{rgb}{0.000000,0.000000,0.000000}%
\pgfsetstrokecolor{currentstroke}%
\pgfsetstrokeopacity{0.000000}%
\pgfsetdash{}{0pt}%
\pgfpathmoveto{\pgfqpoint{2.598663in}{1.034041in}}%
\pgfpathlineto{\pgfqpoint{2.703841in}{1.034041in}}%
\pgfpathlineto{\pgfqpoint{2.703841in}{2.111602in}}%
\pgfpathlineto{\pgfqpoint{2.598663in}{2.111602in}}%
\pgfpathlineto{\pgfqpoint{2.598663in}{1.034041in}}%
\pgfpathclose%
\pgfusepath{fill}%
\end{pgfscope}%
\begin{pgfscope}%
\pgfpathrectangle{\pgfqpoint{0.481944in}{1.034041in}}{\pgfqpoint{7.362500in}{2.695000in}}%
\pgfusepath{clip}%
\pgfsetbuttcap%
\pgfsetmiterjoin%
\definecolor{currentfill}{rgb}{0.207614,0.220467,0.411508}%
\pgfsetfillcolor{currentfill}%
\pgfsetlinewidth{0.000000pt}%
\definecolor{currentstroke}{rgb}{0.000000,0.000000,0.000000}%
\pgfsetstrokecolor{currentstroke}%
\pgfsetstrokeopacity{0.000000}%
\pgfsetdash{}{0pt}%
\pgfpathmoveto{\pgfqpoint{2.730136in}{1.034041in}}%
\pgfpathlineto{\pgfqpoint{2.835315in}{1.034041in}}%
\pgfpathlineto{\pgfqpoint{2.835315in}{3.047539in}}%
\pgfpathlineto{\pgfqpoint{2.730136in}{3.047539in}}%
\pgfpathlineto{\pgfqpoint{2.730136in}{1.034041in}}%
\pgfpathclose%
\pgfusepath{fill}%
\end{pgfscope}%
\begin{pgfscope}%
\pgfpathrectangle{\pgfqpoint{0.481944in}{1.034041in}}{\pgfqpoint{7.362500in}{2.695000in}}%
\pgfusepath{clip}%
\pgfsetbuttcap%
\pgfsetmiterjoin%
\definecolor{currentfill}{rgb}{0.638590,0.764325,0.595722}%
\pgfsetfillcolor{currentfill}%
\pgfsetlinewidth{0.000000pt}%
\definecolor{currentstroke}{rgb}{0.000000,0.000000,0.000000}%
\pgfsetstrokecolor{currentstroke}%
\pgfsetstrokeopacity{0.000000}%
\pgfsetdash{}{0pt}%
\pgfpathmoveto{\pgfqpoint{2.861609in}{1.034041in}}%
\pgfpathlineto{\pgfqpoint{2.966788in}{1.034041in}}%
\pgfpathlineto{\pgfqpoint{2.966788in}{2.023413in}}%
\pgfpathlineto{\pgfqpoint{2.861609in}{2.023413in}}%
\pgfpathlineto{\pgfqpoint{2.861609in}{1.034041in}}%
\pgfpathclose%
\pgfusepath{fill}%
\end{pgfscope}%
\begin{pgfscope}%
\pgfpathrectangle{\pgfqpoint{0.481944in}{1.034041in}}{\pgfqpoint{7.362500in}{2.695000in}}%
\pgfusepath{clip}%
\pgfsetbuttcap%
\pgfsetmiterjoin%
\definecolor{currentfill}{rgb}{0.652651,0.771509,0.595697}%
\pgfsetfillcolor{currentfill}%
\pgfsetlinewidth{0.000000pt}%
\definecolor{currentstroke}{rgb}{0.000000,0.000000,0.000000}%
\pgfsetstrokecolor{currentstroke}%
\pgfsetstrokeopacity{0.000000}%
\pgfsetdash{}{0pt}%
\pgfpathmoveto{\pgfqpoint{2.993083in}{1.034041in}}%
\pgfpathlineto{\pgfqpoint{3.098261in}{1.034041in}}%
\pgfpathlineto{\pgfqpoint{3.098261in}{2.022878in}}%
\pgfpathlineto{\pgfqpoint{2.993083in}{2.022878in}}%
\pgfpathlineto{\pgfqpoint{2.993083in}{1.034041in}}%
\pgfpathclose%
\pgfusepath{fill}%
\end{pgfscope}%
\begin{pgfscope}%
\pgfpathrectangle{\pgfqpoint{0.481944in}{1.034041in}}{\pgfqpoint{7.362500in}{2.695000in}}%
\pgfusepath{clip}%
\pgfsetbuttcap%
\pgfsetmiterjoin%
\definecolor{currentfill}{rgb}{0.614104,0.751679,0.595029}%
\pgfsetfillcolor{currentfill}%
\pgfsetlinewidth{0.000000pt}%
\definecolor{currentstroke}{rgb}{0.000000,0.000000,0.000000}%
\pgfsetstrokecolor{currentstroke}%
\pgfsetstrokeopacity{0.000000}%
\pgfsetdash{}{0pt}%
\pgfpathmoveto{\pgfqpoint{3.124556in}{1.034041in}}%
\pgfpathlineto{\pgfqpoint{3.229734in}{1.034041in}}%
\pgfpathlineto{\pgfqpoint{3.229734in}{2.229981in}}%
\pgfpathlineto{\pgfqpoint{3.124556in}{2.229981in}}%
\pgfpathlineto{\pgfqpoint{3.124556in}{1.034041in}}%
\pgfpathclose%
\pgfusepath{fill}%
\end{pgfscope}%
\begin{pgfscope}%
\pgfpathrectangle{\pgfqpoint{0.481944in}{1.034041in}}{\pgfqpoint{7.362500in}{2.695000in}}%
\pgfusepath{clip}%
\pgfsetbuttcap%
\pgfsetmiterjoin%
\definecolor{currentfill}{rgb}{0.491283,0.682262,0.571685}%
\pgfsetfillcolor{currentfill}%
\pgfsetlinewidth{0.000000pt}%
\definecolor{currentstroke}{rgb}{0.000000,0.000000,0.000000}%
\pgfsetstrokecolor{currentstroke}%
\pgfsetstrokeopacity{0.000000}%
\pgfsetdash{}{0pt}%
\pgfpathmoveto{\pgfqpoint{3.256029in}{1.034041in}}%
\pgfpathlineto{\pgfqpoint{3.361208in}{1.034041in}}%
\pgfpathlineto{\pgfqpoint{3.361208in}{2.786941in}}%
\pgfpathlineto{\pgfqpoint{3.256029in}{2.786941in}}%
\pgfpathlineto{\pgfqpoint{3.256029in}{1.034041in}}%
\pgfpathclose%
\pgfusepath{fill}%
\end{pgfscope}%
\begin{pgfscope}%
\pgfpathrectangle{\pgfqpoint{0.481944in}{1.034041in}}{\pgfqpoint{7.362500in}{2.695000in}}%
\pgfusepath{clip}%
\pgfsetbuttcap%
\pgfsetmiterjoin%
\definecolor{currentfill}{rgb}{0.652651,0.771509,0.595697}%
\pgfsetfillcolor{currentfill}%
\pgfsetlinewidth{0.000000pt}%
\definecolor{currentstroke}{rgb}{0.000000,0.000000,0.000000}%
\pgfsetstrokecolor{currentstroke}%
\pgfsetstrokeopacity{0.000000}%
\pgfsetdash{}{0pt}%
\pgfpathmoveto{\pgfqpoint{3.387502in}{1.034041in}}%
\pgfpathlineto{\pgfqpoint{3.492681in}{1.034041in}}%
\pgfpathlineto{\pgfqpoint{3.492681in}{1.935980in}}%
\pgfpathlineto{\pgfqpoint{3.387502in}{1.935980in}}%
\pgfpathlineto{\pgfqpoint{3.387502in}{1.034041in}}%
\pgfpathclose%
\pgfusepath{fill}%
\end{pgfscope}%
\begin{pgfscope}%
\pgfpathrectangle{\pgfqpoint{0.481944in}{1.034041in}}{\pgfqpoint{7.362500in}{2.695000in}}%
\pgfusepath{clip}%
\pgfsetbuttcap%
\pgfsetmiterjoin%
\definecolor{currentfill}{rgb}{0.638590,0.764325,0.595722}%
\pgfsetfillcolor{currentfill}%
\pgfsetlinewidth{0.000000pt}%
\definecolor{currentstroke}{rgb}{0.000000,0.000000,0.000000}%
\pgfsetstrokecolor{currentstroke}%
\pgfsetstrokeopacity{0.000000}%
\pgfsetdash{}{0pt}%
\pgfpathmoveto{\pgfqpoint{3.518975in}{1.034041in}}%
\pgfpathlineto{\pgfqpoint{3.624154in}{1.034041in}}%
\pgfpathlineto{\pgfqpoint{3.624154in}{1.968911in}}%
\pgfpathlineto{\pgfqpoint{3.518975in}{1.968911in}}%
\pgfpathlineto{\pgfqpoint{3.518975in}{1.034041in}}%
\pgfpathclose%
\pgfusepath{fill}%
\end{pgfscope}%
\begin{pgfscope}%
\pgfpathrectangle{\pgfqpoint{0.481944in}{1.034041in}}{\pgfqpoint{7.362500in}{2.695000in}}%
\pgfusepath{clip}%
\pgfsetbuttcap%
\pgfsetmiterjoin%
\definecolor{currentfill}{rgb}{0.642098,0.766125,0.595747}%
\pgfsetfillcolor{currentfill}%
\pgfsetlinewidth{0.000000pt}%
\definecolor{currentstroke}{rgb}{0.000000,0.000000,0.000000}%
\pgfsetstrokecolor{currentstroke}%
\pgfsetstrokeopacity{0.000000}%
\pgfsetdash{}{0pt}%
\pgfpathmoveto{\pgfqpoint{3.650449in}{1.034041in}}%
\pgfpathlineto{\pgfqpoint{3.755627in}{1.034041in}}%
\pgfpathlineto{\pgfqpoint{3.755627in}{2.104868in}}%
\pgfpathlineto{\pgfqpoint{3.650449in}{2.104868in}}%
\pgfpathlineto{\pgfqpoint{3.650449in}{1.034041in}}%
\pgfpathclose%
\pgfusepath{fill}%
\end{pgfscope}%
\begin{pgfscope}%
\pgfpathrectangle{\pgfqpoint{0.481944in}{1.034041in}}{\pgfqpoint{7.362500in}{2.695000in}}%
\pgfusepath{clip}%
\pgfsetbuttcap%
\pgfsetmiterjoin%
\definecolor{currentfill}{rgb}{0.624593,0.757104,0.595421}%
\pgfsetfillcolor{currentfill}%
\pgfsetlinewidth{0.000000pt}%
\definecolor{currentstroke}{rgb}{0.000000,0.000000,0.000000}%
\pgfsetstrokecolor{currentstroke}%
\pgfsetstrokeopacity{0.000000}%
\pgfsetdash{}{0pt}%
\pgfpathmoveto{\pgfqpoint{3.781922in}{1.034041in}}%
\pgfpathlineto{\pgfqpoint{3.887100in}{1.034041in}}%
\pgfpathlineto{\pgfqpoint{3.887100in}{2.704469in}}%
\pgfpathlineto{\pgfqpoint{3.781922in}{2.704469in}}%
\pgfpathlineto{\pgfqpoint{3.781922in}{1.034041in}}%
\pgfpathclose%
\pgfusepath{fill}%
\end{pgfscope}%
\begin{pgfscope}%
\pgfpathrectangle{\pgfqpoint{0.481944in}{1.034041in}}{\pgfqpoint{7.362500in}{2.695000in}}%
\pgfusepath{clip}%
\pgfsetbuttcap%
\pgfsetmiterjoin%
\definecolor{currentfill}{rgb}{0.610607,0.749870,0.594868}%
\pgfsetfillcolor{currentfill}%
\pgfsetlinewidth{0.000000pt}%
\definecolor{currentstroke}{rgb}{0.000000,0.000000,0.000000}%
\pgfsetstrokecolor{currentstroke}%
\pgfsetstrokeopacity{0.000000}%
\pgfsetdash{}{0pt}%
\pgfpathmoveto{\pgfqpoint{3.913395in}{1.034041in}}%
\pgfpathlineto{\pgfqpoint{4.018574in}{1.034041in}}%
\pgfpathlineto{\pgfqpoint{4.018574in}{1.877043in}}%
\pgfpathlineto{\pgfqpoint{3.913395in}{1.877043in}}%
\pgfpathlineto{\pgfqpoint{3.913395in}{1.034041in}}%
\pgfpathclose%
\pgfusepath{fill}%
\end{pgfscope}%
\begin{pgfscope}%
\pgfpathrectangle{\pgfqpoint{0.481944in}{1.034041in}}{\pgfqpoint{7.362500in}{2.695000in}}%
\pgfusepath{clip}%
\pgfsetbuttcap%
\pgfsetmiterjoin%
\definecolor{currentfill}{rgb}{0.656222,0.773271,0.595544}%
\pgfsetfillcolor{currentfill}%
\pgfsetlinewidth{0.000000pt}%
\definecolor{currentstroke}{rgb}{0.000000,0.000000,0.000000}%
\pgfsetstrokecolor{currentstroke}%
\pgfsetstrokeopacity{0.000000}%
\pgfsetdash{}{0pt}%
\pgfpathmoveto{\pgfqpoint{4.044868in}{1.034041in}}%
\pgfpathlineto{\pgfqpoint{4.150047in}{1.034041in}}%
\pgfpathlineto{\pgfqpoint{4.150047in}{1.641469in}}%
\pgfpathlineto{\pgfqpoint{4.044868in}{1.641469in}}%
\pgfpathlineto{\pgfqpoint{4.044868in}{1.034041in}}%
\pgfpathclose%
\pgfusepath{fill}%
\end{pgfscope}%
\begin{pgfscope}%
\pgfpathrectangle{\pgfqpoint{0.481944in}{1.034041in}}{\pgfqpoint{7.362500in}{2.695000in}}%
\pgfusepath{clip}%
\pgfsetbuttcap%
\pgfsetmiterjoin%
\definecolor{currentfill}{rgb}{0.649123,0.769720,0.595746}%
\pgfsetfillcolor{currentfill}%
\pgfsetlinewidth{0.000000pt}%
\definecolor{currentstroke}{rgb}{0.000000,0.000000,0.000000}%
\pgfsetstrokecolor{currentstroke}%
\pgfsetstrokeopacity{0.000000}%
\pgfsetdash{}{0pt}%
\pgfpathmoveto{\pgfqpoint{4.176341in}{1.034041in}}%
\pgfpathlineto{\pgfqpoint{4.281520in}{1.034041in}}%
\pgfpathlineto{\pgfqpoint{4.281520in}{2.123556in}}%
\pgfpathlineto{\pgfqpoint{4.176341in}{2.123556in}}%
\pgfpathlineto{\pgfqpoint{4.176341in}{1.034041in}}%
\pgfpathclose%
\pgfusepath{fill}%
\end{pgfscope}%
\begin{pgfscope}%
\pgfpathrectangle{\pgfqpoint{0.481944in}{1.034041in}}{\pgfqpoint{7.362500in}{2.695000in}}%
\pgfusepath{clip}%
\pgfsetbuttcap%
\pgfsetmiterjoin%
\definecolor{currentfill}{rgb}{0.165098,0.366532,0.479263}%
\pgfsetfillcolor{currentfill}%
\pgfsetlinewidth{0.000000pt}%
\definecolor{currentstroke}{rgb}{0.000000,0.000000,0.000000}%
\pgfsetstrokecolor{currentstroke}%
\pgfsetstrokeopacity{0.000000}%
\pgfsetdash{}{0pt}%
\pgfpathmoveto{\pgfqpoint{4.307815in}{1.034041in}}%
\pgfpathlineto{\pgfqpoint{4.412993in}{1.034041in}}%
\pgfpathlineto{\pgfqpoint{4.412993in}{3.360345in}}%
\pgfpathlineto{\pgfqpoint{4.307815in}{3.360345in}}%
\pgfpathlineto{\pgfqpoint{4.307815in}{1.034041in}}%
\pgfpathclose%
\pgfusepath{fill}%
\end{pgfscope}%
\begin{pgfscope}%
\pgfpathrectangle{\pgfqpoint{0.481944in}{1.034041in}}{\pgfqpoint{7.362500in}{2.695000in}}%
\pgfusepath{clip}%
\pgfsetbuttcap%
\pgfsetmiterjoin%
\definecolor{currentfill}{rgb}{0.656222,0.773271,0.595544}%
\pgfsetfillcolor{currentfill}%
\pgfsetlinewidth{0.000000pt}%
\definecolor{currentstroke}{rgb}{0.000000,0.000000,0.000000}%
\pgfsetstrokecolor{currentstroke}%
\pgfsetstrokeopacity{0.000000}%
\pgfsetdash{}{0pt}%
\pgfpathmoveto{\pgfqpoint{4.439288in}{1.034041in}}%
\pgfpathlineto{\pgfqpoint{4.544466in}{1.034041in}}%
\pgfpathlineto{\pgfqpoint{4.544466in}{1.810199in}}%
\pgfpathlineto{\pgfqpoint{4.439288in}{1.810199in}}%
\pgfpathlineto{\pgfqpoint{4.439288in}{1.034041in}}%
\pgfpathclose%
\pgfusepath{fill}%
\end{pgfscope}%
\begin{pgfscope}%
\pgfpathrectangle{\pgfqpoint{0.481944in}{1.034041in}}{\pgfqpoint{7.362500in}{2.695000in}}%
\pgfusepath{clip}%
\pgfsetbuttcap%
\pgfsetmiterjoin%
\definecolor{currentfill}{rgb}{0.560465,0.721944,0.587075}%
\pgfsetfillcolor{currentfill}%
\pgfsetlinewidth{0.000000pt}%
\definecolor{currentstroke}{rgb}{0.000000,0.000000,0.000000}%
\pgfsetstrokecolor{currentstroke}%
\pgfsetstrokeopacity{0.000000}%
\pgfsetdash{}{0pt}%
\pgfpathmoveto{\pgfqpoint{4.570761in}{1.034041in}}%
\pgfpathlineto{\pgfqpoint{4.675940in}{1.034041in}}%
\pgfpathlineto{\pgfqpoint{4.675940in}{2.155600in}}%
\pgfpathlineto{\pgfqpoint{4.570761in}{2.155600in}}%
\pgfpathlineto{\pgfqpoint{4.570761in}{1.034041in}}%
\pgfpathclose%
\pgfusepath{fill}%
\end{pgfscope}%
\begin{pgfscope}%
\pgfpathrectangle{\pgfqpoint{0.481944in}{1.034041in}}{\pgfqpoint{7.362500in}{2.695000in}}%
\pgfusepath{clip}%
\pgfsetbuttcap%
\pgfsetmiterjoin%
\definecolor{currentfill}{rgb}{0.523411,0.700989,0.578806}%
\pgfsetfillcolor{currentfill}%
\pgfsetlinewidth{0.000000pt}%
\definecolor{currentstroke}{rgb}{0.000000,0.000000,0.000000}%
\pgfsetstrokecolor{currentstroke}%
\pgfsetstrokeopacity{0.000000}%
\pgfsetdash{}{0pt}%
\pgfpathmoveto{\pgfqpoint{4.702234in}{1.034041in}}%
\pgfpathlineto{\pgfqpoint{4.807413in}{1.034041in}}%
\pgfpathlineto{\pgfqpoint{4.807413in}{2.131662in}}%
\pgfpathlineto{\pgfqpoint{4.702234in}{2.131662in}}%
\pgfpathlineto{\pgfqpoint{4.702234in}{1.034041in}}%
\pgfpathclose%
\pgfusepath{fill}%
\end{pgfscope}%
\begin{pgfscope}%
\pgfpathrectangle{\pgfqpoint{0.481944in}{1.034041in}}{\pgfqpoint{7.362500in}{2.695000in}}%
\pgfusepath{clip}%
\pgfsetbuttcap%
\pgfsetmiterjoin%
\definecolor{currentfill}{rgb}{0.564547,0.724281,0.587944}%
\pgfsetfillcolor{currentfill}%
\pgfsetlinewidth{0.000000pt}%
\definecolor{currentstroke}{rgb}{0.000000,0.000000,0.000000}%
\pgfsetstrokecolor{currentstroke}%
\pgfsetstrokeopacity{0.000000}%
\pgfsetdash{}{0pt}%
\pgfpathmoveto{\pgfqpoint{4.833708in}{1.034041in}}%
\pgfpathlineto{\pgfqpoint{4.938886in}{1.034041in}}%
\pgfpathlineto{\pgfqpoint{4.938886in}{2.149202in}}%
\pgfpathlineto{\pgfqpoint{4.833708in}{2.149202in}}%
\pgfpathlineto{\pgfqpoint{4.833708in}{1.034041in}}%
\pgfpathclose%
\pgfusepath{fill}%
\end{pgfscope}%
\begin{pgfscope}%
\pgfpathrectangle{\pgfqpoint{0.481944in}{1.034041in}}{\pgfqpoint{7.362500in}{2.695000in}}%
\pgfusepath{clip}%
\pgfsetbuttcap%
\pgfsetmiterjoin%
\definecolor{currentfill}{rgb}{0.588929,0.738349,0.592927}%
\pgfsetfillcolor{currentfill}%
\pgfsetlinewidth{0.000000pt}%
\definecolor{currentstroke}{rgb}{0.000000,0.000000,0.000000}%
\pgfsetstrokecolor{currentstroke}%
\pgfsetstrokeopacity{0.000000}%
\pgfsetdash{}{0pt}%
\pgfpathmoveto{\pgfqpoint{4.965181in}{1.034041in}}%
\pgfpathlineto{\pgfqpoint{5.070359in}{1.034041in}}%
\pgfpathlineto{\pgfqpoint{5.070359in}{3.214519in}}%
\pgfpathlineto{\pgfqpoint{4.965181in}{3.214519in}}%
\pgfpathlineto{\pgfqpoint{4.965181in}{1.034041in}}%
\pgfpathclose%
\pgfusepath{fill}%
\end{pgfscope}%
\begin{pgfscope}%
\pgfpathrectangle{\pgfqpoint{0.481944in}{1.034041in}}{\pgfqpoint{7.362500in}{2.695000in}}%
\pgfusepath{clip}%
\pgfsetbuttcap%
\pgfsetmiterjoin%
\definecolor{currentfill}{rgb}{0.656222,0.773271,0.595544}%
\pgfsetfillcolor{currentfill}%
\pgfsetlinewidth{0.000000pt}%
\definecolor{currentstroke}{rgb}{0.000000,0.000000,0.000000}%
\pgfsetstrokecolor{currentstroke}%
\pgfsetstrokeopacity{0.000000}%
\pgfsetdash{}{0pt}%
\pgfpathmoveto{\pgfqpoint{5.096654in}{1.034041in}}%
\pgfpathlineto{\pgfqpoint{5.201833in}{1.034041in}}%
\pgfpathlineto{\pgfqpoint{5.201833in}{1.641469in}}%
\pgfpathlineto{\pgfqpoint{5.096654in}{1.641469in}}%
\pgfpathlineto{\pgfqpoint{5.096654in}{1.034041in}}%
\pgfpathclose%
\pgfusepath{fill}%
\end{pgfscope}%
\begin{pgfscope}%
\pgfpathrectangle{\pgfqpoint{0.481944in}{1.034041in}}{\pgfqpoint{7.362500in}{2.695000in}}%
\pgfusepath{clip}%
\pgfsetbuttcap%
\pgfsetmiterjoin%
\definecolor{currentfill}{rgb}{0.656222,0.773271,0.595544}%
\pgfsetfillcolor{currentfill}%
\pgfsetlinewidth{0.000000pt}%
\definecolor{currentstroke}{rgb}{0.000000,0.000000,0.000000}%
\pgfsetstrokecolor{currentstroke}%
\pgfsetstrokeopacity{0.000000}%
\pgfsetdash{}{0pt}%
\pgfpathmoveto{\pgfqpoint{5.228127in}{1.034041in}}%
\pgfpathlineto{\pgfqpoint{5.333306in}{1.034041in}}%
\pgfpathlineto{\pgfqpoint{5.333306in}{2.504657in}}%
\pgfpathlineto{\pgfqpoint{5.228127in}{2.504657in}}%
\pgfpathlineto{\pgfqpoint{5.228127in}{1.034041in}}%
\pgfpathclose%
\pgfusepath{fill}%
\end{pgfscope}%
\begin{pgfscope}%
\pgfpathrectangle{\pgfqpoint{0.481944in}{1.034041in}}{\pgfqpoint{7.362500in}{2.695000in}}%
\pgfusepath{clip}%
\pgfsetbuttcap%
\pgfsetmiterjoin%
\definecolor{currentfill}{rgb}{0.652651,0.771509,0.595697}%
\pgfsetfillcolor{currentfill}%
\pgfsetlinewidth{0.000000pt}%
\definecolor{currentstroke}{rgb}{0.000000,0.000000,0.000000}%
\pgfsetstrokecolor{currentstroke}%
\pgfsetstrokeopacity{0.000000}%
\pgfsetdash{}{0pt}%
\pgfpathmoveto{\pgfqpoint{5.359600in}{1.034041in}}%
\pgfpathlineto{\pgfqpoint{5.464779in}{1.034041in}}%
\pgfpathlineto{\pgfqpoint{5.464779in}{1.865259in}}%
\pgfpathlineto{\pgfqpoint{5.359600in}{1.865259in}}%
\pgfpathlineto{\pgfqpoint{5.359600in}{1.034041in}}%
\pgfpathclose%
\pgfusepath{fill}%
\end{pgfscope}%
\begin{pgfscope}%
\pgfpathrectangle{\pgfqpoint{0.481944in}{1.034041in}}{\pgfqpoint{7.362500in}{2.695000in}}%
\pgfusepath{clip}%
\pgfsetbuttcap%
\pgfsetmiterjoin%
\definecolor{currentfill}{rgb}{0.656222,0.773271,0.595544}%
\pgfsetfillcolor{currentfill}%
\pgfsetlinewidth{0.000000pt}%
\definecolor{currentstroke}{rgb}{0.000000,0.000000,0.000000}%
\pgfsetstrokecolor{currentstroke}%
\pgfsetstrokeopacity{0.000000}%
\pgfsetdash{}{0pt}%
\pgfpathmoveto{\pgfqpoint{5.491074in}{1.034041in}}%
\pgfpathlineto{\pgfqpoint{5.596252in}{1.034041in}}%
\pgfpathlineto{\pgfqpoint{5.596252in}{3.403012in}}%
\pgfpathlineto{\pgfqpoint{5.491074in}{3.403012in}}%
\pgfpathlineto{\pgfqpoint{5.491074in}{1.034041in}}%
\pgfpathclose%
\pgfusepath{fill}%
\end{pgfscope}%
\begin{pgfscope}%
\pgfpathrectangle{\pgfqpoint{0.481944in}{1.034041in}}{\pgfqpoint{7.362500in}{2.695000in}}%
\pgfusepath{clip}%
\pgfsetbuttcap%
\pgfsetmiterjoin%
\definecolor{currentfill}{rgb}{0.584872,0.735999,0.592127}%
\pgfsetfillcolor{currentfill}%
\pgfsetlinewidth{0.000000pt}%
\definecolor{currentstroke}{rgb}{0.000000,0.000000,0.000000}%
\pgfsetstrokecolor{currentstroke}%
\pgfsetstrokeopacity{0.000000}%
\pgfsetdash{}{0pt}%
\pgfpathmoveto{\pgfqpoint{5.622547in}{1.034041in}}%
\pgfpathlineto{\pgfqpoint{5.727725in}{1.034041in}}%
\pgfpathlineto{\pgfqpoint{5.727725in}{2.254024in}}%
\pgfpathlineto{\pgfqpoint{5.622547in}{2.254024in}}%
\pgfpathlineto{\pgfqpoint{5.622547in}{1.034041in}}%
\pgfpathclose%
\pgfusepath{fill}%
\end{pgfscope}%
\begin{pgfscope}%
\pgfpathrectangle{\pgfqpoint{0.481944in}{1.034041in}}{\pgfqpoint{7.362500in}{2.695000in}}%
\pgfusepath{clip}%
\pgfsetbuttcap%
\pgfsetmiterjoin%
\definecolor{currentfill}{rgb}{0.617599,0.753489,0.595178}%
\pgfsetfillcolor{currentfill}%
\pgfsetlinewidth{0.000000pt}%
\definecolor{currentstroke}{rgb}{0.000000,0.000000,0.000000}%
\pgfsetstrokecolor{currentstroke}%
\pgfsetstrokeopacity{0.000000}%
\pgfsetdash{}{0pt}%
\pgfpathmoveto{\pgfqpoint{5.754020in}{1.034041in}}%
\pgfpathlineto{\pgfqpoint{5.859199in}{1.034041in}}%
\pgfpathlineto{\pgfqpoint{5.859199in}{1.821195in}}%
\pgfpathlineto{\pgfqpoint{5.754020in}{1.821195in}}%
\pgfpathlineto{\pgfqpoint{5.754020in}{1.034041in}}%
\pgfpathclose%
\pgfusepath{fill}%
\end{pgfscope}%
\begin{pgfscope}%
\pgfpathrectangle{\pgfqpoint{0.481944in}{1.034041in}}{\pgfqpoint{7.362500in}{2.695000in}}%
\pgfusepath{clip}%
\pgfsetbuttcap%
\pgfsetmiterjoin%
\definecolor{currentfill}{rgb}{0.649123,0.769720,0.595746}%
\pgfsetfillcolor{currentfill}%
\pgfsetlinewidth{0.000000pt}%
\definecolor{currentstroke}{rgb}{0.000000,0.000000,0.000000}%
\pgfsetstrokecolor{currentstroke}%
\pgfsetstrokeopacity{0.000000}%
\pgfsetdash{}{0pt}%
\pgfpathmoveto{\pgfqpoint{5.885493in}{1.034041in}}%
\pgfpathlineto{\pgfqpoint{5.990672in}{1.034041in}}%
\pgfpathlineto{\pgfqpoint{5.990672in}{2.065523in}}%
\pgfpathlineto{\pgfqpoint{5.885493in}{2.065523in}}%
\pgfpathlineto{\pgfqpoint{5.885493in}{1.034041in}}%
\pgfpathclose%
\pgfusepath{fill}%
\end{pgfscope}%
\begin{pgfscope}%
\pgfpathrectangle{\pgfqpoint{0.481944in}{1.034041in}}{\pgfqpoint{7.362500in}{2.695000in}}%
\pgfusepath{clip}%
\pgfsetbuttcap%
\pgfsetmiterjoin%
\definecolor{currentfill}{rgb}{0.624593,0.757104,0.595421}%
\pgfsetfillcolor{currentfill}%
\pgfsetlinewidth{0.000000pt}%
\definecolor{currentstroke}{rgb}{0.000000,0.000000,0.000000}%
\pgfsetstrokecolor{currentstroke}%
\pgfsetstrokeopacity{0.000000}%
\pgfsetdash{}{0pt}%
\pgfpathmoveto{\pgfqpoint{6.016966in}{1.034041in}}%
\pgfpathlineto{\pgfqpoint{6.122145in}{1.034041in}}%
\pgfpathlineto{\pgfqpoint{6.122145in}{1.969888in}}%
\pgfpathlineto{\pgfqpoint{6.016966in}{1.969888in}}%
\pgfpathlineto{\pgfqpoint{6.016966in}{1.034041in}}%
\pgfpathclose%
\pgfusepath{fill}%
\end{pgfscope}%
\begin{pgfscope}%
\pgfpathrectangle{\pgfqpoint{0.481944in}{1.034041in}}{\pgfqpoint{7.362500in}{2.695000in}}%
\pgfusepath{clip}%
\pgfsetbuttcap%
\pgfsetmiterjoin%
\definecolor{currentfill}{rgb}{0.638590,0.764325,0.595722}%
\pgfsetfillcolor{currentfill}%
\pgfsetlinewidth{0.000000pt}%
\definecolor{currentstroke}{rgb}{0.000000,0.000000,0.000000}%
\pgfsetstrokecolor{currentstroke}%
\pgfsetstrokeopacity{0.000000}%
\pgfsetdash{}{0pt}%
\pgfpathmoveto{\pgfqpoint{6.148440in}{1.034041in}}%
\pgfpathlineto{\pgfqpoint{6.253618in}{1.034041in}}%
\pgfpathlineto{\pgfqpoint{6.253618in}{2.130260in}}%
\pgfpathlineto{\pgfqpoint{6.148440in}{2.130260in}}%
\pgfpathlineto{\pgfqpoint{6.148440in}{1.034041in}}%
\pgfpathclose%
\pgfusepath{fill}%
\end{pgfscope}%
\begin{pgfscope}%
\pgfpathrectangle{\pgfqpoint{0.481944in}{1.034041in}}{\pgfqpoint{7.362500in}{2.695000in}}%
\pgfusepath{clip}%
\pgfsetbuttcap%
\pgfsetmiterjoin%
\definecolor{currentfill}{rgb}{0.515241,0.696325,0.576964}%
\pgfsetfillcolor{currentfill}%
\pgfsetlinewidth{0.000000pt}%
\definecolor{currentstroke}{rgb}{0.000000,0.000000,0.000000}%
\pgfsetstrokecolor{currentstroke}%
\pgfsetstrokeopacity{0.000000}%
\pgfsetdash{}{0pt}%
\pgfpathmoveto{\pgfqpoint{6.279913in}{1.034041in}}%
\pgfpathlineto{\pgfqpoint{6.385091in}{1.034041in}}%
\pgfpathlineto{\pgfqpoint{6.385091in}{2.732160in}}%
\pgfpathlineto{\pgfqpoint{6.279913in}{2.732160in}}%
\pgfpathlineto{\pgfqpoint{6.279913in}{1.034041in}}%
\pgfpathclose%
\pgfusepath{fill}%
\end{pgfscope}%
\begin{pgfscope}%
\pgfpathrectangle{\pgfqpoint{0.481944in}{1.034041in}}{\pgfqpoint{7.362500in}{2.695000in}}%
\pgfusepath{clip}%
\pgfsetbuttcap%
\pgfsetmiterjoin%
\definecolor{currentfill}{rgb}{0.568623,0.726620,0.588802}%
\pgfsetfillcolor{currentfill}%
\pgfsetlinewidth{0.000000pt}%
\definecolor{currentstroke}{rgb}{0.000000,0.000000,0.000000}%
\pgfsetstrokecolor{currentstroke}%
\pgfsetstrokeopacity{0.000000}%
\pgfsetdash{}{0pt}%
\pgfpathmoveto{\pgfqpoint{6.411386in}{1.034041in}}%
\pgfpathlineto{\pgfqpoint{6.516565in}{1.034041in}}%
\pgfpathlineto{\pgfqpoint{6.516565in}{2.207158in}}%
\pgfpathlineto{\pgfqpoint{6.411386in}{2.207158in}}%
\pgfpathlineto{\pgfqpoint{6.411386in}{1.034041in}}%
\pgfpathclose%
\pgfusepath{fill}%
\end{pgfscope}%
\begin{pgfscope}%
\pgfpathrectangle{\pgfqpoint{0.481944in}{1.034041in}}{\pgfqpoint{7.362500in}{2.695000in}}%
\pgfusepath{clip}%
\pgfsetbuttcap%
\pgfsetmiterjoin%
\definecolor{currentfill}{rgb}{0.652651,0.771509,0.595697}%
\pgfsetfillcolor{currentfill}%
\pgfsetlinewidth{0.000000pt}%
\definecolor{currentstroke}{rgb}{0.000000,0.000000,0.000000}%
\pgfsetstrokecolor{currentstroke}%
\pgfsetstrokeopacity{0.000000}%
\pgfsetdash{}{0pt}%
\pgfpathmoveto{\pgfqpoint{6.542859in}{1.034041in}}%
\pgfpathlineto{\pgfqpoint{6.648038in}{1.034041in}}%
\pgfpathlineto{\pgfqpoint{6.648038in}{3.246817in}}%
\pgfpathlineto{\pgfqpoint{6.542859in}{3.246817in}}%
\pgfpathlineto{\pgfqpoint{6.542859in}{1.034041in}}%
\pgfpathclose%
\pgfusepath{fill}%
\end{pgfscope}%
\begin{pgfscope}%
\pgfpathrectangle{\pgfqpoint{0.481944in}{1.034041in}}{\pgfqpoint{7.362500in}{2.695000in}}%
\pgfusepath{clip}%
\pgfsetbuttcap%
\pgfsetmiterjoin%
\definecolor{currentfill}{rgb}{0.552281,0.717276,0.585304}%
\pgfsetfillcolor{currentfill}%
\pgfsetlinewidth{0.000000pt}%
\definecolor{currentstroke}{rgb}{0.000000,0.000000,0.000000}%
\pgfsetstrokecolor{currentstroke}%
\pgfsetstrokeopacity{0.000000}%
\pgfsetdash{}{0pt}%
\pgfpathmoveto{\pgfqpoint{6.674333in}{1.034041in}}%
\pgfpathlineto{\pgfqpoint{6.779511in}{1.034041in}}%
\pgfpathlineto{\pgfqpoint{6.779511in}{3.140655in}}%
\pgfpathlineto{\pgfqpoint{6.674333in}{3.140655in}}%
\pgfpathlineto{\pgfqpoint{6.674333in}{1.034041in}}%
\pgfpathclose%
\pgfusepath{fill}%
\end{pgfscope}%
\begin{pgfscope}%
\pgfpathrectangle{\pgfqpoint{0.481944in}{1.034041in}}{\pgfqpoint{7.362500in}{2.695000in}}%
\pgfusepath{clip}%
\pgfsetbuttcap%
\pgfsetmiterjoin%
\definecolor{currentfill}{rgb}{0.642098,0.766125,0.595747}%
\pgfsetfillcolor{currentfill}%
\pgfsetlinewidth{0.000000pt}%
\definecolor{currentstroke}{rgb}{0.000000,0.000000,0.000000}%
\pgfsetstrokecolor{currentstroke}%
\pgfsetstrokeopacity{0.000000}%
\pgfsetdash{}{0pt}%
\pgfpathmoveto{\pgfqpoint{6.805806in}{1.034041in}}%
\pgfpathlineto{\pgfqpoint{6.910984in}{1.034041in}}%
\pgfpathlineto{\pgfqpoint{6.910984in}{2.198279in}}%
\pgfpathlineto{\pgfqpoint{6.805806in}{2.198279in}}%
\pgfpathlineto{\pgfqpoint{6.805806in}{1.034041in}}%
\pgfpathclose%
\pgfusepath{fill}%
\end{pgfscope}%
\begin{pgfscope}%
\pgfpathrectangle{\pgfqpoint{0.481944in}{1.034041in}}{\pgfqpoint{7.362500in}{2.695000in}}%
\pgfusepath{clip}%
\pgfsetbuttcap%
\pgfsetmiterjoin%
\definecolor{currentfill}{rgb}{0.656222,0.773271,0.595544}%
\pgfsetfillcolor{currentfill}%
\pgfsetlinewidth{0.000000pt}%
\definecolor{currentstroke}{rgb}{0.000000,0.000000,0.000000}%
\pgfsetstrokecolor{currentstroke}%
\pgfsetstrokeopacity{0.000000}%
\pgfsetdash{}{0pt}%
\pgfpathmoveto{\pgfqpoint{6.937279in}{1.034041in}}%
\pgfpathlineto{\pgfqpoint{7.042458in}{1.034041in}}%
\pgfpathlineto{\pgfqpoint{7.042458in}{2.046422in}}%
\pgfpathlineto{\pgfqpoint{6.937279in}{2.046422in}}%
\pgfpathlineto{\pgfqpoint{6.937279in}{1.034041in}}%
\pgfpathclose%
\pgfusepath{fill}%
\end{pgfscope}%
\begin{pgfscope}%
\pgfpathrectangle{\pgfqpoint{0.481944in}{1.034041in}}{\pgfqpoint{7.362500in}{2.695000in}}%
\pgfusepath{clip}%
\pgfsetbuttcap%
\pgfsetmiterjoin%
\definecolor{currentfill}{rgb}{0.645609,0.767924,0.595756}%
\pgfsetfillcolor{currentfill}%
\pgfsetlinewidth{0.000000pt}%
\definecolor{currentstroke}{rgb}{0.000000,0.000000,0.000000}%
\pgfsetstrokecolor{currentstroke}%
\pgfsetstrokeopacity{0.000000}%
\pgfsetdash{}{0pt}%
\pgfpathmoveto{\pgfqpoint{7.068752in}{1.034041in}}%
\pgfpathlineto{\pgfqpoint{7.173931in}{1.034041in}}%
\pgfpathlineto{\pgfqpoint{7.173931in}{1.987366in}}%
\pgfpathlineto{\pgfqpoint{7.068752in}{1.987366in}}%
\pgfpathlineto{\pgfqpoint{7.068752in}{1.034041in}}%
\pgfpathclose%
\pgfusepath{fill}%
\end{pgfscope}%
\begin{pgfscope}%
\pgfpathrectangle{\pgfqpoint{0.481944in}{1.034041in}}{\pgfqpoint{7.362500in}{2.695000in}}%
\pgfusepath{clip}%
\pgfsetbuttcap%
\pgfsetmiterjoin%
\definecolor{currentfill}{rgb}{0.642098,0.766125,0.595747}%
\pgfsetfillcolor{currentfill}%
\pgfsetlinewidth{0.000000pt}%
\definecolor{currentstroke}{rgb}{0.000000,0.000000,0.000000}%
\pgfsetstrokecolor{currentstroke}%
\pgfsetstrokeopacity{0.000000}%
\pgfsetdash{}{0pt}%
\pgfpathmoveto{\pgfqpoint{7.200225in}{1.034041in}}%
\pgfpathlineto{\pgfqpoint{7.305404in}{1.034041in}}%
\pgfpathlineto{\pgfqpoint{7.305404in}{1.900892in}}%
\pgfpathlineto{\pgfqpoint{7.200225in}{1.900892in}}%
\pgfpathlineto{\pgfqpoint{7.200225in}{1.034041in}}%
\pgfpathclose%
\pgfusepath{fill}%
\end{pgfscope}%
\begin{pgfscope}%
\pgfpathrectangle{\pgfqpoint{0.481944in}{1.034041in}}{\pgfqpoint{7.362500in}{2.695000in}}%
\pgfusepath{clip}%
\pgfsetbuttcap%
\pgfsetmiterjoin%
\definecolor{currentfill}{rgb}{0.652651,0.771509,0.595697}%
\pgfsetfillcolor{currentfill}%
\pgfsetlinewidth{0.000000pt}%
\definecolor{currentstroke}{rgb}{0.000000,0.000000,0.000000}%
\pgfsetstrokecolor{currentstroke}%
\pgfsetstrokeopacity{0.000000}%
\pgfsetdash{}{0pt}%
\pgfpathmoveto{\pgfqpoint{7.331699in}{1.034041in}}%
\pgfpathlineto{\pgfqpoint{7.436877in}{1.034041in}}%
\pgfpathlineto{\pgfqpoint{7.436877in}{2.273195in}}%
\pgfpathlineto{\pgfqpoint{7.331699in}{2.273195in}}%
\pgfpathlineto{\pgfqpoint{7.331699in}{1.034041in}}%
\pgfpathclose%
\pgfusepath{fill}%
\end{pgfscope}%
\begin{pgfscope}%
\pgfpathrectangle{\pgfqpoint{0.481944in}{1.034041in}}{\pgfqpoint{7.362500in}{2.695000in}}%
\pgfusepath{clip}%
\pgfsetbuttcap%
\pgfsetmiterjoin%
\definecolor{currentfill}{rgb}{0.652651,0.771509,0.595697}%
\pgfsetfillcolor{currentfill}%
\pgfsetlinewidth{0.000000pt}%
\definecolor{currentstroke}{rgb}{0.000000,0.000000,0.000000}%
\pgfsetstrokecolor{currentstroke}%
\pgfsetstrokeopacity{0.000000}%
\pgfsetdash{}{0pt}%
\pgfpathmoveto{\pgfqpoint{7.463172in}{1.034041in}}%
\pgfpathlineto{\pgfqpoint{7.568350in}{1.034041in}}%
\pgfpathlineto{\pgfqpoint{7.568350in}{2.068919in}}%
\pgfpathlineto{\pgfqpoint{7.463172in}{2.068919in}}%
\pgfpathlineto{\pgfqpoint{7.463172in}{1.034041in}}%
\pgfpathclose%
\pgfusepath{fill}%
\end{pgfscope}%
\begin{pgfscope}%
\pgfpathrectangle{\pgfqpoint{0.481944in}{1.034041in}}{\pgfqpoint{7.362500in}{2.695000in}}%
\pgfusepath{clip}%
\pgfsetbuttcap%
\pgfsetmiterjoin%
\definecolor{currentfill}{rgb}{0.638590,0.764325,0.595722}%
\pgfsetfillcolor{currentfill}%
\pgfsetlinewidth{0.000000pt}%
\definecolor{currentstroke}{rgb}{0.000000,0.000000,0.000000}%
\pgfsetstrokecolor{currentstroke}%
\pgfsetstrokeopacity{0.000000}%
\pgfsetdash{}{0pt}%
\pgfpathmoveto{\pgfqpoint{7.594645in}{1.034041in}}%
\pgfpathlineto{\pgfqpoint{7.699824in}{1.034041in}}%
\pgfpathlineto{\pgfqpoint{7.699824in}{1.887962in}}%
\pgfpathlineto{\pgfqpoint{7.594645in}{1.887962in}}%
\pgfpathlineto{\pgfqpoint{7.594645in}{1.034041in}}%
\pgfpathclose%
\pgfusepath{fill}%
\end{pgfscope}%
\begin{pgfscope}%
\pgfpathrectangle{\pgfqpoint{0.481944in}{1.034041in}}{\pgfqpoint{7.362500in}{2.695000in}}%
\pgfusepath{clip}%
\pgfsetbuttcap%
\pgfsetmiterjoin%
\definecolor{currentfill}{rgb}{0.621095,0.755296,0.595308}%
\pgfsetfillcolor{currentfill}%
\pgfsetlinewidth{0.000000pt}%
\definecolor{currentstroke}{rgb}{0.000000,0.000000,0.000000}%
\pgfsetstrokecolor{currentstroke}%
\pgfsetstrokeopacity{0.000000}%
\pgfsetdash{}{0pt}%
\pgfpathmoveto{\pgfqpoint{7.726118in}{1.034041in}}%
\pgfpathlineto{\pgfqpoint{7.831297in}{1.034041in}}%
\pgfpathlineto{\pgfqpoint{7.831297in}{3.729041in}}%
\pgfpathlineto{\pgfqpoint{7.726118in}{3.729041in}}%
\pgfpathlineto{\pgfqpoint{7.726118in}{1.034041in}}%
\pgfpathclose%
\pgfusepath{fill}%
\end{pgfscope}%
\begin{pgfscope}%
\definecolor{textcolor}{rgb}{0.000000,0.000000,0.000000}%
\pgfsetstrokecolor{textcolor}%
\pgfsetfillcolor{textcolor}%
\pgftext[x=0.563884in, y=0.852615in, left, base,rotate=90.000000]{\color{textcolor}\ttfamily\fontsize{6.000000}{7.200000}\selectfont AMV}%
\end{pgfscope}%
\begin{pgfscope}%
\definecolor{textcolor}{rgb}{0.000000,0.000000,0.000000}%
\pgfsetstrokecolor{textcolor}%
\pgfsetfillcolor{textcolor}%
\pgftext[x=0.695358in, y=0.764072in, left, base,rotate=90.000000]{\color{textcolor}\ttfamily\fontsize{6.000000}{7.200000}\selectfont APRIL}%
\end{pgfscope}%
\begin{pgfscope}%
\definecolor{textcolor}{rgb}{0.000000,0.000000,0.000000}%
\pgfsetstrokecolor{textcolor}%
\pgfsetfillcolor{textcolor}%
\pgftext[x=0.826831in, y=0.542715in, left, base,rotate=90.000000]{\color{textcolor}\ttfamily\fontsize{6.000000}{7.200000}\selectfont APRIL Moto}%
\end{pgfscope}%
\begin{pgfscope}%
\definecolor{textcolor}{rgb}{0.000000,0.000000,0.000000}%
\pgfsetstrokecolor{textcolor}%
\pgfsetfillcolor{textcolor}%
\pgftext[x=0.958304in, y=0.852615in, left, base,rotate=90.000000]{\color{textcolor}\ttfamily\fontsize{6.000000}{7.200000}\selectfont AXA}%
\end{pgfscope}%
\begin{pgfscope}%
\definecolor{textcolor}{rgb}{0.000000,0.000000,0.000000}%
\pgfsetstrokecolor{textcolor}%
\pgfsetfillcolor{textcolor}%
\pgftext[x=1.089777in, y=0.232814in, left, base,rotate=90.000000]{\color{textcolor}\ttfamily\fontsize{6.000000}{7.200000}\selectfont Active Assurances}%
\end{pgfscope}%
\begin{pgfscope}%
\definecolor{textcolor}{rgb}{0.000000,0.000000,0.000000}%
\pgfsetstrokecolor{textcolor}%
\pgfsetfillcolor{textcolor}%
\pgftext[x=1.221251in, y=0.808344in, left, base,rotate=90.000000]{\color{textcolor}\ttfamily\fontsize{6.000000}{7.200000}\selectfont Afer}%
\end{pgfscope}%
\begin{pgfscope}%
\definecolor{textcolor}{rgb}{0.000000,0.000000,0.000000}%
\pgfsetstrokecolor{textcolor}%
\pgfsetfillcolor{textcolor}%
\pgftext[x=1.352724in, y=0.631258in, left, base,rotate=90.000000]{\color{textcolor}\ttfamily\fontsize{6.000000}{7.200000}\selectfont Afi Esca}%
\end{pgfscope}%
\begin{pgfscope}%
\definecolor{textcolor}{rgb}{0.000000,0.000000,0.000000}%
\pgfsetstrokecolor{textcolor}%
\pgfsetfillcolor{textcolor}%
\pgftext[x=1.484197in, y=0.277086in, left, base,rotate=90.000000]{\color{textcolor}\ttfamily\fontsize{6.000000}{7.200000}\selectfont Ag2r La Mondiale}%
\end{pgfscope}%
\begin{pgfscope}%
\definecolor{textcolor}{rgb}{0.000000,0.000000,0.000000}%
\pgfsetstrokecolor{textcolor}%
\pgfsetfillcolor{textcolor}%
\pgftext[x=1.615670in, y=0.675529in, left, base,rotate=90.000000]{\color{textcolor}\ttfamily\fontsize{6.000000}{7.200000}\selectfont Allianz}%
\end{pgfscope}%
\begin{pgfscope}%
\definecolor{textcolor}{rgb}{0.000000,0.000000,0.000000}%
\pgfsetstrokecolor{textcolor}%
\pgfsetfillcolor{textcolor}%
\pgftext[x=1.747143in, y=0.365629in, left, base,rotate=90.000000]{\color{textcolor}\ttfamily\fontsize{6.000000}{7.200000}\selectfont Assur Bon Plan}%
\end{pgfscope}%
\begin{pgfscope}%
\definecolor{textcolor}{rgb}{0.000000,0.000000,0.000000}%
\pgfsetstrokecolor{textcolor}%
\pgfsetfillcolor{textcolor}%
\pgftext[x=1.878617in, y=0.454172in, left, base,rotate=90.000000]{\color{textcolor}\ttfamily\fontsize{6.000000}{7.200000}\selectfont Assur O'Poil}%
\end{pgfscope}%
\begin{pgfscope}%
\definecolor{textcolor}{rgb}{0.000000,0.000000,0.000000}%
\pgfsetstrokecolor{textcolor}%
\pgfsetfillcolor{textcolor}%
\pgftext[x=2.010090in, y=0.498443in, left, base,rotate=90.000000]{\color{textcolor}\ttfamily\fontsize{6.000000}{7.200000}\selectfont AssurOnline}%
\end{pgfscope}%
\begin{pgfscope}%
\definecolor{textcolor}{rgb}{0.000000,0.000000,0.000000}%
\pgfsetstrokecolor{textcolor}%
\pgfsetfillcolor{textcolor}%
\pgftext[x=2.141563in, y=0.365629in, left, base,rotate=90.000000]{\color{textcolor}\ttfamily\fontsize{6.000000}{7.200000}\selectfont CNP Assurances}%
\end{pgfscope}%
\begin{pgfscope}%
\definecolor{textcolor}{rgb}{0.000000,0.000000,0.000000}%
\pgfsetstrokecolor{textcolor}%
\pgfsetfillcolor{textcolor}%
\pgftext[x=2.273036in, y=0.764072in, left, base,rotate=90.000000]{\color{textcolor}\ttfamily\fontsize{6.000000}{7.200000}\selectfont Carac}%
\end{pgfscope}%
\begin{pgfscope}%
\definecolor{textcolor}{rgb}{0.000000,0.000000,0.000000}%
\pgfsetstrokecolor{textcolor}%
\pgfsetfillcolor{textcolor}%
\pgftext[x=2.404509in, y=0.719801in, left, base,rotate=90.000000]{\color{textcolor}\ttfamily\fontsize{6.000000}{7.200000}\selectfont Cardif}%
\end{pgfscope}%
\begin{pgfscope}%
\definecolor{textcolor}{rgb}{0.000000,0.000000,0.000000}%
\pgfsetstrokecolor{textcolor}%
\pgfsetfillcolor{textcolor}%
\pgftext[x=2.535983in, y=0.232814in, left, base,rotate=90.000000]{\color{textcolor}\ttfamily\fontsize{6.000000}{7.200000}\selectfont Cegema Assurances}%
\end{pgfscope}%
\begin{pgfscope}%
\definecolor{textcolor}{rgb}{0.000000,0.000000,0.000000}%
\pgfsetstrokecolor{textcolor}%
\pgfsetfillcolor{textcolor}%
\pgftext[x=2.667456in, y=0.409900in, left, base,rotate=90.000000]{\color{textcolor}\ttfamily\fontsize{6.000000}{7.200000}\selectfont Crédit Mutuel}%
\end{pgfscope}%
\begin{pgfscope}%
\definecolor{textcolor}{rgb}{0.000000,0.000000,0.000000}%
\pgfsetstrokecolor{textcolor}%
\pgfsetfillcolor{textcolor}%
\pgftext[x=2.798929in, y=0.277086in, left, base,rotate=90.000000]{\color{textcolor}\ttfamily\fontsize{6.000000}{7.200000}\selectfont Direct Assurance}%
\end{pgfscope}%
\begin{pgfscope}%
\definecolor{textcolor}{rgb}{0.000000,0.000000,0.000000}%
\pgfsetstrokecolor{textcolor}%
\pgfsetfillcolor{textcolor}%
\pgftext[x=2.930402in, y=0.365629in, left, base,rotate=90.000000]{\color{textcolor}\ttfamily\fontsize{6.000000}{7.200000}\selectfont Eca Assurances}%
\end{pgfscope}%
\begin{pgfscope}%
\definecolor{textcolor}{rgb}{0.000000,0.000000,0.000000}%
\pgfsetstrokecolor{textcolor}%
\pgfsetfillcolor{textcolor}%
\pgftext[x=3.061876in, y=0.365629in, left, base,rotate=90.000000]{\color{textcolor}\ttfamily\fontsize{6.000000}{7.200000}\selectfont Euro-Assurance}%
\end{pgfscope}%
\begin{pgfscope}%
\definecolor{textcolor}{rgb}{0.000000,0.000000,0.000000}%
\pgfsetstrokecolor{textcolor}%
\pgfsetfillcolor{textcolor}%
\pgftext[x=3.193349in, y=0.675529in, left, base,rotate=90.000000]{\color{textcolor}\ttfamily\fontsize{6.000000}{7.200000}\selectfont Eurofil}%
\end{pgfscope}%
\begin{pgfscope}%
\definecolor{textcolor}{rgb}{0.000000,0.000000,0.000000}%
\pgfsetstrokecolor{textcolor}%
\pgfsetfillcolor{textcolor}%
\pgftext[x=3.324822in, y=0.852615in, left, base,rotate=90.000000]{\color{textcolor}\ttfamily\fontsize{6.000000}{7.200000}\selectfont GMF}%
\end{pgfscope}%
\begin{pgfscope}%
\definecolor{textcolor}{rgb}{0.000000,0.000000,0.000000}%
\pgfsetstrokecolor{textcolor}%
\pgfsetfillcolor{textcolor}%
\pgftext[x=3.456295in, y=0.852615in, left, base,rotate=90.000000]{\color{textcolor}\ttfamily\fontsize{6.000000}{7.200000}\selectfont Gan}%
\end{pgfscope}%
\begin{pgfscope}%
\definecolor{textcolor}{rgb}{0.000000,0.000000,0.000000}%
\pgfsetstrokecolor{textcolor}%
\pgfsetfillcolor{textcolor}%
\pgftext[x=3.587768in, y=0.631258in, left, base,rotate=90.000000]{\color{textcolor}\ttfamily\fontsize{6.000000}{7.200000}\selectfont Generali}%
\end{pgfscope}%
\begin{pgfscope}%
\definecolor{textcolor}{rgb}{0.000000,0.000000,0.000000}%
\pgfsetstrokecolor{textcolor}%
\pgfsetfillcolor{textcolor}%
\pgftext[x=3.719242in, y=0.631258in, left, base,rotate=90.000000]{\color{textcolor}\ttfamily\fontsize{6.000000}{7.200000}\selectfont Groupama}%
\end{pgfscope}%
\begin{pgfscope}%
\definecolor{textcolor}{rgb}{0.000000,0.000000,0.000000}%
\pgfsetstrokecolor{textcolor}%
\pgfsetfillcolor{textcolor}%
\pgftext[x=3.850715in, y=0.542715in, left, base,rotate=90.000000]{\color{textcolor}\ttfamily\fontsize{6.000000}{7.200000}\selectfont Génération}%
\end{pgfscope}%
\begin{pgfscope}%
\definecolor{textcolor}{rgb}{0.000000,0.000000,0.000000}%
\pgfsetstrokecolor{textcolor}%
\pgfsetfillcolor{textcolor}%
\pgftext[x=3.982188in, y=0.232814in, left, base,rotate=90.000000]{\color{textcolor}\ttfamily\fontsize{6.000000}{7.200000}\selectfont Harmonie Mutuelle}%
\end{pgfscope}%
\begin{pgfscope}%
\definecolor{textcolor}{rgb}{0.000000,0.000000,0.000000}%
\pgfsetstrokecolor{textcolor}%
\pgfsetfillcolor{textcolor}%
\pgftext[x=4.113661in, y=0.719801in, left, base,rotate=90.000000]{\color{textcolor}\ttfamily\fontsize{6.000000}{7.200000}\selectfont Hiscox}%
\end{pgfscope}%
\begin{pgfscope}%
\definecolor{textcolor}{rgb}{0.000000,0.000000,0.000000}%
\pgfsetstrokecolor{textcolor}%
\pgfsetfillcolor{textcolor}%
\pgftext[x=4.245134in, y=0.586986in, left, base,rotate=90.000000]{\color{textcolor}\ttfamily\fontsize{6.000000}{7.200000}\selectfont Intériale}%
\end{pgfscope}%
\begin{pgfscope}%
\definecolor{textcolor}{rgb}{0.000000,0.000000,0.000000}%
\pgfsetstrokecolor{textcolor}%
\pgfsetfillcolor{textcolor}%
\pgftext[x=4.376608in, y=0.144272in, left, base,rotate=90.000000]{\color{textcolor}\ttfamily\fontsize{6.000000}{7.200000}\selectfont L'olivier Assurance}%
\end{pgfscope}%
\begin{pgfscope}%
\definecolor{textcolor}{rgb}{0.000000,0.000000,0.000000}%
\pgfsetstrokecolor{textcolor}%
\pgfsetfillcolor{textcolor}%
\pgftext[x=4.508081in, y=0.852615in, left, base,rotate=90.000000]{\color{textcolor}\ttfamily\fontsize{6.000000}{7.200000}\selectfont LCL}%
\end{pgfscope}%
\begin{pgfscope}%
\definecolor{textcolor}{rgb}{0.000000,0.000000,0.000000}%
\pgfsetstrokecolor{textcolor}%
\pgfsetfillcolor{textcolor}%
\pgftext[x=4.639554in, y=0.808344in, left, base,rotate=90.000000]{\color{textcolor}\ttfamily\fontsize{6.000000}{7.200000}\selectfont MAAF}%
\end{pgfscope}%
\begin{pgfscope}%
\definecolor{textcolor}{rgb}{0.000000,0.000000,0.000000}%
\pgfsetstrokecolor{textcolor}%
\pgfsetfillcolor{textcolor}%
\pgftext[x=4.771027in, y=0.764072in, left, base,rotate=90.000000]{\color{textcolor}\ttfamily\fontsize{6.000000}{7.200000}\selectfont MACIF}%
\end{pgfscope}%
\begin{pgfscope}%
\definecolor{textcolor}{rgb}{0.000000,0.000000,0.000000}%
\pgfsetstrokecolor{textcolor}%
\pgfsetfillcolor{textcolor}%
\pgftext[x=4.902501in, y=0.808344in, left, base,rotate=90.000000]{\color{textcolor}\ttfamily\fontsize{6.000000}{7.200000}\selectfont MAIF}%
\end{pgfscope}%
\begin{pgfscope}%
\definecolor{textcolor}{rgb}{0.000000,0.000000,0.000000}%
\pgfsetstrokecolor{textcolor}%
\pgfsetfillcolor{textcolor}%
\pgftext[x=5.033974in, y=0.852615in, left, base,rotate=90.000000]{\color{textcolor}\ttfamily\fontsize{6.000000}{7.200000}\selectfont MGP}%
\end{pgfscope}%
\begin{pgfscope}%
\definecolor{textcolor}{rgb}{0.000000,0.000000,0.000000}%
\pgfsetstrokecolor{textcolor}%
\pgfsetfillcolor{textcolor}%
\pgftext[x=5.165447in, y=0.852615in, left, base,rotate=90.000000]{\color{textcolor}\ttfamily\fontsize{6.000000}{7.200000}\selectfont MMA}%
\end{pgfscope}%
\begin{pgfscope}%
\definecolor{textcolor}{rgb}{0.000000,0.000000,0.000000}%
\pgfsetstrokecolor{textcolor}%
\pgfsetfillcolor{textcolor}%
\pgftext[x=5.296920in, y=0.631258in, left, base,rotate=90.000000]{\color{textcolor}\ttfamily\fontsize{6.000000}{7.200000}\selectfont Magnolia}%
\end{pgfscope}%
\begin{pgfscope}%
\definecolor{textcolor}{rgb}{0.000000,0.000000,0.000000}%
\pgfsetstrokecolor{textcolor}%
\pgfsetfillcolor{textcolor}%
\pgftext[x=5.428393in, y=0.277086in, left, base,rotate=90.000000]{\color{textcolor}\ttfamily\fontsize{6.000000}{7.200000}\selectfont Malakoff Humanis}%
\end{pgfscope}%
\begin{pgfscope}%
\definecolor{textcolor}{rgb}{0.000000,0.000000,0.000000}%
\pgfsetstrokecolor{textcolor}%
\pgfsetfillcolor{textcolor}%
\pgftext[x=5.559867in, y=0.808344in, left, base,rotate=90.000000]{\color{textcolor}\ttfamily\fontsize{6.000000}{7.200000}\selectfont Mapa}%
\end{pgfscope}%
\begin{pgfscope}%
\definecolor{textcolor}{rgb}{0.000000,0.000000,0.000000}%
\pgfsetstrokecolor{textcolor}%
\pgfsetfillcolor{textcolor}%
\pgftext[x=5.691340in, y=0.719801in, left, base,rotate=90.000000]{\color{textcolor}\ttfamily\fontsize{6.000000}{7.200000}\selectfont Matmut}%
\end{pgfscope}%
\begin{pgfscope}%
\definecolor{textcolor}{rgb}{0.000000,0.000000,0.000000}%
\pgfsetstrokecolor{textcolor}%
\pgfsetfillcolor{textcolor}%
\pgftext[x=5.822813in, y=0.719801in, left, base,rotate=90.000000]{\color{textcolor}\ttfamily\fontsize{6.000000}{7.200000}\selectfont Mercer}%
\end{pgfscope}%
\begin{pgfscope}%
\definecolor{textcolor}{rgb}{0.000000,0.000000,0.000000}%
\pgfsetstrokecolor{textcolor}%
\pgfsetfillcolor{textcolor}%
\pgftext[x=5.954286in, y=0.675529in, left, base,rotate=90.000000]{\color{textcolor}\ttfamily\fontsize{6.000000}{7.200000}\selectfont MetLife}%
\end{pgfscope}%
\begin{pgfscope}%
\definecolor{textcolor}{rgb}{0.000000,0.000000,0.000000}%
\pgfsetstrokecolor{textcolor}%
\pgfsetfillcolor{textcolor}%
\pgftext[x=6.085759in, y=0.808344in, left, base,rotate=90.000000]{\color{textcolor}\ttfamily\fontsize{6.000000}{7.200000}\selectfont Mgen}%
\end{pgfscope}%
\begin{pgfscope}%
\definecolor{textcolor}{rgb}{0.000000,0.000000,0.000000}%
\pgfsetstrokecolor{textcolor}%
\pgfsetfillcolor{textcolor}%
\pgftext[x=6.217233in, y=0.100000in, left, base,rotate=90.000000]{\color{textcolor}\ttfamily\fontsize{6.000000}{7.200000}\selectfont Mutuelle des Motards}%
\end{pgfscope}%
\begin{pgfscope}%
\definecolor{textcolor}{rgb}{0.000000,0.000000,0.000000}%
\pgfsetstrokecolor{textcolor}%
\pgfsetfillcolor{textcolor}%
\pgftext[x=6.348706in, y=0.365629in, left, base,rotate=90.000000]{\color{textcolor}\ttfamily\fontsize{6.000000}{7.200000}\selectfont Néoliane Santé}%
\end{pgfscope}%
\begin{pgfscope}%
\definecolor{textcolor}{rgb}{0.000000,0.000000,0.000000}%
\pgfsetstrokecolor{textcolor}%
\pgfsetfillcolor{textcolor}%
\pgftext[x=6.480179in, y=0.631258in, left, base,rotate=90.000000]{\color{textcolor}\ttfamily\fontsize{6.000000}{7.200000}\selectfont Pacifica}%
\end{pgfscope}%
\begin{pgfscope}%
\definecolor{textcolor}{rgb}{0.000000,0.000000,0.000000}%
\pgfsetstrokecolor{textcolor}%
\pgfsetfillcolor{textcolor}%
\pgftext[x=6.611652in, y=0.232814in, left, base,rotate=90.000000]{\color{textcolor}\ttfamily\fontsize{6.000000}{7.200000}\selectfont Peyrac Assurances}%
\end{pgfscope}%
\begin{pgfscope}%
\definecolor{textcolor}{rgb}{0.000000,0.000000,0.000000}%
\pgfsetstrokecolor{textcolor}%
\pgfsetfillcolor{textcolor}%
\pgftext[x=6.743126in, y=0.631258in, left, base,rotate=90.000000]{\color{textcolor}\ttfamily\fontsize{6.000000}{7.200000}\selectfont Santiane}%
\end{pgfscope}%
\begin{pgfscope}%
\definecolor{textcolor}{rgb}{0.000000,0.000000,0.000000}%
\pgfsetstrokecolor{textcolor}%
\pgfsetfillcolor{textcolor}%
\pgftext[x=6.874599in, y=0.631258in, left, base,rotate=90.000000]{\color{textcolor}\ttfamily\fontsize{6.000000}{7.200000}\selectfont SantéVet}%
\end{pgfscope}%
\begin{pgfscope}%
\definecolor{textcolor}{rgb}{0.000000,0.000000,0.000000}%
\pgfsetstrokecolor{textcolor}%
\pgfsetfillcolor{textcolor}%
\pgftext[x=7.006072in, y=0.852615in, left, base,rotate=90.000000]{\color{textcolor}\ttfamily\fontsize{6.000000}{7.200000}\selectfont Sma}%
\end{pgfscope}%
\begin{pgfscope}%
\definecolor{textcolor}{rgb}{0.000000,0.000000,0.000000}%
\pgfsetstrokecolor{textcolor}%
\pgfsetfillcolor{textcolor}%
\pgftext[x=7.137545in, y=0.675529in, left, base,rotate=90.000000]{\color{textcolor}\ttfamily\fontsize{6.000000}{7.200000}\selectfont Sogecap}%
\end{pgfscope}%
\begin{pgfscope}%
\definecolor{textcolor}{rgb}{0.000000,0.000000,0.000000}%
\pgfsetstrokecolor{textcolor}%
\pgfsetfillcolor{textcolor}%
\pgftext[x=7.269018in, y=0.631258in, left, base,rotate=90.000000]{\color{textcolor}\ttfamily\fontsize{6.000000}{7.200000}\selectfont Sogessur}%
\end{pgfscope}%
\begin{pgfscope}%
\definecolor{textcolor}{rgb}{0.000000,0.000000,0.000000}%
\pgfsetstrokecolor{textcolor}%
\pgfsetfillcolor{textcolor}%
\pgftext[x=7.400492in, y=0.542715in, left, base,rotate=90.000000]{\color{textcolor}\ttfamily\fontsize{6.000000}{7.200000}\selectfont Solly Azar}%
\end{pgfscope}%
\begin{pgfscope}%
\definecolor{textcolor}{rgb}{0.000000,0.000000,0.000000}%
\pgfsetstrokecolor{textcolor}%
\pgfsetfillcolor{textcolor}%
\pgftext[x=7.531965in, y=0.586986in, left, base,rotate=90.000000]{\color{textcolor}\ttfamily\fontsize{6.000000}{7.200000}\selectfont Suravenir}%
\end{pgfscope}%
\begin{pgfscope}%
\definecolor{textcolor}{rgb}{0.000000,0.000000,0.000000}%
\pgfsetstrokecolor{textcolor}%
\pgfsetfillcolor{textcolor}%
\pgftext[x=7.663438in, y=0.586986in, left, base,rotate=90.000000]{\color{textcolor}\ttfamily\fontsize{6.000000}{7.200000}\selectfont SwissLife}%
\end{pgfscope}%
\begin{pgfscope}%
\definecolor{textcolor}{rgb}{0.000000,0.000000,0.000000}%
\pgfsetstrokecolor{textcolor}%
\pgfsetfillcolor{textcolor}%
\pgftext[x=7.794911in, y=0.719801in, left, base,rotate=90.000000]{\color{textcolor}\ttfamily\fontsize{6.000000}{7.200000}\selectfont Zen'Up}%
\end{pgfscope}%
\begin{pgfscope}%
\definecolor{textcolor}{rgb}{0.000000,0.000000,0.000000}%
\pgfsetstrokecolor{textcolor}%
\pgfsetfillcolor{textcolor}%
\pgftext[x=0.396296in, y=1.003370in, left, base,rotate=90.000000]{\color{textcolor}\ttfamily\fontsize{12.000000}{14.400000}\selectfont \(\displaystyle {0}\)}%
\end{pgfscope}%
\begin{pgfscope}%
\definecolor{textcolor}{rgb}{0.000000,0.000000,0.000000}%
\pgfsetstrokecolor{textcolor}%
\pgfsetfillcolor{textcolor}%
\pgftext[x=0.396296in, y=1.610799in, left, base,rotate=90.000000]{\color{textcolor}\ttfamily\fontsize{12.000000}{14.400000}\selectfont \(\displaystyle {1}\)}%
\end{pgfscope}%
\begin{pgfscope}%
\definecolor{textcolor}{rgb}{0.000000,0.000000,0.000000}%
\pgfsetstrokecolor{textcolor}%
\pgfsetfillcolor{textcolor}%
\pgftext[x=0.396296in, y=2.218227in, left, base,rotate=90.000000]{\color{textcolor}\ttfamily\fontsize{12.000000}{14.400000}\selectfont \(\displaystyle {2}\)}%
\end{pgfscope}%
\begin{pgfscope}%
\definecolor{textcolor}{rgb}{0.000000,0.000000,0.000000}%
\pgfsetstrokecolor{textcolor}%
\pgfsetfillcolor{textcolor}%
\pgftext[x=0.396296in, y=2.825656in, left, base,rotate=90.000000]{\color{textcolor}\ttfamily\fontsize{12.000000}{14.400000}\selectfont \(\displaystyle {3}\)}%
\end{pgfscope}%
\begin{pgfscope}%
\definecolor{textcolor}{rgb}{0.000000,0.000000,0.000000}%
\pgfsetstrokecolor{textcolor}%
\pgfsetfillcolor{textcolor}%
\pgftext[x=0.396296in, y=3.433085in, left, base,rotate=90.000000]{\color{textcolor}\ttfamily\fontsize{12.000000}{14.400000}\selectfont \(\displaystyle {4}\)}%
\end{pgfscope}%
\begin{pgfscope}%
\definecolor{textcolor}{rgb}{0.000000,0.000000,0.000000}%
\pgfsetstrokecolor{textcolor}%
\pgfsetfillcolor{textcolor}%
\pgftext[x=0.238889in,y=2.381541in,,bottom,rotate=90.000000]{\color{textcolor}\ttfamily\fontsize{12.000000}{14.400000}\selectfont Mean note}%
\end{pgfscope}%
\begin{pgfscope}%
\pgfpathrectangle{\pgfqpoint{0.481944in}{1.034041in}}{\pgfqpoint{7.362500in}{2.695000in}}%
\pgfusepath{clip}%
\pgfsetrectcap%
\pgfsetroundjoin%
\pgfsetlinewidth{2.710125pt}%
\definecolor{currentstroke}{rgb}{0.260000,0.260000,0.260000}%
\pgfsetstrokecolor{currentstroke}%
\pgfsetdash{}{0pt}%
\pgfusepath{stroke}%
\end{pgfscope}%
\begin{pgfscope}%
\pgfpathrectangle{\pgfqpoint{0.481944in}{1.034041in}}{\pgfqpoint{7.362500in}{2.695000in}}%
\pgfusepath{clip}%
\pgfsetrectcap%
\pgfsetroundjoin%
\pgfsetlinewidth{2.710125pt}%
\definecolor{currentstroke}{rgb}{0.260000,0.260000,0.260000}%
\pgfsetstrokecolor{currentstroke}%
\pgfsetdash{}{0pt}%
\pgfusepath{stroke}%
\end{pgfscope}%
\begin{pgfscope}%
\pgfpathrectangle{\pgfqpoint{0.481944in}{1.034041in}}{\pgfqpoint{7.362500in}{2.695000in}}%
\pgfusepath{clip}%
\pgfsetrectcap%
\pgfsetroundjoin%
\pgfsetlinewidth{2.710125pt}%
\definecolor{currentstroke}{rgb}{0.260000,0.260000,0.260000}%
\pgfsetstrokecolor{currentstroke}%
\pgfsetdash{}{0pt}%
\pgfusepath{stroke}%
\end{pgfscope}%
\begin{pgfscope}%
\pgfpathrectangle{\pgfqpoint{0.481944in}{1.034041in}}{\pgfqpoint{7.362500in}{2.695000in}}%
\pgfusepath{clip}%
\pgfsetrectcap%
\pgfsetroundjoin%
\pgfsetlinewidth{2.710125pt}%
\definecolor{currentstroke}{rgb}{0.260000,0.260000,0.260000}%
\pgfsetstrokecolor{currentstroke}%
\pgfsetdash{}{0pt}%
\pgfusepath{stroke}%
\end{pgfscope}%
\begin{pgfscope}%
\pgfpathrectangle{\pgfqpoint{0.481944in}{1.034041in}}{\pgfqpoint{7.362500in}{2.695000in}}%
\pgfusepath{clip}%
\pgfsetrectcap%
\pgfsetroundjoin%
\pgfsetlinewidth{2.710125pt}%
\definecolor{currentstroke}{rgb}{0.260000,0.260000,0.260000}%
\pgfsetstrokecolor{currentstroke}%
\pgfsetdash{}{0pt}%
\pgfusepath{stroke}%
\end{pgfscope}%
\begin{pgfscope}%
\pgfpathrectangle{\pgfqpoint{0.481944in}{1.034041in}}{\pgfqpoint{7.362500in}{2.695000in}}%
\pgfusepath{clip}%
\pgfsetrectcap%
\pgfsetroundjoin%
\pgfsetlinewidth{2.710125pt}%
\definecolor{currentstroke}{rgb}{0.260000,0.260000,0.260000}%
\pgfsetstrokecolor{currentstroke}%
\pgfsetdash{}{0pt}%
\pgfusepath{stroke}%
\end{pgfscope}%
\begin{pgfscope}%
\pgfpathrectangle{\pgfqpoint{0.481944in}{1.034041in}}{\pgfqpoint{7.362500in}{2.695000in}}%
\pgfusepath{clip}%
\pgfsetrectcap%
\pgfsetroundjoin%
\pgfsetlinewidth{2.710125pt}%
\definecolor{currentstroke}{rgb}{0.260000,0.260000,0.260000}%
\pgfsetstrokecolor{currentstroke}%
\pgfsetdash{}{0pt}%
\pgfusepath{stroke}%
\end{pgfscope}%
\begin{pgfscope}%
\pgfpathrectangle{\pgfqpoint{0.481944in}{1.034041in}}{\pgfqpoint{7.362500in}{2.695000in}}%
\pgfusepath{clip}%
\pgfsetrectcap%
\pgfsetroundjoin%
\pgfsetlinewidth{2.710125pt}%
\definecolor{currentstroke}{rgb}{0.260000,0.260000,0.260000}%
\pgfsetstrokecolor{currentstroke}%
\pgfsetdash{}{0pt}%
\pgfusepath{stroke}%
\end{pgfscope}%
\begin{pgfscope}%
\pgfpathrectangle{\pgfqpoint{0.481944in}{1.034041in}}{\pgfqpoint{7.362500in}{2.695000in}}%
\pgfusepath{clip}%
\pgfsetrectcap%
\pgfsetroundjoin%
\pgfsetlinewidth{2.710125pt}%
\definecolor{currentstroke}{rgb}{0.260000,0.260000,0.260000}%
\pgfsetstrokecolor{currentstroke}%
\pgfsetdash{}{0pt}%
\pgfusepath{stroke}%
\end{pgfscope}%
\begin{pgfscope}%
\pgfpathrectangle{\pgfqpoint{0.481944in}{1.034041in}}{\pgfqpoint{7.362500in}{2.695000in}}%
\pgfusepath{clip}%
\pgfsetrectcap%
\pgfsetroundjoin%
\pgfsetlinewidth{2.710125pt}%
\definecolor{currentstroke}{rgb}{0.260000,0.260000,0.260000}%
\pgfsetstrokecolor{currentstroke}%
\pgfsetdash{}{0pt}%
\pgfusepath{stroke}%
\end{pgfscope}%
\begin{pgfscope}%
\pgfpathrectangle{\pgfqpoint{0.481944in}{1.034041in}}{\pgfqpoint{7.362500in}{2.695000in}}%
\pgfusepath{clip}%
\pgfsetrectcap%
\pgfsetroundjoin%
\pgfsetlinewidth{2.710125pt}%
\definecolor{currentstroke}{rgb}{0.260000,0.260000,0.260000}%
\pgfsetstrokecolor{currentstroke}%
\pgfsetdash{}{0pt}%
\pgfusepath{stroke}%
\end{pgfscope}%
\begin{pgfscope}%
\pgfpathrectangle{\pgfqpoint{0.481944in}{1.034041in}}{\pgfqpoint{7.362500in}{2.695000in}}%
\pgfusepath{clip}%
\pgfsetrectcap%
\pgfsetroundjoin%
\pgfsetlinewidth{2.710125pt}%
\definecolor{currentstroke}{rgb}{0.260000,0.260000,0.260000}%
\pgfsetstrokecolor{currentstroke}%
\pgfsetdash{}{0pt}%
\pgfusepath{stroke}%
\end{pgfscope}%
\begin{pgfscope}%
\pgfpathrectangle{\pgfqpoint{0.481944in}{1.034041in}}{\pgfqpoint{7.362500in}{2.695000in}}%
\pgfusepath{clip}%
\pgfsetrectcap%
\pgfsetroundjoin%
\pgfsetlinewidth{2.710125pt}%
\definecolor{currentstroke}{rgb}{0.260000,0.260000,0.260000}%
\pgfsetstrokecolor{currentstroke}%
\pgfsetdash{}{0pt}%
\pgfusepath{stroke}%
\end{pgfscope}%
\begin{pgfscope}%
\pgfpathrectangle{\pgfqpoint{0.481944in}{1.034041in}}{\pgfqpoint{7.362500in}{2.695000in}}%
\pgfusepath{clip}%
\pgfsetrectcap%
\pgfsetroundjoin%
\pgfsetlinewidth{2.710125pt}%
\definecolor{currentstroke}{rgb}{0.260000,0.260000,0.260000}%
\pgfsetstrokecolor{currentstroke}%
\pgfsetdash{}{0pt}%
\pgfusepath{stroke}%
\end{pgfscope}%
\begin{pgfscope}%
\pgfpathrectangle{\pgfqpoint{0.481944in}{1.034041in}}{\pgfqpoint{7.362500in}{2.695000in}}%
\pgfusepath{clip}%
\pgfsetrectcap%
\pgfsetroundjoin%
\pgfsetlinewidth{2.710125pt}%
\definecolor{currentstroke}{rgb}{0.260000,0.260000,0.260000}%
\pgfsetstrokecolor{currentstroke}%
\pgfsetdash{}{0pt}%
\pgfusepath{stroke}%
\end{pgfscope}%
\begin{pgfscope}%
\pgfpathrectangle{\pgfqpoint{0.481944in}{1.034041in}}{\pgfqpoint{7.362500in}{2.695000in}}%
\pgfusepath{clip}%
\pgfsetrectcap%
\pgfsetroundjoin%
\pgfsetlinewidth{2.710125pt}%
\definecolor{currentstroke}{rgb}{0.260000,0.260000,0.260000}%
\pgfsetstrokecolor{currentstroke}%
\pgfsetdash{}{0pt}%
\pgfusepath{stroke}%
\end{pgfscope}%
\begin{pgfscope}%
\pgfpathrectangle{\pgfqpoint{0.481944in}{1.034041in}}{\pgfqpoint{7.362500in}{2.695000in}}%
\pgfusepath{clip}%
\pgfsetrectcap%
\pgfsetroundjoin%
\pgfsetlinewidth{2.710125pt}%
\definecolor{currentstroke}{rgb}{0.260000,0.260000,0.260000}%
\pgfsetstrokecolor{currentstroke}%
\pgfsetdash{}{0pt}%
\pgfusepath{stroke}%
\end{pgfscope}%
\begin{pgfscope}%
\pgfpathrectangle{\pgfqpoint{0.481944in}{1.034041in}}{\pgfqpoint{7.362500in}{2.695000in}}%
\pgfusepath{clip}%
\pgfsetrectcap%
\pgfsetroundjoin%
\pgfsetlinewidth{2.710125pt}%
\definecolor{currentstroke}{rgb}{0.260000,0.260000,0.260000}%
\pgfsetstrokecolor{currentstroke}%
\pgfsetdash{}{0pt}%
\pgfusepath{stroke}%
\end{pgfscope}%
\begin{pgfscope}%
\pgfpathrectangle{\pgfqpoint{0.481944in}{1.034041in}}{\pgfqpoint{7.362500in}{2.695000in}}%
\pgfusepath{clip}%
\pgfsetrectcap%
\pgfsetroundjoin%
\pgfsetlinewidth{2.710125pt}%
\definecolor{currentstroke}{rgb}{0.260000,0.260000,0.260000}%
\pgfsetstrokecolor{currentstroke}%
\pgfsetdash{}{0pt}%
\pgfusepath{stroke}%
\end{pgfscope}%
\begin{pgfscope}%
\pgfpathrectangle{\pgfqpoint{0.481944in}{1.034041in}}{\pgfqpoint{7.362500in}{2.695000in}}%
\pgfusepath{clip}%
\pgfsetrectcap%
\pgfsetroundjoin%
\pgfsetlinewidth{2.710125pt}%
\definecolor{currentstroke}{rgb}{0.260000,0.260000,0.260000}%
\pgfsetstrokecolor{currentstroke}%
\pgfsetdash{}{0pt}%
\pgfusepath{stroke}%
\end{pgfscope}%
\begin{pgfscope}%
\pgfpathrectangle{\pgfqpoint{0.481944in}{1.034041in}}{\pgfqpoint{7.362500in}{2.695000in}}%
\pgfusepath{clip}%
\pgfsetrectcap%
\pgfsetroundjoin%
\pgfsetlinewidth{2.710125pt}%
\definecolor{currentstroke}{rgb}{0.260000,0.260000,0.260000}%
\pgfsetstrokecolor{currentstroke}%
\pgfsetdash{}{0pt}%
\pgfusepath{stroke}%
\end{pgfscope}%
\begin{pgfscope}%
\pgfpathrectangle{\pgfqpoint{0.481944in}{1.034041in}}{\pgfqpoint{7.362500in}{2.695000in}}%
\pgfusepath{clip}%
\pgfsetrectcap%
\pgfsetroundjoin%
\pgfsetlinewidth{2.710125pt}%
\definecolor{currentstroke}{rgb}{0.260000,0.260000,0.260000}%
\pgfsetstrokecolor{currentstroke}%
\pgfsetdash{}{0pt}%
\pgfusepath{stroke}%
\end{pgfscope}%
\begin{pgfscope}%
\pgfpathrectangle{\pgfqpoint{0.481944in}{1.034041in}}{\pgfqpoint{7.362500in}{2.695000in}}%
\pgfusepath{clip}%
\pgfsetrectcap%
\pgfsetroundjoin%
\pgfsetlinewidth{2.710125pt}%
\definecolor{currentstroke}{rgb}{0.260000,0.260000,0.260000}%
\pgfsetstrokecolor{currentstroke}%
\pgfsetdash{}{0pt}%
\pgfusepath{stroke}%
\end{pgfscope}%
\begin{pgfscope}%
\pgfpathrectangle{\pgfqpoint{0.481944in}{1.034041in}}{\pgfqpoint{7.362500in}{2.695000in}}%
\pgfusepath{clip}%
\pgfsetrectcap%
\pgfsetroundjoin%
\pgfsetlinewidth{2.710125pt}%
\definecolor{currentstroke}{rgb}{0.260000,0.260000,0.260000}%
\pgfsetstrokecolor{currentstroke}%
\pgfsetdash{}{0pt}%
\pgfusepath{stroke}%
\end{pgfscope}%
\begin{pgfscope}%
\pgfpathrectangle{\pgfqpoint{0.481944in}{1.034041in}}{\pgfqpoint{7.362500in}{2.695000in}}%
\pgfusepath{clip}%
\pgfsetrectcap%
\pgfsetroundjoin%
\pgfsetlinewidth{2.710125pt}%
\definecolor{currentstroke}{rgb}{0.260000,0.260000,0.260000}%
\pgfsetstrokecolor{currentstroke}%
\pgfsetdash{}{0pt}%
\pgfusepath{stroke}%
\end{pgfscope}%
\begin{pgfscope}%
\pgfpathrectangle{\pgfqpoint{0.481944in}{1.034041in}}{\pgfqpoint{7.362500in}{2.695000in}}%
\pgfusepath{clip}%
\pgfsetrectcap%
\pgfsetroundjoin%
\pgfsetlinewidth{2.710125pt}%
\definecolor{currentstroke}{rgb}{0.260000,0.260000,0.260000}%
\pgfsetstrokecolor{currentstroke}%
\pgfsetdash{}{0pt}%
\pgfusepath{stroke}%
\end{pgfscope}%
\begin{pgfscope}%
\pgfpathrectangle{\pgfqpoint{0.481944in}{1.034041in}}{\pgfqpoint{7.362500in}{2.695000in}}%
\pgfusepath{clip}%
\pgfsetrectcap%
\pgfsetroundjoin%
\pgfsetlinewidth{2.710125pt}%
\definecolor{currentstroke}{rgb}{0.260000,0.260000,0.260000}%
\pgfsetstrokecolor{currentstroke}%
\pgfsetdash{}{0pt}%
\pgfusepath{stroke}%
\end{pgfscope}%
\begin{pgfscope}%
\pgfpathrectangle{\pgfqpoint{0.481944in}{1.034041in}}{\pgfqpoint{7.362500in}{2.695000in}}%
\pgfusepath{clip}%
\pgfsetrectcap%
\pgfsetroundjoin%
\pgfsetlinewidth{2.710125pt}%
\definecolor{currentstroke}{rgb}{0.260000,0.260000,0.260000}%
\pgfsetstrokecolor{currentstroke}%
\pgfsetdash{}{0pt}%
\pgfusepath{stroke}%
\end{pgfscope}%
\begin{pgfscope}%
\pgfpathrectangle{\pgfqpoint{0.481944in}{1.034041in}}{\pgfqpoint{7.362500in}{2.695000in}}%
\pgfusepath{clip}%
\pgfsetrectcap%
\pgfsetroundjoin%
\pgfsetlinewidth{2.710125pt}%
\definecolor{currentstroke}{rgb}{0.260000,0.260000,0.260000}%
\pgfsetstrokecolor{currentstroke}%
\pgfsetdash{}{0pt}%
\pgfusepath{stroke}%
\end{pgfscope}%
\begin{pgfscope}%
\pgfpathrectangle{\pgfqpoint{0.481944in}{1.034041in}}{\pgfqpoint{7.362500in}{2.695000in}}%
\pgfusepath{clip}%
\pgfsetrectcap%
\pgfsetroundjoin%
\pgfsetlinewidth{2.710125pt}%
\definecolor{currentstroke}{rgb}{0.260000,0.260000,0.260000}%
\pgfsetstrokecolor{currentstroke}%
\pgfsetdash{}{0pt}%
\pgfusepath{stroke}%
\end{pgfscope}%
\begin{pgfscope}%
\pgfpathrectangle{\pgfqpoint{0.481944in}{1.034041in}}{\pgfqpoint{7.362500in}{2.695000in}}%
\pgfusepath{clip}%
\pgfsetrectcap%
\pgfsetroundjoin%
\pgfsetlinewidth{2.710125pt}%
\definecolor{currentstroke}{rgb}{0.260000,0.260000,0.260000}%
\pgfsetstrokecolor{currentstroke}%
\pgfsetdash{}{0pt}%
\pgfusepath{stroke}%
\end{pgfscope}%
\begin{pgfscope}%
\pgfpathrectangle{\pgfqpoint{0.481944in}{1.034041in}}{\pgfqpoint{7.362500in}{2.695000in}}%
\pgfusepath{clip}%
\pgfsetrectcap%
\pgfsetroundjoin%
\pgfsetlinewidth{2.710125pt}%
\definecolor{currentstroke}{rgb}{0.260000,0.260000,0.260000}%
\pgfsetstrokecolor{currentstroke}%
\pgfsetdash{}{0pt}%
\pgfusepath{stroke}%
\end{pgfscope}%
\begin{pgfscope}%
\pgfpathrectangle{\pgfqpoint{0.481944in}{1.034041in}}{\pgfqpoint{7.362500in}{2.695000in}}%
\pgfusepath{clip}%
\pgfsetrectcap%
\pgfsetroundjoin%
\pgfsetlinewidth{2.710125pt}%
\definecolor{currentstroke}{rgb}{0.260000,0.260000,0.260000}%
\pgfsetstrokecolor{currentstroke}%
\pgfsetdash{}{0pt}%
\pgfusepath{stroke}%
\end{pgfscope}%
\begin{pgfscope}%
\pgfpathrectangle{\pgfqpoint{0.481944in}{1.034041in}}{\pgfqpoint{7.362500in}{2.695000in}}%
\pgfusepath{clip}%
\pgfsetrectcap%
\pgfsetroundjoin%
\pgfsetlinewidth{2.710125pt}%
\definecolor{currentstroke}{rgb}{0.260000,0.260000,0.260000}%
\pgfsetstrokecolor{currentstroke}%
\pgfsetdash{}{0pt}%
\pgfusepath{stroke}%
\end{pgfscope}%
\begin{pgfscope}%
\pgfpathrectangle{\pgfqpoint{0.481944in}{1.034041in}}{\pgfqpoint{7.362500in}{2.695000in}}%
\pgfusepath{clip}%
\pgfsetrectcap%
\pgfsetroundjoin%
\pgfsetlinewidth{2.710125pt}%
\definecolor{currentstroke}{rgb}{0.260000,0.260000,0.260000}%
\pgfsetstrokecolor{currentstroke}%
\pgfsetdash{}{0pt}%
\pgfusepath{stroke}%
\end{pgfscope}%
\begin{pgfscope}%
\pgfpathrectangle{\pgfqpoint{0.481944in}{1.034041in}}{\pgfqpoint{7.362500in}{2.695000in}}%
\pgfusepath{clip}%
\pgfsetrectcap%
\pgfsetroundjoin%
\pgfsetlinewidth{2.710125pt}%
\definecolor{currentstroke}{rgb}{0.260000,0.260000,0.260000}%
\pgfsetstrokecolor{currentstroke}%
\pgfsetdash{}{0pt}%
\pgfusepath{stroke}%
\end{pgfscope}%
\begin{pgfscope}%
\pgfpathrectangle{\pgfqpoint{0.481944in}{1.034041in}}{\pgfqpoint{7.362500in}{2.695000in}}%
\pgfusepath{clip}%
\pgfsetrectcap%
\pgfsetroundjoin%
\pgfsetlinewidth{2.710125pt}%
\definecolor{currentstroke}{rgb}{0.260000,0.260000,0.260000}%
\pgfsetstrokecolor{currentstroke}%
\pgfsetdash{}{0pt}%
\pgfusepath{stroke}%
\end{pgfscope}%
\begin{pgfscope}%
\pgfpathrectangle{\pgfqpoint{0.481944in}{1.034041in}}{\pgfqpoint{7.362500in}{2.695000in}}%
\pgfusepath{clip}%
\pgfsetrectcap%
\pgfsetroundjoin%
\pgfsetlinewidth{2.710125pt}%
\definecolor{currentstroke}{rgb}{0.260000,0.260000,0.260000}%
\pgfsetstrokecolor{currentstroke}%
\pgfsetdash{}{0pt}%
\pgfusepath{stroke}%
\end{pgfscope}%
\begin{pgfscope}%
\pgfpathrectangle{\pgfqpoint{0.481944in}{1.034041in}}{\pgfqpoint{7.362500in}{2.695000in}}%
\pgfusepath{clip}%
\pgfsetrectcap%
\pgfsetroundjoin%
\pgfsetlinewidth{2.710125pt}%
\definecolor{currentstroke}{rgb}{0.260000,0.260000,0.260000}%
\pgfsetstrokecolor{currentstroke}%
\pgfsetdash{}{0pt}%
\pgfusepath{stroke}%
\end{pgfscope}%
\begin{pgfscope}%
\pgfpathrectangle{\pgfqpoint{0.481944in}{1.034041in}}{\pgfqpoint{7.362500in}{2.695000in}}%
\pgfusepath{clip}%
\pgfsetrectcap%
\pgfsetroundjoin%
\pgfsetlinewidth{2.710125pt}%
\definecolor{currentstroke}{rgb}{0.260000,0.260000,0.260000}%
\pgfsetstrokecolor{currentstroke}%
\pgfsetdash{}{0pt}%
\pgfusepath{stroke}%
\end{pgfscope}%
\begin{pgfscope}%
\pgfpathrectangle{\pgfqpoint{0.481944in}{1.034041in}}{\pgfqpoint{7.362500in}{2.695000in}}%
\pgfusepath{clip}%
\pgfsetrectcap%
\pgfsetroundjoin%
\pgfsetlinewidth{2.710125pt}%
\definecolor{currentstroke}{rgb}{0.260000,0.260000,0.260000}%
\pgfsetstrokecolor{currentstroke}%
\pgfsetdash{}{0pt}%
\pgfusepath{stroke}%
\end{pgfscope}%
\begin{pgfscope}%
\pgfpathrectangle{\pgfqpoint{0.481944in}{1.034041in}}{\pgfqpoint{7.362500in}{2.695000in}}%
\pgfusepath{clip}%
\pgfsetrectcap%
\pgfsetroundjoin%
\pgfsetlinewidth{2.710125pt}%
\definecolor{currentstroke}{rgb}{0.260000,0.260000,0.260000}%
\pgfsetstrokecolor{currentstroke}%
\pgfsetdash{}{0pt}%
\pgfusepath{stroke}%
\end{pgfscope}%
\begin{pgfscope}%
\pgfpathrectangle{\pgfqpoint{0.481944in}{1.034041in}}{\pgfqpoint{7.362500in}{2.695000in}}%
\pgfusepath{clip}%
\pgfsetrectcap%
\pgfsetroundjoin%
\pgfsetlinewidth{2.710125pt}%
\definecolor{currentstroke}{rgb}{0.260000,0.260000,0.260000}%
\pgfsetstrokecolor{currentstroke}%
\pgfsetdash{}{0pt}%
\pgfusepath{stroke}%
\end{pgfscope}%
\begin{pgfscope}%
\pgfpathrectangle{\pgfqpoint{0.481944in}{1.034041in}}{\pgfqpoint{7.362500in}{2.695000in}}%
\pgfusepath{clip}%
\pgfsetrectcap%
\pgfsetroundjoin%
\pgfsetlinewidth{2.710125pt}%
\definecolor{currentstroke}{rgb}{0.260000,0.260000,0.260000}%
\pgfsetstrokecolor{currentstroke}%
\pgfsetdash{}{0pt}%
\pgfusepath{stroke}%
\end{pgfscope}%
\begin{pgfscope}%
\pgfpathrectangle{\pgfqpoint{0.481944in}{1.034041in}}{\pgfqpoint{7.362500in}{2.695000in}}%
\pgfusepath{clip}%
\pgfsetrectcap%
\pgfsetroundjoin%
\pgfsetlinewidth{2.710125pt}%
\definecolor{currentstroke}{rgb}{0.260000,0.260000,0.260000}%
\pgfsetstrokecolor{currentstroke}%
\pgfsetdash{}{0pt}%
\pgfusepath{stroke}%
\end{pgfscope}%
\begin{pgfscope}%
\pgfpathrectangle{\pgfqpoint{0.481944in}{1.034041in}}{\pgfqpoint{7.362500in}{2.695000in}}%
\pgfusepath{clip}%
\pgfsetrectcap%
\pgfsetroundjoin%
\pgfsetlinewidth{2.710125pt}%
\definecolor{currentstroke}{rgb}{0.260000,0.260000,0.260000}%
\pgfsetstrokecolor{currentstroke}%
\pgfsetdash{}{0pt}%
\pgfusepath{stroke}%
\end{pgfscope}%
\begin{pgfscope}%
\pgfpathrectangle{\pgfqpoint{0.481944in}{1.034041in}}{\pgfqpoint{7.362500in}{2.695000in}}%
\pgfusepath{clip}%
\pgfsetrectcap%
\pgfsetroundjoin%
\pgfsetlinewidth{2.710125pt}%
\definecolor{currentstroke}{rgb}{0.260000,0.260000,0.260000}%
\pgfsetstrokecolor{currentstroke}%
\pgfsetdash{}{0pt}%
\pgfusepath{stroke}%
\end{pgfscope}%
\begin{pgfscope}%
\pgfpathrectangle{\pgfqpoint{0.481944in}{1.034041in}}{\pgfqpoint{7.362500in}{2.695000in}}%
\pgfusepath{clip}%
\pgfsetrectcap%
\pgfsetroundjoin%
\pgfsetlinewidth{2.710125pt}%
\definecolor{currentstroke}{rgb}{0.260000,0.260000,0.260000}%
\pgfsetstrokecolor{currentstroke}%
\pgfsetdash{}{0pt}%
\pgfusepath{stroke}%
\end{pgfscope}%
\begin{pgfscope}%
\pgfpathrectangle{\pgfqpoint{0.481944in}{1.034041in}}{\pgfqpoint{7.362500in}{2.695000in}}%
\pgfusepath{clip}%
\pgfsetrectcap%
\pgfsetroundjoin%
\pgfsetlinewidth{2.710125pt}%
\definecolor{currentstroke}{rgb}{0.260000,0.260000,0.260000}%
\pgfsetstrokecolor{currentstroke}%
\pgfsetdash{}{0pt}%
\pgfusepath{stroke}%
\end{pgfscope}%
\begin{pgfscope}%
\pgfpathrectangle{\pgfqpoint{0.481944in}{1.034041in}}{\pgfqpoint{7.362500in}{2.695000in}}%
\pgfusepath{clip}%
\pgfsetrectcap%
\pgfsetroundjoin%
\pgfsetlinewidth{2.710125pt}%
\definecolor{currentstroke}{rgb}{0.260000,0.260000,0.260000}%
\pgfsetstrokecolor{currentstroke}%
\pgfsetdash{}{0pt}%
\pgfusepath{stroke}%
\end{pgfscope}%
\begin{pgfscope}%
\pgfpathrectangle{\pgfqpoint{0.481944in}{1.034041in}}{\pgfqpoint{7.362500in}{2.695000in}}%
\pgfusepath{clip}%
\pgfsetrectcap%
\pgfsetroundjoin%
\pgfsetlinewidth{2.710125pt}%
\definecolor{currentstroke}{rgb}{0.260000,0.260000,0.260000}%
\pgfsetstrokecolor{currentstroke}%
\pgfsetdash{}{0pt}%
\pgfusepath{stroke}%
\end{pgfscope}%
\begin{pgfscope}%
\pgfpathrectangle{\pgfqpoint{0.481944in}{1.034041in}}{\pgfqpoint{7.362500in}{2.695000in}}%
\pgfusepath{clip}%
\pgfsetrectcap%
\pgfsetroundjoin%
\pgfsetlinewidth{2.710125pt}%
\definecolor{currentstroke}{rgb}{0.260000,0.260000,0.260000}%
\pgfsetstrokecolor{currentstroke}%
\pgfsetdash{}{0pt}%
\pgfusepath{stroke}%
\end{pgfscope}%
\begin{pgfscope}%
\pgfpathrectangle{\pgfqpoint{0.481944in}{1.034041in}}{\pgfqpoint{7.362500in}{2.695000in}}%
\pgfusepath{clip}%
\pgfsetrectcap%
\pgfsetroundjoin%
\pgfsetlinewidth{2.710125pt}%
\definecolor{currentstroke}{rgb}{0.260000,0.260000,0.260000}%
\pgfsetstrokecolor{currentstroke}%
\pgfsetdash{}{0pt}%
\pgfusepath{stroke}%
\end{pgfscope}%
\begin{pgfscope}%
\pgfpathrectangle{\pgfqpoint{0.481944in}{1.034041in}}{\pgfqpoint{7.362500in}{2.695000in}}%
\pgfusepath{clip}%
\pgfsetrectcap%
\pgfsetroundjoin%
\pgfsetlinewidth{2.710125pt}%
\definecolor{currentstroke}{rgb}{0.260000,0.260000,0.260000}%
\pgfsetstrokecolor{currentstroke}%
\pgfsetdash{}{0pt}%
\pgfusepath{stroke}%
\end{pgfscope}%
\begin{pgfscope}%
\pgfpathrectangle{\pgfqpoint{0.481944in}{1.034041in}}{\pgfqpoint{7.362500in}{2.695000in}}%
\pgfusepath{clip}%
\pgfsetrectcap%
\pgfsetroundjoin%
\pgfsetlinewidth{2.710125pt}%
\definecolor{currentstroke}{rgb}{0.260000,0.260000,0.260000}%
\pgfsetstrokecolor{currentstroke}%
\pgfsetdash{}{0pt}%
\pgfusepath{stroke}%
\end{pgfscope}%
\begin{pgfscope}%
\pgfpathrectangle{\pgfqpoint{0.481944in}{1.034041in}}{\pgfqpoint{7.362500in}{2.695000in}}%
\pgfusepath{clip}%
\pgfsetrectcap%
\pgfsetroundjoin%
\pgfsetlinewidth{2.710125pt}%
\definecolor{currentstroke}{rgb}{0.260000,0.260000,0.260000}%
\pgfsetstrokecolor{currentstroke}%
\pgfsetdash{}{0pt}%
\pgfusepath{stroke}%
\end{pgfscope}%
\begin{pgfscope}%
\pgfsetrectcap%
\pgfsetmiterjoin%
\pgfsetlinewidth{1.003750pt}%
\definecolor{currentstroke}{rgb}{0.000000,0.000000,0.000000}%
\pgfsetstrokecolor{currentstroke}%
\pgfsetdash{}{0pt}%
\pgfpathmoveto{\pgfqpoint{0.481944in}{1.034041in}}%
\pgfpathlineto{\pgfqpoint{0.481944in}{3.729041in}}%
\pgfusepath{stroke}%
\end{pgfscope}%
\begin{pgfscope}%
\pgfsetrectcap%
\pgfsetmiterjoin%
\pgfsetlinewidth{1.003750pt}%
\definecolor{currentstroke}{rgb}{0.000000,0.000000,0.000000}%
\pgfsetstrokecolor{currentstroke}%
\pgfsetdash{}{0pt}%
\pgfpathmoveto{\pgfqpoint{7.844444in}{1.034041in}}%
\pgfpathlineto{\pgfqpoint{7.844444in}{3.729041in}}%
\pgfusepath{stroke}%
\end{pgfscope}%
\begin{pgfscope}%
\pgfsetrectcap%
\pgfsetmiterjoin%
\pgfsetlinewidth{1.003750pt}%
\definecolor{currentstroke}{rgb}{0.000000,0.000000,0.000000}%
\pgfsetstrokecolor{currentstroke}%
\pgfsetdash{}{0pt}%
\pgfpathmoveto{\pgfqpoint{0.481944in}{1.034041in}}%
\pgfpathlineto{\pgfqpoint{7.844444in}{1.034041in}}%
\pgfusepath{stroke}%
\end{pgfscope}%
\begin{pgfscope}%
\pgfsetrectcap%
\pgfsetmiterjoin%
\pgfsetlinewidth{1.003750pt}%
\definecolor{currentstroke}{rgb}{0.000000,0.000000,0.000000}%
\pgfsetstrokecolor{currentstroke}%
\pgfsetdash{}{0pt}%
\pgfpathmoveto{\pgfqpoint{0.481944in}{3.729041in}}%
\pgfpathlineto{\pgfqpoint{7.844444in}{3.729041in}}%
\pgfusepath{stroke}%
\end{pgfscope}%
\end{pgfpicture}%
\makeatother%
\endgroup%

    \caption{Mean note per assureur (colored by number of ratings)}
    \label{fig:mean_note_per_assureur}
\end{figure}
\begin{figure}[H]
    \advance\leftskip-3cm
    %% Creator: Matplotlib, PGF backend
%%
%% To include the figure in your LaTeX document, write
%%   \input{<filename>.pgf}
%%
%% Make sure the required packages are loaded in your preamble
%%   \usepackage{pgf}
%%
%% Also ensure that all the required font packages are loaded; for instance,
%% the lmodern package is sometimes necessary when using math font.
%%   \usepackage{lmodern}
%%
%% Figures using additional raster images can only be included by \input if
%% they are in the same directory as the main LaTeX file. For loading figures
%% from other directories you can use the `import` package
%%   \usepackage{import}
%%
%% and then include the figures with
%%   \import{<path to file>}{<filename>.pgf}
%%
%% Matplotlib used the following preamble
%%
\begingroup%
\makeatletter%
\begin{pgfpicture}%
\pgfpathrectangle{\pgfpointorigin}{\pgfqpoint{7.944444in}{3.829041in}}%
\pgfusepath{use as bounding box, clip}%
\begin{pgfscope}%
\pgfsetbuttcap%
\pgfsetmiterjoin%
\definecolor{currentfill}{rgb}{1.000000,1.000000,1.000000}%
\pgfsetfillcolor{currentfill}%
\pgfsetlinewidth{0.000000pt}%
\definecolor{currentstroke}{rgb}{1.000000,1.000000,1.000000}%
\pgfsetstrokecolor{currentstroke}%
\pgfsetdash{}{0pt}%
\pgfpathmoveto{\pgfqpoint{0.000000in}{0.000000in}}%
\pgfpathlineto{\pgfqpoint{7.944444in}{0.000000in}}%
\pgfpathlineto{\pgfqpoint{7.944444in}{3.829041in}}%
\pgfpathlineto{\pgfqpoint{0.000000in}{3.829041in}}%
\pgfpathlineto{\pgfqpoint{0.000000in}{0.000000in}}%
\pgfpathclose%
\pgfusepath{fill}%
\end{pgfscope}%
\begin{pgfscope}%
\pgfsetbuttcap%
\pgfsetmiterjoin%
\definecolor{currentfill}{rgb}{1.000000,1.000000,1.000000}%
\pgfsetfillcolor{currentfill}%
\pgfsetlinewidth{0.000000pt}%
\definecolor{currentstroke}{rgb}{0.000000,0.000000,0.000000}%
\pgfsetstrokecolor{currentstroke}%
\pgfsetstrokeopacity{0.000000}%
\pgfsetdash{}{0pt}%
\pgfpathmoveto{\pgfqpoint{0.481944in}{1.034041in}}%
\pgfpathlineto{\pgfqpoint{7.844444in}{1.034041in}}%
\pgfpathlineto{\pgfqpoint{7.844444in}{3.729041in}}%
\pgfpathlineto{\pgfqpoint{0.481944in}{3.729041in}}%
\pgfpathlineto{\pgfqpoint{0.481944in}{1.034041in}}%
\pgfpathclose%
\pgfusepath{fill}%
\end{pgfscope}%
\begin{pgfscope}%
\pgfpathrectangle{\pgfqpoint{0.481944in}{1.034041in}}{\pgfqpoint{7.362500in}{2.695000in}}%
\pgfusepath{clip}%
\pgfsetbuttcap%
\pgfsetmiterjoin%
\definecolor{currentfill}{rgb}{0.192291,0.294409,0.451057}%
\pgfsetfillcolor{currentfill}%
\pgfsetlinewidth{0.000000pt}%
\definecolor{currentstroke}{rgb}{0.000000,0.000000,0.000000}%
\pgfsetstrokecolor{currentstroke}%
\pgfsetstrokeopacity{0.000000}%
\pgfsetdash{}{0pt}%
\pgfpathmoveto{\pgfqpoint{0.495091in}{1.034041in}}%
\pgfpathlineto{\pgfqpoint{0.600270in}{1.034041in}}%
\pgfpathlineto{\pgfqpoint{0.600270in}{3.276583in}}%
\pgfpathlineto{\pgfqpoint{0.495091in}{3.276583in}}%
\pgfpathlineto{\pgfqpoint{0.495091in}{1.034041in}}%
\pgfpathclose%
\pgfusepath{fill}%
\end{pgfscope}%
\begin{pgfscope}%
\pgfpathrectangle{\pgfqpoint{0.481944in}{1.034041in}}{\pgfqpoint{7.362500in}{2.695000in}}%
\pgfusepath{clip}%
\pgfsetbuttcap%
\pgfsetmiterjoin%
\definecolor{currentfill}{rgb}{0.180817,0.420502,0.492726}%
\pgfsetfillcolor{currentfill}%
\pgfsetlinewidth{0.000000pt}%
\definecolor{currentstroke}{rgb}{0.000000,0.000000,0.000000}%
\pgfsetstrokecolor{currentstroke}%
\pgfsetstrokeopacity{0.000000}%
\pgfsetdash{}{0pt}%
\pgfpathmoveto{\pgfqpoint{0.626565in}{1.034041in}}%
\pgfpathlineto{\pgfqpoint{0.731743in}{1.034041in}}%
\pgfpathlineto{\pgfqpoint{0.731743in}{2.504319in}}%
\pgfpathlineto{\pgfqpoint{0.626565in}{2.504319in}}%
\pgfpathlineto{\pgfqpoint{0.626565in}{1.034041in}}%
\pgfpathclose%
\pgfusepath{fill}%
\end{pgfscope}%
\begin{pgfscope}%
\pgfpathrectangle{\pgfqpoint{0.481944in}{1.034041in}}{\pgfqpoint{7.362500in}{2.695000in}}%
\pgfusepath{clip}%
\pgfsetbuttcap%
\pgfsetmiterjoin%
\definecolor{currentfill}{rgb}{0.203716,0.253116,0.427350}%
\pgfsetfillcolor{currentfill}%
\pgfsetlinewidth{0.000000pt}%
\definecolor{currentstroke}{rgb}{0.000000,0.000000,0.000000}%
\pgfsetstrokecolor{currentstroke}%
\pgfsetstrokeopacity{0.000000}%
\pgfsetdash{}{0pt}%
\pgfpathmoveto{\pgfqpoint{0.758038in}{1.034041in}}%
\pgfpathlineto{\pgfqpoint{0.863216in}{1.034041in}}%
\pgfpathlineto{\pgfqpoint{0.863216in}{3.431098in}}%
\pgfpathlineto{\pgfqpoint{0.758038in}{3.431098in}}%
\pgfpathlineto{\pgfqpoint{0.758038in}{1.034041in}}%
\pgfpathclose%
\pgfusepath{fill}%
\end{pgfscope}%
\begin{pgfscope}%
\pgfpathrectangle{\pgfqpoint{0.481944in}{1.034041in}}{\pgfqpoint{7.362500in}{2.695000in}}%
\pgfusepath{clip}%
\pgfsetbuttcap%
\pgfsetmiterjoin%
\definecolor{currentfill}{rgb}{0.183729,0.315078,0.461321}%
\pgfsetfillcolor{currentfill}%
\pgfsetlinewidth{0.000000pt}%
\definecolor{currentstroke}{rgb}{0.000000,0.000000,0.000000}%
\pgfsetstrokecolor{currentstroke}%
\pgfsetstrokeopacity{0.000000}%
\pgfsetdash{}{0pt}%
\pgfpathmoveto{\pgfqpoint{0.889511in}{1.034041in}}%
\pgfpathlineto{\pgfqpoint{0.994690in}{1.034041in}}%
\pgfpathlineto{\pgfqpoint{0.994690in}{2.067345in}}%
\pgfpathlineto{\pgfqpoint{0.889511in}{2.067345in}}%
\pgfpathlineto{\pgfqpoint{0.889511in}{1.034041in}}%
\pgfpathclose%
\pgfusepath{fill}%
\end{pgfscope}%
\begin{pgfscope}%
\pgfpathrectangle{\pgfqpoint{0.481944in}{1.034041in}}{\pgfqpoint{7.362500in}{2.695000in}}%
\pgfusepath{clip}%
\pgfsetbuttcap%
\pgfsetmiterjoin%
\definecolor{currentfill}{rgb}{0.164073,0.382217,0.483403}%
\pgfsetfillcolor{currentfill}%
\pgfsetlinewidth{0.000000pt}%
\definecolor{currentstroke}{rgb}{0.000000,0.000000,0.000000}%
\pgfsetstrokecolor{currentstroke}%
\pgfsetstrokeopacity{0.000000}%
\pgfsetdash{}{0pt}%
\pgfpathmoveto{\pgfqpoint{1.020984in}{1.034041in}}%
\pgfpathlineto{\pgfqpoint{1.126163in}{1.034041in}}%
\pgfpathlineto{\pgfqpoint{1.126163in}{2.074055in}}%
\pgfpathlineto{\pgfqpoint{1.020984in}{2.074055in}}%
\pgfpathlineto{\pgfqpoint{1.020984in}{1.034041in}}%
\pgfpathclose%
\pgfusepath{fill}%
\end{pgfscope}%
\begin{pgfscope}%
\pgfpathrectangle{\pgfqpoint{0.481944in}{1.034041in}}{\pgfqpoint{7.362500in}{2.695000in}}%
\pgfusepath{clip}%
\pgfsetbuttcap%
\pgfsetmiterjoin%
\definecolor{currentfill}{rgb}{0.274508,0.512094,0.516671}%
\pgfsetfillcolor{currentfill}%
\pgfsetlinewidth{0.000000pt}%
\definecolor{currentstroke}{rgb}{0.000000,0.000000,0.000000}%
\pgfsetstrokecolor{currentstroke}%
\pgfsetstrokeopacity{0.000000}%
\pgfsetdash{}{0pt}%
\pgfpathmoveto{\pgfqpoint{1.152458in}{1.034041in}}%
\pgfpathlineto{\pgfqpoint{1.257636in}{1.034041in}}%
\pgfpathlineto{\pgfqpoint{1.257636in}{2.105477in}}%
\pgfpathlineto{\pgfqpoint{1.152458in}{2.105477in}}%
\pgfpathlineto{\pgfqpoint{1.152458in}{1.034041in}}%
\pgfpathclose%
\pgfusepath{fill}%
\end{pgfscope}%
\begin{pgfscope}%
\pgfpathrectangle{\pgfqpoint{0.481944in}{1.034041in}}{\pgfqpoint{7.362500in}{2.695000in}}%
\pgfusepath{clip}%
\pgfsetbuttcap%
\pgfsetmiterjoin%
\definecolor{currentfill}{rgb}{0.531667,0.705639,0.580692}%
\pgfsetfillcolor{currentfill}%
\pgfsetlinewidth{0.000000pt}%
\definecolor{currentstroke}{rgb}{0.000000,0.000000,0.000000}%
\pgfsetstrokecolor{currentstroke}%
\pgfsetstrokeopacity{0.000000}%
\pgfsetdash{}{0pt}%
\pgfpathmoveto{\pgfqpoint{1.283931in}{1.034041in}}%
\pgfpathlineto{\pgfqpoint{1.389109in}{1.034041in}}%
\pgfpathlineto{\pgfqpoint{1.389109in}{2.248898in}}%
\pgfpathlineto{\pgfqpoint{1.283931in}{2.248898in}}%
\pgfpathlineto{\pgfqpoint{1.283931in}{1.034041in}}%
\pgfpathclose%
\pgfusepath{fill}%
\end{pgfscope}%
\begin{pgfscope}%
\pgfpathrectangle{\pgfqpoint{0.481944in}{1.034041in}}{\pgfqpoint{7.362500in}{2.695000in}}%
\pgfusepath{clip}%
\pgfsetbuttcap%
\pgfsetmiterjoin%
\definecolor{currentfill}{rgb}{0.165465,0.392113,0.485883}%
\pgfsetfillcolor{currentfill}%
\pgfsetlinewidth{0.000000pt}%
\definecolor{currentstroke}{rgb}{0.000000,0.000000,0.000000}%
\pgfsetstrokecolor{currentstroke}%
\pgfsetstrokeopacity{0.000000}%
\pgfsetdash{}{0pt}%
\pgfpathmoveto{\pgfqpoint{1.415404in}{1.034041in}}%
\pgfpathlineto{\pgfqpoint{1.520583in}{1.034041in}}%
\pgfpathlineto{\pgfqpoint{1.520583in}{1.886176in}}%
\pgfpathlineto{\pgfqpoint{1.415404in}{1.886176in}}%
\pgfpathlineto{\pgfqpoint{1.415404in}{1.034041in}}%
\pgfpathclose%
\pgfusepath{fill}%
\end{pgfscope}%
\begin{pgfscope}%
\pgfpathrectangle{\pgfqpoint{0.481944in}{1.034041in}}{\pgfqpoint{7.362500in}{2.695000in}}%
\pgfusepath{clip}%
\pgfsetbuttcap%
\pgfsetmiterjoin%
\definecolor{currentfill}{rgb}{0.171813,0.343039,0.472200}%
\pgfsetfillcolor{currentfill}%
\pgfsetlinewidth{0.000000pt}%
\definecolor{currentstroke}{rgb}{0.000000,0.000000,0.000000}%
\pgfsetstrokecolor{currentstroke}%
\pgfsetstrokeopacity{0.000000}%
\pgfsetdash{}{0pt}%
\pgfpathmoveto{\pgfqpoint{1.546877in}{1.034041in}}%
\pgfpathlineto{\pgfqpoint{1.652056in}{1.034041in}}%
\pgfpathlineto{\pgfqpoint{1.652056in}{1.967407in}}%
\pgfpathlineto{\pgfqpoint{1.546877in}{1.967407in}}%
\pgfpathlineto{\pgfqpoint{1.546877in}{1.034041in}}%
\pgfpathclose%
\pgfusepath{fill}%
\end{pgfscope}%
\begin{pgfscope}%
\pgfpathrectangle{\pgfqpoint{0.481944in}{1.034041in}}{\pgfqpoint{7.362500in}{2.695000in}}%
\pgfusepath{clip}%
\pgfsetbuttcap%
\pgfsetmiterjoin%
\definecolor{currentfill}{rgb}{0.405274,0.625213,0.552953}%
\pgfsetfillcolor{currentfill}%
\pgfsetlinewidth{0.000000pt}%
\definecolor{currentstroke}{rgb}{0.000000,0.000000,0.000000}%
\pgfsetstrokecolor{currentstroke}%
\pgfsetstrokeopacity{0.000000}%
\pgfsetdash{}{0pt}%
\pgfpathmoveto{\pgfqpoint{1.678350in}{1.034041in}}%
\pgfpathlineto{\pgfqpoint{1.783529in}{1.034041in}}%
\pgfpathlineto{\pgfqpoint{1.783529in}{2.606209in}}%
\pgfpathlineto{\pgfqpoint{1.678350in}{2.606209in}}%
\pgfpathlineto{\pgfqpoint{1.678350in}{1.034041in}}%
\pgfpathclose%
\pgfusepath{fill}%
\end{pgfscope}%
\begin{pgfscope}%
\pgfpathrectangle{\pgfqpoint{0.481944in}{1.034041in}}{\pgfqpoint{7.362500in}{2.695000in}}%
\pgfusepath{clip}%
\pgfsetbuttcap%
\pgfsetmiterjoin%
\definecolor{currentfill}{rgb}{0.321598,0.556089,0.531070}%
\pgfsetfillcolor{currentfill}%
\pgfsetlinewidth{0.000000pt}%
\definecolor{currentstroke}{rgb}{0.000000,0.000000,0.000000}%
\pgfsetstrokecolor{currentstroke}%
\pgfsetstrokeopacity{0.000000}%
\pgfsetdash{}{0pt}%
\pgfpathmoveto{\pgfqpoint{1.809824in}{1.034041in}}%
\pgfpathlineto{\pgfqpoint{1.915002in}{1.034041in}}%
\pgfpathlineto{\pgfqpoint{1.915002in}{2.188155in}}%
\pgfpathlineto{\pgfqpoint{1.809824in}{2.188155in}}%
\pgfpathlineto{\pgfqpoint{1.809824in}{1.034041in}}%
\pgfpathclose%
\pgfusepath{fill}%
\end{pgfscope}%
\begin{pgfscope}%
\pgfpathrectangle{\pgfqpoint{0.481944in}{1.034041in}}{\pgfqpoint{7.362500in}{2.695000in}}%
\pgfusepath{clip}%
\pgfsetbuttcap%
\pgfsetmiterjoin%
\definecolor{currentfill}{rgb}{0.552281,0.717276,0.585304}%
\pgfsetfillcolor{currentfill}%
\pgfsetlinewidth{0.000000pt}%
\definecolor{currentstroke}{rgb}{0.000000,0.000000,0.000000}%
\pgfsetstrokecolor{currentstroke}%
\pgfsetstrokeopacity{0.000000}%
\pgfsetdash{}{0pt}%
\pgfpathmoveto{\pgfqpoint{1.941297in}{1.034041in}}%
\pgfpathlineto{\pgfqpoint{2.046475in}{1.034041in}}%
\pgfpathlineto{\pgfqpoint{2.046475in}{1.908738in}}%
\pgfpathlineto{\pgfqpoint{1.941297in}{1.908738in}}%
\pgfpathlineto{\pgfqpoint{1.941297in}{1.034041in}}%
\pgfpathclose%
\pgfusepath{fill}%
\end{pgfscope}%
\begin{pgfscope}%
\pgfpathrectangle{\pgfqpoint{0.481944in}{1.034041in}}{\pgfqpoint{7.362500in}{2.695000in}}%
\pgfusepath{clip}%
\pgfsetbuttcap%
\pgfsetmiterjoin%
\definecolor{currentfill}{rgb}{0.251824,0.491212,0.511055}%
\pgfsetfillcolor{currentfill}%
\pgfsetlinewidth{0.000000pt}%
\definecolor{currentstroke}{rgb}{0.000000,0.000000,0.000000}%
\pgfsetstrokecolor{currentstroke}%
\pgfsetstrokeopacity{0.000000}%
\pgfsetdash{}{0pt}%
\pgfpathmoveto{\pgfqpoint{2.072770in}{1.034041in}}%
\pgfpathlineto{\pgfqpoint{2.177949in}{1.034041in}}%
\pgfpathlineto{\pgfqpoint{2.177949in}{1.901796in}}%
\pgfpathlineto{\pgfqpoint{2.072770in}{1.901796in}}%
\pgfpathlineto{\pgfqpoint{2.072770in}{1.034041in}}%
\pgfpathclose%
\pgfusepath{fill}%
\end{pgfscope}%
\begin{pgfscope}%
\pgfpathrectangle{\pgfqpoint{0.481944in}{1.034041in}}{\pgfqpoint{7.362500in}{2.695000in}}%
\pgfusepath{clip}%
\pgfsetbuttcap%
\pgfsetmiterjoin%
\definecolor{currentfill}{rgb}{0.472019,0.670464,0.567533}%
\pgfsetfillcolor{currentfill}%
\pgfsetlinewidth{0.000000pt}%
\definecolor{currentstroke}{rgb}{0.000000,0.000000,0.000000}%
\pgfsetstrokecolor{currentstroke}%
\pgfsetstrokeopacity{0.000000}%
\pgfsetdash{}{0pt}%
\pgfpathmoveto{\pgfqpoint{2.204243in}{1.034041in}}%
\pgfpathlineto{\pgfqpoint{2.309422in}{1.034041in}}%
\pgfpathlineto{\pgfqpoint{2.309422in}{2.495666in}}%
\pgfpathlineto{\pgfqpoint{2.204243in}{2.495666in}}%
\pgfpathlineto{\pgfqpoint{2.204243in}{1.034041in}}%
\pgfpathclose%
\pgfusepath{fill}%
\end{pgfscope}%
\begin{pgfscope}%
\pgfpathrectangle{\pgfqpoint{0.481944in}{1.034041in}}{\pgfqpoint{7.362500in}{2.695000in}}%
\pgfusepath{clip}%
\pgfsetbuttcap%
\pgfsetmiterjoin%
\definecolor{currentfill}{rgb}{0.199045,0.441193,0.497870}%
\pgfsetfillcolor{currentfill}%
\pgfsetlinewidth{0.000000pt}%
\definecolor{currentstroke}{rgb}{0.000000,0.000000,0.000000}%
\pgfsetstrokecolor{currentstroke}%
\pgfsetstrokeopacity{0.000000}%
\pgfsetdash{}{0pt}%
\pgfpathmoveto{\pgfqpoint{2.335716in}{1.034041in}}%
\pgfpathlineto{\pgfqpoint{2.440895in}{1.034041in}}%
\pgfpathlineto{\pgfqpoint{2.440895in}{1.836305in}}%
\pgfpathlineto{\pgfqpoint{2.335716in}{1.836305in}}%
\pgfpathlineto{\pgfqpoint{2.335716in}{1.034041in}}%
\pgfpathclose%
\pgfusepath{fill}%
\end{pgfscope}%
\begin{pgfscope}%
\pgfpathrectangle{\pgfqpoint{0.481944in}{1.034041in}}{\pgfqpoint{7.362500in}{2.695000in}}%
\pgfusepath{clip}%
\pgfsetbuttcap%
\pgfsetmiterjoin%
\definecolor{currentfill}{rgb}{0.218070,0.459928,0.502705}%
\pgfsetfillcolor{currentfill}%
\pgfsetlinewidth{0.000000pt}%
\definecolor{currentstroke}{rgb}{0.000000,0.000000,0.000000}%
\pgfsetstrokecolor{currentstroke}%
\pgfsetstrokeopacity{0.000000}%
\pgfsetdash{}{0pt}%
\pgfpathmoveto{\pgfqpoint{2.467190in}{1.034041in}}%
\pgfpathlineto{\pgfqpoint{2.572368in}{1.034041in}}%
\pgfpathlineto{\pgfqpoint{2.572368in}{1.942345in}}%
\pgfpathlineto{\pgfqpoint{2.467190in}{1.942345in}}%
\pgfpathlineto{\pgfqpoint{2.467190in}{1.034041in}}%
\pgfpathclose%
\pgfusepath{fill}%
\end{pgfscope}%
\begin{pgfscope}%
\pgfpathrectangle{\pgfqpoint{0.481944in}{1.034041in}}{\pgfqpoint{7.362500in}{2.695000in}}%
\pgfusepath{clip}%
\pgfsetbuttcap%
\pgfsetmiterjoin%
\definecolor{currentfill}{rgb}{0.263192,0.501649,0.513869}%
\pgfsetfillcolor{currentfill}%
\pgfsetlinewidth{0.000000pt}%
\definecolor{currentstroke}{rgb}{0.000000,0.000000,0.000000}%
\pgfsetstrokecolor{currentstroke}%
\pgfsetstrokeopacity{0.000000}%
\pgfsetdash{}{0pt}%
\pgfpathmoveto{\pgfqpoint{2.598663in}{1.034041in}}%
\pgfpathlineto{\pgfqpoint{2.703841in}{1.034041in}}%
\pgfpathlineto{\pgfqpoint{2.703841in}{2.111602in}}%
\pgfpathlineto{\pgfqpoint{2.598663in}{2.111602in}}%
\pgfpathlineto{\pgfqpoint{2.598663in}{1.034041in}}%
\pgfpathclose%
\pgfusepath{fill}%
\end{pgfscope}%
\begin{pgfscope}%
\pgfpathrectangle{\pgfqpoint{0.481944in}{1.034041in}}{\pgfqpoint{7.362500in}{2.695000in}}%
\pgfusepath{clip}%
\pgfsetbuttcap%
\pgfsetmiterjoin%
\definecolor{currentfill}{rgb}{0.206675,0.230070,0.415765}%
\pgfsetfillcolor{currentfill}%
\pgfsetlinewidth{0.000000pt}%
\definecolor{currentstroke}{rgb}{0.000000,0.000000,0.000000}%
\pgfsetstrokecolor{currentstroke}%
\pgfsetstrokeopacity{0.000000}%
\pgfsetdash{}{0pt}%
\pgfpathmoveto{\pgfqpoint{2.730136in}{1.034041in}}%
\pgfpathlineto{\pgfqpoint{2.835315in}{1.034041in}}%
\pgfpathlineto{\pgfqpoint{2.835315in}{3.047539in}}%
\pgfpathlineto{\pgfqpoint{2.730136in}{3.047539in}}%
\pgfpathlineto{\pgfqpoint{2.730136in}{1.034041in}}%
\pgfpathclose%
\pgfusepath{fill}%
\end{pgfscope}%
\begin{pgfscope}%
\pgfpathrectangle{\pgfqpoint{0.481944in}{1.034041in}}{\pgfqpoint{7.362500in}{2.695000in}}%
\pgfusepath{clip}%
\pgfsetbuttcap%
\pgfsetmiterjoin%
\definecolor{currentfill}{rgb}{0.296193,0.532476,0.523173}%
\pgfsetfillcolor{currentfill}%
\pgfsetlinewidth{0.000000pt}%
\definecolor{currentstroke}{rgb}{0.000000,0.000000,0.000000}%
\pgfsetstrokecolor{currentstroke}%
\pgfsetstrokeopacity{0.000000}%
\pgfsetdash{}{0pt}%
\pgfpathmoveto{\pgfqpoint{2.861609in}{1.034041in}}%
\pgfpathlineto{\pgfqpoint{2.966788in}{1.034041in}}%
\pgfpathlineto{\pgfqpoint{2.966788in}{2.023413in}}%
\pgfpathlineto{\pgfqpoint{2.861609in}{2.023413in}}%
\pgfpathlineto{\pgfqpoint{2.861609in}{1.034041in}}%
\pgfpathclose%
\pgfusepath{fill}%
\end{pgfscope}%
\begin{pgfscope}%
\pgfpathrectangle{\pgfqpoint{0.481944in}{1.034041in}}{\pgfqpoint{7.362500in}{2.695000in}}%
\pgfusepath{clip}%
\pgfsetbuttcap%
\pgfsetmiterjoin%
\definecolor{currentfill}{rgb}{0.421819,0.637162,0.556681}%
\pgfsetfillcolor{currentfill}%
\pgfsetlinewidth{0.000000pt}%
\definecolor{currentstroke}{rgb}{0.000000,0.000000,0.000000}%
\pgfsetstrokecolor{currentstroke}%
\pgfsetstrokeopacity{0.000000}%
\pgfsetdash{}{0pt}%
\pgfpathmoveto{\pgfqpoint{2.993083in}{1.034041in}}%
\pgfpathlineto{\pgfqpoint{3.098261in}{1.034041in}}%
\pgfpathlineto{\pgfqpoint{3.098261in}{2.022878in}}%
\pgfpathlineto{\pgfqpoint{2.993083in}{2.022878in}}%
\pgfpathlineto{\pgfqpoint{2.993083in}{1.034041in}}%
\pgfpathclose%
\pgfusepath{fill}%
\end{pgfscope}%
\begin{pgfscope}%
\pgfpathrectangle{\pgfqpoint{0.481944in}{1.034041in}}{\pgfqpoint{7.362500in}{2.695000in}}%
\pgfusepath{clip}%
\pgfsetbuttcap%
\pgfsetmiterjoin%
\definecolor{currentfill}{rgb}{0.173626,0.410260,0.490252}%
\pgfsetfillcolor{currentfill}%
\pgfsetlinewidth{0.000000pt}%
\definecolor{currentstroke}{rgb}{0.000000,0.000000,0.000000}%
\pgfsetstrokecolor{currentstroke}%
\pgfsetstrokeopacity{0.000000}%
\pgfsetdash{}{0pt}%
\pgfpathmoveto{\pgfqpoint{3.124556in}{1.034041in}}%
\pgfpathlineto{\pgfqpoint{3.229734in}{1.034041in}}%
\pgfpathlineto{\pgfqpoint{3.229734in}{2.229981in}}%
\pgfpathlineto{\pgfqpoint{3.124556in}{2.229981in}}%
\pgfpathlineto{\pgfqpoint{3.124556in}{1.034041in}}%
\pgfpathclose%
\pgfusepath{fill}%
\end{pgfscope}%
\begin{pgfscope}%
\pgfpathrectangle{\pgfqpoint{0.481944in}{1.034041in}}{\pgfqpoint{7.362500in}{2.695000in}}%
\pgfusepath{clip}%
\pgfsetbuttcap%
\pgfsetmiterjoin%
\definecolor{currentfill}{rgb}{0.201564,0.264259,0.433512}%
\pgfsetfillcolor{currentfill}%
\pgfsetlinewidth{0.000000pt}%
\definecolor{currentstroke}{rgb}{0.000000,0.000000,0.000000}%
\pgfsetstrokecolor{currentstroke}%
\pgfsetstrokeopacity{0.000000}%
\pgfsetdash{}{0pt}%
\pgfpathmoveto{\pgfqpoint{3.256029in}{1.034041in}}%
\pgfpathlineto{\pgfqpoint{3.361208in}{1.034041in}}%
\pgfpathlineto{\pgfqpoint{3.361208in}{2.786941in}}%
\pgfpathlineto{\pgfqpoint{3.256029in}{2.786941in}}%
\pgfpathlineto{\pgfqpoint{3.256029in}{1.034041in}}%
\pgfpathclose%
\pgfusepath{fill}%
\end{pgfscope}%
\begin{pgfscope}%
\pgfpathrectangle{\pgfqpoint{0.481944in}{1.034041in}}{\pgfqpoint{7.362500in}{2.695000in}}%
\pgfusepath{clip}%
\pgfsetbuttcap%
\pgfsetmiterjoin%
\definecolor{currentfill}{rgb}{0.453441,0.658609,0.563554}%
\pgfsetfillcolor{currentfill}%
\pgfsetlinewidth{0.000000pt}%
\definecolor{currentstroke}{rgb}{0.000000,0.000000,0.000000}%
\pgfsetstrokecolor{currentstroke}%
\pgfsetstrokeopacity{0.000000}%
\pgfsetdash{}{0pt}%
\pgfpathmoveto{\pgfqpoint{3.387502in}{1.034041in}}%
\pgfpathlineto{\pgfqpoint{3.492681in}{1.034041in}}%
\pgfpathlineto{\pgfqpoint{3.492681in}{1.935980in}}%
\pgfpathlineto{\pgfqpoint{3.387502in}{1.935980in}}%
\pgfpathlineto{\pgfqpoint{3.387502in}{1.034041in}}%
\pgfpathclose%
\pgfusepath{fill}%
\end{pgfscope}%
\begin{pgfscope}%
\pgfpathrectangle{\pgfqpoint{0.481944in}{1.034041in}}{\pgfqpoint{7.362500in}{2.695000in}}%
\pgfusepath{clip}%
\pgfsetbuttcap%
\pgfsetmiterjoin%
\definecolor{currentfill}{rgb}{0.308751,0.544266,0.527144}%
\pgfsetfillcolor{currentfill}%
\pgfsetlinewidth{0.000000pt}%
\definecolor{currentstroke}{rgb}{0.000000,0.000000,0.000000}%
\pgfsetstrokecolor{currentstroke}%
\pgfsetstrokeopacity{0.000000}%
\pgfsetdash{}{0pt}%
\pgfpathmoveto{\pgfqpoint{3.518975in}{1.034041in}}%
\pgfpathlineto{\pgfqpoint{3.624154in}{1.034041in}}%
\pgfpathlineto{\pgfqpoint{3.624154in}{1.968911in}}%
\pgfpathlineto{\pgfqpoint{3.518975in}{1.968911in}}%
\pgfpathlineto{\pgfqpoint{3.518975in}{1.034041in}}%
\pgfpathclose%
\pgfusepath{fill}%
\end{pgfscope}%
\begin{pgfscope}%
\pgfpathrectangle{\pgfqpoint{0.481944in}{1.034041in}}{\pgfqpoint{7.362500in}{2.695000in}}%
\pgfusepath{clip}%
\pgfsetbuttcap%
\pgfsetmiterjoin%
\definecolor{currentfill}{rgb}{0.334797,0.567947,0.534943}%
\pgfsetfillcolor{currentfill}%
\pgfsetlinewidth{0.000000pt}%
\definecolor{currentstroke}{rgb}{0.000000,0.000000,0.000000}%
\pgfsetstrokecolor{currentstroke}%
\pgfsetstrokeopacity{0.000000}%
\pgfsetdash{}{0pt}%
\pgfpathmoveto{\pgfqpoint{3.650449in}{1.034041in}}%
\pgfpathlineto{\pgfqpoint{3.755627in}{1.034041in}}%
\pgfpathlineto{\pgfqpoint{3.755627in}{2.104868in}}%
\pgfpathlineto{\pgfqpoint{3.650449in}{2.104868in}}%
\pgfpathlineto{\pgfqpoint{3.650449in}{1.034041in}}%
\pgfpathclose%
\pgfusepath{fill}%
\end{pgfscope}%
\begin{pgfscope}%
\pgfpathrectangle{\pgfqpoint{0.481944in}{1.034041in}}{\pgfqpoint{7.362500in}{2.695000in}}%
\pgfusepath{clip}%
\pgfsetbuttcap%
\pgfsetmiterjoin%
\definecolor{currentfill}{rgb}{0.229175,0.470352,0.505459}%
\pgfsetfillcolor{currentfill}%
\pgfsetlinewidth{0.000000pt}%
\definecolor{currentstroke}{rgb}{0.000000,0.000000,0.000000}%
\pgfsetstrokecolor{currentstroke}%
\pgfsetstrokeopacity{0.000000}%
\pgfsetdash{}{0pt}%
\pgfpathmoveto{\pgfqpoint{3.781922in}{1.034041in}}%
\pgfpathlineto{\pgfqpoint{3.887100in}{1.034041in}}%
\pgfpathlineto{\pgfqpoint{3.887100in}{2.704469in}}%
\pgfpathlineto{\pgfqpoint{3.781922in}{2.704469in}}%
\pgfpathlineto{\pgfqpoint{3.781922in}{1.034041in}}%
\pgfpathclose%
\pgfusepath{fill}%
\end{pgfscope}%
\begin{pgfscope}%
\pgfpathrectangle{\pgfqpoint{0.481944in}{1.034041in}}{\pgfqpoint{7.362500in}{2.695000in}}%
\pgfusepath{clip}%
\pgfsetbuttcap%
\pgfsetmiterjoin%
\definecolor{currentfill}{rgb}{0.168248,0.400127,0.487820}%
\pgfsetfillcolor{currentfill}%
\pgfsetlinewidth{0.000000pt}%
\definecolor{currentstroke}{rgb}{0.000000,0.000000,0.000000}%
\pgfsetstrokecolor{currentstroke}%
\pgfsetstrokeopacity{0.000000}%
\pgfsetdash{}{0pt}%
\pgfpathmoveto{\pgfqpoint{3.913395in}{1.034041in}}%
\pgfpathlineto{\pgfqpoint{4.018574in}{1.034041in}}%
\pgfpathlineto{\pgfqpoint{4.018574in}{1.877043in}}%
\pgfpathlineto{\pgfqpoint{3.913395in}{1.877043in}}%
\pgfpathlineto{\pgfqpoint{3.913395in}{1.034041in}}%
\pgfpathclose%
\pgfusepath{fill}%
\end{pgfscope}%
\begin{pgfscope}%
\pgfpathrectangle{\pgfqpoint{0.481944in}{1.034041in}}{\pgfqpoint{7.362500in}{2.695000in}}%
\pgfusepath{clip}%
\pgfsetbuttcap%
\pgfsetmiterjoin%
\definecolor{currentfill}{rgb}{0.206675,0.230070,0.415765}%
\pgfsetfillcolor{currentfill}%
\pgfsetlinewidth{0.000000pt}%
\definecolor{currentstroke}{rgb}{0.000000,0.000000,0.000000}%
\pgfsetstrokecolor{currentstroke}%
\pgfsetstrokeopacity{0.000000}%
\pgfsetdash{}{0pt}%
\pgfpathmoveto{\pgfqpoint{4.044868in}{1.034041in}}%
\pgfpathlineto{\pgfqpoint{4.150047in}{1.034041in}}%
\pgfpathlineto{\pgfqpoint{4.150047in}{1.641469in}}%
\pgfpathlineto{\pgfqpoint{4.044868in}{1.641469in}}%
\pgfpathlineto{\pgfqpoint{4.044868in}{1.034041in}}%
\pgfpathclose%
\pgfusepath{fill}%
\end{pgfscope}%
\begin{pgfscope}%
\pgfpathrectangle{\pgfqpoint{0.481944in}{1.034041in}}{\pgfqpoint{7.362500in}{2.695000in}}%
\pgfusepath{clip}%
\pgfsetbuttcap%
\pgfsetmiterjoin%
\definecolor{currentfill}{rgb}{0.374243,0.601306,0.545528}%
\pgfsetfillcolor{currentfill}%
\pgfsetlinewidth{0.000000pt}%
\definecolor{currentstroke}{rgb}{0.000000,0.000000,0.000000}%
\pgfsetstrokecolor{currentstroke}%
\pgfsetstrokeopacity{0.000000}%
\pgfsetdash{}{0pt}%
\pgfpathmoveto{\pgfqpoint{4.176341in}{1.034041in}}%
\pgfpathlineto{\pgfqpoint{4.281520in}{1.034041in}}%
\pgfpathlineto{\pgfqpoint{4.281520in}{2.123556in}}%
\pgfpathlineto{\pgfqpoint{4.176341in}{2.123556in}}%
\pgfpathlineto{\pgfqpoint{4.176341in}{1.034041in}}%
\pgfpathclose%
\pgfusepath{fill}%
\end{pgfscope}%
\begin{pgfscope}%
\pgfpathrectangle{\pgfqpoint{0.481944in}{1.034041in}}{\pgfqpoint{7.362500in}{2.695000in}}%
\pgfusepath{clip}%
\pgfsetbuttcap%
\pgfsetmiterjoin%
\definecolor{currentfill}{rgb}{0.205383,0.241736,0.421423}%
\pgfsetfillcolor{currentfill}%
\pgfsetlinewidth{0.000000pt}%
\definecolor{currentstroke}{rgb}{0.000000,0.000000,0.000000}%
\pgfsetstrokecolor{currentstroke}%
\pgfsetstrokeopacity{0.000000}%
\pgfsetdash{}{0pt}%
\pgfpathmoveto{\pgfqpoint{4.307815in}{1.034041in}}%
\pgfpathlineto{\pgfqpoint{4.412993in}{1.034041in}}%
\pgfpathlineto{\pgfqpoint{4.412993in}{3.360345in}}%
\pgfpathlineto{\pgfqpoint{4.307815in}{3.360345in}}%
\pgfpathlineto{\pgfqpoint{4.307815in}{1.034041in}}%
\pgfpathclose%
\pgfusepath{fill}%
\end{pgfscope}%
\begin{pgfscope}%
\pgfpathrectangle{\pgfqpoint{0.481944in}{1.034041in}}{\pgfqpoint{7.362500in}{2.695000in}}%
\pgfusepath{clip}%
\pgfsetbuttcap%
\pgfsetmiterjoin%
\definecolor{currentfill}{rgb}{0.588929,0.738349,0.592927}%
\pgfsetfillcolor{currentfill}%
\pgfsetlinewidth{0.000000pt}%
\definecolor{currentstroke}{rgb}{0.000000,0.000000,0.000000}%
\pgfsetstrokecolor{currentstroke}%
\pgfsetstrokeopacity{0.000000}%
\pgfsetdash{}{0pt}%
\pgfpathmoveto{\pgfqpoint{4.439288in}{1.034041in}}%
\pgfpathlineto{\pgfqpoint{4.544466in}{1.034041in}}%
\pgfpathlineto{\pgfqpoint{4.544466in}{1.810199in}}%
\pgfpathlineto{\pgfqpoint{4.439288in}{1.810199in}}%
\pgfpathlineto{\pgfqpoint{4.439288in}{1.034041in}}%
\pgfpathclose%
\pgfusepath{fill}%
\end{pgfscope}%
\begin{pgfscope}%
\pgfpathrectangle{\pgfqpoint{0.481944in}{1.034041in}}{\pgfqpoint{7.362500in}{2.695000in}}%
\pgfusepath{clip}%
\pgfsetbuttcap%
\pgfsetmiterjoin%
\definecolor{currentfill}{rgb}{0.179277,0.325175,0.465628}%
\pgfsetfillcolor{currentfill}%
\pgfsetlinewidth{0.000000pt}%
\definecolor{currentstroke}{rgb}{0.000000,0.000000,0.000000}%
\pgfsetstrokecolor{currentstroke}%
\pgfsetstrokeopacity{0.000000}%
\pgfsetdash{}{0pt}%
\pgfpathmoveto{\pgfqpoint{4.570761in}{1.034041in}}%
\pgfpathlineto{\pgfqpoint{4.675940in}{1.034041in}}%
\pgfpathlineto{\pgfqpoint{4.675940in}{2.155600in}}%
\pgfpathlineto{\pgfqpoint{4.570761in}{2.155600in}}%
\pgfpathlineto{\pgfqpoint{4.570761in}{1.034041in}}%
\pgfpathclose%
\pgfusepath{fill}%
\end{pgfscope}%
\begin{pgfscope}%
\pgfpathrectangle{\pgfqpoint{0.481944in}{1.034041in}}{\pgfqpoint{7.362500in}{2.695000in}}%
\pgfusepath{clip}%
\pgfsetbuttcap%
\pgfsetmiterjoin%
\definecolor{currentfill}{rgb}{0.196092,0.283810,0.445045}%
\pgfsetfillcolor{currentfill}%
\pgfsetlinewidth{0.000000pt}%
\definecolor{currentstroke}{rgb}{0.000000,0.000000,0.000000}%
\pgfsetstrokecolor{currentstroke}%
\pgfsetstrokeopacity{0.000000}%
\pgfsetdash{}{0pt}%
\pgfpathmoveto{\pgfqpoint{4.702234in}{1.034041in}}%
\pgfpathlineto{\pgfqpoint{4.807413in}{1.034041in}}%
\pgfpathlineto{\pgfqpoint{4.807413in}{2.131662in}}%
\pgfpathlineto{\pgfqpoint{4.702234in}{2.131662in}}%
\pgfpathlineto{\pgfqpoint{4.702234in}{1.034041in}}%
\pgfpathclose%
\pgfusepath{fill}%
\end{pgfscope}%
\begin{pgfscope}%
\pgfpathrectangle{\pgfqpoint{0.481944in}{1.034041in}}{\pgfqpoint{7.362500in}{2.695000in}}%
\pgfusepath{clip}%
\pgfsetbuttcap%
\pgfsetmiterjoin%
\definecolor{currentfill}{rgb}{0.174980,0.335140,0.469449}%
\pgfsetfillcolor{currentfill}%
\pgfsetlinewidth{0.000000pt}%
\definecolor{currentstroke}{rgb}{0.000000,0.000000,0.000000}%
\pgfsetstrokecolor{currentstroke}%
\pgfsetstrokeopacity{0.000000}%
\pgfsetdash{}{0pt}%
\pgfpathmoveto{\pgfqpoint{4.833708in}{1.034041in}}%
\pgfpathlineto{\pgfqpoint{4.938886in}{1.034041in}}%
\pgfpathlineto{\pgfqpoint{4.938886in}{2.149202in}}%
\pgfpathlineto{\pgfqpoint{4.833708in}{2.149202in}}%
\pgfpathlineto{\pgfqpoint{4.833708in}{1.034041in}}%
\pgfpathclose%
\pgfusepath{fill}%
\end{pgfscope}%
\begin{pgfscope}%
\pgfpathrectangle{\pgfqpoint{0.481944in}{1.034041in}}{\pgfqpoint{7.362500in}{2.695000in}}%
\pgfusepath{clip}%
\pgfsetbuttcap%
\pgfsetmiterjoin%
\definecolor{currentfill}{rgb}{0.164337,0.372401,0.480845}%
\pgfsetfillcolor{currentfill}%
\pgfsetlinewidth{0.000000pt}%
\definecolor{currentstroke}{rgb}{0.000000,0.000000,0.000000}%
\pgfsetstrokecolor{currentstroke}%
\pgfsetstrokeopacity{0.000000}%
\pgfsetdash{}{0pt}%
\pgfpathmoveto{\pgfqpoint{4.965181in}{1.034041in}}%
\pgfpathlineto{\pgfqpoint{5.070359in}{1.034041in}}%
\pgfpathlineto{\pgfqpoint{5.070359in}{3.214519in}}%
\pgfpathlineto{\pgfqpoint{4.965181in}{3.214519in}}%
\pgfpathlineto{\pgfqpoint{4.965181in}{1.034041in}}%
\pgfpathclose%
\pgfusepath{fill}%
\end{pgfscope}%
\begin{pgfscope}%
\pgfpathrectangle{\pgfqpoint{0.481944in}{1.034041in}}{\pgfqpoint{7.362500in}{2.695000in}}%
\pgfusepath{clip}%
\pgfsetbuttcap%
\pgfsetmiterjoin%
\definecolor{currentfill}{rgb}{0.642098,0.766125,0.595747}%
\pgfsetfillcolor{currentfill}%
\pgfsetlinewidth{0.000000pt}%
\definecolor{currentstroke}{rgb}{0.000000,0.000000,0.000000}%
\pgfsetstrokecolor{currentstroke}%
\pgfsetstrokeopacity{0.000000}%
\pgfsetdash{}{0pt}%
\pgfpathmoveto{\pgfqpoint{5.096654in}{1.034041in}}%
\pgfpathlineto{\pgfqpoint{5.201833in}{1.034041in}}%
\pgfpathlineto{\pgfqpoint{5.201833in}{1.641469in}}%
\pgfpathlineto{\pgfqpoint{5.096654in}{1.641469in}}%
\pgfpathlineto{\pgfqpoint{5.096654in}{1.034041in}}%
\pgfpathclose%
\pgfusepath{fill}%
\end{pgfscope}%
\begin{pgfscope}%
\pgfpathrectangle{\pgfqpoint{0.481944in}{1.034041in}}{\pgfqpoint{7.362500in}{2.695000in}}%
\pgfusepath{clip}%
\pgfsetbuttcap%
\pgfsetmiterjoin%
\definecolor{currentfill}{rgb}{0.568623,0.726620,0.588802}%
\pgfsetfillcolor{currentfill}%
\pgfsetlinewidth{0.000000pt}%
\definecolor{currentstroke}{rgb}{0.000000,0.000000,0.000000}%
\pgfsetstrokecolor{currentstroke}%
\pgfsetstrokeopacity{0.000000}%
\pgfsetdash{}{0pt}%
\pgfpathmoveto{\pgfqpoint{5.228127in}{1.034041in}}%
\pgfpathlineto{\pgfqpoint{5.333306in}{1.034041in}}%
\pgfpathlineto{\pgfqpoint{5.333306in}{2.504657in}}%
\pgfpathlineto{\pgfqpoint{5.228127in}{2.504657in}}%
\pgfpathlineto{\pgfqpoint{5.228127in}{1.034041in}}%
\pgfpathclose%
\pgfusepath{fill}%
\end{pgfscope}%
\begin{pgfscope}%
\pgfpathrectangle{\pgfqpoint{0.481944in}{1.034041in}}{\pgfqpoint{7.362500in}{2.695000in}}%
\pgfusepath{clip}%
\pgfsetbuttcap%
\pgfsetmiterjoin%
\definecolor{currentfill}{rgb}{0.439094,0.649090,0.560463}%
\pgfsetfillcolor{currentfill}%
\pgfsetlinewidth{0.000000pt}%
\definecolor{currentstroke}{rgb}{0.000000,0.000000,0.000000}%
\pgfsetstrokecolor{currentstroke}%
\pgfsetstrokeopacity{0.000000}%
\pgfsetdash{}{0pt}%
\pgfpathmoveto{\pgfqpoint{5.359600in}{1.034041in}}%
\pgfpathlineto{\pgfqpoint{5.464779in}{1.034041in}}%
\pgfpathlineto{\pgfqpoint{5.464779in}{1.865259in}}%
\pgfpathlineto{\pgfqpoint{5.359600in}{1.865259in}}%
\pgfpathlineto{\pgfqpoint{5.359600in}{1.034041in}}%
\pgfpathclose%
\pgfusepath{fill}%
\end{pgfscope}%
\begin{pgfscope}%
\pgfpathrectangle{\pgfqpoint{0.481944in}{1.034041in}}{\pgfqpoint{7.362500in}{2.695000in}}%
\pgfusepath{clip}%
\pgfsetbuttcap%
\pgfsetmiterjoin%
\definecolor{currentfill}{rgb}{0.607110,0.748061,0.594692}%
\pgfsetfillcolor{currentfill}%
\pgfsetlinewidth{0.000000pt}%
\definecolor{currentstroke}{rgb}{0.000000,0.000000,0.000000}%
\pgfsetstrokecolor{currentstroke}%
\pgfsetstrokeopacity{0.000000}%
\pgfsetdash{}{0pt}%
\pgfpathmoveto{\pgfqpoint{5.491074in}{1.034041in}}%
\pgfpathlineto{\pgfqpoint{5.596252in}{1.034041in}}%
\pgfpathlineto{\pgfqpoint{5.596252in}{3.403012in}}%
\pgfpathlineto{\pgfqpoint{5.491074in}{3.403012in}}%
\pgfpathlineto{\pgfqpoint{5.491074in}{1.034041in}}%
\pgfpathclose%
\pgfusepath{fill}%
\end{pgfscope}%
\begin{pgfscope}%
\pgfpathrectangle{\pgfqpoint{0.481944in}{1.034041in}}{\pgfqpoint{7.362500in}{2.695000in}}%
\pgfusepath{clip}%
\pgfsetbuttcap%
\pgfsetmiterjoin%
\definecolor{currentfill}{rgb}{0.165839,0.362623,0.478178}%
\pgfsetfillcolor{currentfill}%
\pgfsetlinewidth{0.000000pt}%
\definecolor{currentstroke}{rgb}{0.000000,0.000000,0.000000}%
\pgfsetstrokecolor{currentstroke}%
\pgfsetstrokeopacity{0.000000}%
\pgfsetdash{}{0pt}%
\pgfpathmoveto{\pgfqpoint{5.622547in}{1.034041in}}%
\pgfpathlineto{\pgfqpoint{5.727725in}{1.034041in}}%
\pgfpathlineto{\pgfqpoint{5.727725in}{2.254024in}}%
\pgfpathlineto{\pgfqpoint{5.622547in}{2.254024in}}%
\pgfpathlineto{\pgfqpoint{5.622547in}{1.034041in}}%
\pgfpathclose%
\pgfusepath{fill}%
\end{pgfscope}%
\begin{pgfscope}%
\pgfpathrectangle{\pgfqpoint{0.481944in}{1.034041in}}{\pgfqpoint{7.362500in}{2.695000in}}%
\pgfusepath{clip}%
\pgfsetbuttcap%
\pgfsetmiterjoin%
\definecolor{currentfill}{rgb}{0.189413,0.430822,0.495265}%
\pgfsetfillcolor{currentfill}%
\pgfsetlinewidth{0.000000pt}%
\definecolor{currentstroke}{rgb}{0.000000,0.000000,0.000000}%
\pgfsetstrokecolor{currentstroke}%
\pgfsetstrokeopacity{0.000000}%
\pgfsetdash{}{0pt}%
\pgfpathmoveto{\pgfqpoint{5.754020in}{1.034041in}}%
\pgfpathlineto{\pgfqpoint{5.859199in}{1.034041in}}%
\pgfpathlineto{\pgfqpoint{5.859199in}{1.821195in}}%
\pgfpathlineto{\pgfqpoint{5.754020in}{1.821195in}}%
\pgfpathlineto{\pgfqpoint{5.754020in}{1.034041in}}%
\pgfpathclose%
\pgfusepath{fill}%
\end{pgfscope}%
\begin{pgfscope}%
\pgfpathrectangle{\pgfqpoint{0.481944in}{1.034041in}}{\pgfqpoint{7.362500in}{2.695000in}}%
\pgfusepath{clip}%
\pgfsetbuttcap%
\pgfsetmiterjoin%
\definecolor{currentfill}{rgb}{0.389428,0.613257,0.549244}%
\pgfsetfillcolor{currentfill}%
\pgfsetlinewidth{0.000000pt}%
\definecolor{currentstroke}{rgb}{0.000000,0.000000,0.000000}%
\pgfsetstrokecolor{currentstroke}%
\pgfsetstrokeopacity{0.000000}%
\pgfsetdash{}{0pt}%
\pgfpathmoveto{\pgfqpoint{5.885493in}{1.034041in}}%
\pgfpathlineto{\pgfqpoint{5.990672in}{1.034041in}}%
\pgfpathlineto{\pgfqpoint{5.990672in}{2.065523in}}%
\pgfpathlineto{\pgfqpoint{5.885493in}{2.065523in}}%
\pgfpathlineto{\pgfqpoint{5.885493in}{1.034041in}}%
\pgfpathclose%
\pgfusepath{fill}%
\end{pgfscope}%
\begin{pgfscope}%
\pgfpathrectangle{\pgfqpoint{0.481944in}{1.034041in}}{\pgfqpoint{7.362500in}{2.695000in}}%
\pgfusepath{clip}%
\pgfsetbuttcap%
\pgfsetmiterjoin%
\definecolor{currentfill}{rgb}{0.240460,0.480780,0.508247}%
\pgfsetfillcolor{currentfill}%
\pgfsetlinewidth{0.000000pt}%
\definecolor{currentstroke}{rgb}{0.000000,0.000000,0.000000}%
\pgfsetstrokecolor{currentstroke}%
\pgfsetstrokeopacity{0.000000}%
\pgfsetdash{}{0pt}%
\pgfpathmoveto{\pgfqpoint{6.016966in}{1.034041in}}%
\pgfpathlineto{\pgfqpoint{6.122145in}{1.034041in}}%
\pgfpathlineto{\pgfqpoint{6.122145in}{1.969888in}}%
\pgfpathlineto{\pgfqpoint{6.016966in}{1.969888in}}%
\pgfpathlineto{\pgfqpoint{6.016966in}{1.034041in}}%
\pgfpathclose%
\pgfusepath{fill}%
\end{pgfscope}%
\begin{pgfscope}%
\pgfpathrectangle{\pgfqpoint{0.481944in}{1.034041in}}{\pgfqpoint{7.362500in}{2.695000in}}%
\pgfusepath{clip}%
\pgfsetbuttcap%
\pgfsetmiterjoin%
\definecolor{currentfill}{rgb}{0.308751,0.544266,0.527144}%
\pgfsetfillcolor{currentfill}%
\pgfsetlinewidth{0.000000pt}%
\definecolor{currentstroke}{rgb}{0.000000,0.000000,0.000000}%
\pgfsetstrokecolor{currentstroke}%
\pgfsetstrokeopacity{0.000000}%
\pgfsetdash{}{0pt}%
\pgfpathmoveto{\pgfqpoint{6.148440in}{1.034041in}}%
\pgfpathlineto{\pgfqpoint{6.253618in}{1.034041in}}%
\pgfpathlineto{\pgfqpoint{6.253618in}{2.130260in}}%
\pgfpathlineto{\pgfqpoint{6.148440in}{2.130260in}}%
\pgfpathlineto{\pgfqpoint{6.148440in}{1.034041in}}%
\pgfpathclose%
\pgfusepath{fill}%
\end{pgfscope}%
\begin{pgfscope}%
\pgfpathrectangle{\pgfqpoint{0.481944in}{1.034041in}}{\pgfqpoint{7.362500in}{2.695000in}}%
\pgfusepath{clip}%
\pgfsetbuttcap%
\pgfsetmiterjoin%
\definecolor{currentfill}{rgb}{0.198801,0.275197,0.439878}%
\pgfsetfillcolor{currentfill}%
\pgfsetlinewidth{0.000000pt}%
\definecolor{currentstroke}{rgb}{0.000000,0.000000,0.000000}%
\pgfsetstrokecolor{currentstroke}%
\pgfsetstrokeopacity{0.000000}%
\pgfsetdash{}{0pt}%
\pgfpathmoveto{\pgfqpoint{6.279913in}{1.034041in}}%
\pgfpathlineto{\pgfqpoint{6.385091in}{1.034041in}}%
\pgfpathlineto{\pgfqpoint{6.385091in}{2.732160in}}%
\pgfpathlineto{\pgfqpoint{6.279913in}{2.732160in}}%
\pgfpathlineto{\pgfqpoint{6.279913in}{1.034041in}}%
\pgfpathclose%
\pgfusepath{fill}%
\end{pgfscope}%
\begin{pgfscope}%
\pgfpathrectangle{\pgfqpoint{0.481944in}{1.034041in}}{\pgfqpoint{7.362500in}{2.695000in}}%
\pgfusepath{clip}%
\pgfsetbuttcap%
\pgfsetmiterjoin%
\definecolor{currentfill}{rgb}{0.168406,0.352848,0.475324}%
\pgfsetfillcolor{currentfill}%
\pgfsetlinewidth{0.000000pt}%
\definecolor{currentstroke}{rgb}{0.000000,0.000000,0.000000}%
\pgfsetstrokecolor{currentstroke}%
\pgfsetstrokeopacity{0.000000}%
\pgfsetdash{}{0pt}%
\pgfpathmoveto{\pgfqpoint{6.411386in}{1.034041in}}%
\pgfpathlineto{\pgfqpoint{6.516565in}{1.034041in}}%
\pgfpathlineto{\pgfqpoint{6.516565in}{2.207158in}}%
\pgfpathlineto{\pgfqpoint{6.411386in}{2.207158in}}%
\pgfpathlineto{\pgfqpoint{6.411386in}{1.034041in}}%
\pgfpathclose%
\pgfusepath{fill}%
\end{pgfscope}%
\begin{pgfscope}%
\pgfpathrectangle{\pgfqpoint{0.481944in}{1.034041in}}{\pgfqpoint{7.362500in}{2.695000in}}%
\pgfusepath{clip}%
\pgfsetbuttcap%
\pgfsetmiterjoin%
\definecolor{currentfill}{rgb}{0.491283,0.682262,0.571685}%
\pgfsetfillcolor{currentfill}%
\pgfsetlinewidth{0.000000pt}%
\definecolor{currentstroke}{rgb}{0.000000,0.000000,0.000000}%
\pgfsetstrokecolor{currentstroke}%
\pgfsetstrokeopacity{0.000000}%
\pgfsetdash{}{0pt}%
\pgfpathmoveto{\pgfqpoint{6.542859in}{1.034041in}}%
\pgfpathlineto{\pgfqpoint{6.648038in}{1.034041in}}%
\pgfpathlineto{\pgfqpoint{6.648038in}{3.246817in}}%
\pgfpathlineto{\pgfqpoint{6.542859in}{3.246817in}}%
\pgfpathlineto{\pgfqpoint{6.542859in}{1.034041in}}%
\pgfpathclose%
\pgfusepath{fill}%
\end{pgfscope}%
\begin{pgfscope}%
\pgfpathrectangle{\pgfqpoint{0.481944in}{1.034041in}}{\pgfqpoint{7.362500in}{2.695000in}}%
\pgfusepath{clip}%
\pgfsetbuttcap%
\pgfsetmiterjoin%
\definecolor{currentfill}{rgb}{0.188125,0.304829,0.456476}%
\pgfsetfillcolor{currentfill}%
\pgfsetlinewidth{0.000000pt}%
\definecolor{currentstroke}{rgb}{0.000000,0.000000,0.000000}%
\pgfsetstrokecolor{currentstroke}%
\pgfsetstrokeopacity{0.000000}%
\pgfsetdash{}{0pt}%
\pgfpathmoveto{\pgfqpoint{6.674333in}{1.034041in}}%
\pgfpathlineto{\pgfqpoint{6.779511in}{1.034041in}}%
\pgfpathlineto{\pgfqpoint{6.779511in}{3.140655in}}%
\pgfpathlineto{\pgfqpoint{6.674333in}{3.140655in}}%
\pgfpathlineto{\pgfqpoint{6.674333in}{1.034041in}}%
\pgfpathclose%
\pgfusepath{fill}%
\end{pgfscope}%
\begin{pgfscope}%
\pgfpathrectangle{\pgfqpoint{0.481944in}{1.034041in}}{\pgfqpoint{7.362500in}{2.695000in}}%
\pgfusepath{clip}%
\pgfsetbuttcap%
\pgfsetmiterjoin%
\definecolor{currentfill}{rgb}{0.348423,0.579837,0.538765}%
\pgfsetfillcolor{currentfill}%
\pgfsetlinewidth{0.000000pt}%
\definecolor{currentstroke}{rgb}{0.000000,0.000000,0.000000}%
\pgfsetstrokecolor{currentstroke}%
\pgfsetstrokeopacity{0.000000}%
\pgfsetdash{}{0pt}%
\pgfpathmoveto{\pgfqpoint{6.805806in}{1.034041in}}%
\pgfpathlineto{\pgfqpoint{6.910984in}{1.034041in}}%
\pgfpathlineto{\pgfqpoint{6.910984in}{2.198279in}}%
\pgfpathlineto{\pgfqpoint{6.805806in}{2.198279in}}%
\pgfpathlineto{\pgfqpoint{6.805806in}{1.034041in}}%
\pgfpathclose%
\pgfusepath{fill}%
\end{pgfscope}%
\begin{pgfscope}%
\pgfpathrectangle{\pgfqpoint{0.481944in}{1.034041in}}{\pgfqpoint{7.362500in}{2.695000in}}%
\pgfusepath{clip}%
\pgfsetbuttcap%
\pgfsetmiterjoin%
\definecolor{currentfill}{rgb}{0.624593,0.757104,0.595421}%
\pgfsetfillcolor{currentfill}%
\pgfsetlinewidth{0.000000pt}%
\definecolor{currentstroke}{rgb}{0.000000,0.000000,0.000000}%
\pgfsetstrokecolor{currentstroke}%
\pgfsetstrokeopacity{0.000000}%
\pgfsetdash{}{0pt}%
\pgfpathmoveto{\pgfqpoint{6.937279in}{1.034041in}}%
\pgfpathlineto{\pgfqpoint{7.042458in}{1.034041in}}%
\pgfpathlineto{\pgfqpoint{7.042458in}{2.046422in}}%
\pgfpathlineto{\pgfqpoint{6.937279in}{2.046422in}}%
\pgfpathlineto{\pgfqpoint{6.937279in}{1.034041in}}%
\pgfpathclose%
\pgfusepath{fill}%
\end{pgfscope}%
\begin{pgfscope}%
\pgfpathrectangle{\pgfqpoint{0.481944in}{1.034041in}}{\pgfqpoint{7.362500in}{2.695000in}}%
\pgfusepath{clip}%
\pgfsetbuttcap%
\pgfsetmiterjoin%
\definecolor{currentfill}{rgb}{0.359677,0.589370,0.541787}%
\pgfsetfillcolor{currentfill}%
\pgfsetlinewidth{0.000000pt}%
\definecolor{currentstroke}{rgb}{0.000000,0.000000,0.000000}%
\pgfsetstrokecolor{currentstroke}%
\pgfsetstrokeopacity{0.000000}%
\pgfsetdash{}{0pt}%
\pgfpathmoveto{\pgfqpoint{7.068752in}{1.034041in}}%
\pgfpathlineto{\pgfqpoint{7.173931in}{1.034041in}}%
\pgfpathlineto{\pgfqpoint{7.173931in}{1.987366in}}%
\pgfpathlineto{\pgfqpoint{7.068752in}{1.987366in}}%
\pgfpathlineto{\pgfqpoint{7.068752in}{1.034041in}}%
\pgfpathclose%
\pgfusepath{fill}%
\end{pgfscope}%
\begin{pgfscope}%
\pgfpathrectangle{\pgfqpoint{0.481944in}{1.034041in}}{\pgfqpoint{7.362500in}{2.695000in}}%
\pgfusepath{clip}%
\pgfsetbuttcap%
\pgfsetmiterjoin%
\definecolor{currentfill}{rgb}{0.348423,0.579837,0.538765}%
\pgfsetfillcolor{currentfill}%
\pgfsetlinewidth{0.000000pt}%
\definecolor{currentstroke}{rgb}{0.000000,0.000000,0.000000}%
\pgfsetstrokecolor{currentstroke}%
\pgfsetstrokeopacity{0.000000}%
\pgfsetdash{}{0pt}%
\pgfpathmoveto{\pgfqpoint{7.200225in}{1.034041in}}%
\pgfpathlineto{\pgfqpoint{7.305404in}{1.034041in}}%
\pgfpathlineto{\pgfqpoint{7.305404in}{1.900892in}}%
\pgfpathlineto{\pgfqpoint{7.200225in}{1.900892in}}%
\pgfpathlineto{\pgfqpoint{7.200225in}{1.034041in}}%
\pgfpathclose%
\pgfusepath{fill}%
\end{pgfscope}%
\begin{pgfscope}%
\pgfpathrectangle{\pgfqpoint{0.481944in}{1.034041in}}{\pgfqpoint{7.362500in}{2.695000in}}%
\pgfusepath{clip}%
\pgfsetbuttcap%
\pgfsetmiterjoin%
\definecolor{currentfill}{rgb}{0.552281,0.717276,0.585304}%
\pgfsetfillcolor{currentfill}%
\pgfsetlinewidth{0.000000pt}%
\definecolor{currentstroke}{rgb}{0.000000,0.000000,0.000000}%
\pgfsetstrokecolor{currentstroke}%
\pgfsetstrokeopacity{0.000000}%
\pgfsetdash{}{0pt}%
\pgfpathmoveto{\pgfqpoint{7.331699in}{1.034041in}}%
\pgfpathlineto{\pgfqpoint{7.436877in}{1.034041in}}%
\pgfpathlineto{\pgfqpoint{7.436877in}{2.273195in}}%
\pgfpathlineto{\pgfqpoint{7.331699in}{2.273195in}}%
\pgfpathlineto{\pgfqpoint{7.331699in}{1.034041in}}%
\pgfpathclose%
\pgfusepath{fill}%
\end{pgfscope}%
\begin{pgfscope}%
\pgfpathrectangle{\pgfqpoint{0.481944in}{1.034041in}}{\pgfqpoint{7.362500in}{2.695000in}}%
\pgfusepath{clip}%
\pgfsetbuttcap%
\pgfsetmiterjoin%
\definecolor{currentfill}{rgb}{0.511189,0.693989,0.576059}%
\pgfsetfillcolor{currentfill}%
\pgfsetlinewidth{0.000000pt}%
\definecolor{currentstroke}{rgb}{0.000000,0.000000,0.000000}%
\pgfsetstrokecolor{currentstroke}%
\pgfsetstrokeopacity{0.000000}%
\pgfsetdash{}{0pt}%
\pgfpathmoveto{\pgfqpoint{7.463172in}{1.034041in}}%
\pgfpathlineto{\pgfqpoint{7.568350in}{1.034041in}}%
\pgfpathlineto{\pgfqpoint{7.568350in}{2.068919in}}%
\pgfpathlineto{\pgfqpoint{7.463172in}{2.068919in}}%
\pgfpathlineto{\pgfqpoint{7.463172in}{1.034041in}}%
\pgfpathclose%
\pgfusepath{fill}%
\end{pgfscope}%
\begin{pgfscope}%
\pgfpathrectangle{\pgfqpoint{0.481944in}{1.034041in}}{\pgfqpoint{7.362500in}{2.695000in}}%
\pgfusepath{clip}%
\pgfsetbuttcap%
\pgfsetmiterjoin%
\definecolor{currentfill}{rgb}{0.283759,0.520718,0.519153}%
\pgfsetfillcolor{currentfill}%
\pgfsetlinewidth{0.000000pt}%
\definecolor{currentstroke}{rgb}{0.000000,0.000000,0.000000}%
\pgfsetstrokecolor{currentstroke}%
\pgfsetstrokeopacity{0.000000}%
\pgfsetdash{}{0pt}%
\pgfpathmoveto{\pgfqpoint{7.594645in}{1.034041in}}%
\pgfpathlineto{\pgfqpoint{7.699824in}{1.034041in}}%
\pgfpathlineto{\pgfqpoint{7.699824in}{1.887962in}}%
\pgfpathlineto{\pgfqpoint{7.594645in}{1.887962in}}%
\pgfpathlineto{\pgfqpoint{7.594645in}{1.034041in}}%
\pgfpathclose%
\pgfusepath{fill}%
\end{pgfscope}%
\begin{pgfscope}%
\pgfpathrectangle{\pgfqpoint{0.481944in}{1.034041in}}{\pgfqpoint{7.362500in}{2.695000in}}%
\pgfusepath{clip}%
\pgfsetbuttcap%
\pgfsetmiterjoin%
\definecolor{currentfill}{rgb}{0.209419,0.451595,0.500534}%
\pgfsetfillcolor{currentfill}%
\pgfsetlinewidth{0.000000pt}%
\definecolor{currentstroke}{rgb}{0.000000,0.000000,0.000000}%
\pgfsetstrokecolor{currentstroke}%
\pgfsetstrokeopacity{0.000000}%
\pgfsetdash{}{0pt}%
\pgfpathmoveto{\pgfqpoint{7.726118in}{1.034041in}}%
\pgfpathlineto{\pgfqpoint{7.831297in}{1.034041in}}%
\pgfpathlineto{\pgfqpoint{7.831297in}{3.729041in}}%
\pgfpathlineto{\pgfqpoint{7.726118in}{3.729041in}}%
\pgfpathlineto{\pgfqpoint{7.726118in}{1.034041in}}%
\pgfpathclose%
\pgfusepath{fill}%
\end{pgfscope}%
\begin{pgfscope}%
\definecolor{textcolor}{rgb}{0.000000,0.000000,0.000000}%
\pgfsetstrokecolor{textcolor}%
\pgfsetfillcolor{textcolor}%
\pgftext[x=0.563884in, y=0.852615in, left, base,rotate=90.000000]{\color{textcolor}\ttfamily\fontsize{6.000000}{7.200000}\selectfont AMV}%
\end{pgfscope}%
\begin{pgfscope}%
\definecolor{textcolor}{rgb}{0.000000,0.000000,0.000000}%
\pgfsetstrokecolor{textcolor}%
\pgfsetfillcolor{textcolor}%
\pgftext[x=0.695358in, y=0.764072in, left, base,rotate=90.000000]{\color{textcolor}\ttfamily\fontsize{6.000000}{7.200000}\selectfont APRIL}%
\end{pgfscope}%
\begin{pgfscope}%
\definecolor{textcolor}{rgb}{0.000000,0.000000,0.000000}%
\pgfsetstrokecolor{textcolor}%
\pgfsetfillcolor{textcolor}%
\pgftext[x=0.826831in, y=0.542715in, left, base,rotate=90.000000]{\color{textcolor}\ttfamily\fontsize{6.000000}{7.200000}\selectfont APRIL Moto}%
\end{pgfscope}%
\begin{pgfscope}%
\definecolor{textcolor}{rgb}{0.000000,0.000000,0.000000}%
\pgfsetstrokecolor{textcolor}%
\pgfsetfillcolor{textcolor}%
\pgftext[x=0.958304in, y=0.852615in, left, base,rotate=90.000000]{\color{textcolor}\ttfamily\fontsize{6.000000}{7.200000}\selectfont AXA}%
\end{pgfscope}%
\begin{pgfscope}%
\definecolor{textcolor}{rgb}{0.000000,0.000000,0.000000}%
\pgfsetstrokecolor{textcolor}%
\pgfsetfillcolor{textcolor}%
\pgftext[x=1.089777in, y=0.232814in, left, base,rotate=90.000000]{\color{textcolor}\ttfamily\fontsize{6.000000}{7.200000}\selectfont Active Assurances}%
\end{pgfscope}%
\begin{pgfscope}%
\definecolor{textcolor}{rgb}{0.000000,0.000000,0.000000}%
\pgfsetstrokecolor{textcolor}%
\pgfsetfillcolor{textcolor}%
\pgftext[x=1.221251in, y=0.808344in, left, base,rotate=90.000000]{\color{textcolor}\ttfamily\fontsize{6.000000}{7.200000}\selectfont Afer}%
\end{pgfscope}%
\begin{pgfscope}%
\definecolor{textcolor}{rgb}{0.000000,0.000000,0.000000}%
\pgfsetstrokecolor{textcolor}%
\pgfsetfillcolor{textcolor}%
\pgftext[x=1.352724in, y=0.631258in, left, base,rotate=90.000000]{\color{textcolor}\ttfamily\fontsize{6.000000}{7.200000}\selectfont Afi Esca}%
\end{pgfscope}%
\begin{pgfscope}%
\definecolor{textcolor}{rgb}{0.000000,0.000000,0.000000}%
\pgfsetstrokecolor{textcolor}%
\pgfsetfillcolor{textcolor}%
\pgftext[x=1.484197in, y=0.277086in, left, base,rotate=90.000000]{\color{textcolor}\ttfamily\fontsize{6.000000}{7.200000}\selectfont Ag2r La Mondiale}%
\end{pgfscope}%
\begin{pgfscope}%
\definecolor{textcolor}{rgb}{0.000000,0.000000,0.000000}%
\pgfsetstrokecolor{textcolor}%
\pgfsetfillcolor{textcolor}%
\pgftext[x=1.615670in, y=0.675529in, left, base,rotate=90.000000]{\color{textcolor}\ttfamily\fontsize{6.000000}{7.200000}\selectfont Allianz}%
\end{pgfscope}%
\begin{pgfscope}%
\definecolor{textcolor}{rgb}{0.000000,0.000000,0.000000}%
\pgfsetstrokecolor{textcolor}%
\pgfsetfillcolor{textcolor}%
\pgftext[x=1.747143in, y=0.365629in, left, base,rotate=90.000000]{\color{textcolor}\ttfamily\fontsize{6.000000}{7.200000}\selectfont Assur Bon Plan}%
\end{pgfscope}%
\begin{pgfscope}%
\definecolor{textcolor}{rgb}{0.000000,0.000000,0.000000}%
\pgfsetstrokecolor{textcolor}%
\pgfsetfillcolor{textcolor}%
\pgftext[x=1.878617in, y=0.454172in, left, base,rotate=90.000000]{\color{textcolor}\ttfamily\fontsize{6.000000}{7.200000}\selectfont Assur O'Poil}%
\end{pgfscope}%
\begin{pgfscope}%
\definecolor{textcolor}{rgb}{0.000000,0.000000,0.000000}%
\pgfsetstrokecolor{textcolor}%
\pgfsetfillcolor{textcolor}%
\pgftext[x=2.010090in, y=0.498443in, left, base,rotate=90.000000]{\color{textcolor}\ttfamily\fontsize{6.000000}{7.200000}\selectfont AssurOnline}%
\end{pgfscope}%
\begin{pgfscope}%
\definecolor{textcolor}{rgb}{0.000000,0.000000,0.000000}%
\pgfsetstrokecolor{textcolor}%
\pgfsetfillcolor{textcolor}%
\pgftext[x=2.141563in, y=0.365629in, left, base,rotate=90.000000]{\color{textcolor}\ttfamily\fontsize{6.000000}{7.200000}\selectfont CNP Assurances}%
\end{pgfscope}%
\begin{pgfscope}%
\definecolor{textcolor}{rgb}{0.000000,0.000000,0.000000}%
\pgfsetstrokecolor{textcolor}%
\pgfsetfillcolor{textcolor}%
\pgftext[x=2.273036in, y=0.764072in, left, base,rotate=90.000000]{\color{textcolor}\ttfamily\fontsize{6.000000}{7.200000}\selectfont Carac}%
\end{pgfscope}%
\begin{pgfscope}%
\definecolor{textcolor}{rgb}{0.000000,0.000000,0.000000}%
\pgfsetstrokecolor{textcolor}%
\pgfsetfillcolor{textcolor}%
\pgftext[x=2.404509in, y=0.719801in, left, base,rotate=90.000000]{\color{textcolor}\ttfamily\fontsize{6.000000}{7.200000}\selectfont Cardif}%
\end{pgfscope}%
\begin{pgfscope}%
\definecolor{textcolor}{rgb}{0.000000,0.000000,0.000000}%
\pgfsetstrokecolor{textcolor}%
\pgfsetfillcolor{textcolor}%
\pgftext[x=2.535983in, y=0.232814in, left, base,rotate=90.000000]{\color{textcolor}\ttfamily\fontsize{6.000000}{7.200000}\selectfont Cegema Assurances}%
\end{pgfscope}%
\begin{pgfscope}%
\definecolor{textcolor}{rgb}{0.000000,0.000000,0.000000}%
\pgfsetstrokecolor{textcolor}%
\pgfsetfillcolor{textcolor}%
\pgftext[x=2.667456in, y=0.409900in, left, base,rotate=90.000000]{\color{textcolor}\ttfamily\fontsize{6.000000}{7.200000}\selectfont Crédit Mutuel}%
\end{pgfscope}%
\begin{pgfscope}%
\definecolor{textcolor}{rgb}{0.000000,0.000000,0.000000}%
\pgfsetstrokecolor{textcolor}%
\pgfsetfillcolor{textcolor}%
\pgftext[x=2.798929in, y=0.277086in, left, base,rotate=90.000000]{\color{textcolor}\ttfamily\fontsize{6.000000}{7.200000}\selectfont Direct Assurance}%
\end{pgfscope}%
\begin{pgfscope}%
\definecolor{textcolor}{rgb}{0.000000,0.000000,0.000000}%
\pgfsetstrokecolor{textcolor}%
\pgfsetfillcolor{textcolor}%
\pgftext[x=2.930402in, y=0.365629in, left, base,rotate=90.000000]{\color{textcolor}\ttfamily\fontsize{6.000000}{7.200000}\selectfont Eca Assurances}%
\end{pgfscope}%
\begin{pgfscope}%
\definecolor{textcolor}{rgb}{0.000000,0.000000,0.000000}%
\pgfsetstrokecolor{textcolor}%
\pgfsetfillcolor{textcolor}%
\pgftext[x=3.061876in, y=0.365629in, left, base,rotate=90.000000]{\color{textcolor}\ttfamily\fontsize{6.000000}{7.200000}\selectfont Euro-Assurance}%
\end{pgfscope}%
\begin{pgfscope}%
\definecolor{textcolor}{rgb}{0.000000,0.000000,0.000000}%
\pgfsetstrokecolor{textcolor}%
\pgfsetfillcolor{textcolor}%
\pgftext[x=3.193349in, y=0.675529in, left, base,rotate=90.000000]{\color{textcolor}\ttfamily\fontsize{6.000000}{7.200000}\selectfont Eurofil}%
\end{pgfscope}%
\begin{pgfscope}%
\definecolor{textcolor}{rgb}{0.000000,0.000000,0.000000}%
\pgfsetstrokecolor{textcolor}%
\pgfsetfillcolor{textcolor}%
\pgftext[x=3.324822in, y=0.852615in, left, base,rotate=90.000000]{\color{textcolor}\ttfamily\fontsize{6.000000}{7.200000}\selectfont GMF}%
\end{pgfscope}%
\begin{pgfscope}%
\definecolor{textcolor}{rgb}{0.000000,0.000000,0.000000}%
\pgfsetstrokecolor{textcolor}%
\pgfsetfillcolor{textcolor}%
\pgftext[x=3.456295in, y=0.852615in, left, base,rotate=90.000000]{\color{textcolor}\ttfamily\fontsize{6.000000}{7.200000}\selectfont Gan}%
\end{pgfscope}%
\begin{pgfscope}%
\definecolor{textcolor}{rgb}{0.000000,0.000000,0.000000}%
\pgfsetstrokecolor{textcolor}%
\pgfsetfillcolor{textcolor}%
\pgftext[x=3.587768in, y=0.631258in, left, base,rotate=90.000000]{\color{textcolor}\ttfamily\fontsize{6.000000}{7.200000}\selectfont Generali}%
\end{pgfscope}%
\begin{pgfscope}%
\definecolor{textcolor}{rgb}{0.000000,0.000000,0.000000}%
\pgfsetstrokecolor{textcolor}%
\pgfsetfillcolor{textcolor}%
\pgftext[x=3.719242in, y=0.631258in, left, base,rotate=90.000000]{\color{textcolor}\ttfamily\fontsize{6.000000}{7.200000}\selectfont Groupama}%
\end{pgfscope}%
\begin{pgfscope}%
\definecolor{textcolor}{rgb}{0.000000,0.000000,0.000000}%
\pgfsetstrokecolor{textcolor}%
\pgfsetfillcolor{textcolor}%
\pgftext[x=3.850715in, y=0.542715in, left, base,rotate=90.000000]{\color{textcolor}\ttfamily\fontsize{6.000000}{7.200000}\selectfont Génération}%
\end{pgfscope}%
\begin{pgfscope}%
\definecolor{textcolor}{rgb}{0.000000,0.000000,0.000000}%
\pgfsetstrokecolor{textcolor}%
\pgfsetfillcolor{textcolor}%
\pgftext[x=3.982188in, y=0.232814in, left, base,rotate=90.000000]{\color{textcolor}\ttfamily\fontsize{6.000000}{7.200000}\selectfont Harmonie Mutuelle}%
\end{pgfscope}%
\begin{pgfscope}%
\definecolor{textcolor}{rgb}{0.000000,0.000000,0.000000}%
\pgfsetstrokecolor{textcolor}%
\pgfsetfillcolor{textcolor}%
\pgftext[x=4.113661in, y=0.719801in, left, base,rotate=90.000000]{\color{textcolor}\ttfamily\fontsize{6.000000}{7.200000}\selectfont Hiscox}%
\end{pgfscope}%
\begin{pgfscope}%
\definecolor{textcolor}{rgb}{0.000000,0.000000,0.000000}%
\pgfsetstrokecolor{textcolor}%
\pgfsetfillcolor{textcolor}%
\pgftext[x=4.245134in, y=0.586986in, left, base,rotate=90.000000]{\color{textcolor}\ttfamily\fontsize{6.000000}{7.200000}\selectfont Intériale}%
\end{pgfscope}%
\begin{pgfscope}%
\definecolor{textcolor}{rgb}{0.000000,0.000000,0.000000}%
\pgfsetstrokecolor{textcolor}%
\pgfsetfillcolor{textcolor}%
\pgftext[x=4.376608in, y=0.144272in, left, base,rotate=90.000000]{\color{textcolor}\ttfamily\fontsize{6.000000}{7.200000}\selectfont L'olivier Assurance}%
\end{pgfscope}%
\begin{pgfscope}%
\definecolor{textcolor}{rgb}{0.000000,0.000000,0.000000}%
\pgfsetstrokecolor{textcolor}%
\pgfsetfillcolor{textcolor}%
\pgftext[x=4.508081in, y=0.852615in, left, base,rotate=90.000000]{\color{textcolor}\ttfamily\fontsize{6.000000}{7.200000}\selectfont LCL}%
\end{pgfscope}%
\begin{pgfscope}%
\definecolor{textcolor}{rgb}{0.000000,0.000000,0.000000}%
\pgfsetstrokecolor{textcolor}%
\pgfsetfillcolor{textcolor}%
\pgftext[x=4.639554in, y=0.808344in, left, base,rotate=90.000000]{\color{textcolor}\ttfamily\fontsize{6.000000}{7.200000}\selectfont MAAF}%
\end{pgfscope}%
\begin{pgfscope}%
\definecolor{textcolor}{rgb}{0.000000,0.000000,0.000000}%
\pgfsetstrokecolor{textcolor}%
\pgfsetfillcolor{textcolor}%
\pgftext[x=4.771027in, y=0.764072in, left, base,rotate=90.000000]{\color{textcolor}\ttfamily\fontsize{6.000000}{7.200000}\selectfont MACIF}%
\end{pgfscope}%
\begin{pgfscope}%
\definecolor{textcolor}{rgb}{0.000000,0.000000,0.000000}%
\pgfsetstrokecolor{textcolor}%
\pgfsetfillcolor{textcolor}%
\pgftext[x=4.902501in, y=0.808344in, left, base,rotate=90.000000]{\color{textcolor}\ttfamily\fontsize{6.000000}{7.200000}\selectfont MAIF}%
\end{pgfscope}%
\begin{pgfscope}%
\definecolor{textcolor}{rgb}{0.000000,0.000000,0.000000}%
\pgfsetstrokecolor{textcolor}%
\pgfsetfillcolor{textcolor}%
\pgftext[x=5.033974in, y=0.852615in, left, base,rotate=90.000000]{\color{textcolor}\ttfamily\fontsize{6.000000}{7.200000}\selectfont MGP}%
\end{pgfscope}%
\begin{pgfscope}%
\definecolor{textcolor}{rgb}{0.000000,0.000000,0.000000}%
\pgfsetstrokecolor{textcolor}%
\pgfsetfillcolor{textcolor}%
\pgftext[x=5.165447in, y=0.852615in, left, base,rotate=90.000000]{\color{textcolor}\ttfamily\fontsize{6.000000}{7.200000}\selectfont MMA}%
\end{pgfscope}%
\begin{pgfscope}%
\definecolor{textcolor}{rgb}{0.000000,0.000000,0.000000}%
\pgfsetstrokecolor{textcolor}%
\pgfsetfillcolor{textcolor}%
\pgftext[x=5.296920in, y=0.631258in, left, base,rotate=90.000000]{\color{textcolor}\ttfamily\fontsize{6.000000}{7.200000}\selectfont Magnolia}%
\end{pgfscope}%
\begin{pgfscope}%
\definecolor{textcolor}{rgb}{0.000000,0.000000,0.000000}%
\pgfsetstrokecolor{textcolor}%
\pgfsetfillcolor{textcolor}%
\pgftext[x=5.428393in, y=0.277086in, left, base,rotate=90.000000]{\color{textcolor}\ttfamily\fontsize{6.000000}{7.200000}\selectfont Malakoff Humanis}%
\end{pgfscope}%
\begin{pgfscope}%
\definecolor{textcolor}{rgb}{0.000000,0.000000,0.000000}%
\pgfsetstrokecolor{textcolor}%
\pgfsetfillcolor{textcolor}%
\pgftext[x=5.559867in, y=0.808344in, left, base,rotate=90.000000]{\color{textcolor}\ttfamily\fontsize{6.000000}{7.200000}\selectfont Mapa}%
\end{pgfscope}%
\begin{pgfscope}%
\definecolor{textcolor}{rgb}{0.000000,0.000000,0.000000}%
\pgfsetstrokecolor{textcolor}%
\pgfsetfillcolor{textcolor}%
\pgftext[x=5.691340in, y=0.719801in, left, base,rotate=90.000000]{\color{textcolor}\ttfamily\fontsize{6.000000}{7.200000}\selectfont Matmut}%
\end{pgfscope}%
\begin{pgfscope}%
\definecolor{textcolor}{rgb}{0.000000,0.000000,0.000000}%
\pgfsetstrokecolor{textcolor}%
\pgfsetfillcolor{textcolor}%
\pgftext[x=5.822813in, y=0.719801in, left, base,rotate=90.000000]{\color{textcolor}\ttfamily\fontsize{6.000000}{7.200000}\selectfont Mercer}%
\end{pgfscope}%
\begin{pgfscope}%
\definecolor{textcolor}{rgb}{0.000000,0.000000,0.000000}%
\pgfsetstrokecolor{textcolor}%
\pgfsetfillcolor{textcolor}%
\pgftext[x=5.954286in, y=0.675529in, left, base,rotate=90.000000]{\color{textcolor}\ttfamily\fontsize{6.000000}{7.200000}\selectfont MetLife}%
\end{pgfscope}%
\begin{pgfscope}%
\definecolor{textcolor}{rgb}{0.000000,0.000000,0.000000}%
\pgfsetstrokecolor{textcolor}%
\pgfsetfillcolor{textcolor}%
\pgftext[x=6.085759in, y=0.808344in, left, base,rotate=90.000000]{\color{textcolor}\ttfamily\fontsize{6.000000}{7.200000}\selectfont Mgen}%
\end{pgfscope}%
\begin{pgfscope}%
\definecolor{textcolor}{rgb}{0.000000,0.000000,0.000000}%
\pgfsetstrokecolor{textcolor}%
\pgfsetfillcolor{textcolor}%
\pgftext[x=6.217233in, y=0.100000in, left, base,rotate=90.000000]{\color{textcolor}\ttfamily\fontsize{6.000000}{7.200000}\selectfont Mutuelle des Motards}%
\end{pgfscope}%
\begin{pgfscope}%
\definecolor{textcolor}{rgb}{0.000000,0.000000,0.000000}%
\pgfsetstrokecolor{textcolor}%
\pgfsetfillcolor{textcolor}%
\pgftext[x=6.348706in, y=0.365629in, left, base,rotate=90.000000]{\color{textcolor}\ttfamily\fontsize{6.000000}{7.200000}\selectfont Néoliane Santé}%
\end{pgfscope}%
\begin{pgfscope}%
\definecolor{textcolor}{rgb}{0.000000,0.000000,0.000000}%
\pgfsetstrokecolor{textcolor}%
\pgfsetfillcolor{textcolor}%
\pgftext[x=6.480179in, y=0.631258in, left, base,rotate=90.000000]{\color{textcolor}\ttfamily\fontsize{6.000000}{7.200000}\selectfont Pacifica}%
\end{pgfscope}%
\begin{pgfscope}%
\definecolor{textcolor}{rgb}{0.000000,0.000000,0.000000}%
\pgfsetstrokecolor{textcolor}%
\pgfsetfillcolor{textcolor}%
\pgftext[x=6.611652in, y=0.232814in, left, base,rotate=90.000000]{\color{textcolor}\ttfamily\fontsize{6.000000}{7.200000}\selectfont Peyrac Assurances}%
\end{pgfscope}%
\begin{pgfscope}%
\definecolor{textcolor}{rgb}{0.000000,0.000000,0.000000}%
\pgfsetstrokecolor{textcolor}%
\pgfsetfillcolor{textcolor}%
\pgftext[x=6.743126in, y=0.631258in, left, base,rotate=90.000000]{\color{textcolor}\ttfamily\fontsize{6.000000}{7.200000}\selectfont Santiane}%
\end{pgfscope}%
\begin{pgfscope}%
\definecolor{textcolor}{rgb}{0.000000,0.000000,0.000000}%
\pgfsetstrokecolor{textcolor}%
\pgfsetfillcolor{textcolor}%
\pgftext[x=6.874599in, y=0.631258in, left, base,rotate=90.000000]{\color{textcolor}\ttfamily\fontsize{6.000000}{7.200000}\selectfont SantéVet}%
\end{pgfscope}%
\begin{pgfscope}%
\definecolor{textcolor}{rgb}{0.000000,0.000000,0.000000}%
\pgfsetstrokecolor{textcolor}%
\pgfsetfillcolor{textcolor}%
\pgftext[x=7.006072in, y=0.852615in, left, base,rotate=90.000000]{\color{textcolor}\ttfamily\fontsize{6.000000}{7.200000}\selectfont Sma}%
\end{pgfscope}%
\begin{pgfscope}%
\definecolor{textcolor}{rgb}{0.000000,0.000000,0.000000}%
\pgfsetstrokecolor{textcolor}%
\pgfsetfillcolor{textcolor}%
\pgftext[x=7.137545in, y=0.675529in, left, base,rotate=90.000000]{\color{textcolor}\ttfamily\fontsize{6.000000}{7.200000}\selectfont Sogecap}%
\end{pgfscope}%
\begin{pgfscope}%
\definecolor{textcolor}{rgb}{0.000000,0.000000,0.000000}%
\pgfsetstrokecolor{textcolor}%
\pgfsetfillcolor{textcolor}%
\pgftext[x=7.269018in, y=0.631258in, left, base,rotate=90.000000]{\color{textcolor}\ttfamily\fontsize{6.000000}{7.200000}\selectfont Sogessur}%
\end{pgfscope}%
\begin{pgfscope}%
\definecolor{textcolor}{rgb}{0.000000,0.000000,0.000000}%
\pgfsetstrokecolor{textcolor}%
\pgfsetfillcolor{textcolor}%
\pgftext[x=7.400492in, y=0.542715in, left, base,rotate=90.000000]{\color{textcolor}\ttfamily\fontsize{6.000000}{7.200000}\selectfont Solly Azar}%
\end{pgfscope}%
\begin{pgfscope}%
\definecolor{textcolor}{rgb}{0.000000,0.000000,0.000000}%
\pgfsetstrokecolor{textcolor}%
\pgfsetfillcolor{textcolor}%
\pgftext[x=7.531965in, y=0.586986in, left, base,rotate=90.000000]{\color{textcolor}\ttfamily\fontsize{6.000000}{7.200000}\selectfont Suravenir}%
\end{pgfscope}%
\begin{pgfscope}%
\definecolor{textcolor}{rgb}{0.000000,0.000000,0.000000}%
\pgfsetstrokecolor{textcolor}%
\pgfsetfillcolor{textcolor}%
\pgftext[x=7.663438in, y=0.586986in, left, base,rotate=90.000000]{\color{textcolor}\ttfamily\fontsize{6.000000}{7.200000}\selectfont SwissLife}%
\end{pgfscope}%
\begin{pgfscope}%
\definecolor{textcolor}{rgb}{0.000000,0.000000,0.000000}%
\pgfsetstrokecolor{textcolor}%
\pgfsetfillcolor{textcolor}%
\pgftext[x=7.794911in, y=0.719801in, left, base,rotate=90.000000]{\color{textcolor}\ttfamily\fontsize{6.000000}{7.200000}\selectfont Zen'Up}%
\end{pgfscope}%
\begin{pgfscope}%
\definecolor{textcolor}{rgb}{0.000000,0.000000,0.000000}%
\pgfsetstrokecolor{textcolor}%
\pgfsetfillcolor{textcolor}%
\pgftext[x=0.396296in, y=1.003370in, left, base,rotate=90.000000]{\color{textcolor}\ttfamily\fontsize{12.000000}{14.400000}\selectfont \(\displaystyle {0}\)}%
\end{pgfscope}%
\begin{pgfscope}%
\definecolor{textcolor}{rgb}{0.000000,0.000000,0.000000}%
\pgfsetstrokecolor{textcolor}%
\pgfsetfillcolor{textcolor}%
\pgftext[x=0.396296in, y=1.610799in, left, base,rotate=90.000000]{\color{textcolor}\ttfamily\fontsize{12.000000}{14.400000}\selectfont \(\displaystyle {1}\)}%
\end{pgfscope}%
\begin{pgfscope}%
\definecolor{textcolor}{rgb}{0.000000,0.000000,0.000000}%
\pgfsetstrokecolor{textcolor}%
\pgfsetfillcolor{textcolor}%
\pgftext[x=0.396296in, y=2.218227in, left, base,rotate=90.000000]{\color{textcolor}\ttfamily\fontsize{12.000000}{14.400000}\selectfont \(\displaystyle {2}\)}%
\end{pgfscope}%
\begin{pgfscope}%
\definecolor{textcolor}{rgb}{0.000000,0.000000,0.000000}%
\pgfsetstrokecolor{textcolor}%
\pgfsetfillcolor{textcolor}%
\pgftext[x=0.396296in, y=2.825656in, left, base,rotate=90.000000]{\color{textcolor}\ttfamily\fontsize{12.000000}{14.400000}\selectfont \(\displaystyle {3}\)}%
\end{pgfscope}%
\begin{pgfscope}%
\definecolor{textcolor}{rgb}{0.000000,0.000000,0.000000}%
\pgfsetstrokecolor{textcolor}%
\pgfsetfillcolor{textcolor}%
\pgftext[x=0.396296in, y=3.433085in, left, base,rotate=90.000000]{\color{textcolor}\ttfamily\fontsize{12.000000}{14.400000}\selectfont \(\displaystyle {4}\)}%
\end{pgfscope}%
\begin{pgfscope}%
\definecolor{textcolor}{rgb}{0.000000,0.000000,0.000000}%
\pgfsetstrokecolor{textcolor}%
\pgfsetfillcolor{textcolor}%
\pgftext[x=0.238889in,y=2.381541in,,bottom,rotate=90.000000]{\color{textcolor}\ttfamily\fontsize{12.000000}{14.400000}\selectfont Mean note}%
\end{pgfscope}%
\begin{pgfscope}%
\pgfpathrectangle{\pgfqpoint{0.481944in}{1.034041in}}{\pgfqpoint{7.362500in}{2.695000in}}%
\pgfusepath{clip}%
\pgfsetrectcap%
\pgfsetroundjoin%
\pgfsetlinewidth{2.710125pt}%
\definecolor{currentstroke}{rgb}{0.260000,0.260000,0.260000}%
\pgfsetstrokecolor{currentstroke}%
\pgfsetdash{}{0pt}%
\pgfusepath{stroke}%
\end{pgfscope}%
\begin{pgfscope}%
\pgfpathrectangle{\pgfqpoint{0.481944in}{1.034041in}}{\pgfqpoint{7.362500in}{2.695000in}}%
\pgfusepath{clip}%
\pgfsetrectcap%
\pgfsetroundjoin%
\pgfsetlinewidth{2.710125pt}%
\definecolor{currentstroke}{rgb}{0.260000,0.260000,0.260000}%
\pgfsetstrokecolor{currentstroke}%
\pgfsetdash{}{0pt}%
\pgfusepath{stroke}%
\end{pgfscope}%
\begin{pgfscope}%
\pgfpathrectangle{\pgfqpoint{0.481944in}{1.034041in}}{\pgfqpoint{7.362500in}{2.695000in}}%
\pgfusepath{clip}%
\pgfsetrectcap%
\pgfsetroundjoin%
\pgfsetlinewidth{2.710125pt}%
\definecolor{currentstroke}{rgb}{0.260000,0.260000,0.260000}%
\pgfsetstrokecolor{currentstroke}%
\pgfsetdash{}{0pt}%
\pgfusepath{stroke}%
\end{pgfscope}%
\begin{pgfscope}%
\pgfpathrectangle{\pgfqpoint{0.481944in}{1.034041in}}{\pgfqpoint{7.362500in}{2.695000in}}%
\pgfusepath{clip}%
\pgfsetrectcap%
\pgfsetroundjoin%
\pgfsetlinewidth{2.710125pt}%
\definecolor{currentstroke}{rgb}{0.260000,0.260000,0.260000}%
\pgfsetstrokecolor{currentstroke}%
\pgfsetdash{}{0pt}%
\pgfusepath{stroke}%
\end{pgfscope}%
\begin{pgfscope}%
\pgfpathrectangle{\pgfqpoint{0.481944in}{1.034041in}}{\pgfqpoint{7.362500in}{2.695000in}}%
\pgfusepath{clip}%
\pgfsetrectcap%
\pgfsetroundjoin%
\pgfsetlinewidth{2.710125pt}%
\definecolor{currentstroke}{rgb}{0.260000,0.260000,0.260000}%
\pgfsetstrokecolor{currentstroke}%
\pgfsetdash{}{0pt}%
\pgfusepath{stroke}%
\end{pgfscope}%
\begin{pgfscope}%
\pgfpathrectangle{\pgfqpoint{0.481944in}{1.034041in}}{\pgfqpoint{7.362500in}{2.695000in}}%
\pgfusepath{clip}%
\pgfsetrectcap%
\pgfsetroundjoin%
\pgfsetlinewidth{2.710125pt}%
\definecolor{currentstroke}{rgb}{0.260000,0.260000,0.260000}%
\pgfsetstrokecolor{currentstroke}%
\pgfsetdash{}{0pt}%
\pgfusepath{stroke}%
\end{pgfscope}%
\begin{pgfscope}%
\pgfpathrectangle{\pgfqpoint{0.481944in}{1.034041in}}{\pgfqpoint{7.362500in}{2.695000in}}%
\pgfusepath{clip}%
\pgfsetrectcap%
\pgfsetroundjoin%
\pgfsetlinewidth{2.710125pt}%
\definecolor{currentstroke}{rgb}{0.260000,0.260000,0.260000}%
\pgfsetstrokecolor{currentstroke}%
\pgfsetdash{}{0pt}%
\pgfusepath{stroke}%
\end{pgfscope}%
\begin{pgfscope}%
\pgfpathrectangle{\pgfqpoint{0.481944in}{1.034041in}}{\pgfqpoint{7.362500in}{2.695000in}}%
\pgfusepath{clip}%
\pgfsetrectcap%
\pgfsetroundjoin%
\pgfsetlinewidth{2.710125pt}%
\definecolor{currentstroke}{rgb}{0.260000,0.260000,0.260000}%
\pgfsetstrokecolor{currentstroke}%
\pgfsetdash{}{0pt}%
\pgfusepath{stroke}%
\end{pgfscope}%
\begin{pgfscope}%
\pgfpathrectangle{\pgfqpoint{0.481944in}{1.034041in}}{\pgfqpoint{7.362500in}{2.695000in}}%
\pgfusepath{clip}%
\pgfsetrectcap%
\pgfsetroundjoin%
\pgfsetlinewidth{2.710125pt}%
\definecolor{currentstroke}{rgb}{0.260000,0.260000,0.260000}%
\pgfsetstrokecolor{currentstroke}%
\pgfsetdash{}{0pt}%
\pgfusepath{stroke}%
\end{pgfscope}%
\begin{pgfscope}%
\pgfpathrectangle{\pgfqpoint{0.481944in}{1.034041in}}{\pgfqpoint{7.362500in}{2.695000in}}%
\pgfusepath{clip}%
\pgfsetrectcap%
\pgfsetroundjoin%
\pgfsetlinewidth{2.710125pt}%
\definecolor{currentstroke}{rgb}{0.260000,0.260000,0.260000}%
\pgfsetstrokecolor{currentstroke}%
\pgfsetdash{}{0pt}%
\pgfusepath{stroke}%
\end{pgfscope}%
\begin{pgfscope}%
\pgfpathrectangle{\pgfqpoint{0.481944in}{1.034041in}}{\pgfqpoint{7.362500in}{2.695000in}}%
\pgfusepath{clip}%
\pgfsetrectcap%
\pgfsetroundjoin%
\pgfsetlinewidth{2.710125pt}%
\definecolor{currentstroke}{rgb}{0.260000,0.260000,0.260000}%
\pgfsetstrokecolor{currentstroke}%
\pgfsetdash{}{0pt}%
\pgfusepath{stroke}%
\end{pgfscope}%
\begin{pgfscope}%
\pgfpathrectangle{\pgfqpoint{0.481944in}{1.034041in}}{\pgfqpoint{7.362500in}{2.695000in}}%
\pgfusepath{clip}%
\pgfsetrectcap%
\pgfsetroundjoin%
\pgfsetlinewidth{2.710125pt}%
\definecolor{currentstroke}{rgb}{0.260000,0.260000,0.260000}%
\pgfsetstrokecolor{currentstroke}%
\pgfsetdash{}{0pt}%
\pgfusepath{stroke}%
\end{pgfscope}%
\begin{pgfscope}%
\pgfpathrectangle{\pgfqpoint{0.481944in}{1.034041in}}{\pgfqpoint{7.362500in}{2.695000in}}%
\pgfusepath{clip}%
\pgfsetrectcap%
\pgfsetroundjoin%
\pgfsetlinewidth{2.710125pt}%
\definecolor{currentstroke}{rgb}{0.260000,0.260000,0.260000}%
\pgfsetstrokecolor{currentstroke}%
\pgfsetdash{}{0pt}%
\pgfusepath{stroke}%
\end{pgfscope}%
\begin{pgfscope}%
\pgfpathrectangle{\pgfqpoint{0.481944in}{1.034041in}}{\pgfqpoint{7.362500in}{2.695000in}}%
\pgfusepath{clip}%
\pgfsetrectcap%
\pgfsetroundjoin%
\pgfsetlinewidth{2.710125pt}%
\definecolor{currentstroke}{rgb}{0.260000,0.260000,0.260000}%
\pgfsetstrokecolor{currentstroke}%
\pgfsetdash{}{0pt}%
\pgfusepath{stroke}%
\end{pgfscope}%
\begin{pgfscope}%
\pgfpathrectangle{\pgfqpoint{0.481944in}{1.034041in}}{\pgfqpoint{7.362500in}{2.695000in}}%
\pgfusepath{clip}%
\pgfsetrectcap%
\pgfsetroundjoin%
\pgfsetlinewidth{2.710125pt}%
\definecolor{currentstroke}{rgb}{0.260000,0.260000,0.260000}%
\pgfsetstrokecolor{currentstroke}%
\pgfsetdash{}{0pt}%
\pgfusepath{stroke}%
\end{pgfscope}%
\begin{pgfscope}%
\pgfpathrectangle{\pgfqpoint{0.481944in}{1.034041in}}{\pgfqpoint{7.362500in}{2.695000in}}%
\pgfusepath{clip}%
\pgfsetrectcap%
\pgfsetroundjoin%
\pgfsetlinewidth{2.710125pt}%
\definecolor{currentstroke}{rgb}{0.260000,0.260000,0.260000}%
\pgfsetstrokecolor{currentstroke}%
\pgfsetdash{}{0pt}%
\pgfusepath{stroke}%
\end{pgfscope}%
\begin{pgfscope}%
\pgfpathrectangle{\pgfqpoint{0.481944in}{1.034041in}}{\pgfqpoint{7.362500in}{2.695000in}}%
\pgfusepath{clip}%
\pgfsetrectcap%
\pgfsetroundjoin%
\pgfsetlinewidth{2.710125pt}%
\definecolor{currentstroke}{rgb}{0.260000,0.260000,0.260000}%
\pgfsetstrokecolor{currentstroke}%
\pgfsetdash{}{0pt}%
\pgfusepath{stroke}%
\end{pgfscope}%
\begin{pgfscope}%
\pgfpathrectangle{\pgfqpoint{0.481944in}{1.034041in}}{\pgfqpoint{7.362500in}{2.695000in}}%
\pgfusepath{clip}%
\pgfsetrectcap%
\pgfsetroundjoin%
\pgfsetlinewidth{2.710125pt}%
\definecolor{currentstroke}{rgb}{0.260000,0.260000,0.260000}%
\pgfsetstrokecolor{currentstroke}%
\pgfsetdash{}{0pt}%
\pgfusepath{stroke}%
\end{pgfscope}%
\begin{pgfscope}%
\pgfpathrectangle{\pgfqpoint{0.481944in}{1.034041in}}{\pgfqpoint{7.362500in}{2.695000in}}%
\pgfusepath{clip}%
\pgfsetrectcap%
\pgfsetroundjoin%
\pgfsetlinewidth{2.710125pt}%
\definecolor{currentstroke}{rgb}{0.260000,0.260000,0.260000}%
\pgfsetstrokecolor{currentstroke}%
\pgfsetdash{}{0pt}%
\pgfusepath{stroke}%
\end{pgfscope}%
\begin{pgfscope}%
\pgfpathrectangle{\pgfqpoint{0.481944in}{1.034041in}}{\pgfqpoint{7.362500in}{2.695000in}}%
\pgfusepath{clip}%
\pgfsetrectcap%
\pgfsetroundjoin%
\pgfsetlinewidth{2.710125pt}%
\definecolor{currentstroke}{rgb}{0.260000,0.260000,0.260000}%
\pgfsetstrokecolor{currentstroke}%
\pgfsetdash{}{0pt}%
\pgfusepath{stroke}%
\end{pgfscope}%
\begin{pgfscope}%
\pgfpathrectangle{\pgfqpoint{0.481944in}{1.034041in}}{\pgfqpoint{7.362500in}{2.695000in}}%
\pgfusepath{clip}%
\pgfsetrectcap%
\pgfsetroundjoin%
\pgfsetlinewidth{2.710125pt}%
\definecolor{currentstroke}{rgb}{0.260000,0.260000,0.260000}%
\pgfsetstrokecolor{currentstroke}%
\pgfsetdash{}{0pt}%
\pgfusepath{stroke}%
\end{pgfscope}%
\begin{pgfscope}%
\pgfpathrectangle{\pgfqpoint{0.481944in}{1.034041in}}{\pgfqpoint{7.362500in}{2.695000in}}%
\pgfusepath{clip}%
\pgfsetrectcap%
\pgfsetroundjoin%
\pgfsetlinewidth{2.710125pt}%
\definecolor{currentstroke}{rgb}{0.260000,0.260000,0.260000}%
\pgfsetstrokecolor{currentstroke}%
\pgfsetdash{}{0pt}%
\pgfusepath{stroke}%
\end{pgfscope}%
\begin{pgfscope}%
\pgfpathrectangle{\pgfqpoint{0.481944in}{1.034041in}}{\pgfqpoint{7.362500in}{2.695000in}}%
\pgfusepath{clip}%
\pgfsetrectcap%
\pgfsetroundjoin%
\pgfsetlinewidth{2.710125pt}%
\definecolor{currentstroke}{rgb}{0.260000,0.260000,0.260000}%
\pgfsetstrokecolor{currentstroke}%
\pgfsetdash{}{0pt}%
\pgfusepath{stroke}%
\end{pgfscope}%
\begin{pgfscope}%
\pgfpathrectangle{\pgfqpoint{0.481944in}{1.034041in}}{\pgfqpoint{7.362500in}{2.695000in}}%
\pgfusepath{clip}%
\pgfsetrectcap%
\pgfsetroundjoin%
\pgfsetlinewidth{2.710125pt}%
\definecolor{currentstroke}{rgb}{0.260000,0.260000,0.260000}%
\pgfsetstrokecolor{currentstroke}%
\pgfsetdash{}{0pt}%
\pgfusepath{stroke}%
\end{pgfscope}%
\begin{pgfscope}%
\pgfpathrectangle{\pgfqpoint{0.481944in}{1.034041in}}{\pgfqpoint{7.362500in}{2.695000in}}%
\pgfusepath{clip}%
\pgfsetrectcap%
\pgfsetroundjoin%
\pgfsetlinewidth{2.710125pt}%
\definecolor{currentstroke}{rgb}{0.260000,0.260000,0.260000}%
\pgfsetstrokecolor{currentstroke}%
\pgfsetdash{}{0pt}%
\pgfusepath{stroke}%
\end{pgfscope}%
\begin{pgfscope}%
\pgfpathrectangle{\pgfqpoint{0.481944in}{1.034041in}}{\pgfqpoint{7.362500in}{2.695000in}}%
\pgfusepath{clip}%
\pgfsetrectcap%
\pgfsetroundjoin%
\pgfsetlinewidth{2.710125pt}%
\definecolor{currentstroke}{rgb}{0.260000,0.260000,0.260000}%
\pgfsetstrokecolor{currentstroke}%
\pgfsetdash{}{0pt}%
\pgfusepath{stroke}%
\end{pgfscope}%
\begin{pgfscope}%
\pgfpathrectangle{\pgfqpoint{0.481944in}{1.034041in}}{\pgfqpoint{7.362500in}{2.695000in}}%
\pgfusepath{clip}%
\pgfsetrectcap%
\pgfsetroundjoin%
\pgfsetlinewidth{2.710125pt}%
\definecolor{currentstroke}{rgb}{0.260000,0.260000,0.260000}%
\pgfsetstrokecolor{currentstroke}%
\pgfsetdash{}{0pt}%
\pgfusepath{stroke}%
\end{pgfscope}%
\begin{pgfscope}%
\pgfpathrectangle{\pgfqpoint{0.481944in}{1.034041in}}{\pgfqpoint{7.362500in}{2.695000in}}%
\pgfusepath{clip}%
\pgfsetrectcap%
\pgfsetroundjoin%
\pgfsetlinewidth{2.710125pt}%
\definecolor{currentstroke}{rgb}{0.260000,0.260000,0.260000}%
\pgfsetstrokecolor{currentstroke}%
\pgfsetdash{}{0pt}%
\pgfusepath{stroke}%
\end{pgfscope}%
\begin{pgfscope}%
\pgfpathrectangle{\pgfqpoint{0.481944in}{1.034041in}}{\pgfqpoint{7.362500in}{2.695000in}}%
\pgfusepath{clip}%
\pgfsetrectcap%
\pgfsetroundjoin%
\pgfsetlinewidth{2.710125pt}%
\definecolor{currentstroke}{rgb}{0.260000,0.260000,0.260000}%
\pgfsetstrokecolor{currentstroke}%
\pgfsetdash{}{0pt}%
\pgfusepath{stroke}%
\end{pgfscope}%
\begin{pgfscope}%
\pgfpathrectangle{\pgfqpoint{0.481944in}{1.034041in}}{\pgfqpoint{7.362500in}{2.695000in}}%
\pgfusepath{clip}%
\pgfsetrectcap%
\pgfsetroundjoin%
\pgfsetlinewidth{2.710125pt}%
\definecolor{currentstroke}{rgb}{0.260000,0.260000,0.260000}%
\pgfsetstrokecolor{currentstroke}%
\pgfsetdash{}{0pt}%
\pgfusepath{stroke}%
\end{pgfscope}%
\begin{pgfscope}%
\pgfpathrectangle{\pgfqpoint{0.481944in}{1.034041in}}{\pgfqpoint{7.362500in}{2.695000in}}%
\pgfusepath{clip}%
\pgfsetrectcap%
\pgfsetroundjoin%
\pgfsetlinewidth{2.710125pt}%
\definecolor{currentstroke}{rgb}{0.260000,0.260000,0.260000}%
\pgfsetstrokecolor{currentstroke}%
\pgfsetdash{}{0pt}%
\pgfusepath{stroke}%
\end{pgfscope}%
\begin{pgfscope}%
\pgfpathrectangle{\pgfqpoint{0.481944in}{1.034041in}}{\pgfqpoint{7.362500in}{2.695000in}}%
\pgfusepath{clip}%
\pgfsetrectcap%
\pgfsetroundjoin%
\pgfsetlinewidth{2.710125pt}%
\definecolor{currentstroke}{rgb}{0.260000,0.260000,0.260000}%
\pgfsetstrokecolor{currentstroke}%
\pgfsetdash{}{0pt}%
\pgfusepath{stroke}%
\end{pgfscope}%
\begin{pgfscope}%
\pgfpathrectangle{\pgfqpoint{0.481944in}{1.034041in}}{\pgfqpoint{7.362500in}{2.695000in}}%
\pgfusepath{clip}%
\pgfsetrectcap%
\pgfsetroundjoin%
\pgfsetlinewidth{2.710125pt}%
\definecolor{currentstroke}{rgb}{0.260000,0.260000,0.260000}%
\pgfsetstrokecolor{currentstroke}%
\pgfsetdash{}{0pt}%
\pgfusepath{stroke}%
\end{pgfscope}%
\begin{pgfscope}%
\pgfpathrectangle{\pgfqpoint{0.481944in}{1.034041in}}{\pgfqpoint{7.362500in}{2.695000in}}%
\pgfusepath{clip}%
\pgfsetrectcap%
\pgfsetroundjoin%
\pgfsetlinewidth{2.710125pt}%
\definecolor{currentstroke}{rgb}{0.260000,0.260000,0.260000}%
\pgfsetstrokecolor{currentstroke}%
\pgfsetdash{}{0pt}%
\pgfusepath{stroke}%
\end{pgfscope}%
\begin{pgfscope}%
\pgfpathrectangle{\pgfqpoint{0.481944in}{1.034041in}}{\pgfqpoint{7.362500in}{2.695000in}}%
\pgfusepath{clip}%
\pgfsetrectcap%
\pgfsetroundjoin%
\pgfsetlinewidth{2.710125pt}%
\definecolor{currentstroke}{rgb}{0.260000,0.260000,0.260000}%
\pgfsetstrokecolor{currentstroke}%
\pgfsetdash{}{0pt}%
\pgfusepath{stroke}%
\end{pgfscope}%
\begin{pgfscope}%
\pgfpathrectangle{\pgfqpoint{0.481944in}{1.034041in}}{\pgfqpoint{7.362500in}{2.695000in}}%
\pgfusepath{clip}%
\pgfsetrectcap%
\pgfsetroundjoin%
\pgfsetlinewidth{2.710125pt}%
\definecolor{currentstroke}{rgb}{0.260000,0.260000,0.260000}%
\pgfsetstrokecolor{currentstroke}%
\pgfsetdash{}{0pt}%
\pgfusepath{stroke}%
\end{pgfscope}%
\begin{pgfscope}%
\pgfpathrectangle{\pgfqpoint{0.481944in}{1.034041in}}{\pgfqpoint{7.362500in}{2.695000in}}%
\pgfusepath{clip}%
\pgfsetrectcap%
\pgfsetroundjoin%
\pgfsetlinewidth{2.710125pt}%
\definecolor{currentstroke}{rgb}{0.260000,0.260000,0.260000}%
\pgfsetstrokecolor{currentstroke}%
\pgfsetdash{}{0pt}%
\pgfusepath{stroke}%
\end{pgfscope}%
\begin{pgfscope}%
\pgfpathrectangle{\pgfqpoint{0.481944in}{1.034041in}}{\pgfqpoint{7.362500in}{2.695000in}}%
\pgfusepath{clip}%
\pgfsetrectcap%
\pgfsetroundjoin%
\pgfsetlinewidth{2.710125pt}%
\definecolor{currentstroke}{rgb}{0.260000,0.260000,0.260000}%
\pgfsetstrokecolor{currentstroke}%
\pgfsetdash{}{0pt}%
\pgfusepath{stroke}%
\end{pgfscope}%
\begin{pgfscope}%
\pgfpathrectangle{\pgfqpoint{0.481944in}{1.034041in}}{\pgfqpoint{7.362500in}{2.695000in}}%
\pgfusepath{clip}%
\pgfsetrectcap%
\pgfsetroundjoin%
\pgfsetlinewidth{2.710125pt}%
\definecolor{currentstroke}{rgb}{0.260000,0.260000,0.260000}%
\pgfsetstrokecolor{currentstroke}%
\pgfsetdash{}{0pt}%
\pgfusepath{stroke}%
\end{pgfscope}%
\begin{pgfscope}%
\pgfpathrectangle{\pgfqpoint{0.481944in}{1.034041in}}{\pgfqpoint{7.362500in}{2.695000in}}%
\pgfusepath{clip}%
\pgfsetrectcap%
\pgfsetroundjoin%
\pgfsetlinewidth{2.710125pt}%
\definecolor{currentstroke}{rgb}{0.260000,0.260000,0.260000}%
\pgfsetstrokecolor{currentstroke}%
\pgfsetdash{}{0pt}%
\pgfusepath{stroke}%
\end{pgfscope}%
\begin{pgfscope}%
\pgfpathrectangle{\pgfqpoint{0.481944in}{1.034041in}}{\pgfqpoint{7.362500in}{2.695000in}}%
\pgfusepath{clip}%
\pgfsetrectcap%
\pgfsetroundjoin%
\pgfsetlinewidth{2.710125pt}%
\definecolor{currentstroke}{rgb}{0.260000,0.260000,0.260000}%
\pgfsetstrokecolor{currentstroke}%
\pgfsetdash{}{0pt}%
\pgfusepath{stroke}%
\end{pgfscope}%
\begin{pgfscope}%
\pgfpathrectangle{\pgfqpoint{0.481944in}{1.034041in}}{\pgfqpoint{7.362500in}{2.695000in}}%
\pgfusepath{clip}%
\pgfsetrectcap%
\pgfsetroundjoin%
\pgfsetlinewidth{2.710125pt}%
\definecolor{currentstroke}{rgb}{0.260000,0.260000,0.260000}%
\pgfsetstrokecolor{currentstroke}%
\pgfsetdash{}{0pt}%
\pgfusepath{stroke}%
\end{pgfscope}%
\begin{pgfscope}%
\pgfpathrectangle{\pgfqpoint{0.481944in}{1.034041in}}{\pgfqpoint{7.362500in}{2.695000in}}%
\pgfusepath{clip}%
\pgfsetrectcap%
\pgfsetroundjoin%
\pgfsetlinewidth{2.710125pt}%
\definecolor{currentstroke}{rgb}{0.260000,0.260000,0.260000}%
\pgfsetstrokecolor{currentstroke}%
\pgfsetdash{}{0pt}%
\pgfusepath{stroke}%
\end{pgfscope}%
\begin{pgfscope}%
\pgfpathrectangle{\pgfqpoint{0.481944in}{1.034041in}}{\pgfqpoint{7.362500in}{2.695000in}}%
\pgfusepath{clip}%
\pgfsetrectcap%
\pgfsetroundjoin%
\pgfsetlinewidth{2.710125pt}%
\definecolor{currentstroke}{rgb}{0.260000,0.260000,0.260000}%
\pgfsetstrokecolor{currentstroke}%
\pgfsetdash{}{0pt}%
\pgfusepath{stroke}%
\end{pgfscope}%
\begin{pgfscope}%
\pgfpathrectangle{\pgfqpoint{0.481944in}{1.034041in}}{\pgfqpoint{7.362500in}{2.695000in}}%
\pgfusepath{clip}%
\pgfsetrectcap%
\pgfsetroundjoin%
\pgfsetlinewidth{2.710125pt}%
\definecolor{currentstroke}{rgb}{0.260000,0.260000,0.260000}%
\pgfsetstrokecolor{currentstroke}%
\pgfsetdash{}{0pt}%
\pgfusepath{stroke}%
\end{pgfscope}%
\begin{pgfscope}%
\pgfpathrectangle{\pgfqpoint{0.481944in}{1.034041in}}{\pgfqpoint{7.362500in}{2.695000in}}%
\pgfusepath{clip}%
\pgfsetrectcap%
\pgfsetroundjoin%
\pgfsetlinewidth{2.710125pt}%
\definecolor{currentstroke}{rgb}{0.260000,0.260000,0.260000}%
\pgfsetstrokecolor{currentstroke}%
\pgfsetdash{}{0pt}%
\pgfusepath{stroke}%
\end{pgfscope}%
\begin{pgfscope}%
\pgfpathrectangle{\pgfqpoint{0.481944in}{1.034041in}}{\pgfqpoint{7.362500in}{2.695000in}}%
\pgfusepath{clip}%
\pgfsetrectcap%
\pgfsetroundjoin%
\pgfsetlinewidth{2.710125pt}%
\definecolor{currentstroke}{rgb}{0.260000,0.260000,0.260000}%
\pgfsetstrokecolor{currentstroke}%
\pgfsetdash{}{0pt}%
\pgfusepath{stroke}%
\end{pgfscope}%
\begin{pgfscope}%
\pgfpathrectangle{\pgfqpoint{0.481944in}{1.034041in}}{\pgfqpoint{7.362500in}{2.695000in}}%
\pgfusepath{clip}%
\pgfsetrectcap%
\pgfsetroundjoin%
\pgfsetlinewidth{2.710125pt}%
\definecolor{currentstroke}{rgb}{0.260000,0.260000,0.260000}%
\pgfsetstrokecolor{currentstroke}%
\pgfsetdash{}{0pt}%
\pgfusepath{stroke}%
\end{pgfscope}%
\begin{pgfscope}%
\pgfpathrectangle{\pgfqpoint{0.481944in}{1.034041in}}{\pgfqpoint{7.362500in}{2.695000in}}%
\pgfusepath{clip}%
\pgfsetrectcap%
\pgfsetroundjoin%
\pgfsetlinewidth{2.710125pt}%
\definecolor{currentstroke}{rgb}{0.260000,0.260000,0.260000}%
\pgfsetstrokecolor{currentstroke}%
\pgfsetdash{}{0pt}%
\pgfusepath{stroke}%
\end{pgfscope}%
\begin{pgfscope}%
\pgfpathrectangle{\pgfqpoint{0.481944in}{1.034041in}}{\pgfqpoint{7.362500in}{2.695000in}}%
\pgfusepath{clip}%
\pgfsetrectcap%
\pgfsetroundjoin%
\pgfsetlinewidth{2.710125pt}%
\definecolor{currentstroke}{rgb}{0.260000,0.260000,0.260000}%
\pgfsetstrokecolor{currentstroke}%
\pgfsetdash{}{0pt}%
\pgfusepath{stroke}%
\end{pgfscope}%
\begin{pgfscope}%
\pgfpathrectangle{\pgfqpoint{0.481944in}{1.034041in}}{\pgfqpoint{7.362500in}{2.695000in}}%
\pgfusepath{clip}%
\pgfsetrectcap%
\pgfsetroundjoin%
\pgfsetlinewidth{2.710125pt}%
\definecolor{currentstroke}{rgb}{0.260000,0.260000,0.260000}%
\pgfsetstrokecolor{currentstroke}%
\pgfsetdash{}{0pt}%
\pgfusepath{stroke}%
\end{pgfscope}%
\begin{pgfscope}%
\pgfpathrectangle{\pgfqpoint{0.481944in}{1.034041in}}{\pgfqpoint{7.362500in}{2.695000in}}%
\pgfusepath{clip}%
\pgfsetrectcap%
\pgfsetroundjoin%
\pgfsetlinewidth{2.710125pt}%
\definecolor{currentstroke}{rgb}{0.260000,0.260000,0.260000}%
\pgfsetstrokecolor{currentstroke}%
\pgfsetdash{}{0pt}%
\pgfusepath{stroke}%
\end{pgfscope}%
\begin{pgfscope}%
\pgfpathrectangle{\pgfqpoint{0.481944in}{1.034041in}}{\pgfqpoint{7.362500in}{2.695000in}}%
\pgfusepath{clip}%
\pgfsetrectcap%
\pgfsetroundjoin%
\pgfsetlinewidth{2.710125pt}%
\definecolor{currentstroke}{rgb}{0.260000,0.260000,0.260000}%
\pgfsetstrokecolor{currentstroke}%
\pgfsetdash{}{0pt}%
\pgfusepath{stroke}%
\end{pgfscope}%
\begin{pgfscope}%
\pgfpathrectangle{\pgfqpoint{0.481944in}{1.034041in}}{\pgfqpoint{7.362500in}{2.695000in}}%
\pgfusepath{clip}%
\pgfsetrectcap%
\pgfsetroundjoin%
\pgfsetlinewidth{2.710125pt}%
\definecolor{currentstroke}{rgb}{0.260000,0.260000,0.260000}%
\pgfsetstrokecolor{currentstroke}%
\pgfsetdash{}{0pt}%
\pgfusepath{stroke}%
\end{pgfscope}%
\begin{pgfscope}%
\pgfpathrectangle{\pgfqpoint{0.481944in}{1.034041in}}{\pgfqpoint{7.362500in}{2.695000in}}%
\pgfusepath{clip}%
\pgfsetrectcap%
\pgfsetroundjoin%
\pgfsetlinewidth{2.710125pt}%
\definecolor{currentstroke}{rgb}{0.260000,0.260000,0.260000}%
\pgfsetstrokecolor{currentstroke}%
\pgfsetdash{}{0pt}%
\pgfusepath{stroke}%
\end{pgfscope}%
\begin{pgfscope}%
\pgfpathrectangle{\pgfqpoint{0.481944in}{1.034041in}}{\pgfqpoint{7.362500in}{2.695000in}}%
\pgfusepath{clip}%
\pgfsetrectcap%
\pgfsetroundjoin%
\pgfsetlinewidth{2.710125pt}%
\definecolor{currentstroke}{rgb}{0.260000,0.260000,0.260000}%
\pgfsetstrokecolor{currentstroke}%
\pgfsetdash{}{0pt}%
\pgfusepath{stroke}%
\end{pgfscope}%
\begin{pgfscope}%
\pgfsetrectcap%
\pgfsetmiterjoin%
\pgfsetlinewidth{1.003750pt}%
\definecolor{currentstroke}{rgb}{0.000000,0.000000,0.000000}%
\pgfsetstrokecolor{currentstroke}%
\pgfsetdash{}{0pt}%
\pgfpathmoveto{\pgfqpoint{0.481944in}{1.034041in}}%
\pgfpathlineto{\pgfqpoint{0.481944in}{3.729041in}}%
\pgfusepath{stroke}%
\end{pgfscope}%
\begin{pgfscope}%
\pgfsetrectcap%
\pgfsetmiterjoin%
\pgfsetlinewidth{1.003750pt}%
\definecolor{currentstroke}{rgb}{0.000000,0.000000,0.000000}%
\pgfsetstrokecolor{currentstroke}%
\pgfsetdash{}{0pt}%
\pgfpathmoveto{\pgfqpoint{7.844444in}{1.034041in}}%
\pgfpathlineto{\pgfqpoint{7.844444in}{3.729041in}}%
\pgfusepath{stroke}%
\end{pgfscope}%
\begin{pgfscope}%
\pgfsetrectcap%
\pgfsetmiterjoin%
\pgfsetlinewidth{1.003750pt}%
\definecolor{currentstroke}{rgb}{0.000000,0.000000,0.000000}%
\pgfsetstrokecolor{currentstroke}%
\pgfsetdash{}{0pt}%
\pgfpathmoveto{\pgfqpoint{0.481944in}{1.034041in}}%
\pgfpathlineto{\pgfqpoint{7.844444in}{1.034041in}}%
\pgfusepath{stroke}%
\end{pgfscope}%
\begin{pgfscope}%
\pgfsetrectcap%
\pgfsetmiterjoin%
\pgfsetlinewidth{1.003750pt}%
\definecolor{currentstroke}{rgb}{0.000000,0.000000,0.000000}%
\pgfsetstrokecolor{currentstroke}%
\pgfsetdash{}{0pt}%
\pgfpathmoveto{\pgfqpoint{0.481944in}{3.729041in}}%
\pgfpathlineto{\pgfqpoint{7.844444in}{3.729041in}}%
\pgfusepath{stroke}%
\end{pgfscope}%
\end{pgfpicture}%
\makeatother%
\endgroup%

    \caption{Mean note per assureur (colored by rank)}
    \label{fig:mean_note_per_assureur_linear}
\end{figure}
\restoregeometry

And so naturally we looked which assureurs were the most represented in the train dataset (\cref{fig:nbnote_per_assureur}).

\begin{figure}[H]
    \advance\leftskip-2.5cm
    %% Creator: Matplotlib, PGF backend
%%
%% To include the figure in your LaTeX document, write
%%   \input{<filename>.pgf}
%%
%% Make sure the required packages are loaded in your preamble
%%   \usepackage{pgf}
%%
%% Also ensure that all the required font packages are loaded; for instance,
%% the lmodern package is sometimes necessary when using math font.
%%   \usepackage{lmodern}
%%
%% Figures using additional raster images can only be included by \input if
%% they are in the same directory as the main LaTeX file. For loading figures
%% from other directories you can use the `import` package
%%   \usepackage{import}
%%
%% and then include the figures with
%%   \import{<path to file>}{<filename>.pgf}
%%
%% Matplotlib used the following preamble
%%
\begingroup%
\makeatletter%
\begin{pgfpicture}%
\pgfpathrectangle{\pgfpointorigin}{\pgfqpoint{7.962191in}{5.053099in}}%
\pgfusepath{use as bounding box, clip}%
\begin{pgfscope}%
\pgfsetbuttcap%
\pgfsetmiterjoin%
\definecolor{currentfill}{rgb}{1.000000,1.000000,1.000000}%
\pgfsetfillcolor{currentfill}%
\pgfsetlinewidth{0.000000pt}%
\definecolor{currentstroke}{rgb}{1.000000,1.000000,1.000000}%
\pgfsetstrokecolor{currentstroke}%
\pgfsetdash{}{0pt}%
\pgfpathmoveto{\pgfqpoint{0.000000in}{0.000000in}}%
\pgfpathlineto{\pgfqpoint{7.962191in}{0.000000in}}%
\pgfpathlineto{\pgfqpoint{7.962191in}{5.053099in}}%
\pgfpathlineto{\pgfqpoint{0.000000in}{5.053099in}}%
\pgfpathlineto{\pgfqpoint{0.000000in}{0.000000in}}%
\pgfpathclose%
\pgfusepath{fill}%
\end{pgfscope}%
\begin{pgfscope}%
\pgfsetbuttcap%
\pgfsetmiterjoin%
\definecolor{currentfill}{rgb}{1.000000,1.000000,1.000000}%
\pgfsetfillcolor{currentfill}%
\pgfsetlinewidth{0.000000pt}%
\definecolor{currentstroke}{rgb}{0.000000,0.000000,0.000000}%
\pgfsetstrokecolor{currentstroke}%
\pgfsetstrokeopacity{0.000000}%
\pgfsetdash{}{0pt}%
\pgfpathmoveto{\pgfqpoint{0.499691in}{1.103099in}}%
\pgfpathlineto{\pgfqpoint{7.862191in}{1.103099in}}%
\pgfpathlineto{\pgfqpoint{7.862191in}{4.953099in}}%
\pgfpathlineto{\pgfqpoint{0.499691in}{4.953099in}}%
\pgfpathlineto{\pgfqpoint{0.499691in}{1.103099in}}%
\pgfpathclose%
\pgfusepath{fill}%
\end{pgfscope}%
\begin{pgfscope}%
\pgfpathrectangle{\pgfqpoint{0.499691in}{1.103099in}}{\pgfqpoint{7.362500in}{3.850000in}}%
\pgfusepath{clip}%
\pgfsetbuttcap%
\pgfsetmiterjoin%
\definecolor{currentfill}{rgb}{0.914216,0.537745,0.399510}%
\pgfsetfillcolor{currentfill}%
\pgfsetlinewidth{0.000000pt}%
\definecolor{currentstroke}{rgb}{0.000000,0.000000,0.000000}%
\pgfsetstrokecolor{currentstroke}%
\pgfsetstrokeopacity{0.000000}%
\pgfsetdash{}{0pt}%
\pgfpathmoveto{\pgfqpoint{0.512838in}{1.103099in}}%
\pgfpathlineto{\pgfqpoint{0.618017in}{1.103099in}}%
\pgfpathlineto{\pgfqpoint{0.618017in}{1.530959in}}%
\pgfpathlineto{\pgfqpoint{0.512838in}{1.530959in}}%
\pgfpathlineto{\pgfqpoint{0.512838in}{1.103099in}}%
\pgfpathclose%
\pgfusepath{fill}%
\end{pgfscope}%
\begin{pgfscope}%
\pgfpathrectangle{\pgfqpoint{0.499691in}{1.103099in}}{\pgfqpoint{7.362500in}{3.850000in}}%
\pgfusepath{clip}%
\pgfsetbuttcap%
\pgfsetmiterjoin%
\definecolor{currentfill}{rgb}{0.914216,0.537745,0.399510}%
\pgfsetfillcolor{currentfill}%
\pgfsetlinewidth{0.000000pt}%
\definecolor{currentstroke}{rgb}{0.000000,0.000000,0.000000}%
\pgfsetstrokecolor{currentstroke}%
\pgfsetstrokeopacity{0.000000}%
\pgfsetdash{}{0pt}%
\pgfpathmoveto{\pgfqpoint{0.644312in}{1.103099in}}%
\pgfpathlineto{\pgfqpoint{0.749490in}{1.103099in}}%
\pgfpathlineto{\pgfqpoint{0.749490in}{1.279094in}}%
\pgfpathlineto{\pgfqpoint{0.644312in}{1.279094in}}%
\pgfpathlineto{\pgfqpoint{0.644312in}{1.103099in}}%
\pgfpathclose%
\pgfusepath{fill}%
\end{pgfscope}%
\begin{pgfscope}%
\pgfpathrectangle{\pgfqpoint{0.499691in}{1.103099in}}{\pgfqpoint{7.362500in}{3.850000in}}%
\pgfusepath{clip}%
\pgfsetbuttcap%
\pgfsetmiterjoin%
\definecolor{currentfill}{rgb}{0.914216,0.537745,0.399510}%
\pgfsetfillcolor{currentfill}%
\pgfsetlinewidth{0.000000pt}%
\definecolor{currentstroke}{rgb}{0.000000,0.000000,0.000000}%
\pgfsetstrokecolor{currentstroke}%
\pgfsetstrokeopacity{0.000000}%
\pgfsetdash{}{0pt}%
\pgfpathmoveto{\pgfqpoint{0.775785in}{1.103099in}}%
\pgfpathlineto{\pgfqpoint{0.880963in}{1.103099in}}%
\pgfpathlineto{\pgfqpoint{0.880963in}{1.739293in}}%
\pgfpathlineto{\pgfqpoint{0.775785in}{1.739293in}}%
\pgfpathlineto{\pgfqpoint{0.775785in}{1.103099in}}%
\pgfpathclose%
\pgfusepath{fill}%
\end{pgfscope}%
\begin{pgfscope}%
\pgfpathrectangle{\pgfqpoint{0.499691in}{1.103099in}}{\pgfqpoint{7.362500in}{3.850000in}}%
\pgfusepath{clip}%
\pgfsetbuttcap%
\pgfsetmiterjoin%
\definecolor{currentfill}{rgb}{0.914216,0.537745,0.399510}%
\pgfsetfillcolor{currentfill}%
\pgfsetlinewidth{0.000000pt}%
\definecolor{currentstroke}{rgb}{0.000000,0.000000,0.000000}%
\pgfsetstrokecolor{currentstroke}%
\pgfsetstrokeopacity{0.000000}%
\pgfsetdash{}{0pt}%
\pgfpathmoveto{\pgfqpoint{0.907258in}{1.103099in}}%
\pgfpathlineto{\pgfqpoint{1.012437in}{1.103099in}}%
\pgfpathlineto{\pgfqpoint{1.012437in}{1.494268in}}%
\pgfpathlineto{\pgfqpoint{0.907258in}{1.494268in}}%
\pgfpathlineto{\pgfqpoint{0.907258in}{1.103099in}}%
\pgfpathclose%
\pgfusepath{fill}%
\end{pgfscope}%
\begin{pgfscope}%
\pgfpathrectangle{\pgfqpoint{0.499691in}{1.103099in}}{\pgfqpoint{7.362500in}{3.850000in}}%
\pgfusepath{clip}%
\pgfsetbuttcap%
\pgfsetmiterjoin%
\definecolor{currentfill}{rgb}{0.914216,0.537745,0.399510}%
\pgfsetfillcolor{currentfill}%
\pgfsetlinewidth{0.000000pt}%
\definecolor{currentstroke}{rgb}{0.000000,0.000000,0.000000}%
\pgfsetstrokecolor{currentstroke}%
\pgfsetstrokeopacity{0.000000}%
\pgfsetdash{}{0pt}%
\pgfpathmoveto{\pgfqpoint{1.038731in}{1.103099in}}%
\pgfpathlineto{\pgfqpoint{1.143910in}{1.103099in}}%
\pgfpathlineto{\pgfqpoint{1.143910in}{1.353721in}}%
\pgfpathlineto{\pgfqpoint{1.038731in}{1.353721in}}%
\pgfpathlineto{\pgfqpoint{1.038731in}{1.103099in}}%
\pgfpathclose%
\pgfusepath{fill}%
\end{pgfscope}%
\begin{pgfscope}%
\pgfpathrectangle{\pgfqpoint{0.499691in}{1.103099in}}{\pgfqpoint{7.362500in}{3.850000in}}%
\pgfusepath{clip}%
\pgfsetbuttcap%
\pgfsetmiterjoin%
\definecolor{currentfill}{rgb}{0.914216,0.537745,0.399510}%
\pgfsetfillcolor{currentfill}%
\pgfsetlinewidth{0.000000pt}%
\definecolor{currentstroke}{rgb}{0.000000,0.000000,0.000000}%
\pgfsetstrokecolor{currentstroke}%
\pgfsetstrokeopacity{0.000000}%
\pgfsetdash{}{0pt}%
\pgfpathmoveto{\pgfqpoint{1.170205in}{1.103099in}}%
\pgfpathlineto{\pgfqpoint{1.275383in}{1.103099in}}%
\pgfpathlineto{\pgfqpoint{1.275383in}{1.192651in}}%
\pgfpathlineto{\pgfqpoint{1.170205in}{1.192651in}}%
\pgfpathlineto{\pgfqpoint{1.170205in}{1.103099in}}%
\pgfpathclose%
\pgfusepath{fill}%
\end{pgfscope}%
\begin{pgfscope}%
\pgfpathrectangle{\pgfqpoint{0.499691in}{1.103099in}}{\pgfqpoint{7.362500in}{3.850000in}}%
\pgfusepath{clip}%
\pgfsetbuttcap%
\pgfsetmiterjoin%
\definecolor{currentfill}{rgb}{0.914216,0.537745,0.399510}%
\pgfsetfillcolor{currentfill}%
\pgfsetlinewidth{0.000000pt}%
\definecolor{currentstroke}{rgb}{0.000000,0.000000,0.000000}%
\pgfsetstrokecolor{currentstroke}%
\pgfsetstrokeopacity{0.000000}%
\pgfsetdash{}{0pt}%
\pgfpathmoveto{\pgfqpoint{1.301678in}{1.103099in}}%
\pgfpathlineto{\pgfqpoint{1.406856in}{1.103099in}}%
\pgfpathlineto{\pgfqpoint{1.406856in}{1.119268in}}%
\pgfpathlineto{\pgfqpoint{1.301678in}{1.119268in}}%
\pgfpathlineto{\pgfqpoint{1.301678in}{1.103099in}}%
\pgfpathclose%
\pgfusepath{fill}%
\end{pgfscope}%
\begin{pgfscope}%
\pgfpathrectangle{\pgfqpoint{0.499691in}{1.103099in}}{\pgfqpoint{7.362500in}{3.850000in}}%
\pgfusepath{clip}%
\pgfsetbuttcap%
\pgfsetmiterjoin%
\definecolor{currentfill}{rgb}{0.914216,0.537745,0.399510}%
\pgfsetfillcolor{currentfill}%
\pgfsetlinewidth{0.000000pt}%
\definecolor{currentstroke}{rgb}{0.000000,0.000000,0.000000}%
\pgfsetstrokecolor{currentstroke}%
\pgfsetstrokeopacity{0.000000}%
\pgfsetdash{}{0pt}%
\pgfpathmoveto{\pgfqpoint{1.433151in}{1.103099in}}%
\pgfpathlineto{\pgfqpoint{1.538330in}{1.103099in}}%
\pgfpathlineto{\pgfqpoint{1.538330in}{1.320760in}}%
\pgfpathlineto{\pgfqpoint{1.433151in}{1.320760in}}%
\pgfpathlineto{\pgfqpoint{1.433151in}{1.103099in}}%
\pgfpathclose%
\pgfusepath{fill}%
\end{pgfscope}%
\begin{pgfscope}%
\pgfpathrectangle{\pgfqpoint{0.499691in}{1.103099in}}{\pgfqpoint{7.362500in}{3.850000in}}%
\pgfusepath{clip}%
\pgfsetbuttcap%
\pgfsetmiterjoin%
\definecolor{currentfill}{rgb}{0.914216,0.537745,0.399510}%
\pgfsetfillcolor{currentfill}%
\pgfsetlinewidth{0.000000pt}%
\definecolor{currentstroke}{rgb}{0.000000,0.000000,0.000000}%
\pgfsetstrokecolor{currentstroke}%
\pgfsetstrokeopacity{0.000000}%
\pgfsetdash{}{0pt}%
\pgfpathmoveto{\pgfqpoint{1.564624in}{1.103099in}}%
\pgfpathlineto{\pgfqpoint{1.669803in}{1.103099in}}%
\pgfpathlineto{\pgfqpoint{1.669803in}{1.460064in}}%
\pgfpathlineto{\pgfqpoint{1.564624in}{1.460064in}}%
\pgfpathlineto{\pgfqpoint{1.564624in}{1.103099in}}%
\pgfpathclose%
\pgfusepath{fill}%
\end{pgfscope}%
\begin{pgfscope}%
\pgfpathrectangle{\pgfqpoint{0.499691in}{1.103099in}}{\pgfqpoint{7.362500in}{3.850000in}}%
\pgfusepath{clip}%
\pgfsetbuttcap%
\pgfsetmiterjoin%
\definecolor{currentfill}{rgb}{0.914216,0.537745,0.399510}%
\pgfsetfillcolor{currentfill}%
\pgfsetlinewidth{0.000000pt}%
\definecolor{currentstroke}{rgb}{0.000000,0.000000,0.000000}%
\pgfsetstrokecolor{currentstroke}%
\pgfsetstrokeopacity{0.000000}%
\pgfsetdash{}{0pt}%
\pgfpathmoveto{\pgfqpoint{1.696097in}{1.103099in}}%
\pgfpathlineto{\pgfqpoint{1.801276in}{1.103099in}}%
\pgfpathlineto{\pgfqpoint{1.801276in}{1.134815in}}%
\pgfpathlineto{\pgfqpoint{1.696097in}{1.134815in}}%
\pgfpathlineto{\pgfqpoint{1.696097in}{1.103099in}}%
\pgfpathclose%
\pgfusepath{fill}%
\end{pgfscope}%
\begin{pgfscope}%
\pgfpathrectangle{\pgfqpoint{0.499691in}{1.103099in}}{\pgfqpoint{7.362500in}{3.850000in}}%
\pgfusepath{clip}%
\pgfsetbuttcap%
\pgfsetmiterjoin%
\definecolor{currentfill}{rgb}{0.914216,0.537745,0.399510}%
\pgfsetfillcolor{currentfill}%
\pgfsetlinewidth{0.000000pt}%
\definecolor{currentstroke}{rgb}{0.000000,0.000000,0.000000}%
\pgfsetstrokecolor{currentstroke}%
\pgfsetstrokeopacity{0.000000}%
\pgfsetdash{}{0pt}%
\pgfpathmoveto{\pgfqpoint{1.827571in}{1.103099in}}%
\pgfpathlineto{\pgfqpoint{1.932749in}{1.103099in}}%
\pgfpathlineto{\pgfqpoint{1.932749in}{1.171507in}}%
\pgfpathlineto{\pgfqpoint{1.827571in}{1.171507in}}%
\pgfpathlineto{\pgfqpoint{1.827571in}{1.103099in}}%
\pgfpathclose%
\pgfusepath{fill}%
\end{pgfscope}%
\begin{pgfscope}%
\pgfpathrectangle{\pgfqpoint{0.499691in}{1.103099in}}{\pgfqpoint{7.362500in}{3.850000in}}%
\pgfusepath{clip}%
\pgfsetbuttcap%
\pgfsetmiterjoin%
\definecolor{currentfill}{rgb}{0.914216,0.537745,0.399510}%
\pgfsetfillcolor{currentfill}%
\pgfsetlinewidth{0.000000pt}%
\definecolor{currentstroke}{rgb}{0.000000,0.000000,0.000000}%
\pgfsetstrokecolor{currentstroke}%
\pgfsetstrokeopacity{0.000000}%
\pgfsetdash{}{0pt}%
\pgfpathmoveto{\pgfqpoint{1.959044in}{1.103099in}}%
\pgfpathlineto{\pgfqpoint{2.064222in}{1.103099in}}%
\pgfpathlineto{\pgfqpoint{2.064222in}{1.118646in}}%
\pgfpathlineto{\pgfqpoint{1.959044in}{1.118646in}}%
\pgfpathlineto{\pgfqpoint{1.959044in}{1.103099in}}%
\pgfpathclose%
\pgfusepath{fill}%
\end{pgfscope}%
\begin{pgfscope}%
\pgfpathrectangle{\pgfqpoint{0.499691in}{1.103099in}}{\pgfqpoint{7.362500in}{3.850000in}}%
\pgfusepath{clip}%
\pgfsetbuttcap%
\pgfsetmiterjoin%
\definecolor{currentfill}{rgb}{0.914216,0.537745,0.399510}%
\pgfsetfillcolor{currentfill}%
\pgfsetlinewidth{0.000000pt}%
\definecolor{currentstroke}{rgb}{0.000000,0.000000,0.000000}%
\pgfsetstrokecolor{currentstroke}%
\pgfsetstrokeopacity{0.000000}%
\pgfsetdash{}{0pt}%
\pgfpathmoveto{\pgfqpoint{2.090517in}{1.103099in}}%
\pgfpathlineto{\pgfqpoint{2.195696in}{1.103099in}}%
\pgfpathlineto{\pgfqpoint{2.195696in}{1.194517in}}%
\pgfpathlineto{\pgfqpoint{2.090517in}{1.194517in}}%
\pgfpathlineto{\pgfqpoint{2.090517in}{1.103099in}}%
\pgfpathclose%
\pgfusepath{fill}%
\end{pgfscope}%
\begin{pgfscope}%
\pgfpathrectangle{\pgfqpoint{0.499691in}{1.103099in}}{\pgfqpoint{7.362500in}{3.850000in}}%
\pgfusepath{clip}%
\pgfsetbuttcap%
\pgfsetmiterjoin%
\definecolor{currentfill}{rgb}{0.914216,0.537745,0.399510}%
\pgfsetfillcolor{currentfill}%
\pgfsetlinewidth{0.000000pt}%
\definecolor{currentstroke}{rgb}{0.000000,0.000000,0.000000}%
\pgfsetstrokecolor{currentstroke}%
\pgfsetstrokeopacity{0.000000}%
\pgfsetdash{}{0pt}%
\pgfpathmoveto{\pgfqpoint{2.221990in}{1.103099in}}%
\pgfpathlineto{\pgfqpoint{2.327169in}{1.103099in}}%
\pgfpathlineto{\pgfqpoint{2.327169in}{1.122999in}}%
\pgfpathlineto{\pgfqpoint{2.221990in}{1.122999in}}%
\pgfpathlineto{\pgfqpoint{2.221990in}{1.103099in}}%
\pgfpathclose%
\pgfusepath{fill}%
\end{pgfscope}%
\begin{pgfscope}%
\pgfpathrectangle{\pgfqpoint{0.499691in}{1.103099in}}{\pgfqpoint{7.362500in}{3.850000in}}%
\pgfusepath{clip}%
\pgfsetbuttcap%
\pgfsetmiterjoin%
\definecolor{currentfill}{rgb}{0.914216,0.537745,0.399510}%
\pgfsetfillcolor{currentfill}%
\pgfsetlinewidth{0.000000pt}%
\definecolor{currentstroke}{rgb}{0.000000,0.000000,0.000000}%
\pgfsetstrokecolor{currentstroke}%
\pgfsetstrokeopacity{0.000000}%
\pgfsetdash{}{0pt}%
\pgfpathmoveto{\pgfqpoint{2.353463in}{1.103099in}}%
\pgfpathlineto{\pgfqpoint{2.458642in}{1.103099in}}%
\pgfpathlineto{\pgfqpoint{2.458642in}{1.267900in}}%
\pgfpathlineto{\pgfqpoint{2.353463in}{1.267900in}}%
\pgfpathlineto{\pgfqpoint{2.353463in}{1.103099in}}%
\pgfpathclose%
\pgfusepath{fill}%
\end{pgfscope}%
\begin{pgfscope}%
\pgfpathrectangle{\pgfqpoint{0.499691in}{1.103099in}}{\pgfqpoint{7.362500in}{3.850000in}}%
\pgfusepath{clip}%
\pgfsetbuttcap%
\pgfsetmiterjoin%
\definecolor{currentfill}{rgb}{0.914216,0.537745,0.399510}%
\pgfsetfillcolor{currentfill}%
\pgfsetlinewidth{0.000000pt}%
\definecolor{currentstroke}{rgb}{0.000000,0.000000,0.000000}%
\pgfsetstrokecolor{currentstroke}%
\pgfsetstrokeopacity{0.000000}%
\pgfsetdash{}{0pt}%
\pgfpathmoveto{\pgfqpoint{2.484937in}{1.103099in}}%
\pgfpathlineto{\pgfqpoint{2.590115in}{1.103099in}}%
\pgfpathlineto{\pgfqpoint{2.590115in}{1.236183in}}%
\pgfpathlineto{\pgfqpoint{2.484937in}{1.236183in}}%
\pgfpathlineto{\pgfqpoint{2.484937in}{1.103099in}}%
\pgfpathclose%
\pgfusepath{fill}%
\end{pgfscope}%
\begin{pgfscope}%
\pgfpathrectangle{\pgfqpoint{0.499691in}{1.103099in}}{\pgfqpoint{7.362500in}{3.850000in}}%
\pgfusepath{clip}%
\pgfsetbuttcap%
\pgfsetmiterjoin%
\definecolor{currentfill}{rgb}{0.914216,0.537745,0.399510}%
\pgfsetfillcolor{currentfill}%
\pgfsetlinewidth{0.000000pt}%
\definecolor{currentstroke}{rgb}{0.000000,0.000000,0.000000}%
\pgfsetstrokecolor{currentstroke}%
\pgfsetstrokeopacity{0.000000}%
\pgfsetdash{}{0pt}%
\pgfpathmoveto{\pgfqpoint{2.616410in}{1.103099in}}%
\pgfpathlineto{\pgfqpoint{2.721588in}{1.103099in}}%
\pgfpathlineto{\pgfqpoint{2.721588in}{1.193895in}}%
\pgfpathlineto{\pgfqpoint{2.616410in}{1.193895in}}%
\pgfpathlineto{\pgfqpoint{2.616410in}{1.103099in}}%
\pgfpathclose%
\pgfusepath{fill}%
\end{pgfscope}%
\begin{pgfscope}%
\pgfpathrectangle{\pgfqpoint{0.499691in}{1.103099in}}{\pgfqpoint{7.362500in}{3.850000in}}%
\pgfusepath{clip}%
\pgfsetbuttcap%
\pgfsetmiterjoin%
\definecolor{currentfill}{rgb}{0.914216,0.537745,0.399510}%
\pgfsetfillcolor{currentfill}%
\pgfsetlinewidth{0.000000pt}%
\definecolor{currentstroke}{rgb}{0.000000,0.000000,0.000000}%
\pgfsetstrokecolor{currentstroke}%
\pgfsetstrokeopacity{0.000000}%
\pgfsetdash{}{0pt}%
\pgfpathmoveto{\pgfqpoint{2.747883in}{1.103099in}}%
\pgfpathlineto{\pgfqpoint{2.853062in}{1.103099in}}%
\pgfpathlineto{\pgfqpoint{2.853062in}{4.769765in}}%
\pgfpathlineto{\pgfqpoint{2.747883in}{4.769765in}}%
\pgfpathlineto{\pgfqpoint{2.747883in}{1.103099in}}%
\pgfpathclose%
\pgfusepath{fill}%
\end{pgfscope}%
\begin{pgfscope}%
\pgfpathrectangle{\pgfqpoint{0.499691in}{1.103099in}}{\pgfqpoint{7.362500in}{3.850000in}}%
\pgfusepath{clip}%
\pgfsetbuttcap%
\pgfsetmiterjoin%
\definecolor{currentfill}{rgb}{0.914216,0.537745,0.399510}%
\pgfsetfillcolor{currentfill}%
\pgfsetlinewidth{0.000000pt}%
\definecolor{currentstroke}{rgb}{0.000000,0.000000,0.000000}%
\pgfsetstrokecolor{currentstroke}%
\pgfsetstrokeopacity{0.000000}%
\pgfsetdash{}{0pt}%
\pgfpathmoveto{\pgfqpoint{2.879356in}{1.103099in}}%
\pgfpathlineto{\pgfqpoint{2.984535in}{1.103099in}}%
\pgfpathlineto{\pgfqpoint{2.984535in}{1.185188in}}%
\pgfpathlineto{\pgfqpoint{2.879356in}{1.185188in}}%
\pgfpathlineto{\pgfqpoint{2.879356in}{1.103099in}}%
\pgfpathclose%
\pgfusepath{fill}%
\end{pgfscope}%
\begin{pgfscope}%
\pgfpathrectangle{\pgfqpoint{0.499691in}{1.103099in}}{\pgfqpoint{7.362500in}{3.850000in}}%
\pgfusepath{clip}%
\pgfsetbuttcap%
\pgfsetmiterjoin%
\definecolor{currentfill}{rgb}{0.914216,0.537745,0.399510}%
\pgfsetfillcolor{currentfill}%
\pgfsetlinewidth{0.000000pt}%
\definecolor{currentstroke}{rgb}{0.000000,0.000000,0.000000}%
\pgfsetstrokecolor{currentstroke}%
\pgfsetstrokeopacity{0.000000}%
\pgfsetdash{}{0pt}%
\pgfpathmoveto{\pgfqpoint{3.010830in}{1.103099in}}%
\pgfpathlineto{\pgfqpoint{3.116008in}{1.103099in}}%
\pgfpathlineto{\pgfqpoint{3.116008in}{1.129840in}}%
\pgfpathlineto{\pgfqpoint{3.010830in}{1.129840in}}%
\pgfpathlineto{\pgfqpoint{3.010830in}{1.103099in}}%
\pgfpathclose%
\pgfusepath{fill}%
\end{pgfscope}%
\begin{pgfscope}%
\pgfpathrectangle{\pgfqpoint{0.499691in}{1.103099in}}{\pgfqpoint{7.362500in}{3.850000in}}%
\pgfusepath{clip}%
\pgfsetbuttcap%
\pgfsetmiterjoin%
\definecolor{currentfill}{rgb}{0.914216,0.537745,0.399510}%
\pgfsetfillcolor{currentfill}%
\pgfsetlinewidth{0.000000pt}%
\definecolor{currentstroke}{rgb}{0.000000,0.000000,0.000000}%
\pgfsetstrokecolor{currentstroke}%
\pgfsetstrokeopacity{0.000000}%
\pgfsetdash{}{0pt}%
\pgfpathmoveto{\pgfqpoint{3.142303in}{1.103099in}}%
\pgfpathlineto{\pgfqpoint{3.247481in}{1.103099in}}%
\pgfpathlineto{\pgfqpoint{3.247481in}{1.282825in}}%
\pgfpathlineto{\pgfqpoint{3.142303in}{1.282825in}}%
\pgfpathlineto{\pgfqpoint{3.142303in}{1.103099in}}%
\pgfpathclose%
\pgfusepath{fill}%
\end{pgfscope}%
\begin{pgfscope}%
\pgfpathrectangle{\pgfqpoint{0.499691in}{1.103099in}}{\pgfqpoint{7.362500in}{3.850000in}}%
\pgfusepath{clip}%
\pgfsetbuttcap%
\pgfsetmiterjoin%
\definecolor{currentfill}{rgb}{0.914216,0.537745,0.399510}%
\pgfsetfillcolor{currentfill}%
\pgfsetlinewidth{0.000000pt}%
\definecolor{currentstroke}{rgb}{0.000000,0.000000,0.000000}%
\pgfsetstrokecolor{currentstroke}%
\pgfsetstrokeopacity{0.000000}%
\pgfsetdash{}{0pt}%
\pgfpathmoveto{\pgfqpoint{3.273776in}{1.103099in}}%
\pgfpathlineto{\pgfqpoint{3.378955in}{1.103099in}}%
\pgfpathlineto{\pgfqpoint{3.378955in}{1.723746in}}%
\pgfpathlineto{\pgfqpoint{3.273776in}{1.723746in}}%
\pgfpathlineto{\pgfqpoint{3.273776in}{1.103099in}}%
\pgfpathclose%
\pgfusepath{fill}%
\end{pgfscope}%
\begin{pgfscope}%
\pgfpathrectangle{\pgfqpoint{0.499691in}{1.103099in}}{\pgfqpoint{7.362500in}{3.850000in}}%
\pgfusepath{clip}%
\pgfsetbuttcap%
\pgfsetmiterjoin%
\definecolor{currentfill}{rgb}{0.914216,0.537745,0.399510}%
\pgfsetfillcolor{currentfill}%
\pgfsetlinewidth{0.000000pt}%
\definecolor{currentstroke}{rgb}{0.000000,0.000000,0.000000}%
\pgfsetstrokecolor{currentstroke}%
\pgfsetstrokeopacity{0.000000}%
\pgfsetdash{}{0pt}%
\pgfpathmoveto{\pgfqpoint{3.405249in}{1.103099in}}%
\pgfpathlineto{\pgfqpoint{3.510428in}{1.103099in}}%
\pgfpathlineto{\pgfqpoint{3.510428in}{1.123621in}}%
\pgfpathlineto{\pgfqpoint{3.405249in}{1.123621in}}%
\pgfpathlineto{\pgfqpoint{3.405249in}{1.103099in}}%
\pgfpathclose%
\pgfusepath{fill}%
\end{pgfscope}%
\begin{pgfscope}%
\pgfpathrectangle{\pgfqpoint{0.499691in}{1.103099in}}{\pgfqpoint{7.362500in}{3.850000in}}%
\pgfusepath{clip}%
\pgfsetbuttcap%
\pgfsetmiterjoin%
\definecolor{currentfill}{rgb}{0.914216,0.537745,0.399510}%
\pgfsetfillcolor{currentfill}%
\pgfsetlinewidth{0.000000pt}%
\definecolor{currentstroke}{rgb}{0.000000,0.000000,0.000000}%
\pgfsetstrokecolor{currentstroke}%
\pgfsetstrokeopacity{0.000000}%
\pgfsetdash{}{0pt}%
\pgfpathmoveto{\pgfqpoint{3.536722in}{1.103099in}}%
\pgfpathlineto{\pgfqpoint{3.641901in}{1.103099in}}%
\pgfpathlineto{\pgfqpoint{3.641901in}{1.182701in}}%
\pgfpathlineto{\pgfqpoint{3.536722in}{1.182701in}}%
\pgfpathlineto{\pgfqpoint{3.536722in}{1.103099in}}%
\pgfpathclose%
\pgfusepath{fill}%
\end{pgfscope}%
\begin{pgfscope}%
\pgfpathrectangle{\pgfqpoint{0.499691in}{1.103099in}}{\pgfqpoint{7.362500in}{3.850000in}}%
\pgfusepath{clip}%
\pgfsetbuttcap%
\pgfsetmiterjoin%
\definecolor{currentfill}{rgb}{0.914216,0.537745,0.399510}%
\pgfsetfillcolor{currentfill}%
\pgfsetlinewidth{0.000000pt}%
\definecolor{currentstroke}{rgb}{0.000000,0.000000,0.000000}%
\pgfsetstrokecolor{currentstroke}%
\pgfsetstrokeopacity{0.000000}%
\pgfsetdash{}{0pt}%
\pgfpathmoveto{\pgfqpoint{3.668196in}{1.103099in}}%
\pgfpathlineto{\pgfqpoint{3.773374in}{1.103099in}}%
\pgfpathlineto{\pgfqpoint{3.773374in}{1.163422in}}%
\pgfpathlineto{\pgfqpoint{3.668196in}{1.163422in}}%
\pgfpathlineto{\pgfqpoint{3.668196in}{1.103099in}}%
\pgfpathclose%
\pgfusepath{fill}%
\end{pgfscope}%
\begin{pgfscope}%
\pgfpathrectangle{\pgfqpoint{0.499691in}{1.103099in}}{\pgfqpoint{7.362500in}{3.850000in}}%
\pgfusepath{clip}%
\pgfsetbuttcap%
\pgfsetmiterjoin%
\definecolor{currentfill}{rgb}{0.914216,0.537745,0.399510}%
\pgfsetfillcolor{currentfill}%
\pgfsetlinewidth{0.000000pt}%
\definecolor{currentstroke}{rgb}{0.000000,0.000000,0.000000}%
\pgfsetstrokecolor{currentstroke}%
\pgfsetstrokeopacity{0.000000}%
\pgfsetdash{}{0pt}%
\pgfpathmoveto{\pgfqpoint{3.799669in}{1.103099in}}%
\pgfpathlineto{\pgfqpoint{3.904847in}{1.103099in}}%
\pgfpathlineto{\pgfqpoint{3.904847in}{1.234940in}}%
\pgfpathlineto{\pgfqpoint{3.799669in}{1.234940in}}%
\pgfpathlineto{\pgfqpoint{3.799669in}{1.103099in}}%
\pgfpathclose%
\pgfusepath{fill}%
\end{pgfscope}%
\begin{pgfscope}%
\pgfpathrectangle{\pgfqpoint{0.499691in}{1.103099in}}{\pgfqpoint{7.362500in}{3.850000in}}%
\pgfusepath{clip}%
\pgfsetbuttcap%
\pgfsetmiterjoin%
\definecolor{currentfill}{rgb}{0.914216,0.537745,0.399510}%
\pgfsetfillcolor{currentfill}%
\pgfsetlinewidth{0.000000pt}%
\definecolor{currentstroke}{rgb}{0.000000,0.000000,0.000000}%
\pgfsetstrokecolor{currentstroke}%
\pgfsetstrokeopacity{0.000000}%
\pgfsetdash{}{0pt}%
\pgfpathmoveto{\pgfqpoint{3.931142in}{1.103099in}}%
\pgfpathlineto{\pgfqpoint{4.036321in}{1.103099in}}%
\pgfpathlineto{\pgfqpoint{4.036321in}{1.297129in}}%
\pgfpathlineto{\pgfqpoint{3.931142in}{1.297129in}}%
\pgfpathlineto{\pgfqpoint{3.931142in}{1.103099in}}%
\pgfpathclose%
\pgfusepath{fill}%
\end{pgfscope}%
\begin{pgfscope}%
\pgfpathrectangle{\pgfqpoint{0.499691in}{1.103099in}}{\pgfqpoint{7.362500in}{3.850000in}}%
\pgfusepath{clip}%
\pgfsetbuttcap%
\pgfsetmiterjoin%
\definecolor{currentfill}{rgb}{0.914216,0.537745,0.399510}%
\pgfsetfillcolor{currentfill}%
\pgfsetlinewidth{0.000000pt}%
\definecolor{currentstroke}{rgb}{0.000000,0.000000,0.000000}%
\pgfsetstrokecolor{currentstroke}%
\pgfsetstrokeopacity{0.000000}%
\pgfsetdash{}{0pt}%
\pgfpathmoveto{\pgfqpoint{4.062615in}{1.103099in}}%
\pgfpathlineto{\pgfqpoint{4.167794in}{1.103099in}}%
\pgfpathlineto{\pgfqpoint{4.167794in}{1.103721in}}%
\pgfpathlineto{\pgfqpoint{4.062615in}{1.103721in}}%
\pgfpathlineto{\pgfqpoint{4.062615in}{1.103099in}}%
\pgfpathclose%
\pgfusepath{fill}%
\end{pgfscope}%
\begin{pgfscope}%
\pgfpathrectangle{\pgfqpoint{0.499691in}{1.103099in}}{\pgfqpoint{7.362500in}{3.850000in}}%
\pgfusepath{clip}%
\pgfsetbuttcap%
\pgfsetmiterjoin%
\definecolor{currentfill}{rgb}{0.914216,0.537745,0.399510}%
\pgfsetfillcolor{currentfill}%
\pgfsetlinewidth{0.000000pt}%
\definecolor{currentstroke}{rgb}{0.000000,0.000000,0.000000}%
\pgfsetstrokecolor{currentstroke}%
\pgfsetstrokeopacity{0.000000}%
\pgfsetdash{}{0pt}%
\pgfpathmoveto{\pgfqpoint{4.194088in}{1.103099in}}%
\pgfpathlineto{\pgfqpoint{4.299267in}{1.103099in}}%
\pgfpathlineto{\pgfqpoint{4.299267in}{1.142278in}}%
\pgfpathlineto{\pgfqpoint{4.194088in}{1.142278in}}%
\pgfpathlineto{\pgfqpoint{4.194088in}{1.103099in}}%
\pgfpathclose%
\pgfusepath{fill}%
\end{pgfscope}%
\begin{pgfscope}%
\pgfpathrectangle{\pgfqpoint{0.499691in}{1.103099in}}{\pgfqpoint{7.362500in}{3.850000in}}%
\pgfusepath{clip}%
\pgfsetbuttcap%
\pgfsetmiterjoin%
\definecolor{currentfill}{rgb}{0.914216,0.537745,0.399510}%
\pgfsetfillcolor{currentfill}%
\pgfsetlinewidth{0.000000pt}%
\definecolor{currentstroke}{rgb}{0.000000,0.000000,0.000000}%
\pgfsetstrokecolor{currentstroke}%
\pgfsetstrokeopacity{0.000000}%
\pgfsetdash{}{0pt}%
\pgfpathmoveto{\pgfqpoint{4.325562in}{1.103099in}}%
\pgfpathlineto{\pgfqpoint{4.430740in}{1.103099in}}%
\pgfpathlineto{\pgfqpoint{4.430740in}{3.769765in}}%
\pgfpathlineto{\pgfqpoint{4.325562in}{3.769765in}}%
\pgfpathlineto{\pgfqpoint{4.325562in}{1.103099in}}%
\pgfpathclose%
\pgfusepath{fill}%
\end{pgfscope}%
\begin{pgfscope}%
\pgfpathrectangle{\pgfqpoint{0.499691in}{1.103099in}}{\pgfqpoint{7.362500in}{3.850000in}}%
\pgfusepath{clip}%
\pgfsetbuttcap%
\pgfsetmiterjoin%
\definecolor{currentfill}{rgb}{0.914216,0.537745,0.399510}%
\pgfsetfillcolor{currentfill}%
\pgfsetlinewidth{0.000000pt}%
\definecolor{currentstroke}{rgb}{0.000000,0.000000,0.000000}%
\pgfsetstrokecolor{currentstroke}%
\pgfsetstrokeopacity{0.000000}%
\pgfsetdash{}{0pt}%
\pgfpathmoveto{\pgfqpoint{4.457035in}{1.103099in}}%
\pgfpathlineto{\pgfqpoint{4.562213in}{1.103099in}}%
\pgfpathlineto{\pgfqpoint{4.562213in}{1.114293in}}%
\pgfpathlineto{\pgfqpoint{4.457035in}{1.114293in}}%
\pgfpathlineto{\pgfqpoint{4.457035in}{1.103099in}}%
\pgfpathclose%
\pgfusepath{fill}%
\end{pgfscope}%
\begin{pgfscope}%
\pgfpathrectangle{\pgfqpoint{0.499691in}{1.103099in}}{\pgfqpoint{7.362500in}{3.850000in}}%
\pgfusepath{clip}%
\pgfsetbuttcap%
\pgfsetmiterjoin%
\definecolor{currentfill}{rgb}{0.914216,0.537745,0.399510}%
\pgfsetfillcolor{currentfill}%
\pgfsetlinewidth{0.000000pt}%
\definecolor{currentstroke}{rgb}{0.000000,0.000000,0.000000}%
\pgfsetstrokecolor{currentstroke}%
\pgfsetstrokeopacity{0.000000}%
\pgfsetdash{}{0pt}%
\pgfpathmoveto{\pgfqpoint{4.588508in}{1.103099in}}%
\pgfpathlineto{\pgfqpoint{4.693687in}{1.103099in}}%
\pgfpathlineto{\pgfqpoint{4.693687in}{1.483696in}}%
\pgfpathlineto{\pgfqpoint{4.588508in}{1.483696in}}%
\pgfpathlineto{\pgfqpoint{4.588508in}{1.103099in}}%
\pgfpathclose%
\pgfusepath{fill}%
\end{pgfscope}%
\begin{pgfscope}%
\pgfpathrectangle{\pgfqpoint{0.499691in}{1.103099in}}{\pgfqpoint{7.362500in}{3.850000in}}%
\pgfusepath{clip}%
\pgfsetbuttcap%
\pgfsetmiterjoin%
\definecolor{currentfill}{rgb}{0.914216,0.537745,0.399510}%
\pgfsetfillcolor{currentfill}%
\pgfsetlinewidth{0.000000pt}%
\definecolor{currentstroke}{rgb}{0.000000,0.000000,0.000000}%
\pgfsetstrokecolor{currentstroke}%
\pgfsetstrokeopacity{0.000000}%
\pgfsetdash{}{0pt}%
\pgfpathmoveto{\pgfqpoint{4.719981in}{1.103099in}}%
\pgfpathlineto{\pgfqpoint{4.825160in}{1.103099in}}%
\pgfpathlineto{\pgfqpoint{4.825160in}{1.618646in}}%
\pgfpathlineto{\pgfqpoint{4.719981in}{1.618646in}}%
\pgfpathlineto{\pgfqpoint{4.719981in}{1.103099in}}%
\pgfpathclose%
\pgfusepath{fill}%
\end{pgfscope}%
\begin{pgfscope}%
\pgfpathrectangle{\pgfqpoint{0.499691in}{1.103099in}}{\pgfqpoint{7.362500in}{3.850000in}}%
\pgfusepath{clip}%
\pgfsetbuttcap%
\pgfsetmiterjoin%
\definecolor{currentfill}{rgb}{0.914216,0.537745,0.399510}%
\pgfsetfillcolor{currentfill}%
\pgfsetlinewidth{0.000000pt}%
\definecolor{currentstroke}{rgb}{0.000000,0.000000,0.000000}%
\pgfsetstrokecolor{currentstroke}%
\pgfsetstrokeopacity{0.000000}%
\pgfsetdash{}{0pt}%
\pgfpathmoveto{\pgfqpoint{4.851455in}{1.103099in}}%
\pgfpathlineto{\pgfqpoint{4.956633in}{1.103099in}}%
\pgfpathlineto{\pgfqpoint{4.956633in}{1.470636in}}%
\pgfpathlineto{\pgfqpoint{4.851455in}{1.470636in}}%
\pgfpathlineto{\pgfqpoint{4.851455in}{1.103099in}}%
\pgfpathclose%
\pgfusepath{fill}%
\end{pgfscope}%
\begin{pgfscope}%
\pgfpathrectangle{\pgfqpoint{0.499691in}{1.103099in}}{\pgfqpoint{7.362500in}{3.850000in}}%
\pgfusepath{clip}%
\pgfsetbuttcap%
\pgfsetmiterjoin%
\definecolor{currentfill}{rgb}{0.914216,0.537745,0.399510}%
\pgfsetfillcolor{currentfill}%
\pgfsetlinewidth{0.000000pt}%
\definecolor{currentstroke}{rgb}{0.000000,0.000000,0.000000}%
\pgfsetstrokecolor{currentstroke}%
\pgfsetstrokeopacity{0.000000}%
\pgfsetdash{}{0pt}%
\pgfpathmoveto{\pgfqpoint{4.982928in}{1.103099in}}%
\pgfpathlineto{\pgfqpoint{5.088106in}{1.103099in}}%
\pgfpathlineto{\pgfqpoint{5.088106in}{1.380462in}}%
\pgfpathlineto{\pgfqpoint{4.982928in}{1.380462in}}%
\pgfpathlineto{\pgfqpoint{4.982928in}{1.103099in}}%
\pgfpathclose%
\pgfusepath{fill}%
\end{pgfscope}%
\begin{pgfscope}%
\pgfpathrectangle{\pgfqpoint{0.499691in}{1.103099in}}{\pgfqpoint{7.362500in}{3.850000in}}%
\pgfusepath{clip}%
\pgfsetbuttcap%
\pgfsetmiterjoin%
\definecolor{currentfill}{rgb}{0.914216,0.537745,0.399510}%
\pgfsetfillcolor{currentfill}%
\pgfsetlinewidth{0.000000pt}%
\definecolor{currentstroke}{rgb}{0.000000,0.000000,0.000000}%
\pgfsetstrokecolor{currentstroke}%
\pgfsetstrokeopacity{0.000000}%
\pgfsetdash{}{0pt}%
\pgfpathmoveto{\pgfqpoint{5.114401in}{1.103099in}}%
\pgfpathlineto{\pgfqpoint{5.219580in}{1.103099in}}%
\pgfpathlineto{\pgfqpoint{5.219580in}{1.105586in}}%
\pgfpathlineto{\pgfqpoint{5.114401in}{1.105586in}}%
\pgfpathlineto{\pgfqpoint{5.114401in}{1.103099in}}%
\pgfpathclose%
\pgfusepath{fill}%
\end{pgfscope}%
\begin{pgfscope}%
\pgfpathrectangle{\pgfqpoint{0.499691in}{1.103099in}}{\pgfqpoint{7.362500in}{3.850000in}}%
\pgfusepath{clip}%
\pgfsetbuttcap%
\pgfsetmiterjoin%
\definecolor{currentfill}{rgb}{0.914216,0.537745,0.399510}%
\pgfsetfillcolor{currentfill}%
\pgfsetlinewidth{0.000000pt}%
\definecolor{currentstroke}{rgb}{0.000000,0.000000,0.000000}%
\pgfsetstrokecolor{currentstroke}%
\pgfsetstrokeopacity{0.000000}%
\pgfsetdash{}{0pt}%
\pgfpathmoveto{\pgfqpoint{5.245874in}{1.103099in}}%
\pgfpathlineto{\pgfqpoint{5.351053in}{1.103099in}}%
\pgfpathlineto{\pgfqpoint{5.351053in}{1.114915in}}%
\pgfpathlineto{\pgfqpoint{5.245874in}{1.114915in}}%
\pgfpathlineto{\pgfqpoint{5.245874in}{1.103099in}}%
\pgfpathclose%
\pgfusepath{fill}%
\end{pgfscope}%
\begin{pgfscope}%
\pgfpathrectangle{\pgfqpoint{0.499691in}{1.103099in}}{\pgfqpoint{7.362500in}{3.850000in}}%
\pgfusepath{clip}%
\pgfsetbuttcap%
\pgfsetmiterjoin%
\definecolor{currentfill}{rgb}{0.914216,0.537745,0.399510}%
\pgfsetfillcolor{currentfill}%
\pgfsetlinewidth{0.000000pt}%
\definecolor{currentstroke}{rgb}{0.000000,0.000000,0.000000}%
\pgfsetstrokecolor{currentstroke}%
\pgfsetstrokeopacity{0.000000}%
\pgfsetdash{}{0pt}%
\pgfpathmoveto{\pgfqpoint{5.377347in}{1.103099in}}%
\pgfpathlineto{\pgfqpoint{5.482526in}{1.103099in}}%
\pgfpathlineto{\pgfqpoint{5.482526in}{1.126731in}}%
\pgfpathlineto{\pgfqpoint{5.377347in}{1.126731in}}%
\pgfpathlineto{\pgfqpoint{5.377347in}{1.103099in}}%
\pgfpathclose%
\pgfusepath{fill}%
\end{pgfscope}%
\begin{pgfscope}%
\pgfpathrectangle{\pgfqpoint{0.499691in}{1.103099in}}{\pgfqpoint{7.362500in}{3.850000in}}%
\pgfusepath{clip}%
\pgfsetbuttcap%
\pgfsetmiterjoin%
\definecolor{currentfill}{rgb}{0.914216,0.537745,0.399510}%
\pgfsetfillcolor{currentfill}%
\pgfsetlinewidth{0.000000pt}%
\definecolor{currentstroke}{rgb}{0.000000,0.000000,0.000000}%
\pgfsetstrokecolor{currentstroke}%
\pgfsetstrokeopacity{0.000000}%
\pgfsetdash{}{0pt}%
\pgfpathmoveto{\pgfqpoint{5.508821in}{1.103099in}}%
\pgfpathlineto{\pgfqpoint{5.613999in}{1.103099in}}%
\pgfpathlineto{\pgfqpoint{5.613999in}{1.109318in}}%
\pgfpathlineto{\pgfqpoint{5.508821in}{1.109318in}}%
\pgfpathlineto{\pgfqpoint{5.508821in}{1.103099in}}%
\pgfpathclose%
\pgfusepath{fill}%
\end{pgfscope}%
\begin{pgfscope}%
\pgfpathrectangle{\pgfqpoint{0.499691in}{1.103099in}}{\pgfqpoint{7.362500in}{3.850000in}}%
\pgfusepath{clip}%
\pgfsetbuttcap%
\pgfsetmiterjoin%
\definecolor{currentfill}{rgb}{0.914216,0.537745,0.399510}%
\pgfsetfillcolor{currentfill}%
\pgfsetlinewidth{0.000000pt}%
\definecolor{currentstroke}{rgb}{0.000000,0.000000,0.000000}%
\pgfsetstrokecolor{currentstroke}%
\pgfsetstrokeopacity{0.000000}%
\pgfsetdash{}{0pt}%
\pgfpathmoveto{\pgfqpoint{5.640294in}{1.103099in}}%
\pgfpathlineto{\pgfqpoint{5.745472in}{1.103099in}}%
\pgfpathlineto{\pgfqpoint{5.745472in}{1.397875in}}%
\pgfpathlineto{\pgfqpoint{5.640294in}{1.397875in}}%
\pgfpathlineto{\pgfqpoint{5.640294in}{1.103099in}}%
\pgfpathclose%
\pgfusepath{fill}%
\end{pgfscope}%
\begin{pgfscope}%
\pgfpathrectangle{\pgfqpoint{0.499691in}{1.103099in}}{\pgfqpoint{7.362500in}{3.850000in}}%
\pgfusepath{clip}%
\pgfsetbuttcap%
\pgfsetmiterjoin%
\definecolor{currentfill}{rgb}{0.914216,0.537745,0.399510}%
\pgfsetfillcolor{currentfill}%
\pgfsetlinewidth{0.000000pt}%
\definecolor{currentstroke}{rgb}{0.000000,0.000000,0.000000}%
\pgfsetstrokecolor{currentstroke}%
\pgfsetstrokeopacity{0.000000}%
\pgfsetdash{}{0pt}%
\pgfpathmoveto{\pgfqpoint{5.771767in}{1.103099in}}%
\pgfpathlineto{\pgfqpoint{5.876946in}{1.103099in}}%
\pgfpathlineto{\pgfqpoint{5.876946in}{1.269144in}}%
\pgfpathlineto{\pgfqpoint{5.771767in}{1.269144in}}%
\pgfpathlineto{\pgfqpoint{5.771767in}{1.103099in}}%
\pgfpathclose%
\pgfusepath{fill}%
\end{pgfscope}%
\begin{pgfscope}%
\pgfpathrectangle{\pgfqpoint{0.499691in}{1.103099in}}{\pgfqpoint{7.362500in}{3.850000in}}%
\pgfusepath{clip}%
\pgfsetbuttcap%
\pgfsetmiterjoin%
\definecolor{currentfill}{rgb}{0.914216,0.537745,0.399510}%
\pgfsetfillcolor{currentfill}%
\pgfsetlinewidth{0.000000pt}%
\definecolor{currentstroke}{rgb}{0.000000,0.000000,0.000000}%
\pgfsetstrokecolor{currentstroke}%
\pgfsetstrokeopacity{0.000000}%
\pgfsetdash{}{0pt}%
\pgfpathmoveto{\pgfqpoint{5.903240in}{1.103099in}}%
\pgfpathlineto{\pgfqpoint{6.008419in}{1.103099in}}%
\pgfpathlineto{\pgfqpoint{6.008419in}{1.136059in}}%
\pgfpathlineto{\pgfqpoint{5.903240in}{1.136059in}}%
\pgfpathlineto{\pgfqpoint{5.903240in}{1.103099in}}%
\pgfpathclose%
\pgfusepath{fill}%
\end{pgfscope}%
\begin{pgfscope}%
\pgfpathrectangle{\pgfqpoint{0.499691in}{1.103099in}}{\pgfqpoint{7.362500in}{3.850000in}}%
\pgfusepath{clip}%
\pgfsetbuttcap%
\pgfsetmiterjoin%
\definecolor{currentfill}{rgb}{0.914216,0.537745,0.399510}%
\pgfsetfillcolor{currentfill}%
\pgfsetlinewidth{0.000000pt}%
\definecolor{currentstroke}{rgb}{0.000000,0.000000,0.000000}%
\pgfsetstrokecolor{currentstroke}%
\pgfsetstrokeopacity{0.000000}%
\pgfsetdash{}{0pt}%
\pgfpathmoveto{\pgfqpoint{6.034713in}{1.103099in}}%
\pgfpathlineto{\pgfqpoint{6.139892in}{1.103099in}}%
\pgfpathlineto{\pgfqpoint{6.139892in}{1.233074in}}%
\pgfpathlineto{\pgfqpoint{6.034713in}{1.233074in}}%
\pgfpathlineto{\pgfqpoint{6.034713in}{1.103099in}}%
\pgfpathclose%
\pgfusepath{fill}%
\end{pgfscope}%
\begin{pgfscope}%
\pgfpathrectangle{\pgfqpoint{0.499691in}{1.103099in}}{\pgfqpoint{7.362500in}{3.850000in}}%
\pgfusepath{clip}%
\pgfsetbuttcap%
\pgfsetmiterjoin%
\definecolor{currentfill}{rgb}{0.914216,0.537745,0.399510}%
\pgfsetfillcolor{currentfill}%
\pgfsetlinewidth{0.000000pt}%
\definecolor{currentstroke}{rgb}{0.000000,0.000000,0.000000}%
\pgfsetstrokecolor{currentstroke}%
\pgfsetstrokeopacity{0.000000}%
\pgfsetdash{}{0pt}%
\pgfpathmoveto{\pgfqpoint{6.166187in}{1.103099in}}%
\pgfpathlineto{\pgfqpoint{6.271365in}{1.103099in}}%
\pgfpathlineto{\pgfqpoint{6.271365in}{1.182701in}}%
\pgfpathlineto{\pgfqpoint{6.166187in}{1.182701in}}%
\pgfpathlineto{\pgfqpoint{6.166187in}{1.103099in}}%
\pgfpathclose%
\pgfusepath{fill}%
\end{pgfscope}%
\begin{pgfscope}%
\pgfpathrectangle{\pgfqpoint{0.499691in}{1.103099in}}{\pgfqpoint{7.362500in}{3.850000in}}%
\pgfusepath{clip}%
\pgfsetbuttcap%
\pgfsetmiterjoin%
\definecolor{currentfill}{rgb}{0.914216,0.537745,0.399510}%
\pgfsetfillcolor{currentfill}%
\pgfsetlinewidth{0.000000pt}%
\definecolor{currentstroke}{rgb}{0.000000,0.000000,0.000000}%
\pgfsetstrokecolor{currentstroke}%
\pgfsetstrokeopacity{0.000000}%
\pgfsetdash{}{0pt}%
\pgfpathmoveto{\pgfqpoint{6.297660in}{1.103099in}}%
\pgfpathlineto{\pgfqpoint{6.402838in}{1.103099in}}%
\pgfpathlineto{\pgfqpoint{6.402838in}{1.638547in}}%
\pgfpathlineto{\pgfqpoint{6.297660in}{1.638547in}}%
\pgfpathlineto{\pgfqpoint{6.297660in}{1.103099in}}%
\pgfpathclose%
\pgfusepath{fill}%
\end{pgfscope}%
\begin{pgfscope}%
\pgfpathrectangle{\pgfqpoint{0.499691in}{1.103099in}}{\pgfqpoint{7.362500in}{3.850000in}}%
\pgfusepath{clip}%
\pgfsetbuttcap%
\pgfsetmiterjoin%
\definecolor{currentfill}{rgb}{0.914216,0.537745,0.399510}%
\pgfsetfillcolor{currentfill}%
\pgfsetlinewidth{0.000000pt}%
\definecolor{currentstroke}{rgb}{0.000000,0.000000,0.000000}%
\pgfsetstrokecolor{currentstroke}%
\pgfsetstrokeopacity{0.000000}%
\pgfsetdash{}{0pt}%
\pgfpathmoveto{\pgfqpoint{6.429133in}{1.103099in}}%
\pgfpathlineto{\pgfqpoint{6.534312in}{1.103099in}}%
\pgfpathlineto{\pgfqpoint{6.534312in}{1.447004in}}%
\pgfpathlineto{\pgfqpoint{6.429133in}{1.447004in}}%
\pgfpathlineto{\pgfqpoint{6.429133in}{1.103099in}}%
\pgfpathclose%
\pgfusepath{fill}%
\end{pgfscope}%
\begin{pgfscope}%
\pgfpathrectangle{\pgfqpoint{0.499691in}{1.103099in}}{\pgfqpoint{7.362500in}{3.850000in}}%
\pgfusepath{clip}%
\pgfsetbuttcap%
\pgfsetmiterjoin%
\definecolor{currentfill}{rgb}{0.914216,0.537745,0.399510}%
\pgfsetfillcolor{currentfill}%
\pgfsetlinewidth{0.000000pt}%
\definecolor{currentstroke}{rgb}{0.000000,0.000000,0.000000}%
\pgfsetstrokecolor{currentstroke}%
\pgfsetstrokeopacity{0.000000}%
\pgfsetdash{}{0pt}%
\pgfpathmoveto{\pgfqpoint{6.560606in}{1.103099in}}%
\pgfpathlineto{\pgfqpoint{6.665785in}{1.103099in}}%
\pgfpathlineto{\pgfqpoint{6.665785in}{1.120512in}}%
\pgfpathlineto{\pgfqpoint{6.560606in}{1.120512in}}%
\pgfpathlineto{\pgfqpoint{6.560606in}{1.103099in}}%
\pgfpathclose%
\pgfusepath{fill}%
\end{pgfscope}%
\begin{pgfscope}%
\pgfpathrectangle{\pgfqpoint{0.499691in}{1.103099in}}{\pgfqpoint{7.362500in}{3.850000in}}%
\pgfusepath{clip}%
\pgfsetbuttcap%
\pgfsetmiterjoin%
\definecolor{currentfill}{rgb}{0.914216,0.537745,0.399510}%
\pgfsetfillcolor{currentfill}%
\pgfsetlinewidth{0.000000pt}%
\definecolor{currentstroke}{rgb}{0.000000,0.000000,0.000000}%
\pgfsetstrokecolor{currentstroke}%
\pgfsetstrokeopacity{0.000000}%
\pgfsetdash{}{0pt}%
\pgfpathmoveto{\pgfqpoint{6.692080in}{1.103099in}}%
\pgfpathlineto{\pgfqpoint{6.797258in}{1.103099in}}%
\pgfpathlineto{\pgfqpoint{6.797258in}{1.512303in}}%
\pgfpathlineto{\pgfqpoint{6.692080in}{1.512303in}}%
\pgfpathlineto{\pgfqpoint{6.692080in}{1.103099in}}%
\pgfpathclose%
\pgfusepath{fill}%
\end{pgfscope}%
\begin{pgfscope}%
\pgfpathrectangle{\pgfqpoint{0.499691in}{1.103099in}}{\pgfqpoint{7.362500in}{3.850000in}}%
\pgfusepath{clip}%
\pgfsetbuttcap%
\pgfsetmiterjoin%
\definecolor{currentfill}{rgb}{0.914216,0.537745,0.399510}%
\pgfsetfillcolor{currentfill}%
\pgfsetlinewidth{0.000000pt}%
\definecolor{currentstroke}{rgb}{0.000000,0.000000,0.000000}%
\pgfsetstrokecolor{currentstroke}%
\pgfsetstrokeopacity{0.000000}%
\pgfsetdash{}{0pt}%
\pgfpathmoveto{\pgfqpoint{6.823553in}{1.103099in}}%
\pgfpathlineto{\pgfqpoint{6.928731in}{1.103099in}}%
\pgfpathlineto{\pgfqpoint{6.928731in}{1.162800in}}%
\pgfpathlineto{\pgfqpoint{6.823553in}{1.162800in}}%
\pgfpathlineto{\pgfqpoint{6.823553in}{1.103099in}}%
\pgfpathclose%
\pgfusepath{fill}%
\end{pgfscope}%
\begin{pgfscope}%
\pgfpathrectangle{\pgfqpoint{0.499691in}{1.103099in}}{\pgfqpoint{7.362500in}{3.850000in}}%
\pgfusepath{clip}%
\pgfsetbuttcap%
\pgfsetmiterjoin%
\definecolor{currentfill}{rgb}{0.914216,0.537745,0.399510}%
\pgfsetfillcolor{currentfill}%
\pgfsetlinewidth{0.000000pt}%
\definecolor{currentstroke}{rgb}{0.000000,0.000000,0.000000}%
\pgfsetstrokecolor{currentstroke}%
\pgfsetstrokeopacity{0.000000}%
\pgfsetdash{}{0pt}%
\pgfpathmoveto{\pgfqpoint{6.955026in}{1.103099in}}%
\pgfpathlineto{\pgfqpoint{7.060205in}{1.103099in}}%
\pgfpathlineto{\pgfqpoint{7.060205in}{1.106830in}}%
\pgfpathlineto{\pgfqpoint{6.955026in}{1.106830in}}%
\pgfpathlineto{\pgfqpoint{6.955026in}{1.103099in}}%
\pgfpathclose%
\pgfusepath{fill}%
\end{pgfscope}%
\begin{pgfscope}%
\pgfpathrectangle{\pgfqpoint{0.499691in}{1.103099in}}{\pgfqpoint{7.362500in}{3.850000in}}%
\pgfusepath{clip}%
\pgfsetbuttcap%
\pgfsetmiterjoin%
\definecolor{currentfill}{rgb}{0.914216,0.537745,0.399510}%
\pgfsetfillcolor{currentfill}%
\pgfsetlinewidth{0.000000pt}%
\definecolor{currentstroke}{rgb}{0.000000,0.000000,0.000000}%
\pgfsetstrokecolor{currentstroke}%
\pgfsetstrokeopacity{0.000000}%
\pgfsetdash{}{0pt}%
\pgfpathmoveto{\pgfqpoint{7.086499in}{1.103099in}}%
\pgfpathlineto{\pgfqpoint{7.191678in}{1.103099in}}%
\pgfpathlineto{\pgfqpoint{7.191678in}{1.147875in}}%
\pgfpathlineto{\pgfqpoint{7.086499in}{1.147875in}}%
\pgfpathlineto{\pgfqpoint{7.086499in}{1.103099in}}%
\pgfpathclose%
\pgfusepath{fill}%
\end{pgfscope}%
\begin{pgfscope}%
\pgfpathrectangle{\pgfqpoint{0.499691in}{1.103099in}}{\pgfqpoint{7.362500in}{3.850000in}}%
\pgfusepath{clip}%
\pgfsetbuttcap%
\pgfsetmiterjoin%
\definecolor{currentfill}{rgb}{0.914216,0.537745,0.399510}%
\pgfsetfillcolor{currentfill}%
\pgfsetlinewidth{0.000000pt}%
\definecolor{currentstroke}{rgb}{0.000000,0.000000,0.000000}%
\pgfsetstrokecolor{currentstroke}%
\pgfsetstrokeopacity{0.000000}%
\pgfsetdash{}{0pt}%
\pgfpathmoveto{\pgfqpoint{7.217972in}{1.103099in}}%
\pgfpathlineto{\pgfqpoint{7.323151in}{1.103099in}}%
\pgfpathlineto{\pgfqpoint{7.323151in}{1.162800in}}%
\pgfpathlineto{\pgfqpoint{7.217972in}{1.162800in}}%
\pgfpathlineto{\pgfqpoint{7.217972in}{1.103099in}}%
\pgfpathclose%
\pgfusepath{fill}%
\end{pgfscope}%
\begin{pgfscope}%
\pgfpathrectangle{\pgfqpoint{0.499691in}{1.103099in}}{\pgfqpoint{7.362500in}{3.850000in}}%
\pgfusepath{clip}%
\pgfsetbuttcap%
\pgfsetmiterjoin%
\definecolor{currentfill}{rgb}{0.914216,0.537745,0.399510}%
\pgfsetfillcolor{currentfill}%
\pgfsetlinewidth{0.000000pt}%
\definecolor{currentstroke}{rgb}{0.000000,0.000000,0.000000}%
\pgfsetstrokecolor{currentstroke}%
\pgfsetstrokeopacity{0.000000}%
\pgfsetdash{}{0pt}%
\pgfpathmoveto{\pgfqpoint{7.349446in}{1.103099in}}%
\pgfpathlineto{\pgfqpoint{7.454624in}{1.103099in}}%
\pgfpathlineto{\pgfqpoint{7.454624in}{1.118646in}}%
\pgfpathlineto{\pgfqpoint{7.349446in}{1.118646in}}%
\pgfpathlineto{\pgfqpoint{7.349446in}{1.103099in}}%
\pgfpathclose%
\pgfusepath{fill}%
\end{pgfscope}%
\begin{pgfscope}%
\pgfpathrectangle{\pgfqpoint{0.499691in}{1.103099in}}{\pgfqpoint{7.362500in}{3.850000in}}%
\pgfusepath{clip}%
\pgfsetbuttcap%
\pgfsetmiterjoin%
\definecolor{currentfill}{rgb}{0.914216,0.537745,0.399510}%
\pgfsetfillcolor{currentfill}%
\pgfsetlinewidth{0.000000pt}%
\definecolor{currentstroke}{rgb}{0.000000,0.000000,0.000000}%
\pgfsetstrokecolor{currentstroke}%
\pgfsetstrokeopacity{0.000000}%
\pgfsetdash{}{0pt}%
\pgfpathmoveto{\pgfqpoint{7.480919in}{1.103099in}}%
\pgfpathlineto{\pgfqpoint{7.586097in}{1.103099in}}%
\pgfpathlineto{\pgfqpoint{7.586097in}{1.119890in}}%
\pgfpathlineto{\pgfqpoint{7.480919in}{1.119890in}}%
\pgfpathlineto{\pgfqpoint{7.480919in}{1.103099in}}%
\pgfpathclose%
\pgfusepath{fill}%
\end{pgfscope}%
\begin{pgfscope}%
\pgfpathrectangle{\pgfqpoint{0.499691in}{1.103099in}}{\pgfqpoint{7.362500in}{3.850000in}}%
\pgfusepath{clip}%
\pgfsetbuttcap%
\pgfsetmiterjoin%
\definecolor{currentfill}{rgb}{0.914216,0.537745,0.399510}%
\pgfsetfillcolor{currentfill}%
\pgfsetlinewidth{0.000000pt}%
\definecolor{currentstroke}{rgb}{0.000000,0.000000,0.000000}%
\pgfsetstrokecolor{currentstroke}%
\pgfsetstrokeopacity{0.000000}%
\pgfsetdash{}{0pt}%
\pgfpathmoveto{\pgfqpoint{7.612392in}{1.103099in}}%
\pgfpathlineto{\pgfqpoint{7.717571in}{1.103099in}}%
\pgfpathlineto{\pgfqpoint{7.717571in}{1.188920in}}%
\pgfpathlineto{\pgfqpoint{7.612392in}{1.188920in}}%
\pgfpathlineto{\pgfqpoint{7.612392in}{1.103099in}}%
\pgfpathclose%
\pgfusepath{fill}%
\end{pgfscope}%
\begin{pgfscope}%
\pgfpathrectangle{\pgfqpoint{0.499691in}{1.103099in}}{\pgfqpoint{7.362500in}{3.850000in}}%
\pgfusepath{clip}%
\pgfsetbuttcap%
\pgfsetmiterjoin%
\definecolor{currentfill}{rgb}{0.914216,0.537745,0.399510}%
\pgfsetfillcolor{currentfill}%
\pgfsetlinewidth{0.000000pt}%
\definecolor{currentstroke}{rgb}{0.000000,0.000000,0.000000}%
\pgfsetstrokecolor{currentstroke}%
\pgfsetstrokeopacity{0.000000}%
\pgfsetdash{}{0pt}%
\pgfpathmoveto{\pgfqpoint{7.743865in}{1.103099in}}%
\pgfpathlineto{\pgfqpoint{7.849044in}{1.103099in}}%
\pgfpathlineto{\pgfqpoint{7.849044in}{1.255462in}}%
\pgfpathlineto{\pgfqpoint{7.743865in}{1.255462in}}%
\pgfpathlineto{\pgfqpoint{7.743865in}{1.103099in}}%
\pgfpathclose%
\pgfusepath{fill}%
\end{pgfscope}%
\begin{pgfscope}%
\pgfsetbuttcap%
\pgfsetroundjoin%
\definecolor{currentfill}{rgb}{0.000000,0.000000,0.000000}%
\pgfsetfillcolor{currentfill}%
\pgfsetlinewidth{0.803000pt}%
\definecolor{currentstroke}{rgb}{0.000000,0.000000,0.000000}%
\pgfsetstrokecolor{currentstroke}%
\pgfsetdash{}{0pt}%
\pgfsys@defobject{currentmarker}{\pgfqpoint{0.000000in}{-0.048611in}}{\pgfqpoint{0.000000in}{0.000000in}}{%
\pgfpathmoveto{\pgfqpoint{0.000000in}{0.000000in}}%
\pgfpathlineto{\pgfqpoint{0.000000in}{-0.048611in}}%
\pgfusepath{stroke,fill}%
}%
\begin{pgfscope}%
\pgfsys@transformshift{0.565428in}{1.103099in}%
\pgfsys@useobject{currentmarker}{}%
\end{pgfscope}%
\end{pgfscope}%
\begin{pgfscope}%
\definecolor{textcolor}{rgb}{0.000000,0.000000,0.000000}%
\pgfsetstrokecolor{textcolor}%
\pgfsetfillcolor{textcolor}%
\pgftext[x=0.582789in, y=0.787123in, left, base,rotate=90.000000]{\color{textcolor}\rmfamily\fontsize{5.000000}{6.000000}\selectfont AMV}%
\end{pgfscope}%
\begin{pgfscope}%
\pgfsetbuttcap%
\pgfsetroundjoin%
\definecolor{currentfill}{rgb}{0.000000,0.000000,0.000000}%
\pgfsetfillcolor{currentfill}%
\pgfsetlinewidth{0.803000pt}%
\definecolor{currentstroke}{rgb}{0.000000,0.000000,0.000000}%
\pgfsetstrokecolor{currentstroke}%
\pgfsetdash{}{0pt}%
\pgfsys@defobject{currentmarker}{\pgfqpoint{0.000000in}{-0.048611in}}{\pgfqpoint{0.000000in}{0.000000in}}{%
\pgfpathmoveto{\pgfqpoint{0.000000in}{0.000000in}}%
\pgfpathlineto{\pgfqpoint{0.000000in}{-0.048611in}}%
\pgfusepath{stroke,fill}%
}%
\begin{pgfscope}%
\pgfsys@transformshift{0.696901in}{1.103099in}%
\pgfsys@useobject{currentmarker}{}%
\end{pgfscope}%
\end{pgfscope}%
\begin{pgfscope}%
\definecolor{textcolor}{rgb}{0.000000,0.000000,0.000000}%
\pgfsetstrokecolor{textcolor}%
\pgfsetfillcolor{textcolor}%
\pgftext[x=0.714262in, y=0.715942in, left, base,rotate=90.000000]{\color{textcolor}\rmfamily\fontsize{5.000000}{6.000000}\selectfont APRIL}%
\end{pgfscope}%
\begin{pgfscope}%
\pgfsetbuttcap%
\pgfsetroundjoin%
\definecolor{currentfill}{rgb}{0.000000,0.000000,0.000000}%
\pgfsetfillcolor{currentfill}%
\pgfsetlinewidth{0.803000pt}%
\definecolor{currentstroke}{rgb}{0.000000,0.000000,0.000000}%
\pgfsetstrokecolor{currentstroke}%
\pgfsetdash{}{0pt}%
\pgfsys@defobject{currentmarker}{\pgfqpoint{0.000000in}{-0.048611in}}{\pgfqpoint{0.000000in}{0.000000in}}{%
\pgfpathmoveto{\pgfqpoint{0.000000in}{0.000000in}}%
\pgfpathlineto{\pgfqpoint{0.000000in}{-0.048611in}}%
\pgfusepath{stroke,fill}%
}%
\begin{pgfscope}%
\pgfsys@transformshift{0.828374in}{1.103099in}%
\pgfsys@useobject{currentmarker}{}%
\end{pgfscope}%
\end{pgfscope}%
\begin{pgfscope}%
\definecolor{textcolor}{rgb}{0.000000,0.000000,0.000000}%
\pgfsetstrokecolor{textcolor}%
\pgfsetfillcolor{textcolor}%
\pgftext[x=0.845735in, y=0.468446in, left, base,rotate=90.000000]{\color{textcolor}\rmfamily\fontsize{5.000000}{6.000000}\selectfont APRIL Moto}%
\end{pgfscope}%
\begin{pgfscope}%
\pgfsetbuttcap%
\pgfsetroundjoin%
\definecolor{currentfill}{rgb}{0.000000,0.000000,0.000000}%
\pgfsetfillcolor{currentfill}%
\pgfsetlinewidth{0.803000pt}%
\definecolor{currentstroke}{rgb}{0.000000,0.000000,0.000000}%
\pgfsetstrokecolor{currentstroke}%
\pgfsetdash{}{0pt}%
\pgfsys@defobject{currentmarker}{\pgfqpoint{0.000000in}{-0.048611in}}{\pgfqpoint{0.000000in}{0.000000in}}{%
\pgfpathmoveto{\pgfqpoint{0.000000in}{0.000000in}}%
\pgfpathlineto{\pgfqpoint{0.000000in}{-0.048611in}}%
\pgfusepath{stroke,fill}%
}%
\begin{pgfscope}%
\pgfsys@transformshift{0.959847in}{1.103099in}%
\pgfsys@useobject{currentmarker}{}%
\end{pgfscope}%
\end{pgfscope}%
\begin{pgfscope}%
\definecolor{textcolor}{rgb}{0.000000,0.000000,0.000000}%
\pgfsetstrokecolor{textcolor}%
\pgfsetfillcolor{textcolor}%
\pgftext[x=0.977208in, y=0.801591in, left, base,rotate=90.000000]{\color{textcolor}\rmfamily\fontsize{5.000000}{6.000000}\selectfont AXA}%
\end{pgfscope}%
\begin{pgfscope}%
\pgfsetbuttcap%
\pgfsetroundjoin%
\definecolor{currentfill}{rgb}{0.000000,0.000000,0.000000}%
\pgfsetfillcolor{currentfill}%
\pgfsetlinewidth{0.803000pt}%
\definecolor{currentstroke}{rgb}{0.000000,0.000000,0.000000}%
\pgfsetstrokecolor{currentstroke}%
\pgfsetdash{}{0pt}%
\pgfsys@defobject{currentmarker}{\pgfqpoint{0.000000in}{-0.048611in}}{\pgfqpoint{0.000000in}{0.000000in}}{%
\pgfpathmoveto{\pgfqpoint{0.000000in}{0.000000in}}%
\pgfpathlineto{\pgfqpoint{0.000000in}{-0.048611in}}%
\pgfusepath{stroke,fill}%
}%
\begin{pgfscope}%
\pgfsys@transformshift{1.091321in}{1.103099in}%
\pgfsys@useobject{currentmarker}{}%
\end{pgfscope}%
\end{pgfscope}%
\begin{pgfscope}%
\definecolor{textcolor}{rgb}{0.000000,0.000000,0.000000}%
\pgfsetstrokecolor{textcolor}%
\pgfsetfillcolor{textcolor}%
\pgftext[x=1.108682in, y=0.250948in, left, base,rotate=90.000000]{\color{textcolor}\rmfamily\fontsize{5.000000}{6.000000}\selectfont Active Assurances}%
\end{pgfscope}%
\begin{pgfscope}%
\pgfsetbuttcap%
\pgfsetroundjoin%
\definecolor{currentfill}{rgb}{0.000000,0.000000,0.000000}%
\pgfsetfillcolor{currentfill}%
\pgfsetlinewidth{0.803000pt}%
\definecolor{currentstroke}{rgb}{0.000000,0.000000,0.000000}%
\pgfsetstrokecolor{currentstroke}%
\pgfsetdash{}{0pt}%
\pgfsys@defobject{currentmarker}{\pgfqpoint{0.000000in}{-0.048611in}}{\pgfqpoint{0.000000in}{0.000000in}}{%
\pgfpathmoveto{\pgfqpoint{0.000000in}{0.000000in}}%
\pgfpathlineto{\pgfqpoint{0.000000in}{-0.048611in}}%
\pgfusepath{stroke,fill}%
}%
\begin{pgfscope}%
\pgfsys@transformshift{1.222794in}{1.103099in}%
\pgfsys@useobject{currentmarker}{}%
\end{pgfscope}%
\end{pgfscope}%
\begin{pgfscope}%
\definecolor{textcolor}{rgb}{0.000000,0.000000,0.000000}%
\pgfsetstrokecolor{textcolor}%
\pgfsetfillcolor{textcolor}%
\pgftext[x=1.240155in, y=0.827344in, left, base,rotate=90.000000]{\color{textcolor}\rmfamily\fontsize{5.000000}{6.000000}\selectfont Afer}%
\end{pgfscope}%
\begin{pgfscope}%
\pgfsetbuttcap%
\pgfsetroundjoin%
\definecolor{currentfill}{rgb}{0.000000,0.000000,0.000000}%
\pgfsetfillcolor{currentfill}%
\pgfsetlinewidth{0.803000pt}%
\definecolor{currentstroke}{rgb}{0.000000,0.000000,0.000000}%
\pgfsetstrokecolor{currentstroke}%
\pgfsetdash{}{0pt}%
\pgfsys@defobject{currentmarker}{\pgfqpoint{0.000000in}{-0.048611in}}{\pgfqpoint{0.000000in}{0.000000in}}{%
\pgfpathmoveto{\pgfqpoint{0.000000in}{0.000000in}}%
\pgfpathlineto{\pgfqpoint{0.000000in}{-0.048611in}}%
\pgfusepath{stroke,fill}%
}%
\begin{pgfscope}%
\pgfsys@transformshift{1.354267in}{1.103099in}%
\pgfsys@useobject{currentmarker}{}%
\end{pgfscope}%
\end{pgfscope}%
\begin{pgfscope}%
\definecolor{textcolor}{rgb}{0.000000,0.000000,0.000000}%
\pgfsetstrokecolor{textcolor}%
\pgfsetfillcolor{textcolor}%
\pgftext[x=1.371628in, y=0.656335in, left, base,rotate=90.000000]{\color{textcolor}\rmfamily\fontsize{5.000000}{6.000000}\selectfont Afi Esca}%
\end{pgfscope}%
\begin{pgfscope}%
\pgfsetbuttcap%
\pgfsetroundjoin%
\definecolor{currentfill}{rgb}{0.000000,0.000000,0.000000}%
\pgfsetfillcolor{currentfill}%
\pgfsetlinewidth{0.803000pt}%
\definecolor{currentstroke}{rgb}{0.000000,0.000000,0.000000}%
\pgfsetstrokecolor{currentstroke}%
\pgfsetdash{}{0pt}%
\pgfsys@defobject{currentmarker}{\pgfqpoint{0.000000in}{-0.048611in}}{\pgfqpoint{0.000000in}{0.000000in}}{%
\pgfpathmoveto{\pgfqpoint{0.000000in}{0.000000in}}%
\pgfpathlineto{\pgfqpoint{0.000000in}{-0.048611in}}%
\pgfusepath{stroke,fill}%
}%
\begin{pgfscope}%
\pgfsys@transformshift{1.485740in}{1.103099in}%
\pgfsys@useobject{currentmarker}{}%
\end{pgfscope}%
\end{pgfscope}%
\begin{pgfscope}%
\definecolor{textcolor}{rgb}{0.000000,0.000000,0.000000}%
\pgfsetstrokecolor{textcolor}%
\pgfsetfillcolor{textcolor}%
\pgftext[x=1.503101in, y=0.255480in, left, base,rotate=90.000000]{\color{textcolor}\rmfamily\fontsize{5.000000}{6.000000}\selectfont Ag2r La Mondiale}%
\end{pgfscope}%
\begin{pgfscope}%
\pgfsetbuttcap%
\pgfsetroundjoin%
\definecolor{currentfill}{rgb}{0.000000,0.000000,0.000000}%
\pgfsetfillcolor{currentfill}%
\pgfsetlinewidth{0.803000pt}%
\definecolor{currentstroke}{rgb}{0.000000,0.000000,0.000000}%
\pgfsetstrokecolor{currentstroke}%
\pgfsetdash{}{0pt}%
\pgfsys@defobject{currentmarker}{\pgfqpoint{0.000000in}{-0.048611in}}{\pgfqpoint{0.000000in}{0.000000in}}{%
\pgfpathmoveto{\pgfqpoint{0.000000in}{0.000000in}}%
\pgfpathlineto{\pgfqpoint{0.000000in}{-0.048611in}}%
\pgfusepath{stroke,fill}%
}%
\begin{pgfscope}%
\pgfsys@transformshift{1.617213in}{1.103099in}%
\pgfsys@useobject{currentmarker}{}%
\end{pgfscope}%
\end{pgfscope}%
\begin{pgfscope}%
\definecolor{textcolor}{rgb}{0.000000,0.000000,0.000000}%
\pgfsetstrokecolor{textcolor}%
\pgfsetfillcolor{textcolor}%
\pgftext[x=1.634575in, y=0.712084in, left, base,rotate=90.000000]{\color{textcolor}\rmfamily\fontsize{5.000000}{6.000000}\selectfont Allianz}%
\end{pgfscope}%
\begin{pgfscope}%
\pgfsetbuttcap%
\pgfsetroundjoin%
\definecolor{currentfill}{rgb}{0.000000,0.000000,0.000000}%
\pgfsetfillcolor{currentfill}%
\pgfsetlinewidth{0.803000pt}%
\definecolor{currentstroke}{rgb}{0.000000,0.000000,0.000000}%
\pgfsetstrokecolor{currentstroke}%
\pgfsetdash{}{0pt}%
\pgfsys@defobject{currentmarker}{\pgfqpoint{0.000000in}{-0.048611in}}{\pgfqpoint{0.000000in}{0.000000in}}{%
\pgfpathmoveto{\pgfqpoint{0.000000in}{0.000000in}}%
\pgfpathlineto{\pgfqpoint{0.000000in}{-0.048611in}}%
\pgfusepath{stroke,fill}%
}%
\begin{pgfscope}%
\pgfsys@transformshift{1.748687in}{1.103099in}%
\pgfsys@useobject{currentmarker}{}%
\end{pgfscope}%
\end{pgfscope}%
\begin{pgfscope}%
\definecolor{textcolor}{rgb}{0.000000,0.000000,0.000000}%
\pgfsetstrokecolor{textcolor}%
\pgfsetfillcolor{textcolor}%
\pgftext[x=1.766048in, y=0.352221in, left, base,rotate=90.000000]{\color{textcolor}\rmfamily\fontsize{5.000000}{6.000000}\selectfont Assur Bon Plan}%
\end{pgfscope}%
\begin{pgfscope}%
\pgfsetbuttcap%
\pgfsetroundjoin%
\definecolor{currentfill}{rgb}{0.000000,0.000000,0.000000}%
\pgfsetfillcolor{currentfill}%
\pgfsetlinewidth{0.803000pt}%
\definecolor{currentstroke}{rgb}{0.000000,0.000000,0.000000}%
\pgfsetstrokecolor{currentstroke}%
\pgfsetdash{}{0pt}%
\pgfsys@defobject{currentmarker}{\pgfqpoint{0.000000in}{-0.048611in}}{\pgfqpoint{0.000000in}{0.000000in}}{%
\pgfpathmoveto{\pgfqpoint{0.000000in}{0.000000in}}%
\pgfpathlineto{\pgfqpoint{0.000000in}{-0.048611in}}%
\pgfusepath{stroke,fill}%
}%
\begin{pgfscope}%
\pgfsys@transformshift{1.880160in}{1.103099in}%
\pgfsys@useobject{currentmarker}{}%
\end{pgfscope}%
\end{pgfscope}%
\begin{pgfscope}%
\definecolor{textcolor}{rgb}{0.000000,0.000000,0.000000}%
\pgfsetstrokecolor{textcolor}%
\pgfsetfillcolor{textcolor}%
\pgftext[x=1.897521in, y=0.476452in, left, base,rotate=90.000000]{\color{textcolor}\rmfamily\fontsize{5.000000}{6.000000}\selectfont Assur O'Poil}%
\end{pgfscope}%
\begin{pgfscope}%
\pgfsetbuttcap%
\pgfsetroundjoin%
\definecolor{currentfill}{rgb}{0.000000,0.000000,0.000000}%
\pgfsetfillcolor{currentfill}%
\pgfsetlinewidth{0.803000pt}%
\definecolor{currentstroke}{rgb}{0.000000,0.000000,0.000000}%
\pgfsetstrokecolor{currentstroke}%
\pgfsetdash{}{0pt}%
\pgfsys@defobject{currentmarker}{\pgfqpoint{0.000000in}{-0.048611in}}{\pgfqpoint{0.000000in}{0.000000in}}{%
\pgfpathmoveto{\pgfqpoint{0.000000in}{0.000000in}}%
\pgfpathlineto{\pgfqpoint{0.000000in}{-0.048611in}}%
\pgfusepath{stroke,fill}%
}%
\begin{pgfscope}%
\pgfsys@transformshift{2.011633in}{1.103099in}%
\pgfsys@useobject{currentmarker}{}%
\end{pgfscope}%
\end{pgfscope}%
\begin{pgfscope}%
\definecolor{textcolor}{rgb}{0.000000,0.000000,0.000000}%
\pgfsetstrokecolor{textcolor}%
\pgfsetfillcolor{textcolor}%
\pgftext[x=2.028994in, y=0.497960in, left, base,rotate=90.000000]{\color{textcolor}\rmfamily\fontsize{5.000000}{6.000000}\selectfont AssurOnline}%
\end{pgfscope}%
\begin{pgfscope}%
\pgfsetbuttcap%
\pgfsetroundjoin%
\definecolor{currentfill}{rgb}{0.000000,0.000000,0.000000}%
\pgfsetfillcolor{currentfill}%
\pgfsetlinewidth{0.803000pt}%
\definecolor{currentstroke}{rgb}{0.000000,0.000000,0.000000}%
\pgfsetstrokecolor{currentstroke}%
\pgfsetdash{}{0pt}%
\pgfsys@defobject{currentmarker}{\pgfqpoint{0.000000in}{-0.048611in}}{\pgfqpoint{0.000000in}{0.000000in}}{%
\pgfpathmoveto{\pgfqpoint{0.000000in}{0.000000in}}%
\pgfpathlineto{\pgfqpoint{0.000000in}{-0.048611in}}%
\pgfusepath{stroke,fill}%
}%
\begin{pgfscope}%
\pgfsys@transformshift{2.143106in}{1.103099in}%
\pgfsys@useobject{currentmarker}{}%
\end{pgfscope}%
\end{pgfscope}%
\begin{pgfscope}%
\definecolor{textcolor}{rgb}{0.000000,0.000000,0.000000}%
\pgfsetstrokecolor{textcolor}%
\pgfsetfillcolor{textcolor}%
\pgftext[x=2.160467in, y=0.319621in, left, base,rotate=90.000000]{\color{textcolor}\rmfamily\fontsize{5.000000}{6.000000}\selectfont CNP Assurances}%
\end{pgfscope}%
\begin{pgfscope}%
\pgfsetbuttcap%
\pgfsetroundjoin%
\definecolor{currentfill}{rgb}{0.000000,0.000000,0.000000}%
\pgfsetfillcolor{currentfill}%
\pgfsetlinewidth{0.803000pt}%
\definecolor{currentstroke}{rgb}{0.000000,0.000000,0.000000}%
\pgfsetstrokecolor{currentstroke}%
\pgfsetdash{}{0pt}%
\pgfsys@defobject{currentmarker}{\pgfqpoint{0.000000in}{-0.048611in}}{\pgfqpoint{0.000000in}{0.000000in}}{%
\pgfpathmoveto{\pgfqpoint{0.000000in}{0.000000in}}%
\pgfpathlineto{\pgfqpoint{0.000000in}{-0.048611in}}%
\pgfusepath{stroke,fill}%
}%
\begin{pgfscope}%
\pgfsys@transformshift{2.274580in}{1.103099in}%
\pgfsys@useobject{currentmarker}{}%
\end{pgfscope}%
\end{pgfscope}%
\begin{pgfscope}%
\definecolor{textcolor}{rgb}{0.000000,0.000000,0.000000}%
\pgfsetstrokecolor{textcolor}%
\pgfsetfillcolor{textcolor}%
\pgftext[x=2.291941in, y=0.764747in, left, base,rotate=90.000000]{\color{textcolor}\rmfamily\fontsize{5.000000}{6.000000}\selectfont Carac}%
\end{pgfscope}%
\begin{pgfscope}%
\pgfsetbuttcap%
\pgfsetroundjoin%
\definecolor{currentfill}{rgb}{0.000000,0.000000,0.000000}%
\pgfsetfillcolor{currentfill}%
\pgfsetlinewidth{0.803000pt}%
\definecolor{currentstroke}{rgb}{0.000000,0.000000,0.000000}%
\pgfsetstrokecolor{currentstroke}%
\pgfsetdash{}{0pt}%
\pgfsys@defobject{currentmarker}{\pgfqpoint{0.000000in}{-0.048611in}}{\pgfqpoint{0.000000in}{0.000000in}}{%
\pgfpathmoveto{\pgfqpoint{0.000000in}{0.000000in}}%
\pgfpathlineto{\pgfqpoint{0.000000in}{-0.048611in}}%
\pgfusepath{stroke,fill}%
}%
\begin{pgfscope}%
\pgfsys@transformshift{2.406053in}{1.103099in}%
\pgfsys@useobject{currentmarker}{}%
\end{pgfscope}%
\end{pgfscope}%
\begin{pgfscope}%
\definecolor{textcolor}{rgb}{0.000000,0.000000,0.000000}%
\pgfsetstrokecolor{textcolor}%
\pgfsetfillcolor{textcolor}%
\pgftext[x=2.423414in, y=0.744010in, left, base,rotate=90.000000]{\color{textcolor}\rmfamily\fontsize{5.000000}{6.000000}\selectfont Cardif}%
\end{pgfscope}%
\begin{pgfscope}%
\pgfsetbuttcap%
\pgfsetroundjoin%
\definecolor{currentfill}{rgb}{0.000000,0.000000,0.000000}%
\pgfsetfillcolor{currentfill}%
\pgfsetlinewidth{0.803000pt}%
\definecolor{currentstroke}{rgb}{0.000000,0.000000,0.000000}%
\pgfsetstrokecolor{currentstroke}%
\pgfsetdash{}{0pt}%
\pgfsys@defobject{currentmarker}{\pgfqpoint{0.000000in}{-0.048611in}}{\pgfqpoint{0.000000in}{0.000000in}}{%
\pgfpathmoveto{\pgfqpoint{0.000000in}{0.000000in}}%
\pgfpathlineto{\pgfqpoint{0.000000in}{-0.048611in}}%
\pgfusepath{stroke,fill}%
}%
\begin{pgfscope}%
\pgfsys@transformshift{2.537526in}{1.103099in}%
\pgfsys@useobject{currentmarker}{}%
\end{pgfscope}%
\end{pgfscope}%
\begin{pgfscope}%
\definecolor{textcolor}{rgb}{0.000000,0.000000,0.000000}%
\pgfsetstrokecolor{textcolor}%
\pgfsetfillcolor{textcolor}%
\pgftext[x=2.554887in, y=0.194619in, left, base,rotate=90.000000]{\color{textcolor}\rmfamily\fontsize{5.000000}{6.000000}\selectfont Cegema Assurances}%
\end{pgfscope}%
\begin{pgfscope}%
\pgfsetbuttcap%
\pgfsetroundjoin%
\definecolor{currentfill}{rgb}{0.000000,0.000000,0.000000}%
\pgfsetfillcolor{currentfill}%
\pgfsetlinewidth{0.803000pt}%
\definecolor{currentstroke}{rgb}{0.000000,0.000000,0.000000}%
\pgfsetstrokecolor{currentstroke}%
\pgfsetdash{}{0pt}%
\pgfsys@defobject{currentmarker}{\pgfqpoint{0.000000in}{-0.048611in}}{\pgfqpoint{0.000000in}{0.000000in}}{%
\pgfpathmoveto{\pgfqpoint{0.000000in}{0.000000in}}%
\pgfpathlineto{\pgfqpoint{0.000000in}{-0.048611in}}%
\pgfusepath{stroke,fill}%
}%
\begin{pgfscope}%
\pgfsys@transformshift{2.668999in}{1.103099in}%
\pgfsys@useobject{currentmarker}{}%
\end{pgfscope}%
\end{pgfscope}%
\begin{pgfscope}%
\definecolor{textcolor}{rgb}{0.000000,0.000000,0.000000}%
\pgfsetstrokecolor{textcolor}%
\pgfsetfillcolor{textcolor}%
\pgftext[x=2.686360in, y=0.414048in, left, base,rotate=90.000000]{\color{textcolor}\rmfamily\fontsize{5.000000}{6.000000}\selectfont Crédit Mutuel}%
\end{pgfscope}%
\begin{pgfscope}%
\pgfsetbuttcap%
\pgfsetroundjoin%
\definecolor{currentfill}{rgb}{0.000000,0.000000,0.000000}%
\pgfsetfillcolor{currentfill}%
\pgfsetlinewidth{0.803000pt}%
\definecolor{currentstroke}{rgb}{0.000000,0.000000,0.000000}%
\pgfsetstrokecolor{currentstroke}%
\pgfsetdash{}{0pt}%
\pgfsys@defobject{currentmarker}{\pgfqpoint{0.000000in}{-0.048611in}}{\pgfqpoint{0.000000in}{0.000000in}}{%
\pgfpathmoveto{\pgfqpoint{0.000000in}{0.000000in}}%
\pgfpathlineto{\pgfqpoint{0.000000in}{-0.048611in}}%
\pgfusepath{stroke,fill}%
}%
\begin{pgfscope}%
\pgfsys@transformshift{2.800472in}{1.103099in}%
\pgfsys@useobject{currentmarker}{}%
\end{pgfscope}%
\end{pgfscope}%
\begin{pgfscope}%
\definecolor{textcolor}{rgb}{0.000000,0.000000,0.000000}%
\pgfsetstrokecolor{textcolor}%
\pgfsetfillcolor{textcolor}%
\pgftext[x=2.817833in, y=0.297052in, left, base,rotate=90.000000]{\color{textcolor}\rmfamily\fontsize{5.000000}{6.000000}\selectfont Direct Assurance}%
\end{pgfscope}%
\begin{pgfscope}%
\pgfsetbuttcap%
\pgfsetroundjoin%
\definecolor{currentfill}{rgb}{0.000000,0.000000,0.000000}%
\pgfsetfillcolor{currentfill}%
\pgfsetlinewidth{0.803000pt}%
\definecolor{currentstroke}{rgb}{0.000000,0.000000,0.000000}%
\pgfsetstrokecolor{currentstroke}%
\pgfsetdash{}{0pt}%
\pgfsys@defobject{currentmarker}{\pgfqpoint{0.000000in}{-0.048611in}}{\pgfqpoint{0.000000in}{0.000000in}}{%
\pgfpathmoveto{\pgfqpoint{0.000000in}{0.000000in}}%
\pgfpathlineto{\pgfqpoint{0.000000in}{-0.048611in}}%
\pgfusepath{stroke,fill}%
}%
\begin{pgfscope}%
\pgfsys@transformshift{2.931946in}{1.103099in}%
\pgfsys@useobject{currentmarker}{}%
\end{pgfscope}%
\end{pgfscope}%
\begin{pgfscope}%
\definecolor{textcolor}{rgb}{0.000000,0.000000,0.000000}%
\pgfsetstrokecolor{textcolor}%
\pgfsetfillcolor{textcolor}%
\pgftext[x=2.949307in, y=0.364568in, left, base,rotate=90.000000]{\color{textcolor}\rmfamily\fontsize{5.000000}{6.000000}\selectfont Eca Assurances}%
\end{pgfscope}%
\begin{pgfscope}%
\pgfsetbuttcap%
\pgfsetroundjoin%
\definecolor{currentfill}{rgb}{0.000000,0.000000,0.000000}%
\pgfsetfillcolor{currentfill}%
\pgfsetlinewidth{0.803000pt}%
\definecolor{currentstroke}{rgb}{0.000000,0.000000,0.000000}%
\pgfsetstrokecolor{currentstroke}%
\pgfsetdash{}{0pt}%
\pgfsys@defobject{currentmarker}{\pgfqpoint{0.000000in}{-0.048611in}}{\pgfqpoint{0.000000in}{0.000000in}}{%
\pgfpathmoveto{\pgfqpoint{0.000000in}{0.000000in}}%
\pgfpathlineto{\pgfqpoint{0.000000in}{-0.048611in}}%
\pgfusepath{stroke,fill}%
}%
\begin{pgfscope}%
\pgfsys@transformshift{3.063419in}{1.103099in}%
\pgfsys@useobject{currentmarker}{}%
\end{pgfscope}%
\end{pgfscope}%
\begin{pgfscope}%
\definecolor{textcolor}{rgb}{0.000000,0.000000,0.000000}%
\pgfsetstrokecolor{textcolor}%
\pgfsetfillcolor{textcolor}%
\pgftext[x=3.080780in, y=0.355405in, left, base,rotate=90.000000]{\color{textcolor}\rmfamily\fontsize{5.000000}{6.000000}\selectfont Euro-Assurance}%
\end{pgfscope}%
\begin{pgfscope}%
\pgfsetbuttcap%
\pgfsetroundjoin%
\definecolor{currentfill}{rgb}{0.000000,0.000000,0.000000}%
\pgfsetfillcolor{currentfill}%
\pgfsetlinewidth{0.803000pt}%
\definecolor{currentstroke}{rgb}{0.000000,0.000000,0.000000}%
\pgfsetstrokecolor{currentstroke}%
\pgfsetdash{}{0pt}%
\pgfsys@defobject{currentmarker}{\pgfqpoint{0.000000in}{-0.048611in}}{\pgfqpoint{0.000000in}{0.000000in}}{%
\pgfpathmoveto{\pgfqpoint{0.000000in}{0.000000in}}%
\pgfpathlineto{\pgfqpoint{0.000000in}{-0.048611in}}%
\pgfusepath{stroke,fill}%
}%
\begin{pgfscope}%
\pgfsys@transformshift{3.194892in}{1.103099in}%
\pgfsys@useobject{currentmarker}{}%
\end{pgfscope}%
\end{pgfscope}%
\begin{pgfscope}%
\definecolor{textcolor}{rgb}{0.000000,0.000000,0.000000}%
\pgfsetstrokecolor{textcolor}%
\pgfsetfillcolor{textcolor}%
\pgftext[x=3.212253in, y=0.720090in, left, base,rotate=90.000000]{\color{textcolor}\rmfamily\fontsize{5.000000}{6.000000}\selectfont Eurofil}%
\end{pgfscope}%
\begin{pgfscope}%
\pgfsetbuttcap%
\pgfsetroundjoin%
\definecolor{currentfill}{rgb}{0.000000,0.000000,0.000000}%
\pgfsetfillcolor{currentfill}%
\pgfsetlinewidth{0.803000pt}%
\definecolor{currentstroke}{rgb}{0.000000,0.000000,0.000000}%
\pgfsetstrokecolor{currentstroke}%
\pgfsetdash{}{0pt}%
\pgfsys@defobject{currentmarker}{\pgfqpoint{0.000000in}{-0.048611in}}{\pgfqpoint{0.000000in}{0.000000in}}{%
\pgfpathmoveto{\pgfqpoint{0.000000in}{0.000000in}}%
\pgfpathlineto{\pgfqpoint{0.000000in}{-0.048611in}}%
\pgfusepath{stroke,fill}%
}%
\begin{pgfscope}%
\pgfsys@transformshift{3.326365in}{1.103099in}%
\pgfsys@useobject{currentmarker}{}%
\end{pgfscope}%
\end{pgfscope}%
\begin{pgfscope}%
\definecolor{textcolor}{rgb}{0.000000,0.000000,0.000000}%
\pgfsetstrokecolor{textcolor}%
\pgfsetfillcolor{textcolor}%
\pgftext[x=3.343726in, y=0.791463in, left, base,rotate=90.000000]{\color{textcolor}\rmfamily\fontsize{5.000000}{6.000000}\selectfont GMF}%
\end{pgfscope}%
\begin{pgfscope}%
\pgfsetbuttcap%
\pgfsetroundjoin%
\definecolor{currentfill}{rgb}{0.000000,0.000000,0.000000}%
\pgfsetfillcolor{currentfill}%
\pgfsetlinewidth{0.803000pt}%
\definecolor{currentstroke}{rgb}{0.000000,0.000000,0.000000}%
\pgfsetstrokecolor{currentstroke}%
\pgfsetdash{}{0pt}%
\pgfsys@defobject{currentmarker}{\pgfqpoint{0.000000in}{-0.048611in}}{\pgfqpoint{0.000000in}{0.000000in}}{%
\pgfpathmoveto{\pgfqpoint{0.000000in}{0.000000in}}%
\pgfpathlineto{\pgfqpoint{0.000000in}{-0.048611in}}%
\pgfusepath{stroke,fill}%
}%
\begin{pgfscope}%
\pgfsys@transformshift{3.457838in}{1.103099in}%
\pgfsys@useobject{currentmarker}{}%
\end{pgfscope}%
\end{pgfscope}%
\begin{pgfscope}%
\definecolor{textcolor}{rgb}{0.000000,0.000000,0.000000}%
\pgfsetstrokecolor{textcolor}%
\pgfsetfillcolor{textcolor}%
\pgftext[x=3.475200in, y=0.834770in, left, base,rotate=90.000000]{\color{textcolor}\rmfamily\fontsize{5.000000}{6.000000}\selectfont Gan}%
\end{pgfscope}%
\begin{pgfscope}%
\pgfsetbuttcap%
\pgfsetroundjoin%
\definecolor{currentfill}{rgb}{0.000000,0.000000,0.000000}%
\pgfsetfillcolor{currentfill}%
\pgfsetlinewidth{0.803000pt}%
\definecolor{currentstroke}{rgb}{0.000000,0.000000,0.000000}%
\pgfsetstrokecolor{currentstroke}%
\pgfsetdash{}{0pt}%
\pgfsys@defobject{currentmarker}{\pgfqpoint{0.000000in}{-0.048611in}}{\pgfqpoint{0.000000in}{0.000000in}}{%
\pgfpathmoveto{\pgfqpoint{0.000000in}{0.000000in}}%
\pgfpathlineto{\pgfqpoint{0.000000in}{-0.048611in}}%
\pgfusepath{stroke,fill}%
}%
\begin{pgfscope}%
\pgfsys@transformshift{3.589312in}{1.103099in}%
\pgfsys@useobject{currentmarker}{}%
\end{pgfscope}%
\end{pgfscope}%
\begin{pgfscope}%
\definecolor{textcolor}{rgb}{0.000000,0.000000,0.000000}%
\pgfsetstrokecolor{textcolor}%
\pgfsetfillcolor{textcolor}%
\pgftext[x=3.606673in, y=0.656335in, left, base,rotate=90.000000]{\color{textcolor}\rmfamily\fontsize{5.000000}{6.000000}\selectfont Generali}%
\end{pgfscope}%
\begin{pgfscope}%
\pgfsetbuttcap%
\pgfsetroundjoin%
\definecolor{currentfill}{rgb}{0.000000,0.000000,0.000000}%
\pgfsetfillcolor{currentfill}%
\pgfsetlinewidth{0.803000pt}%
\definecolor{currentstroke}{rgb}{0.000000,0.000000,0.000000}%
\pgfsetstrokecolor{currentstroke}%
\pgfsetdash{}{0pt}%
\pgfsys@defobject{currentmarker}{\pgfqpoint{0.000000in}{-0.048611in}}{\pgfqpoint{0.000000in}{0.000000in}}{%
\pgfpathmoveto{\pgfqpoint{0.000000in}{0.000000in}}%
\pgfpathlineto{\pgfqpoint{0.000000in}{-0.048611in}}%
\pgfusepath{stroke,fill}%
}%
\begin{pgfscope}%
\pgfsys@transformshift{3.720785in}{1.103099in}%
\pgfsys@useobject{currentmarker}{}%
\end{pgfscope}%
\end{pgfscope}%
\begin{pgfscope}%
\definecolor{textcolor}{rgb}{0.000000,0.000000,0.000000}%
\pgfsetstrokecolor{textcolor}%
\pgfsetfillcolor{textcolor}%
\pgftext[x=3.738146in, y=0.574349in, left, base,rotate=90.000000]{\color{textcolor}\rmfamily\fontsize{5.000000}{6.000000}\selectfont Groupama}%
\end{pgfscope}%
\begin{pgfscope}%
\pgfsetbuttcap%
\pgfsetroundjoin%
\definecolor{currentfill}{rgb}{0.000000,0.000000,0.000000}%
\pgfsetfillcolor{currentfill}%
\pgfsetlinewidth{0.803000pt}%
\definecolor{currentstroke}{rgb}{0.000000,0.000000,0.000000}%
\pgfsetstrokecolor{currentstroke}%
\pgfsetdash{}{0pt}%
\pgfsys@defobject{currentmarker}{\pgfqpoint{0.000000in}{-0.048611in}}{\pgfqpoint{0.000000in}{0.000000in}}{%
\pgfpathmoveto{\pgfqpoint{0.000000in}{0.000000in}}%
\pgfpathlineto{\pgfqpoint{0.000000in}{-0.048611in}}%
\pgfusepath{stroke,fill}%
}%
\begin{pgfscope}%
\pgfsys@transformshift{3.852258in}{1.103099in}%
\pgfsys@useobject{currentmarker}{}%
\end{pgfscope}%
\end{pgfscope}%
\begin{pgfscope}%
\definecolor{textcolor}{rgb}{0.000000,0.000000,0.000000}%
\pgfsetstrokecolor{textcolor}%
\pgfsetfillcolor{textcolor}%
\pgftext[x=3.869619in, y=0.547344in, left, base,rotate=90.000000]{\color{textcolor}\rmfamily\fontsize{5.000000}{6.000000}\selectfont Génération}%
\end{pgfscope}%
\begin{pgfscope}%
\pgfsetbuttcap%
\pgfsetroundjoin%
\definecolor{currentfill}{rgb}{0.000000,0.000000,0.000000}%
\pgfsetfillcolor{currentfill}%
\pgfsetlinewidth{0.803000pt}%
\definecolor{currentstroke}{rgb}{0.000000,0.000000,0.000000}%
\pgfsetstrokecolor{currentstroke}%
\pgfsetdash{}{0pt}%
\pgfsys@defobject{currentmarker}{\pgfqpoint{0.000000in}{-0.048611in}}{\pgfqpoint{0.000000in}{0.000000in}}{%
\pgfpathmoveto{\pgfqpoint{0.000000in}{0.000000in}}%
\pgfpathlineto{\pgfqpoint{0.000000in}{-0.048611in}}%
\pgfusepath{stroke,fill}%
}%
\begin{pgfscope}%
\pgfsys@transformshift{3.983731in}{1.103099in}%
\pgfsys@useobject{currentmarker}{}%
\end{pgfscope}%
\end{pgfscope}%
\begin{pgfscope}%
\definecolor{textcolor}{rgb}{0.000000,0.000000,0.000000}%
\pgfsetstrokecolor{textcolor}%
\pgfsetfillcolor{textcolor}%
\pgftext[x=4.001092in, y=0.208990in, left, base,rotate=90.000000]{\color{textcolor}\rmfamily\fontsize{5.000000}{6.000000}\selectfont Harmonie Mutuelle}%
\end{pgfscope}%
\begin{pgfscope}%
\pgfsetbuttcap%
\pgfsetroundjoin%
\definecolor{currentfill}{rgb}{0.000000,0.000000,0.000000}%
\pgfsetfillcolor{currentfill}%
\pgfsetlinewidth{0.803000pt}%
\definecolor{currentstroke}{rgb}{0.000000,0.000000,0.000000}%
\pgfsetstrokecolor{currentstroke}%
\pgfsetdash{}{0pt}%
\pgfsys@defobject{currentmarker}{\pgfqpoint{0.000000in}{-0.048611in}}{\pgfqpoint{0.000000in}{0.000000in}}{%
\pgfpathmoveto{\pgfqpoint{0.000000in}{0.000000in}}%
\pgfpathlineto{\pgfqpoint{0.000000in}{-0.048611in}}%
\pgfusepath{stroke,fill}%
}%
\begin{pgfscope}%
\pgfsys@transformshift{4.115205in}{1.103099in}%
\pgfsys@useobject{currentmarker}{}%
\end{pgfscope}%
\end{pgfscope}%
\begin{pgfscope}%
\definecolor{textcolor}{rgb}{0.000000,0.000000,0.000000}%
\pgfsetstrokecolor{textcolor}%
\pgfsetfillcolor{textcolor}%
\pgftext[x=4.132566in, y=0.734750in, left, base,rotate=90.000000]{\color{textcolor}\rmfamily\fontsize{5.000000}{6.000000}\selectfont Hiscox}%
\end{pgfscope}%
\begin{pgfscope}%
\pgfsetbuttcap%
\pgfsetroundjoin%
\definecolor{currentfill}{rgb}{0.000000,0.000000,0.000000}%
\pgfsetfillcolor{currentfill}%
\pgfsetlinewidth{0.803000pt}%
\definecolor{currentstroke}{rgb}{0.000000,0.000000,0.000000}%
\pgfsetstrokecolor{currentstroke}%
\pgfsetdash{}{0pt}%
\pgfsys@defobject{currentmarker}{\pgfqpoint{0.000000in}{-0.048611in}}{\pgfqpoint{0.000000in}{0.000000in}}{%
\pgfpathmoveto{\pgfqpoint{0.000000in}{0.000000in}}%
\pgfpathlineto{\pgfqpoint{0.000000in}{-0.048611in}}%
\pgfusepath{stroke,fill}%
}%
\begin{pgfscope}%
\pgfsys@transformshift{4.246678in}{1.103099in}%
\pgfsys@useobject{currentmarker}{}%
\end{pgfscope}%
\end{pgfscope}%
\begin{pgfscope}%
\definecolor{textcolor}{rgb}{0.000000,0.000000,0.000000}%
\pgfsetstrokecolor{textcolor}%
\pgfsetfillcolor{textcolor}%
\pgftext[x=4.264039in, y=0.658554in, left, base,rotate=90.000000]{\color{textcolor}\rmfamily\fontsize{5.000000}{6.000000}\selectfont Intériale}%
\end{pgfscope}%
\begin{pgfscope}%
\pgfsetbuttcap%
\pgfsetroundjoin%
\definecolor{currentfill}{rgb}{0.000000,0.000000,0.000000}%
\pgfsetfillcolor{currentfill}%
\pgfsetlinewidth{0.803000pt}%
\definecolor{currentstroke}{rgb}{0.000000,0.000000,0.000000}%
\pgfsetstrokecolor{currentstroke}%
\pgfsetdash{}{0pt}%
\pgfsys@defobject{currentmarker}{\pgfqpoint{0.000000in}{-0.048611in}}{\pgfqpoint{0.000000in}{0.000000in}}{%
\pgfpathmoveto{\pgfqpoint{0.000000in}{0.000000in}}%
\pgfpathlineto{\pgfqpoint{0.000000in}{-0.048611in}}%
\pgfusepath{stroke,fill}%
}%
\begin{pgfscope}%
\pgfsys@transformshift{4.378151in}{1.103099in}%
\pgfsys@useobject{currentmarker}{}%
\end{pgfscope}%
\end{pgfscope}%
\begin{pgfscope}%
\definecolor{textcolor}{rgb}{0.000000,0.000000,0.000000}%
\pgfsetstrokecolor{textcolor}%
\pgfsetfillcolor{textcolor}%
\pgftext[x=4.395512in, y=0.208316in, left, base,rotate=90.000000]{\color{textcolor}\rmfamily\fontsize{5.000000}{6.000000}\selectfont L'olivier Assurance}%
\end{pgfscope}%
\begin{pgfscope}%
\pgfsetbuttcap%
\pgfsetroundjoin%
\definecolor{currentfill}{rgb}{0.000000,0.000000,0.000000}%
\pgfsetfillcolor{currentfill}%
\pgfsetlinewidth{0.803000pt}%
\definecolor{currentstroke}{rgb}{0.000000,0.000000,0.000000}%
\pgfsetstrokecolor{currentstroke}%
\pgfsetdash{}{0pt}%
\pgfsys@defobject{currentmarker}{\pgfqpoint{0.000000in}{-0.048611in}}{\pgfqpoint{0.000000in}{0.000000in}}{%
\pgfpathmoveto{\pgfqpoint{0.000000in}{0.000000in}}%
\pgfpathlineto{\pgfqpoint{0.000000in}{-0.048611in}}%
\pgfusepath{stroke,fill}%
}%
\begin{pgfscope}%
\pgfsys@transformshift{4.509624in}{1.103099in}%
\pgfsys@useobject{currentmarker}{}%
\end{pgfscope}%
\end{pgfscope}%
\begin{pgfscope}%
\definecolor{textcolor}{rgb}{0.000000,0.000000,0.000000}%
\pgfsetstrokecolor{textcolor}%
\pgfsetfillcolor{textcolor}%
\pgftext[x=4.526985in, y=0.823968in, left, base,rotate=90.000000]{\color{textcolor}\rmfamily\fontsize{5.000000}{6.000000}\selectfont LCL}%
\end{pgfscope}%
\begin{pgfscope}%
\pgfsetbuttcap%
\pgfsetroundjoin%
\definecolor{currentfill}{rgb}{0.000000,0.000000,0.000000}%
\pgfsetfillcolor{currentfill}%
\pgfsetlinewidth{0.803000pt}%
\definecolor{currentstroke}{rgb}{0.000000,0.000000,0.000000}%
\pgfsetstrokecolor{currentstroke}%
\pgfsetdash{}{0pt}%
\pgfsys@defobject{currentmarker}{\pgfqpoint{0.000000in}{-0.048611in}}{\pgfqpoint{0.000000in}{0.000000in}}{%
\pgfpathmoveto{\pgfqpoint{0.000000in}{0.000000in}}%
\pgfpathlineto{\pgfqpoint{0.000000in}{-0.048611in}}%
\pgfusepath{stroke,fill}%
}%
\begin{pgfscope}%
\pgfsys@transformshift{4.641097in}{1.103099in}%
\pgfsys@useobject{currentmarker}{}%
\end{pgfscope}%
\end{pgfscope}%
\begin{pgfscope}%
\definecolor{textcolor}{rgb}{0.000000,0.000000,0.000000}%
\pgfsetstrokecolor{textcolor}%
\pgfsetfillcolor{textcolor}%
\pgftext[x=4.658458in, y=0.727033in, left, base,rotate=90.000000]{\color{textcolor}\rmfamily\fontsize{5.000000}{6.000000}\selectfont MAAF}%
\end{pgfscope}%
\begin{pgfscope}%
\pgfsetbuttcap%
\pgfsetroundjoin%
\definecolor{currentfill}{rgb}{0.000000,0.000000,0.000000}%
\pgfsetfillcolor{currentfill}%
\pgfsetlinewidth{0.803000pt}%
\definecolor{currentstroke}{rgb}{0.000000,0.000000,0.000000}%
\pgfsetstrokecolor{currentstroke}%
\pgfsetdash{}{0pt}%
\pgfsys@defobject{currentmarker}{\pgfqpoint{0.000000in}{-0.048611in}}{\pgfqpoint{0.000000in}{0.000000in}}{%
\pgfpathmoveto{\pgfqpoint{0.000000in}{0.000000in}}%
\pgfpathlineto{\pgfqpoint{0.000000in}{-0.048611in}}%
\pgfusepath{stroke,fill}%
}%
\begin{pgfscope}%
\pgfsys@transformshift{4.772571in}{1.103099in}%
\pgfsys@useobject{currentmarker}{}%
\end{pgfscope}%
\end{pgfscope}%
\begin{pgfscope}%
\definecolor{textcolor}{rgb}{0.000000,0.000000,0.000000}%
\pgfsetstrokecolor{textcolor}%
\pgfsetfillcolor{textcolor}%
\pgftext[x=4.789932in, y=0.696652in, left, base,rotate=90.000000]{\color{textcolor}\rmfamily\fontsize{5.000000}{6.000000}\selectfont MACIF}%
\end{pgfscope}%
\begin{pgfscope}%
\pgfsetbuttcap%
\pgfsetroundjoin%
\definecolor{currentfill}{rgb}{0.000000,0.000000,0.000000}%
\pgfsetfillcolor{currentfill}%
\pgfsetlinewidth{0.803000pt}%
\definecolor{currentstroke}{rgb}{0.000000,0.000000,0.000000}%
\pgfsetstrokecolor{currentstroke}%
\pgfsetdash{}{0pt}%
\pgfsys@defobject{currentmarker}{\pgfqpoint{0.000000in}{-0.048611in}}{\pgfqpoint{0.000000in}{0.000000in}}{%
\pgfpathmoveto{\pgfqpoint{0.000000in}{0.000000in}}%
\pgfpathlineto{\pgfqpoint{0.000000in}{-0.048611in}}%
\pgfusepath{stroke,fill}%
}%
\begin{pgfscope}%
\pgfsys@transformshift{4.904044in}{1.103099in}%
\pgfsys@useobject{currentmarker}{}%
\end{pgfscope}%
\end{pgfscope}%
\begin{pgfscope}%
\definecolor{textcolor}{rgb}{0.000000,0.000000,0.000000}%
\pgfsetstrokecolor{textcolor}%
\pgfsetfillcolor{textcolor}%
\pgftext[x=4.921405in, y=0.760792in, left, base,rotate=90.000000]{\color{textcolor}\rmfamily\fontsize{5.000000}{6.000000}\selectfont MAIF}%
\end{pgfscope}%
\begin{pgfscope}%
\pgfsetbuttcap%
\pgfsetroundjoin%
\definecolor{currentfill}{rgb}{0.000000,0.000000,0.000000}%
\pgfsetfillcolor{currentfill}%
\pgfsetlinewidth{0.803000pt}%
\definecolor{currentstroke}{rgb}{0.000000,0.000000,0.000000}%
\pgfsetstrokecolor{currentstroke}%
\pgfsetdash{}{0pt}%
\pgfsys@defobject{currentmarker}{\pgfqpoint{0.000000in}{-0.048611in}}{\pgfqpoint{0.000000in}{0.000000in}}{%
\pgfpathmoveto{\pgfqpoint{0.000000in}{0.000000in}}%
\pgfpathlineto{\pgfqpoint{0.000000in}{-0.048611in}}%
\pgfusepath{stroke,fill}%
}%
\begin{pgfscope}%
\pgfsys@transformshift{5.035517in}{1.103099in}%
\pgfsys@useobject{currentmarker}{}%
\end{pgfscope}%
\end{pgfscope}%
\begin{pgfscope}%
\definecolor{textcolor}{rgb}{0.000000,0.000000,0.000000}%
\pgfsetstrokecolor{textcolor}%
\pgfsetfillcolor{textcolor}%
\pgftext[x=5.052878in, y=0.789052in, left, base,rotate=90.000000]{\color{textcolor}\rmfamily\fontsize{5.000000}{6.000000}\selectfont MGP}%
\end{pgfscope}%
\begin{pgfscope}%
\pgfsetbuttcap%
\pgfsetroundjoin%
\definecolor{currentfill}{rgb}{0.000000,0.000000,0.000000}%
\pgfsetfillcolor{currentfill}%
\pgfsetlinewidth{0.803000pt}%
\definecolor{currentstroke}{rgb}{0.000000,0.000000,0.000000}%
\pgfsetstrokecolor{currentstroke}%
\pgfsetdash{}{0pt}%
\pgfsys@defobject{currentmarker}{\pgfqpoint{0.000000in}{-0.048611in}}{\pgfqpoint{0.000000in}{0.000000in}}{%
\pgfpathmoveto{\pgfqpoint{0.000000in}{0.000000in}}%
\pgfpathlineto{\pgfqpoint{0.000000in}{-0.048611in}}%
\pgfusepath{stroke,fill}%
}%
\begin{pgfscope}%
\pgfsys@transformshift{5.166990in}{1.103099in}%
\pgfsys@useobject{currentmarker}{}%
\end{pgfscope}%
\end{pgfscope}%
\begin{pgfscope}%
\definecolor{textcolor}{rgb}{0.000000,0.000000,0.000000}%
\pgfsetstrokecolor{textcolor}%
\pgfsetfillcolor{textcolor}%
\pgftext[x=5.184351in, y=0.772655in, left, base,rotate=90.000000]{\color{textcolor}\rmfamily\fontsize{5.000000}{6.000000}\selectfont MMA}%
\end{pgfscope}%
\begin{pgfscope}%
\pgfsetbuttcap%
\pgfsetroundjoin%
\definecolor{currentfill}{rgb}{0.000000,0.000000,0.000000}%
\pgfsetfillcolor{currentfill}%
\pgfsetlinewidth{0.803000pt}%
\definecolor{currentstroke}{rgb}{0.000000,0.000000,0.000000}%
\pgfsetstrokecolor{currentstroke}%
\pgfsetdash{}{0pt}%
\pgfsys@defobject{currentmarker}{\pgfqpoint{0.000000in}{-0.048611in}}{\pgfqpoint{0.000000in}{0.000000in}}{%
\pgfpathmoveto{\pgfqpoint{0.000000in}{0.000000in}}%
\pgfpathlineto{\pgfqpoint{0.000000in}{-0.048611in}}%
\pgfusepath{stroke,fill}%
}%
\begin{pgfscope}%
\pgfsys@transformshift{5.298463in}{1.103099in}%
\pgfsys@useobject{currentmarker}{}%
\end{pgfscope}%
\end{pgfscope}%
\begin{pgfscope}%
\definecolor{textcolor}{rgb}{0.000000,0.000000,0.000000}%
\pgfsetstrokecolor{textcolor}%
\pgfsetfillcolor{textcolor}%
\pgftext[x=5.315825in, y=0.626241in, left, base,rotate=90.000000]{\color{textcolor}\rmfamily\fontsize{5.000000}{6.000000}\selectfont Magnolia}%
\end{pgfscope}%
\begin{pgfscope}%
\pgfsetbuttcap%
\pgfsetroundjoin%
\definecolor{currentfill}{rgb}{0.000000,0.000000,0.000000}%
\pgfsetfillcolor{currentfill}%
\pgfsetlinewidth{0.803000pt}%
\definecolor{currentstroke}{rgb}{0.000000,0.000000,0.000000}%
\pgfsetstrokecolor{currentstroke}%
\pgfsetdash{}{0pt}%
\pgfsys@defobject{currentmarker}{\pgfqpoint{0.000000in}{-0.048611in}}{\pgfqpoint{0.000000in}{0.000000in}}{%
\pgfpathmoveto{\pgfqpoint{0.000000in}{0.000000in}}%
\pgfpathlineto{\pgfqpoint{0.000000in}{-0.048611in}}%
\pgfusepath{stroke,fill}%
}%
\begin{pgfscope}%
\pgfsys@transformshift{5.429937in}{1.103099in}%
\pgfsys@useobject{currentmarker}{}%
\end{pgfscope}%
\end{pgfscope}%
\begin{pgfscope}%
\definecolor{textcolor}{rgb}{0.000000,0.000000,0.000000}%
\pgfsetstrokecolor{textcolor}%
\pgfsetfillcolor{textcolor}%
\pgftext[x=5.447298in, y=0.250947in, left, base,rotate=90.000000]{\color{textcolor}\rmfamily\fontsize{5.000000}{6.000000}\selectfont Malakoff Humanis}%
\end{pgfscope}%
\begin{pgfscope}%
\pgfsetbuttcap%
\pgfsetroundjoin%
\definecolor{currentfill}{rgb}{0.000000,0.000000,0.000000}%
\pgfsetfillcolor{currentfill}%
\pgfsetlinewidth{0.803000pt}%
\definecolor{currentstroke}{rgb}{0.000000,0.000000,0.000000}%
\pgfsetstrokecolor{currentstroke}%
\pgfsetdash{}{0pt}%
\pgfsys@defobject{currentmarker}{\pgfqpoint{0.000000in}{-0.048611in}}{\pgfqpoint{0.000000in}{0.000000in}}{%
\pgfpathmoveto{\pgfqpoint{0.000000in}{0.000000in}}%
\pgfpathlineto{\pgfqpoint{0.000000in}{-0.048611in}}%
\pgfusepath{stroke,fill}%
}%
\begin{pgfscope}%
\pgfsys@transformshift{5.561410in}{1.103099in}%
\pgfsys@useobject{currentmarker}{}%
\end{pgfscope}%
\end{pgfscope}%
\begin{pgfscope}%
\definecolor{textcolor}{rgb}{0.000000,0.000000,0.000000}%
\pgfsetstrokecolor{textcolor}%
\pgfsetfillcolor{textcolor}%
\pgftext[x=5.578771in, y=0.776706in, left, base,rotate=90.000000]{\color{textcolor}\rmfamily\fontsize{5.000000}{6.000000}\selectfont Mapa}%
\end{pgfscope}%
\begin{pgfscope}%
\pgfsetbuttcap%
\pgfsetroundjoin%
\definecolor{currentfill}{rgb}{0.000000,0.000000,0.000000}%
\pgfsetfillcolor{currentfill}%
\pgfsetlinewidth{0.803000pt}%
\definecolor{currentstroke}{rgb}{0.000000,0.000000,0.000000}%
\pgfsetstrokecolor{currentstroke}%
\pgfsetdash{}{0pt}%
\pgfsys@defobject{currentmarker}{\pgfqpoint{0.000000in}{-0.048611in}}{\pgfqpoint{0.000000in}{0.000000in}}{%
\pgfpathmoveto{\pgfqpoint{0.000000in}{0.000000in}}%
\pgfpathlineto{\pgfqpoint{0.000000in}{-0.048611in}}%
\pgfusepath{stroke,fill}%
}%
\begin{pgfscope}%
\pgfsys@transformshift{5.692883in}{1.103099in}%
\pgfsys@useobject{currentmarker}{}%
\end{pgfscope}%
\end{pgfscope}%
\begin{pgfscope}%
\definecolor{textcolor}{rgb}{0.000000,0.000000,0.000000}%
\pgfsetstrokecolor{textcolor}%
\pgfsetfillcolor{textcolor}%
\pgftext[x=5.710244in, y=0.674950in, left, base,rotate=90.000000]{\color{textcolor}\rmfamily\fontsize{5.000000}{6.000000}\selectfont Matmut}%
\end{pgfscope}%
\begin{pgfscope}%
\pgfsetbuttcap%
\pgfsetroundjoin%
\definecolor{currentfill}{rgb}{0.000000,0.000000,0.000000}%
\pgfsetfillcolor{currentfill}%
\pgfsetlinewidth{0.803000pt}%
\definecolor{currentstroke}{rgb}{0.000000,0.000000,0.000000}%
\pgfsetstrokecolor{currentstroke}%
\pgfsetdash{}{0pt}%
\pgfsys@defobject{currentmarker}{\pgfqpoint{0.000000in}{-0.048611in}}{\pgfqpoint{0.000000in}{0.000000in}}{%
\pgfpathmoveto{\pgfqpoint{0.000000in}{0.000000in}}%
\pgfpathlineto{\pgfqpoint{0.000000in}{-0.048611in}}%
\pgfusepath{stroke,fill}%
}%
\begin{pgfscope}%
\pgfsys@transformshift{5.824356in}{1.103099in}%
\pgfsys@useobject{currentmarker}{}%
\end{pgfscope}%
\end{pgfscope}%
\begin{pgfscope}%
\definecolor{textcolor}{rgb}{0.000000,0.000000,0.000000}%
\pgfsetstrokecolor{textcolor}%
\pgfsetfillcolor{textcolor}%
\pgftext[x=5.841717in, y=0.720765in, left, base,rotate=90.000000]{\color{textcolor}\rmfamily\fontsize{5.000000}{6.000000}\selectfont Mercer}%
\end{pgfscope}%
\begin{pgfscope}%
\pgfsetbuttcap%
\pgfsetroundjoin%
\definecolor{currentfill}{rgb}{0.000000,0.000000,0.000000}%
\pgfsetfillcolor{currentfill}%
\pgfsetlinewidth{0.803000pt}%
\definecolor{currentstroke}{rgb}{0.000000,0.000000,0.000000}%
\pgfsetstrokecolor{currentstroke}%
\pgfsetdash{}{0pt}%
\pgfsys@defobject{currentmarker}{\pgfqpoint{0.000000in}{-0.048611in}}{\pgfqpoint{0.000000in}{0.000000in}}{%
\pgfpathmoveto{\pgfqpoint{0.000000in}{0.000000in}}%
\pgfpathlineto{\pgfqpoint{0.000000in}{-0.048611in}}%
\pgfusepath{stroke,fill}%
}%
\begin{pgfscope}%
\pgfsys@transformshift{5.955830in}{1.103099in}%
\pgfsys@useobject{currentmarker}{}%
\end{pgfscope}%
\end{pgfscope}%
\begin{pgfscope}%
\definecolor{textcolor}{rgb}{0.000000,0.000000,0.000000}%
\pgfsetstrokecolor{textcolor}%
\pgfsetfillcolor{textcolor}%
\pgftext[x=5.973191in, y=0.684789in, left, base,rotate=90.000000]{\color{textcolor}\rmfamily\fontsize{5.000000}{6.000000}\selectfont MetLife}%
\end{pgfscope}%
\begin{pgfscope}%
\pgfsetbuttcap%
\pgfsetroundjoin%
\definecolor{currentfill}{rgb}{0.000000,0.000000,0.000000}%
\pgfsetfillcolor{currentfill}%
\pgfsetlinewidth{0.803000pt}%
\definecolor{currentstroke}{rgb}{0.000000,0.000000,0.000000}%
\pgfsetstrokecolor{currentstroke}%
\pgfsetdash{}{0pt}%
\pgfsys@defobject{currentmarker}{\pgfqpoint{0.000000in}{-0.048611in}}{\pgfqpoint{0.000000in}{0.000000in}}{%
\pgfpathmoveto{\pgfqpoint{0.000000in}{0.000000in}}%
\pgfpathlineto{\pgfqpoint{0.000000in}{-0.048611in}}%
\pgfusepath{stroke,fill}%
}%
\begin{pgfscope}%
\pgfsys@transformshift{6.087303in}{1.103099in}%
\pgfsys@useobject{currentmarker}{}%
\end{pgfscope}%
\end{pgfscope}%
\begin{pgfscope}%
\definecolor{textcolor}{rgb}{0.000000,0.000000,0.000000}%
\pgfsetstrokecolor{textcolor}%
\pgfsetfillcolor{textcolor}%
\pgftext[x=6.104664in, y=0.781529in, left, base,rotate=90.000000]{\color{textcolor}\rmfamily\fontsize{5.000000}{6.000000}\selectfont Mgen}%
\end{pgfscope}%
\begin{pgfscope}%
\pgfsetbuttcap%
\pgfsetroundjoin%
\definecolor{currentfill}{rgb}{0.000000,0.000000,0.000000}%
\pgfsetfillcolor{currentfill}%
\pgfsetlinewidth{0.803000pt}%
\definecolor{currentstroke}{rgb}{0.000000,0.000000,0.000000}%
\pgfsetstrokecolor{currentstroke}%
\pgfsetdash{}{0pt}%
\pgfsys@defobject{currentmarker}{\pgfqpoint{0.000000in}{-0.048611in}}{\pgfqpoint{0.000000in}{0.000000in}}{%
\pgfpathmoveto{\pgfqpoint{0.000000in}{0.000000in}}%
\pgfpathlineto{\pgfqpoint{0.000000in}{-0.048611in}}%
\pgfusepath{stroke,fill}%
}%
\begin{pgfscope}%
\pgfsys@transformshift{6.218776in}{1.103099in}%
\pgfsys@useobject{currentmarker}{}%
\end{pgfscope}%
\end{pgfscope}%
\begin{pgfscope}%
\definecolor{textcolor}{rgb}{0.000000,0.000000,0.000000}%
\pgfsetstrokecolor{textcolor}%
\pgfsetfillcolor{textcolor}%
\pgftext[x=6.236137in, y=0.100000in, left, base,rotate=90.000000]{\color{textcolor}\rmfamily\fontsize{5.000000}{6.000000}\selectfont Mutuelle des Motards}%
\end{pgfscope}%
\begin{pgfscope}%
\pgfsetbuttcap%
\pgfsetroundjoin%
\definecolor{currentfill}{rgb}{0.000000,0.000000,0.000000}%
\pgfsetfillcolor{currentfill}%
\pgfsetlinewidth{0.803000pt}%
\definecolor{currentstroke}{rgb}{0.000000,0.000000,0.000000}%
\pgfsetstrokecolor{currentstroke}%
\pgfsetdash{}{0pt}%
\pgfsys@defobject{currentmarker}{\pgfqpoint{0.000000in}{-0.048611in}}{\pgfqpoint{0.000000in}{0.000000in}}{%
\pgfpathmoveto{\pgfqpoint{0.000000in}{0.000000in}}%
\pgfpathlineto{\pgfqpoint{0.000000in}{-0.048611in}}%
\pgfusepath{stroke,fill}%
}%
\begin{pgfscope}%
\pgfsys@transformshift{6.350249in}{1.103099in}%
\pgfsys@useobject{currentmarker}{}%
\end{pgfscope}%
\end{pgfscope}%
\begin{pgfscope}%
\definecolor{textcolor}{rgb}{0.000000,0.000000,0.000000}%
\pgfsetstrokecolor{textcolor}%
\pgfsetfillcolor{textcolor}%
\pgftext[x=6.367610in, y=0.388488in, left, base,rotate=90.000000]{\color{textcolor}\rmfamily\fontsize{5.000000}{6.000000}\selectfont Néoliane Santé}%
\end{pgfscope}%
\begin{pgfscope}%
\pgfsetbuttcap%
\pgfsetroundjoin%
\definecolor{currentfill}{rgb}{0.000000,0.000000,0.000000}%
\pgfsetfillcolor{currentfill}%
\pgfsetlinewidth{0.803000pt}%
\definecolor{currentstroke}{rgb}{0.000000,0.000000,0.000000}%
\pgfsetstrokecolor{currentstroke}%
\pgfsetdash{}{0pt}%
\pgfsys@defobject{currentmarker}{\pgfqpoint{0.000000in}{-0.048611in}}{\pgfqpoint{0.000000in}{0.000000in}}{%
\pgfpathmoveto{\pgfqpoint{0.000000in}{0.000000in}}%
\pgfpathlineto{\pgfqpoint{0.000000in}{-0.048611in}}%
\pgfusepath{stroke,fill}%
}%
\begin{pgfscope}%
\pgfsys@transformshift{6.481722in}{1.103099in}%
\pgfsys@useobject{currentmarker}{}%
\end{pgfscope}%
\end{pgfscope}%
\begin{pgfscope}%
\definecolor{textcolor}{rgb}{0.000000,0.000000,0.000000}%
\pgfsetstrokecolor{textcolor}%
\pgfsetfillcolor{textcolor}%
\pgftext[x=6.499083in, y=0.680062in, left, base,rotate=90.000000]{\color{textcolor}\rmfamily\fontsize{5.000000}{6.000000}\selectfont Pacifica}%
\end{pgfscope}%
\begin{pgfscope}%
\pgfsetbuttcap%
\pgfsetroundjoin%
\definecolor{currentfill}{rgb}{0.000000,0.000000,0.000000}%
\pgfsetfillcolor{currentfill}%
\pgfsetlinewidth{0.803000pt}%
\definecolor{currentstroke}{rgb}{0.000000,0.000000,0.000000}%
\pgfsetstrokecolor{currentstroke}%
\pgfsetdash{}{0pt}%
\pgfsys@defobject{currentmarker}{\pgfqpoint{0.000000in}{-0.048611in}}{\pgfqpoint{0.000000in}{0.000000in}}{%
\pgfpathmoveto{\pgfqpoint{0.000000in}{0.000000in}}%
\pgfpathlineto{\pgfqpoint{0.000000in}{-0.048611in}}%
\pgfusepath{stroke,fill}%
}%
\begin{pgfscope}%
\pgfsys@transformshift{6.613196in}{1.103099in}%
\pgfsys@useobject{currentmarker}{}%
\end{pgfscope}%
\end{pgfscope}%
\begin{pgfscope}%
\definecolor{textcolor}{rgb}{0.000000,0.000000,0.000000}%
\pgfsetstrokecolor{textcolor}%
\pgfsetfillcolor{textcolor}%
\pgftext[x=6.630557in, y=0.237252in, left, base,rotate=90.000000]{\color{textcolor}\rmfamily\fontsize{5.000000}{6.000000}\selectfont Peyrac Assurances}%
\end{pgfscope}%
\begin{pgfscope}%
\pgfsetbuttcap%
\pgfsetroundjoin%
\definecolor{currentfill}{rgb}{0.000000,0.000000,0.000000}%
\pgfsetfillcolor{currentfill}%
\pgfsetlinewidth{0.803000pt}%
\definecolor{currentstroke}{rgb}{0.000000,0.000000,0.000000}%
\pgfsetstrokecolor{currentstroke}%
\pgfsetdash{}{0pt}%
\pgfsys@defobject{currentmarker}{\pgfqpoint{0.000000in}{-0.048611in}}{\pgfqpoint{0.000000in}{0.000000in}}{%
\pgfpathmoveto{\pgfqpoint{0.000000in}{0.000000in}}%
\pgfpathlineto{\pgfqpoint{0.000000in}{-0.048611in}}%
\pgfusepath{stroke,fill}%
}%
\begin{pgfscope}%
\pgfsys@transformshift{6.744669in}{1.103099in}%
\pgfsys@useobject{currentmarker}{}%
\end{pgfscope}%
\end{pgfscope}%
\begin{pgfscope}%
\definecolor{textcolor}{rgb}{0.000000,0.000000,0.000000}%
\pgfsetstrokecolor{textcolor}%
\pgfsetfillcolor{textcolor}%
\pgftext[x=6.762030in, y=0.649487in, left, base,rotate=90.000000]{\color{textcolor}\rmfamily\fontsize{5.000000}{6.000000}\selectfont Santiane}%
\end{pgfscope}%
\begin{pgfscope}%
\pgfsetbuttcap%
\pgfsetroundjoin%
\definecolor{currentfill}{rgb}{0.000000,0.000000,0.000000}%
\pgfsetfillcolor{currentfill}%
\pgfsetlinewidth{0.803000pt}%
\definecolor{currentstroke}{rgb}{0.000000,0.000000,0.000000}%
\pgfsetstrokecolor{currentstroke}%
\pgfsetdash{}{0pt}%
\pgfsys@defobject{currentmarker}{\pgfqpoint{0.000000in}{-0.048611in}}{\pgfqpoint{0.000000in}{0.000000in}}{%
\pgfpathmoveto{\pgfqpoint{0.000000in}{0.000000in}}%
\pgfpathlineto{\pgfqpoint{0.000000in}{-0.048611in}}%
\pgfusepath{stroke,fill}%
}%
\begin{pgfscope}%
\pgfsys@transformshift{6.876142in}{1.103099in}%
\pgfsys@useobject{currentmarker}{}%
\end{pgfscope}%
\end{pgfscope}%
\begin{pgfscope}%
\definecolor{textcolor}{rgb}{0.000000,0.000000,0.000000}%
\pgfsetstrokecolor{textcolor}%
\pgfsetfillcolor{textcolor}%
\pgftext[x=6.893503in, y=0.635887in, left, base,rotate=90.000000]{\color{textcolor}\rmfamily\fontsize{5.000000}{6.000000}\selectfont SantéVet}%
\end{pgfscope}%
\begin{pgfscope}%
\pgfsetbuttcap%
\pgfsetroundjoin%
\definecolor{currentfill}{rgb}{0.000000,0.000000,0.000000}%
\pgfsetfillcolor{currentfill}%
\pgfsetlinewidth{0.803000pt}%
\definecolor{currentstroke}{rgb}{0.000000,0.000000,0.000000}%
\pgfsetstrokecolor{currentstroke}%
\pgfsetdash{}{0pt}%
\pgfsys@defobject{currentmarker}{\pgfqpoint{0.000000in}{-0.048611in}}{\pgfqpoint{0.000000in}{0.000000in}}{%
\pgfpathmoveto{\pgfqpoint{0.000000in}{0.000000in}}%
\pgfpathlineto{\pgfqpoint{0.000000in}{-0.048611in}}%
\pgfusepath{stroke,fill}%
}%
\begin{pgfscope}%
\pgfsys@transformshift{7.007615in}{1.103099in}%
\pgfsys@useobject{currentmarker}{}%
\end{pgfscope}%
\end{pgfscope}%
\begin{pgfscope}%
\definecolor{textcolor}{rgb}{0.000000,0.000000,0.000000}%
\pgfsetstrokecolor{textcolor}%
\pgfsetfillcolor{textcolor}%
\pgftext[x=7.024976in, y=0.830333in, left, base,rotate=90.000000]{\color{textcolor}\rmfamily\fontsize{5.000000}{6.000000}\selectfont Sma}%
\end{pgfscope}%
\begin{pgfscope}%
\pgfsetbuttcap%
\pgfsetroundjoin%
\definecolor{currentfill}{rgb}{0.000000,0.000000,0.000000}%
\pgfsetfillcolor{currentfill}%
\pgfsetlinewidth{0.803000pt}%
\definecolor{currentstroke}{rgb}{0.000000,0.000000,0.000000}%
\pgfsetstrokecolor{currentstroke}%
\pgfsetdash{}{0pt}%
\pgfsys@defobject{currentmarker}{\pgfqpoint{0.000000in}{-0.048611in}}{\pgfqpoint{0.000000in}{0.000000in}}{%
\pgfpathmoveto{\pgfqpoint{0.000000in}{0.000000in}}%
\pgfpathlineto{\pgfqpoint{0.000000in}{-0.048611in}}%
\pgfusepath{stroke,fill}%
}%
\begin{pgfscope}%
\pgfsys@transformshift{7.139088in}{1.103099in}%
\pgfsys@useobject{currentmarker}{}%
\end{pgfscope}%
\end{pgfscope}%
\begin{pgfscope}%
\definecolor{textcolor}{rgb}{0.000000,0.000000,0.000000}%
\pgfsetstrokecolor{textcolor}%
\pgfsetfillcolor{textcolor}%
\pgftext[x=7.156450in, y=0.675046in, left, base,rotate=90.000000]{\color{textcolor}\rmfamily\fontsize{5.000000}{6.000000}\selectfont Sogecap}%
\end{pgfscope}%
\begin{pgfscope}%
\pgfsetbuttcap%
\pgfsetroundjoin%
\definecolor{currentfill}{rgb}{0.000000,0.000000,0.000000}%
\pgfsetfillcolor{currentfill}%
\pgfsetlinewidth{0.803000pt}%
\definecolor{currentstroke}{rgb}{0.000000,0.000000,0.000000}%
\pgfsetstrokecolor{currentstroke}%
\pgfsetdash{}{0pt}%
\pgfsys@defobject{currentmarker}{\pgfqpoint{0.000000in}{-0.048611in}}{\pgfqpoint{0.000000in}{0.000000in}}{%
\pgfpathmoveto{\pgfqpoint{0.000000in}{0.000000in}}%
\pgfpathlineto{\pgfqpoint{0.000000in}{-0.048611in}}%
\pgfusepath{stroke,fill}%
}%
\begin{pgfscope}%
\pgfsys@transformshift{7.270562in}{1.103099in}%
\pgfsys@useobject{currentmarker}{}%
\end{pgfscope}%
\end{pgfscope}%
\begin{pgfscope}%
\definecolor{textcolor}{rgb}{0.000000,0.000000,0.000000}%
\pgfsetstrokecolor{textcolor}%
\pgfsetfillcolor{textcolor}%
\pgftext[x=7.287923in, y=0.650933in, left, base,rotate=90.000000]{\color{textcolor}\rmfamily\fontsize{5.000000}{6.000000}\selectfont Sogessur}%
\end{pgfscope}%
\begin{pgfscope}%
\pgfsetbuttcap%
\pgfsetroundjoin%
\definecolor{currentfill}{rgb}{0.000000,0.000000,0.000000}%
\pgfsetfillcolor{currentfill}%
\pgfsetlinewidth{0.803000pt}%
\definecolor{currentstroke}{rgb}{0.000000,0.000000,0.000000}%
\pgfsetstrokecolor{currentstroke}%
\pgfsetdash{}{0pt}%
\pgfsys@defobject{currentmarker}{\pgfqpoint{0.000000in}{-0.048611in}}{\pgfqpoint{0.000000in}{0.000000in}}{%
\pgfpathmoveto{\pgfqpoint{0.000000in}{0.000000in}}%
\pgfpathlineto{\pgfqpoint{0.000000in}{-0.048611in}}%
\pgfusepath{stroke,fill}%
}%
\begin{pgfscope}%
\pgfsys@transformshift{7.402035in}{1.103099in}%
\pgfsys@useobject{currentmarker}{}%
\end{pgfscope}%
\end{pgfscope}%
\begin{pgfscope}%
\definecolor{textcolor}{rgb}{0.000000,0.000000,0.000000}%
\pgfsetstrokecolor{textcolor}%
\pgfsetfillcolor{textcolor}%
\pgftext[x=7.419396in, y=0.572711in, left, base,rotate=90.000000]{\color{textcolor}\rmfamily\fontsize{5.000000}{6.000000}\selectfont Solly Azar}%
\end{pgfscope}%
\begin{pgfscope}%
\pgfsetbuttcap%
\pgfsetroundjoin%
\definecolor{currentfill}{rgb}{0.000000,0.000000,0.000000}%
\pgfsetfillcolor{currentfill}%
\pgfsetlinewidth{0.803000pt}%
\definecolor{currentstroke}{rgb}{0.000000,0.000000,0.000000}%
\pgfsetstrokecolor{currentstroke}%
\pgfsetdash{}{0pt}%
\pgfsys@defobject{currentmarker}{\pgfqpoint{0.000000in}{-0.048611in}}{\pgfqpoint{0.000000in}{0.000000in}}{%
\pgfpathmoveto{\pgfqpoint{0.000000in}{0.000000in}}%
\pgfpathlineto{\pgfqpoint{0.000000in}{-0.048611in}}%
\pgfusepath{stroke,fill}%
}%
\begin{pgfscope}%
\pgfsys@transformshift{7.533508in}{1.103099in}%
\pgfsys@useobject{currentmarker}{}%
\end{pgfscope}%
\end{pgfscope}%
\begin{pgfscope}%
\definecolor{textcolor}{rgb}{0.000000,0.000000,0.000000}%
\pgfsetstrokecolor{textcolor}%
\pgfsetfillcolor{textcolor}%
\pgftext[x=7.550869in, y=0.611871in, left, base,rotate=90.000000]{\color{textcolor}\rmfamily\fontsize{5.000000}{6.000000}\selectfont Suravenir}%
\end{pgfscope}%
\begin{pgfscope}%
\pgfsetbuttcap%
\pgfsetroundjoin%
\definecolor{currentfill}{rgb}{0.000000,0.000000,0.000000}%
\pgfsetfillcolor{currentfill}%
\pgfsetlinewidth{0.803000pt}%
\definecolor{currentstroke}{rgb}{0.000000,0.000000,0.000000}%
\pgfsetstrokecolor{currentstroke}%
\pgfsetdash{}{0pt}%
\pgfsys@defobject{currentmarker}{\pgfqpoint{0.000000in}{-0.048611in}}{\pgfqpoint{0.000000in}{0.000000in}}{%
\pgfpathmoveto{\pgfqpoint{0.000000in}{0.000000in}}%
\pgfpathlineto{\pgfqpoint{0.000000in}{-0.048611in}}%
\pgfusepath{stroke,fill}%
}%
\begin{pgfscope}%
\pgfsys@transformshift{7.664981in}{1.103099in}%
\pgfsys@useobject{currentmarker}{}%
\end{pgfscope}%
\end{pgfscope}%
\begin{pgfscope}%
\definecolor{textcolor}{rgb}{0.000000,0.000000,0.000000}%
\pgfsetstrokecolor{textcolor}%
\pgfsetfillcolor{textcolor}%
\pgftext[x=7.682342in, y=0.624602in, left, base,rotate=90.000000]{\color{textcolor}\rmfamily\fontsize{5.000000}{6.000000}\selectfont SwissLife}%
\end{pgfscope}%
\begin{pgfscope}%
\pgfsetbuttcap%
\pgfsetroundjoin%
\definecolor{currentfill}{rgb}{0.000000,0.000000,0.000000}%
\pgfsetfillcolor{currentfill}%
\pgfsetlinewidth{0.803000pt}%
\definecolor{currentstroke}{rgb}{0.000000,0.000000,0.000000}%
\pgfsetstrokecolor{currentstroke}%
\pgfsetdash{}{0pt}%
\pgfsys@defobject{currentmarker}{\pgfqpoint{0.000000in}{-0.048611in}}{\pgfqpoint{0.000000in}{0.000000in}}{%
\pgfpathmoveto{\pgfqpoint{0.000000in}{0.000000in}}%
\pgfpathlineto{\pgfqpoint{0.000000in}{-0.048611in}}%
\pgfusepath{stroke,fill}%
}%
\begin{pgfscope}%
\pgfsys@transformshift{7.796455in}{1.103099in}%
\pgfsys@useobject{currentmarker}{}%
\end{pgfscope}%
\end{pgfscope}%
\begin{pgfscope}%
\definecolor{textcolor}{rgb}{0.000000,0.000000,0.000000}%
\pgfsetstrokecolor{textcolor}%
\pgfsetfillcolor{textcolor}%
\pgftext[x=7.813816in, y=0.706296in, left, base,rotate=90.000000]{\color{textcolor}\rmfamily\fontsize{5.000000}{6.000000}\selectfont Zen'Up}%
\end{pgfscope}%
\begin{pgfscope}%
\pgfsetbuttcap%
\pgfsetroundjoin%
\definecolor{currentfill}{rgb}{0.000000,0.000000,0.000000}%
\pgfsetfillcolor{currentfill}%
\pgfsetlinewidth{0.803000pt}%
\definecolor{currentstroke}{rgb}{0.000000,0.000000,0.000000}%
\pgfsetstrokecolor{currentstroke}%
\pgfsetdash{}{0pt}%
\pgfsys@defobject{currentmarker}{\pgfqpoint{-0.048611in}{0.000000in}}{\pgfqpoint{-0.000000in}{0.000000in}}{%
\pgfpathmoveto{\pgfqpoint{-0.000000in}{0.000000in}}%
\pgfpathlineto{\pgfqpoint{-0.048611in}{0.000000in}}%
\pgfusepath{stroke,fill}%
}%
\begin{pgfscope}%
\pgfsys@transformshift{0.499691in}{1.103099in}%
\pgfsys@useobject{currentmarker}{}%
\end{pgfscope}%
\end{pgfscope}%
\begin{pgfscope}%
\definecolor{textcolor}{rgb}{0.000000,0.000000,0.000000}%
\pgfsetstrokecolor{textcolor}%
\pgfsetfillcolor{textcolor}%
\pgftext[x=0.375463in, y=1.081879in, left, base,rotate=90.000000]{\color{textcolor}\rmfamily\fontsize{10.000000}{12.000000}\selectfont \(\displaystyle {0}\)}%
\end{pgfscope}%
\begin{pgfscope}%
\pgfsetbuttcap%
\pgfsetroundjoin%
\definecolor{currentfill}{rgb}{0.000000,0.000000,0.000000}%
\pgfsetfillcolor{currentfill}%
\pgfsetlinewidth{0.803000pt}%
\definecolor{currentstroke}{rgb}{0.000000,0.000000,0.000000}%
\pgfsetstrokecolor{currentstroke}%
\pgfsetdash{}{0pt}%
\pgfsys@defobject{currentmarker}{\pgfqpoint{-0.048611in}{0.000000in}}{\pgfqpoint{-0.000000in}{0.000000in}}{%
\pgfpathmoveto{\pgfqpoint{-0.000000in}{0.000000in}}%
\pgfpathlineto{\pgfqpoint{-0.048611in}{0.000000in}}%
\pgfusepath{stroke,fill}%
}%
\begin{pgfscope}%
\pgfsys@transformshift{0.499691in}{1.724989in}%
\pgfsys@useobject{currentmarker}{}%
\end{pgfscope}%
\end{pgfscope}%
\begin{pgfscope}%
\definecolor{textcolor}{rgb}{0.000000,0.000000,0.000000}%
\pgfsetstrokecolor{textcolor}%
\pgfsetfillcolor{textcolor}%
\pgftext[x=0.375463in, y=1.495436in, left, base,rotate=90.000000]{\color{textcolor}\rmfamily\fontsize{10.000000}{12.000000}\selectfont \(\displaystyle {1000}\)}%
\end{pgfscope}%
\begin{pgfscope}%
\pgfsetbuttcap%
\pgfsetroundjoin%
\definecolor{currentfill}{rgb}{0.000000,0.000000,0.000000}%
\pgfsetfillcolor{currentfill}%
\pgfsetlinewidth{0.803000pt}%
\definecolor{currentstroke}{rgb}{0.000000,0.000000,0.000000}%
\pgfsetstrokecolor{currentstroke}%
\pgfsetdash{}{0pt}%
\pgfsys@defobject{currentmarker}{\pgfqpoint{-0.048611in}{0.000000in}}{\pgfqpoint{-0.000000in}{0.000000in}}{%
\pgfpathmoveto{\pgfqpoint{-0.000000in}{0.000000in}}%
\pgfpathlineto{\pgfqpoint{-0.048611in}{0.000000in}}%
\pgfusepath{stroke,fill}%
}%
\begin{pgfscope}%
\pgfsys@transformshift{0.499691in}{2.346880in}%
\pgfsys@useobject{currentmarker}{}%
\end{pgfscope}%
\end{pgfscope}%
\begin{pgfscope}%
\definecolor{textcolor}{rgb}{0.000000,0.000000,0.000000}%
\pgfsetstrokecolor{textcolor}%
\pgfsetfillcolor{textcolor}%
\pgftext[x=0.375463in, y=2.117327in, left, base,rotate=90.000000]{\color{textcolor}\rmfamily\fontsize{10.000000}{12.000000}\selectfont \(\displaystyle {2000}\)}%
\end{pgfscope}%
\begin{pgfscope}%
\pgfsetbuttcap%
\pgfsetroundjoin%
\definecolor{currentfill}{rgb}{0.000000,0.000000,0.000000}%
\pgfsetfillcolor{currentfill}%
\pgfsetlinewidth{0.803000pt}%
\definecolor{currentstroke}{rgb}{0.000000,0.000000,0.000000}%
\pgfsetstrokecolor{currentstroke}%
\pgfsetdash{}{0pt}%
\pgfsys@defobject{currentmarker}{\pgfqpoint{-0.048611in}{0.000000in}}{\pgfqpoint{-0.000000in}{0.000000in}}{%
\pgfpathmoveto{\pgfqpoint{-0.000000in}{0.000000in}}%
\pgfpathlineto{\pgfqpoint{-0.048611in}{0.000000in}}%
\pgfusepath{stroke,fill}%
}%
\begin{pgfscope}%
\pgfsys@transformshift{0.499691in}{2.968770in}%
\pgfsys@useobject{currentmarker}{}%
\end{pgfscope}%
\end{pgfscope}%
\begin{pgfscope}%
\definecolor{textcolor}{rgb}{0.000000,0.000000,0.000000}%
\pgfsetstrokecolor{textcolor}%
\pgfsetfillcolor{textcolor}%
\pgftext[x=0.375463in, y=2.739217in, left, base,rotate=90.000000]{\color{textcolor}\rmfamily\fontsize{10.000000}{12.000000}\selectfont \(\displaystyle {3000}\)}%
\end{pgfscope}%
\begin{pgfscope}%
\pgfsetbuttcap%
\pgfsetroundjoin%
\definecolor{currentfill}{rgb}{0.000000,0.000000,0.000000}%
\pgfsetfillcolor{currentfill}%
\pgfsetlinewidth{0.803000pt}%
\definecolor{currentstroke}{rgb}{0.000000,0.000000,0.000000}%
\pgfsetstrokecolor{currentstroke}%
\pgfsetdash{}{0pt}%
\pgfsys@defobject{currentmarker}{\pgfqpoint{-0.048611in}{0.000000in}}{\pgfqpoint{-0.000000in}{0.000000in}}{%
\pgfpathmoveto{\pgfqpoint{-0.000000in}{0.000000in}}%
\pgfpathlineto{\pgfqpoint{-0.048611in}{0.000000in}}%
\pgfusepath{stroke,fill}%
}%
\begin{pgfscope}%
\pgfsys@transformshift{0.499691in}{3.590661in}%
\pgfsys@useobject{currentmarker}{}%
\end{pgfscope}%
\end{pgfscope}%
\begin{pgfscope}%
\definecolor{textcolor}{rgb}{0.000000,0.000000,0.000000}%
\pgfsetstrokecolor{textcolor}%
\pgfsetfillcolor{textcolor}%
\pgftext[x=0.375463in, y=3.361108in, left, base,rotate=90.000000]{\color{textcolor}\rmfamily\fontsize{10.000000}{12.000000}\selectfont \(\displaystyle {4000}\)}%
\end{pgfscope}%
\begin{pgfscope}%
\pgfsetbuttcap%
\pgfsetroundjoin%
\definecolor{currentfill}{rgb}{0.000000,0.000000,0.000000}%
\pgfsetfillcolor{currentfill}%
\pgfsetlinewidth{0.803000pt}%
\definecolor{currentstroke}{rgb}{0.000000,0.000000,0.000000}%
\pgfsetstrokecolor{currentstroke}%
\pgfsetdash{}{0pt}%
\pgfsys@defobject{currentmarker}{\pgfqpoint{-0.048611in}{0.000000in}}{\pgfqpoint{-0.000000in}{0.000000in}}{%
\pgfpathmoveto{\pgfqpoint{-0.000000in}{0.000000in}}%
\pgfpathlineto{\pgfqpoint{-0.048611in}{0.000000in}}%
\pgfusepath{stroke,fill}%
}%
\begin{pgfscope}%
\pgfsys@transformshift{0.499691in}{4.212551in}%
\pgfsys@useobject{currentmarker}{}%
\end{pgfscope}%
\end{pgfscope}%
\begin{pgfscope}%
\definecolor{textcolor}{rgb}{0.000000,0.000000,0.000000}%
\pgfsetstrokecolor{textcolor}%
\pgfsetfillcolor{textcolor}%
\pgftext[x=0.375463in, y=3.982998in, left, base,rotate=90.000000]{\color{textcolor}\rmfamily\fontsize{10.000000}{12.000000}\selectfont \(\displaystyle {5000}\)}%
\end{pgfscope}%
\begin{pgfscope}%
\pgfsetbuttcap%
\pgfsetroundjoin%
\definecolor{currentfill}{rgb}{0.000000,0.000000,0.000000}%
\pgfsetfillcolor{currentfill}%
\pgfsetlinewidth{0.803000pt}%
\definecolor{currentstroke}{rgb}{0.000000,0.000000,0.000000}%
\pgfsetstrokecolor{currentstroke}%
\pgfsetdash{}{0pt}%
\pgfsys@defobject{currentmarker}{\pgfqpoint{-0.048611in}{0.000000in}}{\pgfqpoint{-0.000000in}{0.000000in}}{%
\pgfpathmoveto{\pgfqpoint{-0.000000in}{0.000000in}}%
\pgfpathlineto{\pgfqpoint{-0.048611in}{0.000000in}}%
\pgfusepath{stroke,fill}%
}%
\begin{pgfscope}%
\pgfsys@transformshift{0.499691in}{4.834442in}%
\pgfsys@useobject{currentmarker}{}%
\end{pgfscope}%
\end{pgfscope}%
\begin{pgfscope}%
\definecolor{textcolor}{rgb}{0.000000,0.000000,0.000000}%
\pgfsetstrokecolor{textcolor}%
\pgfsetfillcolor{textcolor}%
\pgftext[x=0.375463in, y=4.604889in, left, base,rotate=90.000000]{\color{textcolor}\rmfamily\fontsize{10.000000}{12.000000}\selectfont \(\displaystyle {6000}\)}%
\end{pgfscope}%
\begin{pgfscope}%
\definecolor{textcolor}{rgb}{0.000000,0.000000,0.000000}%
\pgfsetstrokecolor{textcolor}%
\pgfsetfillcolor{textcolor}%
\pgftext[x=0.223457in,y=3.028099in,,bottom,rotate=90.000000]{\color{textcolor}\rmfamily\fontsize{10.000000}{12.000000}\selectfont Count}%
\end{pgfscope}%
\begin{pgfscope}%
\pgfpathrectangle{\pgfqpoint{0.499691in}{1.103099in}}{\pgfqpoint{7.362500in}{3.850000in}}%
\pgfusepath{clip}%
\pgfsetrectcap%
\pgfsetroundjoin%
\pgfsetlinewidth{2.710125pt}%
\definecolor{currentstroke}{rgb}{0.260000,0.260000,0.260000}%
\pgfsetstrokecolor{currentstroke}%
\pgfsetdash{}{0pt}%
\pgfusepath{stroke}%
\end{pgfscope}%
\begin{pgfscope}%
\pgfpathrectangle{\pgfqpoint{0.499691in}{1.103099in}}{\pgfqpoint{7.362500in}{3.850000in}}%
\pgfusepath{clip}%
\pgfsetrectcap%
\pgfsetroundjoin%
\pgfsetlinewidth{2.710125pt}%
\definecolor{currentstroke}{rgb}{0.260000,0.260000,0.260000}%
\pgfsetstrokecolor{currentstroke}%
\pgfsetdash{}{0pt}%
\pgfusepath{stroke}%
\end{pgfscope}%
\begin{pgfscope}%
\pgfpathrectangle{\pgfqpoint{0.499691in}{1.103099in}}{\pgfqpoint{7.362500in}{3.850000in}}%
\pgfusepath{clip}%
\pgfsetrectcap%
\pgfsetroundjoin%
\pgfsetlinewidth{2.710125pt}%
\definecolor{currentstroke}{rgb}{0.260000,0.260000,0.260000}%
\pgfsetstrokecolor{currentstroke}%
\pgfsetdash{}{0pt}%
\pgfusepath{stroke}%
\end{pgfscope}%
\begin{pgfscope}%
\pgfpathrectangle{\pgfqpoint{0.499691in}{1.103099in}}{\pgfqpoint{7.362500in}{3.850000in}}%
\pgfusepath{clip}%
\pgfsetrectcap%
\pgfsetroundjoin%
\pgfsetlinewidth{2.710125pt}%
\definecolor{currentstroke}{rgb}{0.260000,0.260000,0.260000}%
\pgfsetstrokecolor{currentstroke}%
\pgfsetdash{}{0pt}%
\pgfusepath{stroke}%
\end{pgfscope}%
\begin{pgfscope}%
\pgfpathrectangle{\pgfqpoint{0.499691in}{1.103099in}}{\pgfqpoint{7.362500in}{3.850000in}}%
\pgfusepath{clip}%
\pgfsetrectcap%
\pgfsetroundjoin%
\pgfsetlinewidth{2.710125pt}%
\definecolor{currentstroke}{rgb}{0.260000,0.260000,0.260000}%
\pgfsetstrokecolor{currentstroke}%
\pgfsetdash{}{0pt}%
\pgfusepath{stroke}%
\end{pgfscope}%
\begin{pgfscope}%
\pgfpathrectangle{\pgfqpoint{0.499691in}{1.103099in}}{\pgfqpoint{7.362500in}{3.850000in}}%
\pgfusepath{clip}%
\pgfsetrectcap%
\pgfsetroundjoin%
\pgfsetlinewidth{2.710125pt}%
\definecolor{currentstroke}{rgb}{0.260000,0.260000,0.260000}%
\pgfsetstrokecolor{currentstroke}%
\pgfsetdash{}{0pt}%
\pgfusepath{stroke}%
\end{pgfscope}%
\begin{pgfscope}%
\pgfpathrectangle{\pgfqpoint{0.499691in}{1.103099in}}{\pgfqpoint{7.362500in}{3.850000in}}%
\pgfusepath{clip}%
\pgfsetrectcap%
\pgfsetroundjoin%
\pgfsetlinewidth{2.710125pt}%
\definecolor{currentstroke}{rgb}{0.260000,0.260000,0.260000}%
\pgfsetstrokecolor{currentstroke}%
\pgfsetdash{}{0pt}%
\pgfusepath{stroke}%
\end{pgfscope}%
\begin{pgfscope}%
\pgfpathrectangle{\pgfqpoint{0.499691in}{1.103099in}}{\pgfqpoint{7.362500in}{3.850000in}}%
\pgfusepath{clip}%
\pgfsetrectcap%
\pgfsetroundjoin%
\pgfsetlinewidth{2.710125pt}%
\definecolor{currentstroke}{rgb}{0.260000,0.260000,0.260000}%
\pgfsetstrokecolor{currentstroke}%
\pgfsetdash{}{0pt}%
\pgfusepath{stroke}%
\end{pgfscope}%
\begin{pgfscope}%
\pgfpathrectangle{\pgfqpoint{0.499691in}{1.103099in}}{\pgfqpoint{7.362500in}{3.850000in}}%
\pgfusepath{clip}%
\pgfsetrectcap%
\pgfsetroundjoin%
\pgfsetlinewidth{2.710125pt}%
\definecolor{currentstroke}{rgb}{0.260000,0.260000,0.260000}%
\pgfsetstrokecolor{currentstroke}%
\pgfsetdash{}{0pt}%
\pgfusepath{stroke}%
\end{pgfscope}%
\begin{pgfscope}%
\pgfpathrectangle{\pgfqpoint{0.499691in}{1.103099in}}{\pgfqpoint{7.362500in}{3.850000in}}%
\pgfusepath{clip}%
\pgfsetrectcap%
\pgfsetroundjoin%
\pgfsetlinewidth{2.710125pt}%
\definecolor{currentstroke}{rgb}{0.260000,0.260000,0.260000}%
\pgfsetstrokecolor{currentstroke}%
\pgfsetdash{}{0pt}%
\pgfusepath{stroke}%
\end{pgfscope}%
\begin{pgfscope}%
\pgfpathrectangle{\pgfqpoint{0.499691in}{1.103099in}}{\pgfqpoint{7.362500in}{3.850000in}}%
\pgfusepath{clip}%
\pgfsetrectcap%
\pgfsetroundjoin%
\pgfsetlinewidth{2.710125pt}%
\definecolor{currentstroke}{rgb}{0.260000,0.260000,0.260000}%
\pgfsetstrokecolor{currentstroke}%
\pgfsetdash{}{0pt}%
\pgfusepath{stroke}%
\end{pgfscope}%
\begin{pgfscope}%
\pgfpathrectangle{\pgfqpoint{0.499691in}{1.103099in}}{\pgfqpoint{7.362500in}{3.850000in}}%
\pgfusepath{clip}%
\pgfsetrectcap%
\pgfsetroundjoin%
\pgfsetlinewidth{2.710125pt}%
\definecolor{currentstroke}{rgb}{0.260000,0.260000,0.260000}%
\pgfsetstrokecolor{currentstroke}%
\pgfsetdash{}{0pt}%
\pgfusepath{stroke}%
\end{pgfscope}%
\begin{pgfscope}%
\pgfpathrectangle{\pgfqpoint{0.499691in}{1.103099in}}{\pgfqpoint{7.362500in}{3.850000in}}%
\pgfusepath{clip}%
\pgfsetrectcap%
\pgfsetroundjoin%
\pgfsetlinewidth{2.710125pt}%
\definecolor{currentstroke}{rgb}{0.260000,0.260000,0.260000}%
\pgfsetstrokecolor{currentstroke}%
\pgfsetdash{}{0pt}%
\pgfusepath{stroke}%
\end{pgfscope}%
\begin{pgfscope}%
\pgfpathrectangle{\pgfqpoint{0.499691in}{1.103099in}}{\pgfqpoint{7.362500in}{3.850000in}}%
\pgfusepath{clip}%
\pgfsetrectcap%
\pgfsetroundjoin%
\pgfsetlinewidth{2.710125pt}%
\definecolor{currentstroke}{rgb}{0.260000,0.260000,0.260000}%
\pgfsetstrokecolor{currentstroke}%
\pgfsetdash{}{0pt}%
\pgfusepath{stroke}%
\end{pgfscope}%
\begin{pgfscope}%
\pgfpathrectangle{\pgfqpoint{0.499691in}{1.103099in}}{\pgfqpoint{7.362500in}{3.850000in}}%
\pgfusepath{clip}%
\pgfsetrectcap%
\pgfsetroundjoin%
\pgfsetlinewidth{2.710125pt}%
\definecolor{currentstroke}{rgb}{0.260000,0.260000,0.260000}%
\pgfsetstrokecolor{currentstroke}%
\pgfsetdash{}{0pt}%
\pgfusepath{stroke}%
\end{pgfscope}%
\begin{pgfscope}%
\pgfpathrectangle{\pgfqpoint{0.499691in}{1.103099in}}{\pgfqpoint{7.362500in}{3.850000in}}%
\pgfusepath{clip}%
\pgfsetrectcap%
\pgfsetroundjoin%
\pgfsetlinewidth{2.710125pt}%
\definecolor{currentstroke}{rgb}{0.260000,0.260000,0.260000}%
\pgfsetstrokecolor{currentstroke}%
\pgfsetdash{}{0pt}%
\pgfusepath{stroke}%
\end{pgfscope}%
\begin{pgfscope}%
\pgfpathrectangle{\pgfqpoint{0.499691in}{1.103099in}}{\pgfqpoint{7.362500in}{3.850000in}}%
\pgfusepath{clip}%
\pgfsetrectcap%
\pgfsetroundjoin%
\pgfsetlinewidth{2.710125pt}%
\definecolor{currentstroke}{rgb}{0.260000,0.260000,0.260000}%
\pgfsetstrokecolor{currentstroke}%
\pgfsetdash{}{0pt}%
\pgfusepath{stroke}%
\end{pgfscope}%
\begin{pgfscope}%
\pgfpathrectangle{\pgfqpoint{0.499691in}{1.103099in}}{\pgfqpoint{7.362500in}{3.850000in}}%
\pgfusepath{clip}%
\pgfsetrectcap%
\pgfsetroundjoin%
\pgfsetlinewidth{2.710125pt}%
\definecolor{currentstroke}{rgb}{0.260000,0.260000,0.260000}%
\pgfsetstrokecolor{currentstroke}%
\pgfsetdash{}{0pt}%
\pgfusepath{stroke}%
\end{pgfscope}%
\begin{pgfscope}%
\pgfpathrectangle{\pgfqpoint{0.499691in}{1.103099in}}{\pgfqpoint{7.362500in}{3.850000in}}%
\pgfusepath{clip}%
\pgfsetrectcap%
\pgfsetroundjoin%
\pgfsetlinewidth{2.710125pt}%
\definecolor{currentstroke}{rgb}{0.260000,0.260000,0.260000}%
\pgfsetstrokecolor{currentstroke}%
\pgfsetdash{}{0pt}%
\pgfusepath{stroke}%
\end{pgfscope}%
\begin{pgfscope}%
\pgfpathrectangle{\pgfqpoint{0.499691in}{1.103099in}}{\pgfqpoint{7.362500in}{3.850000in}}%
\pgfusepath{clip}%
\pgfsetrectcap%
\pgfsetroundjoin%
\pgfsetlinewidth{2.710125pt}%
\definecolor{currentstroke}{rgb}{0.260000,0.260000,0.260000}%
\pgfsetstrokecolor{currentstroke}%
\pgfsetdash{}{0pt}%
\pgfusepath{stroke}%
\end{pgfscope}%
\begin{pgfscope}%
\pgfpathrectangle{\pgfqpoint{0.499691in}{1.103099in}}{\pgfqpoint{7.362500in}{3.850000in}}%
\pgfusepath{clip}%
\pgfsetrectcap%
\pgfsetroundjoin%
\pgfsetlinewidth{2.710125pt}%
\definecolor{currentstroke}{rgb}{0.260000,0.260000,0.260000}%
\pgfsetstrokecolor{currentstroke}%
\pgfsetdash{}{0pt}%
\pgfusepath{stroke}%
\end{pgfscope}%
\begin{pgfscope}%
\pgfpathrectangle{\pgfqpoint{0.499691in}{1.103099in}}{\pgfqpoint{7.362500in}{3.850000in}}%
\pgfusepath{clip}%
\pgfsetrectcap%
\pgfsetroundjoin%
\pgfsetlinewidth{2.710125pt}%
\definecolor{currentstroke}{rgb}{0.260000,0.260000,0.260000}%
\pgfsetstrokecolor{currentstroke}%
\pgfsetdash{}{0pt}%
\pgfusepath{stroke}%
\end{pgfscope}%
\begin{pgfscope}%
\pgfpathrectangle{\pgfqpoint{0.499691in}{1.103099in}}{\pgfqpoint{7.362500in}{3.850000in}}%
\pgfusepath{clip}%
\pgfsetrectcap%
\pgfsetroundjoin%
\pgfsetlinewidth{2.710125pt}%
\definecolor{currentstroke}{rgb}{0.260000,0.260000,0.260000}%
\pgfsetstrokecolor{currentstroke}%
\pgfsetdash{}{0pt}%
\pgfusepath{stroke}%
\end{pgfscope}%
\begin{pgfscope}%
\pgfpathrectangle{\pgfqpoint{0.499691in}{1.103099in}}{\pgfqpoint{7.362500in}{3.850000in}}%
\pgfusepath{clip}%
\pgfsetrectcap%
\pgfsetroundjoin%
\pgfsetlinewidth{2.710125pt}%
\definecolor{currentstroke}{rgb}{0.260000,0.260000,0.260000}%
\pgfsetstrokecolor{currentstroke}%
\pgfsetdash{}{0pt}%
\pgfusepath{stroke}%
\end{pgfscope}%
\begin{pgfscope}%
\pgfpathrectangle{\pgfqpoint{0.499691in}{1.103099in}}{\pgfqpoint{7.362500in}{3.850000in}}%
\pgfusepath{clip}%
\pgfsetrectcap%
\pgfsetroundjoin%
\pgfsetlinewidth{2.710125pt}%
\definecolor{currentstroke}{rgb}{0.260000,0.260000,0.260000}%
\pgfsetstrokecolor{currentstroke}%
\pgfsetdash{}{0pt}%
\pgfusepath{stroke}%
\end{pgfscope}%
\begin{pgfscope}%
\pgfpathrectangle{\pgfqpoint{0.499691in}{1.103099in}}{\pgfqpoint{7.362500in}{3.850000in}}%
\pgfusepath{clip}%
\pgfsetrectcap%
\pgfsetroundjoin%
\pgfsetlinewidth{2.710125pt}%
\definecolor{currentstroke}{rgb}{0.260000,0.260000,0.260000}%
\pgfsetstrokecolor{currentstroke}%
\pgfsetdash{}{0pt}%
\pgfusepath{stroke}%
\end{pgfscope}%
\begin{pgfscope}%
\pgfpathrectangle{\pgfqpoint{0.499691in}{1.103099in}}{\pgfqpoint{7.362500in}{3.850000in}}%
\pgfusepath{clip}%
\pgfsetrectcap%
\pgfsetroundjoin%
\pgfsetlinewidth{2.710125pt}%
\definecolor{currentstroke}{rgb}{0.260000,0.260000,0.260000}%
\pgfsetstrokecolor{currentstroke}%
\pgfsetdash{}{0pt}%
\pgfusepath{stroke}%
\end{pgfscope}%
\begin{pgfscope}%
\pgfpathrectangle{\pgfqpoint{0.499691in}{1.103099in}}{\pgfqpoint{7.362500in}{3.850000in}}%
\pgfusepath{clip}%
\pgfsetrectcap%
\pgfsetroundjoin%
\pgfsetlinewidth{2.710125pt}%
\definecolor{currentstroke}{rgb}{0.260000,0.260000,0.260000}%
\pgfsetstrokecolor{currentstroke}%
\pgfsetdash{}{0pt}%
\pgfusepath{stroke}%
\end{pgfscope}%
\begin{pgfscope}%
\pgfpathrectangle{\pgfqpoint{0.499691in}{1.103099in}}{\pgfqpoint{7.362500in}{3.850000in}}%
\pgfusepath{clip}%
\pgfsetrectcap%
\pgfsetroundjoin%
\pgfsetlinewidth{2.710125pt}%
\definecolor{currentstroke}{rgb}{0.260000,0.260000,0.260000}%
\pgfsetstrokecolor{currentstroke}%
\pgfsetdash{}{0pt}%
\pgfusepath{stroke}%
\end{pgfscope}%
\begin{pgfscope}%
\pgfpathrectangle{\pgfqpoint{0.499691in}{1.103099in}}{\pgfqpoint{7.362500in}{3.850000in}}%
\pgfusepath{clip}%
\pgfsetrectcap%
\pgfsetroundjoin%
\pgfsetlinewidth{2.710125pt}%
\definecolor{currentstroke}{rgb}{0.260000,0.260000,0.260000}%
\pgfsetstrokecolor{currentstroke}%
\pgfsetdash{}{0pt}%
\pgfusepath{stroke}%
\end{pgfscope}%
\begin{pgfscope}%
\pgfpathrectangle{\pgfqpoint{0.499691in}{1.103099in}}{\pgfqpoint{7.362500in}{3.850000in}}%
\pgfusepath{clip}%
\pgfsetrectcap%
\pgfsetroundjoin%
\pgfsetlinewidth{2.710125pt}%
\definecolor{currentstroke}{rgb}{0.260000,0.260000,0.260000}%
\pgfsetstrokecolor{currentstroke}%
\pgfsetdash{}{0pt}%
\pgfusepath{stroke}%
\end{pgfscope}%
\begin{pgfscope}%
\pgfpathrectangle{\pgfqpoint{0.499691in}{1.103099in}}{\pgfqpoint{7.362500in}{3.850000in}}%
\pgfusepath{clip}%
\pgfsetrectcap%
\pgfsetroundjoin%
\pgfsetlinewidth{2.710125pt}%
\definecolor{currentstroke}{rgb}{0.260000,0.260000,0.260000}%
\pgfsetstrokecolor{currentstroke}%
\pgfsetdash{}{0pt}%
\pgfusepath{stroke}%
\end{pgfscope}%
\begin{pgfscope}%
\pgfpathrectangle{\pgfqpoint{0.499691in}{1.103099in}}{\pgfqpoint{7.362500in}{3.850000in}}%
\pgfusepath{clip}%
\pgfsetrectcap%
\pgfsetroundjoin%
\pgfsetlinewidth{2.710125pt}%
\definecolor{currentstroke}{rgb}{0.260000,0.260000,0.260000}%
\pgfsetstrokecolor{currentstroke}%
\pgfsetdash{}{0pt}%
\pgfusepath{stroke}%
\end{pgfscope}%
\begin{pgfscope}%
\pgfpathrectangle{\pgfqpoint{0.499691in}{1.103099in}}{\pgfqpoint{7.362500in}{3.850000in}}%
\pgfusepath{clip}%
\pgfsetrectcap%
\pgfsetroundjoin%
\pgfsetlinewidth{2.710125pt}%
\definecolor{currentstroke}{rgb}{0.260000,0.260000,0.260000}%
\pgfsetstrokecolor{currentstroke}%
\pgfsetdash{}{0pt}%
\pgfusepath{stroke}%
\end{pgfscope}%
\begin{pgfscope}%
\pgfpathrectangle{\pgfqpoint{0.499691in}{1.103099in}}{\pgfqpoint{7.362500in}{3.850000in}}%
\pgfusepath{clip}%
\pgfsetrectcap%
\pgfsetroundjoin%
\pgfsetlinewidth{2.710125pt}%
\definecolor{currentstroke}{rgb}{0.260000,0.260000,0.260000}%
\pgfsetstrokecolor{currentstroke}%
\pgfsetdash{}{0pt}%
\pgfusepath{stroke}%
\end{pgfscope}%
\begin{pgfscope}%
\pgfpathrectangle{\pgfqpoint{0.499691in}{1.103099in}}{\pgfqpoint{7.362500in}{3.850000in}}%
\pgfusepath{clip}%
\pgfsetrectcap%
\pgfsetroundjoin%
\pgfsetlinewidth{2.710125pt}%
\definecolor{currentstroke}{rgb}{0.260000,0.260000,0.260000}%
\pgfsetstrokecolor{currentstroke}%
\pgfsetdash{}{0pt}%
\pgfusepath{stroke}%
\end{pgfscope}%
\begin{pgfscope}%
\pgfpathrectangle{\pgfqpoint{0.499691in}{1.103099in}}{\pgfqpoint{7.362500in}{3.850000in}}%
\pgfusepath{clip}%
\pgfsetrectcap%
\pgfsetroundjoin%
\pgfsetlinewidth{2.710125pt}%
\definecolor{currentstroke}{rgb}{0.260000,0.260000,0.260000}%
\pgfsetstrokecolor{currentstroke}%
\pgfsetdash{}{0pt}%
\pgfusepath{stroke}%
\end{pgfscope}%
\begin{pgfscope}%
\pgfpathrectangle{\pgfqpoint{0.499691in}{1.103099in}}{\pgfqpoint{7.362500in}{3.850000in}}%
\pgfusepath{clip}%
\pgfsetrectcap%
\pgfsetroundjoin%
\pgfsetlinewidth{2.710125pt}%
\definecolor{currentstroke}{rgb}{0.260000,0.260000,0.260000}%
\pgfsetstrokecolor{currentstroke}%
\pgfsetdash{}{0pt}%
\pgfusepath{stroke}%
\end{pgfscope}%
\begin{pgfscope}%
\pgfpathrectangle{\pgfqpoint{0.499691in}{1.103099in}}{\pgfqpoint{7.362500in}{3.850000in}}%
\pgfusepath{clip}%
\pgfsetrectcap%
\pgfsetroundjoin%
\pgfsetlinewidth{2.710125pt}%
\definecolor{currentstroke}{rgb}{0.260000,0.260000,0.260000}%
\pgfsetstrokecolor{currentstroke}%
\pgfsetdash{}{0pt}%
\pgfusepath{stroke}%
\end{pgfscope}%
\begin{pgfscope}%
\pgfpathrectangle{\pgfqpoint{0.499691in}{1.103099in}}{\pgfqpoint{7.362500in}{3.850000in}}%
\pgfusepath{clip}%
\pgfsetrectcap%
\pgfsetroundjoin%
\pgfsetlinewidth{2.710125pt}%
\definecolor{currentstroke}{rgb}{0.260000,0.260000,0.260000}%
\pgfsetstrokecolor{currentstroke}%
\pgfsetdash{}{0pt}%
\pgfusepath{stroke}%
\end{pgfscope}%
\begin{pgfscope}%
\pgfpathrectangle{\pgfqpoint{0.499691in}{1.103099in}}{\pgfqpoint{7.362500in}{3.850000in}}%
\pgfusepath{clip}%
\pgfsetrectcap%
\pgfsetroundjoin%
\pgfsetlinewidth{2.710125pt}%
\definecolor{currentstroke}{rgb}{0.260000,0.260000,0.260000}%
\pgfsetstrokecolor{currentstroke}%
\pgfsetdash{}{0pt}%
\pgfusepath{stroke}%
\end{pgfscope}%
\begin{pgfscope}%
\pgfpathrectangle{\pgfqpoint{0.499691in}{1.103099in}}{\pgfqpoint{7.362500in}{3.850000in}}%
\pgfusepath{clip}%
\pgfsetrectcap%
\pgfsetroundjoin%
\pgfsetlinewidth{2.710125pt}%
\definecolor{currentstroke}{rgb}{0.260000,0.260000,0.260000}%
\pgfsetstrokecolor{currentstroke}%
\pgfsetdash{}{0pt}%
\pgfusepath{stroke}%
\end{pgfscope}%
\begin{pgfscope}%
\pgfpathrectangle{\pgfqpoint{0.499691in}{1.103099in}}{\pgfqpoint{7.362500in}{3.850000in}}%
\pgfusepath{clip}%
\pgfsetrectcap%
\pgfsetroundjoin%
\pgfsetlinewidth{2.710125pt}%
\definecolor{currentstroke}{rgb}{0.260000,0.260000,0.260000}%
\pgfsetstrokecolor{currentstroke}%
\pgfsetdash{}{0pt}%
\pgfusepath{stroke}%
\end{pgfscope}%
\begin{pgfscope}%
\pgfpathrectangle{\pgfqpoint{0.499691in}{1.103099in}}{\pgfqpoint{7.362500in}{3.850000in}}%
\pgfusepath{clip}%
\pgfsetrectcap%
\pgfsetroundjoin%
\pgfsetlinewidth{2.710125pt}%
\definecolor{currentstroke}{rgb}{0.260000,0.260000,0.260000}%
\pgfsetstrokecolor{currentstroke}%
\pgfsetdash{}{0pt}%
\pgfusepath{stroke}%
\end{pgfscope}%
\begin{pgfscope}%
\pgfpathrectangle{\pgfqpoint{0.499691in}{1.103099in}}{\pgfqpoint{7.362500in}{3.850000in}}%
\pgfusepath{clip}%
\pgfsetrectcap%
\pgfsetroundjoin%
\pgfsetlinewidth{2.710125pt}%
\definecolor{currentstroke}{rgb}{0.260000,0.260000,0.260000}%
\pgfsetstrokecolor{currentstroke}%
\pgfsetdash{}{0pt}%
\pgfusepath{stroke}%
\end{pgfscope}%
\begin{pgfscope}%
\pgfpathrectangle{\pgfqpoint{0.499691in}{1.103099in}}{\pgfqpoint{7.362500in}{3.850000in}}%
\pgfusepath{clip}%
\pgfsetrectcap%
\pgfsetroundjoin%
\pgfsetlinewidth{2.710125pt}%
\definecolor{currentstroke}{rgb}{0.260000,0.260000,0.260000}%
\pgfsetstrokecolor{currentstroke}%
\pgfsetdash{}{0pt}%
\pgfusepath{stroke}%
\end{pgfscope}%
\begin{pgfscope}%
\pgfpathrectangle{\pgfqpoint{0.499691in}{1.103099in}}{\pgfqpoint{7.362500in}{3.850000in}}%
\pgfusepath{clip}%
\pgfsetrectcap%
\pgfsetroundjoin%
\pgfsetlinewidth{2.710125pt}%
\definecolor{currentstroke}{rgb}{0.260000,0.260000,0.260000}%
\pgfsetstrokecolor{currentstroke}%
\pgfsetdash{}{0pt}%
\pgfusepath{stroke}%
\end{pgfscope}%
\begin{pgfscope}%
\pgfpathrectangle{\pgfqpoint{0.499691in}{1.103099in}}{\pgfqpoint{7.362500in}{3.850000in}}%
\pgfusepath{clip}%
\pgfsetrectcap%
\pgfsetroundjoin%
\pgfsetlinewidth{2.710125pt}%
\definecolor{currentstroke}{rgb}{0.260000,0.260000,0.260000}%
\pgfsetstrokecolor{currentstroke}%
\pgfsetdash{}{0pt}%
\pgfusepath{stroke}%
\end{pgfscope}%
\begin{pgfscope}%
\pgfpathrectangle{\pgfqpoint{0.499691in}{1.103099in}}{\pgfqpoint{7.362500in}{3.850000in}}%
\pgfusepath{clip}%
\pgfsetrectcap%
\pgfsetroundjoin%
\pgfsetlinewidth{2.710125pt}%
\definecolor{currentstroke}{rgb}{0.260000,0.260000,0.260000}%
\pgfsetstrokecolor{currentstroke}%
\pgfsetdash{}{0pt}%
\pgfusepath{stroke}%
\end{pgfscope}%
\begin{pgfscope}%
\pgfpathrectangle{\pgfqpoint{0.499691in}{1.103099in}}{\pgfqpoint{7.362500in}{3.850000in}}%
\pgfusepath{clip}%
\pgfsetrectcap%
\pgfsetroundjoin%
\pgfsetlinewidth{2.710125pt}%
\definecolor{currentstroke}{rgb}{0.260000,0.260000,0.260000}%
\pgfsetstrokecolor{currentstroke}%
\pgfsetdash{}{0pt}%
\pgfusepath{stroke}%
\end{pgfscope}%
\begin{pgfscope}%
\pgfpathrectangle{\pgfqpoint{0.499691in}{1.103099in}}{\pgfqpoint{7.362500in}{3.850000in}}%
\pgfusepath{clip}%
\pgfsetrectcap%
\pgfsetroundjoin%
\pgfsetlinewidth{2.710125pt}%
\definecolor{currentstroke}{rgb}{0.260000,0.260000,0.260000}%
\pgfsetstrokecolor{currentstroke}%
\pgfsetdash{}{0pt}%
\pgfusepath{stroke}%
\end{pgfscope}%
\begin{pgfscope}%
\pgfpathrectangle{\pgfqpoint{0.499691in}{1.103099in}}{\pgfqpoint{7.362500in}{3.850000in}}%
\pgfusepath{clip}%
\pgfsetrectcap%
\pgfsetroundjoin%
\pgfsetlinewidth{2.710125pt}%
\definecolor{currentstroke}{rgb}{0.260000,0.260000,0.260000}%
\pgfsetstrokecolor{currentstroke}%
\pgfsetdash{}{0pt}%
\pgfusepath{stroke}%
\end{pgfscope}%
\begin{pgfscope}%
\pgfpathrectangle{\pgfqpoint{0.499691in}{1.103099in}}{\pgfqpoint{7.362500in}{3.850000in}}%
\pgfusepath{clip}%
\pgfsetrectcap%
\pgfsetroundjoin%
\pgfsetlinewidth{2.710125pt}%
\definecolor{currentstroke}{rgb}{0.260000,0.260000,0.260000}%
\pgfsetstrokecolor{currentstroke}%
\pgfsetdash{}{0pt}%
\pgfusepath{stroke}%
\end{pgfscope}%
\begin{pgfscope}%
\pgfpathrectangle{\pgfqpoint{0.499691in}{1.103099in}}{\pgfqpoint{7.362500in}{3.850000in}}%
\pgfusepath{clip}%
\pgfsetrectcap%
\pgfsetroundjoin%
\pgfsetlinewidth{2.710125pt}%
\definecolor{currentstroke}{rgb}{0.260000,0.260000,0.260000}%
\pgfsetstrokecolor{currentstroke}%
\pgfsetdash{}{0pt}%
\pgfusepath{stroke}%
\end{pgfscope}%
\begin{pgfscope}%
\pgfpathrectangle{\pgfqpoint{0.499691in}{1.103099in}}{\pgfqpoint{7.362500in}{3.850000in}}%
\pgfusepath{clip}%
\pgfsetrectcap%
\pgfsetroundjoin%
\pgfsetlinewidth{2.710125pt}%
\definecolor{currentstroke}{rgb}{0.260000,0.260000,0.260000}%
\pgfsetstrokecolor{currentstroke}%
\pgfsetdash{}{0pt}%
\pgfusepath{stroke}%
\end{pgfscope}%
\begin{pgfscope}%
\pgfpathrectangle{\pgfqpoint{0.499691in}{1.103099in}}{\pgfqpoint{7.362500in}{3.850000in}}%
\pgfusepath{clip}%
\pgfsetrectcap%
\pgfsetroundjoin%
\pgfsetlinewidth{2.710125pt}%
\definecolor{currentstroke}{rgb}{0.260000,0.260000,0.260000}%
\pgfsetstrokecolor{currentstroke}%
\pgfsetdash{}{0pt}%
\pgfusepath{stroke}%
\end{pgfscope}%
\begin{pgfscope}%
\pgfsetrectcap%
\pgfsetmiterjoin%
\pgfsetlinewidth{0.803000pt}%
\definecolor{currentstroke}{rgb}{0.000000,0.000000,0.000000}%
\pgfsetstrokecolor{currentstroke}%
\pgfsetdash{}{0pt}%
\pgfpathmoveto{\pgfqpoint{0.499691in}{1.103099in}}%
\pgfpathlineto{\pgfqpoint{0.499691in}{4.953099in}}%
\pgfusepath{stroke}%
\end{pgfscope}%
\begin{pgfscope}%
\pgfsetrectcap%
\pgfsetmiterjoin%
\pgfsetlinewidth{0.803000pt}%
\definecolor{currentstroke}{rgb}{0.000000,0.000000,0.000000}%
\pgfsetstrokecolor{currentstroke}%
\pgfsetdash{}{0pt}%
\pgfpathmoveto{\pgfqpoint{7.862191in}{1.103099in}}%
\pgfpathlineto{\pgfqpoint{7.862191in}{4.953099in}}%
\pgfusepath{stroke}%
\end{pgfscope}%
\begin{pgfscope}%
\pgfsetrectcap%
\pgfsetmiterjoin%
\pgfsetlinewidth{0.803000pt}%
\definecolor{currentstroke}{rgb}{0.000000,0.000000,0.000000}%
\pgfsetstrokecolor{currentstroke}%
\pgfsetdash{}{0pt}%
\pgfpathmoveto{\pgfqpoint{0.499691in}{1.103099in}}%
\pgfpathlineto{\pgfqpoint{7.862191in}{1.103099in}}%
\pgfusepath{stroke}%
\end{pgfscope}%
\begin{pgfscope}%
\pgfsetrectcap%
\pgfsetmiterjoin%
\pgfsetlinewidth{0.803000pt}%
\definecolor{currentstroke}{rgb}{0.000000,0.000000,0.000000}%
\pgfsetstrokecolor{currentstroke}%
\pgfsetdash{}{0pt}%
\pgfpathmoveto{\pgfqpoint{0.499691in}{4.953099in}}%
\pgfpathlineto{\pgfqpoint{7.862191in}{4.953099in}}%
\pgfusepath{stroke}%
\end{pgfscope}%
\end{pgfpicture}%
\makeatother%
\endgroup%

    \caption{Number of note per assureur}
    \label{fig:nbnote_per_assureur}
\end{figure}

Finally we watch the number of review per date using a calendar

\newgeometry{top=1cm, bottom=0cm}
\begin{figure}[H]
    %\advance\leftskip-0cm
    \centering
    %% Creator: Matplotlib, PGF backend
%%
%% To include the figure in your LaTeX document, write
%%   \input{<filename>.pgf}
%%
%% Make sure the required packages are loaded in your preamble
%%   \usepackage{pgf}
%%
%% Also ensure that all the required font packages are loaded; for instance,
%% the lmodern package is sometimes necessary when using math font.
%%   \usepackage{lmodern}
%%
%% Figures using additional raster images can only be included by \input if
%% they are in the same directory as the main LaTeX file. For loading figures
%% from other directories you can use the `import` package
%%   \usepackage{import}
%%
%% and then include the figures with
%%   \import{<path to file>}{<filename>.pgf}
%%
%% Matplotlib used the following preamble
%%
\begingroup%
\makeatletter%
\begin{pgfpicture}%
\pgfpathrectangle{\pgfpointorigin}{\pgfqpoint{5.130943in}{10.924340in}}%
\pgfusepath{use as bounding box, clip}%
\begin{pgfscope}%
\pgfsetbuttcap%
\pgfsetmiterjoin%
\definecolor{currentfill}{rgb}{1.000000,1.000000,1.000000}%
\pgfsetfillcolor{currentfill}%
\pgfsetlinewidth{0.000000pt}%
\definecolor{currentstroke}{rgb}{1.000000,1.000000,1.000000}%
\pgfsetstrokecolor{currentstroke}%
\pgfsetdash{}{0pt}%
\pgfpathmoveto{\pgfqpoint{0.000000in}{0.000000in}}%
\pgfpathlineto{\pgfqpoint{5.130943in}{0.000000in}}%
\pgfpathlineto{\pgfqpoint{5.130943in}{10.924340in}}%
\pgfpathlineto{\pgfqpoint{0.000000in}{10.924340in}}%
\pgfpathlineto{\pgfqpoint{0.000000in}{0.000000in}}%
\pgfpathclose%
\pgfusepath{fill}%
\end{pgfscope}%
\begin{pgfscope}%
\pgfpathrectangle{\pgfqpoint{0.380943in}{9.960189in}}{\pgfqpoint{4.650000in}{0.614151in}}%
\pgfusepath{clip}%
\pgfsetbuttcap%
\pgfsetroundjoin%
\pgfsetlinewidth{0.250937pt}%
\definecolor{currentstroke}{rgb}{1.000000,1.000000,1.000000}%
\pgfsetstrokecolor{currentstroke}%
\pgfsetdash{}{0pt}%
\pgfpathmoveto{\pgfqpoint{0.380943in}{10.574340in}}%
\pgfpathlineto{\pgfqpoint{0.468679in}{10.574340in}}%
\pgfpathlineto{\pgfqpoint{0.468679in}{10.486604in}}%
\pgfpathlineto{\pgfqpoint{0.380943in}{10.486604in}}%
\pgfpathlineto{\pgfqpoint{0.380943in}{10.574340in}}%
\pgfusepath{stroke}%
\end{pgfscope}%
\begin{pgfscope}%
\pgfpathrectangle{\pgfqpoint{0.380943in}{9.960189in}}{\pgfqpoint{4.650000in}{0.614151in}}%
\pgfusepath{clip}%
\pgfsetbuttcap%
\pgfsetroundjoin%
\definecolor{currentfill}{rgb}{1.000000,1.000000,0.929412}%
\pgfsetfillcolor{currentfill}%
\pgfsetlinewidth{0.250937pt}%
\definecolor{currentstroke}{rgb}{1.000000,1.000000,1.000000}%
\pgfsetstrokecolor{currentstroke}%
\pgfsetdash{}{0pt}%
\pgfpathmoveto{\pgfqpoint{0.468679in}{10.574340in}}%
\pgfpathlineto{\pgfqpoint{0.556415in}{10.574340in}}%
\pgfpathlineto{\pgfqpoint{0.556415in}{10.486604in}}%
\pgfpathlineto{\pgfqpoint{0.468679in}{10.486604in}}%
\pgfpathlineto{\pgfqpoint{0.468679in}{10.574340in}}%
\pgfusepath{stroke,fill}%
\end{pgfscope}%
\begin{pgfscope}%
\pgfpathrectangle{\pgfqpoint{0.380943in}{9.960189in}}{\pgfqpoint{4.650000in}{0.614151in}}%
\pgfusepath{clip}%
\pgfsetbuttcap%
\pgfsetroundjoin%
\definecolor{currentfill}{rgb}{1.000000,1.000000,0.929412}%
\pgfsetfillcolor{currentfill}%
\pgfsetlinewidth{0.250937pt}%
\definecolor{currentstroke}{rgb}{1.000000,1.000000,1.000000}%
\pgfsetstrokecolor{currentstroke}%
\pgfsetdash{}{0pt}%
\pgfpathmoveto{\pgfqpoint{0.556415in}{10.574340in}}%
\pgfpathlineto{\pgfqpoint{0.644151in}{10.574340in}}%
\pgfpathlineto{\pgfqpoint{0.644151in}{10.486604in}}%
\pgfpathlineto{\pgfqpoint{0.556415in}{10.486604in}}%
\pgfpathlineto{\pgfqpoint{0.556415in}{10.574340in}}%
\pgfusepath{stroke,fill}%
\end{pgfscope}%
\begin{pgfscope}%
\pgfpathrectangle{\pgfqpoint{0.380943in}{9.960189in}}{\pgfqpoint{4.650000in}{0.614151in}}%
\pgfusepath{clip}%
\pgfsetbuttcap%
\pgfsetroundjoin%
\definecolor{currentfill}{rgb}{1.000000,1.000000,0.929412}%
\pgfsetfillcolor{currentfill}%
\pgfsetlinewidth{0.250937pt}%
\definecolor{currentstroke}{rgb}{1.000000,1.000000,1.000000}%
\pgfsetstrokecolor{currentstroke}%
\pgfsetdash{}{0pt}%
\pgfpathmoveto{\pgfqpoint{0.644151in}{10.574340in}}%
\pgfpathlineto{\pgfqpoint{0.731886in}{10.574340in}}%
\pgfpathlineto{\pgfqpoint{0.731886in}{10.486604in}}%
\pgfpathlineto{\pgfqpoint{0.644151in}{10.486604in}}%
\pgfpathlineto{\pgfqpoint{0.644151in}{10.574340in}}%
\pgfusepath{stroke,fill}%
\end{pgfscope}%
\begin{pgfscope}%
\pgfpathrectangle{\pgfqpoint{0.380943in}{9.960189in}}{\pgfqpoint{4.650000in}{0.614151in}}%
\pgfusepath{clip}%
\pgfsetbuttcap%
\pgfsetroundjoin%
\definecolor{currentfill}{rgb}{1.000000,1.000000,0.929412}%
\pgfsetfillcolor{currentfill}%
\pgfsetlinewidth{0.250937pt}%
\definecolor{currentstroke}{rgb}{1.000000,1.000000,1.000000}%
\pgfsetstrokecolor{currentstroke}%
\pgfsetdash{}{0pt}%
\pgfpathmoveto{\pgfqpoint{0.731886in}{10.574340in}}%
\pgfpathlineto{\pgfqpoint{0.819622in}{10.574340in}}%
\pgfpathlineto{\pgfqpoint{0.819622in}{10.486604in}}%
\pgfpathlineto{\pgfqpoint{0.731886in}{10.486604in}}%
\pgfpathlineto{\pgfqpoint{0.731886in}{10.574340in}}%
\pgfusepath{stroke,fill}%
\end{pgfscope}%
\begin{pgfscope}%
\pgfpathrectangle{\pgfqpoint{0.380943in}{9.960189in}}{\pgfqpoint{4.650000in}{0.614151in}}%
\pgfusepath{clip}%
\pgfsetbuttcap%
\pgfsetroundjoin%
\definecolor{currentfill}{rgb}{1.000000,1.000000,0.929412}%
\pgfsetfillcolor{currentfill}%
\pgfsetlinewidth{0.250937pt}%
\definecolor{currentstroke}{rgb}{1.000000,1.000000,1.000000}%
\pgfsetstrokecolor{currentstroke}%
\pgfsetdash{}{0pt}%
\pgfpathmoveto{\pgfqpoint{0.819622in}{10.574340in}}%
\pgfpathlineto{\pgfqpoint{0.907358in}{10.574340in}}%
\pgfpathlineto{\pgfqpoint{0.907358in}{10.486604in}}%
\pgfpathlineto{\pgfqpoint{0.819622in}{10.486604in}}%
\pgfpathlineto{\pgfqpoint{0.819622in}{10.574340in}}%
\pgfusepath{stroke,fill}%
\end{pgfscope}%
\begin{pgfscope}%
\pgfpathrectangle{\pgfqpoint{0.380943in}{9.960189in}}{\pgfqpoint{4.650000in}{0.614151in}}%
\pgfusepath{clip}%
\pgfsetbuttcap%
\pgfsetroundjoin%
\definecolor{currentfill}{rgb}{1.000000,1.000000,0.929412}%
\pgfsetfillcolor{currentfill}%
\pgfsetlinewidth{0.250937pt}%
\definecolor{currentstroke}{rgb}{1.000000,1.000000,1.000000}%
\pgfsetstrokecolor{currentstroke}%
\pgfsetdash{}{0pt}%
\pgfpathmoveto{\pgfqpoint{0.907358in}{10.574340in}}%
\pgfpathlineto{\pgfqpoint{0.995094in}{10.574340in}}%
\pgfpathlineto{\pgfqpoint{0.995094in}{10.486604in}}%
\pgfpathlineto{\pgfqpoint{0.907358in}{10.486604in}}%
\pgfpathlineto{\pgfqpoint{0.907358in}{10.574340in}}%
\pgfusepath{stroke,fill}%
\end{pgfscope}%
\begin{pgfscope}%
\pgfpathrectangle{\pgfqpoint{0.380943in}{9.960189in}}{\pgfqpoint{4.650000in}{0.614151in}}%
\pgfusepath{clip}%
\pgfsetbuttcap%
\pgfsetroundjoin%
\definecolor{currentfill}{rgb}{1.000000,1.000000,0.929412}%
\pgfsetfillcolor{currentfill}%
\pgfsetlinewidth{0.250937pt}%
\definecolor{currentstroke}{rgb}{1.000000,1.000000,1.000000}%
\pgfsetstrokecolor{currentstroke}%
\pgfsetdash{}{0pt}%
\pgfpathmoveto{\pgfqpoint{0.995094in}{10.574340in}}%
\pgfpathlineto{\pgfqpoint{1.082830in}{10.574340in}}%
\pgfpathlineto{\pgfqpoint{1.082830in}{10.486604in}}%
\pgfpathlineto{\pgfqpoint{0.995094in}{10.486604in}}%
\pgfpathlineto{\pgfqpoint{0.995094in}{10.574340in}}%
\pgfusepath{stroke,fill}%
\end{pgfscope}%
\begin{pgfscope}%
\pgfpathrectangle{\pgfqpoint{0.380943in}{9.960189in}}{\pgfqpoint{4.650000in}{0.614151in}}%
\pgfusepath{clip}%
\pgfsetbuttcap%
\pgfsetroundjoin%
\definecolor{currentfill}{rgb}{1.000000,1.000000,0.929412}%
\pgfsetfillcolor{currentfill}%
\pgfsetlinewidth{0.250937pt}%
\definecolor{currentstroke}{rgb}{1.000000,1.000000,1.000000}%
\pgfsetstrokecolor{currentstroke}%
\pgfsetdash{}{0pt}%
\pgfpathmoveto{\pgfqpoint{1.082830in}{10.574340in}}%
\pgfpathlineto{\pgfqpoint{1.170566in}{10.574340in}}%
\pgfpathlineto{\pgfqpoint{1.170566in}{10.486604in}}%
\pgfpathlineto{\pgfqpoint{1.082830in}{10.486604in}}%
\pgfpathlineto{\pgfqpoint{1.082830in}{10.574340in}}%
\pgfusepath{stroke,fill}%
\end{pgfscope}%
\begin{pgfscope}%
\pgfpathrectangle{\pgfqpoint{0.380943in}{9.960189in}}{\pgfqpoint{4.650000in}{0.614151in}}%
\pgfusepath{clip}%
\pgfsetbuttcap%
\pgfsetroundjoin%
\definecolor{currentfill}{rgb}{1.000000,1.000000,0.929412}%
\pgfsetfillcolor{currentfill}%
\pgfsetlinewidth{0.250937pt}%
\definecolor{currentstroke}{rgb}{1.000000,1.000000,1.000000}%
\pgfsetstrokecolor{currentstroke}%
\pgfsetdash{}{0pt}%
\pgfpathmoveto{\pgfqpoint{1.170566in}{10.574340in}}%
\pgfpathlineto{\pgfqpoint{1.258302in}{10.574340in}}%
\pgfpathlineto{\pgfqpoint{1.258302in}{10.486604in}}%
\pgfpathlineto{\pgfqpoint{1.170566in}{10.486604in}}%
\pgfpathlineto{\pgfqpoint{1.170566in}{10.574340in}}%
\pgfusepath{stroke,fill}%
\end{pgfscope}%
\begin{pgfscope}%
\pgfpathrectangle{\pgfqpoint{0.380943in}{9.960189in}}{\pgfqpoint{4.650000in}{0.614151in}}%
\pgfusepath{clip}%
\pgfsetbuttcap%
\pgfsetroundjoin%
\definecolor{currentfill}{rgb}{1.000000,1.000000,0.929412}%
\pgfsetfillcolor{currentfill}%
\pgfsetlinewidth{0.250937pt}%
\definecolor{currentstroke}{rgb}{1.000000,1.000000,1.000000}%
\pgfsetstrokecolor{currentstroke}%
\pgfsetdash{}{0pt}%
\pgfpathmoveto{\pgfqpoint{1.258302in}{10.574340in}}%
\pgfpathlineto{\pgfqpoint{1.346037in}{10.574340in}}%
\pgfpathlineto{\pgfqpoint{1.346037in}{10.486604in}}%
\pgfpathlineto{\pgfqpoint{1.258302in}{10.486604in}}%
\pgfpathlineto{\pgfqpoint{1.258302in}{10.574340in}}%
\pgfusepath{stroke,fill}%
\end{pgfscope}%
\begin{pgfscope}%
\pgfpathrectangle{\pgfqpoint{0.380943in}{9.960189in}}{\pgfqpoint{4.650000in}{0.614151in}}%
\pgfusepath{clip}%
\pgfsetbuttcap%
\pgfsetroundjoin%
\definecolor{currentfill}{rgb}{1.000000,1.000000,0.929412}%
\pgfsetfillcolor{currentfill}%
\pgfsetlinewidth{0.250937pt}%
\definecolor{currentstroke}{rgb}{1.000000,1.000000,1.000000}%
\pgfsetstrokecolor{currentstroke}%
\pgfsetdash{}{0pt}%
\pgfpathmoveto{\pgfqpoint{1.346037in}{10.574340in}}%
\pgfpathlineto{\pgfqpoint{1.433773in}{10.574340in}}%
\pgfpathlineto{\pgfqpoint{1.433773in}{10.486604in}}%
\pgfpathlineto{\pgfqpoint{1.346037in}{10.486604in}}%
\pgfpathlineto{\pgfqpoint{1.346037in}{10.574340in}}%
\pgfusepath{stroke,fill}%
\end{pgfscope}%
\begin{pgfscope}%
\pgfpathrectangle{\pgfqpoint{0.380943in}{9.960189in}}{\pgfqpoint{4.650000in}{0.614151in}}%
\pgfusepath{clip}%
\pgfsetbuttcap%
\pgfsetroundjoin%
\definecolor{currentfill}{rgb}{1.000000,1.000000,0.929412}%
\pgfsetfillcolor{currentfill}%
\pgfsetlinewidth{0.250937pt}%
\definecolor{currentstroke}{rgb}{1.000000,1.000000,1.000000}%
\pgfsetstrokecolor{currentstroke}%
\pgfsetdash{}{0pt}%
\pgfpathmoveto{\pgfqpoint{1.433773in}{10.574340in}}%
\pgfpathlineto{\pgfqpoint{1.521509in}{10.574340in}}%
\pgfpathlineto{\pgfqpoint{1.521509in}{10.486604in}}%
\pgfpathlineto{\pgfqpoint{1.433773in}{10.486604in}}%
\pgfpathlineto{\pgfqpoint{1.433773in}{10.574340in}}%
\pgfusepath{stroke,fill}%
\end{pgfscope}%
\begin{pgfscope}%
\pgfpathrectangle{\pgfqpoint{0.380943in}{9.960189in}}{\pgfqpoint{4.650000in}{0.614151in}}%
\pgfusepath{clip}%
\pgfsetbuttcap%
\pgfsetroundjoin%
\definecolor{currentfill}{rgb}{1.000000,1.000000,0.929412}%
\pgfsetfillcolor{currentfill}%
\pgfsetlinewidth{0.250937pt}%
\definecolor{currentstroke}{rgb}{1.000000,1.000000,1.000000}%
\pgfsetstrokecolor{currentstroke}%
\pgfsetdash{}{0pt}%
\pgfpathmoveto{\pgfqpoint{1.521509in}{10.574340in}}%
\pgfpathlineto{\pgfqpoint{1.609245in}{10.574340in}}%
\pgfpathlineto{\pgfqpoint{1.609245in}{10.486604in}}%
\pgfpathlineto{\pgfqpoint{1.521509in}{10.486604in}}%
\pgfpathlineto{\pgfqpoint{1.521509in}{10.574340in}}%
\pgfusepath{stroke,fill}%
\end{pgfscope}%
\begin{pgfscope}%
\pgfpathrectangle{\pgfqpoint{0.380943in}{9.960189in}}{\pgfqpoint{4.650000in}{0.614151in}}%
\pgfusepath{clip}%
\pgfsetbuttcap%
\pgfsetroundjoin%
\definecolor{currentfill}{rgb}{1.000000,1.000000,0.929412}%
\pgfsetfillcolor{currentfill}%
\pgfsetlinewidth{0.250937pt}%
\definecolor{currentstroke}{rgb}{1.000000,1.000000,1.000000}%
\pgfsetstrokecolor{currentstroke}%
\pgfsetdash{}{0pt}%
\pgfpathmoveto{\pgfqpoint{1.609245in}{10.574340in}}%
\pgfpathlineto{\pgfqpoint{1.696981in}{10.574340in}}%
\pgfpathlineto{\pgfqpoint{1.696981in}{10.486604in}}%
\pgfpathlineto{\pgfqpoint{1.609245in}{10.486604in}}%
\pgfpathlineto{\pgfqpoint{1.609245in}{10.574340in}}%
\pgfusepath{stroke,fill}%
\end{pgfscope}%
\begin{pgfscope}%
\pgfpathrectangle{\pgfqpoint{0.380943in}{9.960189in}}{\pgfqpoint{4.650000in}{0.614151in}}%
\pgfusepath{clip}%
\pgfsetbuttcap%
\pgfsetroundjoin%
\definecolor{currentfill}{rgb}{1.000000,1.000000,0.929412}%
\pgfsetfillcolor{currentfill}%
\pgfsetlinewidth{0.250937pt}%
\definecolor{currentstroke}{rgb}{1.000000,1.000000,1.000000}%
\pgfsetstrokecolor{currentstroke}%
\pgfsetdash{}{0pt}%
\pgfpathmoveto{\pgfqpoint{1.696981in}{10.574340in}}%
\pgfpathlineto{\pgfqpoint{1.784717in}{10.574340in}}%
\pgfpathlineto{\pgfqpoint{1.784717in}{10.486604in}}%
\pgfpathlineto{\pgfqpoint{1.696981in}{10.486604in}}%
\pgfpathlineto{\pgfqpoint{1.696981in}{10.574340in}}%
\pgfusepath{stroke,fill}%
\end{pgfscope}%
\begin{pgfscope}%
\pgfpathrectangle{\pgfqpoint{0.380943in}{9.960189in}}{\pgfqpoint{4.650000in}{0.614151in}}%
\pgfusepath{clip}%
\pgfsetbuttcap%
\pgfsetroundjoin%
\definecolor{currentfill}{rgb}{1.000000,1.000000,0.929412}%
\pgfsetfillcolor{currentfill}%
\pgfsetlinewidth{0.250937pt}%
\definecolor{currentstroke}{rgb}{1.000000,1.000000,1.000000}%
\pgfsetstrokecolor{currentstroke}%
\pgfsetdash{}{0pt}%
\pgfpathmoveto{\pgfqpoint{1.784717in}{10.574340in}}%
\pgfpathlineto{\pgfqpoint{1.872452in}{10.574340in}}%
\pgfpathlineto{\pgfqpoint{1.872452in}{10.486604in}}%
\pgfpathlineto{\pgfqpoint{1.784717in}{10.486604in}}%
\pgfpathlineto{\pgfqpoint{1.784717in}{10.574340in}}%
\pgfusepath{stroke,fill}%
\end{pgfscope}%
\begin{pgfscope}%
\pgfpathrectangle{\pgfqpoint{0.380943in}{9.960189in}}{\pgfqpoint{4.650000in}{0.614151in}}%
\pgfusepath{clip}%
\pgfsetbuttcap%
\pgfsetroundjoin%
\definecolor{currentfill}{rgb}{1.000000,1.000000,0.929412}%
\pgfsetfillcolor{currentfill}%
\pgfsetlinewidth{0.250937pt}%
\definecolor{currentstroke}{rgb}{1.000000,1.000000,1.000000}%
\pgfsetstrokecolor{currentstroke}%
\pgfsetdash{}{0pt}%
\pgfpathmoveto{\pgfqpoint{1.872452in}{10.574340in}}%
\pgfpathlineto{\pgfqpoint{1.960188in}{10.574340in}}%
\pgfpathlineto{\pgfqpoint{1.960188in}{10.486604in}}%
\pgfpathlineto{\pgfqpoint{1.872452in}{10.486604in}}%
\pgfpathlineto{\pgfqpoint{1.872452in}{10.574340in}}%
\pgfusepath{stroke,fill}%
\end{pgfscope}%
\begin{pgfscope}%
\pgfpathrectangle{\pgfqpoint{0.380943in}{9.960189in}}{\pgfqpoint{4.650000in}{0.614151in}}%
\pgfusepath{clip}%
\pgfsetbuttcap%
\pgfsetroundjoin%
\definecolor{currentfill}{rgb}{1.000000,1.000000,0.929412}%
\pgfsetfillcolor{currentfill}%
\pgfsetlinewidth{0.250937pt}%
\definecolor{currentstroke}{rgb}{1.000000,1.000000,1.000000}%
\pgfsetstrokecolor{currentstroke}%
\pgfsetdash{}{0pt}%
\pgfpathmoveto{\pgfqpoint{1.960188in}{10.574340in}}%
\pgfpathlineto{\pgfqpoint{2.047924in}{10.574340in}}%
\pgfpathlineto{\pgfqpoint{2.047924in}{10.486604in}}%
\pgfpathlineto{\pgfqpoint{1.960188in}{10.486604in}}%
\pgfpathlineto{\pgfqpoint{1.960188in}{10.574340in}}%
\pgfusepath{stroke,fill}%
\end{pgfscope}%
\begin{pgfscope}%
\pgfpathrectangle{\pgfqpoint{0.380943in}{9.960189in}}{\pgfqpoint{4.650000in}{0.614151in}}%
\pgfusepath{clip}%
\pgfsetbuttcap%
\pgfsetroundjoin%
\definecolor{currentfill}{rgb}{1.000000,1.000000,0.929412}%
\pgfsetfillcolor{currentfill}%
\pgfsetlinewidth{0.250937pt}%
\definecolor{currentstroke}{rgb}{1.000000,1.000000,1.000000}%
\pgfsetstrokecolor{currentstroke}%
\pgfsetdash{}{0pt}%
\pgfpathmoveto{\pgfqpoint{2.047924in}{10.574340in}}%
\pgfpathlineto{\pgfqpoint{2.135660in}{10.574340in}}%
\pgfpathlineto{\pgfqpoint{2.135660in}{10.486604in}}%
\pgfpathlineto{\pgfqpoint{2.047924in}{10.486604in}}%
\pgfpathlineto{\pgfqpoint{2.047924in}{10.574340in}}%
\pgfusepath{stroke,fill}%
\end{pgfscope}%
\begin{pgfscope}%
\pgfpathrectangle{\pgfqpoint{0.380943in}{9.960189in}}{\pgfqpoint{4.650000in}{0.614151in}}%
\pgfusepath{clip}%
\pgfsetbuttcap%
\pgfsetroundjoin%
\definecolor{currentfill}{rgb}{1.000000,1.000000,0.929412}%
\pgfsetfillcolor{currentfill}%
\pgfsetlinewidth{0.250937pt}%
\definecolor{currentstroke}{rgb}{1.000000,1.000000,1.000000}%
\pgfsetstrokecolor{currentstroke}%
\pgfsetdash{}{0pt}%
\pgfpathmoveto{\pgfqpoint{2.135660in}{10.574340in}}%
\pgfpathlineto{\pgfqpoint{2.223396in}{10.574340in}}%
\pgfpathlineto{\pgfqpoint{2.223396in}{10.486604in}}%
\pgfpathlineto{\pgfqpoint{2.135660in}{10.486604in}}%
\pgfpathlineto{\pgfqpoint{2.135660in}{10.574340in}}%
\pgfusepath{stroke,fill}%
\end{pgfscope}%
\begin{pgfscope}%
\pgfpathrectangle{\pgfqpoint{0.380943in}{9.960189in}}{\pgfqpoint{4.650000in}{0.614151in}}%
\pgfusepath{clip}%
\pgfsetbuttcap%
\pgfsetroundjoin%
\definecolor{currentfill}{rgb}{1.000000,1.000000,0.929412}%
\pgfsetfillcolor{currentfill}%
\pgfsetlinewidth{0.250937pt}%
\definecolor{currentstroke}{rgb}{1.000000,1.000000,1.000000}%
\pgfsetstrokecolor{currentstroke}%
\pgfsetdash{}{0pt}%
\pgfpathmoveto{\pgfqpoint{2.223396in}{10.574340in}}%
\pgfpathlineto{\pgfqpoint{2.311132in}{10.574340in}}%
\pgfpathlineto{\pgfqpoint{2.311132in}{10.486604in}}%
\pgfpathlineto{\pgfqpoint{2.223396in}{10.486604in}}%
\pgfpathlineto{\pgfqpoint{2.223396in}{10.574340in}}%
\pgfusepath{stroke,fill}%
\end{pgfscope}%
\begin{pgfscope}%
\pgfpathrectangle{\pgfqpoint{0.380943in}{9.960189in}}{\pgfqpoint{4.650000in}{0.614151in}}%
\pgfusepath{clip}%
\pgfsetbuttcap%
\pgfsetroundjoin%
\definecolor{currentfill}{rgb}{1.000000,1.000000,0.929412}%
\pgfsetfillcolor{currentfill}%
\pgfsetlinewidth{0.250937pt}%
\definecolor{currentstroke}{rgb}{1.000000,1.000000,1.000000}%
\pgfsetstrokecolor{currentstroke}%
\pgfsetdash{}{0pt}%
\pgfpathmoveto{\pgfqpoint{2.311132in}{10.574340in}}%
\pgfpathlineto{\pgfqpoint{2.398868in}{10.574340in}}%
\pgfpathlineto{\pgfqpoint{2.398868in}{10.486604in}}%
\pgfpathlineto{\pgfqpoint{2.311132in}{10.486604in}}%
\pgfpathlineto{\pgfqpoint{2.311132in}{10.574340in}}%
\pgfusepath{stroke,fill}%
\end{pgfscope}%
\begin{pgfscope}%
\pgfpathrectangle{\pgfqpoint{0.380943in}{9.960189in}}{\pgfqpoint{4.650000in}{0.614151in}}%
\pgfusepath{clip}%
\pgfsetbuttcap%
\pgfsetroundjoin%
\definecolor{currentfill}{rgb}{1.000000,1.000000,0.929412}%
\pgfsetfillcolor{currentfill}%
\pgfsetlinewidth{0.250937pt}%
\definecolor{currentstroke}{rgb}{1.000000,1.000000,1.000000}%
\pgfsetstrokecolor{currentstroke}%
\pgfsetdash{}{0pt}%
\pgfpathmoveto{\pgfqpoint{2.398868in}{10.574340in}}%
\pgfpathlineto{\pgfqpoint{2.486603in}{10.574340in}}%
\pgfpathlineto{\pgfqpoint{2.486603in}{10.486604in}}%
\pgfpathlineto{\pgfqpoint{2.398868in}{10.486604in}}%
\pgfpathlineto{\pgfqpoint{2.398868in}{10.574340in}}%
\pgfusepath{stroke,fill}%
\end{pgfscope}%
\begin{pgfscope}%
\pgfpathrectangle{\pgfqpoint{0.380943in}{9.960189in}}{\pgfqpoint{4.650000in}{0.614151in}}%
\pgfusepath{clip}%
\pgfsetbuttcap%
\pgfsetroundjoin%
\definecolor{currentfill}{rgb}{1.000000,1.000000,0.929412}%
\pgfsetfillcolor{currentfill}%
\pgfsetlinewidth{0.250937pt}%
\definecolor{currentstroke}{rgb}{1.000000,1.000000,1.000000}%
\pgfsetstrokecolor{currentstroke}%
\pgfsetdash{}{0pt}%
\pgfpathmoveto{\pgfqpoint{2.486603in}{10.574340in}}%
\pgfpathlineto{\pgfqpoint{2.574339in}{10.574340in}}%
\pgfpathlineto{\pgfqpoint{2.574339in}{10.486604in}}%
\pgfpathlineto{\pgfqpoint{2.486603in}{10.486604in}}%
\pgfpathlineto{\pgfqpoint{2.486603in}{10.574340in}}%
\pgfusepath{stroke,fill}%
\end{pgfscope}%
\begin{pgfscope}%
\pgfpathrectangle{\pgfqpoint{0.380943in}{9.960189in}}{\pgfqpoint{4.650000in}{0.614151in}}%
\pgfusepath{clip}%
\pgfsetbuttcap%
\pgfsetroundjoin%
\definecolor{currentfill}{rgb}{1.000000,1.000000,0.929412}%
\pgfsetfillcolor{currentfill}%
\pgfsetlinewidth{0.250937pt}%
\definecolor{currentstroke}{rgb}{1.000000,1.000000,1.000000}%
\pgfsetstrokecolor{currentstroke}%
\pgfsetdash{}{0pt}%
\pgfpathmoveto{\pgfqpoint{2.574339in}{10.574340in}}%
\pgfpathlineto{\pgfqpoint{2.662075in}{10.574340in}}%
\pgfpathlineto{\pgfqpoint{2.662075in}{10.486604in}}%
\pgfpathlineto{\pgfqpoint{2.574339in}{10.486604in}}%
\pgfpathlineto{\pgfqpoint{2.574339in}{10.574340in}}%
\pgfusepath{stroke,fill}%
\end{pgfscope}%
\begin{pgfscope}%
\pgfpathrectangle{\pgfqpoint{0.380943in}{9.960189in}}{\pgfqpoint{4.650000in}{0.614151in}}%
\pgfusepath{clip}%
\pgfsetbuttcap%
\pgfsetroundjoin%
\definecolor{currentfill}{rgb}{1.000000,1.000000,0.929412}%
\pgfsetfillcolor{currentfill}%
\pgfsetlinewidth{0.250937pt}%
\definecolor{currentstroke}{rgb}{1.000000,1.000000,1.000000}%
\pgfsetstrokecolor{currentstroke}%
\pgfsetdash{}{0pt}%
\pgfpathmoveto{\pgfqpoint{2.662075in}{10.574340in}}%
\pgfpathlineto{\pgfqpoint{2.749811in}{10.574340in}}%
\pgfpathlineto{\pgfqpoint{2.749811in}{10.486604in}}%
\pgfpathlineto{\pgfqpoint{2.662075in}{10.486604in}}%
\pgfpathlineto{\pgfqpoint{2.662075in}{10.574340in}}%
\pgfusepath{stroke,fill}%
\end{pgfscope}%
\begin{pgfscope}%
\pgfpathrectangle{\pgfqpoint{0.380943in}{9.960189in}}{\pgfqpoint{4.650000in}{0.614151in}}%
\pgfusepath{clip}%
\pgfsetbuttcap%
\pgfsetroundjoin%
\definecolor{currentfill}{rgb}{1.000000,1.000000,0.929412}%
\pgfsetfillcolor{currentfill}%
\pgfsetlinewidth{0.250937pt}%
\definecolor{currentstroke}{rgb}{1.000000,1.000000,1.000000}%
\pgfsetstrokecolor{currentstroke}%
\pgfsetdash{}{0pt}%
\pgfpathmoveto{\pgfqpoint{2.749811in}{10.574340in}}%
\pgfpathlineto{\pgfqpoint{2.837547in}{10.574340in}}%
\pgfpathlineto{\pgfqpoint{2.837547in}{10.486604in}}%
\pgfpathlineto{\pgfqpoint{2.749811in}{10.486604in}}%
\pgfpathlineto{\pgfqpoint{2.749811in}{10.574340in}}%
\pgfusepath{stroke,fill}%
\end{pgfscope}%
\begin{pgfscope}%
\pgfpathrectangle{\pgfqpoint{0.380943in}{9.960189in}}{\pgfqpoint{4.650000in}{0.614151in}}%
\pgfusepath{clip}%
\pgfsetbuttcap%
\pgfsetroundjoin%
\definecolor{currentfill}{rgb}{1.000000,1.000000,0.929412}%
\pgfsetfillcolor{currentfill}%
\pgfsetlinewidth{0.250937pt}%
\definecolor{currentstroke}{rgb}{1.000000,1.000000,1.000000}%
\pgfsetstrokecolor{currentstroke}%
\pgfsetdash{}{0pt}%
\pgfpathmoveto{\pgfqpoint{2.837547in}{10.574340in}}%
\pgfpathlineto{\pgfqpoint{2.925283in}{10.574340in}}%
\pgfpathlineto{\pgfqpoint{2.925283in}{10.486604in}}%
\pgfpathlineto{\pgfqpoint{2.837547in}{10.486604in}}%
\pgfpathlineto{\pgfqpoint{2.837547in}{10.574340in}}%
\pgfusepath{stroke,fill}%
\end{pgfscope}%
\begin{pgfscope}%
\pgfpathrectangle{\pgfqpoint{0.380943in}{9.960189in}}{\pgfqpoint{4.650000in}{0.614151in}}%
\pgfusepath{clip}%
\pgfsetbuttcap%
\pgfsetroundjoin%
\definecolor{currentfill}{rgb}{1.000000,1.000000,0.929412}%
\pgfsetfillcolor{currentfill}%
\pgfsetlinewidth{0.250937pt}%
\definecolor{currentstroke}{rgb}{1.000000,1.000000,1.000000}%
\pgfsetstrokecolor{currentstroke}%
\pgfsetdash{}{0pt}%
\pgfpathmoveto{\pgfqpoint{2.925283in}{10.574340in}}%
\pgfpathlineto{\pgfqpoint{3.013019in}{10.574340in}}%
\pgfpathlineto{\pgfqpoint{3.013019in}{10.486604in}}%
\pgfpathlineto{\pgfqpoint{2.925283in}{10.486604in}}%
\pgfpathlineto{\pgfqpoint{2.925283in}{10.574340in}}%
\pgfusepath{stroke,fill}%
\end{pgfscope}%
\begin{pgfscope}%
\pgfpathrectangle{\pgfqpoint{0.380943in}{9.960189in}}{\pgfqpoint{4.650000in}{0.614151in}}%
\pgfusepath{clip}%
\pgfsetbuttcap%
\pgfsetroundjoin%
\definecolor{currentfill}{rgb}{1.000000,1.000000,0.929412}%
\pgfsetfillcolor{currentfill}%
\pgfsetlinewidth{0.250937pt}%
\definecolor{currentstroke}{rgb}{1.000000,1.000000,1.000000}%
\pgfsetstrokecolor{currentstroke}%
\pgfsetdash{}{0pt}%
\pgfpathmoveto{\pgfqpoint{3.013019in}{10.574340in}}%
\pgfpathlineto{\pgfqpoint{3.100754in}{10.574340in}}%
\pgfpathlineto{\pgfqpoint{3.100754in}{10.486604in}}%
\pgfpathlineto{\pgfqpoint{3.013019in}{10.486604in}}%
\pgfpathlineto{\pgfqpoint{3.013019in}{10.574340in}}%
\pgfusepath{stroke,fill}%
\end{pgfscope}%
\begin{pgfscope}%
\pgfpathrectangle{\pgfqpoint{0.380943in}{9.960189in}}{\pgfqpoint{4.650000in}{0.614151in}}%
\pgfusepath{clip}%
\pgfsetbuttcap%
\pgfsetroundjoin%
\definecolor{currentfill}{rgb}{1.000000,1.000000,0.929412}%
\pgfsetfillcolor{currentfill}%
\pgfsetlinewidth{0.250937pt}%
\definecolor{currentstroke}{rgb}{1.000000,1.000000,1.000000}%
\pgfsetstrokecolor{currentstroke}%
\pgfsetdash{}{0pt}%
\pgfpathmoveto{\pgfqpoint{3.100754in}{10.574340in}}%
\pgfpathlineto{\pgfqpoint{3.188490in}{10.574340in}}%
\pgfpathlineto{\pgfqpoint{3.188490in}{10.486604in}}%
\pgfpathlineto{\pgfqpoint{3.100754in}{10.486604in}}%
\pgfpathlineto{\pgfqpoint{3.100754in}{10.574340in}}%
\pgfusepath{stroke,fill}%
\end{pgfscope}%
\begin{pgfscope}%
\pgfpathrectangle{\pgfqpoint{0.380943in}{9.960189in}}{\pgfqpoint{4.650000in}{0.614151in}}%
\pgfusepath{clip}%
\pgfsetbuttcap%
\pgfsetroundjoin%
\definecolor{currentfill}{rgb}{1.000000,1.000000,0.929412}%
\pgfsetfillcolor{currentfill}%
\pgfsetlinewidth{0.250937pt}%
\definecolor{currentstroke}{rgb}{1.000000,1.000000,1.000000}%
\pgfsetstrokecolor{currentstroke}%
\pgfsetdash{}{0pt}%
\pgfpathmoveto{\pgfqpoint{3.188490in}{10.574340in}}%
\pgfpathlineto{\pgfqpoint{3.276226in}{10.574340in}}%
\pgfpathlineto{\pgfqpoint{3.276226in}{10.486604in}}%
\pgfpathlineto{\pgfqpoint{3.188490in}{10.486604in}}%
\pgfpathlineto{\pgfqpoint{3.188490in}{10.574340in}}%
\pgfusepath{stroke,fill}%
\end{pgfscope}%
\begin{pgfscope}%
\pgfpathrectangle{\pgfqpoint{0.380943in}{9.960189in}}{\pgfqpoint{4.650000in}{0.614151in}}%
\pgfusepath{clip}%
\pgfsetbuttcap%
\pgfsetroundjoin%
\definecolor{currentfill}{rgb}{1.000000,1.000000,0.929412}%
\pgfsetfillcolor{currentfill}%
\pgfsetlinewidth{0.250937pt}%
\definecolor{currentstroke}{rgb}{1.000000,1.000000,1.000000}%
\pgfsetstrokecolor{currentstroke}%
\pgfsetdash{}{0pt}%
\pgfpathmoveto{\pgfqpoint{3.276226in}{10.574340in}}%
\pgfpathlineto{\pgfqpoint{3.363962in}{10.574340in}}%
\pgfpathlineto{\pgfqpoint{3.363962in}{10.486604in}}%
\pgfpathlineto{\pgfqpoint{3.276226in}{10.486604in}}%
\pgfpathlineto{\pgfqpoint{3.276226in}{10.574340in}}%
\pgfusepath{stroke,fill}%
\end{pgfscope}%
\begin{pgfscope}%
\pgfpathrectangle{\pgfqpoint{0.380943in}{9.960189in}}{\pgfqpoint{4.650000in}{0.614151in}}%
\pgfusepath{clip}%
\pgfsetbuttcap%
\pgfsetroundjoin%
\definecolor{currentfill}{rgb}{1.000000,1.000000,0.929412}%
\pgfsetfillcolor{currentfill}%
\pgfsetlinewidth{0.250937pt}%
\definecolor{currentstroke}{rgb}{1.000000,1.000000,1.000000}%
\pgfsetstrokecolor{currentstroke}%
\pgfsetdash{}{0pt}%
\pgfpathmoveto{\pgfqpoint{3.363962in}{10.574340in}}%
\pgfpathlineto{\pgfqpoint{3.451698in}{10.574340in}}%
\pgfpathlineto{\pgfqpoint{3.451698in}{10.486604in}}%
\pgfpathlineto{\pgfqpoint{3.363962in}{10.486604in}}%
\pgfpathlineto{\pgfqpoint{3.363962in}{10.574340in}}%
\pgfusepath{stroke,fill}%
\end{pgfscope}%
\begin{pgfscope}%
\pgfpathrectangle{\pgfqpoint{0.380943in}{9.960189in}}{\pgfqpoint{4.650000in}{0.614151in}}%
\pgfusepath{clip}%
\pgfsetbuttcap%
\pgfsetroundjoin%
\definecolor{currentfill}{rgb}{1.000000,1.000000,0.929412}%
\pgfsetfillcolor{currentfill}%
\pgfsetlinewidth{0.250937pt}%
\definecolor{currentstroke}{rgb}{1.000000,1.000000,1.000000}%
\pgfsetstrokecolor{currentstroke}%
\pgfsetdash{}{0pt}%
\pgfpathmoveto{\pgfqpoint{3.451698in}{10.574340in}}%
\pgfpathlineto{\pgfqpoint{3.539434in}{10.574340in}}%
\pgfpathlineto{\pgfqpoint{3.539434in}{10.486604in}}%
\pgfpathlineto{\pgfqpoint{3.451698in}{10.486604in}}%
\pgfpathlineto{\pgfqpoint{3.451698in}{10.574340in}}%
\pgfusepath{stroke,fill}%
\end{pgfscope}%
\begin{pgfscope}%
\pgfpathrectangle{\pgfqpoint{0.380943in}{9.960189in}}{\pgfqpoint{4.650000in}{0.614151in}}%
\pgfusepath{clip}%
\pgfsetbuttcap%
\pgfsetroundjoin%
\definecolor{currentfill}{rgb}{1.000000,1.000000,0.929412}%
\pgfsetfillcolor{currentfill}%
\pgfsetlinewidth{0.250937pt}%
\definecolor{currentstroke}{rgb}{1.000000,1.000000,1.000000}%
\pgfsetstrokecolor{currentstroke}%
\pgfsetdash{}{0pt}%
\pgfpathmoveto{\pgfqpoint{3.539434in}{10.574340in}}%
\pgfpathlineto{\pgfqpoint{3.627169in}{10.574340in}}%
\pgfpathlineto{\pgfqpoint{3.627169in}{10.486604in}}%
\pgfpathlineto{\pgfqpoint{3.539434in}{10.486604in}}%
\pgfpathlineto{\pgfqpoint{3.539434in}{10.574340in}}%
\pgfusepath{stroke,fill}%
\end{pgfscope}%
\begin{pgfscope}%
\pgfpathrectangle{\pgfqpoint{0.380943in}{9.960189in}}{\pgfqpoint{4.650000in}{0.614151in}}%
\pgfusepath{clip}%
\pgfsetbuttcap%
\pgfsetroundjoin%
\definecolor{currentfill}{rgb}{1.000000,1.000000,0.929412}%
\pgfsetfillcolor{currentfill}%
\pgfsetlinewidth{0.250937pt}%
\definecolor{currentstroke}{rgb}{1.000000,1.000000,1.000000}%
\pgfsetstrokecolor{currentstroke}%
\pgfsetdash{}{0pt}%
\pgfpathmoveto{\pgfqpoint{3.627169in}{10.574340in}}%
\pgfpathlineto{\pgfqpoint{3.714905in}{10.574340in}}%
\pgfpathlineto{\pgfqpoint{3.714905in}{10.486604in}}%
\pgfpathlineto{\pgfqpoint{3.627169in}{10.486604in}}%
\pgfpathlineto{\pgfqpoint{3.627169in}{10.574340in}}%
\pgfusepath{stroke,fill}%
\end{pgfscope}%
\begin{pgfscope}%
\pgfpathrectangle{\pgfqpoint{0.380943in}{9.960189in}}{\pgfqpoint{4.650000in}{0.614151in}}%
\pgfusepath{clip}%
\pgfsetbuttcap%
\pgfsetroundjoin%
\definecolor{currentfill}{rgb}{1.000000,1.000000,0.929412}%
\pgfsetfillcolor{currentfill}%
\pgfsetlinewidth{0.250937pt}%
\definecolor{currentstroke}{rgb}{1.000000,1.000000,1.000000}%
\pgfsetstrokecolor{currentstroke}%
\pgfsetdash{}{0pt}%
\pgfpathmoveto{\pgfqpoint{3.714905in}{10.574340in}}%
\pgfpathlineto{\pgfqpoint{3.802641in}{10.574340in}}%
\pgfpathlineto{\pgfqpoint{3.802641in}{10.486604in}}%
\pgfpathlineto{\pgfqpoint{3.714905in}{10.486604in}}%
\pgfpathlineto{\pgfqpoint{3.714905in}{10.574340in}}%
\pgfusepath{stroke,fill}%
\end{pgfscope}%
\begin{pgfscope}%
\pgfpathrectangle{\pgfqpoint{0.380943in}{9.960189in}}{\pgfqpoint{4.650000in}{0.614151in}}%
\pgfusepath{clip}%
\pgfsetbuttcap%
\pgfsetroundjoin%
\definecolor{currentfill}{rgb}{1.000000,1.000000,0.929412}%
\pgfsetfillcolor{currentfill}%
\pgfsetlinewidth{0.250937pt}%
\definecolor{currentstroke}{rgb}{1.000000,1.000000,1.000000}%
\pgfsetstrokecolor{currentstroke}%
\pgfsetdash{}{0pt}%
\pgfpathmoveto{\pgfqpoint{3.802641in}{10.574340in}}%
\pgfpathlineto{\pgfqpoint{3.890377in}{10.574340in}}%
\pgfpathlineto{\pgfqpoint{3.890377in}{10.486604in}}%
\pgfpathlineto{\pgfqpoint{3.802641in}{10.486604in}}%
\pgfpathlineto{\pgfqpoint{3.802641in}{10.574340in}}%
\pgfusepath{stroke,fill}%
\end{pgfscope}%
\begin{pgfscope}%
\pgfpathrectangle{\pgfqpoint{0.380943in}{9.960189in}}{\pgfqpoint{4.650000in}{0.614151in}}%
\pgfusepath{clip}%
\pgfsetbuttcap%
\pgfsetroundjoin%
\definecolor{currentfill}{rgb}{1.000000,1.000000,0.929412}%
\pgfsetfillcolor{currentfill}%
\pgfsetlinewidth{0.250937pt}%
\definecolor{currentstroke}{rgb}{1.000000,1.000000,1.000000}%
\pgfsetstrokecolor{currentstroke}%
\pgfsetdash{}{0pt}%
\pgfpathmoveto{\pgfqpoint{3.890377in}{10.574340in}}%
\pgfpathlineto{\pgfqpoint{3.978113in}{10.574340in}}%
\pgfpathlineto{\pgfqpoint{3.978113in}{10.486604in}}%
\pgfpathlineto{\pgfqpoint{3.890377in}{10.486604in}}%
\pgfpathlineto{\pgfqpoint{3.890377in}{10.574340in}}%
\pgfusepath{stroke,fill}%
\end{pgfscope}%
\begin{pgfscope}%
\pgfpathrectangle{\pgfqpoint{0.380943in}{9.960189in}}{\pgfqpoint{4.650000in}{0.614151in}}%
\pgfusepath{clip}%
\pgfsetbuttcap%
\pgfsetroundjoin%
\definecolor{currentfill}{rgb}{1.000000,1.000000,0.929412}%
\pgfsetfillcolor{currentfill}%
\pgfsetlinewidth{0.250937pt}%
\definecolor{currentstroke}{rgb}{1.000000,1.000000,1.000000}%
\pgfsetstrokecolor{currentstroke}%
\pgfsetdash{}{0pt}%
\pgfpathmoveto{\pgfqpoint{3.978113in}{10.574340in}}%
\pgfpathlineto{\pgfqpoint{4.065849in}{10.574340in}}%
\pgfpathlineto{\pgfqpoint{4.065849in}{10.486604in}}%
\pgfpathlineto{\pgfqpoint{3.978113in}{10.486604in}}%
\pgfpathlineto{\pgfqpoint{3.978113in}{10.574340in}}%
\pgfusepath{stroke,fill}%
\end{pgfscope}%
\begin{pgfscope}%
\pgfpathrectangle{\pgfqpoint{0.380943in}{9.960189in}}{\pgfqpoint{4.650000in}{0.614151in}}%
\pgfusepath{clip}%
\pgfsetbuttcap%
\pgfsetroundjoin%
\definecolor{currentfill}{rgb}{1.000000,1.000000,0.929412}%
\pgfsetfillcolor{currentfill}%
\pgfsetlinewidth{0.250937pt}%
\definecolor{currentstroke}{rgb}{1.000000,1.000000,1.000000}%
\pgfsetstrokecolor{currentstroke}%
\pgfsetdash{}{0pt}%
\pgfpathmoveto{\pgfqpoint{4.065849in}{10.574340in}}%
\pgfpathlineto{\pgfqpoint{4.153585in}{10.574340in}}%
\pgfpathlineto{\pgfqpoint{4.153585in}{10.486604in}}%
\pgfpathlineto{\pgfqpoint{4.065849in}{10.486604in}}%
\pgfpathlineto{\pgfqpoint{4.065849in}{10.574340in}}%
\pgfusepath{stroke,fill}%
\end{pgfscope}%
\begin{pgfscope}%
\pgfpathrectangle{\pgfqpoint{0.380943in}{9.960189in}}{\pgfqpoint{4.650000in}{0.614151in}}%
\pgfusepath{clip}%
\pgfsetbuttcap%
\pgfsetroundjoin%
\definecolor{currentfill}{rgb}{1.000000,1.000000,0.929412}%
\pgfsetfillcolor{currentfill}%
\pgfsetlinewidth{0.250937pt}%
\definecolor{currentstroke}{rgb}{1.000000,1.000000,1.000000}%
\pgfsetstrokecolor{currentstroke}%
\pgfsetdash{}{0pt}%
\pgfpathmoveto{\pgfqpoint{4.153585in}{10.574340in}}%
\pgfpathlineto{\pgfqpoint{4.241320in}{10.574340in}}%
\pgfpathlineto{\pgfqpoint{4.241320in}{10.486604in}}%
\pgfpathlineto{\pgfqpoint{4.153585in}{10.486604in}}%
\pgfpathlineto{\pgfqpoint{4.153585in}{10.574340in}}%
\pgfusepath{stroke,fill}%
\end{pgfscope}%
\begin{pgfscope}%
\pgfpathrectangle{\pgfqpoint{0.380943in}{9.960189in}}{\pgfqpoint{4.650000in}{0.614151in}}%
\pgfusepath{clip}%
\pgfsetbuttcap%
\pgfsetroundjoin%
\definecolor{currentfill}{rgb}{1.000000,1.000000,0.929412}%
\pgfsetfillcolor{currentfill}%
\pgfsetlinewidth{0.250937pt}%
\definecolor{currentstroke}{rgb}{1.000000,1.000000,1.000000}%
\pgfsetstrokecolor{currentstroke}%
\pgfsetdash{}{0pt}%
\pgfpathmoveto{\pgfqpoint{4.241320in}{10.574340in}}%
\pgfpathlineto{\pgfqpoint{4.329056in}{10.574340in}}%
\pgfpathlineto{\pgfqpoint{4.329056in}{10.486604in}}%
\pgfpathlineto{\pgfqpoint{4.241320in}{10.486604in}}%
\pgfpathlineto{\pgfqpoint{4.241320in}{10.574340in}}%
\pgfusepath{stroke,fill}%
\end{pgfscope}%
\begin{pgfscope}%
\pgfpathrectangle{\pgfqpoint{0.380943in}{9.960189in}}{\pgfqpoint{4.650000in}{0.614151in}}%
\pgfusepath{clip}%
\pgfsetbuttcap%
\pgfsetroundjoin%
\definecolor{currentfill}{rgb}{1.000000,1.000000,0.929412}%
\pgfsetfillcolor{currentfill}%
\pgfsetlinewidth{0.250937pt}%
\definecolor{currentstroke}{rgb}{1.000000,1.000000,1.000000}%
\pgfsetstrokecolor{currentstroke}%
\pgfsetdash{}{0pt}%
\pgfpathmoveto{\pgfqpoint{4.329056in}{10.574340in}}%
\pgfpathlineto{\pgfqpoint{4.416792in}{10.574340in}}%
\pgfpathlineto{\pgfqpoint{4.416792in}{10.486604in}}%
\pgfpathlineto{\pgfqpoint{4.329056in}{10.486604in}}%
\pgfpathlineto{\pgfqpoint{4.329056in}{10.574340in}}%
\pgfusepath{stroke,fill}%
\end{pgfscope}%
\begin{pgfscope}%
\pgfpathrectangle{\pgfqpoint{0.380943in}{9.960189in}}{\pgfqpoint{4.650000in}{0.614151in}}%
\pgfusepath{clip}%
\pgfsetbuttcap%
\pgfsetroundjoin%
\definecolor{currentfill}{rgb}{1.000000,1.000000,0.929412}%
\pgfsetfillcolor{currentfill}%
\pgfsetlinewidth{0.250937pt}%
\definecolor{currentstroke}{rgb}{1.000000,1.000000,1.000000}%
\pgfsetstrokecolor{currentstroke}%
\pgfsetdash{}{0pt}%
\pgfpathmoveto{\pgfqpoint{4.416792in}{10.574340in}}%
\pgfpathlineto{\pgfqpoint{4.504528in}{10.574340in}}%
\pgfpathlineto{\pgfqpoint{4.504528in}{10.486604in}}%
\pgfpathlineto{\pgfqpoint{4.416792in}{10.486604in}}%
\pgfpathlineto{\pgfqpoint{4.416792in}{10.574340in}}%
\pgfusepath{stroke,fill}%
\end{pgfscope}%
\begin{pgfscope}%
\pgfpathrectangle{\pgfqpoint{0.380943in}{9.960189in}}{\pgfqpoint{4.650000in}{0.614151in}}%
\pgfusepath{clip}%
\pgfsetbuttcap%
\pgfsetroundjoin%
\definecolor{currentfill}{rgb}{1.000000,1.000000,0.865975}%
\pgfsetfillcolor{currentfill}%
\pgfsetlinewidth{0.250937pt}%
\definecolor{currentstroke}{rgb}{1.000000,1.000000,1.000000}%
\pgfsetstrokecolor{currentstroke}%
\pgfsetdash{}{0pt}%
\pgfpathmoveto{\pgfqpoint{4.504528in}{10.574340in}}%
\pgfpathlineto{\pgfqpoint{4.592264in}{10.574340in}}%
\pgfpathlineto{\pgfqpoint{4.592264in}{10.486604in}}%
\pgfpathlineto{\pgfqpoint{4.504528in}{10.486604in}}%
\pgfpathlineto{\pgfqpoint{4.504528in}{10.574340in}}%
\pgfusepath{stroke,fill}%
\end{pgfscope}%
\begin{pgfscope}%
\pgfpathrectangle{\pgfqpoint{0.380943in}{9.960189in}}{\pgfqpoint{4.650000in}{0.614151in}}%
\pgfusepath{clip}%
\pgfsetbuttcap%
\pgfsetroundjoin%
\definecolor{currentfill}{rgb}{0.967474,0.895963,0.706344}%
\pgfsetfillcolor{currentfill}%
\pgfsetlinewidth{0.250937pt}%
\definecolor{currentstroke}{rgb}{1.000000,1.000000,1.000000}%
\pgfsetstrokecolor{currentstroke}%
\pgfsetdash{}{0pt}%
\pgfpathmoveto{\pgfqpoint{4.592264in}{10.574340in}}%
\pgfpathlineto{\pgfqpoint{4.680000in}{10.574340in}}%
\pgfpathlineto{\pgfqpoint{4.680000in}{10.486604in}}%
\pgfpathlineto{\pgfqpoint{4.592264in}{10.486604in}}%
\pgfpathlineto{\pgfqpoint{4.592264in}{10.574340in}}%
\pgfusepath{stroke,fill}%
\end{pgfscope}%
\begin{pgfscope}%
\pgfpathrectangle{\pgfqpoint{0.380943in}{9.960189in}}{\pgfqpoint{4.650000in}{0.614151in}}%
\pgfusepath{clip}%
\pgfsetbuttcap%
\pgfsetroundjoin%
\definecolor{currentfill}{rgb}{1.000000,0.522261,0.496886}%
\pgfsetfillcolor{currentfill}%
\pgfsetlinewidth{0.250937pt}%
\definecolor{currentstroke}{rgb}{1.000000,1.000000,1.000000}%
\pgfsetstrokecolor{currentstroke}%
\pgfsetdash{}{0pt}%
\pgfpathmoveto{\pgfqpoint{4.680000in}{10.574340in}}%
\pgfpathlineto{\pgfqpoint{4.767736in}{10.574340in}}%
\pgfpathlineto{\pgfqpoint{4.767736in}{10.486604in}}%
\pgfpathlineto{\pgfqpoint{4.680000in}{10.486604in}}%
\pgfpathlineto{\pgfqpoint{4.680000in}{10.574340in}}%
\pgfusepath{stroke,fill}%
\end{pgfscope}%
\begin{pgfscope}%
\pgfpathrectangle{\pgfqpoint{0.380943in}{9.960189in}}{\pgfqpoint{4.650000in}{0.614151in}}%
\pgfusepath{clip}%
\pgfsetbuttcap%
\pgfsetroundjoin%
\definecolor{currentfill}{rgb}{0.997924,0.685352,0.570242}%
\pgfsetfillcolor{currentfill}%
\pgfsetlinewidth{0.250937pt}%
\definecolor{currentstroke}{rgb}{1.000000,1.000000,1.000000}%
\pgfsetstrokecolor{currentstroke}%
\pgfsetdash{}{0pt}%
\pgfpathmoveto{\pgfqpoint{4.767736in}{10.574340in}}%
\pgfpathlineto{\pgfqpoint{4.855471in}{10.574340in}}%
\pgfpathlineto{\pgfqpoint{4.855471in}{10.486604in}}%
\pgfpathlineto{\pgfqpoint{4.767736in}{10.486604in}}%
\pgfpathlineto{\pgfqpoint{4.767736in}{10.574340in}}%
\pgfusepath{stroke,fill}%
\end{pgfscope}%
\begin{pgfscope}%
\pgfpathrectangle{\pgfqpoint{0.380943in}{9.960189in}}{\pgfqpoint{4.650000in}{0.614151in}}%
\pgfusepath{clip}%
\pgfsetbuttcap%
\pgfsetroundjoin%
\definecolor{currentfill}{rgb}{0.989619,0.788235,0.628374}%
\pgfsetfillcolor{currentfill}%
\pgfsetlinewidth{0.250937pt}%
\definecolor{currentstroke}{rgb}{1.000000,1.000000,1.000000}%
\pgfsetstrokecolor{currentstroke}%
\pgfsetdash{}{0pt}%
\pgfpathmoveto{\pgfqpoint{4.855471in}{10.574340in}}%
\pgfpathlineto{\pgfqpoint{4.943207in}{10.574340in}}%
\pgfpathlineto{\pgfqpoint{4.943207in}{10.486604in}}%
\pgfpathlineto{\pgfqpoint{4.855471in}{10.486604in}}%
\pgfpathlineto{\pgfqpoint{4.855471in}{10.574340in}}%
\pgfusepath{stroke,fill}%
\end{pgfscope}%
\begin{pgfscope}%
\pgfpathrectangle{\pgfqpoint{0.380943in}{9.960189in}}{\pgfqpoint{4.650000in}{0.614151in}}%
\pgfusepath{clip}%
\pgfsetbuttcap%
\pgfsetroundjoin%
\definecolor{currentfill}{rgb}{0.994694,0.745098,0.602999}%
\pgfsetfillcolor{currentfill}%
\pgfsetlinewidth{0.250937pt}%
\definecolor{currentstroke}{rgb}{1.000000,1.000000,1.000000}%
\pgfsetstrokecolor{currentstroke}%
\pgfsetdash{}{0pt}%
\pgfpathmoveto{\pgfqpoint{4.943207in}{10.574340in}}%
\pgfpathlineto{\pgfqpoint{5.030943in}{10.574340in}}%
\pgfpathlineto{\pgfqpoint{5.030943in}{10.486604in}}%
\pgfpathlineto{\pgfqpoint{4.943207in}{10.486604in}}%
\pgfpathlineto{\pgfqpoint{4.943207in}{10.574340in}}%
\pgfusepath{stroke,fill}%
\end{pgfscope}%
\begin{pgfscope}%
\pgfpathrectangle{\pgfqpoint{0.380943in}{9.960189in}}{\pgfqpoint{4.650000in}{0.614151in}}%
\pgfusepath{clip}%
\pgfsetbuttcap%
\pgfsetroundjoin%
\pgfsetlinewidth{0.250937pt}%
\definecolor{currentstroke}{rgb}{1.000000,1.000000,1.000000}%
\pgfsetstrokecolor{currentstroke}%
\pgfsetdash{}{0pt}%
\pgfpathmoveto{\pgfqpoint{0.380943in}{10.486604in}}%
\pgfpathlineto{\pgfqpoint{0.468679in}{10.486604in}}%
\pgfpathlineto{\pgfqpoint{0.468679in}{10.398868in}}%
\pgfpathlineto{\pgfqpoint{0.380943in}{10.398868in}}%
\pgfpathlineto{\pgfqpoint{0.380943in}{10.486604in}}%
\pgfusepath{stroke}%
\end{pgfscope}%
\begin{pgfscope}%
\pgfpathrectangle{\pgfqpoint{0.380943in}{9.960189in}}{\pgfqpoint{4.650000in}{0.614151in}}%
\pgfusepath{clip}%
\pgfsetbuttcap%
\pgfsetroundjoin%
\definecolor{currentfill}{rgb}{1.000000,1.000000,0.929412}%
\pgfsetfillcolor{currentfill}%
\pgfsetlinewidth{0.250937pt}%
\definecolor{currentstroke}{rgb}{1.000000,1.000000,1.000000}%
\pgfsetstrokecolor{currentstroke}%
\pgfsetdash{}{0pt}%
\pgfpathmoveto{\pgfqpoint{0.468679in}{10.486604in}}%
\pgfpathlineto{\pgfqpoint{0.556415in}{10.486604in}}%
\pgfpathlineto{\pgfqpoint{0.556415in}{10.398868in}}%
\pgfpathlineto{\pgfqpoint{0.468679in}{10.398868in}}%
\pgfpathlineto{\pgfqpoint{0.468679in}{10.486604in}}%
\pgfusepath{stroke,fill}%
\end{pgfscope}%
\begin{pgfscope}%
\pgfpathrectangle{\pgfqpoint{0.380943in}{9.960189in}}{\pgfqpoint{4.650000in}{0.614151in}}%
\pgfusepath{clip}%
\pgfsetbuttcap%
\pgfsetroundjoin%
\definecolor{currentfill}{rgb}{1.000000,1.000000,0.929412}%
\pgfsetfillcolor{currentfill}%
\pgfsetlinewidth{0.250937pt}%
\definecolor{currentstroke}{rgb}{1.000000,1.000000,1.000000}%
\pgfsetstrokecolor{currentstroke}%
\pgfsetdash{}{0pt}%
\pgfpathmoveto{\pgfqpoint{0.556415in}{10.486604in}}%
\pgfpathlineto{\pgfqpoint{0.644151in}{10.486604in}}%
\pgfpathlineto{\pgfqpoint{0.644151in}{10.398868in}}%
\pgfpathlineto{\pgfqpoint{0.556415in}{10.398868in}}%
\pgfpathlineto{\pgfqpoint{0.556415in}{10.486604in}}%
\pgfusepath{stroke,fill}%
\end{pgfscope}%
\begin{pgfscope}%
\pgfpathrectangle{\pgfqpoint{0.380943in}{9.960189in}}{\pgfqpoint{4.650000in}{0.614151in}}%
\pgfusepath{clip}%
\pgfsetbuttcap%
\pgfsetroundjoin%
\definecolor{currentfill}{rgb}{1.000000,1.000000,0.929412}%
\pgfsetfillcolor{currentfill}%
\pgfsetlinewidth{0.250937pt}%
\definecolor{currentstroke}{rgb}{1.000000,1.000000,1.000000}%
\pgfsetstrokecolor{currentstroke}%
\pgfsetdash{}{0pt}%
\pgfpathmoveto{\pgfqpoint{0.644151in}{10.486604in}}%
\pgfpathlineto{\pgfqpoint{0.731886in}{10.486604in}}%
\pgfpathlineto{\pgfqpoint{0.731886in}{10.398868in}}%
\pgfpathlineto{\pgfqpoint{0.644151in}{10.398868in}}%
\pgfpathlineto{\pgfqpoint{0.644151in}{10.486604in}}%
\pgfusepath{stroke,fill}%
\end{pgfscope}%
\begin{pgfscope}%
\pgfpathrectangle{\pgfqpoint{0.380943in}{9.960189in}}{\pgfqpoint{4.650000in}{0.614151in}}%
\pgfusepath{clip}%
\pgfsetbuttcap%
\pgfsetroundjoin%
\definecolor{currentfill}{rgb}{1.000000,1.000000,0.929412}%
\pgfsetfillcolor{currentfill}%
\pgfsetlinewidth{0.250937pt}%
\definecolor{currentstroke}{rgb}{1.000000,1.000000,1.000000}%
\pgfsetstrokecolor{currentstroke}%
\pgfsetdash{}{0pt}%
\pgfpathmoveto{\pgfqpoint{0.731886in}{10.486604in}}%
\pgfpathlineto{\pgfqpoint{0.819622in}{10.486604in}}%
\pgfpathlineto{\pgfqpoint{0.819622in}{10.398868in}}%
\pgfpathlineto{\pgfqpoint{0.731886in}{10.398868in}}%
\pgfpathlineto{\pgfqpoint{0.731886in}{10.486604in}}%
\pgfusepath{stroke,fill}%
\end{pgfscope}%
\begin{pgfscope}%
\pgfpathrectangle{\pgfqpoint{0.380943in}{9.960189in}}{\pgfqpoint{4.650000in}{0.614151in}}%
\pgfusepath{clip}%
\pgfsetbuttcap%
\pgfsetroundjoin%
\definecolor{currentfill}{rgb}{1.000000,1.000000,0.929412}%
\pgfsetfillcolor{currentfill}%
\pgfsetlinewidth{0.250937pt}%
\definecolor{currentstroke}{rgb}{1.000000,1.000000,1.000000}%
\pgfsetstrokecolor{currentstroke}%
\pgfsetdash{}{0pt}%
\pgfpathmoveto{\pgfqpoint{0.819622in}{10.486604in}}%
\pgfpathlineto{\pgfqpoint{0.907358in}{10.486604in}}%
\pgfpathlineto{\pgfqpoint{0.907358in}{10.398868in}}%
\pgfpathlineto{\pgfqpoint{0.819622in}{10.398868in}}%
\pgfpathlineto{\pgfqpoint{0.819622in}{10.486604in}}%
\pgfusepath{stroke,fill}%
\end{pgfscope}%
\begin{pgfscope}%
\pgfpathrectangle{\pgfqpoint{0.380943in}{9.960189in}}{\pgfqpoint{4.650000in}{0.614151in}}%
\pgfusepath{clip}%
\pgfsetbuttcap%
\pgfsetroundjoin%
\definecolor{currentfill}{rgb}{1.000000,1.000000,0.929412}%
\pgfsetfillcolor{currentfill}%
\pgfsetlinewidth{0.250937pt}%
\definecolor{currentstroke}{rgb}{1.000000,1.000000,1.000000}%
\pgfsetstrokecolor{currentstroke}%
\pgfsetdash{}{0pt}%
\pgfpathmoveto{\pgfqpoint{0.907358in}{10.486604in}}%
\pgfpathlineto{\pgfqpoint{0.995094in}{10.486604in}}%
\pgfpathlineto{\pgfqpoint{0.995094in}{10.398868in}}%
\pgfpathlineto{\pgfqpoint{0.907358in}{10.398868in}}%
\pgfpathlineto{\pgfqpoint{0.907358in}{10.486604in}}%
\pgfusepath{stroke,fill}%
\end{pgfscope}%
\begin{pgfscope}%
\pgfpathrectangle{\pgfqpoint{0.380943in}{9.960189in}}{\pgfqpoint{4.650000in}{0.614151in}}%
\pgfusepath{clip}%
\pgfsetbuttcap%
\pgfsetroundjoin%
\definecolor{currentfill}{rgb}{1.000000,1.000000,0.929412}%
\pgfsetfillcolor{currentfill}%
\pgfsetlinewidth{0.250937pt}%
\definecolor{currentstroke}{rgb}{1.000000,1.000000,1.000000}%
\pgfsetstrokecolor{currentstroke}%
\pgfsetdash{}{0pt}%
\pgfpathmoveto{\pgfqpoint{0.995094in}{10.486604in}}%
\pgfpathlineto{\pgfqpoint{1.082830in}{10.486604in}}%
\pgfpathlineto{\pgfqpoint{1.082830in}{10.398868in}}%
\pgfpathlineto{\pgfqpoint{0.995094in}{10.398868in}}%
\pgfpathlineto{\pgfqpoint{0.995094in}{10.486604in}}%
\pgfusepath{stroke,fill}%
\end{pgfscope}%
\begin{pgfscope}%
\pgfpathrectangle{\pgfqpoint{0.380943in}{9.960189in}}{\pgfqpoint{4.650000in}{0.614151in}}%
\pgfusepath{clip}%
\pgfsetbuttcap%
\pgfsetroundjoin%
\definecolor{currentfill}{rgb}{1.000000,1.000000,0.929412}%
\pgfsetfillcolor{currentfill}%
\pgfsetlinewidth{0.250937pt}%
\definecolor{currentstroke}{rgb}{1.000000,1.000000,1.000000}%
\pgfsetstrokecolor{currentstroke}%
\pgfsetdash{}{0pt}%
\pgfpathmoveto{\pgfqpoint{1.082830in}{10.486604in}}%
\pgfpathlineto{\pgfqpoint{1.170566in}{10.486604in}}%
\pgfpathlineto{\pgfqpoint{1.170566in}{10.398868in}}%
\pgfpathlineto{\pgfqpoint{1.082830in}{10.398868in}}%
\pgfpathlineto{\pgfqpoint{1.082830in}{10.486604in}}%
\pgfusepath{stroke,fill}%
\end{pgfscope}%
\begin{pgfscope}%
\pgfpathrectangle{\pgfqpoint{0.380943in}{9.960189in}}{\pgfqpoint{4.650000in}{0.614151in}}%
\pgfusepath{clip}%
\pgfsetbuttcap%
\pgfsetroundjoin%
\definecolor{currentfill}{rgb}{1.000000,1.000000,0.929412}%
\pgfsetfillcolor{currentfill}%
\pgfsetlinewidth{0.250937pt}%
\definecolor{currentstroke}{rgb}{1.000000,1.000000,1.000000}%
\pgfsetstrokecolor{currentstroke}%
\pgfsetdash{}{0pt}%
\pgfpathmoveto{\pgfqpoint{1.170566in}{10.486604in}}%
\pgfpathlineto{\pgfqpoint{1.258302in}{10.486604in}}%
\pgfpathlineto{\pgfqpoint{1.258302in}{10.398868in}}%
\pgfpathlineto{\pgfqpoint{1.170566in}{10.398868in}}%
\pgfpathlineto{\pgfqpoint{1.170566in}{10.486604in}}%
\pgfusepath{stroke,fill}%
\end{pgfscope}%
\begin{pgfscope}%
\pgfpathrectangle{\pgfqpoint{0.380943in}{9.960189in}}{\pgfqpoint{4.650000in}{0.614151in}}%
\pgfusepath{clip}%
\pgfsetbuttcap%
\pgfsetroundjoin%
\definecolor{currentfill}{rgb}{1.000000,1.000000,0.929412}%
\pgfsetfillcolor{currentfill}%
\pgfsetlinewidth{0.250937pt}%
\definecolor{currentstroke}{rgb}{1.000000,1.000000,1.000000}%
\pgfsetstrokecolor{currentstroke}%
\pgfsetdash{}{0pt}%
\pgfpathmoveto{\pgfqpoint{1.258302in}{10.486604in}}%
\pgfpathlineto{\pgfqpoint{1.346037in}{10.486604in}}%
\pgfpathlineto{\pgfqpoint{1.346037in}{10.398868in}}%
\pgfpathlineto{\pgfqpoint{1.258302in}{10.398868in}}%
\pgfpathlineto{\pgfqpoint{1.258302in}{10.486604in}}%
\pgfusepath{stroke,fill}%
\end{pgfscope}%
\begin{pgfscope}%
\pgfpathrectangle{\pgfqpoint{0.380943in}{9.960189in}}{\pgfqpoint{4.650000in}{0.614151in}}%
\pgfusepath{clip}%
\pgfsetbuttcap%
\pgfsetroundjoin%
\definecolor{currentfill}{rgb}{1.000000,1.000000,0.929412}%
\pgfsetfillcolor{currentfill}%
\pgfsetlinewidth{0.250937pt}%
\definecolor{currentstroke}{rgb}{1.000000,1.000000,1.000000}%
\pgfsetstrokecolor{currentstroke}%
\pgfsetdash{}{0pt}%
\pgfpathmoveto{\pgfqpoint{1.346037in}{10.486604in}}%
\pgfpathlineto{\pgfqpoint{1.433773in}{10.486604in}}%
\pgfpathlineto{\pgfqpoint{1.433773in}{10.398868in}}%
\pgfpathlineto{\pgfqpoint{1.346037in}{10.398868in}}%
\pgfpathlineto{\pgfqpoint{1.346037in}{10.486604in}}%
\pgfusepath{stroke,fill}%
\end{pgfscope}%
\begin{pgfscope}%
\pgfpathrectangle{\pgfqpoint{0.380943in}{9.960189in}}{\pgfqpoint{4.650000in}{0.614151in}}%
\pgfusepath{clip}%
\pgfsetbuttcap%
\pgfsetroundjoin%
\definecolor{currentfill}{rgb}{1.000000,1.000000,0.929412}%
\pgfsetfillcolor{currentfill}%
\pgfsetlinewidth{0.250937pt}%
\definecolor{currentstroke}{rgb}{1.000000,1.000000,1.000000}%
\pgfsetstrokecolor{currentstroke}%
\pgfsetdash{}{0pt}%
\pgfpathmoveto{\pgfqpoint{1.433773in}{10.486604in}}%
\pgfpathlineto{\pgfqpoint{1.521509in}{10.486604in}}%
\pgfpathlineto{\pgfqpoint{1.521509in}{10.398868in}}%
\pgfpathlineto{\pgfqpoint{1.433773in}{10.398868in}}%
\pgfpathlineto{\pgfqpoint{1.433773in}{10.486604in}}%
\pgfusepath{stroke,fill}%
\end{pgfscope}%
\begin{pgfscope}%
\pgfpathrectangle{\pgfqpoint{0.380943in}{9.960189in}}{\pgfqpoint{4.650000in}{0.614151in}}%
\pgfusepath{clip}%
\pgfsetbuttcap%
\pgfsetroundjoin%
\definecolor{currentfill}{rgb}{1.000000,1.000000,0.929412}%
\pgfsetfillcolor{currentfill}%
\pgfsetlinewidth{0.250937pt}%
\definecolor{currentstroke}{rgb}{1.000000,1.000000,1.000000}%
\pgfsetstrokecolor{currentstroke}%
\pgfsetdash{}{0pt}%
\pgfpathmoveto{\pgfqpoint{1.521509in}{10.486604in}}%
\pgfpathlineto{\pgfqpoint{1.609245in}{10.486604in}}%
\pgfpathlineto{\pgfqpoint{1.609245in}{10.398868in}}%
\pgfpathlineto{\pgfqpoint{1.521509in}{10.398868in}}%
\pgfpathlineto{\pgfqpoint{1.521509in}{10.486604in}}%
\pgfusepath{stroke,fill}%
\end{pgfscope}%
\begin{pgfscope}%
\pgfpathrectangle{\pgfqpoint{0.380943in}{9.960189in}}{\pgfqpoint{4.650000in}{0.614151in}}%
\pgfusepath{clip}%
\pgfsetbuttcap%
\pgfsetroundjoin%
\definecolor{currentfill}{rgb}{1.000000,1.000000,0.929412}%
\pgfsetfillcolor{currentfill}%
\pgfsetlinewidth{0.250937pt}%
\definecolor{currentstroke}{rgb}{1.000000,1.000000,1.000000}%
\pgfsetstrokecolor{currentstroke}%
\pgfsetdash{}{0pt}%
\pgfpathmoveto{\pgfqpoint{1.609245in}{10.486604in}}%
\pgfpathlineto{\pgfqpoint{1.696981in}{10.486604in}}%
\pgfpathlineto{\pgfqpoint{1.696981in}{10.398868in}}%
\pgfpathlineto{\pgfqpoint{1.609245in}{10.398868in}}%
\pgfpathlineto{\pgfqpoint{1.609245in}{10.486604in}}%
\pgfusepath{stroke,fill}%
\end{pgfscope}%
\begin{pgfscope}%
\pgfpathrectangle{\pgfqpoint{0.380943in}{9.960189in}}{\pgfqpoint{4.650000in}{0.614151in}}%
\pgfusepath{clip}%
\pgfsetbuttcap%
\pgfsetroundjoin%
\definecolor{currentfill}{rgb}{1.000000,1.000000,0.929412}%
\pgfsetfillcolor{currentfill}%
\pgfsetlinewidth{0.250937pt}%
\definecolor{currentstroke}{rgb}{1.000000,1.000000,1.000000}%
\pgfsetstrokecolor{currentstroke}%
\pgfsetdash{}{0pt}%
\pgfpathmoveto{\pgfqpoint{1.696981in}{10.486604in}}%
\pgfpathlineto{\pgfqpoint{1.784717in}{10.486604in}}%
\pgfpathlineto{\pgfqpoint{1.784717in}{10.398868in}}%
\pgfpathlineto{\pgfqpoint{1.696981in}{10.398868in}}%
\pgfpathlineto{\pgfqpoint{1.696981in}{10.486604in}}%
\pgfusepath{stroke,fill}%
\end{pgfscope}%
\begin{pgfscope}%
\pgfpathrectangle{\pgfqpoint{0.380943in}{9.960189in}}{\pgfqpoint{4.650000in}{0.614151in}}%
\pgfusepath{clip}%
\pgfsetbuttcap%
\pgfsetroundjoin%
\definecolor{currentfill}{rgb}{1.000000,1.000000,0.929412}%
\pgfsetfillcolor{currentfill}%
\pgfsetlinewidth{0.250937pt}%
\definecolor{currentstroke}{rgb}{1.000000,1.000000,1.000000}%
\pgfsetstrokecolor{currentstroke}%
\pgfsetdash{}{0pt}%
\pgfpathmoveto{\pgfqpoint{1.784717in}{10.486604in}}%
\pgfpathlineto{\pgfqpoint{1.872452in}{10.486604in}}%
\pgfpathlineto{\pgfqpoint{1.872452in}{10.398868in}}%
\pgfpathlineto{\pgfqpoint{1.784717in}{10.398868in}}%
\pgfpathlineto{\pgfqpoint{1.784717in}{10.486604in}}%
\pgfusepath{stroke,fill}%
\end{pgfscope}%
\begin{pgfscope}%
\pgfpathrectangle{\pgfqpoint{0.380943in}{9.960189in}}{\pgfqpoint{4.650000in}{0.614151in}}%
\pgfusepath{clip}%
\pgfsetbuttcap%
\pgfsetroundjoin%
\definecolor{currentfill}{rgb}{1.000000,1.000000,0.929412}%
\pgfsetfillcolor{currentfill}%
\pgfsetlinewidth{0.250937pt}%
\definecolor{currentstroke}{rgb}{1.000000,1.000000,1.000000}%
\pgfsetstrokecolor{currentstroke}%
\pgfsetdash{}{0pt}%
\pgfpathmoveto{\pgfqpoint{1.872452in}{10.486604in}}%
\pgfpathlineto{\pgfqpoint{1.960188in}{10.486604in}}%
\pgfpathlineto{\pgfqpoint{1.960188in}{10.398868in}}%
\pgfpathlineto{\pgfqpoint{1.872452in}{10.398868in}}%
\pgfpathlineto{\pgfqpoint{1.872452in}{10.486604in}}%
\pgfusepath{stroke,fill}%
\end{pgfscope}%
\begin{pgfscope}%
\pgfpathrectangle{\pgfqpoint{0.380943in}{9.960189in}}{\pgfqpoint{4.650000in}{0.614151in}}%
\pgfusepath{clip}%
\pgfsetbuttcap%
\pgfsetroundjoin%
\definecolor{currentfill}{rgb}{1.000000,1.000000,0.929412}%
\pgfsetfillcolor{currentfill}%
\pgfsetlinewidth{0.250937pt}%
\definecolor{currentstroke}{rgb}{1.000000,1.000000,1.000000}%
\pgfsetstrokecolor{currentstroke}%
\pgfsetdash{}{0pt}%
\pgfpathmoveto{\pgfqpoint{1.960188in}{10.486604in}}%
\pgfpathlineto{\pgfqpoint{2.047924in}{10.486604in}}%
\pgfpathlineto{\pgfqpoint{2.047924in}{10.398868in}}%
\pgfpathlineto{\pgfqpoint{1.960188in}{10.398868in}}%
\pgfpathlineto{\pgfqpoint{1.960188in}{10.486604in}}%
\pgfusepath{stroke,fill}%
\end{pgfscope}%
\begin{pgfscope}%
\pgfpathrectangle{\pgfqpoint{0.380943in}{9.960189in}}{\pgfqpoint{4.650000in}{0.614151in}}%
\pgfusepath{clip}%
\pgfsetbuttcap%
\pgfsetroundjoin%
\definecolor{currentfill}{rgb}{1.000000,1.000000,0.929412}%
\pgfsetfillcolor{currentfill}%
\pgfsetlinewidth{0.250937pt}%
\definecolor{currentstroke}{rgb}{1.000000,1.000000,1.000000}%
\pgfsetstrokecolor{currentstroke}%
\pgfsetdash{}{0pt}%
\pgfpathmoveto{\pgfqpoint{2.047924in}{10.486604in}}%
\pgfpathlineto{\pgfqpoint{2.135660in}{10.486604in}}%
\pgfpathlineto{\pgfqpoint{2.135660in}{10.398868in}}%
\pgfpathlineto{\pgfqpoint{2.047924in}{10.398868in}}%
\pgfpathlineto{\pgfqpoint{2.047924in}{10.486604in}}%
\pgfusepath{stroke,fill}%
\end{pgfscope}%
\begin{pgfscope}%
\pgfpathrectangle{\pgfqpoint{0.380943in}{9.960189in}}{\pgfqpoint{4.650000in}{0.614151in}}%
\pgfusepath{clip}%
\pgfsetbuttcap%
\pgfsetroundjoin%
\definecolor{currentfill}{rgb}{1.000000,1.000000,0.929412}%
\pgfsetfillcolor{currentfill}%
\pgfsetlinewidth{0.250937pt}%
\definecolor{currentstroke}{rgb}{1.000000,1.000000,1.000000}%
\pgfsetstrokecolor{currentstroke}%
\pgfsetdash{}{0pt}%
\pgfpathmoveto{\pgfqpoint{2.135660in}{10.486604in}}%
\pgfpathlineto{\pgfqpoint{2.223396in}{10.486604in}}%
\pgfpathlineto{\pgfqpoint{2.223396in}{10.398868in}}%
\pgfpathlineto{\pgfqpoint{2.135660in}{10.398868in}}%
\pgfpathlineto{\pgfqpoint{2.135660in}{10.486604in}}%
\pgfusepath{stroke,fill}%
\end{pgfscope}%
\begin{pgfscope}%
\pgfpathrectangle{\pgfqpoint{0.380943in}{9.960189in}}{\pgfqpoint{4.650000in}{0.614151in}}%
\pgfusepath{clip}%
\pgfsetbuttcap%
\pgfsetroundjoin%
\definecolor{currentfill}{rgb}{1.000000,1.000000,0.929412}%
\pgfsetfillcolor{currentfill}%
\pgfsetlinewidth{0.250937pt}%
\definecolor{currentstroke}{rgb}{1.000000,1.000000,1.000000}%
\pgfsetstrokecolor{currentstroke}%
\pgfsetdash{}{0pt}%
\pgfpathmoveto{\pgfqpoint{2.223396in}{10.486604in}}%
\pgfpathlineto{\pgfqpoint{2.311132in}{10.486604in}}%
\pgfpathlineto{\pgfqpoint{2.311132in}{10.398868in}}%
\pgfpathlineto{\pgfqpoint{2.223396in}{10.398868in}}%
\pgfpathlineto{\pgfqpoint{2.223396in}{10.486604in}}%
\pgfusepath{stroke,fill}%
\end{pgfscope}%
\begin{pgfscope}%
\pgfpathrectangle{\pgfqpoint{0.380943in}{9.960189in}}{\pgfqpoint{4.650000in}{0.614151in}}%
\pgfusepath{clip}%
\pgfsetbuttcap%
\pgfsetroundjoin%
\definecolor{currentfill}{rgb}{1.000000,1.000000,0.929412}%
\pgfsetfillcolor{currentfill}%
\pgfsetlinewidth{0.250937pt}%
\definecolor{currentstroke}{rgb}{1.000000,1.000000,1.000000}%
\pgfsetstrokecolor{currentstroke}%
\pgfsetdash{}{0pt}%
\pgfpathmoveto{\pgfqpoint{2.311132in}{10.486604in}}%
\pgfpathlineto{\pgfqpoint{2.398868in}{10.486604in}}%
\pgfpathlineto{\pgfqpoint{2.398868in}{10.398868in}}%
\pgfpathlineto{\pgfqpoint{2.311132in}{10.398868in}}%
\pgfpathlineto{\pgfqpoint{2.311132in}{10.486604in}}%
\pgfusepath{stroke,fill}%
\end{pgfscope}%
\begin{pgfscope}%
\pgfpathrectangle{\pgfqpoint{0.380943in}{9.960189in}}{\pgfqpoint{4.650000in}{0.614151in}}%
\pgfusepath{clip}%
\pgfsetbuttcap%
\pgfsetroundjoin%
\definecolor{currentfill}{rgb}{1.000000,1.000000,0.929412}%
\pgfsetfillcolor{currentfill}%
\pgfsetlinewidth{0.250937pt}%
\definecolor{currentstroke}{rgb}{1.000000,1.000000,1.000000}%
\pgfsetstrokecolor{currentstroke}%
\pgfsetdash{}{0pt}%
\pgfpathmoveto{\pgfqpoint{2.398868in}{10.486604in}}%
\pgfpathlineto{\pgfqpoint{2.486603in}{10.486604in}}%
\pgfpathlineto{\pgfqpoint{2.486603in}{10.398868in}}%
\pgfpathlineto{\pgfqpoint{2.398868in}{10.398868in}}%
\pgfpathlineto{\pgfqpoint{2.398868in}{10.486604in}}%
\pgfusepath{stroke,fill}%
\end{pgfscope}%
\begin{pgfscope}%
\pgfpathrectangle{\pgfqpoint{0.380943in}{9.960189in}}{\pgfqpoint{4.650000in}{0.614151in}}%
\pgfusepath{clip}%
\pgfsetbuttcap%
\pgfsetroundjoin%
\definecolor{currentfill}{rgb}{1.000000,1.000000,0.929412}%
\pgfsetfillcolor{currentfill}%
\pgfsetlinewidth{0.250937pt}%
\definecolor{currentstroke}{rgb}{1.000000,1.000000,1.000000}%
\pgfsetstrokecolor{currentstroke}%
\pgfsetdash{}{0pt}%
\pgfpathmoveto{\pgfqpoint{2.486603in}{10.486604in}}%
\pgfpathlineto{\pgfqpoint{2.574339in}{10.486604in}}%
\pgfpathlineto{\pgfqpoint{2.574339in}{10.398868in}}%
\pgfpathlineto{\pgfqpoint{2.486603in}{10.398868in}}%
\pgfpathlineto{\pgfqpoint{2.486603in}{10.486604in}}%
\pgfusepath{stroke,fill}%
\end{pgfscope}%
\begin{pgfscope}%
\pgfpathrectangle{\pgfqpoint{0.380943in}{9.960189in}}{\pgfqpoint{4.650000in}{0.614151in}}%
\pgfusepath{clip}%
\pgfsetbuttcap%
\pgfsetroundjoin%
\definecolor{currentfill}{rgb}{1.000000,1.000000,0.929412}%
\pgfsetfillcolor{currentfill}%
\pgfsetlinewidth{0.250937pt}%
\definecolor{currentstroke}{rgb}{1.000000,1.000000,1.000000}%
\pgfsetstrokecolor{currentstroke}%
\pgfsetdash{}{0pt}%
\pgfpathmoveto{\pgfqpoint{2.574339in}{10.486604in}}%
\pgfpathlineto{\pgfqpoint{2.662075in}{10.486604in}}%
\pgfpathlineto{\pgfqpoint{2.662075in}{10.398868in}}%
\pgfpathlineto{\pgfqpoint{2.574339in}{10.398868in}}%
\pgfpathlineto{\pgfqpoint{2.574339in}{10.486604in}}%
\pgfusepath{stroke,fill}%
\end{pgfscope}%
\begin{pgfscope}%
\pgfpathrectangle{\pgfqpoint{0.380943in}{9.960189in}}{\pgfqpoint{4.650000in}{0.614151in}}%
\pgfusepath{clip}%
\pgfsetbuttcap%
\pgfsetroundjoin%
\definecolor{currentfill}{rgb}{1.000000,1.000000,0.929412}%
\pgfsetfillcolor{currentfill}%
\pgfsetlinewidth{0.250937pt}%
\definecolor{currentstroke}{rgb}{1.000000,1.000000,1.000000}%
\pgfsetstrokecolor{currentstroke}%
\pgfsetdash{}{0pt}%
\pgfpathmoveto{\pgfqpoint{2.662075in}{10.486604in}}%
\pgfpathlineto{\pgfqpoint{2.749811in}{10.486604in}}%
\pgfpathlineto{\pgfqpoint{2.749811in}{10.398868in}}%
\pgfpathlineto{\pgfqpoint{2.662075in}{10.398868in}}%
\pgfpathlineto{\pgfqpoint{2.662075in}{10.486604in}}%
\pgfusepath{stroke,fill}%
\end{pgfscope}%
\begin{pgfscope}%
\pgfpathrectangle{\pgfqpoint{0.380943in}{9.960189in}}{\pgfqpoint{4.650000in}{0.614151in}}%
\pgfusepath{clip}%
\pgfsetbuttcap%
\pgfsetroundjoin%
\definecolor{currentfill}{rgb}{1.000000,1.000000,0.929412}%
\pgfsetfillcolor{currentfill}%
\pgfsetlinewidth{0.250937pt}%
\definecolor{currentstroke}{rgb}{1.000000,1.000000,1.000000}%
\pgfsetstrokecolor{currentstroke}%
\pgfsetdash{}{0pt}%
\pgfpathmoveto{\pgfqpoint{2.749811in}{10.486604in}}%
\pgfpathlineto{\pgfqpoint{2.837547in}{10.486604in}}%
\pgfpathlineto{\pgfqpoint{2.837547in}{10.398868in}}%
\pgfpathlineto{\pgfqpoint{2.749811in}{10.398868in}}%
\pgfpathlineto{\pgfqpoint{2.749811in}{10.486604in}}%
\pgfusepath{stroke,fill}%
\end{pgfscope}%
\begin{pgfscope}%
\pgfpathrectangle{\pgfqpoint{0.380943in}{9.960189in}}{\pgfqpoint{4.650000in}{0.614151in}}%
\pgfusepath{clip}%
\pgfsetbuttcap%
\pgfsetroundjoin%
\definecolor{currentfill}{rgb}{1.000000,1.000000,0.929412}%
\pgfsetfillcolor{currentfill}%
\pgfsetlinewidth{0.250937pt}%
\definecolor{currentstroke}{rgb}{1.000000,1.000000,1.000000}%
\pgfsetstrokecolor{currentstroke}%
\pgfsetdash{}{0pt}%
\pgfpathmoveto{\pgfqpoint{2.837547in}{10.486604in}}%
\pgfpathlineto{\pgfqpoint{2.925283in}{10.486604in}}%
\pgfpathlineto{\pgfqpoint{2.925283in}{10.398868in}}%
\pgfpathlineto{\pgfqpoint{2.837547in}{10.398868in}}%
\pgfpathlineto{\pgfqpoint{2.837547in}{10.486604in}}%
\pgfusepath{stroke,fill}%
\end{pgfscope}%
\begin{pgfscope}%
\pgfpathrectangle{\pgfqpoint{0.380943in}{9.960189in}}{\pgfqpoint{4.650000in}{0.614151in}}%
\pgfusepath{clip}%
\pgfsetbuttcap%
\pgfsetroundjoin%
\definecolor{currentfill}{rgb}{1.000000,1.000000,0.929412}%
\pgfsetfillcolor{currentfill}%
\pgfsetlinewidth{0.250937pt}%
\definecolor{currentstroke}{rgb}{1.000000,1.000000,1.000000}%
\pgfsetstrokecolor{currentstroke}%
\pgfsetdash{}{0pt}%
\pgfpathmoveto{\pgfqpoint{2.925283in}{10.486604in}}%
\pgfpathlineto{\pgfqpoint{3.013019in}{10.486604in}}%
\pgfpathlineto{\pgfqpoint{3.013019in}{10.398868in}}%
\pgfpathlineto{\pgfqpoint{2.925283in}{10.398868in}}%
\pgfpathlineto{\pgfqpoint{2.925283in}{10.486604in}}%
\pgfusepath{stroke,fill}%
\end{pgfscope}%
\begin{pgfscope}%
\pgfpathrectangle{\pgfqpoint{0.380943in}{9.960189in}}{\pgfqpoint{4.650000in}{0.614151in}}%
\pgfusepath{clip}%
\pgfsetbuttcap%
\pgfsetroundjoin%
\definecolor{currentfill}{rgb}{1.000000,1.000000,0.929412}%
\pgfsetfillcolor{currentfill}%
\pgfsetlinewidth{0.250937pt}%
\definecolor{currentstroke}{rgb}{1.000000,1.000000,1.000000}%
\pgfsetstrokecolor{currentstroke}%
\pgfsetdash{}{0pt}%
\pgfpathmoveto{\pgfqpoint{3.013019in}{10.486604in}}%
\pgfpathlineto{\pgfqpoint{3.100754in}{10.486604in}}%
\pgfpathlineto{\pgfqpoint{3.100754in}{10.398868in}}%
\pgfpathlineto{\pgfqpoint{3.013019in}{10.398868in}}%
\pgfpathlineto{\pgfqpoint{3.013019in}{10.486604in}}%
\pgfusepath{stroke,fill}%
\end{pgfscope}%
\begin{pgfscope}%
\pgfpathrectangle{\pgfqpoint{0.380943in}{9.960189in}}{\pgfqpoint{4.650000in}{0.614151in}}%
\pgfusepath{clip}%
\pgfsetbuttcap%
\pgfsetroundjoin%
\definecolor{currentfill}{rgb}{1.000000,1.000000,0.929412}%
\pgfsetfillcolor{currentfill}%
\pgfsetlinewidth{0.250937pt}%
\definecolor{currentstroke}{rgb}{1.000000,1.000000,1.000000}%
\pgfsetstrokecolor{currentstroke}%
\pgfsetdash{}{0pt}%
\pgfpathmoveto{\pgfqpoint{3.100754in}{10.486604in}}%
\pgfpathlineto{\pgfqpoint{3.188490in}{10.486604in}}%
\pgfpathlineto{\pgfqpoint{3.188490in}{10.398868in}}%
\pgfpathlineto{\pgfqpoint{3.100754in}{10.398868in}}%
\pgfpathlineto{\pgfqpoint{3.100754in}{10.486604in}}%
\pgfusepath{stroke,fill}%
\end{pgfscope}%
\begin{pgfscope}%
\pgfpathrectangle{\pgfqpoint{0.380943in}{9.960189in}}{\pgfqpoint{4.650000in}{0.614151in}}%
\pgfusepath{clip}%
\pgfsetbuttcap%
\pgfsetroundjoin%
\definecolor{currentfill}{rgb}{1.000000,1.000000,0.929412}%
\pgfsetfillcolor{currentfill}%
\pgfsetlinewidth{0.250937pt}%
\definecolor{currentstroke}{rgb}{1.000000,1.000000,1.000000}%
\pgfsetstrokecolor{currentstroke}%
\pgfsetdash{}{0pt}%
\pgfpathmoveto{\pgfqpoint{3.188490in}{10.486604in}}%
\pgfpathlineto{\pgfqpoint{3.276226in}{10.486604in}}%
\pgfpathlineto{\pgfqpoint{3.276226in}{10.398868in}}%
\pgfpathlineto{\pgfqpoint{3.188490in}{10.398868in}}%
\pgfpathlineto{\pgfqpoint{3.188490in}{10.486604in}}%
\pgfusepath{stroke,fill}%
\end{pgfscope}%
\begin{pgfscope}%
\pgfpathrectangle{\pgfqpoint{0.380943in}{9.960189in}}{\pgfqpoint{4.650000in}{0.614151in}}%
\pgfusepath{clip}%
\pgfsetbuttcap%
\pgfsetroundjoin%
\definecolor{currentfill}{rgb}{1.000000,1.000000,0.929412}%
\pgfsetfillcolor{currentfill}%
\pgfsetlinewidth{0.250937pt}%
\definecolor{currentstroke}{rgb}{1.000000,1.000000,1.000000}%
\pgfsetstrokecolor{currentstroke}%
\pgfsetdash{}{0pt}%
\pgfpathmoveto{\pgfqpoint{3.276226in}{10.486604in}}%
\pgfpathlineto{\pgfqpoint{3.363962in}{10.486604in}}%
\pgfpathlineto{\pgfqpoint{3.363962in}{10.398868in}}%
\pgfpathlineto{\pgfqpoint{3.276226in}{10.398868in}}%
\pgfpathlineto{\pgfqpoint{3.276226in}{10.486604in}}%
\pgfusepath{stroke,fill}%
\end{pgfscope}%
\begin{pgfscope}%
\pgfpathrectangle{\pgfqpoint{0.380943in}{9.960189in}}{\pgfqpoint{4.650000in}{0.614151in}}%
\pgfusepath{clip}%
\pgfsetbuttcap%
\pgfsetroundjoin%
\definecolor{currentfill}{rgb}{1.000000,1.000000,0.929412}%
\pgfsetfillcolor{currentfill}%
\pgfsetlinewidth{0.250937pt}%
\definecolor{currentstroke}{rgb}{1.000000,1.000000,1.000000}%
\pgfsetstrokecolor{currentstroke}%
\pgfsetdash{}{0pt}%
\pgfpathmoveto{\pgfqpoint{3.363962in}{10.486604in}}%
\pgfpathlineto{\pgfqpoint{3.451698in}{10.486604in}}%
\pgfpathlineto{\pgfqpoint{3.451698in}{10.398868in}}%
\pgfpathlineto{\pgfqpoint{3.363962in}{10.398868in}}%
\pgfpathlineto{\pgfqpoint{3.363962in}{10.486604in}}%
\pgfusepath{stroke,fill}%
\end{pgfscope}%
\begin{pgfscope}%
\pgfpathrectangle{\pgfqpoint{0.380943in}{9.960189in}}{\pgfqpoint{4.650000in}{0.614151in}}%
\pgfusepath{clip}%
\pgfsetbuttcap%
\pgfsetroundjoin%
\definecolor{currentfill}{rgb}{1.000000,1.000000,0.929412}%
\pgfsetfillcolor{currentfill}%
\pgfsetlinewidth{0.250937pt}%
\definecolor{currentstroke}{rgb}{1.000000,1.000000,1.000000}%
\pgfsetstrokecolor{currentstroke}%
\pgfsetdash{}{0pt}%
\pgfpathmoveto{\pgfqpoint{3.451698in}{10.486604in}}%
\pgfpathlineto{\pgfqpoint{3.539434in}{10.486604in}}%
\pgfpathlineto{\pgfqpoint{3.539434in}{10.398868in}}%
\pgfpathlineto{\pgfqpoint{3.451698in}{10.398868in}}%
\pgfpathlineto{\pgfqpoint{3.451698in}{10.486604in}}%
\pgfusepath{stroke,fill}%
\end{pgfscope}%
\begin{pgfscope}%
\pgfpathrectangle{\pgfqpoint{0.380943in}{9.960189in}}{\pgfqpoint{4.650000in}{0.614151in}}%
\pgfusepath{clip}%
\pgfsetbuttcap%
\pgfsetroundjoin%
\definecolor{currentfill}{rgb}{1.000000,1.000000,0.929412}%
\pgfsetfillcolor{currentfill}%
\pgfsetlinewidth{0.250937pt}%
\definecolor{currentstroke}{rgb}{1.000000,1.000000,1.000000}%
\pgfsetstrokecolor{currentstroke}%
\pgfsetdash{}{0pt}%
\pgfpathmoveto{\pgfqpoint{3.539434in}{10.486604in}}%
\pgfpathlineto{\pgfqpoint{3.627169in}{10.486604in}}%
\pgfpathlineto{\pgfqpoint{3.627169in}{10.398868in}}%
\pgfpathlineto{\pgfqpoint{3.539434in}{10.398868in}}%
\pgfpathlineto{\pgfqpoint{3.539434in}{10.486604in}}%
\pgfusepath{stroke,fill}%
\end{pgfscope}%
\begin{pgfscope}%
\pgfpathrectangle{\pgfqpoint{0.380943in}{9.960189in}}{\pgfqpoint{4.650000in}{0.614151in}}%
\pgfusepath{clip}%
\pgfsetbuttcap%
\pgfsetroundjoin%
\definecolor{currentfill}{rgb}{1.000000,1.000000,0.929412}%
\pgfsetfillcolor{currentfill}%
\pgfsetlinewidth{0.250937pt}%
\definecolor{currentstroke}{rgb}{1.000000,1.000000,1.000000}%
\pgfsetstrokecolor{currentstroke}%
\pgfsetdash{}{0pt}%
\pgfpathmoveto{\pgfqpoint{3.627169in}{10.486604in}}%
\pgfpathlineto{\pgfqpoint{3.714905in}{10.486604in}}%
\pgfpathlineto{\pgfqpoint{3.714905in}{10.398868in}}%
\pgfpathlineto{\pgfqpoint{3.627169in}{10.398868in}}%
\pgfpathlineto{\pgfqpoint{3.627169in}{10.486604in}}%
\pgfusepath{stroke,fill}%
\end{pgfscope}%
\begin{pgfscope}%
\pgfpathrectangle{\pgfqpoint{0.380943in}{9.960189in}}{\pgfqpoint{4.650000in}{0.614151in}}%
\pgfusepath{clip}%
\pgfsetbuttcap%
\pgfsetroundjoin%
\definecolor{currentfill}{rgb}{1.000000,1.000000,0.929412}%
\pgfsetfillcolor{currentfill}%
\pgfsetlinewidth{0.250937pt}%
\definecolor{currentstroke}{rgb}{1.000000,1.000000,1.000000}%
\pgfsetstrokecolor{currentstroke}%
\pgfsetdash{}{0pt}%
\pgfpathmoveto{\pgfqpoint{3.714905in}{10.486604in}}%
\pgfpathlineto{\pgfqpoint{3.802641in}{10.486604in}}%
\pgfpathlineto{\pgfqpoint{3.802641in}{10.398868in}}%
\pgfpathlineto{\pgfqpoint{3.714905in}{10.398868in}}%
\pgfpathlineto{\pgfqpoint{3.714905in}{10.486604in}}%
\pgfusepath{stroke,fill}%
\end{pgfscope}%
\begin{pgfscope}%
\pgfpathrectangle{\pgfqpoint{0.380943in}{9.960189in}}{\pgfqpoint{4.650000in}{0.614151in}}%
\pgfusepath{clip}%
\pgfsetbuttcap%
\pgfsetroundjoin%
\definecolor{currentfill}{rgb}{1.000000,1.000000,0.929412}%
\pgfsetfillcolor{currentfill}%
\pgfsetlinewidth{0.250937pt}%
\definecolor{currentstroke}{rgb}{1.000000,1.000000,1.000000}%
\pgfsetstrokecolor{currentstroke}%
\pgfsetdash{}{0pt}%
\pgfpathmoveto{\pgfqpoint{3.802641in}{10.486604in}}%
\pgfpathlineto{\pgfqpoint{3.890377in}{10.486604in}}%
\pgfpathlineto{\pgfqpoint{3.890377in}{10.398868in}}%
\pgfpathlineto{\pgfqpoint{3.802641in}{10.398868in}}%
\pgfpathlineto{\pgfqpoint{3.802641in}{10.486604in}}%
\pgfusepath{stroke,fill}%
\end{pgfscope}%
\begin{pgfscope}%
\pgfpathrectangle{\pgfqpoint{0.380943in}{9.960189in}}{\pgfqpoint{4.650000in}{0.614151in}}%
\pgfusepath{clip}%
\pgfsetbuttcap%
\pgfsetroundjoin%
\definecolor{currentfill}{rgb}{1.000000,1.000000,0.929412}%
\pgfsetfillcolor{currentfill}%
\pgfsetlinewidth{0.250937pt}%
\definecolor{currentstroke}{rgb}{1.000000,1.000000,1.000000}%
\pgfsetstrokecolor{currentstroke}%
\pgfsetdash{}{0pt}%
\pgfpathmoveto{\pgfqpoint{3.890377in}{10.486604in}}%
\pgfpathlineto{\pgfqpoint{3.978113in}{10.486604in}}%
\pgfpathlineto{\pgfqpoint{3.978113in}{10.398868in}}%
\pgfpathlineto{\pgfqpoint{3.890377in}{10.398868in}}%
\pgfpathlineto{\pgfqpoint{3.890377in}{10.486604in}}%
\pgfusepath{stroke,fill}%
\end{pgfscope}%
\begin{pgfscope}%
\pgfpathrectangle{\pgfqpoint{0.380943in}{9.960189in}}{\pgfqpoint{4.650000in}{0.614151in}}%
\pgfusepath{clip}%
\pgfsetbuttcap%
\pgfsetroundjoin%
\definecolor{currentfill}{rgb}{1.000000,1.000000,0.929412}%
\pgfsetfillcolor{currentfill}%
\pgfsetlinewidth{0.250937pt}%
\definecolor{currentstroke}{rgb}{1.000000,1.000000,1.000000}%
\pgfsetstrokecolor{currentstroke}%
\pgfsetdash{}{0pt}%
\pgfpathmoveto{\pgfqpoint{3.978113in}{10.486604in}}%
\pgfpathlineto{\pgfqpoint{4.065849in}{10.486604in}}%
\pgfpathlineto{\pgfqpoint{4.065849in}{10.398868in}}%
\pgfpathlineto{\pgfqpoint{3.978113in}{10.398868in}}%
\pgfpathlineto{\pgfqpoint{3.978113in}{10.486604in}}%
\pgfusepath{stroke,fill}%
\end{pgfscope}%
\begin{pgfscope}%
\pgfpathrectangle{\pgfqpoint{0.380943in}{9.960189in}}{\pgfqpoint{4.650000in}{0.614151in}}%
\pgfusepath{clip}%
\pgfsetbuttcap%
\pgfsetroundjoin%
\definecolor{currentfill}{rgb}{1.000000,1.000000,0.929412}%
\pgfsetfillcolor{currentfill}%
\pgfsetlinewidth{0.250937pt}%
\definecolor{currentstroke}{rgb}{1.000000,1.000000,1.000000}%
\pgfsetstrokecolor{currentstroke}%
\pgfsetdash{}{0pt}%
\pgfpathmoveto{\pgfqpoint{4.065849in}{10.486604in}}%
\pgfpathlineto{\pgfqpoint{4.153585in}{10.486604in}}%
\pgfpathlineto{\pgfqpoint{4.153585in}{10.398868in}}%
\pgfpathlineto{\pgfqpoint{4.065849in}{10.398868in}}%
\pgfpathlineto{\pgfqpoint{4.065849in}{10.486604in}}%
\pgfusepath{stroke,fill}%
\end{pgfscope}%
\begin{pgfscope}%
\pgfpathrectangle{\pgfqpoint{0.380943in}{9.960189in}}{\pgfqpoint{4.650000in}{0.614151in}}%
\pgfusepath{clip}%
\pgfsetbuttcap%
\pgfsetroundjoin%
\definecolor{currentfill}{rgb}{1.000000,1.000000,0.929412}%
\pgfsetfillcolor{currentfill}%
\pgfsetlinewidth{0.250937pt}%
\definecolor{currentstroke}{rgb}{1.000000,1.000000,1.000000}%
\pgfsetstrokecolor{currentstroke}%
\pgfsetdash{}{0pt}%
\pgfpathmoveto{\pgfqpoint{4.153585in}{10.486604in}}%
\pgfpathlineto{\pgfqpoint{4.241320in}{10.486604in}}%
\pgfpathlineto{\pgfqpoint{4.241320in}{10.398868in}}%
\pgfpathlineto{\pgfqpoint{4.153585in}{10.398868in}}%
\pgfpathlineto{\pgfqpoint{4.153585in}{10.486604in}}%
\pgfusepath{stroke,fill}%
\end{pgfscope}%
\begin{pgfscope}%
\pgfpathrectangle{\pgfqpoint{0.380943in}{9.960189in}}{\pgfqpoint{4.650000in}{0.614151in}}%
\pgfusepath{clip}%
\pgfsetbuttcap%
\pgfsetroundjoin%
\definecolor{currentfill}{rgb}{1.000000,1.000000,0.929412}%
\pgfsetfillcolor{currentfill}%
\pgfsetlinewidth{0.250937pt}%
\definecolor{currentstroke}{rgb}{1.000000,1.000000,1.000000}%
\pgfsetstrokecolor{currentstroke}%
\pgfsetdash{}{0pt}%
\pgfpathmoveto{\pgfqpoint{4.241320in}{10.486604in}}%
\pgfpathlineto{\pgfqpoint{4.329056in}{10.486604in}}%
\pgfpathlineto{\pgfqpoint{4.329056in}{10.398868in}}%
\pgfpathlineto{\pgfqpoint{4.241320in}{10.398868in}}%
\pgfpathlineto{\pgfqpoint{4.241320in}{10.486604in}}%
\pgfusepath{stroke,fill}%
\end{pgfscope}%
\begin{pgfscope}%
\pgfpathrectangle{\pgfqpoint{0.380943in}{9.960189in}}{\pgfqpoint{4.650000in}{0.614151in}}%
\pgfusepath{clip}%
\pgfsetbuttcap%
\pgfsetroundjoin%
\definecolor{currentfill}{rgb}{1.000000,1.000000,0.929412}%
\pgfsetfillcolor{currentfill}%
\pgfsetlinewidth{0.250937pt}%
\definecolor{currentstroke}{rgb}{1.000000,1.000000,1.000000}%
\pgfsetstrokecolor{currentstroke}%
\pgfsetdash{}{0pt}%
\pgfpathmoveto{\pgfqpoint{4.329056in}{10.486604in}}%
\pgfpathlineto{\pgfqpoint{4.416792in}{10.486604in}}%
\pgfpathlineto{\pgfqpoint{4.416792in}{10.398868in}}%
\pgfpathlineto{\pgfqpoint{4.329056in}{10.398868in}}%
\pgfpathlineto{\pgfqpoint{4.329056in}{10.486604in}}%
\pgfusepath{stroke,fill}%
\end{pgfscope}%
\begin{pgfscope}%
\pgfpathrectangle{\pgfqpoint{0.380943in}{9.960189in}}{\pgfqpoint{4.650000in}{0.614151in}}%
\pgfusepath{clip}%
\pgfsetbuttcap%
\pgfsetroundjoin%
\definecolor{currentfill}{rgb}{1.000000,1.000000,0.929412}%
\pgfsetfillcolor{currentfill}%
\pgfsetlinewidth{0.250937pt}%
\definecolor{currentstroke}{rgb}{1.000000,1.000000,1.000000}%
\pgfsetstrokecolor{currentstroke}%
\pgfsetdash{}{0pt}%
\pgfpathmoveto{\pgfqpoint{4.416792in}{10.486604in}}%
\pgfpathlineto{\pgfqpoint{4.504528in}{10.486604in}}%
\pgfpathlineto{\pgfqpoint{4.504528in}{10.398868in}}%
\pgfpathlineto{\pgfqpoint{4.416792in}{10.398868in}}%
\pgfpathlineto{\pgfqpoint{4.416792in}{10.486604in}}%
\pgfusepath{stroke,fill}%
\end{pgfscope}%
\begin{pgfscope}%
\pgfpathrectangle{\pgfqpoint{0.380943in}{9.960189in}}{\pgfqpoint{4.650000in}{0.614151in}}%
\pgfusepath{clip}%
\pgfsetbuttcap%
\pgfsetroundjoin%
\definecolor{currentfill}{rgb}{0.994694,0.745098,0.602999}%
\pgfsetfillcolor{currentfill}%
\pgfsetlinewidth{0.250937pt}%
\definecolor{currentstroke}{rgb}{1.000000,1.000000,1.000000}%
\pgfsetstrokecolor{currentstroke}%
\pgfsetdash{}{0pt}%
\pgfpathmoveto{\pgfqpoint{4.504528in}{10.486604in}}%
\pgfpathlineto{\pgfqpoint{4.592264in}{10.486604in}}%
\pgfpathlineto{\pgfqpoint{4.592264in}{10.398868in}}%
\pgfpathlineto{\pgfqpoint{4.504528in}{10.398868in}}%
\pgfpathlineto{\pgfqpoint{4.504528in}{10.486604in}}%
\pgfusepath{stroke,fill}%
\end{pgfscope}%
\begin{pgfscope}%
\pgfpathrectangle{\pgfqpoint{0.380943in}{9.960189in}}{\pgfqpoint{4.650000in}{0.614151in}}%
\pgfusepath{clip}%
\pgfsetbuttcap%
\pgfsetroundjoin%
\definecolor{currentfill}{rgb}{0.994694,0.745098,0.602999}%
\pgfsetfillcolor{currentfill}%
\pgfsetlinewidth{0.250937pt}%
\definecolor{currentstroke}{rgb}{1.000000,1.000000,1.000000}%
\pgfsetstrokecolor{currentstroke}%
\pgfsetdash{}{0pt}%
\pgfpathmoveto{\pgfqpoint{4.592264in}{10.486604in}}%
\pgfpathlineto{\pgfqpoint{4.680000in}{10.486604in}}%
\pgfpathlineto{\pgfqpoint{4.680000in}{10.398868in}}%
\pgfpathlineto{\pgfqpoint{4.592264in}{10.398868in}}%
\pgfpathlineto{\pgfqpoint{4.592264in}{10.486604in}}%
\pgfusepath{stroke,fill}%
\end{pgfscope}%
\begin{pgfscope}%
\pgfpathrectangle{\pgfqpoint{0.380943in}{9.960189in}}{\pgfqpoint{4.650000in}{0.614151in}}%
\pgfusepath{clip}%
\pgfsetbuttcap%
\pgfsetroundjoin%
\definecolor{currentfill}{rgb}{1.000000,0.522261,0.496886}%
\pgfsetfillcolor{currentfill}%
\pgfsetlinewidth{0.250937pt}%
\definecolor{currentstroke}{rgb}{1.000000,1.000000,1.000000}%
\pgfsetstrokecolor{currentstroke}%
\pgfsetdash{}{0pt}%
\pgfpathmoveto{\pgfqpoint{4.680000in}{10.486604in}}%
\pgfpathlineto{\pgfqpoint{4.767736in}{10.486604in}}%
\pgfpathlineto{\pgfqpoint{4.767736in}{10.398868in}}%
\pgfpathlineto{\pgfqpoint{4.680000in}{10.398868in}}%
\pgfpathlineto{\pgfqpoint{4.680000in}{10.486604in}}%
\pgfusepath{stroke,fill}%
\end{pgfscope}%
\begin{pgfscope}%
\pgfpathrectangle{\pgfqpoint{0.380943in}{9.960189in}}{\pgfqpoint{4.650000in}{0.614151in}}%
\pgfusepath{clip}%
\pgfsetbuttcap%
\pgfsetroundjoin%
\definecolor{currentfill}{rgb}{1.000000,0.622145,0.537486}%
\pgfsetfillcolor{currentfill}%
\pgfsetlinewidth{0.250937pt}%
\definecolor{currentstroke}{rgb}{1.000000,1.000000,1.000000}%
\pgfsetstrokecolor{currentstroke}%
\pgfsetdash{}{0pt}%
\pgfpathmoveto{\pgfqpoint{4.767736in}{10.486604in}}%
\pgfpathlineto{\pgfqpoint{4.855471in}{10.486604in}}%
\pgfpathlineto{\pgfqpoint{4.855471in}{10.398868in}}%
\pgfpathlineto{\pgfqpoint{4.767736in}{10.398868in}}%
\pgfpathlineto{\pgfqpoint{4.767736in}{10.486604in}}%
\pgfusepath{stroke,fill}%
\end{pgfscope}%
\begin{pgfscope}%
\pgfpathrectangle{\pgfqpoint{0.380943in}{9.960189in}}{\pgfqpoint{4.650000in}{0.614151in}}%
\pgfusepath{clip}%
\pgfsetbuttcap%
\pgfsetroundjoin%
\definecolor{currentfill}{rgb}{0.800000,0.278431,0.278431}%
\pgfsetfillcolor{currentfill}%
\pgfsetlinewidth{0.250937pt}%
\definecolor{currentstroke}{rgb}{1.000000,1.000000,1.000000}%
\pgfsetstrokecolor{currentstroke}%
\pgfsetdash{}{0pt}%
\pgfpathmoveto{\pgfqpoint{4.855471in}{10.486604in}}%
\pgfpathlineto{\pgfqpoint{4.943207in}{10.486604in}}%
\pgfpathlineto{\pgfqpoint{4.943207in}{10.398868in}}%
\pgfpathlineto{\pgfqpoint{4.855471in}{10.398868in}}%
\pgfpathlineto{\pgfqpoint{4.855471in}{10.486604in}}%
\pgfusepath{stroke,fill}%
\end{pgfscope}%
\begin{pgfscope}%
\pgfpathrectangle{\pgfqpoint{0.380943in}{9.960189in}}{\pgfqpoint{4.650000in}{0.614151in}}%
\pgfusepath{clip}%
\pgfsetbuttcap%
\pgfsetroundjoin%
\definecolor{currentfill}{rgb}{0.982699,0.823991,0.657439}%
\pgfsetfillcolor{currentfill}%
\pgfsetlinewidth{0.250937pt}%
\definecolor{currentstroke}{rgb}{1.000000,1.000000,1.000000}%
\pgfsetstrokecolor{currentstroke}%
\pgfsetdash{}{0pt}%
\pgfpathmoveto{\pgfqpoint{4.943207in}{10.486604in}}%
\pgfpathlineto{\pgfqpoint{5.030943in}{10.486604in}}%
\pgfpathlineto{\pgfqpoint{5.030943in}{10.398868in}}%
\pgfpathlineto{\pgfqpoint{4.943207in}{10.398868in}}%
\pgfpathlineto{\pgfqpoint{4.943207in}{10.486604in}}%
\pgfusepath{stroke,fill}%
\end{pgfscope}%
\begin{pgfscope}%
\pgfpathrectangle{\pgfqpoint{0.380943in}{9.960189in}}{\pgfqpoint{4.650000in}{0.614151in}}%
\pgfusepath{clip}%
\pgfsetbuttcap%
\pgfsetroundjoin%
\pgfsetlinewidth{0.250937pt}%
\definecolor{currentstroke}{rgb}{1.000000,1.000000,1.000000}%
\pgfsetstrokecolor{currentstroke}%
\pgfsetdash{}{0pt}%
\pgfpathmoveto{\pgfqpoint{0.380943in}{10.398868in}}%
\pgfpathlineto{\pgfqpoint{0.468679in}{10.398868in}}%
\pgfpathlineto{\pgfqpoint{0.468679in}{10.311132in}}%
\pgfpathlineto{\pgfqpoint{0.380943in}{10.311132in}}%
\pgfpathlineto{\pgfqpoint{0.380943in}{10.398868in}}%
\pgfusepath{stroke}%
\end{pgfscope}%
\begin{pgfscope}%
\pgfpathrectangle{\pgfqpoint{0.380943in}{9.960189in}}{\pgfqpoint{4.650000in}{0.614151in}}%
\pgfusepath{clip}%
\pgfsetbuttcap%
\pgfsetroundjoin%
\definecolor{currentfill}{rgb}{1.000000,1.000000,0.929412}%
\pgfsetfillcolor{currentfill}%
\pgfsetlinewidth{0.250937pt}%
\definecolor{currentstroke}{rgb}{1.000000,1.000000,1.000000}%
\pgfsetstrokecolor{currentstroke}%
\pgfsetdash{}{0pt}%
\pgfpathmoveto{\pgfqpoint{0.468679in}{10.398868in}}%
\pgfpathlineto{\pgfqpoint{0.556415in}{10.398868in}}%
\pgfpathlineto{\pgfqpoint{0.556415in}{10.311132in}}%
\pgfpathlineto{\pgfqpoint{0.468679in}{10.311132in}}%
\pgfpathlineto{\pgfqpoint{0.468679in}{10.398868in}}%
\pgfusepath{stroke,fill}%
\end{pgfscope}%
\begin{pgfscope}%
\pgfpathrectangle{\pgfqpoint{0.380943in}{9.960189in}}{\pgfqpoint{4.650000in}{0.614151in}}%
\pgfusepath{clip}%
\pgfsetbuttcap%
\pgfsetroundjoin%
\definecolor{currentfill}{rgb}{1.000000,1.000000,0.929412}%
\pgfsetfillcolor{currentfill}%
\pgfsetlinewidth{0.250937pt}%
\definecolor{currentstroke}{rgb}{1.000000,1.000000,1.000000}%
\pgfsetstrokecolor{currentstroke}%
\pgfsetdash{}{0pt}%
\pgfpathmoveto{\pgfqpoint{0.556415in}{10.398868in}}%
\pgfpathlineto{\pgfqpoint{0.644151in}{10.398868in}}%
\pgfpathlineto{\pgfqpoint{0.644151in}{10.311132in}}%
\pgfpathlineto{\pgfqpoint{0.556415in}{10.311132in}}%
\pgfpathlineto{\pgfqpoint{0.556415in}{10.398868in}}%
\pgfusepath{stroke,fill}%
\end{pgfscope}%
\begin{pgfscope}%
\pgfpathrectangle{\pgfqpoint{0.380943in}{9.960189in}}{\pgfqpoint{4.650000in}{0.614151in}}%
\pgfusepath{clip}%
\pgfsetbuttcap%
\pgfsetroundjoin%
\definecolor{currentfill}{rgb}{1.000000,1.000000,0.929412}%
\pgfsetfillcolor{currentfill}%
\pgfsetlinewidth{0.250937pt}%
\definecolor{currentstroke}{rgb}{1.000000,1.000000,1.000000}%
\pgfsetstrokecolor{currentstroke}%
\pgfsetdash{}{0pt}%
\pgfpathmoveto{\pgfqpoint{0.644151in}{10.398868in}}%
\pgfpathlineto{\pgfqpoint{0.731886in}{10.398868in}}%
\pgfpathlineto{\pgfqpoint{0.731886in}{10.311132in}}%
\pgfpathlineto{\pgfqpoint{0.644151in}{10.311132in}}%
\pgfpathlineto{\pgfqpoint{0.644151in}{10.398868in}}%
\pgfusepath{stroke,fill}%
\end{pgfscope}%
\begin{pgfscope}%
\pgfpathrectangle{\pgfqpoint{0.380943in}{9.960189in}}{\pgfqpoint{4.650000in}{0.614151in}}%
\pgfusepath{clip}%
\pgfsetbuttcap%
\pgfsetroundjoin%
\definecolor{currentfill}{rgb}{1.000000,1.000000,0.929412}%
\pgfsetfillcolor{currentfill}%
\pgfsetlinewidth{0.250937pt}%
\definecolor{currentstroke}{rgb}{1.000000,1.000000,1.000000}%
\pgfsetstrokecolor{currentstroke}%
\pgfsetdash{}{0pt}%
\pgfpathmoveto{\pgfqpoint{0.731886in}{10.398868in}}%
\pgfpathlineto{\pgfqpoint{0.819622in}{10.398868in}}%
\pgfpathlineto{\pgfqpoint{0.819622in}{10.311132in}}%
\pgfpathlineto{\pgfqpoint{0.731886in}{10.311132in}}%
\pgfpathlineto{\pgfqpoint{0.731886in}{10.398868in}}%
\pgfusepath{stroke,fill}%
\end{pgfscope}%
\begin{pgfscope}%
\pgfpathrectangle{\pgfqpoint{0.380943in}{9.960189in}}{\pgfqpoint{4.650000in}{0.614151in}}%
\pgfusepath{clip}%
\pgfsetbuttcap%
\pgfsetroundjoin%
\definecolor{currentfill}{rgb}{1.000000,1.000000,0.929412}%
\pgfsetfillcolor{currentfill}%
\pgfsetlinewidth{0.250937pt}%
\definecolor{currentstroke}{rgb}{1.000000,1.000000,1.000000}%
\pgfsetstrokecolor{currentstroke}%
\pgfsetdash{}{0pt}%
\pgfpathmoveto{\pgfqpoint{0.819622in}{10.398868in}}%
\pgfpathlineto{\pgfqpoint{0.907358in}{10.398868in}}%
\pgfpathlineto{\pgfqpoint{0.907358in}{10.311132in}}%
\pgfpathlineto{\pgfqpoint{0.819622in}{10.311132in}}%
\pgfpathlineto{\pgfqpoint{0.819622in}{10.398868in}}%
\pgfusepath{stroke,fill}%
\end{pgfscope}%
\begin{pgfscope}%
\pgfpathrectangle{\pgfqpoint{0.380943in}{9.960189in}}{\pgfqpoint{4.650000in}{0.614151in}}%
\pgfusepath{clip}%
\pgfsetbuttcap%
\pgfsetroundjoin%
\definecolor{currentfill}{rgb}{1.000000,1.000000,0.929412}%
\pgfsetfillcolor{currentfill}%
\pgfsetlinewidth{0.250937pt}%
\definecolor{currentstroke}{rgb}{1.000000,1.000000,1.000000}%
\pgfsetstrokecolor{currentstroke}%
\pgfsetdash{}{0pt}%
\pgfpathmoveto{\pgfqpoint{0.907358in}{10.398868in}}%
\pgfpathlineto{\pgfqpoint{0.995094in}{10.398868in}}%
\pgfpathlineto{\pgfqpoint{0.995094in}{10.311132in}}%
\pgfpathlineto{\pgfqpoint{0.907358in}{10.311132in}}%
\pgfpathlineto{\pgfqpoint{0.907358in}{10.398868in}}%
\pgfusepath{stroke,fill}%
\end{pgfscope}%
\begin{pgfscope}%
\pgfpathrectangle{\pgfqpoint{0.380943in}{9.960189in}}{\pgfqpoint{4.650000in}{0.614151in}}%
\pgfusepath{clip}%
\pgfsetbuttcap%
\pgfsetroundjoin%
\definecolor{currentfill}{rgb}{1.000000,1.000000,0.929412}%
\pgfsetfillcolor{currentfill}%
\pgfsetlinewidth{0.250937pt}%
\definecolor{currentstroke}{rgb}{1.000000,1.000000,1.000000}%
\pgfsetstrokecolor{currentstroke}%
\pgfsetdash{}{0pt}%
\pgfpathmoveto{\pgfqpoint{0.995094in}{10.398868in}}%
\pgfpathlineto{\pgfqpoint{1.082830in}{10.398868in}}%
\pgfpathlineto{\pgfqpoint{1.082830in}{10.311132in}}%
\pgfpathlineto{\pgfqpoint{0.995094in}{10.311132in}}%
\pgfpathlineto{\pgfqpoint{0.995094in}{10.398868in}}%
\pgfusepath{stroke,fill}%
\end{pgfscope}%
\begin{pgfscope}%
\pgfpathrectangle{\pgfqpoint{0.380943in}{9.960189in}}{\pgfqpoint{4.650000in}{0.614151in}}%
\pgfusepath{clip}%
\pgfsetbuttcap%
\pgfsetroundjoin%
\definecolor{currentfill}{rgb}{1.000000,1.000000,0.929412}%
\pgfsetfillcolor{currentfill}%
\pgfsetlinewidth{0.250937pt}%
\definecolor{currentstroke}{rgb}{1.000000,1.000000,1.000000}%
\pgfsetstrokecolor{currentstroke}%
\pgfsetdash{}{0pt}%
\pgfpathmoveto{\pgfqpoint{1.082830in}{10.398868in}}%
\pgfpathlineto{\pgfqpoint{1.170566in}{10.398868in}}%
\pgfpathlineto{\pgfqpoint{1.170566in}{10.311132in}}%
\pgfpathlineto{\pgfqpoint{1.082830in}{10.311132in}}%
\pgfpathlineto{\pgfqpoint{1.082830in}{10.398868in}}%
\pgfusepath{stroke,fill}%
\end{pgfscope}%
\begin{pgfscope}%
\pgfpathrectangle{\pgfqpoint{0.380943in}{9.960189in}}{\pgfqpoint{4.650000in}{0.614151in}}%
\pgfusepath{clip}%
\pgfsetbuttcap%
\pgfsetroundjoin%
\definecolor{currentfill}{rgb}{1.000000,1.000000,0.929412}%
\pgfsetfillcolor{currentfill}%
\pgfsetlinewidth{0.250937pt}%
\definecolor{currentstroke}{rgb}{1.000000,1.000000,1.000000}%
\pgfsetstrokecolor{currentstroke}%
\pgfsetdash{}{0pt}%
\pgfpathmoveto{\pgfqpoint{1.170566in}{10.398868in}}%
\pgfpathlineto{\pgfqpoint{1.258302in}{10.398868in}}%
\pgfpathlineto{\pgfqpoint{1.258302in}{10.311132in}}%
\pgfpathlineto{\pgfqpoint{1.170566in}{10.311132in}}%
\pgfpathlineto{\pgfqpoint{1.170566in}{10.398868in}}%
\pgfusepath{stroke,fill}%
\end{pgfscope}%
\begin{pgfscope}%
\pgfpathrectangle{\pgfqpoint{0.380943in}{9.960189in}}{\pgfqpoint{4.650000in}{0.614151in}}%
\pgfusepath{clip}%
\pgfsetbuttcap%
\pgfsetroundjoin%
\definecolor{currentfill}{rgb}{1.000000,1.000000,0.929412}%
\pgfsetfillcolor{currentfill}%
\pgfsetlinewidth{0.250937pt}%
\definecolor{currentstroke}{rgb}{1.000000,1.000000,1.000000}%
\pgfsetstrokecolor{currentstroke}%
\pgfsetdash{}{0pt}%
\pgfpathmoveto{\pgfqpoint{1.258302in}{10.398868in}}%
\pgfpathlineto{\pgfqpoint{1.346037in}{10.398868in}}%
\pgfpathlineto{\pgfqpoint{1.346037in}{10.311132in}}%
\pgfpathlineto{\pgfqpoint{1.258302in}{10.311132in}}%
\pgfpathlineto{\pgfqpoint{1.258302in}{10.398868in}}%
\pgfusepath{stroke,fill}%
\end{pgfscope}%
\begin{pgfscope}%
\pgfpathrectangle{\pgfqpoint{0.380943in}{9.960189in}}{\pgfqpoint{4.650000in}{0.614151in}}%
\pgfusepath{clip}%
\pgfsetbuttcap%
\pgfsetroundjoin%
\definecolor{currentfill}{rgb}{1.000000,1.000000,0.929412}%
\pgfsetfillcolor{currentfill}%
\pgfsetlinewidth{0.250937pt}%
\definecolor{currentstroke}{rgb}{1.000000,1.000000,1.000000}%
\pgfsetstrokecolor{currentstroke}%
\pgfsetdash{}{0pt}%
\pgfpathmoveto{\pgfqpoint{1.346037in}{10.398868in}}%
\pgfpathlineto{\pgfqpoint{1.433773in}{10.398868in}}%
\pgfpathlineto{\pgfqpoint{1.433773in}{10.311132in}}%
\pgfpathlineto{\pgfqpoint{1.346037in}{10.311132in}}%
\pgfpathlineto{\pgfqpoint{1.346037in}{10.398868in}}%
\pgfusepath{stroke,fill}%
\end{pgfscope}%
\begin{pgfscope}%
\pgfpathrectangle{\pgfqpoint{0.380943in}{9.960189in}}{\pgfqpoint{4.650000in}{0.614151in}}%
\pgfusepath{clip}%
\pgfsetbuttcap%
\pgfsetroundjoin%
\definecolor{currentfill}{rgb}{1.000000,1.000000,0.929412}%
\pgfsetfillcolor{currentfill}%
\pgfsetlinewidth{0.250937pt}%
\definecolor{currentstroke}{rgb}{1.000000,1.000000,1.000000}%
\pgfsetstrokecolor{currentstroke}%
\pgfsetdash{}{0pt}%
\pgfpathmoveto{\pgfqpoint{1.433773in}{10.398868in}}%
\pgfpathlineto{\pgfqpoint{1.521509in}{10.398868in}}%
\pgfpathlineto{\pgfqpoint{1.521509in}{10.311132in}}%
\pgfpathlineto{\pgfqpoint{1.433773in}{10.311132in}}%
\pgfpathlineto{\pgfqpoint{1.433773in}{10.398868in}}%
\pgfusepath{stroke,fill}%
\end{pgfscope}%
\begin{pgfscope}%
\pgfpathrectangle{\pgfqpoint{0.380943in}{9.960189in}}{\pgfqpoint{4.650000in}{0.614151in}}%
\pgfusepath{clip}%
\pgfsetbuttcap%
\pgfsetroundjoin%
\definecolor{currentfill}{rgb}{1.000000,1.000000,0.929412}%
\pgfsetfillcolor{currentfill}%
\pgfsetlinewidth{0.250937pt}%
\definecolor{currentstroke}{rgb}{1.000000,1.000000,1.000000}%
\pgfsetstrokecolor{currentstroke}%
\pgfsetdash{}{0pt}%
\pgfpathmoveto{\pgfqpoint{1.521509in}{10.398868in}}%
\pgfpathlineto{\pgfqpoint{1.609245in}{10.398868in}}%
\pgfpathlineto{\pgfqpoint{1.609245in}{10.311132in}}%
\pgfpathlineto{\pgfqpoint{1.521509in}{10.311132in}}%
\pgfpathlineto{\pgfqpoint{1.521509in}{10.398868in}}%
\pgfusepath{stroke,fill}%
\end{pgfscope}%
\begin{pgfscope}%
\pgfpathrectangle{\pgfqpoint{0.380943in}{9.960189in}}{\pgfqpoint{4.650000in}{0.614151in}}%
\pgfusepath{clip}%
\pgfsetbuttcap%
\pgfsetroundjoin%
\definecolor{currentfill}{rgb}{1.000000,1.000000,0.929412}%
\pgfsetfillcolor{currentfill}%
\pgfsetlinewidth{0.250937pt}%
\definecolor{currentstroke}{rgb}{1.000000,1.000000,1.000000}%
\pgfsetstrokecolor{currentstroke}%
\pgfsetdash{}{0pt}%
\pgfpathmoveto{\pgfqpoint{1.609245in}{10.398868in}}%
\pgfpathlineto{\pgfqpoint{1.696981in}{10.398868in}}%
\pgfpathlineto{\pgfqpoint{1.696981in}{10.311132in}}%
\pgfpathlineto{\pgfqpoint{1.609245in}{10.311132in}}%
\pgfpathlineto{\pgfqpoint{1.609245in}{10.398868in}}%
\pgfusepath{stroke,fill}%
\end{pgfscope}%
\begin{pgfscope}%
\pgfpathrectangle{\pgfqpoint{0.380943in}{9.960189in}}{\pgfqpoint{4.650000in}{0.614151in}}%
\pgfusepath{clip}%
\pgfsetbuttcap%
\pgfsetroundjoin%
\definecolor{currentfill}{rgb}{1.000000,1.000000,0.929412}%
\pgfsetfillcolor{currentfill}%
\pgfsetlinewidth{0.250937pt}%
\definecolor{currentstroke}{rgb}{1.000000,1.000000,1.000000}%
\pgfsetstrokecolor{currentstroke}%
\pgfsetdash{}{0pt}%
\pgfpathmoveto{\pgfqpoint{1.696981in}{10.398868in}}%
\pgfpathlineto{\pgfqpoint{1.784717in}{10.398868in}}%
\pgfpathlineto{\pgfqpoint{1.784717in}{10.311132in}}%
\pgfpathlineto{\pgfqpoint{1.696981in}{10.311132in}}%
\pgfpathlineto{\pgfqpoint{1.696981in}{10.398868in}}%
\pgfusepath{stroke,fill}%
\end{pgfscope}%
\begin{pgfscope}%
\pgfpathrectangle{\pgfqpoint{0.380943in}{9.960189in}}{\pgfqpoint{4.650000in}{0.614151in}}%
\pgfusepath{clip}%
\pgfsetbuttcap%
\pgfsetroundjoin%
\definecolor{currentfill}{rgb}{1.000000,1.000000,0.929412}%
\pgfsetfillcolor{currentfill}%
\pgfsetlinewidth{0.250937pt}%
\definecolor{currentstroke}{rgb}{1.000000,1.000000,1.000000}%
\pgfsetstrokecolor{currentstroke}%
\pgfsetdash{}{0pt}%
\pgfpathmoveto{\pgfqpoint{1.784717in}{10.398868in}}%
\pgfpathlineto{\pgfqpoint{1.872452in}{10.398868in}}%
\pgfpathlineto{\pgfqpoint{1.872452in}{10.311132in}}%
\pgfpathlineto{\pgfqpoint{1.784717in}{10.311132in}}%
\pgfpathlineto{\pgfqpoint{1.784717in}{10.398868in}}%
\pgfusepath{stroke,fill}%
\end{pgfscope}%
\begin{pgfscope}%
\pgfpathrectangle{\pgfqpoint{0.380943in}{9.960189in}}{\pgfqpoint{4.650000in}{0.614151in}}%
\pgfusepath{clip}%
\pgfsetbuttcap%
\pgfsetroundjoin%
\definecolor{currentfill}{rgb}{1.000000,1.000000,0.929412}%
\pgfsetfillcolor{currentfill}%
\pgfsetlinewidth{0.250937pt}%
\definecolor{currentstroke}{rgb}{1.000000,1.000000,1.000000}%
\pgfsetstrokecolor{currentstroke}%
\pgfsetdash{}{0pt}%
\pgfpathmoveto{\pgfqpoint{1.872452in}{10.398868in}}%
\pgfpathlineto{\pgfqpoint{1.960188in}{10.398868in}}%
\pgfpathlineto{\pgfqpoint{1.960188in}{10.311132in}}%
\pgfpathlineto{\pgfqpoint{1.872452in}{10.311132in}}%
\pgfpathlineto{\pgfqpoint{1.872452in}{10.398868in}}%
\pgfusepath{stroke,fill}%
\end{pgfscope}%
\begin{pgfscope}%
\pgfpathrectangle{\pgfqpoint{0.380943in}{9.960189in}}{\pgfqpoint{4.650000in}{0.614151in}}%
\pgfusepath{clip}%
\pgfsetbuttcap%
\pgfsetroundjoin%
\definecolor{currentfill}{rgb}{1.000000,1.000000,0.929412}%
\pgfsetfillcolor{currentfill}%
\pgfsetlinewidth{0.250937pt}%
\definecolor{currentstroke}{rgb}{1.000000,1.000000,1.000000}%
\pgfsetstrokecolor{currentstroke}%
\pgfsetdash{}{0pt}%
\pgfpathmoveto{\pgfqpoint{1.960188in}{10.398868in}}%
\pgfpathlineto{\pgfqpoint{2.047924in}{10.398868in}}%
\pgfpathlineto{\pgfqpoint{2.047924in}{10.311132in}}%
\pgfpathlineto{\pgfqpoint{1.960188in}{10.311132in}}%
\pgfpathlineto{\pgfqpoint{1.960188in}{10.398868in}}%
\pgfusepath{stroke,fill}%
\end{pgfscope}%
\begin{pgfscope}%
\pgfpathrectangle{\pgfqpoint{0.380943in}{9.960189in}}{\pgfqpoint{4.650000in}{0.614151in}}%
\pgfusepath{clip}%
\pgfsetbuttcap%
\pgfsetroundjoin%
\definecolor{currentfill}{rgb}{1.000000,1.000000,0.929412}%
\pgfsetfillcolor{currentfill}%
\pgfsetlinewidth{0.250937pt}%
\definecolor{currentstroke}{rgb}{1.000000,1.000000,1.000000}%
\pgfsetstrokecolor{currentstroke}%
\pgfsetdash{}{0pt}%
\pgfpathmoveto{\pgfqpoint{2.047924in}{10.398868in}}%
\pgfpathlineto{\pgfqpoint{2.135660in}{10.398868in}}%
\pgfpathlineto{\pgfqpoint{2.135660in}{10.311132in}}%
\pgfpathlineto{\pgfqpoint{2.047924in}{10.311132in}}%
\pgfpathlineto{\pgfqpoint{2.047924in}{10.398868in}}%
\pgfusepath{stroke,fill}%
\end{pgfscope}%
\begin{pgfscope}%
\pgfpathrectangle{\pgfqpoint{0.380943in}{9.960189in}}{\pgfqpoint{4.650000in}{0.614151in}}%
\pgfusepath{clip}%
\pgfsetbuttcap%
\pgfsetroundjoin%
\definecolor{currentfill}{rgb}{1.000000,1.000000,0.929412}%
\pgfsetfillcolor{currentfill}%
\pgfsetlinewidth{0.250937pt}%
\definecolor{currentstroke}{rgb}{1.000000,1.000000,1.000000}%
\pgfsetstrokecolor{currentstroke}%
\pgfsetdash{}{0pt}%
\pgfpathmoveto{\pgfqpoint{2.135660in}{10.398868in}}%
\pgfpathlineto{\pgfqpoint{2.223396in}{10.398868in}}%
\pgfpathlineto{\pgfqpoint{2.223396in}{10.311132in}}%
\pgfpathlineto{\pgfqpoint{2.135660in}{10.311132in}}%
\pgfpathlineto{\pgfqpoint{2.135660in}{10.398868in}}%
\pgfusepath{stroke,fill}%
\end{pgfscope}%
\begin{pgfscope}%
\pgfpathrectangle{\pgfqpoint{0.380943in}{9.960189in}}{\pgfqpoint{4.650000in}{0.614151in}}%
\pgfusepath{clip}%
\pgfsetbuttcap%
\pgfsetroundjoin%
\definecolor{currentfill}{rgb}{1.000000,1.000000,0.929412}%
\pgfsetfillcolor{currentfill}%
\pgfsetlinewidth{0.250937pt}%
\definecolor{currentstroke}{rgb}{1.000000,1.000000,1.000000}%
\pgfsetstrokecolor{currentstroke}%
\pgfsetdash{}{0pt}%
\pgfpathmoveto{\pgfqpoint{2.223396in}{10.398868in}}%
\pgfpathlineto{\pgfqpoint{2.311132in}{10.398868in}}%
\pgfpathlineto{\pgfqpoint{2.311132in}{10.311132in}}%
\pgfpathlineto{\pgfqpoint{2.223396in}{10.311132in}}%
\pgfpathlineto{\pgfqpoint{2.223396in}{10.398868in}}%
\pgfusepath{stroke,fill}%
\end{pgfscope}%
\begin{pgfscope}%
\pgfpathrectangle{\pgfqpoint{0.380943in}{9.960189in}}{\pgfqpoint{4.650000in}{0.614151in}}%
\pgfusepath{clip}%
\pgfsetbuttcap%
\pgfsetroundjoin%
\definecolor{currentfill}{rgb}{1.000000,1.000000,0.929412}%
\pgfsetfillcolor{currentfill}%
\pgfsetlinewidth{0.250937pt}%
\definecolor{currentstroke}{rgb}{1.000000,1.000000,1.000000}%
\pgfsetstrokecolor{currentstroke}%
\pgfsetdash{}{0pt}%
\pgfpathmoveto{\pgfqpoint{2.311132in}{10.398868in}}%
\pgfpathlineto{\pgfqpoint{2.398868in}{10.398868in}}%
\pgfpathlineto{\pgfqpoint{2.398868in}{10.311132in}}%
\pgfpathlineto{\pgfqpoint{2.311132in}{10.311132in}}%
\pgfpathlineto{\pgfqpoint{2.311132in}{10.398868in}}%
\pgfusepath{stroke,fill}%
\end{pgfscope}%
\begin{pgfscope}%
\pgfpathrectangle{\pgfqpoint{0.380943in}{9.960189in}}{\pgfqpoint{4.650000in}{0.614151in}}%
\pgfusepath{clip}%
\pgfsetbuttcap%
\pgfsetroundjoin%
\definecolor{currentfill}{rgb}{1.000000,1.000000,0.929412}%
\pgfsetfillcolor{currentfill}%
\pgfsetlinewidth{0.250937pt}%
\definecolor{currentstroke}{rgb}{1.000000,1.000000,1.000000}%
\pgfsetstrokecolor{currentstroke}%
\pgfsetdash{}{0pt}%
\pgfpathmoveto{\pgfqpoint{2.398868in}{10.398868in}}%
\pgfpathlineto{\pgfqpoint{2.486603in}{10.398868in}}%
\pgfpathlineto{\pgfqpoint{2.486603in}{10.311132in}}%
\pgfpathlineto{\pgfqpoint{2.398868in}{10.311132in}}%
\pgfpathlineto{\pgfqpoint{2.398868in}{10.398868in}}%
\pgfusepath{stroke,fill}%
\end{pgfscope}%
\begin{pgfscope}%
\pgfpathrectangle{\pgfqpoint{0.380943in}{9.960189in}}{\pgfqpoint{4.650000in}{0.614151in}}%
\pgfusepath{clip}%
\pgfsetbuttcap%
\pgfsetroundjoin%
\definecolor{currentfill}{rgb}{1.000000,1.000000,0.929412}%
\pgfsetfillcolor{currentfill}%
\pgfsetlinewidth{0.250937pt}%
\definecolor{currentstroke}{rgb}{1.000000,1.000000,1.000000}%
\pgfsetstrokecolor{currentstroke}%
\pgfsetdash{}{0pt}%
\pgfpathmoveto{\pgfqpoint{2.486603in}{10.398868in}}%
\pgfpathlineto{\pgfqpoint{2.574339in}{10.398868in}}%
\pgfpathlineto{\pgfqpoint{2.574339in}{10.311132in}}%
\pgfpathlineto{\pgfqpoint{2.486603in}{10.311132in}}%
\pgfpathlineto{\pgfqpoint{2.486603in}{10.398868in}}%
\pgfusepath{stroke,fill}%
\end{pgfscope}%
\begin{pgfscope}%
\pgfpathrectangle{\pgfqpoint{0.380943in}{9.960189in}}{\pgfqpoint{4.650000in}{0.614151in}}%
\pgfusepath{clip}%
\pgfsetbuttcap%
\pgfsetroundjoin%
\definecolor{currentfill}{rgb}{1.000000,1.000000,0.929412}%
\pgfsetfillcolor{currentfill}%
\pgfsetlinewidth{0.250937pt}%
\definecolor{currentstroke}{rgb}{1.000000,1.000000,1.000000}%
\pgfsetstrokecolor{currentstroke}%
\pgfsetdash{}{0pt}%
\pgfpathmoveto{\pgfqpoint{2.574339in}{10.398868in}}%
\pgfpathlineto{\pgfqpoint{2.662075in}{10.398868in}}%
\pgfpathlineto{\pgfqpoint{2.662075in}{10.311132in}}%
\pgfpathlineto{\pgfqpoint{2.574339in}{10.311132in}}%
\pgfpathlineto{\pgfqpoint{2.574339in}{10.398868in}}%
\pgfusepath{stroke,fill}%
\end{pgfscope}%
\begin{pgfscope}%
\pgfpathrectangle{\pgfqpoint{0.380943in}{9.960189in}}{\pgfqpoint{4.650000in}{0.614151in}}%
\pgfusepath{clip}%
\pgfsetbuttcap%
\pgfsetroundjoin%
\definecolor{currentfill}{rgb}{1.000000,1.000000,0.929412}%
\pgfsetfillcolor{currentfill}%
\pgfsetlinewidth{0.250937pt}%
\definecolor{currentstroke}{rgb}{1.000000,1.000000,1.000000}%
\pgfsetstrokecolor{currentstroke}%
\pgfsetdash{}{0pt}%
\pgfpathmoveto{\pgfqpoint{2.662075in}{10.398868in}}%
\pgfpathlineto{\pgfqpoint{2.749811in}{10.398868in}}%
\pgfpathlineto{\pgfqpoint{2.749811in}{10.311132in}}%
\pgfpathlineto{\pgfqpoint{2.662075in}{10.311132in}}%
\pgfpathlineto{\pgfqpoint{2.662075in}{10.398868in}}%
\pgfusepath{stroke,fill}%
\end{pgfscope}%
\begin{pgfscope}%
\pgfpathrectangle{\pgfqpoint{0.380943in}{9.960189in}}{\pgfqpoint{4.650000in}{0.614151in}}%
\pgfusepath{clip}%
\pgfsetbuttcap%
\pgfsetroundjoin%
\definecolor{currentfill}{rgb}{1.000000,1.000000,0.929412}%
\pgfsetfillcolor{currentfill}%
\pgfsetlinewidth{0.250937pt}%
\definecolor{currentstroke}{rgb}{1.000000,1.000000,1.000000}%
\pgfsetstrokecolor{currentstroke}%
\pgfsetdash{}{0pt}%
\pgfpathmoveto{\pgfqpoint{2.749811in}{10.398868in}}%
\pgfpathlineto{\pgfqpoint{2.837547in}{10.398868in}}%
\pgfpathlineto{\pgfqpoint{2.837547in}{10.311132in}}%
\pgfpathlineto{\pgfqpoint{2.749811in}{10.311132in}}%
\pgfpathlineto{\pgfqpoint{2.749811in}{10.398868in}}%
\pgfusepath{stroke,fill}%
\end{pgfscope}%
\begin{pgfscope}%
\pgfpathrectangle{\pgfqpoint{0.380943in}{9.960189in}}{\pgfqpoint{4.650000in}{0.614151in}}%
\pgfusepath{clip}%
\pgfsetbuttcap%
\pgfsetroundjoin%
\definecolor{currentfill}{rgb}{1.000000,1.000000,0.929412}%
\pgfsetfillcolor{currentfill}%
\pgfsetlinewidth{0.250937pt}%
\definecolor{currentstroke}{rgb}{1.000000,1.000000,1.000000}%
\pgfsetstrokecolor{currentstroke}%
\pgfsetdash{}{0pt}%
\pgfpathmoveto{\pgfqpoint{2.837547in}{10.398868in}}%
\pgfpathlineto{\pgfqpoint{2.925283in}{10.398868in}}%
\pgfpathlineto{\pgfqpoint{2.925283in}{10.311132in}}%
\pgfpathlineto{\pgfqpoint{2.837547in}{10.311132in}}%
\pgfpathlineto{\pgfqpoint{2.837547in}{10.398868in}}%
\pgfusepath{stroke,fill}%
\end{pgfscope}%
\begin{pgfscope}%
\pgfpathrectangle{\pgfqpoint{0.380943in}{9.960189in}}{\pgfqpoint{4.650000in}{0.614151in}}%
\pgfusepath{clip}%
\pgfsetbuttcap%
\pgfsetroundjoin%
\definecolor{currentfill}{rgb}{1.000000,1.000000,0.929412}%
\pgfsetfillcolor{currentfill}%
\pgfsetlinewidth{0.250937pt}%
\definecolor{currentstroke}{rgb}{1.000000,1.000000,1.000000}%
\pgfsetstrokecolor{currentstroke}%
\pgfsetdash{}{0pt}%
\pgfpathmoveto{\pgfqpoint{2.925283in}{10.398868in}}%
\pgfpathlineto{\pgfqpoint{3.013019in}{10.398868in}}%
\pgfpathlineto{\pgfqpoint{3.013019in}{10.311132in}}%
\pgfpathlineto{\pgfqpoint{2.925283in}{10.311132in}}%
\pgfpathlineto{\pgfqpoint{2.925283in}{10.398868in}}%
\pgfusepath{stroke,fill}%
\end{pgfscope}%
\begin{pgfscope}%
\pgfpathrectangle{\pgfqpoint{0.380943in}{9.960189in}}{\pgfqpoint{4.650000in}{0.614151in}}%
\pgfusepath{clip}%
\pgfsetbuttcap%
\pgfsetroundjoin%
\definecolor{currentfill}{rgb}{1.000000,1.000000,0.929412}%
\pgfsetfillcolor{currentfill}%
\pgfsetlinewidth{0.250937pt}%
\definecolor{currentstroke}{rgb}{1.000000,1.000000,1.000000}%
\pgfsetstrokecolor{currentstroke}%
\pgfsetdash{}{0pt}%
\pgfpathmoveto{\pgfqpoint{3.013019in}{10.398868in}}%
\pgfpathlineto{\pgfqpoint{3.100754in}{10.398868in}}%
\pgfpathlineto{\pgfqpoint{3.100754in}{10.311132in}}%
\pgfpathlineto{\pgfqpoint{3.013019in}{10.311132in}}%
\pgfpathlineto{\pgfqpoint{3.013019in}{10.398868in}}%
\pgfusepath{stroke,fill}%
\end{pgfscope}%
\begin{pgfscope}%
\pgfpathrectangle{\pgfqpoint{0.380943in}{9.960189in}}{\pgfqpoint{4.650000in}{0.614151in}}%
\pgfusepath{clip}%
\pgfsetbuttcap%
\pgfsetroundjoin%
\definecolor{currentfill}{rgb}{1.000000,1.000000,0.929412}%
\pgfsetfillcolor{currentfill}%
\pgfsetlinewidth{0.250937pt}%
\definecolor{currentstroke}{rgb}{1.000000,1.000000,1.000000}%
\pgfsetstrokecolor{currentstroke}%
\pgfsetdash{}{0pt}%
\pgfpathmoveto{\pgfqpoint{3.100754in}{10.398868in}}%
\pgfpathlineto{\pgfqpoint{3.188490in}{10.398868in}}%
\pgfpathlineto{\pgfqpoint{3.188490in}{10.311132in}}%
\pgfpathlineto{\pgfqpoint{3.100754in}{10.311132in}}%
\pgfpathlineto{\pgfqpoint{3.100754in}{10.398868in}}%
\pgfusepath{stroke,fill}%
\end{pgfscope}%
\begin{pgfscope}%
\pgfpathrectangle{\pgfqpoint{0.380943in}{9.960189in}}{\pgfqpoint{4.650000in}{0.614151in}}%
\pgfusepath{clip}%
\pgfsetbuttcap%
\pgfsetroundjoin%
\definecolor{currentfill}{rgb}{1.000000,1.000000,0.929412}%
\pgfsetfillcolor{currentfill}%
\pgfsetlinewidth{0.250937pt}%
\definecolor{currentstroke}{rgb}{1.000000,1.000000,1.000000}%
\pgfsetstrokecolor{currentstroke}%
\pgfsetdash{}{0pt}%
\pgfpathmoveto{\pgfqpoint{3.188490in}{10.398868in}}%
\pgfpathlineto{\pgfqpoint{3.276226in}{10.398868in}}%
\pgfpathlineto{\pgfqpoint{3.276226in}{10.311132in}}%
\pgfpathlineto{\pgfqpoint{3.188490in}{10.311132in}}%
\pgfpathlineto{\pgfqpoint{3.188490in}{10.398868in}}%
\pgfusepath{stroke,fill}%
\end{pgfscope}%
\begin{pgfscope}%
\pgfpathrectangle{\pgfqpoint{0.380943in}{9.960189in}}{\pgfqpoint{4.650000in}{0.614151in}}%
\pgfusepath{clip}%
\pgfsetbuttcap%
\pgfsetroundjoin%
\definecolor{currentfill}{rgb}{1.000000,1.000000,0.929412}%
\pgfsetfillcolor{currentfill}%
\pgfsetlinewidth{0.250937pt}%
\definecolor{currentstroke}{rgb}{1.000000,1.000000,1.000000}%
\pgfsetstrokecolor{currentstroke}%
\pgfsetdash{}{0pt}%
\pgfpathmoveto{\pgfqpoint{3.276226in}{10.398868in}}%
\pgfpathlineto{\pgfqpoint{3.363962in}{10.398868in}}%
\pgfpathlineto{\pgfqpoint{3.363962in}{10.311132in}}%
\pgfpathlineto{\pgfqpoint{3.276226in}{10.311132in}}%
\pgfpathlineto{\pgfqpoint{3.276226in}{10.398868in}}%
\pgfusepath{stroke,fill}%
\end{pgfscope}%
\begin{pgfscope}%
\pgfpathrectangle{\pgfqpoint{0.380943in}{9.960189in}}{\pgfqpoint{4.650000in}{0.614151in}}%
\pgfusepath{clip}%
\pgfsetbuttcap%
\pgfsetroundjoin%
\definecolor{currentfill}{rgb}{1.000000,1.000000,0.929412}%
\pgfsetfillcolor{currentfill}%
\pgfsetlinewidth{0.250937pt}%
\definecolor{currentstroke}{rgb}{1.000000,1.000000,1.000000}%
\pgfsetstrokecolor{currentstroke}%
\pgfsetdash{}{0pt}%
\pgfpathmoveto{\pgfqpoint{3.363962in}{10.398868in}}%
\pgfpathlineto{\pgfqpoint{3.451698in}{10.398868in}}%
\pgfpathlineto{\pgfqpoint{3.451698in}{10.311132in}}%
\pgfpathlineto{\pgfqpoint{3.363962in}{10.311132in}}%
\pgfpathlineto{\pgfqpoint{3.363962in}{10.398868in}}%
\pgfusepath{stroke,fill}%
\end{pgfscope}%
\begin{pgfscope}%
\pgfpathrectangle{\pgfqpoint{0.380943in}{9.960189in}}{\pgfqpoint{4.650000in}{0.614151in}}%
\pgfusepath{clip}%
\pgfsetbuttcap%
\pgfsetroundjoin%
\definecolor{currentfill}{rgb}{1.000000,1.000000,0.929412}%
\pgfsetfillcolor{currentfill}%
\pgfsetlinewidth{0.250937pt}%
\definecolor{currentstroke}{rgb}{1.000000,1.000000,1.000000}%
\pgfsetstrokecolor{currentstroke}%
\pgfsetdash{}{0pt}%
\pgfpathmoveto{\pgfqpoint{3.451698in}{10.398868in}}%
\pgfpathlineto{\pgfqpoint{3.539434in}{10.398868in}}%
\pgfpathlineto{\pgfqpoint{3.539434in}{10.311132in}}%
\pgfpathlineto{\pgfqpoint{3.451698in}{10.311132in}}%
\pgfpathlineto{\pgfqpoint{3.451698in}{10.398868in}}%
\pgfusepath{stroke,fill}%
\end{pgfscope}%
\begin{pgfscope}%
\pgfpathrectangle{\pgfqpoint{0.380943in}{9.960189in}}{\pgfqpoint{4.650000in}{0.614151in}}%
\pgfusepath{clip}%
\pgfsetbuttcap%
\pgfsetroundjoin%
\definecolor{currentfill}{rgb}{1.000000,1.000000,0.929412}%
\pgfsetfillcolor{currentfill}%
\pgfsetlinewidth{0.250937pt}%
\definecolor{currentstroke}{rgb}{1.000000,1.000000,1.000000}%
\pgfsetstrokecolor{currentstroke}%
\pgfsetdash{}{0pt}%
\pgfpathmoveto{\pgfqpoint{3.539434in}{10.398868in}}%
\pgfpathlineto{\pgfqpoint{3.627169in}{10.398868in}}%
\pgfpathlineto{\pgfqpoint{3.627169in}{10.311132in}}%
\pgfpathlineto{\pgfqpoint{3.539434in}{10.311132in}}%
\pgfpathlineto{\pgfqpoint{3.539434in}{10.398868in}}%
\pgfusepath{stroke,fill}%
\end{pgfscope}%
\begin{pgfscope}%
\pgfpathrectangle{\pgfqpoint{0.380943in}{9.960189in}}{\pgfqpoint{4.650000in}{0.614151in}}%
\pgfusepath{clip}%
\pgfsetbuttcap%
\pgfsetroundjoin%
\definecolor{currentfill}{rgb}{1.000000,1.000000,0.929412}%
\pgfsetfillcolor{currentfill}%
\pgfsetlinewidth{0.250937pt}%
\definecolor{currentstroke}{rgb}{1.000000,1.000000,1.000000}%
\pgfsetstrokecolor{currentstroke}%
\pgfsetdash{}{0pt}%
\pgfpathmoveto{\pgfqpoint{3.627169in}{10.398868in}}%
\pgfpathlineto{\pgfqpoint{3.714905in}{10.398868in}}%
\pgfpathlineto{\pgfqpoint{3.714905in}{10.311132in}}%
\pgfpathlineto{\pgfqpoint{3.627169in}{10.311132in}}%
\pgfpathlineto{\pgfqpoint{3.627169in}{10.398868in}}%
\pgfusepath{stroke,fill}%
\end{pgfscope}%
\begin{pgfscope}%
\pgfpathrectangle{\pgfqpoint{0.380943in}{9.960189in}}{\pgfqpoint{4.650000in}{0.614151in}}%
\pgfusepath{clip}%
\pgfsetbuttcap%
\pgfsetroundjoin%
\definecolor{currentfill}{rgb}{1.000000,1.000000,0.929412}%
\pgfsetfillcolor{currentfill}%
\pgfsetlinewidth{0.250937pt}%
\definecolor{currentstroke}{rgb}{1.000000,1.000000,1.000000}%
\pgfsetstrokecolor{currentstroke}%
\pgfsetdash{}{0pt}%
\pgfpathmoveto{\pgfqpoint{3.714905in}{10.398868in}}%
\pgfpathlineto{\pgfqpoint{3.802641in}{10.398868in}}%
\pgfpathlineto{\pgfqpoint{3.802641in}{10.311132in}}%
\pgfpathlineto{\pgfqpoint{3.714905in}{10.311132in}}%
\pgfpathlineto{\pgfqpoint{3.714905in}{10.398868in}}%
\pgfusepath{stroke,fill}%
\end{pgfscope}%
\begin{pgfscope}%
\pgfpathrectangle{\pgfqpoint{0.380943in}{9.960189in}}{\pgfqpoint{4.650000in}{0.614151in}}%
\pgfusepath{clip}%
\pgfsetbuttcap%
\pgfsetroundjoin%
\definecolor{currentfill}{rgb}{1.000000,1.000000,0.929412}%
\pgfsetfillcolor{currentfill}%
\pgfsetlinewidth{0.250937pt}%
\definecolor{currentstroke}{rgb}{1.000000,1.000000,1.000000}%
\pgfsetstrokecolor{currentstroke}%
\pgfsetdash{}{0pt}%
\pgfpathmoveto{\pgfqpoint{3.802641in}{10.398868in}}%
\pgfpathlineto{\pgfqpoint{3.890377in}{10.398868in}}%
\pgfpathlineto{\pgfqpoint{3.890377in}{10.311132in}}%
\pgfpathlineto{\pgfqpoint{3.802641in}{10.311132in}}%
\pgfpathlineto{\pgfqpoint{3.802641in}{10.398868in}}%
\pgfusepath{stroke,fill}%
\end{pgfscope}%
\begin{pgfscope}%
\pgfpathrectangle{\pgfqpoint{0.380943in}{9.960189in}}{\pgfqpoint{4.650000in}{0.614151in}}%
\pgfusepath{clip}%
\pgfsetbuttcap%
\pgfsetroundjoin%
\definecolor{currentfill}{rgb}{1.000000,1.000000,0.929412}%
\pgfsetfillcolor{currentfill}%
\pgfsetlinewidth{0.250937pt}%
\definecolor{currentstroke}{rgb}{1.000000,1.000000,1.000000}%
\pgfsetstrokecolor{currentstroke}%
\pgfsetdash{}{0pt}%
\pgfpathmoveto{\pgfqpoint{3.890377in}{10.398868in}}%
\pgfpathlineto{\pgfqpoint{3.978113in}{10.398868in}}%
\pgfpathlineto{\pgfqpoint{3.978113in}{10.311132in}}%
\pgfpathlineto{\pgfqpoint{3.890377in}{10.311132in}}%
\pgfpathlineto{\pgfqpoint{3.890377in}{10.398868in}}%
\pgfusepath{stroke,fill}%
\end{pgfscope}%
\begin{pgfscope}%
\pgfpathrectangle{\pgfqpoint{0.380943in}{9.960189in}}{\pgfqpoint{4.650000in}{0.614151in}}%
\pgfusepath{clip}%
\pgfsetbuttcap%
\pgfsetroundjoin%
\definecolor{currentfill}{rgb}{1.000000,1.000000,0.929412}%
\pgfsetfillcolor{currentfill}%
\pgfsetlinewidth{0.250937pt}%
\definecolor{currentstroke}{rgb}{1.000000,1.000000,1.000000}%
\pgfsetstrokecolor{currentstroke}%
\pgfsetdash{}{0pt}%
\pgfpathmoveto{\pgfqpoint{3.978113in}{10.398868in}}%
\pgfpathlineto{\pgfqpoint{4.065849in}{10.398868in}}%
\pgfpathlineto{\pgfqpoint{4.065849in}{10.311132in}}%
\pgfpathlineto{\pgfqpoint{3.978113in}{10.311132in}}%
\pgfpathlineto{\pgfqpoint{3.978113in}{10.398868in}}%
\pgfusepath{stroke,fill}%
\end{pgfscope}%
\begin{pgfscope}%
\pgfpathrectangle{\pgfqpoint{0.380943in}{9.960189in}}{\pgfqpoint{4.650000in}{0.614151in}}%
\pgfusepath{clip}%
\pgfsetbuttcap%
\pgfsetroundjoin%
\definecolor{currentfill}{rgb}{1.000000,1.000000,0.929412}%
\pgfsetfillcolor{currentfill}%
\pgfsetlinewidth{0.250937pt}%
\definecolor{currentstroke}{rgb}{1.000000,1.000000,1.000000}%
\pgfsetstrokecolor{currentstroke}%
\pgfsetdash{}{0pt}%
\pgfpathmoveto{\pgfqpoint{4.065849in}{10.398868in}}%
\pgfpathlineto{\pgfqpoint{4.153585in}{10.398868in}}%
\pgfpathlineto{\pgfqpoint{4.153585in}{10.311132in}}%
\pgfpathlineto{\pgfqpoint{4.065849in}{10.311132in}}%
\pgfpathlineto{\pgfqpoint{4.065849in}{10.398868in}}%
\pgfusepath{stroke,fill}%
\end{pgfscope}%
\begin{pgfscope}%
\pgfpathrectangle{\pgfqpoint{0.380943in}{9.960189in}}{\pgfqpoint{4.650000in}{0.614151in}}%
\pgfusepath{clip}%
\pgfsetbuttcap%
\pgfsetroundjoin%
\definecolor{currentfill}{rgb}{1.000000,1.000000,0.929412}%
\pgfsetfillcolor{currentfill}%
\pgfsetlinewidth{0.250937pt}%
\definecolor{currentstroke}{rgb}{1.000000,1.000000,1.000000}%
\pgfsetstrokecolor{currentstroke}%
\pgfsetdash{}{0pt}%
\pgfpathmoveto{\pgfqpoint{4.153585in}{10.398868in}}%
\pgfpathlineto{\pgfqpoint{4.241320in}{10.398868in}}%
\pgfpathlineto{\pgfqpoint{4.241320in}{10.311132in}}%
\pgfpathlineto{\pgfqpoint{4.153585in}{10.311132in}}%
\pgfpathlineto{\pgfqpoint{4.153585in}{10.398868in}}%
\pgfusepath{stroke,fill}%
\end{pgfscope}%
\begin{pgfscope}%
\pgfpathrectangle{\pgfqpoint{0.380943in}{9.960189in}}{\pgfqpoint{4.650000in}{0.614151in}}%
\pgfusepath{clip}%
\pgfsetbuttcap%
\pgfsetroundjoin%
\definecolor{currentfill}{rgb}{1.000000,1.000000,0.929412}%
\pgfsetfillcolor{currentfill}%
\pgfsetlinewidth{0.250937pt}%
\definecolor{currentstroke}{rgb}{1.000000,1.000000,1.000000}%
\pgfsetstrokecolor{currentstroke}%
\pgfsetdash{}{0pt}%
\pgfpathmoveto{\pgfqpoint{4.241320in}{10.398868in}}%
\pgfpathlineto{\pgfqpoint{4.329056in}{10.398868in}}%
\pgfpathlineto{\pgfqpoint{4.329056in}{10.311132in}}%
\pgfpathlineto{\pgfqpoint{4.241320in}{10.311132in}}%
\pgfpathlineto{\pgfqpoint{4.241320in}{10.398868in}}%
\pgfusepath{stroke,fill}%
\end{pgfscope}%
\begin{pgfscope}%
\pgfpathrectangle{\pgfqpoint{0.380943in}{9.960189in}}{\pgfqpoint{4.650000in}{0.614151in}}%
\pgfusepath{clip}%
\pgfsetbuttcap%
\pgfsetroundjoin%
\definecolor{currentfill}{rgb}{1.000000,1.000000,0.929412}%
\pgfsetfillcolor{currentfill}%
\pgfsetlinewidth{0.250937pt}%
\definecolor{currentstroke}{rgb}{1.000000,1.000000,1.000000}%
\pgfsetstrokecolor{currentstroke}%
\pgfsetdash{}{0pt}%
\pgfpathmoveto{\pgfqpoint{4.329056in}{10.398868in}}%
\pgfpathlineto{\pgfqpoint{4.416792in}{10.398868in}}%
\pgfpathlineto{\pgfqpoint{4.416792in}{10.311132in}}%
\pgfpathlineto{\pgfqpoint{4.329056in}{10.311132in}}%
\pgfpathlineto{\pgfqpoint{4.329056in}{10.398868in}}%
\pgfusepath{stroke,fill}%
\end{pgfscope}%
\begin{pgfscope}%
\pgfpathrectangle{\pgfqpoint{0.380943in}{9.960189in}}{\pgfqpoint{4.650000in}{0.614151in}}%
\pgfusepath{clip}%
\pgfsetbuttcap%
\pgfsetroundjoin%
\definecolor{currentfill}{rgb}{1.000000,1.000000,0.929412}%
\pgfsetfillcolor{currentfill}%
\pgfsetlinewidth{0.250937pt}%
\definecolor{currentstroke}{rgb}{1.000000,1.000000,1.000000}%
\pgfsetstrokecolor{currentstroke}%
\pgfsetdash{}{0pt}%
\pgfpathmoveto{\pgfqpoint{4.416792in}{10.398868in}}%
\pgfpathlineto{\pgfqpoint{4.504528in}{10.398868in}}%
\pgfpathlineto{\pgfqpoint{4.504528in}{10.311132in}}%
\pgfpathlineto{\pgfqpoint{4.416792in}{10.311132in}}%
\pgfpathlineto{\pgfqpoint{4.416792in}{10.398868in}}%
\pgfusepath{stroke,fill}%
\end{pgfscope}%
\begin{pgfscope}%
\pgfpathrectangle{\pgfqpoint{0.380943in}{9.960189in}}{\pgfqpoint{4.650000in}{0.614151in}}%
\pgfusepath{clip}%
\pgfsetbuttcap%
\pgfsetroundjoin%
\definecolor{currentfill}{rgb}{1.000000,0.622145,0.537486}%
\pgfsetfillcolor{currentfill}%
\pgfsetlinewidth{0.250937pt}%
\definecolor{currentstroke}{rgb}{1.000000,1.000000,1.000000}%
\pgfsetstrokecolor{currentstroke}%
\pgfsetdash{}{0pt}%
\pgfpathmoveto{\pgfqpoint{4.504528in}{10.398868in}}%
\pgfpathlineto{\pgfqpoint{4.592264in}{10.398868in}}%
\pgfpathlineto{\pgfqpoint{4.592264in}{10.311132in}}%
\pgfpathlineto{\pgfqpoint{4.504528in}{10.311132in}}%
\pgfpathlineto{\pgfqpoint{4.504528in}{10.398868in}}%
\pgfusepath{stroke,fill}%
\end{pgfscope}%
\begin{pgfscope}%
\pgfpathrectangle{\pgfqpoint{0.380943in}{9.960189in}}{\pgfqpoint{4.650000in}{0.614151in}}%
\pgfusepath{clip}%
\pgfsetbuttcap%
\pgfsetroundjoin%
\definecolor{currentfill}{rgb}{0.989619,0.788235,0.628374}%
\pgfsetfillcolor{currentfill}%
\pgfsetlinewidth{0.250937pt}%
\definecolor{currentstroke}{rgb}{1.000000,1.000000,1.000000}%
\pgfsetstrokecolor{currentstroke}%
\pgfsetdash{}{0pt}%
\pgfpathmoveto{\pgfqpoint{4.592264in}{10.398868in}}%
\pgfpathlineto{\pgfqpoint{4.680000in}{10.398868in}}%
\pgfpathlineto{\pgfqpoint{4.680000in}{10.311132in}}%
\pgfpathlineto{\pgfqpoint{4.592264in}{10.311132in}}%
\pgfpathlineto{\pgfqpoint{4.592264in}{10.398868in}}%
\pgfusepath{stroke,fill}%
\end{pgfscope}%
\begin{pgfscope}%
\pgfpathrectangle{\pgfqpoint{0.380943in}{9.960189in}}{\pgfqpoint{4.650000in}{0.614151in}}%
\pgfusepath{clip}%
\pgfsetbuttcap%
\pgfsetroundjoin%
\definecolor{currentfill}{rgb}{0.982699,0.823991,0.657439}%
\pgfsetfillcolor{currentfill}%
\pgfsetlinewidth{0.250937pt}%
\definecolor{currentstroke}{rgb}{1.000000,1.000000,1.000000}%
\pgfsetstrokecolor{currentstroke}%
\pgfsetdash{}{0pt}%
\pgfpathmoveto{\pgfqpoint{4.680000in}{10.398868in}}%
\pgfpathlineto{\pgfqpoint{4.767736in}{10.398868in}}%
\pgfpathlineto{\pgfqpoint{4.767736in}{10.311132in}}%
\pgfpathlineto{\pgfqpoint{4.680000in}{10.311132in}}%
\pgfpathlineto{\pgfqpoint{4.680000in}{10.398868in}}%
\pgfusepath{stroke,fill}%
\end{pgfscope}%
\begin{pgfscope}%
\pgfpathrectangle{\pgfqpoint{0.380943in}{9.960189in}}{\pgfqpoint{4.650000in}{0.614151in}}%
\pgfusepath{clip}%
\pgfsetbuttcap%
\pgfsetroundjoin%
\definecolor{currentfill}{rgb}{0.997924,0.685352,0.570242}%
\pgfsetfillcolor{currentfill}%
\pgfsetlinewidth{0.250937pt}%
\definecolor{currentstroke}{rgb}{1.000000,1.000000,1.000000}%
\pgfsetstrokecolor{currentstroke}%
\pgfsetdash{}{0pt}%
\pgfpathmoveto{\pgfqpoint{4.767736in}{10.398868in}}%
\pgfpathlineto{\pgfqpoint{4.855471in}{10.398868in}}%
\pgfpathlineto{\pgfqpoint{4.855471in}{10.311132in}}%
\pgfpathlineto{\pgfqpoint{4.767736in}{10.311132in}}%
\pgfpathlineto{\pgfqpoint{4.767736in}{10.398868in}}%
\pgfusepath{stroke,fill}%
\end{pgfscope}%
\begin{pgfscope}%
\pgfpathrectangle{\pgfqpoint{0.380943in}{9.960189in}}{\pgfqpoint{4.650000in}{0.614151in}}%
\pgfusepath{clip}%
\pgfsetbuttcap%
\pgfsetroundjoin%
\definecolor{currentfill}{rgb}{0.989619,0.788235,0.628374}%
\pgfsetfillcolor{currentfill}%
\pgfsetlinewidth{0.250937pt}%
\definecolor{currentstroke}{rgb}{1.000000,1.000000,1.000000}%
\pgfsetstrokecolor{currentstroke}%
\pgfsetdash{}{0pt}%
\pgfpathmoveto{\pgfqpoint{4.855471in}{10.398868in}}%
\pgfpathlineto{\pgfqpoint{4.943207in}{10.398868in}}%
\pgfpathlineto{\pgfqpoint{4.943207in}{10.311132in}}%
\pgfpathlineto{\pgfqpoint{4.855471in}{10.311132in}}%
\pgfpathlineto{\pgfqpoint{4.855471in}{10.398868in}}%
\pgfusepath{stroke,fill}%
\end{pgfscope}%
\begin{pgfscope}%
\pgfpathrectangle{\pgfqpoint{0.380943in}{9.960189in}}{\pgfqpoint{4.650000in}{0.614151in}}%
\pgfusepath{clip}%
\pgfsetbuttcap%
\pgfsetroundjoin%
\definecolor{currentfill}{rgb}{0.963091,0.937255,0.735409}%
\pgfsetfillcolor{currentfill}%
\pgfsetlinewidth{0.250937pt}%
\definecolor{currentstroke}{rgb}{1.000000,1.000000,1.000000}%
\pgfsetstrokecolor{currentstroke}%
\pgfsetdash{}{0pt}%
\pgfpathmoveto{\pgfqpoint{4.943207in}{10.398868in}}%
\pgfpathlineto{\pgfqpoint{5.030943in}{10.398868in}}%
\pgfpathlineto{\pgfqpoint{5.030943in}{10.311132in}}%
\pgfpathlineto{\pgfqpoint{4.943207in}{10.311132in}}%
\pgfpathlineto{\pgfqpoint{4.943207in}{10.398868in}}%
\pgfusepath{stroke,fill}%
\end{pgfscope}%
\begin{pgfscope}%
\pgfpathrectangle{\pgfqpoint{0.380943in}{9.960189in}}{\pgfqpoint{4.650000in}{0.614151in}}%
\pgfusepath{clip}%
\pgfsetbuttcap%
\pgfsetroundjoin%
\pgfsetlinewidth{0.250937pt}%
\definecolor{currentstroke}{rgb}{1.000000,1.000000,1.000000}%
\pgfsetstrokecolor{currentstroke}%
\pgfsetdash{}{0pt}%
\pgfpathmoveto{\pgfqpoint{0.380943in}{10.311132in}}%
\pgfpathlineto{\pgfqpoint{0.468679in}{10.311132in}}%
\pgfpathlineto{\pgfqpoint{0.468679in}{10.223396in}}%
\pgfpathlineto{\pgfqpoint{0.380943in}{10.223396in}}%
\pgfpathlineto{\pgfqpoint{0.380943in}{10.311132in}}%
\pgfusepath{stroke}%
\end{pgfscope}%
\begin{pgfscope}%
\pgfpathrectangle{\pgfqpoint{0.380943in}{9.960189in}}{\pgfqpoint{4.650000in}{0.614151in}}%
\pgfusepath{clip}%
\pgfsetbuttcap%
\pgfsetroundjoin%
\definecolor{currentfill}{rgb}{1.000000,1.000000,0.929412}%
\pgfsetfillcolor{currentfill}%
\pgfsetlinewidth{0.250937pt}%
\definecolor{currentstroke}{rgb}{1.000000,1.000000,1.000000}%
\pgfsetstrokecolor{currentstroke}%
\pgfsetdash{}{0pt}%
\pgfpathmoveto{\pgfqpoint{0.468679in}{10.311132in}}%
\pgfpathlineto{\pgfqpoint{0.556415in}{10.311132in}}%
\pgfpathlineto{\pgfqpoint{0.556415in}{10.223396in}}%
\pgfpathlineto{\pgfqpoint{0.468679in}{10.223396in}}%
\pgfpathlineto{\pgfqpoint{0.468679in}{10.311132in}}%
\pgfusepath{stroke,fill}%
\end{pgfscope}%
\begin{pgfscope}%
\pgfpathrectangle{\pgfqpoint{0.380943in}{9.960189in}}{\pgfqpoint{4.650000in}{0.614151in}}%
\pgfusepath{clip}%
\pgfsetbuttcap%
\pgfsetroundjoin%
\definecolor{currentfill}{rgb}{1.000000,1.000000,0.929412}%
\pgfsetfillcolor{currentfill}%
\pgfsetlinewidth{0.250937pt}%
\definecolor{currentstroke}{rgb}{1.000000,1.000000,1.000000}%
\pgfsetstrokecolor{currentstroke}%
\pgfsetdash{}{0pt}%
\pgfpathmoveto{\pgfqpoint{0.556415in}{10.311132in}}%
\pgfpathlineto{\pgfqpoint{0.644151in}{10.311132in}}%
\pgfpathlineto{\pgfqpoint{0.644151in}{10.223396in}}%
\pgfpathlineto{\pgfqpoint{0.556415in}{10.223396in}}%
\pgfpathlineto{\pgfqpoint{0.556415in}{10.311132in}}%
\pgfusepath{stroke,fill}%
\end{pgfscope}%
\begin{pgfscope}%
\pgfpathrectangle{\pgfqpoint{0.380943in}{9.960189in}}{\pgfqpoint{4.650000in}{0.614151in}}%
\pgfusepath{clip}%
\pgfsetbuttcap%
\pgfsetroundjoin%
\definecolor{currentfill}{rgb}{1.000000,1.000000,0.929412}%
\pgfsetfillcolor{currentfill}%
\pgfsetlinewidth{0.250937pt}%
\definecolor{currentstroke}{rgb}{1.000000,1.000000,1.000000}%
\pgfsetstrokecolor{currentstroke}%
\pgfsetdash{}{0pt}%
\pgfpathmoveto{\pgfqpoint{0.644151in}{10.311132in}}%
\pgfpathlineto{\pgfqpoint{0.731886in}{10.311132in}}%
\pgfpathlineto{\pgfqpoint{0.731886in}{10.223396in}}%
\pgfpathlineto{\pgfqpoint{0.644151in}{10.223396in}}%
\pgfpathlineto{\pgfqpoint{0.644151in}{10.311132in}}%
\pgfusepath{stroke,fill}%
\end{pgfscope}%
\begin{pgfscope}%
\pgfpathrectangle{\pgfqpoint{0.380943in}{9.960189in}}{\pgfqpoint{4.650000in}{0.614151in}}%
\pgfusepath{clip}%
\pgfsetbuttcap%
\pgfsetroundjoin%
\definecolor{currentfill}{rgb}{1.000000,1.000000,0.929412}%
\pgfsetfillcolor{currentfill}%
\pgfsetlinewidth{0.250937pt}%
\definecolor{currentstroke}{rgb}{1.000000,1.000000,1.000000}%
\pgfsetstrokecolor{currentstroke}%
\pgfsetdash{}{0pt}%
\pgfpathmoveto{\pgfqpoint{0.731886in}{10.311132in}}%
\pgfpathlineto{\pgfqpoint{0.819622in}{10.311132in}}%
\pgfpathlineto{\pgfqpoint{0.819622in}{10.223396in}}%
\pgfpathlineto{\pgfqpoint{0.731886in}{10.223396in}}%
\pgfpathlineto{\pgfqpoint{0.731886in}{10.311132in}}%
\pgfusepath{stroke,fill}%
\end{pgfscope}%
\begin{pgfscope}%
\pgfpathrectangle{\pgfqpoint{0.380943in}{9.960189in}}{\pgfqpoint{4.650000in}{0.614151in}}%
\pgfusepath{clip}%
\pgfsetbuttcap%
\pgfsetroundjoin%
\definecolor{currentfill}{rgb}{1.000000,1.000000,0.929412}%
\pgfsetfillcolor{currentfill}%
\pgfsetlinewidth{0.250937pt}%
\definecolor{currentstroke}{rgb}{1.000000,1.000000,1.000000}%
\pgfsetstrokecolor{currentstroke}%
\pgfsetdash{}{0pt}%
\pgfpathmoveto{\pgfqpoint{0.819622in}{10.311132in}}%
\pgfpathlineto{\pgfqpoint{0.907358in}{10.311132in}}%
\pgfpathlineto{\pgfqpoint{0.907358in}{10.223396in}}%
\pgfpathlineto{\pgfqpoint{0.819622in}{10.223396in}}%
\pgfpathlineto{\pgfqpoint{0.819622in}{10.311132in}}%
\pgfusepath{stroke,fill}%
\end{pgfscope}%
\begin{pgfscope}%
\pgfpathrectangle{\pgfqpoint{0.380943in}{9.960189in}}{\pgfqpoint{4.650000in}{0.614151in}}%
\pgfusepath{clip}%
\pgfsetbuttcap%
\pgfsetroundjoin%
\definecolor{currentfill}{rgb}{1.000000,1.000000,0.929412}%
\pgfsetfillcolor{currentfill}%
\pgfsetlinewidth{0.250937pt}%
\definecolor{currentstroke}{rgb}{1.000000,1.000000,1.000000}%
\pgfsetstrokecolor{currentstroke}%
\pgfsetdash{}{0pt}%
\pgfpathmoveto{\pgfqpoint{0.907358in}{10.311132in}}%
\pgfpathlineto{\pgfqpoint{0.995094in}{10.311132in}}%
\pgfpathlineto{\pgfqpoint{0.995094in}{10.223396in}}%
\pgfpathlineto{\pgfqpoint{0.907358in}{10.223396in}}%
\pgfpathlineto{\pgfqpoint{0.907358in}{10.311132in}}%
\pgfusepath{stroke,fill}%
\end{pgfscope}%
\begin{pgfscope}%
\pgfpathrectangle{\pgfqpoint{0.380943in}{9.960189in}}{\pgfqpoint{4.650000in}{0.614151in}}%
\pgfusepath{clip}%
\pgfsetbuttcap%
\pgfsetroundjoin%
\definecolor{currentfill}{rgb}{1.000000,1.000000,0.929412}%
\pgfsetfillcolor{currentfill}%
\pgfsetlinewidth{0.250937pt}%
\definecolor{currentstroke}{rgb}{1.000000,1.000000,1.000000}%
\pgfsetstrokecolor{currentstroke}%
\pgfsetdash{}{0pt}%
\pgfpathmoveto{\pgfqpoint{0.995094in}{10.311132in}}%
\pgfpathlineto{\pgfqpoint{1.082830in}{10.311132in}}%
\pgfpathlineto{\pgfqpoint{1.082830in}{10.223396in}}%
\pgfpathlineto{\pgfqpoint{0.995094in}{10.223396in}}%
\pgfpathlineto{\pgfqpoint{0.995094in}{10.311132in}}%
\pgfusepath{stroke,fill}%
\end{pgfscope}%
\begin{pgfscope}%
\pgfpathrectangle{\pgfqpoint{0.380943in}{9.960189in}}{\pgfqpoint{4.650000in}{0.614151in}}%
\pgfusepath{clip}%
\pgfsetbuttcap%
\pgfsetroundjoin%
\definecolor{currentfill}{rgb}{1.000000,1.000000,0.929412}%
\pgfsetfillcolor{currentfill}%
\pgfsetlinewidth{0.250937pt}%
\definecolor{currentstroke}{rgb}{1.000000,1.000000,1.000000}%
\pgfsetstrokecolor{currentstroke}%
\pgfsetdash{}{0pt}%
\pgfpathmoveto{\pgfqpoint{1.082830in}{10.311132in}}%
\pgfpathlineto{\pgfqpoint{1.170566in}{10.311132in}}%
\pgfpathlineto{\pgfqpoint{1.170566in}{10.223396in}}%
\pgfpathlineto{\pgfqpoint{1.082830in}{10.223396in}}%
\pgfpathlineto{\pgfqpoint{1.082830in}{10.311132in}}%
\pgfusepath{stroke,fill}%
\end{pgfscope}%
\begin{pgfscope}%
\pgfpathrectangle{\pgfqpoint{0.380943in}{9.960189in}}{\pgfqpoint{4.650000in}{0.614151in}}%
\pgfusepath{clip}%
\pgfsetbuttcap%
\pgfsetroundjoin%
\definecolor{currentfill}{rgb}{1.000000,1.000000,0.929412}%
\pgfsetfillcolor{currentfill}%
\pgfsetlinewidth{0.250937pt}%
\definecolor{currentstroke}{rgb}{1.000000,1.000000,1.000000}%
\pgfsetstrokecolor{currentstroke}%
\pgfsetdash{}{0pt}%
\pgfpathmoveto{\pgfqpoint{1.170566in}{10.311132in}}%
\pgfpathlineto{\pgfqpoint{1.258302in}{10.311132in}}%
\pgfpathlineto{\pgfqpoint{1.258302in}{10.223396in}}%
\pgfpathlineto{\pgfqpoint{1.170566in}{10.223396in}}%
\pgfpathlineto{\pgfqpoint{1.170566in}{10.311132in}}%
\pgfusepath{stroke,fill}%
\end{pgfscope}%
\begin{pgfscope}%
\pgfpathrectangle{\pgfqpoint{0.380943in}{9.960189in}}{\pgfqpoint{4.650000in}{0.614151in}}%
\pgfusepath{clip}%
\pgfsetbuttcap%
\pgfsetroundjoin%
\definecolor{currentfill}{rgb}{1.000000,1.000000,0.929412}%
\pgfsetfillcolor{currentfill}%
\pgfsetlinewidth{0.250937pt}%
\definecolor{currentstroke}{rgb}{1.000000,1.000000,1.000000}%
\pgfsetstrokecolor{currentstroke}%
\pgfsetdash{}{0pt}%
\pgfpathmoveto{\pgfqpoint{1.258302in}{10.311132in}}%
\pgfpathlineto{\pgfqpoint{1.346037in}{10.311132in}}%
\pgfpathlineto{\pgfqpoint{1.346037in}{10.223396in}}%
\pgfpathlineto{\pgfqpoint{1.258302in}{10.223396in}}%
\pgfpathlineto{\pgfqpoint{1.258302in}{10.311132in}}%
\pgfusepath{stroke,fill}%
\end{pgfscope}%
\begin{pgfscope}%
\pgfpathrectangle{\pgfqpoint{0.380943in}{9.960189in}}{\pgfqpoint{4.650000in}{0.614151in}}%
\pgfusepath{clip}%
\pgfsetbuttcap%
\pgfsetroundjoin%
\definecolor{currentfill}{rgb}{1.000000,1.000000,0.929412}%
\pgfsetfillcolor{currentfill}%
\pgfsetlinewidth{0.250937pt}%
\definecolor{currentstroke}{rgb}{1.000000,1.000000,1.000000}%
\pgfsetstrokecolor{currentstroke}%
\pgfsetdash{}{0pt}%
\pgfpathmoveto{\pgfqpoint{1.346037in}{10.311132in}}%
\pgfpathlineto{\pgfqpoint{1.433773in}{10.311132in}}%
\pgfpathlineto{\pgfqpoint{1.433773in}{10.223396in}}%
\pgfpathlineto{\pgfqpoint{1.346037in}{10.223396in}}%
\pgfpathlineto{\pgfqpoint{1.346037in}{10.311132in}}%
\pgfusepath{stroke,fill}%
\end{pgfscope}%
\begin{pgfscope}%
\pgfpathrectangle{\pgfqpoint{0.380943in}{9.960189in}}{\pgfqpoint{4.650000in}{0.614151in}}%
\pgfusepath{clip}%
\pgfsetbuttcap%
\pgfsetroundjoin%
\definecolor{currentfill}{rgb}{1.000000,1.000000,0.929412}%
\pgfsetfillcolor{currentfill}%
\pgfsetlinewidth{0.250937pt}%
\definecolor{currentstroke}{rgb}{1.000000,1.000000,1.000000}%
\pgfsetstrokecolor{currentstroke}%
\pgfsetdash{}{0pt}%
\pgfpathmoveto{\pgfqpoint{1.433773in}{10.311132in}}%
\pgfpathlineto{\pgfqpoint{1.521509in}{10.311132in}}%
\pgfpathlineto{\pgfqpoint{1.521509in}{10.223396in}}%
\pgfpathlineto{\pgfqpoint{1.433773in}{10.223396in}}%
\pgfpathlineto{\pgfqpoint{1.433773in}{10.311132in}}%
\pgfusepath{stroke,fill}%
\end{pgfscope}%
\begin{pgfscope}%
\pgfpathrectangle{\pgfqpoint{0.380943in}{9.960189in}}{\pgfqpoint{4.650000in}{0.614151in}}%
\pgfusepath{clip}%
\pgfsetbuttcap%
\pgfsetroundjoin%
\definecolor{currentfill}{rgb}{1.000000,1.000000,0.929412}%
\pgfsetfillcolor{currentfill}%
\pgfsetlinewidth{0.250937pt}%
\definecolor{currentstroke}{rgb}{1.000000,1.000000,1.000000}%
\pgfsetstrokecolor{currentstroke}%
\pgfsetdash{}{0pt}%
\pgfpathmoveto{\pgfqpoint{1.521509in}{10.311132in}}%
\pgfpathlineto{\pgfqpoint{1.609245in}{10.311132in}}%
\pgfpathlineto{\pgfqpoint{1.609245in}{10.223396in}}%
\pgfpathlineto{\pgfqpoint{1.521509in}{10.223396in}}%
\pgfpathlineto{\pgfqpoint{1.521509in}{10.311132in}}%
\pgfusepath{stroke,fill}%
\end{pgfscope}%
\begin{pgfscope}%
\pgfpathrectangle{\pgfqpoint{0.380943in}{9.960189in}}{\pgfqpoint{4.650000in}{0.614151in}}%
\pgfusepath{clip}%
\pgfsetbuttcap%
\pgfsetroundjoin%
\definecolor{currentfill}{rgb}{1.000000,1.000000,0.929412}%
\pgfsetfillcolor{currentfill}%
\pgfsetlinewidth{0.250937pt}%
\definecolor{currentstroke}{rgb}{1.000000,1.000000,1.000000}%
\pgfsetstrokecolor{currentstroke}%
\pgfsetdash{}{0pt}%
\pgfpathmoveto{\pgfqpoint{1.609245in}{10.311132in}}%
\pgfpathlineto{\pgfqpoint{1.696981in}{10.311132in}}%
\pgfpathlineto{\pgfqpoint{1.696981in}{10.223396in}}%
\pgfpathlineto{\pgfqpoint{1.609245in}{10.223396in}}%
\pgfpathlineto{\pgfqpoint{1.609245in}{10.311132in}}%
\pgfusepath{stroke,fill}%
\end{pgfscope}%
\begin{pgfscope}%
\pgfpathrectangle{\pgfqpoint{0.380943in}{9.960189in}}{\pgfqpoint{4.650000in}{0.614151in}}%
\pgfusepath{clip}%
\pgfsetbuttcap%
\pgfsetroundjoin%
\definecolor{currentfill}{rgb}{1.000000,1.000000,0.929412}%
\pgfsetfillcolor{currentfill}%
\pgfsetlinewidth{0.250937pt}%
\definecolor{currentstroke}{rgb}{1.000000,1.000000,1.000000}%
\pgfsetstrokecolor{currentstroke}%
\pgfsetdash{}{0pt}%
\pgfpathmoveto{\pgfqpoint{1.696981in}{10.311132in}}%
\pgfpathlineto{\pgfqpoint{1.784717in}{10.311132in}}%
\pgfpathlineto{\pgfqpoint{1.784717in}{10.223396in}}%
\pgfpathlineto{\pgfqpoint{1.696981in}{10.223396in}}%
\pgfpathlineto{\pgfqpoint{1.696981in}{10.311132in}}%
\pgfusepath{stroke,fill}%
\end{pgfscope}%
\begin{pgfscope}%
\pgfpathrectangle{\pgfqpoint{0.380943in}{9.960189in}}{\pgfqpoint{4.650000in}{0.614151in}}%
\pgfusepath{clip}%
\pgfsetbuttcap%
\pgfsetroundjoin%
\definecolor{currentfill}{rgb}{1.000000,1.000000,0.929412}%
\pgfsetfillcolor{currentfill}%
\pgfsetlinewidth{0.250937pt}%
\definecolor{currentstroke}{rgb}{1.000000,1.000000,1.000000}%
\pgfsetstrokecolor{currentstroke}%
\pgfsetdash{}{0pt}%
\pgfpathmoveto{\pgfqpoint{1.784717in}{10.311132in}}%
\pgfpathlineto{\pgfqpoint{1.872452in}{10.311132in}}%
\pgfpathlineto{\pgfqpoint{1.872452in}{10.223396in}}%
\pgfpathlineto{\pgfqpoint{1.784717in}{10.223396in}}%
\pgfpathlineto{\pgfqpoint{1.784717in}{10.311132in}}%
\pgfusepath{stroke,fill}%
\end{pgfscope}%
\begin{pgfscope}%
\pgfpathrectangle{\pgfqpoint{0.380943in}{9.960189in}}{\pgfqpoint{4.650000in}{0.614151in}}%
\pgfusepath{clip}%
\pgfsetbuttcap%
\pgfsetroundjoin%
\definecolor{currentfill}{rgb}{1.000000,1.000000,0.929412}%
\pgfsetfillcolor{currentfill}%
\pgfsetlinewidth{0.250937pt}%
\definecolor{currentstroke}{rgb}{1.000000,1.000000,1.000000}%
\pgfsetstrokecolor{currentstroke}%
\pgfsetdash{}{0pt}%
\pgfpathmoveto{\pgfqpoint{1.872452in}{10.311132in}}%
\pgfpathlineto{\pgfqpoint{1.960188in}{10.311132in}}%
\pgfpathlineto{\pgfqpoint{1.960188in}{10.223396in}}%
\pgfpathlineto{\pgfqpoint{1.872452in}{10.223396in}}%
\pgfpathlineto{\pgfqpoint{1.872452in}{10.311132in}}%
\pgfusepath{stroke,fill}%
\end{pgfscope}%
\begin{pgfscope}%
\pgfpathrectangle{\pgfqpoint{0.380943in}{9.960189in}}{\pgfqpoint{4.650000in}{0.614151in}}%
\pgfusepath{clip}%
\pgfsetbuttcap%
\pgfsetroundjoin%
\definecolor{currentfill}{rgb}{1.000000,1.000000,0.929412}%
\pgfsetfillcolor{currentfill}%
\pgfsetlinewidth{0.250937pt}%
\definecolor{currentstroke}{rgb}{1.000000,1.000000,1.000000}%
\pgfsetstrokecolor{currentstroke}%
\pgfsetdash{}{0pt}%
\pgfpathmoveto{\pgfqpoint{1.960188in}{10.311132in}}%
\pgfpathlineto{\pgfqpoint{2.047924in}{10.311132in}}%
\pgfpathlineto{\pgfqpoint{2.047924in}{10.223396in}}%
\pgfpathlineto{\pgfqpoint{1.960188in}{10.223396in}}%
\pgfpathlineto{\pgfqpoint{1.960188in}{10.311132in}}%
\pgfusepath{stroke,fill}%
\end{pgfscope}%
\begin{pgfscope}%
\pgfpathrectangle{\pgfqpoint{0.380943in}{9.960189in}}{\pgfqpoint{4.650000in}{0.614151in}}%
\pgfusepath{clip}%
\pgfsetbuttcap%
\pgfsetroundjoin%
\definecolor{currentfill}{rgb}{1.000000,1.000000,0.929412}%
\pgfsetfillcolor{currentfill}%
\pgfsetlinewidth{0.250937pt}%
\definecolor{currentstroke}{rgb}{1.000000,1.000000,1.000000}%
\pgfsetstrokecolor{currentstroke}%
\pgfsetdash{}{0pt}%
\pgfpathmoveto{\pgfqpoint{2.047924in}{10.311132in}}%
\pgfpathlineto{\pgfqpoint{2.135660in}{10.311132in}}%
\pgfpathlineto{\pgfqpoint{2.135660in}{10.223396in}}%
\pgfpathlineto{\pgfqpoint{2.047924in}{10.223396in}}%
\pgfpathlineto{\pgfqpoint{2.047924in}{10.311132in}}%
\pgfusepath{stroke,fill}%
\end{pgfscope}%
\begin{pgfscope}%
\pgfpathrectangle{\pgfqpoint{0.380943in}{9.960189in}}{\pgfqpoint{4.650000in}{0.614151in}}%
\pgfusepath{clip}%
\pgfsetbuttcap%
\pgfsetroundjoin%
\definecolor{currentfill}{rgb}{1.000000,1.000000,0.929412}%
\pgfsetfillcolor{currentfill}%
\pgfsetlinewidth{0.250937pt}%
\definecolor{currentstroke}{rgb}{1.000000,1.000000,1.000000}%
\pgfsetstrokecolor{currentstroke}%
\pgfsetdash{}{0pt}%
\pgfpathmoveto{\pgfqpoint{2.135660in}{10.311132in}}%
\pgfpathlineto{\pgfqpoint{2.223396in}{10.311132in}}%
\pgfpathlineto{\pgfqpoint{2.223396in}{10.223396in}}%
\pgfpathlineto{\pgfqpoint{2.135660in}{10.223396in}}%
\pgfpathlineto{\pgfqpoint{2.135660in}{10.311132in}}%
\pgfusepath{stroke,fill}%
\end{pgfscope}%
\begin{pgfscope}%
\pgfpathrectangle{\pgfqpoint{0.380943in}{9.960189in}}{\pgfqpoint{4.650000in}{0.614151in}}%
\pgfusepath{clip}%
\pgfsetbuttcap%
\pgfsetroundjoin%
\definecolor{currentfill}{rgb}{1.000000,1.000000,0.929412}%
\pgfsetfillcolor{currentfill}%
\pgfsetlinewidth{0.250937pt}%
\definecolor{currentstroke}{rgb}{1.000000,1.000000,1.000000}%
\pgfsetstrokecolor{currentstroke}%
\pgfsetdash{}{0pt}%
\pgfpathmoveto{\pgfqpoint{2.223396in}{10.311132in}}%
\pgfpathlineto{\pgfqpoint{2.311132in}{10.311132in}}%
\pgfpathlineto{\pgfqpoint{2.311132in}{10.223396in}}%
\pgfpathlineto{\pgfqpoint{2.223396in}{10.223396in}}%
\pgfpathlineto{\pgfqpoint{2.223396in}{10.311132in}}%
\pgfusepath{stroke,fill}%
\end{pgfscope}%
\begin{pgfscope}%
\pgfpathrectangle{\pgfqpoint{0.380943in}{9.960189in}}{\pgfqpoint{4.650000in}{0.614151in}}%
\pgfusepath{clip}%
\pgfsetbuttcap%
\pgfsetroundjoin%
\definecolor{currentfill}{rgb}{1.000000,1.000000,0.929412}%
\pgfsetfillcolor{currentfill}%
\pgfsetlinewidth{0.250937pt}%
\definecolor{currentstroke}{rgb}{1.000000,1.000000,1.000000}%
\pgfsetstrokecolor{currentstroke}%
\pgfsetdash{}{0pt}%
\pgfpathmoveto{\pgfqpoint{2.311132in}{10.311132in}}%
\pgfpathlineto{\pgfqpoint{2.398868in}{10.311132in}}%
\pgfpathlineto{\pgfqpoint{2.398868in}{10.223396in}}%
\pgfpathlineto{\pgfqpoint{2.311132in}{10.223396in}}%
\pgfpathlineto{\pgfqpoint{2.311132in}{10.311132in}}%
\pgfusepath{stroke,fill}%
\end{pgfscope}%
\begin{pgfscope}%
\pgfpathrectangle{\pgfqpoint{0.380943in}{9.960189in}}{\pgfqpoint{4.650000in}{0.614151in}}%
\pgfusepath{clip}%
\pgfsetbuttcap%
\pgfsetroundjoin%
\definecolor{currentfill}{rgb}{1.000000,1.000000,0.929412}%
\pgfsetfillcolor{currentfill}%
\pgfsetlinewidth{0.250937pt}%
\definecolor{currentstroke}{rgb}{1.000000,1.000000,1.000000}%
\pgfsetstrokecolor{currentstroke}%
\pgfsetdash{}{0pt}%
\pgfpathmoveto{\pgfqpoint{2.398868in}{10.311132in}}%
\pgfpathlineto{\pgfqpoint{2.486603in}{10.311132in}}%
\pgfpathlineto{\pgfqpoint{2.486603in}{10.223396in}}%
\pgfpathlineto{\pgfqpoint{2.398868in}{10.223396in}}%
\pgfpathlineto{\pgfqpoint{2.398868in}{10.311132in}}%
\pgfusepath{stroke,fill}%
\end{pgfscope}%
\begin{pgfscope}%
\pgfpathrectangle{\pgfqpoint{0.380943in}{9.960189in}}{\pgfqpoint{4.650000in}{0.614151in}}%
\pgfusepath{clip}%
\pgfsetbuttcap%
\pgfsetroundjoin%
\definecolor{currentfill}{rgb}{1.000000,1.000000,0.929412}%
\pgfsetfillcolor{currentfill}%
\pgfsetlinewidth{0.250937pt}%
\definecolor{currentstroke}{rgb}{1.000000,1.000000,1.000000}%
\pgfsetstrokecolor{currentstroke}%
\pgfsetdash{}{0pt}%
\pgfpathmoveto{\pgfqpoint{2.486603in}{10.311132in}}%
\pgfpathlineto{\pgfqpoint{2.574339in}{10.311132in}}%
\pgfpathlineto{\pgfqpoint{2.574339in}{10.223396in}}%
\pgfpathlineto{\pgfqpoint{2.486603in}{10.223396in}}%
\pgfpathlineto{\pgfqpoint{2.486603in}{10.311132in}}%
\pgfusepath{stroke,fill}%
\end{pgfscope}%
\begin{pgfscope}%
\pgfpathrectangle{\pgfqpoint{0.380943in}{9.960189in}}{\pgfqpoint{4.650000in}{0.614151in}}%
\pgfusepath{clip}%
\pgfsetbuttcap%
\pgfsetroundjoin%
\definecolor{currentfill}{rgb}{1.000000,1.000000,0.929412}%
\pgfsetfillcolor{currentfill}%
\pgfsetlinewidth{0.250937pt}%
\definecolor{currentstroke}{rgb}{1.000000,1.000000,1.000000}%
\pgfsetstrokecolor{currentstroke}%
\pgfsetdash{}{0pt}%
\pgfpathmoveto{\pgfqpoint{2.574339in}{10.311132in}}%
\pgfpathlineto{\pgfqpoint{2.662075in}{10.311132in}}%
\pgfpathlineto{\pgfqpoint{2.662075in}{10.223396in}}%
\pgfpathlineto{\pgfqpoint{2.574339in}{10.223396in}}%
\pgfpathlineto{\pgfqpoint{2.574339in}{10.311132in}}%
\pgfusepath{stroke,fill}%
\end{pgfscope}%
\begin{pgfscope}%
\pgfpathrectangle{\pgfqpoint{0.380943in}{9.960189in}}{\pgfqpoint{4.650000in}{0.614151in}}%
\pgfusepath{clip}%
\pgfsetbuttcap%
\pgfsetroundjoin%
\definecolor{currentfill}{rgb}{1.000000,1.000000,0.929412}%
\pgfsetfillcolor{currentfill}%
\pgfsetlinewidth{0.250937pt}%
\definecolor{currentstroke}{rgb}{1.000000,1.000000,1.000000}%
\pgfsetstrokecolor{currentstroke}%
\pgfsetdash{}{0pt}%
\pgfpathmoveto{\pgfqpoint{2.662075in}{10.311132in}}%
\pgfpathlineto{\pgfqpoint{2.749811in}{10.311132in}}%
\pgfpathlineto{\pgfqpoint{2.749811in}{10.223396in}}%
\pgfpathlineto{\pgfqpoint{2.662075in}{10.223396in}}%
\pgfpathlineto{\pgfqpoint{2.662075in}{10.311132in}}%
\pgfusepath{stroke,fill}%
\end{pgfscope}%
\begin{pgfscope}%
\pgfpathrectangle{\pgfqpoint{0.380943in}{9.960189in}}{\pgfqpoint{4.650000in}{0.614151in}}%
\pgfusepath{clip}%
\pgfsetbuttcap%
\pgfsetroundjoin%
\definecolor{currentfill}{rgb}{1.000000,1.000000,0.929412}%
\pgfsetfillcolor{currentfill}%
\pgfsetlinewidth{0.250937pt}%
\definecolor{currentstroke}{rgb}{1.000000,1.000000,1.000000}%
\pgfsetstrokecolor{currentstroke}%
\pgfsetdash{}{0pt}%
\pgfpathmoveto{\pgfqpoint{2.749811in}{10.311132in}}%
\pgfpathlineto{\pgfqpoint{2.837547in}{10.311132in}}%
\pgfpathlineto{\pgfqpoint{2.837547in}{10.223396in}}%
\pgfpathlineto{\pgfqpoint{2.749811in}{10.223396in}}%
\pgfpathlineto{\pgfqpoint{2.749811in}{10.311132in}}%
\pgfusepath{stroke,fill}%
\end{pgfscope}%
\begin{pgfscope}%
\pgfpathrectangle{\pgfqpoint{0.380943in}{9.960189in}}{\pgfqpoint{4.650000in}{0.614151in}}%
\pgfusepath{clip}%
\pgfsetbuttcap%
\pgfsetroundjoin%
\definecolor{currentfill}{rgb}{1.000000,1.000000,0.929412}%
\pgfsetfillcolor{currentfill}%
\pgfsetlinewidth{0.250937pt}%
\definecolor{currentstroke}{rgb}{1.000000,1.000000,1.000000}%
\pgfsetstrokecolor{currentstroke}%
\pgfsetdash{}{0pt}%
\pgfpathmoveto{\pgfqpoint{2.837547in}{10.311132in}}%
\pgfpathlineto{\pgfqpoint{2.925283in}{10.311132in}}%
\pgfpathlineto{\pgfqpoint{2.925283in}{10.223396in}}%
\pgfpathlineto{\pgfqpoint{2.837547in}{10.223396in}}%
\pgfpathlineto{\pgfqpoint{2.837547in}{10.311132in}}%
\pgfusepath{stroke,fill}%
\end{pgfscope}%
\begin{pgfscope}%
\pgfpathrectangle{\pgfqpoint{0.380943in}{9.960189in}}{\pgfqpoint{4.650000in}{0.614151in}}%
\pgfusepath{clip}%
\pgfsetbuttcap%
\pgfsetroundjoin%
\definecolor{currentfill}{rgb}{1.000000,1.000000,0.929412}%
\pgfsetfillcolor{currentfill}%
\pgfsetlinewidth{0.250937pt}%
\definecolor{currentstroke}{rgb}{1.000000,1.000000,1.000000}%
\pgfsetstrokecolor{currentstroke}%
\pgfsetdash{}{0pt}%
\pgfpathmoveto{\pgfqpoint{2.925283in}{10.311132in}}%
\pgfpathlineto{\pgfqpoint{3.013019in}{10.311132in}}%
\pgfpathlineto{\pgfqpoint{3.013019in}{10.223396in}}%
\pgfpathlineto{\pgfqpoint{2.925283in}{10.223396in}}%
\pgfpathlineto{\pgfqpoint{2.925283in}{10.311132in}}%
\pgfusepath{stroke,fill}%
\end{pgfscope}%
\begin{pgfscope}%
\pgfpathrectangle{\pgfqpoint{0.380943in}{9.960189in}}{\pgfqpoint{4.650000in}{0.614151in}}%
\pgfusepath{clip}%
\pgfsetbuttcap%
\pgfsetroundjoin%
\definecolor{currentfill}{rgb}{1.000000,1.000000,0.929412}%
\pgfsetfillcolor{currentfill}%
\pgfsetlinewidth{0.250937pt}%
\definecolor{currentstroke}{rgb}{1.000000,1.000000,1.000000}%
\pgfsetstrokecolor{currentstroke}%
\pgfsetdash{}{0pt}%
\pgfpathmoveto{\pgfqpoint{3.013019in}{10.311132in}}%
\pgfpathlineto{\pgfqpoint{3.100754in}{10.311132in}}%
\pgfpathlineto{\pgfqpoint{3.100754in}{10.223396in}}%
\pgfpathlineto{\pgfqpoint{3.013019in}{10.223396in}}%
\pgfpathlineto{\pgfqpoint{3.013019in}{10.311132in}}%
\pgfusepath{stroke,fill}%
\end{pgfscope}%
\begin{pgfscope}%
\pgfpathrectangle{\pgfqpoint{0.380943in}{9.960189in}}{\pgfqpoint{4.650000in}{0.614151in}}%
\pgfusepath{clip}%
\pgfsetbuttcap%
\pgfsetroundjoin%
\definecolor{currentfill}{rgb}{1.000000,1.000000,0.929412}%
\pgfsetfillcolor{currentfill}%
\pgfsetlinewidth{0.250937pt}%
\definecolor{currentstroke}{rgb}{1.000000,1.000000,1.000000}%
\pgfsetstrokecolor{currentstroke}%
\pgfsetdash{}{0pt}%
\pgfpathmoveto{\pgfqpoint{3.100754in}{10.311132in}}%
\pgfpathlineto{\pgfqpoint{3.188490in}{10.311132in}}%
\pgfpathlineto{\pgfqpoint{3.188490in}{10.223396in}}%
\pgfpathlineto{\pgfqpoint{3.100754in}{10.223396in}}%
\pgfpathlineto{\pgfqpoint{3.100754in}{10.311132in}}%
\pgfusepath{stroke,fill}%
\end{pgfscope}%
\begin{pgfscope}%
\pgfpathrectangle{\pgfqpoint{0.380943in}{9.960189in}}{\pgfqpoint{4.650000in}{0.614151in}}%
\pgfusepath{clip}%
\pgfsetbuttcap%
\pgfsetroundjoin%
\definecolor{currentfill}{rgb}{1.000000,1.000000,0.929412}%
\pgfsetfillcolor{currentfill}%
\pgfsetlinewidth{0.250937pt}%
\definecolor{currentstroke}{rgb}{1.000000,1.000000,1.000000}%
\pgfsetstrokecolor{currentstroke}%
\pgfsetdash{}{0pt}%
\pgfpathmoveto{\pgfqpoint{3.188490in}{10.311132in}}%
\pgfpathlineto{\pgfqpoint{3.276226in}{10.311132in}}%
\pgfpathlineto{\pgfqpoint{3.276226in}{10.223396in}}%
\pgfpathlineto{\pgfqpoint{3.188490in}{10.223396in}}%
\pgfpathlineto{\pgfqpoint{3.188490in}{10.311132in}}%
\pgfusepath{stroke,fill}%
\end{pgfscope}%
\begin{pgfscope}%
\pgfpathrectangle{\pgfqpoint{0.380943in}{9.960189in}}{\pgfqpoint{4.650000in}{0.614151in}}%
\pgfusepath{clip}%
\pgfsetbuttcap%
\pgfsetroundjoin%
\definecolor{currentfill}{rgb}{1.000000,1.000000,0.929412}%
\pgfsetfillcolor{currentfill}%
\pgfsetlinewidth{0.250937pt}%
\definecolor{currentstroke}{rgb}{1.000000,1.000000,1.000000}%
\pgfsetstrokecolor{currentstroke}%
\pgfsetdash{}{0pt}%
\pgfpathmoveto{\pgfqpoint{3.276226in}{10.311132in}}%
\pgfpathlineto{\pgfqpoint{3.363962in}{10.311132in}}%
\pgfpathlineto{\pgfqpoint{3.363962in}{10.223396in}}%
\pgfpathlineto{\pgfqpoint{3.276226in}{10.223396in}}%
\pgfpathlineto{\pgfqpoint{3.276226in}{10.311132in}}%
\pgfusepath{stroke,fill}%
\end{pgfscope}%
\begin{pgfscope}%
\pgfpathrectangle{\pgfqpoint{0.380943in}{9.960189in}}{\pgfqpoint{4.650000in}{0.614151in}}%
\pgfusepath{clip}%
\pgfsetbuttcap%
\pgfsetroundjoin%
\definecolor{currentfill}{rgb}{1.000000,1.000000,0.929412}%
\pgfsetfillcolor{currentfill}%
\pgfsetlinewidth{0.250937pt}%
\definecolor{currentstroke}{rgb}{1.000000,1.000000,1.000000}%
\pgfsetstrokecolor{currentstroke}%
\pgfsetdash{}{0pt}%
\pgfpathmoveto{\pgfqpoint{3.363962in}{10.311132in}}%
\pgfpathlineto{\pgfqpoint{3.451698in}{10.311132in}}%
\pgfpathlineto{\pgfqpoint{3.451698in}{10.223396in}}%
\pgfpathlineto{\pgfqpoint{3.363962in}{10.223396in}}%
\pgfpathlineto{\pgfqpoint{3.363962in}{10.311132in}}%
\pgfusepath{stroke,fill}%
\end{pgfscope}%
\begin{pgfscope}%
\pgfpathrectangle{\pgfqpoint{0.380943in}{9.960189in}}{\pgfqpoint{4.650000in}{0.614151in}}%
\pgfusepath{clip}%
\pgfsetbuttcap%
\pgfsetroundjoin%
\definecolor{currentfill}{rgb}{1.000000,1.000000,0.929412}%
\pgfsetfillcolor{currentfill}%
\pgfsetlinewidth{0.250937pt}%
\definecolor{currentstroke}{rgb}{1.000000,1.000000,1.000000}%
\pgfsetstrokecolor{currentstroke}%
\pgfsetdash{}{0pt}%
\pgfpathmoveto{\pgfqpoint{3.451698in}{10.311132in}}%
\pgfpathlineto{\pgfqpoint{3.539434in}{10.311132in}}%
\pgfpathlineto{\pgfqpoint{3.539434in}{10.223396in}}%
\pgfpathlineto{\pgfqpoint{3.451698in}{10.223396in}}%
\pgfpathlineto{\pgfqpoint{3.451698in}{10.311132in}}%
\pgfusepath{stroke,fill}%
\end{pgfscope}%
\begin{pgfscope}%
\pgfpathrectangle{\pgfqpoint{0.380943in}{9.960189in}}{\pgfqpoint{4.650000in}{0.614151in}}%
\pgfusepath{clip}%
\pgfsetbuttcap%
\pgfsetroundjoin%
\definecolor{currentfill}{rgb}{1.000000,1.000000,0.929412}%
\pgfsetfillcolor{currentfill}%
\pgfsetlinewidth{0.250937pt}%
\definecolor{currentstroke}{rgb}{1.000000,1.000000,1.000000}%
\pgfsetstrokecolor{currentstroke}%
\pgfsetdash{}{0pt}%
\pgfpathmoveto{\pgfqpoint{3.539434in}{10.311132in}}%
\pgfpathlineto{\pgfqpoint{3.627169in}{10.311132in}}%
\pgfpathlineto{\pgfqpoint{3.627169in}{10.223396in}}%
\pgfpathlineto{\pgfqpoint{3.539434in}{10.223396in}}%
\pgfpathlineto{\pgfqpoint{3.539434in}{10.311132in}}%
\pgfusepath{stroke,fill}%
\end{pgfscope}%
\begin{pgfscope}%
\pgfpathrectangle{\pgfqpoint{0.380943in}{9.960189in}}{\pgfqpoint{4.650000in}{0.614151in}}%
\pgfusepath{clip}%
\pgfsetbuttcap%
\pgfsetroundjoin%
\definecolor{currentfill}{rgb}{1.000000,1.000000,0.929412}%
\pgfsetfillcolor{currentfill}%
\pgfsetlinewidth{0.250937pt}%
\definecolor{currentstroke}{rgb}{1.000000,1.000000,1.000000}%
\pgfsetstrokecolor{currentstroke}%
\pgfsetdash{}{0pt}%
\pgfpathmoveto{\pgfqpoint{3.627169in}{10.311132in}}%
\pgfpathlineto{\pgfqpoint{3.714905in}{10.311132in}}%
\pgfpathlineto{\pgfqpoint{3.714905in}{10.223396in}}%
\pgfpathlineto{\pgfqpoint{3.627169in}{10.223396in}}%
\pgfpathlineto{\pgfqpoint{3.627169in}{10.311132in}}%
\pgfusepath{stroke,fill}%
\end{pgfscope}%
\begin{pgfscope}%
\pgfpathrectangle{\pgfqpoint{0.380943in}{9.960189in}}{\pgfqpoint{4.650000in}{0.614151in}}%
\pgfusepath{clip}%
\pgfsetbuttcap%
\pgfsetroundjoin%
\definecolor{currentfill}{rgb}{1.000000,1.000000,0.929412}%
\pgfsetfillcolor{currentfill}%
\pgfsetlinewidth{0.250937pt}%
\definecolor{currentstroke}{rgb}{1.000000,1.000000,1.000000}%
\pgfsetstrokecolor{currentstroke}%
\pgfsetdash{}{0pt}%
\pgfpathmoveto{\pgfqpoint{3.714905in}{10.311132in}}%
\pgfpathlineto{\pgfqpoint{3.802641in}{10.311132in}}%
\pgfpathlineto{\pgfqpoint{3.802641in}{10.223396in}}%
\pgfpathlineto{\pgfqpoint{3.714905in}{10.223396in}}%
\pgfpathlineto{\pgfqpoint{3.714905in}{10.311132in}}%
\pgfusepath{stroke,fill}%
\end{pgfscope}%
\begin{pgfscope}%
\pgfpathrectangle{\pgfqpoint{0.380943in}{9.960189in}}{\pgfqpoint{4.650000in}{0.614151in}}%
\pgfusepath{clip}%
\pgfsetbuttcap%
\pgfsetroundjoin%
\definecolor{currentfill}{rgb}{1.000000,1.000000,0.929412}%
\pgfsetfillcolor{currentfill}%
\pgfsetlinewidth{0.250937pt}%
\definecolor{currentstroke}{rgb}{1.000000,1.000000,1.000000}%
\pgfsetstrokecolor{currentstroke}%
\pgfsetdash{}{0pt}%
\pgfpathmoveto{\pgfqpoint{3.802641in}{10.311132in}}%
\pgfpathlineto{\pgfqpoint{3.890377in}{10.311132in}}%
\pgfpathlineto{\pgfqpoint{3.890377in}{10.223396in}}%
\pgfpathlineto{\pgfqpoint{3.802641in}{10.223396in}}%
\pgfpathlineto{\pgfqpoint{3.802641in}{10.311132in}}%
\pgfusepath{stroke,fill}%
\end{pgfscope}%
\begin{pgfscope}%
\pgfpathrectangle{\pgfqpoint{0.380943in}{9.960189in}}{\pgfqpoint{4.650000in}{0.614151in}}%
\pgfusepath{clip}%
\pgfsetbuttcap%
\pgfsetroundjoin%
\definecolor{currentfill}{rgb}{1.000000,1.000000,0.929412}%
\pgfsetfillcolor{currentfill}%
\pgfsetlinewidth{0.250937pt}%
\definecolor{currentstroke}{rgb}{1.000000,1.000000,1.000000}%
\pgfsetstrokecolor{currentstroke}%
\pgfsetdash{}{0pt}%
\pgfpathmoveto{\pgfqpoint{3.890377in}{10.311132in}}%
\pgfpathlineto{\pgfqpoint{3.978113in}{10.311132in}}%
\pgfpathlineto{\pgfqpoint{3.978113in}{10.223396in}}%
\pgfpathlineto{\pgfqpoint{3.890377in}{10.223396in}}%
\pgfpathlineto{\pgfqpoint{3.890377in}{10.311132in}}%
\pgfusepath{stroke,fill}%
\end{pgfscope}%
\begin{pgfscope}%
\pgfpathrectangle{\pgfqpoint{0.380943in}{9.960189in}}{\pgfqpoint{4.650000in}{0.614151in}}%
\pgfusepath{clip}%
\pgfsetbuttcap%
\pgfsetroundjoin%
\definecolor{currentfill}{rgb}{1.000000,1.000000,0.929412}%
\pgfsetfillcolor{currentfill}%
\pgfsetlinewidth{0.250937pt}%
\definecolor{currentstroke}{rgb}{1.000000,1.000000,1.000000}%
\pgfsetstrokecolor{currentstroke}%
\pgfsetdash{}{0pt}%
\pgfpathmoveto{\pgfqpoint{3.978113in}{10.311132in}}%
\pgfpathlineto{\pgfqpoint{4.065849in}{10.311132in}}%
\pgfpathlineto{\pgfqpoint{4.065849in}{10.223396in}}%
\pgfpathlineto{\pgfqpoint{3.978113in}{10.223396in}}%
\pgfpathlineto{\pgfqpoint{3.978113in}{10.311132in}}%
\pgfusepath{stroke,fill}%
\end{pgfscope}%
\begin{pgfscope}%
\pgfpathrectangle{\pgfqpoint{0.380943in}{9.960189in}}{\pgfqpoint{4.650000in}{0.614151in}}%
\pgfusepath{clip}%
\pgfsetbuttcap%
\pgfsetroundjoin%
\definecolor{currentfill}{rgb}{1.000000,1.000000,0.929412}%
\pgfsetfillcolor{currentfill}%
\pgfsetlinewidth{0.250937pt}%
\definecolor{currentstroke}{rgb}{1.000000,1.000000,1.000000}%
\pgfsetstrokecolor{currentstroke}%
\pgfsetdash{}{0pt}%
\pgfpathmoveto{\pgfqpoint{4.065849in}{10.311132in}}%
\pgfpathlineto{\pgfqpoint{4.153585in}{10.311132in}}%
\pgfpathlineto{\pgfqpoint{4.153585in}{10.223396in}}%
\pgfpathlineto{\pgfqpoint{4.065849in}{10.223396in}}%
\pgfpathlineto{\pgfqpoint{4.065849in}{10.311132in}}%
\pgfusepath{stroke,fill}%
\end{pgfscope}%
\begin{pgfscope}%
\pgfpathrectangle{\pgfqpoint{0.380943in}{9.960189in}}{\pgfqpoint{4.650000in}{0.614151in}}%
\pgfusepath{clip}%
\pgfsetbuttcap%
\pgfsetroundjoin%
\definecolor{currentfill}{rgb}{1.000000,1.000000,0.929412}%
\pgfsetfillcolor{currentfill}%
\pgfsetlinewidth{0.250937pt}%
\definecolor{currentstroke}{rgb}{1.000000,1.000000,1.000000}%
\pgfsetstrokecolor{currentstroke}%
\pgfsetdash{}{0pt}%
\pgfpathmoveto{\pgfqpoint{4.153585in}{10.311132in}}%
\pgfpathlineto{\pgfqpoint{4.241320in}{10.311132in}}%
\pgfpathlineto{\pgfqpoint{4.241320in}{10.223396in}}%
\pgfpathlineto{\pgfqpoint{4.153585in}{10.223396in}}%
\pgfpathlineto{\pgfqpoint{4.153585in}{10.311132in}}%
\pgfusepath{stroke,fill}%
\end{pgfscope}%
\begin{pgfscope}%
\pgfpathrectangle{\pgfqpoint{0.380943in}{9.960189in}}{\pgfqpoint{4.650000in}{0.614151in}}%
\pgfusepath{clip}%
\pgfsetbuttcap%
\pgfsetroundjoin%
\definecolor{currentfill}{rgb}{1.000000,1.000000,0.929412}%
\pgfsetfillcolor{currentfill}%
\pgfsetlinewidth{0.250937pt}%
\definecolor{currentstroke}{rgb}{1.000000,1.000000,1.000000}%
\pgfsetstrokecolor{currentstroke}%
\pgfsetdash{}{0pt}%
\pgfpathmoveto{\pgfqpoint{4.241320in}{10.311132in}}%
\pgfpathlineto{\pgfqpoint{4.329056in}{10.311132in}}%
\pgfpathlineto{\pgfqpoint{4.329056in}{10.223396in}}%
\pgfpathlineto{\pgfqpoint{4.241320in}{10.223396in}}%
\pgfpathlineto{\pgfqpoint{4.241320in}{10.311132in}}%
\pgfusepath{stroke,fill}%
\end{pgfscope}%
\begin{pgfscope}%
\pgfpathrectangle{\pgfqpoint{0.380943in}{9.960189in}}{\pgfqpoint{4.650000in}{0.614151in}}%
\pgfusepath{clip}%
\pgfsetbuttcap%
\pgfsetroundjoin%
\definecolor{currentfill}{rgb}{1.000000,1.000000,0.929412}%
\pgfsetfillcolor{currentfill}%
\pgfsetlinewidth{0.250937pt}%
\definecolor{currentstroke}{rgb}{1.000000,1.000000,1.000000}%
\pgfsetstrokecolor{currentstroke}%
\pgfsetdash{}{0pt}%
\pgfpathmoveto{\pgfqpoint{4.329056in}{10.311132in}}%
\pgfpathlineto{\pgfqpoint{4.416792in}{10.311132in}}%
\pgfpathlineto{\pgfqpoint{4.416792in}{10.223396in}}%
\pgfpathlineto{\pgfqpoint{4.329056in}{10.223396in}}%
\pgfpathlineto{\pgfqpoint{4.329056in}{10.311132in}}%
\pgfusepath{stroke,fill}%
\end{pgfscope}%
\begin{pgfscope}%
\pgfpathrectangle{\pgfqpoint{0.380943in}{9.960189in}}{\pgfqpoint{4.650000in}{0.614151in}}%
\pgfusepath{clip}%
\pgfsetbuttcap%
\pgfsetroundjoin%
\definecolor{currentfill}{rgb}{0.967474,0.895963,0.706344}%
\pgfsetfillcolor{currentfill}%
\pgfsetlinewidth{0.250937pt}%
\definecolor{currentstroke}{rgb}{1.000000,1.000000,1.000000}%
\pgfsetstrokecolor{currentstroke}%
\pgfsetdash{}{0pt}%
\pgfpathmoveto{\pgfqpoint{4.416792in}{10.311132in}}%
\pgfpathlineto{\pgfqpoint{4.504528in}{10.311132in}}%
\pgfpathlineto{\pgfqpoint{4.504528in}{10.223396in}}%
\pgfpathlineto{\pgfqpoint{4.416792in}{10.223396in}}%
\pgfpathlineto{\pgfqpoint{4.416792in}{10.311132in}}%
\pgfusepath{stroke,fill}%
\end{pgfscope}%
\begin{pgfscope}%
\pgfpathrectangle{\pgfqpoint{0.380943in}{9.960189in}}{\pgfqpoint{4.650000in}{0.614151in}}%
\pgfusepath{clip}%
\pgfsetbuttcap%
\pgfsetroundjoin%
\definecolor{currentfill}{rgb}{0.997924,0.685352,0.570242}%
\pgfsetfillcolor{currentfill}%
\pgfsetlinewidth{0.250937pt}%
\definecolor{currentstroke}{rgb}{1.000000,1.000000,1.000000}%
\pgfsetstrokecolor{currentstroke}%
\pgfsetdash{}{0pt}%
\pgfpathmoveto{\pgfqpoint{4.504528in}{10.311132in}}%
\pgfpathlineto{\pgfqpoint{4.592264in}{10.311132in}}%
\pgfpathlineto{\pgfqpoint{4.592264in}{10.223396in}}%
\pgfpathlineto{\pgfqpoint{4.504528in}{10.223396in}}%
\pgfpathlineto{\pgfqpoint{4.504528in}{10.311132in}}%
\pgfusepath{stroke,fill}%
\end{pgfscope}%
\begin{pgfscope}%
\pgfpathrectangle{\pgfqpoint{0.380943in}{9.960189in}}{\pgfqpoint{4.650000in}{0.614151in}}%
\pgfusepath{clip}%
\pgfsetbuttcap%
\pgfsetroundjoin%
\definecolor{currentfill}{rgb}{0.967474,0.895963,0.706344}%
\pgfsetfillcolor{currentfill}%
\pgfsetlinewidth{0.250937pt}%
\definecolor{currentstroke}{rgb}{1.000000,1.000000,1.000000}%
\pgfsetstrokecolor{currentstroke}%
\pgfsetdash{}{0pt}%
\pgfpathmoveto{\pgfqpoint{4.592264in}{10.311132in}}%
\pgfpathlineto{\pgfqpoint{4.680000in}{10.311132in}}%
\pgfpathlineto{\pgfqpoint{4.680000in}{10.223396in}}%
\pgfpathlineto{\pgfqpoint{4.592264in}{10.223396in}}%
\pgfpathlineto{\pgfqpoint{4.592264in}{10.311132in}}%
\pgfusepath{stroke,fill}%
\end{pgfscope}%
\begin{pgfscope}%
\pgfpathrectangle{\pgfqpoint{0.380943in}{9.960189in}}{\pgfqpoint{4.650000in}{0.614151in}}%
\pgfusepath{clip}%
\pgfsetbuttcap%
\pgfsetroundjoin%
\definecolor{currentfill}{rgb}{0.989619,0.788235,0.628374}%
\pgfsetfillcolor{currentfill}%
\pgfsetlinewidth{0.250937pt}%
\definecolor{currentstroke}{rgb}{1.000000,1.000000,1.000000}%
\pgfsetstrokecolor{currentstroke}%
\pgfsetdash{}{0pt}%
\pgfpathmoveto{\pgfqpoint{4.680000in}{10.311132in}}%
\pgfpathlineto{\pgfqpoint{4.767736in}{10.311132in}}%
\pgfpathlineto{\pgfqpoint{4.767736in}{10.223396in}}%
\pgfpathlineto{\pgfqpoint{4.680000in}{10.223396in}}%
\pgfpathlineto{\pgfqpoint{4.680000in}{10.311132in}}%
\pgfusepath{stroke,fill}%
\end{pgfscope}%
\begin{pgfscope}%
\pgfpathrectangle{\pgfqpoint{0.380943in}{9.960189in}}{\pgfqpoint{4.650000in}{0.614151in}}%
\pgfusepath{clip}%
\pgfsetbuttcap%
\pgfsetroundjoin%
\definecolor{currentfill}{rgb}{0.982699,0.823991,0.657439}%
\pgfsetfillcolor{currentfill}%
\pgfsetlinewidth{0.250937pt}%
\definecolor{currentstroke}{rgb}{1.000000,1.000000,1.000000}%
\pgfsetstrokecolor{currentstroke}%
\pgfsetdash{}{0pt}%
\pgfpathmoveto{\pgfqpoint{4.767736in}{10.311132in}}%
\pgfpathlineto{\pgfqpoint{4.855471in}{10.311132in}}%
\pgfpathlineto{\pgfqpoint{4.855471in}{10.223396in}}%
\pgfpathlineto{\pgfqpoint{4.767736in}{10.223396in}}%
\pgfpathlineto{\pgfqpoint{4.767736in}{10.311132in}}%
\pgfusepath{stroke,fill}%
\end{pgfscope}%
\begin{pgfscope}%
\pgfpathrectangle{\pgfqpoint{0.380943in}{9.960189in}}{\pgfqpoint{4.650000in}{0.614151in}}%
\pgfusepath{clip}%
\pgfsetbuttcap%
\pgfsetroundjoin%
\definecolor{currentfill}{rgb}{0.994694,0.745098,0.602999}%
\pgfsetfillcolor{currentfill}%
\pgfsetlinewidth{0.250937pt}%
\definecolor{currentstroke}{rgb}{1.000000,1.000000,1.000000}%
\pgfsetstrokecolor{currentstroke}%
\pgfsetdash{}{0pt}%
\pgfpathmoveto{\pgfqpoint{4.855471in}{10.311132in}}%
\pgfpathlineto{\pgfqpoint{4.943207in}{10.311132in}}%
\pgfpathlineto{\pgfqpoint{4.943207in}{10.223396in}}%
\pgfpathlineto{\pgfqpoint{4.855471in}{10.223396in}}%
\pgfpathlineto{\pgfqpoint{4.855471in}{10.311132in}}%
\pgfusepath{stroke,fill}%
\end{pgfscope}%
\begin{pgfscope}%
\pgfpathrectangle{\pgfqpoint{0.380943in}{9.960189in}}{\pgfqpoint{4.650000in}{0.614151in}}%
\pgfusepath{clip}%
\pgfsetbuttcap%
\pgfsetroundjoin%
\definecolor{currentfill}{rgb}{0.989619,0.788235,0.628374}%
\pgfsetfillcolor{currentfill}%
\pgfsetlinewidth{0.250937pt}%
\definecolor{currentstroke}{rgb}{1.000000,1.000000,1.000000}%
\pgfsetstrokecolor{currentstroke}%
\pgfsetdash{}{0pt}%
\pgfpathmoveto{\pgfqpoint{4.943207in}{10.311132in}}%
\pgfpathlineto{\pgfqpoint{5.030943in}{10.311132in}}%
\pgfpathlineto{\pgfqpoint{5.030943in}{10.223396in}}%
\pgfpathlineto{\pgfqpoint{4.943207in}{10.223396in}}%
\pgfpathlineto{\pgfqpoint{4.943207in}{10.311132in}}%
\pgfusepath{stroke,fill}%
\end{pgfscope}%
\begin{pgfscope}%
\pgfpathrectangle{\pgfqpoint{0.380943in}{9.960189in}}{\pgfqpoint{4.650000in}{0.614151in}}%
\pgfusepath{clip}%
\pgfsetbuttcap%
\pgfsetroundjoin%
\definecolor{currentfill}{rgb}{1.000000,1.000000,0.929412}%
\pgfsetfillcolor{currentfill}%
\pgfsetlinewidth{0.250937pt}%
\definecolor{currentstroke}{rgb}{1.000000,1.000000,1.000000}%
\pgfsetstrokecolor{currentstroke}%
\pgfsetdash{}{0pt}%
\pgfpathmoveto{\pgfqpoint{0.380943in}{10.223396in}}%
\pgfpathlineto{\pgfqpoint{0.468679in}{10.223396in}}%
\pgfpathlineto{\pgfqpoint{0.468679in}{10.135661in}}%
\pgfpathlineto{\pgfqpoint{0.380943in}{10.135661in}}%
\pgfpathlineto{\pgfqpoint{0.380943in}{10.223396in}}%
\pgfusepath{stroke,fill}%
\end{pgfscope}%
\begin{pgfscope}%
\pgfpathrectangle{\pgfqpoint{0.380943in}{9.960189in}}{\pgfqpoint{4.650000in}{0.614151in}}%
\pgfusepath{clip}%
\pgfsetbuttcap%
\pgfsetroundjoin%
\definecolor{currentfill}{rgb}{1.000000,1.000000,0.929412}%
\pgfsetfillcolor{currentfill}%
\pgfsetlinewidth{0.250937pt}%
\definecolor{currentstroke}{rgb}{1.000000,1.000000,1.000000}%
\pgfsetstrokecolor{currentstroke}%
\pgfsetdash{}{0pt}%
\pgfpathmoveto{\pgfqpoint{0.468679in}{10.223396in}}%
\pgfpathlineto{\pgfqpoint{0.556415in}{10.223396in}}%
\pgfpathlineto{\pgfqpoint{0.556415in}{10.135661in}}%
\pgfpathlineto{\pgfqpoint{0.468679in}{10.135661in}}%
\pgfpathlineto{\pgfqpoint{0.468679in}{10.223396in}}%
\pgfusepath{stroke,fill}%
\end{pgfscope}%
\begin{pgfscope}%
\pgfpathrectangle{\pgfqpoint{0.380943in}{9.960189in}}{\pgfqpoint{4.650000in}{0.614151in}}%
\pgfusepath{clip}%
\pgfsetbuttcap%
\pgfsetroundjoin%
\definecolor{currentfill}{rgb}{1.000000,1.000000,0.929412}%
\pgfsetfillcolor{currentfill}%
\pgfsetlinewidth{0.250937pt}%
\definecolor{currentstroke}{rgb}{1.000000,1.000000,1.000000}%
\pgfsetstrokecolor{currentstroke}%
\pgfsetdash{}{0pt}%
\pgfpathmoveto{\pgfqpoint{0.556415in}{10.223396in}}%
\pgfpathlineto{\pgfqpoint{0.644151in}{10.223396in}}%
\pgfpathlineto{\pgfqpoint{0.644151in}{10.135661in}}%
\pgfpathlineto{\pgfqpoint{0.556415in}{10.135661in}}%
\pgfpathlineto{\pgfqpoint{0.556415in}{10.223396in}}%
\pgfusepath{stroke,fill}%
\end{pgfscope}%
\begin{pgfscope}%
\pgfpathrectangle{\pgfqpoint{0.380943in}{9.960189in}}{\pgfqpoint{4.650000in}{0.614151in}}%
\pgfusepath{clip}%
\pgfsetbuttcap%
\pgfsetroundjoin%
\definecolor{currentfill}{rgb}{1.000000,1.000000,0.929412}%
\pgfsetfillcolor{currentfill}%
\pgfsetlinewidth{0.250937pt}%
\definecolor{currentstroke}{rgb}{1.000000,1.000000,1.000000}%
\pgfsetstrokecolor{currentstroke}%
\pgfsetdash{}{0pt}%
\pgfpathmoveto{\pgfqpoint{0.644151in}{10.223396in}}%
\pgfpathlineto{\pgfqpoint{0.731886in}{10.223396in}}%
\pgfpathlineto{\pgfqpoint{0.731886in}{10.135661in}}%
\pgfpathlineto{\pgfqpoint{0.644151in}{10.135661in}}%
\pgfpathlineto{\pgfqpoint{0.644151in}{10.223396in}}%
\pgfusepath{stroke,fill}%
\end{pgfscope}%
\begin{pgfscope}%
\pgfpathrectangle{\pgfqpoint{0.380943in}{9.960189in}}{\pgfqpoint{4.650000in}{0.614151in}}%
\pgfusepath{clip}%
\pgfsetbuttcap%
\pgfsetroundjoin%
\definecolor{currentfill}{rgb}{1.000000,1.000000,0.929412}%
\pgfsetfillcolor{currentfill}%
\pgfsetlinewidth{0.250937pt}%
\definecolor{currentstroke}{rgb}{1.000000,1.000000,1.000000}%
\pgfsetstrokecolor{currentstroke}%
\pgfsetdash{}{0pt}%
\pgfpathmoveto{\pgfqpoint{0.731886in}{10.223396in}}%
\pgfpathlineto{\pgfqpoint{0.819622in}{10.223396in}}%
\pgfpathlineto{\pgfqpoint{0.819622in}{10.135661in}}%
\pgfpathlineto{\pgfqpoint{0.731886in}{10.135661in}}%
\pgfpathlineto{\pgfqpoint{0.731886in}{10.223396in}}%
\pgfusepath{stroke,fill}%
\end{pgfscope}%
\begin{pgfscope}%
\pgfpathrectangle{\pgfqpoint{0.380943in}{9.960189in}}{\pgfqpoint{4.650000in}{0.614151in}}%
\pgfusepath{clip}%
\pgfsetbuttcap%
\pgfsetroundjoin%
\definecolor{currentfill}{rgb}{1.000000,1.000000,0.929412}%
\pgfsetfillcolor{currentfill}%
\pgfsetlinewidth{0.250937pt}%
\definecolor{currentstroke}{rgb}{1.000000,1.000000,1.000000}%
\pgfsetstrokecolor{currentstroke}%
\pgfsetdash{}{0pt}%
\pgfpathmoveto{\pgfqpoint{0.819622in}{10.223396in}}%
\pgfpathlineto{\pgfqpoint{0.907358in}{10.223396in}}%
\pgfpathlineto{\pgfqpoint{0.907358in}{10.135661in}}%
\pgfpathlineto{\pgfqpoint{0.819622in}{10.135661in}}%
\pgfpathlineto{\pgfqpoint{0.819622in}{10.223396in}}%
\pgfusepath{stroke,fill}%
\end{pgfscope}%
\begin{pgfscope}%
\pgfpathrectangle{\pgfqpoint{0.380943in}{9.960189in}}{\pgfqpoint{4.650000in}{0.614151in}}%
\pgfusepath{clip}%
\pgfsetbuttcap%
\pgfsetroundjoin%
\definecolor{currentfill}{rgb}{1.000000,1.000000,0.929412}%
\pgfsetfillcolor{currentfill}%
\pgfsetlinewidth{0.250937pt}%
\definecolor{currentstroke}{rgb}{1.000000,1.000000,1.000000}%
\pgfsetstrokecolor{currentstroke}%
\pgfsetdash{}{0pt}%
\pgfpathmoveto{\pgfqpoint{0.907358in}{10.223396in}}%
\pgfpathlineto{\pgfqpoint{0.995094in}{10.223396in}}%
\pgfpathlineto{\pgfqpoint{0.995094in}{10.135661in}}%
\pgfpathlineto{\pgfqpoint{0.907358in}{10.135661in}}%
\pgfpathlineto{\pgfqpoint{0.907358in}{10.223396in}}%
\pgfusepath{stroke,fill}%
\end{pgfscope}%
\begin{pgfscope}%
\pgfpathrectangle{\pgfqpoint{0.380943in}{9.960189in}}{\pgfqpoint{4.650000in}{0.614151in}}%
\pgfusepath{clip}%
\pgfsetbuttcap%
\pgfsetroundjoin%
\definecolor{currentfill}{rgb}{1.000000,1.000000,0.929412}%
\pgfsetfillcolor{currentfill}%
\pgfsetlinewidth{0.250937pt}%
\definecolor{currentstroke}{rgb}{1.000000,1.000000,1.000000}%
\pgfsetstrokecolor{currentstroke}%
\pgfsetdash{}{0pt}%
\pgfpathmoveto{\pgfqpoint{0.995094in}{10.223396in}}%
\pgfpathlineto{\pgfqpoint{1.082830in}{10.223396in}}%
\pgfpathlineto{\pgfqpoint{1.082830in}{10.135661in}}%
\pgfpathlineto{\pgfqpoint{0.995094in}{10.135661in}}%
\pgfpathlineto{\pgfqpoint{0.995094in}{10.223396in}}%
\pgfusepath{stroke,fill}%
\end{pgfscope}%
\begin{pgfscope}%
\pgfpathrectangle{\pgfqpoint{0.380943in}{9.960189in}}{\pgfqpoint{4.650000in}{0.614151in}}%
\pgfusepath{clip}%
\pgfsetbuttcap%
\pgfsetroundjoin%
\definecolor{currentfill}{rgb}{1.000000,1.000000,0.929412}%
\pgfsetfillcolor{currentfill}%
\pgfsetlinewidth{0.250937pt}%
\definecolor{currentstroke}{rgb}{1.000000,1.000000,1.000000}%
\pgfsetstrokecolor{currentstroke}%
\pgfsetdash{}{0pt}%
\pgfpathmoveto{\pgfqpoint{1.082830in}{10.223396in}}%
\pgfpathlineto{\pgfqpoint{1.170566in}{10.223396in}}%
\pgfpathlineto{\pgfqpoint{1.170566in}{10.135661in}}%
\pgfpathlineto{\pgfqpoint{1.082830in}{10.135661in}}%
\pgfpathlineto{\pgfqpoint{1.082830in}{10.223396in}}%
\pgfusepath{stroke,fill}%
\end{pgfscope}%
\begin{pgfscope}%
\pgfpathrectangle{\pgfqpoint{0.380943in}{9.960189in}}{\pgfqpoint{4.650000in}{0.614151in}}%
\pgfusepath{clip}%
\pgfsetbuttcap%
\pgfsetroundjoin%
\definecolor{currentfill}{rgb}{1.000000,1.000000,0.929412}%
\pgfsetfillcolor{currentfill}%
\pgfsetlinewidth{0.250937pt}%
\definecolor{currentstroke}{rgb}{1.000000,1.000000,1.000000}%
\pgfsetstrokecolor{currentstroke}%
\pgfsetdash{}{0pt}%
\pgfpathmoveto{\pgfqpoint{1.170566in}{10.223396in}}%
\pgfpathlineto{\pgfqpoint{1.258302in}{10.223396in}}%
\pgfpathlineto{\pgfqpoint{1.258302in}{10.135661in}}%
\pgfpathlineto{\pgfqpoint{1.170566in}{10.135661in}}%
\pgfpathlineto{\pgfqpoint{1.170566in}{10.223396in}}%
\pgfusepath{stroke,fill}%
\end{pgfscope}%
\begin{pgfscope}%
\pgfpathrectangle{\pgfqpoint{0.380943in}{9.960189in}}{\pgfqpoint{4.650000in}{0.614151in}}%
\pgfusepath{clip}%
\pgfsetbuttcap%
\pgfsetroundjoin%
\definecolor{currentfill}{rgb}{1.000000,1.000000,0.929412}%
\pgfsetfillcolor{currentfill}%
\pgfsetlinewidth{0.250937pt}%
\definecolor{currentstroke}{rgb}{1.000000,1.000000,1.000000}%
\pgfsetstrokecolor{currentstroke}%
\pgfsetdash{}{0pt}%
\pgfpathmoveto{\pgfqpoint{1.258302in}{10.223396in}}%
\pgfpathlineto{\pgfqpoint{1.346037in}{10.223396in}}%
\pgfpathlineto{\pgfqpoint{1.346037in}{10.135661in}}%
\pgfpathlineto{\pgfqpoint{1.258302in}{10.135661in}}%
\pgfpathlineto{\pgfqpoint{1.258302in}{10.223396in}}%
\pgfusepath{stroke,fill}%
\end{pgfscope}%
\begin{pgfscope}%
\pgfpathrectangle{\pgfqpoint{0.380943in}{9.960189in}}{\pgfqpoint{4.650000in}{0.614151in}}%
\pgfusepath{clip}%
\pgfsetbuttcap%
\pgfsetroundjoin%
\definecolor{currentfill}{rgb}{1.000000,1.000000,0.929412}%
\pgfsetfillcolor{currentfill}%
\pgfsetlinewidth{0.250937pt}%
\definecolor{currentstroke}{rgb}{1.000000,1.000000,1.000000}%
\pgfsetstrokecolor{currentstroke}%
\pgfsetdash{}{0pt}%
\pgfpathmoveto{\pgfqpoint{1.346037in}{10.223396in}}%
\pgfpathlineto{\pgfqpoint{1.433773in}{10.223396in}}%
\pgfpathlineto{\pgfqpoint{1.433773in}{10.135661in}}%
\pgfpathlineto{\pgfqpoint{1.346037in}{10.135661in}}%
\pgfpathlineto{\pgfqpoint{1.346037in}{10.223396in}}%
\pgfusepath{stroke,fill}%
\end{pgfscope}%
\begin{pgfscope}%
\pgfpathrectangle{\pgfqpoint{0.380943in}{9.960189in}}{\pgfqpoint{4.650000in}{0.614151in}}%
\pgfusepath{clip}%
\pgfsetbuttcap%
\pgfsetroundjoin%
\definecolor{currentfill}{rgb}{1.000000,1.000000,0.929412}%
\pgfsetfillcolor{currentfill}%
\pgfsetlinewidth{0.250937pt}%
\definecolor{currentstroke}{rgb}{1.000000,1.000000,1.000000}%
\pgfsetstrokecolor{currentstroke}%
\pgfsetdash{}{0pt}%
\pgfpathmoveto{\pgfqpoint{1.433773in}{10.223396in}}%
\pgfpathlineto{\pgfqpoint{1.521509in}{10.223396in}}%
\pgfpathlineto{\pgfqpoint{1.521509in}{10.135661in}}%
\pgfpathlineto{\pgfqpoint{1.433773in}{10.135661in}}%
\pgfpathlineto{\pgfqpoint{1.433773in}{10.223396in}}%
\pgfusepath{stroke,fill}%
\end{pgfscope}%
\begin{pgfscope}%
\pgfpathrectangle{\pgfqpoint{0.380943in}{9.960189in}}{\pgfqpoint{4.650000in}{0.614151in}}%
\pgfusepath{clip}%
\pgfsetbuttcap%
\pgfsetroundjoin%
\definecolor{currentfill}{rgb}{1.000000,1.000000,0.929412}%
\pgfsetfillcolor{currentfill}%
\pgfsetlinewidth{0.250937pt}%
\definecolor{currentstroke}{rgb}{1.000000,1.000000,1.000000}%
\pgfsetstrokecolor{currentstroke}%
\pgfsetdash{}{0pt}%
\pgfpathmoveto{\pgfqpoint{1.521509in}{10.223396in}}%
\pgfpathlineto{\pgfqpoint{1.609245in}{10.223396in}}%
\pgfpathlineto{\pgfqpoint{1.609245in}{10.135661in}}%
\pgfpathlineto{\pgfqpoint{1.521509in}{10.135661in}}%
\pgfpathlineto{\pgfqpoint{1.521509in}{10.223396in}}%
\pgfusepath{stroke,fill}%
\end{pgfscope}%
\begin{pgfscope}%
\pgfpathrectangle{\pgfqpoint{0.380943in}{9.960189in}}{\pgfqpoint{4.650000in}{0.614151in}}%
\pgfusepath{clip}%
\pgfsetbuttcap%
\pgfsetroundjoin%
\definecolor{currentfill}{rgb}{1.000000,1.000000,0.929412}%
\pgfsetfillcolor{currentfill}%
\pgfsetlinewidth{0.250937pt}%
\definecolor{currentstroke}{rgb}{1.000000,1.000000,1.000000}%
\pgfsetstrokecolor{currentstroke}%
\pgfsetdash{}{0pt}%
\pgfpathmoveto{\pgfqpoint{1.609245in}{10.223396in}}%
\pgfpathlineto{\pgfqpoint{1.696981in}{10.223396in}}%
\pgfpathlineto{\pgfqpoint{1.696981in}{10.135661in}}%
\pgfpathlineto{\pgfqpoint{1.609245in}{10.135661in}}%
\pgfpathlineto{\pgfqpoint{1.609245in}{10.223396in}}%
\pgfusepath{stroke,fill}%
\end{pgfscope}%
\begin{pgfscope}%
\pgfpathrectangle{\pgfqpoint{0.380943in}{9.960189in}}{\pgfqpoint{4.650000in}{0.614151in}}%
\pgfusepath{clip}%
\pgfsetbuttcap%
\pgfsetroundjoin%
\definecolor{currentfill}{rgb}{1.000000,1.000000,0.929412}%
\pgfsetfillcolor{currentfill}%
\pgfsetlinewidth{0.250937pt}%
\definecolor{currentstroke}{rgb}{1.000000,1.000000,1.000000}%
\pgfsetstrokecolor{currentstroke}%
\pgfsetdash{}{0pt}%
\pgfpathmoveto{\pgfqpoint{1.696981in}{10.223396in}}%
\pgfpathlineto{\pgfqpoint{1.784717in}{10.223396in}}%
\pgfpathlineto{\pgfqpoint{1.784717in}{10.135661in}}%
\pgfpathlineto{\pgfqpoint{1.696981in}{10.135661in}}%
\pgfpathlineto{\pgfqpoint{1.696981in}{10.223396in}}%
\pgfusepath{stroke,fill}%
\end{pgfscope}%
\begin{pgfscope}%
\pgfpathrectangle{\pgfqpoint{0.380943in}{9.960189in}}{\pgfqpoint{4.650000in}{0.614151in}}%
\pgfusepath{clip}%
\pgfsetbuttcap%
\pgfsetroundjoin%
\definecolor{currentfill}{rgb}{1.000000,1.000000,0.929412}%
\pgfsetfillcolor{currentfill}%
\pgfsetlinewidth{0.250937pt}%
\definecolor{currentstroke}{rgb}{1.000000,1.000000,1.000000}%
\pgfsetstrokecolor{currentstroke}%
\pgfsetdash{}{0pt}%
\pgfpathmoveto{\pgfqpoint{1.784717in}{10.223396in}}%
\pgfpathlineto{\pgfqpoint{1.872452in}{10.223396in}}%
\pgfpathlineto{\pgfqpoint{1.872452in}{10.135661in}}%
\pgfpathlineto{\pgfqpoint{1.784717in}{10.135661in}}%
\pgfpathlineto{\pgfqpoint{1.784717in}{10.223396in}}%
\pgfusepath{stroke,fill}%
\end{pgfscope}%
\begin{pgfscope}%
\pgfpathrectangle{\pgfqpoint{0.380943in}{9.960189in}}{\pgfqpoint{4.650000in}{0.614151in}}%
\pgfusepath{clip}%
\pgfsetbuttcap%
\pgfsetroundjoin%
\definecolor{currentfill}{rgb}{1.000000,1.000000,0.929412}%
\pgfsetfillcolor{currentfill}%
\pgfsetlinewidth{0.250937pt}%
\definecolor{currentstroke}{rgb}{1.000000,1.000000,1.000000}%
\pgfsetstrokecolor{currentstroke}%
\pgfsetdash{}{0pt}%
\pgfpathmoveto{\pgfqpoint{1.872452in}{10.223396in}}%
\pgfpathlineto{\pgfqpoint{1.960188in}{10.223396in}}%
\pgfpathlineto{\pgfqpoint{1.960188in}{10.135661in}}%
\pgfpathlineto{\pgfqpoint{1.872452in}{10.135661in}}%
\pgfpathlineto{\pgfqpoint{1.872452in}{10.223396in}}%
\pgfusepath{stroke,fill}%
\end{pgfscope}%
\begin{pgfscope}%
\pgfpathrectangle{\pgfqpoint{0.380943in}{9.960189in}}{\pgfqpoint{4.650000in}{0.614151in}}%
\pgfusepath{clip}%
\pgfsetbuttcap%
\pgfsetroundjoin%
\definecolor{currentfill}{rgb}{1.000000,1.000000,0.929412}%
\pgfsetfillcolor{currentfill}%
\pgfsetlinewidth{0.250937pt}%
\definecolor{currentstroke}{rgb}{1.000000,1.000000,1.000000}%
\pgfsetstrokecolor{currentstroke}%
\pgfsetdash{}{0pt}%
\pgfpathmoveto{\pgfqpoint{1.960188in}{10.223396in}}%
\pgfpathlineto{\pgfqpoint{2.047924in}{10.223396in}}%
\pgfpathlineto{\pgfqpoint{2.047924in}{10.135661in}}%
\pgfpathlineto{\pgfqpoint{1.960188in}{10.135661in}}%
\pgfpathlineto{\pgfqpoint{1.960188in}{10.223396in}}%
\pgfusepath{stroke,fill}%
\end{pgfscope}%
\begin{pgfscope}%
\pgfpathrectangle{\pgfqpoint{0.380943in}{9.960189in}}{\pgfqpoint{4.650000in}{0.614151in}}%
\pgfusepath{clip}%
\pgfsetbuttcap%
\pgfsetroundjoin%
\definecolor{currentfill}{rgb}{1.000000,1.000000,0.929412}%
\pgfsetfillcolor{currentfill}%
\pgfsetlinewidth{0.250937pt}%
\definecolor{currentstroke}{rgb}{1.000000,1.000000,1.000000}%
\pgfsetstrokecolor{currentstroke}%
\pgfsetdash{}{0pt}%
\pgfpathmoveto{\pgfqpoint{2.047924in}{10.223396in}}%
\pgfpathlineto{\pgfqpoint{2.135660in}{10.223396in}}%
\pgfpathlineto{\pgfqpoint{2.135660in}{10.135661in}}%
\pgfpathlineto{\pgfqpoint{2.047924in}{10.135661in}}%
\pgfpathlineto{\pgfqpoint{2.047924in}{10.223396in}}%
\pgfusepath{stroke,fill}%
\end{pgfscope}%
\begin{pgfscope}%
\pgfpathrectangle{\pgfqpoint{0.380943in}{9.960189in}}{\pgfqpoint{4.650000in}{0.614151in}}%
\pgfusepath{clip}%
\pgfsetbuttcap%
\pgfsetroundjoin%
\definecolor{currentfill}{rgb}{1.000000,1.000000,0.929412}%
\pgfsetfillcolor{currentfill}%
\pgfsetlinewidth{0.250937pt}%
\definecolor{currentstroke}{rgb}{1.000000,1.000000,1.000000}%
\pgfsetstrokecolor{currentstroke}%
\pgfsetdash{}{0pt}%
\pgfpathmoveto{\pgfqpoint{2.135660in}{10.223396in}}%
\pgfpathlineto{\pgfqpoint{2.223396in}{10.223396in}}%
\pgfpathlineto{\pgfqpoint{2.223396in}{10.135661in}}%
\pgfpathlineto{\pgfqpoint{2.135660in}{10.135661in}}%
\pgfpathlineto{\pgfqpoint{2.135660in}{10.223396in}}%
\pgfusepath{stroke,fill}%
\end{pgfscope}%
\begin{pgfscope}%
\pgfpathrectangle{\pgfqpoint{0.380943in}{9.960189in}}{\pgfqpoint{4.650000in}{0.614151in}}%
\pgfusepath{clip}%
\pgfsetbuttcap%
\pgfsetroundjoin%
\definecolor{currentfill}{rgb}{1.000000,1.000000,0.929412}%
\pgfsetfillcolor{currentfill}%
\pgfsetlinewidth{0.250937pt}%
\definecolor{currentstroke}{rgb}{1.000000,1.000000,1.000000}%
\pgfsetstrokecolor{currentstroke}%
\pgfsetdash{}{0pt}%
\pgfpathmoveto{\pgfqpoint{2.223396in}{10.223396in}}%
\pgfpathlineto{\pgfqpoint{2.311132in}{10.223396in}}%
\pgfpathlineto{\pgfqpoint{2.311132in}{10.135661in}}%
\pgfpathlineto{\pgfqpoint{2.223396in}{10.135661in}}%
\pgfpathlineto{\pgfqpoint{2.223396in}{10.223396in}}%
\pgfusepath{stroke,fill}%
\end{pgfscope}%
\begin{pgfscope}%
\pgfpathrectangle{\pgfqpoint{0.380943in}{9.960189in}}{\pgfqpoint{4.650000in}{0.614151in}}%
\pgfusepath{clip}%
\pgfsetbuttcap%
\pgfsetroundjoin%
\definecolor{currentfill}{rgb}{1.000000,1.000000,0.929412}%
\pgfsetfillcolor{currentfill}%
\pgfsetlinewidth{0.250937pt}%
\definecolor{currentstroke}{rgb}{1.000000,1.000000,1.000000}%
\pgfsetstrokecolor{currentstroke}%
\pgfsetdash{}{0pt}%
\pgfpathmoveto{\pgfqpoint{2.311132in}{10.223396in}}%
\pgfpathlineto{\pgfqpoint{2.398868in}{10.223396in}}%
\pgfpathlineto{\pgfqpoint{2.398868in}{10.135661in}}%
\pgfpathlineto{\pgfqpoint{2.311132in}{10.135661in}}%
\pgfpathlineto{\pgfqpoint{2.311132in}{10.223396in}}%
\pgfusepath{stroke,fill}%
\end{pgfscope}%
\begin{pgfscope}%
\pgfpathrectangle{\pgfqpoint{0.380943in}{9.960189in}}{\pgfqpoint{4.650000in}{0.614151in}}%
\pgfusepath{clip}%
\pgfsetbuttcap%
\pgfsetroundjoin%
\definecolor{currentfill}{rgb}{1.000000,1.000000,0.929412}%
\pgfsetfillcolor{currentfill}%
\pgfsetlinewidth{0.250937pt}%
\definecolor{currentstroke}{rgb}{1.000000,1.000000,1.000000}%
\pgfsetstrokecolor{currentstroke}%
\pgfsetdash{}{0pt}%
\pgfpathmoveto{\pgfqpoint{2.398868in}{10.223396in}}%
\pgfpathlineto{\pgfqpoint{2.486603in}{10.223396in}}%
\pgfpathlineto{\pgfqpoint{2.486603in}{10.135661in}}%
\pgfpathlineto{\pgfqpoint{2.398868in}{10.135661in}}%
\pgfpathlineto{\pgfqpoint{2.398868in}{10.223396in}}%
\pgfusepath{stroke,fill}%
\end{pgfscope}%
\begin{pgfscope}%
\pgfpathrectangle{\pgfqpoint{0.380943in}{9.960189in}}{\pgfqpoint{4.650000in}{0.614151in}}%
\pgfusepath{clip}%
\pgfsetbuttcap%
\pgfsetroundjoin%
\definecolor{currentfill}{rgb}{1.000000,1.000000,0.929412}%
\pgfsetfillcolor{currentfill}%
\pgfsetlinewidth{0.250937pt}%
\definecolor{currentstroke}{rgb}{1.000000,1.000000,1.000000}%
\pgfsetstrokecolor{currentstroke}%
\pgfsetdash{}{0pt}%
\pgfpathmoveto{\pgfqpoint{2.486603in}{10.223396in}}%
\pgfpathlineto{\pgfqpoint{2.574339in}{10.223396in}}%
\pgfpathlineto{\pgfqpoint{2.574339in}{10.135661in}}%
\pgfpathlineto{\pgfqpoint{2.486603in}{10.135661in}}%
\pgfpathlineto{\pgfqpoint{2.486603in}{10.223396in}}%
\pgfusepath{stroke,fill}%
\end{pgfscope}%
\begin{pgfscope}%
\pgfpathrectangle{\pgfqpoint{0.380943in}{9.960189in}}{\pgfqpoint{4.650000in}{0.614151in}}%
\pgfusepath{clip}%
\pgfsetbuttcap%
\pgfsetroundjoin%
\definecolor{currentfill}{rgb}{1.000000,1.000000,0.929412}%
\pgfsetfillcolor{currentfill}%
\pgfsetlinewidth{0.250937pt}%
\definecolor{currentstroke}{rgb}{1.000000,1.000000,1.000000}%
\pgfsetstrokecolor{currentstroke}%
\pgfsetdash{}{0pt}%
\pgfpathmoveto{\pgfqpoint{2.574339in}{10.223396in}}%
\pgfpathlineto{\pgfqpoint{2.662075in}{10.223396in}}%
\pgfpathlineto{\pgfqpoint{2.662075in}{10.135661in}}%
\pgfpathlineto{\pgfqpoint{2.574339in}{10.135661in}}%
\pgfpathlineto{\pgfqpoint{2.574339in}{10.223396in}}%
\pgfusepath{stroke,fill}%
\end{pgfscope}%
\begin{pgfscope}%
\pgfpathrectangle{\pgfqpoint{0.380943in}{9.960189in}}{\pgfqpoint{4.650000in}{0.614151in}}%
\pgfusepath{clip}%
\pgfsetbuttcap%
\pgfsetroundjoin%
\definecolor{currentfill}{rgb}{1.000000,1.000000,0.929412}%
\pgfsetfillcolor{currentfill}%
\pgfsetlinewidth{0.250937pt}%
\definecolor{currentstroke}{rgb}{1.000000,1.000000,1.000000}%
\pgfsetstrokecolor{currentstroke}%
\pgfsetdash{}{0pt}%
\pgfpathmoveto{\pgfqpoint{2.662075in}{10.223396in}}%
\pgfpathlineto{\pgfqpoint{2.749811in}{10.223396in}}%
\pgfpathlineto{\pgfqpoint{2.749811in}{10.135661in}}%
\pgfpathlineto{\pgfqpoint{2.662075in}{10.135661in}}%
\pgfpathlineto{\pgfqpoint{2.662075in}{10.223396in}}%
\pgfusepath{stroke,fill}%
\end{pgfscope}%
\begin{pgfscope}%
\pgfpathrectangle{\pgfqpoint{0.380943in}{9.960189in}}{\pgfqpoint{4.650000in}{0.614151in}}%
\pgfusepath{clip}%
\pgfsetbuttcap%
\pgfsetroundjoin%
\definecolor{currentfill}{rgb}{1.000000,1.000000,0.929412}%
\pgfsetfillcolor{currentfill}%
\pgfsetlinewidth{0.250937pt}%
\definecolor{currentstroke}{rgb}{1.000000,1.000000,1.000000}%
\pgfsetstrokecolor{currentstroke}%
\pgfsetdash{}{0pt}%
\pgfpathmoveto{\pgfqpoint{2.749811in}{10.223396in}}%
\pgfpathlineto{\pgfqpoint{2.837547in}{10.223396in}}%
\pgfpathlineto{\pgfqpoint{2.837547in}{10.135661in}}%
\pgfpathlineto{\pgfqpoint{2.749811in}{10.135661in}}%
\pgfpathlineto{\pgfqpoint{2.749811in}{10.223396in}}%
\pgfusepath{stroke,fill}%
\end{pgfscope}%
\begin{pgfscope}%
\pgfpathrectangle{\pgfqpoint{0.380943in}{9.960189in}}{\pgfqpoint{4.650000in}{0.614151in}}%
\pgfusepath{clip}%
\pgfsetbuttcap%
\pgfsetroundjoin%
\definecolor{currentfill}{rgb}{1.000000,1.000000,0.929412}%
\pgfsetfillcolor{currentfill}%
\pgfsetlinewidth{0.250937pt}%
\definecolor{currentstroke}{rgb}{1.000000,1.000000,1.000000}%
\pgfsetstrokecolor{currentstroke}%
\pgfsetdash{}{0pt}%
\pgfpathmoveto{\pgfqpoint{2.837547in}{10.223396in}}%
\pgfpathlineto{\pgfqpoint{2.925283in}{10.223396in}}%
\pgfpathlineto{\pgfqpoint{2.925283in}{10.135661in}}%
\pgfpathlineto{\pgfqpoint{2.837547in}{10.135661in}}%
\pgfpathlineto{\pgfqpoint{2.837547in}{10.223396in}}%
\pgfusepath{stroke,fill}%
\end{pgfscope}%
\begin{pgfscope}%
\pgfpathrectangle{\pgfqpoint{0.380943in}{9.960189in}}{\pgfqpoint{4.650000in}{0.614151in}}%
\pgfusepath{clip}%
\pgfsetbuttcap%
\pgfsetroundjoin%
\definecolor{currentfill}{rgb}{1.000000,1.000000,0.929412}%
\pgfsetfillcolor{currentfill}%
\pgfsetlinewidth{0.250937pt}%
\definecolor{currentstroke}{rgb}{1.000000,1.000000,1.000000}%
\pgfsetstrokecolor{currentstroke}%
\pgfsetdash{}{0pt}%
\pgfpathmoveto{\pgfqpoint{2.925283in}{10.223396in}}%
\pgfpathlineto{\pgfqpoint{3.013019in}{10.223396in}}%
\pgfpathlineto{\pgfqpoint{3.013019in}{10.135661in}}%
\pgfpathlineto{\pgfqpoint{2.925283in}{10.135661in}}%
\pgfpathlineto{\pgfqpoint{2.925283in}{10.223396in}}%
\pgfusepath{stroke,fill}%
\end{pgfscope}%
\begin{pgfscope}%
\pgfpathrectangle{\pgfqpoint{0.380943in}{9.960189in}}{\pgfqpoint{4.650000in}{0.614151in}}%
\pgfusepath{clip}%
\pgfsetbuttcap%
\pgfsetroundjoin%
\definecolor{currentfill}{rgb}{1.000000,1.000000,0.929412}%
\pgfsetfillcolor{currentfill}%
\pgfsetlinewidth{0.250937pt}%
\definecolor{currentstroke}{rgb}{1.000000,1.000000,1.000000}%
\pgfsetstrokecolor{currentstroke}%
\pgfsetdash{}{0pt}%
\pgfpathmoveto{\pgfqpoint{3.013019in}{10.223396in}}%
\pgfpathlineto{\pgfqpoint{3.100754in}{10.223396in}}%
\pgfpathlineto{\pgfqpoint{3.100754in}{10.135661in}}%
\pgfpathlineto{\pgfqpoint{3.013019in}{10.135661in}}%
\pgfpathlineto{\pgfqpoint{3.013019in}{10.223396in}}%
\pgfusepath{stroke,fill}%
\end{pgfscope}%
\begin{pgfscope}%
\pgfpathrectangle{\pgfqpoint{0.380943in}{9.960189in}}{\pgfqpoint{4.650000in}{0.614151in}}%
\pgfusepath{clip}%
\pgfsetbuttcap%
\pgfsetroundjoin%
\definecolor{currentfill}{rgb}{1.000000,1.000000,0.929412}%
\pgfsetfillcolor{currentfill}%
\pgfsetlinewidth{0.250937pt}%
\definecolor{currentstroke}{rgb}{1.000000,1.000000,1.000000}%
\pgfsetstrokecolor{currentstroke}%
\pgfsetdash{}{0pt}%
\pgfpathmoveto{\pgfqpoint{3.100754in}{10.223396in}}%
\pgfpathlineto{\pgfqpoint{3.188490in}{10.223396in}}%
\pgfpathlineto{\pgfqpoint{3.188490in}{10.135661in}}%
\pgfpathlineto{\pgfqpoint{3.100754in}{10.135661in}}%
\pgfpathlineto{\pgfqpoint{3.100754in}{10.223396in}}%
\pgfusepath{stroke,fill}%
\end{pgfscope}%
\begin{pgfscope}%
\pgfpathrectangle{\pgfqpoint{0.380943in}{9.960189in}}{\pgfqpoint{4.650000in}{0.614151in}}%
\pgfusepath{clip}%
\pgfsetbuttcap%
\pgfsetroundjoin%
\definecolor{currentfill}{rgb}{1.000000,1.000000,0.929412}%
\pgfsetfillcolor{currentfill}%
\pgfsetlinewidth{0.250937pt}%
\definecolor{currentstroke}{rgb}{1.000000,1.000000,1.000000}%
\pgfsetstrokecolor{currentstroke}%
\pgfsetdash{}{0pt}%
\pgfpathmoveto{\pgfqpoint{3.188490in}{10.223396in}}%
\pgfpathlineto{\pgfqpoint{3.276226in}{10.223396in}}%
\pgfpathlineto{\pgfqpoint{3.276226in}{10.135661in}}%
\pgfpathlineto{\pgfqpoint{3.188490in}{10.135661in}}%
\pgfpathlineto{\pgfqpoint{3.188490in}{10.223396in}}%
\pgfusepath{stroke,fill}%
\end{pgfscope}%
\begin{pgfscope}%
\pgfpathrectangle{\pgfqpoint{0.380943in}{9.960189in}}{\pgfqpoint{4.650000in}{0.614151in}}%
\pgfusepath{clip}%
\pgfsetbuttcap%
\pgfsetroundjoin%
\definecolor{currentfill}{rgb}{1.000000,1.000000,0.929412}%
\pgfsetfillcolor{currentfill}%
\pgfsetlinewidth{0.250937pt}%
\definecolor{currentstroke}{rgb}{1.000000,1.000000,1.000000}%
\pgfsetstrokecolor{currentstroke}%
\pgfsetdash{}{0pt}%
\pgfpathmoveto{\pgfqpoint{3.276226in}{10.223396in}}%
\pgfpathlineto{\pgfqpoint{3.363962in}{10.223396in}}%
\pgfpathlineto{\pgfqpoint{3.363962in}{10.135661in}}%
\pgfpathlineto{\pgfqpoint{3.276226in}{10.135661in}}%
\pgfpathlineto{\pgfqpoint{3.276226in}{10.223396in}}%
\pgfusepath{stroke,fill}%
\end{pgfscope}%
\begin{pgfscope}%
\pgfpathrectangle{\pgfqpoint{0.380943in}{9.960189in}}{\pgfqpoint{4.650000in}{0.614151in}}%
\pgfusepath{clip}%
\pgfsetbuttcap%
\pgfsetroundjoin%
\definecolor{currentfill}{rgb}{1.000000,1.000000,0.929412}%
\pgfsetfillcolor{currentfill}%
\pgfsetlinewidth{0.250937pt}%
\definecolor{currentstroke}{rgb}{1.000000,1.000000,1.000000}%
\pgfsetstrokecolor{currentstroke}%
\pgfsetdash{}{0pt}%
\pgfpathmoveto{\pgfqpoint{3.363962in}{10.223396in}}%
\pgfpathlineto{\pgfqpoint{3.451698in}{10.223396in}}%
\pgfpathlineto{\pgfqpoint{3.451698in}{10.135661in}}%
\pgfpathlineto{\pgfqpoint{3.363962in}{10.135661in}}%
\pgfpathlineto{\pgfqpoint{3.363962in}{10.223396in}}%
\pgfusepath{stroke,fill}%
\end{pgfscope}%
\begin{pgfscope}%
\pgfpathrectangle{\pgfqpoint{0.380943in}{9.960189in}}{\pgfqpoint{4.650000in}{0.614151in}}%
\pgfusepath{clip}%
\pgfsetbuttcap%
\pgfsetroundjoin%
\definecolor{currentfill}{rgb}{1.000000,1.000000,0.929412}%
\pgfsetfillcolor{currentfill}%
\pgfsetlinewidth{0.250937pt}%
\definecolor{currentstroke}{rgb}{1.000000,1.000000,1.000000}%
\pgfsetstrokecolor{currentstroke}%
\pgfsetdash{}{0pt}%
\pgfpathmoveto{\pgfqpoint{3.451698in}{10.223396in}}%
\pgfpathlineto{\pgfqpoint{3.539434in}{10.223396in}}%
\pgfpathlineto{\pgfqpoint{3.539434in}{10.135661in}}%
\pgfpathlineto{\pgfqpoint{3.451698in}{10.135661in}}%
\pgfpathlineto{\pgfqpoint{3.451698in}{10.223396in}}%
\pgfusepath{stroke,fill}%
\end{pgfscope}%
\begin{pgfscope}%
\pgfpathrectangle{\pgfqpoint{0.380943in}{9.960189in}}{\pgfqpoint{4.650000in}{0.614151in}}%
\pgfusepath{clip}%
\pgfsetbuttcap%
\pgfsetroundjoin%
\definecolor{currentfill}{rgb}{1.000000,1.000000,0.929412}%
\pgfsetfillcolor{currentfill}%
\pgfsetlinewidth{0.250937pt}%
\definecolor{currentstroke}{rgb}{1.000000,1.000000,1.000000}%
\pgfsetstrokecolor{currentstroke}%
\pgfsetdash{}{0pt}%
\pgfpathmoveto{\pgfqpoint{3.539434in}{10.223396in}}%
\pgfpathlineto{\pgfqpoint{3.627169in}{10.223396in}}%
\pgfpathlineto{\pgfqpoint{3.627169in}{10.135661in}}%
\pgfpathlineto{\pgfqpoint{3.539434in}{10.135661in}}%
\pgfpathlineto{\pgfqpoint{3.539434in}{10.223396in}}%
\pgfusepath{stroke,fill}%
\end{pgfscope}%
\begin{pgfscope}%
\pgfpathrectangle{\pgfqpoint{0.380943in}{9.960189in}}{\pgfqpoint{4.650000in}{0.614151in}}%
\pgfusepath{clip}%
\pgfsetbuttcap%
\pgfsetroundjoin%
\definecolor{currentfill}{rgb}{1.000000,1.000000,0.929412}%
\pgfsetfillcolor{currentfill}%
\pgfsetlinewidth{0.250937pt}%
\definecolor{currentstroke}{rgb}{1.000000,1.000000,1.000000}%
\pgfsetstrokecolor{currentstroke}%
\pgfsetdash{}{0pt}%
\pgfpathmoveto{\pgfqpoint{3.627169in}{10.223396in}}%
\pgfpathlineto{\pgfqpoint{3.714905in}{10.223396in}}%
\pgfpathlineto{\pgfqpoint{3.714905in}{10.135661in}}%
\pgfpathlineto{\pgfqpoint{3.627169in}{10.135661in}}%
\pgfpathlineto{\pgfqpoint{3.627169in}{10.223396in}}%
\pgfusepath{stroke,fill}%
\end{pgfscope}%
\begin{pgfscope}%
\pgfpathrectangle{\pgfqpoint{0.380943in}{9.960189in}}{\pgfqpoint{4.650000in}{0.614151in}}%
\pgfusepath{clip}%
\pgfsetbuttcap%
\pgfsetroundjoin%
\definecolor{currentfill}{rgb}{1.000000,1.000000,0.929412}%
\pgfsetfillcolor{currentfill}%
\pgfsetlinewidth{0.250937pt}%
\definecolor{currentstroke}{rgb}{1.000000,1.000000,1.000000}%
\pgfsetstrokecolor{currentstroke}%
\pgfsetdash{}{0pt}%
\pgfpathmoveto{\pgfqpoint{3.714905in}{10.223396in}}%
\pgfpathlineto{\pgfqpoint{3.802641in}{10.223396in}}%
\pgfpathlineto{\pgfqpoint{3.802641in}{10.135661in}}%
\pgfpathlineto{\pgfqpoint{3.714905in}{10.135661in}}%
\pgfpathlineto{\pgfqpoint{3.714905in}{10.223396in}}%
\pgfusepath{stroke,fill}%
\end{pgfscope}%
\begin{pgfscope}%
\pgfpathrectangle{\pgfqpoint{0.380943in}{9.960189in}}{\pgfqpoint{4.650000in}{0.614151in}}%
\pgfusepath{clip}%
\pgfsetbuttcap%
\pgfsetroundjoin%
\definecolor{currentfill}{rgb}{1.000000,1.000000,0.929412}%
\pgfsetfillcolor{currentfill}%
\pgfsetlinewidth{0.250937pt}%
\definecolor{currentstroke}{rgb}{1.000000,1.000000,1.000000}%
\pgfsetstrokecolor{currentstroke}%
\pgfsetdash{}{0pt}%
\pgfpathmoveto{\pgfqpoint{3.802641in}{10.223396in}}%
\pgfpathlineto{\pgfqpoint{3.890377in}{10.223396in}}%
\pgfpathlineto{\pgfqpoint{3.890377in}{10.135661in}}%
\pgfpathlineto{\pgfqpoint{3.802641in}{10.135661in}}%
\pgfpathlineto{\pgfqpoint{3.802641in}{10.223396in}}%
\pgfusepath{stroke,fill}%
\end{pgfscope}%
\begin{pgfscope}%
\pgfpathrectangle{\pgfqpoint{0.380943in}{9.960189in}}{\pgfqpoint{4.650000in}{0.614151in}}%
\pgfusepath{clip}%
\pgfsetbuttcap%
\pgfsetroundjoin%
\definecolor{currentfill}{rgb}{1.000000,1.000000,0.929412}%
\pgfsetfillcolor{currentfill}%
\pgfsetlinewidth{0.250937pt}%
\definecolor{currentstroke}{rgb}{1.000000,1.000000,1.000000}%
\pgfsetstrokecolor{currentstroke}%
\pgfsetdash{}{0pt}%
\pgfpathmoveto{\pgfqpoint{3.890377in}{10.223396in}}%
\pgfpathlineto{\pgfqpoint{3.978113in}{10.223396in}}%
\pgfpathlineto{\pgfqpoint{3.978113in}{10.135661in}}%
\pgfpathlineto{\pgfqpoint{3.890377in}{10.135661in}}%
\pgfpathlineto{\pgfqpoint{3.890377in}{10.223396in}}%
\pgfusepath{stroke,fill}%
\end{pgfscope}%
\begin{pgfscope}%
\pgfpathrectangle{\pgfqpoint{0.380943in}{9.960189in}}{\pgfqpoint{4.650000in}{0.614151in}}%
\pgfusepath{clip}%
\pgfsetbuttcap%
\pgfsetroundjoin%
\definecolor{currentfill}{rgb}{1.000000,1.000000,0.929412}%
\pgfsetfillcolor{currentfill}%
\pgfsetlinewidth{0.250937pt}%
\definecolor{currentstroke}{rgb}{1.000000,1.000000,1.000000}%
\pgfsetstrokecolor{currentstroke}%
\pgfsetdash{}{0pt}%
\pgfpathmoveto{\pgfqpoint{3.978113in}{10.223396in}}%
\pgfpathlineto{\pgfqpoint{4.065849in}{10.223396in}}%
\pgfpathlineto{\pgfqpoint{4.065849in}{10.135661in}}%
\pgfpathlineto{\pgfqpoint{3.978113in}{10.135661in}}%
\pgfpathlineto{\pgfqpoint{3.978113in}{10.223396in}}%
\pgfusepath{stroke,fill}%
\end{pgfscope}%
\begin{pgfscope}%
\pgfpathrectangle{\pgfqpoint{0.380943in}{9.960189in}}{\pgfqpoint{4.650000in}{0.614151in}}%
\pgfusepath{clip}%
\pgfsetbuttcap%
\pgfsetroundjoin%
\definecolor{currentfill}{rgb}{1.000000,1.000000,0.929412}%
\pgfsetfillcolor{currentfill}%
\pgfsetlinewidth{0.250937pt}%
\definecolor{currentstroke}{rgb}{1.000000,1.000000,1.000000}%
\pgfsetstrokecolor{currentstroke}%
\pgfsetdash{}{0pt}%
\pgfpathmoveto{\pgfqpoint{4.065849in}{10.223396in}}%
\pgfpathlineto{\pgfqpoint{4.153585in}{10.223396in}}%
\pgfpathlineto{\pgfqpoint{4.153585in}{10.135661in}}%
\pgfpathlineto{\pgfqpoint{4.065849in}{10.135661in}}%
\pgfpathlineto{\pgfqpoint{4.065849in}{10.223396in}}%
\pgfusepath{stroke,fill}%
\end{pgfscope}%
\begin{pgfscope}%
\pgfpathrectangle{\pgfqpoint{0.380943in}{9.960189in}}{\pgfqpoint{4.650000in}{0.614151in}}%
\pgfusepath{clip}%
\pgfsetbuttcap%
\pgfsetroundjoin%
\definecolor{currentfill}{rgb}{1.000000,1.000000,0.929412}%
\pgfsetfillcolor{currentfill}%
\pgfsetlinewidth{0.250937pt}%
\definecolor{currentstroke}{rgb}{1.000000,1.000000,1.000000}%
\pgfsetstrokecolor{currentstroke}%
\pgfsetdash{}{0pt}%
\pgfpathmoveto{\pgfqpoint{4.153585in}{10.223396in}}%
\pgfpathlineto{\pgfqpoint{4.241320in}{10.223396in}}%
\pgfpathlineto{\pgfqpoint{4.241320in}{10.135661in}}%
\pgfpathlineto{\pgfqpoint{4.153585in}{10.135661in}}%
\pgfpathlineto{\pgfqpoint{4.153585in}{10.223396in}}%
\pgfusepath{stroke,fill}%
\end{pgfscope}%
\begin{pgfscope}%
\pgfpathrectangle{\pgfqpoint{0.380943in}{9.960189in}}{\pgfqpoint{4.650000in}{0.614151in}}%
\pgfusepath{clip}%
\pgfsetbuttcap%
\pgfsetroundjoin%
\definecolor{currentfill}{rgb}{1.000000,1.000000,0.929412}%
\pgfsetfillcolor{currentfill}%
\pgfsetlinewidth{0.250937pt}%
\definecolor{currentstroke}{rgb}{1.000000,1.000000,1.000000}%
\pgfsetstrokecolor{currentstroke}%
\pgfsetdash{}{0pt}%
\pgfpathmoveto{\pgfqpoint{4.241320in}{10.223396in}}%
\pgfpathlineto{\pgfqpoint{4.329056in}{10.223396in}}%
\pgfpathlineto{\pgfqpoint{4.329056in}{10.135661in}}%
\pgfpathlineto{\pgfqpoint{4.241320in}{10.135661in}}%
\pgfpathlineto{\pgfqpoint{4.241320in}{10.223396in}}%
\pgfusepath{stroke,fill}%
\end{pgfscope}%
\begin{pgfscope}%
\pgfpathrectangle{\pgfqpoint{0.380943in}{9.960189in}}{\pgfqpoint{4.650000in}{0.614151in}}%
\pgfusepath{clip}%
\pgfsetbuttcap%
\pgfsetroundjoin%
\definecolor{currentfill}{rgb}{1.000000,1.000000,0.929412}%
\pgfsetfillcolor{currentfill}%
\pgfsetlinewidth{0.250937pt}%
\definecolor{currentstroke}{rgb}{1.000000,1.000000,1.000000}%
\pgfsetstrokecolor{currentstroke}%
\pgfsetdash{}{0pt}%
\pgfpathmoveto{\pgfqpoint{4.329056in}{10.223396in}}%
\pgfpathlineto{\pgfqpoint{4.416792in}{10.223396in}}%
\pgfpathlineto{\pgfqpoint{4.416792in}{10.135661in}}%
\pgfpathlineto{\pgfqpoint{4.329056in}{10.135661in}}%
\pgfpathlineto{\pgfqpoint{4.329056in}{10.223396in}}%
\pgfusepath{stroke,fill}%
\end{pgfscope}%
\begin{pgfscope}%
\pgfpathrectangle{\pgfqpoint{0.380943in}{9.960189in}}{\pgfqpoint{4.650000in}{0.614151in}}%
\pgfusepath{clip}%
\pgfsetbuttcap%
\pgfsetroundjoin%
\definecolor{currentfill}{rgb}{0.963091,0.919493,0.720185}%
\pgfsetfillcolor{currentfill}%
\pgfsetlinewidth{0.250937pt}%
\definecolor{currentstroke}{rgb}{1.000000,1.000000,1.000000}%
\pgfsetstrokecolor{currentstroke}%
\pgfsetdash{}{0pt}%
\pgfpathmoveto{\pgfqpoint{4.416792in}{10.223396in}}%
\pgfpathlineto{\pgfqpoint{4.504528in}{10.223396in}}%
\pgfpathlineto{\pgfqpoint{4.504528in}{10.135661in}}%
\pgfpathlineto{\pgfqpoint{4.416792in}{10.135661in}}%
\pgfpathlineto{\pgfqpoint{4.416792in}{10.223396in}}%
\pgfusepath{stroke,fill}%
\end{pgfscope}%
\begin{pgfscope}%
\pgfpathrectangle{\pgfqpoint{0.380943in}{9.960189in}}{\pgfqpoint{4.650000in}{0.614151in}}%
\pgfusepath{clip}%
\pgfsetbuttcap%
\pgfsetroundjoin%
\definecolor{currentfill}{rgb}{0.997924,0.685352,0.570242}%
\pgfsetfillcolor{currentfill}%
\pgfsetlinewidth{0.250937pt}%
\definecolor{currentstroke}{rgb}{1.000000,1.000000,1.000000}%
\pgfsetstrokecolor{currentstroke}%
\pgfsetdash{}{0pt}%
\pgfpathmoveto{\pgfqpoint{4.504528in}{10.223396in}}%
\pgfpathlineto{\pgfqpoint{4.592264in}{10.223396in}}%
\pgfpathlineto{\pgfqpoint{4.592264in}{10.135661in}}%
\pgfpathlineto{\pgfqpoint{4.504528in}{10.135661in}}%
\pgfpathlineto{\pgfqpoint{4.504528in}{10.223396in}}%
\pgfusepath{stroke,fill}%
\end{pgfscope}%
\begin{pgfscope}%
\pgfpathrectangle{\pgfqpoint{0.380943in}{9.960189in}}{\pgfqpoint{4.650000in}{0.614151in}}%
\pgfusepath{clip}%
\pgfsetbuttcap%
\pgfsetroundjoin%
\definecolor{currentfill}{rgb}{0.975087,0.857901,0.686044}%
\pgfsetfillcolor{currentfill}%
\pgfsetlinewidth{0.250937pt}%
\definecolor{currentstroke}{rgb}{1.000000,1.000000,1.000000}%
\pgfsetstrokecolor{currentstroke}%
\pgfsetdash{}{0pt}%
\pgfpathmoveto{\pgfqpoint{4.592264in}{10.223396in}}%
\pgfpathlineto{\pgfqpoint{4.680000in}{10.223396in}}%
\pgfpathlineto{\pgfqpoint{4.680000in}{10.135661in}}%
\pgfpathlineto{\pgfqpoint{4.592264in}{10.135661in}}%
\pgfpathlineto{\pgfqpoint{4.592264in}{10.223396in}}%
\pgfusepath{stroke,fill}%
\end{pgfscope}%
\begin{pgfscope}%
\pgfpathrectangle{\pgfqpoint{0.380943in}{9.960189in}}{\pgfqpoint{4.650000in}{0.614151in}}%
\pgfusepath{clip}%
\pgfsetbuttcap%
\pgfsetroundjoin%
\definecolor{currentfill}{rgb}{0.989619,0.788235,0.628374}%
\pgfsetfillcolor{currentfill}%
\pgfsetlinewidth{0.250937pt}%
\definecolor{currentstroke}{rgb}{1.000000,1.000000,1.000000}%
\pgfsetstrokecolor{currentstroke}%
\pgfsetdash{}{0pt}%
\pgfpathmoveto{\pgfqpoint{4.680000in}{10.223396in}}%
\pgfpathlineto{\pgfqpoint{4.767736in}{10.223396in}}%
\pgfpathlineto{\pgfqpoint{4.767736in}{10.135661in}}%
\pgfpathlineto{\pgfqpoint{4.680000in}{10.135661in}}%
\pgfpathlineto{\pgfqpoint{4.680000in}{10.223396in}}%
\pgfusepath{stroke,fill}%
\end{pgfscope}%
\begin{pgfscope}%
\pgfpathrectangle{\pgfqpoint{0.380943in}{9.960189in}}{\pgfqpoint{4.650000in}{0.614151in}}%
\pgfusepath{clip}%
\pgfsetbuttcap%
\pgfsetroundjoin%
\definecolor{currentfill}{rgb}{1.000000,0.522261,0.496886}%
\pgfsetfillcolor{currentfill}%
\pgfsetlinewidth{0.250937pt}%
\definecolor{currentstroke}{rgb}{1.000000,1.000000,1.000000}%
\pgfsetstrokecolor{currentstroke}%
\pgfsetdash{}{0pt}%
\pgfpathmoveto{\pgfqpoint{4.767736in}{10.223396in}}%
\pgfpathlineto{\pgfqpoint{4.855471in}{10.223396in}}%
\pgfpathlineto{\pgfqpoint{4.855471in}{10.135661in}}%
\pgfpathlineto{\pgfqpoint{4.767736in}{10.135661in}}%
\pgfpathlineto{\pgfqpoint{4.767736in}{10.223396in}}%
\pgfusepath{stroke,fill}%
\end{pgfscope}%
\begin{pgfscope}%
\pgfpathrectangle{\pgfqpoint{0.380943in}{9.960189in}}{\pgfqpoint{4.650000in}{0.614151in}}%
\pgfusepath{clip}%
\pgfsetbuttcap%
\pgfsetroundjoin%
\definecolor{currentfill}{rgb}{0.994694,0.745098,0.602999}%
\pgfsetfillcolor{currentfill}%
\pgfsetlinewidth{0.250937pt}%
\definecolor{currentstroke}{rgb}{1.000000,1.000000,1.000000}%
\pgfsetstrokecolor{currentstroke}%
\pgfsetdash{}{0pt}%
\pgfpathmoveto{\pgfqpoint{4.855471in}{10.223396in}}%
\pgfpathlineto{\pgfqpoint{4.943207in}{10.223396in}}%
\pgfpathlineto{\pgfqpoint{4.943207in}{10.135661in}}%
\pgfpathlineto{\pgfqpoint{4.855471in}{10.135661in}}%
\pgfpathlineto{\pgfqpoint{4.855471in}{10.223396in}}%
\pgfusepath{stroke,fill}%
\end{pgfscope}%
\begin{pgfscope}%
\pgfpathrectangle{\pgfqpoint{0.380943in}{9.960189in}}{\pgfqpoint{4.650000in}{0.614151in}}%
\pgfusepath{clip}%
\pgfsetbuttcap%
\pgfsetroundjoin%
\definecolor{currentfill}{rgb}{0.975087,0.857901,0.686044}%
\pgfsetfillcolor{currentfill}%
\pgfsetlinewidth{0.250937pt}%
\definecolor{currentstroke}{rgb}{1.000000,1.000000,1.000000}%
\pgfsetstrokecolor{currentstroke}%
\pgfsetdash{}{0pt}%
\pgfpathmoveto{\pgfqpoint{4.943207in}{10.223396in}}%
\pgfpathlineto{\pgfqpoint{5.030943in}{10.223396in}}%
\pgfpathlineto{\pgfqpoint{5.030943in}{10.135661in}}%
\pgfpathlineto{\pgfqpoint{4.943207in}{10.135661in}}%
\pgfpathlineto{\pgfqpoint{4.943207in}{10.223396in}}%
\pgfusepath{stroke,fill}%
\end{pgfscope}%
\begin{pgfscope}%
\pgfpathrectangle{\pgfqpoint{0.380943in}{9.960189in}}{\pgfqpoint{4.650000in}{0.614151in}}%
\pgfusepath{clip}%
\pgfsetbuttcap%
\pgfsetroundjoin%
\definecolor{currentfill}{rgb}{1.000000,1.000000,0.929412}%
\pgfsetfillcolor{currentfill}%
\pgfsetlinewidth{0.250937pt}%
\definecolor{currentstroke}{rgb}{1.000000,1.000000,1.000000}%
\pgfsetstrokecolor{currentstroke}%
\pgfsetdash{}{0pt}%
\pgfpathmoveto{\pgfqpoint{0.380943in}{10.135661in}}%
\pgfpathlineto{\pgfqpoint{0.468679in}{10.135661in}}%
\pgfpathlineto{\pgfqpoint{0.468679in}{10.047925in}}%
\pgfpathlineto{\pgfqpoint{0.380943in}{10.047925in}}%
\pgfpathlineto{\pgfqpoint{0.380943in}{10.135661in}}%
\pgfusepath{stroke,fill}%
\end{pgfscope}%
\begin{pgfscope}%
\pgfpathrectangle{\pgfqpoint{0.380943in}{9.960189in}}{\pgfqpoint{4.650000in}{0.614151in}}%
\pgfusepath{clip}%
\pgfsetbuttcap%
\pgfsetroundjoin%
\definecolor{currentfill}{rgb}{1.000000,1.000000,0.929412}%
\pgfsetfillcolor{currentfill}%
\pgfsetlinewidth{0.250937pt}%
\definecolor{currentstroke}{rgb}{1.000000,1.000000,1.000000}%
\pgfsetstrokecolor{currentstroke}%
\pgfsetdash{}{0pt}%
\pgfpathmoveto{\pgfqpoint{0.468679in}{10.135661in}}%
\pgfpathlineto{\pgfqpoint{0.556415in}{10.135661in}}%
\pgfpathlineto{\pgfqpoint{0.556415in}{10.047925in}}%
\pgfpathlineto{\pgfqpoint{0.468679in}{10.047925in}}%
\pgfpathlineto{\pgfqpoint{0.468679in}{10.135661in}}%
\pgfusepath{stroke,fill}%
\end{pgfscope}%
\begin{pgfscope}%
\pgfpathrectangle{\pgfqpoint{0.380943in}{9.960189in}}{\pgfqpoint{4.650000in}{0.614151in}}%
\pgfusepath{clip}%
\pgfsetbuttcap%
\pgfsetroundjoin%
\definecolor{currentfill}{rgb}{1.000000,1.000000,0.929412}%
\pgfsetfillcolor{currentfill}%
\pgfsetlinewidth{0.250937pt}%
\definecolor{currentstroke}{rgb}{1.000000,1.000000,1.000000}%
\pgfsetstrokecolor{currentstroke}%
\pgfsetdash{}{0pt}%
\pgfpathmoveto{\pgfqpoint{0.556415in}{10.135661in}}%
\pgfpathlineto{\pgfqpoint{0.644151in}{10.135661in}}%
\pgfpathlineto{\pgfqpoint{0.644151in}{10.047925in}}%
\pgfpathlineto{\pgfqpoint{0.556415in}{10.047925in}}%
\pgfpathlineto{\pgfqpoint{0.556415in}{10.135661in}}%
\pgfusepath{stroke,fill}%
\end{pgfscope}%
\begin{pgfscope}%
\pgfpathrectangle{\pgfqpoint{0.380943in}{9.960189in}}{\pgfqpoint{4.650000in}{0.614151in}}%
\pgfusepath{clip}%
\pgfsetbuttcap%
\pgfsetroundjoin%
\definecolor{currentfill}{rgb}{1.000000,1.000000,0.929412}%
\pgfsetfillcolor{currentfill}%
\pgfsetlinewidth{0.250937pt}%
\definecolor{currentstroke}{rgb}{1.000000,1.000000,1.000000}%
\pgfsetstrokecolor{currentstroke}%
\pgfsetdash{}{0pt}%
\pgfpathmoveto{\pgfqpoint{0.644151in}{10.135661in}}%
\pgfpathlineto{\pgfqpoint{0.731886in}{10.135661in}}%
\pgfpathlineto{\pgfqpoint{0.731886in}{10.047925in}}%
\pgfpathlineto{\pgfqpoint{0.644151in}{10.047925in}}%
\pgfpathlineto{\pgfqpoint{0.644151in}{10.135661in}}%
\pgfusepath{stroke,fill}%
\end{pgfscope}%
\begin{pgfscope}%
\pgfpathrectangle{\pgfqpoint{0.380943in}{9.960189in}}{\pgfqpoint{4.650000in}{0.614151in}}%
\pgfusepath{clip}%
\pgfsetbuttcap%
\pgfsetroundjoin%
\definecolor{currentfill}{rgb}{1.000000,1.000000,0.929412}%
\pgfsetfillcolor{currentfill}%
\pgfsetlinewidth{0.250937pt}%
\definecolor{currentstroke}{rgb}{1.000000,1.000000,1.000000}%
\pgfsetstrokecolor{currentstroke}%
\pgfsetdash{}{0pt}%
\pgfpathmoveto{\pgfqpoint{0.731886in}{10.135661in}}%
\pgfpathlineto{\pgfqpoint{0.819622in}{10.135661in}}%
\pgfpathlineto{\pgfqpoint{0.819622in}{10.047925in}}%
\pgfpathlineto{\pgfqpoint{0.731886in}{10.047925in}}%
\pgfpathlineto{\pgfqpoint{0.731886in}{10.135661in}}%
\pgfusepath{stroke,fill}%
\end{pgfscope}%
\begin{pgfscope}%
\pgfpathrectangle{\pgfqpoint{0.380943in}{9.960189in}}{\pgfqpoint{4.650000in}{0.614151in}}%
\pgfusepath{clip}%
\pgfsetbuttcap%
\pgfsetroundjoin%
\definecolor{currentfill}{rgb}{1.000000,1.000000,0.929412}%
\pgfsetfillcolor{currentfill}%
\pgfsetlinewidth{0.250937pt}%
\definecolor{currentstroke}{rgb}{1.000000,1.000000,1.000000}%
\pgfsetstrokecolor{currentstroke}%
\pgfsetdash{}{0pt}%
\pgfpathmoveto{\pgfqpoint{0.819622in}{10.135661in}}%
\pgfpathlineto{\pgfqpoint{0.907358in}{10.135661in}}%
\pgfpathlineto{\pgfqpoint{0.907358in}{10.047925in}}%
\pgfpathlineto{\pgfqpoint{0.819622in}{10.047925in}}%
\pgfpathlineto{\pgfqpoint{0.819622in}{10.135661in}}%
\pgfusepath{stroke,fill}%
\end{pgfscope}%
\begin{pgfscope}%
\pgfpathrectangle{\pgfqpoint{0.380943in}{9.960189in}}{\pgfqpoint{4.650000in}{0.614151in}}%
\pgfusepath{clip}%
\pgfsetbuttcap%
\pgfsetroundjoin%
\definecolor{currentfill}{rgb}{1.000000,1.000000,0.929412}%
\pgfsetfillcolor{currentfill}%
\pgfsetlinewidth{0.250937pt}%
\definecolor{currentstroke}{rgb}{1.000000,1.000000,1.000000}%
\pgfsetstrokecolor{currentstroke}%
\pgfsetdash{}{0pt}%
\pgfpathmoveto{\pgfqpoint{0.907358in}{10.135661in}}%
\pgfpathlineto{\pgfqpoint{0.995094in}{10.135661in}}%
\pgfpathlineto{\pgfqpoint{0.995094in}{10.047925in}}%
\pgfpathlineto{\pgfqpoint{0.907358in}{10.047925in}}%
\pgfpathlineto{\pgfqpoint{0.907358in}{10.135661in}}%
\pgfusepath{stroke,fill}%
\end{pgfscope}%
\begin{pgfscope}%
\pgfpathrectangle{\pgfqpoint{0.380943in}{9.960189in}}{\pgfqpoint{4.650000in}{0.614151in}}%
\pgfusepath{clip}%
\pgfsetbuttcap%
\pgfsetroundjoin%
\definecolor{currentfill}{rgb}{1.000000,1.000000,0.929412}%
\pgfsetfillcolor{currentfill}%
\pgfsetlinewidth{0.250937pt}%
\definecolor{currentstroke}{rgb}{1.000000,1.000000,1.000000}%
\pgfsetstrokecolor{currentstroke}%
\pgfsetdash{}{0pt}%
\pgfpathmoveto{\pgfqpoint{0.995094in}{10.135661in}}%
\pgfpathlineto{\pgfqpoint{1.082830in}{10.135661in}}%
\pgfpathlineto{\pgfqpoint{1.082830in}{10.047925in}}%
\pgfpathlineto{\pgfqpoint{0.995094in}{10.047925in}}%
\pgfpathlineto{\pgfqpoint{0.995094in}{10.135661in}}%
\pgfusepath{stroke,fill}%
\end{pgfscope}%
\begin{pgfscope}%
\pgfpathrectangle{\pgfqpoint{0.380943in}{9.960189in}}{\pgfqpoint{4.650000in}{0.614151in}}%
\pgfusepath{clip}%
\pgfsetbuttcap%
\pgfsetroundjoin%
\definecolor{currentfill}{rgb}{1.000000,1.000000,0.929412}%
\pgfsetfillcolor{currentfill}%
\pgfsetlinewidth{0.250937pt}%
\definecolor{currentstroke}{rgb}{1.000000,1.000000,1.000000}%
\pgfsetstrokecolor{currentstroke}%
\pgfsetdash{}{0pt}%
\pgfpathmoveto{\pgfqpoint{1.082830in}{10.135661in}}%
\pgfpathlineto{\pgfqpoint{1.170566in}{10.135661in}}%
\pgfpathlineto{\pgfqpoint{1.170566in}{10.047925in}}%
\pgfpathlineto{\pgfqpoint{1.082830in}{10.047925in}}%
\pgfpathlineto{\pgfqpoint{1.082830in}{10.135661in}}%
\pgfusepath{stroke,fill}%
\end{pgfscope}%
\begin{pgfscope}%
\pgfpathrectangle{\pgfqpoint{0.380943in}{9.960189in}}{\pgfqpoint{4.650000in}{0.614151in}}%
\pgfusepath{clip}%
\pgfsetbuttcap%
\pgfsetroundjoin%
\definecolor{currentfill}{rgb}{1.000000,1.000000,0.929412}%
\pgfsetfillcolor{currentfill}%
\pgfsetlinewidth{0.250937pt}%
\definecolor{currentstroke}{rgb}{1.000000,1.000000,1.000000}%
\pgfsetstrokecolor{currentstroke}%
\pgfsetdash{}{0pt}%
\pgfpathmoveto{\pgfqpoint{1.170566in}{10.135661in}}%
\pgfpathlineto{\pgfqpoint{1.258302in}{10.135661in}}%
\pgfpathlineto{\pgfqpoint{1.258302in}{10.047925in}}%
\pgfpathlineto{\pgfqpoint{1.170566in}{10.047925in}}%
\pgfpathlineto{\pgfqpoint{1.170566in}{10.135661in}}%
\pgfusepath{stroke,fill}%
\end{pgfscope}%
\begin{pgfscope}%
\pgfpathrectangle{\pgfqpoint{0.380943in}{9.960189in}}{\pgfqpoint{4.650000in}{0.614151in}}%
\pgfusepath{clip}%
\pgfsetbuttcap%
\pgfsetroundjoin%
\definecolor{currentfill}{rgb}{1.000000,1.000000,0.929412}%
\pgfsetfillcolor{currentfill}%
\pgfsetlinewidth{0.250937pt}%
\definecolor{currentstroke}{rgb}{1.000000,1.000000,1.000000}%
\pgfsetstrokecolor{currentstroke}%
\pgfsetdash{}{0pt}%
\pgfpathmoveto{\pgfqpoint{1.258302in}{10.135661in}}%
\pgfpathlineto{\pgfqpoint{1.346037in}{10.135661in}}%
\pgfpathlineto{\pgfqpoint{1.346037in}{10.047925in}}%
\pgfpathlineto{\pgfqpoint{1.258302in}{10.047925in}}%
\pgfpathlineto{\pgfqpoint{1.258302in}{10.135661in}}%
\pgfusepath{stroke,fill}%
\end{pgfscope}%
\begin{pgfscope}%
\pgfpathrectangle{\pgfqpoint{0.380943in}{9.960189in}}{\pgfqpoint{4.650000in}{0.614151in}}%
\pgfusepath{clip}%
\pgfsetbuttcap%
\pgfsetroundjoin%
\definecolor{currentfill}{rgb}{1.000000,1.000000,0.929412}%
\pgfsetfillcolor{currentfill}%
\pgfsetlinewidth{0.250937pt}%
\definecolor{currentstroke}{rgb}{1.000000,1.000000,1.000000}%
\pgfsetstrokecolor{currentstroke}%
\pgfsetdash{}{0pt}%
\pgfpathmoveto{\pgfqpoint{1.346037in}{10.135661in}}%
\pgfpathlineto{\pgfqpoint{1.433773in}{10.135661in}}%
\pgfpathlineto{\pgfqpoint{1.433773in}{10.047925in}}%
\pgfpathlineto{\pgfqpoint{1.346037in}{10.047925in}}%
\pgfpathlineto{\pgfqpoint{1.346037in}{10.135661in}}%
\pgfusepath{stroke,fill}%
\end{pgfscope}%
\begin{pgfscope}%
\pgfpathrectangle{\pgfqpoint{0.380943in}{9.960189in}}{\pgfqpoint{4.650000in}{0.614151in}}%
\pgfusepath{clip}%
\pgfsetbuttcap%
\pgfsetroundjoin%
\definecolor{currentfill}{rgb}{1.000000,1.000000,0.929412}%
\pgfsetfillcolor{currentfill}%
\pgfsetlinewidth{0.250937pt}%
\definecolor{currentstroke}{rgb}{1.000000,1.000000,1.000000}%
\pgfsetstrokecolor{currentstroke}%
\pgfsetdash{}{0pt}%
\pgfpathmoveto{\pgfqpoint{1.433773in}{10.135661in}}%
\pgfpathlineto{\pgfqpoint{1.521509in}{10.135661in}}%
\pgfpathlineto{\pgfqpoint{1.521509in}{10.047925in}}%
\pgfpathlineto{\pgfqpoint{1.433773in}{10.047925in}}%
\pgfpathlineto{\pgfqpoint{1.433773in}{10.135661in}}%
\pgfusepath{stroke,fill}%
\end{pgfscope}%
\begin{pgfscope}%
\pgfpathrectangle{\pgfqpoint{0.380943in}{9.960189in}}{\pgfqpoint{4.650000in}{0.614151in}}%
\pgfusepath{clip}%
\pgfsetbuttcap%
\pgfsetroundjoin%
\definecolor{currentfill}{rgb}{1.000000,1.000000,0.929412}%
\pgfsetfillcolor{currentfill}%
\pgfsetlinewidth{0.250937pt}%
\definecolor{currentstroke}{rgb}{1.000000,1.000000,1.000000}%
\pgfsetstrokecolor{currentstroke}%
\pgfsetdash{}{0pt}%
\pgfpathmoveto{\pgfqpoint{1.521509in}{10.135661in}}%
\pgfpathlineto{\pgfqpoint{1.609245in}{10.135661in}}%
\pgfpathlineto{\pgfqpoint{1.609245in}{10.047925in}}%
\pgfpathlineto{\pgfqpoint{1.521509in}{10.047925in}}%
\pgfpathlineto{\pgfqpoint{1.521509in}{10.135661in}}%
\pgfusepath{stroke,fill}%
\end{pgfscope}%
\begin{pgfscope}%
\pgfpathrectangle{\pgfqpoint{0.380943in}{9.960189in}}{\pgfqpoint{4.650000in}{0.614151in}}%
\pgfusepath{clip}%
\pgfsetbuttcap%
\pgfsetroundjoin%
\definecolor{currentfill}{rgb}{1.000000,1.000000,0.929412}%
\pgfsetfillcolor{currentfill}%
\pgfsetlinewidth{0.250937pt}%
\definecolor{currentstroke}{rgb}{1.000000,1.000000,1.000000}%
\pgfsetstrokecolor{currentstroke}%
\pgfsetdash{}{0pt}%
\pgfpathmoveto{\pgfqpoint{1.609245in}{10.135661in}}%
\pgfpathlineto{\pgfqpoint{1.696981in}{10.135661in}}%
\pgfpathlineto{\pgfqpoint{1.696981in}{10.047925in}}%
\pgfpathlineto{\pgfqpoint{1.609245in}{10.047925in}}%
\pgfpathlineto{\pgfqpoint{1.609245in}{10.135661in}}%
\pgfusepath{stroke,fill}%
\end{pgfscope}%
\begin{pgfscope}%
\pgfpathrectangle{\pgfqpoint{0.380943in}{9.960189in}}{\pgfqpoint{4.650000in}{0.614151in}}%
\pgfusepath{clip}%
\pgfsetbuttcap%
\pgfsetroundjoin%
\definecolor{currentfill}{rgb}{1.000000,1.000000,0.929412}%
\pgfsetfillcolor{currentfill}%
\pgfsetlinewidth{0.250937pt}%
\definecolor{currentstroke}{rgb}{1.000000,1.000000,1.000000}%
\pgfsetstrokecolor{currentstroke}%
\pgfsetdash{}{0pt}%
\pgfpathmoveto{\pgfqpoint{1.696981in}{10.135661in}}%
\pgfpathlineto{\pgfqpoint{1.784717in}{10.135661in}}%
\pgfpathlineto{\pgfqpoint{1.784717in}{10.047925in}}%
\pgfpathlineto{\pgfqpoint{1.696981in}{10.047925in}}%
\pgfpathlineto{\pgfqpoint{1.696981in}{10.135661in}}%
\pgfusepath{stroke,fill}%
\end{pgfscope}%
\begin{pgfscope}%
\pgfpathrectangle{\pgfqpoint{0.380943in}{9.960189in}}{\pgfqpoint{4.650000in}{0.614151in}}%
\pgfusepath{clip}%
\pgfsetbuttcap%
\pgfsetroundjoin%
\definecolor{currentfill}{rgb}{1.000000,1.000000,0.929412}%
\pgfsetfillcolor{currentfill}%
\pgfsetlinewidth{0.250937pt}%
\definecolor{currentstroke}{rgb}{1.000000,1.000000,1.000000}%
\pgfsetstrokecolor{currentstroke}%
\pgfsetdash{}{0pt}%
\pgfpathmoveto{\pgfqpoint{1.784717in}{10.135661in}}%
\pgfpathlineto{\pgfqpoint{1.872452in}{10.135661in}}%
\pgfpathlineto{\pgfqpoint{1.872452in}{10.047925in}}%
\pgfpathlineto{\pgfqpoint{1.784717in}{10.047925in}}%
\pgfpathlineto{\pgfqpoint{1.784717in}{10.135661in}}%
\pgfusepath{stroke,fill}%
\end{pgfscope}%
\begin{pgfscope}%
\pgfpathrectangle{\pgfqpoint{0.380943in}{9.960189in}}{\pgfqpoint{4.650000in}{0.614151in}}%
\pgfusepath{clip}%
\pgfsetbuttcap%
\pgfsetroundjoin%
\definecolor{currentfill}{rgb}{1.000000,1.000000,0.929412}%
\pgfsetfillcolor{currentfill}%
\pgfsetlinewidth{0.250937pt}%
\definecolor{currentstroke}{rgb}{1.000000,1.000000,1.000000}%
\pgfsetstrokecolor{currentstroke}%
\pgfsetdash{}{0pt}%
\pgfpathmoveto{\pgfqpoint{1.872452in}{10.135661in}}%
\pgfpathlineto{\pgfqpoint{1.960188in}{10.135661in}}%
\pgfpathlineto{\pgfqpoint{1.960188in}{10.047925in}}%
\pgfpathlineto{\pgfqpoint{1.872452in}{10.047925in}}%
\pgfpathlineto{\pgfqpoint{1.872452in}{10.135661in}}%
\pgfusepath{stroke,fill}%
\end{pgfscope}%
\begin{pgfscope}%
\pgfpathrectangle{\pgfqpoint{0.380943in}{9.960189in}}{\pgfqpoint{4.650000in}{0.614151in}}%
\pgfusepath{clip}%
\pgfsetbuttcap%
\pgfsetroundjoin%
\definecolor{currentfill}{rgb}{1.000000,1.000000,0.929412}%
\pgfsetfillcolor{currentfill}%
\pgfsetlinewidth{0.250937pt}%
\definecolor{currentstroke}{rgb}{1.000000,1.000000,1.000000}%
\pgfsetstrokecolor{currentstroke}%
\pgfsetdash{}{0pt}%
\pgfpathmoveto{\pgfqpoint{1.960188in}{10.135661in}}%
\pgfpathlineto{\pgfqpoint{2.047924in}{10.135661in}}%
\pgfpathlineto{\pgfqpoint{2.047924in}{10.047925in}}%
\pgfpathlineto{\pgfqpoint{1.960188in}{10.047925in}}%
\pgfpathlineto{\pgfqpoint{1.960188in}{10.135661in}}%
\pgfusepath{stroke,fill}%
\end{pgfscope}%
\begin{pgfscope}%
\pgfpathrectangle{\pgfqpoint{0.380943in}{9.960189in}}{\pgfqpoint{4.650000in}{0.614151in}}%
\pgfusepath{clip}%
\pgfsetbuttcap%
\pgfsetroundjoin%
\definecolor{currentfill}{rgb}{1.000000,1.000000,0.929412}%
\pgfsetfillcolor{currentfill}%
\pgfsetlinewidth{0.250937pt}%
\definecolor{currentstroke}{rgb}{1.000000,1.000000,1.000000}%
\pgfsetstrokecolor{currentstroke}%
\pgfsetdash{}{0pt}%
\pgfpathmoveto{\pgfqpoint{2.047924in}{10.135661in}}%
\pgfpathlineto{\pgfqpoint{2.135660in}{10.135661in}}%
\pgfpathlineto{\pgfqpoint{2.135660in}{10.047925in}}%
\pgfpathlineto{\pgfqpoint{2.047924in}{10.047925in}}%
\pgfpathlineto{\pgfqpoint{2.047924in}{10.135661in}}%
\pgfusepath{stroke,fill}%
\end{pgfscope}%
\begin{pgfscope}%
\pgfpathrectangle{\pgfqpoint{0.380943in}{9.960189in}}{\pgfqpoint{4.650000in}{0.614151in}}%
\pgfusepath{clip}%
\pgfsetbuttcap%
\pgfsetroundjoin%
\definecolor{currentfill}{rgb}{1.000000,1.000000,0.929412}%
\pgfsetfillcolor{currentfill}%
\pgfsetlinewidth{0.250937pt}%
\definecolor{currentstroke}{rgb}{1.000000,1.000000,1.000000}%
\pgfsetstrokecolor{currentstroke}%
\pgfsetdash{}{0pt}%
\pgfpathmoveto{\pgfqpoint{2.135660in}{10.135661in}}%
\pgfpathlineto{\pgfqpoint{2.223396in}{10.135661in}}%
\pgfpathlineto{\pgfqpoint{2.223396in}{10.047925in}}%
\pgfpathlineto{\pgfqpoint{2.135660in}{10.047925in}}%
\pgfpathlineto{\pgfqpoint{2.135660in}{10.135661in}}%
\pgfusepath{stroke,fill}%
\end{pgfscope}%
\begin{pgfscope}%
\pgfpathrectangle{\pgfqpoint{0.380943in}{9.960189in}}{\pgfqpoint{4.650000in}{0.614151in}}%
\pgfusepath{clip}%
\pgfsetbuttcap%
\pgfsetroundjoin%
\definecolor{currentfill}{rgb}{1.000000,1.000000,0.929412}%
\pgfsetfillcolor{currentfill}%
\pgfsetlinewidth{0.250937pt}%
\definecolor{currentstroke}{rgb}{1.000000,1.000000,1.000000}%
\pgfsetstrokecolor{currentstroke}%
\pgfsetdash{}{0pt}%
\pgfpathmoveto{\pgfqpoint{2.223396in}{10.135661in}}%
\pgfpathlineto{\pgfqpoint{2.311132in}{10.135661in}}%
\pgfpathlineto{\pgfqpoint{2.311132in}{10.047925in}}%
\pgfpathlineto{\pgfqpoint{2.223396in}{10.047925in}}%
\pgfpathlineto{\pgfqpoint{2.223396in}{10.135661in}}%
\pgfusepath{stroke,fill}%
\end{pgfscope}%
\begin{pgfscope}%
\pgfpathrectangle{\pgfqpoint{0.380943in}{9.960189in}}{\pgfqpoint{4.650000in}{0.614151in}}%
\pgfusepath{clip}%
\pgfsetbuttcap%
\pgfsetroundjoin%
\definecolor{currentfill}{rgb}{1.000000,1.000000,0.929412}%
\pgfsetfillcolor{currentfill}%
\pgfsetlinewidth{0.250937pt}%
\definecolor{currentstroke}{rgb}{1.000000,1.000000,1.000000}%
\pgfsetstrokecolor{currentstroke}%
\pgfsetdash{}{0pt}%
\pgfpathmoveto{\pgfqpoint{2.311132in}{10.135661in}}%
\pgfpathlineto{\pgfqpoint{2.398868in}{10.135661in}}%
\pgfpathlineto{\pgfqpoint{2.398868in}{10.047925in}}%
\pgfpathlineto{\pgfqpoint{2.311132in}{10.047925in}}%
\pgfpathlineto{\pgfqpoint{2.311132in}{10.135661in}}%
\pgfusepath{stroke,fill}%
\end{pgfscope}%
\begin{pgfscope}%
\pgfpathrectangle{\pgfqpoint{0.380943in}{9.960189in}}{\pgfqpoint{4.650000in}{0.614151in}}%
\pgfusepath{clip}%
\pgfsetbuttcap%
\pgfsetroundjoin%
\definecolor{currentfill}{rgb}{1.000000,1.000000,0.929412}%
\pgfsetfillcolor{currentfill}%
\pgfsetlinewidth{0.250937pt}%
\definecolor{currentstroke}{rgb}{1.000000,1.000000,1.000000}%
\pgfsetstrokecolor{currentstroke}%
\pgfsetdash{}{0pt}%
\pgfpathmoveto{\pgfqpoint{2.398868in}{10.135661in}}%
\pgfpathlineto{\pgfqpoint{2.486603in}{10.135661in}}%
\pgfpathlineto{\pgfqpoint{2.486603in}{10.047925in}}%
\pgfpathlineto{\pgfqpoint{2.398868in}{10.047925in}}%
\pgfpathlineto{\pgfqpoint{2.398868in}{10.135661in}}%
\pgfusepath{stroke,fill}%
\end{pgfscope}%
\begin{pgfscope}%
\pgfpathrectangle{\pgfqpoint{0.380943in}{9.960189in}}{\pgfqpoint{4.650000in}{0.614151in}}%
\pgfusepath{clip}%
\pgfsetbuttcap%
\pgfsetroundjoin%
\definecolor{currentfill}{rgb}{1.000000,1.000000,0.929412}%
\pgfsetfillcolor{currentfill}%
\pgfsetlinewidth{0.250937pt}%
\definecolor{currentstroke}{rgb}{1.000000,1.000000,1.000000}%
\pgfsetstrokecolor{currentstroke}%
\pgfsetdash{}{0pt}%
\pgfpathmoveto{\pgfqpoint{2.486603in}{10.135661in}}%
\pgfpathlineto{\pgfqpoint{2.574339in}{10.135661in}}%
\pgfpathlineto{\pgfqpoint{2.574339in}{10.047925in}}%
\pgfpathlineto{\pgfqpoint{2.486603in}{10.047925in}}%
\pgfpathlineto{\pgfqpoint{2.486603in}{10.135661in}}%
\pgfusepath{stroke,fill}%
\end{pgfscope}%
\begin{pgfscope}%
\pgfpathrectangle{\pgfqpoint{0.380943in}{9.960189in}}{\pgfqpoint{4.650000in}{0.614151in}}%
\pgfusepath{clip}%
\pgfsetbuttcap%
\pgfsetroundjoin%
\definecolor{currentfill}{rgb}{1.000000,1.000000,0.929412}%
\pgfsetfillcolor{currentfill}%
\pgfsetlinewidth{0.250937pt}%
\definecolor{currentstroke}{rgb}{1.000000,1.000000,1.000000}%
\pgfsetstrokecolor{currentstroke}%
\pgfsetdash{}{0pt}%
\pgfpathmoveto{\pgfqpoint{2.574339in}{10.135661in}}%
\pgfpathlineto{\pgfqpoint{2.662075in}{10.135661in}}%
\pgfpathlineto{\pgfqpoint{2.662075in}{10.047925in}}%
\pgfpathlineto{\pgfqpoint{2.574339in}{10.047925in}}%
\pgfpathlineto{\pgfqpoint{2.574339in}{10.135661in}}%
\pgfusepath{stroke,fill}%
\end{pgfscope}%
\begin{pgfscope}%
\pgfpathrectangle{\pgfqpoint{0.380943in}{9.960189in}}{\pgfqpoint{4.650000in}{0.614151in}}%
\pgfusepath{clip}%
\pgfsetbuttcap%
\pgfsetroundjoin%
\definecolor{currentfill}{rgb}{1.000000,1.000000,0.929412}%
\pgfsetfillcolor{currentfill}%
\pgfsetlinewidth{0.250937pt}%
\definecolor{currentstroke}{rgb}{1.000000,1.000000,1.000000}%
\pgfsetstrokecolor{currentstroke}%
\pgfsetdash{}{0pt}%
\pgfpathmoveto{\pgfqpoint{2.662075in}{10.135661in}}%
\pgfpathlineto{\pgfqpoint{2.749811in}{10.135661in}}%
\pgfpathlineto{\pgfqpoint{2.749811in}{10.047925in}}%
\pgfpathlineto{\pgfqpoint{2.662075in}{10.047925in}}%
\pgfpathlineto{\pgfqpoint{2.662075in}{10.135661in}}%
\pgfusepath{stroke,fill}%
\end{pgfscope}%
\begin{pgfscope}%
\pgfpathrectangle{\pgfqpoint{0.380943in}{9.960189in}}{\pgfqpoint{4.650000in}{0.614151in}}%
\pgfusepath{clip}%
\pgfsetbuttcap%
\pgfsetroundjoin%
\definecolor{currentfill}{rgb}{1.000000,1.000000,0.929412}%
\pgfsetfillcolor{currentfill}%
\pgfsetlinewidth{0.250937pt}%
\definecolor{currentstroke}{rgb}{1.000000,1.000000,1.000000}%
\pgfsetstrokecolor{currentstroke}%
\pgfsetdash{}{0pt}%
\pgfpathmoveto{\pgfqpoint{2.749811in}{10.135661in}}%
\pgfpathlineto{\pgfqpoint{2.837547in}{10.135661in}}%
\pgfpathlineto{\pgfqpoint{2.837547in}{10.047925in}}%
\pgfpathlineto{\pgfqpoint{2.749811in}{10.047925in}}%
\pgfpathlineto{\pgfqpoint{2.749811in}{10.135661in}}%
\pgfusepath{stroke,fill}%
\end{pgfscope}%
\begin{pgfscope}%
\pgfpathrectangle{\pgfqpoint{0.380943in}{9.960189in}}{\pgfqpoint{4.650000in}{0.614151in}}%
\pgfusepath{clip}%
\pgfsetbuttcap%
\pgfsetroundjoin%
\definecolor{currentfill}{rgb}{1.000000,1.000000,0.929412}%
\pgfsetfillcolor{currentfill}%
\pgfsetlinewidth{0.250937pt}%
\definecolor{currentstroke}{rgb}{1.000000,1.000000,1.000000}%
\pgfsetstrokecolor{currentstroke}%
\pgfsetdash{}{0pt}%
\pgfpathmoveto{\pgfqpoint{2.837547in}{10.135661in}}%
\pgfpathlineto{\pgfqpoint{2.925283in}{10.135661in}}%
\pgfpathlineto{\pgfqpoint{2.925283in}{10.047925in}}%
\pgfpathlineto{\pgfqpoint{2.837547in}{10.047925in}}%
\pgfpathlineto{\pgfqpoint{2.837547in}{10.135661in}}%
\pgfusepath{stroke,fill}%
\end{pgfscope}%
\begin{pgfscope}%
\pgfpathrectangle{\pgfqpoint{0.380943in}{9.960189in}}{\pgfqpoint{4.650000in}{0.614151in}}%
\pgfusepath{clip}%
\pgfsetbuttcap%
\pgfsetroundjoin%
\definecolor{currentfill}{rgb}{1.000000,1.000000,0.929412}%
\pgfsetfillcolor{currentfill}%
\pgfsetlinewidth{0.250937pt}%
\definecolor{currentstroke}{rgb}{1.000000,1.000000,1.000000}%
\pgfsetstrokecolor{currentstroke}%
\pgfsetdash{}{0pt}%
\pgfpathmoveto{\pgfqpoint{2.925283in}{10.135661in}}%
\pgfpathlineto{\pgfqpoint{3.013019in}{10.135661in}}%
\pgfpathlineto{\pgfqpoint{3.013019in}{10.047925in}}%
\pgfpathlineto{\pgfqpoint{2.925283in}{10.047925in}}%
\pgfpathlineto{\pgfqpoint{2.925283in}{10.135661in}}%
\pgfusepath{stroke,fill}%
\end{pgfscope}%
\begin{pgfscope}%
\pgfpathrectangle{\pgfqpoint{0.380943in}{9.960189in}}{\pgfqpoint{4.650000in}{0.614151in}}%
\pgfusepath{clip}%
\pgfsetbuttcap%
\pgfsetroundjoin%
\definecolor{currentfill}{rgb}{1.000000,1.000000,0.929412}%
\pgfsetfillcolor{currentfill}%
\pgfsetlinewidth{0.250937pt}%
\definecolor{currentstroke}{rgb}{1.000000,1.000000,1.000000}%
\pgfsetstrokecolor{currentstroke}%
\pgfsetdash{}{0pt}%
\pgfpathmoveto{\pgfqpoint{3.013019in}{10.135661in}}%
\pgfpathlineto{\pgfqpoint{3.100754in}{10.135661in}}%
\pgfpathlineto{\pgfqpoint{3.100754in}{10.047925in}}%
\pgfpathlineto{\pgfqpoint{3.013019in}{10.047925in}}%
\pgfpathlineto{\pgfqpoint{3.013019in}{10.135661in}}%
\pgfusepath{stroke,fill}%
\end{pgfscope}%
\begin{pgfscope}%
\pgfpathrectangle{\pgfqpoint{0.380943in}{9.960189in}}{\pgfqpoint{4.650000in}{0.614151in}}%
\pgfusepath{clip}%
\pgfsetbuttcap%
\pgfsetroundjoin%
\definecolor{currentfill}{rgb}{1.000000,1.000000,0.929412}%
\pgfsetfillcolor{currentfill}%
\pgfsetlinewidth{0.250937pt}%
\definecolor{currentstroke}{rgb}{1.000000,1.000000,1.000000}%
\pgfsetstrokecolor{currentstroke}%
\pgfsetdash{}{0pt}%
\pgfpathmoveto{\pgfqpoint{3.100754in}{10.135661in}}%
\pgfpathlineto{\pgfqpoint{3.188490in}{10.135661in}}%
\pgfpathlineto{\pgfqpoint{3.188490in}{10.047925in}}%
\pgfpathlineto{\pgfqpoint{3.100754in}{10.047925in}}%
\pgfpathlineto{\pgfqpoint{3.100754in}{10.135661in}}%
\pgfusepath{stroke,fill}%
\end{pgfscope}%
\begin{pgfscope}%
\pgfpathrectangle{\pgfqpoint{0.380943in}{9.960189in}}{\pgfqpoint{4.650000in}{0.614151in}}%
\pgfusepath{clip}%
\pgfsetbuttcap%
\pgfsetroundjoin%
\definecolor{currentfill}{rgb}{1.000000,1.000000,0.929412}%
\pgfsetfillcolor{currentfill}%
\pgfsetlinewidth{0.250937pt}%
\definecolor{currentstroke}{rgb}{1.000000,1.000000,1.000000}%
\pgfsetstrokecolor{currentstroke}%
\pgfsetdash{}{0pt}%
\pgfpathmoveto{\pgfqpoint{3.188490in}{10.135661in}}%
\pgfpathlineto{\pgfqpoint{3.276226in}{10.135661in}}%
\pgfpathlineto{\pgfqpoint{3.276226in}{10.047925in}}%
\pgfpathlineto{\pgfqpoint{3.188490in}{10.047925in}}%
\pgfpathlineto{\pgfqpoint{3.188490in}{10.135661in}}%
\pgfusepath{stroke,fill}%
\end{pgfscope}%
\begin{pgfscope}%
\pgfpathrectangle{\pgfqpoint{0.380943in}{9.960189in}}{\pgfqpoint{4.650000in}{0.614151in}}%
\pgfusepath{clip}%
\pgfsetbuttcap%
\pgfsetroundjoin%
\definecolor{currentfill}{rgb}{1.000000,1.000000,0.929412}%
\pgfsetfillcolor{currentfill}%
\pgfsetlinewidth{0.250937pt}%
\definecolor{currentstroke}{rgb}{1.000000,1.000000,1.000000}%
\pgfsetstrokecolor{currentstroke}%
\pgfsetdash{}{0pt}%
\pgfpathmoveto{\pgfqpoint{3.276226in}{10.135661in}}%
\pgfpathlineto{\pgfqpoint{3.363962in}{10.135661in}}%
\pgfpathlineto{\pgfqpoint{3.363962in}{10.047925in}}%
\pgfpathlineto{\pgfqpoint{3.276226in}{10.047925in}}%
\pgfpathlineto{\pgfqpoint{3.276226in}{10.135661in}}%
\pgfusepath{stroke,fill}%
\end{pgfscope}%
\begin{pgfscope}%
\pgfpathrectangle{\pgfqpoint{0.380943in}{9.960189in}}{\pgfqpoint{4.650000in}{0.614151in}}%
\pgfusepath{clip}%
\pgfsetbuttcap%
\pgfsetroundjoin%
\definecolor{currentfill}{rgb}{1.000000,1.000000,0.929412}%
\pgfsetfillcolor{currentfill}%
\pgfsetlinewidth{0.250937pt}%
\definecolor{currentstroke}{rgb}{1.000000,1.000000,1.000000}%
\pgfsetstrokecolor{currentstroke}%
\pgfsetdash{}{0pt}%
\pgfpathmoveto{\pgfqpoint{3.363962in}{10.135661in}}%
\pgfpathlineto{\pgfqpoint{3.451698in}{10.135661in}}%
\pgfpathlineto{\pgfqpoint{3.451698in}{10.047925in}}%
\pgfpathlineto{\pgfqpoint{3.363962in}{10.047925in}}%
\pgfpathlineto{\pgfqpoint{3.363962in}{10.135661in}}%
\pgfusepath{stroke,fill}%
\end{pgfscope}%
\begin{pgfscope}%
\pgfpathrectangle{\pgfqpoint{0.380943in}{9.960189in}}{\pgfqpoint{4.650000in}{0.614151in}}%
\pgfusepath{clip}%
\pgfsetbuttcap%
\pgfsetroundjoin%
\definecolor{currentfill}{rgb}{1.000000,1.000000,0.929412}%
\pgfsetfillcolor{currentfill}%
\pgfsetlinewidth{0.250937pt}%
\definecolor{currentstroke}{rgb}{1.000000,1.000000,1.000000}%
\pgfsetstrokecolor{currentstroke}%
\pgfsetdash{}{0pt}%
\pgfpathmoveto{\pgfqpoint{3.451698in}{10.135661in}}%
\pgfpathlineto{\pgfqpoint{3.539434in}{10.135661in}}%
\pgfpathlineto{\pgfqpoint{3.539434in}{10.047925in}}%
\pgfpathlineto{\pgfqpoint{3.451698in}{10.047925in}}%
\pgfpathlineto{\pgfqpoint{3.451698in}{10.135661in}}%
\pgfusepath{stroke,fill}%
\end{pgfscope}%
\begin{pgfscope}%
\pgfpathrectangle{\pgfqpoint{0.380943in}{9.960189in}}{\pgfqpoint{4.650000in}{0.614151in}}%
\pgfusepath{clip}%
\pgfsetbuttcap%
\pgfsetroundjoin%
\definecolor{currentfill}{rgb}{1.000000,1.000000,0.929412}%
\pgfsetfillcolor{currentfill}%
\pgfsetlinewidth{0.250937pt}%
\definecolor{currentstroke}{rgb}{1.000000,1.000000,1.000000}%
\pgfsetstrokecolor{currentstroke}%
\pgfsetdash{}{0pt}%
\pgfpathmoveto{\pgfqpoint{3.539434in}{10.135661in}}%
\pgfpathlineto{\pgfqpoint{3.627169in}{10.135661in}}%
\pgfpathlineto{\pgfqpoint{3.627169in}{10.047925in}}%
\pgfpathlineto{\pgfqpoint{3.539434in}{10.047925in}}%
\pgfpathlineto{\pgfqpoint{3.539434in}{10.135661in}}%
\pgfusepath{stroke,fill}%
\end{pgfscope}%
\begin{pgfscope}%
\pgfpathrectangle{\pgfqpoint{0.380943in}{9.960189in}}{\pgfqpoint{4.650000in}{0.614151in}}%
\pgfusepath{clip}%
\pgfsetbuttcap%
\pgfsetroundjoin%
\definecolor{currentfill}{rgb}{1.000000,1.000000,0.929412}%
\pgfsetfillcolor{currentfill}%
\pgfsetlinewidth{0.250937pt}%
\definecolor{currentstroke}{rgb}{1.000000,1.000000,1.000000}%
\pgfsetstrokecolor{currentstroke}%
\pgfsetdash{}{0pt}%
\pgfpathmoveto{\pgfqpoint{3.627169in}{10.135661in}}%
\pgfpathlineto{\pgfqpoint{3.714905in}{10.135661in}}%
\pgfpathlineto{\pgfqpoint{3.714905in}{10.047925in}}%
\pgfpathlineto{\pgfqpoint{3.627169in}{10.047925in}}%
\pgfpathlineto{\pgfqpoint{3.627169in}{10.135661in}}%
\pgfusepath{stroke,fill}%
\end{pgfscope}%
\begin{pgfscope}%
\pgfpathrectangle{\pgfqpoint{0.380943in}{9.960189in}}{\pgfqpoint{4.650000in}{0.614151in}}%
\pgfusepath{clip}%
\pgfsetbuttcap%
\pgfsetroundjoin%
\definecolor{currentfill}{rgb}{1.000000,1.000000,0.929412}%
\pgfsetfillcolor{currentfill}%
\pgfsetlinewidth{0.250937pt}%
\definecolor{currentstroke}{rgb}{1.000000,1.000000,1.000000}%
\pgfsetstrokecolor{currentstroke}%
\pgfsetdash{}{0pt}%
\pgfpathmoveto{\pgfqpoint{3.714905in}{10.135661in}}%
\pgfpathlineto{\pgfqpoint{3.802641in}{10.135661in}}%
\pgfpathlineto{\pgfqpoint{3.802641in}{10.047925in}}%
\pgfpathlineto{\pgfqpoint{3.714905in}{10.047925in}}%
\pgfpathlineto{\pgfqpoint{3.714905in}{10.135661in}}%
\pgfusepath{stroke,fill}%
\end{pgfscope}%
\begin{pgfscope}%
\pgfpathrectangle{\pgfqpoint{0.380943in}{9.960189in}}{\pgfqpoint{4.650000in}{0.614151in}}%
\pgfusepath{clip}%
\pgfsetbuttcap%
\pgfsetroundjoin%
\definecolor{currentfill}{rgb}{1.000000,1.000000,0.929412}%
\pgfsetfillcolor{currentfill}%
\pgfsetlinewidth{0.250937pt}%
\definecolor{currentstroke}{rgb}{1.000000,1.000000,1.000000}%
\pgfsetstrokecolor{currentstroke}%
\pgfsetdash{}{0pt}%
\pgfpathmoveto{\pgfqpoint{3.802641in}{10.135661in}}%
\pgfpathlineto{\pgfqpoint{3.890377in}{10.135661in}}%
\pgfpathlineto{\pgfqpoint{3.890377in}{10.047925in}}%
\pgfpathlineto{\pgfqpoint{3.802641in}{10.047925in}}%
\pgfpathlineto{\pgfqpoint{3.802641in}{10.135661in}}%
\pgfusepath{stroke,fill}%
\end{pgfscope}%
\begin{pgfscope}%
\pgfpathrectangle{\pgfqpoint{0.380943in}{9.960189in}}{\pgfqpoint{4.650000in}{0.614151in}}%
\pgfusepath{clip}%
\pgfsetbuttcap%
\pgfsetroundjoin%
\definecolor{currentfill}{rgb}{1.000000,1.000000,0.929412}%
\pgfsetfillcolor{currentfill}%
\pgfsetlinewidth{0.250937pt}%
\definecolor{currentstroke}{rgb}{1.000000,1.000000,1.000000}%
\pgfsetstrokecolor{currentstroke}%
\pgfsetdash{}{0pt}%
\pgfpathmoveto{\pgfqpoint{3.890377in}{10.135661in}}%
\pgfpathlineto{\pgfqpoint{3.978113in}{10.135661in}}%
\pgfpathlineto{\pgfqpoint{3.978113in}{10.047925in}}%
\pgfpathlineto{\pgfqpoint{3.890377in}{10.047925in}}%
\pgfpathlineto{\pgfqpoint{3.890377in}{10.135661in}}%
\pgfusepath{stroke,fill}%
\end{pgfscope}%
\begin{pgfscope}%
\pgfpathrectangle{\pgfqpoint{0.380943in}{9.960189in}}{\pgfqpoint{4.650000in}{0.614151in}}%
\pgfusepath{clip}%
\pgfsetbuttcap%
\pgfsetroundjoin%
\definecolor{currentfill}{rgb}{1.000000,1.000000,0.929412}%
\pgfsetfillcolor{currentfill}%
\pgfsetlinewidth{0.250937pt}%
\definecolor{currentstroke}{rgb}{1.000000,1.000000,1.000000}%
\pgfsetstrokecolor{currentstroke}%
\pgfsetdash{}{0pt}%
\pgfpathmoveto{\pgfqpoint{3.978113in}{10.135661in}}%
\pgfpathlineto{\pgfqpoint{4.065849in}{10.135661in}}%
\pgfpathlineto{\pgfqpoint{4.065849in}{10.047925in}}%
\pgfpathlineto{\pgfqpoint{3.978113in}{10.047925in}}%
\pgfpathlineto{\pgfqpoint{3.978113in}{10.135661in}}%
\pgfusepath{stroke,fill}%
\end{pgfscope}%
\begin{pgfscope}%
\pgfpathrectangle{\pgfqpoint{0.380943in}{9.960189in}}{\pgfqpoint{4.650000in}{0.614151in}}%
\pgfusepath{clip}%
\pgfsetbuttcap%
\pgfsetroundjoin%
\definecolor{currentfill}{rgb}{1.000000,1.000000,0.929412}%
\pgfsetfillcolor{currentfill}%
\pgfsetlinewidth{0.250937pt}%
\definecolor{currentstroke}{rgb}{1.000000,1.000000,1.000000}%
\pgfsetstrokecolor{currentstroke}%
\pgfsetdash{}{0pt}%
\pgfpathmoveto{\pgfqpoint{4.065849in}{10.135661in}}%
\pgfpathlineto{\pgfqpoint{4.153585in}{10.135661in}}%
\pgfpathlineto{\pgfqpoint{4.153585in}{10.047925in}}%
\pgfpathlineto{\pgfqpoint{4.065849in}{10.047925in}}%
\pgfpathlineto{\pgfqpoint{4.065849in}{10.135661in}}%
\pgfusepath{stroke,fill}%
\end{pgfscope}%
\begin{pgfscope}%
\pgfpathrectangle{\pgfqpoint{0.380943in}{9.960189in}}{\pgfqpoint{4.650000in}{0.614151in}}%
\pgfusepath{clip}%
\pgfsetbuttcap%
\pgfsetroundjoin%
\definecolor{currentfill}{rgb}{1.000000,1.000000,0.929412}%
\pgfsetfillcolor{currentfill}%
\pgfsetlinewidth{0.250937pt}%
\definecolor{currentstroke}{rgb}{1.000000,1.000000,1.000000}%
\pgfsetstrokecolor{currentstroke}%
\pgfsetdash{}{0pt}%
\pgfpathmoveto{\pgfqpoint{4.153585in}{10.135661in}}%
\pgfpathlineto{\pgfqpoint{4.241320in}{10.135661in}}%
\pgfpathlineto{\pgfqpoint{4.241320in}{10.047925in}}%
\pgfpathlineto{\pgfqpoint{4.153585in}{10.047925in}}%
\pgfpathlineto{\pgfqpoint{4.153585in}{10.135661in}}%
\pgfusepath{stroke,fill}%
\end{pgfscope}%
\begin{pgfscope}%
\pgfpathrectangle{\pgfqpoint{0.380943in}{9.960189in}}{\pgfqpoint{4.650000in}{0.614151in}}%
\pgfusepath{clip}%
\pgfsetbuttcap%
\pgfsetroundjoin%
\definecolor{currentfill}{rgb}{1.000000,1.000000,0.929412}%
\pgfsetfillcolor{currentfill}%
\pgfsetlinewidth{0.250937pt}%
\definecolor{currentstroke}{rgb}{1.000000,1.000000,1.000000}%
\pgfsetstrokecolor{currentstroke}%
\pgfsetdash{}{0pt}%
\pgfpathmoveto{\pgfqpoint{4.241320in}{10.135661in}}%
\pgfpathlineto{\pgfqpoint{4.329056in}{10.135661in}}%
\pgfpathlineto{\pgfqpoint{4.329056in}{10.047925in}}%
\pgfpathlineto{\pgfqpoint{4.241320in}{10.047925in}}%
\pgfpathlineto{\pgfqpoint{4.241320in}{10.135661in}}%
\pgfusepath{stroke,fill}%
\end{pgfscope}%
\begin{pgfscope}%
\pgfpathrectangle{\pgfqpoint{0.380943in}{9.960189in}}{\pgfqpoint{4.650000in}{0.614151in}}%
\pgfusepath{clip}%
\pgfsetbuttcap%
\pgfsetroundjoin%
\definecolor{currentfill}{rgb}{1.000000,1.000000,0.929412}%
\pgfsetfillcolor{currentfill}%
\pgfsetlinewidth{0.250937pt}%
\definecolor{currentstroke}{rgb}{1.000000,1.000000,1.000000}%
\pgfsetstrokecolor{currentstroke}%
\pgfsetdash{}{0pt}%
\pgfpathmoveto{\pgfqpoint{4.329056in}{10.135661in}}%
\pgfpathlineto{\pgfqpoint{4.416792in}{10.135661in}}%
\pgfpathlineto{\pgfqpoint{4.416792in}{10.047925in}}%
\pgfpathlineto{\pgfqpoint{4.329056in}{10.047925in}}%
\pgfpathlineto{\pgfqpoint{4.329056in}{10.135661in}}%
\pgfusepath{stroke,fill}%
\end{pgfscope}%
\begin{pgfscope}%
\pgfpathrectangle{\pgfqpoint{0.380943in}{9.960189in}}{\pgfqpoint{4.650000in}{0.614151in}}%
\pgfusepath{clip}%
\pgfsetbuttcap%
\pgfsetroundjoin%
\definecolor{currentfill}{rgb}{0.963091,0.919493,0.720185}%
\pgfsetfillcolor{currentfill}%
\pgfsetlinewidth{0.250937pt}%
\definecolor{currentstroke}{rgb}{1.000000,1.000000,1.000000}%
\pgfsetstrokecolor{currentstroke}%
\pgfsetdash{}{0pt}%
\pgfpathmoveto{\pgfqpoint{4.416792in}{10.135661in}}%
\pgfpathlineto{\pgfqpoint{4.504528in}{10.135661in}}%
\pgfpathlineto{\pgfqpoint{4.504528in}{10.047925in}}%
\pgfpathlineto{\pgfqpoint{4.416792in}{10.047925in}}%
\pgfpathlineto{\pgfqpoint{4.416792in}{10.135661in}}%
\pgfusepath{stroke,fill}%
\end{pgfscope}%
\begin{pgfscope}%
\pgfpathrectangle{\pgfqpoint{0.380943in}{9.960189in}}{\pgfqpoint{4.650000in}{0.614151in}}%
\pgfusepath{clip}%
\pgfsetbuttcap%
\pgfsetroundjoin%
\definecolor{currentfill}{rgb}{0.994694,0.745098,0.602999}%
\pgfsetfillcolor{currentfill}%
\pgfsetlinewidth{0.250937pt}%
\definecolor{currentstroke}{rgb}{1.000000,1.000000,1.000000}%
\pgfsetstrokecolor{currentstroke}%
\pgfsetdash{}{0pt}%
\pgfpathmoveto{\pgfqpoint{4.504528in}{10.135661in}}%
\pgfpathlineto{\pgfqpoint{4.592264in}{10.135661in}}%
\pgfpathlineto{\pgfqpoint{4.592264in}{10.047925in}}%
\pgfpathlineto{\pgfqpoint{4.504528in}{10.047925in}}%
\pgfpathlineto{\pgfqpoint{4.504528in}{10.135661in}}%
\pgfusepath{stroke,fill}%
\end{pgfscope}%
\begin{pgfscope}%
\pgfpathrectangle{\pgfqpoint{0.380943in}{9.960189in}}{\pgfqpoint{4.650000in}{0.614151in}}%
\pgfusepath{clip}%
\pgfsetbuttcap%
\pgfsetroundjoin%
\definecolor{currentfill}{rgb}{0.963091,0.937255,0.735409}%
\pgfsetfillcolor{currentfill}%
\pgfsetlinewidth{0.250937pt}%
\definecolor{currentstroke}{rgb}{1.000000,1.000000,1.000000}%
\pgfsetstrokecolor{currentstroke}%
\pgfsetdash{}{0pt}%
\pgfpathmoveto{\pgfqpoint{4.592264in}{10.135661in}}%
\pgfpathlineto{\pgfqpoint{4.680000in}{10.135661in}}%
\pgfpathlineto{\pgfqpoint{4.680000in}{10.047925in}}%
\pgfpathlineto{\pgfqpoint{4.592264in}{10.047925in}}%
\pgfpathlineto{\pgfqpoint{4.592264in}{10.135661in}}%
\pgfusepath{stroke,fill}%
\end{pgfscope}%
\begin{pgfscope}%
\pgfpathrectangle{\pgfqpoint{0.380943in}{9.960189in}}{\pgfqpoint{4.650000in}{0.614151in}}%
\pgfusepath{clip}%
\pgfsetbuttcap%
\pgfsetroundjoin%
\definecolor{currentfill}{rgb}{0.982699,0.823991,0.657439}%
\pgfsetfillcolor{currentfill}%
\pgfsetlinewidth{0.250937pt}%
\definecolor{currentstroke}{rgb}{1.000000,1.000000,1.000000}%
\pgfsetstrokecolor{currentstroke}%
\pgfsetdash{}{0pt}%
\pgfpathmoveto{\pgfqpoint{4.680000in}{10.135661in}}%
\pgfpathlineto{\pgfqpoint{4.767736in}{10.135661in}}%
\pgfpathlineto{\pgfqpoint{4.767736in}{10.047925in}}%
\pgfpathlineto{\pgfqpoint{4.680000in}{10.047925in}}%
\pgfpathlineto{\pgfqpoint{4.680000in}{10.135661in}}%
\pgfusepath{stroke,fill}%
\end{pgfscope}%
\begin{pgfscope}%
\pgfpathrectangle{\pgfqpoint{0.380943in}{9.960189in}}{\pgfqpoint{4.650000in}{0.614151in}}%
\pgfusepath{clip}%
\pgfsetbuttcap%
\pgfsetroundjoin%
\definecolor{currentfill}{rgb}{0.963091,0.919493,0.720185}%
\pgfsetfillcolor{currentfill}%
\pgfsetlinewidth{0.250937pt}%
\definecolor{currentstroke}{rgb}{1.000000,1.000000,1.000000}%
\pgfsetstrokecolor{currentstroke}%
\pgfsetdash{}{0pt}%
\pgfpathmoveto{\pgfqpoint{4.767736in}{10.135661in}}%
\pgfpathlineto{\pgfqpoint{4.855471in}{10.135661in}}%
\pgfpathlineto{\pgfqpoint{4.855471in}{10.047925in}}%
\pgfpathlineto{\pgfqpoint{4.767736in}{10.047925in}}%
\pgfpathlineto{\pgfqpoint{4.767736in}{10.135661in}}%
\pgfusepath{stroke,fill}%
\end{pgfscope}%
\begin{pgfscope}%
\pgfpathrectangle{\pgfqpoint{0.380943in}{9.960189in}}{\pgfqpoint{4.650000in}{0.614151in}}%
\pgfusepath{clip}%
\pgfsetbuttcap%
\pgfsetroundjoin%
\definecolor{currentfill}{rgb}{0.963091,0.919493,0.720185}%
\pgfsetfillcolor{currentfill}%
\pgfsetlinewidth{0.250937pt}%
\definecolor{currentstroke}{rgb}{1.000000,1.000000,1.000000}%
\pgfsetstrokecolor{currentstroke}%
\pgfsetdash{}{0pt}%
\pgfpathmoveto{\pgfqpoint{4.855471in}{10.135661in}}%
\pgfpathlineto{\pgfqpoint{4.943207in}{10.135661in}}%
\pgfpathlineto{\pgfqpoint{4.943207in}{10.047925in}}%
\pgfpathlineto{\pgfqpoint{4.855471in}{10.047925in}}%
\pgfpathlineto{\pgfqpoint{4.855471in}{10.135661in}}%
\pgfusepath{stroke,fill}%
\end{pgfscope}%
\begin{pgfscope}%
\pgfpathrectangle{\pgfqpoint{0.380943in}{9.960189in}}{\pgfqpoint{4.650000in}{0.614151in}}%
\pgfusepath{clip}%
\pgfsetbuttcap%
\pgfsetroundjoin%
\definecolor{currentfill}{rgb}{0.967474,0.895963,0.706344}%
\pgfsetfillcolor{currentfill}%
\pgfsetlinewidth{0.250937pt}%
\definecolor{currentstroke}{rgb}{1.000000,1.000000,1.000000}%
\pgfsetstrokecolor{currentstroke}%
\pgfsetdash{}{0pt}%
\pgfpathmoveto{\pgfqpoint{4.943207in}{10.135661in}}%
\pgfpathlineto{\pgfqpoint{5.030943in}{10.135661in}}%
\pgfpathlineto{\pgfqpoint{5.030943in}{10.047925in}}%
\pgfpathlineto{\pgfqpoint{4.943207in}{10.047925in}}%
\pgfpathlineto{\pgfqpoint{4.943207in}{10.135661in}}%
\pgfusepath{stroke,fill}%
\end{pgfscope}%
\begin{pgfscope}%
\pgfpathrectangle{\pgfqpoint{0.380943in}{9.960189in}}{\pgfqpoint{4.650000in}{0.614151in}}%
\pgfusepath{clip}%
\pgfsetbuttcap%
\pgfsetroundjoin%
\definecolor{currentfill}{rgb}{1.000000,1.000000,0.929412}%
\pgfsetfillcolor{currentfill}%
\pgfsetlinewidth{0.250937pt}%
\definecolor{currentstroke}{rgb}{1.000000,1.000000,1.000000}%
\pgfsetstrokecolor{currentstroke}%
\pgfsetdash{}{0pt}%
\pgfpathmoveto{\pgfqpoint{0.380943in}{10.047925in}}%
\pgfpathlineto{\pgfqpoint{0.468679in}{10.047925in}}%
\pgfpathlineto{\pgfqpoint{0.468679in}{9.960189in}}%
\pgfpathlineto{\pgfqpoint{0.380943in}{9.960189in}}%
\pgfpathlineto{\pgfqpoint{0.380943in}{10.047925in}}%
\pgfusepath{stroke,fill}%
\end{pgfscope}%
\begin{pgfscope}%
\pgfpathrectangle{\pgfqpoint{0.380943in}{9.960189in}}{\pgfqpoint{4.650000in}{0.614151in}}%
\pgfusepath{clip}%
\pgfsetbuttcap%
\pgfsetroundjoin%
\definecolor{currentfill}{rgb}{1.000000,1.000000,0.929412}%
\pgfsetfillcolor{currentfill}%
\pgfsetlinewidth{0.250937pt}%
\definecolor{currentstroke}{rgb}{1.000000,1.000000,1.000000}%
\pgfsetstrokecolor{currentstroke}%
\pgfsetdash{}{0pt}%
\pgfpathmoveto{\pgfqpoint{0.468679in}{10.047925in}}%
\pgfpathlineto{\pgfqpoint{0.556415in}{10.047925in}}%
\pgfpathlineto{\pgfqpoint{0.556415in}{9.960189in}}%
\pgfpathlineto{\pgfqpoint{0.468679in}{9.960189in}}%
\pgfpathlineto{\pgfqpoint{0.468679in}{10.047925in}}%
\pgfusepath{stroke,fill}%
\end{pgfscope}%
\begin{pgfscope}%
\pgfpathrectangle{\pgfqpoint{0.380943in}{9.960189in}}{\pgfqpoint{4.650000in}{0.614151in}}%
\pgfusepath{clip}%
\pgfsetbuttcap%
\pgfsetroundjoin%
\definecolor{currentfill}{rgb}{1.000000,1.000000,0.929412}%
\pgfsetfillcolor{currentfill}%
\pgfsetlinewidth{0.250937pt}%
\definecolor{currentstroke}{rgb}{1.000000,1.000000,1.000000}%
\pgfsetstrokecolor{currentstroke}%
\pgfsetdash{}{0pt}%
\pgfpathmoveto{\pgfqpoint{0.556415in}{10.047925in}}%
\pgfpathlineto{\pgfqpoint{0.644151in}{10.047925in}}%
\pgfpathlineto{\pgfqpoint{0.644151in}{9.960189in}}%
\pgfpathlineto{\pgfqpoint{0.556415in}{9.960189in}}%
\pgfpathlineto{\pgfqpoint{0.556415in}{10.047925in}}%
\pgfusepath{stroke,fill}%
\end{pgfscope}%
\begin{pgfscope}%
\pgfpathrectangle{\pgfqpoint{0.380943in}{9.960189in}}{\pgfqpoint{4.650000in}{0.614151in}}%
\pgfusepath{clip}%
\pgfsetbuttcap%
\pgfsetroundjoin%
\definecolor{currentfill}{rgb}{1.000000,1.000000,0.929412}%
\pgfsetfillcolor{currentfill}%
\pgfsetlinewidth{0.250937pt}%
\definecolor{currentstroke}{rgb}{1.000000,1.000000,1.000000}%
\pgfsetstrokecolor{currentstroke}%
\pgfsetdash{}{0pt}%
\pgfpathmoveto{\pgfqpoint{0.644151in}{10.047925in}}%
\pgfpathlineto{\pgfqpoint{0.731886in}{10.047925in}}%
\pgfpathlineto{\pgfqpoint{0.731886in}{9.960189in}}%
\pgfpathlineto{\pgfqpoint{0.644151in}{9.960189in}}%
\pgfpathlineto{\pgfqpoint{0.644151in}{10.047925in}}%
\pgfusepath{stroke,fill}%
\end{pgfscope}%
\begin{pgfscope}%
\pgfpathrectangle{\pgfqpoint{0.380943in}{9.960189in}}{\pgfqpoint{4.650000in}{0.614151in}}%
\pgfusepath{clip}%
\pgfsetbuttcap%
\pgfsetroundjoin%
\definecolor{currentfill}{rgb}{1.000000,1.000000,0.929412}%
\pgfsetfillcolor{currentfill}%
\pgfsetlinewidth{0.250937pt}%
\definecolor{currentstroke}{rgb}{1.000000,1.000000,1.000000}%
\pgfsetstrokecolor{currentstroke}%
\pgfsetdash{}{0pt}%
\pgfpathmoveto{\pgfqpoint{0.731886in}{10.047925in}}%
\pgfpathlineto{\pgfqpoint{0.819622in}{10.047925in}}%
\pgfpathlineto{\pgfqpoint{0.819622in}{9.960189in}}%
\pgfpathlineto{\pgfqpoint{0.731886in}{9.960189in}}%
\pgfpathlineto{\pgfqpoint{0.731886in}{10.047925in}}%
\pgfusepath{stroke,fill}%
\end{pgfscope}%
\begin{pgfscope}%
\pgfpathrectangle{\pgfqpoint{0.380943in}{9.960189in}}{\pgfqpoint{4.650000in}{0.614151in}}%
\pgfusepath{clip}%
\pgfsetbuttcap%
\pgfsetroundjoin%
\definecolor{currentfill}{rgb}{1.000000,1.000000,0.929412}%
\pgfsetfillcolor{currentfill}%
\pgfsetlinewidth{0.250937pt}%
\definecolor{currentstroke}{rgb}{1.000000,1.000000,1.000000}%
\pgfsetstrokecolor{currentstroke}%
\pgfsetdash{}{0pt}%
\pgfpathmoveto{\pgfqpoint{0.819622in}{10.047925in}}%
\pgfpathlineto{\pgfqpoint{0.907358in}{10.047925in}}%
\pgfpathlineto{\pgfqpoint{0.907358in}{9.960189in}}%
\pgfpathlineto{\pgfqpoint{0.819622in}{9.960189in}}%
\pgfpathlineto{\pgfqpoint{0.819622in}{10.047925in}}%
\pgfusepath{stroke,fill}%
\end{pgfscope}%
\begin{pgfscope}%
\pgfpathrectangle{\pgfqpoint{0.380943in}{9.960189in}}{\pgfqpoint{4.650000in}{0.614151in}}%
\pgfusepath{clip}%
\pgfsetbuttcap%
\pgfsetroundjoin%
\definecolor{currentfill}{rgb}{1.000000,1.000000,0.929412}%
\pgfsetfillcolor{currentfill}%
\pgfsetlinewidth{0.250937pt}%
\definecolor{currentstroke}{rgb}{1.000000,1.000000,1.000000}%
\pgfsetstrokecolor{currentstroke}%
\pgfsetdash{}{0pt}%
\pgfpathmoveto{\pgfqpoint{0.907358in}{10.047925in}}%
\pgfpathlineto{\pgfqpoint{0.995094in}{10.047925in}}%
\pgfpathlineto{\pgfqpoint{0.995094in}{9.960189in}}%
\pgfpathlineto{\pgfqpoint{0.907358in}{9.960189in}}%
\pgfpathlineto{\pgfqpoint{0.907358in}{10.047925in}}%
\pgfusepath{stroke,fill}%
\end{pgfscope}%
\begin{pgfscope}%
\pgfpathrectangle{\pgfqpoint{0.380943in}{9.960189in}}{\pgfqpoint{4.650000in}{0.614151in}}%
\pgfusepath{clip}%
\pgfsetbuttcap%
\pgfsetroundjoin%
\definecolor{currentfill}{rgb}{1.000000,1.000000,0.929412}%
\pgfsetfillcolor{currentfill}%
\pgfsetlinewidth{0.250937pt}%
\definecolor{currentstroke}{rgb}{1.000000,1.000000,1.000000}%
\pgfsetstrokecolor{currentstroke}%
\pgfsetdash{}{0pt}%
\pgfpathmoveto{\pgfqpoint{0.995094in}{10.047925in}}%
\pgfpathlineto{\pgfqpoint{1.082830in}{10.047925in}}%
\pgfpathlineto{\pgfqpoint{1.082830in}{9.960189in}}%
\pgfpathlineto{\pgfqpoint{0.995094in}{9.960189in}}%
\pgfpathlineto{\pgfqpoint{0.995094in}{10.047925in}}%
\pgfusepath{stroke,fill}%
\end{pgfscope}%
\begin{pgfscope}%
\pgfpathrectangle{\pgfqpoint{0.380943in}{9.960189in}}{\pgfqpoint{4.650000in}{0.614151in}}%
\pgfusepath{clip}%
\pgfsetbuttcap%
\pgfsetroundjoin%
\definecolor{currentfill}{rgb}{1.000000,1.000000,0.929412}%
\pgfsetfillcolor{currentfill}%
\pgfsetlinewidth{0.250937pt}%
\definecolor{currentstroke}{rgb}{1.000000,1.000000,1.000000}%
\pgfsetstrokecolor{currentstroke}%
\pgfsetdash{}{0pt}%
\pgfpathmoveto{\pgfqpoint{1.082830in}{10.047925in}}%
\pgfpathlineto{\pgfqpoint{1.170566in}{10.047925in}}%
\pgfpathlineto{\pgfqpoint{1.170566in}{9.960189in}}%
\pgfpathlineto{\pgfqpoint{1.082830in}{9.960189in}}%
\pgfpathlineto{\pgfqpoint{1.082830in}{10.047925in}}%
\pgfusepath{stroke,fill}%
\end{pgfscope}%
\begin{pgfscope}%
\pgfpathrectangle{\pgfqpoint{0.380943in}{9.960189in}}{\pgfqpoint{4.650000in}{0.614151in}}%
\pgfusepath{clip}%
\pgfsetbuttcap%
\pgfsetroundjoin%
\definecolor{currentfill}{rgb}{1.000000,1.000000,0.929412}%
\pgfsetfillcolor{currentfill}%
\pgfsetlinewidth{0.250937pt}%
\definecolor{currentstroke}{rgb}{1.000000,1.000000,1.000000}%
\pgfsetstrokecolor{currentstroke}%
\pgfsetdash{}{0pt}%
\pgfpathmoveto{\pgfqpoint{1.170566in}{10.047925in}}%
\pgfpathlineto{\pgfqpoint{1.258302in}{10.047925in}}%
\pgfpathlineto{\pgfqpoint{1.258302in}{9.960189in}}%
\pgfpathlineto{\pgfqpoint{1.170566in}{9.960189in}}%
\pgfpathlineto{\pgfqpoint{1.170566in}{10.047925in}}%
\pgfusepath{stroke,fill}%
\end{pgfscope}%
\begin{pgfscope}%
\pgfpathrectangle{\pgfqpoint{0.380943in}{9.960189in}}{\pgfqpoint{4.650000in}{0.614151in}}%
\pgfusepath{clip}%
\pgfsetbuttcap%
\pgfsetroundjoin%
\definecolor{currentfill}{rgb}{1.000000,1.000000,0.929412}%
\pgfsetfillcolor{currentfill}%
\pgfsetlinewidth{0.250937pt}%
\definecolor{currentstroke}{rgb}{1.000000,1.000000,1.000000}%
\pgfsetstrokecolor{currentstroke}%
\pgfsetdash{}{0pt}%
\pgfpathmoveto{\pgfqpoint{1.258302in}{10.047925in}}%
\pgfpathlineto{\pgfqpoint{1.346037in}{10.047925in}}%
\pgfpathlineto{\pgfqpoint{1.346037in}{9.960189in}}%
\pgfpathlineto{\pgfqpoint{1.258302in}{9.960189in}}%
\pgfpathlineto{\pgfqpoint{1.258302in}{10.047925in}}%
\pgfusepath{stroke,fill}%
\end{pgfscope}%
\begin{pgfscope}%
\pgfpathrectangle{\pgfqpoint{0.380943in}{9.960189in}}{\pgfqpoint{4.650000in}{0.614151in}}%
\pgfusepath{clip}%
\pgfsetbuttcap%
\pgfsetroundjoin%
\definecolor{currentfill}{rgb}{1.000000,1.000000,0.929412}%
\pgfsetfillcolor{currentfill}%
\pgfsetlinewidth{0.250937pt}%
\definecolor{currentstroke}{rgb}{1.000000,1.000000,1.000000}%
\pgfsetstrokecolor{currentstroke}%
\pgfsetdash{}{0pt}%
\pgfpathmoveto{\pgfqpoint{1.346037in}{10.047925in}}%
\pgfpathlineto{\pgfqpoint{1.433773in}{10.047925in}}%
\pgfpathlineto{\pgfqpoint{1.433773in}{9.960189in}}%
\pgfpathlineto{\pgfqpoint{1.346037in}{9.960189in}}%
\pgfpathlineto{\pgfqpoint{1.346037in}{10.047925in}}%
\pgfusepath{stroke,fill}%
\end{pgfscope}%
\begin{pgfscope}%
\pgfpathrectangle{\pgfqpoint{0.380943in}{9.960189in}}{\pgfqpoint{4.650000in}{0.614151in}}%
\pgfusepath{clip}%
\pgfsetbuttcap%
\pgfsetroundjoin%
\definecolor{currentfill}{rgb}{1.000000,1.000000,0.929412}%
\pgfsetfillcolor{currentfill}%
\pgfsetlinewidth{0.250937pt}%
\definecolor{currentstroke}{rgb}{1.000000,1.000000,1.000000}%
\pgfsetstrokecolor{currentstroke}%
\pgfsetdash{}{0pt}%
\pgfpathmoveto{\pgfqpoint{1.433773in}{10.047925in}}%
\pgfpathlineto{\pgfqpoint{1.521509in}{10.047925in}}%
\pgfpathlineto{\pgfqpoint{1.521509in}{9.960189in}}%
\pgfpathlineto{\pgfqpoint{1.433773in}{9.960189in}}%
\pgfpathlineto{\pgfqpoint{1.433773in}{10.047925in}}%
\pgfusepath{stroke,fill}%
\end{pgfscope}%
\begin{pgfscope}%
\pgfpathrectangle{\pgfqpoint{0.380943in}{9.960189in}}{\pgfqpoint{4.650000in}{0.614151in}}%
\pgfusepath{clip}%
\pgfsetbuttcap%
\pgfsetroundjoin%
\definecolor{currentfill}{rgb}{1.000000,1.000000,0.929412}%
\pgfsetfillcolor{currentfill}%
\pgfsetlinewidth{0.250937pt}%
\definecolor{currentstroke}{rgb}{1.000000,1.000000,1.000000}%
\pgfsetstrokecolor{currentstroke}%
\pgfsetdash{}{0pt}%
\pgfpathmoveto{\pgfqpoint{1.521509in}{10.047925in}}%
\pgfpathlineto{\pgfqpoint{1.609245in}{10.047925in}}%
\pgfpathlineto{\pgfqpoint{1.609245in}{9.960189in}}%
\pgfpathlineto{\pgfqpoint{1.521509in}{9.960189in}}%
\pgfpathlineto{\pgfqpoint{1.521509in}{10.047925in}}%
\pgfusepath{stroke,fill}%
\end{pgfscope}%
\begin{pgfscope}%
\pgfpathrectangle{\pgfqpoint{0.380943in}{9.960189in}}{\pgfqpoint{4.650000in}{0.614151in}}%
\pgfusepath{clip}%
\pgfsetbuttcap%
\pgfsetroundjoin%
\definecolor{currentfill}{rgb}{1.000000,1.000000,0.929412}%
\pgfsetfillcolor{currentfill}%
\pgfsetlinewidth{0.250937pt}%
\definecolor{currentstroke}{rgb}{1.000000,1.000000,1.000000}%
\pgfsetstrokecolor{currentstroke}%
\pgfsetdash{}{0pt}%
\pgfpathmoveto{\pgfqpoint{1.609245in}{10.047925in}}%
\pgfpathlineto{\pgfqpoint{1.696981in}{10.047925in}}%
\pgfpathlineto{\pgfqpoint{1.696981in}{9.960189in}}%
\pgfpathlineto{\pgfqpoint{1.609245in}{9.960189in}}%
\pgfpathlineto{\pgfqpoint{1.609245in}{10.047925in}}%
\pgfusepath{stroke,fill}%
\end{pgfscope}%
\begin{pgfscope}%
\pgfpathrectangle{\pgfqpoint{0.380943in}{9.960189in}}{\pgfqpoint{4.650000in}{0.614151in}}%
\pgfusepath{clip}%
\pgfsetbuttcap%
\pgfsetroundjoin%
\definecolor{currentfill}{rgb}{1.000000,1.000000,0.929412}%
\pgfsetfillcolor{currentfill}%
\pgfsetlinewidth{0.250937pt}%
\definecolor{currentstroke}{rgb}{1.000000,1.000000,1.000000}%
\pgfsetstrokecolor{currentstroke}%
\pgfsetdash{}{0pt}%
\pgfpathmoveto{\pgfqpoint{1.696981in}{10.047925in}}%
\pgfpathlineto{\pgfqpoint{1.784717in}{10.047925in}}%
\pgfpathlineto{\pgfqpoint{1.784717in}{9.960189in}}%
\pgfpathlineto{\pgfqpoint{1.696981in}{9.960189in}}%
\pgfpathlineto{\pgfqpoint{1.696981in}{10.047925in}}%
\pgfusepath{stroke,fill}%
\end{pgfscope}%
\begin{pgfscope}%
\pgfpathrectangle{\pgfqpoint{0.380943in}{9.960189in}}{\pgfqpoint{4.650000in}{0.614151in}}%
\pgfusepath{clip}%
\pgfsetbuttcap%
\pgfsetroundjoin%
\definecolor{currentfill}{rgb}{1.000000,1.000000,0.929412}%
\pgfsetfillcolor{currentfill}%
\pgfsetlinewidth{0.250937pt}%
\definecolor{currentstroke}{rgb}{1.000000,1.000000,1.000000}%
\pgfsetstrokecolor{currentstroke}%
\pgfsetdash{}{0pt}%
\pgfpathmoveto{\pgfqpoint{1.784717in}{10.047925in}}%
\pgfpathlineto{\pgfqpoint{1.872452in}{10.047925in}}%
\pgfpathlineto{\pgfqpoint{1.872452in}{9.960189in}}%
\pgfpathlineto{\pgfqpoint{1.784717in}{9.960189in}}%
\pgfpathlineto{\pgfqpoint{1.784717in}{10.047925in}}%
\pgfusepath{stroke,fill}%
\end{pgfscope}%
\begin{pgfscope}%
\pgfpathrectangle{\pgfqpoint{0.380943in}{9.960189in}}{\pgfqpoint{4.650000in}{0.614151in}}%
\pgfusepath{clip}%
\pgfsetbuttcap%
\pgfsetroundjoin%
\definecolor{currentfill}{rgb}{1.000000,1.000000,0.929412}%
\pgfsetfillcolor{currentfill}%
\pgfsetlinewidth{0.250937pt}%
\definecolor{currentstroke}{rgb}{1.000000,1.000000,1.000000}%
\pgfsetstrokecolor{currentstroke}%
\pgfsetdash{}{0pt}%
\pgfpathmoveto{\pgfqpoint{1.872452in}{10.047925in}}%
\pgfpathlineto{\pgfqpoint{1.960188in}{10.047925in}}%
\pgfpathlineto{\pgfqpoint{1.960188in}{9.960189in}}%
\pgfpathlineto{\pgfqpoint{1.872452in}{9.960189in}}%
\pgfpathlineto{\pgfqpoint{1.872452in}{10.047925in}}%
\pgfusepath{stroke,fill}%
\end{pgfscope}%
\begin{pgfscope}%
\pgfpathrectangle{\pgfqpoint{0.380943in}{9.960189in}}{\pgfqpoint{4.650000in}{0.614151in}}%
\pgfusepath{clip}%
\pgfsetbuttcap%
\pgfsetroundjoin%
\definecolor{currentfill}{rgb}{1.000000,1.000000,0.929412}%
\pgfsetfillcolor{currentfill}%
\pgfsetlinewidth{0.250937pt}%
\definecolor{currentstroke}{rgb}{1.000000,1.000000,1.000000}%
\pgfsetstrokecolor{currentstroke}%
\pgfsetdash{}{0pt}%
\pgfpathmoveto{\pgfqpoint{1.960188in}{10.047925in}}%
\pgfpathlineto{\pgfqpoint{2.047924in}{10.047925in}}%
\pgfpathlineto{\pgfqpoint{2.047924in}{9.960189in}}%
\pgfpathlineto{\pgfqpoint{1.960188in}{9.960189in}}%
\pgfpathlineto{\pgfqpoint{1.960188in}{10.047925in}}%
\pgfusepath{stroke,fill}%
\end{pgfscope}%
\begin{pgfscope}%
\pgfpathrectangle{\pgfqpoint{0.380943in}{9.960189in}}{\pgfqpoint{4.650000in}{0.614151in}}%
\pgfusepath{clip}%
\pgfsetbuttcap%
\pgfsetroundjoin%
\definecolor{currentfill}{rgb}{1.000000,1.000000,0.929412}%
\pgfsetfillcolor{currentfill}%
\pgfsetlinewidth{0.250937pt}%
\definecolor{currentstroke}{rgb}{1.000000,1.000000,1.000000}%
\pgfsetstrokecolor{currentstroke}%
\pgfsetdash{}{0pt}%
\pgfpathmoveto{\pgfqpoint{2.047924in}{10.047925in}}%
\pgfpathlineto{\pgfqpoint{2.135660in}{10.047925in}}%
\pgfpathlineto{\pgfqpoint{2.135660in}{9.960189in}}%
\pgfpathlineto{\pgfqpoint{2.047924in}{9.960189in}}%
\pgfpathlineto{\pgfqpoint{2.047924in}{10.047925in}}%
\pgfusepath{stroke,fill}%
\end{pgfscope}%
\begin{pgfscope}%
\pgfpathrectangle{\pgfqpoint{0.380943in}{9.960189in}}{\pgfqpoint{4.650000in}{0.614151in}}%
\pgfusepath{clip}%
\pgfsetbuttcap%
\pgfsetroundjoin%
\definecolor{currentfill}{rgb}{1.000000,1.000000,0.929412}%
\pgfsetfillcolor{currentfill}%
\pgfsetlinewidth{0.250937pt}%
\definecolor{currentstroke}{rgb}{1.000000,1.000000,1.000000}%
\pgfsetstrokecolor{currentstroke}%
\pgfsetdash{}{0pt}%
\pgfpathmoveto{\pgfqpoint{2.135660in}{10.047925in}}%
\pgfpathlineto{\pgfqpoint{2.223396in}{10.047925in}}%
\pgfpathlineto{\pgfqpoint{2.223396in}{9.960189in}}%
\pgfpathlineto{\pgfqpoint{2.135660in}{9.960189in}}%
\pgfpathlineto{\pgfqpoint{2.135660in}{10.047925in}}%
\pgfusepath{stroke,fill}%
\end{pgfscope}%
\begin{pgfscope}%
\pgfpathrectangle{\pgfqpoint{0.380943in}{9.960189in}}{\pgfqpoint{4.650000in}{0.614151in}}%
\pgfusepath{clip}%
\pgfsetbuttcap%
\pgfsetroundjoin%
\definecolor{currentfill}{rgb}{1.000000,1.000000,0.929412}%
\pgfsetfillcolor{currentfill}%
\pgfsetlinewidth{0.250937pt}%
\definecolor{currentstroke}{rgb}{1.000000,1.000000,1.000000}%
\pgfsetstrokecolor{currentstroke}%
\pgfsetdash{}{0pt}%
\pgfpathmoveto{\pgfqpoint{2.223396in}{10.047925in}}%
\pgfpathlineto{\pgfqpoint{2.311132in}{10.047925in}}%
\pgfpathlineto{\pgfqpoint{2.311132in}{9.960189in}}%
\pgfpathlineto{\pgfqpoint{2.223396in}{9.960189in}}%
\pgfpathlineto{\pgfqpoint{2.223396in}{10.047925in}}%
\pgfusepath{stroke,fill}%
\end{pgfscope}%
\begin{pgfscope}%
\pgfpathrectangle{\pgfqpoint{0.380943in}{9.960189in}}{\pgfqpoint{4.650000in}{0.614151in}}%
\pgfusepath{clip}%
\pgfsetbuttcap%
\pgfsetroundjoin%
\definecolor{currentfill}{rgb}{1.000000,1.000000,0.929412}%
\pgfsetfillcolor{currentfill}%
\pgfsetlinewidth{0.250937pt}%
\definecolor{currentstroke}{rgb}{1.000000,1.000000,1.000000}%
\pgfsetstrokecolor{currentstroke}%
\pgfsetdash{}{0pt}%
\pgfpathmoveto{\pgfqpoint{2.311132in}{10.047925in}}%
\pgfpathlineto{\pgfqpoint{2.398868in}{10.047925in}}%
\pgfpathlineto{\pgfqpoint{2.398868in}{9.960189in}}%
\pgfpathlineto{\pgfqpoint{2.311132in}{9.960189in}}%
\pgfpathlineto{\pgfqpoint{2.311132in}{10.047925in}}%
\pgfusepath{stroke,fill}%
\end{pgfscope}%
\begin{pgfscope}%
\pgfpathrectangle{\pgfqpoint{0.380943in}{9.960189in}}{\pgfqpoint{4.650000in}{0.614151in}}%
\pgfusepath{clip}%
\pgfsetbuttcap%
\pgfsetroundjoin%
\definecolor{currentfill}{rgb}{1.000000,1.000000,0.929412}%
\pgfsetfillcolor{currentfill}%
\pgfsetlinewidth{0.250937pt}%
\definecolor{currentstroke}{rgb}{1.000000,1.000000,1.000000}%
\pgfsetstrokecolor{currentstroke}%
\pgfsetdash{}{0pt}%
\pgfpathmoveto{\pgfqpoint{2.398868in}{10.047925in}}%
\pgfpathlineto{\pgfqpoint{2.486603in}{10.047925in}}%
\pgfpathlineto{\pgfqpoint{2.486603in}{9.960189in}}%
\pgfpathlineto{\pgfqpoint{2.398868in}{9.960189in}}%
\pgfpathlineto{\pgfqpoint{2.398868in}{10.047925in}}%
\pgfusepath{stroke,fill}%
\end{pgfscope}%
\begin{pgfscope}%
\pgfpathrectangle{\pgfqpoint{0.380943in}{9.960189in}}{\pgfqpoint{4.650000in}{0.614151in}}%
\pgfusepath{clip}%
\pgfsetbuttcap%
\pgfsetroundjoin%
\definecolor{currentfill}{rgb}{1.000000,1.000000,0.929412}%
\pgfsetfillcolor{currentfill}%
\pgfsetlinewidth{0.250937pt}%
\definecolor{currentstroke}{rgb}{1.000000,1.000000,1.000000}%
\pgfsetstrokecolor{currentstroke}%
\pgfsetdash{}{0pt}%
\pgfpathmoveto{\pgfqpoint{2.486603in}{10.047925in}}%
\pgfpathlineto{\pgfqpoint{2.574339in}{10.047925in}}%
\pgfpathlineto{\pgfqpoint{2.574339in}{9.960189in}}%
\pgfpathlineto{\pgfqpoint{2.486603in}{9.960189in}}%
\pgfpathlineto{\pgfqpoint{2.486603in}{10.047925in}}%
\pgfusepath{stroke,fill}%
\end{pgfscope}%
\begin{pgfscope}%
\pgfpathrectangle{\pgfqpoint{0.380943in}{9.960189in}}{\pgfqpoint{4.650000in}{0.614151in}}%
\pgfusepath{clip}%
\pgfsetbuttcap%
\pgfsetroundjoin%
\definecolor{currentfill}{rgb}{1.000000,1.000000,0.929412}%
\pgfsetfillcolor{currentfill}%
\pgfsetlinewidth{0.250937pt}%
\definecolor{currentstroke}{rgb}{1.000000,1.000000,1.000000}%
\pgfsetstrokecolor{currentstroke}%
\pgfsetdash{}{0pt}%
\pgfpathmoveto{\pgfqpoint{2.574339in}{10.047925in}}%
\pgfpathlineto{\pgfqpoint{2.662075in}{10.047925in}}%
\pgfpathlineto{\pgfqpoint{2.662075in}{9.960189in}}%
\pgfpathlineto{\pgfqpoint{2.574339in}{9.960189in}}%
\pgfpathlineto{\pgfqpoint{2.574339in}{10.047925in}}%
\pgfusepath{stroke,fill}%
\end{pgfscope}%
\begin{pgfscope}%
\pgfpathrectangle{\pgfqpoint{0.380943in}{9.960189in}}{\pgfqpoint{4.650000in}{0.614151in}}%
\pgfusepath{clip}%
\pgfsetbuttcap%
\pgfsetroundjoin%
\definecolor{currentfill}{rgb}{1.000000,1.000000,0.929412}%
\pgfsetfillcolor{currentfill}%
\pgfsetlinewidth{0.250937pt}%
\definecolor{currentstroke}{rgb}{1.000000,1.000000,1.000000}%
\pgfsetstrokecolor{currentstroke}%
\pgfsetdash{}{0pt}%
\pgfpathmoveto{\pgfqpoint{2.662075in}{10.047925in}}%
\pgfpathlineto{\pgfqpoint{2.749811in}{10.047925in}}%
\pgfpathlineto{\pgfqpoint{2.749811in}{9.960189in}}%
\pgfpathlineto{\pgfqpoint{2.662075in}{9.960189in}}%
\pgfpathlineto{\pgfqpoint{2.662075in}{10.047925in}}%
\pgfusepath{stroke,fill}%
\end{pgfscope}%
\begin{pgfscope}%
\pgfpathrectangle{\pgfqpoint{0.380943in}{9.960189in}}{\pgfqpoint{4.650000in}{0.614151in}}%
\pgfusepath{clip}%
\pgfsetbuttcap%
\pgfsetroundjoin%
\definecolor{currentfill}{rgb}{1.000000,1.000000,0.929412}%
\pgfsetfillcolor{currentfill}%
\pgfsetlinewidth{0.250937pt}%
\definecolor{currentstroke}{rgb}{1.000000,1.000000,1.000000}%
\pgfsetstrokecolor{currentstroke}%
\pgfsetdash{}{0pt}%
\pgfpathmoveto{\pgfqpoint{2.749811in}{10.047925in}}%
\pgfpathlineto{\pgfqpoint{2.837547in}{10.047925in}}%
\pgfpathlineto{\pgfqpoint{2.837547in}{9.960189in}}%
\pgfpathlineto{\pgfqpoint{2.749811in}{9.960189in}}%
\pgfpathlineto{\pgfqpoint{2.749811in}{10.047925in}}%
\pgfusepath{stroke,fill}%
\end{pgfscope}%
\begin{pgfscope}%
\pgfpathrectangle{\pgfqpoint{0.380943in}{9.960189in}}{\pgfqpoint{4.650000in}{0.614151in}}%
\pgfusepath{clip}%
\pgfsetbuttcap%
\pgfsetroundjoin%
\definecolor{currentfill}{rgb}{1.000000,1.000000,0.929412}%
\pgfsetfillcolor{currentfill}%
\pgfsetlinewidth{0.250937pt}%
\definecolor{currentstroke}{rgb}{1.000000,1.000000,1.000000}%
\pgfsetstrokecolor{currentstroke}%
\pgfsetdash{}{0pt}%
\pgfpathmoveto{\pgfqpoint{2.837547in}{10.047925in}}%
\pgfpathlineto{\pgfqpoint{2.925283in}{10.047925in}}%
\pgfpathlineto{\pgfqpoint{2.925283in}{9.960189in}}%
\pgfpathlineto{\pgfqpoint{2.837547in}{9.960189in}}%
\pgfpathlineto{\pgfqpoint{2.837547in}{10.047925in}}%
\pgfusepath{stroke,fill}%
\end{pgfscope}%
\begin{pgfscope}%
\pgfpathrectangle{\pgfqpoint{0.380943in}{9.960189in}}{\pgfqpoint{4.650000in}{0.614151in}}%
\pgfusepath{clip}%
\pgfsetbuttcap%
\pgfsetroundjoin%
\definecolor{currentfill}{rgb}{1.000000,1.000000,0.929412}%
\pgfsetfillcolor{currentfill}%
\pgfsetlinewidth{0.250937pt}%
\definecolor{currentstroke}{rgb}{1.000000,1.000000,1.000000}%
\pgfsetstrokecolor{currentstroke}%
\pgfsetdash{}{0pt}%
\pgfpathmoveto{\pgfqpoint{2.925283in}{10.047925in}}%
\pgfpathlineto{\pgfqpoint{3.013019in}{10.047925in}}%
\pgfpathlineto{\pgfqpoint{3.013019in}{9.960189in}}%
\pgfpathlineto{\pgfqpoint{2.925283in}{9.960189in}}%
\pgfpathlineto{\pgfqpoint{2.925283in}{10.047925in}}%
\pgfusepath{stroke,fill}%
\end{pgfscope}%
\begin{pgfscope}%
\pgfpathrectangle{\pgfqpoint{0.380943in}{9.960189in}}{\pgfqpoint{4.650000in}{0.614151in}}%
\pgfusepath{clip}%
\pgfsetbuttcap%
\pgfsetroundjoin%
\definecolor{currentfill}{rgb}{1.000000,1.000000,0.929412}%
\pgfsetfillcolor{currentfill}%
\pgfsetlinewidth{0.250937pt}%
\definecolor{currentstroke}{rgb}{1.000000,1.000000,1.000000}%
\pgfsetstrokecolor{currentstroke}%
\pgfsetdash{}{0pt}%
\pgfpathmoveto{\pgfqpoint{3.013019in}{10.047925in}}%
\pgfpathlineto{\pgfqpoint{3.100754in}{10.047925in}}%
\pgfpathlineto{\pgfqpoint{3.100754in}{9.960189in}}%
\pgfpathlineto{\pgfqpoint{3.013019in}{9.960189in}}%
\pgfpathlineto{\pgfqpoint{3.013019in}{10.047925in}}%
\pgfusepath{stroke,fill}%
\end{pgfscope}%
\begin{pgfscope}%
\pgfpathrectangle{\pgfqpoint{0.380943in}{9.960189in}}{\pgfqpoint{4.650000in}{0.614151in}}%
\pgfusepath{clip}%
\pgfsetbuttcap%
\pgfsetroundjoin%
\definecolor{currentfill}{rgb}{1.000000,1.000000,0.929412}%
\pgfsetfillcolor{currentfill}%
\pgfsetlinewidth{0.250937pt}%
\definecolor{currentstroke}{rgb}{1.000000,1.000000,1.000000}%
\pgfsetstrokecolor{currentstroke}%
\pgfsetdash{}{0pt}%
\pgfpathmoveto{\pgfqpoint{3.100754in}{10.047925in}}%
\pgfpathlineto{\pgfqpoint{3.188490in}{10.047925in}}%
\pgfpathlineto{\pgfqpoint{3.188490in}{9.960189in}}%
\pgfpathlineto{\pgfqpoint{3.100754in}{9.960189in}}%
\pgfpathlineto{\pgfqpoint{3.100754in}{10.047925in}}%
\pgfusepath{stroke,fill}%
\end{pgfscope}%
\begin{pgfscope}%
\pgfpathrectangle{\pgfqpoint{0.380943in}{9.960189in}}{\pgfqpoint{4.650000in}{0.614151in}}%
\pgfusepath{clip}%
\pgfsetbuttcap%
\pgfsetroundjoin%
\definecolor{currentfill}{rgb}{1.000000,1.000000,0.929412}%
\pgfsetfillcolor{currentfill}%
\pgfsetlinewidth{0.250937pt}%
\definecolor{currentstroke}{rgb}{1.000000,1.000000,1.000000}%
\pgfsetstrokecolor{currentstroke}%
\pgfsetdash{}{0pt}%
\pgfpathmoveto{\pgfqpoint{3.188490in}{10.047925in}}%
\pgfpathlineto{\pgfqpoint{3.276226in}{10.047925in}}%
\pgfpathlineto{\pgfqpoint{3.276226in}{9.960189in}}%
\pgfpathlineto{\pgfqpoint{3.188490in}{9.960189in}}%
\pgfpathlineto{\pgfqpoint{3.188490in}{10.047925in}}%
\pgfusepath{stroke,fill}%
\end{pgfscope}%
\begin{pgfscope}%
\pgfpathrectangle{\pgfqpoint{0.380943in}{9.960189in}}{\pgfqpoint{4.650000in}{0.614151in}}%
\pgfusepath{clip}%
\pgfsetbuttcap%
\pgfsetroundjoin%
\definecolor{currentfill}{rgb}{1.000000,1.000000,0.929412}%
\pgfsetfillcolor{currentfill}%
\pgfsetlinewidth{0.250937pt}%
\definecolor{currentstroke}{rgb}{1.000000,1.000000,1.000000}%
\pgfsetstrokecolor{currentstroke}%
\pgfsetdash{}{0pt}%
\pgfpathmoveto{\pgfqpoint{3.276226in}{10.047925in}}%
\pgfpathlineto{\pgfqpoint{3.363962in}{10.047925in}}%
\pgfpathlineto{\pgfqpoint{3.363962in}{9.960189in}}%
\pgfpathlineto{\pgfqpoint{3.276226in}{9.960189in}}%
\pgfpathlineto{\pgfqpoint{3.276226in}{10.047925in}}%
\pgfusepath{stroke,fill}%
\end{pgfscope}%
\begin{pgfscope}%
\pgfpathrectangle{\pgfqpoint{0.380943in}{9.960189in}}{\pgfqpoint{4.650000in}{0.614151in}}%
\pgfusepath{clip}%
\pgfsetbuttcap%
\pgfsetroundjoin%
\definecolor{currentfill}{rgb}{1.000000,1.000000,0.929412}%
\pgfsetfillcolor{currentfill}%
\pgfsetlinewidth{0.250937pt}%
\definecolor{currentstroke}{rgb}{1.000000,1.000000,1.000000}%
\pgfsetstrokecolor{currentstroke}%
\pgfsetdash{}{0pt}%
\pgfpathmoveto{\pgfqpoint{3.363962in}{10.047925in}}%
\pgfpathlineto{\pgfqpoint{3.451698in}{10.047925in}}%
\pgfpathlineto{\pgfqpoint{3.451698in}{9.960189in}}%
\pgfpathlineto{\pgfqpoint{3.363962in}{9.960189in}}%
\pgfpathlineto{\pgfqpoint{3.363962in}{10.047925in}}%
\pgfusepath{stroke,fill}%
\end{pgfscope}%
\begin{pgfscope}%
\pgfpathrectangle{\pgfqpoint{0.380943in}{9.960189in}}{\pgfqpoint{4.650000in}{0.614151in}}%
\pgfusepath{clip}%
\pgfsetbuttcap%
\pgfsetroundjoin%
\definecolor{currentfill}{rgb}{1.000000,1.000000,0.929412}%
\pgfsetfillcolor{currentfill}%
\pgfsetlinewidth{0.250937pt}%
\definecolor{currentstroke}{rgb}{1.000000,1.000000,1.000000}%
\pgfsetstrokecolor{currentstroke}%
\pgfsetdash{}{0pt}%
\pgfpathmoveto{\pgfqpoint{3.451698in}{10.047925in}}%
\pgfpathlineto{\pgfqpoint{3.539434in}{10.047925in}}%
\pgfpathlineto{\pgfqpoint{3.539434in}{9.960189in}}%
\pgfpathlineto{\pgfqpoint{3.451698in}{9.960189in}}%
\pgfpathlineto{\pgfqpoint{3.451698in}{10.047925in}}%
\pgfusepath{stroke,fill}%
\end{pgfscope}%
\begin{pgfscope}%
\pgfpathrectangle{\pgfqpoint{0.380943in}{9.960189in}}{\pgfqpoint{4.650000in}{0.614151in}}%
\pgfusepath{clip}%
\pgfsetbuttcap%
\pgfsetroundjoin%
\definecolor{currentfill}{rgb}{1.000000,1.000000,0.929412}%
\pgfsetfillcolor{currentfill}%
\pgfsetlinewidth{0.250937pt}%
\definecolor{currentstroke}{rgb}{1.000000,1.000000,1.000000}%
\pgfsetstrokecolor{currentstroke}%
\pgfsetdash{}{0pt}%
\pgfpathmoveto{\pgfqpoint{3.539434in}{10.047925in}}%
\pgfpathlineto{\pgfqpoint{3.627169in}{10.047925in}}%
\pgfpathlineto{\pgfqpoint{3.627169in}{9.960189in}}%
\pgfpathlineto{\pgfqpoint{3.539434in}{9.960189in}}%
\pgfpathlineto{\pgfqpoint{3.539434in}{10.047925in}}%
\pgfusepath{stroke,fill}%
\end{pgfscope}%
\begin{pgfscope}%
\pgfpathrectangle{\pgfqpoint{0.380943in}{9.960189in}}{\pgfqpoint{4.650000in}{0.614151in}}%
\pgfusepath{clip}%
\pgfsetbuttcap%
\pgfsetroundjoin%
\definecolor{currentfill}{rgb}{1.000000,1.000000,0.929412}%
\pgfsetfillcolor{currentfill}%
\pgfsetlinewidth{0.250937pt}%
\definecolor{currentstroke}{rgb}{1.000000,1.000000,1.000000}%
\pgfsetstrokecolor{currentstroke}%
\pgfsetdash{}{0pt}%
\pgfpathmoveto{\pgfqpoint{3.627169in}{10.047925in}}%
\pgfpathlineto{\pgfqpoint{3.714905in}{10.047925in}}%
\pgfpathlineto{\pgfqpoint{3.714905in}{9.960189in}}%
\pgfpathlineto{\pgfqpoint{3.627169in}{9.960189in}}%
\pgfpathlineto{\pgfqpoint{3.627169in}{10.047925in}}%
\pgfusepath{stroke,fill}%
\end{pgfscope}%
\begin{pgfscope}%
\pgfpathrectangle{\pgfqpoint{0.380943in}{9.960189in}}{\pgfqpoint{4.650000in}{0.614151in}}%
\pgfusepath{clip}%
\pgfsetbuttcap%
\pgfsetroundjoin%
\definecolor{currentfill}{rgb}{1.000000,1.000000,0.929412}%
\pgfsetfillcolor{currentfill}%
\pgfsetlinewidth{0.250937pt}%
\definecolor{currentstroke}{rgb}{1.000000,1.000000,1.000000}%
\pgfsetstrokecolor{currentstroke}%
\pgfsetdash{}{0pt}%
\pgfpathmoveto{\pgfqpoint{3.714905in}{10.047925in}}%
\pgfpathlineto{\pgfqpoint{3.802641in}{10.047925in}}%
\pgfpathlineto{\pgfqpoint{3.802641in}{9.960189in}}%
\pgfpathlineto{\pgfqpoint{3.714905in}{9.960189in}}%
\pgfpathlineto{\pgfqpoint{3.714905in}{10.047925in}}%
\pgfusepath{stroke,fill}%
\end{pgfscope}%
\begin{pgfscope}%
\pgfpathrectangle{\pgfqpoint{0.380943in}{9.960189in}}{\pgfqpoint{4.650000in}{0.614151in}}%
\pgfusepath{clip}%
\pgfsetbuttcap%
\pgfsetroundjoin%
\definecolor{currentfill}{rgb}{1.000000,1.000000,0.929412}%
\pgfsetfillcolor{currentfill}%
\pgfsetlinewidth{0.250937pt}%
\definecolor{currentstroke}{rgb}{1.000000,1.000000,1.000000}%
\pgfsetstrokecolor{currentstroke}%
\pgfsetdash{}{0pt}%
\pgfpathmoveto{\pgfqpoint{3.802641in}{10.047925in}}%
\pgfpathlineto{\pgfqpoint{3.890377in}{10.047925in}}%
\pgfpathlineto{\pgfqpoint{3.890377in}{9.960189in}}%
\pgfpathlineto{\pgfqpoint{3.802641in}{9.960189in}}%
\pgfpathlineto{\pgfqpoint{3.802641in}{10.047925in}}%
\pgfusepath{stroke,fill}%
\end{pgfscope}%
\begin{pgfscope}%
\pgfpathrectangle{\pgfqpoint{0.380943in}{9.960189in}}{\pgfqpoint{4.650000in}{0.614151in}}%
\pgfusepath{clip}%
\pgfsetbuttcap%
\pgfsetroundjoin%
\definecolor{currentfill}{rgb}{1.000000,1.000000,0.929412}%
\pgfsetfillcolor{currentfill}%
\pgfsetlinewidth{0.250937pt}%
\definecolor{currentstroke}{rgb}{1.000000,1.000000,1.000000}%
\pgfsetstrokecolor{currentstroke}%
\pgfsetdash{}{0pt}%
\pgfpathmoveto{\pgfqpoint{3.890377in}{10.047925in}}%
\pgfpathlineto{\pgfqpoint{3.978113in}{10.047925in}}%
\pgfpathlineto{\pgfqpoint{3.978113in}{9.960189in}}%
\pgfpathlineto{\pgfqpoint{3.890377in}{9.960189in}}%
\pgfpathlineto{\pgfqpoint{3.890377in}{10.047925in}}%
\pgfusepath{stroke,fill}%
\end{pgfscope}%
\begin{pgfscope}%
\pgfpathrectangle{\pgfqpoint{0.380943in}{9.960189in}}{\pgfqpoint{4.650000in}{0.614151in}}%
\pgfusepath{clip}%
\pgfsetbuttcap%
\pgfsetroundjoin%
\definecolor{currentfill}{rgb}{1.000000,1.000000,0.929412}%
\pgfsetfillcolor{currentfill}%
\pgfsetlinewidth{0.250937pt}%
\definecolor{currentstroke}{rgb}{1.000000,1.000000,1.000000}%
\pgfsetstrokecolor{currentstroke}%
\pgfsetdash{}{0pt}%
\pgfpathmoveto{\pgfqpoint{3.978113in}{10.047925in}}%
\pgfpathlineto{\pgfqpoint{4.065849in}{10.047925in}}%
\pgfpathlineto{\pgfqpoint{4.065849in}{9.960189in}}%
\pgfpathlineto{\pgfqpoint{3.978113in}{9.960189in}}%
\pgfpathlineto{\pgfqpoint{3.978113in}{10.047925in}}%
\pgfusepath{stroke,fill}%
\end{pgfscope}%
\begin{pgfscope}%
\pgfpathrectangle{\pgfqpoint{0.380943in}{9.960189in}}{\pgfqpoint{4.650000in}{0.614151in}}%
\pgfusepath{clip}%
\pgfsetbuttcap%
\pgfsetroundjoin%
\definecolor{currentfill}{rgb}{1.000000,1.000000,0.929412}%
\pgfsetfillcolor{currentfill}%
\pgfsetlinewidth{0.250937pt}%
\definecolor{currentstroke}{rgb}{1.000000,1.000000,1.000000}%
\pgfsetstrokecolor{currentstroke}%
\pgfsetdash{}{0pt}%
\pgfpathmoveto{\pgfqpoint{4.065849in}{10.047925in}}%
\pgfpathlineto{\pgfqpoint{4.153585in}{10.047925in}}%
\pgfpathlineto{\pgfqpoint{4.153585in}{9.960189in}}%
\pgfpathlineto{\pgfqpoint{4.065849in}{9.960189in}}%
\pgfpathlineto{\pgfqpoint{4.065849in}{10.047925in}}%
\pgfusepath{stroke,fill}%
\end{pgfscope}%
\begin{pgfscope}%
\pgfpathrectangle{\pgfqpoint{0.380943in}{9.960189in}}{\pgfqpoint{4.650000in}{0.614151in}}%
\pgfusepath{clip}%
\pgfsetbuttcap%
\pgfsetroundjoin%
\definecolor{currentfill}{rgb}{1.000000,1.000000,0.929412}%
\pgfsetfillcolor{currentfill}%
\pgfsetlinewidth{0.250937pt}%
\definecolor{currentstroke}{rgb}{1.000000,1.000000,1.000000}%
\pgfsetstrokecolor{currentstroke}%
\pgfsetdash{}{0pt}%
\pgfpathmoveto{\pgfqpoint{4.153585in}{10.047925in}}%
\pgfpathlineto{\pgfqpoint{4.241320in}{10.047925in}}%
\pgfpathlineto{\pgfqpoint{4.241320in}{9.960189in}}%
\pgfpathlineto{\pgfqpoint{4.153585in}{9.960189in}}%
\pgfpathlineto{\pgfqpoint{4.153585in}{10.047925in}}%
\pgfusepath{stroke,fill}%
\end{pgfscope}%
\begin{pgfscope}%
\pgfpathrectangle{\pgfqpoint{0.380943in}{9.960189in}}{\pgfqpoint{4.650000in}{0.614151in}}%
\pgfusepath{clip}%
\pgfsetbuttcap%
\pgfsetroundjoin%
\definecolor{currentfill}{rgb}{1.000000,1.000000,0.929412}%
\pgfsetfillcolor{currentfill}%
\pgfsetlinewidth{0.250937pt}%
\definecolor{currentstroke}{rgb}{1.000000,1.000000,1.000000}%
\pgfsetstrokecolor{currentstroke}%
\pgfsetdash{}{0pt}%
\pgfpathmoveto{\pgfqpoint{4.241320in}{10.047925in}}%
\pgfpathlineto{\pgfqpoint{4.329056in}{10.047925in}}%
\pgfpathlineto{\pgfqpoint{4.329056in}{9.960189in}}%
\pgfpathlineto{\pgfqpoint{4.241320in}{9.960189in}}%
\pgfpathlineto{\pgfqpoint{4.241320in}{10.047925in}}%
\pgfusepath{stroke,fill}%
\end{pgfscope}%
\begin{pgfscope}%
\pgfpathrectangle{\pgfqpoint{0.380943in}{9.960189in}}{\pgfqpoint{4.650000in}{0.614151in}}%
\pgfusepath{clip}%
\pgfsetbuttcap%
\pgfsetroundjoin%
\definecolor{currentfill}{rgb}{1.000000,1.000000,0.929412}%
\pgfsetfillcolor{currentfill}%
\pgfsetlinewidth{0.250937pt}%
\definecolor{currentstroke}{rgb}{1.000000,1.000000,1.000000}%
\pgfsetstrokecolor{currentstroke}%
\pgfsetdash{}{0pt}%
\pgfpathmoveto{\pgfqpoint{4.329056in}{10.047925in}}%
\pgfpathlineto{\pgfqpoint{4.416792in}{10.047925in}}%
\pgfpathlineto{\pgfqpoint{4.416792in}{9.960189in}}%
\pgfpathlineto{\pgfqpoint{4.329056in}{9.960189in}}%
\pgfpathlineto{\pgfqpoint{4.329056in}{10.047925in}}%
\pgfusepath{stroke,fill}%
\end{pgfscope}%
\begin{pgfscope}%
\pgfpathrectangle{\pgfqpoint{0.380943in}{9.960189in}}{\pgfqpoint{4.650000in}{0.614151in}}%
\pgfusepath{clip}%
\pgfsetbuttcap%
\pgfsetroundjoin%
\definecolor{currentfill}{rgb}{0.988466,0.980392,0.801384}%
\pgfsetfillcolor{currentfill}%
\pgfsetlinewidth{0.250937pt}%
\definecolor{currentstroke}{rgb}{1.000000,1.000000,1.000000}%
\pgfsetstrokecolor{currentstroke}%
\pgfsetdash{}{0pt}%
\pgfpathmoveto{\pgfqpoint{4.416792in}{10.047925in}}%
\pgfpathlineto{\pgfqpoint{4.504528in}{10.047925in}}%
\pgfpathlineto{\pgfqpoint{4.504528in}{9.960189in}}%
\pgfpathlineto{\pgfqpoint{4.416792in}{9.960189in}}%
\pgfpathlineto{\pgfqpoint{4.416792in}{10.047925in}}%
\pgfusepath{stroke,fill}%
\end{pgfscope}%
\begin{pgfscope}%
\pgfpathrectangle{\pgfqpoint{0.380943in}{9.960189in}}{\pgfqpoint{4.650000in}{0.614151in}}%
\pgfusepath{clip}%
\pgfsetbuttcap%
\pgfsetroundjoin%
\definecolor{currentfill}{rgb}{0.963091,0.919493,0.720185}%
\pgfsetfillcolor{currentfill}%
\pgfsetlinewidth{0.250937pt}%
\definecolor{currentstroke}{rgb}{1.000000,1.000000,1.000000}%
\pgfsetstrokecolor{currentstroke}%
\pgfsetdash{}{0pt}%
\pgfpathmoveto{\pgfqpoint{4.504528in}{10.047925in}}%
\pgfpathlineto{\pgfqpoint{4.592264in}{10.047925in}}%
\pgfpathlineto{\pgfqpoint{4.592264in}{9.960189in}}%
\pgfpathlineto{\pgfqpoint{4.504528in}{9.960189in}}%
\pgfpathlineto{\pgfqpoint{4.504528in}{10.047925in}}%
\pgfusepath{stroke,fill}%
\end{pgfscope}%
\begin{pgfscope}%
\pgfpathrectangle{\pgfqpoint{0.380943in}{9.960189in}}{\pgfqpoint{4.650000in}{0.614151in}}%
\pgfusepath{clip}%
\pgfsetbuttcap%
\pgfsetroundjoin%
\definecolor{currentfill}{rgb}{0.975087,0.857901,0.686044}%
\pgfsetfillcolor{currentfill}%
\pgfsetlinewidth{0.250937pt}%
\definecolor{currentstroke}{rgb}{1.000000,1.000000,1.000000}%
\pgfsetstrokecolor{currentstroke}%
\pgfsetdash{}{0pt}%
\pgfpathmoveto{\pgfqpoint{4.592264in}{10.047925in}}%
\pgfpathlineto{\pgfqpoint{4.680000in}{10.047925in}}%
\pgfpathlineto{\pgfqpoint{4.680000in}{9.960189in}}%
\pgfpathlineto{\pgfqpoint{4.592264in}{9.960189in}}%
\pgfpathlineto{\pgfqpoint{4.592264in}{10.047925in}}%
\pgfusepath{stroke,fill}%
\end{pgfscope}%
\begin{pgfscope}%
\pgfpathrectangle{\pgfqpoint{0.380943in}{9.960189in}}{\pgfqpoint{4.650000in}{0.614151in}}%
\pgfusepath{clip}%
\pgfsetbuttcap%
\pgfsetroundjoin%
\definecolor{currentfill}{rgb}{0.988466,0.980392,0.801384}%
\pgfsetfillcolor{currentfill}%
\pgfsetlinewidth{0.250937pt}%
\definecolor{currentstroke}{rgb}{1.000000,1.000000,1.000000}%
\pgfsetstrokecolor{currentstroke}%
\pgfsetdash{}{0pt}%
\pgfpathmoveto{\pgfqpoint{4.680000in}{10.047925in}}%
\pgfpathlineto{\pgfqpoint{4.767736in}{10.047925in}}%
\pgfpathlineto{\pgfqpoint{4.767736in}{9.960189in}}%
\pgfpathlineto{\pgfqpoint{4.680000in}{9.960189in}}%
\pgfpathlineto{\pgfqpoint{4.680000in}{10.047925in}}%
\pgfusepath{stroke,fill}%
\end{pgfscope}%
\begin{pgfscope}%
\pgfpathrectangle{\pgfqpoint{0.380943in}{9.960189in}}{\pgfqpoint{4.650000in}{0.614151in}}%
\pgfusepath{clip}%
\pgfsetbuttcap%
\pgfsetroundjoin%
\definecolor{currentfill}{rgb}{0.963091,0.919493,0.720185}%
\pgfsetfillcolor{currentfill}%
\pgfsetlinewidth{0.250937pt}%
\definecolor{currentstroke}{rgb}{1.000000,1.000000,1.000000}%
\pgfsetstrokecolor{currentstroke}%
\pgfsetdash{}{0pt}%
\pgfpathmoveto{\pgfqpoint{4.767736in}{10.047925in}}%
\pgfpathlineto{\pgfqpoint{4.855471in}{10.047925in}}%
\pgfpathlineto{\pgfqpoint{4.855471in}{9.960189in}}%
\pgfpathlineto{\pgfqpoint{4.767736in}{9.960189in}}%
\pgfpathlineto{\pgfqpoint{4.767736in}{10.047925in}}%
\pgfusepath{stroke,fill}%
\end{pgfscope}%
\begin{pgfscope}%
\pgfpathrectangle{\pgfqpoint{0.380943in}{9.960189in}}{\pgfqpoint{4.650000in}{0.614151in}}%
\pgfusepath{clip}%
\pgfsetbuttcap%
\pgfsetroundjoin%
\definecolor{currentfill}{rgb}{1.000000,1.000000,0.865975}%
\pgfsetfillcolor{currentfill}%
\pgfsetlinewidth{0.250937pt}%
\definecolor{currentstroke}{rgb}{1.000000,1.000000,1.000000}%
\pgfsetstrokecolor{currentstroke}%
\pgfsetdash{}{0pt}%
\pgfpathmoveto{\pgfqpoint{4.855471in}{10.047925in}}%
\pgfpathlineto{\pgfqpoint{4.943207in}{10.047925in}}%
\pgfpathlineto{\pgfqpoint{4.943207in}{9.960189in}}%
\pgfpathlineto{\pgfqpoint{4.855471in}{9.960189in}}%
\pgfpathlineto{\pgfqpoint{4.855471in}{10.047925in}}%
\pgfusepath{stroke,fill}%
\end{pgfscope}%
\begin{pgfscope}%
\pgfpathrectangle{\pgfqpoint{0.380943in}{9.960189in}}{\pgfqpoint{4.650000in}{0.614151in}}%
\pgfusepath{clip}%
\pgfsetbuttcap%
\pgfsetroundjoin%
\pgfsetlinewidth{0.250937pt}%
\definecolor{currentstroke}{rgb}{1.000000,1.000000,1.000000}%
\pgfsetstrokecolor{currentstroke}%
\pgfsetdash{}{0pt}%
\pgfpathmoveto{\pgfqpoint{4.943207in}{10.047925in}}%
\pgfpathlineto{\pgfqpoint{5.030943in}{10.047925in}}%
\pgfpathlineto{\pgfqpoint{5.030943in}{9.960189in}}%
\pgfpathlineto{\pgfqpoint{4.943207in}{9.960189in}}%
\pgfpathlineto{\pgfqpoint{4.943207in}{10.047925in}}%
\pgfusepath{stroke}%
\end{pgfscope}%
\begin{pgfscope}%
\pgfsetbuttcap%
\pgfsetroundjoin%
\definecolor{currentfill}{rgb}{0.000000,0.000000,0.000000}%
\pgfsetfillcolor{currentfill}%
\pgfsetlinewidth{0.803000pt}%
\definecolor{currentstroke}{rgb}{0.000000,0.000000,0.000000}%
\pgfsetstrokecolor{currentstroke}%
\pgfsetdash{}{0pt}%
\pgfsys@defobject{currentmarker}{\pgfqpoint{0.000000in}{-0.048611in}}{\pgfqpoint{0.000000in}{0.000000in}}{%
\pgfpathmoveto{\pgfqpoint{0.000000in}{0.000000in}}%
\pgfpathlineto{\pgfqpoint{0.000000in}{-0.048611in}}%
\pgfusepath{stroke,fill}%
}%
\begin{pgfscope}%
\pgfsys@transformshift{0.600283in}{9.960189in}%
\pgfsys@useobject{currentmarker}{}%
\end{pgfscope}%
\end{pgfscope}%
\begin{pgfscope}%
\definecolor{textcolor}{rgb}{0.000000,0.000000,0.000000}%
\pgfsetstrokecolor{textcolor}%
\pgfsetfillcolor{textcolor}%
\pgftext[x=0.600283in,y=9.862967in,,top]{\color{textcolor}\rmfamily\fontsize{8.000000}{9.600000}\selectfont Jan}%
\end{pgfscope}%
\begin{pgfscope}%
\pgfsetbuttcap%
\pgfsetroundjoin%
\definecolor{currentfill}{rgb}{0.000000,0.000000,0.000000}%
\pgfsetfillcolor{currentfill}%
\pgfsetlinewidth{0.803000pt}%
\definecolor{currentstroke}{rgb}{0.000000,0.000000,0.000000}%
\pgfsetstrokecolor{currentstroke}%
\pgfsetdash{}{0pt}%
\pgfsys@defobject{currentmarker}{\pgfqpoint{0.000000in}{-0.048611in}}{\pgfqpoint{0.000000in}{0.000000in}}{%
\pgfpathmoveto{\pgfqpoint{0.000000in}{0.000000in}}%
\pgfpathlineto{\pgfqpoint{0.000000in}{-0.048611in}}%
\pgfusepath{stroke,fill}%
}%
\begin{pgfscope}%
\pgfsys@transformshift{1.038962in}{9.960189in}%
\pgfsys@useobject{currentmarker}{}%
\end{pgfscope}%
\end{pgfscope}%
\begin{pgfscope}%
\definecolor{textcolor}{rgb}{0.000000,0.000000,0.000000}%
\pgfsetstrokecolor{textcolor}%
\pgfsetfillcolor{textcolor}%
\pgftext[x=1.038962in,y=9.862967in,,top]{\color{textcolor}\rmfamily\fontsize{8.000000}{9.600000}\selectfont Feb}%
\end{pgfscope}%
\begin{pgfscope}%
\pgfsetbuttcap%
\pgfsetroundjoin%
\definecolor{currentfill}{rgb}{0.000000,0.000000,0.000000}%
\pgfsetfillcolor{currentfill}%
\pgfsetlinewidth{0.803000pt}%
\definecolor{currentstroke}{rgb}{0.000000,0.000000,0.000000}%
\pgfsetstrokecolor{currentstroke}%
\pgfsetdash{}{0pt}%
\pgfsys@defobject{currentmarker}{\pgfqpoint{0.000000in}{-0.048611in}}{\pgfqpoint{0.000000in}{0.000000in}}{%
\pgfpathmoveto{\pgfqpoint{0.000000in}{0.000000in}}%
\pgfpathlineto{\pgfqpoint{0.000000in}{-0.048611in}}%
\pgfusepath{stroke,fill}%
}%
\begin{pgfscope}%
\pgfsys@transformshift{1.389905in}{9.960189in}%
\pgfsys@useobject{currentmarker}{}%
\end{pgfscope}%
\end{pgfscope}%
\begin{pgfscope}%
\definecolor{textcolor}{rgb}{0.000000,0.000000,0.000000}%
\pgfsetstrokecolor{textcolor}%
\pgfsetfillcolor{textcolor}%
\pgftext[x=1.389905in,y=9.862967in,,top]{\color{textcolor}\rmfamily\fontsize{8.000000}{9.600000}\selectfont Mar}%
\end{pgfscope}%
\begin{pgfscope}%
\pgfsetbuttcap%
\pgfsetroundjoin%
\definecolor{currentfill}{rgb}{0.000000,0.000000,0.000000}%
\pgfsetfillcolor{currentfill}%
\pgfsetlinewidth{0.803000pt}%
\definecolor{currentstroke}{rgb}{0.000000,0.000000,0.000000}%
\pgfsetstrokecolor{currentstroke}%
\pgfsetdash{}{0pt}%
\pgfsys@defobject{currentmarker}{\pgfqpoint{0.000000in}{-0.048611in}}{\pgfqpoint{0.000000in}{0.000000in}}{%
\pgfpathmoveto{\pgfqpoint{0.000000in}{0.000000in}}%
\pgfpathlineto{\pgfqpoint{0.000000in}{-0.048611in}}%
\pgfusepath{stroke,fill}%
}%
\begin{pgfscope}%
\pgfsys@transformshift{1.740849in}{9.960189in}%
\pgfsys@useobject{currentmarker}{}%
\end{pgfscope}%
\end{pgfscope}%
\begin{pgfscope}%
\definecolor{textcolor}{rgb}{0.000000,0.000000,0.000000}%
\pgfsetstrokecolor{textcolor}%
\pgfsetfillcolor{textcolor}%
\pgftext[x=1.740849in,y=9.862967in,,top]{\color{textcolor}\rmfamily\fontsize{8.000000}{9.600000}\selectfont Apr}%
\end{pgfscope}%
\begin{pgfscope}%
\pgfsetbuttcap%
\pgfsetroundjoin%
\definecolor{currentfill}{rgb}{0.000000,0.000000,0.000000}%
\pgfsetfillcolor{currentfill}%
\pgfsetlinewidth{0.803000pt}%
\definecolor{currentstroke}{rgb}{0.000000,0.000000,0.000000}%
\pgfsetstrokecolor{currentstroke}%
\pgfsetdash{}{0pt}%
\pgfsys@defobject{currentmarker}{\pgfqpoint{0.000000in}{-0.048611in}}{\pgfqpoint{0.000000in}{0.000000in}}{%
\pgfpathmoveto{\pgfqpoint{0.000000in}{0.000000in}}%
\pgfpathlineto{\pgfqpoint{0.000000in}{-0.048611in}}%
\pgfusepath{stroke,fill}%
}%
\begin{pgfscope}%
\pgfsys@transformshift{2.135660in}{9.960189in}%
\pgfsys@useobject{currentmarker}{}%
\end{pgfscope}%
\end{pgfscope}%
\begin{pgfscope}%
\definecolor{textcolor}{rgb}{0.000000,0.000000,0.000000}%
\pgfsetstrokecolor{textcolor}%
\pgfsetfillcolor{textcolor}%
\pgftext[x=2.135660in,y=9.862967in,,top]{\color{textcolor}\rmfamily\fontsize{8.000000}{9.600000}\selectfont May}%
\end{pgfscope}%
\begin{pgfscope}%
\pgfsetbuttcap%
\pgfsetroundjoin%
\definecolor{currentfill}{rgb}{0.000000,0.000000,0.000000}%
\pgfsetfillcolor{currentfill}%
\pgfsetlinewidth{0.803000pt}%
\definecolor{currentstroke}{rgb}{0.000000,0.000000,0.000000}%
\pgfsetstrokecolor{currentstroke}%
\pgfsetdash{}{0pt}%
\pgfsys@defobject{currentmarker}{\pgfqpoint{0.000000in}{-0.048611in}}{\pgfqpoint{0.000000in}{0.000000in}}{%
\pgfpathmoveto{\pgfqpoint{0.000000in}{0.000000in}}%
\pgfpathlineto{\pgfqpoint{0.000000in}{-0.048611in}}%
\pgfusepath{stroke,fill}%
}%
\begin{pgfscope}%
\pgfsys@transformshift{2.530471in}{9.960189in}%
\pgfsys@useobject{currentmarker}{}%
\end{pgfscope}%
\end{pgfscope}%
\begin{pgfscope}%
\definecolor{textcolor}{rgb}{0.000000,0.000000,0.000000}%
\pgfsetstrokecolor{textcolor}%
\pgfsetfillcolor{textcolor}%
\pgftext[x=2.530471in,y=9.862967in,,top]{\color{textcolor}\rmfamily\fontsize{8.000000}{9.600000}\selectfont Jun}%
\end{pgfscope}%
\begin{pgfscope}%
\pgfsetbuttcap%
\pgfsetroundjoin%
\definecolor{currentfill}{rgb}{0.000000,0.000000,0.000000}%
\pgfsetfillcolor{currentfill}%
\pgfsetlinewidth{0.803000pt}%
\definecolor{currentstroke}{rgb}{0.000000,0.000000,0.000000}%
\pgfsetstrokecolor{currentstroke}%
\pgfsetdash{}{0pt}%
\pgfsys@defobject{currentmarker}{\pgfqpoint{0.000000in}{-0.048611in}}{\pgfqpoint{0.000000in}{0.000000in}}{%
\pgfpathmoveto{\pgfqpoint{0.000000in}{0.000000in}}%
\pgfpathlineto{\pgfqpoint{0.000000in}{-0.048611in}}%
\pgfusepath{stroke,fill}%
}%
\begin{pgfscope}%
\pgfsys@transformshift{2.881415in}{9.960189in}%
\pgfsys@useobject{currentmarker}{}%
\end{pgfscope}%
\end{pgfscope}%
\begin{pgfscope}%
\definecolor{textcolor}{rgb}{0.000000,0.000000,0.000000}%
\pgfsetstrokecolor{textcolor}%
\pgfsetfillcolor{textcolor}%
\pgftext[x=2.881415in,y=9.862967in,,top]{\color{textcolor}\rmfamily\fontsize{8.000000}{9.600000}\selectfont Jul}%
\end{pgfscope}%
\begin{pgfscope}%
\pgfsetbuttcap%
\pgfsetroundjoin%
\definecolor{currentfill}{rgb}{0.000000,0.000000,0.000000}%
\pgfsetfillcolor{currentfill}%
\pgfsetlinewidth{0.803000pt}%
\definecolor{currentstroke}{rgb}{0.000000,0.000000,0.000000}%
\pgfsetstrokecolor{currentstroke}%
\pgfsetdash{}{0pt}%
\pgfsys@defobject{currentmarker}{\pgfqpoint{0.000000in}{-0.048611in}}{\pgfqpoint{0.000000in}{0.000000in}}{%
\pgfpathmoveto{\pgfqpoint{0.000000in}{0.000000in}}%
\pgfpathlineto{\pgfqpoint{0.000000in}{-0.048611in}}%
\pgfusepath{stroke,fill}%
}%
\begin{pgfscope}%
\pgfsys@transformshift{3.320094in}{9.960189in}%
\pgfsys@useobject{currentmarker}{}%
\end{pgfscope}%
\end{pgfscope}%
\begin{pgfscope}%
\definecolor{textcolor}{rgb}{0.000000,0.000000,0.000000}%
\pgfsetstrokecolor{textcolor}%
\pgfsetfillcolor{textcolor}%
\pgftext[x=3.320094in,y=9.862967in,,top]{\color{textcolor}\rmfamily\fontsize{8.000000}{9.600000}\selectfont Aug}%
\end{pgfscope}%
\begin{pgfscope}%
\pgfsetbuttcap%
\pgfsetroundjoin%
\definecolor{currentfill}{rgb}{0.000000,0.000000,0.000000}%
\pgfsetfillcolor{currentfill}%
\pgfsetlinewidth{0.803000pt}%
\definecolor{currentstroke}{rgb}{0.000000,0.000000,0.000000}%
\pgfsetstrokecolor{currentstroke}%
\pgfsetdash{}{0pt}%
\pgfsys@defobject{currentmarker}{\pgfqpoint{0.000000in}{-0.048611in}}{\pgfqpoint{0.000000in}{0.000000in}}{%
\pgfpathmoveto{\pgfqpoint{0.000000in}{0.000000in}}%
\pgfpathlineto{\pgfqpoint{0.000000in}{-0.048611in}}%
\pgfusepath{stroke,fill}%
}%
\begin{pgfscope}%
\pgfsys@transformshift{3.671037in}{9.960189in}%
\pgfsys@useobject{currentmarker}{}%
\end{pgfscope}%
\end{pgfscope}%
\begin{pgfscope}%
\definecolor{textcolor}{rgb}{0.000000,0.000000,0.000000}%
\pgfsetstrokecolor{textcolor}%
\pgfsetfillcolor{textcolor}%
\pgftext[x=3.671037in,y=9.862967in,,top]{\color{textcolor}\rmfamily\fontsize{8.000000}{9.600000}\selectfont Sep}%
\end{pgfscope}%
\begin{pgfscope}%
\pgfsetbuttcap%
\pgfsetroundjoin%
\definecolor{currentfill}{rgb}{0.000000,0.000000,0.000000}%
\pgfsetfillcolor{currentfill}%
\pgfsetlinewidth{0.803000pt}%
\definecolor{currentstroke}{rgb}{0.000000,0.000000,0.000000}%
\pgfsetstrokecolor{currentstroke}%
\pgfsetdash{}{0pt}%
\pgfsys@defobject{currentmarker}{\pgfqpoint{0.000000in}{-0.048611in}}{\pgfqpoint{0.000000in}{0.000000in}}{%
\pgfpathmoveto{\pgfqpoint{0.000000in}{0.000000in}}%
\pgfpathlineto{\pgfqpoint{0.000000in}{-0.048611in}}%
\pgfusepath{stroke,fill}%
}%
\begin{pgfscope}%
\pgfsys@transformshift{4.065849in}{9.960189in}%
\pgfsys@useobject{currentmarker}{}%
\end{pgfscope}%
\end{pgfscope}%
\begin{pgfscope}%
\definecolor{textcolor}{rgb}{0.000000,0.000000,0.000000}%
\pgfsetstrokecolor{textcolor}%
\pgfsetfillcolor{textcolor}%
\pgftext[x=4.065849in,y=9.862967in,,top]{\color{textcolor}\rmfamily\fontsize{8.000000}{9.600000}\selectfont Oct}%
\end{pgfscope}%
\begin{pgfscope}%
\pgfsetbuttcap%
\pgfsetroundjoin%
\definecolor{currentfill}{rgb}{0.000000,0.000000,0.000000}%
\pgfsetfillcolor{currentfill}%
\pgfsetlinewidth{0.803000pt}%
\definecolor{currentstroke}{rgb}{0.000000,0.000000,0.000000}%
\pgfsetstrokecolor{currentstroke}%
\pgfsetdash{}{0pt}%
\pgfsys@defobject{currentmarker}{\pgfqpoint{0.000000in}{-0.048611in}}{\pgfqpoint{0.000000in}{0.000000in}}{%
\pgfpathmoveto{\pgfqpoint{0.000000in}{0.000000in}}%
\pgfpathlineto{\pgfqpoint{0.000000in}{-0.048611in}}%
\pgfusepath{stroke,fill}%
}%
\begin{pgfscope}%
\pgfsys@transformshift{4.460660in}{9.960189in}%
\pgfsys@useobject{currentmarker}{}%
\end{pgfscope}%
\end{pgfscope}%
\begin{pgfscope}%
\definecolor{textcolor}{rgb}{0.000000,0.000000,0.000000}%
\pgfsetstrokecolor{textcolor}%
\pgfsetfillcolor{textcolor}%
\pgftext[x=4.460660in,y=9.862967in,,top]{\color{textcolor}\rmfamily\fontsize{8.000000}{9.600000}\selectfont Nov}%
\end{pgfscope}%
\begin{pgfscope}%
\pgfsetbuttcap%
\pgfsetroundjoin%
\definecolor{currentfill}{rgb}{0.000000,0.000000,0.000000}%
\pgfsetfillcolor{currentfill}%
\pgfsetlinewidth{0.803000pt}%
\definecolor{currentstroke}{rgb}{0.000000,0.000000,0.000000}%
\pgfsetstrokecolor{currentstroke}%
\pgfsetdash{}{0pt}%
\pgfsys@defobject{currentmarker}{\pgfqpoint{0.000000in}{-0.048611in}}{\pgfqpoint{0.000000in}{0.000000in}}{%
\pgfpathmoveto{\pgfqpoint{0.000000in}{0.000000in}}%
\pgfpathlineto{\pgfqpoint{0.000000in}{-0.048611in}}%
\pgfusepath{stroke,fill}%
}%
\begin{pgfscope}%
\pgfsys@transformshift{4.811603in}{9.960189in}%
\pgfsys@useobject{currentmarker}{}%
\end{pgfscope}%
\end{pgfscope}%
\begin{pgfscope}%
\definecolor{textcolor}{rgb}{0.000000,0.000000,0.000000}%
\pgfsetstrokecolor{textcolor}%
\pgfsetfillcolor{textcolor}%
\pgftext[x=4.811603in,y=9.862967in,,top]{\color{textcolor}\rmfamily\fontsize{8.000000}{9.600000}\selectfont Dec}%
\end{pgfscope}%
\begin{pgfscope}%
\pgfsetbuttcap%
\pgfsetroundjoin%
\definecolor{currentfill}{rgb}{0.000000,0.000000,0.000000}%
\pgfsetfillcolor{currentfill}%
\pgfsetlinewidth{0.803000pt}%
\definecolor{currentstroke}{rgb}{0.000000,0.000000,0.000000}%
\pgfsetstrokecolor{currentstroke}%
\pgfsetdash{}{0pt}%
\pgfsys@defobject{currentmarker}{\pgfqpoint{-0.048611in}{0.000000in}}{\pgfqpoint{-0.000000in}{0.000000in}}{%
\pgfpathmoveto{\pgfqpoint{-0.000000in}{0.000000in}}%
\pgfpathlineto{\pgfqpoint{-0.048611in}{0.000000in}}%
\pgfusepath{stroke,fill}%
}%
\begin{pgfscope}%
\pgfsys@transformshift{0.380943in}{10.530472in}%
\pgfsys@useobject{currentmarker}{}%
\end{pgfscope}%
\end{pgfscope}%
\begin{pgfscope}%
\definecolor{textcolor}{rgb}{0.000000,0.000000,0.000000}%
\pgfsetstrokecolor{textcolor}%
\pgfsetfillcolor{textcolor}%
\pgftext[x=0.113117in, y=10.491892in, left, base]{\color{textcolor}\rmfamily\fontsize{8.000000}{9.600000}\selectfont M}%
\end{pgfscope}%
\begin{pgfscope}%
\pgfsetbuttcap%
\pgfsetroundjoin%
\definecolor{currentfill}{rgb}{0.000000,0.000000,0.000000}%
\pgfsetfillcolor{currentfill}%
\pgfsetlinewidth{0.803000pt}%
\definecolor{currentstroke}{rgb}{0.000000,0.000000,0.000000}%
\pgfsetstrokecolor{currentstroke}%
\pgfsetdash{}{0pt}%
\pgfsys@defobject{currentmarker}{\pgfqpoint{-0.048611in}{0.000000in}}{\pgfqpoint{-0.000000in}{0.000000in}}{%
\pgfpathmoveto{\pgfqpoint{-0.000000in}{0.000000in}}%
\pgfpathlineto{\pgfqpoint{-0.048611in}{0.000000in}}%
\pgfusepath{stroke,fill}%
}%
\begin{pgfscope}%
\pgfsys@transformshift{0.380943in}{10.442736in}%
\pgfsys@useobject{currentmarker}{}%
\end{pgfscope}%
\end{pgfscope}%
\begin{pgfscope}%
\definecolor{textcolor}{rgb}{0.000000,0.000000,0.000000}%
\pgfsetstrokecolor{textcolor}%
\pgfsetfillcolor{textcolor}%
\pgftext[x=0.135957in, y=10.404156in, left, base]{\color{textcolor}\rmfamily\fontsize{8.000000}{9.600000}\selectfont T}%
\end{pgfscope}%
\begin{pgfscope}%
\pgfsetbuttcap%
\pgfsetroundjoin%
\definecolor{currentfill}{rgb}{0.000000,0.000000,0.000000}%
\pgfsetfillcolor{currentfill}%
\pgfsetlinewidth{0.803000pt}%
\definecolor{currentstroke}{rgb}{0.000000,0.000000,0.000000}%
\pgfsetstrokecolor{currentstroke}%
\pgfsetdash{}{0pt}%
\pgfsys@defobject{currentmarker}{\pgfqpoint{-0.048611in}{0.000000in}}{\pgfqpoint{-0.000000in}{0.000000in}}{%
\pgfpathmoveto{\pgfqpoint{-0.000000in}{0.000000in}}%
\pgfpathlineto{\pgfqpoint{-0.048611in}{0.000000in}}%
\pgfusepath{stroke,fill}%
}%
\begin{pgfscope}%
\pgfsys@transformshift{0.380943in}{10.355000in}%
\pgfsys@useobject{currentmarker}{}%
\end{pgfscope}%
\end{pgfscope}%
\begin{pgfscope}%
\definecolor{textcolor}{rgb}{0.000000,0.000000,0.000000}%
\pgfsetstrokecolor{textcolor}%
\pgfsetfillcolor{textcolor}%
\pgftext[x=0.100000in, y=10.316420in, left, base]{\color{textcolor}\rmfamily\fontsize{8.000000}{9.600000}\selectfont W}%
\end{pgfscope}%
\begin{pgfscope}%
\pgfsetbuttcap%
\pgfsetroundjoin%
\definecolor{currentfill}{rgb}{0.000000,0.000000,0.000000}%
\pgfsetfillcolor{currentfill}%
\pgfsetlinewidth{0.803000pt}%
\definecolor{currentstroke}{rgb}{0.000000,0.000000,0.000000}%
\pgfsetstrokecolor{currentstroke}%
\pgfsetdash{}{0pt}%
\pgfsys@defobject{currentmarker}{\pgfqpoint{-0.048611in}{0.000000in}}{\pgfqpoint{-0.000000in}{0.000000in}}{%
\pgfpathmoveto{\pgfqpoint{-0.000000in}{0.000000in}}%
\pgfpathlineto{\pgfqpoint{-0.048611in}{0.000000in}}%
\pgfusepath{stroke,fill}%
}%
\begin{pgfscope}%
\pgfsys@transformshift{0.380943in}{10.267264in}%
\pgfsys@useobject{currentmarker}{}%
\end{pgfscope}%
\end{pgfscope}%
\begin{pgfscope}%
\definecolor{textcolor}{rgb}{0.000000,0.000000,0.000000}%
\pgfsetstrokecolor{textcolor}%
\pgfsetfillcolor{textcolor}%
\pgftext[x=0.135957in, y=10.228684in, left, base]{\color{textcolor}\rmfamily\fontsize{8.000000}{9.600000}\selectfont T}%
\end{pgfscope}%
\begin{pgfscope}%
\pgfsetbuttcap%
\pgfsetroundjoin%
\definecolor{currentfill}{rgb}{0.000000,0.000000,0.000000}%
\pgfsetfillcolor{currentfill}%
\pgfsetlinewidth{0.803000pt}%
\definecolor{currentstroke}{rgb}{0.000000,0.000000,0.000000}%
\pgfsetstrokecolor{currentstroke}%
\pgfsetdash{}{0pt}%
\pgfsys@defobject{currentmarker}{\pgfqpoint{-0.048611in}{0.000000in}}{\pgfqpoint{-0.000000in}{0.000000in}}{%
\pgfpathmoveto{\pgfqpoint{-0.000000in}{0.000000in}}%
\pgfpathlineto{\pgfqpoint{-0.048611in}{0.000000in}}%
\pgfusepath{stroke,fill}%
}%
\begin{pgfscope}%
\pgfsys@transformshift{0.380943in}{10.179529in}%
\pgfsys@useobject{currentmarker}{}%
\end{pgfscope}%
\end{pgfscope}%
\begin{pgfscope}%
\definecolor{textcolor}{rgb}{0.000000,0.000000,0.000000}%
\pgfsetstrokecolor{textcolor}%
\pgfsetfillcolor{textcolor}%
\pgftext[x=0.144213in, y=10.140948in, left, base]{\color{textcolor}\rmfamily\fontsize{8.000000}{9.600000}\selectfont F}%
\end{pgfscope}%
\begin{pgfscope}%
\pgfsetbuttcap%
\pgfsetroundjoin%
\definecolor{currentfill}{rgb}{0.000000,0.000000,0.000000}%
\pgfsetfillcolor{currentfill}%
\pgfsetlinewidth{0.803000pt}%
\definecolor{currentstroke}{rgb}{0.000000,0.000000,0.000000}%
\pgfsetstrokecolor{currentstroke}%
\pgfsetdash{}{0pt}%
\pgfsys@defobject{currentmarker}{\pgfqpoint{-0.048611in}{0.000000in}}{\pgfqpoint{-0.000000in}{0.000000in}}{%
\pgfpathmoveto{\pgfqpoint{-0.000000in}{0.000000in}}%
\pgfpathlineto{\pgfqpoint{-0.048611in}{0.000000in}}%
\pgfusepath{stroke,fill}%
}%
\begin{pgfscope}%
\pgfsys@transformshift{0.380943in}{10.091793in}%
\pgfsys@useobject{currentmarker}{}%
\end{pgfscope}%
\end{pgfscope}%
\begin{pgfscope}%
\definecolor{textcolor}{rgb}{0.000000,0.000000,0.000000}%
\pgfsetstrokecolor{textcolor}%
\pgfsetfillcolor{textcolor}%
\pgftext[x=0.155633in, y=10.053212in, left, base]{\color{textcolor}\rmfamily\fontsize{8.000000}{9.600000}\selectfont S}%
\end{pgfscope}%
\begin{pgfscope}%
\pgfsetbuttcap%
\pgfsetroundjoin%
\definecolor{currentfill}{rgb}{0.000000,0.000000,0.000000}%
\pgfsetfillcolor{currentfill}%
\pgfsetlinewidth{0.803000pt}%
\definecolor{currentstroke}{rgb}{0.000000,0.000000,0.000000}%
\pgfsetstrokecolor{currentstroke}%
\pgfsetdash{}{0pt}%
\pgfsys@defobject{currentmarker}{\pgfqpoint{-0.048611in}{0.000000in}}{\pgfqpoint{-0.000000in}{0.000000in}}{%
\pgfpathmoveto{\pgfqpoint{-0.000000in}{0.000000in}}%
\pgfpathlineto{\pgfqpoint{-0.048611in}{0.000000in}}%
\pgfusepath{stroke,fill}%
}%
\begin{pgfscope}%
\pgfsys@transformshift{0.380943in}{10.004057in}%
\pgfsys@useobject{currentmarker}{}%
\end{pgfscope}%
\end{pgfscope}%
\begin{pgfscope}%
\definecolor{textcolor}{rgb}{0.000000,0.000000,0.000000}%
\pgfsetstrokecolor{textcolor}%
\pgfsetfillcolor{textcolor}%
\pgftext[x=0.155633in, y=9.965477in, left, base]{\color{textcolor}\rmfamily\fontsize{8.000000}{9.600000}\selectfont S}%
\end{pgfscope}%
\begin{pgfscope}%
\definecolor{textcolor}{rgb}{0.000000,0.000000,0.000000}%
\pgfsetstrokecolor{textcolor}%
\pgfsetfillcolor{textcolor}%
\pgftext[x=2.705943in,y=10.741007in,,]{\color{textcolor}\ttfamily\fontsize{14.400000}{17.280000}\selectfont 2016}%
\end{pgfscope}%
\begin{pgfscope}%
\pgfpathrectangle{\pgfqpoint{0.380943in}{8.035189in}}{\pgfqpoint{4.650000in}{0.614151in}}%
\pgfusepath{clip}%
\pgfsetbuttcap%
\pgfsetroundjoin%
\pgfsetlinewidth{0.250937pt}%
\definecolor{currentstroke}{rgb}{1.000000,1.000000,1.000000}%
\pgfsetstrokecolor{currentstroke}%
\pgfsetdash{}{0pt}%
\pgfpathmoveto{\pgfqpoint{0.380943in}{8.649340in}}%
\pgfpathlineto{\pgfqpoint{0.468679in}{8.649340in}}%
\pgfpathlineto{\pgfqpoint{0.468679in}{8.561604in}}%
\pgfpathlineto{\pgfqpoint{0.380943in}{8.561604in}}%
\pgfpathlineto{\pgfqpoint{0.380943in}{8.649340in}}%
\pgfusepath{stroke}%
\end{pgfscope}%
\begin{pgfscope}%
\pgfpathrectangle{\pgfqpoint{0.380943in}{8.035189in}}{\pgfqpoint{4.650000in}{0.614151in}}%
\pgfusepath{clip}%
\pgfsetbuttcap%
\pgfsetroundjoin%
\definecolor{currentfill}{rgb}{0.986251,0.808597,0.643230}%
\pgfsetfillcolor{currentfill}%
\pgfsetlinewidth{0.250937pt}%
\definecolor{currentstroke}{rgb}{1.000000,1.000000,1.000000}%
\pgfsetstrokecolor{currentstroke}%
\pgfsetdash{}{0pt}%
\pgfpathmoveto{\pgfqpoint{0.468679in}{8.649340in}}%
\pgfpathlineto{\pgfqpoint{0.556415in}{8.649340in}}%
\pgfpathlineto{\pgfqpoint{0.556415in}{8.561604in}}%
\pgfpathlineto{\pgfqpoint{0.468679in}{8.561604in}}%
\pgfpathlineto{\pgfqpoint{0.468679in}{8.649340in}}%
\pgfusepath{stroke,fill}%
\end{pgfscope}%
\begin{pgfscope}%
\pgfpathrectangle{\pgfqpoint{0.380943in}{8.035189in}}{\pgfqpoint{4.650000in}{0.614151in}}%
\pgfusepath{clip}%
\pgfsetbuttcap%
\pgfsetroundjoin%
\definecolor{currentfill}{rgb}{0.963768,0.915433,0.717478}%
\pgfsetfillcolor{currentfill}%
\pgfsetlinewidth{0.250937pt}%
\definecolor{currentstroke}{rgb}{1.000000,1.000000,1.000000}%
\pgfsetstrokecolor{currentstroke}%
\pgfsetdash{}{0pt}%
\pgfpathmoveto{\pgfqpoint{0.556415in}{8.649340in}}%
\pgfpathlineto{\pgfqpoint{0.644151in}{8.649340in}}%
\pgfpathlineto{\pgfqpoint{0.644151in}{8.561604in}}%
\pgfpathlineto{\pgfqpoint{0.556415in}{8.561604in}}%
\pgfpathlineto{\pgfqpoint{0.556415in}{8.649340in}}%
\pgfusepath{stroke,fill}%
\end{pgfscope}%
\begin{pgfscope}%
\pgfpathrectangle{\pgfqpoint{0.380943in}{8.035189in}}{\pgfqpoint{4.650000in}{0.614151in}}%
\pgfusepath{clip}%
\pgfsetbuttcap%
\pgfsetroundjoin%
\definecolor{currentfill}{rgb}{0.996909,0.711742,0.584452}%
\pgfsetfillcolor{currentfill}%
\pgfsetlinewidth{0.250937pt}%
\definecolor{currentstroke}{rgb}{1.000000,1.000000,1.000000}%
\pgfsetstrokecolor{currentstroke}%
\pgfsetdash{}{0pt}%
\pgfpathmoveto{\pgfqpoint{0.644151in}{8.649340in}}%
\pgfpathlineto{\pgfqpoint{0.731886in}{8.649340in}}%
\pgfpathlineto{\pgfqpoint{0.731886in}{8.561604in}}%
\pgfpathlineto{\pgfqpoint{0.644151in}{8.561604in}}%
\pgfpathlineto{\pgfqpoint{0.644151in}{8.649340in}}%
\pgfusepath{stroke,fill}%
\end{pgfscope}%
\begin{pgfscope}%
\pgfpathrectangle{\pgfqpoint{0.380943in}{8.035189in}}{\pgfqpoint{4.650000in}{0.614151in}}%
\pgfusepath{clip}%
\pgfsetbuttcap%
\pgfsetroundjoin%
\definecolor{currentfill}{rgb}{0.999616,0.641369,0.546559}%
\pgfsetfillcolor{currentfill}%
\pgfsetlinewidth{0.250937pt}%
\definecolor{currentstroke}{rgb}{1.000000,1.000000,1.000000}%
\pgfsetstrokecolor{currentstroke}%
\pgfsetdash{}{0pt}%
\pgfpathmoveto{\pgfqpoint{0.731886in}{8.649340in}}%
\pgfpathlineto{\pgfqpoint{0.819622in}{8.649340in}}%
\pgfpathlineto{\pgfqpoint{0.819622in}{8.561604in}}%
\pgfpathlineto{\pgfqpoint{0.731886in}{8.561604in}}%
\pgfpathlineto{\pgfqpoint{0.731886in}{8.649340in}}%
\pgfusepath{stroke,fill}%
\end{pgfscope}%
\begin{pgfscope}%
\pgfpathrectangle{\pgfqpoint{0.380943in}{8.035189in}}{\pgfqpoint{4.650000in}{0.614151in}}%
\pgfusepath{clip}%
\pgfsetbuttcap%
\pgfsetroundjoin%
\definecolor{currentfill}{rgb}{0.996909,0.711742,0.584452}%
\pgfsetfillcolor{currentfill}%
\pgfsetlinewidth{0.250937pt}%
\definecolor{currentstroke}{rgb}{1.000000,1.000000,1.000000}%
\pgfsetstrokecolor{currentstroke}%
\pgfsetdash{}{0pt}%
\pgfpathmoveto{\pgfqpoint{0.819622in}{8.649340in}}%
\pgfpathlineto{\pgfqpoint{0.907358in}{8.649340in}}%
\pgfpathlineto{\pgfqpoint{0.907358in}{8.561604in}}%
\pgfpathlineto{\pgfqpoint{0.819622in}{8.561604in}}%
\pgfpathlineto{\pgfqpoint{0.819622in}{8.649340in}}%
\pgfusepath{stroke,fill}%
\end{pgfscope}%
\begin{pgfscope}%
\pgfpathrectangle{\pgfqpoint{0.380943in}{8.035189in}}{\pgfqpoint{4.650000in}{0.614151in}}%
\pgfusepath{clip}%
\pgfsetbuttcap%
\pgfsetroundjoin%
\definecolor{currentfill}{rgb}{0.970012,0.883276,0.699577}%
\pgfsetfillcolor{currentfill}%
\pgfsetlinewidth{0.250937pt}%
\definecolor{currentstroke}{rgb}{1.000000,1.000000,1.000000}%
\pgfsetstrokecolor{currentstroke}%
\pgfsetdash{}{0pt}%
\pgfpathmoveto{\pgfqpoint{0.907358in}{8.649340in}}%
\pgfpathlineto{\pgfqpoint{0.995094in}{8.649340in}}%
\pgfpathlineto{\pgfqpoint{0.995094in}{8.561604in}}%
\pgfpathlineto{\pgfqpoint{0.907358in}{8.561604in}}%
\pgfpathlineto{\pgfqpoint{0.907358in}{8.649340in}}%
\pgfusepath{stroke,fill}%
\end{pgfscope}%
\begin{pgfscope}%
\pgfpathrectangle{\pgfqpoint{0.380943in}{8.035189in}}{\pgfqpoint{4.650000in}{0.614151in}}%
\pgfusepath{clip}%
\pgfsetbuttcap%
\pgfsetroundjoin%
\definecolor{currentfill}{rgb}{1.000000,0.531903,0.500946}%
\pgfsetfillcolor{currentfill}%
\pgfsetlinewidth{0.250937pt}%
\definecolor{currentstroke}{rgb}{1.000000,1.000000,1.000000}%
\pgfsetstrokecolor{currentstroke}%
\pgfsetdash{}{0pt}%
\pgfpathmoveto{\pgfqpoint{0.995094in}{8.649340in}}%
\pgfpathlineto{\pgfqpoint{1.082830in}{8.649340in}}%
\pgfpathlineto{\pgfqpoint{1.082830in}{8.561604in}}%
\pgfpathlineto{\pgfqpoint{0.995094in}{8.561604in}}%
\pgfpathlineto{\pgfqpoint{0.995094in}{8.649340in}}%
\pgfusepath{stroke,fill}%
\end{pgfscope}%
\begin{pgfscope}%
\pgfpathrectangle{\pgfqpoint{0.380943in}{8.035189in}}{\pgfqpoint{4.650000in}{0.614151in}}%
\pgfusepath{clip}%
\pgfsetbuttcap%
\pgfsetroundjoin%
\definecolor{currentfill}{rgb}{1.000000,0.531903,0.500946}%
\pgfsetfillcolor{currentfill}%
\pgfsetlinewidth{0.250937pt}%
\definecolor{currentstroke}{rgb}{1.000000,1.000000,1.000000}%
\pgfsetstrokecolor{currentstroke}%
\pgfsetdash{}{0pt}%
\pgfpathmoveto{\pgfqpoint{1.082830in}{8.649340in}}%
\pgfpathlineto{\pgfqpoint{1.170566in}{8.649340in}}%
\pgfpathlineto{\pgfqpoint{1.170566in}{8.561604in}}%
\pgfpathlineto{\pgfqpoint{1.082830in}{8.561604in}}%
\pgfpathlineto{\pgfqpoint{1.082830in}{8.649340in}}%
\pgfusepath{stroke,fill}%
\end{pgfscope}%
\begin{pgfscope}%
\pgfpathrectangle{\pgfqpoint{0.380943in}{8.035189in}}{\pgfqpoint{4.650000in}{0.614151in}}%
\pgfusepath{clip}%
\pgfsetbuttcap%
\pgfsetroundjoin%
\definecolor{currentfill}{rgb}{1.000000,0.584929,0.522599}%
\pgfsetfillcolor{currentfill}%
\pgfsetlinewidth{0.250937pt}%
\definecolor{currentstroke}{rgb}{1.000000,1.000000,1.000000}%
\pgfsetstrokecolor{currentstroke}%
\pgfsetdash{}{0pt}%
\pgfpathmoveto{\pgfqpoint{1.170566in}{8.649340in}}%
\pgfpathlineto{\pgfqpoint{1.258302in}{8.649340in}}%
\pgfpathlineto{\pgfqpoint{1.258302in}{8.561604in}}%
\pgfpathlineto{\pgfqpoint{1.170566in}{8.561604in}}%
\pgfpathlineto{\pgfqpoint{1.170566in}{8.649340in}}%
\pgfusepath{stroke,fill}%
\end{pgfscope}%
\begin{pgfscope}%
\pgfpathrectangle{\pgfqpoint{0.380943in}{8.035189in}}{\pgfqpoint{4.650000in}{0.614151in}}%
\pgfusepath{clip}%
\pgfsetbuttcap%
\pgfsetroundjoin%
\definecolor{currentfill}{rgb}{0.963768,0.915433,0.717478}%
\pgfsetfillcolor{currentfill}%
\pgfsetlinewidth{0.250937pt}%
\definecolor{currentstroke}{rgb}{1.000000,1.000000,1.000000}%
\pgfsetstrokecolor{currentstroke}%
\pgfsetdash{}{0pt}%
\pgfpathmoveto{\pgfqpoint{1.258302in}{8.649340in}}%
\pgfpathlineto{\pgfqpoint{1.346037in}{8.649340in}}%
\pgfpathlineto{\pgfqpoint{1.346037in}{8.561604in}}%
\pgfpathlineto{\pgfqpoint{1.258302in}{8.561604in}}%
\pgfpathlineto{\pgfqpoint{1.258302in}{8.649340in}}%
\pgfusepath{stroke,fill}%
\end{pgfscope}%
\begin{pgfscope}%
\pgfpathrectangle{\pgfqpoint{0.380943in}{8.035189in}}{\pgfqpoint{4.650000in}{0.614151in}}%
\pgfusepath{clip}%
\pgfsetbuttcap%
\pgfsetroundjoin%
\definecolor{currentfill}{rgb}{0.992326,0.765229,0.614840}%
\pgfsetfillcolor{currentfill}%
\pgfsetlinewidth{0.250937pt}%
\definecolor{currentstroke}{rgb}{1.000000,1.000000,1.000000}%
\pgfsetstrokecolor{currentstroke}%
\pgfsetdash{}{0pt}%
\pgfpathmoveto{\pgfqpoint{1.346037in}{8.649340in}}%
\pgfpathlineto{\pgfqpoint{1.433773in}{8.649340in}}%
\pgfpathlineto{\pgfqpoint{1.433773in}{8.561604in}}%
\pgfpathlineto{\pgfqpoint{1.346037in}{8.561604in}}%
\pgfpathlineto{\pgfqpoint{1.346037in}{8.649340in}}%
\pgfusepath{stroke,fill}%
\end{pgfscope}%
\begin{pgfscope}%
\pgfpathrectangle{\pgfqpoint{0.380943in}{8.035189in}}{\pgfqpoint{4.650000in}{0.614151in}}%
\pgfusepath{clip}%
\pgfsetbuttcap%
\pgfsetroundjoin%
\definecolor{currentfill}{rgb}{0.978131,0.843783,0.675709}%
\pgfsetfillcolor{currentfill}%
\pgfsetlinewidth{0.250937pt}%
\definecolor{currentstroke}{rgb}{1.000000,1.000000,1.000000}%
\pgfsetstrokecolor{currentstroke}%
\pgfsetdash{}{0pt}%
\pgfpathmoveto{\pgfqpoint{1.433773in}{8.649340in}}%
\pgfpathlineto{\pgfqpoint{1.521509in}{8.649340in}}%
\pgfpathlineto{\pgfqpoint{1.521509in}{8.561604in}}%
\pgfpathlineto{\pgfqpoint{1.433773in}{8.561604in}}%
\pgfpathlineto{\pgfqpoint{1.433773in}{8.649340in}}%
\pgfusepath{stroke,fill}%
\end{pgfscope}%
\begin{pgfscope}%
\pgfpathrectangle{\pgfqpoint{0.380943in}{8.035189in}}{\pgfqpoint{4.650000in}{0.614151in}}%
\pgfusepath{clip}%
\pgfsetbuttcap%
\pgfsetroundjoin%
\definecolor{currentfill}{rgb}{0.992326,0.765229,0.614840}%
\pgfsetfillcolor{currentfill}%
\pgfsetlinewidth{0.250937pt}%
\definecolor{currentstroke}{rgb}{1.000000,1.000000,1.000000}%
\pgfsetstrokecolor{currentstroke}%
\pgfsetdash{}{0pt}%
\pgfpathmoveto{\pgfqpoint{1.521509in}{8.649340in}}%
\pgfpathlineto{\pgfqpoint{1.609245in}{8.649340in}}%
\pgfpathlineto{\pgfqpoint{1.609245in}{8.561604in}}%
\pgfpathlineto{\pgfqpoint{1.521509in}{8.561604in}}%
\pgfpathlineto{\pgfqpoint{1.521509in}{8.649340in}}%
\pgfusepath{stroke,fill}%
\end{pgfscope}%
\begin{pgfscope}%
\pgfpathrectangle{\pgfqpoint{0.380943in}{8.035189in}}{\pgfqpoint{4.650000in}{0.614151in}}%
\pgfusepath{clip}%
\pgfsetbuttcap%
\pgfsetroundjoin%
\definecolor{currentfill}{rgb}{0.961061,0.931672,0.728304}%
\pgfsetfillcolor{currentfill}%
\pgfsetlinewidth{0.250937pt}%
\definecolor{currentstroke}{rgb}{1.000000,1.000000,1.000000}%
\pgfsetstrokecolor{currentstroke}%
\pgfsetdash{}{0pt}%
\pgfpathmoveto{\pgfqpoint{1.609245in}{8.649340in}}%
\pgfpathlineto{\pgfqpoint{1.696981in}{8.649340in}}%
\pgfpathlineto{\pgfqpoint{1.696981in}{8.561604in}}%
\pgfpathlineto{\pgfqpoint{1.609245in}{8.561604in}}%
\pgfpathlineto{\pgfqpoint{1.609245in}{8.649340in}}%
\pgfusepath{stroke,fill}%
\end{pgfscope}%
\begin{pgfscope}%
\pgfpathrectangle{\pgfqpoint{0.380943in}{8.035189in}}{\pgfqpoint{4.650000in}{0.614151in}}%
\pgfusepath{clip}%
\pgfsetbuttcap%
\pgfsetroundjoin%
\definecolor{currentfill}{rgb}{0.961061,0.931672,0.728304}%
\pgfsetfillcolor{currentfill}%
\pgfsetlinewidth{0.250937pt}%
\definecolor{currentstroke}{rgb}{1.000000,1.000000,1.000000}%
\pgfsetstrokecolor{currentstroke}%
\pgfsetdash{}{0pt}%
\pgfpathmoveto{\pgfqpoint{1.696981in}{8.649340in}}%
\pgfpathlineto{\pgfqpoint{1.784717in}{8.649340in}}%
\pgfpathlineto{\pgfqpoint{1.784717in}{8.561604in}}%
\pgfpathlineto{\pgfqpoint{1.696981in}{8.561604in}}%
\pgfpathlineto{\pgfqpoint{1.696981in}{8.649340in}}%
\pgfusepath{stroke,fill}%
\end{pgfscope}%
\begin{pgfscope}%
\pgfpathrectangle{\pgfqpoint{0.380943in}{8.035189in}}{\pgfqpoint{4.650000in}{0.614151in}}%
\pgfusepath{clip}%
\pgfsetbuttcap%
\pgfsetroundjoin%
\definecolor{currentfill}{rgb}{1.000000,1.000000,0.861745}%
\pgfsetfillcolor{currentfill}%
\pgfsetlinewidth{0.250937pt}%
\definecolor{currentstroke}{rgb}{1.000000,1.000000,1.000000}%
\pgfsetstrokecolor{currentstroke}%
\pgfsetdash{}{0pt}%
\pgfpathmoveto{\pgfqpoint{1.784717in}{8.649340in}}%
\pgfpathlineto{\pgfqpoint{1.872452in}{8.649340in}}%
\pgfpathlineto{\pgfqpoint{1.872452in}{8.561604in}}%
\pgfpathlineto{\pgfqpoint{1.784717in}{8.561604in}}%
\pgfpathlineto{\pgfqpoint{1.784717in}{8.649340in}}%
\pgfusepath{stroke,fill}%
\end{pgfscope}%
\begin{pgfscope}%
\pgfpathrectangle{\pgfqpoint{0.380943in}{8.035189in}}{\pgfqpoint{4.650000in}{0.614151in}}%
\pgfusepath{clip}%
\pgfsetbuttcap%
\pgfsetroundjoin%
\definecolor{currentfill}{rgb}{0.970012,0.883276,0.699577}%
\pgfsetfillcolor{currentfill}%
\pgfsetlinewidth{0.250937pt}%
\definecolor{currentstroke}{rgb}{1.000000,1.000000,1.000000}%
\pgfsetstrokecolor{currentstroke}%
\pgfsetdash{}{0pt}%
\pgfpathmoveto{\pgfqpoint{1.872452in}{8.649340in}}%
\pgfpathlineto{\pgfqpoint{1.960188in}{8.649340in}}%
\pgfpathlineto{\pgfqpoint{1.960188in}{8.561604in}}%
\pgfpathlineto{\pgfqpoint{1.872452in}{8.561604in}}%
\pgfpathlineto{\pgfqpoint{1.872452in}{8.649340in}}%
\pgfusepath{stroke,fill}%
\end{pgfscope}%
\begin{pgfscope}%
\pgfpathrectangle{\pgfqpoint{0.380943in}{8.035189in}}{\pgfqpoint{4.650000in}{0.614151in}}%
\pgfusepath{clip}%
\pgfsetbuttcap%
\pgfsetroundjoin%
\definecolor{currentfill}{rgb}{1.000000,1.000000,0.861745}%
\pgfsetfillcolor{currentfill}%
\pgfsetlinewidth{0.250937pt}%
\definecolor{currentstroke}{rgb}{1.000000,1.000000,1.000000}%
\pgfsetstrokecolor{currentstroke}%
\pgfsetdash{}{0pt}%
\pgfpathmoveto{\pgfqpoint{1.960188in}{8.649340in}}%
\pgfpathlineto{\pgfqpoint{2.047924in}{8.649340in}}%
\pgfpathlineto{\pgfqpoint{2.047924in}{8.561604in}}%
\pgfpathlineto{\pgfqpoint{1.960188in}{8.561604in}}%
\pgfpathlineto{\pgfqpoint{1.960188in}{8.649340in}}%
\pgfusepath{stroke,fill}%
\end{pgfscope}%
\begin{pgfscope}%
\pgfpathrectangle{\pgfqpoint{0.380943in}{8.035189in}}{\pgfqpoint{4.650000in}{0.614151in}}%
\pgfusepath{clip}%
\pgfsetbuttcap%
\pgfsetroundjoin%
\definecolor{currentfill}{rgb}{0.961061,0.931672,0.728304}%
\pgfsetfillcolor{currentfill}%
\pgfsetlinewidth{0.250937pt}%
\definecolor{currentstroke}{rgb}{1.000000,1.000000,1.000000}%
\pgfsetstrokecolor{currentstroke}%
\pgfsetdash{}{0pt}%
\pgfpathmoveto{\pgfqpoint{2.047924in}{8.649340in}}%
\pgfpathlineto{\pgfqpoint{2.135660in}{8.649340in}}%
\pgfpathlineto{\pgfqpoint{2.135660in}{8.561604in}}%
\pgfpathlineto{\pgfqpoint{2.047924in}{8.561604in}}%
\pgfpathlineto{\pgfqpoint{2.047924in}{8.649340in}}%
\pgfusepath{stroke,fill}%
\end{pgfscope}%
\begin{pgfscope}%
\pgfpathrectangle{\pgfqpoint{0.380943in}{8.035189in}}{\pgfqpoint{4.650000in}{0.614151in}}%
\pgfusepath{clip}%
\pgfsetbuttcap%
\pgfsetroundjoin%
\definecolor{currentfill}{rgb}{0.986251,0.808597,0.643230}%
\pgfsetfillcolor{currentfill}%
\pgfsetlinewidth{0.250937pt}%
\definecolor{currentstroke}{rgb}{1.000000,1.000000,1.000000}%
\pgfsetstrokecolor{currentstroke}%
\pgfsetdash{}{0pt}%
\pgfpathmoveto{\pgfqpoint{2.135660in}{8.649340in}}%
\pgfpathlineto{\pgfqpoint{2.223396in}{8.649340in}}%
\pgfpathlineto{\pgfqpoint{2.223396in}{8.561604in}}%
\pgfpathlineto{\pgfqpoint{2.135660in}{8.561604in}}%
\pgfpathlineto{\pgfqpoint{2.135660in}{8.649340in}}%
\pgfusepath{stroke,fill}%
\end{pgfscope}%
\begin{pgfscope}%
\pgfpathrectangle{\pgfqpoint{0.380943in}{8.035189in}}{\pgfqpoint{4.650000in}{0.614151in}}%
\pgfusepath{clip}%
\pgfsetbuttcap%
\pgfsetroundjoin%
\definecolor{currentfill}{rgb}{0.961061,0.931672,0.728304}%
\pgfsetfillcolor{currentfill}%
\pgfsetlinewidth{0.250937pt}%
\definecolor{currentstroke}{rgb}{1.000000,1.000000,1.000000}%
\pgfsetstrokecolor{currentstroke}%
\pgfsetdash{}{0pt}%
\pgfpathmoveto{\pgfqpoint{2.223396in}{8.649340in}}%
\pgfpathlineto{\pgfqpoint{2.311132in}{8.649340in}}%
\pgfpathlineto{\pgfqpoint{2.311132in}{8.561604in}}%
\pgfpathlineto{\pgfqpoint{2.223396in}{8.561604in}}%
\pgfpathlineto{\pgfqpoint{2.223396in}{8.649340in}}%
\pgfusepath{stroke,fill}%
\end{pgfscope}%
\begin{pgfscope}%
\pgfpathrectangle{\pgfqpoint{0.380943in}{8.035189in}}{\pgfqpoint{4.650000in}{0.614151in}}%
\pgfusepath{clip}%
\pgfsetbuttcap%
\pgfsetroundjoin%
\definecolor{currentfill}{rgb}{0.970012,0.883276,0.699577}%
\pgfsetfillcolor{currentfill}%
\pgfsetlinewidth{0.250937pt}%
\definecolor{currentstroke}{rgb}{1.000000,1.000000,1.000000}%
\pgfsetstrokecolor{currentstroke}%
\pgfsetdash{}{0pt}%
\pgfpathmoveto{\pgfqpoint{2.311132in}{8.649340in}}%
\pgfpathlineto{\pgfqpoint{2.398868in}{8.649340in}}%
\pgfpathlineto{\pgfqpoint{2.398868in}{8.561604in}}%
\pgfpathlineto{\pgfqpoint{2.311132in}{8.561604in}}%
\pgfpathlineto{\pgfqpoint{2.311132in}{8.649340in}}%
\pgfusepath{stroke,fill}%
\end{pgfscope}%
\begin{pgfscope}%
\pgfpathrectangle{\pgfqpoint{0.380943in}{8.035189in}}{\pgfqpoint{4.650000in}{0.614151in}}%
\pgfusepath{clip}%
\pgfsetbuttcap%
\pgfsetroundjoin%
\definecolor{currentfill}{rgb}{0.961061,0.931672,0.728304}%
\pgfsetfillcolor{currentfill}%
\pgfsetlinewidth{0.250937pt}%
\definecolor{currentstroke}{rgb}{1.000000,1.000000,1.000000}%
\pgfsetstrokecolor{currentstroke}%
\pgfsetdash{}{0pt}%
\pgfpathmoveto{\pgfqpoint{2.398868in}{8.649340in}}%
\pgfpathlineto{\pgfqpoint{2.486603in}{8.649340in}}%
\pgfpathlineto{\pgfqpoint{2.486603in}{8.561604in}}%
\pgfpathlineto{\pgfqpoint{2.398868in}{8.561604in}}%
\pgfpathlineto{\pgfqpoint{2.398868in}{8.649340in}}%
\pgfusepath{stroke,fill}%
\end{pgfscope}%
\begin{pgfscope}%
\pgfpathrectangle{\pgfqpoint{0.380943in}{8.035189in}}{\pgfqpoint{4.650000in}{0.614151in}}%
\pgfusepath{clip}%
\pgfsetbuttcap%
\pgfsetroundjoin%
\definecolor{currentfill}{rgb}{0.970012,0.883276,0.699577}%
\pgfsetfillcolor{currentfill}%
\pgfsetlinewidth{0.250937pt}%
\definecolor{currentstroke}{rgb}{1.000000,1.000000,1.000000}%
\pgfsetstrokecolor{currentstroke}%
\pgfsetdash{}{0pt}%
\pgfpathmoveto{\pgfqpoint{2.486603in}{8.649340in}}%
\pgfpathlineto{\pgfqpoint{2.574339in}{8.649340in}}%
\pgfpathlineto{\pgfqpoint{2.574339in}{8.561604in}}%
\pgfpathlineto{\pgfqpoint{2.486603in}{8.561604in}}%
\pgfpathlineto{\pgfqpoint{2.486603in}{8.649340in}}%
\pgfusepath{stroke,fill}%
\end{pgfscope}%
\begin{pgfscope}%
\pgfpathrectangle{\pgfqpoint{0.380943in}{8.035189in}}{\pgfqpoint{4.650000in}{0.614151in}}%
\pgfusepath{clip}%
\pgfsetbuttcap%
\pgfsetroundjoin%
\definecolor{currentfill}{rgb}{0.978131,0.843783,0.675709}%
\pgfsetfillcolor{currentfill}%
\pgfsetlinewidth{0.250937pt}%
\definecolor{currentstroke}{rgb}{1.000000,1.000000,1.000000}%
\pgfsetstrokecolor{currentstroke}%
\pgfsetdash{}{0pt}%
\pgfpathmoveto{\pgfqpoint{2.574339in}{8.649340in}}%
\pgfpathlineto{\pgfqpoint{2.662075in}{8.649340in}}%
\pgfpathlineto{\pgfqpoint{2.662075in}{8.561604in}}%
\pgfpathlineto{\pgfqpoint{2.574339in}{8.561604in}}%
\pgfpathlineto{\pgfqpoint{2.574339in}{8.649340in}}%
\pgfusepath{stroke,fill}%
\end{pgfscope}%
\begin{pgfscope}%
\pgfpathrectangle{\pgfqpoint{0.380943in}{8.035189in}}{\pgfqpoint{4.650000in}{0.614151in}}%
\pgfusepath{clip}%
\pgfsetbuttcap%
\pgfsetroundjoin%
\definecolor{currentfill}{rgb}{0.992326,0.765229,0.614840}%
\pgfsetfillcolor{currentfill}%
\pgfsetlinewidth{0.250937pt}%
\definecolor{currentstroke}{rgb}{1.000000,1.000000,1.000000}%
\pgfsetstrokecolor{currentstroke}%
\pgfsetdash{}{0pt}%
\pgfpathmoveto{\pgfqpoint{2.662075in}{8.649340in}}%
\pgfpathlineto{\pgfqpoint{2.749811in}{8.649340in}}%
\pgfpathlineto{\pgfqpoint{2.749811in}{8.561604in}}%
\pgfpathlineto{\pgfqpoint{2.662075in}{8.561604in}}%
\pgfpathlineto{\pgfqpoint{2.662075in}{8.649340in}}%
\pgfusepath{stroke,fill}%
\end{pgfscope}%
\begin{pgfscope}%
\pgfpathrectangle{\pgfqpoint{0.380943in}{8.035189in}}{\pgfqpoint{4.650000in}{0.614151in}}%
\pgfusepath{clip}%
\pgfsetbuttcap%
\pgfsetroundjoin%
\definecolor{currentfill}{rgb}{0.986251,0.808597,0.643230}%
\pgfsetfillcolor{currentfill}%
\pgfsetlinewidth{0.250937pt}%
\definecolor{currentstroke}{rgb}{1.000000,1.000000,1.000000}%
\pgfsetstrokecolor{currentstroke}%
\pgfsetdash{}{0pt}%
\pgfpathmoveto{\pgfqpoint{2.749811in}{8.649340in}}%
\pgfpathlineto{\pgfqpoint{2.837547in}{8.649340in}}%
\pgfpathlineto{\pgfqpoint{2.837547in}{8.561604in}}%
\pgfpathlineto{\pgfqpoint{2.749811in}{8.561604in}}%
\pgfpathlineto{\pgfqpoint{2.749811in}{8.649340in}}%
\pgfusepath{stroke,fill}%
\end{pgfscope}%
\begin{pgfscope}%
\pgfpathrectangle{\pgfqpoint{0.380943in}{8.035189in}}{\pgfqpoint{4.650000in}{0.614151in}}%
\pgfusepath{clip}%
\pgfsetbuttcap%
\pgfsetroundjoin%
\definecolor{currentfill}{rgb}{0.992326,0.765229,0.614840}%
\pgfsetfillcolor{currentfill}%
\pgfsetlinewidth{0.250937pt}%
\definecolor{currentstroke}{rgb}{1.000000,1.000000,1.000000}%
\pgfsetstrokecolor{currentstroke}%
\pgfsetdash{}{0pt}%
\pgfpathmoveto{\pgfqpoint{2.837547in}{8.649340in}}%
\pgfpathlineto{\pgfqpoint{2.925283in}{8.649340in}}%
\pgfpathlineto{\pgfqpoint{2.925283in}{8.561604in}}%
\pgfpathlineto{\pgfqpoint{2.837547in}{8.561604in}}%
\pgfpathlineto{\pgfqpoint{2.837547in}{8.649340in}}%
\pgfusepath{stroke,fill}%
\end{pgfscope}%
\begin{pgfscope}%
\pgfpathrectangle{\pgfqpoint{0.380943in}{8.035189in}}{\pgfqpoint{4.650000in}{0.614151in}}%
\pgfusepath{clip}%
\pgfsetbuttcap%
\pgfsetroundjoin%
\definecolor{currentfill}{rgb}{0.963768,0.915433,0.717478}%
\pgfsetfillcolor{currentfill}%
\pgfsetlinewidth{0.250937pt}%
\definecolor{currentstroke}{rgb}{1.000000,1.000000,1.000000}%
\pgfsetstrokecolor{currentstroke}%
\pgfsetdash{}{0pt}%
\pgfpathmoveto{\pgfqpoint{2.925283in}{8.649340in}}%
\pgfpathlineto{\pgfqpoint{3.013019in}{8.649340in}}%
\pgfpathlineto{\pgfqpoint{3.013019in}{8.561604in}}%
\pgfpathlineto{\pgfqpoint{2.925283in}{8.561604in}}%
\pgfpathlineto{\pgfqpoint{2.925283in}{8.649340in}}%
\pgfusepath{stroke,fill}%
\end{pgfscope}%
\begin{pgfscope}%
\pgfpathrectangle{\pgfqpoint{0.380943in}{8.035189in}}{\pgfqpoint{4.650000in}{0.614151in}}%
\pgfusepath{clip}%
\pgfsetbuttcap%
\pgfsetroundjoin%
\definecolor{currentfill}{rgb}{0.992326,0.765229,0.614840}%
\pgfsetfillcolor{currentfill}%
\pgfsetlinewidth{0.250937pt}%
\definecolor{currentstroke}{rgb}{1.000000,1.000000,1.000000}%
\pgfsetstrokecolor{currentstroke}%
\pgfsetdash{}{0pt}%
\pgfpathmoveto{\pgfqpoint{3.013019in}{8.649340in}}%
\pgfpathlineto{\pgfqpoint{3.100754in}{8.649340in}}%
\pgfpathlineto{\pgfqpoint{3.100754in}{8.561604in}}%
\pgfpathlineto{\pgfqpoint{3.013019in}{8.561604in}}%
\pgfpathlineto{\pgfqpoint{3.013019in}{8.649340in}}%
\pgfusepath{stroke,fill}%
\end{pgfscope}%
\begin{pgfscope}%
\pgfpathrectangle{\pgfqpoint{0.380943in}{8.035189in}}{\pgfqpoint{4.650000in}{0.614151in}}%
\pgfusepath{clip}%
\pgfsetbuttcap%
\pgfsetroundjoin%
\definecolor{currentfill}{rgb}{0.961061,0.931672,0.728304}%
\pgfsetfillcolor{currentfill}%
\pgfsetlinewidth{0.250937pt}%
\definecolor{currentstroke}{rgb}{1.000000,1.000000,1.000000}%
\pgfsetstrokecolor{currentstroke}%
\pgfsetdash{}{0pt}%
\pgfpathmoveto{\pgfqpoint{3.100754in}{8.649340in}}%
\pgfpathlineto{\pgfqpoint{3.188490in}{8.649340in}}%
\pgfpathlineto{\pgfqpoint{3.188490in}{8.561604in}}%
\pgfpathlineto{\pgfqpoint{3.100754in}{8.561604in}}%
\pgfpathlineto{\pgfqpoint{3.100754in}{8.649340in}}%
\pgfusepath{stroke,fill}%
\end{pgfscope}%
\begin{pgfscope}%
\pgfpathrectangle{\pgfqpoint{0.380943in}{8.035189in}}{\pgfqpoint{4.650000in}{0.614151in}}%
\pgfusepath{clip}%
\pgfsetbuttcap%
\pgfsetroundjoin%
\definecolor{currentfill}{rgb}{0.961061,0.931672,0.728304}%
\pgfsetfillcolor{currentfill}%
\pgfsetlinewidth{0.250937pt}%
\definecolor{currentstroke}{rgb}{1.000000,1.000000,1.000000}%
\pgfsetstrokecolor{currentstroke}%
\pgfsetdash{}{0pt}%
\pgfpathmoveto{\pgfqpoint{3.188490in}{8.649340in}}%
\pgfpathlineto{\pgfqpoint{3.276226in}{8.649340in}}%
\pgfpathlineto{\pgfqpoint{3.276226in}{8.561604in}}%
\pgfpathlineto{\pgfqpoint{3.188490in}{8.561604in}}%
\pgfpathlineto{\pgfqpoint{3.188490in}{8.649340in}}%
\pgfusepath{stroke,fill}%
\end{pgfscope}%
\begin{pgfscope}%
\pgfpathrectangle{\pgfqpoint{0.380943in}{8.035189in}}{\pgfqpoint{4.650000in}{0.614151in}}%
\pgfusepath{clip}%
\pgfsetbuttcap%
\pgfsetroundjoin%
\definecolor{currentfill}{rgb}{0.985083,0.974641,0.792587}%
\pgfsetfillcolor{currentfill}%
\pgfsetlinewidth{0.250937pt}%
\definecolor{currentstroke}{rgb}{1.000000,1.000000,1.000000}%
\pgfsetstrokecolor{currentstroke}%
\pgfsetdash{}{0pt}%
\pgfpathmoveto{\pgfqpoint{3.276226in}{8.649340in}}%
\pgfpathlineto{\pgfqpoint{3.363962in}{8.649340in}}%
\pgfpathlineto{\pgfqpoint{3.363962in}{8.561604in}}%
\pgfpathlineto{\pgfqpoint{3.276226in}{8.561604in}}%
\pgfpathlineto{\pgfqpoint{3.276226in}{8.649340in}}%
\pgfusepath{stroke,fill}%
\end{pgfscope}%
\begin{pgfscope}%
\pgfpathrectangle{\pgfqpoint{0.380943in}{8.035189in}}{\pgfqpoint{4.650000in}{0.614151in}}%
\pgfusepath{clip}%
\pgfsetbuttcap%
\pgfsetroundjoin%
\definecolor{currentfill}{rgb}{0.978131,0.843783,0.675709}%
\pgfsetfillcolor{currentfill}%
\pgfsetlinewidth{0.250937pt}%
\definecolor{currentstroke}{rgb}{1.000000,1.000000,1.000000}%
\pgfsetstrokecolor{currentstroke}%
\pgfsetdash{}{0pt}%
\pgfpathmoveto{\pgfqpoint{3.363962in}{8.649340in}}%
\pgfpathlineto{\pgfqpoint{3.451698in}{8.649340in}}%
\pgfpathlineto{\pgfqpoint{3.451698in}{8.561604in}}%
\pgfpathlineto{\pgfqpoint{3.363962in}{8.561604in}}%
\pgfpathlineto{\pgfqpoint{3.363962in}{8.649340in}}%
\pgfusepath{stroke,fill}%
\end{pgfscope}%
\begin{pgfscope}%
\pgfpathrectangle{\pgfqpoint{0.380943in}{8.035189in}}{\pgfqpoint{4.650000in}{0.614151in}}%
\pgfusepath{clip}%
\pgfsetbuttcap%
\pgfsetroundjoin%
\definecolor{currentfill}{rgb}{0.970012,0.883276,0.699577}%
\pgfsetfillcolor{currentfill}%
\pgfsetlinewidth{0.250937pt}%
\definecolor{currentstroke}{rgb}{1.000000,1.000000,1.000000}%
\pgfsetstrokecolor{currentstroke}%
\pgfsetdash{}{0pt}%
\pgfpathmoveto{\pgfqpoint{3.451698in}{8.649340in}}%
\pgfpathlineto{\pgfqpoint{3.539434in}{8.649340in}}%
\pgfpathlineto{\pgfqpoint{3.539434in}{8.561604in}}%
\pgfpathlineto{\pgfqpoint{3.451698in}{8.561604in}}%
\pgfpathlineto{\pgfqpoint{3.451698in}{8.649340in}}%
\pgfusepath{stroke,fill}%
\end{pgfscope}%
\begin{pgfscope}%
\pgfpathrectangle{\pgfqpoint{0.380943in}{8.035189in}}{\pgfqpoint{4.650000in}{0.614151in}}%
\pgfusepath{clip}%
\pgfsetbuttcap%
\pgfsetroundjoin%
\definecolor{currentfill}{rgb}{0.986251,0.808597,0.643230}%
\pgfsetfillcolor{currentfill}%
\pgfsetlinewidth{0.250937pt}%
\definecolor{currentstroke}{rgb}{1.000000,1.000000,1.000000}%
\pgfsetstrokecolor{currentstroke}%
\pgfsetdash{}{0pt}%
\pgfpathmoveto{\pgfqpoint{3.539434in}{8.649340in}}%
\pgfpathlineto{\pgfqpoint{3.627169in}{8.649340in}}%
\pgfpathlineto{\pgfqpoint{3.627169in}{8.561604in}}%
\pgfpathlineto{\pgfqpoint{3.539434in}{8.561604in}}%
\pgfpathlineto{\pgfqpoint{3.539434in}{8.649340in}}%
\pgfusepath{stroke,fill}%
\end{pgfscope}%
\begin{pgfscope}%
\pgfpathrectangle{\pgfqpoint{0.380943in}{8.035189in}}{\pgfqpoint{4.650000in}{0.614151in}}%
\pgfusepath{clip}%
\pgfsetbuttcap%
\pgfsetroundjoin%
\definecolor{currentfill}{rgb}{0.986251,0.808597,0.643230}%
\pgfsetfillcolor{currentfill}%
\pgfsetlinewidth{0.250937pt}%
\definecolor{currentstroke}{rgb}{1.000000,1.000000,1.000000}%
\pgfsetstrokecolor{currentstroke}%
\pgfsetdash{}{0pt}%
\pgfpathmoveto{\pgfqpoint{3.627169in}{8.649340in}}%
\pgfpathlineto{\pgfqpoint{3.714905in}{8.649340in}}%
\pgfpathlineto{\pgfqpoint{3.714905in}{8.561604in}}%
\pgfpathlineto{\pgfqpoint{3.627169in}{8.561604in}}%
\pgfpathlineto{\pgfqpoint{3.627169in}{8.649340in}}%
\pgfusepath{stroke,fill}%
\end{pgfscope}%
\begin{pgfscope}%
\pgfpathrectangle{\pgfqpoint{0.380943in}{8.035189in}}{\pgfqpoint{4.650000in}{0.614151in}}%
\pgfusepath{clip}%
\pgfsetbuttcap%
\pgfsetroundjoin%
\definecolor{currentfill}{rgb}{0.970012,0.883276,0.699577}%
\pgfsetfillcolor{currentfill}%
\pgfsetlinewidth{0.250937pt}%
\definecolor{currentstroke}{rgb}{1.000000,1.000000,1.000000}%
\pgfsetstrokecolor{currentstroke}%
\pgfsetdash{}{0pt}%
\pgfpathmoveto{\pgfqpoint{3.714905in}{8.649340in}}%
\pgfpathlineto{\pgfqpoint{3.802641in}{8.649340in}}%
\pgfpathlineto{\pgfqpoint{3.802641in}{8.561604in}}%
\pgfpathlineto{\pgfqpoint{3.714905in}{8.561604in}}%
\pgfpathlineto{\pgfqpoint{3.714905in}{8.649340in}}%
\pgfusepath{stroke,fill}%
\end{pgfscope}%
\begin{pgfscope}%
\pgfpathrectangle{\pgfqpoint{0.380943in}{8.035189in}}{\pgfqpoint{4.650000in}{0.614151in}}%
\pgfusepath{clip}%
\pgfsetbuttcap%
\pgfsetroundjoin%
\definecolor{currentfill}{rgb}{0.963768,0.915433,0.717478}%
\pgfsetfillcolor{currentfill}%
\pgfsetlinewidth{0.250937pt}%
\definecolor{currentstroke}{rgb}{1.000000,1.000000,1.000000}%
\pgfsetstrokecolor{currentstroke}%
\pgfsetdash{}{0pt}%
\pgfpathmoveto{\pgfqpoint{3.802641in}{8.649340in}}%
\pgfpathlineto{\pgfqpoint{3.890377in}{8.649340in}}%
\pgfpathlineto{\pgfqpoint{3.890377in}{8.561604in}}%
\pgfpathlineto{\pgfqpoint{3.802641in}{8.561604in}}%
\pgfpathlineto{\pgfqpoint{3.802641in}{8.649340in}}%
\pgfusepath{stroke,fill}%
\end{pgfscope}%
\begin{pgfscope}%
\pgfpathrectangle{\pgfqpoint{0.380943in}{8.035189in}}{\pgfqpoint{4.650000in}{0.614151in}}%
\pgfusepath{clip}%
\pgfsetbuttcap%
\pgfsetroundjoin%
\definecolor{currentfill}{rgb}{0.961061,0.931672,0.728304}%
\pgfsetfillcolor{currentfill}%
\pgfsetlinewidth{0.250937pt}%
\definecolor{currentstroke}{rgb}{1.000000,1.000000,1.000000}%
\pgfsetstrokecolor{currentstroke}%
\pgfsetdash{}{0pt}%
\pgfpathmoveto{\pgfqpoint{3.890377in}{8.649340in}}%
\pgfpathlineto{\pgfqpoint{3.978113in}{8.649340in}}%
\pgfpathlineto{\pgfqpoint{3.978113in}{8.561604in}}%
\pgfpathlineto{\pgfqpoint{3.890377in}{8.561604in}}%
\pgfpathlineto{\pgfqpoint{3.890377in}{8.649340in}}%
\pgfusepath{stroke,fill}%
\end{pgfscope}%
\begin{pgfscope}%
\pgfpathrectangle{\pgfqpoint{0.380943in}{8.035189in}}{\pgfqpoint{4.650000in}{0.614151in}}%
\pgfusepath{clip}%
\pgfsetbuttcap%
\pgfsetroundjoin%
\definecolor{currentfill}{rgb}{0.865975,0.344406,0.344406}%
\pgfsetfillcolor{currentfill}%
\pgfsetlinewidth{0.250937pt}%
\definecolor{currentstroke}{rgb}{1.000000,1.000000,1.000000}%
\pgfsetstrokecolor{currentstroke}%
\pgfsetdash{}{0pt}%
\pgfpathmoveto{\pgfqpoint{3.978113in}{8.649340in}}%
\pgfpathlineto{\pgfqpoint{4.065849in}{8.649340in}}%
\pgfpathlineto{\pgfqpoint{4.065849in}{8.561604in}}%
\pgfpathlineto{\pgfqpoint{3.978113in}{8.561604in}}%
\pgfpathlineto{\pgfqpoint{3.978113in}{8.649340in}}%
\pgfusepath{stroke,fill}%
\end{pgfscope}%
\begin{pgfscope}%
\pgfpathrectangle{\pgfqpoint{0.380943in}{8.035189in}}{\pgfqpoint{4.650000in}{0.614151in}}%
\pgfusepath{clip}%
\pgfsetbuttcap%
\pgfsetroundjoin%
\definecolor{currentfill}{rgb}{0.961061,0.931672,0.728304}%
\pgfsetfillcolor{currentfill}%
\pgfsetlinewidth{0.250937pt}%
\definecolor{currentstroke}{rgb}{1.000000,1.000000,1.000000}%
\pgfsetstrokecolor{currentstroke}%
\pgfsetdash{}{0pt}%
\pgfpathmoveto{\pgfqpoint{4.065849in}{8.649340in}}%
\pgfpathlineto{\pgfqpoint{4.153585in}{8.649340in}}%
\pgfpathlineto{\pgfqpoint{4.153585in}{8.561604in}}%
\pgfpathlineto{\pgfqpoint{4.065849in}{8.561604in}}%
\pgfpathlineto{\pgfqpoint{4.065849in}{8.649340in}}%
\pgfusepath{stroke,fill}%
\end{pgfscope}%
\begin{pgfscope}%
\pgfpathrectangle{\pgfqpoint{0.380943in}{8.035189in}}{\pgfqpoint{4.650000in}{0.614151in}}%
\pgfusepath{clip}%
\pgfsetbuttcap%
\pgfsetroundjoin%
\definecolor{currentfill}{rgb}{0.999616,0.641369,0.546559}%
\pgfsetfillcolor{currentfill}%
\pgfsetlinewidth{0.250937pt}%
\definecolor{currentstroke}{rgb}{1.000000,1.000000,1.000000}%
\pgfsetstrokecolor{currentstroke}%
\pgfsetdash{}{0pt}%
\pgfpathmoveto{\pgfqpoint{4.153585in}{8.649340in}}%
\pgfpathlineto{\pgfqpoint{4.241320in}{8.649340in}}%
\pgfpathlineto{\pgfqpoint{4.241320in}{8.561604in}}%
\pgfpathlineto{\pgfqpoint{4.153585in}{8.561604in}}%
\pgfpathlineto{\pgfqpoint{4.153585in}{8.649340in}}%
\pgfusepath{stroke,fill}%
\end{pgfscope}%
\begin{pgfscope}%
\pgfpathrectangle{\pgfqpoint{0.380943in}{8.035189in}}{\pgfqpoint{4.650000in}{0.614151in}}%
\pgfusepath{clip}%
\pgfsetbuttcap%
\pgfsetroundjoin%
\definecolor{currentfill}{rgb}{0.978131,0.843783,0.675709}%
\pgfsetfillcolor{currentfill}%
\pgfsetlinewidth{0.250937pt}%
\definecolor{currentstroke}{rgb}{1.000000,1.000000,1.000000}%
\pgfsetstrokecolor{currentstroke}%
\pgfsetdash{}{0pt}%
\pgfpathmoveto{\pgfqpoint{4.241320in}{8.649340in}}%
\pgfpathlineto{\pgfqpoint{4.329056in}{8.649340in}}%
\pgfpathlineto{\pgfqpoint{4.329056in}{8.561604in}}%
\pgfpathlineto{\pgfqpoint{4.241320in}{8.561604in}}%
\pgfpathlineto{\pgfqpoint{4.241320in}{8.649340in}}%
\pgfusepath{stroke,fill}%
\end{pgfscope}%
\begin{pgfscope}%
\pgfpathrectangle{\pgfqpoint{0.380943in}{8.035189in}}{\pgfqpoint{4.650000in}{0.614151in}}%
\pgfusepath{clip}%
\pgfsetbuttcap%
\pgfsetroundjoin%
\definecolor{currentfill}{rgb}{0.986251,0.808597,0.643230}%
\pgfsetfillcolor{currentfill}%
\pgfsetlinewidth{0.250937pt}%
\definecolor{currentstroke}{rgb}{1.000000,1.000000,1.000000}%
\pgfsetstrokecolor{currentstroke}%
\pgfsetdash{}{0pt}%
\pgfpathmoveto{\pgfqpoint{4.329056in}{8.649340in}}%
\pgfpathlineto{\pgfqpoint{4.416792in}{8.649340in}}%
\pgfpathlineto{\pgfqpoint{4.416792in}{8.561604in}}%
\pgfpathlineto{\pgfqpoint{4.329056in}{8.561604in}}%
\pgfpathlineto{\pgfqpoint{4.329056in}{8.649340in}}%
\pgfusepath{stroke,fill}%
\end{pgfscope}%
\begin{pgfscope}%
\pgfpathrectangle{\pgfqpoint{0.380943in}{8.035189in}}{\pgfqpoint{4.650000in}{0.614151in}}%
\pgfusepath{clip}%
\pgfsetbuttcap%
\pgfsetroundjoin%
\definecolor{currentfill}{rgb}{0.992326,0.765229,0.614840}%
\pgfsetfillcolor{currentfill}%
\pgfsetlinewidth{0.250937pt}%
\definecolor{currentstroke}{rgb}{1.000000,1.000000,1.000000}%
\pgfsetstrokecolor{currentstroke}%
\pgfsetdash{}{0pt}%
\pgfpathmoveto{\pgfqpoint{4.416792in}{8.649340in}}%
\pgfpathlineto{\pgfqpoint{4.504528in}{8.649340in}}%
\pgfpathlineto{\pgfqpoint{4.504528in}{8.561604in}}%
\pgfpathlineto{\pgfqpoint{4.416792in}{8.561604in}}%
\pgfpathlineto{\pgfqpoint{4.416792in}{8.649340in}}%
\pgfusepath{stroke,fill}%
\end{pgfscope}%
\begin{pgfscope}%
\pgfpathrectangle{\pgfqpoint{0.380943in}{8.035189in}}{\pgfqpoint{4.650000in}{0.614151in}}%
\pgfusepath{clip}%
\pgfsetbuttcap%
\pgfsetroundjoin%
\definecolor{currentfill}{rgb}{0.986251,0.808597,0.643230}%
\pgfsetfillcolor{currentfill}%
\pgfsetlinewidth{0.250937pt}%
\definecolor{currentstroke}{rgb}{1.000000,1.000000,1.000000}%
\pgfsetstrokecolor{currentstroke}%
\pgfsetdash{}{0pt}%
\pgfpathmoveto{\pgfqpoint{4.504528in}{8.649340in}}%
\pgfpathlineto{\pgfqpoint{4.592264in}{8.649340in}}%
\pgfpathlineto{\pgfqpoint{4.592264in}{8.561604in}}%
\pgfpathlineto{\pgfqpoint{4.504528in}{8.561604in}}%
\pgfpathlineto{\pgfqpoint{4.504528in}{8.649340in}}%
\pgfusepath{stroke,fill}%
\end{pgfscope}%
\begin{pgfscope}%
\pgfpathrectangle{\pgfqpoint{0.380943in}{8.035189in}}{\pgfqpoint{4.650000in}{0.614151in}}%
\pgfusepath{clip}%
\pgfsetbuttcap%
\pgfsetroundjoin%
\definecolor{currentfill}{rgb}{0.996909,0.711742,0.584452}%
\pgfsetfillcolor{currentfill}%
\pgfsetlinewidth{0.250937pt}%
\definecolor{currentstroke}{rgb}{1.000000,1.000000,1.000000}%
\pgfsetstrokecolor{currentstroke}%
\pgfsetdash{}{0pt}%
\pgfpathmoveto{\pgfqpoint{4.592264in}{8.649340in}}%
\pgfpathlineto{\pgfqpoint{4.680000in}{8.649340in}}%
\pgfpathlineto{\pgfqpoint{4.680000in}{8.561604in}}%
\pgfpathlineto{\pgfqpoint{4.592264in}{8.561604in}}%
\pgfpathlineto{\pgfqpoint{4.592264in}{8.649340in}}%
\pgfusepath{stroke,fill}%
\end{pgfscope}%
\begin{pgfscope}%
\pgfpathrectangle{\pgfqpoint{0.380943in}{8.035189in}}{\pgfqpoint{4.650000in}{0.614151in}}%
\pgfusepath{clip}%
\pgfsetbuttcap%
\pgfsetroundjoin%
\definecolor{currentfill}{rgb}{0.978131,0.843783,0.675709}%
\pgfsetfillcolor{currentfill}%
\pgfsetlinewidth{0.250937pt}%
\definecolor{currentstroke}{rgb}{1.000000,1.000000,1.000000}%
\pgfsetstrokecolor{currentstroke}%
\pgfsetdash{}{0pt}%
\pgfpathmoveto{\pgfqpoint{4.680000in}{8.649340in}}%
\pgfpathlineto{\pgfqpoint{4.767736in}{8.649340in}}%
\pgfpathlineto{\pgfqpoint{4.767736in}{8.561604in}}%
\pgfpathlineto{\pgfqpoint{4.680000in}{8.561604in}}%
\pgfpathlineto{\pgfqpoint{4.680000in}{8.649340in}}%
\pgfusepath{stroke,fill}%
\end{pgfscope}%
\begin{pgfscope}%
\pgfpathrectangle{\pgfqpoint{0.380943in}{8.035189in}}{\pgfqpoint{4.650000in}{0.614151in}}%
\pgfusepath{clip}%
\pgfsetbuttcap%
\pgfsetroundjoin%
\definecolor{currentfill}{rgb}{1.000000,0.584929,0.522599}%
\pgfsetfillcolor{currentfill}%
\pgfsetlinewidth{0.250937pt}%
\definecolor{currentstroke}{rgb}{1.000000,1.000000,1.000000}%
\pgfsetstrokecolor{currentstroke}%
\pgfsetdash{}{0pt}%
\pgfpathmoveto{\pgfqpoint{4.767736in}{8.649340in}}%
\pgfpathlineto{\pgfqpoint{4.855471in}{8.649340in}}%
\pgfpathlineto{\pgfqpoint{4.855471in}{8.561604in}}%
\pgfpathlineto{\pgfqpoint{4.767736in}{8.561604in}}%
\pgfpathlineto{\pgfqpoint{4.767736in}{8.649340in}}%
\pgfusepath{stroke,fill}%
\end{pgfscope}%
\begin{pgfscope}%
\pgfpathrectangle{\pgfqpoint{0.380943in}{8.035189in}}{\pgfqpoint{4.650000in}{0.614151in}}%
\pgfusepath{clip}%
\pgfsetbuttcap%
\pgfsetroundjoin%
\definecolor{currentfill}{rgb}{0.970012,0.883276,0.699577}%
\pgfsetfillcolor{currentfill}%
\pgfsetlinewidth{0.250937pt}%
\definecolor{currentstroke}{rgb}{1.000000,1.000000,1.000000}%
\pgfsetstrokecolor{currentstroke}%
\pgfsetdash{}{0pt}%
\pgfpathmoveto{\pgfqpoint{4.855471in}{8.649340in}}%
\pgfpathlineto{\pgfqpoint{4.943207in}{8.649340in}}%
\pgfpathlineto{\pgfqpoint{4.943207in}{8.561604in}}%
\pgfpathlineto{\pgfqpoint{4.855471in}{8.561604in}}%
\pgfpathlineto{\pgfqpoint{4.855471in}{8.649340in}}%
\pgfusepath{stroke,fill}%
\end{pgfscope}%
\begin{pgfscope}%
\pgfpathrectangle{\pgfqpoint{0.380943in}{8.035189in}}{\pgfqpoint{4.650000in}{0.614151in}}%
\pgfusepath{clip}%
\pgfsetbuttcap%
\pgfsetroundjoin%
\definecolor{currentfill}{rgb}{0.985083,0.974641,0.792587}%
\pgfsetfillcolor{currentfill}%
\pgfsetlinewidth{0.250937pt}%
\definecolor{currentstroke}{rgb}{1.000000,1.000000,1.000000}%
\pgfsetstrokecolor{currentstroke}%
\pgfsetdash{}{0pt}%
\pgfpathmoveto{\pgfqpoint{4.943207in}{8.649340in}}%
\pgfpathlineto{\pgfqpoint{5.030943in}{8.649340in}}%
\pgfpathlineto{\pgfqpoint{5.030943in}{8.561604in}}%
\pgfpathlineto{\pgfqpoint{4.943207in}{8.561604in}}%
\pgfpathlineto{\pgfqpoint{4.943207in}{8.649340in}}%
\pgfusepath{stroke,fill}%
\end{pgfscope}%
\begin{pgfscope}%
\pgfpathrectangle{\pgfqpoint{0.380943in}{8.035189in}}{\pgfqpoint{4.650000in}{0.614151in}}%
\pgfusepath{clip}%
\pgfsetbuttcap%
\pgfsetroundjoin%
\pgfsetlinewidth{0.250937pt}%
\definecolor{currentstroke}{rgb}{1.000000,1.000000,1.000000}%
\pgfsetstrokecolor{currentstroke}%
\pgfsetdash{}{0pt}%
\pgfpathmoveto{\pgfqpoint{0.380943in}{8.561604in}}%
\pgfpathlineto{\pgfqpoint{0.468679in}{8.561604in}}%
\pgfpathlineto{\pgfqpoint{0.468679in}{8.473868in}}%
\pgfpathlineto{\pgfqpoint{0.380943in}{8.473868in}}%
\pgfpathlineto{\pgfqpoint{0.380943in}{8.561604in}}%
\pgfusepath{stroke}%
\end{pgfscope}%
\begin{pgfscope}%
\pgfpathrectangle{\pgfqpoint{0.380943in}{8.035189in}}{\pgfqpoint{4.650000in}{0.614151in}}%
\pgfusepath{clip}%
\pgfsetbuttcap%
\pgfsetroundjoin%
\definecolor{currentfill}{rgb}{1.000000,0.584929,0.522599}%
\pgfsetfillcolor{currentfill}%
\pgfsetlinewidth{0.250937pt}%
\definecolor{currentstroke}{rgb}{1.000000,1.000000,1.000000}%
\pgfsetstrokecolor{currentstroke}%
\pgfsetdash{}{0pt}%
\pgfpathmoveto{\pgfqpoint{0.468679in}{8.561604in}}%
\pgfpathlineto{\pgfqpoint{0.556415in}{8.561604in}}%
\pgfpathlineto{\pgfqpoint{0.556415in}{8.473868in}}%
\pgfpathlineto{\pgfqpoint{0.468679in}{8.473868in}}%
\pgfpathlineto{\pgfqpoint{0.468679in}{8.561604in}}%
\pgfusepath{stroke,fill}%
\end{pgfscope}%
\begin{pgfscope}%
\pgfpathrectangle{\pgfqpoint{0.380943in}{8.035189in}}{\pgfqpoint{4.650000in}{0.614151in}}%
\pgfusepath{clip}%
\pgfsetbuttcap%
\pgfsetroundjoin%
\definecolor{currentfill}{rgb}{0.996909,0.711742,0.584452}%
\pgfsetfillcolor{currentfill}%
\pgfsetlinewidth{0.250937pt}%
\definecolor{currentstroke}{rgb}{1.000000,1.000000,1.000000}%
\pgfsetstrokecolor{currentstroke}%
\pgfsetdash{}{0pt}%
\pgfpathmoveto{\pgfqpoint{0.556415in}{8.561604in}}%
\pgfpathlineto{\pgfqpoint{0.644151in}{8.561604in}}%
\pgfpathlineto{\pgfqpoint{0.644151in}{8.473868in}}%
\pgfpathlineto{\pgfqpoint{0.556415in}{8.473868in}}%
\pgfpathlineto{\pgfqpoint{0.556415in}{8.561604in}}%
\pgfusepath{stroke,fill}%
\end{pgfscope}%
\begin{pgfscope}%
\pgfpathrectangle{\pgfqpoint{0.380943in}{8.035189in}}{\pgfqpoint{4.650000in}{0.614151in}}%
\pgfusepath{clip}%
\pgfsetbuttcap%
\pgfsetroundjoin%
\definecolor{currentfill}{rgb}{0.986251,0.808597,0.643230}%
\pgfsetfillcolor{currentfill}%
\pgfsetlinewidth{0.250937pt}%
\definecolor{currentstroke}{rgb}{1.000000,1.000000,1.000000}%
\pgfsetstrokecolor{currentstroke}%
\pgfsetdash{}{0pt}%
\pgfpathmoveto{\pgfqpoint{0.644151in}{8.561604in}}%
\pgfpathlineto{\pgfqpoint{0.731886in}{8.561604in}}%
\pgfpathlineto{\pgfqpoint{0.731886in}{8.473868in}}%
\pgfpathlineto{\pgfqpoint{0.644151in}{8.473868in}}%
\pgfpathlineto{\pgfqpoint{0.644151in}{8.561604in}}%
\pgfusepath{stroke,fill}%
\end{pgfscope}%
\begin{pgfscope}%
\pgfpathrectangle{\pgfqpoint{0.380943in}{8.035189in}}{\pgfqpoint{4.650000in}{0.614151in}}%
\pgfusepath{clip}%
\pgfsetbuttcap%
\pgfsetroundjoin%
\definecolor{currentfill}{rgb}{0.961061,0.931672,0.728304}%
\pgfsetfillcolor{currentfill}%
\pgfsetlinewidth{0.250937pt}%
\definecolor{currentstroke}{rgb}{1.000000,1.000000,1.000000}%
\pgfsetstrokecolor{currentstroke}%
\pgfsetdash{}{0pt}%
\pgfpathmoveto{\pgfqpoint{0.731886in}{8.561604in}}%
\pgfpathlineto{\pgfqpoint{0.819622in}{8.561604in}}%
\pgfpathlineto{\pgfqpoint{0.819622in}{8.473868in}}%
\pgfpathlineto{\pgfqpoint{0.731886in}{8.473868in}}%
\pgfpathlineto{\pgfqpoint{0.731886in}{8.561604in}}%
\pgfusepath{stroke,fill}%
\end{pgfscope}%
\begin{pgfscope}%
\pgfpathrectangle{\pgfqpoint{0.380943in}{8.035189in}}{\pgfqpoint{4.650000in}{0.614151in}}%
\pgfusepath{clip}%
\pgfsetbuttcap%
\pgfsetroundjoin%
\definecolor{currentfill}{rgb}{0.999616,0.641369,0.546559}%
\pgfsetfillcolor{currentfill}%
\pgfsetlinewidth{0.250937pt}%
\definecolor{currentstroke}{rgb}{1.000000,1.000000,1.000000}%
\pgfsetstrokecolor{currentstroke}%
\pgfsetdash{}{0pt}%
\pgfpathmoveto{\pgfqpoint{0.819622in}{8.561604in}}%
\pgfpathlineto{\pgfqpoint{0.907358in}{8.561604in}}%
\pgfpathlineto{\pgfqpoint{0.907358in}{8.473868in}}%
\pgfpathlineto{\pgfqpoint{0.819622in}{8.473868in}}%
\pgfpathlineto{\pgfqpoint{0.819622in}{8.561604in}}%
\pgfusepath{stroke,fill}%
\end{pgfscope}%
\begin{pgfscope}%
\pgfpathrectangle{\pgfqpoint{0.380943in}{8.035189in}}{\pgfqpoint{4.650000in}{0.614151in}}%
\pgfusepath{clip}%
\pgfsetbuttcap%
\pgfsetroundjoin%
\definecolor{currentfill}{rgb}{1.000000,0.480477,0.479293}%
\pgfsetfillcolor{currentfill}%
\pgfsetlinewidth{0.250937pt}%
\definecolor{currentstroke}{rgb}{1.000000,1.000000,1.000000}%
\pgfsetstrokecolor{currentstroke}%
\pgfsetdash{}{0pt}%
\pgfpathmoveto{\pgfqpoint{0.907358in}{8.561604in}}%
\pgfpathlineto{\pgfqpoint{0.995094in}{8.561604in}}%
\pgfpathlineto{\pgfqpoint{0.995094in}{8.473868in}}%
\pgfpathlineto{\pgfqpoint{0.907358in}{8.473868in}}%
\pgfpathlineto{\pgfqpoint{0.907358in}{8.561604in}}%
\pgfusepath{stroke,fill}%
\end{pgfscope}%
\begin{pgfscope}%
\pgfpathrectangle{\pgfqpoint{0.380943in}{8.035189in}}{\pgfqpoint{4.650000in}{0.614151in}}%
\pgfusepath{clip}%
\pgfsetbuttcap%
\pgfsetroundjoin%
\definecolor{currentfill}{rgb}{0.970012,0.883276,0.699577}%
\pgfsetfillcolor{currentfill}%
\pgfsetlinewidth{0.250937pt}%
\definecolor{currentstroke}{rgb}{1.000000,1.000000,1.000000}%
\pgfsetstrokecolor{currentstroke}%
\pgfsetdash{}{0pt}%
\pgfpathmoveto{\pgfqpoint{0.995094in}{8.561604in}}%
\pgfpathlineto{\pgfqpoint{1.082830in}{8.561604in}}%
\pgfpathlineto{\pgfqpoint{1.082830in}{8.473868in}}%
\pgfpathlineto{\pgfqpoint{0.995094in}{8.473868in}}%
\pgfpathlineto{\pgfqpoint{0.995094in}{8.561604in}}%
\pgfusepath{stroke,fill}%
\end{pgfscope}%
\begin{pgfscope}%
\pgfpathrectangle{\pgfqpoint{0.380943in}{8.035189in}}{\pgfqpoint{4.650000in}{0.614151in}}%
\pgfusepath{clip}%
\pgfsetbuttcap%
\pgfsetroundjoin%
\definecolor{currentfill}{rgb}{0.970012,0.883276,0.699577}%
\pgfsetfillcolor{currentfill}%
\pgfsetlinewidth{0.250937pt}%
\definecolor{currentstroke}{rgb}{1.000000,1.000000,1.000000}%
\pgfsetstrokecolor{currentstroke}%
\pgfsetdash{}{0pt}%
\pgfpathmoveto{\pgfqpoint{1.082830in}{8.561604in}}%
\pgfpathlineto{\pgfqpoint{1.170566in}{8.561604in}}%
\pgfpathlineto{\pgfqpoint{1.170566in}{8.473868in}}%
\pgfpathlineto{\pgfqpoint{1.082830in}{8.473868in}}%
\pgfpathlineto{\pgfqpoint{1.082830in}{8.561604in}}%
\pgfusepath{stroke,fill}%
\end{pgfscope}%
\begin{pgfscope}%
\pgfpathrectangle{\pgfqpoint{0.380943in}{8.035189in}}{\pgfqpoint{4.650000in}{0.614151in}}%
\pgfusepath{clip}%
\pgfsetbuttcap%
\pgfsetroundjoin%
\definecolor{currentfill}{rgb}{1.000000,0.480477,0.479293}%
\pgfsetfillcolor{currentfill}%
\pgfsetlinewidth{0.250937pt}%
\definecolor{currentstroke}{rgb}{1.000000,1.000000,1.000000}%
\pgfsetstrokecolor{currentstroke}%
\pgfsetdash{}{0pt}%
\pgfpathmoveto{\pgfqpoint{1.170566in}{8.561604in}}%
\pgfpathlineto{\pgfqpoint{1.258302in}{8.561604in}}%
\pgfpathlineto{\pgfqpoint{1.258302in}{8.473868in}}%
\pgfpathlineto{\pgfqpoint{1.170566in}{8.473868in}}%
\pgfpathlineto{\pgfqpoint{1.170566in}{8.561604in}}%
\pgfusepath{stroke,fill}%
\end{pgfscope}%
\begin{pgfscope}%
\pgfpathrectangle{\pgfqpoint{0.380943in}{8.035189in}}{\pgfqpoint{4.650000in}{0.614151in}}%
\pgfusepath{clip}%
\pgfsetbuttcap%
\pgfsetroundjoin%
\definecolor{currentfill}{rgb}{0.970012,0.883276,0.699577}%
\pgfsetfillcolor{currentfill}%
\pgfsetlinewidth{0.250937pt}%
\definecolor{currentstroke}{rgb}{1.000000,1.000000,1.000000}%
\pgfsetstrokecolor{currentstroke}%
\pgfsetdash{}{0pt}%
\pgfpathmoveto{\pgfqpoint{1.258302in}{8.561604in}}%
\pgfpathlineto{\pgfqpoint{1.346037in}{8.561604in}}%
\pgfpathlineto{\pgfqpoint{1.346037in}{8.473868in}}%
\pgfpathlineto{\pgfqpoint{1.258302in}{8.473868in}}%
\pgfpathlineto{\pgfqpoint{1.258302in}{8.561604in}}%
\pgfusepath{stroke,fill}%
\end{pgfscope}%
\begin{pgfscope}%
\pgfpathrectangle{\pgfqpoint{0.380943in}{8.035189in}}{\pgfqpoint{4.650000in}{0.614151in}}%
\pgfusepath{clip}%
\pgfsetbuttcap%
\pgfsetroundjoin%
\definecolor{currentfill}{rgb}{0.999616,0.641369,0.546559}%
\pgfsetfillcolor{currentfill}%
\pgfsetlinewidth{0.250937pt}%
\definecolor{currentstroke}{rgb}{1.000000,1.000000,1.000000}%
\pgfsetstrokecolor{currentstroke}%
\pgfsetdash{}{0pt}%
\pgfpathmoveto{\pgfqpoint{1.346037in}{8.561604in}}%
\pgfpathlineto{\pgfqpoint{1.433773in}{8.561604in}}%
\pgfpathlineto{\pgfqpoint{1.433773in}{8.473868in}}%
\pgfpathlineto{\pgfqpoint{1.346037in}{8.473868in}}%
\pgfpathlineto{\pgfqpoint{1.346037in}{8.561604in}}%
\pgfusepath{stroke,fill}%
\end{pgfscope}%
\begin{pgfscope}%
\pgfpathrectangle{\pgfqpoint{0.380943in}{8.035189in}}{\pgfqpoint{4.650000in}{0.614151in}}%
\pgfusepath{clip}%
\pgfsetbuttcap%
\pgfsetroundjoin%
\definecolor{currentfill}{rgb}{0.970012,0.883276,0.699577}%
\pgfsetfillcolor{currentfill}%
\pgfsetlinewidth{0.250937pt}%
\definecolor{currentstroke}{rgb}{1.000000,1.000000,1.000000}%
\pgfsetstrokecolor{currentstroke}%
\pgfsetdash{}{0pt}%
\pgfpathmoveto{\pgfqpoint{1.433773in}{8.561604in}}%
\pgfpathlineto{\pgfqpoint{1.521509in}{8.561604in}}%
\pgfpathlineto{\pgfqpoint{1.521509in}{8.473868in}}%
\pgfpathlineto{\pgfqpoint{1.433773in}{8.473868in}}%
\pgfpathlineto{\pgfqpoint{1.433773in}{8.561604in}}%
\pgfusepath{stroke,fill}%
\end{pgfscope}%
\begin{pgfscope}%
\pgfpathrectangle{\pgfqpoint{0.380943in}{8.035189in}}{\pgfqpoint{4.650000in}{0.614151in}}%
\pgfusepath{clip}%
\pgfsetbuttcap%
\pgfsetroundjoin%
\definecolor{currentfill}{rgb}{0.999616,0.641369,0.546559}%
\pgfsetfillcolor{currentfill}%
\pgfsetlinewidth{0.250937pt}%
\definecolor{currentstroke}{rgb}{1.000000,1.000000,1.000000}%
\pgfsetstrokecolor{currentstroke}%
\pgfsetdash{}{0pt}%
\pgfpathmoveto{\pgfqpoint{1.521509in}{8.561604in}}%
\pgfpathlineto{\pgfqpoint{1.609245in}{8.561604in}}%
\pgfpathlineto{\pgfqpoint{1.609245in}{8.473868in}}%
\pgfpathlineto{\pgfqpoint{1.521509in}{8.473868in}}%
\pgfpathlineto{\pgfqpoint{1.521509in}{8.561604in}}%
\pgfusepath{stroke,fill}%
\end{pgfscope}%
\begin{pgfscope}%
\pgfpathrectangle{\pgfqpoint{0.380943in}{8.035189in}}{\pgfqpoint{4.650000in}{0.614151in}}%
\pgfusepath{clip}%
\pgfsetbuttcap%
\pgfsetroundjoin%
\definecolor{currentfill}{rgb}{0.996909,0.711742,0.584452}%
\pgfsetfillcolor{currentfill}%
\pgfsetlinewidth{0.250937pt}%
\definecolor{currentstroke}{rgb}{1.000000,1.000000,1.000000}%
\pgfsetstrokecolor{currentstroke}%
\pgfsetdash{}{0pt}%
\pgfpathmoveto{\pgfqpoint{1.609245in}{8.561604in}}%
\pgfpathlineto{\pgfqpoint{1.696981in}{8.561604in}}%
\pgfpathlineto{\pgfqpoint{1.696981in}{8.473868in}}%
\pgfpathlineto{\pgfqpoint{1.609245in}{8.473868in}}%
\pgfpathlineto{\pgfqpoint{1.609245in}{8.561604in}}%
\pgfusepath{stroke,fill}%
\end{pgfscope}%
\begin{pgfscope}%
\pgfpathrectangle{\pgfqpoint{0.380943in}{8.035189in}}{\pgfqpoint{4.650000in}{0.614151in}}%
\pgfusepath{clip}%
\pgfsetbuttcap%
\pgfsetroundjoin%
\definecolor{currentfill}{rgb}{0.978131,0.843783,0.675709}%
\pgfsetfillcolor{currentfill}%
\pgfsetlinewidth{0.250937pt}%
\definecolor{currentstroke}{rgb}{1.000000,1.000000,1.000000}%
\pgfsetstrokecolor{currentstroke}%
\pgfsetdash{}{0pt}%
\pgfpathmoveto{\pgfqpoint{1.696981in}{8.561604in}}%
\pgfpathlineto{\pgfqpoint{1.784717in}{8.561604in}}%
\pgfpathlineto{\pgfqpoint{1.784717in}{8.473868in}}%
\pgfpathlineto{\pgfqpoint{1.696981in}{8.473868in}}%
\pgfpathlineto{\pgfqpoint{1.696981in}{8.561604in}}%
\pgfusepath{stroke,fill}%
\end{pgfscope}%
\begin{pgfscope}%
\pgfpathrectangle{\pgfqpoint{0.380943in}{8.035189in}}{\pgfqpoint{4.650000in}{0.614151in}}%
\pgfusepath{clip}%
\pgfsetbuttcap%
\pgfsetroundjoin%
\definecolor{currentfill}{rgb}{0.970012,0.883276,0.699577}%
\pgfsetfillcolor{currentfill}%
\pgfsetlinewidth{0.250937pt}%
\definecolor{currentstroke}{rgb}{1.000000,1.000000,1.000000}%
\pgfsetstrokecolor{currentstroke}%
\pgfsetdash{}{0pt}%
\pgfpathmoveto{\pgfqpoint{1.784717in}{8.561604in}}%
\pgfpathlineto{\pgfqpoint{1.872452in}{8.561604in}}%
\pgfpathlineto{\pgfqpoint{1.872452in}{8.473868in}}%
\pgfpathlineto{\pgfqpoint{1.784717in}{8.473868in}}%
\pgfpathlineto{\pgfqpoint{1.784717in}{8.561604in}}%
\pgfusepath{stroke,fill}%
\end{pgfscope}%
\begin{pgfscope}%
\pgfpathrectangle{\pgfqpoint{0.380943in}{8.035189in}}{\pgfqpoint{4.650000in}{0.614151in}}%
\pgfusepath{clip}%
\pgfsetbuttcap%
\pgfsetroundjoin%
\definecolor{currentfill}{rgb}{0.999616,0.641369,0.546559}%
\pgfsetfillcolor{currentfill}%
\pgfsetlinewidth{0.250937pt}%
\definecolor{currentstroke}{rgb}{1.000000,1.000000,1.000000}%
\pgfsetstrokecolor{currentstroke}%
\pgfsetdash{}{0pt}%
\pgfpathmoveto{\pgfqpoint{1.872452in}{8.561604in}}%
\pgfpathlineto{\pgfqpoint{1.960188in}{8.561604in}}%
\pgfpathlineto{\pgfqpoint{1.960188in}{8.473868in}}%
\pgfpathlineto{\pgfqpoint{1.872452in}{8.473868in}}%
\pgfpathlineto{\pgfqpoint{1.872452in}{8.561604in}}%
\pgfusepath{stroke,fill}%
\end{pgfscope}%
\begin{pgfscope}%
\pgfpathrectangle{\pgfqpoint{0.380943in}{8.035189in}}{\pgfqpoint{4.650000in}{0.614151in}}%
\pgfusepath{clip}%
\pgfsetbuttcap%
\pgfsetroundjoin%
\definecolor{currentfill}{rgb}{0.963768,0.915433,0.717478}%
\pgfsetfillcolor{currentfill}%
\pgfsetlinewidth{0.250937pt}%
\definecolor{currentstroke}{rgb}{1.000000,1.000000,1.000000}%
\pgfsetstrokecolor{currentstroke}%
\pgfsetdash{}{0pt}%
\pgfpathmoveto{\pgfqpoint{1.960188in}{8.561604in}}%
\pgfpathlineto{\pgfqpoint{2.047924in}{8.561604in}}%
\pgfpathlineto{\pgfqpoint{2.047924in}{8.473868in}}%
\pgfpathlineto{\pgfqpoint{1.960188in}{8.473868in}}%
\pgfpathlineto{\pgfqpoint{1.960188in}{8.561604in}}%
\pgfusepath{stroke,fill}%
\end{pgfscope}%
\begin{pgfscope}%
\pgfpathrectangle{\pgfqpoint{0.380943in}{8.035189in}}{\pgfqpoint{4.650000in}{0.614151in}}%
\pgfusepath{clip}%
\pgfsetbuttcap%
\pgfsetroundjoin%
\definecolor{currentfill}{rgb}{0.961061,0.931672,0.728304}%
\pgfsetfillcolor{currentfill}%
\pgfsetlinewidth{0.250937pt}%
\definecolor{currentstroke}{rgb}{1.000000,1.000000,1.000000}%
\pgfsetstrokecolor{currentstroke}%
\pgfsetdash{}{0pt}%
\pgfpathmoveto{\pgfqpoint{2.047924in}{8.561604in}}%
\pgfpathlineto{\pgfqpoint{2.135660in}{8.561604in}}%
\pgfpathlineto{\pgfqpoint{2.135660in}{8.473868in}}%
\pgfpathlineto{\pgfqpoint{2.047924in}{8.473868in}}%
\pgfpathlineto{\pgfqpoint{2.047924in}{8.561604in}}%
\pgfusepath{stroke,fill}%
\end{pgfscope}%
\begin{pgfscope}%
\pgfpathrectangle{\pgfqpoint{0.380943in}{8.035189in}}{\pgfqpoint{4.650000in}{0.614151in}}%
\pgfusepath{clip}%
\pgfsetbuttcap%
\pgfsetroundjoin%
\definecolor{currentfill}{rgb}{0.978131,0.843783,0.675709}%
\pgfsetfillcolor{currentfill}%
\pgfsetlinewidth{0.250937pt}%
\definecolor{currentstroke}{rgb}{1.000000,1.000000,1.000000}%
\pgfsetstrokecolor{currentstroke}%
\pgfsetdash{}{0pt}%
\pgfpathmoveto{\pgfqpoint{2.135660in}{8.561604in}}%
\pgfpathlineto{\pgfqpoint{2.223396in}{8.561604in}}%
\pgfpathlineto{\pgfqpoint{2.223396in}{8.473868in}}%
\pgfpathlineto{\pgfqpoint{2.135660in}{8.473868in}}%
\pgfpathlineto{\pgfqpoint{2.135660in}{8.561604in}}%
\pgfusepath{stroke,fill}%
\end{pgfscope}%
\begin{pgfscope}%
\pgfpathrectangle{\pgfqpoint{0.380943in}{8.035189in}}{\pgfqpoint{4.650000in}{0.614151in}}%
\pgfusepath{clip}%
\pgfsetbuttcap%
\pgfsetroundjoin%
\definecolor{currentfill}{rgb}{1.000000,0.584929,0.522599}%
\pgfsetfillcolor{currentfill}%
\pgfsetlinewidth{0.250937pt}%
\definecolor{currentstroke}{rgb}{1.000000,1.000000,1.000000}%
\pgfsetstrokecolor{currentstroke}%
\pgfsetdash{}{0pt}%
\pgfpathmoveto{\pgfqpoint{2.223396in}{8.561604in}}%
\pgfpathlineto{\pgfqpoint{2.311132in}{8.561604in}}%
\pgfpathlineto{\pgfqpoint{2.311132in}{8.473868in}}%
\pgfpathlineto{\pgfqpoint{2.223396in}{8.473868in}}%
\pgfpathlineto{\pgfqpoint{2.223396in}{8.561604in}}%
\pgfusepath{stroke,fill}%
\end{pgfscope}%
\begin{pgfscope}%
\pgfpathrectangle{\pgfqpoint{0.380943in}{8.035189in}}{\pgfqpoint{4.650000in}{0.614151in}}%
\pgfusepath{clip}%
\pgfsetbuttcap%
\pgfsetroundjoin%
\definecolor{currentfill}{rgb}{0.961061,0.931672,0.728304}%
\pgfsetfillcolor{currentfill}%
\pgfsetlinewidth{0.250937pt}%
\definecolor{currentstroke}{rgb}{1.000000,1.000000,1.000000}%
\pgfsetstrokecolor{currentstroke}%
\pgfsetdash{}{0pt}%
\pgfpathmoveto{\pgfqpoint{2.311132in}{8.561604in}}%
\pgfpathlineto{\pgfqpoint{2.398868in}{8.561604in}}%
\pgfpathlineto{\pgfqpoint{2.398868in}{8.473868in}}%
\pgfpathlineto{\pgfqpoint{2.311132in}{8.473868in}}%
\pgfpathlineto{\pgfqpoint{2.311132in}{8.561604in}}%
\pgfusepath{stroke,fill}%
\end{pgfscope}%
\begin{pgfscope}%
\pgfpathrectangle{\pgfqpoint{0.380943in}{8.035189in}}{\pgfqpoint{4.650000in}{0.614151in}}%
\pgfusepath{clip}%
\pgfsetbuttcap%
\pgfsetroundjoin%
\definecolor{currentfill}{rgb}{0.970012,0.883276,0.699577}%
\pgfsetfillcolor{currentfill}%
\pgfsetlinewidth{0.250937pt}%
\definecolor{currentstroke}{rgb}{1.000000,1.000000,1.000000}%
\pgfsetstrokecolor{currentstroke}%
\pgfsetdash{}{0pt}%
\pgfpathmoveto{\pgfqpoint{2.398868in}{8.561604in}}%
\pgfpathlineto{\pgfqpoint{2.486603in}{8.561604in}}%
\pgfpathlineto{\pgfqpoint{2.486603in}{8.473868in}}%
\pgfpathlineto{\pgfqpoint{2.398868in}{8.473868in}}%
\pgfpathlineto{\pgfqpoint{2.398868in}{8.561604in}}%
\pgfusepath{stroke,fill}%
\end{pgfscope}%
\begin{pgfscope}%
\pgfpathrectangle{\pgfqpoint{0.380943in}{8.035189in}}{\pgfqpoint{4.650000in}{0.614151in}}%
\pgfusepath{clip}%
\pgfsetbuttcap%
\pgfsetroundjoin%
\definecolor{currentfill}{rgb}{0.970012,0.883276,0.699577}%
\pgfsetfillcolor{currentfill}%
\pgfsetlinewidth{0.250937pt}%
\definecolor{currentstroke}{rgb}{1.000000,1.000000,1.000000}%
\pgfsetstrokecolor{currentstroke}%
\pgfsetdash{}{0pt}%
\pgfpathmoveto{\pgfqpoint{2.486603in}{8.561604in}}%
\pgfpathlineto{\pgfqpoint{2.574339in}{8.561604in}}%
\pgfpathlineto{\pgfqpoint{2.574339in}{8.473868in}}%
\pgfpathlineto{\pgfqpoint{2.486603in}{8.473868in}}%
\pgfpathlineto{\pgfqpoint{2.486603in}{8.561604in}}%
\pgfusepath{stroke,fill}%
\end{pgfscope}%
\begin{pgfscope}%
\pgfpathrectangle{\pgfqpoint{0.380943in}{8.035189in}}{\pgfqpoint{4.650000in}{0.614151in}}%
\pgfusepath{clip}%
\pgfsetbuttcap%
\pgfsetroundjoin%
\definecolor{currentfill}{rgb}{0.992326,0.765229,0.614840}%
\pgfsetfillcolor{currentfill}%
\pgfsetlinewidth{0.250937pt}%
\definecolor{currentstroke}{rgb}{1.000000,1.000000,1.000000}%
\pgfsetstrokecolor{currentstroke}%
\pgfsetdash{}{0pt}%
\pgfpathmoveto{\pgfqpoint{2.574339in}{8.561604in}}%
\pgfpathlineto{\pgfqpoint{2.662075in}{8.561604in}}%
\pgfpathlineto{\pgfqpoint{2.662075in}{8.473868in}}%
\pgfpathlineto{\pgfqpoint{2.574339in}{8.473868in}}%
\pgfpathlineto{\pgfqpoint{2.574339in}{8.561604in}}%
\pgfusepath{stroke,fill}%
\end{pgfscope}%
\begin{pgfscope}%
\pgfpathrectangle{\pgfqpoint{0.380943in}{8.035189in}}{\pgfqpoint{4.650000in}{0.614151in}}%
\pgfusepath{clip}%
\pgfsetbuttcap%
\pgfsetroundjoin%
\definecolor{currentfill}{rgb}{0.970012,0.883276,0.699577}%
\pgfsetfillcolor{currentfill}%
\pgfsetlinewidth{0.250937pt}%
\definecolor{currentstroke}{rgb}{1.000000,1.000000,1.000000}%
\pgfsetstrokecolor{currentstroke}%
\pgfsetdash{}{0pt}%
\pgfpathmoveto{\pgfqpoint{2.662075in}{8.561604in}}%
\pgfpathlineto{\pgfqpoint{2.749811in}{8.561604in}}%
\pgfpathlineto{\pgfqpoint{2.749811in}{8.473868in}}%
\pgfpathlineto{\pgfqpoint{2.662075in}{8.473868in}}%
\pgfpathlineto{\pgfqpoint{2.662075in}{8.561604in}}%
\pgfusepath{stroke,fill}%
\end{pgfscope}%
\begin{pgfscope}%
\pgfpathrectangle{\pgfqpoint{0.380943in}{8.035189in}}{\pgfqpoint{4.650000in}{0.614151in}}%
\pgfusepath{clip}%
\pgfsetbuttcap%
\pgfsetroundjoin%
\definecolor{currentfill}{rgb}{0.961061,0.931672,0.728304}%
\pgfsetfillcolor{currentfill}%
\pgfsetlinewidth{0.250937pt}%
\definecolor{currentstroke}{rgb}{1.000000,1.000000,1.000000}%
\pgfsetstrokecolor{currentstroke}%
\pgfsetdash{}{0pt}%
\pgfpathmoveto{\pgfqpoint{2.749811in}{8.561604in}}%
\pgfpathlineto{\pgfqpoint{2.837547in}{8.561604in}}%
\pgfpathlineto{\pgfqpoint{2.837547in}{8.473868in}}%
\pgfpathlineto{\pgfqpoint{2.749811in}{8.473868in}}%
\pgfpathlineto{\pgfqpoint{2.749811in}{8.561604in}}%
\pgfusepath{stroke,fill}%
\end{pgfscope}%
\begin{pgfscope}%
\pgfpathrectangle{\pgfqpoint{0.380943in}{8.035189in}}{\pgfqpoint{4.650000in}{0.614151in}}%
\pgfusepath{clip}%
\pgfsetbuttcap%
\pgfsetroundjoin%
\definecolor{currentfill}{rgb}{0.999616,0.641369,0.546559}%
\pgfsetfillcolor{currentfill}%
\pgfsetlinewidth{0.250937pt}%
\definecolor{currentstroke}{rgb}{1.000000,1.000000,1.000000}%
\pgfsetstrokecolor{currentstroke}%
\pgfsetdash{}{0pt}%
\pgfpathmoveto{\pgfqpoint{2.837547in}{8.561604in}}%
\pgfpathlineto{\pgfqpoint{2.925283in}{8.561604in}}%
\pgfpathlineto{\pgfqpoint{2.925283in}{8.473868in}}%
\pgfpathlineto{\pgfqpoint{2.837547in}{8.473868in}}%
\pgfpathlineto{\pgfqpoint{2.837547in}{8.561604in}}%
\pgfusepath{stroke,fill}%
\end{pgfscope}%
\begin{pgfscope}%
\pgfpathrectangle{\pgfqpoint{0.380943in}{8.035189in}}{\pgfqpoint{4.650000in}{0.614151in}}%
\pgfusepath{clip}%
\pgfsetbuttcap%
\pgfsetroundjoin%
\definecolor{currentfill}{rgb}{0.978131,0.843783,0.675709}%
\pgfsetfillcolor{currentfill}%
\pgfsetlinewidth{0.250937pt}%
\definecolor{currentstroke}{rgb}{1.000000,1.000000,1.000000}%
\pgfsetstrokecolor{currentstroke}%
\pgfsetdash{}{0pt}%
\pgfpathmoveto{\pgfqpoint{2.925283in}{8.561604in}}%
\pgfpathlineto{\pgfqpoint{3.013019in}{8.561604in}}%
\pgfpathlineto{\pgfqpoint{3.013019in}{8.473868in}}%
\pgfpathlineto{\pgfqpoint{2.925283in}{8.473868in}}%
\pgfpathlineto{\pgfqpoint{2.925283in}{8.561604in}}%
\pgfusepath{stroke,fill}%
\end{pgfscope}%
\begin{pgfscope}%
\pgfpathrectangle{\pgfqpoint{0.380943in}{8.035189in}}{\pgfqpoint{4.650000in}{0.614151in}}%
\pgfusepath{clip}%
\pgfsetbuttcap%
\pgfsetroundjoin%
\definecolor{currentfill}{rgb}{0.999616,0.641369,0.546559}%
\pgfsetfillcolor{currentfill}%
\pgfsetlinewidth{0.250937pt}%
\definecolor{currentstroke}{rgb}{1.000000,1.000000,1.000000}%
\pgfsetstrokecolor{currentstroke}%
\pgfsetdash{}{0pt}%
\pgfpathmoveto{\pgfqpoint{3.013019in}{8.561604in}}%
\pgfpathlineto{\pgfqpoint{3.100754in}{8.561604in}}%
\pgfpathlineto{\pgfqpoint{3.100754in}{8.473868in}}%
\pgfpathlineto{\pgfqpoint{3.013019in}{8.473868in}}%
\pgfpathlineto{\pgfqpoint{3.013019in}{8.561604in}}%
\pgfusepath{stroke,fill}%
\end{pgfscope}%
\begin{pgfscope}%
\pgfpathrectangle{\pgfqpoint{0.380943in}{8.035189in}}{\pgfqpoint{4.650000in}{0.614151in}}%
\pgfusepath{clip}%
\pgfsetbuttcap%
\pgfsetroundjoin%
\definecolor{currentfill}{rgb}{0.961061,0.931672,0.728304}%
\pgfsetfillcolor{currentfill}%
\pgfsetlinewidth{0.250937pt}%
\definecolor{currentstroke}{rgb}{1.000000,1.000000,1.000000}%
\pgfsetstrokecolor{currentstroke}%
\pgfsetdash{}{0pt}%
\pgfpathmoveto{\pgfqpoint{3.100754in}{8.561604in}}%
\pgfpathlineto{\pgfqpoint{3.188490in}{8.561604in}}%
\pgfpathlineto{\pgfqpoint{3.188490in}{8.473868in}}%
\pgfpathlineto{\pgfqpoint{3.100754in}{8.473868in}}%
\pgfpathlineto{\pgfqpoint{3.100754in}{8.561604in}}%
\pgfusepath{stroke,fill}%
\end{pgfscope}%
\begin{pgfscope}%
\pgfpathrectangle{\pgfqpoint{0.380943in}{8.035189in}}{\pgfqpoint{4.650000in}{0.614151in}}%
\pgfusepath{clip}%
\pgfsetbuttcap%
\pgfsetroundjoin%
\definecolor{currentfill}{rgb}{0.978131,0.843783,0.675709}%
\pgfsetfillcolor{currentfill}%
\pgfsetlinewidth{0.250937pt}%
\definecolor{currentstroke}{rgb}{1.000000,1.000000,1.000000}%
\pgfsetstrokecolor{currentstroke}%
\pgfsetdash{}{0pt}%
\pgfpathmoveto{\pgfqpoint{3.188490in}{8.561604in}}%
\pgfpathlineto{\pgfqpoint{3.276226in}{8.561604in}}%
\pgfpathlineto{\pgfqpoint{3.276226in}{8.473868in}}%
\pgfpathlineto{\pgfqpoint{3.188490in}{8.473868in}}%
\pgfpathlineto{\pgfqpoint{3.188490in}{8.561604in}}%
\pgfusepath{stroke,fill}%
\end{pgfscope}%
\begin{pgfscope}%
\pgfpathrectangle{\pgfqpoint{0.380943in}{8.035189in}}{\pgfqpoint{4.650000in}{0.614151in}}%
\pgfusepath{clip}%
\pgfsetbuttcap%
\pgfsetroundjoin%
\definecolor{currentfill}{rgb}{0.961061,0.931672,0.728304}%
\pgfsetfillcolor{currentfill}%
\pgfsetlinewidth{0.250937pt}%
\definecolor{currentstroke}{rgb}{1.000000,1.000000,1.000000}%
\pgfsetstrokecolor{currentstroke}%
\pgfsetdash{}{0pt}%
\pgfpathmoveto{\pgfqpoint{3.276226in}{8.561604in}}%
\pgfpathlineto{\pgfqpoint{3.363962in}{8.561604in}}%
\pgfpathlineto{\pgfqpoint{3.363962in}{8.473868in}}%
\pgfpathlineto{\pgfqpoint{3.276226in}{8.473868in}}%
\pgfpathlineto{\pgfqpoint{3.276226in}{8.561604in}}%
\pgfusepath{stroke,fill}%
\end{pgfscope}%
\begin{pgfscope}%
\pgfpathrectangle{\pgfqpoint{0.380943in}{8.035189in}}{\pgfqpoint{4.650000in}{0.614151in}}%
\pgfusepath{clip}%
\pgfsetbuttcap%
\pgfsetroundjoin%
\definecolor{currentfill}{rgb}{0.970012,0.883276,0.699577}%
\pgfsetfillcolor{currentfill}%
\pgfsetlinewidth{0.250937pt}%
\definecolor{currentstroke}{rgb}{1.000000,1.000000,1.000000}%
\pgfsetstrokecolor{currentstroke}%
\pgfsetdash{}{0pt}%
\pgfpathmoveto{\pgfqpoint{3.363962in}{8.561604in}}%
\pgfpathlineto{\pgfqpoint{3.451698in}{8.561604in}}%
\pgfpathlineto{\pgfqpoint{3.451698in}{8.473868in}}%
\pgfpathlineto{\pgfqpoint{3.363962in}{8.473868in}}%
\pgfpathlineto{\pgfqpoint{3.363962in}{8.561604in}}%
\pgfusepath{stroke,fill}%
\end{pgfscope}%
\begin{pgfscope}%
\pgfpathrectangle{\pgfqpoint{0.380943in}{8.035189in}}{\pgfqpoint{4.650000in}{0.614151in}}%
\pgfusepath{clip}%
\pgfsetbuttcap%
\pgfsetroundjoin%
\definecolor{currentfill}{rgb}{0.963768,0.915433,0.717478}%
\pgfsetfillcolor{currentfill}%
\pgfsetlinewidth{0.250937pt}%
\definecolor{currentstroke}{rgb}{1.000000,1.000000,1.000000}%
\pgfsetstrokecolor{currentstroke}%
\pgfsetdash{}{0pt}%
\pgfpathmoveto{\pgfqpoint{3.451698in}{8.561604in}}%
\pgfpathlineto{\pgfqpoint{3.539434in}{8.561604in}}%
\pgfpathlineto{\pgfqpoint{3.539434in}{8.473868in}}%
\pgfpathlineto{\pgfqpoint{3.451698in}{8.473868in}}%
\pgfpathlineto{\pgfqpoint{3.451698in}{8.561604in}}%
\pgfusepath{stroke,fill}%
\end{pgfscope}%
\begin{pgfscope}%
\pgfpathrectangle{\pgfqpoint{0.380943in}{8.035189in}}{\pgfqpoint{4.650000in}{0.614151in}}%
\pgfusepath{clip}%
\pgfsetbuttcap%
\pgfsetroundjoin%
\definecolor{currentfill}{rgb}{0.986251,0.808597,0.643230}%
\pgfsetfillcolor{currentfill}%
\pgfsetlinewidth{0.250937pt}%
\definecolor{currentstroke}{rgb}{1.000000,1.000000,1.000000}%
\pgfsetstrokecolor{currentstroke}%
\pgfsetdash{}{0pt}%
\pgfpathmoveto{\pgfqpoint{3.539434in}{8.561604in}}%
\pgfpathlineto{\pgfqpoint{3.627169in}{8.561604in}}%
\pgfpathlineto{\pgfqpoint{3.627169in}{8.473868in}}%
\pgfpathlineto{\pgfqpoint{3.539434in}{8.473868in}}%
\pgfpathlineto{\pgfqpoint{3.539434in}{8.561604in}}%
\pgfusepath{stroke,fill}%
\end{pgfscope}%
\begin{pgfscope}%
\pgfpathrectangle{\pgfqpoint{0.380943in}{8.035189in}}{\pgfqpoint{4.650000in}{0.614151in}}%
\pgfusepath{clip}%
\pgfsetbuttcap%
\pgfsetroundjoin%
\definecolor{currentfill}{rgb}{1.000000,0.584929,0.522599}%
\pgfsetfillcolor{currentfill}%
\pgfsetlinewidth{0.250937pt}%
\definecolor{currentstroke}{rgb}{1.000000,1.000000,1.000000}%
\pgfsetstrokecolor{currentstroke}%
\pgfsetdash{}{0pt}%
\pgfpathmoveto{\pgfqpoint{3.627169in}{8.561604in}}%
\pgfpathlineto{\pgfqpoint{3.714905in}{8.561604in}}%
\pgfpathlineto{\pgfqpoint{3.714905in}{8.473868in}}%
\pgfpathlineto{\pgfqpoint{3.627169in}{8.473868in}}%
\pgfpathlineto{\pgfqpoint{3.627169in}{8.561604in}}%
\pgfusepath{stroke,fill}%
\end{pgfscope}%
\begin{pgfscope}%
\pgfpathrectangle{\pgfqpoint{0.380943in}{8.035189in}}{\pgfqpoint{4.650000in}{0.614151in}}%
\pgfusepath{clip}%
\pgfsetbuttcap%
\pgfsetroundjoin%
\definecolor{currentfill}{rgb}{0.986251,0.808597,0.643230}%
\pgfsetfillcolor{currentfill}%
\pgfsetlinewidth{0.250937pt}%
\definecolor{currentstroke}{rgb}{1.000000,1.000000,1.000000}%
\pgfsetstrokecolor{currentstroke}%
\pgfsetdash{}{0pt}%
\pgfpathmoveto{\pgfqpoint{3.714905in}{8.561604in}}%
\pgfpathlineto{\pgfqpoint{3.802641in}{8.561604in}}%
\pgfpathlineto{\pgfqpoint{3.802641in}{8.473868in}}%
\pgfpathlineto{\pgfqpoint{3.714905in}{8.473868in}}%
\pgfpathlineto{\pgfqpoint{3.714905in}{8.561604in}}%
\pgfusepath{stroke,fill}%
\end{pgfscope}%
\begin{pgfscope}%
\pgfpathrectangle{\pgfqpoint{0.380943in}{8.035189in}}{\pgfqpoint{4.650000in}{0.614151in}}%
\pgfusepath{clip}%
\pgfsetbuttcap%
\pgfsetroundjoin%
\definecolor{currentfill}{rgb}{0.970012,0.883276,0.699577}%
\pgfsetfillcolor{currentfill}%
\pgfsetlinewidth{0.250937pt}%
\definecolor{currentstroke}{rgb}{1.000000,1.000000,1.000000}%
\pgfsetstrokecolor{currentstroke}%
\pgfsetdash{}{0pt}%
\pgfpathmoveto{\pgfqpoint{3.802641in}{8.561604in}}%
\pgfpathlineto{\pgfqpoint{3.890377in}{8.561604in}}%
\pgfpathlineto{\pgfqpoint{3.890377in}{8.473868in}}%
\pgfpathlineto{\pgfqpoint{3.802641in}{8.473868in}}%
\pgfpathlineto{\pgfqpoint{3.802641in}{8.561604in}}%
\pgfusepath{stroke,fill}%
\end{pgfscope}%
\begin{pgfscope}%
\pgfpathrectangle{\pgfqpoint{0.380943in}{8.035189in}}{\pgfqpoint{4.650000in}{0.614151in}}%
\pgfusepath{clip}%
\pgfsetbuttcap%
\pgfsetroundjoin%
\definecolor{currentfill}{rgb}{0.986251,0.808597,0.643230}%
\pgfsetfillcolor{currentfill}%
\pgfsetlinewidth{0.250937pt}%
\definecolor{currentstroke}{rgb}{1.000000,1.000000,1.000000}%
\pgfsetstrokecolor{currentstroke}%
\pgfsetdash{}{0pt}%
\pgfpathmoveto{\pgfqpoint{3.890377in}{8.561604in}}%
\pgfpathlineto{\pgfqpoint{3.978113in}{8.561604in}}%
\pgfpathlineto{\pgfqpoint{3.978113in}{8.473868in}}%
\pgfpathlineto{\pgfqpoint{3.890377in}{8.473868in}}%
\pgfpathlineto{\pgfqpoint{3.890377in}{8.561604in}}%
\pgfusepath{stroke,fill}%
\end{pgfscope}%
\begin{pgfscope}%
\pgfpathrectangle{\pgfqpoint{0.380943in}{8.035189in}}{\pgfqpoint{4.650000in}{0.614151in}}%
\pgfusepath{clip}%
\pgfsetbuttcap%
\pgfsetroundjoin%
\definecolor{currentfill}{rgb}{0.978131,0.843783,0.675709}%
\pgfsetfillcolor{currentfill}%
\pgfsetlinewidth{0.250937pt}%
\definecolor{currentstroke}{rgb}{1.000000,1.000000,1.000000}%
\pgfsetstrokecolor{currentstroke}%
\pgfsetdash{}{0pt}%
\pgfpathmoveto{\pgfqpoint{3.978113in}{8.561604in}}%
\pgfpathlineto{\pgfqpoint{4.065849in}{8.561604in}}%
\pgfpathlineto{\pgfqpoint{4.065849in}{8.473868in}}%
\pgfpathlineto{\pgfqpoint{3.978113in}{8.473868in}}%
\pgfpathlineto{\pgfqpoint{3.978113in}{8.561604in}}%
\pgfusepath{stroke,fill}%
\end{pgfscope}%
\begin{pgfscope}%
\pgfpathrectangle{\pgfqpoint{0.380943in}{8.035189in}}{\pgfqpoint{4.650000in}{0.614151in}}%
\pgfusepath{clip}%
\pgfsetbuttcap%
\pgfsetroundjoin%
\definecolor{currentfill}{rgb}{0.963768,0.915433,0.717478}%
\pgfsetfillcolor{currentfill}%
\pgfsetlinewidth{0.250937pt}%
\definecolor{currentstroke}{rgb}{1.000000,1.000000,1.000000}%
\pgfsetstrokecolor{currentstroke}%
\pgfsetdash{}{0pt}%
\pgfpathmoveto{\pgfqpoint{4.065849in}{8.561604in}}%
\pgfpathlineto{\pgfqpoint{4.153585in}{8.561604in}}%
\pgfpathlineto{\pgfqpoint{4.153585in}{8.473868in}}%
\pgfpathlineto{\pgfqpoint{4.065849in}{8.473868in}}%
\pgfpathlineto{\pgfqpoint{4.065849in}{8.561604in}}%
\pgfusepath{stroke,fill}%
\end{pgfscope}%
\begin{pgfscope}%
\pgfpathrectangle{\pgfqpoint{0.380943in}{8.035189in}}{\pgfqpoint{4.650000in}{0.614151in}}%
\pgfusepath{clip}%
\pgfsetbuttcap%
\pgfsetroundjoin%
\definecolor{currentfill}{rgb}{0.999616,0.641369,0.546559}%
\pgfsetfillcolor{currentfill}%
\pgfsetlinewidth{0.250937pt}%
\definecolor{currentstroke}{rgb}{1.000000,1.000000,1.000000}%
\pgfsetstrokecolor{currentstroke}%
\pgfsetdash{}{0pt}%
\pgfpathmoveto{\pgfqpoint{4.153585in}{8.561604in}}%
\pgfpathlineto{\pgfqpoint{4.241320in}{8.561604in}}%
\pgfpathlineto{\pgfqpoint{4.241320in}{8.473868in}}%
\pgfpathlineto{\pgfqpoint{4.153585in}{8.473868in}}%
\pgfpathlineto{\pgfqpoint{4.153585in}{8.561604in}}%
\pgfusepath{stroke,fill}%
\end{pgfscope}%
\begin{pgfscope}%
\pgfpathrectangle{\pgfqpoint{0.380943in}{8.035189in}}{\pgfqpoint{4.650000in}{0.614151in}}%
\pgfusepath{clip}%
\pgfsetbuttcap%
\pgfsetroundjoin%
\definecolor{currentfill}{rgb}{0.992326,0.765229,0.614840}%
\pgfsetfillcolor{currentfill}%
\pgfsetlinewidth{0.250937pt}%
\definecolor{currentstroke}{rgb}{1.000000,1.000000,1.000000}%
\pgfsetstrokecolor{currentstroke}%
\pgfsetdash{}{0pt}%
\pgfpathmoveto{\pgfqpoint{4.241320in}{8.561604in}}%
\pgfpathlineto{\pgfqpoint{4.329056in}{8.561604in}}%
\pgfpathlineto{\pgfqpoint{4.329056in}{8.473868in}}%
\pgfpathlineto{\pgfqpoint{4.241320in}{8.473868in}}%
\pgfpathlineto{\pgfqpoint{4.241320in}{8.561604in}}%
\pgfusepath{stroke,fill}%
\end{pgfscope}%
\begin{pgfscope}%
\pgfpathrectangle{\pgfqpoint{0.380943in}{8.035189in}}{\pgfqpoint{4.650000in}{0.614151in}}%
\pgfusepath{clip}%
\pgfsetbuttcap%
\pgfsetroundjoin%
\definecolor{currentfill}{rgb}{0.978131,0.843783,0.675709}%
\pgfsetfillcolor{currentfill}%
\pgfsetlinewidth{0.250937pt}%
\definecolor{currentstroke}{rgb}{1.000000,1.000000,1.000000}%
\pgfsetstrokecolor{currentstroke}%
\pgfsetdash{}{0pt}%
\pgfpathmoveto{\pgfqpoint{4.329056in}{8.561604in}}%
\pgfpathlineto{\pgfqpoint{4.416792in}{8.561604in}}%
\pgfpathlineto{\pgfqpoint{4.416792in}{8.473868in}}%
\pgfpathlineto{\pgfqpoint{4.329056in}{8.473868in}}%
\pgfpathlineto{\pgfqpoint{4.329056in}{8.561604in}}%
\pgfusepath{stroke,fill}%
\end{pgfscope}%
\begin{pgfscope}%
\pgfpathrectangle{\pgfqpoint{0.380943in}{8.035189in}}{\pgfqpoint{4.650000in}{0.614151in}}%
\pgfusepath{clip}%
\pgfsetbuttcap%
\pgfsetroundjoin%
\definecolor{currentfill}{rgb}{0.986251,0.808597,0.643230}%
\pgfsetfillcolor{currentfill}%
\pgfsetlinewidth{0.250937pt}%
\definecolor{currentstroke}{rgb}{1.000000,1.000000,1.000000}%
\pgfsetstrokecolor{currentstroke}%
\pgfsetdash{}{0pt}%
\pgfpathmoveto{\pgfqpoint{4.416792in}{8.561604in}}%
\pgfpathlineto{\pgfqpoint{4.504528in}{8.561604in}}%
\pgfpathlineto{\pgfqpoint{4.504528in}{8.473868in}}%
\pgfpathlineto{\pgfqpoint{4.416792in}{8.473868in}}%
\pgfpathlineto{\pgfqpoint{4.416792in}{8.561604in}}%
\pgfusepath{stroke,fill}%
\end{pgfscope}%
\begin{pgfscope}%
\pgfpathrectangle{\pgfqpoint{0.380943in}{8.035189in}}{\pgfqpoint{4.650000in}{0.614151in}}%
\pgfusepath{clip}%
\pgfsetbuttcap%
\pgfsetroundjoin%
\definecolor{currentfill}{rgb}{0.999616,0.641369,0.546559}%
\pgfsetfillcolor{currentfill}%
\pgfsetlinewidth{0.250937pt}%
\definecolor{currentstroke}{rgb}{1.000000,1.000000,1.000000}%
\pgfsetstrokecolor{currentstroke}%
\pgfsetdash{}{0pt}%
\pgfpathmoveto{\pgfqpoint{4.504528in}{8.561604in}}%
\pgfpathlineto{\pgfqpoint{4.592264in}{8.561604in}}%
\pgfpathlineto{\pgfqpoint{4.592264in}{8.473868in}}%
\pgfpathlineto{\pgfqpoint{4.504528in}{8.473868in}}%
\pgfpathlineto{\pgfqpoint{4.504528in}{8.561604in}}%
\pgfusepath{stroke,fill}%
\end{pgfscope}%
\begin{pgfscope}%
\pgfpathrectangle{\pgfqpoint{0.380943in}{8.035189in}}{\pgfqpoint{4.650000in}{0.614151in}}%
\pgfusepath{clip}%
\pgfsetbuttcap%
\pgfsetroundjoin%
\definecolor{currentfill}{rgb}{0.999616,0.641369,0.546559}%
\pgfsetfillcolor{currentfill}%
\pgfsetlinewidth{0.250937pt}%
\definecolor{currentstroke}{rgb}{1.000000,1.000000,1.000000}%
\pgfsetstrokecolor{currentstroke}%
\pgfsetdash{}{0pt}%
\pgfpathmoveto{\pgfqpoint{4.592264in}{8.561604in}}%
\pgfpathlineto{\pgfqpoint{4.680000in}{8.561604in}}%
\pgfpathlineto{\pgfqpoint{4.680000in}{8.473868in}}%
\pgfpathlineto{\pgfqpoint{4.592264in}{8.473868in}}%
\pgfpathlineto{\pgfqpoint{4.592264in}{8.561604in}}%
\pgfusepath{stroke,fill}%
\end{pgfscope}%
\begin{pgfscope}%
\pgfpathrectangle{\pgfqpoint{0.380943in}{8.035189in}}{\pgfqpoint{4.650000in}{0.614151in}}%
\pgfusepath{clip}%
\pgfsetbuttcap%
\pgfsetroundjoin%
\definecolor{currentfill}{rgb}{0.800000,0.278431,0.278431}%
\pgfsetfillcolor{currentfill}%
\pgfsetlinewidth{0.250937pt}%
\definecolor{currentstroke}{rgb}{1.000000,1.000000,1.000000}%
\pgfsetstrokecolor{currentstroke}%
\pgfsetdash{}{0pt}%
\pgfpathmoveto{\pgfqpoint{4.680000in}{8.561604in}}%
\pgfpathlineto{\pgfqpoint{4.767736in}{8.561604in}}%
\pgfpathlineto{\pgfqpoint{4.767736in}{8.473868in}}%
\pgfpathlineto{\pgfqpoint{4.680000in}{8.473868in}}%
\pgfpathlineto{\pgfqpoint{4.680000in}{8.561604in}}%
\pgfusepath{stroke,fill}%
\end{pgfscope}%
\begin{pgfscope}%
\pgfpathrectangle{\pgfqpoint{0.380943in}{8.035189in}}{\pgfqpoint{4.650000in}{0.614151in}}%
\pgfusepath{clip}%
\pgfsetbuttcap%
\pgfsetroundjoin%
\definecolor{currentfill}{rgb}{0.986251,0.808597,0.643230}%
\pgfsetfillcolor{currentfill}%
\pgfsetlinewidth{0.250937pt}%
\definecolor{currentstroke}{rgb}{1.000000,1.000000,1.000000}%
\pgfsetstrokecolor{currentstroke}%
\pgfsetdash{}{0pt}%
\pgfpathmoveto{\pgfqpoint{4.767736in}{8.561604in}}%
\pgfpathlineto{\pgfqpoint{4.855471in}{8.561604in}}%
\pgfpathlineto{\pgfqpoint{4.855471in}{8.473868in}}%
\pgfpathlineto{\pgfqpoint{4.767736in}{8.473868in}}%
\pgfpathlineto{\pgfqpoint{4.767736in}{8.561604in}}%
\pgfusepath{stroke,fill}%
\end{pgfscope}%
\begin{pgfscope}%
\pgfpathrectangle{\pgfqpoint{0.380943in}{8.035189in}}{\pgfqpoint{4.650000in}{0.614151in}}%
\pgfusepath{clip}%
\pgfsetbuttcap%
\pgfsetroundjoin%
\definecolor{currentfill}{rgb}{0.999616,0.641369,0.546559}%
\pgfsetfillcolor{currentfill}%
\pgfsetlinewidth{0.250937pt}%
\definecolor{currentstroke}{rgb}{1.000000,1.000000,1.000000}%
\pgfsetstrokecolor{currentstroke}%
\pgfsetdash{}{0pt}%
\pgfpathmoveto{\pgfqpoint{4.855471in}{8.561604in}}%
\pgfpathlineto{\pgfqpoint{4.943207in}{8.561604in}}%
\pgfpathlineto{\pgfqpoint{4.943207in}{8.473868in}}%
\pgfpathlineto{\pgfqpoint{4.855471in}{8.473868in}}%
\pgfpathlineto{\pgfqpoint{4.855471in}{8.561604in}}%
\pgfusepath{stroke,fill}%
\end{pgfscope}%
\begin{pgfscope}%
\pgfpathrectangle{\pgfqpoint{0.380943in}{8.035189in}}{\pgfqpoint{4.650000in}{0.614151in}}%
\pgfusepath{clip}%
\pgfsetbuttcap%
\pgfsetroundjoin%
\definecolor{currentfill}{rgb}{0.963768,0.915433,0.717478}%
\pgfsetfillcolor{currentfill}%
\pgfsetlinewidth{0.250937pt}%
\definecolor{currentstroke}{rgb}{1.000000,1.000000,1.000000}%
\pgfsetstrokecolor{currentstroke}%
\pgfsetdash{}{0pt}%
\pgfpathmoveto{\pgfqpoint{4.943207in}{8.561604in}}%
\pgfpathlineto{\pgfqpoint{5.030943in}{8.561604in}}%
\pgfpathlineto{\pgfqpoint{5.030943in}{8.473868in}}%
\pgfpathlineto{\pgfqpoint{4.943207in}{8.473868in}}%
\pgfpathlineto{\pgfqpoint{4.943207in}{8.561604in}}%
\pgfusepath{stroke,fill}%
\end{pgfscope}%
\begin{pgfscope}%
\pgfpathrectangle{\pgfqpoint{0.380943in}{8.035189in}}{\pgfqpoint{4.650000in}{0.614151in}}%
\pgfusepath{clip}%
\pgfsetbuttcap%
\pgfsetroundjoin%
\pgfsetlinewidth{0.250937pt}%
\definecolor{currentstroke}{rgb}{1.000000,1.000000,1.000000}%
\pgfsetstrokecolor{currentstroke}%
\pgfsetdash{}{0pt}%
\pgfpathmoveto{\pgfqpoint{0.380943in}{8.473868in}}%
\pgfpathlineto{\pgfqpoint{0.468679in}{8.473868in}}%
\pgfpathlineto{\pgfqpoint{0.468679in}{8.386132in}}%
\pgfpathlineto{\pgfqpoint{0.380943in}{8.386132in}}%
\pgfpathlineto{\pgfqpoint{0.380943in}{8.473868in}}%
\pgfusepath{stroke}%
\end{pgfscope}%
\begin{pgfscope}%
\pgfpathrectangle{\pgfqpoint{0.380943in}{8.035189in}}{\pgfqpoint{4.650000in}{0.614151in}}%
\pgfusepath{clip}%
\pgfsetbuttcap%
\pgfsetroundjoin%
\definecolor{currentfill}{rgb}{0.996909,0.711742,0.584452}%
\pgfsetfillcolor{currentfill}%
\pgfsetlinewidth{0.250937pt}%
\definecolor{currentstroke}{rgb}{1.000000,1.000000,1.000000}%
\pgfsetstrokecolor{currentstroke}%
\pgfsetdash{}{0pt}%
\pgfpathmoveto{\pgfqpoint{0.468679in}{8.473868in}}%
\pgfpathlineto{\pgfqpoint{0.556415in}{8.473868in}}%
\pgfpathlineto{\pgfqpoint{0.556415in}{8.386132in}}%
\pgfpathlineto{\pgfqpoint{0.468679in}{8.386132in}}%
\pgfpathlineto{\pgfqpoint{0.468679in}{8.473868in}}%
\pgfusepath{stroke,fill}%
\end{pgfscope}%
\begin{pgfscope}%
\pgfpathrectangle{\pgfqpoint{0.380943in}{8.035189in}}{\pgfqpoint{4.650000in}{0.614151in}}%
\pgfusepath{clip}%
\pgfsetbuttcap%
\pgfsetroundjoin%
\definecolor{currentfill}{rgb}{0.970012,0.883276,0.699577}%
\pgfsetfillcolor{currentfill}%
\pgfsetlinewidth{0.250937pt}%
\definecolor{currentstroke}{rgb}{1.000000,1.000000,1.000000}%
\pgfsetstrokecolor{currentstroke}%
\pgfsetdash{}{0pt}%
\pgfpathmoveto{\pgfqpoint{0.556415in}{8.473868in}}%
\pgfpathlineto{\pgfqpoint{0.644151in}{8.473868in}}%
\pgfpathlineto{\pgfqpoint{0.644151in}{8.386132in}}%
\pgfpathlineto{\pgfqpoint{0.556415in}{8.386132in}}%
\pgfpathlineto{\pgfqpoint{0.556415in}{8.473868in}}%
\pgfusepath{stroke,fill}%
\end{pgfscope}%
\begin{pgfscope}%
\pgfpathrectangle{\pgfqpoint{0.380943in}{8.035189in}}{\pgfqpoint{4.650000in}{0.614151in}}%
\pgfusepath{clip}%
\pgfsetbuttcap%
\pgfsetroundjoin%
\definecolor{currentfill}{rgb}{0.996909,0.711742,0.584452}%
\pgfsetfillcolor{currentfill}%
\pgfsetlinewidth{0.250937pt}%
\definecolor{currentstroke}{rgb}{1.000000,1.000000,1.000000}%
\pgfsetstrokecolor{currentstroke}%
\pgfsetdash{}{0pt}%
\pgfpathmoveto{\pgfqpoint{0.644151in}{8.473868in}}%
\pgfpathlineto{\pgfqpoint{0.731886in}{8.473868in}}%
\pgfpathlineto{\pgfqpoint{0.731886in}{8.386132in}}%
\pgfpathlineto{\pgfqpoint{0.644151in}{8.386132in}}%
\pgfpathlineto{\pgfqpoint{0.644151in}{8.473868in}}%
\pgfusepath{stroke,fill}%
\end{pgfscope}%
\begin{pgfscope}%
\pgfpathrectangle{\pgfqpoint{0.380943in}{8.035189in}}{\pgfqpoint{4.650000in}{0.614151in}}%
\pgfusepath{clip}%
\pgfsetbuttcap%
\pgfsetroundjoin%
\definecolor{currentfill}{rgb}{0.978131,0.843783,0.675709}%
\pgfsetfillcolor{currentfill}%
\pgfsetlinewidth{0.250937pt}%
\definecolor{currentstroke}{rgb}{1.000000,1.000000,1.000000}%
\pgfsetstrokecolor{currentstroke}%
\pgfsetdash{}{0pt}%
\pgfpathmoveto{\pgfqpoint{0.731886in}{8.473868in}}%
\pgfpathlineto{\pgfqpoint{0.819622in}{8.473868in}}%
\pgfpathlineto{\pgfqpoint{0.819622in}{8.386132in}}%
\pgfpathlineto{\pgfqpoint{0.731886in}{8.386132in}}%
\pgfpathlineto{\pgfqpoint{0.731886in}{8.473868in}}%
\pgfusepath{stroke,fill}%
\end{pgfscope}%
\begin{pgfscope}%
\pgfpathrectangle{\pgfqpoint{0.380943in}{8.035189in}}{\pgfqpoint{4.650000in}{0.614151in}}%
\pgfusepath{clip}%
\pgfsetbuttcap%
\pgfsetroundjoin%
\definecolor{currentfill}{rgb}{0.992326,0.765229,0.614840}%
\pgfsetfillcolor{currentfill}%
\pgfsetlinewidth{0.250937pt}%
\definecolor{currentstroke}{rgb}{1.000000,1.000000,1.000000}%
\pgfsetstrokecolor{currentstroke}%
\pgfsetdash{}{0pt}%
\pgfpathmoveto{\pgfqpoint{0.819622in}{8.473868in}}%
\pgfpathlineto{\pgfqpoint{0.907358in}{8.473868in}}%
\pgfpathlineto{\pgfqpoint{0.907358in}{8.386132in}}%
\pgfpathlineto{\pgfqpoint{0.819622in}{8.386132in}}%
\pgfpathlineto{\pgfqpoint{0.819622in}{8.473868in}}%
\pgfusepath{stroke,fill}%
\end{pgfscope}%
\begin{pgfscope}%
\pgfpathrectangle{\pgfqpoint{0.380943in}{8.035189in}}{\pgfqpoint{4.650000in}{0.614151in}}%
\pgfusepath{clip}%
\pgfsetbuttcap%
\pgfsetroundjoin%
\definecolor{currentfill}{rgb}{0.996909,0.711742,0.584452}%
\pgfsetfillcolor{currentfill}%
\pgfsetlinewidth{0.250937pt}%
\definecolor{currentstroke}{rgb}{1.000000,1.000000,1.000000}%
\pgfsetstrokecolor{currentstroke}%
\pgfsetdash{}{0pt}%
\pgfpathmoveto{\pgfqpoint{0.907358in}{8.473868in}}%
\pgfpathlineto{\pgfqpoint{0.995094in}{8.473868in}}%
\pgfpathlineto{\pgfqpoint{0.995094in}{8.386132in}}%
\pgfpathlineto{\pgfqpoint{0.907358in}{8.386132in}}%
\pgfpathlineto{\pgfqpoint{0.907358in}{8.473868in}}%
\pgfusepath{stroke,fill}%
\end{pgfscope}%
\begin{pgfscope}%
\pgfpathrectangle{\pgfqpoint{0.380943in}{8.035189in}}{\pgfqpoint{4.650000in}{0.614151in}}%
\pgfusepath{clip}%
\pgfsetbuttcap%
\pgfsetroundjoin%
\definecolor{currentfill}{rgb}{0.986251,0.808597,0.643230}%
\pgfsetfillcolor{currentfill}%
\pgfsetlinewidth{0.250937pt}%
\definecolor{currentstroke}{rgb}{1.000000,1.000000,1.000000}%
\pgfsetstrokecolor{currentstroke}%
\pgfsetdash{}{0pt}%
\pgfpathmoveto{\pgfqpoint{0.995094in}{8.473868in}}%
\pgfpathlineto{\pgfqpoint{1.082830in}{8.473868in}}%
\pgfpathlineto{\pgfqpoint{1.082830in}{8.386132in}}%
\pgfpathlineto{\pgfqpoint{0.995094in}{8.386132in}}%
\pgfpathlineto{\pgfqpoint{0.995094in}{8.473868in}}%
\pgfusepath{stroke,fill}%
\end{pgfscope}%
\begin{pgfscope}%
\pgfpathrectangle{\pgfqpoint{0.380943in}{8.035189in}}{\pgfqpoint{4.650000in}{0.614151in}}%
\pgfusepath{clip}%
\pgfsetbuttcap%
\pgfsetroundjoin%
\definecolor{currentfill}{rgb}{0.963768,0.915433,0.717478}%
\pgfsetfillcolor{currentfill}%
\pgfsetlinewidth{0.250937pt}%
\definecolor{currentstroke}{rgb}{1.000000,1.000000,1.000000}%
\pgfsetstrokecolor{currentstroke}%
\pgfsetdash{}{0pt}%
\pgfpathmoveto{\pgfqpoint{1.082830in}{8.473868in}}%
\pgfpathlineto{\pgfqpoint{1.170566in}{8.473868in}}%
\pgfpathlineto{\pgfqpoint{1.170566in}{8.386132in}}%
\pgfpathlineto{\pgfqpoint{1.082830in}{8.386132in}}%
\pgfpathlineto{\pgfqpoint{1.082830in}{8.473868in}}%
\pgfusepath{stroke,fill}%
\end{pgfscope}%
\begin{pgfscope}%
\pgfpathrectangle{\pgfqpoint{0.380943in}{8.035189in}}{\pgfqpoint{4.650000in}{0.614151in}}%
\pgfusepath{clip}%
\pgfsetbuttcap%
\pgfsetroundjoin%
\definecolor{currentfill}{rgb}{0.961061,0.931672,0.728304}%
\pgfsetfillcolor{currentfill}%
\pgfsetlinewidth{0.250937pt}%
\definecolor{currentstroke}{rgb}{1.000000,1.000000,1.000000}%
\pgfsetstrokecolor{currentstroke}%
\pgfsetdash{}{0pt}%
\pgfpathmoveto{\pgfqpoint{1.170566in}{8.473868in}}%
\pgfpathlineto{\pgfqpoint{1.258302in}{8.473868in}}%
\pgfpathlineto{\pgfqpoint{1.258302in}{8.386132in}}%
\pgfpathlineto{\pgfqpoint{1.170566in}{8.386132in}}%
\pgfpathlineto{\pgfqpoint{1.170566in}{8.473868in}}%
\pgfusepath{stroke,fill}%
\end{pgfscope}%
\begin{pgfscope}%
\pgfpathrectangle{\pgfqpoint{0.380943in}{8.035189in}}{\pgfqpoint{4.650000in}{0.614151in}}%
\pgfusepath{clip}%
\pgfsetbuttcap%
\pgfsetroundjoin%
\definecolor{currentfill}{rgb}{0.978131,0.843783,0.675709}%
\pgfsetfillcolor{currentfill}%
\pgfsetlinewidth{0.250937pt}%
\definecolor{currentstroke}{rgb}{1.000000,1.000000,1.000000}%
\pgfsetstrokecolor{currentstroke}%
\pgfsetdash{}{0pt}%
\pgfpathmoveto{\pgfqpoint{1.258302in}{8.473868in}}%
\pgfpathlineto{\pgfqpoint{1.346037in}{8.473868in}}%
\pgfpathlineto{\pgfqpoint{1.346037in}{8.386132in}}%
\pgfpathlineto{\pgfqpoint{1.258302in}{8.386132in}}%
\pgfpathlineto{\pgfqpoint{1.258302in}{8.473868in}}%
\pgfusepath{stroke,fill}%
\end{pgfscope}%
\begin{pgfscope}%
\pgfpathrectangle{\pgfqpoint{0.380943in}{8.035189in}}{\pgfqpoint{4.650000in}{0.614151in}}%
\pgfusepath{clip}%
\pgfsetbuttcap%
\pgfsetroundjoin%
\definecolor{currentfill}{rgb}{0.986251,0.808597,0.643230}%
\pgfsetfillcolor{currentfill}%
\pgfsetlinewidth{0.250937pt}%
\definecolor{currentstroke}{rgb}{1.000000,1.000000,1.000000}%
\pgfsetstrokecolor{currentstroke}%
\pgfsetdash{}{0pt}%
\pgfpathmoveto{\pgfqpoint{1.346037in}{8.473868in}}%
\pgfpathlineto{\pgfqpoint{1.433773in}{8.473868in}}%
\pgfpathlineto{\pgfqpoint{1.433773in}{8.386132in}}%
\pgfpathlineto{\pgfqpoint{1.346037in}{8.386132in}}%
\pgfpathlineto{\pgfqpoint{1.346037in}{8.473868in}}%
\pgfusepath{stroke,fill}%
\end{pgfscope}%
\begin{pgfscope}%
\pgfpathrectangle{\pgfqpoint{0.380943in}{8.035189in}}{\pgfqpoint{4.650000in}{0.614151in}}%
\pgfusepath{clip}%
\pgfsetbuttcap%
\pgfsetroundjoin%
\definecolor{currentfill}{rgb}{0.800000,0.278431,0.278431}%
\pgfsetfillcolor{currentfill}%
\pgfsetlinewidth{0.250937pt}%
\definecolor{currentstroke}{rgb}{1.000000,1.000000,1.000000}%
\pgfsetstrokecolor{currentstroke}%
\pgfsetdash{}{0pt}%
\pgfpathmoveto{\pgfqpoint{1.433773in}{8.473868in}}%
\pgfpathlineto{\pgfqpoint{1.521509in}{8.473868in}}%
\pgfpathlineto{\pgfqpoint{1.521509in}{8.386132in}}%
\pgfpathlineto{\pgfqpoint{1.433773in}{8.386132in}}%
\pgfpathlineto{\pgfqpoint{1.433773in}{8.473868in}}%
\pgfusepath{stroke,fill}%
\end{pgfscope}%
\begin{pgfscope}%
\pgfpathrectangle{\pgfqpoint{0.380943in}{8.035189in}}{\pgfqpoint{4.650000in}{0.614151in}}%
\pgfusepath{clip}%
\pgfsetbuttcap%
\pgfsetroundjoin%
\definecolor{currentfill}{rgb}{0.970012,0.883276,0.699577}%
\pgfsetfillcolor{currentfill}%
\pgfsetlinewidth{0.250937pt}%
\definecolor{currentstroke}{rgb}{1.000000,1.000000,1.000000}%
\pgfsetstrokecolor{currentstroke}%
\pgfsetdash{}{0pt}%
\pgfpathmoveto{\pgfqpoint{1.521509in}{8.473868in}}%
\pgfpathlineto{\pgfqpoint{1.609245in}{8.473868in}}%
\pgfpathlineto{\pgfqpoint{1.609245in}{8.386132in}}%
\pgfpathlineto{\pgfqpoint{1.521509in}{8.386132in}}%
\pgfpathlineto{\pgfqpoint{1.521509in}{8.473868in}}%
\pgfusepath{stroke,fill}%
\end{pgfscope}%
\begin{pgfscope}%
\pgfpathrectangle{\pgfqpoint{0.380943in}{8.035189in}}{\pgfqpoint{4.650000in}{0.614151in}}%
\pgfusepath{clip}%
\pgfsetbuttcap%
\pgfsetroundjoin%
\definecolor{currentfill}{rgb}{0.978131,0.843783,0.675709}%
\pgfsetfillcolor{currentfill}%
\pgfsetlinewidth{0.250937pt}%
\definecolor{currentstroke}{rgb}{1.000000,1.000000,1.000000}%
\pgfsetstrokecolor{currentstroke}%
\pgfsetdash{}{0pt}%
\pgfpathmoveto{\pgfqpoint{1.609245in}{8.473868in}}%
\pgfpathlineto{\pgfqpoint{1.696981in}{8.473868in}}%
\pgfpathlineto{\pgfqpoint{1.696981in}{8.386132in}}%
\pgfpathlineto{\pgfqpoint{1.609245in}{8.386132in}}%
\pgfpathlineto{\pgfqpoint{1.609245in}{8.473868in}}%
\pgfusepath{stroke,fill}%
\end{pgfscope}%
\begin{pgfscope}%
\pgfpathrectangle{\pgfqpoint{0.380943in}{8.035189in}}{\pgfqpoint{4.650000in}{0.614151in}}%
\pgfusepath{clip}%
\pgfsetbuttcap%
\pgfsetroundjoin%
\definecolor{currentfill}{rgb}{0.963768,0.915433,0.717478}%
\pgfsetfillcolor{currentfill}%
\pgfsetlinewidth{0.250937pt}%
\definecolor{currentstroke}{rgb}{1.000000,1.000000,1.000000}%
\pgfsetstrokecolor{currentstroke}%
\pgfsetdash{}{0pt}%
\pgfpathmoveto{\pgfqpoint{1.696981in}{8.473868in}}%
\pgfpathlineto{\pgfqpoint{1.784717in}{8.473868in}}%
\pgfpathlineto{\pgfqpoint{1.784717in}{8.386132in}}%
\pgfpathlineto{\pgfqpoint{1.696981in}{8.386132in}}%
\pgfpathlineto{\pgfqpoint{1.696981in}{8.473868in}}%
\pgfusepath{stroke,fill}%
\end{pgfscope}%
\begin{pgfscope}%
\pgfpathrectangle{\pgfqpoint{0.380943in}{8.035189in}}{\pgfqpoint{4.650000in}{0.614151in}}%
\pgfusepath{clip}%
\pgfsetbuttcap%
\pgfsetroundjoin%
\definecolor{currentfill}{rgb}{0.986251,0.808597,0.643230}%
\pgfsetfillcolor{currentfill}%
\pgfsetlinewidth{0.250937pt}%
\definecolor{currentstroke}{rgb}{1.000000,1.000000,1.000000}%
\pgfsetstrokecolor{currentstroke}%
\pgfsetdash{}{0pt}%
\pgfpathmoveto{\pgfqpoint{1.784717in}{8.473868in}}%
\pgfpathlineto{\pgfqpoint{1.872452in}{8.473868in}}%
\pgfpathlineto{\pgfqpoint{1.872452in}{8.386132in}}%
\pgfpathlineto{\pgfqpoint{1.784717in}{8.386132in}}%
\pgfpathlineto{\pgfqpoint{1.784717in}{8.473868in}}%
\pgfusepath{stroke,fill}%
\end{pgfscope}%
\begin{pgfscope}%
\pgfpathrectangle{\pgfqpoint{0.380943in}{8.035189in}}{\pgfqpoint{4.650000in}{0.614151in}}%
\pgfusepath{clip}%
\pgfsetbuttcap%
\pgfsetroundjoin%
\definecolor{currentfill}{rgb}{0.986251,0.808597,0.643230}%
\pgfsetfillcolor{currentfill}%
\pgfsetlinewidth{0.250937pt}%
\definecolor{currentstroke}{rgb}{1.000000,1.000000,1.000000}%
\pgfsetstrokecolor{currentstroke}%
\pgfsetdash{}{0pt}%
\pgfpathmoveto{\pgfqpoint{1.872452in}{8.473868in}}%
\pgfpathlineto{\pgfqpoint{1.960188in}{8.473868in}}%
\pgfpathlineto{\pgfqpoint{1.960188in}{8.386132in}}%
\pgfpathlineto{\pgfqpoint{1.872452in}{8.386132in}}%
\pgfpathlineto{\pgfqpoint{1.872452in}{8.473868in}}%
\pgfusepath{stroke,fill}%
\end{pgfscope}%
\begin{pgfscope}%
\pgfpathrectangle{\pgfqpoint{0.380943in}{8.035189in}}{\pgfqpoint{4.650000in}{0.614151in}}%
\pgfusepath{clip}%
\pgfsetbuttcap%
\pgfsetroundjoin%
\definecolor{currentfill}{rgb}{0.999616,0.641369,0.546559}%
\pgfsetfillcolor{currentfill}%
\pgfsetlinewidth{0.250937pt}%
\definecolor{currentstroke}{rgb}{1.000000,1.000000,1.000000}%
\pgfsetstrokecolor{currentstroke}%
\pgfsetdash{}{0pt}%
\pgfpathmoveto{\pgfqpoint{1.960188in}{8.473868in}}%
\pgfpathlineto{\pgfqpoint{2.047924in}{8.473868in}}%
\pgfpathlineto{\pgfqpoint{2.047924in}{8.386132in}}%
\pgfpathlineto{\pgfqpoint{1.960188in}{8.386132in}}%
\pgfpathlineto{\pgfqpoint{1.960188in}{8.473868in}}%
\pgfusepath{stroke,fill}%
\end{pgfscope}%
\begin{pgfscope}%
\pgfpathrectangle{\pgfqpoint{0.380943in}{8.035189in}}{\pgfqpoint{4.650000in}{0.614151in}}%
\pgfusepath{clip}%
\pgfsetbuttcap%
\pgfsetroundjoin%
\definecolor{currentfill}{rgb}{0.992326,0.765229,0.614840}%
\pgfsetfillcolor{currentfill}%
\pgfsetlinewidth{0.250937pt}%
\definecolor{currentstroke}{rgb}{1.000000,1.000000,1.000000}%
\pgfsetstrokecolor{currentstroke}%
\pgfsetdash{}{0pt}%
\pgfpathmoveto{\pgfqpoint{2.047924in}{8.473868in}}%
\pgfpathlineto{\pgfqpoint{2.135660in}{8.473868in}}%
\pgfpathlineto{\pgfqpoint{2.135660in}{8.386132in}}%
\pgfpathlineto{\pgfqpoint{2.047924in}{8.386132in}}%
\pgfpathlineto{\pgfqpoint{2.047924in}{8.473868in}}%
\pgfusepath{stroke,fill}%
\end{pgfscope}%
\begin{pgfscope}%
\pgfpathrectangle{\pgfqpoint{0.380943in}{8.035189in}}{\pgfqpoint{4.650000in}{0.614151in}}%
\pgfusepath{clip}%
\pgfsetbuttcap%
\pgfsetroundjoin%
\definecolor{currentfill}{rgb}{0.986251,0.808597,0.643230}%
\pgfsetfillcolor{currentfill}%
\pgfsetlinewidth{0.250937pt}%
\definecolor{currentstroke}{rgb}{1.000000,1.000000,1.000000}%
\pgfsetstrokecolor{currentstroke}%
\pgfsetdash{}{0pt}%
\pgfpathmoveto{\pgfqpoint{2.135660in}{8.473868in}}%
\pgfpathlineto{\pgfqpoint{2.223396in}{8.473868in}}%
\pgfpathlineto{\pgfqpoint{2.223396in}{8.386132in}}%
\pgfpathlineto{\pgfqpoint{2.135660in}{8.386132in}}%
\pgfpathlineto{\pgfqpoint{2.135660in}{8.473868in}}%
\pgfusepath{stroke,fill}%
\end{pgfscope}%
\begin{pgfscope}%
\pgfpathrectangle{\pgfqpoint{0.380943in}{8.035189in}}{\pgfqpoint{4.650000in}{0.614151in}}%
\pgfusepath{clip}%
\pgfsetbuttcap%
\pgfsetroundjoin%
\definecolor{currentfill}{rgb}{0.986251,0.808597,0.643230}%
\pgfsetfillcolor{currentfill}%
\pgfsetlinewidth{0.250937pt}%
\definecolor{currentstroke}{rgb}{1.000000,1.000000,1.000000}%
\pgfsetstrokecolor{currentstroke}%
\pgfsetdash{}{0pt}%
\pgfpathmoveto{\pgfqpoint{2.223396in}{8.473868in}}%
\pgfpathlineto{\pgfqpoint{2.311132in}{8.473868in}}%
\pgfpathlineto{\pgfqpoint{2.311132in}{8.386132in}}%
\pgfpathlineto{\pgfqpoint{2.223396in}{8.386132in}}%
\pgfpathlineto{\pgfqpoint{2.223396in}{8.473868in}}%
\pgfusepath{stroke,fill}%
\end{pgfscope}%
\begin{pgfscope}%
\pgfpathrectangle{\pgfqpoint{0.380943in}{8.035189in}}{\pgfqpoint{4.650000in}{0.614151in}}%
\pgfusepath{clip}%
\pgfsetbuttcap%
\pgfsetroundjoin%
\definecolor{currentfill}{rgb}{0.992326,0.765229,0.614840}%
\pgfsetfillcolor{currentfill}%
\pgfsetlinewidth{0.250937pt}%
\definecolor{currentstroke}{rgb}{1.000000,1.000000,1.000000}%
\pgfsetstrokecolor{currentstroke}%
\pgfsetdash{}{0pt}%
\pgfpathmoveto{\pgfqpoint{2.311132in}{8.473868in}}%
\pgfpathlineto{\pgfqpoint{2.398868in}{8.473868in}}%
\pgfpathlineto{\pgfqpoint{2.398868in}{8.386132in}}%
\pgfpathlineto{\pgfqpoint{2.311132in}{8.386132in}}%
\pgfpathlineto{\pgfqpoint{2.311132in}{8.473868in}}%
\pgfusepath{stroke,fill}%
\end{pgfscope}%
\begin{pgfscope}%
\pgfpathrectangle{\pgfqpoint{0.380943in}{8.035189in}}{\pgfqpoint{4.650000in}{0.614151in}}%
\pgfusepath{clip}%
\pgfsetbuttcap%
\pgfsetroundjoin%
\definecolor{currentfill}{rgb}{0.963768,0.915433,0.717478}%
\pgfsetfillcolor{currentfill}%
\pgfsetlinewidth{0.250937pt}%
\definecolor{currentstroke}{rgb}{1.000000,1.000000,1.000000}%
\pgfsetstrokecolor{currentstroke}%
\pgfsetdash{}{0pt}%
\pgfpathmoveto{\pgfqpoint{2.398868in}{8.473868in}}%
\pgfpathlineto{\pgfqpoint{2.486603in}{8.473868in}}%
\pgfpathlineto{\pgfqpoint{2.486603in}{8.386132in}}%
\pgfpathlineto{\pgfqpoint{2.398868in}{8.386132in}}%
\pgfpathlineto{\pgfqpoint{2.398868in}{8.473868in}}%
\pgfusepath{stroke,fill}%
\end{pgfscope}%
\begin{pgfscope}%
\pgfpathrectangle{\pgfqpoint{0.380943in}{8.035189in}}{\pgfqpoint{4.650000in}{0.614151in}}%
\pgfusepath{clip}%
\pgfsetbuttcap%
\pgfsetroundjoin%
\definecolor{currentfill}{rgb}{0.961061,0.931672,0.728304}%
\pgfsetfillcolor{currentfill}%
\pgfsetlinewidth{0.250937pt}%
\definecolor{currentstroke}{rgb}{1.000000,1.000000,1.000000}%
\pgfsetstrokecolor{currentstroke}%
\pgfsetdash{}{0pt}%
\pgfpathmoveto{\pgfqpoint{2.486603in}{8.473868in}}%
\pgfpathlineto{\pgfqpoint{2.574339in}{8.473868in}}%
\pgfpathlineto{\pgfqpoint{2.574339in}{8.386132in}}%
\pgfpathlineto{\pgfqpoint{2.486603in}{8.386132in}}%
\pgfpathlineto{\pgfqpoint{2.486603in}{8.473868in}}%
\pgfusepath{stroke,fill}%
\end{pgfscope}%
\begin{pgfscope}%
\pgfpathrectangle{\pgfqpoint{0.380943in}{8.035189in}}{\pgfqpoint{4.650000in}{0.614151in}}%
\pgfusepath{clip}%
\pgfsetbuttcap%
\pgfsetroundjoin%
\definecolor{currentfill}{rgb}{0.986251,0.808597,0.643230}%
\pgfsetfillcolor{currentfill}%
\pgfsetlinewidth{0.250937pt}%
\definecolor{currentstroke}{rgb}{1.000000,1.000000,1.000000}%
\pgfsetstrokecolor{currentstroke}%
\pgfsetdash{}{0pt}%
\pgfpathmoveto{\pgfqpoint{2.574339in}{8.473868in}}%
\pgfpathlineto{\pgfqpoint{2.662075in}{8.473868in}}%
\pgfpathlineto{\pgfqpoint{2.662075in}{8.386132in}}%
\pgfpathlineto{\pgfqpoint{2.574339in}{8.386132in}}%
\pgfpathlineto{\pgfqpoint{2.574339in}{8.473868in}}%
\pgfusepath{stroke,fill}%
\end{pgfscope}%
\begin{pgfscope}%
\pgfpathrectangle{\pgfqpoint{0.380943in}{8.035189in}}{\pgfqpoint{4.650000in}{0.614151in}}%
\pgfusepath{clip}%
\pgfsetbuttcap%
\pgfsetroundjoin%
\definecolor{currentfill}{rgb}{1.000000,0.531903,0.500946}%
\pgfsetfillcolor{currentfill}%
\pgfsetlinewidth{0.250937pt}%
\definecolor{currentstroke}{rgb}{1.000000,1.000000,1.000000}%
\pgfsetstrokecolor{currentstroke}%
\pgfsetdash{}{0pt}%
\pgfpathmoveto{\pgfqpoint{2.662075in}{8.473868in}}%
\pgfpathlineto{\pgfqpoint{2.749811in}{8.473868in}}%
\pgfpathlineto{\pgfqpoint{2.749811in}{8.386132in}}%
\pgfpathlineto{\pgfqpoint{2.662075in}{8.386132in}}%
\pgfpathlineto{\pgfqpoint{2.662075in}{8.473868in}}%
\pgfusepath{stroke,fill}%
\end{pgfscope}%
\begin{pgfscope}%
\pgfpathrectangle{\pgfqpoint{0.380943in}{8.035189in}}{\pgfqpoint{4.650000in}{0.614151in}}%
\pgfusepath{clip}%
\pgfsetbuttcap%
\pgfsetroundjoin%
\definecolor{currentfill}{rgb}{0.996909,0.711742,0.584452}%
\pgfsetfillcolor{currentfill}%
\pgfsetlinewidth{0.250937pt}%
\definecolor{currentstroke}{rgb}{1.000000,1.000000,1.000000}%
\pgfsetstrokecolor{currentstroke}%
\pgfsetdash{}{0pt}%
\pgfpathmoveto{\pgfqpoint{2.749811in}{8.473868in}}%
\pgfpathlineto{\pgfqpoint{2.837547in}{8.473868in}}%
\pgfpathlineto{\pgfqpoint{2.837547in}{8.386132in}}%
\pgfpathlineto{\pgfqpoint{2.749811in}{8.386132in}}%
\pgfpathlineto{\pgfqpoint{2.749811in}{8.473868in}}%
\pgfusepath{stroke,fill}%
\end{pgfscope}%
\begin{pgfscope}%
\pgfpathrectangle{\pgfqpoint{0.380943in}{8.035189in}}{\pgfqpoint{4.650000in}{0.614151in}}%
\pgfusepath{clip}%
\pgfsetbuttcap%
\pgfsetroundjoin%
\definecolor{currentfill}{rgb}{0.970012,0.883276,0.699577}%
\pgfsetfillcolor{currentfill}%
\pgfsetlinewidth{0.250937pt}%
\definecolor{currentstroke}{rgb}{1.000000,1.000000,1.000000}%
\pgfsetstrokecolor{currentstroke}%
\pgfsetdash{}{0pt}%
\pgfpathmoveto{\pgfqpoint{2.837547in}{8.473868in}}%
\pgfpathlineto{\pgfqpoint{2.925283in}{8.473868in}}%
\pgfpathlineto{\pgfqpoint{2.925283in}{8.386132in}}%
\pgfpathlineto{\pgfqpoint{2.837547in}{8.386132in}}%
\pgfpathlineto{\pgfqpoint{2.837547in}{8.473868in}}%
\pgfusepath{stroke,fill}%
\end{pgfscope}%
\begin{pgfscope}%
\pgfpathrectangle{\pgfqpoint{0.380943in}{8.035189in}}{\pgfqpoint{4.650000in}{0.614151in}}%
\pgfusepath{clip}%
\pgfsetbuttcap%
\pgfsetroundjoin%
\definecolor{currentfill}{rgb}{0.978131,0.843783,0.675709}%
\pgfsetfillcolor{currentfill}%
\pgfsetlinewidth{0.250937pt}%
\definecolor{currentstroke}{rgb}{1.000000,1.000000,1.000000}%
\pgfsetstrokecolor{currentstroke}%
\pgfsetdash{}{0pt}%
\pgfpathmoveto{\pgfqpoint{2.925283in}{8.473868in}}%
\pgfpathlineto{\pgfqpoint{3.013019in}{8.473868in}}%
\pgfpathlineto{\pgfqpoint{3.013019in}{8.386132in}}%
\pgfpathlineto{\pgfqpoint{2.925283in}{8.386132in}}%
\pgfpathlineto{\pgfqpoint{2.925283in}{8.473868in}}%
\pgfusepath{stroke,fill}%
\end{pgfscope}%
\begin{pgfscope}%
\pgfpathrectangle{\pgfqpoint{0.380943in}{8.035189in}}{\pgfqpoint{4.650000in}{0.614151in}}%
\pgfusepath{clip}%
\pgfsetbuttcap%
\pgfsetroundjoin%
\definecolor{currentfill}{rgb}{0.865975,0.344406,0.344406}%
\pgfsetfillcolor{currentfill}%
\pgfsetlinewidth{0.250937pt}%
\definecolor{currentstroke}{rgb}{1.000000,1.000000,1.000000}%
\pgfsetstrokecolor{currentstroke}%
\pgfsetdash{}{0pt}%
\pgfpathmoveto{\pgfqpoint{3.013019in}{8.473868in}}%
\pgfpathlineto{\pgfqpoint{3.100754in}{8.473868in}}%
\pgfpathlineto{\pgfqpoint{3.100754in}{8.386132in}}%
\pgfpathlineto{\pgfqpoint{3.013019in}{8.386132in}}%
\pgfpathlineto{\pgfqpoint{3.013019in}{8.473868in}}%
\pgfusepath{stroke,fill}%
\end{pgfscope}%
\begin{pgfscope}%
\pgfpathrectangle{\pgfqpoint{0.380943in}{8.035189in}}{\pgfqpoint{4.650000in}{0.614151in}}%
\pgfusepath{clip}%
\pgfsetbuttcap%
\pgfsetroundjoin%
\definecolor{currentfill}{rgb}{0.978131,0.843783,0.675709}%
\pgfsetfillcolor{currentfill}%
\pgfsetlinewidth{0.250937pt}%
\definecolor{currentstroke}{rgb}{1.000000,1.000000,1.000000}%
\pgfsetstrokecolor{currentstroke}%
\pgfsetdash{}{0pt}%
\pgfpathmoveto{\pgfqpoint{3.100754in}{8.473868in}}%
\pgfpathlineto{\pgfqpoint{3.188490in}{8.473868in}}%
\pgfpathlineto{\pgfqpoint{3.188490in}{8.386132in}}%
\pgfpathlineto{\pgfqpoint{3.100754in}{8.386132in}}%
\pgfpathlineto{\pgfqpoint{3.100754in}{8.473868in}}%
\pgfusepath{stroke,fill}%
\end{pgfscope}%
\begin{pgfscope}%
\pgfpathrectangle{\pgfqpoint{0.380943in}{8.035189in}}{\pgfqpoint{4.650000in}{0.614151in}}%
\pgfusepath{clip}%
\pgfsetbuttcap%
\pgfsetroundjoin%
\definecolor{currentfill}{rgb}{0.978131,0.843783,0.675709}%
\pgfsetfillcolor{currentfill}%
\pgfsetlinewidth{0.250937pt}%
\definecolor{currentstroke}{rgb}{1.000000,1.000000,1.000000}%
\pgfsetstrokecolor{currentstroke}%
\pgfsetdash{}{0pt}%
\pgfpathmoveto{\pgfqpoint{3.188490in}{8.473868in}}%
\pgfpathlineto{\pgfqpoint{3.276226in}{8.473868in}}%
\pgfpathlineto{\pgfqpoint{3.276226in}{8.386132in}}%
\pgfpathlineto{\pgfqpoint{3.188490in}{8.386132in}}%
\pgfpathlineto{\pgfqpoint{3.188490in}{8.473868in}}%
\pgfusepath{stroke,fill}%
\end{pgfscope}%
\begin{pgfscope}%
\pgfpathrectangle{\pgfqpoint{0.380943in}{8.035189in}}{\pgfqpoint{4.650000in}{0.614151in}}%
\pgfusepath{clip}%
\pgfsetbuttcap%
\pgfsetroundjoin%
\definecolor{currentfill}{rgb}{0.992326,0.765229,0.614840}%
\pgfsetfillcolor{currentfill}%
\pgfsetlinewidth{0.250937pt}%
\definecolor{currentstroke}{rgb}{1.000000,1.000000,1.000000}%
\pgfsetstrokecolor{currentstroke}%
\pgfsetdash{}{0pt}%
\pgfpathmoveto{\pgfqpoint{3.276226in}{8.473868in}}%
\pgfpathlineto{\pgfqpoint{3.363962in}{8.473868in}}%
\pgfpathlineto{\pgfqpoint{3.363962in}{8.386132in}}%
\pgfpathlineto{\pgfqpoint{3.276226in}{8.386132in}}%
\pgfpathlineto{\pgfqpoint{3.276226in}{8.473868in}}%
\pgfusepath{stroke,fill}%
\end{pgfscope}%
\begin{pgfscope}%
\pgfpathrectangle{\pgfqpoint{0.380943in}{8.035189in}}{\pgfqpoint{4.650000in}{0.614151in}}%
\pgfusepath{clip}%
\pgfsetbuttcap%
\pgfsetroundjoin%
\definecolor{currentfill}{rgb}{0.992326,0.765229,0.614840}%
\pgfsetfillcolor{currentfill}%
\pgfsetlinewidth{0.250937pt}%
\definecolor{currentstroke}{rgb}{1.000000,1.000000,1.000000}%
\pgfsetstrokecolor{currentstroke}%
\pgfsetdash{}{0pt}%
\pgfpathmoveto{\pgfqpoint{3.363962in}{8.473868in}}%
\pgfpathlineto{\pgfqpoint{3.451698in}{8.473868in}}%
\pgfpathlineto{\pgfqpoint{3.451698in}{8.386132in}}%
\pgfpathlineto{\pgfqpoint{3.363962in}{8.386132in}}%
\pgfpathlineto{\pgfqpoint{3.363962in}{8.473868in}}%
\pgfusepath{stroke,fill}%
\end{pgfscope}%
\begin{pgfscope}%
\pgfpathrectangle{\pgfqpoint{0.380943in}{8.035189in}}{\pgfqpoint{4.650000in}{0.614151in}}%
\pgfusepath{clip}%
\pgfsetbuttcap%
\pgfsetroundjoin%
\definecolor{currentfill}{rgb}{0.963768,0.915433,0.717478}%
\pgfsetfillcolor{currentfill}%
\pgfsetlinewidth{0.250937pt}%
\definecolor{currentstroke}{rgb}{1.000000,1.000000,1.000000}%
\pgfsetstrokecolor{currentstroke}%
\pgfsetdash{}{0pt}%
\pgfpathmoveto{\pgfqpoint{3.451698in}{8.473868in}}%
\pgfpathlineto{\pgfqpoint{3.539434in}{8.473868in}}%
\pgfpathlineto{\pgfqpoint{3.539434in}{8.386132in}}%
\pgfpathlineto{\pgfqpoint{3.451698in}{8.386132in}}%
\pgfpathlineto{\pgfqpoint{3.451698in}{8.473868in}}%
\pgfusepath{stroke,fill}%
\end{pgfscope}%
\begin{pgfscope}%
\pgfpathrectangle{\pgfqpoint{0.380943in}{8.035189in}}{\pgfqpoint{4.650000in}{0.614151in}}%
\pgfusepath{clip}%
\pgfsetbuttcap%
\pgfsetroundjoin%
\definecolor{currentfill}{rgb}{0.978131,0.843783,0.675709}%
\pgfsetfillcolor{currentfill}%
\pgfsetlinewidth{0.250937pt}%
\definecolor{currentstroke}{rgb}{1.000000,1.000000,1.000000}%
\pgfsetstrokecolor{currentstroke}%
\pgfsetdash{}{0pt}%
\pgfpathmoveto{\pgfqpoint{3.539434in}{8.473868in}}%
\pgfpathlineto{\pgfqpoint{3.627169in}{8.473868in}}%
\pgfpathlineto{\pgfqpoint{3.627169in}{8.386132in}}%
\pgfpathlineto{\pgfqpoint{3.539434in}{8.386132in}}%
\pgfpathlineto{\pgfqpoint{3.539434in}{8.473868in}}%
\pgfusepath{stroke,fill}%
\end{pgfscope}%
\begin{pgfscope}%
\pgfpathrectangle{\pgfqpoint{0.380943in}{8.035189in}}{\pgfqpoint{4.650000in}{0.614151in}}%
\pgfusepath{clip}%
\pgfsetbuttcap%
\pgfsetroundjoin%
\definecolor{currentfill}{rgb}{0.970012,0.883276,0.699577}%
\pgfsetfillcolor{currentfill}%
\pgfsetlinewidth{0.250937pt}%
\definecolor{currentstroke}{rgb}{1.000000,1.000000,1.000000}%
\pgfsetstrokecolor{currentstroke}%
\pgfsetdash{}{0pt}%
\pgfpathmoveto{\pgfqpoint{3.627169in}{8.473868in}}%
\pgfpathlineto{\pgfqpoint{3.714905in}{8.473868in}}%
\pgfpathlineto{\pgfqpoint{3.714905in}{8.386132in}}%
\pgfpathlineto{\pgfqpoint{3.627169in}{8.386132in}}%
\pgfpathlineto{\pgfqpoint{3.627169in}{8.473868in}}%
\pgfusepath{stroke,fill}%
\end{pgfscope}%
\begin{pgfscope}%
\pgfpathrectangle{\pgfqpoint{0.380943in}{8.035189in}}{\pgfqpoint{4.650000in}{0.614151in}}%
\pgfusepath{clip}%
\pgfsetbuttcap%
\pgfsetroundjoin%
\definecolor{currentfill}{rgb}{0.986251,0.808597,0.643230}%
\pgfsetfillcolor{currentfill}%
\pgfsetlinewidth{0.250937pt}%
\definecolor{currentstroke}{rgb}{1.000000,1.000000,1.000000}%
\pgfsetstrokecolor{currentstroke}%
\pgfsetdash{}{0pt}%
\pgfpathmoveto{\pgfqpoint{3.714905in}{8.473868in}}%
\pgfpathlineto{\pgfqpoint{3.802641in}{8.473868in}}%
\pgfpathlineto{\pgfqpoint{3.802641in}{8.386132in}}%
\pgfpathlineto{\pgfqpoint{3.714905in}{8.386132in}}%
\pgfpathlineto{\pgfqpoint{3.714905in}{8.473868in}}%
\pgfusepath{stroke,fill}%
\end{pgfscope}%
\begin{pgfscope}%
\pgfpathrectangle{\pgfqpoint{0.380943in}{8.035189in}}{\pgfqpoint{4.650000in}{0.614151in}}%
\pgfusepath{clip}%
\pgfsetbuttcap%
\pgfsetroundjoin%
\definecolor{currentfill}{rgb}{0.970012,0.883276,0.699577}%
\pgfsetfillcolor{currentfill}%
\pgfsetlinewidth{0.250937pt}%
\definecolor{currentstroke}{rgb}{1.000000,1.000000,1.000000}%
\pgfsetstrokecolor{currentstroke}%
\pgfsetdash{}{0pt}%
\pgfpathmoveto{\pgfqpoint{3.802641in}{8.473868in}}%
\pgfpathlineto{\pgfqpoint{3.890377in}{8.473868in}}%
\pgfpathlineto{\pgfqpoint{3.890377in}{8.386132in}}%
\pgfpathlineto{\pgfqpoint{3.802641in}{8.386132in}}%
\pgfpathlineto{\pgfqpoint{3.802641in}{8.473868in}}%
\pgfusepath{stroke,fill}%
\end{pgfscope}%
\begin{pgfscope}%
\pgfpathrectangle{\pgfqpoint{0.380943in}{8.035189in}}{\pgfqpoint{4.650000in}{0.614151in}}%
\pgfusepath{clip}%
\pgfsetbuttcap%
\pgfsetroundjoin%
\definecolor{currentfill}{rgb}{0.986251,0.808597,0.643230}%
\pgfsetfillcolor{currentfill}%
\pgfsetlinewidth{0.250937pt}%
\definecolor{currentstroke}{rgb}{1.000000,1.000000,1.000000}%
\pgfsetstrokecolor{currentstroke}%
\pgfsetdash{}{0pt}%
\pgfpathmoveto{\pgfqpoint{3.890377in}{8.473868in}}%
\pgfpathlineto{\pgfqpoint{3.978113in}{8.473868in}}%
\pgfpathlineto{\pgfqpoint{3.978113in}{8.386132in}}%
\pgfpathlineto{\pgfqpoint{3.890377in}{8.386132in}}%
\pgfpathlineto{\pgfqpoint{3.890377in}{8.473868in}}%
\pgfusepath{stroke,fill}%
\end{pgfscope}%
\begin{pgfscope}%
\pgfpathrectangle{\pgfqpoint{0.380943in}{8.035189in}}{\pgfqpoint{4.650000in}{0.614151in}}%
\pgfusepath{clip}%
\pgfsetbuttcap%
\pgfsetroundjoin%
\definecolor{currentfill}{rgb}{0.986251,0.808597,0.643230}%
\pgfsetfillcolor{currentfill}%
\pgfsetlinewidth{0.250937pt}%
\definecolor{currentstroke}{rgb}{1.000000,1.000000,1.000000}%
\pgfsetstrokecolor{currentstroke}%
\pgfsetdash{}{0pt}%
\pgfpathmoveto{\pgfqpoint{3.978113in}{8.473868in}}%
\pgfpathlineto{\pgfqpoint{4.065849in}{8.473868in}}%
\pgfpathlineto{\pgfqpoint{4.065849in}{8.386132in}}%
\pgfpathlineto{\pgfqpoint{3.978113in}{8.386132in}}%
\pgfpathlineto{\pgfqpoint{3.978113in}{8.473868in}}%
\pgfusepath{stroke,fill}%
\end{pgfscope}%
\begin{pgfscope}%
\pgfpathrectangle{\pgfqpoint{0.380943in}{8.035189in}}{\pgfqpoint{4.650000in}{0.614151in}}%
\pgfusepath{clip}%
\pgfsetbuttcap%
\pgfsetroundjoin%
\definecolor{currentfill}{rgb}{0.992326,0.765229,0.614840}%
\pgfsetfillcolor{currentfill}%
\pgfsetlinewidth{0.250937pt}%
\definecolor{currentstroke}{rgb}{1.000000,1.000000,1.000000}%
\pgfsetstrokecolor{currentstroke}%
\pgfsetdash{}{0pt}%
\pgfpathmoveto{\pgfqpoint{4.065849in}{8.473868in}}%
\pgfpathlineto{\pgfqpoint{4.153585in}{8.473868in}}%
\pgfpathlineto{\pgfqpoint{4.153585in}{8.386132in}}%
\pgfpathlineto{\pgfqpoint{4.065849in}{8.386132in}}%
\pgfpathlineto{\pgfqpoint{4.065849in}{8.473868in}}%
\pgfusepath{stroke,fill}%
\end{pgfscope}%
\begin{pgfscope}%
\pgfpathrectangle{\pgfqpoint{0.380943in}{8.035189in}}{\pgfqpoint{4.650000in}{0.614151in}}%
\pgfusepath{clip}%
\pgfsetbuttcap%
\pgfsetroundjoin%
\definecolor{currentfill}{rgb}{0.978131,0.843783,0.675709}%
\pgfsetfillcolor{currentfill}%
\pgfsetlinewidth{0.250937pt}%
\definecolor{currentstroke}{rgb}{1.000000,1.000000,1.000000}%
\pgfsetstrokecolor{currentstroke}%
\pgfsetdash{}{0pt}%
\pgfpathmoveto{\pgfqpoint{4.153585in}{8.473868in}}%
\pgfpathlineto{\pgfqpoint{4.241320in}{8.473868in}}%
\pgfpathlineto{\pgfqpoint{4.241320in}{8.386132in}}%
\pgfpathlineto{\pgfqpoint{4.153585in}{8.386132in}}%
\pgfpathlineto{\pgfqpoint{4.153585in}{8.473868in}}%
\pgfusepath{stroke,fill}%
\end{pgfscope}%
\begin{pgfscope}%
\pgfpathrectangle{\pgfqpoint{0.380943in}{8.035189in}}{\pgfqpoint{4.650000in}{0.614151in}}%
\pgfusepath{clip}%
\pgfsetbuttcap%
\pgfsetroundjoin%
\definecolor{currentfill}{rgb}{0.963768,0.915433,0.717478}%
\pgfsetfillcolor{currentfill}%
\pgfsetlinewidth{0.250937pt}%
\definecolor{currentstroke}{rgb}{1.000000,1.000000,1.000000}%
\pgfsetstrokecolor{currentstroke}%
\pgfsetdash{}{0pt}%
\pgfpathmoveto{\pgfqpoint{4.241320in}{8.473868in}}%
\pgfpathlineto{\pgfqpoint{4.329056in}{8.473868in}}%
\pgfpathlineto{\pgfqpoint{4.329056in}{8.386132in}}%
\pgfpathlineto{\pgfqpoint{4.241320in}{8.386132in}}%
\pgfpathlineto{\pgfqpoint{4.241320in}{8.473868in}}%
\pgfusepath{stroke,fill}%
\end{pgfscope}%
\begin{pgfscope}%
\pgfpathrectangle{\pgfqpoint{0.380943in}{8.035189in}}{\pgfqpoint{4.650000in}{0.614151in}}%
\pgfusepath{clip}%
\pgfsetbuttcap%
\pgfsetroundjoin%
\definecolor{currentfill}{rgb}{0.963768,0.915433,0.717478}%
\pgfsetfillcolor{currentfill}%
\pgfsetlinewidth{0.250937pt}%
\definecolor{currentstroke}{rgb}{1.000000,1.000000,1.000000}%
\pgfsetstrokecolor{currentstroke}%
\pgfsetdash{}{0pt}%
\pgfpathmoveto{\pgfqpoint{4.329056in}{8.473868in}}%
\pgfpathlineto{\pgfqpoint{4.416792in}{8.473868in}}%
\pgfpathlineto{\pgfqpoint{4.416792in}{8.386132in}}%
\pgfpathlineto{\pgfqpoint{4.329056in}{8.386132in}}%
\pgfpathlineto{\pgfqpoint{4.329056in}{8.473868in}}%
\pgfusepath{stroke,fill}%
\end{pgfscope}%
\begin{pgfscope}%
\pgfpathrectangle{\pgfqpoint{0.380943in}{8.035189in}}{\pgfqpoint{4.650000in}{0.614151in}}%
\pgfusepath{clip}%
\pgfsetbuttcap%
\pgfsetroundjoin%
\definecolor{currentfill}{rgb}{0.970012,0.883276,0.699577}%
\pgfsetfillcolor{currentfill}%
\pgfsetlinewidth{0.250937pt}%
\definecolor{currentstroke}{rgb}{1.000000,1.000000,1.000000}%
\pgfsetstrokecolor{currentstroke}%
\pgfsetdash{}{0pt}%
\pgfpathmoveto{\pgfqpoint{4.416792in}{8.473868in}}%
\pgfpathlineto{\pgfqpoint{4.504528in}{8.473868in}}%
\pgfpathlineto{\pgfqpoint{4.504528in}{8.386132in}}%
\pgfpathlineto{\pgfqpoint{4.416792in}{8.386132in}}%
\pgfpathlineto{\pgfqpoint{4.416792in}{8.473868in}}%
\pgfusepath{stroke,fill}%
\end{pgfscope}%
\begin{pgfscope}%
\pgfpathrectangle{\pgfqpoint{0.380943in}{8.035189in}}{\pgfqpoint{4.650000in}{0.614151in}}%
\pgfusepath{clip}%
\pgfsetbuttcap%
\pgfsetroundjoin%
\definecolor{currentfill}{rgb}{0.992326,0.765229,0.614840}%
\pgfsetfillcolor{currentfill}%
\pgfsetlinewidth{0.250937pt}%
\definecolor{currentstroke}{rgb}{1.000000,1.000000,1.000000}%
\pgfsetstrokecolor{currentstroke}%
\pgfsetdash{}{0pt}%
\pgfpathmoveto{\pgfqpoint{4.504528in}{8.473868in}}%
\pgfpathlineto{\pgfqpoint{4.592264in}{8.473868in}}%
\pgfpathlineto{\pgfqpoint{4.592264in}{8.386132in}}%
\pgfpathlineto{\pgfqpoint{4.504528in}{8.386132in}}%
\pgfpathlineto{\pgfqpoint{4.504528in}{8.473868in}}%
\pgfusepath{stroke,fill}%
\end{pgfscope}%
\begin{pgfscope}%
\pgfpathrectangle{\pgfqpoint{0.380943in}{8.035189in}}{\pgfqpoint{4.650000in}{0.614151in}}%
\pgfusepath{clip}%
\pgfsetbuttcap%
\pgfsetroundjoin%
\definecolor{currentfill}{rgb}{0.978131,0.843783,0.675709}%
\pgfsetfillcolor{currentfill}%
\pgfsetlinewidth{0.250937pt}%
\definecolor{currentstroke}{rgb}{1.000000,1.000000,1.000000}%
\pgfsetstrokecolor{currentstroke}%
\pgfsetdash{}{0pt}%
\pgfpathmoveto{\pgfqpoint{4.592264in}{8.473868in}}%
\pgfpathlineto{\pgfqpoint{4.680000in}{8.473868in}}%
\pgfpathlineto{\pgfqpoint{4.680000in}{8.386132in}}%
\pgfpathlineto{\pgfqpoint{4.592264in}{8.386132in}}%
\pgfpathlineto{\pgfqpoint{4.592264in}{8.473868in}}%
\pgfusepath{stroke,fill}%
\end{pgfscope}%
\begin{pgfscope}%
\pgfpathrectangle{\pgfqpoint{0.380943in}{8.035189in}}{\pgfqpoint{4.650000in}{0.614151in}}%
\pgfusepath{clip}%
\pgfsetbuttcap%
\pgfsetroundjoin%
\definecolor{currentfill}{rgb}{0.978131,0.843783,0.675709}%
\pgfsetfillcolor{currentfill}%
\pgfsetlinewidth{0.250937pt}%
\definecolor{currentstroke}{rgb}{1.000000,1.000000,1.000000}%
\pgfsetstrokecolor{currentstroke}%
\pgfsetdash{}{0pt}%
\pgfpathmoveto{\pgfqpoint{4.680000in}{8.473868in}}%
\pgfpathlineto{\pgfqpoint{4.767736in}{8.473868in}}%
\pgfpathlineto{\pgfqpoint{4.767736in}{8.386132in}}%
\pgfpathlineto{\pgfqpoint{4.680000in}{8.386132in}}%
\pgfpathlineto{\pgfqpoint{4.680000in}{8.473868in}}%
\pgfusepath{stroke,fill}%
\end{pgfscope}%
\begin{pgfscope}%
\pgfpathrectangle{\pgfqpoint{0.380943in}{8.035189in}}{\pgfqpoint{4.650000in}{0.614151in}}%
\pgfusepath{clip}%
\pgfsetbuttcap%
\pgfsetroundjoin%
\definecolor{currentfill}{rgb}{0.996909,0.711742,0.584452}%
\pgfsetfillcolor{currentfill}%
\pgfsetlinewidth{0.250937pt}%
\definecolor{currentstroke}{rgb}{1.000000,1.000000,1.000000}%
\pgfsetstrokecolor{currentstroke}%
\pgfsetdash{}{0pt}%
\pgfpathmoveto{\pgfqpoint{4.767736in}{8.473868in}}%
\pgfpathlineto{\pgfqpoint{4.855471in}{8.473868in}}%
\pgfpathlineto{\pgfqpoint{4.855471in}{8.386132in}}%
\pgfpathlineto{\pgfqpoint{4.767736in}{8.386132in}}%
\pgfpathlineto{\pgfqpoint{4.767736in}{8.473868in}}%
\pgfusepath{stroke,fill}%
\end{pgfscope}%
\begin{pgfscope}%
\pgfpathrectangle{\pgfqpoint{0.380943in}{8.035189in}}{\pgfqpoint{4.650000in}{0.614151in}}%
\pgfusepath{clip}%
\pgfsetbuttcap%
\pgfsetroundjoin%
\definecolor{currentfill}{rgb}{0.992326,0.765229,0.614840}%
\pgfsetfillcolor{currentfill}%
\pgfsetlinewidth{0.250937pt}%
\definecolor{currentstroke}{rgb}{1.000000,1.000000,1.000000}%
\pgfsetstrokecolor{currentstroke}%
\pgfsetdash{}{0pt}%
\pgfpathmoveto{\pgfqpoint{4.855471in}{8.473868in}}%
\pgfpathlineto{\pgfqpoint{4.943207in}{8.473868in}}%
\pgfpathlineto{\pgfqpoint{4.943207in}{8.386132in}}%
\pgfpathlineto{\pgfqpoint{4.855471in}{8.386132in}}%
\pgfpathlineto{\pgfqpoint{4.855471in}{8.473868in}}%
\pgfusepath{stroke,fill}%
\end{pgfscope}%
\begin{pgfscope}%
\pgfpathrectangle{\pgfqpoint{0.380943in}{8.035189in}}{\pgfqpoint{4.650000in}{0.614151in}}%
\pgfusepath{clip}%
\pgfsetbuttcap%
\pgfsetroundjoin%
\definecolor{currentfill}{rgb}{0.986251,0.808597,0.643230}%
\pgfsetfillcolor{currentfill}%
\pgfsetlinewidth{0.250937pt}%
\definecolor{currentstroke}{rgb}{1.000000,1.000000,1.000000}%
\pgfsetstrokecolor{currentstroke}%
\pgfsetdash{}{0pt}%
\pgfpathmoveto{\pgfqpoint{4.943207in}{8.473868in}}%
\pgfpathlineto{\pgfqpoint{5.030943in}{8.473868in}}%
\pgfpathlineto{\pgfqpoint{5.030943in}{8.386132in}}%
\pgfpathlineto{\pgfqpoint{4.943207in}{8.386132in}}%
\pgfpathlineto{\pgfqpoint{4.943207in}{8.473868in}}%
\pgfusepath{stroke,fill}%
\end{pgfscope}%
\begin{pgfscope}%
\pgfpathrectangle{\pgfqpoint{0.380943in}{8.035189in}}{\pgfqpoint{4.650000in}{0.614151in}}%
\pgfusepath{clip}%
\pgfsetbuttcap%
\pgfsetroundjoin%
\pgfsetlinewidth{0.250937pt}%
\definecolor{currentstroke}{rgb}{1.000000,1.000000,1.000000}%
\pgfsetstrokecolor{currentstroke}%
\pgfsetdash{}{0pt}%
\pgfpathmoveto{\pgfqpoint{0.380943in}{8.386132in}}%
\pgfpathlineto{\pgfqpoint{0.468679in}{8.386132in}}%
\pgfpathlineto{\pgfqpoint{0.468679in}{8.298396in}}%
\pgfpathlineto{\pgfqpoint{0.380943in}{8.298396in}}%
\pgfpathlineto{\pgfqpoint{0.380943in}{8.386132in}}%
\pgfusepath{stroke}%
\end{pgfscope}%
\begin{pgfscope}%
\pgfpathrectangle{\pgfqpoint{0.380943in}{8.035189in}}{\pgfqpoint{4.650000in}{0.614151in}}%
\pgfusepath{clip}%
\pgfsetbuttcap%
\pgfsetroundjoin%
\definecolor{currentfill}{rgb}{0.978131,0.843783,0.675709}%
\pgfsetfillcolor{currentfill}%
\pgfsetlinewidth{0.250937pt}%
\definecolor{currentstroke}{rgb}{1.000000,1.000000,1.000000}%
\pgfsetstrokecolor{currentstroke}%
\pgfsetdash{}{0pt}%
\pgfpathmoveto{\pgfqpoint{0.468679in}{8.386132in}}%
\pgfpathlineto{\pgfqpoint{0.556415in}{8.386132in}}%
\pgfpathlineto{\pgfqpoint{0.556415in}{8.298396in}}%
\pgfpathlineto{\pgfqpoint{0.468679in}{8.298396in}}%
\pgfpathlineto{\pgfqpoint{0.468679in}{8.386132in}}%
\pgfusepath{stroke,fill}%
\end{pgfscope}%
\begin{pgfscope}%
\pgfpathrectangle{\pgfqpoint{0.380943in}{8.035189in}}{\pgfqpoint{4.650000in}{0.614151in}}%
\pgfusepath{clip}%
\pgfsetbuttcap%
\pgfsetroundjoin%
\definecolor{currentfill}{rgb}{1.000000,0.531903,0.500946}%
\pgfsetfillcolor{currentfill}%
\pgfsetlinewidth{0.250937pt}%
\definecolor{currentstroke}{rgb}{1.000000,1.000000,1.000000}%
\pgfsetstrokecolor{currentstroke}%
\pgfsetdash{}{0pt}%
\pgfpathmoveto{\pgfqpoint{0.556415in}{8.386132in}}%
\pgfpathlineto{\pgfqpoint{0.644151in}{8.386132in}}%
\pgfpathlineto{\pgfqpoint{0.644151in}{8.298396in}}%
\pgfpathlineto{\pgfqpoint{0.556415in}{8.298396in}}%
\pgfpathlineto{\pgfqpoint{0.556415in}{8.386132in}}%
\pgfusepath{stroke,fill}%
\end{pgfscope}%
\begin{pgfscope}%
\pgfpathrectangle{\pgfqpoint{0.380943in}{8.035189in}}{\pgfqpoint{4.650000in}{0.614151in}}%
\pgfusepath{clip}%
\pgfsetbuttcap%
\pgfsetroundjoin%
\definecolor{currentfill}{rgb}{1.000000,0.531903,0.500946}%
\pgfsetfillcolor{currentfill}%
\pgfsetlinewidth{0.250937pt}%
\definecolor{currentstroke}{rgb}{1.000000,1.000000,1.000000}%
\pgfsetstrokecolor{currentstroke}%
\pgfsetdash{}{0pt}%
\pgfpathmoveto{\pgfqpoint{0.644151in}{8.386132in}}%
\pgfpathlineto{\pgfqpoint{0.731886in}{8.386132in}}%
\pgfpathlineto{\pgfqpoint{0.731886in}{8.298396in}}%
\pgfpathlineto{\pgfqpoint{0.644151in}{8.298396in}}%
\pgfpathlineto{\pgfqpoint{0.644151in}{8.386132in}}%
\pgfusepath{stroke,fill}%
\end{pgfscope}%
\begin{pgfscope}%
\pgfpathrectangle{\pgfqpoint{0.380943in}{8.035189in}}{\pgfqpoint{4.650000in}{0.614151in}}%
\pgfusepath{clip}%
\pgfsetbuttcap%
\pgfsetroundjoin%
\definecolor{currentfill}{rgb}{1.000000,0.480477,0.479293}%
\pgfsetfillcolor{currentfill}%
\pgfsetlinewidth{0.250937pt}%
\definecolor{currentstroke}{rgb}{1.000000,1.000000,1.000000}%
\pgfsetstrokecolor{currentstroke}%
\pgfsetdash{}{0pt}%
\pgfpathmoveto{\pgfqpoint{0.731886in}{8.386132in}}%
\pgfpathlineto{\pgfqpoint{0.819622in}{8.386132in}}%
\pgfpathlineto{\pgfqpoint{0.819622in}{8.298396in}}%
\pgfpathlineto{\pgfqpoint{0.731886in}{8.298396in}}%
\pgfpathlineto{\pgfqpoint{0.731886in}{8.386132in}}%
\pgfusepath{stroke,fill}%
\end{pgfscope}%
\begin{pgfscope}%
\pgfpathrectangle{\pgfqpoint{0.380943in}{8.035189in}}{\pgfqpoint{4.650000in}{0.614151in}}%
\pgfusepath{clip}%
\pgfsetbuttcap%
\pgfsetroundjoin%
\definecolor{currentfill}{rgb}{0.970012,0.883276,0.699577}%
\pgfsetfillcolor{currentfill}%
\pgfsetlinewidth{0.250937pt}%
\definecolor{currentstroke}{rgb}{1.000000,1.000000,1.000000}%
\pgfsetstrokecolor{currentstroke}%
\pgfsetdash{}{0pt}%
\pgfpathmoveto{\pgfqpoint{0.819622in}{8.386132in}}%
\pgfpathlineto{\pgfqpoint{0.907358in}{8.386132in}}%
\pgfpathlineto{\pgfqpoint{0.907358in}{8.298396in}}%
\pgfpathlineto{\pgfqpoint{0.819622in}{8.298396in}}%
\pgfpathlineto{\pgfqpoint{0.819622in}{8.386132in}}%
\pgfusepath{stroke,fill}%
\end{pgfscope}%
\begin{pgfscope}%
\pgfpathrectangle{\pgfqpoint{0.380943in}{8.035189in}}{\pgfqpoint{4.650000in}{0.614151in}}%
\pgfusepath{clip}%
\pgfsetbuttcap%
\pgfsetroundjoin%
\definecolor{currentfill}{rgb}{0.986251,0.808597,0.643230}%
\pgfsetfillcolor{currentfill}%
\pgfsetlinewidth{0.250937pt}%
\definecolor{currentstroke}{rgb}{1.000000,1.000000,1.000000}%
\pgfsetstrokecolor{currentstroke}%
\pgfsetdash{}{0pt}%
\pgfpathmoveto{\pgfqpoint{0.907358in}{8.386132in}}%
\pgfpathlineto{\pgfqpoint{0.995094in}{8.386132in}}%
\pgfpathlineto{\pgfqpoint{0.995094in}{8.298396in}}%
\pgfpathlineto{\pgfqpoint{0.907358in}{8.298396in}}%
\pgfpathlineto{\pgfqpoint{0.907358in}{8.386132in}}%
\pgfusepath{stroke,fill}%
\end{pgfscope}%
\begin{pgfscope}%
\pgfpathrectangle{\pgfqpoint{0.380943in}{8.035189in}}{\pgfqpoint{4.650000in}{0.614151in}}%
\pgfusepath{clip}%
\pgfsetbuttcap%
\pgfsetroundjoin%
\definecolor{currentfill}{rgb}{0.986251,0.808597,0.643230}%
\pgfsetfillcolor{currentfill}%
\pgfsetlinewidth{0.250937pt}%
\definecolor{currentstroke}{rgb}{1.000000,1.000000,1.000000}%
\pgfsetstrokecolor{currentstroke}%
\pgfsetdash{}{0pt}%
\pgfpathmoveto{\pgfqpoint{0.995094in}{8.386132in}}%
\pgfpathlineto{\pgfqpoint{1.082830in}{8.386132in}}%
\pgfpathlineto{\pgfqpoint{1.082830in}{8.298396in}}%
\pgfpathlineto{\pgfqpoint{0.995094in}{8.298396in}}%
\pgfpathlineto{\pgfqpoint{0.995094in}{8.386132in}}%
\pgfusepath{stroke,fill}%
\end{pgfscope}%
\begin{pgfscope}%
\pgfpathrectangle{\pgfqpoint{0.380943in}{8.035189in}}{\pgfqpoint{4.650000in}{0.614151in}}%
\pgfusepath{clip}%
\pgfsetbuttcap%
\pgfsetroundjoin%
\definecolor{currentfill}{rgb}{1.000000,0.480477,0.479293}%
\pgfsetfillcolor{currentfill}%
\pgfsetlinewidth{0.250937pt}%
\definecolor{currentstroke}{rgb}{1.000000,1.000000,1.000000}%
\pgfsetstrokecolor{currentstroke}%
\pgfsetdash{}{0pt}%
\pgfpathmoveto{\pgfqpoint{1.082830in}{8.386132in}}%
\pgfpathlineto{\pgfqpoint{1.170566in}{8.386132in}}%
\pgfpathlineto{\pgfqpoint{1.170566in}{8.298396in}}%
\pgfpathlineto{\pgfqpoint{1.082830in}{8.298396in}}%
\pgfpathlineto{\pgfqpoint{1.082830in}{8.386132in}}%
\pgfusepath{stroke,fill}%
\end{pgfscope}%
\begin{pgfscope}%
\pgfpathrectangle{\pgfqpoint{0.380943in}{8.035189in}}{\pgfqpoint{4.650000in}{0.614151in}}%
\pgfusepath{clip}%
\pgfsetbuttcap%
\pgfsetroundjoin%
\definecolor{currentfill}{rgb}{1.000000,0.531903,0.500946}%
\pgfsetfillcolor{currentfill}%
\pgfsetlinewidth{0.250937pt}%
\definecolor{currentstroke}{rgb}{1.000000,1.000000,1.000000}%
\pgfsetstrokecolor{currentstroke}%
\pgfsetdash{}{0pt}%
\pgfpathmoveto{\pgfqpoint{1.170566in}{8.386132in}}%
\pgfpathlineto{\pgfqpoint{1.258302in}{8.386132in}}%
\pgfpathlineto{\pgfqpoint{1.258302in}{8.298396in}}%
\pgfpathlineto{\pgfqpoint{1.170566in}{8.298396in}}%
\pgfpathlineto{\pgfqpoint{1.170566in}{8.386132in}}%
\pgfusepath{stroke,fill}%
\end{pgfscope}%
\begin{pgfscope}%
\pgfpathrectangle{\pgfqpoint{0.380943in}{8.035189in}}{\pgfqpoint{4.650000in}{0.614151in}}%
\pgfusepath{clip}%
\pgfsetbuttcap%
\pgfsetroundjoin%
\definecolor{currentfill}{rgb}{0.986251,0.808597,0.643230}%
\pgfsetfillcolor{currentfill}%
\pgfsetlinewidth{0.250937pt}%
\definecolor{currentstroke}{rgb}{1.000000,1.000000,1.000000}%
\pgfsetstrokecolor{currentstroke}%
\pgfsetdash{}{0pt}%
\pgfpathmoveto{\pgfqpoint{1.258302in}{8.386132in}}%
\pgfpathlineto{\pgfqpoint{1.346037in}{8.386132in}}%
\pgfpathlineto{\pgfqpoint{1.346037in}{8.298396in}}%
\pgfpathlineto{\pgfqpoint{1.258302in}{8.298396in}}%
\pgfpathlineto{\pgfqpoint{1.258302in}{8.386132in}}%
\pgfusepath{stroke,fill}%
\end{pgfscope}%
\begin{pgfscope}%
\pgfpathrectangle{\pgfqpoint{0.380943in}{8.035189in}}{\pgfqpoint{4.650000in}{0.614151in}}%
\pgfusepath{clip}%
\pgfsetbuttcap%
\pgfsetroundjoin%
\definecolor{currentfill}{rgb}{0.992326,0.765229,0.614840}%
\pgfsetfillcolor{currentfill}%
\pgfsetlinewidth{0.250937pt}%
\definecolor{currentstroke}{rgb}{1.000000,1.000000,1.000000}%
\pgfsetstrokecolor{currentstroke}%
\pgfsetdash{}{0pt}%
\pgfpathmoveto{\pgfqpoint{1.346037in}{8.386132in}}%
\pgfpathlineto{\pgfqpoint{1.433773in}{8.386132in}}%
\pgfpathlineto{\pgfqpoint{1.433773in}{8.298396in}}%
\pgfpathlineto{\pgfqpoint{1.346037in}{8.298396in}}%
\pgfpathlineto{\pgfqpoint{1.346037in}{8.386132in}}%
\pgfusepath{stroke,fill}%
\end{pgfscope}%
\begin{pgfscope}%
\pgfpathrectangle{\pgfqpoint{0.380943in}{8.035189in}}{\pgfqpoint{4.650000in}{0.614151in}}%
\pgfusepath{clip}%
\pgfsetbuttcap%
\pgfsetroundjoin%
\definecolor{currentfill}{rgb}{0.961061,0.931672,0.728304}%
\pgfsetfillcolor{currentfill}%
\pgfsetlinewidth{0.250937pt}%
\definecolor{currentstroke}{rgb}{1.000000,1.000000,1.000000}%
\pgfsetstrokecolor{currentstroke}%
\pgfsetdash{}{0pt}%
\pgfpathmoveto{\pgfqpoint{1.433773in}{8.386132in}}%
\pgfpathlineto{\pgfqpoint{1.521509in}{8.386132in}}%
\pgfpathlineto{\pgfqpoint{1.521509in}{8.298396in}}%
\pgfpathlineto{\pgfqpoint{1.433773in}{8.298396in}}%
\pgfpathlineto{\pgfqpoint{1.433773in}{8.386132in}}%
\pgfusepath{stroke,fill}%
\end{pgfscope}%
\begin{pgfscope}%
\pgfpathrectangle{\pgfqpoint{0.380943in}{8.035189in}}{\pgfqpoint{4.650000in}{0.614151in}}%
\pgfusepath{clip}%
\pgfsetbuttcap%
\pgfsetroundjoin%
\definecolor{currentfill}{rgb}{0.992326,0.765229,0.614840}%
\pgfsetfillcolor{currentfill}%
\pgfsetlinewidth{0.250937pt}%
\definecolor{currentstroke}{rgb}{1.000000,1.000000,1.000000}%
\pgfsetstrokecolor{currentstroke}%
\pgfsetdash{}{0pt}%
\pgfpathmoveto{\pgfqpoint{1.521509in}{8.386132in}}%
\pgfpathlineto{\pgfqpoint{1.609245in}{8.386132in}}%
\pgfpathlineto{\pgfqpoint{1.609245in}{8.298396in}}%
\pgfpathlineto{\pgfqpoint{1.521509in}{8.298396in}}%
\pgfpathlineto{\pgfqpoint{1.521509in}{8.386132in}}%
\pgfusepath{stroke,fill}%
\end{pgfscope}%
\begin{pgfscope}%
\pgfpathrectangle{\pgfqpoint{0.380943in}{8.035189in}}{\pgfqpoint{4.650000in}{0.614151in}}%
\pgfusepath{clip}%
\pgfsetbuttcap%
\pgfsetroundjoin%
\definecolor{currentfill}{rgb}{0.986251,0.808597,0.643230}%
\pgfsetfillcolor{currentfill}%
\pgfsetlinewidth{0.250937pt}%
\definecolor{currentstroke}{rgb}{1.000000,1.000000,1.000000}%
\pgfsetstrokecolor{currentstroke}%
\pgfsetdash{}{0pt}%
\pgfpathmoveto{\pgfqpoint{1.609245in}{8.386132in}}%
\pgfpathlineto{\pgfqpoint{1.696981in}{8.386132in}}%
\pgfpathlineto{\pgfqpoint{1.696981in}{8.298396in}}%
\pgfpathlineto{\pgfqpoint{1.609245in}{8.298396in}}%
\pgfpathlineto{\pgfqpoint{1.609245in}{8.386132in}}%
\pgfusepath{stroke,fill}%
\end{pgfscope}%
\begin{pgfscope}%
\pgfpathrectangle{\pgfqpoint{0.380943in}{8.035189in}}{\pgfqpoint{4.650000in}{0.614151in}}%
\pgfusepath{clip}%
\pgfsetbuttcap%
\pgfsetroundjoin%
\definecolor{currentfill}{rgb}{0.992326,0.765229,0.614840}%
\pgfsetfillcolor{currentfill}%
\pgfsetlinewidth{0.250937pt}%
\definecolor{currentstroke}{rgb}{1.000000,1.000000,1.000000}%
\pgfsetstrokecolor{currentstroke}%
\pgfsetdash{}{0pt}%
\pgfpathmoveto{\pgfqpoint{1.696981in}{8.386132in}}%
\pgfpathlineto{\pgfqpoint{1.784717in}{8.386132in}}%
\pgfpathlineto{\pgfqpoint{1.784717in}{8.298396in}}%
\pgfpathlineto{\pgfqpoint{1.696981in}{8.298396in}}%
\pgfpathlineto{\pgfqpoint{1.696981in}{8.386132in}}%
\pgfusepath{stroke,fill}%
\end{pgfscope}%
\begin{pgfscope}%
\pgfpathrectangle{\pgfqpoint{0.380943in}{8.035189in}}{\pgfqpoint{4.650000in}{0.614151in}}%
\pgfusepath{clip}%
\pgfsetbuttcap%
\pgfsetroundjoin%
\definecolor{currentfill}{rgb}{0.996909,0.711742,0.584452}%
\pgfsetfillcolor{currentfill}%
\pgfsetlinewidth{0.250937pt}%
\definecolor{currentstroke}{rgb}{1.000000,1.000000,1.000000}%
\pgfsetstrokecolor{currentstroke}%
\pgfsetdash{}{0pt}%
\pgfpathmoveto{\pgfqpoint{1.784717in}{8.386132in}}%
\pgfpathlineto{\pgfqpoint{1.872452in}{8.386132in}}%
\pgfpathlineto{\pgfqpoint{1.872452in}{8.298396in}}%
\pgfpathlineto{\pgfqpoint{1.784717in}{8.298396in}}%
\pgfpathlineto{\pgfqpoint{1.784717in}{8.386132in}}%
\pgfusepath{stroke,fill}%
\end{pgfscope}%
\begin{pgfscope}%
\pgfpathrectangle{\pgfqpoint{0.380943in}{8.035189in}}{\pgfqpoint{4.650000in}{0.614151in}}%
\pgfusepath{clip}%
\pgfsetbuttcap%
\pgfsetroundjoin%
\definecolor{currentfill}{rgb}{0.970012,0.883276,0.699577}%
\pgfsetfillcolor{currentfill}%
\pgfsetlinewidth{0.250937pt}%
\definecolor{currentstroke}{rgb}{1.000000,1.000000,1.000000}%
\pgfsetstrokecolor{currentstroke}%
\pgfsetdash{}{0pt}%
\pgfpathmoveto{\pgfqpoint{1.872452in}{8.386132in}}%
\pgfpathlineto{\pgfqpoint{1.960188in}{8.386132in}}%
\pgfpathlineto{\pgfqpoint{1.960188in}{8.298396in}}%
\pgfpathlineto{\pgfqpoint{1.872452in}{8.298396in}}%
\pgfpathlineto{\pgfqpoint{1.872452in}{8.386132in}}%
\pgfusepath{stroke,fill}%
\end{pgfscope}%
\begin{pgfscope}%
\pgfpathrectangle{\pgfqpoint{0.380943in}{8.035189in}}{\pgfqpoint{4.650000in}{0.614151in}}%
\pgfusepath{clip}%
\pgfsetbuttcap%
\pgfsetroundjoin%
\definecolor{currentfill}{rgb}{0.985083,0.974641,0.792587}%
\pgfsetfillcolor{currentfill}%
\pgfsetlinewidth{0.250937pt}%
\definecolor{currentstroke}{rgb}{1.000000,1.000000,1.000000}%
\pgfsetstrokecolor{currentstroke}%
\pgfsetdash{}{0pt}%
\pgfpathmoveto{\pgfqpoint{1.960188in}{8.386132in}}%
\pgfpathlineto{\pgfqpoint{2.047924in}{8.386132in}}%
\pgfpathlineto{\pgfqpoint{2.047924in}{8.298396in}}%
\pgfpathlineto{\pgfqpoint{1.960188in}{8.298396in}}%
\pgfpathlineto{\pgfqpoint{1.960188in}{8.386132in}}%
\pgfusepath{stroke,fill}%
\end{pgfscope}%
\begin{pgfscope}%
\pgfpathrectangle{\pgfqpoint{0.380943in}{8.035189in}}{\pgfqpoint{4.650000in}{0.614151in}}%
\pgfusepath{clip}%
\pgfsetbuttcap%
\pgfsetroundjoin%
\definecolor{currentfill}{rgb}{0.986251,0.808597,0.643230}%
\pgfsetfillcolor{currentfill}%
\pgfsetlinewidth{0.250937pt}%
\definecolor{currentstroke}{rgb}{1.000000,1.000000,1.000000}%
\pgfsetstrokecolor{currentstroke}%
\pgfsetdash{}{0pt}%
\pgfpathmoveto{\pgfqpoint{2.047924in}{8.386132in}}%
\pgfpathlineto{\pgfqpoint{2.135660in}{8.386132in}}%
\pgfpathlineto{\pgfqpoint{2.135660in}{8.298396in}}%
\pgfpathlineto{\pgfqpoint{2.047924in}{8.298396in}}%
\pgfpathlineto{\pgfqpoint{2.047924in}{8.386132in}}%
\pgfusepath{stroke,fill}%
\end{pgfscope}%
\begin{pgfscope}%
\pgfpathrectangle{\pgfqpoint{0.380943in}{8.035189in}}{\pgfqpoint{4.650000in}{0.614151in}}%
\pgfusepath{clip}%
\pgfsetbuttcap%
\pgfsetroundjoin%
\definecolor{currentfill}{rgb}{0.978131,0.843783,0.675709}%
\pgfsetfillcolor{currentfill}%
\pgfsetlinewidth{0.250937pt}%
\definecolor{currentstroke}{rgb}{1.000000,1.000000,1.000000}%
\pgfsetstrokecolor{currentstroke}%
\pgfsetdash{}{0pt}%
\pgfpathmoveto{\pgfqpoint{2.135660in}{8.386132in}}%
\pgfpathlineto{\pgfqpoint{2.223396in}{8.386132in}}%
\pgfpathlineto{\pgfqpoint{2.223396in}{8.298396in}}%
\pgfpathlineto{\pgfqpoint{2.135660in}{8.298396in}}%
\pgfpathlineto{\pgfqpoint{2.135660in}{8.386132in}}%
\pgfusepath{stroke,fill}%
\end{pgfscope}%
\begin{pgfscope}%
\pgfpathrectangle{\pgfqpoint{0.380943in}{8.035189in}}{\pgfqpoint{4.650000in}{0.614151in}}%
\pgfusepath{clip}%
\pgfsetbuttcap%
\pgfsetroundjoin%
\definecolor{currentfill}{rgb}{1.000000,1.000000,0.861745}%
\pgfsetfillcolor{currentfill}%
\pgfsetlinewidth{0.250937pt}%
\definecolor{currentstroke}{rgb}{1.000000,1.000000,1.000000}%
\pgfsetstrokecolor{currentstroke}%
\pgfsetdash{}{0pt}%
\pgfpathmoveto{\pgfqpoint{2.223396in}{8.386132in}}%
\pgfpathlineto{\pgfqpoint{2.311132in}{8.386132in}}%
\pgfpathlineto{\pgfqpoint{2.311132in}{8.298396in}}%
\pgfpathlineto{\pgfqpoint{2.223396in}{8.298396in}}%
\pgfpathlineto{\pgfqpoint{2.223396in}{8.386132in}}%
\pgfusepath{stroke,fill}%
\end{pgfscope}%
\begin{pgfscope}%
\pgfpathrectangle{\pgfqpoint{0.380943in}{8.035189in}}{\pgfqpoint{4.650000in}{0.614151in}}%
\pgfusepath{clip}%
\pgfsetbuttcap%
\pgfsetroundjoin%
\definecolor{currentfill}{rgb}{0.992326,0.765229,0.614840}%
\pgfsetfillcolor{currentfill}%
\pgfsetlinewidth{0.250937pt}%
\definecolor{currentstroke}{rgb}{1.000000,1.000000,1.000000}%
\pgfsetstrokecolor{currentstroke}%
\pgfsetdash{}{0pt}%
\pgfpathmoveto{\pgfqpoint{2.311132in}{8.386132in}}%
\pgfpathlineto{\pgfqpoint{2.398868in}{8.386132in}}%
\pgfpathlineto{\pgfqpoint{2.398868in}{8.298396in}}%
\pgfpathlineto{\pgfqpoint{2.311132in}{8.298396in}}%
\pgfpathlineto{\pgfqpoint{2.311132in}{8.386132in}}%
\pgfusepath{stroke,fill}%
\end{pgfscope}%
\begin{pgfscope}%
\pgfpathrectangle{\pgfqpoint{0.380943in}{8.035189in}}{\pgfqpoint{4.650000in}{0.614151in}}%
\pgfusepath{clip}%
\pgfsetbuttcap%
\pgfsetroundjoin%
\definecolor{currentfill}{rgb}{0.978131,0.843783,0.675709}%
\pgfsetfillcolor{currentfill}%
\pgfsetlinewidth{0.250937pt}%
\definecolor{currentstroke}{rgb}{1.000000,1.000000,1.000000}%
\pgfsetstrokecolor{currentstroke}%
\pgfsetdash{}{0pt}%
\pgfpathmoveto{\pgfqpoint{2.398868in}{8.386132in}}%
\pgfpathlineto{\pgfqpoint{2.486603in}{8.386132in}}%
\pgfpathlineto{\pgfqpoint{2.486603in}{8.298396in}}%
\pgfpathlineto{\pgfqpoint{2.398868in}{8.298396in}}%
\pgfpathlineto{\pgfqpoint{2.398868in}{8.386132in}}%
\pgfusepath{stroke,fill}%
\end{pgfscope}%
\begin{pgfscope}%
\pgfpathrectangle{\pgfqpoint{0.380943in}{8.035189in}}{\pgfqpoint{4.650000in}{0.614151in}}%
\pgfusepath{clip}%
\pgfsetbuttcap%
\pgfsetroundjoin%
\definecolor{currentfill}{rgb}{0.970012,0.883276,0.699577}%
\pgfsetfillcolor{currentfill}%
\pgfsetlinewidth{0.250937pt}%
\definecolor{currentstroke}{rgb}{1.000000,1.000000,1.000000}%
\pgfsetstrokecolor{currentstroke}%
\pgfsetdash{}{0pt}%
\pgfpathmoveto{\pgfqpoint{2.486603in}{8.386132in}}%
\pgfpathlineto{\pgfqpoint{2.574339in}{8.386132in}}%
\pgfpathlineto{\pgfqpoint{2.574339in}{8.298396in}}%
\pgfpathlineto{\pgfqpoint{2.486603in}{8.298396in}}%
\pgfpathlineto{\pgfqpoint{2.486603in}{8.386132in}}%
\pgfusepath{stroke,fill}%
\end{pgfscope}%
\begin{pgfscope}%
\pgfpathrectangle{\pgfqpoint{0.380943in}{8.035189in}}{\pgfqpoint{4.650000in}{0.614151in}}%
\pgfusepath{clip}%
\pgfsetbuttcap%
\pgfsetroundjoin%
\definecolor{currentfill}{rgb}{0.970012,0.883276,0.699577}%
\pgfsetfillcolor{currentfill}%
\pgfsetlinewidth{0.250937pt}%
\definecolor{currentstroke}{rgb}{1.000000,1.000000,1.000000}%
\pgfsetstrokecolor{currentstroke}%
\pgfsetdash{}{0pt}%
\pgfpathmoveto{\pgfqpoint{2.574339in}{8.386132in}}%
\pgfpathlineto{\pgfqpoint{2.662075in}{8.386132in}}%
\pgfpathlineto{\pgfqpoint{2.662075in}{8.298396in}}%
\pgfpathlineto{\pgfqpoint{2.574339in}{8.298396in}}%
\pgfpathlineto{\pgfqpoint{2.574339in}{8.386132in}}%
\pgfusepath{stroke,fill}%
\end{pgfscope}%
\begin{pgfscope}%
\pgfpathrectangle{\pgfqpoint{0.380943in}{8.035189in}}{\pgfqpoint{4.650000in}{0.614151in}}%
\pgfusepath{clip}%
\pgfsetbuttcap%
\pgfsetroundjoin%
\definecolor{currentfill}{rgb}{0.961061,0.931672,0.728304}%
\pgfsetfillcolor{currentfill}%
\pgfsetlinewidth{0.250937pt}%
\definecolor{currentstroke}{rgb}{1.000000,1.000000,1.000000}%
\pgfsetstrokecolor{currentstroke}%
\pgfsetdash{}{0pt}%
\pgfpathmoveto{\pgfqpoint{2.662075in}{8.386132in}}%
\pgfpathlineto{\pgfqpoint{2.749811in}{8.386132in}}%
\pgfpathlineto{\pgfqpoint{2.749811in}{8.298396in}}%
\pgfpathlineto{\pgfqpoint{2.662075in}{8.298396in}}%
\pgfpathlineto{\pgfqpoint{2.662075in}{8.386132in}}%
\pgfusepath{stroke,fill}%
\end{pgfscope}%
\begin{pgfscope}%
\pgfpathrectangle{\pgfqpoint{0.380943in}{8.035189in}}{\pgfqpoint{4.650000in}{0.614151in}}%
\pgfusepath{clip}%
\pgfsetbuttcap%
\pgfsetroundjoin%
\definecolor{currentfill}{rgb}{0.985083,0.974641,0.792587}%
\pgfsetfillcolor{currentfill}%
\pgfsetlinewidth{0.250937pt}%
\definecolor{currentstroke}{rgb}{1.000000,1.000000,1.000000}%
\pgfsetstrokecolor{currentstroke}%
\pgfsetdash{}{0pt}%
\pgfpathmoveto{\pgfqpoint{2.749811in}{8.386132in}}%
\pgfpathlineto{\pgfqpoint{2.837547in}{8.386132in}}%
\pgfpathlineto{\pgfqpoint{2.837547in}{8.298396in}}%
\pgfpathlineto{\pgfqpoint{2.749811in}{8.298396in}}%
\pgfpathlineto{\pgfqpoint{2.749811in}{8.386132in}}%
\pgfusepath{stroke,fill}%
\end{pgfscope}%
\begin{pgfscope}%
\pgfpathrectangle{\pgfqpoint{0.380943in}{8.035189in}}{\pgfqpoint{4.650000in}{0.614151in}}%
\pgfusepath{clip}%
\pgfsetbuttcap%
\pgfsetroundjoin%
\definecolor{currentfill}{rgb}{0.978131,0.843783,0.675709}%
\pgfsetfillcolor{currentfill}%
\pgfsetlinewidth{0.250937pt}%
\definecolor{currentstroke}{rgb}{1.000000,1.000000,1.000000}%
\pgfsetstrokecolor{currentstroke}%
\pgfsetdash{}{0pt}%
\pgfpathmoveto{\pgfqpoint{2.837547in}{8.386132in}}%
\pgfpathlineto{\pgfqpoint{2.925283in}{8.386132in}}%
\pgfpathlineto{\pgfqpoint{2.925283in}{8.298396in}}%
\pgfpathlineto{\pgfqpoint{2.837547in}{8.298396in}}%
\pgfpathlineto{\pgfqpoint{2.837547in}{8.386132in}}%
\pgfusepath{stroke,fill}%
\end{pgfscope}%
\begin{pgfscope}%
\pgfpathrectangle{\pgfqpoint{0.380943in}{8.035189in}}{\pgfqpoint{4.650000in}{0.614151in}}%
\pgfusepath{clip}%
\pgfsetbuttcap%
\pgfsetroundjoin%
\definecolor{currentfill}{rgb}{0.978131,0.843783,0.675709}%
\pgfsetfillcolor{currentfill}%
\pgfsetlinewidth{0.250937pt}%
\definecolor{currentstroke}{rgb}{1.000000,1.000000,1.000000}%
\pgfsetstrokecolor{currentstroke}%
\pgfsetdash{}{0pt}%
\pgfpathmoveto{\pgfqpoint{2.925283in}{8.386132in}}%
\pgfpathlineto{\pgfqpoint{3.013019in}{8.386132in}}%
\pgfpathlineto{\pgfqpoint{3.013019in}{8.298396in}}%
\pgfpathlineto{\pgfqpoint{2.925283in}{8.298396in}}%
\pgfpathlineto{\pgfqpoint{2.925283in}{8.386132in}}%
\pgfusepath{stroke,fill}%
\end{pgfscope}%
\begin{pgfscope}%
\pgfpathrectangle{\pgfqpoint{0.380943in}{8.035189in}}{\pgfqpoint{4.650000in}{0.614151in}}%
\pgfusepath{clip}%
\pgfsetbuttcap%
\pgfsetroundjoin%
\definecolor{currentfill}{rgb}{0.986251,0.808597,0.643230}%
\pgfsetfillcolor{currentfill}%
\pgfsetlinewidth{0.250937pt}%
\definecolor{currentstroke}{rgb}{1.000000,1.000000,1.000000}%
\pgfsetstrokecolor{currentstroke}%
\pgfsetdash{}{0pt}%
\pgfpathmoveto{\pgfqpoint{3.013019in}{8.386132in}}%
\pgfpathlineto{\pgfqpoint{3.100754in}{8.386132in}}%
\pgfpathlineto{\pgfqpoint{3.100754in}{8.298396in}}%
\pgfpathlineto{\pgfqpoint{3.013019in}{8.298396in}}%
\pgfpathlineto{\pgfqpoint{3.013019in}{8.386132in}}%
\pgfusepath{stroke,fill}%
\end{pgfscope}%
\begin{pgfscope}%
\pgfpathrectangle{\pgfqpoint{0.380943in}{8.035189in}}{\pgfqpoint{4.650000in}{0.614151in}}%
\pgfusepath{clip}%
\pgfsetbuttcap%
\pgfsetroundjoin%
\definecolor{currentfill}{rgb}{0.961061,0.931672,0.728304}%
\pgfsetfillcolor{currentfill}%
\pgfsetlinewidth{0.250937pt}%
\definecolor{currentstroke}{rgb}{1.000000,1.000000,1.000000}%
\pgfsetstrokecolor{currentstroke}%
\pgfsetdash{}{0pt}%
\pgfpathmoveto{\pgfqpoint{3.100754in}{8.386132in}}%
\pgfpathlineto{\pgfqpoint{3.188490in}{8.386132in}}%
\pgfpathlineto{\pgfqpoint{3.188490in}{8.298396in}}%
\pgfpathlineto{\pgfqpoint{3.100754in}{8.298396in}}%
\pgfpathlineto{\pgfqpoint{3.100754in}{8.386132in}}%
\pgfusepath{stroke,fill}%
\end{pgfscope}%
\begin{pgfscope}%
\pgfpathrectangle{\pgfqpoint{0.380943in}{8.035189in}}{\pgfqpoint{4.650000in}{0.614151in}}%
\pgfusepath{clip}%
\pgfsetbuttcap%
\pgfsetroundjoin%
\definecolor{currentfill}{rgb}{0.963768,0.915433,0.717478}%
\pgfsetfillcolor{currentfill}%
\pgfsetlinewidth{0.250937pt}%
\definecolor{currentstroke}{rgb}{1.000000,1.000000,1.000000}%
\pgfsetstrokecolor{currentstroke}%
\pgfsetdash{}{0pt}%
\pgfpathmoveto{\pgfqpoint{3.188490in}{8.386132in}}%
\pgfpathlineto{\pgfqpoint{3.276226in}{8.386132in}}%
\pgfpathlineto{\pgfqpoint{3.276226in}{8.298396in}}%
\pgfpathlineto{\pgfqpoint{3.188490in}{8.298396in}}%
\pgfpathlineto{\pgfqpoint{3.188490in}{8.386132in}}%
\pgfusepath{stroke,fill}%
\end{pgfscope}%
\begin{pgfscope}%
\pgfpathrectangle{\pgfqpoint{0.380943in}{8.035189in}}{\pgfqpoint{4.650000in}{0.614151in}}%
\pgfusepath{clip}%
\pgfsetbuttcap%
\pgfsetroundjoin%
\definecolor{currentfill}{rgb}{0.986251,0.808597,0.643230}%
\pgfsetfillcolor{currentfill}%
\pgfsetlinewidth{0.250937pt}%
\definecolor{currentstroke}{rgb}{1.000000,1.000000,1.000000}%
\pgfsetstrokecolor{currentstroke}%
\pgfsetdash{}{0pt}%
\pgfpathmoveto{\pgfqpoint{3.276226in}{8.386132in}}%
\pgfpathlineto{\pgfqpoint{3.363962in}{8.386132in}}%
\pgfpathlineto{\pgfqpoint{3.363962in}{8.298396in}}%
\pgfpathlineto{\pgfqpoint{3.276226in}{8.298396in}}%
\pgfpathlineto{\pgfqpoint{3.276226in}{8.386132in}}%
\pgfusepath{stroke,fill}%
\end{pgfscope}%
\begin{pgfscope}%
\pgfpathrectangle{\pgfqpoint{0.380943in}{8.035189in}}{\pgfqpoint{4.650000in}{0.614151in}}%
\pgfusepath{clip}%
\pgfsetbuttcap%
\pgfsetroundjoin%
\definecolor{currentfill}{rgb}{0.999616,0.641369,0.546559}%
\pgfsetfillcolor{currentfill}%
\pgfsetlinewidth{0.250937pt}%
\definecolor{currentstroke}{rgb}{1.000000,1.000000,1.000000}%
\pgfsetstrokecolor{currentstroke}%
\pgfsetdash{}{0pt}%
\pgfpathmoveto{\pgfqpoint{3.363962in}{8.386132in}}%
\pgfpathlineto{\pgfqpoint{3.451698in}{8.386132in}}%
\pgfpathlineto{\pgfqpoint{3.451698in}{8.298396in}}%
\pgfpathlineto{\pgfqpoint{3.363962in}{8.298396in}}%
\pgfpathlineto{\pgfqpoint{3.363962in}{8.386132in}}%
\pgfusepath{stroke,fill}%
\end{pgfscope}%
\begin{pgfscope}%
\pgfpathrectangle{\pgfqpoint{0.380943in}{8.035189in}}{\pgfqpoint{4.650000in}{0.614151in}}%
\pgfusepath{clip}%
\pgfsetbuttcap%
\pgfsetroundjoin%
\definecolor{currentfill}{rgb}{0.978131,0.843783,0.675709}%
\pgfsetfillcolor{currentfill}%
\pgfsetlinewidth{0.250937pt}%
\definecolor{currentstroke}{rgb}{1.000000,1.000000,1.000000}%
\pgfsetstrokecolor{currentstroke}%
\pgfsetdash{}{0pt}%
\pgfpathmoveto{\pgfqpoint{3.451698in}{8.386132in}}%
\pgfpathlineto{\pgfqpoint{3.539434in}{8.386132in}}%
\pgfpathlineto{\pgfqpoint{3.539434in}{8.298396in}}%
\pgfpathlineto{\pgfqpoint{3.451698in}{8.298396in}}%
\pgfpathlineto{\pgfqpoint{3.451698in}{8.386132in}}%
\pgfusepath{stroke,fill}%
\end{pgfscope}%
\begin{pgfscope}%
\pgfpathrectangle{\pgfqpoint{0.380943in}{8.035189in}}{\pgfqpoint{4.650000in}{0.614151in}}%
\pgfusepath{clip}%
\pgfsetbuttcap%
\pgfsetroundjoin%
\definecolor{currentfill}{rgb}{0.970012,0.883276,0.699577}%
\pgfsetfillcolor{currentfill}%
\pgfsetlinewidth{0.250937pt}%
\definecolor{currentstroke}{rgb}{1.000000,1.000000,1.000000}%
\pgfsetstrokecolor{currentstroke}%
\pgfsetdash{}{0pt}%
\pgfpathmoveto{\pgfqpoint{3.539434in}{8.386132in}}%
\pgfpathlineto{\pgfqpoint{3.627169in}{8.386132in}}%
\pgfpathlineto{\pgfqpoint{3.627169in}{8.298396in}}%
\pgfpathlineto{\pgfqpoint{3.539434in}{8.298396in}}%
\pgfpathlineto{\pgfqpoint{3.539434in}{8.386132in}}%
\pgfusepath{stroke,fill}%
\end{pgfscope}%
\begin{pgfscope}%
\pgfpathrectangle{\pgfqpoint{0.380943in}{8.035189in}}{\pgfqpoint{4.650000in}{0.614151in}}%
\pgfusepath{clip}%
\pgfsetbuttcap%
\pgfsetroundjoin%
\definecolor{currentfill}{rgb}{0.986251,0.808597,0.643230}%
\pgfsetfillcolor{currentfill}%
\pgfsetlinewidth{0.250937pt}%
\definecolor{currentstroke}{rgb}{1.000000,1.000000,1.000000}%
\pgfsetstrokecolor{currentstroke}%
\pgfsetdash{}{0pt}%
\pgfpathmoveto{\pgfqpoint{3.627169in}{8.386132in}}%
\pgfpathlineto{\pgfqpoint{3.714905in}{8.386132in}}%
\pgfpathlineto{\pgfqpoint{3.714905in}{8.298396in}}%
\pgfpathlineto{\pgfqpoint{3.627169in}{8.298396in}}%
\pgfpathlineto{\pgfqpoint{3.627169in}{8.386132in}}%
\pgfusepath{stroke,fill}%
\end{pgfscope}%
\begin{pgfscope}%
\pgfpathrectangle{\pgfqpoint{0.380943in}{8.035189in}}{\pgfqpoint{4.650000in}{0.614151in}}%
\pgfusepath{clip}%
\pgfsetbuttcap%
\pgfsetroundjoin%
\definecolor{currentfill}{rgb}{0.961061,0.931672,0.728304}%
\pgfsetfillcolor{currentfill}%
\pgfsetlinewidth{0.250937pt}%
\definecolor{currentstroke}{rgb}{1.000000,1.000000,1.000000}%
\pgfsetstrokecolor{currentstroke}%
\pgfsetdash{}{0pt}%
\pgfpathmoveto{\pgfqpoint{3.714905in}{8.386132in}}%
\pgfpathlineto{\pgfqpoint{3.802641in}{8.386132in}}%
\pgfpathlineto{\pgfqpoint{3.802641in}{8.298396in}}%
\pgfpathlineto{\pgfqpoint{3.714905in}{8.298396in}}%
\pgfpathlineto{\pgfqpoint{3.714905in}{8.386132in}}%
\pgfusepath{stroke,fill}%
\end{pgfscope}%
\begin{pgfscope}%
\pgfpathrectangle{\pgfqpoint{0.380943in}{8.035189in}}{\pgfqpoint{4.650000in}{0.614151in}}%
\pgfusepath{clip}%
\pgfsetbuttcap%
\pgfsetroundjoin%
\definecolor{currentfill}{rgb}{1.000000,0.584929,0.522599}%
\pgfsetfillcolor{currentfill}%
\pgfsetlinewidth{0.250937pt}%
\definecolor{currentstroke}{rgb}{1.000000,1.000000,1.000000}%
\pgfsetstrokecolor{currentstroke}%
\pgfsetdash{}{0pt}%
\pgfpathmoveto{\pgfqpoint{3.802641in}{8.386132in}}%
\pgfpathlineto{\pgfqpoint{3.890377in}{8.386132in}}%
\pgfpathlineto{\pgfqpoint{3.890377in}{8.298396in}}%
\pgfpathlineto{\pgfqpoint{3.802641in}{8.298396in}}%
\pgfpathlineto{\pgfqpoint{3.802641in}{8.386132in}}%
\pgfusepath{stroke,fill}%
\end{pgfscope}%
\begin{pgfscope}%
\pgfpathrectangle{\pgfqpoint{0.380943in}{8.035189in}}{\pgfqpoint{4.650000in}{0.614151in}}%
\pgfusepath{clip}%
\pgfsetbuttcap%
\pgfsetroundjoin%
\definecolor{currentfill}{rgb}{0.992326,0.765229,0.614840}%
\pgfsetfillcolor{currentfill}%
\pgfsetlinewidth{0.250937pt}%
\definecolor{currentstroke}{rgb}{1.000000,1.000000,1.000000}%
\pgfsetstrokecolor{currentstroke}%
\pgfsetdash{}{0pt}%
\pgfpathmoveto{\pgfqpoint{3.890377in}{8.386132in}}%
\pgfpathlineto{\pgfqpoint{3.978113in}{8.386132in}}%
\pgfpathlineto{\pgfqpoint{3.978113in}{8.298396in}}%
\pgfpathlineto{\pgfqpoint{3.890377in}{8.298396in}}%
\pgfpathlineto{\pgfqpoint{3.890377in}{8.386132in}}%
\pgfusepath{stroke,fill}%
\end{pgfscope}%
\begin{pgfscope}%
\pgfpathrectangle{\pgfqpoint{0.380943in}{8.035189in}}{\pgfqpoint{4.650000in}{0.614151in}}%
\pgfusepath{clip}%
\pgfsetbuttcap%
\pgfsetroundjoin%
\definecolor{currentfill}{rgb}{0.986251,0.808597,0.643230}%
\pgfsetfillcolor{currentfill}%
\pgfsetlinewidth{0.250937pt}%
\definecolor{currentstroke}{rgb}{1.000000,1.000000,1.000000}%
\pgfsetstrokecolor{currentstroke}%
\pgfsetdash{}{0pt}%
\pgfpathmoveto{\pgfqpoint{3.978113in}{8.386132in}}%
\pgfpathlineto{\pgfqpoint{4.065849in}{8.386132in}}%
\pgfpathlineto{\pgfqpoint{4.065849in}{8.298396in}}%
\pgfpathlineto{\pgfqpoint{3.978113in}{8.298396in}}%
\pgfpathlineto{\pgfqpoint{3.978113in}{8.386132in}}%
\pgfusepath{stroke,fill}%
\end{pgfscope}%
\begin{pgfscope}%
\pgfpathrectangle{\pgfqpoint{0.380943in}{8.035189in}}{\pgfqpoint{4.650000in}{0.614151in}}%
\pgfusepath{clip}%
\pgfsetbuttcap%
\pgfsetroundjoin%
\definecolor{currentfill}{rgb}{1.000000,0.480477,0.479293}%
\pgfsetfillcolor{currentfill}%
\pgfsetlinewidth{0.250937pt}%
\definecolor{currentstroke}{rgb}{1.000000,1.000000,1.000000}%
\pgfsetstrokecolor{currentstroke}%
\pgfsetdash{}{0pt}%
\pgfpathmoveto{\pgfqpoint{4.065849in}{8.386132in}}%
\pgfpathlineto{\pgfqpoint{4.153585in}{8.386132in}}%
\pgfpathlineto{\pgfqpoint{4.153585in}{8.298396in}}%
\pgfpathlineto{\pgfqpoint{4.065849in}{8.298396in}}%
\pgfpathlineto{\pgfqpoint{4.065849in}{8.386132in}}%
\pgfusepath{stroke,fill}%
\end{pgfscope}%
\begin{pgfscope}%
\pgfpathrectangle{\pgfqpoint{0.380943in}{8.035189in}}{\pgfqpoint{4.650000in}{0.614151in}}%
\pgfusepath{clip}%
\pgfsetbuttcap%
\pgfsetroundjoin%
\definecolor{currentfill}{rgb}{0.992326,0.765229,0.614840}%
\pgfsetfillcolor{currentfill}%
\pgfsetlinewidth{0.250937pt}%
\definecolor{currentstroke}{rgb}{1.000000,1.000000,1.000000}%
\pgfsetstrokecolor{currentstroke}%
\pgfsetdash{}{0pt}%
\pgfpathmoveto{\pgfqpoint{4.153585in}{8.386132in}}%
\pgfpathlineto{\pgfqpoint{4.241320in}{8.386132in}}%
\pgfpathlineto{\pgfqpoint{4.241320in}{8.298396in}}%
\pgfpathlineto{\pgfqpoint{4.153585in}{8.298396in}}%
\pgfpathlineto{\pgfqpoint{4.153585in}{8.386132in}}%
\pgfusepath{stroke,fill}%
\end{pgfscope}%
\begin{pgfscope}%
\pgfpathrectangle{\pgfqpoint{0.380943in}{8.035189in}}{\pgfqpoint{4.650000in}{0.614151in}}%
\pgfusepath{clip}%
\pgfsetbuttcap%
\pgfsetroundjoin%
\definecolor{currentfill}{rgb}{0.986251,0.808597,0.643230}%
\pgfsetfillcolor{currentfill}%
\pgfsetlinewidth{0.250937pt}%
\definecolor{currentstroke}{rgb}{1.000000,1.000000,1.000000}%
\pgfsetstrokecolor{currentstroke}%
\pgfsetdash{}{0pt}%
\pgfpathmoveto{\pgfqpoint{4.241320in}{8.386132in}}%
\pgfpathlineto{\pgfqpoint{4.329056in}{8.386132in}}%
\pgfpathlineto{\pgfqpoint{4.329056in}{8.298396in}}%
\pgfpathlineto{\pgfqpoint{4.241320in}{8.298396in}}%
\pgfpathlineto{\pgfqpoint{4.241320in}{8.386132in}}%
\pgfusepath{stroke,fill}%
\end{pgfscope}%
\begin{pgfscope}%
\pgfpathrectangle{\pgfqpoint{0.380943in}{8.035189in}}{\pgfqpoint{4.650000in}{0.614151in}}%
\pgfusepath{clip}%
\pgfsetbuttcap%
\pgfsetroundjoin%
\definecolor{currentfill}{rgb}{0.961061,0.931672,0.728304}%
\pgfsetfillcolor{currentfill}%
\pgfsetlinewidth{0.250937pt}%
\definecolor{currentstroke}{rgb}{1.000000,1.000000,1.000000}%
\pgfsetstrokecolor{currentstroke}%
\pgfsetdash{}{0pt}%
\pgfpathmoveto{\pgfqpoint{4.329056in}{8.386132in}}%
\pgfpathlineto{\pgfqpoint{4.416792in}{8.386132in}}%
\pgfpathlineto{\pgfqpoint{4.416792in}{8.298396in}}%
\pgfpathlineto{\pgfqpoint{4.329056in}{8.298396in}}%
\pgfpathlineto{\pgfqpoint{4.329056in}{8.386132in}}%
\pgfusepath{stroke,fill}%
\end{pgfscope}%
\begin{pgfscope}%
\pgfpathrectangle{\pgfqpoint{0.380943in}{8.035189in}}{\pgfqpoint{4.650000in}{0.614151in}}%
\pgfusepath{clip}%
\pgfsetbuttcap%
\pgfsetroundjoin%
\definecolor{currentfill}{rgb}{0.999616,0.641369,0.546559}%
\pgfsetfillcolor{currentfill}%
\pgfsetlinewidth{0.250937pt}%
\definecolor{currentstroke}{rgb}{1.000000,1.000000,1.000000}%
\pgfsetstrokecolor{currentstroke}%
\pgfsetdash{}{0pt}%
\pgfpathmoveto{\pgfqpoint{4.416792in}{8.386132in}}%
\pgfpathlineto{\pgfqpoint{4.504528in}{8.386132in}}%
\pgfpathlineto{\pgfqpoint{4.504528in}{8.298396in}}%
\pgfpathlineto{\pgfqpoint{4.416792in}{8.298396in}}%
\pgfpathlineto{\pgfqpoint{4.416792in}{8.386132in}}%
\pgfusepath{stroke,fill}%
\end{pgfscope}%
\begin{pgfscope}%
\pgfpathrectangle{\pgfqpoint{0.380943in}{8.035189in}}{\pgfqpoint{4.650000in}{0.614151in}}%
\pgfusepath{clip}%
\pgfsetbuttcap%
\pgfsetroundjoin%
\definecolor{currentfill}{rgb}{0.986251,0.808597,0.643230}%
\pgfsetfillcolor{currentfill}%
\pgfsetlinewidth{0.250937pt}%
\definecolor{currentstroke}{rgb}{1.000000,1.000000,1.000000}%
\pgfsetstrokecolor{currentstroke}%
\pgfsetdash{}{0pt}%
\pgfpathmoveto{\pgfqpoint{4.504528in}{8.386132in}}%
\pgfpathlineto{\pgfqpoint{4.592264in}{8.386132in}}%
\pgfpathlineto{\pgfqpoint{4.592264in}{8.298396in}}%
\pgfpathlineto{\pgfqpoint{4.504528in}{8.298396in}}%
\pgfpathlineto{\pgfqpoint{4.504528in}{8.386132in}}%
\pgfusepath{stroke,fill}%
\end{pgfscope}%
\begin{pgfscope}%
\pgfpathrectangle{\pgfqpoint{0.380943in}{8.035189in}}{\pgfqpoint{4.650000in}{0.614151in}}%
\pgfusepath{clip}%
\pgfsetbuttcap%
\pgfsetroundjoin%
\definecolor{currentfill}{rgb}{0.935025,0.413456,0.413456}%
\pgfsetfillcolor{currentfill}%
\pgfsetlinewidth{0.250937pt}%
\definecolor{currentstroke}{rgb}{1.000000,1.000000,1.000000}%
\pgfsetstrokecolor{currentstroke}%
\pgfsetdash{}{0pt}%
\pgfpathmoveto{\pgfqpoint{4.592264in}{8.386132in}}%
\pgfpathlineto{\pgfqpoint{4.680000in}{8.386132in}}%
\pgfpathlineto{\pgfqpoint{4.680000in}{8.298396in}}%
\pgfpathlineto{\pgfqpoint{4.592264in}{8.298396in}}%
\pgfpathlineto{\pgfqpoint{4.592264in}{8.386132in}}%
\pgfusepath{stroke,fill}%
\end{pgfscope}%
\begin{pgfscope}%
\pgfpathrectangle{\pgfqpoint{0.380943in}{8.035189in}}{\pgfqpoint{4.650000in}{0.614151in}}%
\pgfusepath{clip}%
\pgfsetbuttcap%
\pgfsetroundjoin%
\definecolor{currentfill}{rgb}{0.999616,0.641369,0.546559}%
\pgfsetfillcolor{currentfill}%
\pgfsetlinewidth{0.250937pt}%
\definecolor{currentstroke}{rgb}{1.000000,1.000000,1.000000}%
\pgfsetstrokecolor{currentstroke}%
\pgfsetdash{}{0pt}%
\pgfpathmoveto{\pgfqpoint{4.680000in}{8.386132in}}%
\pgfpathlineto{\pgfqpoint{4.767736in}{8.386132in}}%
\pgfpathlineto{\pgfqpoint{4.767736in}{8.298396in}}%
\pgfpathlineto{\pgfqpoint{4.680000in}{8.298396in}}%
\pgfpathlineto{\pgfqpoint{4.680000in}{8.386132in}}%
\pgfusepath{stroke,fill}%
\end{pgfscope}%
\begin{pgfscope}%
\pgfpathrectangle{\pgfqpoint{0.380943in}{8.035189in}}{\pgfqpoint{4.650000in}{0.614151in}}%
\pgfusepath{clip}%
\pgfsetbuttcap%
\pgfsetroundjoin%
\definecolor{currentfill}{rgb}{0.996909,0.711742,0.584452}%
\pgfsetfillcolor{currentfill}%
\pgfsetlinewidth{0.250937pt}%
\definecolor{currentstroke}{rgb}{1.000000,1.000000,1.000000}%
\pgfsetstrokecolor{currentstroke}%
\pgfsetdash{}{0pt}%
\pgfpathmoveto{\pgfqpoint{4.767736in}{8.386132in}}%
\pgfpathlineto{\pgfqpoint{4.855471in}{8.386132in}}%
\pgfpathlineto{\pgfqpoint{4.855471in}{8.298396in}}%
\pgfpathlineto{\pgfqpoint{4.767736in}{8.298396in}}%
\pgfpathlineto{\pgfqpoint{4.767736in}{8.386132in}}%
\pgfusepath{stroke,fill}%
\end{pgfscope}%
\begin{pgfscope}%
\pgfpathrectangle{\pgfqpoint{0.380943in}{8.035189in}}{\pgfqpoint{4.650000in}{0.614151in}}%
\pgfusepath{clip}%
\pgfsetbuttcap%
\pgfsetroundjoin%
\definecolor{currentfill}{rgb}{0.978131,0.843783,0.675709}%
\pgfsetfillcolor{currentfill}%
\pgfsetlinewidth{0.250937pt}%
\definecolor{currentstroke}{rgb}{1.000000,1.000000,1.000000}%
\pgfsetstrokecolor{currentstroke}%
\pgfsetdash{}{0pt}%
\pgfpathmoveto{\pgfqpoint{4.855471in}{8.386132in}}%
\pgfpathlineto{\pgfqpoint{4.943207in}{8.386132in}}%
\pgfpathlineto{\pgfqpoint{4.943207in}{8.298396in}}%
\pgfpathlineto{\pgfqpoint{4.855471in}{8.298396in}}%
\pgfpathlineto{\pgfqpoint{4.855471in}{8.386132in}}%
\pgfusepath{stroke,fill}%
\end{pgfscope}%
\begin{pgfscope}%
\pgfpathrectangle{\pgfqpoint{0.380943in}{8.035189in}}{\pgfqpoint{4.650000in}{0.614151in}}%
\pgfusepath{clip}%
\pgfsetbuttcap%
\pgfsetroundjoin%
\definecolor{currentfill}{rgb}{0.970012,0.883276,0.699577}%
\pgfsetfillcolor{currentfill}%
\pgfsetlinewidth{0.250937pt}%
\definecolor{currentstroke}{rgb}{1.000000,1.000000,1.000000}%
\pgfsetstrokecolor{currentstroke}%
\pgfsetdash{}{0pt}%
\pgfpathmoveto{\pgfqpoint{4.943207in}{8.386132in}}%
\pgfpathlineto{\pgfqpoint{5.030943in}{8.386132in}}%
\pgfpathlineto{\pgfqpoint{5.030943in}{8.298396in}}%
\pgfpathlineto{\pgfqpoint{4.943207in}{8.298396in}}%
\pgfpathlineto{\pgfqpoint{4.943207in}{8.386132in}}%
\pgfusepath{stroke,fill}%
\end{pgfscope}%
\begin{pgfscope}%
\pgfpathrectangle{\pgfqpoint{0.380943in}{8.035189in}}{\pgfqpoint{4.650000in}{0.614151in}}%
\pgfusepath{clip}%
\pgfsetbuttcap%
\pgfsetroundjoin%
\pgfsetlinewidth{0.250937pt}%
\definecolor{currentstroke}{rgb}{1.000000,1.000000,1.000000}%
\pgfsetstrokecolor{currentstroke}%
\pgfsetdash{}{0pt}%
\pgfpathmoveto{\pgfqpoint{0.380943in}{8.298396in}}%
\pgfpathlineto{\pgfqpoint{0.468679in}{8.298396in}}%
\pgfpathlineto{\pgfqpoint{0.468679in}{8.210661in}}%
\pgfpathlineto{\pgfqpoint{0.380943in}{8.210661in}}%
\pgfpathlineto{\pgfqpoint{0.380943in}{8.298396in}}%
\pgfusepath{stroke}%
\end{pgfscope}%
\begin{pgfscope}%
\pgfpathrectangle{\pgfqpoint{0.380943in}{8.035189in}}{\pgfqpoint{4.650000in}{0.614151in}}%
\pgfusepath{clip}%
\pgfsetbuttcap%
\pgfsetroundjoin%
\definecolor{currentfill}{rgb}{0.999616,0.641369,0.546559}%
\pgfsetfillcolor{currentfill}%
\pgfsetlinewidth{0.250937pt}%
\definecolor{currentstroke}{rgb}{1.000000,1.000000,1.000000}%
\pgfsetstrokecolor{currentstroke}%
\pgfsetdash{}{0pt}%
\pgfpathmoveto{\pgfqpoint{0.468679in}{8.298396in}}%
\pgfpathlineto{\pgfqpoint{0.556415in}{8.298396in}}%
\pgfpathlineto{\pgfqpoint{0.556415in}{8.210661in}}%
\pgfpathlineto{\pgfqpoint{0.468679in}{8.210661in}}%
\pgfpathlineto{\pgfqpoint{0.468679in}{8.298396in}}%
\pgfusepath{stroke,fill}%
\end{pgfscope}%
\begin{pgfscope}%
\pgfpathrectangle{\pgfqpoint{0.380943in}{8.035189in}}{\pgfqpoint{4.650000in}{0.614151in}}%
\pgfusepath{clip}%
\pgfsetbuttcap%
\pgfsetroundjoin%
\definecolor{currentfill}{rgb}{0.986251,0.808597,0.643230}%
\pgfsetfillcolor{currentfill}%
\pgfsetlinewidth{0.250937pt}%
\definecolor{currentstroke}{rgb}{1.000000,1.000000,1.000000}%
\pgfsetstrokecolor{currentstroke}%
\pgfsetdash{}{0pt}%
\pgfpathmoveto{\pgfqpoint{0.556415in}{8.298396in}}%
\pgfpathlineto{\pgfqpoint{0.644151in}{8.298396in}}%
\pgfpathlineto{\pgfqpoint{0.644151in}{8.210661in}}%
\pgfpathlineto{\pgfqpoint{0.556415in}{8.210661in}}%
\pgfpathlineto{\pgfqpoint{0.556415in}{8.298396in}}%
\pgfusepath{stroke,fill}%
\end{pgfscope}%
\begin{pgfscope}%
\pgfpathrectangle{\pgfqpoint{0.380943in}{8.035189in}}{\pgfqpoint{4.650000in}{0.614151in}}%
\pgfusepath{clip}%
\pgfsetbuttcap%
\pgfsetroundjoin%
\definecolor{currentfill}{rgb}{0.970012,0.883276,0.699577}%
\pgfsetfillcolor{currentfill}%
\pgfsetlinewidth{0.250937pt}%
\definecolor{currentstroke}{rgb}{1.000000,1.000000,1.000000}%
\pgfsetstrokecolor{currentstroke}%
\pgfsetdash{}{0pt}%
\pgfpathmoveto{\pgfqpoint{0.644151in}{8.298396in}}%
\pgfpathlineto{\pgfqpoint{0.731886in}{8.298396in}}%
\pgfpathlineto{\pgfqpoint{0.731886in}{8.210661in}}%
\pgfpathlineto{\pgfqpoint{0.644151in}{8.210661in}}%
\pgfpathlineto{\pgfqpoint{0.644151in}{8.298396in}}%
\pgfusepath{stroke,fill}%
\end{pgfscope}%
\begin{pgfscope}%
\pgfpathrectangle{\pgfqpoint{0.380943in}{8.035189in}}{\pgfqpoint{4.650000in}{0.614151in}}%
\pgfusepath{clip}%
\pgfsetbuttcap%
\pgfsetroundjoin%
\definecolor{currentfill}{rgb}{0.986251,0.808597,0.643230}%
\pgfsetfillcolor{currentfill}%
\pgfsetlinewidth{0.250937pt}%
\definecolor{currentstroke}{rgb}{1.000000,1.000000,1.000000}%
\pgfsetstrokecolor{currentstroke}%
\pgfsetdash{}{0pt}%
\pgfpathmoveto{\pgfqpoint{0.731886in}{8.298396in}}%
\pgfpathlineto{\pgfqpoint{0.819622in}{8.298396in}}%
\pgfpathlineto{\pgfqpoint{0.819622in}{8.210661in}}%
\pgfpathlineto{\pgfqpoint{0.731886in}{8.210661in}}%
\pgfpathlineto{\pgfqpoint{0.731886in}{8.298396in}}%
\pgfusepath{stroke,fill}%
\end{pgfscope}%
\begin{pgfscope}%
\pgfpathrectangle{\pgfqpoint{0.380943in}{8.035189in}}{\pgfqpoint{4.650000in}{0.614151in}}%
\pgfusepath{clip}%
\pgfsetbuttcap%
\pgfsetroundjoin%
\definecolor{currentfill}{rgb}{0.978131,0.843783,0.675709}%
\pgfsetfillcolor{currentfill}%
\pgfsetlinewidth{0.250937pt}%
\definecolor{currentstroke}{rgb}{1.000000,1.000000,1.000000}%
\pgfsetstrokecolor{currentstroke}%
\pgfsetdash{}{0pt}%
\pgfpathmoveto{\pgfqpoint{0.819622in}{8.298396in}}%
\pgfpathlineto{\pgfqpoint{0.907358in}{8.298396in}}%
\pgfpathlineto{\pgfqpoint{0.907358in}{8.210661in}}%
\pgfpathlineto{\pgfqpoint{0.819622in}{8.210661in}}%
\pgfpathlineto{\pgfqpoint{0.819622in}{8.298396in}}%
\pgfusepath{stroke,fill}%
\end{pgfscope}%
\begin{pgfscope}%
\pgfpathrectangle{\pgfqpoint{0.380943in}{8.035189in}}{\pgfqpoint{4.650000in}{0.614151in}}%
\pgfusepath{clip}%
\pgfsetbuttcap%
\pgfsetroundjoin%
\definecolor{currentfill}{rgb}{0.992326,0.765229,0.614840}%
\pgfsetfillcolor{currentfill}%
\pgfsetlinewidth{0.250937pt}%
\definecolor{currentstroke}{rgb}{1.000000,1.000000,1.000000}%
\pgfsetstrokecolor{currentstroke}%
\pgfsetdash{}{0pt}%
\pgfpathmoveto{\pgfqpoint{0.907358in}{8.298396in}}%
\pgfpathlineto{\pgfqpoint{0.995094in}{8.298396in}}%
\pgfpathlineto{\pgfqpoint{0.995094in}{8.210661in}}%
\pgfpathlineto{\pgfqpoint{0.907358in}{8.210661in}}%
\pgfpathlineto{\pgfqpoint{0.907358in}{8.298396in}}%
\pgfusepath{stroke,fill}%
\end{pgfscope}%
\begin{pgfscope}%
\pgfpathrectangle{\pgfqpoint{0.380943in}{8.035189in}}{\pgfqpoint{4.650000in}{0.614151in}}%
\pgfusepath{clip}%
\pgfsetbuttcap%
\pgfsetroundjoin%
\definecolor{currentfill}{rgb}{0.999616,0.641369,0.546559}%
\pgfsetfillcolor{currentfill}%
\pgfsetlinewidth{0.250937pt}%
\definecolor{currentstroke}{rgb}{1.000000,1.000000,1.000000}%
\pgfsetstrokecolor{currentstroke}%
\pgfsetdash{}{0pt}%
\pgfpathmoveto{\pgfqpoint{0.995094in}{8.298396in}}%
\pgfpathlineto{\pgfqpoint{1.082830in}{8.298396in}}%
\pgfpathlineto{\pgfqpoint{1.082830in}{8.210661in}}%
\pgfpathlineto{\pgfqpoint{0.995094in}{8.210661in}}%
\pgfpathlineto{\pgfqpoint{0.995094in}{8.298396in}}%
\pgfusepath{stroke,fill}%
\end{pgfscope}%
\begin{pgfscope}%
\pgfpathrectangle{\pgfqpoint{0.380943in}{8.035189in}}{\pgfqpoint{4.650000in}{0.614151in}}%
\pgfusepath{clip}%
\pgfsetbuttcap%
\pgfsetroundjoin%
\definecolor{currentfill}{rgb}{0.992326,0.765229,0.614840}%
\pgfsetfillcolor{currentfill}%
\pgfsetlinewidth{0.250937pt}%
\definecolor{currentstroke}{rgb}{1.000000,1.000000,1.000000}%
\pgfsetstrokecolor{currentstroke}%
\pgfsetdash{}{0pt}%
\pgfpathmoveto{\pgfqpoint{1.082830in}{8.298396in}}%
\pgfpathlineto{\pgfqpoint{1.170566in}{8.298396in}}%
\pgfpathlineto{\pgfqpoint{1.170566in}{8.210661in}}%
\pgfpathlineto{\pgfqpoint{1.082830in}{8.210661in}}%
\pgfpathlineto{\pgfqpoint{1.082830in}{8.298396in}}%
\pgfusepath{stroke,fill}%
\end{pgfscope}%
\begin{pgfscope}%
\pgfpathrectangle{\pgfqpoint{0.380943in}{8.035189in}}{\pgfqpoint{4.650000in}{0.614151in}}%
\pgfusepath{clip}%
\pgfsetbuttcap%
\pgfsetroundjoin%
\definecolor{currentfill}{rgb}{0.978131,0.843783,0.675709}%
\pgfsetfillcolor{currentfill}%
\pgfsetlinewidth{0.250937pt}%
\definecolor{currentstroke}{rgb}{1.000000,1.000000,1.000000}%
\pgfsetstrokecolor{currentstroke}%
\pgfsetdash{}{0pt}%
\pgfpathmoveto{\pgfqpoint{1.170566in}{8.298396in}}%
\pgfpathlineto{\pgfqpoint{1.258302in}{8.298396in}}%
\pgfpathlineto{\pgfqpoint{1.258302in}{8.210661in}}%
\pgfpathlineto{\pgfqpoint{1.170566in}{8.210661in}}%
\pgfpathlineto{\pgfqpoint{1.170566in}{8.298396in}}%
\pgfusepath{stroke,fill}%
\end{pgfscope}%
\begin{pgfscope}%
\pgfpathrectangle{\pgfqpoint{0.380943in}{8.035189in}}{\pgfqpoint{4.650000in}{0.614151in}}%
\pgfusepath{clip}%
\pgfsetbuttcap%
\pgfsetroundjoin%
\definecolor{currentfill}{rgb}{0.985083,0.974641,0.792587}%
\pgfsetfillcolor{currentfill}%
\pgfsetlinewidth{0.250937pt}%
\definecolor{currentstroke}{rgb}{1.000000,1.000000,1.000000}%
\pgfsetstrokecolor{currentstroke}%
\pgfsetdash{}{0pt}%
\pgfpathmoveto{\pgfqpoint{1.258302in}{8.298396in}}%
\pgfpathlineto{\pgfqpoint{1.346037in}{8.298396in}}%
\pgfpathlineto{\pgfqpoint{1.346037in}{8.210661in}}%
\pgfpathlineto{\pgfqpoint{1.258302in}{8.210661in}}%
\pgfpathlineto{\pgfqpoint{1.258302in}{8.298396in}}%
\pgfusepath{stroke,fill}%
\end{pgfscope}%
\begin{pgfscope}%
\pgfpathrectangle{\pgfqpoint{0.380943in}{8.035189in}}{\pgfqpoint{4.650000in}{0.614151in}}%
\pgfusepath{clip}%
\pgfsetbuttcap%
\pgfsetroundjoin%
\definecolor{currentfill}{rgb}{0.970012,0.883276,0.699577}%
\pgfsetfillcolor{currentfill}%
\pgfsetlinewidth{0.250937pt}%
\definecolor{currentstroke}{rgb}{1.000000,1.000000,1.000000}%
\pgfsetstrokecolor{currentstroke}%
\pgfsetdash{}{0pt}%
\pgfpathmoveto{\pgfqpoint{1.346037in}{8.298396in}}%
\pgfpathlineto{\pgfqpoint{1.433773in}{8.298396in}}%
\pgfpathlineto{\pgfqpoint{1.433773in}{8.210661in}}%
\pgfpathlineto{\pgfqpoint{1.346037in}{8.210661in}}%
\pgfpathlineto{\pgfqpoint{1.346037in}{8.298396in}}%
\pgfusepath{stroke,fill}%
\end{pgfscope}%
\begin{pgfscope}%
\pgfpathrectangle{\pgfqpoint{0.380943in}{8.035189in}}{\pgfqpoint{4.650000in}{0.614151in}}%
\pgfusepath{clip}%
\pgfsetbuttcap%
\pgfsetroundjoin%
\definecolor{currentfill}{rgb}{0.992326,0.765229,0.614840}%
\pgfsetfillcolor{currentfill}%
\pgfsetlinewidth{0.250937pt}%
\definecolor{currentstroke}{rgb}{1.000000,1.000000,1.000000}%
\pgfsetstrokecolor{currentstroke}%
\pgfsetdash{}{0pt}%
\pgfpathmoveto{\pgfqpoint{1.433773in}{8.298396in}}%
\pgfpathlineto{\pgfqpoint{1.521509in}{8.298396in}}%
\pgfpathlineto{\pgfqpoint{1.521509in}{8.210661in}}%
\pgfpathlineto{\pgfqpoint{1.433773in}{8.210661in}}%
\pgfpathlineto{\pgfqpoint{1.433773in}{8.298396in}}%
\pgfusepath{stroke,fill}%
\end{pgfscope}%
\begin{pgfscope}%
\pgfpathrectangle{\pgfqpoint{0.380943in}{8.035189in}}{\pgfqpoint{4.650000in}{0.614151in}}%
\pgfusepath{clip}%
\pgfsetbuttcap%
\pgfsetroundjoin%
\definecolor{currentfill}{rgb}{0.961061,0.931672,0.728304}%
\pgfsetfillcolor{currentfill}%
\pgfsetlinewidth{0.250937pt}%
\definecolor{currentstroke}{rgb}{1.000000,1.000000,1.000000}%
\pgfsetstrokecolor{currentstroke}%
\pgfsetdash{}{0pt}%
\pgfpathmoveto{\pgfqpoint{1.521509in}{8.298396in}}%
\pgfpathlineto{\pgfqpoint{1.609245in}{8.298396in}}%
\pgfpathlineto{\pgfqpoint{1.609245in}{8.210661in}}%
\pgfpathlineto{\pgfqpoint{1.521509in}{8.210661in}}%
\pgfpathlineto{\pgfqpoint{1.521509in}{8.298396in}}%
\pgfusepath{stroke,fill}%
\end{pgfscope}%
\begin{pgfscope}%
\pgfpathrectangle{\pgfqpoint{0.380943in}{8.035189in}}{\pgfqpoint{4.650000in}{0.614151in}}%
\pgfusepath{clip}%
\pgfsetbuttcap%
\pgfsetroundjoin%
\definecolor{currentfill}{rgb}{0.986251,0.808597,0.643230}%
\pgfsetfillcolor{currentfill}%
\pgfsetlinewidth{0.250937pt}%
\definecolor{currentstroke}{rgb}{1.000000,1.000000,1.000000}%
\pgfsetstrokecolor{currentstroke}%
\pgfsetdash{}{0pt}%
\pgfpathmoveto{\pgfqpoint{1.609245in}{8.298396in}}%
\pgfpathlineto{\pgfqpoint{1.696981in}{8.298396in}}%
\pgfpathlineto{\pgfqpoint{1.696981in}{8.210661in}}%
\pgfpathlineto{\pgfqpoint{1.609245in}{8.210661in}}%
\pgfpathlineto{\pgfqpoint{1.609245in}{8.298396in}}%
\pgfusepath{stroke,fill}%
\end{pgfscope}%
\begin{pgfscope}%
\pgfpathrectangle{\pgfqpoint{0.380943in}{8.035189in}}{\pgfqpoint{4.650000in}{0.614151in}}%
\pgfusepath{clip}%
\pgfsetbuttcap%
\pgfsetroundjoin%
\definecolor{currentfill}{rgb}{0.985083,0.974641,0.792587}%
\pgfsetfillcolor{currentfill}%
\pgfsetlinewidth{0.250937pt}%
\definecolor{currentstroke}{rgb}{1.000000,1.000000,1.000000}%
\pgfsetstrokecolor{currentstroke}%
\pgfsetdash{}{0pt}%
\pgfpathmoveto{\pgfqpoint{1.696981in}{8.298396in}}%
\pgfpathlineto{\pgfqpoint{1.784717in}{8.298396in}}%
\pgfpathlineto{\pgfqpoint{1.784717in}{8.210661in}}%
\pgfpathlineto{\pgfqpoint{1.696981in}{8.210661in}}%
\pgfpathlineto{\pgfqpoint{1.696981in}{8.298396in}}%
\pgfusepath{stroke,fill}%
\end{pgfscope}%
\begin{pgfscope}%
\pgfpathrectangle{\pgfqpoint{0.380943in}{8.035189in}}{\pgfqpoint{4.650000in}{0.614151in}}%
\pgfusepath{clip}%
\pgfsetbuttcap%
\pgfsetroundjoin%
\definecolor{currentfill}{rgb}{0.963768,0.915433,0.717478}%
\pgfsetfillcolor{currentfill}%
\pgfsetlinewidth{0.250937pt}%
\definecolor{currentstroke}{rgb}{1.000000,1.000000,1.000000}%
\pgfsetstrokecolor{currentstroke}%
\pgfsetdash{}{0pt}%
\pgfpathmoveto{\pgfqpoint{1.784717in}{8.298396in}}%
\pgfpathlineto{\pgfqpoint{1.872452in}{8.298396in}}%
\pgfpathlineto{\pgfqpoint{1.872452in}{8.210661in}}%
\pgfpathlineto{\pgfqpoint{1.784717in}{8.210661in}}%
\pgfpathlineto{\pgfqpoint{1.784717in}{8.298396in}}%
\pgfusepath{stroke,fill}%
\end{pgfscope}%
\begin{pgfscope}%
\pgfpathrectangle{\pgfqpoint{0.380943in}{8.035189in}}{\pgfqpoint{4.650000in}{0.614151in}}%
\pgfusepath{clip}%
\pgfsetbuttcap%
\pgfsetroundjoin%
\definecolor{currentfill}{rgb}{0.963768,0.915433,0.717478}%
\pgfsetfillcolor{currentfill}%
\pgfsetlinewidth{0.250937pt}%
\definecolor{currentstroke}{rgb}{1.000000,1.000000,1.000000}%
\pgfsetstrokecolor{currentstroke}%
\pgfsetdash{}{0pt}%
\pgfpathmoveto{\pgfqpoint{1.872452in}{8.298396in}}%
\pgfpathlineto{\pgfqpoint{1.960188in}{8.298396in}}%
\pgfpathlineto{\pgfqpoint{1.960188in}{8.210661in}}%
\pgfpathlineto{\pgfqpoint{1.872452in}{8.210661in}}%
\pgfpathlineto{\pgfqpoint{1.872452in}{8.298396in}}%
\pgfusepath{stroke,fill}%
\end{pgfscope}%
\begin{pgfscope}%
\pgfpathrectangle{\pgfqpoint{0.380943in}{8.035189in}}{\pgfqpoint{4.650000in}{0.614151in}}%
\pgfusepath{clip}%
\pgfsetbuttcap%
\pgfsetroundjoin%
\definecolor{currentfill}{rgb}{0.963768,0.915433,0.717478}%
\pgfsetfillcolor{currentfill}%
\pgfsetlinewidth{0.250937pt}%
\definecolor{currentstroke}{rgb}{1.000000,1.000000,1.000000}%
\pgfsetstrokecolor{currentstroke}%
\pgfsetdash{}{0pt}%
\pgfpathmoveto{\pgfqpoint{1.960188in}{8.298396in}}%
\pgfpathlineto{\pgfqpoint{2.047924in}{8.298396in}}%
\pgfpathlineto{\pgfqpoint{2.047924in}{8.210661in}}%
\pgfpathlineto{\pgfqpoint{1.960188in}{8.210661in}}%
\pgfpathlineto{\pgfqpoint{1.960188in}{8.298396in}}%
\pgfusepath{stroke,fill}%
\end{pgfscope}%
\begin{pgfscope}%
\pgfpathrectangle{\pgfqpoint{0.380943in}{8.035189in}}{\pgfqpoint{4.650000in}{0.614151in}}%
\pgfusepath{clip}%
\pgfsetbuttcap%
\pgfsetroundjoin%
\definecolor{currentfill}{rgb}{0.985083,0.974641,0.792587}%
\pgfsetfillcolor{currentfill}%
\pgfsetlinewidth{0.250937pt}%
\definecolor{currentstroke}{rgb}{1.000000,1.000000,1.000000}%
\pgfsetstrokecolor{currentstroke}%
\pgfsetdash{}{0pt}%
\pgfpathmoveto{\pgfqpoint{2.047924in}{8.298396in}}%
\pgfpathlineto{\pgfqpoint{2.135660in}{8.298396in}}%
\pgfpathlineto{\pgfqpoint{2.135660in}{8.210661in}}%
\pgfpathlineto{\pgfqpoint{2.047924in}{8.210661in}}%
\pgfpathlineto{\pgfqpoint{2.047924in}{8.298396in}}%
\pgfusepath{stroke,fill}%
\end{pgfscope}%
\begin{pgfscope}%
\pgfpathrectangle{\pgfqpoint{0.380943in}{8.035189in}}{\pgfqpoint{4.650000in}{0.614151in}}%
\pgfusepath{clip}%
\pgfsetbuttcap%
\pgfsetroundjoin%
\definecolor{currentfill}{rgb}{0.986251,0.808597,0.643230}%
\pgfsetfillcolor{currentfill}%
\pgfsetlinewidth{0.250937pt}%
\definecolor{currentstroke}{rgb}{1.000000,1.000000,1.000000}%
\pgfsetstrokecolor{currentstroke}%
\pgfsetdash{}{0pt}%
\pgfpathmoveto{\pgfqpoint{2.135660in}{8.298396in}}%
\pgfpathlineto{\pgfqpoint{2.223396in}{8.298396in}}%
\pgfpathlineto{\pgfqpoint{2.223396in}{8.210661in}}%
\pgfpathlineto{\pgfqpoint{2.135660in}{8.210661in}}%
\pgfpathlineto{\pgfqpoint{2.135660in}{8.298396in}}%
\pgfusepath{stroke,fill}%
\end{pgfscope}%
\begin{pgfscope}%
\pgfpathrectangle{\pgfqpoint{0.380943in}{8.035189in}}{\pgfqpoint{4.650000in}{0.614151in}}%
\pgfusepath{clip}%
\pgfsetbuttcap%
\pgfsetroundjoin%
\definecolor{currentfill}{rgb}{0.963768,0.915433,0.717478}%
\pgfsetfillcolor{currentfill}%
\pgfsetlinewidth{0.250937pt}%
\definecolor{currentstroke}{rgb}{1.000000,1.000000,1.000000}%
\pgfsetstrokecolor{currentstroke}%
\pgfsetdash{}{0pt}%
\pgfpathmoveto{\pgfqpoint{2.223396in}{8.298396in}}%
\pgfpathlineto{\pgfqpoint{2.311132in}{8.298396in}}%
\pgfpathlineto{\pgfqpoint{2.311132in}{8.210661in}}%
\pgfpathlineto{\pgfqpoint{2.223396in}{8.210661in}}%
\pgfpathlineto{\pgfqpoint{2.223396in}{8.298396in}}%
\pgfusepath{stroke,fill}%
\end{pgfscope}%
\begin{pgfscope}%
\pgfpathrectangle{\pgfqpoint{0.380943in}{8.035189in}}{\pgfqpoint{4.650000in}{0.614151in}}%
\pgfusepath{clip}%
\pgfsetbuttcap%
\pgfsetroundjoin%
\definecolor{currentfill}{rgb}{0.996909,0.711742,0.584452}%
\pgfsetfillcolor{currentfill}%
\pgfsetlinewidth{0.250937pt}%
\definecolor{currentstroke}{rgb}{1.000000,1.000000,1.000000}%
\pgfsetstrokecolor{currentstroke}%
\pgfsetdash{}{0pt}%
\pgfpathmoveto{\pgfqpoint{2.311132in}{8.298396in}}%
\pgfpathlineto{\pgfqpoint{2.398868in}{8.298396in}}%
\pgfpathlineto{\pgfqpoint{2.398868in}{8.210661in}}%
\pgfpathlineto{\pgfqpoint{2.311132in}{8.210661in}}%
\pgfpathlineto{\pgfqpoint{2.311132in}{8.298396in}}%
\pgfusepath{stroke,fill}%
\end{pgfscope}%
\begin{pgfscope}%
\pgfpathrectangle{\pgfqpoint{0.380943in}{8.035189in}}{\pgfqpoint{4.650000in}{0.614151in}}%
\pgfusepath{clip}%
\pgfsetbuttcap%
\pgfsetroundjoin%
\definecolor{currentfill}{rgb}{0.992326,0.765229,0.614840}%
\pgfsetfillcolor{currentfill}%
\pgfsetlinewidth{0.250937pt}%
\definecolor{currentstroke}{rgb}{1.000000,1.000000,1.000000}%
\pgfsetstrokecolor{currentstroke}%
\pgfsetdash{}{0pt}%
\pgfpathmoveto{\pgfqpoint{2.398868in}{8.298396in}}%
\pgfpathlineto{\pgfqpoint{2.486603in}{8.298396in}}%
\pgfpathlineto{\pgfqpoint{2.486603in}{8.210661in}}%
\pgfpathlineto{\pgfqpoint{2.398868in}{8.210661in}}%
\pgfpathlineto{\pgfqpoint{2.398868in}{8.298396in}}%
\pgfusepath{stroke,fill}%
\end{pgfscope}%
\begin{pgfscope}%
\pgfpathrectangle{\pgfqpoint{0.380943in}{8.035189in}}{\pgfqpoint{4.650000in}{0.614151in}}%
\pgfusepath{clip}%
\pgfsetbuttcap%
\pgfsetroundjoin%
\definecolor{currentfill}{rgb}{0.970012,0.883276,0.699577}%
\pgfsetfillcolor{currentfill}%
\pgfsetlinewidth{0.250937pt}%
\definecolor{currentstroke}{rgb}{1.000000,1.000000,1.000000}%
\pgfsetstrokecolor{currentstroke}%
\pgfsetdash{}{0pt}%
\pgfpathmoveto{\pgfqpoint{2.486603in}{8.298396in}}%
\pgfpathlineto{\pgfqpoint{2.574339in}{8.298396in}}%
\pgfpathlineto{\pgfqpoint{2.574339in}{8.210661in}}%
\pgfpathlineto{\pgfqpoint{2.486603in}{8.210661in}}%
\pgfpathlineto{\pgfqpoint{2.486603in}{8.298396in}}%
\pgfusepath{stroke,fill}%
\end{pgfscope}%
\begin{pgfscope}%
\pgfpathrectangle{\pgfqpoint{0.380943in}{8.035189in}}{\pgfqpoint{4.650000in}{0.614151in}}%
\pgfusepath{clip}%
\pgfsetbuttcap%
\pgfsetroundjoin%
\definecolor{currentfill}{rgb}{0.986251,0.808597,0.643230}%
\pgfsetfillcolor{currentfill}%
\pgfsetlinewidth{0.250937pt}%
\definecolor{currentstroke}{rgb}{1.000000,1.000000,1.000000}%
\pgfsetstrokecolor{currentstroke}%
\pgfsetdash{}{0pt}%
\pgfpathmoveto{\pgfqpoint{2.574339in}{8.298396in}}%
\pgfpathlineto{\pgfqpoint{2.662075in}{8.298396in}}%
\pgfpathlineto{\pgfqpoint{2.662075in}{8.210661in}}%
\pgfpathlineto{\pgfqpoint{2.574339in}{8.210661in}}%
\pgfpathlineto{\pgfqpoint{2.574339in}{8.298396in}}%
\pgfusepath{stroke,fill}%
\end{pgfscope}%
\begin{pgfscope}%
\pgfpathrectangle{\pgfqpoint{0.380943in}{8.035189in}}{\pgfqpoint{4.650000in}{0.614151in}}%
\pgfusepath{clip}%
\pgfsetbuttcap%
\pgfsetroundjoin%
\definecolor{currentfill}{rgb}{0.963768,0.915433,0.717478}%
\pgfsetfillcolor{currentfill}%
\pgfsetlinewidth{0.250937pt}%
\definecolor{currentstroke}{rgb}{1.000000,1.000000,1.000000}%
\pgfsetstrokecolor{currentstroke}%
\pgfsetdash{}{0pt}%
\pgfpathmoveto{\pgfqpoint{2.662075in}{8.298396in}}%
\pgfpathlineto{\pgfqpoint{2.749811in}{8.298396in}}%
\pgfpathlineto{\pgfqpoint{2.749811in}{8.210661in}}%
\pgfpathlineto{\pgfqpoint{2.662075in}{8.210661in}}%
\pgfpathlineto{\pgfqpoint{2.662075in}{8.298396in}}%
\pgfusepath{stroke,fill}%
\end{pgfscope}%
\begin{pgfscope}%
\pgfpathrectangle{\pgfqpoint{0.380943in}{8.035189in}}{\pgfqpoint{4.650000in}{0.614151in}}%
\pgfusepath{clip}%
\pgfsetbuttcap%
\pgfsetroundjoin%
\definecolor{currentfill}{rgb}{0.985083,0.974641,0.792587}%
\pgfsetfillcolor{currentfill}%
\pgfsetlinewidth{0.250937pt}%
\definecolor{currentstroke}{rgb}{1.000000,1.000000,1.000000}%
\pgfsetstrokecolor{currentstroke}%
\pgfsetdash{}{0pt}%
\pgfpathmoveto{\pgfqpoint{2.749811in}{8.298396in}}%
\pgfpathlineto{\pgfqpoint{2.837547in}{8.298396in}}%
\pgfpathlineto{\pgfqpoint{2.837547in}{8.210661in}}%
\pgfpathlineto{\pgfqpoint{2.749811in}{8.210661in}}%
\pgfpathlineto{\pgfqpoint{2.749811in}{8.298396in}}%
\pgfusepath{stroke,fill}%
\end{pgfscope}%
\begin{pgfscope}%
\pgfpathrectangle{\pgfqpoint{0.380943in}{8.035189in}}{\pgfqpoint{4.650000in}{0.614151in}}%
\pgfusepath{clip}%
\pgfsetbuttcap%
\pgfsetroundjoin%
\definecolor{currentfill}{rgb}{0.963768,0.915433,0.717478}%
\pgfsetfillcolor{currentfill}%
\pgfsetlinewidth{0.250937pt}%
\definecolor{currentstroke}{rgb}{1.000000,1.000000,1.000000}%
\pgfsetstrokecolor{currentstroke}%
\pgfsetdash{}{0pt}%
\pgfpathmoveto{\pgfqpoint{2.837547in}{8.298396in}}%
\pgfpathlineto{\pgfqpoint{2.925283in}{8.298396in}}%
\pgfpathlineto{\pgfqpoint{2.925283in}{8.210661in}}%
\pgfpathlineto{\pgfqpoint{2.837547in}{8.210661in}}%
\pgfpathlineto{\pgfqpoint{2.837547in}{8.298396in}}%
\pgfusepath{stroke,fill}%
\end{pgfscope}%
\begin{pgfscope}%
\pgfpathrectangle{\pgfqpoint{0.380943in}{8.035189in}}{\pgfqpoint{4.650000in}{0.614151in}}%
\pgfusepath{clip}%
\pgfsetbuttcap%
\pgfsetroundjoin%
\definecolor{currentfill}{rgb}{0.978131,0.843783,0.675709}%
\pgfsetfillcolor{currentfill}%
\pgfsetlinewidth{0.250937pt}%
\definecolor{currentstroke}{rgb}{1.000000,1.000000,1.000000}%
\pgfsetstrokecolor{currentstroke}%
\pgfsetdash{}{0pt}%
\pgfpathmoveto{\pgfqpoint{2.925283in}{8.298396in}}%
\pgfpathlineto{\pgfqpoint{3.013019in}{8.298396in}}%
\pgfpathlineto{\pgfqpoint{3.013019in}{8.210661in}}%
\pgfpathlineto{\pgfqpoint{2.925283in}{8.210661in}}%
\pgfpathlineto{\pgfqpoint{2.925283in}{8.298396in}}%
\pgfusepath{stroke,fill}%
\end{pgfscope}%
\begin{pgfscope}%
\pgfpathrectangle{\pgfqpoint{0.380943in}{8.035189in}}{\pgfqpoint{4.650000in}{0.614151in}}%
\pgfusepath{clip}%
\pgfsetbuttcap%
\pgfsetroundjoin%
\definecolor{currentfill}{rgb}{0.961061,0.931672,0.728304}%
\pgfsetfillcolor{currentfill}%
\pgfsetlinewidth{0.250937pt}%
\definecolor{currentstroke}{rgb}{1.000000,1.000000,1.000000}%
\pgfsetstrokecolor{currentstroke}%
\pgfsetdash{}{0pt}%
\pgfpathmoveto{\pgfqpoint{3.013019in}{8.298396in}}%
\pgfpathlineto{\pgfqpoint{3.100754in}{8.298396in}}%
\pgfpathlineto{\pgfqpoint{3.100754in}{8.210661in}}%
\pgfpathlineto{\pgfqpoint{3.013019in}{8.210661in}}%
\pgfpathlineto{\pgfqpoint{3.013019in}{8.298396in}}%
\pgfusepath{stroke,fill}%
\end{pgfscope}%
\begin{pgfscope}%
\pgfpathrectangle{\pgfqpoint{0.380943in}{8.035189in}}{\pgfqpoint{4.650000in}{0.614151in}}%
\pgfusepath{clip}%
\pgfsetbuttcap%
\pgfsetroundjoin%
\definecolor{currentfill}{rgb}{0.961061,0.931672,0.728304}%
\pgfsetfillcolor{currentfill}%
\pgfsetlinewidth{0.250937pt}%
\definecolor{currentstroke}{rgb}{1.000000,1.000000,1.000000}%
\pgfsetstrokecolor{currentstroke}%
\pgfsetdash{}{0pt}%
\pgfpathmoveto{\pgfqpoint{3.100754in}{8.298396in}}%
\pgfpathlineto{\pgfqpoint{3.188490in}{8.298396in}}%
\pgfpathlineto{\pgfqpoint{3.188490in}{8.210661in}}%
\pgfpathlineto{\pgfqpoint{3.100754in}{8.210661in}}%
\pgfpathlineto{\pgfqpoint{3.100754in}{8.298396in}}%
\pgfusepath{stroke,fill}%
\end{pgfscope}%
\begin{pgfscope}%
\pgfpathrectangle{\pgfqpoint{0.380943in}{8.035189in}}{\pgfqpoint{4.650000in}{0.614151in}}%
\pgfusepath{clip}%
\pgfsetbuttcap%
\pgfsetroundjoin%
\definecolor{currentfill}{rgb}{0.978131,0.843783,0.675709}%
\pgfsetfillcolor{currentfill}%
\pgfsetlinewidth{0.250937pt}%
\definecolor{currentstroke}{rgb}{1.000000,1.000000,1.000000}%
\pgfsetstrokecolor{currentstroke}%
\pgfsetdash{}{0pt}%
\pgfpathmoveto{\pgfqpoint{3.188490in}{8.298396in}}%
\pgfpathlineto{\pgfqpoint{3.276226in}{8.298396in}}%
\pgfpathlineto{\pgfqpoint{3.276226in}{8.210661in}}%
\pgfpathlineto{\pgfqpoint{3.188490in}{8.210661in}}%
\pgfpathlineto{\pgfqpoint{3.188490in}{8.298396in}}%
\pgfusepath{stroke,fill}%
\end{pgfscope}%
\begin{pgfscope}%
\pgfpathrectangle{\pgfqpoint{0.380943in}{8.035189in}}{\pgfqpoint{4.650000in}{0.614151in}}%
\pgfusepath{clip}%
\pgfsetbuttcap%
\pgfsetroundjoin%
\definecolor{currentfill}{rgb}{0.986251,0.808597,0.643230}%
\pgfsetfillcolor{currentfill}%
\pgfsetlinewidth{0.250937pt}%
\definecolor{currentstroke}{rgb}{1.000000,1.000000,1.000000}%
\pgfsetstrokecolor{currentstroke}%
\pgfsetdash{}{0pt}%
\pgfpathmoveto{\pgfqpoint{3.276226in}{8.298396in}}%
\pgfpathlineto{\pgfqpoint{3.363962in}{8.298396in}}%
\pgfpathlineto{\pgfqpoint{3.363962in}{8.210661in}}%
\pgfpathlineto{\pgfqpoint{3.276226in}{8.210661in}}%
\pgfpathlineto{\pgfqpoint{3.276226in}{8.298396in}}%
\pgfusepath{stroke,fill}%
\end{pgfscope}%
\begin{pgfscope}%
\pgfpathrectangle{\pgfqpoint{0.380943in}{8.035189in}}{\pgfqpoint{4.650000in}{0.614151in}}%
\pgfusepath{clip}%
\pgfsetbuttcap%
\pgfsetroundjoin%
\definecolor{currentfill}{rgb}{0.961061,0.931672,0.728304}%
\pgfsetfillcolor{currentfill}%
\pgfsetlinewidth{0.250937pt}%
\definecolor{currentstroke}{rgb}{1.000000,1.000000,1.000000}%
\pgfsetstrokecolor{currentstroke}%
\pgfsetdash{}{0pt}%
\pgfpathmoveto{\pgfqpoint{3.363962in}{8.298396in}}%
\pgfpathlineto{\pgfqpoint{3.451698in}{8.298396in}}%
\pgfpathlineto{\pgfqpoint{3.451698in}{8.210661in}}%
\pgfpathlineto{\pgfqpoint{3.363962in}{8.210661in}}%
\pgfpathlineto{\pgfqpoint{3.363962in}{8.298396in}}%
\pgfusepath{stroke,fill}%
\end{pgfscope}%
\begin{pgfscope}%
\pgfpathrectangle{\pgfqpoint{0.380943in}{8.035189in}}{\pgfqpoint{4.650000in}{0.614151in}}%
\pgfusepath{clip}%
\pgfsetbuttcap%
\pgfsetroundjoin%
\definecolor{currentfill}{rgb}{0.961061,0.931672,0.728304}%
\pgfsetfillcolor{currentfill}%
\pgfsetlinewidth{0.250937pt}%
\definecolor{currentstroke}{rgb}{1.000000,1.000000,1.000000}%
\pgfsetstrokecolor{currentstroke}%
\pgfsetdash{}{0pt}%
\pgfpathmoveto{\pgfqpoint{3.451698in}{8.298396in}}%
\pgfpathlineto{\pgfqpoint{3.539434in}{8.298396in}}%
\pgfpathlineto{\pgfqpoint{3.539434in}{8.210661in}}%
\pgfpathlineto{\pgfqpoint{3.451698in}{8.210661in}}%
\pgfpathlineto{\pgfqpoint{3.451698in}{8.298396in}}%
\pgfusepath{stroke,fill}%
\end{pgfscope}%
\begin{pgfscope}%
\pgfpathrectangle{\pgfqpoint{0.380943in}{8.035189in}}{\pgfqpoint{4.650000in}{0.614151in}}%
\pgfusepath{clip}%
\pgfsetbuttcap%
\pgfsetroundjoin%
\definecolor{currentfill}{rgb}{1.000000,0.584929,0.522599}%
\pgfsetfillcolor{currentfill}%
\pgfsetlinewidth{0.250937pt}%
\definecolor{currentstroke}{rgb}{1.000000,1.000000,1.000000}%
\pgfsetstrokecolor{currentstroke}%
\pgfsetdash{}{0pt}%
\pgfpathmoveto{\pgfqpoint{3.539434in}{8.298396in}}%
\pgfpathlineto{\pgfqpoint{3.627169in}{8.298396in}}%
\pgfpathlineto{\pgfqpoint{3.627169in}{8.210661in}}%
\pgfpathlineto{\pgfqpoint{3.539434in}{8.210661in}}%
\pgfpathlineto{\pgfqpoint{3.539434in}{8.298396in}}%
\pgfusepath{stroke,fill}%
\end{pgfscope}%
\begin{pgfscope}%
\pgfpathrectangle{\pgfqpoint{0.380943in}{8.035189in}}{\pgfqpoint{4.650000in}{0.614151in}}%
\pgfusepath{clip}%
\pgfsetbuttcap%
\pgfsetroundjoin%
\definecolor{currentfill}{rgb}{0.800000,0.278431,0.278431}%
\pgfsetfillcolor{currentfill}%
\pgfsetlinewidth{0.250937pt}%
\definecolor{currentstroke}{rgb}{1.000000,1.000000,1.000000}%
\pgfsetstrokecolor{currentstroke}%
\pgfsetdash{}{0pt}%
\pgfpathmoveto{\pgfqpoint{3.627169in}{8.298396in}}%
\pgfpathlineto{\pgfqpoint{3.714905in}{8.298396in}}%
\pgfpathlineto{\pgfqpoint{3.714905in}{8.210661in}}%
\pgfpathlineto{\pgfqpoint{3.627169in}{8.210661in}}%
\pgfpathlineto{\pgfqpoint{3.627169in}{8.298396in}}%
\pgfusepath{stroke,fill}%
\end{pgfscope}%
\begin{pgfscope}%
\pgfpathrectangle{\pgfqpoint{0.380943in}{8.035189in}}{\pgfqpoint{4.650000in}{0.614151in}}%
\pgfusepath{clip}%
\pgfsetbuttcap%
\pgfsetroundjoin%
\definecolor{currentfill}{rgb}{0.970012,0.883276,0.699577}%
\pgfsetfillcolor{currentfill}%
\pgfsetlinewidth{0.250937pt}%
\definecolor{currentstroke}{rgb}{1.000000,1.000000,1.000000}%
\pgfsetstrokecolor{currentstroke}%
\pgfsetdash{}{0pt}%
\pgfpathmoveto{\pgfqpoint{3.714905in}{8.298396in}}%
\pgfpathlineto{\pgfqpoint{3.802641in}{8.298396in}}%
\pgfpathlineto{\pgfqpoint{3.802641in}{8.210661in}}%
\pgfpathlineto{\pgfqpoint{3.714905in}{8.210661in}}%
\pgfpathlineto{\pgfqpoint{3.714905in}{8.298396in}}%
\pgfusepath{stroke,fill}%
\end{pgfscope}%
\begin{pgfscope}%
\pgfpathrectangle{\pgfqpoint{0.380943in}{8.035189in}}{\pgfqpoint{4.650000in}{0.614151in}}%
\pgfusepath{clip}%
\pgfsetbuttcap%
\pgfsetroundjoin%
\definecolor{currentfill}{rgb}{0.970012,0.883276,0.699577}%
\pgfsetfillcolor{currentfill}%
\pgfsetlinewidth{0.250937pt}%
\definecolor{currentstroke}{rgb}{1.000000,1.000000,1.000000}%
\pgfsetstrokecolor{currentstroke}%
\pgfsetdash{}{0pt}%
\pgfpathmoveto{\pgfqpoint{3.802641in}{8.298396in}}%
\pgfpathlineto{\pgfqpoint{3.890377in}{8.298396in}}%
\pgfpathlineto{\pgfqpoint{3.890377in}{8.210661in}}%
\pgfpathlineto{\pgfqpoint{3.802641in}{8.210661in}}%
\pgfpathlineto{\pgfqpoint{3.802641in}{8.298396in}}%
\pgfusepath{stroke,fill}%
\end{pgfscope}%
\begin{pgfscope}%
\pgfpathrectangle{\pgfqpoint{0.380943in}{8.035189in}}{\pgfqpoint{4.650000in}{0.614151in}}%
\pgfusepath{clip}%
\pgfsetbuttcap%
\pgfsetroundjoin%
\definecolor{currentfill}{rgb}{0.986251,0.808597,0.643230}%
\pgfsetfillcolor{currentfill}%
\pgfsetlinewidth{0.250937pt}%
\definecolor{currentstroke}{rgb}{1.000000,1.000000,1.000000}%
\pgfsetstrokecolor{currentstroke}%
\pgfsetdash{}{0pt}%
\pgfpathmoveto{\pgfqpoint{3.890377in}{8.298396in}}%
\pgfpathlineto{\pgfqpoint{3.978113in}{8.298396in}}%
\pgfpathlineto{\pgfqpoint{3.978113in}{8.210661in}}%
\pgfpathlineto{\pgfqpoint{3.890377in}{8.210661in}}%
\pgfpathlineto{\pgfqpoint{3.890377in}{8.298396in}}%
\pgfusepath{stroke,fill}%
\end{pgfscope}%
\begin{pgfscope}%
\pgfpathrectangle{\pgfqpoint{0.380943in}{8.035189in}}{\pgfqpoint{4.650000in}{0.614151in}}%
\pgfusepath{clip}%
\pgfsetbuttcap%
\pgfsetroundjoin%
\definecolor{currentfill}{rgb}{0.986251,0.808597,0.643230}%
\pgfsetfillcolor{currentfill}%
\pgfsetlinewidth{0.250937pt}%
\definecolor{currentstroke}{rgb}{1.000000,1.000000,1.000000}%
\pgfsetstrokecolor{currentstroke}%
\pgfsetdash{}{0pt}%
\pgfpathmoveto{\pgfqpoint{3.978113in}{8.298396in}}%
\pgfpathlineto{\pgfqpoint{4.065849in}{8.298396in}}%
\pgfpathlineto{\pgfqpoint{4.065849in}{8.210661in}}%
\pgfpathlineto{\pgfqpoint{3.978113in}{8.210661in}}%
\pgfpathlineto{\pgfqpoint{3.978113in}{8.298396in}}%
\pgfusepath{stroke,fill}%
\end{pgfscope}%
\begin{pgfscope}%
\pgfpathrectangle{\pgfqpoint{0.380943in}{8.035189in}}{\pgfqpoint{4.650000in}{0.614151in}}%
\pgfusepath{clip}%
\pgfsetbuttcap%
\pgfsetroundjoin%
\definecolor{currentfill}{rgb}{0.970012,0.883276,0.699577}%
\pgfsetfillcolor{currentfill}%
\pgfsetlinewidth{0.250937pt}%
\definecolor{currentstroke}{rgb}{1.000000,1.000000,1.000000}%
\pgfsetstrokecolor{currentstroke}%
\pgfsetdash{}{0pt}%
\pgfpathmoveto{\pgfqpoint{4.065849in}{8.298396in}}%
\pgfpathlineto{\pgfqpoint{4.153585in}{8.298396in}}%
\pgfpathlineto{\pgfqpoint{4.153585in}{8.210661in}}%
\pgfpathlineto{\pgfqpoint{4.065849in}{8.210661in}}%
\pgfpathlineto{\pgfqpoint{4.065849in}{8.298396in}}%
\pgfusepath{stroke,fill}%
\end{pgfscope}%
\begin{pgfscope}%
\pgfpathrectangle{\pgfqpoint{0.380943in}{8.035189in}}{\pgfqpoint{4.650000in}{0.614151in}}%
\pgfusepath{clip}%
\pgfsetbuttcap%
\pgfsetroundjoin%
\definecolor{currentfill}{rgb}{0.996909,0.711742,0.584452}%
\pgfsetfillcolor{currentfill}%
\pgfsetlinewidth{0.250937pt}%
\definecolor{currentstroke}{rgb}{1.000000,1.000000,1.000000}%
\pgfsetstrokecolor{currentstroke}%
\pgfsetdash{}{0pt}%
\pgfpathmoveto{\pgfqpoint{4.153585in}{8.298396in}}%
\pgfpathlineto{\pgfqpoint{4.241320in}{8.298396in}}%
\pgfpathlineto{\pgfqpoint{4.241320in}{8.210661in}}%
\pgfpathlineto{\pgfqpoint{4.153585in}{8.210661in}}%
\pgfpathlineto{\pgfqpoint{4.153585in}{8.298396in}}%
\pgfusepath{stroke,fill}%
\end{pgfscope}%
\begin{pgfscope}%
\pgfpathrectangle{\pgfqpoint{0.380943in}{8.035189in}}{\pgfqpoint{4.650000in}{0.614151in}}%
\pgfusepath{clip}%
\pgfsetbuttcap%
\pgfsetroundjoin%
\definecolor{currentfill}{rgb}{0.978131,0.843783,0.675709}%
\pgfsetfillcolor{currentfill}%
\pgfsetlinewidth{0.250937pt}%
\definecolor{currentstroke}{rgb}{1.000000,1.000000,1.000000}%
\pgfsetstrokecolor{currentstroke}%
\pgfsetdash{}{0pt}%
\pgfpathmoveto{\pgfqpoint{4.241320in}{8.298396in}}%
\pgfpathlineto{\pgfqpoint{4.329056in}{8.298396in}}%
\pgfpathlineto{\pgfqpoint{4.329056in}{8.210661in}}%
\pgfpathlineto{\pgfqpoint{4.241320in}{8.210661in}}%
\pgfpathlineto{\pgfqpoint{4.241320in}{8.298396in}}%
\pgfusepath{stroke,fill}%
\end{pgfscope}%
\begin{pgfscope}%
\pgfpathrectangle{\pgfqpoint{0.380943in}{8.035189in}}{\pgfqpoint{4.650000in}{0.614151in}}%
\pgfusepath{clip}%
\pgfsetbuttcap%
\pgfsetroundjoin%
\definecolor{currentfill}{rgb}{0.970012,0.883276,0.699577}%
\pgfsetfillcolor{currentfill}%
\pgfsetlinewidth{0.250937pt}%
\definecolor{currentstroke}{rgb}{1.000000,1.000000,1.000000}%
\pgfsetstrokecolor{currentstroke}%
\pgfsetdash{}{0pt}%
\pgfpathmoveto{\pgfqpoint{4.329056in}{8.298396in}}%
\pgfpathlineto{\pgfqpoint{4.416792in}{8.298396in}}%
\pgfpathlineto{\pgfqpoint{4.416792in}{8.210661in}}%
\pgfpathlineto{\pgfqpoint{4.329056in}{8.210661in}}%
\pgfpathlineto{\pgfqpoint{4.329056in}{8.298396in}}%
\pgfusepath{stroke,fill}%
\end{pgfscope}%
\begin{pgfscope}%
\pgfpathrectangle{\pgfqpoint{0.380943in}{8.035189in}}{\pgfqpoint{4.650000in}{0.614151in}}%
\pgfusepath{clip}%
\pgfsetbuttcap%
\pgfsetroundjoin%
\definecolor{currentfill}{rgb}{0.963768,0.915433,0.717478}%
\pgfsetfillcolor{currentfill}%
\pgfsetlinewidth{0.250937pt}%
\definecolor{currentstroke}{rgb}{1.000000,1.000000,1.000000}%
\pgfsetstrokecolor{currentstroke}%
\pgfsetdash{}{0pt}%
\pgfpathmoveto{\pgfqpoint{4.416792in}{8.298396in}}%
\pgfpathlineto{\pgfqpoint{4.504528in}{8.298396in}}%
\pgfpathlineto{\pgfqpoint{4.504528in}{8.210661in}}%
\pgfpathlineto{\pgfqpoint{4.416792in}{8.210661in}}%
\pgfpathlineto{\pgfqpoint{4.416792in}{8.298396in}}%
\pgfusepath{stroke,fill}%
\end{pgfscope}%
\begin{pgfscope}%
\pgfpathrectangle{\pgfqpoint{0.380943in}{8.035189in}}{\pgfqpoint{4.650000in}{0.614151in}}%
\pgfusepath{clip}%
\pgfsetbuttcap%
\pgfsetroundjoin%
\definecolor{currentfill}{rgb}{0.999616,0.641369,0.546559}%
\pgfsetfillcolor{currentfill}%
\pgfsetlinewidth{0.250937pt}%
\definecolor{currentstroke}{rgb}{1.000000,1.000000,1.000000}%
\pgfsetstrokecolor{currentstroke}%
\pgfsetdash{}{0pt}%
\pgfpathmoveto{\pgfqpoint{4.504528in}{8.298396in}}%
\pgfpathlineto{\pgfqpoint{4.592264in}{8.298396in}}%
\pgfpathlineto{\pgfqpoint{4.592264in}{8.210661in}}%
\pgfpathlineto{\pgfqpoint{4.504528in}{8.210661in}}%
\pgfpathlineto{\pgfqpoint{4.504528in}{8.298396in}}%
\pgfusepath{stroke,fill}%
\end{pgfscope}%
\begin{pgfscope}%
\pgfpathrectangle{\pgfqpoint{0.380943in}{8.035189in}}{\pgfqpoint{4.650000in}{0.614151in}}%
\pgfusepath{clip}%
\pgfsetbuttcap%
\pgfsetroundjoin%
\definecolor{currentfill}{rgb}{0.986251,0.808597,0.643230}%
\pgfsetfillcolor{currentfill}%
\pgfsetlinewidth{0.250937pt}%
\definecolor{currentstroke}{rgb}{1.000000,1.000000,1.000000}%
\pgfsetstrokecolor{currentstroke}%
\pgfsetdash{}{0pt}%
\pgfpathmoveto{\pgfqpoint{4.592264in}{8.298396in}}%
\pgfpathlineto{\pgfqpoint{4.680000in}{8.298396in}}%
\pgfpathlineto{\pgfqpoint{4.680000in}{8.210661in}}%
\pgfpathlineto{\pgfqpoint{4.592264in}{8.210661in}}%
\pgfpathlineto{\pgfqpoint{4.592264in}{8.298396in}}%
\pgfusepath{stroke,fill}%
\end{pgfscope}%
\begin{pgfscope}%
\pgfpathrectangle{\pgfqpoint{0.380943in}{8.035189in}}{\pgfqpoint{4.650000in}{0.614151in}}%
\pgfusepath{clip}%
\pgfsetbuttcap%
\pgfsetroundjoin%
\definecolor{currentfill}{rgb}{0.985083,0.974641,0.792587}%
\pgfsetfillcolor{currentfill}%
\pgfsetlinewidth{0.250937pt}%
\definecolor{currentstroke}{rgb}{1.000000,1.000000,1.000000}%
\pgfsetstrokecolor{currentstroke}%
\pgfsetdash{}{0pt}%
\pgfpathmoveto{\pgfqpoint{4.680000in}{8.298396in}}%
\pgfpathlineto{\pgfqpoint{4.767736in}{8.298396in}}%
\pgfpathlineto{\pgfqpoint{4.767736in}{8.210661in}}%
\pgfpathlineto{\pgfqpoint{4.680000in}{8.210661in}}%
\pgfpathlineto{\pgfqpoint{4.680000in}{8.298396in}}%
\pgfusepath{stroke,fill}%
\end{pgfscope}%
\begin{pgfscope}%
\pgfpathrectangle{\pgfqpoint{0.380943in}{8.035189in}}{\pgfqpoint{4.650000in}{0.614151in}}%
\pgfusepath{clip}%
\pgfsetbuttcap%
\pgfsetroundjoin%
\definecolor{currentfill}{rgb}{0.986251,0.808597,0.643230}%
\pgfsetfillcolor{currentfill}%
\pgfsetlinewidth{0.250937pt}%
\definecolor{currentstroke}{rgb}{1.000000,1.000000,1.000000}%
\pgfsetstrokecolor{currentstroke}%
\pgfsetdash{}{0pt}%
\pgfpathmoveto{\pgfqpoint{4.767736in}{8.298396in}}%
\pgfpathlineto{\pgfqpoint{4.855471in}{8.298396in}}%
\pgfpathlineto{\pgfqpoint{4.855471in}{8.210661in}}%
\pgfpathlineto{\pgfqpoint{4.767736in}{8.210661in}}%
\pgfpathlineto{\pgfqpoint{4.767736in}{8.298396in}}%
\pgfusepath{stroke,fill}%
\end{pgfscope}%
\begin{pgfscope}%
\pgfpathrectangle{\pgfqpoint{0.380943in}{8.035189in}}{\pgfqpoint{4.650000in}{0.614151in}}%
\pgfusepath{clip}%
\pgfsetbuttcap%
\pgfsetroundjoin%
\definecolor{currentfill}{rgb}{1.000000,0.584929,0.522599}%
\pgfsetfillcolor{currentfill}%
\pgfsetlinewidth{0.250937pt}%
\definecolor{currentstroke}{rgb}{1.000000,1.000000,1.000000}%
\pgfsetstrokecolor{currentstroke}%
\pgfsetdash{}{0pt}%
\pgfpathmoveto{\pgfqpoint{4.855471in}{8.298396in}}%
\pgfpathlineto{\pgfqpoint{4.943207in}{8.298396in}}%
\pgfpathlineto{\pgfqpoint{4.943207in}{8.210661in}}%
\pgfpathlineto{\pgfqpoint{4.855471in}{8.210661in}}%
\pgfpathlineto{\pgfqpoint{4.855471in}{8.298396in}}%
\pgfusepath{stroke,fill}%
\end{pgfscope}%
\begin{pgfscope}%
\pgfpathrectangle{\pgfqpoint{0.380943in}{8.035189in}}{\pgfqpoint{4.650000in}{0.614151in}}%
\pgfusepath{clip}%
\pgfsetbuttcap%
\pgfsetroundjoin%
\definecolor{currentfill}{rgb}{0.978131,0.843783,0.675709}%
\pgfsetfillcolor{currentfill}%
\pgfsetlinewidth{0.250937pt}%
\definecolor{currentstroke}{rgb}{1.000000,1.000000,1.000000}%
\pgfsetstrokecolor{currentstroke}%
\pgfsetdash{}{0pt}%
\pgfpathmoveto{\pgfqpoint{4.943207in}{8.298396in}}%
\pgfpathlineto{\pgfqpoint{5.030943in}{8.298396in}}%
\pgfpathlineto{\pgfqpoint{5.030943in}{8.210661in}}%
\pgfpathlineto{\pgfqpoint{4.943207in}{8.210661in}}%
\pgfpathlineto{\pgfqpoint{4.943207in}{8.298396in}}%
\pgfusepath{stroke,fill}%
\end{pgfscope}%
\begin{pgfscope}%
\pgfpathrectangle{\pgfqpoint{0.380943in}{8.035189in}}{\pgfqpoint{4.650000in}{0.614151in}}%
\pgfusepath{clip}%
\pgfsetbuttcap%
\pgfsetroundjoin%
\pgfsetlinewidth{0.250937pt}%
\definecolor{currentstroke}{rgb}{1.000000,1.000000,1.000000}%
\pgfsetstrokecolor{currentstroke}%
\pgfsetdash{}{0pt}%
\pgfpathmoveto{\pgfqpoint{0.380943in}{8.210661in}}%
\pgfpathlineto{\pgfqpoint{0.468679in}{8.210661in}}%
\pgfpathlineto{\pgfqpoint{0.468679in}{8.122925in}}%
\pgfpathlineto{\pgfqpoint{0.380943in}{8.122925in}}%
\pgfpathlineto{\pgfqpoint{0.380943in}{8.210661in}}%
\pgfusepath{stroke}%
\end{pgfscope}%
\begin{pgfscope}%
\pgfpathrectangle{\pgfqpoint{0.380943in}{8.035189in}}{\pgfqpoint{4.650000in}{0.614151in}}%
\pgfusepath{clip}%
\pgfsetbuttcap%
\pgfsetroundjoin%
\definecolor{currentfill}{rgb}{0.961061,0.931672,0.728304}%
\pgfsetfillcolor{currentfill}%
\pgfsetlinewidth{0.250937pt}%
\definecolor{currentstroke}{rgb}{1.000000,1.000000,1.000000}%
\pgfsetstrokecolor{currentstroke}%
\pgfsetdash{}{0pt}%
\pgfpathmoveto{\pgfqpoint{0.468679in}{8.210661in}}%
\pgfpathlineto{\pgfqpoint{0.556415in}{8.210661in}}%
\pgfpathlineto{\pgfqpoint{0.556415in}{8.122925in}}%
\pgfpathlineto{\pgfqpoint{0.468679in}{8.122925in}}%
\pgfpathlineto{\pgfqpoint{0.468679in}{8.210661in}}%
\pgfusepath{stroke,fill}%
\end{pgfscope}%
\begin{pgfscope}%
\pgfpathrectangle{\pgfqpoint{0.380943in}{8.035189in}}{\pgfqpoint{4.650000in}{0.614151in}}%
\pgfusepath{clip}%
\pgfsetbuttcap%
\pgfsetroundjoin%
\definecolor{currentfill}{rgb}{0.978131,0.843783,0.675709}%
\pgfsetfillcolor{currentfill}%
\pgfsetlinewidth{0.250937pt}%
\definecolor{currentstroke}{rgb}{1.000000,1.000000,1.000000}%
\pgfsetstrokecolor{currentstroke}%
\pgfsetdash{}{0pt}%
\pgfpathmoveto{\pgfqpoint{0.556415in}{8.210661in}}%
\pgfpathlineto{\pgfqpoint{0.644151in}{8.210661in}}%
\pgfpathlineto{\pgfqpoint{0.644151in}{8.122925in}}%
\pgfpathlineto{\pgfqpoint{0.556415in}{8.122925in}}%
\pgfpathlineto{\pgfqpoint{0.556415in}{8.210661in}}%
\pgfusepath{stroke,fill}%
\end{pgfscope}%
\begin{pgfscope}%
\pgfpathrectangle{\pgfqpoint{0.380943in}{8.035189in}}{\pgfqpoint{4.650000in}{0.614151in}}%
\pgfusepath{clip}%
\pgfsetbuttcap%
\pgfsetroundjoin%
\definecolor{currentfill}{rgb}{0.996909,0.711742,0.584452}%
\pgfsetfillcolor{currentfill}%
\pgfsetlinewidth{0.250937pt}%
\definecolor{currentstroke}{rgb}{1.000000,1.000000,1.000000}%
\pgfsetstrokecolor{currentstroke}%
\pgfsetdash{}{0pt}%
\pgfpathmoveto{\pgfqpoint{0.644151in}{8.210661in}}%
\pgfpathlineto{\pgfqpoint{0.731886in}{8.210661in}}%
\pgfpathlineto{\pgfqpoint{0.731886in}{8.122925in}}%
\pgfpathlineto{\pgfqpoint{0.644151in}{8.122925in}}%
\pgfpathlineto{\pgfqpoint{0.644151in}{8.210661in}}%
\pgfusepath{stroke,fill}%
\end{pgfscope}%
\begin{pgfscope}%
\pgfpathrectangle{\pgfqpoint{0.380943in}{8.035189in}}{\pgfqpoint{4.650000in}{0.614151in}}%
\pgfusepath{clip}%
\pgfsetbuttcap%
\pgfsetroundjoin%
\definecolor{currentfill}{rgb}{0.978131,0.843783,0.675709}%
\pgfsetfillcolor{currentfill}%
\pgfsetlinewidth{0.250937pt}%
\definecolor{currentstroke}{rgb}{1.000000,1.000000,1.000000}%
\pgfsetstrokecolor{currentstroke}%
\pgfsetdash{}{0pt}%
\pgfpathmoveto{\pgfqpoint{0.731886in}{8.210661in}}%
\pgfpathlineto{\pgfqpoint{0.819622in}{8.210661in}}%
\pgfpathlineto{\pgfqpoint{0.819622in}{8.122925in}}%
\pgfpathlineto{\pgfqpoint{0.731886in}{8.122925in}}%
\pgfpathlineto{\pgfqpoint{0.731886in}{8.210661in}}%
\pgfusepath{stroke,fill}%
\end{pgfscope}%
\begin{pgfscope}%
\pgfpathrectangle{\pgfqpoint{0.380943in}{8.035189in}}{\pgfqpoint{4.650000in}{0.614151in}}%
\pgfusepath{clip}%
\pgfsetbuttcap%
\pgfsetroundjoin%
\definecolor{currentfill}{rgb}{0.978131,0.843783,0.675709}%
\pgfsetfillcolor{currentfill}%
\pgfsetlinewidth{0.250937pt}%
\definecolor{currentstroke}{rgb}{1.000000,1.000000,1.000000}%
\pgfsetstrokecolor{currentstroke}%
\pgfsetdash{}{0pt}%
\pgfpathmoveto{\pgfqpoint{0.819622in}{8.210661in}}%
\pgfpathlineto{\pgfqpoint{0.907358in}{8.210661in}}%
\pgfpathlineto{\pgfqpoint{0.907358in}{8.122925in}}%
\pgfpathlineto{\pgfqpoint{0.819622in}{8.122925in}}%
\pgfpathlineto{\pgfqpoint{0.819622in}{8.210661in}}%
\pgfusepath{stroke,fill}%
\end{pgfscope}%
\begin{pgfscope}%
\pgfpathrectangle{\pgfqpoint{0.380943in}{8.035189in}}{\pgfqpoint{4.650000in}{0.614151in}}%
\pgfusepath{clip}%
\pgfsetbuttcap%
\pgfsetroundjoin%
\definecolor{currentfill}{rgb}{0.985083,0.974641,0.792587}%
\pgfsetfillcolor{currentfill}%
\pgfsetlinewidth{0.250937pt}%
\definecolor{currentstroke}{rgb}{1.000000,1.000000,1.000000}%
\pgfsetstrokecolor{currentstroke}%
\pgfsetdash{}{0pt}%
\pgfpathmoveto{\pgfqpoint{0.907358in}{8.210661in}}%
\pgfpathlineto{\pgfqpoint{0.995094in}{8.210661in}}%
\pgfpathlineto{\pgfqpoint{0.995094in}{8.122925in}}%
\pgfpathlineto{\pgfqpoint{0.907358in}{8.122925in}}%
\pgfpathlineto{\pgfqpoint{0.907358in}{8.210661in}}%
\pgfusepath{stroke,fill}%
\end{pgfscope}%
\begin{pgfscope}%
\pgfpathrectangle{\pgfqpoint{0.380943in}{8.035189in}}{\pgfqpoint{4.650000in}{0.614151in}}%
\pgfusepath{clip}%
\pgfsetbuttcap%
\pgfsetroundjoin%
\definecolor{currentfill}{rgb}{0.985083,0.974641,0.792587}%
\pgfsetfillcolor{currentfill}%
\pgfsetlinewidth{0.250937pt}%
\definecolor{currentstroke}{rgb}{1.000000,1.000000,1.000000}%
\pgfsetstrokecolor{currentstroke}%
\pgfsetdash{}{0pt}%
\pgfpathmoveto{\pgfqpoint{0.995094in}{8.210661in}}%
\pgfpathlineto{\pgfqpoint{1.082830in}{8.210661in}}%
\pgfpathlineto{\pgfqpoint{1.082830in}{8.122925in}}%
\pgfpathlineto{\pgfqpoint{0.995094in}{8.122925in}}%
\pgfpathlineto{\pgfqpoint{0.995094in}{8.210661in}}%
\pgfusepath{stroke,fill}%
\end{pgfscope}%
\begin{pgfscope}%
\pgfpathrectangle{\pgfqpoint{0.380943in}{8.035189in}}{\pgfqpoint{4.650000in}{0.614151in}}%
\pgfusepath{clip}%
\pgfsetbuttcap%
\pgfsetroundjoin%
\definecolor{currentfill}{rgb}{0.961061,0.931672,0.728304}%
\pgfsetfillcolor{currentfill}%
\pgfsetlinewidth{0.250937pt}%
\definecolor{currentstroke}{rgb}{1.000000,1.000000,1.000000}%
\pgfsetstrokecolor{currentstroke}%
\pgfsetdash{}{0pt}%
\pgfpathmoveto{\pgfqpoint{1.082830in}{8.210661in}}%
\pgfpathlineto{\pgfqpoint{1.170566in}{8.210661in}}%
\pgfpathlineto{\pgfqpoint{1.170566in}{8.122925in}}%
\pgfpathlineto{\pgfqpoint{1.082830in}{8.122925in}}%
\pgfpathlineto{\pgfqpoint{1.082830in}{8.210661in}}%
\pgfusepath{stroke,fill}%
\end{pgfscope}%
\begin{pgfscope}%
\pgfpathrectangle{\pgfqpoint{0.380943in}{8.035189in}}{\pgfqpoint{4.650000in}{0.614151in}}%
\pgfusepath{clip}%
\pgfsetbuttcap%
\pgfsetroundjoin%
\definecolor{currentfill}{rgb}{0.978131,0.843783,0.675709}%
\pgfsetfillcolor{currentfill}%
\pgfsetlinewidth{0.250937pt}%
\definecolor{currentstroke}{rgb}{1.000000,1.000000,1.000000}%
\pgfsetstrokecolor{currentstroke}%
\pgfsetdash{}{0pt}%
\pgfpathmoveto{\pgfqpoint{1.170566in}{8.210661in}}%
\pgfpathlineto{\pgfqpoint{1.258302in}{8.210661in}}%
\pgfpathlineto{\pgfqpoint{1.258302in}{8.122925in}}%
\pgfpathlineto{\pgfqpoint{1.170566in}{8.122925in}}%
\pgfpathlineto{\pgfqpoint{1.170566in}{8.210661in}}%
\pgfusepath{stroke,fill}%
\end{pgfscope}%
\begin{pgfscope}%
\pgfpathrectangle{\pgfqpoint{0.380943in}{8.035189in}}{\pgfqpoint{4.650000in}{0.614151in}}%
\pgfusepath{clip}%
\pgfsetbuttcap%
\pgfsetroundjoin%
\definecolor{currentfill}{rgb}{0.978131,0.843783,0.675709}%
\pgfsetfillcolor{currentfill}%
\pgfsetlinewidth{0.250937pt}%
\definecolor{currentstroke}{rgb}{1.000000,1.000000,1.000000}%
\pgfsetstrokecolor{currentstroke}%
\pgfsetdash{}{0pt}%
\pgfpathmoveto{\pgfqpoint{1.258302in}{8.210661in}}%
\pgfpathlineto{\pgfqpoint{1.346037in}{8.210661in}}%
\pgfpathlineto{\pgfqpoint{1.346037in}{8.122925in}}%
\pgfpathlineto{\pgfqpoint{1.258302in}{8.122925in}}%
\pgfpathlineto{\pgfqpoint{1.258302in}{8.210661in}}%
\pgfusepath{stroke,fill}%
\end{pgfscope}%
\begin{pgfscope}%
\pgfpathrectangle{\pgfqpoint{0.380943in}{8.035189in}}{\pgfqpoint{4.650000in}{0.614151in}}%
\pgfusepath{clip}%
\pgfsetbuttcap%
\pgfsetroundjoin%
\definecolor{currentfill}{rgb}{0.961061,0.931672,0.728304}%
\pgfsetfillcolor{currentfill}%
\pgfsetlinewidth{0.250937pt}%
\definecolor{currentstroke}{rgb}{1.000000,1.000000,1.000000}%
\pgfsetstrokecolor{currentstroke}%
\pgfsetdash{}{0pt}%
\pgfpathmoveto{\pgfqpoint{1.346037in}{8.210661in}}%
\pgfpathlineto{\pgfqpoint{1.433773in}{8.210661in}}%
\pgfpathlineto{\pgfqpoint{1.433773in}{8.122925in}}%
\pgfpathlineto{\pgfqpoint{1.346037in}{8.122925in}}%
\pgfpathlineto{\pgfqpoint{1.346037in}{8.210661in}}%
\pgfusepath{stroke,fill}%
\end{pgfscope}%
\begin{pgfscope}%
\pgfpathrectangle{\pgfqpoint{0.380943in}{8.035189in}}{\pgfqpoint{4.650000in}{0.614151in}}%
\pgfusepath{clip}%
\pgfsetbuttcap%
\pgfsetroundjoin%
\definecolor{currentfill}{rgb}{0.985083,0.974641,0.792587}%
\pgfsetfillcolor{currentfill}%
\pgfsetlinewidth{0.250937pt}%
\definecolor{currentstroke}{rgb}{1.000000,1.000000,1.000000}%
\pgfsetstrokecolor{currentstroke}%
\pgfsetdash{}{0pt}%
\pgfpathmoveto{\pgfqpoint{1.433773in}{8.210661in}}%
\pgfpathlineto{\pgfqpoint{1.521509in}{8.210661in}}%
\pgfpathlineto{\pgfqpoint{1.521509in}{8.122925in}}%
\pgfpathlineto{\pgfqpoint{1.433773in}{8.122925in}}%
\pgfpathlineto{\pgfqpoint{1.433773in}{8.210661in}}%
\pgfusepath{stroke,fill}%
\end{pgfscope}%
\begin{pgfscope}%
\pgfpathrectangle{\pgfqpoint{0.380943in}{8.035189in}}{\pgfqpoint{4.650000in}{0.614151in}}%
\pgfusepath{clip}%
\pgfsetbuttcap%
\pgfsetroundjoin%
\definecolor{currentfill}{rgb}{1.000000,1.000000,0.861745}%
\pgfsetfillcolor{currentfill}%
\pgfsetlinewidth{0.250937pt}%
\definecolor{currentstroke}{rgb}{1.000000,1.000000,1.000000}%
\pgfsetstrokecolor{currentstroke}%
\pgfsetdash{}{0pt}%
\pgfpathmoveto{\pgfqpoint{1.521509in}{8.210661in}}%
\pgfpathlineto{\pgfqpoint{1.609245in}{8.210661in}}%
\pgfpathlineto{\pgfqpoint{1.609245in}{8.122925in}}%
\pgfpathlineto{\pgfqpoint{1.521509in}{8.122925in}}%
\pgfpathlineto{\pgfqpoint{1.521509in}{8.210661in}}%
\pgfusepath{stroke,fill}%
\end{pgfscope}%
\begin{pgfscope}%
\pgfpathrectangle{\pgfqpoint{0.380943in}{8.035189in}}{\pgfqpoint{4.650000in}{0.614151in}}%
\pgfusepath{clip}%
\pgfsetbuttcap%
\pgfsetroundjoin%
\definecolor{currentfill}{rgb}{1.000000,1.000000,0.861745}%
\pgfsetfillcolor{currentfill}%
\pgfsetlinewidth{0.250937pt}%
\definecolor{currentstroke}{rgb}{1.000000,1.000000,1.000000}%
\pgfsetstrokecolor{currentstroke}%
\pgfsetdash{}{0pt}%
\pgfpathmoveto{\pgfqpoint{1.609245in}{8.210661in}}%
\pgfpathlineto{\pgfqpoint{1.696981in}{8.210661in}}%
\pgfpathlineto{\pgfqpoint{1.696981in}{8.122925in}}%
\pgfpathlineto{\pgfqpoint{1.609245in}{8.122925in}}%
\pgfpathlineto{\pgfqpoint{1.609245in}{8.210661in}}%
\pgfusepath{stroke,fill}%
\end{pgfscope}%
\begin{pgfscope}%
\pgfpathrectangle{\pgfqpoint{0.380943in}{8.035189in}}{\pgfqpoint{4.650000in}{0.614151in}}%
\pgfusepath{clip}%
\pgfsetbuttcap%
\pgfsetroundjoin%
\definecolor{currentfill}{rgb}{0.985083,0.974641,0.792587}%
\pgfsetfillcolor{currentfill}%
\pgfsetlinewidth{0.250937pt}%
\definecolor{currentstroke}{rgb}{1.000000,1.000000,1.000000}%
\pgfsetstrokecolor{currentstroke}%
\pgfsetdash{}{0pt}%
\pgfpathmoveto{\pgfqpoint{1.696981in}{8.210661in}}%
\pgfpathlineto{\pgfqpoint{1.784717in}{8.210661in}}%
\pgfpathlineto{\pgfqpoint{1.784717in}{8.122925in}}%
\pgfpathlineto{\pgfqpoint{1.696981in}{8.122925in}}%
\pgfpathlineto{\pgfqpoint{1.696981in}{8.210661in}}%
\pgfusepath{stroke,fill}%
\end{pgfscope}%
\begin{pgfscope}%
\pgfpathrectangle{\pgfqpoint{0.380943in}{8.035189in}}{\pgfqpoint{4.650000in}{0.614151in}}%
\pgfusepath{clip}%
\pgfsetbuttcap%
\pgfsetroundjoin%
\definecolor{currentfill}{rgb}{0.985083,0.974641,0.792587}%
\pgfsetfillcolor{currentfill}%
\pgfsetlinewidth{0.250937pt}%
\definecolor{currentstroke}{rgb}{1.000000,1.000000,1.000000}%
\pgfsetstrokecolor{currentstroke}%
\pgfsetdash{}{0pt}%
\pgfpathmoveto{\pgfqpoint{1.784717in}{8.210661in}}%
\pgfpathlineto{\pgfqpoint{1.872452in}{8.210661in}}%
\pgfpathlineto{\pgfqpoint{1.872452in}{8.122925in}}%
\pgfpathlineto{\pgfqpoint{1.784717in}{8.122925in}}%
\pgfpathlineto{\pgfqpoint{1.784717in}{8.210661in}}%
\pgfusepath{stroke,fill}%
\end{pgfscope}%
\begin{pgfscope}%
\pgfpathrectangle{\pgfqpoint{0.380943in}{8.035189in}}{\pgfqpoint{4.650000in}{0.614151in}}%
\pgfusepath{clip}%
\pgfsetbuttcap%
\pgfsetroundjoin%
\definecolor{currentfill}{rgb}{0.978131,0.843783,0.675709}%
\pgfsetfillcolor{currentfill}%
\pgfsetlinewidth{0.250937pt}%
\definecolor{currentstroke}{rgb}{1.000000,1.000000,1.000000}%
\pgfsetstrokecolor{currentstroke}%
\pgfsetdash{}{0pt}%
\pgfpathmoveto{\pgfqpoint{1.872452in}{8.210661in}}%
\pgfpathlineto{\pgfqpoint{1.960188in}{8.210661in}}%
\pgfpathlineto{\pgfqpoint{1.960188in}{8.122925in}}%
\pgfpathlineto{\pgfqpoint{1.872452in}{8.122925in}}%
\pgfpathlineto{\pgfqpoint{1.872452in}{8.210661in}}%
\pgfusepath{stroke,fill}%
\end{pgfscope}%
\begin{pgfscope}%
\pgfpathrectangle{\pgfqpoint{0.380943in}{8.035189in}}{\pgfqpoint{4.650000in}{0.614151in}}%
\pgfusepath{clip}%
\pgfsetbuttcap%
\pgfsetroundjoin%
\definecolor{currentfill}{rgb}{0.985083,0.974641,0.792587}%
\pgfsetfillcolor{currentfill}%
\pgfsetlinewidth{0.250937pt}%
\definecolor{currentstroke}{rgb}{1.000000,1.000000,1.000000}%
\pgfsetstrokecolor{currentstroke}%
\pgfsetdash{}{0pt}%
\pgfpathmoveto{\pgfqpoint{1.960188in}{8.210661in}}%
\pgfpathlineto{\pgfqpoint{2.047924in}{8.210661in}}%
\pgfpathlineto{\pgfqpoint{2.047924in}{8.122925in}}%
\pgfpathlineto{\pgfqpoint{1.960188in}{8.122925in}}%
\pgfpathlineto{\pgfqpoint{1.960188in}{8.210661in}}%
\pgfusepath{stroke,fill}%
\end{pgfscope}%
\begin{pgfscope}%
\pgfpathrectangle{\pgfqpoint{0.380943in}{8.035189in}}{\pgfqpoint{4.650000in}{0.614151in}}%
\pgfusepath{clip}%
\pgfsetbuttcap%
\pgfsetroundjoin%
\definecolor{currentfill}{rgb}{0.963768,0.915433,0.717478}%
\pgfsetfillcolor{currentfill}%
\pgfsetlinewidth{0.250937pt}%
\definecolor{currentstroke}{rgb}{1.000000,1.000000,1.000000}%
\pgfsetstrokecolor{currentstroke}%
\pgfsetdash{}{0pt}%
\pgfpathmoveto{\pgfqpoint{2.047924in}{8.210661in}}%
\pgfpathlineto{\pgfqpoint{2.135660in}{8.210661in}}%
\pgfpathlineto{\pgfqpoint{2.135660in}{8.122925in}}%
\pgfpathlineto{\pgfqpoint{2.047924in}{8.122925in}}%
\pgfpathlineto{\pgfqpoint{2.047924in}{8.210661in}}%
\pgfusepath{stroke,fill}%
\end{pgfscope}%
\begin{pgfscope}%
\pgfpathrectangle{\pgfqpoint{0.380943in}{8.035189in}}{\pgfqpoint{4.650000in}{0.614151in}}%
\pgfusepath{clip}%
\pgfsetbuttcap%
\pgfsetroundjoin%
\definecolor{currentfill}{rgb}{0.970012,0.883276,0.699577}%
\pgfsetfillcolor{currentfill}%
\pgfsetlinewidth{0.250937pt}%
\definecolor{currentstroke}{rgb}{1.000000,1.000000,1.000000}%
\pgfsetstrokecolor{currentstroke}%
\pgfsetdash{}{0pt}%
\pgfpathmoveto{\pgfqpoint{2.135660in}{8.210661in}}%
\pgfpathlineto{\pgfqpoint{2.223396in}{8.210661in}}%
\pgfpathlineto{\pgfqpoint{2.223396in}{8.122925in}}%
\pgfpathlineto{\pgfqpoint{2.135660in}{8.122925in}}%
\pgfpathlineto{\pgfqpoint{2.135660in}{8.210661in}}%
\pgfusepath{stroke,fill}%
\end{pgfscope}%
\begin{pgfscope}%
\pgfpathrectangle{\pgfqpoint{0.380943in}{8.035189in}}{\pgfqpoint{4.650000in}{0.614151in}}%
\pgfusepath{clip}%
\pgfsetbuttcap%
\pgfsetroundjoin%
\definecolor{currentfill}{rgb}{0.985083,0.974641,0.792587}%
\pgfsetfillcolor{currentfill}%
\pgfsetlinewidth{0.250937pt}%
\definecolor{currentstroke}{rgb}{1.000000,1.000000,1.000000}%
\pgfsetstrokecolor{currentstroke}%
\pgfsetdash{}{0pt}%
\pgfpathmoveto{\pgfqpoint{2.223396in}{8.210661in}}%
\pgfpathlineto{\pgfqpoint{2.311132in}{8.210661in}}%
\pgfpathlineto{\pgfqpoint{2.311132in}{8.122925in}}%
\pgfpathlineto{\pgfqpoint{2.223396in}{8.122925in}}%
\pgfpathlineto{\pgfqpoint{2.223396in}{8.210661in}}%
\pgfusepath{stroke,fill}%
\end{pgfscope}%
\begin{pgfscope}%
\pgfpathrectangle{\pgfqpoint{0.380943in}{8.035189in}}{\pgfqpoint{4.650000in}{0.614151in}}%
\pgfusepath{clip}%
\pgfsetbuttcap%
\pgfsetroundjoin%
\definecolor{currentfill}{rgb}{0.970012,0.883276,0.699577}%
\pgfsetfillcolor{currentfill}%
\pgfsetlinewidth{0.250937pt}%
\definecolor{currentstroke}{rgb}{1.000000,1.000000,1.000000}%
\pgfsetstrokecolor{currentstroke}%
\pgfsetdash{}{0pt}%
\pgfpathmoveto{\pgfqpoint{2.311132in}{8.210661in}}%
\pgfpathlineto{\pgfqpoint{2.398868in}{8.210661in}}%
\pgfpathlineto{\pgfqpoint{2.398868in}{8.122925in}}%
\pgfpathlineto{\pgfqpoint{2.311132in}{8.122925in}}%
\pgfpathlineto{\pgfqpoint{2.311132in}{8.210661in}}%
\pgfusepath{stroke,fill}%
\end{pgfscope}%
\begin{pgfscope}%
\pgfpathrectangle{\pgfqpoint{0.380943in}{8.035189in}}{\pgfqpoint{4.650000in}{0.614151in}}%
\pgfusepath{clip}%
\pgfsetbuttcap%
\pgfsetroundjoin%
\definecolor{currentfill}{rgb}{0.978131,0.843783,0.675709}%
\pgfsetfillcolor{currentfill}%
\pgfsetlinewidth{0.250937pt}%
\definecolor{currentstroke}{rgb}{1.000000,1.000000,1.000000}%
\pgfsetstrokecolor{currentstroke}%
\pgfsetdash{}{0pt}%
\pgfpathmoveto{\pgfqpoint{2.398868in}{8.210661in}}%
\pgfpathlineto{\pgfqpoint{2.486603in}{8.210661in}}%
\pgfpathlineto{\pgfqpoint{2.486603in}{8.122925in}}%
\pgfpathlineto{\pgfqpoint{2.398868in}{8.122925in}}%
\pgfpathlineto{\pgfqpoint{2.398868in}{8.210661in}}%
\pgfusepath{stroke,fill}%
\end{pgfscope}%
\begin{pgfscope}%
\pgfpathrectangle{\pgfqpoint{0.380943in}{8.035189in}}{\pgfqpoint{4.650000in}{0.614151in}}%
\pgfusepath{clip}%
\pgfsetbuttcap%
\pgfsetroundjoin%
\definecolor{currentfill}{rgb}{0.985083,0.974641,0.792587}%
\pgfsetfillcolor{currentfill}%
\pgfsetlinewidth{0.250937pt}%
\definecolor{currentstroke}{rgb}{1.000000,1.000000,1.000000}%
\pgfsetstrokecolor{currentstroke}%
\pgfsetdash{}{0pt}%
\pgfpathmoveto{\pgfqpoint{2.486603in}{8.210661in}}%
\pgfpathlineto{\pgfqpoint{2.574339in}{8.210661in}}%
\pgfpathlineto{\pgfqpoint{2.574339in}{8.122925in}}%
\pgfpathlineto{\pgfqpoint{2.486603in}{8.122925in}}%
\pgfpathlineto{\pgfqpoint{2.486603in}{8.210661in}}%
\pgfusepath{stroke,fill}%
\end{pgfscope}%
\begin{pgfscope}%
\pgfpathrectangle{\pgfqpoint{0.380943in}{8.035189in}}{\pgfqpoint{4.650000in}{0.614151in}}%
\pgfusepath{clip}%
\pgfsetbuttcap%
\pgfsetroundjoin%
\definecolor{currentfill}{rgb}{1.000000,1.000000,0.861745}%
\pgfsetfillcolor{currentfill}%
\pgfsetlinewidth{0.250937pt}%
\definecolor{currentstroke}{rgb}{1.000000,1.000000,1.000000}%
\pgfsetstrokecolor{currentstroke}%
\pgfsetdash{}{0pt}%
\pgfpathmoveto{\pgfqpoint{2.574339in}{8.210661in}}%
\pgfpathlineto{\pgfqpoint{2.662075in}{8.210661in}}%
\pgfpathlineto{\pgfqpoint{2.662075in}{8.122925in}}%
\pgfpathlineto{\pgfqpoint{2.574339in}{8.122925in}}%
\pgfpathlineto{\pgfqpoint{2.574339in}{8.210661in}}%
\pgfusepath{stroke,fill}%
\end{pgfscope}%
\begin{pgfscope}%
\pgfpathrectangle{\pgfqpoint{0.380943in}{8.035189in}}{\pgfqpoint{4.650000in}{0.614151in}}%
\pgfusepath{clip}%
\pgfsetbuttcap%
\pgfsetroundjoin%
\definecolor{currentfill}{rgb}{0.970012,0.883276,0.699577}%
\pgfsetfillcolor{currentfill}%
\pgfsetlinewidth{0.250937pt}%
\definecolor{currentstroke}{rgb}{1.000000,1.000000,1.000000}%
\pgfsetstrokecolor{currentstroke}%
\pgfsetdash{}{0pt}%
\pgfpathmoveto{\pgfqpoint{2.662075in}{8.210661in}}%
\pgfpathlineto{\pgfqpoint{2.749811in}{8.210661in}}%
\pgfpathlineto{\pgfqpoint{2.749811in}{8.122925in}}%
\pgfpathlineto{\pgfqpoint{2.662075in}{8.122925in}}%
\pgfpathlineto{\pgfqpoint{2.662075in}{8.210661in}}%
\pgfusepath{stroke,fill}%
\end{pgfscope}%
\begin{pgfscope}%
\pgfpathrectangle{\pgfqpoint{0.380943in}{8.035189in}}{\pgfqpoint{4.650000in}{0.614151in}}%
\pgfusepath{clip}%
\pgfsetbuttcap%
\pgfsetroundjoin%
\definecolor{currentfill}{rgb}{0.961061,0.931672,0.728304}%
\pgfsetfillcolor{currentfill}%
\pgfsetlinewidth{0.250937pt}%
\definecolor{currentstroke}{rgb}{1.000000,1.000000,1.000000}%
\pgfsetstrokecolor{currentstroke}%
\pgfsetdash{}{0pt}%
\pgfpathmoveto{\pgfqpoint{2.749811in}{8.210661in}}%
\pgfpathlineto{\pgfqpoint{2.837547in}{8.210661in}}%
\pgfpathlineto{\pgfqpoint{2.837547in}{8.122925in}}%
\pgfpathlineto{\pgfqpoint{2.749811in}{8.122925in}}%
\pgfpathlineto{\pgfqpoint{2.749811in}{8.210661in}}%
\pgfusepath{stroke,fill}%
\end{pgfscope}%
\begin{pgfscope}%
\pgfpathrectangle{\pgfqpoint{0.380943in}{8.035189in}}{\pgfqpoint{4.650000in}{0.614151in}}%
\pgfusepath{clip}%
\pgfsetbuttcap%
\pgfsetroundjoin%
\definecolor{currentfill}{rgb}{0.985083,0.974641,0.792587}%
\pgfsetfillcolor{currentfill}%
\pgfsetlinewidth{0.250937pt}%
\definecolor{currentstroke}{rgb}{1.000000,1.000000,1.000000}%
\pgfsetstrokecolor{currentstroke}%
\pgfsetdash{}{0pt}%
\pgfpathmoveto{\pgfqpoint{2.837547in}{8.210661in}}%
\pgfpathlineto{\pgfqpoint{2.925283in}{8.210661in}}%
\pgfpathlineto{\pgfqpoint{2.925283in}{8.122925in}}%
\pgfpathlineto{\pgfqpoint{2.837547in}{8.122925in}}%
\pgfpathlineto{\pgfqpoint{2.837547in}{8.210661in}}%
\pgfusepath{stroke,fill}%
\end{pgfscope}%
\begin{pgfscope}%
\pgfpathrectangle{\pgfqpoint{0.380943in}{8.035189in}}{\pgfqpoint{4.650000in}{0.614151in}}%
\pgfusepath{clip}%
\pgfsetbuttcap%
\pgfsetroundjoin%
\definecolor{currentfill}{rgb}{0.963768,0.915433,0.717478}%
\pgfsetfillcolor{currentfill}%
\pgfsetlinewidth{0.250937pt}%
\definecolor{currentstroke}{rgb}{1.000000,1.000000,1.000000}%
\pgfsetstrokecolor{currentstroke}%
\pgfsetdash{}{0pt}%
\pgfpathmoveto{\pgfqpoint{2.925283in}{8.210661in}}%
\pgfpathlineto{\pgfqpoint{3.013019in}{8.210661in}}%
\pgfpathlineto{\pgfqpoint{3.013019in}{8.122925in}}%
\pgfpathlineto{\pgfqpoint{2.925283in}{8.122925in}}%
\pgfpathlineto{\pgfqpoint{2.925283in}{8.210661in}}%
\pgfusepath{stroke,fill}%
\end{pgfscope}%
\begin{pgfscope}%
\pgfpathrectangle{\pgfqpoint{0.380943in}{8.035189in}}{\pgfqpoint{4.650000in}{0.614151in}}%
\pgfusepath{clip}%
\pgfsetbuttcap%
\pgfsetroundjoin%
\definecolor{currentfill}{rgb}{0.961061,0.931672,0.728304}%
\pgfsetfillcolor{currentfill}%
\pgfsetlinewidth{0.250937pt}%
\definecolor{currentstroke}{rgb}{1.000000,1.000000,1.000000}%
\pgfsetstrokecolor{currentstroke}%
\pgfsetdash{}{0pt}%
\pgfpathmoveto{\pgfqpoint{3.013019in}{8.210661in}}%
\pgfpathlineto{\pgfqpoint{3.100754in}{8.210661in}}%
\pgfpathlineto{\pgfqpoint{3.100754in}{8.122925in}}%
\pgfpathlineto{\pgfqpoint{3.013019in}{8.122925in}}%
\pgfpathlineto{\pgfqpoint{3.013019in}{8.210661in}}%
\pgfusepath{stroke,fill}%
\end{pgfscope}%
\begin{pgfscope}%
\pgfpathrectangle{\pgfqpoint{0.380943in}{8.035189in}}{\pgfqpoint{4.650000in}{0.614151in}}%
\pgfusepath{clip}%
\pgfsetbuttcap%
\pgfsetroundjoin%
\definecolor{currentfill}{rgb}{1.000000,1.000000,0.861745}%
\pgfsetfillcolor{currentfill}%
\pgfsetlinewidth{0.250937pt}%
\definecolor{currentstroke}{rgb}{1.000000,1.000000,1.000000}%
\pgfsetstrokecolor{currentstroke}%
\pgfsetdash{}{0pt}%
\pgfpathmoveto{\pgfqpoint{3.100754in}{8.210661in}}%
\pgfpathlineto{\pgfqpoint{3.188490in}{8.210661in}}%
\pgfpathlineto{\pgfqpoint{3.188490in}{8.122925in}}%
\pgfpathlineto{\pgfqpoint{3.100754in}{8.122925in}}%
\pgfpathlineto{\pgfqpoint{3.100754in}{8.210661in}}%
\pgfusepath{stroke,fill}%
\end{pgfscope}%
\begin{pgfscope}%
\pgfpathrectangle{\pgfqpoint{0.380943in}{8.035189in}}{\pgfqpoint{4.650000in}{0.614151in}}%
\pgfusepath{clip}%
\pgfsetbuttcap%
\pgfsetroundjoin%
\definecolor{currentfill}{rgb}{0.963768,0.915433,0.717478}%
\pgfsetfillcolor{currentfill}%
\pgfsetlinewidth{0.250937pt}%
\definecolor{currentstroke}{rgb}{1.000000,1.000000,1.000000}%
\pgfsetstrokecolor{currentstroke}%
\pgfsetdash{}{0pt}%
\pgfpathmoveto{\pgfqpoint{3.188490in}{8.210661in}}%
\pgfpathlineto{\pgfqpoint{3.276226in}{8.210661in}}%
\pgfpathlineto{\pgfqpoint{3.276226in}{8.122925in}}%
\pgfpathlineto{\pgfqpoint{3.188490in}{8.122925in}}%
\pgfpathlineto{\pgfqpoint{3.188490in}{8.210661in}}%
\pgfusepath{stroke,fill}%
\end{pgfscope}%
\begin{pgfscope}%
\pgfpathrectangle{\pgfqpoint{0.380943in}{8.035189in}}{\pgfqpoint{4.650000in}{0.614151in}}%
\pgfusepath{clip}%
\pgfsetbuttcap%
\pgfsetroundjoin%
\definecolor{currentfill}{rgb}{0.961061,0.931672,0.728304}%
\pgfsetfillcolor{currentfill}%
\pgfsetlinewidth{0.250937pt}%
\definecolor{currentstroke}{rgb}{1.000000,1.000000,1.000000}%
\pgfsetstrokecolor{currentstroke}%
\pgfsetdash{}{0pt}%
\pgfpathmoveto{\pgfqpoint{3.276226in}{8.210661in}}%
\pgfpathlineto{\pgfqpoint{3.363962in}{8.210661in}}%
\pgfpathlineto{\pgfqpoint{3.363962in}{8.122925in}}%
\pgfpathlineto{\pgfqpoint{3.276226in}{8.122925in}}%
\pgfpathlineto{\pgfqpoint{3.276226in}{8.210661in}}%
\pgfusepath{stroke,fill}%
\end{pgfscope}%
\begin{pgfscope}%
\pgfpathrectangle{\pgfqpoint{0.380943in}{8.035189in}}{\pgfqpoint{4.650000in}{0.614151in}}%
\pgfusepath{clip}%
\pgfsetbuttcap%
\pgfsetroundjoin%
\definecolor{currentfill}{rgb}{1.000000,1.000000,0.861745}%
\pgfsetfillcolor{currentfill}%
\pgfsetlinewidth{0.250937pt}%
\definecolor{currentstroke}{rgb}{1.000000,1.000000,1.000000}%
\pgfsetstrokecolor{currentstroke}%
\pgfsetdash{}{0pt}%
\pgfpathmoveto{\pgfqpoint{3.363962in}{8.210661in}}%
\pgfpathlineto{\pgfqpoint{3.451698in}{8.210661in}}%
\pgfpathlineto{\pgfqpoint{3.451698in}{8.122925in}}%
\pgfpathlineto{\pgfqpoint{3.363962in}{8.122925in}}%
\pgfpathlineto{\pgfqpoint{3.363962in}{8.210661in}}%
\pgfusepath{stroke,fill}%
\end{pgfscope}%
\begin{pgfscope}%
\pgfpathrectangle{\pgfqpoint{0.380943in}{8.035189in}}{\pgfqpoint{4.650000in}{0.614151in}}%
\pgfusepath{clip}%
\pgfsetbuttcap%
\pgfsetroundjoin%
\definecolor{currentfill}{rgb}{1.000000,1.000000,0.861745}%
\pgfsetfillcolor{currentfill}%
\pgfsetlinewidth{0.250937pt}%
\definecolor{currentstroke}{rgb}{1.000000,1.000000,1.000000}%
\pgfsetstrokecolor{currentstroke}%
\pgfsetdash{}{0pt}%
\pgfpathmoveto{\pgfqpoint{3.451698in}{8.210661in}}%
\pgfpathlineto{\pgfqpoint{3.539434in}{8.210661in}}%
\pgfpathlineto{\pgfqpoint{3.539434in}{8.122925in}}%
\pgfpathlineto{\pgfqpoint{3.451698in}{8.122925in}}%
\pgfpathlineto{\pgfqpoint{3.451698in}{8.210661in}}%
\pgfusepath{stroke,fill}%
\end{pgfscope}%
\begin{pgfscope}%
\pgfpathrectangle{\pgfqpoint{0.380943in}{8.035189in}}{\pgfqpoint{4.650000in}{0.614151in}}%
\pgfusepath{clip}%
\pgfsetbuttcap%
\pgfsetroundjoin%
\definecolor{currentfill}{rgb}{0.961061,0.931672,0.728304}%
\pgfsetfillcolor{currentfill}%
\pgfsetlinewidth{0.250937pt}%
\definecolor{currentstroke}{rgb}{1.000000,1.000000,1.000000}%
\pgfsetstrokecolor{currentstroke}%
\pgfsetdash{}{0pt}%
\pgfpathmoveto{\pgfqpoint{3.539434in}{8.210661in}}%
\pgfpathlineto{\pgfqpoint{3.627169in}{8.210661in}}%
\pgfpathlineto{\pgfqpoint{3.627169in}{8.122925in}}%
\pgfpathlineto{\pgfqpoint{3.539434in}{8.122925in}}%
\pgfpathlineto{\pgfqpoint{3.539434in}{8.210661in}}%
\pgfusepath{stroke,fill}%
\end{pgfscope}%
\begin{pgfscope}%
\pgfpathrectangle{\pgfqpoint{0.380943in}{8.035189in}}{\pgfqpoint{4.650000in}{0.614151in}}%
\pgfusepath{clip}%
\pgfsetbuttcap%
\pgfsetroundjoin%
\definecolor{currentfill}{rgb}{0.978131,0.843783,0.675709}%
\pgfsetfillcolor{currentfill}%
\pgfsetlinewidth{0.250937pt}%
\definecolor{currentstroke}{rgb}{1.000000,1.000000,1.000000}%
\pgfsetstrokecolor{currentstroke}%
\pgfsetdash{}{0pt}%
\pgfpathmoveto{\pgfqpoint{3.627169in}{8.210661in}}%
\pgfpathlineto{\pgfqpoint{3.714905in}{8.210661in}}%
\pgfpathlineto{\pgfqpoint{3.714905in}{8.122925in}}%
\pgfpathlineto{\pgfqpoint{3.627169in}{8.122925in}}%
\pgfpathlineto{\pgfqpoint{3.627169in}{8.210661in}}%
\pgfusepath{stroke,fill}%
\end{pgfscope}%
\begin{pgfscope}%
\pgfpathrectangle{\pgfqpoint{0.380943in}{8.035189in}}{\pgfqpoint{4.650000in}{0.614151in}}%
\pgfusepath{clip}%
\pgfsetbuttcap%
\pgfsetroundjoin%
\definecolor{currentfill}{rgb}{0.963768,0.915433,0.717478}%
\pgfsetfillcolor{currentfill}%
\pgfsetlinewidth{0.250937pt}%
\definecolor{currentstroke}{rgb}{1.000000,1.000000,1.000000}%
\pgfsetstrokecolor{currentstroke}%
\pgfsetdash{}{0pt}%
\pgfpathmoveto{\pgfqpoint{3.714905in}{8.210661in}}%
\pgfpathlineto{\pgfqpoint{3.802641in}{8.210661in}}%
\pgfpathlineto{\pgfqpoint{3.802641in}{8.122925in}}%
\pgfpathlineto{\pgfqpoint{3.714905in}{8.122925in}}%
\pgfpathlineto{\pgfqpoint{3.714905in}{8.210661in}}%
\pgfusepath{stroke,fill}%
\end{pgfscope}%
\begin{pgfscope}%
\pgfpathrectangle{\pgfqpoint{0.380943in}{8.035189in}}{\pgfqpoint{4.650000in}{0.614151in}}%
\pgfusepath{clip}%
\pgfsetbuttcap%
\pgfsetroundjoin%
\definecolor{currentfill}{rgb}{0.985083,0.974641,0.792587}%
\pgfsetfillcolor{currentfill}%
\pgfsetlinewidth{0.250937pt}%
\definecolor{currentstroke}{rgb}{1.000000,1.000000,1.000000}%
\pgfsetstrokecolor{currentstroke}%
\pgfsetdash{}{0pt}%
\pgfpathmoveto{\pgfqpoint{3.802641in}{8.210661in}}%
\pgfpathlineto{\pgfqpoint{3.890377in}{8.210661in}}%
\pgfpathlineto{\pgfqpoint{3.890377in}{8.122925in}}%
\pgfpathlineto{\pgfqpoint{3.802641in}{8.122925in}}%
\pgfpathlineto{\pgfqpoint{3.802641in}{8.210661in}}%
\pgfusepath{stroke,fill}%
\end{pgfscope}%
\begin{pgfscope}%
\pgfpathrectangle{\pgfqpoint{0.380943in}{8.035189in}}{\pgfqpoint{4.650000in}{0.614151in}}%
\pgfusepath{clip}%
\pgfsetbuttcap%
\pgfsetroundjoin%
\definecolor{currentfill}{rgb}{0.985083,0.974641,0.792587}%
\pgfsetfillcolor{currentfill}%
\pgfsetlinewidth{0.250937pt}%
\definecolor{currentstroke}{rgb}{1.000000,1.000000,1.000000}%
\pgfsetstrokecolor{currentstroke}%
\pgfsetdash{}{0pt}%
\pgfpathmoveto{\pgfqpoint{3.890377in}{8.210661in}}%
\pgfpathlineto{\pgfqpoint{3.978113in}{8.210661in}}%
\pgfpathlineto{\pgfqpoint{3.978113in}{8.122925in}}%
\pgfpathlineto{\pgfqpoint{3.890377in}{8.122925in}}%
\pgfpathlineto{\pgfqpoint{3.890377in}{8.210661in}}%
\pgfusepath{stroke,fill}%
\end{pgfscope}%
\begin{pgfscope}%
\pgfpathrectangle{\pgfqpoint{0.380943in}{8.035189in}}{\pgfqpoint{4.650000in}{0.614151in}}%
\pgfusepath{clip}%
\pgfsetbuttcap%
\pgfsetroundjoin%
\definecolor{currentfill}{rgb}{0.978131,0.843783,0.675709}%
\pgfsetfillcolor{currentfill}%
\pgfsetlinewidth{0.250937pt}%
\definecolor{currentstroke}{rgb}{1.000000,1.000000,1.000000}%
\pgfsetstrokecolor{currentstroke}%
\pgfsetdash{}{0pt}%
\pgfpathmoveto{\pgfqpoint{3.978113in}{8.210661in}}%
\pgfpathlineto{\pgfqpoint{4.065849in}{8.210661in}}%
\pgfpathlineto{\pgfqpoint{4.065849in}{8.122925in}}%
\pgfpathlineto{\pgfqpoint{3.978113in}{8.122925in}}%
\pgfpathlineto{\pgfqpoint{3.978113in}{8.210661in}}%
\pgfusepath{stroke,fill}%
\end{pgfscope}%
\begin{pgfscope}%
\pgfpathrectangle{\pgfqpoint{0.380943in}{8.035189in}}{\pgfqpoint{4.650000in}{0.614151in}}%
\pgfusepath{clip}%
\pgfsetbuttcap%
\pgfsetroundjoin%
\definecolor{currentfill}{rgb}{0.985083,0.974641,0.792587}%
\pgfsetfillcolor{currentfill}%
\pgfsetlinewidth{0.250937pt}%
\definecolor{currentstroke}{rgb}{1.000000,1.000000,1.000000}%
\pgfsetstrokecolor{currentstroke}%
\pgfsetdash{}{0pt}%
\pgfpathmoveto{\pgfqpoint{4.065849in}{8.210661in}}%
\pgfpathlineto{\pgfqpoint{4.153585in}{8.210661in}}%
\pgfpathlineto{\pgfqpoint{4.153585in}{8.122925in}}%
\pgfpathlineto{\pgfqpoint{4.065849in}{8.122925in}}%
\pgfpathlineto{\pgfqpoint{4.065849in}{8.210661in}}%
\pgfusepath{stroke,fill}%
\end{pgfscope}%
\begin{pgfscope}%
\pgfpathrectangle{\pgfqpoint{0.380943in}{8.035189in}}{\pgfqpoint{4.650000in}{0.614151in}}%
\pgfusepath{clip}%
\pgfsetbuttcap%
\pgfsetroundjoin%
\definecolor{currentfill}{rgb}{0.970012,0.883276,0.699577}%
\pgfsetfillcolor{currentfill}%
\pgfsetlinewidth{0.250937pt}%
\definecolor{currentstroke}{rgb}{1.000000,1.000000,1.000000}%
\pgfsetstrokecolor{currentstroke}%
\pgfsetdash{}{0pt}%
\pgfpathmoveto{\pgfqpoint{4.153585in}{8.210661in}}%
\pgfpathlineto{\pgfqpoint{4.241320in}{8.210661in}}%
\pgfpathlineto{\pgfqpoint{4.241320in}{8.122925in}}%
\pgfpathlineto{\pgfqpoint{4.153585in}{8.122925in}}%
\pgfpathlineto{\pgfqpoint{4.153585in}{8.210661in}}%
\pgfusepath{stroke,fill}%
\end{pgfscope}%
\begin{pgfscope}%
\pgfpathrectangle{\pgfqpoint{0.380943in}{8.035189in}}{\pgfqpoint{4.650000in}{0.614151in}}%
\pgfusepath{clip}%
\pgfsetbuttcap%
\pgfsetroundjoin%
\definecolor{currentfill}{rgb}{0.963768,0.915433,0.717478}%
\pgfsetfillcolor{currentfill}%
\pgfsetlinewidth{0.250937pt}%
\definecolor{currentstroke}{rgb}{1.000000,1.000000,1.000000}%
\pgfsetstrokecolor{currentstroke}%
\pgfsetdash{}{0pt}%
\pgfpathmoveto{\pgfqpoint{4.241320in}{8.210661in}}%
\pgfpathlineto{\pgfqpoint{4.329056in}{8.210661in}}%
\pgfpathlineto{\pgfqpoint{4.329056in}{8.122925in}}%
\pgfpathlineto{\pgfqpoint{4.241320in}{8.122925in}}%
\pgfpathlineto{\pgfqpoint{4.241320in}{8.210661in}}%
\pgfusepath{stroke,fill}%
\end{pgfscope}%
\begin{pgfscope}%
\pgfpathrectangle{\pgfqpoint{0.380943in}{8.035189in}}{\pgfqpoint{4.650000in}{0.614151in}}%
\pgfusepath{clip}%
\pgfsetbuttcap%
\pgfsetroundjoin%
\definecolor{currentfill}{rgb}{0.985083,0.974641,0.792587}%
\pgfsetfillcolor{currentfill}%
\pgfsetlinewidth{0.250937pt}%
\definecolor{currentstroke}{rgb}{1.000000,1.000000,1.000000}%
\pgfsetstrokecolor{currentstroke}%
\pgfsetdash{}{0pt}%
\pgfpathmoveto{\pgfqpoint{4.329056in}{8.210661in}}%
\pgfpathlineto{\pgfqpoint{4.416792in}{8.210661in}}%
\pgfpathlineto{\pgfqpoint{4.416792in}{8.122925in}}%
\pgfpathlineto{\pgfqpoint{4.329056in}{8.122925in}}%
\pgfpathlineto{\pgfqpoint{4.329056in}{8.210661in}}%
\pgfusepath{stroke,fill}%
\end{pgfscope}%
\begin{pgfscope}%
\pgfpathrectangle{\pgfqpoint{0.380943in}{8.035189in}}{\pgfqpoint{4.650000in}{0.614151in}}%
\pgfusepath{clip}%
\pgfsetbuttcap%
\pgfsetroundjoin%
\definecolor{currentfill}{rgb}{0.963768,0.915433,0.717478}%
\pgfsetfillcolor{currentfill}%
\pgfsetlinewidth{0.250937pt}%
\definecolor{currentstroke}{rgb}{1.000000,1.000000,1.000000}%
\pgfsetstrokecolor{currentstroke}%
\pgfsetdash{}{0pt}%
\pgfpathmoveto{\pgfqpoint{4.416792in}{8.210661in}}%
\pgfpathlineto{\pgfqpoint{4.504528in}{8.210661in}}%
\pgfpathlineto{\pgfqpoint{4.504528in}{8.122925in}}%
\pgfpathlineto{\pgfqpoint{4.416792in}{8.122925in}}%
\pgfpathlineto{\pgfqpoint{4.416792in}{8.210661in}}%
\pgfusepath{stroke,fill}%
\end{pgfscope}%
\begin{pgfscope}%
\pgfpathrectangle{\pgfqpoint{0.380943in}{8.035189in}}{\pgfqpoint{4.650000in}{0.614151in}}%
\pgfusepath{clip}%
\pgfsetbuttcap%
\pgfsetroundjoin%
\definecolor{currentfill}{rgb}{0.961061,0.931672,0.728304}%
\pgfsetfillcolor{currentfill}%
\pgfsetlinewidth{0.250937pt}%
\definecolor{currentstroke}{rgb}{1.000000,1.000000,1.000000}%
\pgfsetstrokecolor{currentstroke}%
\pgfsetdash{}{0pt}%
\pgfpathmoveto{\pgfqpoint{4.504528in}{8.210661in}}%
\pgfpathlineto{\pgfqpoint{4.592264in}{8.210661in}}%
\pgfpathlineto{\pgfqpoint{4.592264in}{8.122925in}}%
\pgfpathlineto{\pgfqpoint{4.504528in}{8.122925in}}%
\pgfpathlineto{\pgfqpoint{4.504528in}{8.210661in}}%
\pgfusepath{stroke,fill}%
\end{pgfscope}%
\begin{pgfscope}%
\pgfpathrectangle{\pgfqpoint{0.380943in}{8.035189in}}{\pgfqpoint{4.650000in}{0.614151in}}%
\pgfusepath{clip}%
\pgfsetbuttcap%
\pgfsetroundjoin%
\definecolor{currentfill}{rgb}{0.985083,0.974641,0.792587}%
\pgfsetfillcolor{currentfill}%
\pgfsetlinewidth{0.250937pt}%
\definecolor{currentstroke}{rgb}{1.000000,1.000000,1.000000}%
\pgfsetstrokecolor{currentstroke}%
\pgfsetdash{}{0pt}%
\pgfpathmoveto{\pgfqpoint{4.592264in}{8.210661in}}%
\pgfpathlineto{\pgfqpoint{4.680000in}{8.210661in}}%
\pgfpathlineto{\pgfqpoint{4.680000in}{8.122925in}}%
\pgfpathlineto{\pgfqpoint{4.592264in}{8.122925in}}%
\pgfpathlineto{\pgfqpoint{4.592264in}{8.210661in}}%
\pgfusepath{stroke,fill}%
\end{pgfscope}%
\begin{pgfscope}%
\pgfpathrectangle{\pgfqpoint{0.380943in}{8.035189in}}{\pgfqpoint{4.650000in}{0.614151in}}%
\pgfusepath{clip}%
\pgfsetbuttcap%
\pgfsetroundjoin%
\definecolor{currentfill}{rgb}{0.970012,0.883276,0.699577}%
\pgfsetfillcolor{currentfill}%
\pgfsetlinewidth{0.250937pt}%
\definecolor{currentstroke}{rgb}{1.000000,1.000000,1.000000}%
\pgfsetstrokecolor{currentstroke}%
\pgfsetdash{}{0pt}%
\pgfpathmoveto{\pgfqpoint{4.680000in}{8.210661in}}%
\pgfpathlineto{\pgfqpoint{4.767736in}{8.210661in}}%
\pgfpathlineto{\pgfqpoint{4.767736in}{8.122925in}}%
\pgfpathlineto{\pgfqpoint{4.680000in}{8.122925in}}%
\pgfpathlineto{\pgfqpoint{4.680000in}{8.210661in}}%
\pgfusepath{stroke,fill}%
\end{pgfscope}%
\begin{pgfscope}%
\pgfpathrectangle{\pgfqpoint{0.380943in}{8.035189in}}{\pgfqpoint{4.650000in}{0.614151in}}%
\pgfusepath{clip}%
\pgfsetbuttcap%
\pgfsetroundjoin%
\definecolor{currentfill}{rgb}{0.970012,0.883276,0.699577}%
\pgfsetfillcolor{currentfill}%
\pgfsetlinewidth{0.250937pt}%
\definecolor{currentstroke}{rgb}{1.000000,1.000000,1.000000}%
\pgfsetstrokecolor{currentstroke}%
\pgfsetdash{}{0pt}%
\pgfpathmoveto{\pgfqpoint{4.767736in}{8.210661in}}%
\pgfpathlineto{\pgfqpoint{4.855471in}{8.210661in}}%
\pgfpathlineto{\pgfqpoint{4.855471in}{8.122925in}}%
\pgfpathlineto{\pgfqpoint{4.767736in}{8.122925in}}%
\pgfpathlineto{\pgfqpoint{4.767736in}{8.210661in}}%
\pgfusepath{stroke,fill}%
\end{pgfscope}%
\begin{pgfscope}%
\pgfpathrectangle{\pgfqpoint{0.380943in}{8.035189in}}{\pgfqpoint{4.650000in}{0.614151in}}%
\pgfusepath{clip}%
\pgfsetbuttcap%
\pgfsetroundjoin%
\definecolor{currentfill}{rgb}{0.963768,0.915433,0.717478}%
\pgfsetfillcolor{currentfill}%
\pgfsetlinewidth{0.250937pt}%
\definecolor{currentstroke}{rgb}{1.000000,1.000000,1.000000}%
\pgfsetstrokecolor{currentstroke}%
\pgfsetdash{}{0pt}%
\pgfpathmoveto{\pgfqpoint{4.855471in}{8.210661in}}%
\pgfpathlineto{\pgfqpoint{4.943207in}{8.210661in}}%
\pgfpathlineto{\pgfqpoint{4.943207in}{8.122925in}}%
\pgfpathlineto{\pgfqpoint{4.855471in}{8.122925in}}%
\pgfpathlineto{\pgfqpoint{4.855471in}{8.210661in}}%
\pgfusepath{stroke,fill}%
\end{pgfscope}%
\begin{pgfscope}%
\pgfpathrectangle{\pgfqpoint{0.380943in}{8.035189in}}{\pgfqpoint{4.650000in}{0.614151in}}%
\pgfusepath{clip}%
\pgfsetbuttcap%
\pgfsetroundjoin%
\definecolor{currentfill}{rgb}{0.970012,0.883276,0.699577}%
\pgfsetfillcolor{currentfill}%
\pgfsetlinewidth{0.250937pt}%
\definecolor{currentstroke}{rgb}{1.000000,1.000000,1.000000}%
\pgfsetstrokecolor{currentstroke}%
\pgfsetdash{}{0pt}%
\pgfpathmoveto{\pgfqpoint{4.943207in}{8.210661in}}%
\pgfpathlineto{\pgfqpoint{5.030943in}{8.210661in}}%
\pgfpathlineto{\pgfqpoint{5.030943in}{8.122925in}}%
\pgfpathlineto{\pgfqpoint{4.943207in}{8.122925in}}%
\pgfpathlineto{\pgfqpoint{4.943207in}{8.210661in}}%
\pgfusepath{stroke,fill}%
\end{pgfscope}%
\begin{pgfscope}%
\pgfpathrectangle{\pgfqpoint{0.380943in}{8.035189in}}{\pgfqpoint{4.650000in}{0.614151in}}%
\pgfusepath{clip}%
\pgfsetbuttcap%
\pgfsetroundjoin%
\definecolor{currentfill}{rgb}{0.970012,0.883276,0.699577}%
\pgfsetfillcolor{currentfill}%
\pgfsetlinewidth{0.250937pt}%
\definecolor{currentstroke}{rgb}{1.000000,1.000000,1.000000}%
\pgfsetstrokecolor{currentstroke}%
\pgfsetdash{}{0pt}%
\pgfpathmoveto{\pgfqpoint{0.380943in}{8.122925in}}%
\pgfpathlineto{\pgfqpoint{0.468679in}{8.122925in}}%
\pgfpathlineto{\pgfqpoint{0.468679in}{8.035189in}}%
\pgfpathlineto{\pgfqpoint{0.380943in}{8.035189in}}%
\pgfpathlineto{\pgfqpoint{0.380943in}{8.122925in}}%
\pgfusepath{stroke,fill}%
\end{pgfscope}%
\begin{pgfscope}%
\pgfpathrectangle{\pgfqpoint{0.380943in}{8.035189in}}{\pgfqpoint{4.650000in}{0.614151in}}%
\pgfusepath{clip}%
\pgfsetbuttcap%
\pgfsetroundjoin%
\definecolor{currentfill}{rgb}{0.985083,0.974641,0.792587}%
\pgfsetfillcolor{currentfill}%
\pgfsetlinewidth{0.250937pt}%
\definecolor{currentstroke}{rgb}{1.000000,1.000000,1.000000}%
\pgfsetstrokecolor{currentstroke}%
\pgfsetdash{}{0pt}%
\pgfpathmoveto{\pgfqpoint{0.468679in}{8.122925in}}%
\pgfpathlineto{\pgfqpoint{0.556415in}{8.122925in}}%
\pgfpathlineto{\pgfqpoint{0.556415in}{8.035189in}}%
\pgfpathlineto{\pgfqpoint{0.468679in}{8.035189in}}%
\pgfpathlineto{\pgfqpoint{0.468679in}{8.122925in}}%
\pgfusepath{stroke,fill}%
\end{pgfscope}%
\begin{pgfscope}%
\pgfpathrectangle{\pgfqpoint{0.380943in}{8.035189in}}{\pgfqpoint{4.650000in}{0.614151in}}%
\pgfusepath{clip}%
\pgfsetbuttcap%
\pgfsetroundjoin%
\definecolor{currentfill}{rgb}{0.985083,0.974641,0.792587}%
\pgfsetfillcolor{currentfill}%
\pgfsetlinewidth{0.250937pt}%
\definecolor{currentstroke}{rgb}{1.000000,1.000000,1.000000}%
\pgfsetstrokecolor{currentstroke}%
\pgfsetdash{}{0pt}%
\pgfpathmoveto{\pgfqpoint{0.556415in}{8.122925in}}%
\pgfpathlineto{\pgfqpoint{0.644151in}{8.122925in}}%
\pgfpathlineto{\pgfqpoint{0.644151in}{8.035189in}}%
\pgfpathlineto{\pgfqpoint{0.556415in}{8.035189in}}%
\pgfpathlineto{\pgfqpoint{0.556415in}{8.122925in}}%
\pgfusepath{stroke,fill}%
\end{pgfscope}%
\begin{pgfscope}%
\pgfpathrectangle{\pgfqpoint{0.380943in}{8.035189in}}{\pgfqpoint{4.650000in}{0.614151in}}%
\pgfusepath{clip}%
\pgfsetbuttcap%
\pgfsetroundjoin%
\definecolor{currentfill}{rgb}{0.963768,0.915433,0.717478}%
\pgfsetfillcolor{currentfill}%
\pgfsetlinewidth{0.250937pt}%
\definecolor{currentstroke}{rgb}{1.000000,1.000000,1.000000}%
\pgfsetstrokecolor{currentstroke}%
\pgfsetdash{}{0pt}%
\pgfpathmoveto{\pgfqpoint{0.644151in}{8.122925in}}%
\pgfpathlineto{\pgfqpoint{0.731886in}{8.122925in}}%
\pgfpathlineto{\pgfqpoint{0.731886in}{8.035189in}}%
\pgfpathlineto{\pgfqpoint{0.644151in}{8.035189in}}%
\pgfpathlineto{\pgfqpoint{0.644151in}{8.122925in}}%
\pgfusepath{stroke,fill}%
\end{pgfscope}%
\begin{pgfscope}%
\pgfpathrectangle{\pgfqpoint{0.380943in}{8.035189in}}{\pgfqpoint{4.650000in}{0.614151in}}%
\pgfusepath{clip}%
\pgfsetbuttcap%
\pgfsetroundjoin%
\definecolor{currentfill}{rgb}{0.970012,0.883276,0.699577}%
\pgfsetfillcolor{currentfill}%
\pgfsetlinewidth{0.250937pt}%
\definecolor{currentstroke}{rgb}{1.000000,1.000000,1.000000}%
\pgfsetstrokecolor{currentstroke}%
\pgfsetdash{}{0pt}%
\pgfpathmoveto{\pgfqpoint{0.731886in}{8.122925in}}%
\pgfpathlineto{\pgfqpoint{0.819622in}{8.122925in}}%
\pgfpathlineto{\pgfqpoint{0.819622in}{8.035189in}}%
\pgfpathlineto{\pgfqpoint{0.731886in}{8.035189in}}%
\pgfpathlineto{\pgfqpoint{0.731886in}{8.122925in}}%
\pgfusepath{stroke,fill}%
\end{pgfscope}%
\begin{pgfscope}%
\pgfpathrectangle{\pgfqpoint{0.380943in}{8.035189in}}{\pgfqpoint{4.650000in}{0.614151in}}%
\pgfusepath{clip}%
\pgfsetbuttcap%
\pgfsetroundjoin%
\definecolor{currentfill}{rgb}{1.000000,1.000000,0.861745}%
\pgfsetfillcolor{currentfill}%
\pgfsetlinewidth{0.250937pt}%
\definecolor{currentstroke}{rgb}{1.000000,1.000000,1.000000}%
\pgfsetstrokecolor{currentstroke}%
\pgfsetdash{}{0pt}%
\pgfpathmoveto{\pgfqpoint{0.819622in}{8.122925in}}%
\pgfpathlineto{\pgfqpoint{0.907358in}{8.122925in}}%
\pgfpathlineto{\pgfqpoint{0.907358in}{8.035189in}}%
\pgfpathlineto{\pgfqpoint{0.819622in}{8.035189in}}%
\pgfpathlineto{\pgfqpoint{0.819622in}{8.122925in}}%
\pgfusepath{stroke,fill}%
\end{pgfscope}%
\begin{pgfscope}%
\pgfpathrectangle{\pgfqpoint{0.380943in}{8.035189in}}{\pgfqpoint{4.650000in}{0.614151in}}%
\pgfusepath{clip}%
\pgfsetbuttcap%
\pgfsetroundjoin%
\definecolor{currentfill}{rgb}{0.970012,0.883276,0.699577}%
\pgfsetfillcolor{currentfill}%
\pgfsetlinewidth{0.250937pt}%
\definecolor{currentstroke}{rgb}{1.000000,1.000000,1.000000}%
\pgfsetstrokecolor{currentstroke}%
\pgfsetdash{}{0pt}%
\pgfpathmoveto{\pgfqpoint{0.907358in}{8.122925in}}%
\pgfpathlineto{\pgfqpoint{0.995094in}{8.122925in}}%
\pgfpathlineto{\pgfqpoint{0.995094in}{8.035189in}}%
\pgfpathlineto{\pgfqpoint{0.907358in}{8.035189in}}%
\pgfpathlineto{\pgfqpoint{0.907358in}{8.122925in}}%
\pgfusepath{stroke,fill}%
\end{pgfscope}%
\begin{pgfscope}%
\pgfpathrectangle{\pgfqpoint{0.380943in}{8.035189in}}{\pgfqpoint{4.650000in}{0.614151in}}%
\pgfusepath{clip}%
\pgfsetbuttcap%
\pgfsetroundjoin%
\definecolor{currentfill}{rgb}{0.985083,0.974641,0.792587}%
\pgfsetfillcolor{currentfill}%
\pgfsetlinewidth{0.250937pt}%
\definecolor{currentstroke}{rgb}{1.000000,1.000000,1.000000}%
\pgfsetstrokecolor{currentstroke}%
\pgfsetdash{}{0pt}%
\pgfpathmoveto{\pgfqpoint{0.995094in}{8.122925in}}%
\pgfpathlineto{\pgfqpoint{1.082830in}{8.122925in}}%
\pgfpathlineto{\pgfqpoint{1.082830in}{8.035189in}}%
\pgfpathlineto{\pgfqpoint{0.995094in}{8.035189in}}%
\pgfpathlineto{\pgfqpoint{0.995094in}{8.122925in}}%
\pgfusepath{stroke,fill}%
\end{pgfscope}%
\begin{pgfscope}%
\pgfpathrectangle{\pgfqpoint{0.380943in}{8.035189in}}{\pgfqpoint{4.650000in}{0.614151in}}%
\pgfusepath{clip}%
\pgfsetbuttcap%
\pgfsetroundjoin%
\definecolor{currentfill}{rgb}{0.978131,0.843783,0.675709}%
\pgfsetfillcolor{currentfill}%
\pgfsetlinewidth{0.250937pt}%
\definecolor{currentstroke}{rgb}{1.000000,1.000000,1.000000}%
\pgfsetstrokecolor{currentstroke}%
\pgfsetdash{}{0pt}%
\pgfpathmoveto{\pgfqpoint{1.082830in}{8.122925in}}%
\pgfpathlineto{\pgfqpoint{1.170566in}{8.122925in}}%
\pgfpathlineto{\pgfqpoint{1.170566in}{8.035189in}}%
\pgfpathlineto{\pgfqpoint{1.082830in}{8.035189in}}%
\pgfpathlineto{\pgfqpoint{1.082830in}{8.122925in}}%
\pgfusepath{stroke,fill}%
\end{pgfscope}%
\begin{pgfscope}%
\pgfpathrectangle{\pgfqpoint{0.380943in}{8.035189in}}{\pgfqpoint{4.650000in}{0.614151in}}%
\pgfusepath{clip}%
\pgfsetbuttcap%
\pgfsetroundjoin%
\definecolor{currentfill}{rgb}{0.961061,0.931672,0.728304}%
\pgfsetfillcolor{currentfill}%
\pgfsetlinewidth{0.250937pt}%
\definecolor{currentstroke}{rgb}{1.000000,1.000000,1.000000}%
\pgfsetstrokecolor{currentstroke}%
\pgfsetdash{}{0pt}%
\pgfpathmoveto{\pgfqpoint{1.170566in}{8.122925in}}%
\pgfpathlineto{\pgfqpoint{1.258302in}{8.122925in}}%
\pgfpathlineto{\pgfqpoint{1.258302in}{8.035189in}}%
\pgfpathlineto{\pgfqpoint{1.170566in}{8.035189in}}%
\pgfpathlineto{\pgfqpoint{1.170566in}{8.122925in}}%
\pgfusepath{stroke,fill}%
\end{pgfscope}%
\begin{pgfscope}%
\pgfpathrectangle{\pgfqpoint{0.380943in}{8.035189in}}{\pgfqpoint{4.650000in}{0.614151in}}%
\pgfusepath{clip}%
\pgfsetbuttcap%
\pgfsetroundjoin%
\definecolor{currentfill}{rgb}{0.961061,0.931672,0.728304}%
\pgfsetfillcolor{currentfill}%
\pgfsetlinewidth{0.250937pt}%
\definecolor{currentstroke}{rgb}{1.000000,1.000000,1.000000}%
\pgfsetstrokecolor{currentstroke}%
\pgfsetdash{}{0pt}%
\pgfpathmoveto{\pgfqpoint{1.258302in}{8.122925in}}%
\pgfpathlineto{\pgfqpoint{1.346037in}{8.122925in}}%
\pgfpathlineto{\pgfqpoint{1.346037in}{8.035189in}}%
\pgfpathlineto{\pgfqpoint{1.258302in}{8.035189in}}%
\pgfpathlineto{\pgfqpoint{1.258302in}{8.122925in}}%
\pgfusepath{stroke,fill}%
\end{pgfscope}%
\begin{pgfscope}%
\pgfpathrectangle{\pgfqpoint{0.380943in}{8.035189in}}{\pgfqpoint{4.650000in}{0.614151in}}%
\pgfusepath{clip}%
\pgfsetbuttcap%
\pgfsetroundjoin%
\definecolor{currentfill}{rgb}{0.963768,0.915433,0.717478}%
\pgfsetfillcolor{currentfill}%
\pgfsetlinewidth{0.250937pt}%
\definecolor{currentstroke}{rgb}{1.000000,1.000000,1.000000}%
\pgfsetstrokecolor{currentstroke}%
\pgfsetdash{}{0pt}%
\pgfpathmoveto{\pgfqpoint{1.346037in}{8.122925in}}%
\pgfpathlineto{\pgfqpoint{1.433773in}{8.122925in}}%
\pgfpathlineto{\pgfqpoint{1.433773in}{8.035189in}}%
\pgfpathlineto{\pgfqpoint{1.346037in}{8.035189in}}%
\pgfpathlineto{\pgfqpoint{1.346037in}{8.122925in}}%
\pgfusepath{stroke,fill}%
\end{pgfscope}%
\begin{pgfscope}%
\pgfpathrectangle{\pgfqpoint{0.380943in}{8.035189in}}{\pgfqpoint{4.650000in}{0.614151in}}%
\pgfusepath{clip}%
\pgfsetbuttcap%
\pgfsetroundjoin%
\definecolor{currentfill}{rgb}{0.961061,0.931672,0.728304}%
\pgfsetfillcolor{currentfill}%
\pgfsetlinewidth{0.250937pt}%
\definecolor{currentstroke}{rgb}{1.000000,1.000000,1.000000}%
\pgfsetstrokecolor{currentstroke}%
\pgfsetdash{}{0pt}%
\pgfpathmoveto{\pgfqpoint{1.433773in}{8.122925in}}%
\pgfpathlineto{\pgfqpoint{1.521509in}{8.122925in}}%
\pgfpathlineto{\pgfqpoint{1.521509in}{8.035189in}}%
\pgfpathlineto{\pgfqpoint{1.433773in}{8.035189in}}%
\pgfpathlineto{\pgfqpoint{1.433773in}{8.122925in}}%
\pgfusepath{stroke,fill}%
\end{pgfscope}%
\begin{pgfscope}%
\pgfpathrectangle{\pgfqpoint{0.380943in}{8.035189in}}{\pgfqpoint{4.650000in}{0.614151in}}%
\pgfusepath{clip}%
\pgfsetbuttcap%
\pgfsetroundjoin%
\definecolor{currentfill}{rgb}{0.963768,0.915433,0.717478}%
\pgfsetfillcolor{currentfill}%
\pgfsetlinewidth{0.250937pt}%
\definecolor{currentstroke}{rgb}{1.000000,1.000000,1.000000}%
\pgfsetstrokecolor{currentstroke}%
\pgfsetdash{}{0pt}%
\pgfpathmoveto{\pgfqpoint{1.521509in}{8.122925in}}%
\pgfpathlineto{\pgfqpoint{1.609245in}{8.122925in}}%
\pgfpathlineto{\pgfqpoint{1.609245in}{8.035189in}}%
\pgfpathlineto{\pgfqpoint{1.521509in}{8.035189in}}%
\pgfpathlineto{\pgfqpoint{1.521509in}{8.122925in}}%
\pgfusepath{stroke,fill}%
\end{pgfscope}%
\begin{pgfscope}%
\pgfpathrectangle{\pgfqpoint{0.380943in}{8.035189in}}{\pgfqpoint{4.650000in}{0.614151in}}%
\pgfusepath{clip}%
\pgfsetbuttcap%
\pgfsetroundjoin%
\definecolor{currentfill}{rgb}{0.961061,0.931672,0.728304}%
\pgfsetfillcolor{currentfill}%
\pgfsetlinewidth{0.250937pt}%
\definecolor{currentstroke}{rgb}{1.000000,1.000000,1.000000}%
\pgfsetstrokecolor{currentstroke}%
\pgfsetdash{}{0pt}%
\pgfpathmoveto{\pgfqpoint{1.609245in}{8.122925in}}%
\pgfpathlineto{\pgfqpoint{1.696981in}{8.122925in}}%
\pgfpathlineto{\pgfqpoint{1.696981in}{8.035189in}}%
\pgfpathlineto{\pgfqpoint{1.609245in}{8.035189in}}%
\pgfpathlineto{\pgfqpoint{1.609245in}{8.122925in}}%
\pgfusepath{stroke,fill}%
\end{pgfscope}%
\begin{pgfscope}%
\pgfpathrectangle{\pgfqpoint{0.380943in}{8.035189in}}{\pgfqpoint{4.650000in}{0.614151in}}%
\pgfusepath{clip}%
\pgfsetbuttcap%
\pgfsetroundjoin%
\definecolor{currentfill}{rgb}{1.000000,1.000000,0.861745}%
\pgfsetfillcolor{currentfill}%
\pgfsetlinewidth{0.250937pt}%
\definecolor{currentstroke}{rgb}{1.000000,1.000000,1.000000}%
\pgfsetstrokecolor{currentstroke}%
\pgfsetdash{}{0pt}%
\pgfpathmoveto{\pgfqpoint{1.696981in}{8.122925in}}%
\pgfpathlineto{\pgfqpoint{1.784717in}{8.122925in}}%
\pgfpathlineto{\pgfqpoint{1.784717in}{8.035189in}}%
\pgfpathlineto{\pgfqpoint{1.696981in}{8.035189in}}%
\pgfpathlineto{\pgfqpoint{1.696981in}{8.122925in}}%
\pgfusepath{stroke,fill}%
\end{pgfscope}%
\begin{pgfscope}%
\pgfpathrectangle{\pgfqpoint{0.380943in}{8.035189in}}{\pgfqpoint{4.650000in}{0.614151in}}%
\pgfusepath{clip}%
\pgfsetbuttcap%
\pgfsetroundjoin%
\definecolor{currentfill}{rgb}{0.963768,0.915433,0.717478}%
\pgfsetfillcolor{currentfill}%
\pgfsetlinewidth{0.250937pt}%
\definecolor{currentstroke}{rgb}{1.000000,1.000000,1.000000}%
\pgfsetstrokecolor{currentstroke}%
\pgfsetdash{}{0pt}%
\pgfpathmoveto{\pgfqpoint{1.784717in}{8.122925in}}%
\pgfpathlineto{\pgfqpoint{1.872452in}{8.122925in}}%
\pgfpathlineto{\pgfqpoint{1.872452in}{8.035189in}}%
\pgfpathlineto{\pgfqpoint{1.784717in}{8.035189in}}%
\pgfpathlineto{\pgfqpoint{1.784717in}{8.122925in}}%
\pgfusepath{stroke,fill}%
\end{pgfscope}%
\begin{pgfscope}%
\pgfpathrectangle{\pgfqpoint{0.380943in}{8.035189in}}{\pgfqpoint{4.650000in}{0.614151in}}%
\pgfusepath{clip}%
\pgfsetbuttcap%
\pgfsetroundjoin%
\definecolor{currentfill}{rgb}{0.961061,0.931672,0.728304}%
\pgfsetfillcolor{currentfill}%
\pgfsetlinewidth{0.250937pt}%
\definecolor{currentstroke}{rgb}{1.000000,1.000000,1.000000}%
\pgfsetstrokecolor{currentstroke}%
\pgfsetdash{}{0pt}%
\pgfpathmoveto{\pgfqpoint{1.872452in}{8.122925in}}%
\pgfpathlineto{\pgfqpoint{1.960188in}{8.122925in}}%
\pgfpathlineto{\pgfqpoint{1.960188in}{8.035189in}}%
\pgfpathlineto{\pgfqpoint{1.872452in}{8.035189in}}%
\pgfpathlineto{\pgfqpoint{1.872452in}{8.122925in}}%
\pgfusepath{stroke,fill}%
\end{pgfscope}%
\begin{pgfscope}%
\pgfpathrectangle{\pgfqpoint{0.380943in}{8.035189in}}{\pgfqpoint{4.650000in}{0.614151in}}%
\pgfusepath{clip}%
\pgfsetbuttcap%
\pgfsetroundjoin%
\definecolor{currentfill}{rgb}{1.000000,1.000000,0.861745}%
\pgfsetfillcolor{currentfill}%
\pgfsetlinewidth{0.250937pt}%
\definecolor{currentstroke}{rgb}{1.000000,1.000000,1.000000}%
\pgfsetstrokecolor{currentstroke}%
\pgfsetdash{}{0pt}%
\pgfpathmoveto{\pgfqpoint{1.960188in}{8.122925in}}%
\pgfpathlineto{\pgfqpoint{2.047924in}{8.122925in}}%
\pgfpathlineto{\pgfqpoint{2.047924in}{8.035189in}}%
\pgfpathlineto{\pgfqpoint{1.960188in}{8.035189in}}%
\pgfpathlineto{\pgfqpoint{1.960188in}{8.122925in}}%
\pgfusepath{stroke,fill}%
\end{pgfscope}%
\begin{pgfscope}%
\pgfpathrectangle{\pgfqpoint{0.380943in}{8.035189in}}{\pgfqpoint{4.650000in}{0.614151in}}%
\pgfusepath{clip}%
\pgfsetbuttcap%
\pgfsetroundjoin%
\definecolor{currentfill}{rgb}{1.000000,1.000000,0.861745}%
\pgfsetfillcolor{currentfill}%
\pgfsetlinewidth{0.250937pt}%
\definecolor{currentstroke}{rgb}{1.000000,1.000000,1.000000}%
\pgfsetstrokecolor{currentstroke}%
\pgfsetdash{}{0pt}%
\pgfpathmoveto{\pgfqpoint{2.047924in}{8.122925in}}%
\pgfpathlineto{\pgfqpoint{2.135660in}{8.122925in}}%
\pgfpathlineto{\pgfqpoint{2.135660in}{8.035189in}}%
\pgfpathlineto{\pgfqpoint{2.047924in}{8.035189in}}%
\pgfpathlineto{\pgfqpoint{2.047924in}{8.122925in}}%
\pgfusepath{stroke,fill}%
\end{pgfscope}%
\begin{pgfscope}%
\pgfpathrectangle{\pgfqpoint{0.380943in}{8.035189in}}{\pgfqpoint{4.650000in}{0.614151in}}%
\pgfusepath{clip}%
\pgfsetbuttcap%
\pgfsetroundjoin%
\definecolor{currentfill}{rgb}{1.000000,1.000000,0.861745}%
\pgfsetfillcolor{currentfill}%
\pgfsetlinewidth{0.250937pt}%
\definecolor{currentstroke}{rgb}{1.000000,1.000000,1.000000}%
\pgfsetstrokecolor{currentstroke}%
\pgfsetdash{}{0pt}%
\pgfpathmoveto{\pgfqpoint{2.135660in}{8.122925in}}%
\pgfpathlineto{\pgfqpoint{2.223396in}{8.122925in}}%
\pgfpathlineto{\pgfqpoint{2.223396in}{8.035189in}}%
\pgfpathlineto{\pgfqpoint{2.135660in}{8.035189in}}%
\pgfpathlineto{\pgfqpoint{2.135660in}{8.122925in}}%
\pgfusepath{stroke,fill}%
\end{pgfscope}%
\begin{pgfscope}%
\pgfpathrectangle{\pgfqpoint{0.380943in}{8.035189in}}{\pgfqpoint{4.650000in}{0.614151in}}%
\pgfusepath{clip}%
\pgfsetbuttcap%
\pgfsetroundjoin%
\definecolor{currentfill}{rgb}{0.963768,0.915433,0.717478}%
\pgfsetfillcolor{currentfill}%
\pgfsetlinewidth{0.250937pt}%
\definecolor{currentstroke}{rgb}{1.000000,1.000000,1.000000}%
\pgfsetstrokecolor{currentstroke}%
\pgfsetdash{}{0pt}%
\pgfpathmoveto{\pgfqpoint{2.223396in}{8.122925in}}%
\pgfpathlineto{\pgfqpoint{2.311132in}{8.122925in}}%
\pgfpathlineto{\pgfqpoint{2.311132in}{8.035189in}}%
\pgfpathlineto{\pgfqpoint{2.223396in}{8.035189in}}%
\pgfpathlineto{\pgfqpoint{2.223396in}{8.122925in}}%
\pgfusepath{stroke,fill}%
\end{pgfscope}%
\begin{pgfscope}%
\pgfpathrectangle{\pgfqpoint{0.380943in}{8.035189in}}{\pgfqpoint{4.650000in}{0.614151in}}%
\pgfusepath{clip}%
\pgfsetbuttcap%
\pgfsetroundjoin%
\definecolor{currentfill}{rgb}{0.961061,0.931672,0.728304}%
\pgfsetfillcolor{currentfill}%
\pgfsetlinewidth{0.250937pt}%
\definecolor{currentstroke}{rgb}{1.000000,1.000000,1.000000}%
\pgfsetstrokecolor{currentstroke}%
\pgfsetdash{}{0pt}%
\pgfpathmoveto{\pgfqpoint{2.311132in}{8.122925in}}%
\pgfpathlineto{\pgfqpoint{2.398868in}{8.122925in}}%
\pgfpathlineto{\pgfqpoint{2.398868in}{8.035189in}}%
\pgfpathlineto{\pgfqpoint{2.311132in}{8.035189in}}%
\pgfpathlineto{\pgfqpoint{2.311132in}{8.122925in}}%
\pgfusepath{stroke,fill}%
\end{pgfscope}%
\begin{pgfscope}%
\pgfpathrectangle{\pgfqpoint{0.380943in}{8.035189in}}{\pgfqpoint{4.650000in}{0.614151in}}%
\pgfusepath{clip}%
\pgfsetbuttcap%
\pgfsetroundjoin%
\definecolor{currentfill}{rgb}{0.961061,0.931672,0.728304}%
\pgfsetfillcolor{currentfill}%
\pgfsetlinewidth{0.250937pt}%
\definecolor{currentstroke}{rgb}{1.000000,1.000000,1.000000}%
\pgfsetstrokecolor{currentstroke}%
\pgfsetdash{}{0pt}%
\pgfpathmoveto{\pgfqpoint{2.398868in}{8.122925in}}%
\pgfpathlineto{\pgfqpoint{2.486603in}{8.122925in}}%
\pgfpathlineto{\pgfqpoint{2.486603in}{8.035189in}}%
\pgfpathlineto{\pgfqpoint{2.398868in}{8.035189in}}%
\pgfpathlineto{\pgfqpoint{2.398868in}{8.122925in}}%
\pgfusepath{stroke,fill}%
\end{pgfscope}%
\begin{pgfscope}%
\pgfpathrectangle{\pgfqpoint{0.380943in}{8.035189in}}{\pgfqpoint{4.650000in}{0.614151in}}%
\pgfusepath{clip}%
\pgfsetbuttcap%
\pgfsetroundjoin%
\definecolor{currentfill}{rgb}{0.963768,0.915433,0.717478}%
\pgfsetfillcolor{currentfill}%
\pgfsetlinewidth{0.250937pt}%
\definecolor{currentstroke}{rgb}{1.000000,1.000000,1.000000}%
\pgfsetstrokecolor{currentstroke}%
\pgfsetdash{}{0pt}%
\pgfpathmoveto{\pgfqpoint{2.486603in}{8.122925in}}%
\pgfpathlineto{\pgfqpoint{2.574339in}{8.122925in}}%
\pgfpathlineto{\pgfqpoint{2.574339in}{8.035189in}}%
\pgfpathlineto{\pgfqpoint{2.486603in}{8.035189in}}%
\pgfpathlineto{\pgfqpoint{2.486603in}{8.122925in}}%
\pgfusepath{stroke,fill}%
\end{pgfscope}%
\begin{pgfscope}%
\pgfpathrectangle{\pgfqpoint{0.380943in}{8.035189in}}{\pgfqpoint{4.650000in}{0.614151in}}%
\pgfusepath{clip}%
\pgfsetbuttcap%
\pgfsetroundjoin%
\definecolor{currentfill}{rgb}{1.000000,1.000000,0.861745}%
\pgfsetfillcolor{currentfill}%
\pgfsetlinewidth{0.250937pt}%
\definecolor{currentstroke}{rgb}{1.000000,1.000000,1.000000}%
\pgfsetstrokecolor{currentstroke}%
\pgfsetdash{}{0pt}%
\pgfpathmoveto{\pgfqpoint{2.574339in}{8.122925in}}%
\pgfpathlineto{\pgfqpoint{2.662075in}{8.122925in}}%
\pgfpathlineto{\pgfqpoint{2.662075in}{8.035189in}}%
\pgfpathlineto{\pgfqpoint{2.574339in}{8.035189in}}%
\pgfpathlineto{\pgfqpoint{2.574339in}{8.122925in}}%
\pgfusepath{stroke,fill}%
\end{pgfscope}%
\begin{pgfscope}%
\pgfpathrectangle{\pgfqpoint{0.380943in}{8.035189in}}{\pgfqpoint{4.650000in}{0.614151in}}%
\pgfusepath{clip}%
\pgfsetbuttcap%
\pgfsetroundjoin%
\definecolor{currentfill}{rgb}{1.000000,1.000000,0.861745}%
\pgfsetfillcolor{currentfill}%
\pgfsetlinewidth{0.250937pt}%
\definecolor{currentstroke}{rgb}{1.000000,1.000000,1.000000}%
\pgfsetstrokecolor{currentstroke}%
\pgfsetdash{}{0pt}%
\pgfpathmoveto{\pgfqpoint{2.662075in}{8.122925in}}%
\pgfpathlineto{\pgfqpoint{2.749811in}{8.122925in}}%
\pgfpathlineto{\pgfqpoint{2.749811in}{8.035189in}}%
\pgfpathlineto{\pgfqpoint{2.662075in}{8.035189in}}%
\pgfpathlineto{\pgfqpoint{2.662075in}{8.122925in}}%
\pgfusepath{stroke,fill}%
\end{pgfscope}%
\begin{pgfscope}%
\pgfpathrectangle{\pgfqpoint{0.380943in}{8.035189in}}{\pgfqpoint{4.650000in}{0.614151in}}%
\pgfusepath{clip}%
\pgfsetbuttcap%
\pgfsetroundjoin%
\definecolor{currentfill}{rgb}{1.000000,1.000000,0.929412}%
\pgfsetfillcolor{currentfill}%
\pgfsetlinewidth{0.250937pt}%
\definecolor{currentstroke}{rgb}{1.000000,1.000000,1.000000}%
\pgfsetstrokecolor{currentstroke}%
\pgfsetdash{}{0pt}%
\pgfpathmoveto{\pgfqpoint{2.749811in}{8.122925in}}%
\pgfpathlineto{\pgfqpoint{2.837547in}{8.122925in}}%
\pgfpathlineto{\pgfqpoint{2.837547in}{8.035189in}}%
\pgfpathlineto{\pgfqpoint{2.749811in}{8.035189in}}%
\pgfpathlineto{\pgfqpoint{2.749811in}{8.122925in}}%
\pgfusepath{stroke,fill}%
\end{pgfscope}%
\begin{pgfscope}%
\pgfpathrectangle{\pgfqpoint{0.380943in}{8.035189in}}{\pgfqpoint{4.650000in}{0.614151in}}%
\pgfusepath{clip}%
\pgfsetbuttcap%
\pgfsetroundjoin%
\definecolor{currentfill}{rgb}{1.000000,1.000000,0.861745}%
\pgfsetfillcolor{currentfill}%
\pgfsetlinewidth{0.250937pt}%
\definecolor{currentstroke}{rgb}{1.000000,1.000000,1.000000}%
\pgfsetstrokecolor{currentstroke}%
\pgfsetdash{}{0pt}%
\pgfpathmoveto{\pgfqpoint{2.837547in}{8.122925in}}%
\pgfpathlineto{\pgfqpoint{2.925283in}{8.122925in}}%
\pgfpathlineto{\pgfqpoint{2.925283in}{8.035189in}}%
\pgfpathlineto{\pgfqpoint{2.837547in}{8.035189in}}%
\pgfpathlineto{\pgfqpoint{2.837547in}{8.122925in}}%
\pgfusepath{stroke,fill}%
\end{pgfscope}%
\begin{pgfscope}%
\pgfpathrectangle{\pgfqpoint{0.380943in}{8.035189in}}{\pgfqpoint{4.650000in}{0.614151in}}%
\pgfusepath{clip}%
\pgfsetbuttcap%
\pgfsetroundjoin%
\definecolor{currentfill}{rgb}{0.970012,0.883276,0.699577}%
\pgfsetfillcolor{currentfill}%
\pgfsetlinewidth{0.250937pt}%
\definecolor{currentstroke}{rgb}{1.000000,1.000000,1.000000}%
\pgfsetstrokecolor{currentstroke}%
\pgfsetdash{}{0pt}%
\pgfpathmoveto{\pgfqpoint{2.925283in}{8.122925in}}%
\pgfpathlineto{\pgfqpoint{3.013019in}{8.122925in}}%
\pgfpathlineto{\pgfqpoint{3.013019in}{8.035189in}}%
\pgfpathlineto{\pgfqpoint{2.925283in}{8.035189in}}%
\pgfpathlineto{\pgfqpoint{2.925283in}{8.122925in}}%
\pgfusepath{stroke,fill}%
\end{pgfscope}%
\begin{pgfscope}%
\pgfpathrectangle{\pgfqpoint{0.380943in}{8.035189in}}{\pgfqpoint{4.650000in}{0.614151in}}%
\pgfusepath{clip}%
\pgfsetbuttcap%
\pgfsetroundjoin%
\definecolor{currentfill}{rgb}{0.961061,0.931672,0.728304}%
\pgfsetfillcolor{currentfill}%
\pgfsetlinewidth{0.250937pt}%
\definecolor{currentstroke}{rgb}{1.000000,1.000000,1.000000}%
\pgfsetstrokecolor{currentstroke}%
\pgfsetdash{}{0pt}%
\pgfpathmoveto{\pgfqpoint{3.013019in}{8.122925in}}%
\pgfpathlineto{\pgfqpoint{3.100754in}{8.122925in}}%
\pgfpathlineto{\pgfqpoint{3.100754in}{8.035189in}}%
\pgfpathlineto{\pgfqpoint{3.013019in}{8.035189in}}%
\pgfpathlineto{\pgfqpoint{3.013019in}{8.122925in}}%
\pgfusepath{stroke,fill}%
\end{pgfscope}%
\begin{pgfscope}%
\pgfpathrectangle{\pgfqpoint{0.380943in}{8.035189in}}{\pgfqpoint{4.650000in}{0.614151in}}%
\pgfusepath{clip}%
\pgfsetbuttcap%
\pgfsetroundjoin%
\definecolor{currentfill}{rgb}{1.000000,1.000000,0.929412}%
\pgfsetfillcolor{currentfill}%
\pgfsetlinewidth{0.250937pt}%
\definecolor{currentstroke}{rgb}{1.000000,1.000000,1.000000}%
\pgfsetstrokecolor{currentstroke}%
\pgfsetdash{}{0pt}%
\pgfpathmoveto{\pgfqpoint{3.100754in}{8.122925in}}%
\pgfpathlineto{\pgfqpoint{3.188490in}{8.122925in}}%
\pgfpathlineto{\pgfqpoint{3.188490in}{8.035189in}}%
\pgfpathlineto{\pgfqpoint{3.100754in}{8.035189in}}%
\pgfpathlineto{\pgfqpoint{3.100754in}{8.122925in}}%
\pgfusepath{stroke,fill}%
\end{pgfscope}%
\begin{pgfscope}%
\pgfpathrectangle{\pgfqpoint{0.380943in}{8.035189in}}{\pgfqpoint{4.650000in}{0.614151in}}%
\pgfusepath{clip}%
\pgfsetbuttcap%
\pgfsetroundjoin%
\definecolor{currentfill}{rgb}{1.000000,1.000000,0.861745}%
\pgfsetfillcolor{currentfill}%
\pgfsetlinewidth{0.250937pt}%
\definecolor{currentstroke}{rgb}{1.000000,1.000000,1.000000}%
\pgfsetstrokecolor{currentstroke}%
\pgfsetdash{}{0pt}%
\pgfpathmoveto{\pgfqpoint{3.188490in}{8.122925in}}%
\pgfpathlineto{\pgfqpoint{3.276226in}{8.122925in}}%
\pgfpathlineto{\pgfqpoint{3.276226in}{8.035189in}}%
\pgfpathlineto{\pgfqpoint{3.188490in}{8.035189in}}%
\pgfpathlineto{\pgfqpoint{3.188490in}{8.122925in}}%
\pgfusepath{stroke,fill}%
\end{pgfscope}%
\begin{pgfscope}%
\pgfpathrectangle{\pgfqpoint{0.380943in}{8.035189in}}{\pgfqpoint{4.650000in}{0.614151in}}%
\pgfusepath{clip}%
\pgfsetbuttcap%
\pgfsetroundjoin%
\definecolor{currentfill}{rgb}{0.963768,0.915433,0.717478}%
\pgfsetfillcolor{currentfill}%
\pgfsetlinewidth{0.250937pt}%
\definecolor{currentstroke}{rgb}{1.000000,1.000000,1.000000}%
\pgfsetstrokecolor{currentstroke}%
\pgfsetdash{}{0pt}%
\pgfpathmoveto{\pgfqpoint{3.276226in}{8.122925in}}%
\pgfpathlineto{\pgfqpoint{3.363962in}{8.122925in}}%
\pgfpathlineto{\pgfqpoint{3.363962in}{8.035189in}}%
\pgfpathlineto{\pgfqpoint{3.276226in}{8.035189in}}%
\pgfpathlineto{\pgfqpoint{3.276226in}{8.122925in}}%
\pgfusepath{stroke,fill}%
\end{pgfscope}%
\begin{pgfscope}%
\pgfpathrectangle{\pgfqpoint{0.380943in}{8.035189in}}{\pgfqpoint{4.650000in}{0.614151in}}%
\pgfusepath{clip}%
\pgfsetbuttcap%
\pgfsetroundjoin%
\definecolor{currentfill}{rgb}{0.978131,0.843783,0.675709}%
\pgfsetfillcolor{currentfill}%
\pgfsetlinewidth{0.250937pt}%
\definecolor{currentstroke}{rgb}{1.000000,1.000000,1.000000}%
\pgfsetstrokecolor{currentstroke}%
\pgfsetdash{}{0pt}%
\pgfpathmoveto{\pgfqpoint{3.363962in}{8.122925in}}%
\pgfpathlineto{\pgfqpoint{3.451698in}{8.122925in}}%
\pgfpathlineto{\pgfqpoint{3.451698in}{8.035189in}}%
\pgfpathlineto{\pgfqpoint{3.363962in}{8.035189in}}%
\pgfpathlineto{\pgfqpoint{3.363962in}{8.122925in}}%
\pgfusepath{stroke,fill}%
\end{pgfscope}%
\begin{pgfscope}%
\pgfpathrectangle{\pgfqpoint{0.380943in}{8.035189in}}{\pgfqpoint{4.650000in}{0.614151in}}%
\pgfusepath{clip}%
\pgfsetbuttcap%
\pgfsetroundjoin%
\definecolor{currentfill}{rgb}{0.961061,0.931672,0.728304}%
\pgfsetfillcolor{currentfill}%
\pgfsetlinewidth{0.250937pt}%
\definecolor{currentstroke}{rgb}{1.000000,1.000000,1.000000}%
\pgfsetstrokecolor{currentstroke}%
\pgfsetdash{}{0pt}%
\pgfpathmoveto{\pgfqpoint{3.451698in}{8.122925in}}%
\pgfpathlineto{\pgfqpoint{3.539434in}{8.122925in}}%
\pgfpathlineto{\pgfqpoint{3.539434in}{8.035189in}}%
\pgfpathlineto{\pgfqpoint{3.451698in}{8.035189in}}%
\pgfpathlineto{\pgfqpoint{3.451698in}{8.122925in}}%
\pgfusepath{stroke,fill}%
\end{pgfscope}%
\begin{pgfscope}%
\pgfpathrectangle{\pgfqpoint{0.380943in}{8.035189in}}{\pgfqpoint{4.650000in}{0.614151in}}%
\pgfusepath{clip}%
\pgfsetbuttcap%
\pgfsetroundjoin%
\definecolor{currentfill}{rgb}{1.000000,1.000000,0.861745}%
\pgfsetfillcolor{currentfill}%
\pgfsetlinewidth{0.250937pt}%
\definecolor{currentstroke}{rgb}{1.000000,1.000000,1.000000}%
\pgfsetstrokecolor{currentstroke}%
\pgfsetdash{}{0pt}%
\pgfpathmoveto{\pgfqpoint{3.539434in}{8.122925in}}%
\pgfpathlineto{\pgfqpoint{3.627169in}{8.122925in}}%
\pgfpathlineto{\pgfqpoint{3.627169in}{8.035189in}}%
\pgfpathlineto{\pgfqpoint{3.539434in}{8.035189in}}%
\pgfpathlineto{\pgfqpoint{3.539434in}{8.122925in}}%
\pgfusepath{stroke,fill}%
\end{pgfscope}%
\begin{pgfscope}%
\pgfpathrectangle{\pgfqpoint{0.380943in}{8.035189in}}{\pgfqpoint{4.650000in}{0.614151in}}%
\pgfusepath{clip}%
\pgfsetbuttcap%
\pgfsetroundjoin%
\definecolor{currentfill}{rgb}{0.999616,0.641369,0.546559}%
\pgfsetfillcolor{currentfill}%
\pgfsetlinewidth{0.250937pt}%
\definecolor{currentstroke}{rgb}{1.000000,1.000000,1.000000}%
\pgfsetstrokecolor{currentstroke}%
\pgfsetdash{}{0pt}%
\pgfpathmoveto{\pgfqpoint{3.627169in}{8.122925in}}%
\pgfpathlineto{\pgfqpoint{3.714905in}{8.122925in}}%
\pgfpathlineto{\pgfqpoint{3.714905in}{8.035189in}}%
\pgfpathlineto{\pgfqpoint{3.627169in}{8.035189in}}%
\pgfpathlineto{\pgfqpoint{3.627169in}{8.122925in}}%
\pgfusepath{stroke,fill}%
\end{pgfscope}%
\begin{pgfscope}%
\pgfpathrectangle{\pgfqpoint{0.380943in}{8.035189in}}{\pgfqpoint{4.650000in}{0.614151in}}%
\pgfusepath{clip}%
\pgfsetbuttcap%
\pgfsetroundjoin%
\definecolor{currentfill}{rgb}{0.978131,0.843783,0.675709}%
\pgfsetfillcolor{currentfill}%
\pgfsetlinewidth{0.250937pt}%
\definecolor{currentstroke}{rgb}{1.000000,1.000000,1.000000}%
\pgfsetstrokecolor{currentstroke}%
\pgfsetdash{}{0pt}%
\pgfpathmoveto{\pgfqpoint{3.714905in}{8.122925in}}%
\pgfpathlineto{\pgfqpoint{3.802641in}{8.122925in}}%
\pgfpathlineto{\pgfqpoint{3.802641in}{8.035189in}}%
\pgfpathlineto{\pgfqpoint{3.714905in}{8.035189in}}%
\pgfpathlineto{\pgfqpoint{3.714905in}{8.122925in}}%
\pgfusepath{stroke,fill}%
\end{pgfscope}%
\begin{pgfscope}%
\pgfpathrectangle{\pgfqpoint{0.380943in}{8.035189in}}{\pgfqpoint{4.650000in}{0.614151in}}%
\pgfusepath{clip}%
\pgfsetbuttcap%
\pgfsetroundjoin%
\definecolor{currentfill}{rgb}{0.985083,0.974641,0.792587}%
\pgfsetfillcolor{currentfill}%
\pgfsetlinewidth{0.250937pt}%
\definecolor{currentstroke}{rgb}{1.000000,1.000000,1.000000}%
\pgfsetstrokecolor{currentstroke}%
\pgfsetdash{}{0pt}%
\pgfpathmoveto{\pgfqpoint{3.802641in}{8.122925in}}%
\pgfpathlineto{\pgfqpoint{3.890377in}{8.122925in}}%
\pgfpathlineto{\pgfqpoint{3.890377in}{8.035189in}}%
\pgfpathlineto{\pgfqpoint{3.802641in}{8.035189in}}%
\pgfpathlineto{\pgfqpoint{3.802641in}{8.122925in}}%
\pgfusepath{stroke,fill}%
\end{pgfscope}%
\begin{pgfscope}%
\pgfpathrectangle{\pgfqpoint{0.380943in}{8.035189in}}{\pgfqpoint{4.650000in}{0.614151in}}%
\pgfusepath{clip}%
\pgfsetbuttcap%
\pgfsetroundjoin%
\definecolor{currentfill}{rgb}{0.961061,0.931672,0.728304}%
\pgfsetfillcolor{currentfill}%
\pgfsetlinewidth{0.250937pt}%
\definecolor{currentstroke}{rgb}{1.000000,1.000000,1.000000}%
\pgfsetstrokecolor{currentstroke}%
\pgfsetdash{}{0pt}%
\pgfpathmoveto{\pgfqpoint{3.890377in}{8.122925in}}%
\pgfpathlineto{\pgfqpoint{3.978113in}{8.122925in}}%
\pgfpathlineto{\pgfqpoint{3.978113in}{8.035189in}}%
\pgfpathlineto{\pgfqpoint{3.890377in}{8.035189in}}%
\pgfpathlineto{\pgfqpoint{3.890377in}{8.122925in}}%
\pgfusepath{stroke,fill}%
\end{pgfscope}%
\begin{pgfscope}%
\pgfpathrectangle{\pgfqpoint{0.380943in}{8.035189in}}{\pgfqpoint{4.650000in}{0.614151in}}%
\pgfusepath{clip}%
\pgfsetbuttcap%
\pgfsetroundjoin%
\definecolor{currentfill}{rgb}{0.978131,0.843783,0.675709}%
\pgfsetfillcolor{currentfill}%
\pgfsetlinewidth{0.250937pt}%
\definecolor{currentstroke}{rgb}{1.000000,1.000000,1.000000}%
\pgfsetstrokecolor{currentstroke}%
\pgfsetdash{}{0pt}%
\pgfpathmoveto{\pgfqpoint{3.978113in}{8.122925in}}%
\pgfpathlineto{\pgfqpoint{4.065849in}{8.122925in}}%
\pgfpathlineto{\pgfqpoint{4.065849in}{8.035189in}}%
\pgfpathlineto{\pgfqpoint{3.978113in}{8.035189in}}%
\pgfpathlineto{\pgfqpoint{3.978113in}{8.122925in}}%
\pgfusepath{stroke,fill}%
\end{pgfscope}%
\begin{pgfscope}%
\pgfpathrectangle{\pgfqpoint{0.380943in}{8.035189in}}{\pgfqpoint{4.650000in}{0.614151in}}%
\pgfusepath{clip}%
\pgfsetbuttcap%
\pgfsetroundjoin%
\definecolor{currentfill}{rgb}{0.985083,0.974641,0.792587}%
\pgfsetfillcolor{currentfill}%
\pgfsetlinewidth{0.250937pt}%
\definecolor{currentstroke}{rgb}{1.000000,1.000000,1.000000}%
\pgfsetstrokecolor{currentstroke}%
\pgfsetdash{}{0pt}%
\pgfpathmoveto{\pgfqpoint{4.065849in}{8.122925in}}%
\pgfpathlineto{\pgfqpoint{4.153585in}{8.122925in}}%
\pgfpathlineto{\pgfqpoint{4.153585in}{8.035189in}}%
\pgfpathlineto{\pgfqpoint{4.065849in}{8.035189in}}%
\pgfpathlineto{\pgfqpoint{4.065849in}{8.122925in}}%
\pgfusepath{stroke,fill}%
\end{pgfscope}%
\begin{pgfscope}%
\pgfpathrectangle{\pgfqpoint{0.380943in}{8.035189in}}{\pgfqpoint{4.650000in}{0.614151in}}%
\pgfusepath{clip}%
\pgfsetbuttcap%
\pgfsetroundjoin%
\definecolor{currentfill}{rgb}{0.970012,0.883276,0.699577}%
\pgfsetfillcolor{currentfill}%
\pgfsetlinewidth{0.250937pt}%
\definecolor{currentstroke}{rgb}{1.000000,1.000000,1.000000}%
\pgfsetstrokecolor{currentstroke}%
\pgfsetdash{}{0pt}%
\pgfpathmoveto{\pgfqpoint{4.153585in}{8.122925in}}%
\pgfpathlineto{\pgfqpoint{4.241320in}{8.122925in}}%
\pgfpathlineto{\pgfqpoint{4.241320in}{8.035189in}}%
\pgfpathlineto{\pgfqpoint{4.153585in}{8.035189in}}%
\pgfpathlineto{\pgfqpoint{4.153585in}{8.122925in}}%
\pgfusepath{stroke,fill}%
\end{pgfscope}%
\begin{pgfscope}%
\pgfpathrectangle{\pgfqpoint{0.380943in}{8.035189in}}{\pgfqpoint{4.650000in}{0.614151in}}%
\pgfusepath{clip}%
\pgfsetbuttcap%
\pgfsetroundjoin%
\definecolor{currentfill}{rgb}{0.963768,0.915433,0.717478}%
\pgfsetfillcolor{currentfill}%
\pgfsetlinewidth{0.250937pt}%
\definecolor{currentstroke}{rgb}{1.000000,1.000000,1.000000}%
\pgfsetstrokecolor{currentstroke}%
\pgfsetdash{}{0pt}%
\pgfpathmoveto{\pgfqpoint{4.241320in}{8.122925in}}%
\pgfpathlineto{\pgfqpoint{4.329056in}{8.122925in}}%
\pgfpathlineto{\pgfqpoint{4.329056in}{8.035189in}}%
\pgfpathlineto{\pgfqpoint{4.241320in}{8.035189in}}%
\pgfpathlineto{\pgfqpoint{4.241320in}{8.122925in}}%
\pgfusepath{stroke,fill}%
\end{pgfscope}%
\begin{pgfscope}%
\pgfpathrectangle{\pgfqpoint{0.380943in}{8.035189in}}{\pgfqpoint{4.650000in}{0.614151in}}%
\pgfusepath{clip}%
\pgfsetbuttcap%
\pgfsetroundjoin%
\definecolor{currentfill}{rgb}{0.985083,0.974641,0.792587}%
\pgfsetfillcolor{currentfill}%
\pgfsetlinewidth{0.250937pt}%
\definecolor{currentstroke}{rgb}{1.000000,1.000000,1.000000}%
\pgfsetstrokecolor{currentstroke}%
\pgfsetdash{}{0pt}%
\pgfpathmoveto{\pgfqpoint{4.329056in}{8.122925in}}%
\pgfpathlineto{\pgfqpoint{4.416792in}{8.122925in}}%
\pgfpathlineto{\pgfqpoint{4.416792in}{8.035189in}}%
\pgfpathlineto{\pgfqpoint{4.329056in}{8.035189in}}%
\pgfpathlineto{\pgfqpoint{4.329056in}{8.122925in}}%
\pgfusepath{stroke,fill}%
\end{pgfscope}%
\begin{pgfscope}%
\pgfpathrectangle{\pgfqpoint{0.380943in}{8.035189in}}{\pgfqpoint{4.650000in}{0.614151in}}%
\pgfusepath{clip}%
\pgfsetbuttcap%
\pgfsetroundjoin%
\definecolor{currentfill}{rgb}{0.985083,0.974641,0.792587}%
\pgfsetfillcolor{currentfill}%
\pgfsetlinewidth{0.250937pt}%
\definecolor{currentstroke}{rgb}{1.000000,1.000000,1.000000}%
\pgfsetstrokecolor{currentstroke}%
\pgfsetdash{}{0pt}%
\pgfpathmoveto{\pgfqpoint{4.416792in}{8.122925in}}%
\pgfpathlineto{\pgfqpoint{4.504528in}{8.122925in}}%
\pgfpathlineto{\pgfqpoint{4.504528in}{8.035189in}}%
\pgfpathlineto{\pgfqpoint{4.416792in}{8.035189in}}%
\pgfpathlineto{\pgfqpoint{4.416792in}{8.122925in}}%
\pgfusepath{stroke,fill}%
\end{pgfscope}%
\begin{pgfscope}%
\pgfpathrectangle{\pgfqpoint{0.380943in}{8.035189in}}{\pgfqpoint{4.650000in}{0.614151in}}%
\pgfusepath{clip}%
\pgfsetbuttcap%
\pgfsetroundjoin%
\definecolor{currentfill}{rgb}{0.985083,0.974641,0.792587}%
\pgfsetfillcolor{currentfill}%
\pgfsetlinewidth{0.250937pt}%
\definecolor{currentstroke}{rgb}{1.000000,1.000000,1.000000}%
\pgfsetstrokecolor{currentstroke}%
\pgfsetdash{}{0pt}%
\pgfpathmoveto{\pgfqpoint{4.504528in}{8.122925in}}%
\pgfpathlineto{\pgfqpoint{4.592264in}{8.122925in}}%
\pgfpathlineto{\pgfqpoint{4.592264in}{8.035189in}}%
\pgfpathlineto{\pgfqpoint{4.504528in}{8.035189in}}%
\pgfpathlineto{\pgfqpoint{4.504528in}{8.122925in}}%
\pgfusepath{stroke,fill}%
\end{pgfscope}%
\begin{pgfscope}%
\pgfpathrectangle{\pgfqpoint{0.380943in}{8.035189in}}{\pgfqpoint{4.650000in}{0.614151in}}%
\pgfusepath{clip}%
\pgfsetbuttcap%
\pgfsetroundjoin%
\definecolor{currentfill}{rgb}{0.963768,0.915433,0.717478}%
\pgfsetfillcolor{currentfill}%
\pgfsetlinewidth{0.250937pt}%
\definecolor{currentstroke}{rgb}{1.000000,1.000000,1.000000}%
\pgfsetstrokecolor{currentstroke}%
\pgfsetdash{}{0pt}%
\pgfpathmoveto{\pgfqpoint{4.592264in}{8.122925in}}%
\pgfpathlineto{\pgfqpoint{4.680000in}{8.122925in}}%
\pgfpathlineto{\pgfqpoint{4.680000in}{8.035189in}}%
\pgfpathlineto{\pgfqpoint{4.592264in}{8.035189in}}%
\pgfpathlineto{\pgfqpoint{4.592264in}{8.122925in}}%
\pgfusepath{stroke,fill}%
\end{pgfscope}%
\begin{pgfscope}%
\pgfpathrectangle{\pgfqpoint{0.380943in}{8.035189in}}{\pgfqpoint{4.650000in}{0.614151in}}%
\pgfusepath{clip}%
\pgfsetbuttcap%
\pgfsetroundjoin%
\definecolor{currentfill}{rgb}{0.978131,0.843783,0.675709}%
\pgfsetfillcolor{currentfill}%
\pgfsetlinewidth{0.250937pt}%
\definecolor{currentstroke}{rgb}{1.000000,1.000000,1.000000}%
\pgfsetstrokecolor{currentstroke}%
\pgfsetdash{}{0pt}%
\pgfpathmoveto{\pgfqpoint{4.680000in}{8.122925in}}%
\pgfpathlineto{\pgfqpoint{4.767736in}{8.122925in}}%
\pgfpathlineto{\pgfqpoint{4.767736in}{8.035189in}}%
\pgfpathlineto{\pgfqpoint{4.680000in}{8.035189in}}%
\pgfpathlineto{\pgfqpoint{4.680000in}{8.122925in}}%
\pgfusepath{stroke,fill}%
\end{pgfscope}%
\begin{pgfscope}%
\pgfpathrectangle{\pgfqpoint{0.380943in}{8.035189in}}{\pgfqpoint{4.650000in}{0.614151in}}%
\pgfusepath{clip}%
\pgfsetbuttcap%
\pgfsetroundjoin%
\definecolor{currentfill}{rgb}{1.000000,1.000000,0.861745}%
\pgfsetfillcolor{currentfill}%
\pgfsetlinewidth{0.250937pt}%
\definecolor{currentstroke}{rgb}{1.000000,1.000000,1.000000}%
\pgfsetstrokecolor{currentstroke}%
\pgfsetdash{}{0pt}%
\pgfpathmoveto{\pgfqpoint{4.767736in}{8.122925in}}%
\pgfpathlineto{\pgfqpoint{4.855471in}{8.122925in}}%
\pgfpathlineto{\pgfqpoint{4.855471in}{8.035189in}}%
\pgfpathlineto{\pgfqpoint{4.767736in}{8.035189in}}%
\pgfpathlineto{\pgfqpoint{4.767736in}{8.122925in}}%
\pgfusepath{stroke,fill}%
\end{pgfscope}%
\begin{pgfscope}%
\pgfpathrectangle{\pgfqpoint{0.380943in}{8.035189in}}{\pgfqpoint{4.650000in}{0.614151in}}%
\pgfusepath{clip}%
\pgfsetbuttcap%
\pgfsetroundjoin%
\definecolor{currentfill}{rgb}{1.000000,1.000000,0.861745}%
\pgfsetfillcolor{currentfill}%
\pgfsetlinewidth{0.250937pt}%
\definecolor{currentstroke}{rgb}{1.000000,1.000000,1.000000}%
\pgfsetstrokecolor{currentstroke}%
\pgfsetdash{}{0pt}%
\pgfpathmoveto{\pgfqpoint{4.855471in}{8.122925in}}%
\pgfpathlineto{\pgfqpoint{4.943207in}{8.122925in}}%
\pgfpathlineto{\pgfqpoint{4.943207in}{8.035189in}}%
\pgfpathlineto{\pgfqpoint{4.855471in}{8.035189in}}%
\pgfpathlineto{\pgfqpoint{4.855471in}{8.122925in}}%
\pgfusepath{stroke,fill}%
\end{pgfscope}%
\begin{pgfscope}%
\pgfpathrectangle{\pgfqpoint{0.380943in}{8.035189in}}{\pgfqpoint{4.650000in}{0.614151in}}%
\pgfusepath{clip}%
\pgfsetbuttcap%
\pgfsetroundjoin%
\definecolor{currentfill}{rgb}{0.985083,0.974641,0.792587}%
\pgfsetfillcolor{currentfill}%
\pgfsetlinewidth{0.250937pt}%
\definecolor{currentstroke}{rgb}{1.000000,1.000000,1.000000}%
\pgfsetstrokecolor{currentstroke}%
\pgfsetdash{}{0pt}%
\pgfpathmoveto{\pgfqpoint{4.943207in}{8.122925in}}%
\pgfpathlineto{\pgfqpoint{5.030943in}{8.122925in}}%
\pgfpathlineto{\pgfqpoint{5.030943in}{8.035189in}}%
\pgfpathlineto{\pgfqpoint{4.943207in}{8.035189in}}%
\pgfpathlineto{\pgfqpoint{4.943207in}{8.122925in}}%
\pgfusepath{stroke,fill}%
\end{pgfscope}%
\begin{pgfscope}%
\pgfsetbuttcap%
\pgfsetroundjoin%
\definecolor{currentfill}{rgb}{0.000000,0.000000,0.000000}%
\pgfsetfillcolor{currentfill}%
\pgfsetlinewidth{0.803000pt}%
\definecolor{currentstroke}{rgb}{0.000000,0.000000,0.000000}%
\pgfsetstrokecolor{currentstroke}%
\pgfsetdash{}{0pt}%
\pgfsys@defobject{currentmarker}{\pgfqpoint{0.000000in}{-0.048611in}}{\pgfqpoint{0.000000in}{0.000000in}}{%
\pgfpathmoveto{\pgfqpoint{0.000000in}{0.000000in}}%
\pgfpathlineto{\pgfqpoint{0.000000in}{-0.048611in}}%
\pgfusepath{stroke,fill}%
}%
\begin{pgfscope}%
\pgfsys@transformshift{0.644151in}{8.035189in}%
\pgfsys@useobject{currentmarker}{}%
\end{pgfscope}%
\end{pgfscope}%
\begin{pgfscope}%
\definecolor{textcolor}{rgb}{0.000000,0.000000,0.000000}%
\pgfsetstrokecolor{textcolor}%
\pgfsetfillcolor{textcolor}%
\pgftext[x=0.644151in,y=7.937967in,,top]{\color{textcolor}\rmfamily\fontsize{8.000000}{9.600000}\selectfont Jan}%
\end{pgfscope}%
\begin{pgfscope}%
\pgfsetbuttcap%
\pgfsetroundjoin%
\definecolor{currentfill}{rgb}{0.000000,0.000000,0.000000}%
\pgfsetfillcolor{currentfill}%
\pgfsetlinewidth{0.803000pt}%
\definecolor{currentstroke}{rgb}{0.000000,0.000000,0.000000}%
\pgfsetstrokecolor{currentstroke}%
\pgfsetdash{}{0pt}%
\pgfsys@defobject{currentmarker}{\pgfqpoint{0.000000in}{-0.048611in}}{\pgfqpoint{0.000000in}{0.000000in}}{%
\pgfpathmoveto{\pgfqpoint{0.000000in}{0.000000in}}%
\pgfpathlineto{\pgfqpoint{0.000000in}{-0.048611in}}%
\pgfusepath{stroke,fill}%
}%
\begin{pgfscope}%
\pgfsys@transformshift{1.038962in}{8.035189in}%
\pgfsys@useobject{currentmarker}{}%
\end{pgfscope}%
\end{pgfscope}%
\begin{pgfscope}%
\definecolor{textcolor}{rgb}{0.000000,0.000000,0.000000}%
\pgfsetstrokecolor{textcolor}%
\pgfsetfillcolor{textcolor}%
\pgftext[x=1.038962in,y=7.937967in,,top]{\color{textcolor}\rmfamily\fontsize{8.000000}{9.600000}\selectfont Feb}%
\end{pgfscope}%
\begin{pgfscope}%
\pgfsetbuttcap%
\pgfsetroundjoin%
\definecolor{currentfill}{rgb}{0.000000,0.000000,0.000000}%
\pgfsetfillcolor{currentfill}%
\pgfsetlinewidth{0.803000pt}%
\definecolor{currentstroke}{rgb}{0.000000,0.000000,0.000000}%
\pgfsetstrokecolor{currentstroke}%
\pgfsetdash{}{0pt}%
\pgfsys@defobject{currentmarker}{\pgfqpoint{0.000000in}{-0.048611in}}{\pgfqpoint{0.000000in}{0.000000in}}{%
\pgfpathmoveto{\pgfqpoint{0.000000in}{0.000000in}}%
\pgfpathlineto{\pgfqpoint{0.000000in}{-0.048611in}}%
\pgfusepath{stroke,fill}%
}%
\begin{pgfscope}%
\pgfsys@transformshift{1.389905in}{8.035189in}%
\pgfsys@useobject{currentmarker}{}%
\end{pgfscope}%
\end{pgfscope}%
\begin{pgfscope}%
\definecolor{textcolor}{rgb}{0.000000,0.000000,0.000000}%
\pgfsetstrokecolor{textcolor}%
\pgfsetfillcolor{textcolor}%
\pgftext[x=1.389905in,y=7.937967in,,top]{\color{textcolor}\rmfamily\fontsize{8.000000}{9.600000}\selectfont Mar}%
\end{pgfscope}%
\begin{pgfscope}%
\pgfsetbuttcap%
\pgfsetroundjoin%
\definecolor{currentfill}{rgb}{0.000000,0.000000,0.000000}%
\pgfsetfillcolor{currentfill}%
\pgfsetlinewidth{0.803000pt}%
\definecolor{currentstroke}{rgb}{0.000000,0.000000,0.000000}%
\pgfsetstrokecolor{currentstroke}%
\pgfsetdash{}{0pt}%
\pgfsys@defobject{currentmarker}{\pgfqpoint{0.000000in}{-0.048611in}}{\pgfqpoint{0.000000in}{0.000000in}}{%
\pgfpathmoveto{\pgfqpoint{0.000000in}{0.000000in}}%
\pgfpathlineto{\pgfqpoint{0.000000in}{-0.048611in}}%
\pgfusepath{stroke,fill}%
}%
\begin{pgfscope}%
\pgfsys@transformshift{1.740849in}{8.035189in}%
\pgfsys@useobject{currentmarker}{}%
\end{pgfscope}%
\end{pgfscope}%
\begin{pgfscope}%
\definecolor{textcolor}{rgb}{0.000000,0.000000,0.000000}%
\pgfsetstrokecolor{textcolor}%
\pgfsetfillcolor{textcolor}%
\pgftext[x=1.740849in,y=7.937967in,,top]{\color{textcolor}\rmfamily\fontsize{8.000000}{9.600000}\selectfont Apr}%
\end{pgfscope}%
\begin{pgfscope}%
\pgfsetbuttcap%
\pgfsetroundjoin%
\definecolor{currentfill}{rgb}{0.000000,0.000000,0.000000}%
\pgfsetfillcolor{currentfill}%
\pgfsetlinewidth{0.803000pt}%
\definecolor{currentstroke}{rgb}{0.000000,0.000000,0.000000}%
\pgfsetstrokecolor{currentstroke}%
\pgfsetdash{}{0pt}%
\pgfsys@defobject{currentmarker}{\pgfqpoint{0.000000in}{-0.048611in}}{\pgfqpoint{0.000000in}{0.000000in}}{%
\pgfpathmoveto{\pgfqpoint{0.000000in}{0.000000in}}%
\pgfpathlineto{\pgfqpoint{0.000000in}{-0.048611in}}%
\pgfusepath{stroke,fill}%
}%
\begin{pgfscope}%
\pgfsys@transformshift{2.179528in}{8.035189in}%
\pgfsys@useobject{currentmarker}{}%
\end{pgfscope}%
\end{pgfscope}%
\begin{pgfscope}%
\definecolor{textcolor}{rgb}{0.000000,0.000000,0.000000}%
\pgfsetstrokecolor{textcolor}%
\pgfsetfillcolor{textcolor}%
\pgftext[x=2.179528in,y=7.937967in,,top]{\color{textcolor}\rmfamily\fontsize{8.000000}{9.600000}\selectfont May}%
\end{pgfscope}%
\begin{pgfscope}%
\pgfsetbuttcap%
\pgfsetroundjoin%
\definecolor{currentfill}{rgb}{0.000000,0.000000,0.000000}%
\pgfsetfillcolor{currentfill}%
\pgfsetlinewidth{0.803000pt}%
\definecolor{currentstroke}{rgb}{0.000000,0.000000,0.000000}%
\pgfsetstrokecolor{currentstroke}%
\pgfsetdash{}{0pt}%
\pgfsys@defobject{currentmarker}{\pgfqpoint{0.000000in}{-0.048611in}}{\pgfqpoint{0.000000in}{0.000000in}}{%
\pgfpathmoveto{\pgfqpoint{0.000000in}{0.000000in}}%
\pgfpathlineto{\pgfqpoint{0.000000in}{-0.048611in}}%
\pgfusepath{stroke,fill}%
}%
\begin{pgfscope}%
\pgfsys@transformshift{2.530471in}{8.035189in}%
\pgfsys@useobject{currentmarker}{}%
\end{pgfscope}%
\end{pgfscope}%
\begin{pgfscope}%
\definecolor{textcolor}{rgb}{0.000000,0.000000,0.000000}%
\pgfsetstrokecolor{textcolor}%
\pgfsetfillcolor{textcolor}%
\pgftext[x=2.530471in,y=7.937967in,,top]{\color{textcolor}\rmfamily\fontsize{8.000000}{9.600000}\selectfont Jun}%
\end{pgfscope}%
\begin{pgfscope}%
\pgfsetbuttcap%
\pgfsetroundjoin%
\definecolor{currentfill}{rgb}{0.000000,0.000000,0.000000}%
\pgfsetfillcolor{currentfill}%
\pgfsetlinewidth{0.803000pt}%
\definecolor{currentstroke}{rgb}{0.000000,0.000000,0.000000}%
\pgfsetstrokecolor{currentstroke}%
\pgfsetdash{}{0pt}%
\pgfsys@defobject{currentmarker}{\pgfqpoint{0.000000in}{-0.048611in}}{\pgfqpoint{0.000000in}{0.000000in}}{%
\pgfpathmoveto{\pgfqpoint{0.000000in}{0.000000in}}%
\pgfpathlineto{\pgfqpoint{0.000000in}{-0.048611in}}%
\pgfusepath{stroke,fill}%
}%
\begin{pgfscope}%
\pgfsys@transformshift{2.925283in}{8.035189in}%
\pgfsys@useobject{currentmarker}{}%
\end{pgfscope}%
\end{pgfscope}%
\begin{pgfscope}%
\definecolor{textcolor}{rgb}{0.000000,0.000000,0.000000}%
\pgfsetstrokecolor{textcolor}%
\pgfsetfillcolor{textcolor}%
\pgftext[x=2.925283in,y=7.937967in,,top]{\color{textcolor}\rmfamily\fontsize{8.000000}{9.600000}\selectfont Jul}%
\end{pgfscope}%
\begin{pgfscope}%
\pgfsetbuttcap%
\pgfsetroundjoin%
\definecolor{currentfill}{rgb}{0.000000,0.000000,0.000000}%
\pgfsetfillcolor{currentfill}%
\pgfsetlinewidth{0.803000pt}%
\definecolor{currentstroke}{rgb}{0.000000,0.000000,0.000000}%
\pgfsetstrokecolor{currentstroke}%
\pgfsetdash{}{0pt}%
\pgfsys@defobject{currentmarker}{\pgfqpoint{0.000000in}{-0.048611in}}{\pgfqpoint{0.000000in}{0.000000in}}{%
\pgfpathmoveto{\pgfqpoint{0.000000in}{0.000000in}}%
\pgfpathlineto{\pgfqpoint{0.000000in}{-0.048611in}}%
\pgfusepath{stroke,fill}%
}%
\begin{pgfscope}%
\pgfsys@transformshift{3.320094in}{8.035189in}%
\pgfsys@useobject{currentmarker}{}%
\end{pgfscope}%
\end{pgfscope}%
\begin{pgfscope}%
\definecolor{textcolor}{rgb}{0.000000,0.000000,0.000000}%
\pgfsetstrokecolor{textcolor}%
\pgfsetfillcolor{textcolor}%
\pgftext[x=3.320094in,y=7.937967in,,top]{\color{textcolor}\rmfamily\fontsize{8.000000}{9.600000}\selectfont Aug}%
\end{pgfscope}%
\begin{pgfscope}%
\pgfsetbuttcap%
\pgfsetroundjoin%
\definecolor{currentfill}{rgb}{0.000000,0.000000,0.000000}%
\pgfsetfillcolor{currentfill}%
\pgfsetlinewidth{0.803000pt}%
\definecolor{currentstroke}{rgb}{0.000000,0.000000,0.000000}%
\pgfsetstrokecolor{currentstroke}%
\pgfsetdash{}{0pt}%
\pgfsys@defobject{currentmarker}{\pgfqpoint{0.000000in}{-0.048611in}}{\pgfqpoint{0.000000in}{0.000000in}}{%
\pgfpathmoveto{\pgfqpoint{0.000000in}{0.000000in}}%
\pgfpathlineto{\pgfqpoint{0.000000in}{-0.048611in}}%
\pgfusepath{stroke,fill}%
}%
\begin{pgfscope}%
\pgfsys@transformshift{3.671037in}{8.035189in}%
\pgfsys@useobject{currentmarker}{}%
\end{pgfscope}%
\end{pgfscope}%
\begin{pgfscope}%
\definecolor{textcolor}{rgb}{0.000000,0.000000,0.000000}%
\pgfsetstrokecolor{textcolor}%
\pgfsetfillcolor{textcolor}%
\pgftext[x=3.671037in,y=7.937967in,,top]{\color{textcolor}\rmfamily\fontsize{8.000000}{9.600000}\selectfont Sep}%
\end{pgfscope}%
\begin{pgfscope}%
\pgfsetbuttcap%
\pgfsetroundjoin%
\definecolor{currentfill}{rgb}{0.000000,0.000000,0.000000}%
\pgfsetfillcolor{currentfill}%
\pgfsetlinewidth{0.803000pt}%
\definecolor{currentstroke}{rgb}{0.000000,0.000000,0.000000}%
\pgfsetstrokecolor{currentstroke}%
\pgfsetdash{}{0pt}%
\pgfsys@defobject{currentmarker}{\pgfqpoint{0.000000in}{-0.048611in}}{\pgfqpoint{0.000000in}{0.000000in}}{%
\pgfpathmoveto{\pgfqpoint{0.000000in}{0.000000in}}%
\pgfpathlineto{\pgfqpoint{0.000000in}{-0.048611in}}%
\pgfusepath{stroke,fill}%
}%
\begin{pgfscope}%
\pgfsys@transformshift{4.065849in}{8.035189in}%
\pgfsys@useobject{currentmarker}{}%
\end{pgfscope}%
\end{pgfscope}%
\begin{pgfscope}%
\definecolor{textcolor}{rgb}{0.000000,0.000000,0.000000}%
\pgfsetstrokecolor{textcolor}%
\pgfsetfillcolor{textcolor}%
\pgftext[x=4.065849in,y=7.937967in,,top]{\color{textcolor}\rmfamily\fontsize{8.000000}{9.600000}\selectfont Oct}%
\end{pgfscope}%
\begin{pgfscope}%
\pgfsetbuttcap%
\pgfsetroundjoin%
\definecolor{currentfill}{rgb}{0.000000,0.000000,0.000000}%
\pgfsetfillcolor{currentfill}%
\pgfsetlinewidth{0.803000pt}%
\definecolor{currentstroke}{rgb}{0.000000,0.000000,0.000000}%
\pgfsetstrokecolor{currentstroke}%
\pgfsetdash{}{0pt}%
\pgfsys@defobject{currentmarker}{\pgfqpoint{0.000000in}{-0.048611in}}{\pgfqpoint{0.000000in}{0.000000in}}{%
\pgfpathmoveto{\pgfqpoint{0.000000in}{0.000000in}}%
\pgfpathlineto{\pgfqpoint{0.000000in}{-0.048611in}}%
\pgfusepath{stroke,fill}%
}%
\begin{pgfscope}%
\pgfsys@transformshift{4.460660in}{8.035189in}%
\pgfsys@useobject{currentmarker}{}%
\end{pgfscope}%
\end{pgfscope}%
\begin{pgfscope}%
\definecolor{textcolor}{rgb}{0.000000,0.000000,0.000000}%
\pgfsetstrokecolor{textcolor}%
\pgfsetfillcolor{textcolor}%
\pgftext[x=4.460660in,y=7.937967in,,top]{\color{textcolor}\rmfamily\fontsize{8.000000}{9.600000}\selectfont Nov}%
\end{pgfscope}%
\begin{pgfscope}%
\pgfsetbuttcap%
\pgfsetroundjoin%
\definecolor{currentfill}{rgb}{0.000000,0.000000,0.000000}%
\pgfsetfillcolor{currentfill}%
\pgfsetlinewidth{0.803000pt}%
\definecolor{currentstroke}{rgb}{0.000000,0.000000,0.000000}%
\pgfsetstrokecolor{currentstroke}%
\pgfsetdash{}{0pt}%
\pgfsys@defobject{currentmarker}{\pgfqpoint{0.000000in}{-0.048611in}}{\pgfqpoint{0.000000in}{0.000000in}}{%
\pgfpathmoveto{\pgfqpoint{0.000000in}{0.000000in}}%
\pgfpathlineto{\pgfqpoint{0.000000in}{-0.048611in}}%
\pgfusepath{stroke,fill}%
}%
\begin{pgfscope}%
\pgfsys@transformshift{4.811603in}{8.035189in}%
\pgfsys@useobject{currentmarker}{}%
\end{pgfscope}%
\end{pgfscope}%
\begin{pgfscope}%
\definecolor{textcolor}{rgb}{0.000000,0.000000,0.000000}%
\pgfsetstrokecolor{textcolor}%
\pgfsetfillcolor{textcolor}%
\pgftext[x=4.811603in,y=7.937967in,,top]{\color{textcolor}\rmfamily\fontsize{8.000000}{9.600000}\selectfont Dec}%
\end{pgfscope}%
\begin{pgfscope}%
\pgfsetbuttcap%
\pgfsetroundjoin%
\definecolor{currentfill}{rgb}{0.000000,0.000000,0.000000}%
\pgfsetfillcolor{currentfill}%
\pgfsetlinewidth{0.803000pt}%
\definecolor{currentstroke}{rgb}{0.000000,0.000000,0.000000}%
\pgfsetstrokecolor{currentstroke}%
\pgfsetdash{}{0pt}%
\pgfsys@defobject{currentmarker}{\pgfqpoint{-0.048611in}{0.000000in}}{\pgfqpoint{-0.000000in}{0.000000in}}{%
\pgfpathmoveto{\pgfqpoint{-0.000000in}{0.000000in}}%
\pgfpathlineto{\pgfqpoint{-0.048611in}{0.000000in}}%
\pgfusepath{stroke,fill}%
}%
\begin{pgfscope}%
\pgfsys@transformshift{0.380943in}{8.605472in}%
\pgfsys@useobject{currentmarker}{}%
\end{pgfscope}%
\end{pgfscope}%
\begin{pgfscope}%
\definecolor{textcolor}{rgb}{0.000000,0.000000,0.000000}%
\pgfsetstrokecolor{textcolor}%
\pgfsetfillcolor{textcolor}%
\pgftext[x=0.113117in, y=8.566892in, left, base]{\color{textcolor}\rmfamily\fontsize{8.000000}{9.600000}\selectfont M}%
\end{pgfscope}%
\begin{pgfscope}%
\pgfsetbuttcap%
\pgfsetroundjoin%
\definecolor{currentfill}{rgb}{0.000000,0.000000,0.000000}%
\pgfsetfillcolor{currentfill}%
\pgfsetlinewidth{0.803000pt}%
\definecolor{currentstroke}{rgb}{0.000000,0.000000,0.000000}%
\pgfsetstrokecolor{currentstroke}%
\pgfsetdash{}{0pt}%
\pgfsys@defobject{currentmarker}{\pgfqpoint{-0.048611in}{0.000000in}}{\pgfqpoint{-0.000000in}{0.000000in}}{%
\pgfpathmoveto{\pgfqpoint{-0.000000in}{0.000000in}}%
\pgfpathlineto{\pgfqpoint{-0.048611in}{0.000000in}}%
\pgfusepath{stroke,fill}%
}%
\begin{pgfscope}%
\pgfsys@transformshift{0.380943in}{8.517736in}%
\pgfsys@useobject{currentmarker}{}%
\end{pgfscope}%
\end{pgfscope}%
\begin{pgfscope}%
\definecolor{textcolor}{rgb}{0.000000,0.000000,0.000000}%
\pgfsetstrokecolor{textcolor}%
\pgfsetfillcolor{textcolor}%
\pgftext[x=0.135957in, y=8.479156in, left, base]{\color{textcolor}\rmfamily\fontsize{8.000000}{9.600000}\selectfont T}%
\end{pgfscope}%
\begin{pgfscope}%
\pgfsetbuttcap%
\pgfsetroundjoin%
\definecolor{currentfill}{rgb}{0.000000,0.000000,0.000000}%
\pgfsetfillcolor{currentfill}%
\pgfsetlinewidth{0.803000pt}%
\definecolor{currentstroke}{rgb}{0.000000,0.000000,0.000000}%
\pgfsetstrokecolor{currentstroke}%
\pgfsetdash{}{0pt}%
\pgfsys@defobject{currentmarker}{\pgfqpoint{-0.048611in}{0.000000in}}{\pgfqpoint{-0.000000in}{0.000000in}}{%
\pgfpathmoveto{\pgfqpoint{-0.000000in}{0.000000in}}%
\pgfpathlineto{\pgfqpoint{-0.048611in}{0.000000in}}%
\pgfusepath{stroke,fill}%
}%
\begin{pgfscope}%
\pgfsys@transformshift{0.380943in}{8.430000in}%
\pgfsys@useobject{currentmarker}{}%
\end{pgfscope}%
\end{pgfscope}%
\begin{pgfscope}%
\definecolor{textcolor}{rgb}{0.000000,0.000000,0.000000}%
\pgfsetstrokecolor{textcolor}%
\pgfsetfillcolor{textcolor}%
\pgftext[x=0.100000in, y=8.391420in, left, base]{\color{textcolor}\rmfamily\fontsize{8.000000}{9.600000}\selectfont W}%
\end{pgfscope}%
\begin{pgfscope}%
\pgfsetbuttcap%
\pgfsetroundjoin%
\definecolor{currentfill}{rgb}{0.000000,0.000000,0.000000}%
\pgfsetfillcolor{currentfill}%
\pgfsetlinewidth{0.803000pt}%
\definecolor{currentstroke}{rgb}{0.000000,0.000000,0.000000}%
\pgfsetstrokecolor{currentstroke}%
\pgfsetdash{}{0pt}%
\pgfsys@defobject{currentmarker}{\pgfqpoint{-0.048611in}{0.000000in}}{\pgfqpoint{-0.000000in}{0.000000in}}{%
\pgfpathmoveto{\pgfqpoint{-0.000000in}{0.000000in}}%
\pgfpathlineto{\pgfqpoint{-0.048611in}{0.000000in}}%
\pgfusepath{stroke,fill}%
}%
\begin{pgfscope}%
\pgfsys@transformshift{0.380943in}{8.342264in}%
\pgfsys@useobject{currentmarker}{}%
\end{pgfscope}%
\end{pgfscope}%
\begin{pgfscope}%
\definecolor{textcolor}{rgb}{0.000000,0.000000,0.000000}%
\pgfsetstrokecolor{textcolor}%
\pgfsetfillcolor{textcolor}%
\pgftext[x=0.135957in, y=8.303684in, left, base]{\color{textcolor}\rmfamily\fontsize{8.000000}{9.600000}\selectfont T}%
\end{pgfscope}%
\begin{pgfscope}%
\pgfsetbuttcap%
\pgfsetroundjoin%
\definecolor{currentfill}{rgb}{0.000000,0.000000,0.000000}%
\pgfsetfillcolor{currentfill}%
\pgfsetlinewidth{0.803000pt}%
\definecolor{currentstroke}{rgb}{0.000000,0.000000,0.000000}%
\pgfsetstrokecolor{currentstroke}%
\pgfsetdash{}{0pt}%
\pgfsys@defobject{currentmarker}{\pgfqpoint{-0.048611in}{0.000000in}}{\pgfqpoint{-0.000000in}{0.000000in}}{%
\pgfpathmoveto{\pgfqpoint{-0.000000in}{0.000000in}}%
\pgfpathlineto{\pgfqpoint{-0.048611in}{0.000000in}}%
\pgfusepath{stroke,fill}%
}%
\begin{pgfscope}%
\pgfsys@transformshift{0.380943in}{8.254529in}%
\pgfsys@useobject{currentmarker}{}%
\end{pgfscope}%
\end{pgfscope}%
\begin{pgfscope}%
\definecolor{textcolor}{rgb}{0.000000,0.000000,0.000000}%
\pgfsetstrokecolor{textcolor}%
\pgfsetfillcolor{textcolor}%
\pgftext[x=0.144213in, y=8.215948in, left, base]{\color{textcolor}\rmfamily\fontsize{8.000000}{9.600000}\selectfont F}%
\end{pgfscope}%
\begin{pgfscope}%
\pgfsetbuttcap%
\pgfsetroundjoin%
\definecolor{currentfill}{rgb}{0.000000,0.000000,0.000000}%
\pgfsetfillcolor{currentfill}%
\pgfsetlinewidth{0.803000pt}%
\definecolor{currentstroke}{rgb}{0.000000,0.000000,0.000000}%
\pgfsetstrokecolor{currentstroke}%
\pgfsetdash{}{0pt}%
\pgfsys@defobject{currentmarker}{\pgfqpoint{-0.048611in}{0.000000in}}{\pgfqpoint{-0.000000in}{0.000000in}}{%
\pgfpathmoveto{\pgfqpoint{-0.000000in}{0.000000in}}%
\pgfpathlineto{\pgfqpoint{-0.048611in}{0.000000in}}%
\pgfusepath{stroke,fill}%
}%
\begin{pgfscope}%
\pgfsys@transformshift{0.380943in}{8.166793in}%
\pgfsys@useobject{currentmarker}{}%
\end{pgfscope}%
\end{pgfscope}%
\begin{pgfscope}%
\definecolor{textcolor}{rgb}{0.000000,0.000000,0.000000}%
\pgfsetstrokecolor{textcolor}%
\pgfsetfillcolor{textcolor}%
\pgftext[x=0.155633in, y=8.128212in, left, base]{\color{textcolor}\rmfamily\fontsize{8.000000}{9.600000}\selectfont S}%
\end{pgfscope}%
\begin{pgfscope}%
\pgfsetbuttcap%
\pgfsetroundjoin%
\definecolor{currentfill}{rgb}{0.000000,0.000000,0.000000}%
\pgfsetfillcolor{currentfill}%
\pgfsetlinewidth{0.803000pt}%
\definecolor{currentstroke}{rgb}{0.000000,0.000000,0.000000}%
\pgfsetstrokecolor{currentstroke}%
\pgfsetdash{}{0pt}%
\pgfsys@defobject{currentmarker}{\pgfqpoint{-0.048611in}{0.000000in}}{\pgfqpoint{-0.000000in}{0.000000in}}{%
\pgfpathmoveto{\pgfqpoint{-0.000000in}{0.000000in}}%
\pgfpathlineto{\pgfqpoint{-0.048611in}{0.000000in}}%
\pgfusepath{stroke,fill}%
}%
\begin{pgfscope}%
\pgfsys@transformshift{0.380943in}{8.079057in}%
\pgfsys@useobject{currentmarker}{}%
\end{pgfscope}%
\end{pgfscope}%
\begin{pgfscope}%
\definecolor{textcolor}{rgb}{0.000000,0.000000,0.000000}%
\pgfsetstrokecolor{textcolor}%
\pgfsetfillcolor{textcolor}%
\pgftext[x=0.155633in, y=8.040477in, left, base]{\color{textcolor}\rmfamily\fontsize{8.000000}{9.600000}\selectfont S}%
\end{pgfscope}%
\begin{pgfscope}%
\definecolor{textcolor}{rgb}{0.000000,0.000000,0.000000}%
\pgfsetstrokecolor{textcolor}%
\pgfsetfillcolor{textcolor}%
\pgftext[x=2.705943in,y=8.816007in,,]{\color{textcolor}\ttfamily\fontsize{14.400000}{17.280000}\selectfont 2017}%
\end{pgfscope}%
\begin{pgfscope}%
\pgfpathrectangle{\pgfqpoint{0.380943in}{6.110189in}}{\pgfqpoint{4.650000in}{0.614151in}}%
\pgfusepath{clip}%
\pgfsetbuttcap%
\pgfsetroundjoin%
\definecolor{currentfill}{rgb}{0.991849,0.986144,0.810181}%
\pgfsetfillcolor{currentfill}%
\pgfsetlinewidth{0.250937pt}%
\definecolor{currentstroke}{rgb}{1.000000,1.000000,1.000000}%
\pgfsetstrokecolor{currentstroke}%
\pgfsetdash{}{0pt}%
\pgfpathmoveto{\pgfqpoint{0.380943in}{6.724340in}}%
\pgfpathlineto{\pgfqpoint{0.468679in}{6.724340in}}%
\pgfpathlineto{\pgfqpoint{0.468679in}{6.636604in}}%
\pgfpathlineto{\pgfqpoint{0.380943in}{6.636604in}}%
\pgfpathlineto{\pgfqpoint{0.380943in}{6.724340in}}%
\pgfusepath{stroke,fill}%
\end{pgfscope}%
\begin{pgfscope}%
\pgfpathrectangle{\pgfqpoint{0.380943in}{6.110189in}}{\pgfqpoint{4.650000in}{0.614151in}}%
\pgfusepath{clip}%
\pgfsetbuttcap%
\pgfsetroundjoin%
\definecolor{currentfill}{rgb}{0.992326,0.765229,0.614840}%
\pgfsetfillcolor{currentfill}%
\pgfsetlinewidth{0.250937pt}%
\definecolor{currentstroke}{rgb}{1.000000,1.000000,1.000000}%
\pgfsetstrokecolor{currentstroke}%
\pgfsetdash{}{0pt}%
\pgfpathmoveto{\pgfqpoint{0.468679in}{6.724340in}}%
\pgfpathlineto{\pgfqpoint{0.556415in}{6.724340in}}%
\pgfpathlineto{\pgfqpoint{0.556415in}{6.636604in}}%
\pgfpathlineto{\pgfqpoint{0.468679in}{6.636604in}}%
\pgfpathlineto{\pgfqpoint{0.468679in}{6.724340in}}%
\pgfusepath{stroke,fill}%
\end{pgfscope}%
\begin{pgfscope}%
\pgfpathrectangle{\pgfqpoint{0.380943in}{6.110189in}}{\pgfqpoint{4.650000in}{0.614151in}}%
\pgfusepath{clip}%
\pgfsetbuttcap%
\pgfsetroundjoin%
\definecolor{currentfill}{rgb}{0.981546,0.459977,0.459977}%
\pgfsetfillcolor{currentfill}%
\pgfsetlinewidth{0.250937pt}%
\definecolor{currentstroke}{rgb}{1.000000,1.000000,1.000000}%
\pgfsetstrokecolor{currentstroke}%
\pgfsetdash{}{0pt}%
\pgfpathmoveto{\pgfqpoint{0.556415in}{6.724340in}}%
\pgfpathlineto{\pgfqpoint{0.644151in}{6.724340in}}%
\pgfpathlineto{\pgfqpoint{0.644151in}{6.636604in}}%
\pgfpathlineto{\pgfqpoint{0.556415in}{6.636604in}}%
\pgfpathlineto{\pgfqpoint{0.556415in}{6.724340in}}%
\pgfusepath{stroke,fill}%
\end{pgfscope}%
\begin{pgfscope}%
\pgfpathrectangle{\pgfqpoint{0.380943in}{6.110189in}}{\pgfqpoint{4.650000in}{0.614151in}}%
\pgfusepath{clip}%
\pgfsetbuttcap%
\pgfsetroundjoin%
\definecolor{currentfill}{rgb}{0.968166,0.945882,0.748604}%
\pgfsetfillcolor{currentfill}%
\pgfsetlinewidth{0.250937pt}%
\definecolor{currentstroke}{rgb}{1.000000,1.000000,1.000000}%
\pgfsetstrokecolor{currentstroke}%
\pgfsetdash{}{0pt}%
\pgfpathmoveto{\pgfqpoint{0.644151in}{6.724340in}}%
\pgfpathlineto{\pgfqpoint{0.731886in}{6.724340in}}%
\pgfpathlineto{\pgfqpoint{0.731886in}{6.636604in}}%
\pgfpathlineto{\pgfqpoint{0.644151in}{6.636604in}}%
\pgfpathlineto{\pgfqpoint{0.644151in}{6.724340in}}%
\pgfusepath{stroke,fill}%
\end{pgfscope}%
\begin{pgfscope}%
\pgfpathrectangle{\pgfqpoint{0.380943in}{6.110189in}}{\pgfqpoint{4.650000in}{0.614151in}}%
\pgfusepath{clip}%
\pgfsetbuttcap%
\pgfsetroundjoin%
\definecolor{currentfill}{rgb}{0.962414,0.923552,0.722891}%
\pgfsetfillcolor{currentfill}%
\pgfsetlinewidth{0.250937pt}%
\definecolor{currentstroke}{rgb}{1.000000,1.000000,1.000000}%
\pgfsetstrokecolor{currentstroke}%
\pgfsetdash{}{0pt}%
\pgfpathmoveto{\pgfqpoint{0.731886in}{6.724340in}}%
\pgfpathlineto{\pgfqpoint{0.819622in}{6.724340in}}%
\pgfpathlineto{\pgfqpoint{0.819622in}{6.636604in}}%
\pgfpathlineto{\pgfqpoint{0.731886in}{6.636604in}}%
\pgfpathlineto{\pgfqpoint{0.731886in}{6.724340in}}%
\pgfusepath{stroke,fill}%
\end{pgfscope}%
\begin{pgfscope}%
\pgfpathrectangle{\pgfqpoint{0.380943in}{6.110189in}}{\pgfqpoint{4.650000in}{0.614151in}}%
\pgfusepath{clip}%
\pgfsetbuttcap%
\pgfsetroundjoin%
\definecolor{currentfill}{rgb}{0.992326,0.765229,0.614840}%
\pgfsetfillcolor{currentfill}%
\pgfsetlinewidth{0.250937pt}%
\definecolor{currentstroke}{rgb}{1.000000,1.000000,1.000000}%
\pgfsetstrokecolor{currentstroke}%
\pgfsetdash{}{0pt}%
\pgfpathmoveto{\pgfqpoint{0.819622in}{6.724340in}}%
\pgfpathlineto{\pgfqpoint{0.907358in}{6.724340in}}%
\pgfpathlineto{\pgfqpoint{0.907358in}{6.636604in}}%
\pgfpathlineto{\pgfqpoint{0.819622in}{6.636604in}}%
\pgfpathlineto{\pgfqpoint{0.819622in}{6.724340in}}%
\pgfusepath{stroke,fill}%
\end{pgfscope}%
\begin{pgfscope}%
\pgfpathrectangle{\pgfqpoint{0.380943in}{6.110189in}}{\pgfqpoint{4.650000in}{0.614151in}}%
\pgfusepath{clip}%
\pgfsetbuttcap%
\pgfsetroundjoin%
\definecolor{currentfill}{rgb}{0.996571,0.720538,0.589189}%
\pgfsetfillcolor{currentfill}%
\pgfsetlinewidth{0.250937pt}%
\definecolor{currentstroke}{rgb}{1.000000,1.000000,1.000000}%
\pgfsetstrokecolor{currentstroke}%
\pgfsetdash{}{0pt}%
\pgfpathmoveto{\pgfqpoint{0.907358in}{6.724340in}}%
\pgfpathlineto{\pgfqpoint{0.995094in}{6.724340in}}%
\pgfpathlineto{\pgfqpoint{0.995094in}{6.636604in}}%
\pgfpathlineto{\pgfqpoint{0.907358in}{6.636604in}}%
\pgfpathlineto{\pgfqpoint{0.907358in}{6.724340in}}%
\pgfusepath{stroke,fill}%
\end{pgfscope}%
\begin{pgfscope}%
\pgfpathrectangle{\pgfqpoint{0.380943in}{6.110189in}}{\pgfqpoint{4.650000in}{0.614151in}}%
\pgfusepath{clip}%
\pgfsetbuttcap%
\pgfsetroundjoin%
\definecolor{currentfill}{rgb}{0.972549,0.870588,0.692810}%
\pgfsetfillcolor{currentfill}%
\pgfsetlinewidth{0.250937pt}%
\definecolor{currentstroke}{rgb}{1.000000,1.000000,1.000000}%
\pgfsetstrokecolor{currentstroke}%
\pgfsetdash{}{0pt}%
\pgfpathmoveto{\pgfqpoint{0.995094in}{6.724340in}}%
\pgfpathlineto{\pgfqpoint{1.082830in}{6.724340in}}%
\pgfpathlineto{\pgfqpoint{1.082830in}{6.636604in}}%
\pgfpathlineto{\pgfqpoint{0.995094in}{6.636604in}}%
\pgfpathlineto{\pgfqpoint{0.995094in}{6.724340in}}%
\pgfusepath{stroke,fill}%
\end{pgfscope}%
\begin{pgfscope}%
\pgfpathrectangle{\pgfqpoint{0.380943in}{6.110189in}}{\pgfqpoint{4.650000in}{0.614151in}}%
\pgfusepath{clip}%
\pgfsetbuttcap%
\pgfsetroundjoin%
\definecolor{currentfill}{rgb}{1.000000,0.557862,0.511772}%
\pgfsetfillcolor{currentfill}%
\pgfsetlinewidth{0.250937pt}%
\definecolor{currentstroke}{rgb}{1.000000,1.000000,1.000000}%
\pgfsetstrokecolor{currentstroke}%
\pgfsetdash{}{0pt}%
\pgfpathmoveto{\pgfqpoint{1.082830in}{6.724340in}}%
\pgfpathlineto{\pgfqpoint{1.170566in}{6.724340in}}%
\pgfpathlineto{\pgfqpoint{1.170566in}{6.636604in}}%
\pgfpathlineto{\pgfqpoint{1.082830in}{6.636604in}}%
\pgfpathlineto{\pgfqpoint{1.082830in}{6.724340in}}%
\pgfusepath{stroke,fill}%
\end{pgfscope}%
\begin{pgfscope}%
\pgfpathrectangle{\pgfqpoint{0.380943in}{6.110189in}}{\pgfqpoint{4.650000in}{0.614151in}}%
\pgfusepath{clip}%
\pgfsetbuttcap%
\pgfsetroundjoin%
\definecolor{currentfill}{rgb}{0.986759,0.806398,0.641200}%
\pgfsetfillcolor{currentfill}%
\pgfsetlinewidth{0.250937pt}%
\definecolor{currentstroke}{rgb}{1.000000,1.000000,1.000000}%
\pgfsetstrokecolor{currentstroke}%
\pgfsetdash{}{0pt}%
\pgfpathmoveto{\pgfqpoint{1.170566in}{6.724340in}}%
\pgfpathlineto{\pgfqpoint{1.258302in}{6.724340in}}%
\pgfpathlineto{\pgfqpoint{1.258302in}{6.636604in}}%
\pgfpathlineto{\pgfqpoint{1.170566in}{6.636604in}}%
\pgfpathlineto{\pgfqpoint{1.170566in}{6.724340in}}%
\pgfusepath{stroke,fill}%
\end{pgfscope}%
\begin{pgfscope}%
\pgfpathrectangle{\pgfqpoint{0.380943in}{6.110189in}}{\pgfqpoint{4.650000in}{0.614151in}}%
\pgfusepath{clip}%
\pgfsetbuttcap%
\pgfsetroundjoin%
\definecolor{currentfill}{rgb}{0.979654,0.837186,0.669619}%
\pgfsetfillcolor{currentfill}%
\pgfsetlinewidth{0.250937pt}%
\definecolor{currentstroke}{rgb}{1.000000,1.000000,1.000000}%
\pgfsetstrokecolor{currentstroke}%
\pgfsetdash{}{0pt}%
\pgfpathmoveto{\pgfqpoint{1.258302in}{6.724340in}}%
\pgfpathlineto{\pgfqpoint{1.346037in}{6.724340in}}%
\pgfpathlineto{\pgfqpoint{1.346037in}{6.636604in}}%
\pgfpathlineto{\pgfqpoint{1.258302in}{6.636604in}}%
\pgfpathlineto{\pgfqpoint{1.258302in}{6.724340in}}%
\pgfusepath{stroke,fill}%
\end{pgfscope}%
\begin{pgfscope}%
\pgfpathrectangle{\pgfqpoint{0.380943in}{6.110189in}}{\pgfqpoint{4.650000in}{0.614151in}}%
\pgfusepath{clip}%
\pgfsetbuttcap%
\pgfsetroundjoin%
\definecolor{currentfill}{rgb}{0.996571,0.720538,0.589189}%
\pgfsetfillcolor{currentfill}%
\pgfsetlinewidth{0.250937pt}%
\definecolor{currentstroke}{rgb}{1.000000,1.000000,1.000000}%
\pgfsetstrokecolor{currentstroke}%
\pgfsetdash{}{0pt}%
\pgfpathmoveto{\pgfqpoint{1.346037in}{6.724340in}}%
\pgfpathlineto{\pgfqpoint{1.433773in}{6.724340in}}%
\pgfpathlineto{\pgfqpoint{1.433773in}{6.636604in}}%
\pgfpathlineto{\pgfqpoint{1.346037in}{6.636604in}}%
\pgfpathlineto{\pgfqpoint{1.346037in}{6.724340in}}%
\pgfusepath{stroke,fill}%
\end{pgfscope}%
\begin{pgfscope}%
\pgfpathrectangle{\pgfqpoint{0.380943in}{6.110189in}}{\pgfqpoint{4.650000in}{0.614151in}}%
\pgfusepath{clip}%
\pgfsetbuttcap%
\pgfsetroundjoin%
\definecolor{currentfill}{rgb}{0.979654,0.837186,0.669619}%
\pgfsetfillcolor{currentfill}%
\pgfsetlinewidth{0.250937pt}%
\definecolor{currentstroke}{rgb}{1.000000,1.000000,1.000000}%
\pgfsetstrokecolor{currentstroke}%
\pgfsetdash{}{0pt}%
\pgfpathmoveto{\pgfqpoint{1.433773in}{6.724340in}}%
\pgfpathlineto{\pgfqpoint{1.521509in}{6.724340in}}%
\pgfpathlineto{\pgfqpoint{1.521509in}{6.636604in}}%
\pgfpathlineto{\pgfqpoint{1.433773in}{6.636604in}}%
\pgfpathlineto{\pgfqpoint{1.433773in}{6.724340in}}%
\pgfusepath{stroke,fill}%
\end{pgfscope}%
\begin{pgfscope}%
\pgfpathrectangle{\pgfqpoint{0.380943in}{6.110189in}}{\pgfqpoint{4.650000in}{0.614151in}}%
\pgfusepath{clip}%
\pgfsetbuttcap%
\pgfsetroundjoin%
\definecolor{currentfill}{rgb}{0.991849,0.986144,0.810181}%
\pgfsetfillcolor{currentfill}%
\pgfsetlinewidth{0.250937pt}%
\definecolor{currentstroke}{rgb}{1.000000,1.000000,1.000000}%
\pgfsetstrokecolor{currentstroke}%
\pgfsetdash{}{0pt}%
\pgfpathmoveto{\pgfqpoint{1.521509in}{6.724340in}}%
\pgfpathlineto{\pgfqpoint{1.609245in}{6.724340in}}%
\pgfpathlineto{\pgfqpoint{1.609245in}{6.636604in}}%
\pgfpathlineto{\pgfqpoint{1.521509in}{6.636604in}}%
\pgfpathlineto{\pgfqpoint{1.521509in}{6.724340in}}%
\pgfusepath{stroke,fill}%
\end{pgfscope}%
\begin{pgfscope}%
\pgfpathrectangle{\pgfqpoint{0.380943in}{6.110189in}}{\pgfqpoint{4.650000in}{0.614151in}}%
\pgfusepath{clip}%
\pgfsetbuttcap%
\pgfsetroundjoin%
\definecolor{currentfill}{rgb}{0.979654,0.837186,0.669619}%
\pgfsetfillcolor{currentfill}%
\pgfsetlinewidth{0.250937pt}%
\definecolor{currentstroke}{rgb}{1.000000,1.000000,1.000000}%
\pgfsetstrokecolor{currentstroke}%
\pgfsetdash{}{0pt}%
\pgfpathmoveto{\pgfqpoint{1.609245in}{6.724340in}}%
\pgfpathlineto{\pgfqpoint{1.696981in}{6.724340in}}%
\pgfpathlineto{\pgfqpoint{1.696981in}{6.636604in}}%
\pgfpathlineto{\pgfqpoint{1.609245in}{6.636604in}}%
\pgfpathlineto{\pgfqpoint{1.609245in}{6.724340in}}%
\pgfusepath{stroke,fill}%
\end{pgfscope}%
\begin{pgfscope}%
\pgfpathrectangle{\pgfqpoint{0.380943in}{6.110189in}}{\pgfqpoint{4.650000in}{0.614151in}}%
\pgfusepath{clip}%
\pgfsetbuttcap%
\pgfsetroundjoin%
\definecolor{currentfill}{rgb}{0.986759,0.806398,0.641200}%
\pgfsetfillcolor{currentfill}%
\pgfsetlinewidth{0.250937pt}%
\definecolor{currentstroke}{rgb}{1.000000,1.000000,1.000000}%
\pgfsetstrokecolor{currentstroke}%
\pgfsetdash{}{0pt}%
\pgfpathmoveto{\pgfqpoint{1.696981in}{6.724340in}}%
\pgfpathlineto{\pgfqpoint{1.784717in}{6.724340in}}%
\pgfpathlineto{\pgfqpoint{1.784717in}{6.636604in}}%
\pgfpathlineto{\pgfqpoint{1.696981in}{6.636604in}}%
\pgfpathlineto{\pgfqpoint{1.696981in}{6.724340in}}%
\pgfusepath{stroke,fill}%
\end{pgfscope}%
\begin{pgfscope}%
\pgfpathrectangle{\pgfqpoint{0.380943in}{6.110189in}}{\pgfqpoint{4.650000in}{0.614151in}}%
\pgfusepath{clip}%
\pgfsetbuttcap%
\pgfsetroundjoin%
\definecolor{currentfill}{rgb}{0.986759,0.806398,0.641200}%
\pgfsetfillcolor{currentfill}%
\pgfsetlinewidth{0.250937pt}%
\definecolor{currentstroke}{rgb}{1.000000,1.000000,1.000000}%
\pgfsetstrokecolor{currentstroke}%
\pgfsetdash{}{0pt}%
\pgfpathmoveto{\pgfqpoint{1.784717in}{6.724340in}}%
\pgfpathlineto{\pgfqpoint{1.872452in}{6.724340in}}%
\pgfpathlineto{\pgfqpoint{1.872452in}{6.636604in}}%
\pgfpathlineto{\pgfqpoint{1.784717in}{6.636604in}}%
\pgfpathlineto{\pgfqpoint{1.784717in}{6.724340in}}%
\pgfusepath{stroke,fill}%
\end{pgfscope}%
\begin{pgfscope}%
\pgfpathrectangle{\pgfqpoint{0.380943in}{6.110189in}}{\pgfqpoint{4.650000in}{0.614151in}}%
\pgfusepath{clip}%
\pgfsetbuttcap%
\pgfsetroundjoin%
\definecolor{currentfill}{rgb}{1.000000,0.557862,0.511772}%
\pgfsetfillcolor{currentfill}%
\pgfsetlinewidth{0.250937pt}%
\definecolor{currentstroke}{rgb}{1.000000,1.000000,1.000000}%
\pgfsetstrokecolor{currentstroke}%
\pgfsetdash{}{0pt}%
\pgfpathmoveto{\pgfqpoint{1.872452in}{6.724340in}}%
\pgfpathlineto{\pgfqpoint{1.960188in}{6.724340in}}%
\pgfpathlineto{\pgfqpoint{1.960188in}{6.636604in}}%
\pgfpathlineto{\pgfqpoint{1.872452in}{6.636604in}}%
\pgfpathlineto{\pgfqpoint{1.872452in}{6.724340in}}%
\pgfusepath{stroke,fill}%
\end{pgfscope}%
\begin{pgfscope}%
\pgfpathrectangle{\pgfqpoint{0.380943in}{6.110189in}}{\pgfqpoint{4.650000in}{0.614151in}}%
\pgfusepath{clip}%
\pgfsetbuttcap%
\pgfsetroundjoin%
\definecolor{currentfill}{rgb}{0.965444,0.906113,0.711757}%
\pgfsetfillcolor{currentfill}%
\pgfsetlinewidth{0.250937pt}%
\definecolor{currentstroke}{rgb}{1.000000,1.000000,1.000000}%
\pgfsetstrokecolor{currentstroke}%
\pgfsetdash{}{0pt}%
\pgfpathmoveto{\pgfqpoint{1.960188in}{6.724340in}}%
\pgfpathlineto{\pgfqpoint{2.047924in}{6.724340in}}%
\pgfpathlineto{\pgfqpoint{2.047924in}{6.636604in}}%
\pgfpathlineto{\pgfqpoint{1.960188in}{6.636604in}}%
\pgfpathlineto{\pgfqpoint{1.960188in}{6.724340in}}%
\pgfusepath{stroke,fill}%
\end{pgfscope}%
\begin{pgfscope}%
\pgfpathrectangle{\pgfqpoint{0.380943in}{6.110189in}}{\pgfqpoint{4.650000in}{0.614151in}}%
\pgfusepath{clip}%
\pgfsetbuttcap%
\pgfsetroundjoin%
\definecolor{currentfill}{rgb}{0.996571,0.720538,0.589189}%
\pgfsetfillcolor{currentfill}%
\pgfsetlinewidth{0.250937pt}%
\definecolor{currentstroke}{rgb}{1.000000,1.000000,1.000000}%
\pgfsetstrokecolor{currentstroke}%
\pgfsetdash{}{0pt}%
\pgfpathmoveto{\pgfqpoint{2.047924in}{6.724340in}}%
\pgfpathlineto{\pgfqpoint{2.135660in}{6.724340in}}%
\pgfpathlineto{\pgfqpoint{2.135660in}{6.636604in}}%
\pgfpathlineto{\pgfqpoint{2.047924in}{6.636604in}}%
\pgfpathlineto{\pgfqpoint{2.047924in}{6.724340in}}%
\pgfusepath{stroke,fill}%
\end{pgfscope}%
\begin{pgfscope}%
\pgfpathrectangle{\pgfqpoint{0.380943in}{6.110189in}}{\pgfqpoint{4.650000in}{0.614151in}}%
\pgfusepath{clip}%
\pgfsetbuttcap%
\pgfsetroundjoin%
\definecolor{currentfill}{rgb}{1.000000,1.000000,0.870204}%
\pgfsetfillcolor{currentfill}%
\pgfsetlinewidth{0.250937pt}%
\definecolor{currentstroke}{rgb}{1.000000,1.000000,1.000000}%
\pgfsetstrokecolor{currentstroke}%
\pgfsetdash{}{0pt}%
\pgfpathmoveto{\pgfqpoint{2.135660in}{6.724340in}}%
\pgfpathlineto{\pgfqpoint{2.223396in}{6.724340in}}%
\pgfpathlineto{\pgfqpoint{2.223396in}{6.636604in}}%
\pgfpathlineto{\pgfqpoint{2.135660in}{6.636604in}}%
\pgfpathlineto{\pgfqpoint{2.135660in}{6.724340in}}%
\pgfusepath{stroke,fill}%
\end{pgfscope}%
\begin{pgfscope}%
\pgfpathrectangle{\pgfqpoint{0.380943in}{6.110189in}}{\pgfqpoint{4.650000in}{0.614151in}}%
\pgfusepath{clip}%
\pgfsetbuttcap%
\pgfsetroundjoin%
\definecolor{currentfill}{rgb}{0.962414,0.923552,0.722891}%
\pgfsetfillcolor{currentfill}%
\pgfsetlinewidth{0.250937pt}%
\definecolor{currentstroke}{rgb}{1.000000,1.000000,1.000000}%
\pgfsetstrokecolor{currentstroke}%
\pgfsetdash{}{0pt}%
\pgfpathmoveto{\pgfqpoint{2.223396in}{6.724340in}}%
\pgfpathlineto{\pgfqpoint{2.311132in}{6.724340in}}%
\pgfpathlineto{\pgfqpoint{2.311132in}{6.636604in}}%
\pgfpathlineto{\pgfqpoint{2.223396in}{6.636604in}}%
\pgfpathlineto{\pgfqpoint{2.223396in}{6.724340in}}%
\pgfusepath{stroke,fill}%
\end{pgfscope}%
\begin{pgfscope}%
\pgfpathrectangle{\pgfqpoint{0.380943in}{6.110189in}}{\pgfqpoint{4.650000in}{0.614151in}}%
\pgfusepath{clip}%
\pgfsetbuttcap%
\pgfsetroundjoin%
\definecolor{currentfill}{rgb}{0.992326,0.765229,0.614840}%
\pgfsetfillcolor{currentfill}%
\pgfsetlinewidth{0.250937pt}%
\definecolor{currentstroke}{rgb}{1.000000,1.000000,1.000000}%
\pgfsetstrokecolor{currentstroke}%
\pgfsetdash{}{0pt}%
\pgfpathmoveto{\pgfqpoint{2.311132in}{6.724340in}}%
\pgfpathlineto{\pgfqpoint{2.398868in}{6.724340in}}%
\pgfpathlineto{\pgfqpoint{2.398868in}{6.636604in}}%
\pgfpathlineto{\pgfqpoint{2.311132in}{6.636604in}}%
\pgfpathlineto{\pgfqpoint{2.311132in}{6.724340in}}%
\pgfusepath{stroke,fill}%
\end{pgfscope}%
\begin{pgfscope}%
\pgfpathrectangle{\pgfqpoint{0.380943in}{6.110189in}}{\pgfqpoint{4.650000in}{0.614151in}}%
\pgfusepath{clip}%
\pgfsetbuttcap%
\pgfsetroundjoin%
\definecolor{currentfill}{rgb}{0.992326,0.765229,0.614840}%
\pgfsetfillcolor{currentfill}%
\pgfsetlinewidth{0.250937pt}%
\definecolor{currentstroke}{rgb}{1.000000,1.000000,1.000000}%
\pgfsetstrokecolor{currentstroke}%
\pgfsetdash{}{0pt}%
\pgfpathmoveto{\pgfqpoint{2.398868in}{6.724340in}}%
\pgfpathlineto{\pgfqpoint{2.486603in}{6.724340in}}%
\pgfpathlineto{\pgfqpoint{2.486603in}{6.636604in}}%
\pgfpathlineto{\pgfqpoint{2.398868in}{6.636604in}}%
\pgfpathlineto{\pgfqpoint{2.398868in}{6.724340in}}%
\pgfusepath{stroke,fill}%
\end{pgfscope}%
\begin{pgfscope}%
\pgfpathrectangle{\pgfqpoint{0.380943in}{6.110189in}}{\pgfqpoint{4.650000in}{0.614151in}}%
\pgfusepath{clip}%
\pgfsetbuttcap%
\pgfsetroundjoin%
\definecolor{currentfill}{rgb}{0.996571,0.720538,0.589189}%
\pgfsetfillcolor{currentfill}%
\pgfsetlinewidth{0.250937pt}%
\definecolor{currentstroke}{rgb}{1.000000,1.000000,1.000000}%
\pgfsetstrokecolor{currentstroke}%
\pgfsetdash{}{0pt}%
\pgfpathmoveto{\pgfqpoint{2.486603in}{6.724340in}}%
\pgfpathlineto{\pgfqpoint{2.574339in}{6.724340in}}%
\pgfpathlineto{\pgfqpoint{2.574339in}{6.636604in}}%
\pgfpathlineto{\pgfqpoint{2.486603in}{6.636604in}}%
\pgfpathlineto{\pgfqpoint{2.486603in}{6.724340in}}%
\pgfusepath{stroke,fill}%
\end{pgfscope}%
\begin{pgfscope}%
\pgfpathrectangle{\pgfqpoint{0.380943in}{6.110189in}}{\pgfqpoint{4.650000in}{0.614151in}}%
\pgfusepath{clip}%
\pgfsetbuttcap%
\pgfsetroundjoin%
\definecolor{currentfill}{rgb}{0.986759,0.806398,0.641200}%
\pgfsetfillcolor{currentfill}%
\pgfsetlinewidth{0.250937pt}%
\definecolor{currentstroke}{rgb}{1.000000,1.000000,1.000000}%
\pgfsetstrokecolor{currentstroke}%
\pgfsetdash{}{0pt}%
\pgfpathmoveto{\pgfqpoint{2.574339in}{6.724340in}}%
\pgfpathlineto{\pgfqpoint{2.662075in}{6.724340in}}%
\pgfpathlineto{\pgfqpoint{2.662075in}{6.636604in}}%
\pgfpathlineto{\pgfqpoint{2.574339in}{6.636604in}}%
\pgfpathlineto{\pgfqpoint{2.574339in}{6.724340in}}%
\pgfusepath{stroke,fill}%
\end{pgfscope}%
\begin{pgfscope}%
\pgfpathrectangle{\pgfqpoint{0.380943in}{6.110189in}}{\pgfqpoint{4.650000in}{0.614151in}}%
\pgfusepath{clip}%
\pgfsetbuttcap%
\pgfsetroundjoin%
\definecolor{currentfill}{rgb}{0.972549,0.870588,0.692810}%
\pgfsetfillcolor{currentfill}%
\pgfsetlinewidth{0.250937pt}%
\definecolor{currentstroke}{rgb}{1.000000,1.000000,1.000000}%
\pgfsetstrokecolor{currentstroke}%
\pgfsetdash{}{0pt}%
\pgfpathmoveto{\pgfqpoint{2.662075in}{6.724340in}}%
\pgfpathlineto{\pgfqpoint{2.749811in}{6.724340in}}%
\pgfpathlineto{\pgfqpoint{2.749811in}{6.636604in}}%
\pgfpathlineto{\pgfqpoint{2.662075in}{6.636604in}}%
\pgfpathlineto{\pgfqpoint{2.662075in}{6.724340in}}%
\pgfusepath{stroke,fill}%
\end{pgfscope}%
\begin{pgfscope}%
\pgfpathrectangle{\pgfqpoint{0.380943in}{6.110189in}}{\pgfqpoint{4.650000in}{0.614151in}}%
\pgfusepath{clip}%
\pgfsetbuttcap%
\pgfsetroundjoin%
\definecolor{currentfill}{rgb}{0.992326,0.765229,0.614840}%
\pgfsetfillcolor{currentfill}%
\pgfsetlinewidth{0.250937pt}%
\definecolor{currentstroke}{rgb}{1.000000,1.000000,1.000000}%
\pgfsetstrokecolor{currentstroke}%
\pgfsetdash{}{0pt}%
\pgfpathmoveto{\pgfqpoint{2.749811in}{6.724340in}}%
\pgfpathlineto{\pgfqpoint{2.837547in}{6.724340in}}%
\pgfpathlineto{\pgfqpoint{2.837547in}{6.636604in}}%
\pgfpathlineto{\pgfqpoint{2.749811in}{6.636604in}}%
\pgfpathlineto{\pgfqpoint{2.749811in}{6.724340in}}%
\pgfusepath{stroke,fill}%
\end{pgfscope}%
\begin{pgfscope}%
\pgfpathrectangle{\pgfqpoint{0.380943in}{6.110189in}}{\pgfqpoint{4.650000in}{0.614151in}}%
\pgfusepath{clip}%
\pgfsetbuttcap%
\pgfsetroundjoin%
\definecolor{currentfill}{rgb}{0.986759,0.806398,0.641200}%
\pgfsetfillcolor{currentfill}%
\pgfsetlinewidth{0.250937pt}%
\definecolor{currentstroke}{rgb}{1.000000,1.000000,1.000000}%
\pgfsetstrokecolor{currentstroke}%
\pgfsetdash{}{0pt}%
\pgfpathmoveto{\pgfqpoint{2.837547in}{6.724340in}}%
\pgfpathlineto{\pgfqpoint{2.925283in}{6.724340in}}%
\pgfpathlineto{\pgfqpoint{2.925283in}{6.636604in}}%
\pgfpathlineto{\pgfqpoint{2.837547in}{6.636604in}}%
\pgfpathlineto{\pgfqpoint{2.837547in}{6.724340in}}%
\pgfusepath{stroke,fill}%
\end{pgfscope}%
\begin{pgfscope}%
\pgfpathrectangle{\pgfqpoint{0.380943in}{6.110189in}}{\pgfqpoint{4.650000in}{0.614151in}}%
\pgfusepath{clip}%
\pgfsetbuttcap%
\pgfsetroundjoin%
\definecolor{currentfill}{rgb}{0.965444,0.906113,0.711757}%
\pgfsetfillcolor{currentfill}%
\pgfsetlinewidth{0.250937pt}%
\definecolor{currentstroke}{rgb}{1.000000,1.000000,1.000000}%
\pgfsetstrokecolor{currentstroke}%
\pgfsetdash{}{0pt}%
\pgfpathmoveto{\pgfqpoint{2.925283in}{6.724340in}}%
\pgfpathlineto{\pgfqpoint{3.013019in}{6.724340in}}%
\pgfpathlineto{\pgfqpoint{3.013019in}{6.636604in}}%
\pgfpathlineto{\pgfqpoint{2.925283in}{6.636604in}}%
\pgfpathlineto{\pgfqpoint{2.925283in}{6.724340in}}%
\pgfusepath{stroke,fill}%
\end{pgfscope}%
\begin{pgfscope}%
\pgfpathrectangle{\pgfqpoint{0.380943in}{6.110189in}}{\pgfqpoint{4.650000in}{0.614151in}}%
\pgfusepath{clip}%
\pgfsetbuttcap%
\pgfsetroundjoin%
\definecolor{currentfill}{rgb}{0.979654,0.837186,0.669619}%
\pgfsetfillcolor{currentfill}%
\pgfsetlinewidth{0.250937pt}%
\definecolor{currentstroke}{rgb}{1.000000,1.000000,1.000000}%
\pgfsetstrokecolor{currentstroke}%
\pgfsetdash{}{0pt}%
\pgfpathmoveto{\pgfqpoint{3.013019in}{6.724340in}}%
\pgfpathlineto{\pgfqpoint{3.100754in}{6.724340in}}%
\pgfpathlineto{\pgfqpoint{3.100754in}{6.636604in}}%
\pgfpathlineto{\pgfqpoint{3.013019in}{6.636604in}}%
\pgfpathlineto{\pgfqpoint{3.013019in}{6.724340in}}%
\pgfusepath{stroke,fill}%
\end{pgfscope}%
\begin{pgfscope}%
\pgfpathrectangle{\pgfqpoint{0.380943in}{6.110189in}}{\pgfqpoint{4.650000in}{0.614151in}}%
\pgfusepath{clip}%
\pgfsetbuttcap%
\pgfsetroundjoin%
\definecolor{currentfill}{rgb}{0.962414,0.923552,0.722891}%
\pgfsetfillcolor{currentfill}%
\pgfsetlinewidth{0.250937pt}%
\definecolor{currentstroke}{rgb}{1.000000,1.000000,1.000000}%
\pgfsetstrokecolor{currentstroke}%
\pgfsetdash{}{0pt}%
\pgfpathmoveto{\pgfqpoint{3.100754in}{6.724340in}}%
\pgfpathlineto{\pgfqpoint{3.188490in}{6.724340in}}%
\pgfpathlineto{\pgfqpoint{3.188490in}{6.636604in}}%
\pgfpathlineto{\pgfqpoint{3.100754in}{6.636604in}}%
\pgfpathlineto{\pgfqpoint{3.100754in}{6.724340in}}%
\pgfusepath{stroke,fill}%
\end{pgfscope}%
\begin{pgfscope}%
\pgfpathrectangle{\pgfqpoint{0.380943in}{6.110189in}}{\pgfqpoint{4.650000in}{0.614151in}}%
\pgfusepath{clip}%
\pgfsetbuttcap%
\pgfsetroundjoin%
\definecolor{currentfill}{rgb}{0.992326,0.765229,0.614840}%
\pgfsetfillcolor{currentfill}%
\pgfsetlinewidth{0.250937pt}%
\definecolor{currentstroke}{rgb}{1.000000,1.000000,1.000000}%
\pgfsetstrokecolor{currentstroke}%
\pgfsetdash{}{0pt}%
\pgfpathmoveto{\pgfqpoint{3.188490in}{6.724340in}}%
\pgfpathlineto{\pgfqpoint{3.276226in}{6.724340in}}%
\pgfpathlineto{\pgfqpoint{3.276226in}{6.636604in}}%
\pgfpathlineto{\pgfqpoint{3.188490in}{6.636604in}}%
\pgfpathlineto{\pgfqpoint{3.188490in}{6.724340in}}%
\pgfusepath{stroke,fill}%
\end{pgfscope}%
\begin{pgfscope}%
\pgfpathrectangle{\pgfqpoint{0.380943in}{6.110189in}}{\pgfqpoint{4.650000in}{0.614151in}}%
\pgfusepath{clip}%
\pgfsetbuttcap%
\pgfsetroundjoin%
\definecolor{currentfill}{rgb}{0.968166,0.945882,0.748604}%
\pgfsetfillcolor{currentfill}%
\pgfsetlinewidth{0.250937pt}%
\definecolor{currentstroke}{rgb}{1.000000,1.000000,1.000000}%
\pgfsetstrokecolor{currentstroke}%
\pgfsetdash{}{0pt}%
\pgfpathmoveto{\pgfqpoint{3.276226in}{6.724340in}}%
\pgfpathlineto{\pgfqpoint{3.363962in}{6.724340in}}%
\pgfpathlineto{\pgfqpoint{3.363962in}{6.636604in}}%
\pgfpathlineto{\pgfqpoint{3.276226in}{6.636604in}}%
\pgfpathlineto{\pgfqpoint{3.276226in}{6.724340in}}%
\pgfusepath{stroke,fill}%
\end{pgfscope}%
\begin{pgfscope}%
\pgfpathrectangle{\pgfqpoint{0.380943in}{6.110189in}}{\pgfqpoint{4.650000in}{0.614151in}}%
\pgfusepath{clip}%
\pgfsetbuttcap%
\pgfsetroundjoin%
\definecolor{currentfill}{rgb}{0.991849,0.986144,0.810181}%
\pgfsetfillcolor{currentfill}%
\pgfsetlinewidth{0.250937pt}%
\definecolor{currentstroke}{rgb}{1.000000,1.000000,1.000000}%
\pgfsetstrokecolor{currentstroke}%
\pgfsetdash{}{0pt}%
\pgfpathmoveto{\pgfqpoint{3.363962in}{6.724340in}}%
\pgfpathlineto{\pgfqpoint{3.451698in}{6.724340in}}%
\pgfpathlineto{\pgfqpoint{3.451698in}{6.636604in}}%
\pgfpathlineto{\pgfqpoint{3.363962in}{6.636604in}}%
\pgfpathlineto{\pgfqpoint{3.363962in}{6.724340in}}%
\pgfusepath{stroke,fill}%
\end{pgfscope}%
\begin{pgfscope}%
\pgfpathrectangle{\pgfqpoint{0.380943in}{6.110189in}}{\pgfqpoint{4.650000in}{0.614151in}}%
\pgfusepath{clip}%
\pgfsetbuttcap%
\pgfsetroundjoin%
\definecolor{currentfill}{rgb}{0.962414,0.923552,0.722891}%
\pgfsetfillcolor{currentfill}%
\pgfsetlinewidth{0.250937pt}%
\definecolor{currentstroke}{rgb}{1.000000,1.000000,1.000000}%
\pgfsetstrokecolor{currentstroke}%
\pgfsetdash{}{0pt}%
\pgfpathmoveto{\pgfqpoint{3.451698in}{6.724340in}}%
\pgfpathlineto{\pgfqpoint{3.539434in}{6.724340in}}%
\pgfpathlineto{\pgfqpoint{3.539434in}{6.636604in}}%
\pgfpathlineto{\pgfqpoint{3.451698in}{6.636604in}}%
\pgfpathlineto{\pgfqpoint{3.451698in}{6.724340in}}%
\pgfusepath{stroke,fill}%
\end{pgfscope}%
\begin{pgfscope}%
\pgfpathrectangle{\pgfqpoint{0.380943in}{6.110189in}}{\pgfqpoint{4.650000in}{0.614151in}}%
\pgfusepath{clip}%
\pgfsetbuttcap%
\pgfsetroundjoin%
\definecolor{currentfill}{rgb}{0.965444,0.906113,0.711757}%
\pgfsetfillcolor{currentfill}%
\pgfsetlinewidth{0.250937pt}%
\definecolor{currentstroke}{rgb}{1.000000,1.000000,1.000000}%
\pgfsetstrokecolor{currentstroke}%
\pgfsetdash{}{0pt}%
\pgfpathmoveto{\pgfqpoint{3.539434in}{6.724340in}}%
\pgfpathlineto{\pgfqpoint{3.627169in}{6.724340in}}%
\pgfpathlineto{\pgfqpoint{3.627169in}{6.636604in}}%
\pgfpathlineto{\pgfqpoint{3.539434in}{6.636604in}}%
\pgfpathlineto{\pgfqpoint{3.539434in}{6.724340in}}%
\pgfusepath{stroke,fill}%
\end{pgfscope}%
\begin{pgfscope}%
\pgfpathrectangle{\pgfqpoint{0.380943in}{6.110189in}}{\pgfqpoint{4.650000in}{0.614151in}}%
\pgfusepath{clip}%
\pgfsetbuttcap%
\pgfsetroundjoin%
\definecolor{currentfill}{rgb}{0.986759,0.806398,0.641200}%
\pgfsetfillcolor{currentfill}%
\pgfsetlinewidth{0.250937pt}%
\definecolor{currentstroke}{rgb}{1.000000,1.000000,1.000000}%
\pgfsetstrokecolor{currentstroke}%
\pgfsetdash{}{0pt}%
\pgfpathmoveto{\pgfqpoint{3.627169in}{6.724340in}}%
\pgfpathlineto{\pgfqpoint{3.714905in}{6.724340in}}%
\pgfpathlineto{\pgfqpoint{3.714905in}{6.636604in}}%
\pgfpathlineto{\pgfqpoint{3.627169in}{6.636604in}}%
\pgfpathlineto{\pgfqpoint{3.627169in}{6.724340in}}%
\pgfusepath{stroke,fill}%
\end{pgfscope}%
\begin{pgfscope}%
\pgfpathrectangle{\pgfqpoint{0.380943in}{6.110189in}}{\pgfqpoint{4.650000in}{0.614151in}}%
\pgfusepath{clip}%
\pgfsetbuttcap%
\pgfsetroundjoin%
\definecolor{currentfill}{rgb}{0.968166,0.945882,0.748604}%
\pgfsetfillcolor{currentfill}%
\pgfsetlinewidth{0.250937pt}%
\definecolor{currentstroke}{rgb}{1.000000,1.000000,1.000000}%
\pgfsetstrokecolor{currentstroke}%
\pgfsetdash{}{0pt}%
\pgfpathmoveto{\pgfqpoint{3.714905in}{6.724340in}}%
\pgfpathlineto{\pgfqpoint{3.802641in}{6.724340in}}%
\pgfpathlineto{\pgfqpoint{3.802641in}{6.636604in}}%
\pgfpathlineto{\pgfqpoint{3.714905in}{6.636604in}}%
\pgfpathlineto{\pgfqpoint{3.714905in}{6.724340in}}%
\pgfusepath{stroke,fill}%
\end{pgfscope}%
\begin{pgfscope}%
\pgfpathrectangle{\pgfqpoint{0.380943in}{6.110189in}}{\pgfqpoint{4.650000in}{0.614151in}}%
\pgfusepath{clip}%
\pgfsetbuttcap%
\pgfsetroundjoin%
\definecolor{currentfill}{rgb}{0.996571,0.720538,0.589189}%
\pgfsetfillcolor{currentfill}%
\pgfsetlinewidth{0.250937pt}%
\definecolor{currentstroke}{rgb}{1.000000,1.000000,1.000000}%
\pgfsetstrokecolor{currentstroke}%
\pgfsetdash{}{0pt}%
\pgfpathmoveto{\pgfqpoint{3.802641in}{6.724340in}}%
\pgfpathlineto{\pgfqpoint{3.890377in}{6.724340in}}%
\pgfpathlineto{\pgfqpoint{3.890377in}{6.636604in}}%
\pgfpathlineto{\pgfqpoint{3.802641in}{6.636604in}}%
\pgfpathlineto{\pgfqpoint{3.802641in}{6.724340in}}%
\pgfusepath{stroke,fill}%
\end{pgfscope}%
\begin{pgfscope}%
\pgfpathrectangle{\pgfqpoint{0.380943in}{6.110189in}}{\pgfqpoint{4.650000in}{0.614151in}}%
\pgfusepath{clip}%
\pgfsetbuttcap%
\pgfsetroundjoin%
\definecolor{currentfill}{rgb}{0.861576,0.340008,0.340008}%
\pgfsetfillcolor{currentfill}%
\pgfsetlinewidth{0.250937pt}%
\definecolor{currentstroke}{rgb}{1.000000,1.000000,1.000000}%
\pgfsetstrokecolor{currentstroke}%
\pgfsetdash{}{0pt}%
\pgfpathmoveto{\pgfqpoint{3.890377in}{6.724340in}}%
\pgfpathlineto{\pgfqpoint{3.978113in}{6.724340in}}%
\pgfpathlineto{\pgfqpoint{3.978113in}{6.636604in}}%
\pgfpathlineto{\pgfqpoint{3.890377in}{6.636604in}}%
\pgfpathlineto{\pgfqpoint{3.890377in}{6.724340in}}%
\pgfusepath{stroke,fill}%
\end{pgfscope}%
\begin{pgfscope}%
\pgfpathrectangle{\pgfqpoint{0.380943in}{6.110189in}}{\pgfqpoint{4.650000in}{0.614151in}}%
\pgfusepath{clip}%
\pgfsetbuttcap%
\pgfsetroundjoin%
\definecolor{currentfill}{rgb}{0.979654,0.837186,0.669619}%
\pgfsetfillcolor{currentfill}%
\pgfsetlinewidth{0.250937pt}%
\definecolor{currentstroke}{rgb}{1.000000,1.000000,1.000000}%
\pgfsetstrokecolor{currentstroke}%
\pgfsetdash{}{0pt}%
\pgfpathmoveto{\pgfqpoint{3.978113in}{6.724340in}}%
\pgfpathlineto{\pgfqpoint{4.065849in}{6.724340in}}%
\pgfpathlineto{\pgfqpoint{4.065849in}{6.636604in}}%
\pgfpathlineto{\pgfqpoint{3.978113in}{6.636604in}}%
\pgfpathlineto{\pgfqpoint{3.978113in}{6.724340in}}%
\pgfusepath{stroke,fill}%
\end{pgfscope}%
\begin{pgfscope}%
\pgfpathrectangle{\pgfqpoint{0.380943in}{6.110189in}}{\pgfqpoint{4.650000in}{0.614151in}}%
\pgfusepath{clip}%
\pgfsetbuttcap%
\pgfsetroundjoin%
\definecolor{currentfill}{rgb}{0.992326,0.765229,0.614840}%
\pgfsetfillcolor{currentfill}%
\pgfsetlinewidth{0.250937pt}%
\definecolor{currentstroke}{rgb}{1.000000,1.000000,1.000000}%
\pgfsetstrokecolor{currentstroke}%
\pgfsetdash{}{0pt}%
\pgfpathmoveto{\pgfqpoint{4.065849in}{6.724340in}}%
\pgfpathlineto{\pgfqpoint{4.153585in}{6.724340in}}%
\pgfpathlineto{\pgfqpoint{4.153585in}{6.636604in}}%
\pgfpathlineto{\pgfqpoint{4.065849in}{6.636604in}}%
\pgfpathlineto{\pgfqpoint{4.065849in}{6.724340in}}%
\pgfusepath{stroke,fill}%
\end{pgfscope}%
\begin{pgfscope}%
\pgfpathrectangle{\pgfqpoint{0.380943in}{6.110189in}}{\pgfqpoint{4.650000in}{0.614151in}}%
\pgfusepath{clip}%
\pgfsetbuttcap%
\pgfsetroundjoin%
\definecolor{currentfill}{rgb}{0.972549,0.870588,0.692810}%
\pgfsetfillcolor{currentfill}%
\pgfsetlinewidth{0.250937pt}%
\definecolor{currentstroke}{rgb}{1.000000,1.000000,1.000000}%
\pgfsetstrokecolor{currentstroke}%
\pgfsetdash{}{0pt}%
\pgfpathmoveto{\pgfqpoint{4.153585in}{6.724340in}}%
\pgfpathlineto{\pgfqpoint{4.241320in}{6.724340in}}%
\pgfpathlineto{\pgfqpoint{4.241320in}{6.636604in}}%
\pgfpathlineto{\pgfqpoint{4.153585in}{6.636604in}}%
\pgfpathlineto{\pgfqpoint{4.153585in}{6.724340in}}%
\pgfusepath{stroke,fill}%
\end{pgfscope}%
\begin{pgfscope}%
\pgfpathrectangle{\pgfqpoint{0.380943in}{6.110189in}}{\pgfqpoint{4.650000in}{0.614151in}}%
\pgfusepath{clip}%
\pgfsetbuttcap%
\pgfsetroundjoin%
\definecolor{currentfill}{rgb}{0.965444,0.906113,0.711757}%
\pgfsetfillcolor{currentfill}%
\pgfsetlinewidth{0.250937pt}%
\definecolor{currentstroke}{rgb}{1.000000,1.000000,1.000000}%
\pgfsetstrokecolor{currentstroke}%
\pgfsetdash{}{0pt}%
\pgfpathmoveto{\pgfqpoint{4.241320in}{6.724340in}}%
\pgfpathlineto{\pgfqpoint{4.329056in}{6.724340in}}%
\pgfpathlineto{\pgfqpoint{4.329056in}{6.636604in}}%
\pgfpathlineto{\pgfqpoint{4.241320in}{6.636604in}}%
\pgfpathlineto{\pgfqpoint{4.241320in}{6.724340in}}%
\pgfusepath{stroke,fill}%
\end{pgfscope}%
\begin{pgfscope}%
\pgfpathrectangle{\pgfqpoint{0.380943in}{6.110189in}}{\pgfqpoint{4.650000in}{0.614151in}}%
\pgfusepath{clip}%
\pgfsetbuttcap%
\pgfsetroundjoin%
\definecolor{currentfill}{rgb}{0.965444,0.906113,0.711757}%
\pgfsetfillcolor{currentfill}%
\pgfsetlinewidth{0.250937pt}%
\definecolor{currentstroke}{rgb}{1.000000,1.000000,1.000000}%
\pgfsetstrokecolor{currentstroke}%
\pgfsetdash{}{0pt}%
\pgfpathmoveto{\pgfqpoint{4.329056in}{6.724340in}}%
\pgfpathlineto{\pgfqpoint{4.416792in}{6.724340in}}%
\pgfpathlineto{\pgfqpoint{4.416792in}{6.636604in}}%
\pgfpathlineto{\pgfqpoint{4.329056in}{6.636604in}}%
\pgfpathlineto{\pgfqpoint{4.329056in}{6.724340in}}%
\pgfusepath{stroke,fill}%
\end{pgfscope}%
\begin{pgfscope}%
\pgfpathrectangle{\pgfqpoint{0.380943in}{6.110189in}}{\pgfqpoint{4.650000in}{0.614151in}}%
\pgfusepath{clip}%
\pgfsetbuttcap%
\pgfsetroundjoin%
\definecolor{currentfill}{rgb}{0.962414,0.923552,0.722891}%
\pgfsetfillcolor{currentfill}%
\pgfsetlinewidth{0.250937pt}%
\definecolor{currentstroke}{rgb}{1.000000,1.000000,1.000000}%
\pgfsetstrokecolor{currentstroke}%
\pgfsetdash{}{0pt}%
\pgfpathmoveto{\pgfqpoint{4.416792in}{6.724340in}}%
\pgfpathlineto{\pgfqpoint{4.504528in}{6.724340in}}%
\pgfpathlineto{\pgfqpoint{4.504528in}{6.636604in}}%
\pgfpathlineto{\pgfqpoint{4.416792in}{6.636604in}}%
\pgfpathlineto{\pgfqpoint{4.416792in}{6.724340in}}%
\pgfusepath{stroke,fill}%
\end{pgfscope}%
\begin{pgfscope}%
\pgfpathrectangle{\pgfqpoint{0.380943in}{6.110189in}}{\pgfqpoint{4.650000in}{0.614151in}}%
\pgfusepath{clip}%
\pgfsetbuttcap%
\pgfsetroundjoin%
\definecolor{currentfill}{rgb}{0.972549,0.870588,0.692810}%
\pgfsetfillcolor{currentfill}%
\pgfsetlinewidth{0.250937pt}%
\definecolor{currentstroke}{rgb}{1.000000,1.000000,1.000000}%
\pgfsetstrokecolor{currentstroke}%
\pgfsetdash{}{0pt}%
\pgfpathmoveto{\pgfqpoint{4.504528in}{6.724340in}}%
\pgfpathlineto{\pgfqpoint{4.592264in}{6.724340in}}%
\pgfpathlineto{\pgfqpoint{4.592264in}{6.636604in}}%
\pgfpathlineto{\pgfqpoint{4.504528in}{6.636604in}}%
\pgfpathlineto{\pgfqpoint{4.504528in}{6.724340in}}%
\pgfusepath{stroke,fill}%
\end{pgfscope}%
\begin{pgfscope}%
\pgfpathrectangle{\pgfqpoint{0.380943in}{6.110189in}}{\pgfqpoint{4.650000in}{0.614151in}}%
\pgfusepath{clip}%
\pgfsetbuttcap%
\pgfsetroundjoin%
\definecolor{currentfill}{rgb}{0.986759,0.806398,0.641200}%
\pgfsetfillcolor{currentfill}%
\pgfsetlinewidth{0.250937pt}%
\definecolor{currentstroke}{rgb}{1.000000,1.000000,1.000000}%
\pgfsetstrokecolor{currentstroke}%
\pgfsetdash{}{0pt}%
\pgfpathmoveto{\pgfqpoint{4.592264in}{6.724340in}}%
\pgfpathlineto{\pgfqpoint{4.680000in}{6.724340in}}%
\pgfpathlineto{\pgfqpoint{4.680000in}{6.636604in}}%
\pgfpathlineto{\pgfqpoint{4.592264in}{6.636604in}}%
\pgfpathlineto{\pgfqpoint{4.592264in}{6.724340in}}%
\pgfusepath{stroke,fill}%
\end{pgfscope}%
\begin{pgfscope}%
\pgfpathrectangle{\pgfqpoint{0.380943in}{6.110189in}}{\pgfqpoint{4.650000in}{0.614151in}}%
\pgfusepath{clip}%
\pgfsetbuttcap%
\pgfsetroundjoin%
\definecolor{currentfill}{rgb}{0.968166,0.945882,0.748604}%
\pgfsetfillcolor{currentfill}%
\pgfsetlinewidth{0.250937pt}%
\definecolor{currentstroke}{rgb}{1.000000,1.000000,1.000000}%
\pgfsetstrokecolor{currentstroke}%
\pgfsetdash{}{0pt}%
\pgfpathmoveto{\pgfqpoint{4.680000in}{6.724340in}}%
\pgfpathlineto{\pgfqpoint{4.767736in}{6.724340in}}%
\pgfpathlineto{\pgfqpoint{4.767736in}{6.636604in}}%
\pgfpathlineto{\pgfqpoint{4.680000in}{6.636604in}}%
\pgfpathlineto{\pgfqpoint{4.680000in}{6.724340in}}%
\pgfusepath{stroke,fill}%
\end{pgfscope}%
\begin{pgfscope}%
\pgfpathrectangle{\pgfqpoint{0.380943in}{6.110189in}}{\pgfqpoint{4.650000in}{0.614151in}}%
\pgfusepath{clip}%
\pgfsetbuttcap%
\pgfsetroundjoin%
\definecolor{currentfill}{rgb}{0.979654,0.837186,0.669619}%
\pgfsetfillcolor{currentfill}%
\pgfsetlinewidth{0.250937pt}%
\definecolor{currentstroke}{rgb}{1.000000,1.000000,1.000000}%
\pgfsetstrokecolor{currentstroke}%
\pgfsetdash{}{0pt}%
\pgfpathmoveto{\pgfqpoint{4.767736in}{6.724340in}}%
\pgfpathlineto{\pgfqpoint{4.855471in}{6.724340in}}%
\pgfpathlineto{\pgfqpoint{4.855471in}{6.636604in}}%
\pgfpathlineto{\pgfqpoint{4.767736in}{6.636604in}}%
\pgfpathlineto{\pgfqpoint{4.767736in}{6.724340in}}%
\pgfusepath{stroke,fill}%
\end{pgfscope}%
\begin{pgfscope}%
\pgfpathrectangle{\pgfqpoint{0.380943in}{6.110189in}}{\pgfqpoint{4.650000in}{0.614151in}}%
\pgfusepath{clip}%
\pgfsetbuttcap%
\pgfsetroundjoin%
\definecolor{currentfill}{rgb}{0.962414,0.923552,0.722891}%
\pgfsetfillcolor{currentfill}%
\pgfsetlinewidth{0.250937pt}%
\definecolor{currentstroke}{rgb}{1.000000,1.000000,1.000000}%
\pgfsetstrokecolor{currentstroke}%
\pgfsetdash{}{0pt}%
\pgfpathmoveto{\pgfqpoint{4.855471in}{6.724340in}}%
\pgfpathlineto{\pgfqpoint{4.943207in}{6.724340in}}%
\pgfpathlineto{\pgfqpoint{4.943207in}{6.636604in}}%
\pgfpathlineto{\pgfqpoint{4.855471in}{6.636604in}}%
\pgfpathlineto{\pgfqpoint{4.855471in}{6.724340in}}%
\pgfusepath{stroke,fill}%
\end{pgfscope}%
\begin{pgfscope}%
\pgfpathrectangle{\pgfqpoint{0.380943in}{6.110189in}}{\pgfqpoint{4.650000in}{0.614151in}}%
\pgfusepath{clip}%
\pgfsetbuttcap%
\pgfsetroundjoin%
\definecolor{currentfill}{rgb}{0.986759,0.806398,0.641200}%
\pgfsetfillcolor{currentfill}%
\pgfsetlinewidth{0.250937pt}%
\definecolor{currentstroke}{rgb}{1.000000,1.000000,1.000000}%
\pgfsetstrokecolor{currentstroke}%
\pgfsetdash{}{0pt}%
\pgfpathmoveto{\pgfqpoint{4.943207in}{6.724340in}}%
\pgfpathlineto{\pgfqpoint{5.030943in}{6.724340in}}%
\pgfpathlineto{\pgfqpoint{5.030943in}{6.636604in}}%
\pgfpathlineto{\pgfqpoint{4.943207in}{6.636604in}}%
\pgfpathlineto{\pgfqpoint{4.943207in}{6.724340in}}%
\pgfusepath{stroke,fill}%
\end{pgfscope}%
\begin{pgfscope}%
\pgfpathrectangle{\pgfqpoint{0.380943in}{6.110189in}}{\pgfqpoint{4.650000in}{0.614151in}}%
\pgfusepath{clip}%
\pgfsetbuttcap%
\pgfsetroundjoin%
\definecolor{currentfill}{rgb}{0.998939,0.658962,0.556032}%
\pgfsetfillcolor{currentfill}%
\pgfsetlinewidth{0.250937pt}%
\definecolor{currentstroke}{rgb}{1.000000,1.000000,1.000000}%
\pgfsetstrokecolor{currentstroke}%
\pgfsetdash{}{0pt}%
\pgfpathmoveto{\pgfqpoint{0.380943in}{6.636604in}}%
\pgfpathlineto{\pgfqpoint{0.468679in}{6.636604in}}%
\pgfpathlineto{\pgfqpoint{0.468679in}{6.548868in}}%
\pgfpathlineto{\pgfqpoint{0.380943in}{6.548868in}}%
\pgfpathlineto{\pgfqpoint{0.380943in}{6.636604in}}%
\pgfusepath{stroke,fill}%
\end{pgfscope}%
\begin{pgfscope}%
\pgfpathrectangle{\pgfqpoint{0.380943in}{6.110189in}}{\pgfqpoint{4.650000in}{0.614151in}}%
\pgfusepath{clip}%
\pgfsetbuttcap%
\pgfsetroundjoin%
\definecolor{currentfill}{rgb}{0.992326,0.765229,0.614840}%
\pgfsetfillcolor{currentfill}%
\pgfsetlinewidth{0.250937pt}%
\definecolor{currentstroke}{rgb}{1.000000,1.000000,1.000000}%
\pgfsetstrokecolor{currentstroke}%
\pgfsetdash{}{0pt}%
\pgfpathmoveto{\pgfqpoint{0.468679in}{6.636604in}}%
\pgfpathlineto{\pgfqpoint{0.556415in}{6.636604in}}%
\pgfpathlineto{\pgfqpoint{0.556415in}{6.548868in}}%
\pgfpathlineto{\pgfqpoint{0.468679in}{6.548868in}}%
\pgfpathlineto{\pgfqpoint{0.468679in}{6.636604in}}%
\pgfusepath{stroke,fill}%
\end{pgfscope}%
\begin{pgfscope}%
\pgfpathrectangle{\pgfqpoint{0.380943in}{6.110189in}}{\pgfqpoint{4.650000in}{0.614151in}}%
\pgfusepath{clip}%
\pgfsetbuttcap%
\pgfsetroundjoin%
\definecolor{currentfill}{rgb}{0.972549,0.870588,0.692810}%
\pgfsetfillcolor{currentfill}%
\pgfsetlinewidth{0.250937pt}%
\definecolor{currentstroke}{rgb}{1.000000,1.000000,1.000000}%
\pgfsetstrokecolor{currentstroke}%
\pgfsetdash{}{0pt}%
\pgfpathmoveto{\pgfqpoint{0.556415in}{6.636604in}}%
\pgfpathlineto{\pgfqpoint{0.644151in}{6.636604in}}%
\pgfpathlineto{\pgfqpoint{0.644151in}{6.548868in}}%
\pgfpathlineto{\pgfqpoint{0.556415in}{6.548868in}}%
\pgfpathlineto{\pgfqpoint{0.556415in}{6.636604in}}%
\pgfusepath{stroke,fill}%
\end{pgfscope}%
\begin{pgfscope}%
\pgfpathrectangle{\pgfqpoint{0.380943in}{6.110189in}}{\pgfqpoint{4.650000in}{0.614151in}}%
\pgfusepath{clip}%
\pgfsetbuttcap%
\pgfsetroundjoin%
\definecolor{currentfill}{rgb}{0.979654,0.837186,0.669619}%
\pgfsetfillcolor{currentfill}%
\pgfsetlinewidth{0.250937pt}%
\definecolor{currentstroke}{rgb}{1.000000,1.000000,1.000000}%
\pgfsetstrokecolor{currentstroke}%
\pgfsetdash{}{0pt}%
\pgfpathmoveto{\pgfqpoint{0.644151in}{6.636604in}}%
\pgfpathlineto{\pgfqpoint{0.731886in}{6.636604in}}%
\pgfpathlineto{\pgfqpoint{0.731886in}{6.548868in}}%
\pgfpathlineto{\pgfqpoint{0.644151in}{6.548868in}}%
\pgfpathlineto{\pgfqpoint{0.644151in}{6.636604in}}%
\pgfusepath{stroke,fill}%
\end{pgfscope}%
\begin{pgfscope}%
\pgfpathrectangle{\pgfqpoint{0.380943in}{6.110189in}}{\pgfqpoint{4.650000in}{0.614151in}}%
\pgfusepath{clip}%
\pgfsetbuttcap%
\pgfsetroundjoin%
\definecolor{currentfill}{rgb}{0.998939,0.658962,0.556032}%
\pgfsetfillcolor{currentfill}%
\pgfsetlinewidth{0.250937pt}%
\definecolor{currentstroke}{rgb}{1.000000,1.000000,1.000000}%
\pgfsetstrokecolor{currentstroke}%
\pgfsetdash{}{0pt}%
\pgfpathmoveto{\pgfqpoint{0.731886in}{6.636604in}}%
\pgfpathlineto{\pgfqpoint{0.819622in}{6.636604in}}%
\pgfpathlineto{\pgfqpoint{0.819622in}{6.548868in}}%
\pgfpathlineto{\pgfqpoint{0.731886in}{6.548868in}}%
\pgfpathlineto{\pgfqpoint{0.731886in}{6.636604in}}%
\pgfusepath{stroke,fill}%
\end{pgfscope}%
\begin{pgfscope}%
\pgfpathrectangle{\pgfqpoint{0.380943in}{6.110189in}}{\pgfqpoint{4.650000in}{0.614151in}}%
\pgfusepath{clip}%
\pgfsetbuttcap%
\pgfsetroundjoin%
\definecolor{currentfill}{rgb}{0.996571,0.720538,0.589189}%
\pgfsetfillcolor{currentfill}%
\pgfsetlinewidth{0.250937pt}%
\definecolor{currentstroke}{rgb}{1.000000,1.000000,1.000000}%
\pgfsetstrokecolor{currentstroke}%
\pgfsetdash{}{0pt}%
\pgfpathmoveto{\pgfqpoint{0.819622in}{6.636604in}}%
\pgfpathlineto{\pgfqpoint{0.907358in}{6.636604in}}%
\pgfpathlineto{\pgfqpoint{0.907358in}{6.548868in}}%
\pgfpathlineto{\pgfqpoint{0.819622in}{6.548868in}}%
\pgfpathlineto{\pgfqpoint{0.819622in}{6.636604in}}%
\pgfusepath{stroke,fill}%
\end{pgfscope}%
\begin{pgfscope}%
\pgfpathrectangle{\pgfqpoint{0.380943in}{6.110189in}}{\pgfqpoint{4.650000in}{0.614151in}}%
\pgfusepath{clip}%
\pgfsetbuttcap%
\pgfsetroundjoin%
\definecolor{currentfill}{rgb}{0.979654,0.837186,0.669619}%
\pgfsetfillcolor{currentfill}%
\pgfsetlinewidth{0.250937pt}%
\definecolor{currentstroke}{rgb}{1.000000,1.000000,1.000000}%
\pgfsetstrokecolor{currentstroke}%
\pgfsetdash{}{0pt}%
\pgfpathmoveto{\pgfqpoint{0.907358in}{6.636604in}}%
\pgfpathlineto{\pgfqpoint{0.995094in}{6.636604in}}%
\pgfpathlineto{\pgfqpoint{0.995094in}{6.548868in}}%
\pgfpathlineto{\pgfqpoint{0.907358in}{6.548868in}}%
\pgfpathlineto{\pgfqpoint{0.907358in}{6.636604in}}%
\pgfusepath{stroke,fill}%
\end{pgfscope}%
\begin{pgfscope}%
\pgfpathrectangle{\pgfqpoint{0.380943in}{6.110189in}}{\pgfqpoint{4.650000in}{0.614151in}}%
\pgfusepath{clip}%
\pgfsetbuttcap%
\pgfsetroundjoin%
\definecolor{currentfill}{rgb}{0.979654,0.837186,0.669619}%
\pgfsetfillcolor{currentfill}%
\pgfsetlinewidth{0.250937pt}%
\definecolor{currentstroke}{rgb}{1.000000,1.000000,1.000000}%
\pgfsetstrokecolor{currentstroke}%
\pgfsetdash{}{0pt}%
\pgfpathmoveto{\pgfqpoint{0.995094in}{6.636604in}}%
\pgfpathlineto{\pgfqpoint{1.082830in}{6.636604in}}%
\pgfpathlineto{\pgfqpoint{1.082830in}{6.548868in}}%
\pgfpathlineto{\pgfqpoint{0.995094in}{6.548868in}}%
\pgfpathlineto{\pgfqpoint{0.995094in}{6.636604in}}%
\pgfusepath{stroke,fill}%
\end{pgfscope}%
\begin{pgfscope}%
\pgfpathrectangle{\pgfqpoint{0.380943in}{6.110189in}}{\pgfqpoint{4.650000in}{0.614151in}}%
\pgfusepath{clip}%
\pgfsetbuttcap%
\pgfsetroundjoin%
\definecolor{currentfill}{rgb}{0.965444,0.906113,0.711757}%
\pgfsetfillcolor{currentfill}%
\pgfsetlinewidth{0.250937pt}%
\definecolor{currentstroke}{rgb}{1.000000,1.000000,1.000000}%
\pgfsetstrokecolor{currentstroke}%
\pgfsetdash{}{0pt}%
\pgfpathmoveto{\pgfqpoint{1.082830in}{6.636604in}}%
\pgfpathlineto{\pgfqpoint{1.170566in}{6.636604in}}%
\pgfpathlineto{\pgfqpoint{1.170566in}{6.548868in}}%
\pgfpathlineto{\pgfqpoint{1.082830in}{6.548868in}}%
\pgfpathlineto{\pgfqpoint{1.082830in}{6.636604in}}%
\pgfusepath{stroke,fill}%
\end{pgfscope}%
\begin{pgfscope}%
\pgfpathrectangle{\pgfqpoint{0.380943in}{6.110189in}}{\pgfqpoint{4.650000in}{0.614151in}}%
\pgfusepath{clip}%
\pgfsetbuttcap%
\pgfsetroundjoin%
\definecolor{currentfill}{rgb}{0.992326,0.765229,0.614840}%
\pgfsetfillcolor{currentfill}%
\pgfsetlinewidth{0.250937pt}%
\definecolor{currentstroke}{rgb}{1.000000,1.000000,1.000000}%
\pgfsetstrokecolor{currentstroke}%
\pgfsetdash{}{0pt}%
\pgfpathmoveto{\pgfqpoint{1.170566in}{6.636604in}}%
\pgfpathlineto{\pgfqpoint{1.258302in}{6.636604in}}%
\pgfpathlineto{\pgfqpoint{1.258302in}{6.548868in}}%
\pgfpathlineto{\pgfqpoint{1.170566in}{6.548868in}}%
\pgfpathlineto{\pgfqpoint{1.170566in}{6.636604in}}%
\pgfusepath{stroke,fill}%
\end{pgfscope}%
\begin{pgfscope}%
\pgfpathrectangle{\pgfqpoint{0.380943in}{6.110189in}}{\pgfqpoint{4.650000in}{0.614151in}}%
\pgfusepath{clip}%
\pgfsetbuttcap%
\pgfsetroundjoin%
\definecolor{currentfill}{rgb}{1.000000,0.509404,0.491473}%
\pgfsetfillcolor{currentfill}%
\pgfsetlinewidth{0.250937pt}%
\definecolor{currentstroke}{rgb}{1.000000,1.000000,1.000000}%
\pgfsetstrokecolor{currentstroke}%
\pgfsetdash{}{0pt}%
\pgfpathmoveto{\pgfqpoint{1.258302in}{6.636604in}}%
\pgfpathlineto{\pgfqpoint{1.346037in}{6.636604in}}%
\pgfpathlineto{\pgfqpoint{1.346037in}{6.548868in}}%
\pgfpathlineto{\pgfqpoint{1.258302in}{6.548868in}}%
\pgfpathlineto{\pgfqpoint{1.258302in}{6.636604in}}%
\pgfusepath{stroke,fill}%
\end{pgfscope}%
\begin{pgfscope}%
\pgfpathrectangle{\pgfqpoint{0.380943in}{6.110189in}}{\pgfqpoint{4.650000in}{0.614151in}}%
\pgfusepath{clip}%
\pgfsetbuttcap%
\pgfsetroundjoin%
\definecolor{currentfill}{rgb}{0.972549,0.870588,0.692810}%
\pgfsetfillcolor{currentfill}%
\pgfsetlinewidth{0.250937pt}%
\definecolor{currentstroke}{rgb}{1.000000,1.000000,1.000000}%
\pgfsetstrokecolor{currentstroke}%
\pgfsetdash{}{0pt}%
\pgfpathmoveto{\pgfqpoint{1.346037in}{6.636604in}}%
\pgfpathlineto{\pgfqpoint{1.433773in}{6.636604in}}%
\pgfpathlineto{\pgfqpoint{1.433773in}{6.548868in}}%
\pgfpathlineto{\pgfqpoint{1.346037in}{6.548868in}}%
\pgfpathlineto{\pgfqpoint{1.346037in}{6.636604in}}%
\pgfusepath{stroke,fill}%
\end{pgfscope}%
\begin{pgfscope}%
\pgfpathrectangle{\pgfqpoint{0.380943in}{6.110189in}}{\pgfqpoint{4.650000in}{0.614151in}}%
\pgfusepath{clip}%
\pgfsetbuttcap%
\pgfsetroundjoin%
\definecolor{currentfill}{rgb}{0.996571,0.720538,0.589189}%
\pgfsetfillcolor{currentfill}%
\pgfsetlinewidth{0.250937pt}%
\definecolor{currentstroke}{rgb}{1.000000,1.000000,1.000000}%
\pgfsetstrokecolor{currentstroke}%
\pgfsetdash{}{0pt}%
\pgfpathmoveto{\pgfqpoint{1.433773in}{6.636604in}}%
\pgfpathlineto{\pgfqpoint{1.521509in}{6.636604in}}%
\pgfpathlineto{\pgfqpoint{1.521509in}{6.548868in}}%
\pgfpathlineto{\pgfqpoint{1.433773in}{6.548868in}}%
\pgfpathlineto{\pgfqpoint{1.433773in}{6.636604in}}%
\pgfusepath{stroke,fill}%
\end{pgfscope}%
\begin{pgfscope}%
\pgfpathrectangle{\pgfqpoint{0.380943in}{6.110189in}}{\pgfqpoint{4.650000in}{0.614151in}}%
\pgfusepath{clip}%
\pgfsetbuttcap%
\pgfsetroundjoin%
\definecolor{currentfill}{rgb}{0.996571,0.720538,0.589189}%
\pgfsetfillcolor{currentfill}%
\pgfsetlinewidth{0.250937pt}%
\definecolor{currentstroke}{rgb}{1.000000,1.000000,1.000000}%
\pgfsetstrokecolor{currentstroke}%
\pgfsetdash{}{0pt}%
\pgfpathmoveto{\pgfqpoint{1.521509in}{6.636604in}}%
\pgfpathlineto{\pgfqpoint{1.609245in}{6.636604in}}%
\pgfpathlineto{\pgfqpoint{1.609245in}{6.548868in}}%
\pgfpathlineto{\pgfqpoint{1.521509in}{6.548868in}}%
\pgfpathlineto{\pgfqpoint{1.521509in}{6.636604in}}%
\pgfusepath{stroke,fill}%
\end{pgfscope}%
\begin{pgfscope}%
\pgfpathrectangle{\pgfqpoint{0.380943in}{6.110189in}}{\pgfqpoint{4.650000in}{0.614151in}}%
\pgfusepath{clip}%
\pgfsetbuttcap%
\pgfsetroundjoin%
\definecolor{currentfill}{rgb}{1.000000,0.509404,0.491473}%
\pgfsetfillcolor{currentfill}%
\pgfsetlinewidth{0.250937pt}%
\definecolor{currentstroke}{rgb}{1.000000,1.000000,1.000000}%
\pgfsetstrokecolor{currentstroke}%
\pgfsetdash{}{0pt}%
\pgfpathmoveto{\pgfqpoint{1.609245in}{6.636604in}}%
\pgfpathlineto{\pgfqpoint{1.696981in}{6.636604in}}%
\pgfpathlineto{\pgfqpoint{1.696981in}{6.548868in}}%
\pgfpathlineto{\pgfqpoint{1.609245in}{6.548868in}}%
\pgfpathlineto{\pgfqpoint{1.609245in}{6.636604in}}%
\pgfusepath{stroke,fill}%
\end{pgfscope}%
\begin{pgfscope}%
\pgfpathrectangle{\pgfqpoint{0.380943in}{6.110189in}}{\pgfqpoint{4.650000in}{0.614151in}}%
\pgfusepath{clip}%
\pgfsetbuttcap%
\pgfsetroundjoin%
\definecolor{currentfill}{rgb}{0.986759,0.806398,0.641200}%
\pgfsetfillcolor{currentfill}%
\pgfsetlinewidth{0.250937pt}%
\definecolor{currentstroke}{rgb}{1.000000,1.000000,1.000000}%
\pgfsetstrokecolor{currentstroke}%
\pgfsetdash{}{0pt}%
\pgfpathmoveto{\pgfqpoint{1.696981in}{6.636604in}}%
\pgfpathlineto{\pgfqpoint{1.784717in}{6.636604in}}%
\pgfpathlineto{\pgfqpoint{1.784717in}{6.548868in}}%
\pgfpathlineto{\pgfqpoint{1.696981in}{6.548868in}}%
\pgfpathlineto{\pgfqpoint{1.696981in}{6.636604in}}%
\pgfusepath{stroke,fill}%
\end{pgfscope}%
\begin{pgfscope}%
\pgfpathrectangle{\pgfqpoint{0.380943in}{6.110189in}}{\pgfqpoint{4.650000in}{0.614151in}}%
\pgfusepath{clip}%
\pgfsetbuttcap%
\pgfsetroundjoin%
\definecolor{currentfill}{rgb}{0.992326,0.765229,0.614840}%
\pgfsetfillcolor{currentfill}%
\pgfsetlinewidth{0.250937pt}%
\definecolor{currentstroke}{rgb}{1.000000,1.000000,1.000000}%
\pgfsetstrokecolor{currentstroke}%
\pgfsetdash{}{0pt}%
\pgfpathmoveto{\pgfqpoint{1.784717in}{6.636604in}}%
\pgfpathlineto{\pgfqpoint{1.872452in}{6.636604in}}%
\pgfpathlineto{\pgfqpoint{1.872452in}{6.548868in}}%
\pgfpathlineto{\pgfqpoint{1.784717in}{6.548868in}}%
\pgfpathlineto{\pgfqpoint{1.784717in}{6.636604in}}%
\pgfusepath{stroke,fill}%
\end{pgfscope}%
\begin{pgfscope}%
\pgfpathrectangle{\pgfqpoint{0.380943in}{6.110189in}}{\pgfqpoint{4.650000in}{0.614151in}}%
\pgfusepath{clip}%
\pgfsetbuttcap%
\pgfsetroundjoin%
\definecolor{currentfill}{rgb}{0.991849,0.986144,0.810181}%
\pgfsetfillcolor{currentfill}%
\pgfsetlinewidth{0.250937pt}%
\definecolor{currentstroke}{rgb}{1.000000,1.000000,1.000000}%
\pgfsetstrokecolor{currentstroke}%
\pgfsetdash{}{0pt}%
\pgfpathmoveto{\pgfqpoint{1.872452in}{6.636604in}}%
\pgfpathlineto{\pgfqpoint{1.960188in}{6.636604in}}%
\pgfpathlineto{\pgfqpoint{1.960188in}{6.548868in}}%
\pgfpathlineto{\pgfqpoint{1.872452in}{6.548868in}}%
\pgfpathlineto{\pgfqpoint{1.872452in}{6.636604in}}%
\pgfusepath{stroke,fill}%
\end{pgfscope}%
\begin{pgfscope}%
\pgfpathrectangle{\pgfqpoint{0.380943in}{6.110189in}}{\pgfqpoint{4.650000in}{0.614151in}}%
\pgfusepath{clip}%
\pgfsetbuttcap%
\pgfsetroundjoin%
\definecolor{currentfill}{rgb}{0.991849,0.986144,0.810181}%
\pgfsetfillcolor{currentfill}%
\pgfsetlinewidth{0.250937pt}%
\definecolor{currentstroke}{rgb}{1.000000,1.000000,1.000000}%
\pgfsetstrokecolor{currentstroke}%
\pgfsetdash{}{0pt}%
\pgfpathmoveto{\pgfqpoint{1.960188in}{6.636604in}}%
\pgfpathlineto{\pgfqpoint{2.047924in}{6.636604in}}%
\pgfpathlineto{\pgfqpoint{2.047924in}{6.548868in}}%
\pgfpathlineto{\pgfqpoint{1.960188in}{6.548868in}}%
\pgfpathlineto{\pgfqpoint{1.960188in}{6.636604in}}%
\pgfusepath{stroke,fill}%
\end{pgfscope}%
\begin{pgfscope}%
\pgfpathrectangle{\pgfqpoint{0.380943in}{6.110189in}}{\pgfqpoint{4.650000in}{0.614151in}}%
\pgfusepath{clip}%
\pgfsetbuttcap%
\pgfsetroundjoin%
\definecolor{currentfill}{rgb}{0.986759,0.806398,0.641200}%
\pgfsetfillcolor{currentfill}%
\pgfsetlinewidth{0.250937pt}%
\definecolor{currentstroke}{rgb}{1.000000,1.000000,1.000000}%
\pgfsetstrokecolor{currentstroke}%
\pgfsetdash{}{0pt}%
\pgfpathmoveto{\pgfqpoint{2.047924in}{6.636604in}}%
\pgfpathlineto{\pgfqpoint{2.135660in}{6.636604in}}%
\pgfpathlineto{\pgfqpoint{2.135660in}{6.548868in}}%
\pgfpathlineto{\pgfqpoint{2.047924in}{6.548868in}}%
\pgfpathlineto{\pgfqpoint{2.047924in}{6.636604in}}%
\pgfusepath{stroke,fill}%
\end{pgfscope}%
\begin{pgfscope}%
\pgfpathrectangle{\pgfqpoint{0.380943in}{6.110189in}}{\pgfqpoint{4.650000in}{0.614151in}}%
\pgfusepath{clip}%
\pgfsetbuttcap%
\pgfsetroundjoin%
\definecolor{currentfill}{rgb}{0.972549,0.870588,0.692810}%
\pgfsetfillcolor{currentfill}%
\pgfsetlinewidth{0.250937pt}%
\definecolor{currentstroke}{rgb}{1.000000,1.000000,1.000000}%
\pgfsetstrokecolor{currentstroke}%
\pgfsetdash{}{0pt}%
\pgfpathmoveto{\pgfqpoint{2.135660in}{6.636604in}}%
\pgfpathlineto{\pgfqpoint{2.223396in}{6.636604in}}%
\pgfpathlineto{\pgfqpoint{2.223396in}{6.548868in}}%
\pgfpathlineto{\pgfqpoint{2.135660in}{6.548868in}}%
\pgfpathlineto{\pgfqpoint{2.135660in}{6.636604in}}%
\pgfusepath{stroke,fill}%
\end{pgfscope}%
\begin{pgfscope}%
\pgfpathrectangle{\pgfqpoint{0.380943in}{6.110189in}}{\pgfqpoint{4.650000in}{0.614151in}}%
\pgfusepath{clip}%
\pgfsetbuttcap%
\pgfsetroundjoin%
\definecolor{currentfill}{rgb}{0.979654,0.837186,0.669619}%
\pgfsetfillcolor{currentfill}%
\pgfsetlinewidth{0.250937pt}%
\definecolor{currentstroke}{rgb}{1.000000,1.000000,1.000000}%
\pgfsetstrokecolor{currentstroke}%
\pgfsetdash{}{0pt}%
\pgfpathmoveto{\pgfqpoint{2.223396in}{6.636604in}}%
\pgfpathlineto{\pgfqpoint{2.311132in}{6.636604in}}%
\pgfpathlineto{\pgfqpoint{2.311132in}{6.548868in}}%
\pgfpathlineto{\pgfqpoint{2.223396in}{6.548868in}}%
\pgfpathlineto{\pgfqpoint{2.223396in}{6.636604in}}%
\pgfusepath{stroke,fill}%
\end{pgfscope}%
\begin{pgfscope}%
\pgfpathrectangle{\pgfqpoint{0.380943in}{6.110189in}}{\pgfqpoint{4.650000in}{0.614151in}}%
\pgfusepath{clip}%
\pgfsetbuttcap%
\pgfsetroundjoin%
\definecolor{currentfill}{rgb}{0.861576,0.340008,0.340008}%
\pgfsetfillcolor{currentfill}%
\pgfsetlinewidth{0.250937pt}%
\definecolor{currentstroke}{rgb}{1.000000,1.000000,1.000000}%
\pgfsetstrokecolor{currentstroke}%
\pgfsetdash{}{0pt}%
\pgfpathmoveto{\pgfqpoint{2.311132in}{6.636604in}}%
\pgfpathlineto{\pgfqpoint{2.398868in}{6.636604in}}%
\pgfpathlineto{\pgfqpoint{2.398868in}{6.548868in}}%
\pgfpathlineto{\pgfqpoint{2.311132in}{6.548868in}}%
\pgfpathlineto{\pgfqpoint{2.311132in}{6.636604in}}%
\pgfusepath{stroke,fill}%
\end{pgfscope}%
\begin{pgfscope}%
\pgfpathrectangle{\pgfqpoint{0.380943in}{6.110189in}}{\pgfqpoint{4.650000in}{0.614151in}}%
\pgfusepath{clip}%
\pgfsetbuttcap%
\pgfsetroundjoin%
\definecolor{currentfill}{rgb}{1.000000,0.509404,0.491473}%
\pgfsetfillcolor{currentfill}%
\pgfsetlinewidth{0.250937pt}%
\definecolor{currentstroke}{rgb}{1.000000,1.000000,1.000000}%
\pgfsetstrokecolor{currentstroke}%
\pgfsetdash{}{0pt}%
\pgfpathmoveto{\pgfqpoint{2.398868in}{6.636604in}}%
\pgfpathlineto{\pgfqpoint{2.486603in}{6.636604in}}%
\pgfpathlineto{\pgfqpoint{2.486603in}{6.548868in}}%
\pgfpathlineto{\pgfqpoint{2.398868in}{6.548868in}}%
\pgfpathlineto{\pgfqpoint{2.398868in}{6.636604in}}%
\pgfusepath{stroke,fill}%
\end{pgfscope}%
\begin{pgfscope}%
\pgfpathrectangle{\pgfqpoint{0.380943in}{6.110189in}}{\pgfqpoint{4.650000in}{0.614151in}}%
\pgfusepath{clip}%
\pgfsetbuttcap%
\pgfsetroundjoin%
\definecolor{currentfill}{rgb}{1.000000,0.605229,0.530719}%
\pgfsetfillcolor{currentfill}%
\pgfsetlinewidth{0.250937pt}%
\definecolor{currentstroke}{rgb}{1.000000,1.000000,1.000000}%
\pgfsetstrokecolor{currentstroke}%
\pgfsetdash{}{0pt}%
\pgfpathmoveto{\pgfqpoint{2.486603in}{6.636604in}}%
\pgfpathlineto{\pgfqpoint{2.574339in}{6.636604in}}%
\pgfpathlineto{\pgfqpoint{2.574339in}{6.548868in}}%
\pgfpathlineto{\pgfqpoint{2.486603in}{6.548868in}}%
\pgfpathlineto{\pgfqpoint{2.486603in}{6.636604in}}%
\pgfusepath{stroke,fill}%
\end{pgfscope}%
\begin{pgfscope}%
\pgfpathrectangle{\pgfqpoint{0.380943in}{6.110189in}}{\pgfqpoint{4.650000in}{0.614151in}}%
\pgfusepath{clip}%
\pgfsetbuttcap%
\pgfsetroundjoin%
\definecolor{currentfill}{rgb}{0.972549,0.870588,0.692810}%
\pgfsetfillcolor{currentfill}%
\pgfsetlinewidth{0.250937pt}%
\definecolor{currentstroke}{rgb}{1.000000,1.000000,1.000000}%
\pgfsetstrokecolor{currentstroke}%
\pgfsetdash{}{0pt}%
\pgfpathmoveto{\pgfqpoint{2.574339in}{6.636604in}}%
\pgfpathlineto{\pgfqpoint{2.662075in}{6.636604in}}%
\pgfpathlineto{\pgfqpoint{2.662075in}{6.548868in}}%
\pgfpathlineto{\pgfqpoint{2.574339in}{6.548868in}}%
\pgfpathlineto{\pgfqpoint{2.574339in}{6.636604in}}%
\pgfusepath{stroke,fill}%
\end{pgfscope}%
\begin{pgfscope}%
\pgfpathrectangle{\pgfqpoint{0.380943in}{6.110189in}}{\pgfqpoint{4.650000in}{0.614151in}}%
\pgfusepath{clip}%
\pgfsetbuttcap%
\pgfsetroundjoin%
\definecolor{currentfill}{rgb}{0.962414,0.923552,0.722891}%
\pgfsetfillcolor{currentfill}%
\pgfsetlinewidth{0.250937pt}%
\definecolor{currentstroke}{rgb}{1.000000,1.000000,1.000000}%
\pgfsetstrokecolor{currentstroke}%
\pgfsetdash{}{0pt}%
\pgfpathmoveto{\pgfqpoint{2.662075in}{6.636604in}}%
\pgfpathlineto{\pgfqpoint{2.749811in}{6.636604in}}%
\pgfpathlineto{\pgfqpoint{2.749811in}{6.548868in}}%
\pgfpathlineto{\pgfqpoint{2.662075in}{6.548868in}}%
\pgfpathlineto{\pgfqpoint{2.662075in}{6.636604in}}%
\pgfusepath{stroke,fill}%
\end{pgfscope}%
\begin{pgfscope}%
\pgfpathrectangle{\pgfqpoint{0.380943in}{6.110189in}}{\pgfqpoint{4.650000in}{0.614151in}}%
\pgfusepath{clip}%
\pgfsetbuttcap%
\pgfsetroundjoin%
\definecolor{currentfill}{rgb}{0.972549,0.870588,0.692810}%
\pgfsetfillcolor{currentfill}%
\pgfsetlinewidth{0.250937pt}%
\definecolor{currentstroke}{rgb}{1.000000,1.000000,1.000000}%
\pgfsetstrokecolor{currentstroke}%
\pgfsetdash{}{0pt}%
\pgfpathmoveto{\pgfqpoint{2.749811in}{6.636604in}}%
\pgfpathlineto{\pgfqpoint{2.837547in}{6.636604in}}%
\pgfpathlineto{\pgfqpoint{2.837547in}{6.548868in}}%
\pgfpathlineto{\pgfqpoint{2.749811in}{6.548868in}}%
\pgfpathlineto{\pgfqpoint{2.749811in}{6.636604in}}%
\pgfusepath{stroke,fill}%
\end{pgfscope}%
\begin{pgfscope}%
\pgfpathrectangle{\pgfqpoint{0.380943in}{6.110189in}}{\pgfqpoint{4.650000in}{0.614151in}}%
\pgfusepath{clip}%
\pgfsetbuttcap%
\pgfsetroundjoin%
\definecolor{currentfill}{rgb}{0.992326,0.765229,0.614840}%
\pgfsetfillcolor{currentfill}%
\pgfsetlinewidth{0.250937pt}%
\definecolor{currentstroke}{rgb}{1.000000,1.000000,1.000000}%
\pgfsetstrokecolor{currentstroke}%
\pgfsetdash{}{0pt}%
\pgfpathmoveto{\pgfqpoint{2.837547in}{6.636604in}}%
\pgfpathlineto{\pgfqpoint{2.925283in}{6.636604in}}%
\pgfpathlineto{\pgfqpoint{2.925283in}{6.548868in}}%
\pgfpathlineto{\pgfqpoint{2.837547in}{6.548868in}}%
\pgfpathlineto{\pgfqpoint{2.837547in}{6.636604in}}%
\pgfusepath{stroke,fill}%
\end{pgfscope}%
\begin{pgfscope}%
\pgfpathrectangle{\pgfqpoint{0.380943in}{6.110189in}}{\pgfqpoint{4.650000in}{0.614151in}}%
\pgfusepath{clip}%
\pgfsetbuttcap%
\pgfsetroundjoin%
\definecolor{currentfill}{rgb}{0.968166,0.945882,0.748604}%
\pgfsetfillcolor{currentfill}%
\pgfsetlinewidth{0.250937pt}%
\definecolor{currentstroke}{rgb}{1.000000,1.000000,1.000000}%
\pgfsetstrokecolor{currentstroke}%
\pgfsetdash{}{0pt}%
\pgfpathmoveto{\pgfqpoint{2.925283in}{6.636604in}}%
\pgfpathlineto{\pgfqpoint{3.013019in}{6.636604in}}%
\pgfpathlineto{\pgfqpoint{3.013019in}{6.548868in}}%
\pgfpathlineto{\pgfqpoint{2.925283in}{6.548868in}}%
\pgfpathlineto{\pgfqpoint{2.925283in}{6.636604in}}%
\pgfusepath{stroke,fill}%
\end{pgfscope}%
\begin{pgfscope}%
\pgfpathrectangle{\pgfqpoint{0.380943in}{6.110189in}}{\pgfqpoint{4.650000in}{0.614151in}}%
\pgfusepath{clip}%
\pgfsetbuttcap%
\pgfsetroundjoin%
\definecolor{currentfill}{rgb}{0.979654,0.837186,0.669619}%
\pgfsetfillcolor{currentfill}%
\pgfsetlinewidth{0.250937pt}%
\definecolor{currentstroke}{rgb}{1.000000,1.000000,1.000000}%
\pgfsetstrokecolor{currentstroke}%
\pgfsetdash{}{0pt}%
\pgfpathmoveto{\pgfqpoint{3.013019in}{6.636604in}}%
\pgfpathlineto{\pgfqpoint{3.100754in}{6.636604in}}%
\pgfpathlineto{\pgfqpoint{3.100754in}{6.548868in}}%
\pgfpathlineto{\pgfqpoint{3.013019in}{6.548868in}}%
\pgfpathlineto{\pgfqpoint{3.013019in}{6.636604in}}%
\pgfusepath{stroke,fill}%
\end{pgfscope}%
\begin{pgfscope}%
\pgfpathrectangle{\pgfqpoint{0.380943in}{6.110189in}}{\pgfqpoint{4.650000in}{0.614151in}}%
\pgfusepath{clip}%
\pgfsetbuttcap%
\pgfsetroundjoin%
\definecolor{currentfill}{rgb}{0.972549,0.870588,0.692810}%
\pgfsetfillcolor{currentfill}%
\pgfsetlinewidth{0.250937pt}%
\definecolor{currentstroke}{rgb}{1.000000,1.000000,1.000000}%
\pgfsetstrokecolor{currentstroke}%
\pgfsetdash{}{0pt}%
\pgfpathmoveto{\pgfqpoint{3.100754in}{6.636604in}}%
\pgfpathlineto{\pgfqpoint{3.188490in}{6.636604in}}%
\pgfpathlineto{\pgfqpoint{3.188490in}{6.548868in}}%
\pgfpathlineto{\pgfqpoint{3.100754in}{6.548868in}}%
\pgfpathlineto{\pgfqpoint{3.100754in}{6.636604in}}%
\pgfusepath{stroke,fill}%
\end{pgfscope}%
\begin{pgfscope}%
\pgfpathrectangle{\pgfqpoint{0.380943in}{6.110189in}}{\pgfqpoint{4.650000in}{0.614151in}}%
\pgfusepath{clip}%
\pgfsetbuttcap%
\pgfsetroundjoin%
\definecolor{currentfill}{rgb}{0.972549,0.870588,0.692810}%
\pgfsetfillcolor{currentfill}%
\pgfsetlinewidth{0.250937pt}%
\definecolor{currentstroke}{rgb}{1.000000,1.000000,1.000000}%
\pgfsetstrokecolor{currentstroke}%
\pgfsetdash{}{0pt}%
\pgfpathmoveto{\pgfqpoint{3.188490in}{6.636604in}}%
\pgfpathlineto{\pgfqpoint{3.276226in}{6.636604in}}%
\pgfpathlineto{\pgfqpoint{3.276226in}{6.548868in}}%
\pgfpathlineto{\pgfqpoint{3.188490in}{6.548868in}}%
\pgfpathlineto{\pgfqpoint{3.188490in}{6.636604in}}%
\pgfusepath{stroke,fill}%
\end{pgfscope}%
\begin{pgfscope}%
\pgfpathrectangle{\pgfqpoint{0.380943in}{6.110189in}}{\pgfqpoint{4.650000in}{0.614151in}}%
\pgfusepath{clip}%
\pgfsetbuttcap%
\pgfsetroundjoin%
\definecolor{currentfill}{rgb}{0.972549,0.870588,0.692810}%
\pgfsetfillcolor{currentfill}%
\pgfsetlinewidth{0.250937pt}%
\definecolor{currentstroke}{rgb}{1.000000,1.000000,1.000000}%
\pgfsetstrokecolor{currentstroke}%
\pgfsetdash{}{0pt}%
\pgfpathmoveto{\pgfqpoint{3.276226in}{6.636604in}}%
\pgfpathlineto{\pgfqpoint{3.363962in}{6.636604in}}%
\pgfpathlineto{\pgfqpoint{3.363962in}{6.548868in}}%
\pgfpathlineto{\pgfqpoint{3.276226in}{6.548868in}}%
\pgfpathlineto{\pgfqpoint{3.276226in}{6.636604in}}%
\pgfusepath{stroke,fill}%
\end{pgfscope}%
\begin{pgfscope}%
\pgfpathrectangle{\pgfqpoint{0.380943in}{6.110189in}}{\pgfqpoint{4.650000in}{0.614151in}}%
\pgfusepath{clip}%
\pgfsetbuttcap%
\pgfsetroundjoin%
\definecolor{currentfill}{rgb}{0.965444,0.906113,0.711757}%
\pgfsetfillcolor{currentfill}%
\pgfsetlinewidth{0.250937pt}%
\definecolor{currentstroke}{rgb}{1.000000,1.000000,1.000000}%
\pgfsetstrokecolor{currentstroke}%
\pgfsetdash{}{0pt}%
\pgfpathmoveto{\pgfqpoint{3.363962in}{6.636604in}}%
\pgfpathlineto{\pgfqpoint{3.451698in}{6.636604in}}%
\pgfpathlineto{\pgfqpoint{3.451698in}{6.548868in}}%
\pgfpathlineto{\pgfqpoint{3.363962in}{6.548868in}}%
\pgfpathlineto{\pgfqpoint{3.363962in}{6.636604in}}%
\pgfusepath{stroke,fill}%
\end{pgfscope}%
\begin{pgfscope}%
\pgfpathrectangle{\pgfqpoint{0.380943in}{6.110189in}}{\pgfqpoint{4.650000in}{0.614151in}}%
\pgfusepath{clip}%
\pgfsetbuttcap%
\pgfsetroundjoin%
\definecolor{currentfill}{rgb}{0.992326,0.765229,0.614840}%
\pgfsetfillcolor{currentfill}%
\pgfsetlinewidth{0.250937pt}%
\definecolor{currentstroke}{rgb}{1.000000,1.000000,1.000000}%
\pgfsetstrokecolor{currentstroke}%
\pgfsetdash{}{0pt}%
\pgfpathmoveto{\pgfqpoint{3.451698in}{6.636604in}}%
\pgfpathlineto{\pgfqpoint{3.539434in}{6.636604in}}%
\pgfpathlineto{\pgfqpoint{3.539434in}{6.548868in}}%
\pgfpathlineto{\pgfqpoint{3.451698in}{6.548868in}}%
\pgfpathlineto{\pgfqpoint{3.451698in}{6.636604in}}%
\pgfusepath{stroke,fill}%
\end{pgfscope}%
\begin{pgfscope}%
\pgfpathrectangle{\pgfqpoint{0.380943in}{6.110189in}}{\pgfqpoint{4.650000in}{0.614151in}}%
\pgfusepath{clip}%
\pgfsetbuttcap%
\pgfsetroundjoin%
\definecolor{currentfill}{rgb}{0.962414,0.923552,0.722891}%
\pgfsetfillcolor{currentfill}%
\pgfsetlinewidth{0.250937pt}%
\definecolor{currentstroke}{rgb}{1.000000,1.000000,1.000000}%
\pgfsetstrokecolor{currentstroke}%
\pgfsetdash{}{0pt}%
\pgfpathmoveto{\pgfqpoint{3.539434in}{6.636604in}}%
\pgfpathlineto{\pgfqpoint{3.627169in}{6.636604in}}%
\pgfpathlineto{\pgfqpoint{3.627169in}{6.548868in}}%
\pgfpathlineto{\pgfqpoint{3.539434in}{6.548868in}}%
\pgfpathlineto{\pgfqpoint{3.539434in}{6.636604in}}%
\pgfusepath{stroke,fill}%
\end{pgfscope}%
\begin{pgfscope}%
\pgfpathrectangle{\pgfqpoint{0.380943in}{6.110189in}}{\pgfqpoint{4.650000in}{0.614151in}}%
\pgfusepath{clip}%
\pgfsetbuttcap%
\pgfsetroundjoin%
\definecolor{currentfill}{rgb}{0.992326,0.765229,0.614840}%
\pgfsetfillcolor{currentfill}%
\pgfsetlinewidth{0.250937pt}%
\definecolor{currentstroke}{rgb}{1.000000,1.000000,1.000000}%
\pgfsetstrokecolor{currentstroke}%
\pgfsetdash{}{0pt}%
\pgfpathmoveto{\pgfqpoint{3.627169in}{6.636604in}}%
\pgfpathlineto{\pgfqpoint{3.714905in}{6.636604in}}%
\pgfpathlineto{\pgfqpoint{3.714905in}{6.548868in}}%
\pgfpathlineto{\pgfqpoint{3.627169in}{6.548868in}}%
\pgfpathlineto{\pgfqpoint{3.627169in}{6.636604in}}%
\pgfusepath{stroke,fill}%
\end{pgfscope}%
\begin{pgfscope}%
\pgfpathrectangle{\pgfqpoint{0.380943in}{6.110189in}}{\pgfqpoint{4.650000in}{0.614151in}}%
\pgfusepath{clip}%
\pgfsetbuttcap%
\pgfsetroundjoin%
\definecolor{currentfill}{rgb}{0.986759,0.806398,0.641200}%
\pgfsetfillcolor{currentfill}%
\pgfsetlinewidth{0.250937pt}%
\definecolor{currentstroke}{rgb}{1.000000,1.000000,1.000000}%
\pgfsetstrokecolor{currentstroke}%
\pgfsetdash{}{0pt}%
\pgfpathmoveto{\pgfqpoint{3.714905in}{6.636604in}}%
\pgfpathlineto{\pgfqpoint{3.802641in}{6.636604in}}%
\pgfpathlineto{\pgfqpoint{3.802641in}{6.548868in}}%
\pgfpathlineto{\pgfqpoint{3.714905in}{6.548868in}}%
\pgfpathlineto{\pgfqpoint{3.714905in}{6.636604in}}%
\pgfusepath{stroke,fill}%
\end{pgfscope}%
\begin{pgfscope}%
\pgfpathrectangle{\pgfqpoint{0.380943in}{6.110189in}}{\pgfqpoint{4.650000in}{0.614151in}}%
\pgfusepath{clip}%
\pgfsetbuttcap%
\pgfsetroundjoin%
\definecolor{currentfill}{rgb}{0.972549,0.870588,0.692810}%
\pgfsetfillcolor{currentfill}%
\pgfsetlinewidth{0.250937pt}%
\definecolor{currentstroke}{rgb}{1.000000,1.000000,1.000000}%
\pgfsetstrokecolor{currentstroke}%
\pgfsetdash{}{0pt}%
\pgfpathmoveto{\pgfqpoint{3.802641in}{6.636604in}}%
\pgfpathlineto{\pgfqpoint{3.890377in}{6.636604in}}%
\pgfpathlineto{\pgfqpoint{3.890377in}{6.548868in}}%
\pgfpathlineto{\pgfqpoint{3.802641in}{6.548868in}}%
\pgfpathlineto{\pgfqpoint{3.802641in}{6.636604in}}%
\pgfusepath{stroke,fill}%
\end{pgfscope}%
\begin{pgfscope}%
\pgfpathrectangle{\pgfqpoint{0.380943in}{6.110189in}}{\pgfqpoint{4.650000in}{0.614151in}}%
\pgfusepath{clip}%
\pgfsetbuttcap%
\pgfsetroundjoin%
\definecolor{currentfill}{rgb}{0.992326,0.765229,0.614840}%
\pgfsetfillcolor{currentfill}%
\pgfsetlinewidth{0.250937pt}%
\definecolor{currentstroke}{rgb}{1.000000,1.000000,1.000000}%
\pgfsetstrokecolor{currentstroke}%
\pgfsetdash{}{0pt}%
\pgfpathmoveto{\pgfqpoint{3.890377in}{6.636604in}}%
\pgfpathlineto{\pgfqpoint{3.978113in}{6.636604in}}%
\pgfpathlineto{\pgfqpoint{3.978113in}{6.548868in}}%
\pgfpathlineto{\pgfqpoint{3.890377in}{6.548868in}}%
\pgfpathlineto{\pgfqpoint{3.890377in}{6.636604in}}%
\pgfusepath{stroke,fill}%
\end{pgfscope}%
\begin{pgfscope}%
\pgfpathrectangle{\pgfqpoint{0.380943in}{6.110189in}}{\pgfqpoint{4.650000in}{0.614151in}}%
\pgfusepath{clip}%
\pgfsetbuttcap%
\pgfsetroundjoin%
\definecolor{currentfill}{rgb}{0.800000,0.278431,0.278431}%
\pgfsetfillcolor{currentfill}%
\pgfsetlinewidth{0.250937pt}%
\definecolor{currentstroke}{rgb}{1.000000,1.000000,1.000000}%
\pgfsetstrokecolor{currentstroke}%
\pgfsetdash{}{0pt}%
\pgfpathmoveto{\pgfqpoint{3.978113in}{6.636604in}}%
\pgfpathlineto{\pgfqpoint{4.065849in}{6.636604in}}%
\pgfpathlineto{\pgfqpoint{4.065849in}{6.548868in}}%
\pgfpathlineto{\pgfqpoint{3.978113in}{6.548868in}}%
\pgfpathlineto{\pgfqpoint{3.978113in}{6.636604in}}%
\pgfusepath{stroke,fill}%
\end{pgfscope}%
\begin{pgfscope}%
\pgfpathrectangle{\pgfqpoint{0.380943in}{6.110189in}}{\pgfqpoint{4.650000in}{0.614151in}}%
\pgfusepath{clip}%
\pgfsetbuttcap%
\pgfsetroundjoin%
\definecolor{currentfill}{rgb}{0.979654,0.837186,0.669619}%
\pgfsetfillcolor{currentfill}%
\pgfsetlinewidth{0.250937pt}%
\definecolor{currentstroke}{rgb}{1.000000,1.000000,1.000000}%
\pgfsetstrokecolor{currentstroke}%
\pgfsetdash{}{0pt}%
\pgfpathmoveto{\pgfqpoint{4.065849in}{6.636604in}}%
\pgfpathlineto{\pgfqpoint{4.153585in}{6.636604in}}%
\pgfpathlineto{\pgfqpoint{4.153585in}{6.548868in}}%
\pgfpathlineto{\pgfqpoint{4.065849in}{6.548868in}}%
\pgfpathlineto{\pgfqpoint{4.065849in}{6.636604in}}%
\pgfusepath{stroke,fill}%
\end{pgfscope}%
\begin{pgfscope}%
\pgfpathrectangle{\pgfqpoint{0.380943in}{6.110189in}}{\pgfqpoint{4.650000in}{0.614151in}}%
\pgfusepath{clip}%
\pgfsetbuttcap%
\pgfsetroundjoin%
\definecolor{currentfill}{rgb}{0.992326,0.765229,0.614840}%
\pgfsetfillcolor{currentfill}%
\pgfsetlinewidth{0.250937pt}%
\definecolor{currentstroke}{rgb}{1.000000,1.000000,1.000000}%
\pgfsetstrokecolor{currentstroke}%
\pgfsetdash{}{0pt}%
\pgfpathmoveto{\pgfqpoint{4.153585in}{6.636604in}}%
\pgfpathlineto{\pgfqpoint{4.241320in}{6.636604in}}%
\pgfpathlineto{\pgfqpoint{4.241320in}{6.548868in}}%
\pgfpathlineto{\pgfqpoint{4.153585in}{6.548868in}}%
\pgfpathlineto{\pgfqpoint{4.153585in}{6.636604in}}%
\pgfusepath{stroke,fill}%
\end{pgfscope}%
\begin{pgfscope}%
\pgfpathrectangle{\pgfqpoint{0.380943in}{6.110189in}}{\pgfqpoint{4.650000in}{0.614151in}}%
\pgfusepath{clip}%
\pgfsetbuttcap%
\pgfsetroundjoin%
\definecolor{currentfill}{rgb}{0.972549,0.870588,0.692810}%
\pgfsetfillcolor{currentfill}%
\pgfsetlinewidth{0.250937pt}%
\definecolor{currentstroke}{rgb}{1.000000,1.000000,1.000000}%
\pgfsetstrokecolor{currentstroke}%
\pgfsetdash{}{0pt}%
\pgfpathmoveto{\pgfqpoint{4.241320in}{6.636604in}}%
\pgfpathlineto{\pgfqpoint{4.329056in}{6.636604in}}%
\pgfpathlineto{\pgfqpoint{4.329056in}{6.548868in}}%
\pgfpathlineto{\pgfqpoint{4.241320in}{6.548868in}}%
\pgfpathlineto{\pgfqpoint{4.241320in}{6.636604in}}%
\pgfusepath{stroke,fill}%
\end{pgfscope}%
\begin{pgfscope}%
\pgfpathrectangle{\pgfqpoint{0.380943in}{6.110189in}}{\pgfqpoint{4.650000in}{0.614151in}}%
\pgfusepath{clip}%
\pgfsetbuttcap%
\pgfsetroundjoin%
\definecolor{currentfill}{rgb}{0.965444,0.906113,0.711757}%
\pgfsetfillcolor{currentfill}%
\pgfsetlinewidth{0.250937pt}%
\definecolor{currentstroke}{rgb}{1.000000,1.000000,1.000000}%
\pgfsetstrokecolor{currentstroke}%
\pgfsetdash{}{0pt}%
\pgfpathmoveto{\pgfqpoint{4.329056in}{6.636604in}}%
\pgfpathlineto{\pgfqpoint{4.416792in}{6.636604in}}%
\pgfpathlineto{\pgfqpoint{4.416792in}{6.548868in}}%
\pgfpathlineto{\pgfqpoint{4.329056in}{6.548868in}}%
\pgfpathlineto{\pgfqpoint{4.329056in}{6.636604in}}%
\pgfusepath{stroke,fill}%
\end{pgfscope}%
\begin{pgfscope}%
\pgfpathrectangle{\pgfqpoint{0.380943in}{6.110189in}}{\pgfqpoint{4.650000in}{0.614151in}}%
\pgfusepath{clip}%
\pgfsetbuttcap%
\pgfsetroundjoin%
\definecolor{currentfill}{rgb}{0.992326,0.765229,0.614840}%
\pgfsetfillcolor{currentfill}%
\pgfsetlinewidth{0.250937pt}%
\definecolor{currentstroke}{rgb}{1.000000,1.000000,1.000000}%
\pgfsetstrokecolor{currentstroke}%
\pgfsetdash{}{0pt}%
\pgfpathmoveto{\pgfqpoint{4.416792in}{6.636604in}}%
\pgfpathlineto{\pgfqpoint{4.504528in}{6.636604in}}%
\pgfpathlineto{\pgfqpoint{4.504528in}{6.548868in}}%
\pgfpathlineto{\pgfqpoint{4.416792in}{6.548868in}}%
\pgfpathlineto{\pgfqpoint{4.416792in}{6.636604in}}%
\pgfusepath{stroke,fill}%
\end{pgfscope}%
\begin{pgfscope}%
\pgfpathrectangle{\pgfqpoint{0.380943in}{6.110189in}}{\pgfqpoint{4.650000in}{0.614151in}}%
\pgfusepath{clip}%
\pgfsetbuttcap%
\pgfsetroundjoin%
\definecolor{currentfill}{rgb}{0.998939,0.658962,0.556032}%
\pgfsetfillcolor{currentfill}%
\pgfsetlinewidth{0.250937pt}%
\definecolor{currentstroke}{rgb}{1.000000,1.000000,1.000000}%
\pgfsetstrokecolor{currentstroke}%
\pgfsetdash{}{0pt}%
\pgfpathmoveto{\pgfqpoint{4.504528in}{6.636604in}}%
\pgfpathlineto{\pgfqpoint{4.592264in}{6.636604in}}%
\pgfpathlineto{\pgfqpoint{4.592264in}{6.548868in}}%
\pgfpathlineto{\pgfqpoint{4.504528in}{6.548868in}}%
\pgfpathlineto{\pgfqpoint{4.504528in}{6.636604in}}%
\pgfusepath{stroke,fill}%
\end{pgfscope}%
\begin{pgfscope}%
\pgfpathrectangle{\pgfqpoint{0.380943in}{6.110189in}}{\pgfqpoint{4.650000in}{0.614151in}}%
\pgfusepath{clip}%
\pgfsetbuttcap%
\pgfsetroundjoin%
\definecolor{currentfill}{rgb}{0.986759,0.806398,0.641200}%
\pgfsetfillcolor{currentfill}%
\pgfsetlinewidth{0.250937pt}%
\definecolor{currentstroke}{rgb}{1.000000,1.000000,1.000000}%
\pgfsetstrokecolor{currentstroke}%
\pgfsetdash{}{0pt}%
\pgfpathmoveto{\pgfqpoint{4.592264in}{6.636604in}}%
\pgfpathlineto{\pgfqpoint{4.680000in}{6.636604in}}%
\pgfpathlineto{\pgfqpoint{4.680000in}{6.548868in}}%
\pgfpathlineto{\pgfqpoint{4.592264in}{6.548868in}}%
\pgfpathlineto{\pgfqpoint{4.592264in}{6.636604in}}%
\pgfusepath{stroke,fill}%
\end{pgfscope}%
\begin{pgfscope}%
\pgfpathrectangle{\pgfqpoint{0.380943in}{6.110189in}}{\pgfqpoint{4.650000in}{0.614151in}}%
\pgfusepath{clip}%
\pgfsetbuttcap%
\pgfsetroundjoin%
\definecolor{currentfill}{rgb}{1.000000,0.605229,0.530719}%
\pgfsetfillcolor{currentfill}%
\pgfsetlinewidth{0.250937pt}%
\definecolor{currentstroke}{rgb}{1.000000,1.000000,1.000000}%
\pgfsetstrokecolor{currentstroke}%
\pgfsetdash{}{0pt}%
\pgfpathmoveto{\pgfqpoint{4.680000in}{6.636604in}}%
\pgfpathlineto{\pgfqpoint{4.767736in}{6.636604in}}%
\pgfpathlineto{\pgfqpoint{4.767736in}{6.548868in}}%
\pgfpathlineto{\pgfqpoint{4.680000in}{6.548868in}}%
\pgfpathlineto{\pgfqpoint{4.680000in}{6.636604in}}%
\pgfusepath{stroke,fill}%
\end{pgfscope}%
\begin{pgfscope}%
\pgfpathrectangle{\pgfqpoint{0.380943in}{6.110189in}}{\pgfqpoint{4.650000in}{0.614151in}}%
\pgfusepath{clip}%
\pgfsetbuttcap%
\pgfsetroundjoin%
\definecolor{currentfill}{rgb}{0.965444,0.906113,0.711757}%
\pgfsetfillcolor{currentfill}%
\pgfsetlinewidth{0.250937pt}%
\definecolor{currentstroke}{rgb}{1.000000,1.000000,1.000000}%
\pgfsetstrokecolor{currentstroke}%
\pgfsetdash{}{0pt}%
\pgfpathmoveto{\pgfqpoint{4.767736in}{6.636604in}}%
\pgfpathlineto{\pgfqpoint{4.855471in}{6.636604in}}%
\pgfpathlineto{\pgfqpoint{4.855471in}{6.548868in}}%
\pgfpathlineto{\pgfqpoint{4.767736in}{6.548868in}}%
\pgfpathlineto{\pgfqpoint{4.767736in}{6.636604in}}%
\pgfusepath{stroke,fill}%
\end{pgfscope}%
\begin{pgfscope}%
\pgfpathrectangle{\pgfqpoint{0.380943in}{6.110189in}}{\pgfqpoint{4.650000in}{0.614151in}}%
\pgfusepath{clip}%
\pgfsetbuttcap%
\pgfsetroundjoin%
\definecolor{currentfill}{rgb}{1.000000,1.000000,0.870204}%
\pgfsetfillcolor{currentfill}%
\pgfsetlinewidth{0.250937pt}%
\definecolor{currentstroke}{rgb}{1.000000,1.000000,1.000000}%
\pgfsetstrokecolor{currentstroke}%
\pgfsetdash{}{0pt}%
\pgfpathmoveto{\pgfqpoint{4.855471in}{6.636604in}}%
\pgfpathlineto{\pgfqpoint{4.943207in}{6.636604in}}%
\pgfpathlineto{\pgfqpoint{4.943207in}{6.548868in}}%
\pgfpathlineto{\pgfqpoint{4.855471in}{6.548868in}}%
\pgfpathlineto{\pgfqpoint{4.855471in}{6.636604in}}%
\pgfusepath{stroke,fill}%
\end{pgfscope}%
\begin{pgfscope}%
\pgfpathrectangle{\pgfqpoint{0.380943in}{6.110189in}}{\pgfqpoint{4.650000in}{0.614151in}}%
\pgfusepath{clip}%
\pgfsetbuttcap%
\pgfsetroundjoin%
\pgfsetlinewidth{0.250937pt}%
\definecolor{currentstroke}{rgb}{1.000000,1.000000,1.000000}%
\pgfsetstrokecolor{currentstroke}%
\pgfsetdash{}{0pt}%
\pgfpathmoveto{\pgfqpoint{4.943207in}{6.636604in}}%
\pgfpathlineto{\pgfqpoint{5.030943in}{6.636604in}}%
\pgfpathlineto{\pgfqpoint{5.030943in}{6.548868in}}%
\pgfpathlineto{\pgfqpoint{4.943207in}{6.548868in}}%
\pgfpathlineto{\pgfqpoint{4.943207in}{6.636604in}}%
\pgfusepath{stroke}%
\end{pgfscope}%
\begin{pgfscope}%
\pgfpathrectangle{\pgfqpoint{0.380943in}{6.110189in}}{\pgfqpoint{4.650000in}{0.614151in}}%
\pgfusepath{clip}%
\pgfsetbuttcap%
\pgfsetroundjoin%
\definecolor{currentfill}{rgb}{0.981546,0.459977,0.459977}%
\pgfsetfillcolor{currentfill}%
\pgfsetlinewidth{0.250937pt}%
\definecolor{currentstroke}{rgb}{1.000000,1.000000,1.000000}%
\pgfsetstrokecolor{currentstroke}%
\pgfsetdash{}{0pt}%
\pgfpathmoveto{\pgfqpoint{0.380943in}{6.548868in}}%
\pgfpathlineto{\pgfqpoint{0.468679in}{6.548868in}}%
\pgfpathlineto{\pgfqpoint{0.468679in}{6.461132in}}%
\pgfpathlineto{\pgfqpoint{0.380943in}{6.461132in}}%
\pgfpathlineto{\pgfqpoint{0.380943in}{6.548868in}}%
\pgfusepath{stroke,fill}%
\end{pgfscope}%
\begin{pgfscope}%
\pgfpathrectangle{\pgfqpoint{0.380943in}{6.110189in}}{\pgfqpoint{4.650000in}{0.614151in}}%
\pgfusepath{clip}%
\pgfsetbuttcap%
\pgfsetroundjoin%
\definecolor{currentfill}{rgb}{0.965444,0.906113,0.711757}%
\pgfsetfillcolor{currentfill}%
\pgfsetlinewidth{0.250937pt}%
\definecolor{currentstroke}{rgb}{1.000000,1.000000,1.000000}%
\pgfsetstrokecolor{currentstroke}%
\pgfsetdash{}{0pt}%
\pgfpathmoveto{\pgfqpoint{0.468679in}{6.548868in}}%
\pgfpathlineto{\pgfqpoint{0.556415in}{6.548868in}}%
\pgfpathlineto{\pgfqpoint{0.556415in}{6.461132in}}%
\pgfpathlineto{\pgfqpoint{0.468679in}{6.461132in}}%
\pgfpathlineto{\pgfqpoint{0.468679in}{6.548868in}}%
\pgfusepath{stroke,fill}%
\end{pgfscope}%
\begin{pgfscope}%
\pgfpathrectangle{\pgfqpoint{0.380943in}{6.110189in}}{\pgfqpoint{4.650000in}{0.614151in}}%
\pgfusepath{clip}%
\pgfsetbuttcap%
\pgfsetroundjoin%
\definecolor{currentfill}{rgb}{0.986759,0.806398,0.641200}%
\pgfsetfillcolor{currentfill}%
\pgfsetlinewidth{0.250937pt}%
\definecolor{currentstroke}{rgb}{1.000000,1.000000,1.000000}%
\pgfsetstrokecolor{currentstroke}%
\pgfsetdash{}{0pt}%
\pgfpathmoveto{\pgfqpoint{0.556415in}{6.548868in}}%
\pgfpathlineto{\pgfqpoint{0.644151in}{6.548868in}}%
\pgfpathlineto{\pgfqpoint{0.644151in}{6.461132in}}%
\pgfpathlineto{\pgfqpoint{0.556415in}{6.461132in}}%
\pgfpathlineto{\pgfqpoint{0.556415in}{6.548868in}}%
\pgfusepath{stroke,fill}%
\end{pgfscope}%
\begin{pgfscope}%
\pgfpathrectangle{\pgfqpoint{0.380943in}{6.110189in}}{\pgfqpoint{4.650000in}{0.614151in}}%
\pgfusepath{clip}%
\pgfsetbuttcap%
\pgfsetroundjoin%
\definecolor{currentfill}{rgb}{1.000000,0.509404,0.491473}%
\pgfsetfillcolor{currentfill}%
\pgfsetlinewidth{0.250937pt}%
\definecolor{currentstroke}{rgb}{1.000000,1.000000,1.000000}%
\pgfsetstrokecolor{currentstroke}%
\pgfsetdash{}{0pt}%
\pgfpathmoveto{\pgfqpoint{0.644151in}{6.548868in}}%
\pgfpathlineto{\pgfqpoint{0.731886in}{6.548868in}}%
\pgfpathlineto{\pgfqpoint{0.731886in}{6.461132in}}%
\pgfpathlineto{\pgfqpoint{0.644151in}{6.461132in}}%
\pgfpathlineto{\pgfqpoint{0.644151in}{6.548868in}}%
\pgfusepath{stroke,fill}%
\end{pgfscope}%
\begin{pgfscope}%
\pgfpathrectangle{\pgfqpoint{0.380943in}{6.110189in}}{\pgfqpoint{4.650000in}{0.614151in}}%
\pgfusepath{clip}%
\pgfsetbuttcap%
\pgfsetroundjoin%
\definecolor{currentfill}{rgb}{0.965444,0.906113,0.711757}%
\pgfsetfillcolor{currentfill}%
\pgfsetlinewidth{0.250937pt}%
\definecolor{currentstroke}{rgb}{1.000000,1.000000,1.000000}%
\pgfsetstrokecolor{currentstroke}%
\pgfsetdash{}{0pt}%
\pgfpathmoveto{\pgfqpoint{0.731886in}{6.548868in}}%
\pgfpathlineto{\pgfqpoint{0.819622in}{6.548868in}}%
\pgfpathlineto{\pgfqpoint{0.819622in}{6.461132in}}%
\pgfpathlineto{\pgfqpoint{0.731886in}{6.461132in}}%
\pgfpathlineto{\pgfqpoint{0.731886in}{6.548868in}}%
\pgfusepath{stroke,fill}%
\end{pgfscope}%
\begin{pgfscope}%
\pgfpathrectangle{\pgfqpoint{0.380943in}{6.110189in}}{\pgfqpoint{4.650000in}{0.614151in}}%
\pgfusepath{clip}%
\pgfsetbuttcap%
\pgfsetroundjoin%
\definecolor{currentfill}{rgb}{1.000000,0.605229,0.530719}%
\pgfsetfillcolor{currentfill}%
\pgfsetlinewidth{0.250937pt}%
\definecolor{currentstroke}{rgb}{1.000000,1.000000,1.000000}%
\pgfsetstrokecolor{currentstroke}%
\pgfsetdash{}{0pt}%
\pgfpathmoveto{\pgfqpoint{0.819622in}{6.548868in}}%
\pgfpathlineto{\pgfqpoint{0.907358in}{6.548868in}}%
\pgfpathlineto{\pgfqpoint{0.907358in}{6.461132in}}%
\pgfpathlineto{\pgfqpoint{0.819622in}{6.461132in}}%
\pgfpathlineto{\pgfqpoint{0.819622in}{6.548868in}}%
\pgfusepath{stroke,fill}%
\end{pgfscope}%
\begin{pgfscope}%
\pgfpathrectangle{\pgfqpoint{0.380943in}{6.110189in}}{\pgfqpoint{4.650000in}{0.614151in}}%
\pgfusepath{clip}%
\pgfsetbuttcap%
\pgfsetroundjoin%
\definecolor{currentfill}{rgb}{0.996571,0.720538,0.589189}%
\pgfsetfillcolor{currentfill}%
\pgfsetlinewidth{0.250937pt}%
\definecolor{currentstroke}{rgb}{1.000000,1.000000,1.000000}%
\pgfsetstrokecolor{currentstroke}%
\pgfsetdash{}{0pt}%
\pgfpathmoveto{\pgfqpoint{0.907358in}{6.548868in}}%
\pgfpathlineto{\pgfqpoint{0.995094in}{6.548868in}}%
\pgfpathlineto{\pgfqpoint{0.995094in}{6.461132in}}%
\pgfpathlineto{\pgfqpoint{0.907358in}{6.461132in}}%
\pgfpathlineto{\pgfqpoint{0.907358in}{6.548868in}}%
\pgfusepath{stroke,fill}%
\end{pgfscope}%
\begin{pgfscope}%
\pgfpathrectangle{\pgfqpoint{0.380943in}{6.110189in}}{\pgfqpoint{4.650000in}{0.614151in}}%
\pgfusepath{clip}%
\pgfsetbuttcap%
\pgfsetroundjoin%
\definecolor{currentfill}{rgb}{0.996571,0.720538,0.589189}%
\pgfsetfillcolor{currentfill}%
\pgfsetlinewidth{0.250937pt}%
\definecolor{currentstroke}{rgb}{1.000000,1.000000,1.000000}%
\pgfsetstrokecolor{currentstroke}%
\pgfsetdash{}{0pt}%
\pgfpathmoveto{\pgfqpoint{0.995094in}{6.548868in}}%
\pgfpathlineto{\pgfqpoint{1.082830in}{6.548868in}}%
\pgfpathlineto{\pgfqpoint{1.082830in}{6.461132in}}%
\pgfpathlineto{\pgfqpoint{0.995094in}{6.461132in}}%
\pgfpathlineto{\pgfqpoint{0.995094in}{6.548868in}}%
\pgfusepath{stroke,fill}%
\end{pgfscope}%
\begin{pgfscope}%
\pgfpathrectangle{\pgfqpoint{0.380943in}{6.110189in}}{\pgfqpoint{4.650000in}{0.614151in}}%
\pgfusepath{clip}%
\pgfsetbuttcap%
\pgfsetroundjoin%
\definecolor{currentfill}{rgb}{0.986759,0.806398,0.641200}%
\pgfsetfillcolor{currentfill}%
\pgfsetlinewidth{0.250937pt}%
\definecolor{currentstroke}{rgb}{1.000000,1.000000,1.000000}%
\pgfsetstrokecolor{currentstroke}%
\pgfsetdash{}{0pt}%
\pgfpathmoveto{\pgfqpoint{1.082830in}{6.548868in}}%
\pgfpathlineto{\pgfqpoint{1.170566in}{6.548868in}}%
\pgfpathlineto{\pgfqpoint{1.170566in}{6.461132in}}%
\pgfpathlineto{\pgfqpoint{1.082830in}{6.461132in}}%
\pgfpathlineto{\pgfqpoint{1.082830in}{6.548868in}}%
\pgfusepath{stroke,fill}%
\end{pgfscope}%
\begin{pgfscope}%
\pgfpathrectangle{\pgfqpoint{0.380943in}{6.110189in}}{\pgfqpoint{4.650000in}{0.614151in}}%
\pgfusepath{clip}%
\pgfsetbuttcap%
\pgfsetroundjoin%
\definecolor{currentfill}{rgb}{0.998939,0.658962,0.556032}%
\pgfsetfillcolor{currentfill}%
\pgfsetlinewidth{0.250937pt}%
\definecolor{currentstroke}{rgb}{1.000000,1.000000,1.000000}%
\pgfsetstrokecolor{currentstroke}%
\pgfsetdash{}{0pt}%
\pgfpathmoveto{\pgfqpoint{1.170566in}{6.548868in}}%
\pgfpathlineto{\pgfqpoint{1.258302in}{6.548868in}}%
\pgfpathlineto{\pgfqpoint{1.258302in}{6.461132in}}%
\pgfpathlineto{\pgfqpoint{1.170566in}{6.461132in}}%
\pgfpathlineto{\pgfqpoint{1.170566in}{6.548868in}}%
\pgfusepath{stroke,fill}%
\end{pgfscope}%
\begin{pgfscope}%
\pgfpathrectangle{\pgfqpoint{0.380943in}{6.110189in}}{\pgfqpoint{4.650000in}{0.614151in}}%
\pgfusepath{clip}%
\pgfsetbuttcap%
\pgfsetroundjoin%
\definecolor{currentfill}{rgb}{0.922338,0.400769,0.400769}%
\pgfsetfillcolor{currentfill}%
\pgfsetlinewidth{0.250937pt}%
\definecolor{currentstroke}{rgb}{1.000000,1.000000,1.000000}%
\pgfsetstrokecolor{currentstroke}%
\pgfsetdash{}{0pt}%
\pgfpathmoveto{\pgfqpoint{1.258302in}{6.548868in}}%
\pgfpathlineto{\pgfqpoint{1.346037in}{6.548868in}}%
\pgfpathlineto{\pgfqpoint{1.346037in}{6.461132in}}%
\pgfpathlineto{\pgfqpoint{1.258302in}{6.461132in}}%
\pgfpathlineto{\pgfqpoint{1.258302in}{6.548868in}}%
\pgfusepath{stroke,fill}%
\end{pgfscope}%
\begin{pgfscope}%
\pgfpathrectangle{\pgfqpoint{0.380943in}{6.110189in}}{\pgfqpoint{4.650000in}{0.614151in}}%
\pgfusepath{clip}%
\pgfsetbuttcap%
\pgfsetroundjoin%
\definecolor{currentfill}{rgb}{0.998939,0.658962,0.556032}%
\pgfsetfillcolor{currentfill}%
\pgfsetlinewidth{0.250937pt}%
\definecolor{currentstroke}{rgb}{1.000000,1.000000,1.000000}%
\pgfsetstrokecolor{currentstroke}%
\pgfsetdash{}{0pt}%
\pgfpathmoveto{\pgfqpoint{1.346037in}{6.548868in}}%
\pgfpathlineto{\pgfqpoint{1.433773in}{6.548868in}}%
\pgfpathlineto{\pgfqpoint{1.433773in}{6.461132in}}%
\pgfpathlineto{\pgfqpoint{1.346037in}{6.461132in}}%
\pgfpathlineto{\pgfqpoint{1.346037in}{6.548868in}}%
\pgfusepath{stroke,fill}%
\end{pgfscope}%
\begin{pgfscope}%
\pgfpathrectangle{\pgfqpoint{0.380943in}{6.110189in}}{\pgfqpoint{4.650000in}{0.614151in}}%
\pgfusepath{clip}%
\pgfsetbuttcap%
\pgfsetroundjoin%
\definecolor{currentfill}{rgb}{0.962414,0.923552,0.722891}%
\pgfsetfillcolor{currentfill}%
\pgfsetlinewidth{0.250937pt}%
\definecolor{currentstroke}{rgb}{1.000000,1.000000,1.000000}%
\pgfsetstrokecolor{currentstroke}%
\pgfsetdash{}{0pt}%
\pgfpathmoveto{\pgfqpoint{1.433773in}{6.548868in}}%
\pgfpathlineto{\pgfqpoint{1.521509in}{6.548868in}}%
\pgfpathlineto{\pgfqpoint{1.521509in}{6.461132in}}%
\pgfpathlineto{\pgfqpoint{1.433773in}{6.461132in}}%
\pgfpathlineto{\pgfqpoint{1.433773in}{6.548868in}}%
\pgfusepath{stroke,fill}%
\end{pgfscope}%
\begin{pgfscope}%
\pgfpathrectangle{\pgfqpoint{0.380943in}{6.110189in}}{\pgfqpoint{4.650000in}{0.614151in}}%
\pgfusepath{clip}%
\pgfsetbuttcap%
\pgfsetroundjoin%
\definecolor{currentfill}{rgb}{0.922338,0.400769,0.400769}%
\pgfsetfillcolor{currentfill}%
\pgfsetlinewidth{0.250937pt}%
\definecolor{currentstroke}{rgb}{1.000000,1.000000,1.000000}%
\pgfsetstrokecolor{currentstroke}%
\pgfsetdash{}{0pt}%
\pgfpathmoveto{\pgfqpoint{1.521509in}{6.548868in}}%
\pgfpathlineto{\pgfqpoint{1.609245in}{6.548868in}}%
\pgfpathlineto{\pgfqpoint{1.609245in}{6.461132in}}%
\pgfpathlineto{\pgfqpoint{1.521509in}{6.461132in}}%
\pgfpathlineto{\pgfqpoint{1.521509in}{6.548868in}}%
\pgfusepath{stroke,fill}%
\end{pgfscope}%
\begin{pgfscope}%
\pgfpathrectangle{\pgfqpoint{0.380943in}{6.110189in}}{\pgfqpoint{4.650000in}{0.614151in}}%
\pgfusepath{clip}%
\pgfsetbuttcap%
\pgfsetroundjoin%
\definecolor{currentfill}{rgb}{1.000000,0.605229,0.530719}%
\pgfsetfillcolor{currentfill}%
\pgfsetlinewidth{0.250937pt}%
\definecolor{currentstroke}{rgb}{1.000000,1.000000,1.000000}%
\pgfsetstrokecolor{currentstroke}%
\pgfsetdash{}{0pt}%
\pgfpathmoveto{\pgfqpoint{1.609245in}{6.548868in}}%
\pgfpathlineto{\pgfqpoint{1.696981in}{6.548868in}}%
\pgfpathlineto{\pgfqpoint{1.696981in}{6.461132in}}%
\pgfpathlineto{\pgfqpoint{1.609245in}{6.461132in}}%
\pgfpathlineto{\pgfqpoint{1.609245in}{6.548868in}}%
\pgfusepath{stroke,fill}%
\end{pgfscope}%
\begin{pgfscope}%
\pgfpathrectangle{\pgfqpoint{0.380943in}{6.110189in}}{\pgfqpoint{4.650000in}{0.614151in}}%
\pgfusepath{clip}%
\pgfsetbuttcap%
\pgfsetroundjoin%
\definecolor{currentfill}{rgb}{1.000000,0.509404,0.491473}%
\pgfsetfillcolor{currentfill}%
\pgfsetlinewidth{0.250937pt}%
\definecolor{currentstroke}{rgb}{1.000000,1.000000,1.000000}%
\pgfsetstrokecolor{currentstroke}%
\pgfsetdash{}{0pt}%
\pgfpathmoveto{\pgfqpoint{1.696981in}{6.548868in}}%
\pgfpathlineto{\pgfqpoint{1.784717in}{6.548868in}}%
\pgfpathlineto{\pgfqpoint{1.784717in}{6.461132in}}%
\pgfpathlineto{\pgfqpoint{1.696981in}{6.461132in}}%
\pgfpathlineto{\pgfqpoint{1.696981in}{6.548868in}}%
\pgfusepath{stroke,fill}%
\end{pgfscope}%
\begin{pgfscope}%
\pgfpathrectangle{\pgfqpoint{0.380943in}{6.110189in}}{\pgfqpoint{4.650000in}{0.614151in}}%
\pgfusepath{clip}%
\pgfsetbuttcap%
\pgfsetroundjoin%
\definecolor{currentfill}{rgb}{0.992326,0.765229,0.614840}%
\pgfsetfillcolor{currentfill}%
\pgfsetlinewidth{0.250937pt}%
\definecolor{currentstroke}{rgb}{1.000000,1.000000,1.000000}%
\pgfsetstrokecolor{currentstroke}%
\pgfsetdash{}{0pt}%
\pgfpathmoveto{\pgfqpoint{1.784717in}{6.548868in}}%
\pgfpathlineto{\pgfqpoint{1.872452in}{6.548868in}}%
\pgfpathlineto{\pgfqpoint{1.872452in}{6.461132in}}%
\pgfpathlineto{\pgfqpoint{1.784717in}{6.461132in}}%
\pgfpathlineto{\pgfqpoint{1.784717in}{6.548868in}}%
\pgfusepath{stroke,fill}%
\end{pgfscope}%
\begin{pgfscope}%
\pgfpathrectangle{\pgfqpoint{0.380943in}{6.110189in}}{\pgfqpoint{4.650000in}{0.614151in}}%
\pgfusepath{clip}%
\pgfsetbuttcap%
\pgfsetroundjoin%
\definecolor{currentfill}{rgb}{0.996571,0.720538,0.589189}%
\pgfsetfillcolor{currentfill}%
\pgfsetlinewidth{0.250937pt}%
\definecolor{currentstroke}{rgb}{1.000000,1.000000,1.000000}%
\pgfsetstrokecolor{currentstroke}%
\pgfsetdash{}{0pt}%
\pgfpathmoveto{\pgfqpoint{1.872452in}{6.548868in}}%
\pgfpathlineto{\pgfqpoint{1.960188in}{6.548868in}}%
\pgfpathlineto{\pgfqpoint{1.960188in}{6.461132in}}%
\pgfpathlineto{\pgfqpoint{1.872452in}{6.461132in}}%
\pgfpathlineto{\pgfqpoint{1.872452in}{6.548868in}}%
\pgfusepath{stroke,fill}%
\end{pgfscope}%
\begin{pgfscope}%
\pgfpathrectangle{\pgfqpoint{0.380943in}{6.110189in}}{\pgfqpoint{4.650000in}{0.614151in}}%
\pgfusepath{clip}%
\pgfsetbuttcap%
\pgfsetroundjoin%
\definecolor{currentfill}{rgb}{0.965444,0.906113,0.711757}%
\pgfsetfillcolor{currentfill}%
\pgfsetlinewidth{0.250937pt}%
\definecolor{currentstroke}{rgb}{1.000000,1.000000,1.000000}%
\pgfsetstrokecolor{currentstroke}%
\pgfsetdash{}{0pt}%
\pgfpathmoveto{\pgfqpoint{1.960188in}{6.548868in}}%
\pgfpathlineto{\pgfqpoint{2.047924in}{6.548868in}}%
\pgfpathlineto{\pgfqpoint{2.047924in}{6.461132in}}%
\pgfpathlineto{\pgfqpoint{1.960188in}{6.461132in}}%
\pgfpathlineto{\pgfqpoint{1.960188in}{6.548868in}}%
\pgfusepath{stroke,fill}%
\end{pgfscope}%
\begin{pgfscope}%
\pgfpathrectangle{\pgfqpoint{0.380943in}{6.110189in}}{\pgfqpoint{4.650000in}{0.614151in}}%
\pgfusepath{clip}%
\pgfsetbuttcap%
\pgfsetroundjoin%
\definecolor{currentfill}{rgb}{0.968166,0.945882,0.748604}%
\pgfsetfillcolor{currentfill}%
\pgfsetlinewidth{0.250937pt}%
\definecolor{currentstroke}{rgb}{1.000000,1.000000,1.000000}%
\pgfsetstrokecolor{currentstroke}%
\pgfsetdash{}{0pt}%
\pgfpathmoveto{\pgfqpoint{2.047924in}{6.548868in}}%
\pgfpathlineto{\pgfqpoint{2.135660in}{6.548868in}}%
\pgfpathlineto{\pgfqpoint{2.135660in}{6.461132in}}%
\pgfpathlineto{\pgfqpoint{2.047924in}{6.461132in}}%
\pgfpathlineto{\pgfqpoint{2.047924in}{6.548868in}}%
\pgfusepath{stroke,fill}%
\end{pgfscope}%
\begin{pgfscope}%
\pgfpathrectangle{\pgfqpoint{0.380943in}{6.110189in}}{\pgfqpoint{4.650000in}{0.614151in}}%
\pgfusepath{clip}%
\pgfsetbuttcap%
\pgfsetroundjoin%
\definecolor{currentfill}{rgb}{0.968166,0.945882,0.748604}%
\pgfsetfillcolor{currentfill}%
\pgfsetlinewidth{0.250937pt}%
\definecolor{currentstroke}{rgb}{1.000000,1.000000,1.000000}%
\pgfsetstrokecolor{currentstroke}%
\pgfsetdash{}{0pt}%
\pgfpathmoveto{\pgfqpoint{2.135660in}{6.548868in}}%
\pgfpathlineto{\pgfqpoint{2.223396in}{6.548868in}}%
\pgfpathlineto{\pgfqpoint{2.223396in}{6.461132in}}%
\pgfpathlineto{\pgfqpoint{2.135660in}{6.461132in}}%
\pgfpathlineto{\pgfqpoint{2.135660in}{6.548868in}}%
\pgfusepath{stroke,fill}%
\end{pgfscope}%
\begin{pgfscope}%
\pgfpathrectangle{\pgfqpoint{0.380943in}{6.110189in}}{\pgfqpoint{4.650000in}{0.614151in}}%
\pgfusepath{clip}%
\pgfsetbuttcap%
\pgfsetroundjoin%
\definecolor{currentfill}{rgb}{1.000000,0.605229,0.530719}%
\pgfsetfillcolor{currentfill}%
\pgfsetlinewidth{0.250937pt}%
\definecolor{currentstroke}{rgb}{1.000000,1.000000,1.000000}%
\pgfsetstrokecolor{currentstroke}%
\pgfsetdash{}{0pt}%
\pgfpathmoveto{\pgfqpoint{2.223396in}{6.548868in}}%
\pgfpathlineto{\pgfqpoint{2.311132in}{6.548868in}}%
\pgfpathlineto{\pgfqpoint{2.311132in}{6.461132in}}%
\pgfpathlineto{\pgfqpoint{2.223396in}{6.461132in}}%
\pgfpathlineto{\pgfqpoint{2.223396in}{6.548868in}}%
\pgfusepath{stroke,fill}%
\end{pgfscope}%
\begin{pgfscope}%
\pgfpathrectangle{\pgfqpoint{0.380943in}{6.110189in}}{\pgfqpoint{4.650000in}{0.614151in}}%
\pgfusepath{clip}%
\pgfsetbuttcap%
\pgfsetroundjoin%
\definecolor{currentfill}{rgb}{1.000000,0.605229,0.530719}%
\pgfsetfillcolor{currentfill}%
\pgfsetlinewidth{0.250937pt}%
\definecolor{currentstroke}{rgb}{1.000000,1.000000,1.000000}%
\pgfsetstrokecolor{currentstroke}%
\pgfsetdash{}{0pt}%
\pgfpathmoveto{\pgfqpoint{2.311132in}{6.548868in}}%
\pgfpathlineto{\pgfqpoint{2.398868in}{6.548868in}}%
\pgfpathlineto{\pgfqpoint{2.398868in}{6.461132in}}%
\pgfpathlineto{\pgfqpoint{2.311132in}{6.461132in}}%
\pgfpathlineto{\pgfqpoint{2.311132in}{6.548868in}}%
\pgfusepath{stroke,fill}%
\end{pgfscope}%
\begin{pgfscope}%
\pgfpathrectangle{\pgfqpoint{0.380943in}{6.110189in}}{\pgfqpoint{4.650000in}{0.614151in}}%
\pgfusepath{clip}%
\pgfsetbuttcap%
\pgfsetroundjoin%
\definecolor{currentfill}{rgb}{0.992326,0.765229,0.614840}%
\pgfsetfillcolor{currentfill}%
\pgfsetlinewidth{0.250937pt}%
\definecolor{currentstroke}{rgb}{1.000000,1.000000,1.000000}%
\pgfsetstrokecolor{currentstroke}%
\pgfsetdash{}{0pt}%
\pgfpathmoveto{\pgfqpoint{2.398868in}{6.548868in}}%
\pgfpathlineto{\pgfqpoint{2.486603in}{6.548868in}}%
\pgfpathlineto{\pgfqpoint{2.486603in}{6.461132in}}%
\pgfpathlineto{\pgfqpoint{2.398868in}{6.461132in}}%
\pgfpathlineto{\pgfqpoint{2.398868in}{6.548868in}}%
\pgfusepath{stroke,fill}%
\end{pgfscope}%
\begin{pgfscope}%
\pgfpathrectangle{\pgfqpoint{0.380943in}{6.110189in}}{\pgfqpoint{4.650000in}{0.614151in}}%
\pgfusepath{clip}%
\pgfsetbuttcap%
\pgfsetroundjoin%
\definecolor{currentfill}{rgb}{0.968166,0.945882,0.748604}%
\pgfsetfillcolor{currentfill}%
\pgfsetlinewidth{0.250937pt}%
\definecolor{currentstroke}{rgb}{1.000000,1.000000,1.000000}%
\pgfsetstrokecolor{currentstroke}%
\pgfsetdash{}{0pt}%
\pgfpathmoveto{\pgfqpoint{2.486603in}{6.548868in}}%
\pgfpathlineto{\pgfqpoint{2.574339in}{6.548868in}}%
\pgfpathlineto{\pgfqpoint{2.574339in}{6.461132in}}%
\pgfpathlineto{\pgfqpoint{2.486603in}{6.461132in}}%
\pgfpathlineto{\pgfqpoint{2.486603in}{6.548868in}}%
\pgfusepath{stroke,fill}%
\end{pgfscope}%
\begin{pgfscope}%
\pgfpathrectangle{\pgfqpoint{0.380943in}{6.110189in}}{\pgfqpoint{4.650000in}{0.614151in}}%
\pgfusepath{clip}%
\pgfsetbuttcap%
\pgfsetroundjoin%
\definecolor{currentfill}{rgb}{0.962414,0.923552,0.722891}%
\pgfsetfillcolor{currentfill}%
\pgfsetlinewidth{0.250937pt}%
\definecolor{currentstroke}{rgb}{1.000000,1.000000,1.000000}%
\pgfsetstrokecolor{currentstroke}%
\pgfsetdash{}{0pt}%
\pgfpathmoveto{\pgfqpoint{2.574339in}{6.548868in}}%
\pgfpathlineto{\pgfqpoint{2.662075in}{6.548868in}}%
\pgfpathlineto{\pgfqpoint{2.662075in}{6.461132in}}%
\pgfpathlineto{\pgfqpoint{2.574339in}{6.461132in}}%
\pgfpathlineto{\pgfqpoint{2.574339in}{6.548868in}}%
\pgfusepath{stroke,fill}%
\end{pgfscope}%
\begin{pgfscope}%
\pgfpathrectangle{\pgfqpoint{0.380943in}{6.110189in}}{\pgfqpoint{4.650000in}{0.614151in}}%
\pgfusepath{clip}%
\pgfsetbuttcap%
\pgfsetroundjoin%
\definecolor{currentfill}{rgb}{0.962414,0.923552,0.722891}%
\pgfsetfillcolor{currentfill}%
\pgfsetlinewidth{0.250937pt}%
\definecolor{currentstroke}{rgb}{1.000000,1.000000,1.000000}%
\pgfsetstrokecolor{currentstroke}%
\pgfsetdash{}{0pt}%
\pgfpathmoveto{\pgfqpoint{2.662075in}{6.548868in}}%
\pgfpathlineto{\pgfqpoint{2.749811in}{6.548868in}}%
\pgfpathlineto{\pgfqpoint{2.749811in}{6.461132in}}%
\pgfpathlineto{\pgfqpoint{2.662075in}{6.461132in}}%
\pgfpathlineto{\pgfqpoint{2.662075in}{6.548868in}}%
\pgfusepath{stroke,fill}%
\end{pgfscope}%
\begin{pgfscope}%
\pgfpathrectangle{\pgfqpoint{0.380943in}{6.110189in}}{\pgfqpoint{4.650000in}{0.614151in}}%
\pgfusepath{clip}%
\pgfsetbuttcap%
\pgfsetroundjoin%
\definecolor{currentfill}{rgb}{0.972549,0.870588,0.692810}%
\pgfsetfillcolor{currentfill}%
\pgfsetlinewidth{0.250937pt}%
\definecolor{currentstroke}{rgb}{1.000000,1.000000,1.000000}%
\pgfsetstrokecolor{currentstroke}%
\pgfsetdash{}{0pt}%
\pgfpathmoveto{\pgfqpoint{2.749811in}{6.548868in}}%
\pgfpathlineto{\pgfqpoint{2.837547in}{6.548868in}}%
\pgfpathlineto{\pgfqpoint{2.837547in}{6.461132in}}%
\pgfpathlineto{\pgfqpoint{2.749811in}{6.461132in}}%
\pgfpathlineto{\pgfqpoint{2.749811in}{6.548868in}}%
\pgfusepath{stroke,fill}%
\end{pgfscope}%
\begin{pgfscope}%
\pgfpathrectangle{\pgfqpoint{0.380943in}{6.110189in}}{\pgfqpoint{4.650000in}{0.614151in}}%
\pgfusepath{clip}%
\pgfsetbuttcap%
\pgfsetroundjoin%
\definecolor{currentfill}{rgb}{0.965444,0.906113,0.711757}%
\pgfsetfillcolor{currentfill}%
\pgfsetlinewidth{0.250937pt}%
\definecolor{currentstroke}{rgb}{1.000000,1.000000,1.000000}%
\pgfsetstrokecolor{currentstroke}%
\pgfsetdash{}{0pt}%
\pgfpathmoveto{\pgfqpoint{2.837547in}{6.548868in}}%
\pgfpathlineto{\pgfqpoint{2.925283in}{6.548868in}}%
\pgfpathlineto{\pgfqpoint{2.925283in}{6.461132in}}%
\pgfpathlineto{\pgfqpoint{2.837547in}{6.461132in}}%
\pgfpathlineto{\pgfqpoint{2.837547in}{6.548868in}}%
\pgfusepath{stroke,fill}%
\end{pgfscope}%
\begin{pgfscope}%
\pgfpathrectangle{\pgfqpoint{0.380943in}{6.110189in}}{\pgfqpoint{4.650000in}{0.614151in}}%
\pgfusepath{clip}%
\pgfsetbuttcap%
\pgfsetroundjoin%
\definecolor{currentfill}{rgb}{0.979654,0.837186,0.669619}%
\pgfsetfillcolor{currentfill}%
\pgfsetlinewidth{0.250937pt}%
\definecolor{currentstroke}{rgb}{1.000000,1.000000,1.000000}%
\pgfsetstrokecolor{currentstroke}%
\pgfsetdash{}{0pt}%
\pgfpathmoveto{\pgfqpoint{2.925283in}{6.548868in}}%
\pgfpathlineto{\pgfqpoint{3.013019in}{6.548868in}}%
\pgfpathlineto{\pgfqpoint{3.013019in}{6.461132in}}%
\pgfpathlineto{\pgfqpoint{2.925283in}{6.461132in}}%
\pgfpathlineto{\pgfqpoint{2.925283in}{6.548868in}}%
\pgfusepath{stroke,fill}%
\end{pgfscope}%
\begin{pgfscope}%
\pgfpathrectangle{\pgfqpoint{0.380943in}{6.110189in}}{\pgfqpoint{4.650000in}{0.614151in}}%
\pgfusepath{clip}%
\pgfsetbuttcap%
\pgfsetroundjoin%
\definecolor{currentfill}{rgb}{0.968166,0.945882,0.748604}%
\pgfsetfillcolor{currentfill}%
\pgfsetlinewidth{0.250937pt}%
\definecolor{currentstroke}{rgb}{1.000000,1.000000,1.000000}%
\pgfsetstrokecolor{currentstroke}%
\pgfsetdash{}{0pt}%
\pgfpathmoveto{\pgfqpoint{3.013019in}{6.548868in}}%
\pgfpathlineto{\pgfqpoint{3.100754in}{6.548868in}}%
\pgfpathlineto{\pgfqpoint{3.100754in}{6.461132in}}%
\pgfpathlineto{\pgfqpoint{3.013019in}{6.461132in}}%
\pgfpathlineto{\pgfqpoint{3.013019in}{6.548868in}}%
\pgfusepath{stroke,fill}%
\end{pgfscope}%
\begin{pgfscope}%
\pgfpathrectangle{\pgfqpoint{0.380943in}{6.110189in}}{\pgfqpoint{4.650000in}{0.614151in}}%
\pgfusepath{clip}%
\pgfsetbuttcap%
\pgfsetroundjoin%
\definecolor{currentfill}{rgb}{1.000000,0.605229,0.530719}%
\pgfsetfillcolor{currentfill}%
\pgfsetlinewidth{0.250937pt}%
\definecolor{currentstroke}{rgb}{1.000000,1.000000,1.000000}%
\pgfsetstrokecolor{currentstroke}%
\pgfsetdash{}{0pt}%
\pgfpathmoveto{\pgfqpoint{3.100754in}{6.548868in}}%
\pgfpathlineto{\pgfqpoint{3.188490in}{6.548868in}}%
\pgfpathlineto{\pgfqpoint{3.188490in}{6.461132in}}%
\pgfpathlineto{\pgfqpoint{3.100754in}{6.461132in}}%
\pgfpathlineto{\pgfqpoint{3.100754in}{6.548868in}}%
\pgfusepath{stroke,fill}%
\end{pgfscope}%
\begin{pgfscope}%
\pgfpathrectangle{\pgfqpoint{0.380943in}{6.110189in}}{\pgfqpoint{4.650000in}{0.614151in}}%
\pgfusepath{clip}%
\pgfsetbuttcap%
\pgfsetroundjoin%
\definecolor{currentfill}{rgb}{1.000000,1.000000,0.870204}%
\pgfsetfillcolor{currentfill}%
\pgfsetlinewidth{0.250937pt}%
\definecolor{currentstroke}{rgb}{1.000000,1.000000,1.000000}%
\pgfsetstrokecolor{currentstroke}%
\pgfsetdash{}{0pt}%
\pgfpathmoveto{\pgfqpoint{3.188490in}{6.548868in}}%
\pgfpathlineto{\pgfqpoint{3.276226in}{6.548868in}}%
\pgfpathlineto{\pgfqpoint{3.276226in}{6.461132in}}%
\pgfpathlineto{\pgfqpoint{3.188490in}{6.461132in}}%
\pgfpathlineto{\pgfqpoint{3.188490in}{6.548868in}}%
\pgfusepath{stroke,fill}%
\end{pgfscope}%
\begin{pgfscope}%
\pgfpathrectangle{\pgfqpoint{0.380943in}{6.110189in}}{\pgfqpoint{4.650000in}{0.614151in}}%
\pgfusepath{clip}%
\pgfsetbuttcap%
\pgfsetroundjoin%
\definecolor{currentfill}{rgb}{0.962414,0.923552,0.722891}%
\pgfsetfillcolor{currentfill}%
\pgfsetlinewidth{0.250937pt}%
\definecolor{currentstroke}{rgb}{1.000000,1.000000,1.000000}%
\pgfsetstrokecolor{currentstroke}%
\pgfsetdash{}{0pt}%
\pgfpathmoveto{\pgfqpoint{3.276226in}{6.548868in}}%
\pgfpathlineto{\pgfqpoint{3.363962in}{6.548868in}}%
\pgfpathlineto{\pgfqpoint{3.363962in}{6.461132in}}%
\pgfpathlineto{\pgfqpoint{3.276226in}{6.461132in}}%
\pgfpathlineto{\pgfqpoint{3.276226in}{6.548868in}}%
\pgfusepath{stroke,fill}%
\end{pgfscope}%
\begin{pgfscope}%
\pgfpathrectangle{\pgfqpoint{0.380943in}{6.110189in}}{\pgfqpoint{4.650000in}{0.614151in}}%
\pgfusepath{clip}%
\pgfsetbuttcap%
\pgfsetroundjoin%
\definecolor{currentfill}{rgb}{0.992326,0.765229,0.614840}%
\pgfsetfillcolor{currentfill}%
\pgfsetlinewidth{0.250937pt}%
\definecolor{currentstroke}{rgb}{1.000000,1.000000,1.000000}%
\pgfsetstrokecolor{currentstroke}%
\pgfsetdash{}{0pt}%
\pgfpathmoveto{\pgfqpoint{3.363962in}{6.548868in}}%
\pgfpathlineto{\pgfqpoint{3.451698in}{6.548868in}}%
\pgfpathlineto{\pgfqpoint{3.451698in}{6.461132in}}%
\pgfpathlineto{\pgfqpoint{3.363962in}{6.461132in}}%
\pgfpathlineto{\pgfqpoint{3.363962in}{6.548868in}}%
\pgfusepath{stroke,fill}%
\end{pgfscope}%
\begin{pgfscope}%
\pgfpathrectangle{\pgfqpoint{0.380943in}{6.110189in}}{\pgfqpoint{4.650000in}{0.614151in}}%
\pgfusepath{clip}%
\pgfsetbuttcap%
\pgfsetroundjoin%
\definecolor{currentfill}{rgb}{0.996571,0.720538,0.589189}%
\pgfsetfillcolor{currentfill}%
\pgfsetlinewidth{0.250937pt}%
\definecolor{currentstroke}{rgb}{1.000000,1.000000,1.000000}%
\pgfsetstrokecolor{currentstroke}%
\pgfsetdash{}{0pt}%
\pgfpathmoveto{\pgfqpoint{3.451698in}{6.548868in}}%
\pgfpathlineto{\pgfqpoint{3.539434in}{6.548868in}}%
\pgfpathlineto{\pgfqpoint{3.539434in}{6.461132in}}%
\pgfpathlineto{\pgfqpoint{3.451698in}{6.461132in}}%
\pgfpathlineto{\pgfqpoint{3.451698in}{6.548868in}}%
\pgfusepath{stroke,fill}%
\end{pgfscope}%
\begin{pgfscope}%
\pgfpathrectangle{\pgfqpoint{0.380943in}{6.110189in}}{\pgfqpoint{4.650000in}{0.614151in}}%
\pgfusepath{clip}%
\pgfsetbuttcap%
\pgfsetroundjoin%
\definecolor{currentfill}{rgb}{0.962414,0.923552,0.722891}%
\pgfsetfillcolor{currentfill}%
\pgfsetlinewidth{0.250937pt}%
\definecolor{currentstroke}{rgb}{1.000000,1.000000,1.000000}%
\pgfsetstrokecolor{currentstroke}%
\pgfsetdash{}{0pt}%
\pgfpathmoveto{\pgfqpoint{3.539434in}{6.548868in}}%
\pgfpathlineto{\pgfqpoint{3.627169in}{6.548868in}}%
\pgfpathlineto{\pgfqpoint{3.627169in}{6.461132in}}%
\pgfpathlineto{\pgfqpoint{3.539434in}{6.461132in}}%
\pgfpathlineto{\pgfqpoint{3.539434in}{6.548868in}}%
\pgfusepath{stroke,fill}%
\end{pgfscope}%
\begin{pgfscope}%
\pgfpathrectangle{\pgfqpoint{0.380943in}{6.110189in}}{\pgfqpoint{4.650000in}{0.614151in}}%
\pgfusepath{clip}%
\pgfsetbuttcap%
\pgfsetroundjoin%
\definecolor{currentfill}{rgb}{0.979654,0.837186,0.669619}%
\pgfsetfillcolor{currentfill}%
\pgfsetlinewidth{0.250937pt}%
\definecolor{currentstroke}{rgb}{1.000000,1.000000,1.000000}%
\pgfsetstrokecolor{currentstroke}%
\pgfsetdash{}{0pt}%
\pgfpathmoveto{\pgfqpoint{3.627169in}{6.548868in}}%
\pgfpathlineto{\pgfqpoint{3.714905in}{6.548868in}}%
\pgfpathlineto{\pgfqpoint{3.714905in}{6.461132in}}%
\pgfpathlineto{\pgfqpoint{3.627169in}{6.461132in}}%
\pgfpathlineto{\pgfqpoint{3.627169in}{6.548868in}}%
\pgfusepath{stroke,fill}%
\end{pgfscope}%
\begin{pgfscope}%
\pgfpathrectangle{\pgfqpoint{0.380943in}{6.110189in}}{\pgfqpoint{4.650000in}{0.614151in}}%
\pgfusepath{clip}%
\pgfsetbuttcap%
\pgfsetroundjoin%
\definecolor{currentfill}{rgb}{0.992326,0.765229,0.614840}%
\pgfsetfillcolor{currentfill}%
\pgfsetlinewidth{0.250937pt}%
\definecolor{currentstroke}{rgb}{1.000000,1.000000,1.000000}%
\pgfsetstrokecolor{currentstroke}%
\pgfsetdash{}{0pt}%
\pgfpathmoveto{\pgfqpoint{3.714905in}{6.548868in}}%
\pgfpathlineto{\pgfqpoint{3.802641in}{6.548868in}}%
\pgfpathlineto{\pgfqpoint{3.802641in}{6.461132in}}%
\pgfpathlineto{\pgfqpoint{3.714905in}{6.461132in}}%
\pgfpathlineto{\pgfqpoint{3.714905in}{6.548868in}}%
\pgfusepath{stroke,fill}%
\end{pgfscope}%
\begin{pgfscope}%
\pgfpathrectangle{\pgfqpoint{0.380943in}{6.110189in}}{\pgfqpoint{4.650000in}{0.614151in}}%
\pgfusepath{clip}%
\pgfsetbuttcap%
\pgfsetroundjoin%
\definecolor{currentfill}{rgb}{0.965444,0.906113,0.711757}%
\pgfsetfillcolor{currentfill}%
\pgfsetlinewidth{0.250937pt}%
\definecolor{currentstroke}{rgb}{1.000000,1.000000,1.000000}%
\pgfsetstrokecolor{currentstroke}%
\pgfsetdash{}{0pt}%
\pgfpathmoveto{\pgfqpoint{3.802641in}{6.548868in}}%
\pgfpathlineto{\pgfqpoint{3.890377in}{6.548868in}}%
\pgfpathlineto{\pgfqpoint{3.890377in}{6.461132in}}%
\pgfpathlineto{\pgfqpoint{3.802641in}{6.461132in}}%
\pgfpathlineto{\pgfqpoint{3.802641in}{6.548868in}}%
\pgfusepath{stroke,fill}%
\end{pgfscope}%
\begin{pgfscope}%
\pgfpathrectangle{\pgfqpoint{0.380943in}{6.110189in}}{\pgfqpoint{4.650000in}{0.614151in}}%
\pgfusepath{clip}%
\pgfsetbuttcap%
\pgfsetroundjoin%
\definecolor{currentfill}{rgb}{0.992326,0.765229,0.614840}%
\pgfsetfillcolor{currentfill}%
\pgfsetlinewidth{0.250937pt}%
\definecolor{currentstroke}{rgb}{1.000000,1.000000,1.000000}%
\pgfsetstrokecolor{currentstroke}%
\pgfsetdash{}{0pt}%
\pgfpathmoveto{\pgfqpoint{3.890377in}{6.548868in}}%
\pgfpathlineto{\pgfqpoint{3.978113in}{6.548868in}}%
\pgfpathlineto{\pgfqpoint{3.978113in}{6.461132in}}%
\pgfpathlineto{\pgfqpoint{3.890377in}{6.461132in}}%
\pgfpathlineto{\pgfqpoint{3.890377in}{6.548868in}}%
\pgfusepath{stroke,fill}%
\end{pgfscope}%
\begin{pgfscope}%
\pgfpathrectangle{\pgfqpoint{0.380943in}{6.110189in}}{\pgfqpoint{4.650000in}{0.614151in}}%
\pgfusepath{clip}%
\pgfsetbuttcap%
\pgfsetroundjoin%
\definecolor{currentfill}{rgb}{0.998939,0.658962,0.556032}%
\pgfsetfillcolor{currentfill}%
\pgfsetlinewidth{0.250937pt}%
\definecolor{currentstroke}{rgb}{1.000000,1.000000,1.000000}%
\pgfsetstrokecolor{currentstroke}%
\pgfsetdash{}{0pt}%
\pgfpathmoveto{\pgfqpoint{3.978113in}{6.548868in}}%
\pgfpathlineto{\pgfqpoint{4.065849in}{6.548868in}}%
\pgfpathlineto{\pgfqpoint{4.065849in}{6.461132in}}%
\pgfpathlineto{\pgfqpoint{3.978113in}{6.461132in}}%
\pgfpathlineto{\pgfqpoint{3.978113in}{6.548868in}}%
\pgfusepath{stroke,fill}%
\end{pgfscope}%
\begin{pgfscope}%
\pgfpathrectangle{\pgfqpoint{0.380943in}{6.110189in}}{\pgfqpoint{4.650000in}{0.614151in}}%
\pgfusepath{clip}%
\pgfsetbuttcap%
\pgfsetroundjoin%
\definecolor{currentfill}{rgb}{0.986759,0.806398,0.641200}%
\pgfsetfillcolor{currentfill}%
\pgfsetlinewidth{0.250937pt}%
\definecolor{currentstroke}{rgb}{1.000000,1.000000,1.000000}%
\pgfsetstrokecolor{currentstroke}%
\pgfsetdash{}{0pt}%
\pgfpathmoveto{\pgfqpoint{4.065849in}{6.548868in}}%
\pgfpathlineto{\pgfqpoint{4.153585in}{6.548868in}}%
\pgfpathlineto{\pgfqpoint{4.153585in}{6.461132in}}%
\pgfpathlineto{\pgfqpoint{4.065849in}{6.461132in}}%
\pgfpathlineto{\pgfqpoint{4.065849in}{6.548868in}}%
\pgfusepath{stroke,fill}%
\end{pgfscope}%
\begin{pgfscope}%
\pgfpathrectangle{\pgfqpoint{0.380943in}{6.110189in}}{\pgfqpoint{4.650000in}{0.614151in}}%
\pgfusepath{clip}%
\pgfsetbuttcap%
\pgfsetroundjoin%
\definecolor{currentfill}{rgb}{0.986759,0.806398,0.641200}%
\pgfsetfillcolor{currentfill}%
\pgfsetlinewidth{0.250937pt}%
\definecolor{currentstroke}{rgb}{1.000000,1.000000,1.000000}%
\pgfsetstrokecolor{currentstroke}%
\pgfsetdash{}{0pt}%
\pgfpathmoveto{\pgfqpoint{4.153585in}{6.548868in}}%
\pgfpathlineto{\pgfqpoint{4.241320in}{6.548868in}}%
\pgfpathlineto{\pgfqpoint{4.241320in}{6.461132in}}%
\pgfpathlineto{\pgfqpoint{4.153585in}{6.461132in}}%
\pgfpathlineto{\pgfqpoint{4.153585in}{6.548868in}}%
\pgfusepath{stroke,fill}%
\end{pgfscope}%
\begin{pgfscope}%
\pgfpathrectangle{\pgfqpoint{0.380943in}{6.110189in}}{\pgfqpoint{4.650000in}{0.614151in}}%
\pgfusepath{clip}%
\pgfsetbuttcap%
\pgfsetroundjoin%
\definecolor{currentfill}{rgb}{0.992326,0.765229,0.614840}%
\pgfsetfillcolor{currentfill}%
\pgfsetlinewidth{0.250937pt}%
\definecolor{currentstroke}{rgb}{1.000000,1.000000,1.000000}%
\pgfsetstrokecolor{currentstroke}%
\pgfsetdash{}{0pt}%
\pgfpathmoveto{\pgfqpoint{4.241320in}{6.548868in}}%
\pgfpathlineto{\pgfqpoint{4.329056in}{6.548868in}}%
\pgfpathlineto{\pgfqpoint{4.329056in}{6.461132in}}%
\pgfpathlineto{\pgfqpoint{4.241320in}{6.461132in}}%
\pgfpathlineto{\pgfqpoint{4.241320in}{6.548868in}}%
\pgfusepath{stroke,fill}%
\end{pgfscope}%
\begin{pgfscope}%
\pgfpathrectangle{\pgfqpoint{0.380943in}{6.110189in}}{\pgfqpoint{4.650000in}{0.614151in}}%
\pgfusepath{clip}%
\pgfsetbuttcap%
\pgfsetroundjoin%
\definecolor{currentfill}{rgb}{0.979654,0.837186,0.669619}%
\pgfsetfillcolor{currentfill}%
\pgfsetlinewidth{0.250937pt}%
\definecolor{currentstroke}{rgb}{1.000000,1.000000,1.000000}%
\pgfsetstrokecolor{currentstroke}%
\pgfsetdash{}{0pt}%
\pgfpathmoveto{\pgfqpoint{4.329056in}{6.548868in}}%
\pgfpathlineto{\pgfqpoint{4.416792in}{6.548868in}}%
\pgfpathlineto{\pgfqpoint{4.416792in}{6.461132in}}%
\pgfpathlineto{\pgfqpoint{4.329056in}{6.461132in}}%
\pgfpathlineto{\pgfqpoint{4.329056in}{6.548868in}}%
\pgfusepath{stroke,fill}%
\end{pgfscope}%
\begin{pgfscope}%
\pgfpathrectangle{\pgfqpoint{0.380943in}{6.110189in}}{\pgfqpoint{4.650000in}{0.614151in}}%
\pgfusepath{clip}%
\pgfsetbuttcap%
\pgfsetroundjoin%
\definecolor{currentfill}{rgb}{0.979654,0.837186,0.669619}%
\pgfsetfillcolor{currentfill}%
\pgfsetlinewidth{0.250937pt}%
\definecolor{currentstroke}{rgb}{1.000000,1.000000,1.000000}%
\pgfsetstrokecolor{currentstroke}%
\pgfsetdash{}{0pt}%
\pgfpathmoveto{\pgfqpoint{4.416792in}{6.548868in}}%
\pgfpathlineto{\pgfqpoint{4.504528in}{6.548868in}}%
\pgfpathlineto{\pgfqpoint{4.504528in}{6.461132in}}%
\pgfpathlineto{\pgfqpoint{4.416792in}{6.461132in}}%
\pgfpathlineto{\pgfqpoint{4.416792in}{6.548868in}}%
\pgfusepath{stroke,fill}%
\end{pgfscope}%
\begin{pgfscope}%
\pgfpathrectangle{\pgfqpoint{0.380943in}{6.110189in}}{\pgfqpoint{4.650000in}{0.614151in}}%
\pgfusepath{clip}%
\pgfsetbuttcap%
\pgfsetroundjoin%
\definecolor{currentfill}{rgb}{0.979654,0.837186,0.669619}%
\pgfsetfillcolor{currentfill}%
\pgfsetlinewidth{0.250937pt}%
\definecolor{currentstroke}{rgb}{1.000000,1.000000,1.000000}%
\pgfsetstrokecolor{currentstroke}%
\pgfsetdash{}{0pt}%
\pgfpathmoveto{\pgfqpoint{4.504528in}{6.548868in}}%
\pgfpathlineto{\pgfqpoint{4.592264in}{6.548868in}}%
\pgfpathlineto{\pgfqpoint{4.592264in}{6.461132in}}%
\pgfpathlineto{\pgfqpoint{4.504528in}{6.461132in}}%
\pgfpathlineto{\pgfqpoint{4.504528in}{6.548868in}}%
\pgfusepath{stroke,fill}%
\end{pgfscope}%
\begin{pgfscope}%
\pgfpathrectangle{\pgfqpoint{0.380943in}{6.110189in}}{\pgfqpoint{4.650000in}{0.614151in}}%
\pgfusepath{clip}%
\pgfsetbuttcap%
\pgfsetroundjoin%
\definecolor{currentfill}{rgb}{0.962414,0.923552,0.722891}%
\pgfsetfillcolor{currentfill}%
\pgfsetlinewidth{0.250937pt}%
\definecolor{currentstroke}{rgb}{1.000000,1.000000,1.000000}%
\pgfsetstrokecolor{currentstroke}%
\pgfsetdash{}{0pt}%
\pgfpathmoveto{\pgfqpoint{4.592264in}{6.548868in}}%
\pgfpathlineto{\pgfqpoint{4.680000in}{6.548868in}}%
\pgfpathlineto{\pgfqpoint{4.680000in}{6.461132in}}%
\pgfpathlineto{\pgfqpoint{4.592264in}{6.461132in}}%
\pgfpathlineto{\pgfqpoint{4.592264in}{6.548868in}}%
\pgfusepath{stroke,fill}%
\end{pgfscope}%
\begin{pgfscope}%
\pgfpathrectangle{\pgfqpoint{0.380943in}{6.110189in}}{\pgfqpoint{4.650000in}{0.614151in}}%
\pgfusepath{clip}%
\pgfsetbuttcap%
\pgfsetroundjoin%
\definecolor{currentfill}{rgb}{0.965444,0.906113,0.711757}%
\pgfsetfillcolor{currentfill}%
\pgfsetlinewidth{0.250937pt}%
\definecolor{currentstroke}{rgb}{1.000000,1.000000,1.000000}%
\pgfsetstrokecolor{currentstroke}%
\pgfsetdash{}{0pt}%
\pgfpathmoveto{\pgfqpoint{4.680000in}{6.548868in}}%
\pgfpathlineto{\pgfqpoint{4.767736in}{6.548868in}}%
\pgfpathlineto{\pgfqpoint{4.767736in}{6.461132in}}%
\pgfpathlineto{\pgfqpoint{4.680000in}{6.461132in}}%
\pgfpathlineto{\pgfqpoint{4.680000in}{6.548868in}}%
\pgfusepath{stroke,fill}%
\end{pgfscope}%
\begin{pgfscope}%
\pgfpathrectangle{\pgfqpoint{0.380943in}{6.110189in}}{\pgfqpoint{4.650000in}{0.614151in}}%
\pgfusepath{clip}%
\pgfsetbuttcap%
\pgfsetroundjoin%
\definecolor{currentfill}{rgb}{1.000000,0.605229,0.530719}%
\pgfsetfillcolor{currentfill}%
\pgfsetlinewidth{0.250937pt}%
\definecolor{currentstroke}{rgb}{1.000000,1.000000,1.000000}%
\pgfsetstrokecolor{currentstroke}%
\pgfsetdash{}{0pt}%
\pgfpathmoveto{\pgfqpoint{4.767736in}{6.548868in}}%
\pgfpathlineto{\pgfqpoint{4.855471in}{6.548868in}}%
\pgfpathlineto{\pgfqpoint{4.855471in}{6.461132in}}%
\pgfpathlineto{\pgfqpoint{4.767736in}{6.461132in}}%
\pgfpathlineto{\pgfqpoint{4.767736in}{6.548868in}}%
\pgfusepath{stroke,fill}%
\end{pgfscope}%
\begin{pgfscope}%
\pgfpathrectangle{\pgfqpoint{0.380943in}{6.110189in}}{\pgfqpoint{4.650000in}{0.614151in}}%
\pgfusepath{clip}%
\pgfsetbuttcap%
\pgfsetroundjoin%
\definecolor{currentfill}{rgb}{0.986759,0.806398,0.641200}%
\pgfsetfillcolor{currentfill}%
\pgfsetlinewidth{0.250937pt}%
\definecolor{currentstroke}{rgb}{1.000000,1.000000,1.000000}%
\pgfsetstrokecolor{currentstroke}%
\pgfsetdash{}{0pt}%
\pgfpathmoveto{\pgfqpoint{4.855471in}{6.548868in}}%
\pgfpathlineto{\pgfqpoint{4.943207in}{6.548868in}}%
\pgfpathlineto{\pgfqpoint{4.943207in}{6.461132in}}%
\pgfpathlineto{\pgfqpoint{4.855471in}{6.461132in}}%
\pgfpathlineto{\pgfqpoint{4.855471in}{6.548868in}}%
\pgfusepath{stroke,fill}%
\end{pgfscope}%
\begin{pgfscope}%
\pgfpathrectangle{\pgfqpoint{0.380943in}{6.110189in}}{\pgfqpoint{4.650000in}{0.614151in}}%
\pgfusepath{clip}%
\pgfsetbuttcap%
\pgfsetroundjoin%
\pgfsetlinewidth{0.250937pt}%
\definecolor{currentstroke}{rgb}{1.000000,1.000000,1.000000}%
\pgfsetstrokecolor{currentstroke}%
\pgfsetdash{}{0pt}%
\pgfpathmoveto{\pgfqpoint{4.943207in}{6.548868in}}%
\pgfpathlineto{\pgfqpoint{5.030943in}{6.548868in}}%
\pgfpathlineto{\pgfqpoint{5.030943in}{6.461132in}}%
\pgfpathlineto{\pgfqpoint{4.943207in}{6.461132in}}%
\pgfpathlineto{\pgfqpoint{4.943207in}{6.548868in}}%
\pgfusepath{stroke}%
\end{pgfscope}%
\begin{pgfscope}%
\pgfpathrectangle{\pgfqpoint{0.380943in}{6.110189in}}{\pgfqpoint{4.650000in}{0.614151in}}%
\pgfusepath{clip}%
\pgfsetbuttcap%
\pgfsetroundjoin%
\definecolor{currentfill}{rgb}{0.986759,0.806398,0.641200}%
\pgfsetfillcolor{currentfill}%
\pgfsetlinewidth{0.250937pt}%
\definecolor{currentstroke}{rgb}{1.000000,1.000000,1.000000}%
\pgfsetstrokecolor{currentstroke}%
\pgfsetdash{}{0pt}%
\pgfpathmoveto{\pgfqpoint{0.380943in}{6.461132in}}%
\pgfpathlineto{\pgfqpoint{0.468679in}{6.461132in}}%
\pgfpathlineto{\pgfqpoint{0.468679in}{6.373396in}}%
\pgfpathlineto{\pgfqpoint{0.380943in}{6.373396in}}%
\pgfpathlineto{\pgfqpoint{0.380943in}{6.461132in}}%
\pgfusepath{stroke,fill}%
\end{pgfscope}%
\begin{pgfscope}%
\pgfpathrectangle{\pgfqpoint{0.380943in}{6.110189in}}{\pgfqpoint{4.650000in}{0.614151in}}%
\pgfusepath{clip}%
\pgfsetbuttcap%
\pgfsetroundjoin%
\definecolor{currentfill}{rgb}{0.981546,0.459977,0.459977}%
\pgfsetfillcolor{currentfill}%
\pgfsetlinewidth{0.250937pt}%
\definecolor{currentstroke}{rgb}{1.000000,1.000000,1.000000}%
\pgfsetstrokecolor{currentstroke}%
\pgfsetdash{}{0pt}%
\pgfpathmoveto{\pgfqpoint{0.468679in}{6.461132in}}%
\pgfpathlineto{\pgfqpoint{0.556415in}{6.461132in}}%
\pgfpathlineto{\pgfqpoint{0.556415in}{6.373396in}}%
\pgfpathlineto{\pgfqpoint{0.468679in}{6.373396in}}%
\pgfpathlineto{\pgfqpoint{0.468679in}{6.461132in}}%
\pgfusepath{stroke,fill}%
\end{pgfscope}%
\begin{pgfscope}%
\pgfpathrectangle{\pgfqpoint{0.380943in}{6.110189in}}{\pgfqpoint{4.650000in}{0.614151in}}%
\pgfusepath{clip}%
\pgfsetbuttcap%
\pgfsetroundjoin%
\definecolor{currentfill}{rgb}{0.979654,0.837186,0.669619}%
\pgfsetfillcolor{currentfill}%
\pgfsetlinewidth{0.250937pt}%
\definecolor{currentstroke}{rgb}{1.000000,1.000000,1.000000}%
\pgfsetstrokecolor{currentstroke}%
\pgfsetdash{}{0pt}%
\pgfpathmoveto{\pgfqpoint{0.556415in}{6.461132in}}%
\pgfpathlineto{\pgfqpoint{0.644151in}{6.461132in}}%
\pgfpathlineto{\pgfqpoint{0.644151in}{6.373396in}}%
\pgfpathlineto{\pgfqpoint{0.556415in}{6.373396in}}%
\pgfpathlineto{\pgfqpoint{0.556415in}{6.461132in}}%
\pgfusepath{stroke,fill}%
\end{pgfscope}%
\begin{pgfscope}%
\pgfpathrectangle{\pgfqpoint{0.380943in}{6.110189in}}{\pgfqpoint{4.650000in}{0.614151in}}%
\pgfusepath{clip}%
\pgfsetbuttcap%
\pgfsetroundjoin%
\definecolor{currentfill}{rgb}{0.979654,0.837186,0.669619}%
\pgfsetfillcolor{currentfill}%
\pgfsetlinewidth{0.250937pt}%
\definecolor{currentstroke}{rgb}{1.000000,1.000000,1.000000}%
\pgfsetstrokecolor{currentstroke}%
\pgfsetdash{}{0pt}%
\pgfpathmoveto{\pgfqpoint{0.644151in}{6.461132in}}%
\pgfpathlineto{\pgfqpoint{0.731886in}{6.461132in}}%
\pgfpathlineto{\pgfqpoint{0.731886in}{6.373396in}}%
\pgfpathlineto{\pgfqpoint{0.644151in}{6.373396in}}%
\pgfpathlineto{\pgfqpoint{0.644151in}{6.461132in}}%
\pgfusepath{stroke,fill}%
\end{pgfscope}%
\begin{pgfscope}%
\pgfpathrectangle{\pgfqpoint{0.380943in}{6.110189in}}{\pgfqpoint{4.650000in}{0.614151in}}%
\pgfusepath{clip}%
\pgfsetbuttcap%
\pgfsetroundjoin%
\definecolor{currentfill}{rgb}{0.979654,0.837186,0.669619}%
\pgfsetfillcolor{currentfill}%
\pgfsetlinewidth{0.250937pt}%
\definecolor{currentstroke}{rgb}{1.000000,1.000000,1.000000}%
\pgfsetstrokecolor{currentstroke}%
\pgfsetdash{}{0pt}%
\pgfpathmoveto{\pgfqpoint{0.731886in}{6.461132in}}%
\pgfpathlineto{\pgfqpoint{0.819622in}{6.461132in}}%
\pgfpathlineto{\pgfqpoint{0.819622in}{6.373396in}}%
\pgfpathlineto{\pgfqpoint{0.731886in}{6.373396in}}%
\pgfpathlineto{\pgfqpoint{0.731886in}{6.461132in}}%
\pgfusepath{stroke,fill}%
\end{pgfscope}%
\begin{pgfscope}%
\pgfpathrectangle{\pgfqpoint{0.380943in}{6.110189in}}{\pgfqpoint{4.650000in}{0.614151in}}%
\pgfusepath{clip}%
\pgfsetbuttcap%
\pgfsetroundjoin%
\definecolor{currentfill}{rgb}{0.972549,0.870588,0.692810}%
\pgfsetfillcolor{currentfill}%
\pgfsetlinewidth{0.250937pt}%
\definecolor{currentstroke}{rgb}{1.000000,1.000000,1.000000}%
\pgfsetstrokecolor{currentstroke}%
\pgfsetdash{}{0pt}%
\pgfpathmoveto{\pgfqpoint{0.819622in}{6.461132in}}%
\pgfpathlineto{\pgfqpoint{0.907358in}{6.461132in}}%
\pgfpathlineto{\pgfqpoint{0.907358in}{6.373396in}}%
\pgfpathlineto{\pgfqpoint{0.819622in}{6.373396in}}%
\pgfpathlineto{\pgfqpoint{0.819622in}{6.461132in}}%
\pgfusepath{stroke,fill}%
\end{pgfscope}%
\begin{pgfscope}%
\pgfpathrectangle{\pgfqpoint{0.380943in}{6.110189in}}{\pgfqpoint{4.650000in}{0.614151in}}%
\pgfusepath{clip}%
\pgfsetbuttcap%
\pgfsetroundjoin%
\definecolor{currentfill}{rgb}{0.986759,0.806398,0.641200}%
\pgfsetfillcolor{currentfill}%
\pgfsetlinewidth{0.250937pt}%
\definecolor{currentstroke}{rgb}{1.000000,1.000000,1.000000}%
\pgfsetstrokecolor{currentstroke}%
\pgfsetdash{}{0pt}%
\pgfpathmoveto{\pgfqpoint{0.907358in}{6.461132in}}%
\pgfpathlineto{\pgfqpoint{0.995094in}{6.461132in}}%
\pgfpathlineto{\pgfqpoint{0.995094in}{6.373396in}}%
\pgfpathlineto{\pgfqpoint{0.907358in}{6.373396in}}%
\pgfpathlineto{\pgfqpoint{0.907358in}{6.461132in}}%
\pgfusepath{stroke,fill}%
\end{pgfscope}%
\begin{pgfscope}%
\pgfpathrectangle{\pgfqpoint{0.380943in}{6.110189in}}{\pgfqpoint{4.650000in}{0.614151in}}%
\pgfusepath{clip}%
\pgfsetbuttcap%
\pgfsetroundjoin%
\definecolor{currentfill}{rgb}{0.986759,0.806398,0.641200}%
\pgfsetfillcolor{currentfill}%
\pgfsetlinewidth{0.250937pt}%
\definecolor{currentstroke}{rgb}{1.000000,1.000000,1.000000}%
\pgfsetstrokecolor{currentstroke}%
\pgfsetdash{}{0pt}%
\pgfpathmoveto{\pgfqpoint{0.995094in}{6.461132in}}%
\pgfpathlineto{\pgfqpoint{1.082830in}{6.461132in}}%
\pgfpathlineto{\pgfqpoint{1.082830in}{6.373396in}}%
\pgfpathlineto{\pgfqpoint{0.995094in}{6.373396in}}%
\pgfpathlineto{\pgfqpoint{0.995094in}{6.461132in}}%
\pgfusepath{stroke,fill}%
\end{pgfscope}%
\begin{pgfscope}%
\pgfpathrectangle{\pgfqpoint{0.380943in}{6.110189in}}{\pgfqpoint{4.650000in}{0.614151in}}%
\pgfusepath{clip}%
\pgfsetbuttcap%
\pgfsetroundjoin%
\definecolor{currentfill}{rgb}{0.981546,0.459977,0.459977}%
\pgfsetfillcolor{currentfill}%
\pgfsetlinewidth{0.250937pt}%
\definecolor{currentstroke}{rgb}{1.000000,1.000000,1.000000}%
\pgfsetstrokecolor{currentstroke}%
\pgfsetdash{}{0pt}%
\pgfpathmoveto{\pgfqpoint{1.082830in}{6.461132in}}%
\pgfpathlineto{\pgfqpoint{1.170566in}{6.461132in}}%
\pgfpathlineto{\pgfqpoint{1.170566in}{6.373396in}}%
\pgfpathlineto{\pgfqpoint{1.082830in}{6.373396in}}%
\pgfpathlineto{\pgfqpoint{1.082830in}{6.461132in}}%
\pgfusepath{stroke,fill}%
\end{pgfscope}%
\begin{pgfscope}%
\pgfpathrectangle{\pgfqpoint{0.380943in}{6.110189in}}{\pgfqpoint{4.650000in}{0.614151in}}%
\pgfusepath{clip}%
\pgfsetbuttcap%
\pgfsetroundjoin%
\definecolor{currentfill}{rgb}{0.979654,0.837186,0.669619}%
\pgfsetfillcolor{currentfill}%
\pgfsetlinewidth{0.250937pt}%
\definecolor{currentstroke}{rgb}{1.000000,1.000000,1.000000}%
\pgfsetstrokecolor{currentstroke}%
\pgfsetdash{}{0pt}%
\pgfpathmoveto{\pgfqpoint{1.170566in}{6.461132in}}%
\pgfpathlineto{\pgfqpoint{1.258302in}{6.461132in}}%
\pgfpathlineto{\pgfqpoint{1.258302in}{6.373396in}}%
\pgfpathlineto{\pgfqpoint{1.170566in}{6.373396in}}%
\pgfpathlineto{\pgfqpoint{1.170566in}{6.461132in}}%
\pgfusepath{stroke,fill}%
\end{pgfscope}%
\begin{pgfscope}%
\pgfpathrectangle{\pgfqpoint{0.380943in}{6.110189in}}{\pgfqpoint{4.650000in}{0.614151in}}%
\pgfusepath{clip}%
\pgfsetbuttcap%
\pgfsetroundjoin%
\definecolor{currentfill}{rgb}{1.000000,0.557862,0.511772}%
\pgfsetfillcolor{currentfill}%
\pgfsetlinewidth{0.250937pt}%
\definecolor{currentstroke}{rgb}{1.000000,1.000000,1.000000}%
\pgfsetstrokecolor{currentstroke}%
\pgfsetdash{}{0pt}%
\pgfpathmoveto{\pgfqpoint{1.258302in}{6.461132in}}%
\pgfpathlineto{\pgfqpoint{1.346037in}{6.461132in}}%
\pgfpathlineto{\pgfqpoint{1.346037in}{6.373396in}}%
\pgfpathlineto{\pgfqpoint{1.258302in}{6.373396in}}%
\pgfpathlineto{\pgfqpoint{1.258302in}{6.461132in}}%
\pgfusepath{stroke,fill}%
\end{pgfscope}%
\begin{pgfscope}%
\pgfpathrectangle{\pgfqpoint{0.380943in}{6.110189in}}{\pgfqpoint{4.650000in}{0.614151in}}%
\pgfusepath{clip}%
\pgfsetbuttcap%
\pgfsetroundjoin%
\definecolor{currentfill}{rgb}{0.998939,0.658962,0.556032}%
\pgfsetfillcolor{currentfill}%
\pgfsetlinewidth{0.250937pt}%
\definecolor{currentstroke}{rgb}{1.000000,1.000000,1.000000}%
\pgfsetstrokecolor{currentstroke}%
\pgfsetdash{}{0pt}%
\pgfpathmoveto{\pgfqpoint{1.346037in}{6.461132in}}%
\pgfpathlineto{\pgfqpoint{1.433773in}{6.461132in}}%
\pgfpathlineto{\pgfqpoint{1.433773in}{6.373396in}}%
\pgfpathlineto{\pgfqpoint{1.346037in}{6.373396in}}%
\pgfpathlineto{\pgfqpoint{1.346037in}{6.461132in}}%
\pgfusepath{stroke,fill}%
\end{pgfscope}%
\begin{pgfscope}%
\pgfpathrectangle{\pgfqpoint{0.380943in}{6.110189in}}{\pgfqpoint{4.650000in}{0.614151in}}%
\pgfusepath{clip}%
\pgfsetbuttcap%
\pgfsetroundjoin%
\definecolor{currentfill}{rgb}{0.986759,0.806398,0.641200}%
\pgfsetfillcolor{currentfill}%
\pgfsetlinewidth{0.250937pt}%
\definecolor{currentstroke}{rgb}{1.000000,1.000000,1.000000}%
\pgfsetstrokecolor{currentstroke}%
\pgfsetdash{}{0pt}%
\pgfpathmoveto{\pgfqpoint{1.433773in}{6.461132in}}%
\pgfpathlineto{\pgfqpoint{1.521509in}{6.461132in}}%
\pgfpathlineto{\pgfqpoint{1.521509in}{6.373396in}}%
\pgfpathlineto{\pgfqpoint{1.433773in}{6.373396in}}%
\pgfpathlineto{\pgfqpoint{1.433773in}{6.461132in}}%
\pgfusepath{stroke,fill}%
\end{pgfscope}%
\begin{pgfscope}%
\pgfpathrectangle{\pgfqpoint{0.380943in}{6.110189in}}{\pgfqpoint{4.650000in}{0.614151in}}%
\pgfusepath{clip}%
\pgfsetbuttcap%
\pgfsetroundjoin%
\definecolor{currentfill}{rgb}{0.992326,0.765229,0.614840}%
\pgfsetfillcolor{currentfill}%
\pgfsetlinewidth{0.250937pt}%
\definecolor{currentstroke}{rgb}{1.000000,1.000000,1.000000}%
\pgfsetstrokecolor{currentstroke}%
\pgfsetdash{}{0pt}%
\pgfpathmoveto{\pgfqpoint{1.521509in}{6.461132in}}%
\pgfpathlineto{\pgfqpoint{1.609245in}{6.461132in}}%
\pgfpathlineto{\pgfqpoint{1.609245in}{6.373396in}}%
\pgfpathlineto{\pgfqpoint{1.521509in}{6.373396in}}%
\pgfpathlineto{\pgfqpoint{1.521509in}{6.461132in}}%
\pgfusepath{stroke,fill}%
\end{pgfscope}%
\begin{pgfscope}%
\pgfpathrectangle{\pgfqpoint{0.380943in}{6.110189in}}{\pgfqpoint{4.650000in}{0.614151in}}%
\pgfusepath{clip}%
\pgfsetbuttcap%
\pgfsetroundjoin%
\definecolor{currentfill}{rgb}{0.800000,0.278431,0.278431}%
\pgfsetfillcolor{currentfill}%
\pgfsetlinewidth{0.250937pt}%
\definecolor{currentstroke}{rgb}{1.000000,1.000000,1.000000}%
\pgfsetstrokecolor{currentstroke}%
\pgfsetdash{}{0pt}%
\pgfpathmoveto{\pgfqpoint{1.609245in}{6.461132in}}%
\pgfpathlineto{\pgfqpoint{1.696981in}{6.461132in}}%
\pgfpathlineto{\pgfqpoint{1.696981in}{6.373396in}}%
\pgfpathlineto{\pgfqpoint{1.609245in}{6.373396in}}%
\pgfpathlineto{\pgfqpoint{1.609245in}{6.461132in}}%
\pgfusepath{stroke,fill}%
\end{pgfscope}%
\begin{pgfscope}%
\pgfpathrectangle{\pgfqpoint{0.380943in}{6.110189in}}{\pgfqpoint{4.650000in}{0.614151in}}%
\pgfusepath{clip}%
\pgfsetbuttcap%
\pgfsetroundjoin%
\definecolor{currentfill}{rgb}{0.992326,0.765229,0.614840}%
\pgfsetfillcolor{currentfill}%
\pgfsetlinewidth{0.250937pt}%
\definecolor{currentstroke}{rgb}{1.000000,1.000000,1.000000}%
\pgfsetstrokecolor{currentstroke}%
\pgfsetdash{}{0pt}%
\pgfpathmoveto{\pgfqpoint{1.696981in}{6.461132in}}%
\pgfpathlineto{\pgfqpoint{1.784717in}{6.461132in}}%
\pgfpathlineto{\pgfqpoint{1.784717in}{6.373396in}}%
\pgfpathlineto{\pgfqpoint{1.696981in}{6.373396in}}%
\pgfpathlineto{\pgfqpoint{1.696981in}{6.461132in}}%
\pgfusepath{stroke,fill}%
\end{pgfscope}%
\begin{pgfscope}%
\pgfpathrectangle{\pgfqpoint{0.380943in}{6.110189in}}{\pgfqpoint{4.650000in}{0.614151in}}%
\pgfusepath{clip}%
\pgfsetbuttcap%
\pgfsetroundjoin%
\definecolor{currentfill}{rgb}{0.992326,0.765229,0.614840}%
\pgfsetfillcolor{currentfill}%
\pgfsetlinewidth{0.250937pt}%
\definecolor{currentstroke}{rgb}{1.000000,1.000000,1.000000}%
\pgfsetstrokecolor{currentstroke}%
\pgfsetdash{}{0pt}%
\pgfpathmoveto{\pgfqpoint{1.784717in}{6.461132in}}%
\pgfpathlineto{\pgfqpoint{1.872452in}{6.461132in}}%
\pgfpathlineto{\pgfqpoint{1.872452in}{6.373396in}}%
\pgfpathlineto{\pgfqpoint{1.784717in}{6.373396in}}%
\pgfpathlineto{\pgfqpoint{1.784717in}{6.461132in}}%
\pgfusepath{stroke,fill}%
\end{pgfscope}%
\begin{pgfscope}%
\pgfpathrectangle{\pgfqpoint{0.380943in}{6.110189in}}{\pgfqpoint{4.650000in}{0.614151in}}%
\pgfusepath{clip}%
\pgfsetbuttcap%
\pgfsetroundjoin%
\definecolor{currentfill}{rgb}{1.000000,0.605229,0.530719}%
\pgfsetfillcolor{currentfill}%
\pgfsetlinewidth{0.250937pt}%
\definecolor{currentstroke}{rgb}{1.000000,1.000000,1.000000}%
\pgfsetstrokecolor{currentstroke}%
\pgfsetdash{}{0pt}%
\pgfpathmoveto{\pgfqpoint{1.872452in}{6.461132in}}%
\pgfpathlineto{\pgfqpoint{1.960188in}{6.461132in}}%
\pgfpathlineto{\pgfqpoint{1.960188in}{6.373396in}}%
\pgfpathlineto{\pgfqpoint{1.872452in}{6.373396in}}%
\pgfpathlineto{\pgfqpoint{1.872452in}{6.461132in}}%
\pgfusepath{stroke,fill}%
\end{pgfscope}%
\begin{pgfscope}%
\pgfpathrectangle{\pgfqpoint{0.380943in}{6.110189in}}{\pgfqpoint{4.650000in}{0.614151in}}%
\pgfusepath{clip}%
\pgfsetbuttcap%
\pgfsetroundjoin%
\definecolor{currentfill}{rgb}{1.000000,1.000000,0.929412}%
\pgfsetfillcolor{currentfill}%
\pgfsetlinewidth{0.250937pt}%
\definecolor{currentstroke}{rgb}{1.000000,1.000000,1.000000}%
\pgfsetstrokecolor{currentstroke}%
\pgfsetdash{}{0pt}%
\pgfpathmoveto{\pgfqpoint{1.960188in}{6.461132in}}%
\pgfpathlineto{\pgfqpoint{2.047924in}{6.461132in}}%
\pgfpathlineto{\pgfqpoint{2.047924in}{6.373396in}}%
\pgfpathlineto{\pgfqpoint{1.960188in}{6.373396in}}%
\pgfpathlineto{\pgfqpoint{1.960188in}{6.461132in}}%
\pgfusepath{stroke,fill}%
\end{pgfscope}%
\begin{pgfscope}%
\pgfpathrectangle{\pgfqpoint{0.380943in}{6.110189in}}{\pgfqpoint{4.650000in}{0.614151in}}%
\pgfusepath{clip}%
\pgfsetbuttcap%
\pgfsetroundjoin%
\definecolor{currentfill}{rgb}{0.965444,0.906113,0.711757}%
\pgfsetfillcolor{currentfill}%
\pgfsetlinewidth{0.250937pt}%
\definecolor{currentstroke}{rgb}{1.000000,1.000000,1.000000}%
\pgfsetstrokecolor{currentstroke}%
\pgfsetdash{}{0pt}%
\pgfpathmoveto{\pgfqpoint{2.047924in}{6.461132in}}%
\pgfpathlineto{\pgfqpoint{2.135660in}{6.461132in}}%
\pgfpathlineto{\pgfqpoint{2.135660in}{6.373396in}}%
\pgfpathlineto{\pgfqpoint{2.047924in}{6.373396in}}%
\pgfpathlineto{\pgfqpoint{2.047924in}{6.461132in}}%
\pgfusepath{stroke,fill}%
\end{pgfscope}%
\begin{pgfscope}%
\pgfpathrectangle{\pgfqpoint{0.380943in}{6.110189in}}{\pgfqpoint{4.650000in}{0.614151in}}%
\pgfusepath{clip}%
\pgfsetbuttcap%
\pgfsetroundjoin%
\definecolor{currentfill}{rgb}{0.996571,0.720538,0.589189}%
\pgfsetfillcolor{currentfill}%
\pgfsetlinewidth{0.250937pt}%
\definecolor{currentstroke}{rgb}{1.000000,1.000000,1.000000}%
\pgfsetstrokecolor{currentstroke}%
\pgfsetdash{}{0pt}%
\pgfpathmoveto{\pgfqpoint{2.135660in}{6.461132in}}%
\pgfpathlineto{\pgfqpoint{2.223396in}{6.461132in}}%
\pgfpathlineto{\pgfqpoint{2.223396in}{6.373396in}}%
\pgfpathlineto{\pgfqpoint{2.135660in}{6.373396in}}%
\pgfpathlineto{\pgfqpoint{2.135660in}{6.461132in}}%
\pgfusepath{stroke,fill}%
\end{pgfscope}%
\begin{pgfscope}%
\pgfpathrectangle{\pgfqpoint{0.380943in}{6.110189in}}{\pgfqpoint{4.650000in}{0.614151in}}%
\pgfusepath{clip}%
\pgfsetbuttcap%
\pgfsetroundjoin%
\definecolor{currentfill}{rgb}{0.996571,0.720538,0.589189}%
\pgfsetfillcolor{currentfill}%
\pgfsetlinewidth{0.250937pt}%
\definecolor{currentstroke}{rgb}{1.000000,1.000000,1.000000}%
\pgfsetstrokecolor{currentstroke}%
\pgfsetdash{}{0pt}%
\pgfpathmoveto{\pgfqpoint{2.223396in}{6.461132in}}%
\pgfpathlineto{\pgfqpoint{2.311132in}{6.461132in}}%
\pgfpathlineto{\pgfqpoint{2.311132in}{6.373396in}}%
\pgfpathlineto{\pgfqpoint{2.223396in}{6.373396in}}%
\pgfpathlineto{\pgfqpoint{2.223396in}{6.461132in}}%
\pgfusepath{stroke,fill}%
\end{pgfscope}%
\begin{pgfscope}%
\pgfpathrectangle{\pgfqpoint{0.380943in}{6.110189in}}{\pgfqpoint{4.650000in}{0.614151in}}%
\pgfusepath{clip}%
\pgfsetbuttcap%
\pgfsetroundjoin%
\definecolor{currentfill}{rgb}{0.992326,0.765229,0.614840}%
\pgfsetfillcolor{currentfill}%
\pgfsetlinewidth{0.250937pt}%
\definecolor{currentstroke}{rgb}{1.000000,1.000000,1.000000}%
\pgfsetstrokecolor{currentstroke}%
\pgfsetdash{}{0pt}%
\pgfpathmoveto{\pgfqpoint{2.311132in}{6.461132in}}%
\pgfpathlineto{\pgfqpoint{2.398868in}{6.461132in}}%
\pgfpathlineto{\pgfqpoint{2.398868in}{6.373396in}}%
\pgfpathlineto{\pgfqpoint{2.311132in}{6.373396in}}%
\pgfpathlineto{\pgfqpoint{2.311132in}{6.461132in}}%
\pgfusepath{stroke,fill}%
\end{pgfscope}%
\begin{pgfscope}%
\pgfpathrectangle{\pgfqpoint{0.380943in}{6.110189in}}{\pgfqpoint{4.650000in}{0.614151in}}%
\pgfusepath{clip}%
\pgfsetbuttcap%
\pgfsetroundjoin%
\definecolor{currentfill}{rgb}{0.972549,0.870588,0.692810}%
\pgfsetfillcolor{currentfill}%
\pgfsetlinewidth{0.250937pt}%
\definecolor{currentstroke}{rgb}{1.000000,1.000000,1.000000}%
\pgfsetstrokecolor{currentstroke}%
\pgfsetdash{}{0pt}%
\pgfpathmoveto{\pgfqpoint{2.398868in}{6.461132in}}%
\pgfpathlineto{\pgfqpoint{2.486603in}{6.461132in}}%
\pgfpathlineto{\pgfqpoint{2.486603in}{6.373396in}}%
\pgfpathlineto{\pgfqpoint{2.398868in}{6.373396in}}%
\pgfpathlineto{\pgfqpoint{2.398868in}{6.461132in}}%
\pgfusepath{stroke,fill}%
\end{pgfscope}%
\begin{pgfscope}%
\pgfpathrectangle{\pgfqpoint{0.380943in}{6.110189in}}{\pgfqpoint{4.650000in}{0.614151in}}%
\pgfusepath{clip}%
\pgfsetbuttcap%
\pgfsetroundjoin%
\definecolor{currentfill}{rgb}{0.979654,0.837186,0.669619}%
\pgfsetfillcolor{currentfill}%
\pgfsetlinewidth{0.250937pt}%
\definecolor{currentstroke}{rgb}{1.000000,1.000000,1.000000}%
\pgfsetstrokecolor{currentstroke}%
\pgfsetdash{}{0pt}%
\pgfpathmoveto{\pgfqpoint{2.486603in}{6.461132in}}%
\pgfpathlineto{\pgfqpoint{2.574339in}{6.461132in}}%
\pgfpathlineto{\pgfqpoint{2.574339in}{6.373396in}}%
\pgfpathlineto{\pgfqpoint{2.486603in}{6.373396in}}%
\pgfpathlineto{\pgfqpoint{2.486603in}{6.461132in}}%
\pgfusepath{stroke,fill}%
\end{pgfscope}%
\begin{pgfscope}%
\pgfpathrectangle{\pgfqpoint{0.380943in}{6.110189in}}{\pgfqpoint{4.650000in}{0.614151in}}%
\pgfusepath{clip}%
\pgfsetbuttcap%
\pgfsetroundjoin%
\definecolor{currentfill}{rgb}{0.996571,0.720538,0.589189}%
\pgfsetfillcolor{currentfill}%
\pgfsetlinewidth{0.250937pt}%
\definecolor{currentstroke}{rgb}{1.000000,1.000000,1.000000}%
\pgfsetstrokecolor{currentstroke}%
\pgfsetdash{}{0pt}%
\pgfpathmoveto{\pgfqpoint{2.574339in}{6.461132in}}%
\pgfpathlineto{\pgfqpoint{2.662075in}{6.461132in}}%
\pgfpathlineto{\pgfqpoint{2.662075in}{6.373396in}}%
\pgfpathlineto{\pgfqpoint{2.574339in}{6.373396in}}%
\pgfpathlineto{\pgfqpoint{2.574339in}{6.461132in}}%
\pgfusepath{stroke,fill}%
\end{pgfscope}%
\begin{pgfscope}%
\pgfpathrectangle{\pgfqpoint{0.380943in}{6.110189in}}{\pgfqpoint{4.650000in}{0.614151in}}%
\pgfusepath{clip}%
\pgfsetbuttcap%
\pgfsetroundjoin%
\definecolor{currentfill}{rgb}{0.986759,0.806398,0.641200}%
\pgfsetfillcolor{currentfill}%
\pgfsetlinewidth{0.250937pt}%
\definecolor{currentstroke}{rgb}{1.000000,1.000000,1.000000}%
\pgfsetstrokecolor{currentstroke}%
\pgfsetdash{}{0pt}%
\pgfpathmoveto{\pgfqpoint{2.662075in}{6.461132in}}%
\pgfpathlineto{\pgfqpoint{2.749811in}{6.461132in}}%
\pgfpathlineto{\pgfqpoint{2.749811in}{6.373396in}}%
\pgfpathlineto{\pgfqpoint{2.662075in}{6.373396in}}%
\pgfpathlineto{\pgfqpoint{2.662075in}{6.461132in}}%
\pgfusepath{stroke,fill}%
\end{pgfscope}%
\begin{pgfscope}%
\pgfpathrectangle{\pgfqpoint{0.380943in}{6.110189in}}{\pgfqpoint{4.650000in}{0.614151in}}%
\pgfusepath{clip}%
\pgfsetbuttcap%
\pgfsetroundjoin%
\definecolor{currentfill}{rgb}{0.972549,0.870588,0.692810}%
\pgfsetfillcolor{currentfill}%
\pgfsetlinewidth{0.250937pt}%
\definecolor{currentstroke}{rgb}{1.000000,1.000000,1.000000}%
\pgfsetstrokecolor{currentstroke}%
\pgfsetdash{}{0pt}%
\pgfpathmoveto{\pgfqpoint{2.749811in}{6.461132in}}%
\pgfpathlineto{\pgfqpoint{2.837547in}{6.461132in}}%
\pgfpathlineto{\pgfqpoint{2.837547in}{6.373396in}}%
\pgfpathlineto{\pgfqpoint{2.749811in}{6.373396in}}%
\pgfpathlineto{\pgfqpoint{2.749811in}{6.461132in}}%
\pgfusepath{stroke,fill}%
\end{pgfscope}%
\begin{pgfscope}%
\pgfpathrectangle{\pgfqpoint{0.380943in}{6.110189in}}{\pgfqpoint{4.650000in}{0.614151in}}%
\pgfusepath{clip}%
\pgfsetbuttcap%
\pgfsetroundjoin%
\definecolor{currentfill}{rgb}{0.998939,0.658962,0.556032}%
\pgfsetfillcolor{currentfill}%
\pgfsetlinewidth{0.250937pt}%
\definecolor{currentstroke}{rgb}{1.000000,1.000000,1.000000}%
\pgfsetstrokecolor{currentstroke}%
\pgfsetdash{}{0pt}%
\pgfpathmoveto{\pgfqpoint{2.837547in}{6.461132in}}%
\pgfpathlineto{\pgfqpoint{2.925283in}{6.461132in}}%
\pgfpathlineto{\pgfqpoint{2.925283in}{6.373396in}}%
\pgfpathlineto{\pgfqpoint{2.837547in}{6.373396in}}%
\pgfpathlineto{\pgfqpoint{2.837547in}{6.461132in}}%
\pgfusepath{stroke,fill}%
\end{pgfscope}%
\begin{pgfscope}%
\pgfpathrectangle{\pgfqpoint{0.380943in}{6.110189in}}{\pgfqpoint{4.650000in}{0.614151in}}%
\pgfusepath{clip}%
\pgfsetbuttcap%
\pgfsetroundjoin%
\definecolor{currentfill}{rgb}{0.992326,0.765229,0.614840}%
\pgfsetfillcolor{currentfill}%
\pgfsetlinewidth{0.250937pt}%
\definecolor{currentstroke}{rgb}{1.000000,1.000000,1.000000}%
\pgfsetstrokecolor{currentstroke}%
\pgfsetdash{}{0pt}%
\pgfpathmoveto{\pgfqpoint{2.925283in}{6.461132in}}%
\pgfpathlineto{\pgfqpoint{3.013019in}{6.461132in}}%
\pgfpathlineto{\pgfqpoint{3.013019in}{6.373396in}}%
\pgfpathlineto{\pgfqpoint{2.925283in}{6.373396in}}%
\pgfpathlineto{\pgfqpoint{2.925283in}{6.461132in}}%
\pgfusepath{stroke,fill}%
\end{pgfscope}%
\begin{pgfscope}%
\pgfpathrectangle{\pgfqpoint{0.380943in}{6.110189in}}{\pgfqpoint{4.650000in}{0.614151in}}%
\pgfusepath{clip}%
\pgfsetbuttcap%
\pgfsetroundjoin%
\definecolor{currentfill}{rgb}{0.965444,0.906113,0.711757}%
\pgfsetfillcolor{currentfill}%
\pgfsetlinewidth{0.250937pt}%
\definecolor{currentstroke}{rgb}{1.000000,1.000000,1.000000}%
\pgfsetstrokecolor{currentstroke}%
\pgfsetdash{}{0pt}%
\pgfpathmoveto{\pgfqpoint{3.013019in}{6.461132in}}%
\pgfpathlineto{\pgfqpoint{3.100754in}{6.461132in}}%
\pgfpathlineto{\pgfqpoint{3.100754in}{6.373396in}}%
\pgfpathlineto{\pgfqpoint{3.013019in}{6.373396in}}%
\pgfpathlineto{\pgfqpoint{3.013019in}{6.461132in}}%
\pgfusepath{stroke,fill}%
\end{pgfscope}%
\begin{pgfscope}%
\pgfpathrectangle{\pgfqpoint{0.380943in}{6.110189in}}{\pgfqpoint{4.650000in}{0.614151in}}%
\pgfusepath{clip}%
\pgfsetbuttcap%
\pgfsetroundjoin%
\definecolor{currentfill}{rgb}{0.965444,0.906113,0.711757}%
\pgfsetfillcolor{currentfill}%
\pgfsetlinewidth{0.250937pt}%
\definecolor{currentstroke}{rgb}{1.000000,1.000000,1.000000}%
\pgfsetstrokecolor{currentstroke}%
\pgfsetdash{}{0pt}%
\pgfpathmoveto{\pgfqpoint{3.100754in}{6.461132in}}%
\pgfpathlineto{\pgfqpoint{3.188490in}{6.461132in}}%
\pgfpathlineto{\pgfqpoint{3.188490in}{6.373396in}}%
\pgfpathlineto{\pgfqpoint{3.100754in}{6.373396in}}%
\pgfpathlineto{\pgfqpoint{3.100754in}{6.461132in}}%
\pgfusepath{stroke,fill}%
\end{pgfscope}%
\begin{pgfscope}%
\pgfpathrectangle{\pgfqpoint{0.380943in}{6.110189in}}{\pgfqpoint{4.650000in}{0.614151in}}%
\pgfusepath{clip}%
\pgfsetbuttcap%
\pgfsetroundjoin%
\definecolor{currentfill}{rgb}{0.972549,0.870588,0.692810}%
\pgfsetfillcolor{currentfill}%
\pgfsetlinewidth{0.250937pt}%
\definecolor{currentstroke}{rgb}{1.000000,1.000000,1.000000}%
\pgfsetstrokecolor{currentstroke}%
\pgfsetdash{}{0pt}%
\pgfpathmoveto{\pgfqpoint{3.188490in}{6.461132in}}%
\pgfpathlineto{\pgfqpoint{3.276226in}{6.461132in}}%
\pgfpathlineto{\pgfqpoint{3.276226in}{6.373396in}}%
\pgfpathlineto{\pgfqpoint{3.188490in}{6.373396in}}%
\pgfpathlineto{\pgfqpoint{3.188490in}{6.461132in}}%
\pgfusepath{stroke,fill}%
\end{pgfscope}%
\begin{pgfscope}%
\pgfpathrectangle{\pgfqpoint{0.380943in}{6.110189in}}{\pgfqpoint{4.650000in}{0.614151in}}%
\pgfusepath{clip}%
\pgfsetbuttcap%
\pgfsetroundjoin%
\definecolor{currentfill}{rgb}{0.962414,0.923552,0.722891}%
\pgfsetfillcolor{currentfill}%
\pgfsetlinewidth{0.250937pt}%
\definecolor{currentstroke}{rgb}{1.000000,1.000000,1.000000}%
\pgfsetstrokecolor{currentstroke}%
\pgfsetdash{}{0pt}%
\pgfpathmoveto{\pgfqpoint{3.276226in}{6.461132in}}%
\pgfpathlineto{\pgfqpoint{3.363962in}{6.461132in}}%
\pgfpathlineto{\pgfqpoint{3.363962in}{6.373396in}}%
\pgfpathlineto{\pgfqpoint{3.276226in}{6.373396in}}%
\pgfpathlineto{\pgfqpoint{3.276226in}{6.461132in}}%
\pgfusepath{stroke,fill}%
\end{pgfscope}%
\begin{pgfscope}%
\pgfpathrectangle{\pgfqpoint{0.380943in}{6.110189in}}{\pgfqpoint{4.650000in}{0.614151in}}%
\pgfusepath{clip}%
\pgfsetbuttcap%
\pgfsetroundjoin%
\definecolor{currentfill}{rgb}{0.968166,0.945882,0.748604}%
\pgfsetfillcolor{currentfill}%
\pgfsetlinewidth{0.250937pt}%
\definecolor{currentstroke}{rgb}{1.000000,1.000000,1.000000}%
\pgfsetstrokecolor{currentstroke}%
\pgfsetdash{}{0pt}%
\pgfpathmoveto{\pgfqpoint{3.363962in}{6.461132in}}%
\pgfpathlineto{\pgfqpoint{3.451698in}{6.461132in}}%
\pgfpathlineto{\pgfqpoint{3.451698in}{6.373396in}}%
\pgfpathlineto{\pgfqpoint{3.363962in}{6.373396in}}%
\pgfpathlineto{\pgfqpoint{3.363962in}{6.461132in}}%
\pgfusepath{stroke,fill}%
\end{pgfscope}%
\begin{pgfscope}%
\pgfpathrectangle{\pgfqpoint{0.380943in}{6.110189in}}{\pgfqpoint{4.650000in}{0.614151in}}%
\pgfusepath{clip}%
\pgfsetbuttcap%
\pgfsetroundjoin%
\definecolor{currentfill}{rgb}{0.972549,0.870588,0.692810}%
\pgfsetfillcolor{currentfill}%
\pgfsetlinewidth{0.250937pt}%
\definecolor{currentstroke}{rgb}{1.000000,1.000000,1.000000}%
\pgfsetstrokecolor{currentstroke}%
\pgfsetdash{}{0pt}%
\pgfpathmoveto{\pgfqpoint{3.451698in}{6.461132in}}%
\pgfpathlineto{\pgfqpoint{3.539434in}{6.461132in}}%
\pgfpathlineto{\pgfqpoint{3.539434in}{6.373396in}}%
\pgfpathlineto{\pgfqpoint{3.451698in}{6.373396in}}%
\pgfpathlineto{\pgfqpoint{3.451698in}{6.461132in}}%
\pgfusepath{stroke,fill}%
\end{pgfscope}%
\begin{pgfscope}%
\pgfpathrectangle{\pgfqpoint{0.380943in}{6.110189in}}{\pgfqpoint{4.650000in}{0.614151in}}%
\pgfusepath{clip}%
\pgfsetbuttcap%
\pgfsetroundjoin%
\definecolor{currentfill}{rgb}{0.962414,0.923552,0.722891}%
\pgfsetfillcolor{currentfill}%
\pgfsetlinewidth{0.250937pt}%
\definecolor{currentstroke}{rgb}{1.000000,1.000000,1.000000}%
\pgfsetstrokecolor{currentstroke}%
\pgfsetdash{}{0pt}%
\pgfpathmoveto{\pgfqpoint{3.539434in}{6.461132in}}%
\pgfpathlineto{\pgfqpoint{3.627169in}{6.461132in}}%
\pgfpathlineto{\pgfqpoint{3.627169in}{6.373396in}}%
\pgfpathlineto{\pgfqpoint{3.539434in}{6.373396in}}%
\pgfpathlineto{\pgfqpoint{3.539434in}{6.461132in}}%
\pgfusepath{stroke,fill}%
\end{pgfscope}%
\begin{pgfscope}%
\pgfpathrectangle{\pgfqpoint{0.380943in}{6.110189in}}{\pgfqpoint{4.650000in}{0.614151in}}%
\pgfusepath{clip}%
\pgfsetbuttcap%
\pgfsetroundjoin%
\definecolor{currentfill}{rgb}{0.979654,0.837186,0.669619}%
\pgfsetfillcolor{currentfill}%
\pgfsetlinewidth{0.250937pt}%
\definecolor{currentstroke}{rgb}{1.000000,1.000000,1.000000}%
\pgfsetstrokecolor{currentstroke}%
\pgfsetdash{}{0pt}%
\pgfpathmoveto{\pgfqpoint{3.627169in}{6.461132in}}%
\pgfpathlineto{\pgfqpoint{3.714905in}{6.461132in}}%
\pgfpathlineto{\pgfqpoint{3.714905in}{6.373396in}}%
\pgfpathlineto{\pgfqpoint{3.627169in}{6.373396in}}%
\pgfpathlineto{\pgfqpoint{3.627169in}{6.461132in}}%
\pgfusepath{stroke,fill}%
\end{pgfscope}%
\begin{pgfscope}%
\pgfpathrectangle{\pgfqpoint{0.380943in}{6.110189in}}{\pgfqpoint{4.650000in}{0.614151in}}%
\pgfusepath{clip}%
\pgfsetbuttcap%
\pgfsetroundjoin%
\definecolor{currentfill}{rgb}{0.968166,0.945882,0.748604}%
\pgfsetfillcolor{currentfill}%
\pgfsetlinewidth{0.250937pt}%
\definecolor{currentstroke}{rgb}{1.000000,1.000000,1.000000}%
\pgfsetstrokecolor{currentstroke}%
\pgfsetdash{}{0pt}%
\pgfpathmoveto{\pgfqpoint{3.714905in}{6.461132in}}%
\pgfpathlineto{\pgfqpoint{3.802641in}{6.461132in}}%
\pgfpathlineto{\pgfqpoint{3.802641in}{6.373396in}}%
\pgfpathlineto{\pgfqpoint{3.714905in}{6.373396in}}%
\pgfpathlineto{\pgfqpoint{3.714905in}{6.461132in}}%
\pgfusepath{stroke,fill}%
\end{pgfscope}%
\begin{pgfscope}%
\pgfpathrectangle{\pgfqpoint{0.380943in}{6.110189in}}{\pgfqpoint{4.650000in}{0.614151in}}%
\pgfusepath{clip}%
\pgfsetbuttcap%
\pgfsetroundjoin%
\definecolor{currentfill}{rgb}{0.998939,0.658962,0.556032}%
\pgfsetfillcolor{currentfill}%
\pgfsetlinewidth{0.250937pt}%
\definecolor{currentstroke}{rgb}{1.000000,1.000000,1.000000}%
\pgfsetstrokecolor{currentstroke}%
\pgfsetdash{}{0pt}%
\pgfpathmoveto{\pgfqpoint{3.802641in}{6.461132in}}%
\pgfpathlineto{\pgfqpoint{3.890377in}{6.461132in}}%
\pgfpathlineto{\pgfqpoint{3.890377in}{6.373396in}}%
\pgfpathlineto{\pgfqpoint{3.802641in}{6.373396in}}%
\pgfpathlineto{\pgfqpoint{3.802641in}{6.461132in}}%
\pgfusepath{stroke,fill}%
\end{pgfscope}%
\begin{pgfscope}%
\pgfpathrectangle{\pgfqpoint{0.380943in}{6.110189in}}{\pgfqpoint{4.650000in}{0.614151in}}%
\pgfusepath{clip}%
\pgfsetbuttcap%
\pgfsetroundjoin%
\definecolor{currentfill}{rgb}{0.972549,0.870588,0.692810}%
\pgfsetfillcolor{currentfill}%
\pgfsetlinewidth{0.250937pt}%
\definecolor{currentstroke}{rgb}{1.000000,1.000000,1.000000}%
\pgfsetstrokecolor{currentstroke}%
\pgfsetdash{}{0pt}%
\pgfpathmoveto{\pgfqpoint{3.890377in}{6.461132in}}%
\pgfpathlineto{\pgfqpoint{3.978113in}{6.461132in}}%
\pgfpathlineto{\pgfqpoint{3.978113in}{6.373396in}}%
\pgfpathlineto{\pgfqpoint{3.890377in}{6.373396in}}%
\pgfpathlineto{\pgfqpoint{3.890377in}{6.461132in}}%
\pgfusepath{stroke,fill}%
\end{pgfscope}%
\begin{pgfscope}%
\pgfpathrectangle{\pgfqpoint{0.380943in}{6.110189in}}{\pgfqpoint{4.650000in}{0.614151in}}%
\pgfusepath{clip}%
\pgfsetbuttcap%
\pgfsetroundjoin%
\definecolor{currentfill}{rgb}{1.000000,0.557862,0.511772}%
\pgfsetfillcolor{currentfill}%
\pgfsetlinewidth{0.250937pt}%
\definecolor{currentstroke}{rgb}{1.000000,1.000000,1.000000}%
\pgfsetstrokecolor{currentstroke}%
\pgfsetdash{}{0pt}%
\pgfpathmoveto{\pgfqpoint{3.978113in}{6.461132in}}%
\pgfpathlineto{\pgfqpoint{4.065849in}{6.461132in}}%
\pgfpathlineto{\pgfqpoint{4.065849in}{6.373396in}}%
\pgfpathlineto{\pgfqpoint{3.978113in}{6.373396in}}%
\pgfpathlineto{\pgfqpoint{3.978113in}{6.461132in}}%
\pgfusepath{stroke,fill}%
\end{pgfscope}%
\begin{pgfscope}%
\pgfpathrectangle{\pgfqpoint{0.380943in}{6.110189in}}{\pgfqpoint{4.650000in}{0.614151in}}%
\pgfusepath{clip}%
\pgfsetbuttcap%
\pgfsetroundjoin%
\definecolor{currentfill}{rgb}{0.992326,0.765229,0.614840}%
\pgfsetfillcolor{currentfill}%
\pgfsetlinewidth{0.250937pt}%
\definecolor{currentstroke}{rgb}{1.000000,1.000000,1.000000}%
\pgfsetstrokecolor{currentstroke}%
\pgfsetdash{}{0pt}%
\pgfpathmoveto{\pgfqpoint{4.065849in}{6.461132in}}%
\pgfpathlineto{\pgfqpoint{4.153585in}{6.461132in}}%
\pgfpathlineto{\pgfqpoint{4.153585in}{6.373396in}}%
\pgfpathlineto{\pgfqpoint{4.065849in}{6.373396in}}%
\pgfpathlineto{\pgfqpoint{4.065849in}{6.461132in}}%
\pgfusepath{stroke,fill}%
\end{pgfscope}%
\begin{pgfscope}%
\pgfpathrectangle{\pgfqpoint{0.380943in}{6.110189in}}{\pgfqpoint{4.650000in}{0.614151in}}%
\pgfusepath{clip}%
\pgfsetbuttcap%
\pgfsetroundjoin%
\definecolor{currentfill}{rgb}{0.968166,0.945882,0.748604}%
\pgfsetfillcolor{currentfill}%
\pgfsetlinewidth{0.250937pt}%
\definecolor{currentstroke}{rgb}{1.000000,1.000000,1.000000}%
\pgfsetstrokecolor{currentstroke}%
\pgfsetdash{}{0pt}%
\pgfpathmoveto{\pgfqpoint{4.153585in}{6.461132in}}%
\pgfpathlineto{\pgfqpoint{4.241320in}{6.461132in}}%
\pgfpathlineto{\pgfqpoint{4.241320in}{6.373396in}}%
\pgfpathlineto{\pgfqpoint{4.153585in}{6.373396in}}%
\pgfpathlineto{\pgfqpoint{4.153585in}{6.461132in}}%
\pgfusepath{stroke,fill}%
\end{pgfscope}%
\begin{pgfscope}%
\pgfpathrectangle{\pgfqpoint{0.380943in}{6.110189in}}{\pgfqpoint{4.650000in}{0.614151in}}%
\pgfusepath{clip}%
\pgfsetbuttcap%
\pgfsetroundjoin%
\definecolor{currentfill}{rgb}{0.992326,0.765229,0.614840}%
\pgfsetfillcolor{currentfill}%
\pgfsetlinewidth{0.250937pt}%
\definecolor{currentstroke}{rgb}{1.000000,1.000000,1.000000}%
\pgfsetstrokecolor{currentstroke}%
\pgfsetdash{}{0pt}%
\pgfpathmoveto{\pgfqpoint{4.241320in}{6.461132in}}%
\pgfpathlineto{\pgfqpoint{4.329056in}{6.461132in}}%
\pgfpathlineto{\pgfqpoint{4.329056in}{6.373396in}}%
\pgfpathlineto{\pgfqpoint{4.241320in}{6.373396in}}%
\pgfpathlineto{\pgfqpoint{4.241320in}{6.461132in}}%
\pgfusepath{stroke,fill}%
\end{pgfscope}%
\begin{pgfscope}%
\pgfpathrectangle{\pgfqpoint{0.380943in}{6.110189in}}{\pgfqpoint{4.650000in}{0.614151in}}%
\pgfusepath{clip}%
\pgfsetbuttcap%
\pgfsetroundjoin%
\definecolor{currentfill}{rgb}{1.000000,0.605229,0.530719}%
\pgfsetfillcolor{currentfill}%
\pgfsetlinewidth{0.250937pt}%
\definecolor{currentstroke}{rgb}{1.000000,1.000000,1.000000}%
\pgfsetstrokecolor{currentstroke}%
\pgfsetdash{}{0pt}%
\pgfpathmoveto{\pgfqpoint{4.329056in}{6.461132in}}%
\pgfpathlineto{\pgfqpoint{4.416792in}{6.461132in}}%
\pgfpathlineto{\pgfqpoint{4.416792in}{6.373396in}}%
\pgfpathlineto{\pgfqpoint{4.329056in}{6.373396in}}%
\pgfpathlineto{\pgfqpoint{4.329056in}{6.461132in}}%
\pgfusepath{stroke,fill}%
\end{pgfscope}%
\begin{pgfscope}%
\pgfpathrectangle{\pgfqpoint{0.380943in}{6.110189in}}{\pgfqpoint{4.650000in}{0.614151in}}%
\pgfusepath{clip}%
\pgfsetbuttcap%
\pgfsetroundjoin%
\definecolor{currentfill}{rgb}{0.972549,0.870588,0.692810}%
\pgfsetfillcolor{currentfill}%
\pgfsetlinewidth{0.250937pt}%
\definecolor{currentstroke}{rgb}{1.000000,1.000000,1.000000}%
\pgfsetstrokecolor{currentstroke}%
\pgfsetdash{}{0pt}%
\pgfpathmoveto{\pgfqpoint{4.416792in}{6.461132in}}%
\pgfpathlineto{\pgfqpoint{4.504528in}{6.461132in}}%
\pgfpathlineto{\pgfqpoint{4.504528in}{6.373396in}}%
\pgfpathlineto{\pgfqpoint{4.416792in}{6.373396in}}%
\pgfpathlineto{\pgfqpoint{4.416792in}{6.461132in}}%
\pgfusepath{stroke,fill}%
\end{pgfscope}%
\begin{pgfscope}%
\pgfpathrectangle{\pgfqpoint{0.380943in}{6.110189in}}{\pgfqpoint{4.650000in}{0.614151in}}%
\pgfusepath{clip}%
\pgfsetbuttcap%
\pgfsetroundjoin%
\definecolor{currentfill}{rgb}{0.996571,0.720538,0.589189}%
\pgfsetfillcolor{currentfill}%
\pgfsetlinewidth{0.250937pt}%
\definecolor{currentstroke}{rgb}{1.000000,1.000000,1.000000}%
\pgfsetstrokecolor{currentstroke}%
\pgfsetdash{}{0pt}%
\pgfpathmoveto{\pgfqpoint{4.504528in}{6.461132in}}%
\pgfpathlineto{\pgfqpoint{4.592264in}{6.461132in}}%
\pgfpathlineto{\pgfqpoint{4.592264in}{6.373396in}}%
\pgfpathlineto{\pgfqpoint{4.504528in}{6.373396in}}%
\pgfpathlineto{\pgfqpoint{4.504528in}{6.461132in}}%
\pgfusepath{stroke,fill}%
\end{pgfscope}%
\begin{pgfscope}%
\pgfpathrectangle{\pgfqpoint{0.380943in}{6.110189in}}{\pgfqpoint{4.650000in}{0.614151in}}%
\pgfusepath{clip}%
\pgfsetbuttcap%
\pgfsetroundjoin%
\definecolor{currentfill}{rgb}{1.000000,1.000000,0.929412}%
\pgfsetfillcolor{currentfill}%
\pgfsetlinewidth{0.250937pt}%
\definecolor{currentstroke}{rgb}{1.000000,1.000000,1.000000}%
\pgfsetstrokecolor{currentstroke}%
\pgfsetdash{}{0pt}%
\pgfpathmoveto{\pgfqpoint{4.592264in}{6.461132in}}%
\pgfpathlineto{\pgfqpoint{4.680000in}{6.461132in}}%
\pgfpathlineto{\pgfqpoint{4.680000in}{6.373396in}}%
\pgfpathlineto{\pgfqpoint{4.592264in}{6.373396in}}%
\pgfpathlineto{\pgfqpoint{4.592264in}{6.461132in}}%
\pgfusepath{stroke,fill}%
\end{pgfscope}%
\begin{pgfscope}%
\pgfpathrectangle{\pgfqpoint{0.380943in}{6.110189in}}{\pgfqpoint{4.650000in}{0.614151in}}%
\pgfusepath{clip}%
\pgfsetbuttcap%
\pgfsetroundjoin%
\definecolor{currentfill}{rgb}{0.979654,0.837186,0.669619}%
\pgfsetfillcolor{currentfill}%
\pgfsetlinewidth{0.250937pt}%
\definecolor{currentstroke}{rgb}{1.000000,1.000000,1.000000}%
\pgfsetstrokecolor{currentstroke}%
\pgfsetdash{}{0pt}%
\pgfpathmoveto{\pgfqpoint{4.680000in}{6.461132in}}%
\pgfpathlineto{\pgfqpoint{4.767736in}{6.461132in}}%
\pgfpathlineto{\pgfqpoint{4.767736in}{6.373396in}}%
\pgfpathlineto{\pgfqpoint{4.680000in}{6.373396in}}%
\pgfpathlineto{\pgfqpoint{4.680000in}{6.461132in}}%
\pgfusepath{stroke,fill}%
\end{pgfscope}%
\begin{pgfscope}%
\pgfpathrectangle{\pgfqpoint{0.380943in}{6.110189in}}{\pgfqpoint{4.650000in}{0.614151in}}%
\pgfusepath{clip}%
\pgfsetbuttcap%
\pgfsetroundjoin%
\definecolor{currentfill}{rgb}{0.972549,0.870588,0.692810}%
\pgfsetfillcolor{currentfill}%
\pgfsetlinewidth{0.250937pt}%
\definecolor{currentstroke}{rgb}{1.000000,1.000000,1.000000}%
\pgfsetstrokecolor{currentstroke}%
\pgfsetdash{}{0pt}%
\pgfpathmoveto{\pgfqpoint{4.767736in}{6.461132in}}%
\pgfpathlineto{\pgfqpoint{4.855471in}{6.461132in}}%
\pgfpathlineto{\pgfqpoint{4.855471in}{6.373396in}}%
\pgfpathlineto{\pgfqpoint{4.767736in}{6.373396in}}%
\pgfpathlineto{\pgfqpoint{4.767736in}{6.461132in}}%
\pgfusepath{stroke,fill}%
\end{pgfscope}%
\begin{pgfscope}%
\pgfpathrectangle{\pgfqpoint{0.380943in}{6.110189in}}{\pgfqpoint{4.650000in}{0.614151in}}%
\pgfusepath{clip}%
\pgfsetbuttcap%
\pgfsetroundjoin%
\definecolor{currentfill}{rgb}{0.979654,0.837186,0.669619}%
\pgfsetfillcolor{currentfill}%
\pgfsetlinewidth{0.250937pt}%
\definecolor{currentstroke}{rgb}{1.000000,1.000000,1.000000}%
\pgfsetstrokecolor{currentstroke}%
\pgfsetdash{}{0pt}%
\pgfpathmoveto{\pgfqpoint{4.855471in}{6.461132in}}%
\pgfpathlineto{\pgfqpoint{4.943207in}{6.461132in}}%
\pgfpathlineto{\pgfqpoint{4.943207in}{6.373396in}}%
\pgfpathlineto{\pgfqpoint{4.855471in}{6.373396in}}%
\pgfpathlineto{\pgfqpoint{4.855471in}{6.461132in}}%
\pgfusepath{stroke,fill}%
\end{pgfscope}%
\begin{pgfscope}%
\pgfpathrectangle{\pgfqpoint{0.380943in}{6.110189in}}{\pgfqpoint{4.650000in}{0.614151in}}%
\pgfusepath{clip}%
\pgfsetbuttcap%
\pgfsetroundjoin%
\pgfsetlinewidth{0.250937pt}%
\definecolor{currentstroke}{rgb}{1.000000,1.000000,1.000000}%
\pgfsetstrokecolor{currentstroke}%
\pgfsetdash{}{0pt}%
\pgfpathmoveto{\pgfqpoint{4.943207in}{6.461132in}}%
\pgfpathlineto{\pgfqpoint{5.030943in}{6.461132in}}%
\pgfpathlineto{\pgfqpoint{5.030943in}{6.373396in}}%
\pgfpathlineto{\pgfqpoint{4.943207in}{6.373396in}}%
\pgfpathlineto{\pgfqpoint{4.943207in}{6.461132in}}%
\pgfusepath{stroke}%
\end{pgfscope}%
\begin{pgfscope}%
\pgfpathrectangle{\pgfqpoint{0.380943in}{6.110189in}}{\pgfqpoint{4.650000in}{0.614151in}}%
\pgfusepath{clip}%
\pgfsetbuttcap%
\pgfsetroundjoin%
\definecolor{currentfill}{rgb}{1.000000,0.605229,0.530719}%
\pgfsetfillcolor{currentfill}%
\pgfsetlinewidth{0.250937pt}%
\definecolor{currentstroke}{rgb}{1.000000,1.000000,1.000000}%
\pgfsetstrokecolor{currentstroke}%
\pgfsetdash{}{0pt}%
\pgfpathmoveto{\pgfqpoint{0.380943in}{6.373396in}}%
\pgfpathlineto{\pgfqpoint{0.468679in}{6.373396in}}%
\pgfpathlineto{\pgfqpoint{0.468679in}{6.285661in}}%
\pgfpathlineto{\pgfqpoint{0.380943in}{6.285661in}}%
\pgfpathlineto{\pgfqpoint{0.380943in}{6.373396in}}%
\pgfusepath{stroke,fill}%
\end{pgfscope}%
\begin{pgfscope}%
\pgfpathrectangle{\pgfqpoint{0.380943in}{6.110189in}}{\pgfqpoint{4.650000in}{0.614151in}}%
\pgfusepath{clip}%
\pgfsetbuttcap%
\pgfsetroundjoin%
\definecolor{currentfill}{rgb}{0.998939,0.658962,0.556032}%
\pgfsetfillcolor{currentfill}%
\pgfsetlinewidth{0.250937pt}%
\definecolor{currentstroke}{rgb}{1.000000,1.000000,1.000000}%
\pgfsetstrokecolor{currentstroke}%
\pgfsetdash{}{0pt}%
\pgfpathmoveto{\pgfqpoint{0.468679in}{6.373396in}}%
\pgfpathlineto{\pgfqpoint{0.556415in}{6.373396in}}%
\pgfpathlineto{\pgfqpoint{0.556415in}{6.285661in}}%
\pgfpathlineto{\pgfqpoint{0.468679in}{6.285661in}}%
\pgfpathlineto{\pgfqpoint{0.468679in}{6.373396in}}%
\pgfusepath{stroke,fill}%
\end{pgfscope}%
\begin{pgfscope}%
\pgfpathrectangle{\pgfqpoint{0.380943in}{6.110189in}}{\pgfqpoint{4.650000in}{0.614151in}}%
\pgfusepath{clip}%
\pgfsetbuttcap%
\pgfsetroundjoin%
\definecolor{currentfill}{rgb}{0.992326,0.765229,0.614840}%
\pgfsetfillcolor{currentfill}%
\pgfsetlinewidth{0.250937pt}%
\definecolor{currentstroke}{rgb}{1.000000,1.000000,1.000000}%
\pgfsetstrokecolor{currentstroke}%
\pgfsetdash{}{0pt}%
\pgfpathmoveto{\pgfqpoint{0.556415in}{6.373396in}}%
\pgfpathlineto{\pgfqpoint{0.644151in}{6.373396in}}%
\pgfpathlineto{\pgfqpoint{0.644151in}{6.285661in}}%
\pgfpathlineto{\pgfqpoint{0.556415in}{6.285661in}}%
\pgfpathlineto{\pgfqpoint{0.556415in}{6.373396in}}%
\pgfusepath{stroke,fill}%
\end{pgfscope}%
\begin{pgfscope}%
\pgfpathrectangle{\pgfqpoint{0.380943in}{6.110189in}}{\pgfqpoint{4.650000in}{0.614151in}}%
\pgfusepath{clip}%
\pgfsetbuttcap%
\pgfsetroundjoin%
\definecolor{currentfill}{rgb}{1.000000,0.605229,0.530719}%
\pgfsetfillcolor{currentfill}%
\pgfsetlinewidth{0.250937pt}%
\definecolor{currentstroke}{rgb}{1.000000,1.000000,1.000000}%
\pgfsetstrokecolor{currentstroke}%
\pgfsetdash{}{0pt}%
\pgfpathmoveto{\pgfqpoint{0.644151in}{6.373396in}}%
\pgfpathlineto{\pgfqpoint{0.731886in}{6.373396in}}%
\pgfpathlineto{\pgfqpoint{0.731886in}{6.285661in}}%
\pgfpathlineto{\pgfqpoint{0.644151in}{6.285661in}}%
\pgfpathlineto{\pgfqpoint{0.644151in}{6.373396in}}%
\pgfusepath{stroke,fill}%
\end{pgfscope}%
\begin{pgfscope}%
\pgfpathrectangle{\pgfqpoint{0.380943in}{6.110189in}}{\pgfqpoint{4.650000in}{0.614151in}}%
\pgfusepath{clip}%
\pgfsetbuttcap%
\pgfsetroundjoin%
\definecolor{currentfill}{rgb}{0.981546,0.459977,0.459977}%
\pgfsetfillcolor{currentfill}%
\pgfsetlinewidth{0.250937pt}%
\definecolor{currentstroke}{rgb}{1.000000,1.000000,1.000000}%
\pgfsetstrokecolor{currentstroke}%
\pgfsetdash{}{0pt}%
\pgfpathmoveto{\pgfqpoint{0.731886in}{6.373396in}}%
\pgfpathlineto{\pgfqpoint{0.819622in}{6.373396in}}%
\pgfpathlineto{\pgfqpoint{0.819622in}{6.285661in}}%
\pgfpathlineto{\pgfqpoint{0.731886in}{6.285661in}}%
\pgfpathlineto{\pgfqpoint{0.731886in}{6.373396in}}%
\pgfusepath{stroke,fill}%
\end{pgfscope}%
\begin{pgfscope}%
\pgfpathrectangle{\pgfqpoint{0.380943in}{6.110189in}}{\pgfqpoint{4.650000in}{0.614151in}}%
\pgfusepath{clip}%
\pgfsetbuttcap%
\pgfsetroundjoin%
\definecolor{currentfill}{rgb}{0.992326,0.765229,0.614840}%
\pgfsetfillcolor{currentfill}%
\pgfsetlinewidth{0.250937pt}%
\definecolor{currentstroke}{rgb}{1.000000,1.000000,1.000000}%
\pgfsetstrokecolor{currentstroke}%
\pgfsetdash{}{0pt}%
\pgfpathmoveto{\pgfqpoint{0.819622in}{6.373396in}}%
\pgfpathlineto{\pgfqpoint{0.907358in}{6.373396in}}%
\pgfpathlineto{\pgfqpoint{0.907358in}{6.285661in}}%
\pgfpathlineto{\pgfqpoint{0.819622in}{6.285661in}}%
\pgfpathlineto{\pgfqpoint{0.819622in}{6.373396in}}%
\pgfusepath{stroke,fill}%
\end{pgfscope}%
\begin{pgfscope}%
\pgfpathrectangle{\pgfqpoint{0.380943in}{6.110189in}}{\pgfqpoint{4.650000in}{0.614151in}}%
\pgfusepath{clip}%
\pgfsetbuttcap%
\pgfsetroundjoin%
\definecolor{currentfill}{rgb}{0.968166,0.945882,0.748604}%
\pgfsetfillcolor{currentfill}%
\pgfsetlinewidth{0.250937pt}%
\definecolor{currentstroke}{rgb}{1.000000,1.000000,1.000000}%
\pgfsetstrokecolor{currentstroke}%
\pgfsetdash{}{0pt}%
\pgfpathmoveto{\pgfqpoint{0.907358in}{6.373396in}}%
\pgfpathlineto{\pgfqpoint{0.995094in}{6.373396in}}%
\pgfpathlineto{\pgfqpoint{0.995094in}{6.285661in}}%
\pgfpathlineto{\pgfqpoint{0.907358in}{6.285661in}}%
\pgfpathlineto{\pgfqpoint{0.907358in}{6.373396in}}%
\pgfusepath{stroke,fill}%
\end{pgfscope}%
\begin{pgfscope}%
\pgfpathrectangle{\pgfqpoint{0.380943in}{6.110189in}}{\pgfqpoint{4.650000in}{0.614151in}}%
\pgfusepath{clip}%
\pgfsetbuttcap%
\pgfsetroundjoin%
\definecolor{currentfill}{rgb}{0.979654,0.837186,0.669619}%
\pgfsetfillcolor{currentfill}%
\pgfsetlinewidth{0.250937pt}%
\definecolor{currentstroke}{rgb}{1.000000,1.000000,1.000000}%
\pgfsetstrokecolor{currentstroke}%
\pgfsetdash{}{0pt}%
\pgfpathmoveto{\pgfqpoint{0.995094in}{6.373396in}}%
\pgfpathlineto{\pgfqpoint{1.082830in}{6.373396in}}%
\pgfpathlineto{\pgfqpoint{1.082830in}{6.285661in}}%
\pgfpathlineto{\pgfqpoint{0.995094in}{6.285661in}}%
\pgfpathlineto{\pgfqpoint{0.995094in}{6.373396in}}%
\pgfusepath{stroke,fill}%
\end{pgfscope}%
\begin{pgfscope}%
\pgfpathrectangle{\pgfqpoint{0.380943in}{6.110189in}}{\pgfqpoint{4.650000in}{0.614151in}}%
\pgfusepath{clip}%
\pgfsetbuttcap%
\pgfsetroundjoin%
\definecolor{currentfill}{rgb}{0.992326,0.765229,0.614840}%
\pgfsetfillcolor{currentfill}%
\pgfsetlinewidth{0.250937pt}%
\definecolor{currentstroke}{rgb}{1.000000,1.000000,1.000000}%
\pgfsetstrokecolor{currentstroke}%
\pgfsetdash{}{0pt}%
\pgfpathmoveto{\pgfqpoint{1.082830in}{6.373396in}}%
\pgfpathlineto{\pgfqpoint{1.170566in}{6.373396in}}%
\pgfpathlineto{\pgfqpoint{1.170566in}{6.285661in}}%
\pgfpathlineto{\pgfqpoint{1.082830in}{6.285661in}}%
\pgfpathlineto{\pgfqpoint{1.082830in}{6.373396in}}%
\pgfusepath{stroke,fill}%
\end{pgfscope}%
\begin{pgfscope}%
\pgfpathrectangle{\pgfqpoint{0.380943in}{6.110189in}}{\pgfqpoint{4.650000in}{0.614151in}}%
\pgfusepath{clip}%
\pgfsetbuttcap%
\pgfsetroundjoin%
\definecolor{currentfill}{rgb}{0.972549,0.870588,0.692810}%
\pgfsetfillcolor{currentfill}%
\pgfsetlinewidth{0.250937pt}%
\definecolor{currentstroke}{rgb}{1.000000,1.000000,1.000000}%
\pgfsetstrokecolor{currentstroke}%
\pgfsetdash{}{0pt}%
\pgfpathmoveto{\pgfqpoint{1.170566in}{6.373396in}}%
\pgfpathlineto{\pgfqpoint{1.258302in}{6.373396in}}%
\pgfpathlineto{\pgfqpoint{1.258302in}{6.285661in}}%
\pgfpathlineto{\pgfqpoint{1.170566in}{6.285661in}}%
\pgfpathlineto{\pgfqpoint{1.170566in}{6.373396in}}%
\pgfusepath{stroke,fill}%
\end{pgfscope}%
\begin{pgfscope}%
\pgfpathrectangle{\pgfqpoint{0.380943in}{6.110189in}}{\pgfqpoint{4.650000in}{0.614151in}}%
\pgfusepath{clip}%
\pgfsetbuttcap%
\pgfsetroundjoin%
\definecolor{currentfill}{rgb}{0.992326,0.765229,0.614840}%
\pgfsetfillcolor{currentfill}%
\pgfsetlinewidth{0.250937pt}%
\definecolor{currentstroke}{rgb}{1.000000,1.000000,1.000000}%
\pgfsetstrokecolor{currentstroke}%
\pgfsetdash{}{0pt}%
\pgfpathmoveto{\pgfqpoint{1.258302in}{6.373396in}}%
\pgfpathlineto{\pgfqpoint{1.346037in}{6.373396in}}%
\pgfpathlineto{\pgfqpoint{1.346037in}{6.285661in}}%
\pgfpathlineto{\pgfqpoint{1.258302in}{6.285661in}}%
\pgfpathlineto{\pgfqpoint{1.258302in}{6.373396in}}%
\pgfusepath{stroke,fill}%
\end{pgfscope}%
\begin{pgfscope}%
\pgfpathrectangle{\pgfqpoint{0.380943in}{6.110189in}}{\pgfqpoint{4.650000in}{0.614151in}}%
\pgfusepath{clip}%
\pgfsetbuttcap%
\pgfsetroundjoin%
\definecolor{currentfill}{rgb}{0.992326,0.765229,0.614840}%
\pgfsetfillcolor{currentfill}%
\pgfsetlinewidth{0.250937pt}%
\definecolor{currentstroke}{rgb}{1.000000,1.000000,1.000000}%
\pgfsetstrokecolor{currentstroke}%
\pgfsetdash{}{0pt}%
\pgfpathmoveto{\pgfqpoint{1.346037in}{6.373396in}}%
\pgfpathlineto{\pgfqpoint{1.433773in}{6.373396in}}%
\pgfpathlineto{\pgfqpoint{1.433773in}{6.285661in}}%
\pgfpathlineto{\pgfqpoint{1.346037in}{6.285661in}}%
\pgfpathlineto{\pgfqpoint{1.346037in}{6.373396in}}%
\pgfusepath{stroke,fill}%
\end{pgfscope}%
\begin{pgfscope}%
\pgfpathrectangle{\pgfqpoint{0.380943in}{6.110189in}}{\pgfqpoint{4.650000in}{0.614151in}}%
\pgfusepath{clip}%
\pgfsetbuttcap%
\pgfsetroundjoin%
\definecolor{currentfill}{rgb}{0.998939,0.658962,0.556032}%
\pgfsetfillcolor{currentfill}%
\pgfsetlinewidth{0.250937pt}%
\definecolor{currentstroke}{rgb}{1.000000,1.000000,1.000000}%
\pgfsetstrokecolor{currentstroke}%
\pgfsetdash{}{0pt}%
\pgfpathmoveto{\pgfqpoint{1.433773in}{6.373396in}}%
\pgfpathlineto{\pgfqpoint{1.521509in}{6.373396in}}%
\pgfpathlineto{\pgfqpoint{1.521509in}{6.285661in}}%
\pgfpathlineto{\pgfqpoint{1.433773in}{6.285661in}}%
\pgfpathlineto{\pgfqpoint{1.433773in}{6.373396in}}%
\pgfusepath{stroke,fill}%
\end{pgfscope}%
\begin{pgfscope}%
\pgfpathrectangle{\pgfqpoint{0.380943in}{6.110189in}}{\pgfqpoint{4.650000in}{0.614151in}}%
\pgfusepath{clip}%
\pgfsetbuttcap%
\pgfsetroundjoin%
\definecolor{currentfill}{rgb}{0.986759,0.806398,0.641200}%
\pgfsetfillcolor{currentfill}%
\pgfsetlinewidth{0.250937pt}%
\definecolor{currentstroke}{rgb}{1.000000,1.000000,1.000000}%
\pgfsetstrokecolor{currentstroke}%
\pgfsetdash{}{0pt}%
\pgfpathmoveto{\pgfqpoint{1.521509in}{6.373396in}}%
\pgfpathlineto{\pgfqpoint{1.609245in}{6.373396in}}%
\pgfpathlineto{\pgfqpoint{1.609245in}{6.285661in}}%
\pgfpathlineto{\pgfqpoint{1.521509in}{6.285661in}}%
\pgfpathlineto{\pgfqpoint{1.521509in}{6.373396in}}%
\pgfusepath{stroke,fill}%
\end{pgfscope}%
\begin{pgfscope}%
\pgfpathrectangle{\pgfqpoint{0.380943in}{6.110189in}}{\pgfqpoint{4.650000in}{0.614151in}}%
\pgfusepath{clip}%
\pgfsetbuttcap%
\pgfsetroundjoin%
\definecolor{currentfill}{rgb}{0.996571,0.720538,0.589189}%
\pgfsetfillcolor{currentfill}%
\pgfsetlinewidth{0.250937pt}%
\definecolor{currentstroke}{rgb}{1.000000,1.000000,1.000000}%
\pgfsetstrokecolor{currentstroke}%
\pgfsetdash{}{0pt}%
\pgfpathmoveto{\pgfqpoint{1.609245in}{6.373396in}}%
\pgfpathlineto{\pgfqpoint{1.696981in}{6.373396in}}%
\pgfpathlineto{\pgfqpoint{1.696981in}{6.285661in}}%
\pgfpathlineto{\pgfqpoint{1.609245in}{6.285661in}}%
\pgfpathlineto{\pgfqpoint{1.609245in}{6.373396in}}%
\pgfusepath{stroke,fill}%
\end{pgfscope}%
\begin{pgfscope}%
\pgfpathrectangle{\pgfqpoint{0.380943in}{6.110189in}}{\pgfqpoint{4.650000in}{0.614151in}}%
\pgfusepath{clip}%
\pgfsetbuttcap%
\pgfsetroundjoin%
\definecolor{currentfill}{rgb}{0.986759,0.806398,0.641200}%
\pgfsetfillcolor{currentfill}%
\pgfsetlinewidth{0.250937pt}%
\definecolor{currentstroke}{rgb}{1.000000,1.000000,1.000000}%
\pgfsetstrokecolor{currentstroke}%
\pgfsetdash{}{0pt}%
\pgfpathmoveto{\pgfqpoint{1.696981in}{6.373396in}}%
\pgfpathlineto{\pgfqpoint{1.784717in}{6.373396in}}%
\pgfpathlineto{\pgfqpoint{1.784717in}{6.285661in}}%
\pgfpathlineto{\pgfqpoint{1.696981in}{6.285661in}}%
\pgfpathlineto{\pgfqpoint{1.696981in}{6.373396in}}%
\pgfusepath{stroke,fill}%
\end{pgfscope}%
\begin{pgfscope}%
\pgfpathrectangle{\pgfqpoint{0.380943in}{6.110189in}}{\pgfqpoint{4.650000in}{0.614151in}}%
\pgfusepath{clip}%
\pgfsetbuttcap%
\pgfsetroundjoin%
\definecolor{currentfill}{rgb}{0.962414,0.923552,0.722891}%
\pgfsetfillcolor{currentfill}%
\pgfsetlinewidth{0.250937pt}%
\definecolor{currentstroke}{rgb}{1.000000,1.000000,1.000000}%
\pgfsetstrokecolor{currentstroke}%
\pgfsetdash{}{0pt}%
\pgfpathmoveto{\pgfqpoint{1.784717in}{6.373396in}}%
\pgfpathlineto{\pgfqpoint{1.872452in}{6.373396in}}%
\pgfpathlineto{\pgfqpoint{1.872452in}{6.285661in}}%
\pgfpathlineto{\pgfqpoint{1.784717in}{6.285661in}}%
\pgfpathlineto{\pgfqpoint{1.784717in}{6.373396in}}%
\pgfusepath{stroke,fill}%
\end{pgfscope}%
\begin{pgfscope}%
\pgfpathrectangle{\pgfqpoint{0.380943in}{6.110189in}}{\pgfqpoint{4.650000in}{0.614151in}}%
\pgfusepath{clip}%
\pgfsetbuttcap%
\pgfsetroundjoin%
\definecolor{currentfill}{rgb}{0.986759,0.806398,0.641200}%
\pgfsetfillcolor{currentfill}%
\pgfsetlinewidth{0.250937pt}%
\definecolor{currentstroke}{rgb}{1.000000,1.000000,1.000000}%
\pgfsetstrokecolor{currentstroke}%
\pgfsetdash{}{0pt}%
\pgfpathmoveto{\pgfqpoint{1.872452in}{6.373396in}}%
\pgfpathlineto{\pgfqpoint{1.960188in}{6.373396in}}%
\pgfpathlineto{\pgfqpoint{1.960188in}{6.285661in}}%
\pgfpathlineto{\pgfqpoint{1.872452in}{6.285661in}}%
\pgfpathlineto{\pgfqpoint{1.872452in}{6.373396in}}%
\pgfusepath{stroke,fill}%
\end{pgfscope}%
\begin{pgfscope}%
\pgfpathrectangle{\pgfqpoint{0.380943in}{6.110189in}}{\pgfqpoint{4.650000in}{0.614151in}}%
\pgfusepath{clip}%
\pgfsetbuttcap%
\pgfsetroundjoin%
\definecolor{currentfill}{rgb}{0.986759,0.806398,0.641200}%
\pgfsetfillcolor{currentfill}%
\pgfsetlinewidth{0.250937pt}%
\definecolor{currentstroke}{rgb}{1.000000,1.000000,1.000000}%
\pgfsetstrokecolor{currentstroke}%
\pgfsetdash{}{0pt}%
\pgfpathmoveto{\pgfqpoint{1.960188in}{6.373396in}}%
\pgfpathlineto{\pgfqpoint{2.047924in}{6.373396in}}%
\pgfpathlineto{\pgfqpoint{2.047924in}{6.285661in}}%
\pgfpathlineto{\pgfqpoint{1.960188in}{6.285661in}}%
\pgfpathlineto{\pgfqpoint{1.960188in}{6.373396in}}%
\pgfusepath{stroke,fill}%
\end{pgfscope}%
\begin{pgfscope}%
\pgfpathrectangle{\pgfqpoint{0.380943in}{6.110189in}}{\pgfqpoint{4.650000in}{0.614151in}}%
\pgfusepath{clip}%
\pgfsetbuttcap%
\pgfsetroundjoin%
\definecolor{currentfill}{rgb}{0.972549,0.870588,0.692810}%
\pgfsetfillcolor{currentfill}%
\pgfsetlinewidth{0.250937pt}%
\definecolor{currentstroke}{rgb}{1.000000,1.000000,1.000000}%
\pgfsetstrokecolor{currentstroke}%
\pgfsetdash{}{0pt}%
\pgfpathmoveto{\pgfqpoint{2.047924in}{6.373396in}}%
\pgfpathlineto{\pgfqpoint{2.135660in}{6.373396in}}%
\pgfpathlineto{\pgfqpoint{2.135660in}{6.285661in}}%
\pgfpathlineto{\pgfqpoint{2.047924in}{6.285661in}}%
\pgfpathlineto{\pgfqpoint{2.047924in}{6.373396in}}%
\pgfusepath{stroke,fill}%
\end{pgfscope}%
\begin{pgfscope}%
\pgfpathrectangle{\pgfqpoint{0.380943in}{6.110189in}}{\pgfqpoint{4.650000in}{0.614151in}}%
\pgfusepath{clip}%
\pgfsetbuttcap%
\pgfsetroundjoin%
\definecolor{currentfill}{rgb}{0.996571,0.720538,0.589189}%
\pgfsetfillcolor{currentfill}%
\pgfsetlinewidth{0.250937pt}%
\definecolor{currentstroke}{rgb}{1.000000,1.000000,1.000000}%
\pgfsetstrokecolor{currentstroke}%
\pgfsetdash{}{0pt}%
\pgfpathmoveto{\pgfqpoint{2.135660in}{6.373396in}}%
\pgfpathlineto{\pgfqpoint{2.223396in}{6.373396in}}%
\pgfpathlineto{\pgfqpoint{2.223396in}{6.285661in}}%
\pgfpathlineto{\pgfqpoint{2.135660in}{6.285661in}}%
\pgfpathlineto{\pgfqpoint{2.135660in}{6.373396in}}%
\pgfusepath{stroke,fill}%
\end{pgfscope}%
\begin{pgfscope}%
\pgfpathrectangle{\pgfqpoint{0.380943in}{6.110189in}}{\pgfqpoint{4.650000in}{0.614151in}}%
\pgfusepath{clip}%
\pgfsetbuttcap%
\pgfsetroundjoin%
\definecolor{currentfill}{rgb}{0.992326,0.765229,0.614840}%
\pgfsetfillcolor{currentfill}%
\pgfsetlinewidth{0.250937pt}%
\definecolor{currentstroke}{rgb}{1.000000,1.000000,1.000000}%
\pgfsetstrokecolor{currentstroke}%
\pgfsetdash{}{0pt}%
\pgfpathmoveto{\pgfqpoint{2.223396in}{6.373396in}}%
\pgfpathlineto{\pgfqpoint{2.311132in}{6.373396in}}%
\pgfpathlineto{\pgfqpoint{2.311132in}{6.285661in}}%
\pgfpathlineto{\pgfqpoint{2.223396in}{6.285661in}}%
\pgfpathlineto{\pgfqpoint{2.223396in}{6.373396in}}%
\pgfusepath{stroke,fill}%
\end{pgfscope}%
\begin{pgfscope}%
\pgfpathrectangle{\pgfqpoint{0.380943in}{6.110189in}}{\pgfqpoint{4.650000in}{0.614151in}}%
\pgfusepath{clip}%
\pgfsetbuttcap%
\pgfsetroundjoin%
\definecolor{currentfill}{rgb}{0.986759,0.806398,0.641200}%
\pgfsetfillcolor{currentfill}%
\pgfsetlinewidth{0.250937pt}%
\definecolor{currentstroke}{rgb}{1.000000,1.000000,1.000000}%
\pgfsetstrokecolor{currentstroke}%
\pgfsetdash{}{0pt}%
\pgfpathmoveto{\pgfqpoint{2.311132in}{6.373396in}}%
\pgfpathlineto{\pgfqpoint{2.398868in}{6.373396in}}%
\pgfpathlineto{\pgfqpoint{2.398868in}{6.285661in}}%
\pgfpathlineto{\pgfqpoint{2.311132in}{6.285661in}}%
\pgfpathlineto{\pgfqpoint{2.311132in}{6.373396in}}%
\pgfusepath{stroke,fill}%
\end{pgfscope}%
\begin{pgfscope}%
\pgfpathrectangle{\pgfqpoint{0.380943in}{6.110189in}}{\pgfqpoint{4.650000in}{0.614151in}}%
\pgfusepath{clip}%
\pgfsetbuttcap%
\pgfsetroundjoin%
\definecolor{currentfill}{rgb}{0.968166,0.945882,0.748604}%
\pgfsetfillcolor{currentfill}%
\pgfsetlinewidth{0.250937pt}%
\definecolor{currentstroke}{rgb}{1.000000,1.000000,1.000000}%
\pgfsetstrokecolor{currentstroke}%
\pgfsetdash{}{0pt}%
\pgfpathmoveto{\pgfqpoint{2.398868in}{6.373396in}}%
\pgfpathlineto{\pgfqpoint{2.486603in}{6.373396in}}%
\pgfpathlineto{\pgfqpoint{2.486603in}{6.285661in}}%
\pgfpathlineto{\pgfqpoint{2.398868in}{6.285661in}}%
\pgfpathlineto{\pgfqpoint{2.398868in}{6.373396in}}%
\pgfusepath{stroke,fill}%
\end{pgfscope}%
\begin{pgfscope}%
\pgfpathrectangle{\pgfqpoint{0.380943in}{6.110189in}}{\pgfqpoint{4.650000in}{0.614151in}}%
\pgfusepath{clip}%
\pgfsetbuttcap%
\pgfsetroundjoin%
\definecolor{currentfill}{rgb}{0.965444,0.906113,0.711757}%
\pgfsetfillcolor{currentfill}%
\pgfsetlinewidth{0.250937pt}%
\definecolor{currentstroke}{rgb}{1.000000,1.000000,1.000000}%
\pgfsetstrokecolor{currentstroke}%
\pgfsetdash{}{0pt}%
\pgfpathmoveto{\pgfqpoint{2.486603in}{6.373396in}}%
\pgfpathlineto{\pgfqpoint{2.574339in}{6.373396in}}%
\pgfpathlineto{\pgfqpoint{2.574339in}{6.285661in}}%
\pgfpathlineto{\pgfqpoint{2.486603in}{6.285661in}}%
\pgfpathlineto{\pgfqpoint{2.486603in}{6.373396in}}%
\pgfusepath{stroke,fill}%
\end{pgfscope}%
\begin{pgfscope}%
\pgfpathrectangle{\pgfqpoint{0.380943in}{6.110189in}}{\pgfqpoint{4.650000in}{0.614151in}}%
\pgfusepath{clip}%
\pgfsetbuttcap%
\pgfsetroundjoin%
\definecolor{currentfill}{rgb}{0.979654,0.837186,0.669619}%
\pgfsetfillcolor{currentfill}%
\pgfsetlinewidth{0.250937pt}%
\definecolor{currentstroke}{rgb}{1.000000,1.000000,1.000000}%
\pgfsetstrokecolor{currentstroke}%
\pgfsetdash{}{0pt}%
\pgfpathmoveto{\pgfqpoint{2.574339in}{6.373396in}}%
\pgfpathlineto{\pgfqpoint{2.662075in}{6.373396in}}%
\pgfpathlineto{\pgfqpoint{2.662075in}{6.285661in}}%
\pgfpathlineto{\pgfqpoint{2.574339in}{6.285661in}}%
\pgfpathlineto{\pgfqpoint{2.574339in}{6.373396in}}%
\pgfusepath{stroke,fill}%
\end{pgfscope}%
\begin{pgfscope}%
\pgfpathrectangle{\pgfqpoint{0.380943in}{6.110189in}}{\pgfqpoint{4.650000in}{0.614151in}}%
\pgfusepath{clip}%
\pgfsetbuttcap%
\pgfsetroundjoin%
\definecolor{currentfill}{rgb}{0.962414,0.923552,0.722891}%
\pgfsetfillcolor{currentfill}%
\pgfsetlinewidth{0.250937pt}%
\definecolor{currentstroke}{rgb}{1.000000,1.000000,1.000000}%
\pgfsetstrokecolor{currentstroke}%
\pgfsetdash{}{0pt}%
\pgfpathmoveto{\pgfqpoint{2.662075in}{6.373396in}}%
\pgfpathlineto{\pgfqpoint{2.749811in}{6.373396in}}%
\pgfpathlineto{\pgfqpoint{2.749811in}{6.285661in}}%
\pgfpathlineto{\pgfqpoint{2.662075in}{6.285661in}}%
\pgfpathlineto{\pgfqpoint{2.662075in}{6.373396in}}%
\pgfusepath{stroke,fill}%
\end{pgfscope}%
\begin{pgfscope}%
\pgfpathrectangle{\pgfqpoint{0.380943in}{6.110189in}}{\pgfqpoint{4.650000in}{0.614151in}}%
\pgfusepath{clip}%
\pgfsetbuttcap%
\pgfsetroundjoin%
\definecolor{currentfill}{rgb}{0.992326,0.765229,0.614840}%
\pgfsetfillcolor{currentfill}%
\pgfsetlinewidth{0.250937pt}%
\definecolor{currentstroke}{rgb}{1.000000,1.000000,1.000000}%
\pgfsetstrokecolor{currentstroke}%
\pgfsetdash{}{0pt}%
\pgfpathmoveto{\pgfqpoint{2.749811in}{6.373396in}}%
\pgfpathlineto{\pgfqpoint{2.837547in}{6.373396in}}%
\pgfpathlineto{\pgfqpoint{2.837547in}{6.285661in}}%
\pgfpathlineto{\pgfqpoint{2.749811in}{6.285661in}}%
\pgfpathlineto{\pgfqpoint{2.749811in}{6.373396in}}%
\pgfusepath{stroke,fill}%
\end{pgfscope}%
\begin{pgfscope}%
\pgfpathrectangle{\pgfqpoint{0.380943in}{6.110189in}}{\pgfqpoint{4.650000in}{0.614151in}}%
\pgfusepath{clip}%
\pgfsetbuttcap%
\pgfsetroundjoin%
\definecolor{currentfill}{rgb}{1.000000,1.000000,0.870204}%
\pgfsetfillcolor{currentfill}%
\pgfsetlinewidth{0.250937pt}%
\definecolor{currentstroke}{rgb}{1.000000,1.000000,1.000000}%
\pgfsetstrokecolor{currentstroke}%
\pgfsetdash{}{0pt}%
\pgfpathmoveto{\pgfqpoint{2.837547in}{6.373396in}}%
\pgfpathlineto{\pgfqpoint{2.925283in}{6.373396in}}%
\pgfpathlineto{\pgfqpoint{2.925283in}{6.285661in}}%
\pgfpathlineto{\pgfqpoint{2.837547in}{6.285661in}}%
\pgfpathlineto{\pgfqpoint{2.837547in}{6.373396in}}%
\pgfusepath{stroke,fill}%
\end{pgfscope}%
\begin{pgfscope}%
\pgfpathrectangle{\pgfqpoint{0.380943in}{6.110189in}}{\pgfqpoint{4.650000in}{0.614151in}}%
\pgfusepath{clip}%
\pgfsetbuttcap%
\pgfsetroundjoin%
\definecolor{currentfill}{rgb}{0.972549,0.870588,0.692810}%
\pgfsetfillcolor{currentfill}%
\pgfsetlinewidth{0.250937pt}%
\definecolor{currentstroke}{rgb}{1.000000,1.000000,1.000000}%
\pgfsetstrokecolor{currentstroke}%
\pgfsetdash{}{0pt}%
\pgfpathmoveto{\pgfqpoint{2.925283in}{6.373396in}}%
\pgfpathlineto{\pgfqpoint{3.013019in}{6.373396in}}%
\pgfpathlineto{\pgfqpoint{3.013019in}{6.285661in}}%
\pgfpathlineto{\pgfqpoint{2.925283in}{6.285661in}}%
\pgfpathlineto{\pgfqpoint{2.925283in}{6.373396in}}%
\pgfusepath{stroke,fill}%
\end{pgfscope}%
\begin{pgfscope}%
\pgfpathrectangle{\pgfqpoint{0.380943in}{6.110189in}}{\pgfqpoint{4.650000in}{0.614151in}}%
\pgfusepath{clip}%
\pgfsetbuttcap%
\pgfsetroundjoin%
\definecolor{currentfill}{rgb}{0.962414,0.923552,0.722891}%
\pgfsetfillcolor{currentfill}%
\pgfsetlinewidth{0.250937pt}%
\definecolor{currentstroke}{rgb}{1.000000,1.000000,1.000000}%
\pgfsetstrokecolor{currentstroke}%
\pgfsetdash{}{0pt}%
\pgfpathmoveto{\pgfqpoint{3.013019in}{6.373396in}}%
\pgfpathlineto{\pgfqpoint{3.100754in}{6.373396in}}%
\pgfpathlineto{\pgfqpoint{3.100754in}{6.285661in}}%
\pgfpathlineto{\pgfqpoint{3.013019in}{6.285661in}}%
\pgfpathlineto{\pgfqpoint{3.013019in}{6.373396in}}%
\pgfusepath{stroke,fill}%
\end{pgfscope}%
\begin{pgfscope}%
\pgfpathrectangle{\pgfqpoint{0.380943in}{6.110189in}}{\pgfqpoint{4.650000in}{0.614151in}}%
\pgfusepath{clip}%
\pgfsetbuttcap%
\pgfsetroundjoin%
\definecolor{currentfill}{rgb}{0.962414,0.923552,0.722891}%
\pgfsetfillcolor{currentfill}%
\pgfsetlinewidth{0.250937pt}%
\definecolor{currentstroke}{rgb}{1.000000,1.000000,1.000000}%
\pgfsetstrokecolor{currentstroke}%
\pgfsetdash{}{0pt}%
\pgfpathmoveto{\pgfqpoint{3.100754in}{6.373396in}}%
\pgfpathlineto{\pgfqpoint{3.188490in}{6.373396in}}%
\pgfpathlineto{\pgfqpoint{3.188490in}{6.285661in}}%
\pgfpathlineto{\pgfqpoint{3.100754in}{6.285661in}}%
\pgfpathlineto{\pgfqpoint{3.100754in}{6.373396in}}%
\pgfusepath{stroke,fill}%
\end{pgfscope}%
\begin{pgfscope}%
\pgfpathrectangle{\pgfqpoint{0.380943in}{6.110189in}}{\pgfqpoint{4.650000in}{0.614151in}}%
\pgfusepath{clip}%
\pgfsetbuttcap%
\pgfsetroundjoin%
\definecolor{currentfill}{rgb}{0.986759,0.806398,0.641200}%
\pgfsetfillcolor{currentfill}%
\pgfsetlinewidth{0.250937pt}%
\definecolor{currentstroke}{rgb}{1.000000,1.000000,1.000000}%
\pgfsetstrokecolor{currentstroke}%
\pgfsetdash{}{0pt}%
\pgfpathmoveto{\pgfqpoint{3.188490in}{6.373396in}}%
\pgfpathlineto{\pgfqpoint{3.276226in}{6.373396in}}%
\pgfpathlineto{\pgfqpoint{3.276226in}{6.285661in}}%
\pgfpathlineto{\pgfqpoint{3.188490in}{6.285661in}}%
\pgfpathlineto{\pgfqpoint{3.188490in}{6.373396in}}%
\pgfusepath{stroke,fill}%
\end{pgfscope}%
\begin{pgfscope}%
\pgfpathrectangle{\pgfqpoint{0.380943in}{6.110189in}}{\pgfqpoint{4.650000in}{0.614151in}}%
\pgfusepath{clip}%
\pgfsetbuttcap%
\pgfsetroundjoin%
\definecolor{currentfill}{rgb}{0.962414,0.923552,0.722891}%
\pgfsetfillcolor{currentfill}%
\pgfsetlinewidth{0.250937pt}%
\definecolor{currentstroke}{rgb}{1.000000,1.000000,1.000000}%
\pgfsetstrokecolor{currentstroke}%
\pgfsetdash{}{0pt}%
\pgfpathmoveto{\pgfqpoint{3.276226in}{6.373396in}}%
\pgfpathlineto{\pgfqpoint{3.363962in}{6.373396in}}%
\pgfpathlineto{\pgfqpoint{3.363962in}{6.285661in}}%
\pgfpathlineto{\pgfqpoint{3.276226in}{6.285661in}}%
\pgfpathlineto{\pgfqpoint{3.276226in}{6.373396in}}%
\pgfusepath{stroke,fill}%
\end{pgfscope}%
\begin{pgfscope}%
\pgfpathrectangle{\pgfqpoint{0.380943in}{6.110189in}}{\pgfqpoint{4.650000in}{0.614151in}}%
\pgfusepath{clip}%
\pgfsetbuttcap%
\pgfsetroundjoin%
\definecolor{currentfill}{rgb}{0.979654,0.837186,0.669619}%
\pgfsetfillcolor{currentfill}%
\pgfsetlinewidth{0.250937pt}%
\definecolor{currentstroke}{rgb}{1.000000,1.000000,1.000000}%
\pgfsetstrokecolor{currentstroke}%
\pgfsetdash{}{0pt}%
\pgfpathmoveto{\pgfqpoint{3.363962in}{6.373396in}}%
\pgfpathlineto{\pgfqpoint{3.451698in}{6.373396in}}%
\pgfpathlineto{\pgfqpoint{3.451698in}{6.285661in}}%
\pgfpathlineto{\pgfqpoint{3.363962in}{6.285661in}}%
\pgfpathlineto{\pgfqpoint{3.363962in}{6.373396in}}%
\pgfusepath{stroke,fill}%
\end{pgfscope}%
\begin{pgfscope}%
\pgfpathrectangle{\pgfqpoint{0.380943in}{6.110189in}}{\pgfqpoint{4.650000in}{0.614151in}}%
\pgfusepath{clip}%
\pgfsetbuttcap%
\pgfsetroundjoin%
\definecolor{currentfill}{rgb}{0.962414,0.923552,0.722891}%
\pgfsetfillcolor{currentfill}%
\pgfsetlinewidth{0.250937pt}%
\definecolor{currentstroke}{rgb}{1.000000,1.000000,1.000000}%
\pgfsetstrokecolor{currentstroke}%
\pgfsetdash{}{0pt}%
\pgfpathmoveto{\pgfqpoint{3.451698in}{6.373396in}}%
\pgfpathlineto{\pgfqpoint{3.539434in}{6.373396in}}%
\pgfpathlineto{\pgfqpoint{3.539434in}{6.285661in}}%
\pgfpathlineto{\pgfqpoint{3.451698in}{6.285661in}}%
\pgfpathlineto{\pgfqpoint{3.451698in}{6.373396in}}%
\pgfusepath{stroke,fill}%
\end{pgfscope}%
\begin{pgfscope}%
\pgfpathrectangle{\pgfqpoint{0.380943in}{6.110189in}}{\pgfqpoint{4.650000in}{0.614151in}}%
\pgfusepath{clip}%
\pgfsetbuttcap%
\pgfsetroundjoin%
\definecolor{currentfill}{rgb}{0.992326,0.765229,0.614840}%
\pgfsetfillcolor{currentfill}%
\pgfsetlinewidth{0.250937pt}%
\definecolor{currentstroke}{rgb}{1.000000,1.000000,1.000000}%
\pgfsetstrokecolor{currentstroke}%
\pgfsetdash{}{0pt}%
\pgfpathmoveto{\pgfqpoint{3.539434in}{6.373396in}}%
\pgfpathlineto{\pgfqpoint{3.627169in}{6.373396in}}%
\pgfpathlineto{\pgfqpoint{3.627169in}{6.285661in}}%
\pgfpathlineto{\pgfqpoint{3.539434in}{6.285661in}}%
\pgfpathlineto{\pgfqpoint{3.539434in}{6.373396in}}%
\pgfusepath{stroke,fill}%
\end{pgfscope}%
\begin{pgfscope}%
\pgfpathrectangle{\pgfqpoint{0.380943in}{6.110189in}}{\pgfqpoint{4.650000in}{0.614151in}}%
\pgfusepath{clip}%
\pgfsetbuttcap%
\pgfsetroundjoin%
\definecolor{currentfill}{rgb}{0.972549,0.870588,0.692810}%
\pgfsetfillcolor{currentfill}%
\pgfsetlinewidth{0.250937pt}%
\definecolor{currentstroke}{rgb}{1.000000,1.000000,1.000000}%
\pgfsetstrokecolor{currentstroke}%
\pgfsetdash{}{0pt}%
\pgfpathmoveto{\pgfqpoint{3.627169in}{6.373396in}}%
\pgfpathlineto{\pgfqpoint{3.714905in}{6.373396in}}%
\pgfpathlineto{\pgfqpoint{3.714905in}{6.285661in}}%
\pgfpathlineto{\pgfqpoint{3.627169in}{6.285661in}}%
\pgfpathlineto{\pgfqpoint{3.627169in}{6.373396in}}%
\pgfusepath{stroke,fill}%
\end{pgfscope}%
\begin{pgfscope}%
\pgfpathrectangle{\pgfqpoint{0.380943in}{6.110189in}}{\pgfqpoint{4.650000in}{0.614151in}}%
\pgfusepath{clip}%
\pgfsetbuttcap%
\pgfsetroundjoin%
\definecolor{currentfill}{rgb}{1.000000,0.605229,0.530719}%
\pgfsetfillcolor{currentfill}%
\pgfsetlinewidth{0.250937pt}%
\definecolor{currentstroke}{rgb}{1.000000,1.000000,1.000000}%
\pgfsetstrokecolor{currentstroke}%
\pgfsetdash{}{0pt}%
\pgfpathmoveto{\pgfqpoint{3.714905in}{6.373396in}}%
\pgfpathlineto{\pgfqpoint{3.802641in}{6.373396in}}%
\pgfpathlineto{\pgfqpoint{3.802641in}{6.285661in}}%
\pgfpathlineto{\pgfqpoint{3.714905in}{6.285661in}}%
\pgfpathlineto{\pgfqpoint{3.714905in}{6.373396in}}%
\pgfusepath{stroke,fill}%
\end{pgfscope}%
\begin{pgfscope}%
\pgfpathrectangle{\pgfqpoint{0.380943in}{6.110189in}}{\pgfqpoint{4.650000in}{0.614151in}}%
\pgfusepath{clip}%
\pgfsetbuttcap%
\pgfsetroundjoin%
\definecolor{currentfill}{rgb}{0.998939,0.658962,0.556032}%
\pgfsetfillcolor{currentfill}%
\pgfsetlinewidth{0.250937pt}%
\definecolor{currentstroke}{rgb}{1.000000,1.000000,1.000000}%
\pgfsetstrokecolor{currentstroke}%
\pgfsetdash{}{0pt}%
\pgfpathmoveto{\pgfqpoint{3.802641in}{6.373396in}}%
\pgfpathlineto{\pgfqpoint{3.890377in}{6.373396in}}%
\pgfpathlineto{\pgfqpoint{3.890377in}{6.285661in}}%
\pgfpathlineto{\pgfqpoint{3.802641in}{6.285661in}}%
\pgfpathlineto{\pgfqpoint{3.802641in}{6.373396in}}%
\pgfusepath{stroke,fill}%
\end{pgfscope}%
\begin{pgfscope}%
\pgfpathrectangle{\pgfqpoint{0.380943in}{6.110189in}}{\pgfqpoint{4.650000in}{0.614151in}}%
\pgfusepath{clip}%
\pgfsetbuttcap%
\pgfsetroundjoin%
\definecolor{currentfill}{rgb}{0.992326,0.765229,0.614840}%
\pgfsetfillcolor{currentfill}%
\pgfsetlinewidth{0.250937pt}%
\definecolor{currentstroke}{rgb}{1.000000,1.000000,1.000000}%
\pgfsetstrokecolor{currentstroke}%
\pgfsetdash{}{0pt}%
\pgfpathmoveto{\pgfqpoint{3.890377in}{6.373396in}}%
\pgfpathlineto{\pgfqpoint{3.978113in}{6.373396in}}%
\pgfpathlineto{\pgfqpoint{3.978113in}{6.285661in}}%
\pgfpathlineto{\pgfqpoint{3.890377in}{6.285661in}}%
\pgfpathlineto{\pgfqpoint{3.890377in}{6.373396in}}%
\pgfusepath{stroke,fill}%
\end{pgfscope}%
\begin{pgfscope}%
\pgfpathrectangle{\pgfqpoint{0.380943in}{6.110189in}}{\pgfqpoint{4.650000in}{0.614151in}}%
\pgfusepath{clip}%
\pgfsetbuttcap%
\pgfsetroundjoin%
\definecolor{currentfill}{rgb}{0.996571,0.720538,0.589189}%
\pgfsetfillcolor{currentfill}%
\pgfsetlinewidth{0.250937pt}%
\definecolor{currentstroke}{rgb}{1.000000,1.000000,1.000000}%
\pgfsetstrokecolor{currentstroke}%
\pgfsetdash{}{0pt}%
\pgfpathmoveto{\pgfqpoint{3.978113in}{6.373396in}}%
\pgfpathlineto{\pgfqpoint{4.065849in}{6.373396in}}%
\pgfpathlineto{\pgfqpoint{4.065849in}{6.285661in}}%
\pgfpathlineto{\pgfqpoint{3.978113in}{6.285661in}}%
\pgfpathlineto{\pgfqpoint{3.978113in}{6.373396in}}%
\pgfusepath{stroke,fill}%
\end{pgfscope}%
\begin{pgfscope}%
\pgfpathrectangle{\pgfqpoint{0.380943in}{6.110189in}}{\pgfqpoint{4.650000in}{0.614151in}}%
\pgfusepath{clip}%
\pgfsetbuttcap%
\pgfsetroundjoin%
\definecolor{currentfill}{rgb}{0.996571,0.720538,0.589189}%
\pgfsetfillcolor{currentfill}%
\pgfsetlinewidth{0.250937pt}%
\definecolor{currentstroke}{rgb}{1.000000,1.000000,1.000000}%
\pgfsetstrokecolor{currentstroke}%
\pgfsetdash{}{0pt}%
\pgfpathmoveto{\pgfqpoint{4.065849in}{6.373396in}}%
\pgfpathlineto{\pgfqpoint{4.153585in}{6.373396in}}%
\pgfpathlineto{\pgfqpoint{4.153585in}{6.285661in}}%
\pgfpathlineto{\pgfqpoint{4.065849in}{6.285661in}}%
\pgfpathlineto{\pgfqpoint{4.065849in}{6.373396in}}%
\pgfusepath{stroke,fill}%
\end{pgfscope}%
\begin{pgfscope}%
\pgfpathrectangle{\pgfqpoint{0.380943in}{6.110189in}}{\pgfqpoint{4.650000in}{0.614151in}}%
\pgfusepath{clip}%
\pgfsetbuttcap%
\pgfsetroundjoin%
\definecolor{currentfill}{rgb}{0.979654,0.837186,0.669619}%
\pgfsetfillcolor{currentfill}%
\pgfsetlinewidth{0.250937pt}%
\definecolor{currentstroke}{rgb}{1.000000,1.000000,1.000000}%
\pgfsetstrokecolor{currentstroke}%
\pgfsetdash{}{0pt}%
\pgfpathmoveto{\pgfqpoint{4.153585in}{6.373396in}}%
\pgfpathlineto{\pgfqpoint{4.241320in}{6.373396in}}%
\pgfpathlineto{\pgfqpoint{4.241320in}{6.285661in}}%
\pgfpathlineto{\pgfqpoint{4.153585in}{6.285661in}}%
\pgfpathlineto{\pgfqpoint{4.153585in}{6.373396in}}%
\pgfusepath{stroke,fill}%
\end{pgfscope}%
\begin{pgfscope}%
\pgfpathrectangle{\pgfqpoint{0.380943in}{6.110189in}}{\pgfqpoint{4.650000in}{0.614151in}}%
\pgfusepath{clip}%
\pgfsetbuttcap%
\pgfsetroundjoin%
\definecolor{currentfill}{rgb}{0.986759,0.806398,0.641200}%
\pgfsetfillcolor{currentfill}%
\pgfsetlinewidth{0.250937pt}%
\definecolor{currentstroke}{rgb}{1.000000,1.000000,1.000000}%
\pgfsetstrokecolor{currentstroke}%
\pgfsetdash{}{0pt}%
\pgfpathmoveto{\pgfqpoint{4.241320in}{6.373396in}}%
\pgfpathlineto{\pgfqpoint{4.329056in}{6.373396in}}%
\pgfpathlineto{\pgfqpoint{4.329056in}{6.285661in}}%
\pgfpathlineto{\pgfqpoint{4.241320in}{6.285661in}}%
\pgfpathlineto{\pgfqpoint{4.241320in}{6.373396in}}%
\pgfusepath{stroke,fill}%
\end{pgfscope}%
\begin{pgfscope}%
\pgfpathrectangle{\pgfqpoint{0.380943in}{6.110189in}}{\pgfqpoint{4.650000in}{0.614151in}}%
\pgfusepath{clip}%
\pgfsetbuttcap%
\pgfsetroundjoin%
\definecolor{currentfill}{rgb}{0.968166,0.945882,0.748604}%
\pgfsetfillcolor{currentfill}%
\pgfsetlinewidth{0.250937pt}%
\definecolor{currentstroke}{rgb}{1.000000,1.000000,1.000000}%
\pgfsetstrokecolor{currentstroke}%
\pgfsetdash{}{0pt}%
\pgfpathmoveto{\pgfqpoint{4.329056in}{6.373396in}}%
\pgfpathlineto{\pgfqpoint{4.416792in}{6.373396in}}%
\pgfpathlineto{\pgfqpoint{4.416792in}{6.285661in}}%
\pgfpathlineto{\pgfqpoint{4.329056in}{6.285661in}}%
\pgfpathlineto{\pgfqpoint{4.329056in}{6.373396in}}%
\pgfusepath{stroke,fill}%
\end{pgfscope}%
\begin{pgfscope}%
\pgfpathrectangle{\pgfqpoint{0.380943in}{6.110189in}}{\pgfqpoint{4.650000in}{0.614151in}}%
\pgfusepath{clip}%
\pgfsetbuttcap%
\pgfsetroundjoin%
\definecolor{currentfill}{rgb}{0.968166,0.945882,0.748604}%
\pgfsetfillcolor{currentfill}%
\pgfsetlinewidth{0.250937pt}%
\definecolor{currentstroke}{rgb}{1.000000,1.000000,1.000000}%
\pgfsetstrokecolor{currentstroke}%
\pgfsetdash{}{0pt}%
\pgfpathmoveto{\pgfqpoint{4.416792in}{6.373396in}}%
\pgfpathlineto{\pgfqpoint{4.504528in}{6.373396in}}%
\pgfpathlineto{\pgfqpoint{4.504528in}{6.285661in}}%
\pgfpathlineto{\pgfqpoint{4.416792in}{6.285661in}}%
\pgfpathlineto{\pgfqpoint{4.416792in}{6.373396in}}%
\pgfusepath{stroke,fill}%
\end{pgfscope}%
\begin{pgfscope}%
\pgfpathrectangle{\pgfqpoint{0.380943in}{6.110189in}}{\pgfqpoint{4.650000in}{0.614151in}}%
\pgfusepath{clip}%
\pgfsetbuttcap%
\pgfsetroundjoin%
\definecolor{currentfill}{rgb}{0.986759,0.806398,0.641200}%
\pgfsetfillcolor{currentfill}%
\pgfsetlinewidth{0.250937pt}%
\definecolor{currentstroke}{rgb}{1.000000,1.000000,1.000000}%
\pgfsetstrokecolor{currentstroke}%
\pgfsetdash{}{0pt}%
\pgfpathmoveto{\pgfqpoint{4.504528in}{6.373396in}}%
\pgfpathlineto{\pgfqpoint{4.592264in}{6.373396in}}%
\pgfpathlineto{\pgfqpoint{4.592264in}{6.285661in}}%
\pgfpathlineto{\pgfqpoint{4.504528in}{6.285661in}}%
\pgfpathlineto{\pgfqpoint{4.504528in}{6.373396in}}%
\pgfusepath{stroke,fill}%
\end{pgfscope}%
\begin{pgfscope}%
\pgfpathrectangle{\pgfqpoint{0.380943in}{6.110189in}}{\pgfqpoint{4.650000in}{0.614151in}}%
\pgfusepath{clip}%
\pgfsetbuttcap%
\pgfsetroundjoin%
\definecolor{currentfill}{rgb}{0.972549,0.870588,0.692810}%
\pgfsetfillcolor{currentfill}%
\pgfsetlinewidth{0.250937pt}%
\definecolor{currentstroke}{rgb}{1.000000,1.000000,1.000000}%
\pgfsetstrokecolor{currentstroke}%
\pgfsetdash{}{0pt}%
\pgfpathmoveto{\pgfqpoint{4.592264in}{6.373396in}}%
\pgfpathlineto{\pgfqpoint{4.680000in}{6.373396in}}%
\pgfpathlineto{\pgfqpoint{4.680000in}{6.285661in}}%
\pgfpathlineto{\pgfqpoint{4.592264in}{6.285661in}}%
\pgfpathlineto{\pgfqpoint{4.592264in}{6.373396in}}%
\pgfusepath{stroke,fill}%
\end{pgfscope}%
\begin{pgfscope}%
\pgfpathrectangle{\pgfqpoint{0.380943in}{6.110189in}}{\pgfqpoint{4.650000in}{0.614151in}}%
\pgfusepath{clip}%
\pgfsetbuttcap%
\pgfsetroundjoin%
\definecolor{currentfill}{rgb}{0.965444,0.906113,0.711757}%
\pgfsetfillcolor{currentfill}%
\pgfsetlinewidth{0.250937pt}%
\definecolor{currentstroke}{rgb}{1.000000,1.000000,1.000000}%
\pgfsetstrokecolor{currentstroke}%
\pgfsetdash{}{0pt}%
\pgfpathmoveto{\pgfqpoint{4.680000in}{6.373396in}}%
\pgfpathlineto{\pgfqpoint{4.767736in}{6.373396in}}%
\pgfpathlineto{\pgfqpoint{4.767736in}{6.285661in}}%
\pgfpathlineto{\pgfqpoint{4.680000in}{6.285661in}}%
\pgfpathlineto{\pgfqpoint{4.680000in}{6.373396in}}%
\pgfusepath{stroke,fill}%
\end{pgfscope}%
\begin{pgfscope}%
\pgfpathrectangle{\pgfqpoint{0.380943in}{6.110189in}}{\pgfqpoint{4.650000in}{0.614151in}}%
\pgfusepath{clip}%
\pgfsetbuttcap%
\pgfsetroundjoin%
\definecolor{currentfill}{rgb}{0.986759,0.806398,0.641200}%
\pgfsetfillcolor{currentfill}%
\pgfsetlinewidth{0.250937pt}%
\definecolor{currentstroke}{rgb}{1.000000,1.000000,1.000000}%
\pgfsetstrokecolor{currentstroke}%
\pgfsetdash{}{0pt}%
\pgfpathmoveto{\pgfqpoint{4.767736in}{6.373396in}}%
\pgfpathlineto{\pgfqpoint{4.855471in}{6.373396in}}%
\pgfpathlineto{\pgfqpoint{4.855471in}{6.285661in}}%
\pgfpathlineto{\pgfqpoint{4.767736in}{6.285661in}}%
\pgfpathlineto{\pgfqpoint{4.767736in}{6.373396in}}%
\pgfusepath{stroke,fill}%
\end{pgfscope}%
\begin{pgfscope}%
\pgfpathrectangle{\pgfqpoint{0.380943in}{6.110189in}}{\pgfqpoint{4.650000in}{0.614151in}}%
\pgfusepath{clip}%
\pgfsetbuttcap%
\pgfsetroundjoin%
\definecolor{currentfill}{rgb}{0.962414,0.923552,0.722891}%
\pgfsetfillcolor{currentfill}%
\pgfsetlinewidth{0.250937pt}%
\definecolor{currentstroke}{rgb}{1.000000,1.000000,1.000000}%
\pgfsetstrokecolor{currentstroke}%
\pgfsetdash{}{0pt}%
\pgfpathmoveto{\pgfqpoint{4.855471in}{6.373396in}}%
\pgfpathlineto{\pgfqpoint{4.943207in}{6.373396in}}%
\pgfpathlineto{\pgfqpoint{4.943207in}{6.285661in}}%
\pgfpathlineto{\pgfqpoint{4.855471in}{6.285661in}}%
\pgfpathlineto{\pgfqpoint{4.855471in}{6.373396in}}%
\pgfusepath{stroke,fill}%
\end{pgfscope}%
\begin{pgfscope}%
\pgfpathrectangle{\pgfqpoint{0.380943in}{6.110189in}}{\pgfqpoint{4.650000in}{0.614151in}}%
\pgfusepath{clip}%
\pgfsetbuttcap%
\pgfsetroundjoin%
\pgfsetlinewidth{0.250937pt}%
\definecolor{currentstroke}{rgb}{1.000000,1.000000,1.000000}%
\pgfsetstrokecolor{currentstroke}%
\pgfsetdash{}{0pt}%
\pgfpathmoveto{\pgfqpoint{4.943207in}{6.373396in}}%
\pgfpathlineto{\pgfqpoint{5.030943in}{6.373396in}}%
\pgfpathlineto{\pgfqpoint{5.030943in}{6.285661in}}%
\pgfpathlineto{\pgfqpoint{4.943207in}{6.285661in}}%
\pgfpathlineto{\pgfqpoint{4.943207in}{6.373396in}}%
\pgfusepath{stroke}%
\end{pgfscope}%
\begin{pgfscope}%
\pgfpathrectangle{\pgfqpoint{0.380943in}{6.110189in}}{\pgfqpoint{4.650000in}{0.614151in}}%
\pgfusepath{clip}%
\pgfsetbuttcap%
\pgfsetroundjoin%
\definecolor{currentfill}{rgb}{0.972549,0.870588,0.692810}%
\pgfsetfillcolor{currentfill}%
\pgfsetlinewidth{0.250937pt}%
\definecolor{currentstroke}{rgb}{1.000000,1.000000,1.000000}%
\pgfsetstrokecolor{currentstroke}%
\pgfsetdash{}{0pt}%
\pgfpathmoveto{\pgfqpoint{0.380943in}{6.285661in}}%
\pgfpathlineto{\pgfqpoint{0.468679in}{6.285661in}}%
\pgfpathlineto{\pgfqpoint{0.468679in}{6.197925in}}%
\pgfpathlineto{\pgfqpoint{0.380943in}{6.197925in}}%
\pgfpathlineto{\pgfqpoint{0.380943in}{6.285661in}}%
\pgfusepath{stroke,fill}%
\end{pgfscope}%
\begin{pgfscope}%
\pgfpathrectangle{\pgfqpoint{0.380943in}{6.110189in}}{\pgfqpoint{4.650000in}{0.614151in}}%
\pgfusepath{clip}%
\pgfsetbuttcap%
\pgfsetroundjoin%
\definecolor{currentfill}{rgb}{0.965444,0.906113,0.711757}%
\pgfsetfillcolor{currentfill}%
\pgfsetlinewidth{0.250937pt}%
\definecolor{currentstroke}{rgb}{1.000000,1.000000,1.000000}%
\pgfsetstrokecolor{currentstroke}%
\pgfsetdash{}{0pt}%
\pgfpathmoveto{\pgfqpoint{0.468679in}{6.285661in}}%
\pgfpathlineto{\pgfqpoint{0.556415in}{6.285661in}}%
\pgfpathlineto{\pgfqpoint{0.556415in}{6.197925in}}%
\pgfpathlineto{\pgfqpoint{0.468679in}{6.197925in}}%
\pgfpathlineto{\pgfqpoint{0.468679in}{6.285661in}}%
\pgfusepath{stroke,fill}%
\end{pgfscope}%
\begin{pgfscope}%
\pgfpathrectangle{\pgfqpoint{0.380943in}{6.110189in}}{\pgfqpoint{4.650000in}{0.614151in}}%
\pgfusepath{clip}%
\pgfsetbuttcap%
\pgfsetroundjoin%
\definecolor{currentfill}{rgb}{0.991849,0.986144,0.810181}%
\pgfsetfillcolor{currentfill}%
\pgfsetlinewidth{0.250937pt}%
\definecolor{currentstroke}{rgb}{1.000000,1.000000,1.000000}%
\pgfsetstrokecolor{currentstroke}%
\pgfsetdash{}{0pt}%
\pgfpathmoveto{\pgfqpoint{0.556415in}{6.285661in}}%
\pgfpathlineto{\pgfqpoint{0.644151in}{6.285661in}}%
\pgfpathlineto{\pgfqpoint{0.644151in}{6.197925in}}%
\pgfpathlineto{\pgfqpoint{0.556415in}{6.197925in}}%
\pgfpathlineto{\pgfqpoint{0.556415in}{6.285661in}}%
\pgfusepath{stroke,fill}%
\end{pgfscope}%
\begin{pgfscope}%
\pgfpathrectangle{\pgfqpoint{0.380943in}{6.110189in}}{\pgfqpoint{4.650000in}{0.614151in}}%
\pgfusepath{clip}%
\pgfsetbuttcap%
\pgfsetroundjoin%
\definecolor{currentfill}{rgb}{0.965444,0.906113,0.711757}%
\pgfsetfillcolor{currentfill}%
\pgfsetlinewidth{0.250937pt}%
\definecolor{currentstroke}{rgb}{1.000000,1.000000,1.000000}%
\pgfsetstrokecolor{currentstroke}%
\pgfsetdash{}{0pt}%
\pgfpathmoveto{\pgfqpoint{0.644151in}{6.285661in}}%
\pgfpathlineto{\pgfqpoint{0.731886in}{6.285661in}}%
\pgfpathlineto{\pgfqpoint{0.731886in}{6.197925in}}%
\pgfpathlineto{\pgfqpoint{0.644151in}{6.197925in}}%
\pgfpathlineto{\pgfqpoint{0.644151in}{6.285661in}}%
\pgfusepath{stroke,fill}%
\end{pgfscope}%
\begin{pgfscope}%
\pgfpathrectangle{\pgfqpoint{0.380943in}{6.110189in}}{\pgfqpoint{4.650000in}{0.614151in}}%
\pgfusepath{clip}%
\pgfsetbuttcap%
\pgfsetroundjoin%
\definecolor{currentfill}{rgb}{1.000000,1.000000,0.870204}%
\pgfsetfillcolor{currentfill}%
\pgfsetlinewidth{0.250937pt}%
\definecolor{currentstroke}{rgb}{1.000000,1.000000,1.000000}%
\pgfsetstrokecolor{currentstroke}%
\pgfsetdash{}{0pt}%
\pgfpathmoveto{\pgfqpoint{0.731886in}{6.285661in}}%
\pgfpathlineto{\pgfqpoint{0.819622in}{6.285661in}}%
\pgfpathlineto{\pgfqpoint{0.819622in}{6.197925in}}%
\pgfpathlineto{\pgfqpoint{0.731886in}{6.197925in}}%
\pgfpathlineto{\pgfqpoint{0.731886in}{6.285661in}}%
\pgfusepath{stroke,fill}%
\end{pgfscope}%
\begin{pgfscope}%
\pgfpathrectangle{\pgfqpoint{0.380943in}{6.110189in}}{\pgfqpoint{4.650000in}{0.614151in}}%
\pgfusepath{clip}%
\pgfsetbuttcap%
\pgfsetroundjoin%
\definecolor{currentfill}{rgb}{0.968166,0.945882,0.748604}%
\pgfsetfillcolor{currentfill}%
\pgfsetlinewidth{0.250937pt}%
\definecolor{currentstroke}{rgb}{1.000000,1.000000,1.000000}%
\pgfsetstrokecolor{currentstroke}%
\pgfsetdash{}{0pt}%
\pgfpathmoveto{\pgfqpoint{0.819622in}{6.285661in}}%
\pgfpathlineto{\pgfqpoint{0.907358in}{6.285661in}}%
\pgfpathlineto{\pgfqpoint{0.907358in}{6.197925in}}%
\pgfpathlineto{\pgfqpoint{0.819622in}{6.197925in}}%
\pgfpathlineto{\pgfqpoint{0.819622in}{6.285661in}}%
\pgfusepath{stroke,fill}%
\end{pgfscope}%
\begin{pgfscope}%
\pgfpathrectangle{\pgfqpoint{0.380943in}{6.110189in}}{\pgfqpoint{4.650000in}{0.614151in}}%
\pgfusepath{clip}%
\pgfsetbuttcap%
\pgfsetroundjoin%
\definecolor{currentfill}{rgb}{0.965444,0.906113,0.711757}%
\pgfsetfillcolor{currentfill}%
\pgfsetlinewidth{0.250937pt}%
\definecolor{currentstroke}{rgb}{1.000000,1.000000,1.000000}%
\pgfsetstrokecolor{currentstroke}%
\pgfsetdash{}{0pt}%
\pgfpathmoveto{\pgfqpoint{0.907358in}{6.285661in}}%
\pgfpathlineto{\pgfqpoint{0.995094in}{6.285661in}}%
\pgfpathlineto{\pgfqpoint{0.995094in}{6.197925in}}%
\pgfpathlineto{\pgfqpoint{0.907358in}{6.197925in}}%
\pgfpathlineto{\pgfqpoint{0.907358in}{6.285661in}}%
\pgfusepath{stroke,fill}%
\end{pgfscope}%
\begin{pgfscope}%
\pgfpathrectangle{\pgfqpoint{0.380943in}{6.110189in}}{\pgfqpoint{4.650000in}{0.614151in}}%
\pgfusepath{clip}%
\pgfsetbuttcap%
\pgfsetroundjoin%
\definecolor{currentfill}{rgb}{0.965444,0.906113,0.711757}%
\pgfsetfillcolor{currentfill}%
\pgfsetlinewidth{0.250937pt}%
\definecolor{currentstroke}{rgb}{1.000000,1.000000,1.000000}%
\pgfsetstrokecolor{currentstroke}%
\pgfsetdash{}{0pt}%
\pgfpathmoveto{\pgfqpoint{0.995094in}{6.285661in}}%
\pgfpathlineto{\pgfqpoint{1.082830in}{6.285661in}}%
\pgfpathlineto{\pgfqpoint{1.082830in}{6.197925in}}%
\pgfpathlineto{\pgfqpoint{0.995094in}{6.197925in}}%
\pgfpathlineto{\pgfqpoint{0.995094in}{6.285661in}}%
\pgfusepath{stroke,fill}%
\end{pgfscope}%
\begin{pgfscope}%
\pgfpathrectangle{\pgfqpoint{0.380943in}{6.110189in}}{\pgfqpoint{4.650000in}{0.614151in}}%
\pgfusepath{clip}%
\pgfsetbuttcap%
\pgfsetroundjoin%
\definecolor{currentfill}{rgb}{0.965444,0.906113,0.711757}%
\pgfsetfillcolor{currentfill}%
\pgfsetlinewidth{0.250937pt}%
\definecolor{currentstroke}{rgb}{1.000000,1.000000,1.000000}%
\pgfsetstrokecolor{currentstroke}%
\pgfsetdash{}{0pt}%
\pgfpathmoveto{\pgfqpoint{1.082830in}{6.285661in}}%
\pgfpathlineto{\pgfqpoint{1.170566in}{6.285661in}}%
\pgfpathlineto{\pgfqpoint{1.170566in}{6.197925in}}%
\pgfpathlineto{\pgfqpoint{1.082830in}{6.197925in}}%
\pgfpathlineto{\pgfqpoint{1.082830in}{6.285661in}}%
\pgfusepath{stroke,fill}%
\end{pgfscope}%
\begin{pgfscope}%
\pgfpathrectangle{\pgfqpoint{0.380943in}{6.110189in}}{\pgfqpoint{4.650000in}{0.614151in}}%
\pgfusepath{clip}%
\pgfsetbuttcap%
\pgfsetroundjoin%
\definecolor{currentfill}{rgb}{0.968166,0.945882,0.748604}%
\pgfsetfillcolor{currentfill}%
\pgfsetlinewidth{0.250937pt}%
\definecolor{currentstroke}{rgb}{1.000000,1.000000,1.000000}%
\pgfsetstrokecolor{currentstroke}%
\pgfsetdash{}{0pt}%
\pgfpathmoveto{\pgfqpoint{1.170566in}{6.285661in}}%
\pgfpathlineto{\pgfqpoint{1.258302in}{6.285661in}}%
\pgfpathlineto{\pgfqpoint{1.258302in}{6.197925in}}%
\pgfpathlineto{\pgfqpoint{1.170566in}{6.197925in}}%
\pgfpathlineto{\pgfqpoint{1.170566in}{6.285661in}}%
\pgfusepath{stroke,fill}%
\end{pgfscope}%
\begin{pgfscope}%
\pgfpathrectangle{\pgfqpoint{0.380943in}{6.110189in}}{\pgfqpoint{4.650000in}{0.614151in}}%
\pgfusepath{clip}%
\pgfsetbuttcap%
\pgfsetroundjoin%
\definecolor{currentfill}{rgb}{0.991849,0.986144,0.810181}%
\pgfsetfillcolor{currentfill}%
\pgfsetlinewidth{0.250937pt}%
\definecolor{currentstroke}{rgb}{1.000000,1.000000,1.000000}%
\pgfsetstrokecolor{currentstroke}%
\pgfsetdash{}{0pt}%
\pgfpathmoveto{\pgfqpoint{1.258302in}{6.285661in}}%
\pgfpathlineto{\pgfqpoint{1.346037in}{6.285661in}}%
\pgfpathlineto{\pgfqpoint{1.346037in}{6.197925in}}%
\pgfpathlineto{\pgfqpoint{1.258302in}{6.197925in}}%
\pgfpathlineto{\pgfqpoint{1.258302in}{6.285661in}}%
\pgfusepath{stroke,fill}%
\end{pgfscope}%
\begin{pgfscope}%
\pgfpathrectangle{\pgfqpoint{0.380943in}{6.110189in}}{\pgfqpoint{4.650000in}{0.614151in}}%
\pgfusepath{clip}%
\pgfsetbuttcap%
\pgfsetroundjoin%
\definecolor{currentfill}{rgb}{0.972549,0.870588,0.692810}%
\pgfsetfillcolor{currentfill}%
\pgfsetlinewidth{0.250937pt}%
\definecolor{currentstroke}{rgb}{1.000000,1.000000,1.000000}%
\pgfsetstrokecolor{currentstroke}%
\pgfsetdash{}{0pt}%
\pgfpathmoveto{\pgfqpoint{1.346037in}{6.285661in}}%
\pgfpathlineto{\pgfqpoint{1.433773in}{6.285661in}}%
\pgfpathlineto{\pgfqpoint{1.433773in}{6.197925in}}%
\pgfpathlineto{\pgfqpoint{1.346037in}{6.197925in}}%
\pgfpathlineto{\pgfqpoint{1.346037in}{6.285661in}}%
\pgfusepath{stroke,fill}%
\end{pgfscope}%
\begin{pgfscope}%
\pgfpathrectangle{\pgfqpoint{0.380943in}{6.110189in}}{\pgfqpoint{4.650000in}{0.614151in}}%
\pgfusepath{clip}%
\pgfsetbuttcap%
\pgfsetroundjoin%
\definecolor{currentfill}{rgb}{0.965444,0.906113,0.711757}%
\pgfsetfillcolor{currentfill}%
\pgfsetlinewidth{0.250937pt}%
\definecolor{currentstroke}{rgb}{1.000000,1.000000,1.000000}%
\pgfsetstrokecolor{currentstroke}%
\pgfsetdash{}{0pt}%
\pgfpathmoveto{\pgfqpoint{1.433773in}{6.285661in}}%
\pgfpathlineto{\pgfqpoint{1.521509in}{6.285661in}}%
\pgfpathlineto{\pgfqpoint{1.521509in}{6.197925in}}%
\pgfpathlineto{\pgfqpoint{1.433773in}{6.197925in}}%
\pgfpathlineto{\pgfqpoint{1.433773in}{6.285661in}}%
\pgfusepath{stroke,fill}%
\end{pgfscope}%
\begin{pgfscope}%
\pgfpathrectangle{\pgfqpoint{0.380943in}{6.110189in}}{\pgfqpoint{4.650000in}{0.614151in}}%
\pgfusepath{clip}%
\pgfsetbuttcap%
\pgfsetroundjoin%
\definecolor{currentfill}{rgb}{0.991849,0.986144,0.810181}%
\pgfsetfillcolor{currentfill}%
\pgfsetlinewidth{0.250937pt}%
\definecolor{currentstroke}{rgb}{1.000000,1.000000,1.000000}%
\pgfsetstrokecolor{currentstroke}%
\pgfsetdash{}{0pt}%
\pgfpathmoveto{\pgfqpoint{1.521509in}{6.285661in}}%
\pgfpathlineto{\pgfqpoint{1.609245in}{6.285661in}}%
\pgfpathlineto{\pgfqpoint{1.609245in}{6.197925in}}%
\pgfpathlineto{\pgfqpoint{1.521509in}{6.197925in}}%
\pgfpathlineto{\pgfqpoint{1.521509in}{6.285661in}}%
\pgfusepath{stroke,fill}%
\end{pgfscope}%
\begin{pgfscope}%
\pgfpathrectangle{\pgfqpoint{0.380943in}{6.110189in}}{\pgfqpoint{4.650000in}{0.614151in}}%
\pgfusepath{clip}%
\pgfsetbuttcap%
\pgfsetroundjoin%
\definecolor{currentfill}{rgb}{0.965444,0.906113,0.711757}%
\pgfsetfillcolor{currentfill}%
\pgfsetlinewidth{0.250937pt}%
\definecolor{currentstroke}{rgb}{1.000000,1.000000,1.000000}%
\pgfsetstrokecolor{currentstroke}%
\pgfsetdash{}{0pt}%
\pgfpathmoveto{\pgfqpoint{1.609245in}{6.285661in}}%
\pgfpathlineto{\pgfqpoint{1.696981in}{6.285661in}}%
\pgfpathlineto{\pgfqpoint{1.696981in}{6.197925in}}%
\pgfpathlineto{\pgfqpoint{1.609245in}{6.197925in}}%
\pgfpathlineto{\pgfqpoint{1.609245in}{6.285661in}}%
\pgfusepath{stroke,fill}%
\end{pgfscope}%
\begin{pgfscope}%
\pgfpathrectangle{\pgfqpoint{0.380943in}{6.110189in}}{\pgfqpoint{4.650000in}{0.614151in}}%
\pgfusepath{clip}%
\pgfsetbuttcap%
\pgfsetroundjoin%
\definecolor{currentfill}{rgb}{1.000000,1.000000,0.870204}%
\pgfsetfillcolor{currentfill}%
\pgfsetlinewidth{0.250937pt}%
\definecolor{currentstroke}{rgb}{1.000000,1.000000,1.000000}%
\pgfsetstrokecolor{currentstroke}%
\pgfsetdash{}{0pt}%
\pgfpathmoveto{\pgfqpoint{1.696981in}{6.285661in}}%
\pgfpathlineto{\pgfqpoint{1.784717in}{6.285661in}}%
\pgfpathlineto{\pgfqpoint{1.784717in}{6.197925in}}%
\pgfpathlineto{\pgfqpoint{1.696981in}{6.197925in}}%
\pgfpathlineto{\pgfqpoint{1.696981in}{6.285661in}}%
\pgfusepath{stroke,fill}%
\end{pgfscope}%
\begin{pgfscope}%
\pgfpathrectangle{\pgfqpoint{0.380943in}{6.110189in}}{\pgfqpoint{4.650000in}{0.614151in}}%
\pgfusepath{clip}%
\pgfsetbuttcap%
\pgfsetroundjoin%
\definecolor{currentfill}{rgb}{1.000000,1.000000,0.870204}%
\pgfsetfillcolor{currentfill}%
\pgfsetlinewidth{0.250937pt}%
\definecolor{currentstroke}{rgb}{1.000000,1.000000,1.000000}%
\pgfsetstrokecolor{currentstroke}%
\pgfsetdash{}{0pt}%
\pgfpathmoveto{\pgfqpoint{1.784717in}{6.285661in}}%
\pgfpathlineto{\pgfqpoint{1.872452in}{6.285661in}}%
\pgfpathlineto{\pgfqpoint{1.872452in}{6.197925in}}%
\pgfpathlineto{\pgfqpoint{1.784717in}{6.197925in}}%
\pgfpathlineto{\pgfqpoint{1.784717in}{6.285661in}}%
\pgfusepath{stroke,fill}%
\end{pgfscope}%
\begin{pgfscope}%
\pgfpathrectangle{\pgfqpoint{0.380943in}{6.110189in}}{\pgfqpoint{4.650000in}{0.614151in}}%
\pgfusepath{clip}%
\pgfsetbuttcap%
\pgfsetroundjoin%
\definecolor{currentfill}{rgb}{0.968166,0.945882,0.748604}%
\pgfsetfillcolor{currentfill}%
\pgfsetlinewidth{0.250937pt}%
\definecolor{currentstroke}{rgb}{1.000000,1.000000,1.000000}%
\pgfsetstrokecolor{currentstroke}%
\pgfsetdash{}{0pt}%
\pgfpathmoveto{\pgfqpoint{1.872452in}{6.285661in}}%
\pgfpathlineto{\pgfqpoint{1.960188in}{6.285661in}}%
\pgfpathlineto{\pgfqpoint{1.960188in}{6.197925in}}%
\pgfpathlineto{\pgfqpoint{1.872452in}{6.197925in}}%
\pgfpathlineto{\pgfqpoint{1.872452in}{6.285661in}}%
\pgfusepath{stroke,fill}%
\end{pgfscope}%
\begin{pgfscope}%
\pgfpathrectangle{\pgfqpoint{0.380943in}{6.110189in}}{\pgfqpoint{4.650000in}{0.614151in}}%
\pgfusepath{clip}%
\pgfsetbuttcap%
\pgfsetroundjoin%
\definecolor{currentfill}{rgb}{0.962414,0.923552,0.722891}%
\pgfsetfillcolor{currentfill}%
\pgfsetlinewidth{0.250937pt}%
\definecolor{currentstroke}{rgb}{1.000000,1.000000,1.000000}%
\pgfsetstrokecolor{currentstroke}%
\pgfsetdash{}{0pt}%
\pgfpathmoveto{\pgfqpoint{1.960188in}{6.285661in}}%
\pgfpathlineto{\pgfqpoint{2.047924in}{6.285661in}}%
\pgfpathlineto{\pgfqpoint{2.047924in}{6.197925in}}%
\pgfpathlineto{\pgfqpoint{1.960188in}{6.197925in}}%
\pgfpathlineto{\pgfqpoint{1.960188in}{6.285661in}}%
\pgfusepath{stroke,fill}%
\end{pgfscope}%
\begin{pgfscope}%
\pgfpathrectangle{\pgfqpoint{0.380943in}{6.110189in}}{\pgfqpoint{4.650000in}{0.614151in}}%
\pgfusepath{clip}%
\pgfsetbuttcap%
\pgfsetroundjoin%
\definecolor{currentfill}{rgb}{0.991849,0.986144,0.810181}%
\pgfsetfillcolor{currentfill}%
\pgfsetlinewidth{0.250937pt}%
\definecolor{currentstroke}{rgb}{1.000000,1.000000,1.000000}%
\pgfsetstrokecolor{currentstroke}%
\pgfsetdash{}{0pt}%
\pgfpathmoveto{\pgfqpoint{2.047924in}{6.285661in}}%
\pgfpathlineto{\pgfqpoint{2.135660in}{6.285661in}}%
\pgfpathlineto{\pgfqpoint{2.135660in}{6.197925in}}%
\pgfpathlineto{\pgfqpoint{2.047924in}{6.197925in}}%
\pgfpathlineto{\pgfqpoint{2.047924in}{6.285661in}}%
\pgfusepath{stroke,fill}%
\end{pgfscope}%
\begin{pgfscope}%
\pgfpathrectangle{\pgfqpoint{0.380943in}{6.110189in}}{\pgfqpoint{4.650000in}{0.614151in}}%
\pgfusepath{clip}%
\pgfsetbuttcap%
\pgfsetroundjoin%
\definecolor{currentfill}{rgb}{0.991849,0.986144,0.810181}%
\pgfsetfillcolor{currentfill}%
\pgfsetlinewidth{0.250937pt}%
\definecolor{currentstroke}{rgb}{1.000000,1.000000,1.000000}%
\pgfsetstrokecolor{currentstroke}%
\pgfsetdash{}{0pt}%
\pgfpathmoveto{\pgfqpoint{2.135660in}{6.285661in}}%
\pgfpathlineto{\pgfqpoint{2.223396in}{6.285661in}}%
\pgfpathlineto{\pgfqpoint{2.223396in}{6.197925in}}%
\pgfpathlineto{\pgfqpoint{2.135660in}{6.197925in}}%
\pgfpathlineto{\pgfqpoint{2.135660in}{6.285661in}}%
\pgfusepath{stroke,fill}%
\end{pgfscope}%
\begin{pgfscope}%
\pgfpathrectangle{\pgfqpoint{0.380943in}{6.110189in}}{\pgfqpoint{4.650000in}{0.614151in}}%
\pgfusepath{clip}%
\pgfsetbuttcap%
\pgfsetroundjoin%
\definecolor{currentfill}{rgb}{0.962414,0.923552,0.722891}%
\pgfsetfillcolor{currentfill}%
\pgfsetlinewidth{0.250937pt}%
\definecolor{currentstroke}{rgb}{1.000000,1.000000,1.000000}%
\pgfsetstrokecolor{currentstroke}%
\pgfsetdash{}{0pt}%
\pgfpathmoveto{\pgfqpoint{2.223396in}{6.285661in}}%
\pgfpathlineto{\pgfqpoint{2.311132in}{6.285661in}}%
\pgfpathlineto{\pgfqpoint{2.311132in}{6.197925in}}%
\pgfpathlineto{\pgfqpoint{2.223396in}{6.197925in}}%
\pgfpathlineto{\pgfqpoint{2.223396in}{6.285661in}}%
\pgfusepath{stroke,fill}%
\end{pgfscope}%
\begin{pgfscope}%
\pgfpathrectangle{\pgfqpoint{0.380943in}{6.110189in}}{\pgfqpoint{4.650000in}{0.614151in}}%
\pgfusepath{clip}%
\pgfsetbuttcap%
\pgfsetroundjoin%
\definecolor{currentfill}{rgb}{0.979654,0.837186,0.669619}%
\pgfsetfillcolor{currentfill}%
\pgfsetlinewidth{0.250937pt}%
\definecolor{currentstroke}{rgb}{1.000000,1.000000,1.000000}%
\pgfsetstrokecolor{currentstroke}%
\pgfsetdash{}{0pt}%
\pgfpathmoveto{\pgfqpoint{2.311132in}{6.285661in}}%
\pgfpathlineto{\pgfqpoint{2.398868in}{6.285661in}}%
\pgfpathlineto{\pgfqpoint{2.398868in}{6.197925in}}%
\pgfpathlineto{\pgfqpoint{2.311132in}{6.197925in}}%
\pgfpathlineto{\pgfqpoint{2.311132in}{6.285661in}}%
\pgfusepath{stroke,fill}%
\end{pgfscope}%
\begin{pgfscope}%
\pgfpathrectangle{\pgfqpoint{0.380943in}{6.110189in}}{\pgfqpoint{4.650000in}{0.614151in}}%
\pgfusepath{clip}%
\pgfsetbuttcap%
\pgfsetroundjoin%
\definecolor{currentfill}{rgb}{0.968166,0.945882,0.748604}%
\pgfsetfillcolor{currentfill}%
\pgfsetlinewidth{0.250937pt}%
\definecolor{currentstroke}{rgb}{1.000000,1.000000,1.000000}%
\pgfsetstrokecolor{currentstroke}%
\pgfsetdash{}{0pt}%
\pgfpathmoveto{\pgfqpoint{2.398868in}{6.285661in}}%
\pgfpathlineto{\pgfqpoint{2.486603in}{6.285661in}}%
\pgfpathlineto{\pgfqpoint{2.486603in}{6.197925in}}%
\pgfpathlineto{\pgfqpoint{2.398868in}{6.197925in}}%
\pgfpathlineto{\pgfqpoint{2.398868in}{6.285661in}}%
\pgfusepath{stroke,fill}%
\end{pgfscope}%
\begin{pgfscope}%
\pgfpathrectangle{\pgfqpoint{0.380943in}{6.110189in}}{\pgfqpoint{4.650000in}{0.614151in}}%
\pgfusepath{clip}%
\pgfsetbuttcap%
\pgfsetroundjoin%
\definecolor{currentfill}{rgb}{1.000000,1.000000,0.870204}%
\pgfsetfillcolor{currentfill}%
\pgfsetlinewidth{0.250937pt}%
\definecolor{currentstroke}{rgb}{1.000000,1.000000,1.000000}%
\pgfsetstrokecolor{currentstroke}%
\pgfsetdash{}{0pt}%
\pgfpathmoveto{\pgfqpoint{2.486603in}{6.285661in}}%
\pgfpathlineto{\pgfqpoint{2.574339in}{6.285661in}}%
\pgfpathlineto{\pgfqpoint{2.574339in}{6.197925in}}%
\pgfpathlineto{\pgfqpoint{2.486603in}{6.197925in}}%
\pgfpathlineto{\pgfqpoint{2.486603in}{6.285661in}}%
\pgfusepath{stroke,fill}%
\end{pgfscope}%
\begin{pgfscope}%
\pgfpathrectangle{\pgfqpoint{0.380943in}{6.110189in}}{\pgfqpoint{4.650000in}{0.614151in}}%
\pgfusepath{clip}%
\pgfsetbuttcap%
\pgfsetroundjoin%
\definecolor{currentfill}{rgb}{0.991849,0.986144,0.810181}%
\pgfsetfillcolor{currentfill}%
\pgfsetlinewidth{0.250937pt}%
\definecolor{currentstroke}{rgb}{1.000000,1.000000,1.000000}%
\pgfsetstrokecolor{currentstroke}%
\pgfsetdash{}{0pt}%
\pgfpathmoveto{\pgfqpoint{2.574339in}{6.285661in}}%
\pgfpathlineto{\pgfqpoint{2.662075in}{6.285661in}}%
\pgfpathlineto{\pgfqpoint{2.662075in}{6.197925in}}%
\pgfpathlineto{\pgfqpoint{2.574339in}{6.197925in}}%
\pgfpathlineto{\pgfqpoint{2.574339in}{6.285661in}}%
\pgfusepath{stroke,fill}%
\end{pgfscope}%
\begin{pgfscope}%
\pgfpathrectangle{\pgfqpoint{0.380943in}{6.110189in}}{\pgfqpoint{4.650000in}{0.614151in}}%
\pgfusepath{clip}%
\pgfsetbuttcap%
\pgfsetroundjoin%
\definecolor{currentfill}{rgb}{0.991849,0.986144,0.810181}%
\pgfsetfillcolor{currentfill}%
\pgfsetlinewidth{0.250937pt}%
\definecolor{currentstroke}{rgb}{1.000000,1.000000,1.000000}%
\pgfsetstrokecolor{currentstroke}%
\pgfsetdash{}{0pt}%
\pgfpathmoveto{\pgfqpoint{2.662075in}{6.285661in}}%
\pgfpathlineto{\pgfqpoint{2.749811in}{6.285661in}}%
\pgfpathlineto{\pgfqpoint{2.749811in}{6.197925in}}%
\pgfpathlineto{\pgfqpoint{2.662075in}{6.197925in}}%
\pgfpathlineto{\pgfqpoint{2.662075in}{6.285661in}}%
\pgfusepath{stroke,fill}%
\end{pgfscope}%
\begin{pgfscope}%
\pgfpathrectangle{\pgfqpoint{0.380943in}{6.110189in}}{\pgfqpoint{4.650000in}{0.614151in}}%
\pgfusepath{clip}%
\pgfsetbuttcap%
\pgfsetroundjoin%
\definecolor{currentfill}{rgb}{1.000000,1.000000,0.929412}%
\pgfsetfillcolor{currentfill}%
\pgfsetlinewidth{0.250937pt}%
\definecolor{currentstroke}{rgb}{1.000000,1.000000,1.000000}%
\pgfsetstrokecolor{currentstroke}%
\pgfsetdash{}{0pt}%
\pgfpathmoveto{\pgfqpoint{2.749811in}{6.285661in}}%
\pgfpathlineto{\pgfqpoint{2.837547in}{6.285661in}}%
\pgfpathlineto{\pgfqpoint{2.837547in}{6.197925in}}%
\pgfpathlineto{\pgfqpoint{2.749811in}{6.197925in}}%
\pgfpathlineto{\pgfqpoint{2.749811in}{6.285661in}}%
\pgfusepath{stroke,fill}%
\end{pgfscope}%
\begin{pgfscope}%
\pgfpathrectangle{\pgfqpoint{0.380943in}{6.110189in}}{\pgfqpoint{4.650000in}{0.614151in}}%
\pgfusepath{clip}%
\pgfsetbuttcap%
\pgfsetroundjoin%
\definecolor{currentfill}{rgb}{0.968166,0.945882,0.748604}%
\pgfsetfillcolor{currentfill}%
\pgfsetlinewidth{0.250937pt}%
\definecolor{currentstroke}{rgb}{1.000000,1.000000,1.000000}%
\pgfsetstrokecolor{currentstroke}%
\pgfsetdash{}{0pt}%
\pgfpathmoveto{\pgfqpoint{2.837547in}{6.285661in}}%
\pgfpathlineto{\pgfqpoint{2.925283in}{6.285661in}}%
\pgfpathlineto{\pgfqpoint{2.925283in}{6.197925in}}%
\pgfpathlineto{\pgfqpoint{2.837547in}{6.197925in}}%
\pgfpathlineto{\pgfqpoint{2.837547in}{6.285661in}}%
\pgfusepath{stroke,fill}%
\end{pgfscope}%
\begin{pgfscope}%
\pgfpathrectangle{\pgfqpoint{0.380943in}{6.110189in}}{\pgfqpoint{4.650000in}{0.614151in}}%
\pgfusepath{clip}%
\pgfsetbuttcap%
\pgfsetroundjoin%
\definecolor{currentfill}{rgb}{1.000000,1.000000,0.870204}%
\pgfsetfillcolor{currentfill}%
\pgfsetlinewidth{0.250937pt}%
\definecolor{currentstroke}{rgb}{1.000000,1.000000,1.000000}%
\pgfsetstrokecolor{currentstroke}%
\pgfsetdash{}{0pt}%
\pgfpathmoveto{\pgfqpoint{2.925283in}{6.285661in}}%
\pgfpathlineto{\pgfqpoint{3.013019in}{6.285661in}}%
\pgfpathlineto{\pgfqpoint{3.013019in}{6.197925in}}%
\pgfpathlineto{\pgfqpoint{2.925283in}{6.197925in}}%
\pgfpathlineto{\pgfqpoint{2.925283in}{6.285661in}}%
\pgfusepath{stroke,fill}%
\end{pgfscope}%
\begin{pgfscope}%
\pgfpathrectangle{\pgfqpoint{0.380943in}{6.110189in}}{\pgfqpoint{4.650000in}{0.614151in}}%
\pgfusepath{clip}%
\pgfsetbuttcap%
\pgfsetroundjoin%
\definecolor{currentfill}{rgb}{0.968166,0.945882,0.748604}%
\pgfsetfillcolor{currentfill}%
\pgfsetlinewidth{0.250937pt}%
\definecolor{currentstroke}{rgb}{1.000000,1.000000,1.000000}%
\pgfsetstrokecolor{currentstroke}%
\pgfsetdash{}{0pt}%
\pgfpathmoveto{\pgfqpoint{3.013019in}{6.285661in}}%
\pgfpathlineto{\pgfqpoint{3.100754in}{6.285661in}}%
\pgfpathlineto{\pgfqpoint{3.100754in}{6.197925in}}%
\pgfpathlineto{\pgfqpoint{3.013019in}{6.197925in}}%
\pgfpathlineto{\pgfqpoint{3.013019in}{6.285661in}}%
\pgfusepath{stroke,fill}%
\end{pgfscope}%
\begin{pgfscope}%
\pgfpathrectangle{\pgfqpoint{0.380943in}{6.110189in}}{\pgfqpoint{4.650000in}{0.614151in}}%
\pgfusepath{clip}%
\pgfsetbuttcap%
\pgfsetroundjoin%
\definecolor{currentfill}{rgb}{1.000000,1.000000,0.929412}%
\pgfsetfillcolor{currentfill}%
\pgfsetlinewidth{0.250937pt}%
\definecolor{currentstroke}{rgb}{1.000000,1.000000,1.000000}%
\pgfsetstrokecolor{currentstroke}%
\pgfsetdash{}{0pt}%
\pgfpathmoveto{\pgfqpoint{3.100754in}{6.285661in}}%
\pgfpathlineto{\pgfqpoint{3.188490in}{6.285661in}}%
\pgfpathlineto{\pgfqpoint{3.188490in}{6.197925in}}%
\pgfpathlineto{\pgfqpoint{3.100754in}{6.197925in}}%
\pgfpathlineto{\pgfqpoint{3.100754in}{6.285661in}}%
\pgfusepath{stroke,fill}%
\end{pgfscope}%
\begin{pgfscope}%
\pgfpathrectangle{\pgfqpoint{0.380943in}{6.110189in}}{\pgfqpoint{4.650000in}{0.614151in}}%
\pgfusepath{clip}%
\pgfsetbuttcap%
\pgfsetroundjoin%
\definecolor{currentfill}{rgb}{1.000000,1.000000,0.870204}%
\pgfsetfillcolor{currentfill}%
\pgfsetlinewidth{0.250937pt}%
\definecolor{currentstroke}{rgb}{1.000000,1.000000,1.000000}%
\pgfsetstrokecolor{currentstroke}%
\pgfsetdash{}{0pt}%
\pgfpathmoveto{\pgfqpoint{3.188490in}{6.285661in}}%
\pgfpathlineto{\pgfqpoint{3.276226in}{6.285661in}}%
\pgfpathlineto{\pgfqpoint{3.276226in}{6.197925in}}%
\pgfpathlineto{\pgfqpoint{3.188490in}{6.197925in}}%
\pgfpathlineto{\pgfqpoint{3.188490in}{6.285661in}}%
\pgfusepath{stroke,fill}%
\end{pgfscope}%
\begin{pgfscope}%
\pgfpathrectangle{\pgfqpoint{0.380943in}{6.110189in}}{\pgfqpoint{4.650000in}{0.614151in}}%
\pgfusepath{clip}%
\pgfsetbuttcap%
\pgfsetroundjoin%
\definecolor{currentfill}{rgb}{0.991849,0.986144,0.810181}%
\pgfsetfillcolor{currentfill}%
\pgfsetlinewidth{0.250937pt}%
\definecolor{currentstroke}{rgb}{1.000000,1.000000,1.000000}%
\pgfsetstrokecolor{currentstroke}%
\pgfsetdash{}{0pt}%
\pgfpathmoveto{\pgfqpoint{3.276226in}{6.285661in}}%
\pgfpathlineto{\pgfqpoint{3.363962in}{6.285661in}}%
\pgfpathlineto{\pgfqpoint{3.363962in}{6.197925in}}%
\pgfpathlineto{\pgfqpoint{3.276226in}{6.197925in}}%
\pgfpathlineto{\pgfqpoint{3.276226in}{6.285661in}}%
\pgfusepath{stroke,fill}%
\end{pgfscope}%
\begin{pgfscope}%
\pgfpathrectangle{\pgfqpoint{0.380943in}{6.110189in}}{\pgfqpoint{4.650000in}{0.614151in}}%
\pgfusepath{clip}%
\pgfsetbuttcap%
\pgfsetroundjoin%
\definecolor{currentfill}{rgb}{0.972549,0.870588,0.692810}%
\pgfsetfillcolor{currentfill}%
\pgfsetlinewidth{0.250937pt}%
\definecolor{currentstroke}{rgb}{1.000000,1.000000,1.000000}%
\pgfsetstrokecolor{currentstroke}%
\pgfsetdash{}{0pt}%
\pgfpathmoveto{\pgfqpoint{3.363962in}{6.285661in}}%
\pgfpathlineto{\pgfqpoint{3.451698in}{6.285661in}}%
\pgfpathlineto{\pgfqpoint{3.451698in}{6.197925in}}%
\pgfpathlineto{\pgfqpoint{3.363962in}{6.197925in}}%
\pgfpathlineto{\pgfqpoint{3.363962in}{6.285661in}}%
\pgfusepath{stroke,fill}%
\end{pgfscope}%
\begin{pgfscope}%
\pgfpathrectangle{\pgfqpoint{0.380943in}{6.110189in}}{\pgfqpoint{4.650000in}{0.614151in}}%
\pgfusepath{clip}%
\pgfsetbuttcap%
\pgfsetroundjoin%
\definecolor{currentfill}{rgb}{0.962414,0.923552,0.722891}%
\pgfsetfillcolor{currentfill}%
\pgfsetlinewidth{0.250937pt}%
\definecolor{currentstroke}{rgb}{1.000000,1.000000,1.000000}%
\pgfsetstrokecolor{currentstroke}%
\pgfsetdash{}{0pt}%
\pgfpathmoveto{\pgfqpoint{3.451698in}{6.285661in}}%
\pgfpathlineto{\pgfqpoint{3.539434in}{6.285661in}}%
\pgfpathlineto{\pgfqpoint{3.539434in}{6.197925in}}%
\pgfpathlineto{\pgfqpoint{3.451698in}{6.197925in}}%
\pgfpathlineto{\pgfqpoint{3.451698in}{6.285661in}}%
\pgfusepath{stroke,fill}%
\end{pgfscope}%
\begin{pgfscope}%
\pgfpathrectangle{\pgfqpoint{0.380943in}{6.110189in}}{\pgfqpoint{4.650000in}{0.614151in}}%
\pgfusepath{clip}%
\pgfsetbuttcap%
\pgfsetroundjoin%
\definecolor{currentfill}{rgb}{0.991849,0.986144,0.810181}%
\pgfsetfillcolor{currentfill}%
\pgfsetlinewidth{0.250937pt}%
\definecolor{currentstroke}{rgb}{1.000000,1.000000,1.000000}%
\pgfsetstrokecolor{currentstroke}%
\pgfsetdash{}{0pt}%
\pgfpathmoveto{\pgfqpoint{3.539434in}{6.285661in}}%
\pgfpathlineto{\pgfqpoint{3.627169in}{6.285661in}}%
\pgfpathlineto{\pgfqpoint{3.627169in}{6.197925in}}%
\pgfpathlineto{\pgfqpoint{3.539434in}{6.197925in}}%
\pgfpathlineto{\pgfqpoint{3.539434in}{6.285661in}}%
\pgfusepath{stroke,fill}%
\end{pgfscope}%
\begin{pgfscope}%
\pgfpathrectangle{\pgfqpoint{0.380943in}{6.110189in}}{\pgfqpoint{4.650000in}{0.614151in}}%
\pgfusepath{clip}%
\pgfsetbuttcap%
\pgfsetroundjoin%
\definecolor{currentfill}{rgb}{0.991849,0.986144,0.810181}%
\pgfsetfillcolor{currentfill}%
\pgfsetlinewidth{0.250937pt}%
\definecolor{currentstroke}{rgb}{1.000000,1.000000,1.000000}%
\pgfsetstrokecolor{currentstroke}%
\pgfsetdash{}{0pt}%
\pgfpathmoveto{\pgfqpoint{3.627169in}{6.285661in}}%
\pgfpathlineto{\pgfqpoint{3.714905in}{6.285661in}}%
\pgfpathlineto{\pgfqpoint{3.714905in}{6.197925in}}%
\pgfpathlineto{\pgfqpoint{3.627169in}{6.197925in}}%
\pgfpathlineto{\pgfqpoint{3.627169in}{6.285661in}}%
\pgfusepath{stroke,fill}%
\end{pgfscope}%
\begin{pgfscope}%
\pgfpathrectangle{\pgfqpoint{0.380943in}{6.110189in}}{\pgfqpoint{4.650000in}{0.614151in}}%
\pgfusepath{clip}%
\pgfsetbuttcap%
\pgfsetroundjoin%
\definecolor{currentfill}{rgb}{0.991849,0.986144,0.810181}%
\pgfsetfillcolor{currentfill}%
\pgfsetlinewidth{0.250937pt}%
\definecolor{currentstroke}{rgb}{1.000000,1.000000,1.000000}%
\pgfsetstrokecolor{currentstroke}%
\pgfsetdash{}{0pt}%
\pgfpathmoveto{\pgfqpoint{3.714905in}{6.285661in}}%
\pgfpathlineto{\pgfqpoint{3.802641in}{6.285661in}}%
\pgfpathlineto{\pgfqpoint{3.802641in}{6.197925in}}%
\pgfpathlineto{\pgfqpoint{3.714905in}{6.197925in}}%
\pgfpathlineto{\pgfqpoint{3.714905in}{6.285661in}}%
\pgfusepath{stroke,fill}%
\end{pgfscope}%
\begin{pgfscope}%
\pgfpathrectangle{\pgfqpoint{0.380943in}{6.110189in}}{\pgfqpoint{4.650000in}{0.614151in}}%
\pgfusepath{clip}%
\pgfsetbuttcap%
\pgfsetroundjoin%
\definecolor{currentfill}{rgb}{0.991849,0.986144,0.810181}%
\pgfsetfillcolor{currentfill}%
\pgfsetlinewidth{0.250937pt}%
\definecolor{currentstroke}{rgb}{1.000000,1.000000,1.000000}%
\pgfsetstrokecolor{currentstroke}%
\pgfsetdash{}{0pt}%
\pgfpathmoveto{\pgfqpoint{3.802641in}{6.285661in}}%
\pgfpathlineto{\pgfqpoint{3.890377in}{6.285661in}}%
\pgfpathlineto{\pgfqpoint{3.890377in}{6.197925in}}%
\pgfpathlineto{\pgfqpoint{3.802641in}{6.197925in}}%
\pgfpathlineto{\pgfqpoint{3.802641in}{6.285661in}}%
\pgfusepath{stroke,fill}%
\end{pgfscope}%
\begin{pgfscope}%
\pgfpathrectangle{\pgfqpoint{0.380943in}{6.110189in}}{\pgfqpoint{4.650000in}{0.614151in}}%
\pgfusepath{clip}%
\pgfsetbuttcap%
\pgfsetroundjoin%
\definecolor{currentfill}{rgb}{0.962414,0.923552,0.722891}%
\pgfsetfillcolor{currentfill}%
\pgfsetlinewidth{0.250937pt}%
\definecolor{currentstroke}{rgb}{1.000000,1.000000,1.000000}%
\pgfsetstrokecolor{currentstroke}%
\pgfsetdash{}{0pt}%
\pgfpathmoveto{\pgfqpoint{3.890377in}{6.285661in}}%
\pgfpathlineto{\pgfqpoint{3.978113in}{6.285661in}}%
\pgfpathlineto{\pgfqpoint{3.978113in}{6.197925in}}%
\pgfpathlineto{\pgfqpoint{3.890377in}{6.197925in}}%
\pgfpathlineto{\pgfqpoint{3.890377in}{6.285661in}}%
\pgfusepath{stroke,fill}%
\end{pgfscope}%
\begin{pgfscope}%
\pgfpathrectangle{\pgfqpoint{0.380943in}{6.110189in}}{\pgfqpoint{4.650000in}{0.614151in}}%
\pgfusepath{clip}%
\pgfsetbuttcap%
\pgfsetroundjoin%
\definecolor{currentfill}{rgb}{0.979654,0.837186,0.669619}%
\pgfsetfillcolor{currentfill}%
\pgfsetlinewidth{0.250937pt}%
\definecolor{currentstroke}{rgb}{1.000000,1.000000,1.000000}%
\pgfsetstrokecolor{currentstroke}%
\pgfsetdash{}{0pt}%
\pgfpathmoveto{\pgfqpoint{3.978113in}{6.285661in}}%
\pgfpathlineto{\pgfqpoint{4.065849in}{6.285661in}}%
\pgfpathlineto{\pgfqpoint{4.065849in}{6.197925in}}%
\pgfpathlineto{\pgfqpoint{3.978113in}{6.197925in}}%
\pgfpathlineto{\pgfqpoint{3.978113in}{6.285661in}}%
\pgfusepath{stroke,fill}%
\end{pgfscope}%
\begin{pgfscope}%
\pgfpathrectangle{\pgfqpoint{0.380943in}{6.110189in}}{\pgfqpoint{4.650000in}{0.614151in}}%
\pgfusepath{clip}%
\pgfsetbuttcap%
\pgfsetroundjoin%
\definecolor{currentfill}{rgb}{0.998939,0.658962,0.556032}%
\pgfsetfillcolor{currentfill}%
\pgfsetlinewidth{0.250937pt}%
\definecolor{currentstroke}{rgb}{1.000000,1.000000,1.000000}%
\pgfsetstrokecolor{currentstroke}%
\pgfsetdash{}{0pt}%
\pgfpathmoveto{\pgfqpoint{4.065849in}{6.285661in}}%
\pgfpathlineto{\pgfqpoint{4.153585in}{6.285661in}}%
\pgfpathlineto{\pgfqpoint{4.153585in}{6.197925in}}%
\pgfpathlineto{\pgfqpoint{4.065849in}{6.197925in}}%
\pgfpathlineto{\pgfqpoint{4.065849in}{6.285661in}}%
\pgfusepath{stroke,fill}%
\end{pgfscope}%
\begin{pgfscope}%
\pgfpathrectangle{\pgfqpoint{0.380943in}{6.110189in}}{\pgfqpoint{4.650000in}{0.614151in}}%
\pgfusepath{clip}%
\pgfsetbuttcap%
\pgfsetroundjoin%
\definecolor{currentfill}{rgb}{0.972549,0.870588,0.692810}%
\pgfsetfillcolor{currentfill}%
\pgfsetlinewidth{0.250937pt}%
\definecolor{currentstroke}{rgb}{1.000000,1.000000,1.000000}%
\pgfsetstrokecolor{currentstroke}%
\pgfsetdash{}{0pt}%
\pgfpathmoveto{\pgfqpoint{4.153585in}{6.285661in}}%
\pgfpathlineto{\pgfqpoint{4.241320in}{6.285661in}}%
\pgfpathlineto{\pgfqpoint{4.241320in}{6.197925in}}%
\pgfpathlineto{\pgfqpoint{4.153585in}{6.197925in}}%
\pgfpathlineto{\pgfqpoint{4.153585in}{6.285661in}}%
\pgfusepath{stroke,fill}%
\end{pgfscope}%
\begin{pgfscope}%
\pgfpathrectangle{\pgfqpoint{0.380943in}{6.110189in}}{\pgfqpoint{4.650000in}{0.614151in}}%
\pgfusepath{clip}%
\pgfsetbuttcap%
\pgfsetroundjoin%
\definecolor{currentfill}{rgb}{0.965444,0.906113,0.711757}%
\pgfsetfillcolor{currentfill}%
\pgfsetlinewidth{0.250937pt}%
\definecolor{currentstroke}{rgb}{1.000000,1.000000,1.000000}%
\pgfsetstrokecolor{currentstroke}%
\pgfsetdash{}{0pt}%
\pgfpathmoveto{\pgfqpoint{4.241320in}{6.285661in}}%
\pgfpathlineto{\pgfqpoint{4.329056in}{6.285661in}}%
\pgfpathlineto{\pgfqpoint{4.329056in}{6.197925in}}%
\pgfpathlineto{\pgfqpoint{4.241320in}{6.197925in}}%
\pgfpathlineto{\pgfqpoint{4.241320in}{6.285661in}}%
\pgfusepath{stroke,fill}%
\end{pgfscope}%
\begin{pgfscope}%
\pgfpathrectangle{\pgfqpoint{0.380943in}{6.110189in}}{\pgfqpoint{4.650000in}{0.614151in}}%
\pgfusepath{clip}%
\pgfsetbuttcap%
\pgfsetroundjoin%
\definecolor{currentfill}{rgb}{0.968166,0.945882,0.748604}%
\pgfsetfillcolor{currentfill}%
\pgfsetlinewidth{0.250937pt}%
\definecolor{currentstroke}{rgb}{1.000000,1.000000,1.000000}%
\pgfsetstrokecolor{currentstroke}%
\pgfsetdash{}{0pt}%
\pgfpathmoveto{\pgfqpoint{4.329056in}{6.285661in}}%
\pgfpathlineto{\pgfqpoint{4.416792in}{6.285661in}}%
\pgfpathlineto{\pgfqpoint{4.416792in}{6.197925in}}%
\pgfpathlineto{\pgfqpoint{4.329056in}{6.197925in}}%
\pgfpathlineto{\pgfqpoint{4.329056in}{6.285661in}}%
\pgfusepath{stroke,fill}%
\end{pgfscope}%
\begin{pgfscope}%
\pgfpathrectangle{\pgfqpoint{0.380943in}{6.110189in}}{\pgfqpoint{4.650000in}{0.614151in}}%
\pgfusepath{clip}%
\pgfsetbuttcap%
\pgfsetroundjoin%
\definecolor{currentfill}{rgb}{0.968166,0.945882,0.748604}%
\pgfsetfillcolor{currentfill}%
\pgfsetlinewidth{0.250937pt}%
\definecolor{currentstroke}{rgb}{1.000000,1.000000,1.000000}%
\pgfsetstrokecolor{currentstroke}%
\pgfsetdash{}{0pt}%
\pgfpathmoveto{\pgfqpoint{4.416792in}{6.285661in}}%
\pgfpathlineto{\pgfqpoint{4.504528in}{6.285661in}}%
\pgfpathlineto{\pgfqpoint{4.504528in}{6.197925in}}%
\pgfpathlineto{\pgfqpoint{4.416792in}{6.197925in}}%
\pgfpathlineto{\pgfqpoint{4.416792in}{6.285661in}}%
\pgfusepath{stroke,fill}%
\end{pgfscope}%
\begin{pgfscope}%
\pgfpathrectangle{\pgfqpoint{0.380943in}{6.110189in}}{\pgfqpoint{4.650000in}{0.614151in}}%
\pgfusepath{clip}%
\pgfsetbuttcap%
\pgfsetroundjoin%
\definecolor{currentfill}{rgb}{1.000000,1.000000,0.870204}%
\pgfsetfillcolor{currentfill}%
\pgfsetlinewidth{0.250937pt}%
\definecolor{currentstroke}{rgb}{1.000000,1.000000,1.000000}%
\pgfsetstrokecolor{currentstroke}%
\pgfsetdash{}{0pt}%
\pgfpathmoveto{\pgfqpoint{4.504528in}{6.285661in}}%
\pgfpathlineto{\pgfqpoint{4.592264in}{6.285661in}}%
\pgfpathlineto{\pgfqpoint{4.592264in}{6.197925in}}%
\pgfpathlineto{\pgfqpoint{4.504528in}{6.197925in}}%
\pgfpathlineto{\pgfqpoint{4.504528in}{6.285661in}}%
\pgfusepath{stroke,fill}%
\end{pgfscope}%
\begin{pgfscope}%
\pgfpathrectangle{\pgfqpoint{0.380943in}{6.110189in}}{\pgfqpoint{4.650000in}{0.614151in}}%
\pgfusepath{clip}%
\pgfsetbuttcap%
\pgfsetroundjoin%
\definecolor{currentfill}{rgb}{1.000000,1.000000,0.929412}%
\pgfsetfillcolor{currentfill}%
\pgfsetlinewidth{0.250937pt}%
\definecolor{currentstroke}{rgb}{1.000000,1.000000,1.000000}%
\pgfsetstrokecolor{currentstroke}%
\pgfsetdash{}{0pt}%
\pgfpathmoveto{\pgfqpoint{4.592264in}{6.285661in}}%
\pgfpathlineto{\pgfqpoint{4.680000in}{6.285661in}}%
\pgfpathlineto{\pgfqpoint{4.680000in}{6.197925in}}%
\pgfpathlineto{\pgfqpoint{4.592264in}{6.197925in}}%
\pgfpathlineto{\pgfqpoint{4.592264in}{6.285661in}}%
\pgfusepath{stroke,fill}%
\end{pgfscope}%
\begin{pgfscope}%
\pgfpathrectangle{\pgfqpoint{0.380943in}{6.110189in}}{\pgfqpoint{4.650000in}{0.614151in}}%
\pgfusepath{clip}%
\pgfsetbuttcap%
\pgfsetroundjoin%
\definecolor{currentfill}{rgb}{0.991849,0.986144,0.810181}%
\pgfsetfillcolor{currentfill}%
\pgfsetlinewidth{0.250937pt}%
\definecolor{currentstroke}{rgb}{1.000000,1.000000,1.000000}%
\pgfsetstrokecolor{currentstroke}%
\pgfsetdash{}{0pt}%
\pgfpathmoveto{\pgfqpoint{4.680000in}{6.285661in}}%
\pgfpathlineto{\pgfqpoint{4.767736in}{6.285661in}}%
\pgfpathlineto{\pgfqpoint{4.767736in}{6.197925in}}%
\pgfpathlineto{\pgfqpoint{4.680000in}{6.197925in}}%
\pgfpathlineto{\pgfqpoint{4.680000in}{6.285661in}}%
\pgfusepath{stroke,fill}%
\end{pgfscope}%
\begin{pgfscope}%
\pgfpathrectangle{\pgfqpoint{0.380943in}{6.110189in}}{\pgfqpoint{4.650000in}{0.614151in}}%
\pgfusepath{clip}%
\pgfsetbuttcap%
\pgfsetroundjoin%
\definecolor{currentfill}{rgb}{0.965444,0.906113,0.711757}%
\pgfsetfillcolor{currentfill}%
\pgfsetlinewidth{0.250937pt}%
\definecolor{currentstroke}{rgb}{1.000000,1.000000,1.000000}%
\pgfsetstrokecolor{currentstroke}%
\pgfsetdash{}{0pt}%
\pgfpathmoveto{\pgfqpoint{4.767736in}{6.285661in}}%
\pgfpathlineto{\pgfqpoint{4.855471in}{6.285661in}}%
\pgfpathlineto{\pgfqpoint{4.855471in}{6.197925in}}%
\pgfpathlineto{\pgfqpoint{4.767736in}{6.197925in}}%
\pgfpathlineto{\pgfqpoint{4.767736in}{6.285661in}}%
\pgfusepath{stroke,fill}%
\end{pgfscope}%
\begin{pgfscope}%
\pgfpathrectangle{\pgfqpoint{0.380943in}{6.110189in}}{\pgfqpoint{4.650000in}{0.614151in}}%
\pgfusepath{clip}%
\pgfsetbuttcap%
\pgfsetroundjoin%
\definecolor{currentfill}{rgb}{0.965444,0.906113,0.711757}%
\pgfsetfillcolor{currentfill}%
\pgfsetlinewidth{0.250937pt}%
\definecolor{currentstroke}{rgb}{1.000000,1.000000,1.000000}%
\pgfsetstrokecolor{currentstroke}%
\pgfsetdash{}{0pt}%
\pgfpathmoveto{\pgfqpoint{4.855471in}{6.285661in}}%
\pgfpathlineto{\pgfqpoint{4.943207in}{6.285661in}}%
\pgfpathlineto{\pgfqpoint{4.943207in}{6.197925in}}%
\pgfpathlineto{\pgfqpoint{4.855471in}{6.197925in}}%
\pgfpathlineto{\pgfqpoint{4.855471in}{6.285661in}}%
\pgfusepath{stroke,fill}%
\end{pgfscope}%
\begin{pgfscope}%
\pgfpathrectangle{\pgfqpoint{0.380943in}{6.110189in}}{\pgfqpoint{4.650000in}{0.614151in}}%
\pgfusepath{clip}%
\pgfsetbuttcap%
\pgfsetroundjoin%
\pgfsetlinewidth{0.250937pt}%
\definecolor{currentstroke}{rgb}{1.000000,1.000000,1.000000}%
\pgfsetstrokecolor{currentstroke}%
\pgfsetdash{}{0pt}%
\pgfpathmoveto{\pgfqpoint{4.943207in}{6.285661in}}%
\pgfpathlineto{\pgfqpoint{5.030943in}{6.285661in}}%
\pgfpathlineto{\pgfqpoint{5.030943in}{6.197925in}}%
\pgfpathlineto{\pgfqpoint{4.943207in}{6.197925in}}%
\pgfpathlineto{\pgfqpoint{4.943207in}{6.285661in}}%
\pgfusepath{stroke}%
\end{pgfscope}%
\begin{pgfscope}%
\pgfpathrectangle{\pgfqpoint{0.380943in}{6.110189in}}{\pgfqpoint{4.650000in}{0.614151in}}%
\pgfusepath{clip}%
\pgfsetbuttcap%
\pgfsetroundjoin%
\definecolor{currentfill}{rgb}{0.965444,0.906113,0.711757}%
\pgfsetfillcolor{currentfill}%
\pgfsetlinewidth{0.250937pt}%
\definecolor{currentstroke}{rgb}{1.000000,1.000000,1.000000}%
\pgfsetstrokecolor{currentstroke}%
\pgfsetdash{}{0pt}%
\pgfpathmoveto{\pgfqpoint{0.380943in}{6.197925in}}%
\pgfpathlineto{\pgfqpoint{0.468679in}{6.197925in}}%
\pgfpathlineto{\pgfqpoint{0.468679in}{6.110189in}}%
\pgfpathlineto{\pgfqpoint{0.380943in}{6.110189in}}%
\pgfpathlineto{\pgfqpoint{0.380943in}{6.197925in}}%
\pgfusepath{stroke,fill}%
\end{pgfscope}%
\begin{pgfscope}%
\pgfpathrectangle{\pgfqpoint{0.380943in}{6.110189in}}{\pgfqpoint{4.650000in}{0.614151in}}%
\pgfusepath{clip}%
\pgfsetbuttcap%
\pgfsetroundjoin%
\definecolor{currentfill}{rgb}{0.968166,0.945882,0.748604}%
\pgfsetfillcolor{currentfill}%
\pgfsetlinewidth{0.250937pt}%
\definecolor{currentstroke}{rgb}{1.000000,1.000000,1.000000}%
\pgfsetstrokecolor{currentstroke}%
\pgfsetdash{}{0pt}%
\pgfpathmoveto{\pgfqpoint{0.468679in}{6.197925in}}%
\pgfpathlineto{\pgfqpoint{0.556415in}{6.197925in}}%
\pgfpathlineto{\pgfqpoint{0.556415in}{6.110189in}}%
\pgfpathlineto{\pgfqpoint{0.468679in}{6.110189in}}%
\pgfpathlineto{\pgfqpoint{0.468679in}{6.197925in}}%
\pgfusepath{stroke,fill}%
\end{pgfscope}%
\begin{pgfscope}%
\pgfpathrectangle{\pgfqpoint{0.380943in}{6.110189in}}{\pgfqpoint{4.650000in}{0.614151in}}%
\pgfusepath{clip}%
\pgfsetbuttcap%
\pgfsetroundjoin%
\definecolor{currentfill}{rgb}{1.000000,1.000000,0.870204}%
\pgfsetfillcolor{currentfill}%
\pgfsetlinewidth{0.250937pt}%
\definecolor{currentstroke}{rgb}{1.000000,1.000000,1.000000}%
\pgfsetstrokecolor{currentstroke}%
\pgfsetdash{}{0pt}%
\pgfpathmoveto{\pgfqpoint{0.556415in}{6.197925in}}%
\pgfpathlineto{\pgfqpoint{0.644151in}{6.197925in}}%
\pgfpathlineto{\pgfqpoint{0.644151in}{6.110189in}}%
\pgfpathlineto{\pgfqpoint{0.556415in}{6.110189in}}%
\pgfpathlineto{\pgfqpoint{0.556415in}{6.197925in}}%
\pgfusepath{stroke,fill}%
\end{pgfscope}%
\begin{pgfscope}%
\pgfpathrectangle{\pgfqpoint{0.380943in}{6.110189in}}{\pgfqpoint{4.650000in}{0.614151in}}%
\pgfusepath{clip}%
\pgfsetbuttcap%
\pgfsetroundjoin%
\definecolor{currentfill}{rgb}{0.968166,0.945882,0.748604}%
\pgfsetfillcolor{currentfill}%
\pgfsetlinewidth{0.250937pt}%
\definecolor{currentstroke}{rgb}{1.000000,1.000000,1.000000}%
\pgfsetstrokecolor{currentstroke}%
\pgfsetdash{}{0pt}%
\pgfpathmoveto{\pgfqpoint{0.644151in}{6.197925in}}%
\pgfpathlineto{\pgfqpoint{0.731886in}{6.197925in}}%
\pgfpathlineto{\pgfqpoint{0.731886in}{6.110189in}}%
\pgfpathlineto{\pgfqpoint{0.644151in}{6.110189in}}%
\pgfpathlineto{\pgfqpoint{0.644151in}{6.197925in}}%
\pgfusepath{stroke,fill}%
\end{pgfscope}%
\begin{pgfscope}%
\pgfpathrectangle{\pgfqpoint{0.380943in}{6.110189in}}{\pgfqpoint{4.650000in}{0.614151in}}%
\pgfusepath{clip}%
\pgfsetbuttcap%
\pgfsetroundjoin%
\definecolor{currentfill}{rgb}{0.991849,0.986144,0.810181}%
\pgfsetfillcolor{currentfill}%
\pgfsetlinewidth{0.250937pt}%
\definecolor{currentstroke}{rgb}{1.000000,1.000000,1.000000}%
\pgfsetstrokecolor{currentstroke}%
\pgfsetdash{}{0pt}%
\pgfpathmoveto{\pgfqpoint{0.731886in}{6.197925in}}%
\pgfpathlineto{\pgfqpoint{0.819622in}{6.197925in}}%
\pgfpathlineto{\pgfqpoint{0.819622in}{6.110189in}}%
\pgfpathlineto{\pgfqpoint{0.731886in}{6.110189in}}%
\pgfpathlineto{\pgfqpoint{0.731886in}{6.197925in}}%
\pgfusepath{stroke,fill}%
\end{pgfscope}%
\begin{pgfscope}%
\pgfpathrectangle{\pgfqpoint{0.380943in}{6.110189in}}{\pgfqpoint{4.650000in}{0.614151in}}%
\pgfusepath{clip}%
\pgfsetbuttcap%
\pgfsetroundjoin%
\definecolor{currentfill}{rgb}{0.965444,0.906113,0.711757}%
\pgfsetfillcolor{currentfill}%
\pgfsetlinewidth{0.250937pt}%
\definecolor{currentstroke}{rgb}{1.000000,1.000000,1.000000}%
\pgfsetstrokecolor{currentstroke}%
\pgfsetdash{}{0pt}%
\pgfpathmoveto{\pgfqpoint{0.819622in}{6.197925in}}%
\pgfpathlineto{\pgfqpoint{0.907358in}{6.197925in}}%
\pgfpathlineto{\pgfqpoint{0.907358in}{6.110189in}}%
\pgfpathlineto{\pgfqpoint{0.819622in}{6.110189in}}%
\pgfpathlineto{\pgfqpoint{0.819622in}{6.197925in}}%
\pgfusepath{stroke,fill}%
\end{pgfscope}%
\begin{pgfscope}%
\pgfpathrectangle{\pgfqpoint{0.380943in}{6.110189in}}{\pgfqpoint{4.650000in}{0.614151in}}%
\pgfusepath{clip}%
\pgfsetbuttcap%
\pgfsetroundjoin%
\definecolor{currentfill}{rgb}{0.972549,0.870588,0.692810}%
\pgfsetfillcolor{currentfill}%
\pgfsetlinewidth{0.250937pt}%
\definecolor{currentstroke}{rgb}{1.000000,1.000000,1.000000}%
\pgfsetstrokecolor{currentstroke}%
\pgfsetdash{}{0pt}%
\pgfpathmoveto{\pgfqpoint{0.907358in}{6.197925in}}%
\pgfpathlineto{\pgfqpoint{0.995094in}{6.197925in}}%
\pgfpathlineto{\pgfqpoint{0.995094in}{6.110189in}}%
\pgfpathlineto{\pgfqpoint{0.907358in}{6.110189in}}%
\pgfpathlineto{\pgfqpoint{0.907358in}{6.197925in}}%
\pgfusepath{stroke,fill}%
\end{pgfscope}%
\begin{pgfscope}%
\pgfpathrectangle{\pgfqpoint{0.380943in}{6.110189in}}{\pgfqpoint{4.650000in}{0.614151in}}%
\pgfusepath{clip}%
\pgfsetbuttcap%
\pgfsetroundjoin%
\definecolor{currentfill}{rgb}{0.979654,0.837186,0.669619}%
\pgfsetfillcolor{currentfill}%
\pgfsetlinewidth{0.250937pt}%
\definecolor{currentstroke}{rgb}{1.000000,1.000000,1.000000}%
\pgfsetstrokecolor{currentstroke}%
\pgfsetdash{}{0pt}%
\pgfpathmoveto{\pgfqpoint{0.995094in}{6.197925in}}%
\pgfpathlineto{\pgfqpoint{1.082830in}{6.197925in}}%
\pgfpathlineto{\pgfqpoint{1.082830in}{6.110189in}}%
\pgfpathlineto{\pgfqpoint{0.995094in}{6.110189in}}%
\pgfpathlineto{\pgfqpoint{0.995094in}{6.197925in}}%
\pgfusepath{stroke,fill}%
\end{pgfscope}%
\begin{pgfscope}%
\pgfpathrectangle{\pgfqpoint{0.380943in}{6.110189in}}{\pgfqpoint{4.650000in}{0.614151in}}%
\pgfusepath{clip}%
\pgfsetbuttcap%
\pgfsetroundjoin%
\definecolor{currentfill}{rgb}{0.968166,0.945882,0.748604}%
\pgfsetfillcolor{currentfill}%
\pgfsetlinewidth{0.250937pt}%
\definecolor{currentstroke}{rgb}{1.000000,1.000000,1.000000}%
\pgfsetstrokecolor{currentstroke}%
\pgfsetdash{}{0pt}%
\pgfpathmoveto{\pgfqpoint{1.082830in}{6.197925in}}%
\pgfpathlineto{\pgfqpoint{1.170566in}{6.197925in}}%
\pgfpathlineto{\pgfqpoint{1.170566in}{6.110189in}}%
\pgfpathlineto{\pgfqpoint{1.082830in}{6.110189in}}%
\pgfpathlineto{\pgfqpoint{1.082830in}{6.197925in}}%
\pgfusepath{stroke,fill}%
\end{pgfscope}%
\begin{pgfscope}%
\pgfpathrectangle{\pgfqpoint{0.380943in}{6.110189in}}{\pgfqpoint{4.650000in}{0.614151in}}%
\pgfusepath{clip}%
\pgfsetbuttcap%
\pgfsetroundjoin%
\definecolor{currentfill}{rgb}{0.972549,0.870588,0.692810}%
\pgfsetfillcolor{currentfill}%
\pgfsetlinewidth{0.250937pt}%
\definecolor{currentstroke}{rgb}{1.000000,1.000000,1.000000}%
\pgfsetstrokecolor{currentstroke}%
\pgfsetdash{}{0pt}%
\pgfpathmoveto{\pgfqpoint{1.170566in}{6.197925in}}%
\pgfpathlineto{\pgfqpoint{1.258302in}{6.197925in}}%
\pgfpathlineto{\pgfqpoint{1.258302in}{6.110189in}}%
\pgfpathlineto{\pgfqpoint{1.170566in}{6.110189in}}%
\pgfpathlineto{\pgfqpoint{1.170566in}{6.197925in}}%
\pgfusepath{stroke,fill}%
\end{pgfscope}%
\begin{pgfscope}%
\pgfpathrectangle{\pgfqpoint{0.380943in}{6.110189in}}{\pgfqpoint{4.650000in}{0.614151in}}%
\pgfusepath{clip}%
\pgfsetbuttcap%
\pgfsetroundjoin%
\definecolor{currentfill}{rgb}{1.000000,1.000000,0.870204}%
\pgfsetfillcolor{currentfill}%
\pgfsetlinewidth{0.250937pt}%
\definecolor{currentstroke}{rgb}{1.000000,1.000000,1.000000}%
\pgfsetstrokecolor{currentstroke}%
\pgfsetdash{}{0pt}%
\pgfpathmoveto{\pgfqpoint{1.258302in}{6.197925in}}%
\pgfpathlineto{\pgfqpoint{1.346037in}{6.197925in}}%
\pgfpathlineto{\pgfqpoint{1.346037in}{6.110189in}}%
\pgfpathlineto{\pgfqpoint{1.258302in}{6.110189in}}%
\pgfpathlineto{\pgfqpoint{1.258302in}{6.197925in}}%
\pgfusepath{stroke,fill}%
\end{pgfscope}%
\begin{pgfscope}%
\pgfpathrectangle{\pgfqpoint{0.380943in}{6.110189in}}{\pgfqpoint{4.650000in}{0.614151in}}%
\pgfusepath{clip}%
\pgfsetbuttcap%
\pgfsetroundjoin%
\definecolor{currentfill}{rgb}{1.000000,1.000000,0.870204}%
\pgfsetfillcolor{currentfill}%
\pgfsetlinewidth{0.250937pt}%
\definecolor{currentstroke}{rgb}{1.000000,1.000000,1.000000}%
\pgfsetstrokecolor{currentstroke}%
\pgfsetdash{}{0pt}%
\pgfpathmoveto{\pgfqpoint{1.346037in}{6.197925in}}%
\pgfpathlineto{\pgfqpoint{1.433773in}{6.197925in}}%
\pgfpathlineto{\pgfqpoint{1.433773in}{6.110189in}}%
\pgfpathlineto{\pgfqpoint{1.346037in}{6.110189in}}%
\pgfpathlineto{\pgfqpoint{1.346037in}{6.197925in}}%
\pgfusepath{stroke,fill}%
\end{pgfscope}%
\begin{pgfscope}%
\pgfpathrectangle{\pgfqpoint{0.380943in}{6.110189in}}{\pgfqpoint{4.650000in}{0.614151in}}%
\pgfusepath{clip}%
\pgfsetbuttcap%
\pgfsetroundjoin%
\definecolor{currentfill}{rgb}{1.000000,1.000000,0.870204}%
\pgfsetfillcolor{currentfill}%
\pgfsetlinewidth{0.250937pt}%
\definecolor{currentstroke}{rgb}{1.000000,1.000000,1.000000}%
\pgfsetstrokecolor{currentstroke}%
\pgfsetdash{}{0pt}%
\pgfpathmoveto{\pgfqpoint{1.433773in}{6.197925in}}%
\pgfpathlineto{\pgfqpoint{1.521509in}{6.197925in}}%
\pgfpathlineto{\pgfqpoint{1.521509in}{6.110189in}}%
\pgfpathlineto{\pgfqpoint{1.433773in}{6.110189in}}%
\pgfpathlineto{\pgfqpoint{1.433773in}{6.197925in}}%
\pgfusepath{stroke,fill}%
\end{pgfscope}%
\begin{pgfscope}%
\pgfpathrectangle{\pgfqpoint{0.380943in}{6.110189in}}{\pgfqpoint{4.650000in}{0.614151in}}%
\pgfusepath{clip}%
\pgfsetbuttcap%
\pgfsetroundjoin%
\definecolor{currentfill}{rgb}{0.965444,0.906113,0.711757}%
\pgfsetfillcolor{currentfill}%
\pgfsetlinewidth{0.250937pt}%
\definecolor{currentstroke}{rgb}{1.000000,1.000000,1.000000}%
\pgfsetstrokecolor{currentstroke}%
\pgfsetdash{}{0pt}%
\pgfpathmoveto{\pgfqpoint{1.521509in}{6.197925in}}%
\pgfpathlineto{\pgfqpoint{1.609245in}{6.197925in}}%
\pgfpathlineto{\pgfqpoint{1.609245in}{6.110189in}}%
\pgfpathlineto{\pgfqpoint{1.521509in}{6.110189in}}%
\pgfpathlineto{\pgfqpoint{1.521509in}{6.197925in}}%
\pgfusepath{stroke,fill}%
\end{pgfscope}%
\begin{pgfscope}%
\pgfpathrectangle{\pgfqpoint{0.380943in}{6.110189in}}{\pgfqpoint{4.650000in}{0.614151in}}%
\pgfusepath{clip}%
\pgfsetbuttcap%
\pgfsetroundjoin%
\definecolor{currentfill}{rgb}{0.991849,0.986144,0.810181}%
\pgfsetfillcolor{currentfill}%
\pgfsetlinewidth{0.250937pt}%
\definecolor{currentstroke}{rgb}{1.000000,1.000000,1.000000}%
\pgfsetstrokecolor{currentstroke}%
\pgfsetdash{}{0pt}%
\pgfpathmoveto{\pgfqpoint{1.609245in}{6.197925in}}%
\pgfpathlineto{\pgfqpoint{1.696981in}{6.197925in}}%
\pgfpathlineto{\pgfqpoint{1.696981in}{6.110189in}}%
\pgfpathlineto{\pgfqpoint{1.609245in}{6.110189in}}%
\pgfpathlineto{\pgfqpoint{1.609245in}{6.197925in}}%
\pgfusepath{stroke,fill}%
\end{pgfscope}%
\begin{pgfscope}%
\pgfpathrectangle{\pgfqpoint{0.380943in}{6.110189in}}{\pgfqpoint{4.650000in}{0.614151in}}%
\pgfusepath{clip}%
\pgfsetbuttcap%
\pgfsetroundjoin%
\definecolor{currentfill}{rgb}{0.991849,0.986144,0.810181}%
\pgfsetfillcolor{currentfill}%
\pgfsetlinewidth{0.250937pt}%
\definecolor{currentstroke}{rgb}{1.000000,1.000000,1.000000}%
\pgfsetstrokecolor{currentstroke}%
\pgfsetdash{}{0pt}%
\pgfpathmoveto{\pgfqpoint{1.696981in}{6.197925in}}%
\pgfpathlineto{\pgfqpoint{1.784717in}{6.197925in}}%
\pgfpathlineto{\pgfqpoint{1.784717in}{6.110189in}}%
\pgfpathlineto{\pgfqpoint{1.696981in}{6.110189in}}%
\pgfpathlineto{\pgfqpoint{1.696981in}{6.197925in}}%
\pgfusepath{stroke,fill}%
\end{pgfscope}%
\begin{pgfscope}%
\pgfpathrectangle{\pgfqpoint{0.380943in}{6.110189in}}{\pgfqpoint{4.650000in}{0.614151in}}%
\pgfusepath{clip}%
\pgfsetbuttcap%
\pgfsetroundjoin%
\definecolor{currentfill}{rgb}{0.979654,0.837186,0.669619}%
\pgfsetfillcolor{currentfill}%
\pgfsetlinewidth{0.250937pt}%
\definecolor{currentstroke}{rgb}{1.000000,1.000000,1.000000}%
\pgfsetstrokecolor{currentstroke}%
\pgfsetdash{}{0pt}%
\pgfpathmoveto{\pgfqpoint{1.784717in}{6.197925in}}%
\pgfpathlineto{\pgfqpoint{1.872452in}{6.197925in}}%
\pgfpathlineto{\pgfqpoint{1.872452in}{6.110189in}}%
\pgfpathlineto{\pgfqpoint{1.784717in}{6.110189in}}%
\pgfpathlineto{\pgfqpoint{1.784717in}{6.197925in}}%
\pgfusepath{stroke,fill}%
\end{pgfscope}%
\begin{pgfscope}%
\pgfpathrectangle{\pgfqpoint{0.380943in}{6.110189in}}{\pgfqpoint{4.650000in}{0.614151in}}%
\pgfusepath{clip}%
\pgfsetbuttcap%
\pgfsetroundjoin%
\definecolor{currentfill}{rgb}{0.962414,0.923552,0.722891}%
\pgfsetfillcolor{currentfill}%
\pgfsetlinewidth{0.250937pt}%
\definecolor{currentstroke}{rgb}{1.000000,1.000000,1.000000}%
\pgfsetstrokecolor{currentstroke}%
\pgfsetdash{}{0pt}%
\pgfpathmoveto{\pgfqpoint{1.872452in}{6.197925in}}%
\pgfpathlineto{\pgfqpoint{1.960188in}{6.197925in}}%
\pgfpathlineto{\pgfqpoint{1.960188in}{6.110189in}}%
\pgfpathlineto{\pgfqpoint{1.872452in}{6.110189in}}%
\pgfpathlineto{\pgfqpoint{1.872452in}{6.197925in}}%
\pgfusepath{stroke,fill}%
\end{pgfscope}%
\begin{pgfscope}%
\pgfpathrectangle{\pgfqpoint{0.380943in}{6.110189in}}{\pgfqpoint{4.650000in}{0.614151in}}%
\pgfusepath{clip}%
\pgfsetbuttcap%
\pgfsetroundjoin%
\definecolor{currentfill}{rgb}{0.968166,0.945882,0.748604}%
\pgfsetfillcolor{currentfill}%
\pgfsetlinewidth{0.250937pt}%
\definecolor{currentstroke}{rgb}{1.000000,1.000000,1.000000}%
\pgfsetstrokecolor{currentstroke}%
\pgfsetdash{}{0pt}%
\pgfpathmoveto{\pgfqpoint{1.960188in}{6.197925in}}%
\pgfpathlineto{\pgfqpoint{2.047924in}{6.197925in}}%
\pgfpathlineto{\pgfqpoint{2.047924in}{6.110189in}}%
\pgfpathlineto{\pgfqpoint{1.960188in}{6.110189in}}%
\pgfpathlineto{\pgfqpoint{1.960188in}{6.197925in}}%
\pgfusepath{stroke,fill}%
\end{pgfscope}%
\begin{pgfscope}%
\pgfpathrectangle{\pgfqpoint{0.380943in}{6.110189in}}{\pgfqpoint{4.650000in}{0.614151in}}%
\pgfusepath{clip}%
\pgfsetbuttcap%
\pgfsetroundjoin%
\definecolor{currentfill}{rgb}{0.962414,0.923552,0.722891}%
\pgfsetfillcolor{currentfill}%
\pgfsetlinewidth{0.250937pt}%
\definecolor{currentstroke}{rgb}{1.000000,1.000000,1.000000}%
\pgfsetstrokecolor{currentstroke}%
\pgfsetdash{}{0pt}%
\pgfpathmoveto{\pgfqpoint{2.047924in}{6.197925in}}%
\pgfpathlineto{\pgfqpoint{2.135660in}{6.197925in}}%
\pgfpathlineto{\pgfqpoint{2.135660in}{6.110189in}}%
\pgfpathlineto{\pgfqpoint{2.047924in}{6.110189in}}%
\pgfpathlineto{\pgfqpoint{2.047924in}{6.197925in}}%
\pgfusepath{stroke,fill}%
\end{pgfscope}%
\begin{pgfscope}%
\pgfpathrectangle{\pgfqpoint{0.380943in}{6.110189in}}{\pgfqpoint{4.650000in}{0.614151in}}%
\pgfusepath{clip}%
\pgfsetbuttcap%
\pgfsetroundjoin%
\definecolor{currentfill}{rgb}{0.991849,0.986144,0.810181}%
\pgfsetfillcolor{currentfill}%
\pgfsetlinewidth{0.250937pt}%
\definecolor{currentstroke}{rgb}{1.000000,1.000000,1.000000}%
\pgfsetstrokecolor{currentstroke}%
\pgfsetdash{}{0pt}%
\pgfpathmoveto{\pgfqpoint{2.135660in}{6.197925in}}%
\pgfpathlineto{\pgfqpoint{2.223396in}{6.197925in}}%
\pgfpathlineto{\pgfqpoint{2.223396in}{6.110189in}}%
\pgfpathlineto{\pgfqpoint{2.135660in}{6.110189in}}%
\pgfpathlineto{\pgfqpoint{2.135660in}{6.197925in}}%
\pgfusepath{stroke,fill}%
\end{pgfscope}%
\begin{pgfscope}%
\pgfpathrectangle{\pgfqpoint{0.380943in}{6.110189in}}{\pgfqpoint{4.650000in}{0.614151in}}%
\pgfusepath{clip}%
\pgfsetbuttcap%
\pgfsetroundjoin%
\definecolor{currentfill}{rgb}{0.991849,0.986144,0.810181}%
\pgfsetfillcolor{currentfill}%
\pgfsetlinewidth{0.250937pt}%
\definecolor{currentstroke}{rgb}{1.000000,1.000000,1.000000}%
\pgfsetstrokecolor{currentstroke}%
\pgfsetdash{}{0pt}%
\pgfpathmoveto{\pgfqpoint{2.223396in}{6.197925in}}%
\pgfpathlineto{\pgfqpoint{2.311132in}{6.197925in}}%
\pgfpathlineto{\pgfqpoint{2.311132in}{6.110189in}}%
\pgfpathlineto{\pgfqpoint{2.223396in}{6.110189in}}%
\pgfpathlineto{\pgfqpoint{2.223396in}{6.197925in}}%
\pgfusepath{stroke,fill}%
\end{pgfscope}%
\begin{pgfscope}%
\pgfpathrectangle{\pgfqpoint{0.380943in}{6.110189in}}{\pgfqpoint{4.650000in}{0.614151in}}%
\pgfusepath{clip}%
\pgfsetbuttcap%
\pgfsetroundjoin%
\definecolor{currentfill}{rgb}{0.991849,0.986144,0.810181}%
\pgfsetfillcolor{currentfill}%
\pgfsetlinewidth{0.250937pt}%
\definecolor{currentstroke}{rgb}{1.000000,1.000000,1.000000}%
\pgfsetstrokecolor{currentstroke}%
\pgfsetdash{}{0pt}%
\pgfpathmoveto{\pgfqpoint{2.311132in}{6.197925in}}%
\pgfpathlineto{\pgfqpoint{2.398868in}{6.197925in}}%
\pgfpathlineto{\pgfqpoint{2.398868in}{6.110189in}}%
\pgfpathlineto{\pgfqpoint{2.311132in}{6.110189in}}%
\pgfpathlineto{\pgfqpoint{2.311132in}{6.197925in}}%
\pgfusepath{stroke,fill}%
\end{pgfscope}%
\begin{pgfscope}%
\pgfpathrectangle{\pgfqpoint{0.380943in}{6.110189in}}{\pgfqpoint{4.650000in}{0.614151in}}%
\pgfusepath{clip}%
\pgfsetbuttcap%
\pgfsetroundjoin%
\definecolor{currentfill}{rgb}{0.965444,0.906113,0.711757}%
\pgfsetfillcolor{currentfill}%
\pgfsetlinewidth{0.250937pt}%
\definecolor{currentstroke}{rgb}{1.000000,1.000000,1.000000}%
\pgfsetstrokecolor{currentstroke}%
\pgfsetdash{}{0pt}%
\pgfpathmoveto{\pgfqpoint{2.398868in}{6.197925in}}%
\pgfpathlineto{\pgfqpoint{2.486603in}{6.197925in}}%
\pgfpathlineto{\pgfqpoint{2.486603in}{6.110189in}}%
\pgfpathlineto{\pgfqpoint{2.398868in}{6.110189in}}%
\pgfpathlineto{\pgfqpoint{2.398868in}{6.197925in}}%
\pgfusepath{stroke,fill}%
\end{pgfscope}%
\begin{pgfscope}%
\pgfpathrectangle{\pgfqpoint{0.380943in}{6.110189in}}{\pgfqpoint{4.650000in}{0.614151in}}%
\pgfusepath{clip}%
\pgfsetbuttcap%
\pgfsetroundjoin%
\definecolor{currentfill}{rgb}{0.991849,0.986144,0.810181}%
\pgfsetfillcolor{currentfill}%
\pgfsetlinewidth{0.250937pt}%
\definecolor{currentstroke}{rgb}{1.000000,1.000000,1.000000}%
\pgfsetstrokecolor{currentstroke}%
\pgfsetdash{}{0pt}%
\pgfpathmoveto{\pgfqpoint{2.486603in}{6.197925in}}%
\pgfpathlineto{\pgfqpoint{2.574339in}{6.197925in}}%
\pgfpathlineto{\pgfqpoint{2.574339in}{6.110189in}}%
\pgfpathlineto{\pgfqpoint{2.486603in}{6.110189in}}%
\pgfpathlineto{\pgfqpoint{2.486603in}{6.197925in}}%
\pgfusepath{stroke,fill}%
\end{pgfscope}%
\begin{pgfscope}%
\pgfpathrectangle{\pgfqpoint{0.380943in}{6.110189in}}{\pgfqpoint{4.650000in}{0.614151in}}%
\pgfusepath{clip}%
\pgfsetbuttcap%
\pgfsetroundjoin%
\definecolor{currentfill}{rgb}{1.000000,1.000000,0.929412}%
\pgfsetfillcolor{currentfill}%
\pgfsetlinewidth{0.250937pt}%
\definecolor{currentstroke}{rgb}{1.000000,1.000000,1.000000}%
\pgfsetstrokecolor{currentstroke}%
\pgfsetdash{}{0pt}%
\pgfpathmoveto{\pgfqpoint{2.574339in}{6.197925in}}%
\pgfpathlineto{\pgfqpoint{2.662075in}{6.197925in}}%
\pgfpathlineto{\pgfqpoint{2.662075in}{6.110189in}}%
\pgfpathlineto{\pgfqpoint{2.574339in}{6.110189in}}%
\pgfpathlineto{\pgfqpoint{2.574339in}{6.197925in}}%
\pgfusepath{stroke,fill}%
\end{pgfscope}%
\begin{pgfscope}%
\pgfpathrectangle{\pgfqpoint{0.380943in}{6.110189in}}{\pgfqpoint{4.650000in}{0.614151in}}%
\pgfusepath{clip}%
\pgfsetbuttcap%
\pgfsetroundjoin%
\definecolor{currentfill}{rgb}{0.968166,0.945882,0.748604}%
\pgfsetfillcolor{currentfill}%
\pgfsetlinewidth{0.250937pt}%
\definecolor{currentstroke}{rgb}{1.000000,1.000000,1.000000}%
\pgfsetstrokecolor{currentstroke}%
\pgfsetdash{}{0pt}%
\pgfpathmoveto{\pgfqpoint{2.662075in}{6.197925in}}%
\pgfpathlineto{\pgfqpoint{2.749811in}{6.197925in}}%
\pgfpathlineto{\pgfqpoint{2.749811in}{6.110189in}}%
\pgfpathlineto{\pgfqpoint{2.662075in}{6.110189in}}%
\pgfpathlineto{\pgfqpoint{2.662075in}{6.197925in}}%
\pgfusepath{stroke,fill}%
\end{pgfscope}%
\begin{pgfscope}%
\pgfpathrectangle{\pgfqpoint{0.380943in}{6.110189in}}{\pgfqpoint{4.650000in}{0.614151in}}%
\pgfusepath{clip}%
\pgfsetbuttcap%
\pgfsetroundjoin%
\definecolor{currentfill}{rgb}{0.965444,0.906113,0.711757}%
\pgfsetfillcolor{currentfill}%
\pgfsetlinewidth{0.250937pt}%
\definecolor{currentstroke}{rgb}{1.000000,1.000000,1.000000}%
\pgfsetstrokecolor{currentstroke}%
\pgfsetdash{}{0pt}%
\pgfpathmoveto{\pgfqpoint{2.749811in}{6.197925in}}%
\pgfpathlineto{\pgfqpoint{2.837547in}{6.197925in}}%
\pgfpathlineto{\pgfqpoint{2.837547in}{6.110189in}}%
\pgfpathlineto{\pgfqpoint{2.749811in}{6.110189in}}%
\pgfpathlineto{\pgfqpoint{2.749811in}{6.197925in}}%
\pgfusepath{stroke,fill}%
\end{pgfscope}%
\begin{pgfscope}%
\pgfpathrectangle{\pgfqpoint{0.380943in}{6.110189in}}{\pgfqpoint{4.650000in}{0.614151in}}%
\pgfusepath{clip}%
\pgfsetbuttcap%
\pgfsetroundjoin%
\definecolor{currentfill}{rgb}{1.000000,1.000000,0.929412}%
\pgfsetfillcolor{currentfill}%
\pgfsetlinewidth{0.250937pt}%
\definecolor{currentstroke}{rgb}{1.000000,1.000000,1.000000}%
\pgfsetstrokecolor{currentstroke}%
\pgfsetdash{}{0pt}%
\pgfpathmoveto{\pgfqpoint{2.837547in}{6.197925in}}%
\pgfpathlineto{\pgfqpoint{2.925283in}{6.197925in}}%
\pgfpathlineto{\pgfqpoint{2.925283in}{6.110189in}}%
\pgfpathlineto{\pgfqpoint{2.837547in}{6.110189in}}%
\pgfpathlineto{\pgfqpoint{2.837547in}{6.197925in}}%
\pgfusepath{stroke,fill}%
\end{pgfscope}%
\begin{pgfscope}%
\pgfpathrectangle{\pgfqpoint{0.380943in}{6.110189in}}{\pgfqpoint{4.650000in}{0.614151in}}%
\pgfusepath{clip}%
\pgfsetbuttcap%
\pgfsetroundjoin%
\definecolor{currentfill}{rgb}{1.000000,1.000000,0.870204}%
\pgfsetfillcolor{currentfill}%
\pgfsetlinewidth{0.250937pt}%
\definecolor{currentstroke}{rgb}{1.000000,1.000000,1.000000}%
\pgfsetstrokecolor{currentstroke}%
\pgfsetdash{}{0pt}%
\pgfpathmoveto{\pgfqpoint{2.925283in}{6.197925in}}%
\pgfpathlineto{\pgfqpoint{3.013019in}{6.197925in}}%
\pgfpathlineto{\pgfqpoint{3.013019in}{6.110189in}}%
\pgfpathlineto{\pgfqpoint{2.925283in}{6.110189in}}%
\pgfpathlineto{\pgfqpoint{2.925283in}{6.197925in}}%
\pgfusepath{stroke,fill}%
\end{pgfscope}%
\begin{pgfscope}%
\pgfpathrectangle{\pgfqpoint{0.380943in}{6.110189in}}{\pgfqpoint{4.650000in}{0.614151in}}%
\pgfusepath{clip}%
\pgfsetbuttcap%
\pgfsetroundjoin%
\definecolor{currentfill}{rgb}{0.962414,0.923552,0.722891}%
\pgfsetfillcolor{currentfill}%
\pgfsetlinewidth{0.250937pt}%
\definecolor{currentstroke}{rgb}{1.000000,1.000000,1.000000}%
\pgfsetstrokecolor{currentstroke}%
\pgfsetdash{}{0pt}%
\pgfpathmoveto{\pgfqpoint{3.013019in}{6.197925in}}%
\pgfpathlineto{\pgfqpoint{3.100754in}{6.197925in}}%
\pgfpathlineto{\pgfqpoint{3.100754in}{6.110189in}}%
\pgfpathlineto{\pgfqpoint{3.013019in}{6.110189in}}%
\pgfpathlineto{\pgfqpoint{3.013019in}{6.197925in}}%
\pgfusepath{stroke,fill}%
\end{pgfscope}%
\begin{pgfscope}%
\pgfpathrectangle{\pgfqpoint{0.380943in}{6.110189in}}{\pgfqpoint{4.650000in}{0.614151in}}%
\pgfusepath{clip}%
\pgfsetbuttcap%
\pgfsetroundjoin%
\definecolor{currentfill}{rgb}{1.000000,1.000000,0.870204}%
\pgfsetfillcolor{currentfill}%
\pgfsetlinewidth{0.250937pt}%
\definecolor{currentstroke}{rgb}{1.000000,1.000000,1.000000}%
\pgfsetstrokecolor{currentstroke}%
\pgfsetdash{}{0pt}%
\pgfpathmoveto{\pgfqpoint{3.100754in}{6.197925in}}%
\pgfpathlineto{\pgfqpoint{3.188490in}{6.197925in}}%
\pgfpathlineto{\pgfqpoint{3.188490in}{6.110189in}}%
\pgfpathlineto{\pgfqpoint{3.100754in}{6.110189in}}%
\pgfpathlineto{\pgfqpoint{3.100754in}{6.197925in}}%
\pgfusepath{stroke,fill}%
\end{pgfscope}%
\begin{pgfscope}%
\pgfpathrectangle{\pgfqpoint{0.380943in}{6.110189in}}{\pgfqpoint{4.650000in}{0.614151in}}%
\pgfusepath{clip}%
\pgfsetbuttcap%
\pgfsetroundjoin%
\definecolor{currentfill}{rgb}{1.000000,1.000000,0.929412}%
\pgfsetfillcolor{currentfill}%
\pgfsetlinewidth{0.250937pt}%
\definecolor{currentstroke}{rgb}{1.000000,1.000000,1.000000}%
\pgfsetstrokecolor{currentstroke}%
\pgfsetdash{}{0pt}%
\pgfpathmoveto{\pgfqpoint{3.188490in}{6.197925in}}%
\pgfpathlineto{\pgfqpoint{3.276226in}{6.197925in}}%
\pgfpathlineto{\pgfqpoint{3.276226in}{6.110189in}}%
\pgfpathlineto{\pgfqpoint{3.188490in}{6.110189in}}%
\pgfpathlineto{\pgfqpoint{3.188490in}{6.197925in}}%
\pgfusepath{stroke,fill}%
\end{pgfscope}%
\begin{pgfscope}%
\pgfpathrectangle{\pgfqpoint{0.380943in}{6.110189in}}{\pgfqpoint{4.650000in}{0.614151in}}%
\pgfusepath{clip}%
\pgfsetbuttcap%
\pgfsetroundjoin%
\definecolor{currentfill}{rgb}{0.968166,0.945882,0.748604}%
\pgfsetfillcolor{currentfill}%
\pgfsetlinewidth{0.250937pt}%
\definecolor{currentstroke}{rgb}{1.000000,1.000000,1.000000}%
\pgfsetstrokecolor{currentstroke}%
\pgfsetdash{}{0pt}%
\pgfpathmoveto{\pgfqpoint{3.276226in}{6.197925in}}%
\pgfpathlineto{\pgfqpoint{3.363962in}{6.197925in}}%
\pgfpathlineto{\pgfqpoint{3.363962in}{6.110189in}}%
\pgfpathlineto{\pgfqpoint{3.276226in}{6.110189in}}%
\pgfpathlineto{\pgfqpoint{3.276226in}{6.197925in}}%
\pgfusepath{stroke,fill}%
\end{pgfscope}%
\begin{pgfscope}%
\pgfpathrectangle{\pgfqpoint{0.380943in}{6.110189in}}{\pgfqpoint{4.650000in}{0.614151in}}%
\pgfusepath{clip}%
\pgfsetbuttcap%
\pgfsetroundjoin%
\definecolor{currentfill}{rgb}{1.000000,1.000000,0.870204}%
\pgfsetfillcolor{currentfill}%
\pgfsetlinewidth{0.250937pt}%
\definecolor{currentstroke}{rgb}{1.000000,1.000000,1.000000}%
\pgfsetstrokecolor{currentstroke}%
\pgfsetdash{}{0pt}%
\pgfpathmoveto{\pgfqpoint{3.363962in}{6.197925in}}%
\pgfpathlineto{\pgfqpoint{3.451698in}{6.197925in}}%
\pgfpathlineto{\pgfqpoint{3.451698in}{6.110189in}}%
\pgfpathlineto{\pgfqpoint{3.363962in}{6.110189in}}%
\pgfpathlineto{\pgfqpoint{3.363962in}{6.197925in}}%
\pgfusepath{stroke,fill}%
\end{pgfscope}%
\begin{pgfscope}%
\pgfpathrectangle{\pgfqpoint{0.380943in}{6.110189in}}{\pgfqpoint{4.650000in}{0.614151in}}%
\pgfusepath{clip}%
\pgfsetbuttcap%
\pgfsetroundjoin%
\definecolor{currentfill}{rgb}{1.000000,1.000000,0.870204}%
\pgfsetfillcolor{currentfill}%
\pgfsetlinewidth{0.250937pt}%
\definecolor{currentstroke}{rgb}{1.000000,1.000000,1.000000}%
\pgfsetstrokecolor{currentstroke}%
\pgfsetdash{}{0pt}%
\pgfpathmoveto{\pgfqpoint{3.451698in}{6.197925in}}%
\pgfpathlineto{\pgfqpoint{3.539434in}{6.197925in}}%
\pgfpathlineto{\pgfqpoint{3.539434in}{6.110189in}}%
\pgfpathlineto{\pgfqpoint{3.451698in}{6.110189in}}%
\pgfpathlineto{\pgfqpoint{3.451698in}{6.197925in}}%
\pgfusepath{stroke,fill}%
\end{pgfscope}%
\begin{pgfscope}%
\pgfpathrectangle{\pgfqpoint{0.380943in}{6.110189in}}{\pgfqpoint{4.650000in}{0.614151in}}%
\pgfusepath{clip}%
\pgfsetbuttcap%
\pgfsetroundjoin%
\definecolor{currentfill}{rgb}{0.991849,0.986144,0.810181}%
\pgfsetfillcolor{currentfill}%
\pgfsetlinewidth{0.250937pt}%
\definecolor{currentstroke}{rgb}{1.000000,1.000000,1.000000}%
\pgfsetstrokecolor{currentstroke}%
\pgfsetdash{}{0pt}%
\pgfpathmoveto{\pgfqpoint{3.539434in}{6.197925in}}%
\pgfpathlineto{\pgfqpoint{3.627169in}{6.197925in}}%
\pgfpathlineto{\pgfqpoint{3.627169in}{6.110189in}}%
\pgfpathlineto{\pgfqpoint{3.539434in}{6.110189in}}%
\pgfpathlineto{\pgfqpoint{3.539434in}{6.197925in}}%
\pgfusepath{stroke,fill}%
\end{pgfscope}%
\begin{pgfscope}%
\pgfpathrectangle{\pgfqpoint{0.380943in}{6.110189in}}{\pgfqpoint{4.650000in}{0.614151in}}%
\pgfusepath{clip}%
\pgfsetbuttcap%
\pgfsetroundjoin%
\definecolor{currentfill}{rgb}{0.968166,0.945882,0.748604}%
\pgfsetfillcolor{currentfill}%
\pgfsetlinewidth{0.250937pt}%
\definecolor{currentstroke}{rgb}{1.000000,1.000000,1.000000}%
\pgfsetstrokecolor{currentstroke}%
\pgfsetdash{}{0pt}%
\pgfpathmoveto{\pgfqpoint{3.627169in}{6.197925in}}%
\pgfpathlineto{\pgfqpoint{3.714905in}{6.197925in}}%
\pgfpathlineto{\pgfqpoint{3.714905in}{6.110189in}}%
\pgfpathlineto{\pgfqpoint{3.627169in}{6.110189in}}%
\pgfpathlineto{\pgfqpoint{3.627169in}{6.197925in}}%
\pgfusepath{stroke,fill}%
\end{pgfscope}%
\begin{pgfscope}%
\pgfpathrectangle{\pgfqpoint{0.380943in}{6.110189in}}{\pgfqpoint{4.650000in}{0.614151in}}%
\pgfusepath{clip}%
\pgfsetbuttcap%
\pgfsetroundjoin%
\definecolor{currentfill}{rgb}{0.965444,0.906113,0.711757}%
\pgfsetfillcolor{currentfill}%
\pgfsetlinewidth{0.250937pt}%
\definecolor{currentstroke}{rgb}{1.000000,1.000000,1.000000}%
\pgfsetstrokecolor{currentstroke}%
\pgfsetdash{}{0pt}%
\pgfpathmoveto{\pgfqpoint{3.714905in}{6.197925in}}%
\pgfpathlineto{\pgfqpoint{3.802641in}{6.197925in}}%
\pgfpathlineto{\pgfqpoint{3.802641in}{6.110189in}}%
\pgfpathlineto{\pgfqpoint{3.714905in}{6.110189in}}%
\pgfpathlineto{\pgfqpoint{3.714905in}{6.197925in}}%
\pgfusepath{stroke,fill}%
\end{pgfscope}%
\begin{pgfscope}%
\pgfpathrectangle{\pgfqpoint{0.380943in}{6.110189in}}{\pgfqpoint{4.650000in}{0.614151in}}%
\pgfusepath{clip}%
\pgfsetbuttcap%
\pgfsetroundjoin%
\definecolor{currentfill}{rgb}{0.991849,0.986144,0.810181}%
\pgfsetfillcolor{currentfill}%
\pgfsetlinewidth{0.250937pt}%
\definecolor{currentstroke}{rgb}{1.000000,1.000000,1.000000}%
\pgfsetstrokecolor{currentstroke}%
\pgfsetdash{}{0pt}%
\pgfpathmoveto{\pgfqpoint{3.802641in}{6.197925in}}%
\pgfpathlineto{\pgfqpoint{3.890377in}{6.197925in}}%
\pgfpathlineto{\pgfqpoint{3.890377in}{6.110189in}}%
\pgfpathlineto{\pgfqpoint{3.802641in}{6.110189in}}%
\pgfpathlineto{\pgfqpoint{3.802641in}{6.197925in}}%
\pgfusepath{stroke,fill}%
\end{pgfscope}%
\begin{pgfscope}%
\pgfpathrectangle{\pgfqpoint{0.380943in}{6.110189in}}{\pgfqpoint{4.650000in}{0.614151in}}%
\pgfusepath{clip}%
\pgfsetbuttcap%
\pgfsetroundjoin%
\definecolor{currentfill}{rgb}{0.986759,0.806398,0.641200}%
\pgfsetfillcolor{currentfill}%
\pgfsetlinewidth{0.250937pt}%
\definecolor{currentstroke}{rgb}{1.000000,1.000000,1.000000}%
\pgfsetstrokecolor{currentstroke}%
\pgfsetdash{}{0pt}%
\pgfpathmoveto{\pgfqpoint{3.890377in}{6.197925in}}%
\pgfpathlineto{\pgfqpoint{3.978113in}{6.197925in}}%
\pgfpathlineto{\pgfqpoint{3.978113in}{6.110189in}}%
\pgfpathlineto{\pgfqpoint{3.890377in}{6.110189in}}%
\pgfpathlineto{\pgfqpoint{3.890377in}{6.197925in}}%
\pgfusepath{stroke,fill}%
\end{pgfscope}%
\begin{pgfscope}%
\pgfpathrectangle{\pgfqpoint{0.380943in}{6.110189in}}{\pgfqpoint{4.650000in}{0.614151in}}%
\pgfusepath{clip}%
\pgfsetbuttcap%
\pgfsetroundjoin%
\definecolor{currentfill}{rgb}{0.965444,0.906113,0.711757}%
\pgfsetfillcolor{currentfill}%
\pgfsetlinewidth{0.250937pt}%
\definecolor{currentstroke}{rgb}{1.000000,1.000000,1.000000}%
\pgfsetstrokecolor{currentstroke}%
\pgfsetdash{}{0pt}%
\pgfpathmoveto{\pgfqpoint{3.978113in}{6.197925in}}%
\pgfpathlineto{\pgfqpoint{4.065849in}{6.197925in}}%
\pgfpathlineto{\pgfqpoint{4.065849in}{6.110189in}}%
\pgfpathlineto{\pgfqpoint{3.978113in}{6.110189in}}%
\pgfpathlineto{\pgfqpoint{3.978113in}{6.197925in}}%
\pgfusepath{stroke,fill}%
\end{pgfscope}%
\begin{pgfscope}%
\pgfpathrectangle{\pgfqpoint{0.380943in}{6.110189in}}{\pgfqpoint{4.650000in}{0.614151in}}%
\pgfusepath{clip}%
\pgfsetbuttcap%
\pgfsetroundjoin%
\definecolor{currentfill}{rgb}{0.968166,0.945882,0.748604}%
\pgfsetfillcolor{currentfill}%
\pgfsetlinewidth{0.250937pt}%
\definecolor{currentstroke}{rgb}{1.000000,1.000000,1.000000}%
\pgfsetstrokecolor{currentstroke}%
\pgfsetdash{}{0pt}%
\pgfpathmoveto{\pgfqpoint{4.065849in}{6.197925in}}%
\pgfpathlineto{\pgfqpoint{4.153585in}{6.197925in}}%
\pgfpathlineto{\pgfqpoint{4.153585in}{6.110189in}}%
\pgfpathlineto{\pgfqpoint{4.065849in}{6.110189in}}%
\pgfpathlineto{\pgfqpoint{4.065849in}{6.197925in}}%
\pgfusepath{stroke,fill}%
\end{pgfscope}%
\begin{pgfscope}%
\pgfpathrectangle{\pgfqpoint{0.380943in}{6.110189in}}{\pgfqpoint{4.650000in}{0.614151in}}%
\pgfusepath{clip}%
\pgfsetbuttcap%
\pgfsetroundjoin%
\definecolor{currentfill}{rgb}{0.962414,0.923552,0.722891}%
\pgfsetfillcolor{currentfill}%
\pgfsetlinewidth{0.250937pt}%
\definecolor{currentstroke}{rgb}{1.000000,1.000000,1.000000}%
\pgfsetstrokecolor{currentstroke}%
\pgfsetdash{}{0pt}%
\pgfpathmoveto{\pgfqpoint{4.153585in}{6.197925in}}%
\pgfpathlineto{\pgfqpoint{4.241320in}{6.197925in}}%
\pgfpathlineto{\pgfqpoint{4.241320in}{6.110189in}}%
\pgfpathlineto{\pgfqpoint{4.153585in}{6.110189in}}%
\pgfpathlineto{\pgfqpoint{4.153585in}{6.197925in}}%
\pgfusepath{stroke,fill}%
\end{pgfscope}%
\begin{pgfscope}%
\pgfpathrectangle{\pgfqpoint{0.380943in}{6.110189in}}{\pgfqpoint{4.650000in}{0.614151in}}%
\pgfusepath{clip}%
\pgfsetbuttcap%
\pgfsetroundjoin%
\definecolor{currentfill}{rgb}{1.000000,1.000000,0.870204}%
\pgfsetfillcolor{currentfill}%
\pgfsetlinewidth{0.250937pt}%
\definecolor{currentstroke}{rgb}{1.000000,1.000000,1.000000}%
\pgfsetstrokecolor{currentstroke}%
\pgfsetdash{}{0pt}%
\pgfpathmoveto{\pgfqpoint{4.241320in}{6.197925in}}%
\pgfpathlineto{\pgfqpoint{4.329056in}{6.197925in}}%
\pgfpathlineto{\pgfqpoint{4.329056in}{6.110189in}}%
\pgfpathlineto{\pgfqpoint{4.241320in}{6.110189in}}%
\pgfpathlineto{\pgfqpoint{4.241320in}{6.197925in}}%
\pgfusepath{stroke,fill}%
\end{pgfscope}%
\begin{pgfscope}%
\pgfpathrectangle{\pgfqpoint{0.380943in}{6.110189in}}{\pgfqpoint{4.650000in}{0.614151in}}%
\pgfusepath{clip}%
\pgfsetbuttcap%
\pgfsetroundjoin%
\definecolor{currentfill}{rgb}{0.991849,0.986144,0.810181}%
\pgfsetfillcolor{currentfill}%
\pgfsetlinewidth{0.250937pt}%
\definecolor{currentstroke}{rgb}{1.000000,1.000000,1.000000}%
\pgfsetstrokecolor{currentstroke}%
\pgfsetdash{}{0pt}%
\pgfpathmoveto{\pgfqpoint{4.329056in}{6.197925in}}%
\pgfpathlineto{\pgfqpoint{4.416792in}{6.197925in}}%
\pgfpathlineto{\pgfqpoint{4.416792in}{6.110189in}}%
\pgfpathlineto{\pgfqpoint{4.329056in}{6.110189in}}%
\pgfpathlineto{\pgfqpoint{4.329056in}{6.197925in}}%
\pgfusepath{stroke,fill}%
\end{pgfscope}%
\begin{pgfscope}%
\pgfpathrectangle{\pgfqpoint{0.380943in}{6.110189in}}{\pgfqpoint{4.650000in}{0.614151in}}%
\pgfusepath{clip}%
\pgfsetbuttcap%
\pgfsetroundjoin%
\definecolor{currentfill}{rgb}{0.968166,0.945882,0.748604}%
\pgfsetfillcolor{currentfill}%
\pgfsetlinewidth{0.250937pt}%
\definecolor{currentstroke}{rgb}{1.000000,1.000000,1.000000}%
\pgfsetstrokecolor{currentstroke}%
\pgfsetdash{}{0pt}%
\pgfpathmoveto{\pgfqpoint{4.416792in}{6.197925in}}%
\pgfpathlineto{\pgfqpoint{4.504528in}{6.197925in}}%
\pgfpathlineto{\pgfqpoint{4.504528in}{6.110189in}}%
\pgfpathlineto{\pgfqpoint{4.416792in}{6.110189in}}%
\pgfpathlineto{\pgfqpoint{4.416792in}{6.197925in}}%
\pgfusepath{stroke,fill}%
\end{pgfscope}%
\begin{pgfscope}%
\pgfpathrectangle{\pgfqpoint{0.380943in}{6.110189in}}{\pgfqpoint{4.650000in}{0.614151in}}%
\pgfusepath{clip}%
\pgfsetbuttcap%
\pgfsetroundjoin%
\definecolor{currentfill}{rgb}{0.968166,0.945882,0.748604}%
\pgfsetfillcolor{currentfill}%
\pgfsetlinewidth{0.250937pt}%
\definecolor{currentstroke}{rgb}{1.000000,1.000000,1.000000}%
\pgfsetstrokecolor{currentstroke}%
\pgfsetdash{}{0pt}%
\pgfpathmoveto{\pgfqpoint{4.504528in}{6.197925in}}%
\pgfpathlineto{\pgfqpoint{4.592264in}{6.197925in}}%
\pgfpathlineto{\pgfqpoint{4.592264in}{6.110189in}}%
\pgfpathlineto{\pgfqpoint{4.504528in}{6.110189in}}%
\pgfpathlineto{\pgfqpoint{4.504528in}{6.197925in}}%
\pgfusepath{stroke,fill}%
\end{pgfscope}%
\begin{pgfscope}%
\pgfpathrectangle{\pgfqpoint{0.380943in}{6.110189in}}{\pgfqpoint{4.650000in}{0.614151in}}%
\pgfusepath{clip}%
\pgfsetbuttcap%
\pgfsetroundjoin%
\definecolor{currentfill}{rgb}{0.962414,0.923552,0.722891}%
\pgfsetfillcolor{currentfill}%
\pgfsetlinewidth{0.250937pt}%
\definecolor{currentstroke}{rgb}{1.000000,1.000000,1.000000}%
\pgfsetstrokecolor{currentstroke}%
\pgfsetdash{}{0pt}%
\pgfpathmoveto{\pgfqpoint{4.592264in}{6.197925in}}%
\pgfpathlineto{\pgfqpoint{4.680000in}{6.197925in}}%
\pgfpathlineto{\pgfqpoint{4.680000in}{6.110189in}}%
\pgfpathlineto{\pgfqpoint{4.592264in}{6.110189in}}%
\pgfpathlineto{\pgfqpoint{4.592264in}{6.197925in}}%
\pgfusepath{stroke,fill}%
\end{pgfscope}%
\begin{pgfscope}%
\pgfpathrectangle{\pgfqpoint{0.380943in}{6.110189in}}{\pgfqpoint{4.650000in}{0.614151in}}%
\pgfusepath{clip}%
\pgfsetbuttcap%
\pgfsetroundjoin%
\definecolor{currentfill}{rgb}{1.000000,1.000000,0.870204}%
\pgfsetfillcolor{currentfill}%
\pgfsetlinewidth{0.250937pt}%
\definecolor{currentstroke}{rgb}{1.000000,1.000000,1.000000}%
\pgfsetstrokecolor{currentstroke}%
\pgfsetdash{}{0pt}%
\pgfpathmoveto{\pgfqpoint{4.680000in}{6.197925in}}%
\pgfpathlineto{\pgfqpoint{4.767736in}{6.197925in}}%
\pgfpathlineto{\pgfqpoint{4.767736in}{6.110189in}}%
\pgfpathlineto{\pgfqpoint{4.680000in}{6.110189in}}%
\pgfpathlineto{\pgfqpoint{4.680000in}{6.197925in}}%
\pgfusepath{stroke,fill}%
\end{pgfscope}%
\begin{pgfscope}%
\pgfpathrectangle{\pgfqpoint{0.380943in}{6.110189in}}{\pgfqpoint{4.650000in}{0.614151in}}%
\pgfusepath{clip}%
\pgfsetbuttcap%
\pgfsetroundjoin%
\definecolor{currentfill}{rgb}{1.000000,1.000000,0.929412}%
\pgfsetfillcolor{currentfill}%
\pgfsetlinewidth{0.250937pt}%
\definecolor{currentstroke}{rgb}{1.000000,1.000000,1.000000}%
\pgfsetstrokecolor{currentstroke}%
\pgfsetdash{}{0pt}%
\pgfpathmoveto{\pgfqpoint{4.767736in}{6.197925in}}%
\pgfpathlineto{\pgfqpoint{4.855471in}{6.197925in}}%
\pgfpathlineto{\pgfqpoint{4.855471in}{6.110189in}}%
\pgfpathlineto{\pgfqpoint{4.767736in}{6.110189in}}%
\pgfpathlineto{\pgfqpoint{4.767736in}{6.197925in}}%
\pgfusepath{stroke,fill}%
\end{pgfscope}%
\begin{pgfscope}%
\pgfpathrectangle{\pgfqpoint{0.380943in}{6.110189in}}{\pgfqpoint{4.650000in}{0.614151in}}%
\pgfusepath{clip}%
\pgfsetbuttcap%
\pgfsetroundjoin%
\definecolor{currentfill}{rgb}{0.991849,0.986144,0.810181}%
\pgfsetfillcolor{currentfill}%
\pgfsetlinewidth{0.250937pt}%
\definecolor{currentstroke}{rgb}{1.000000,1.000000,1.000000}%
\pgfsetstrokecolor{currentstroke}%
\pgfsetdash{}{0pt}%
\pgfpathmoveto{\pgfqpoint{4.855471in}{6.197925in}}%
\pgfpathlineto{\pgfqpoint{4.943207in}{6.197925in}}%
\pgfpathlineto{\pgfqpoint{4.943207in}{6.110189in}}%
\pgfpathlineto{\pgfqpoint{4.855471in}{6.110189in}}%
\pgfpathlineto{\pgfqpoint{4.855471in}{6.197925in}}%
\pgfusepath{stroke,fill}%
\end{pgfscope}%
\begin{pgfscope}%
\pgfpathrectangle{\pgfqpoint{0.380943in}{6.110189in}}{\pgfqpoint{4.650000in}{0.614151in}}%
\pgfusepath{clip}%
\pgfsetbuttcap%
\pgfsetroundjoin%
\pgfsetlinewidth{0.250937pt}%
\definecolor{currentstroke}{rgb}{1.000000,1.000000,1.000000}%
\pgfsetstrokecolor{currentstroke}%
\pgfsetdash{}{0pt}%
\pgfpathmoveto{\pgfqpoint{4.943207in}{6.197925in}}%
\pgfpathlineto{\pgfqpoint{5.030943in}{6.197925in}}%
\pgfpathlineto{\pgfqpoint{5.030943in}{6.110189in}}%
\pgfpathlineto{\pgfqpoint{4.943207in}{6.110189in}}%
\pgfpathlineto{\pgfqpoint{4.943207in}{6.197925in}}%
\pgfusepath{stroke}%
\end{pgfscope}%
\begin{pgfscope}%
\pgfsetbuttcap%
\pgfsetroundjoin%
\definecolor{currentfill}{rgb}{0.000000,0.000000,0.000000}%
\pgfsetfillcolor{currentfill}%
\pgfsetlinewidth{0.803000pt}%
\definecolor{currentstroke}{rgb}{0.000000,0.000000,0.000000}%
\pgfsetstrokecolor{currentstroke}%
\pgfsetdash{}{0pt}%
\pgfsys@defobject{currentmarker}{\pgfqpoint{0.000000in}{-0.048611in}}{\pgfqpoint{0.000000in}{0.000000in}}{%
\pgfpathmoveto{\pgfqpoint{0.000000in}{0.000000in}}%
\pgfpathlineto{\pgfqpoint{0.000000in}{-0.048611in}}%
\pgfusepath{stroke,fill}%
}%
\begin{pgfscope}%
\pgfsys@transformshift{0.600283in}{6.110189in}%
\pgfsys@useobject{currentmarker}{}%
\end{pgfscope}%
\end{pgfscope}%
\begin{pgfscope}%
\definecolor{textcolor}{rgb}{0.000000,0.000000,0.000000}%
\pgfsetstrokecolor{textcolor}%
\pgfsetfillcolor{textcolor}%
\pgftext[x=0.600283in,y=6.012967in,,top]{\color{textcolor}\rmfamily\fontsize{8.000000}{9.600000}\selectfont Jan}%
\end{pgfscope}%
\begin{pgfscope}%
\pgfsetbuttcap%
\pgfsetroundjoin%
\definecolor{currentfill}{rgb}{0.000000,0.000000,0.000000}%
\pgfsetfillcolor{currentfill}%
\pgfsetlinewidth{0.803000pt}%
\definecolor{currentstroke}{rgb}{0.000000,0.000000,0.000000}%
\pgfsetstrokecolor{currentstroke}%
\pgfsetdash{}{0pt}%
\pgfsys@defobject{currentmarker}{\pgfqpoint{0.000000in}{-0.048611in}}{\pgfqpoint{0.000000in}{0.000000in}}{%
\pgfpathmoveto{\pgfqpoint{0.000000in}{0.000000in}}%
\pgfpathlineto{\pgfqpoint{0.000000in}{-0.048611in}}%
\pgfusepath{stroke,fill}%
}%
\begin{pgfscope}%
\pgfsys@transformshift{0.951226in}{6.110189in}%
\pgfsys@useobject{currentmarker}{}%
\end{pgfscope}%
\end{pgfscope}%
\begin{pgfscope}%
\definecolor{textcolor}{rgb}{0.000000,0.000000,0.000000}%
\pgfsetstrokecolor{textcolor}%
\pgfsetfillcolor{textcolor}%
\pgftext[x=0.951226in,y=6.012967in,,top]{\color{textcolor}\rmfamily\fontsize{8.000000}{9.600000}\selectfont Feb}%
\end{pgfscope}%
\begin{pgfscope}%
\pgfsetbuttcap%
\pgfsetroundjoin%
\definecolor{currentfill}{rgb}{0.000000,0.000000,0.000000}%
\pgfsetfillcolor{currentfill}%
\pgfsetlinewidth{0.803000pt}%
\definecolor{currentstroke}{rgb}{0.000000,0.000000,0.000000}%
\pgfsetstrokecolor{currentstroke}%
\pgfsetdash{}{0pt}%
\pgfsys@defobject{currentmarker}{\pgfqpoint{0.000000in}{-0.048611in}}{\pgfqpoint{0.000000in}{0.000000in}}{%
\pgfpathmoveto{\pgfqpoint{0.000000in}{0.000000in}}%
\pgfpathlineto{\pgfqpoint{0.000000in}{-0.048611in}}%
\pgfusepath{stroke,fill}%
}%
\begin{pgfscope}%
\pgfsys@transformshift{1.302169in}{6.110189in}%
\pgfsys@useobject{currentmarker}{}%
\end{pgfscope}%
\end{pgfscope}%
\begin{pgfscope}%
\definecolor{textcolor}{rgb}{0.000000,0.000000,0.000000}%
\pgfsetstrokecolor{textcolor}%
\pgfsetfillcolor{textcolor}%
\pgftext[x=1.302169in,y=6.012967in,,top]{\color{textcolor}\rmfamily\fontsize{8.000000}{9.600000}\selectfont Mar}%
\end{pgfscope}%
\begin{pgfscope}%
\pgfsetbuttcap%
\pgfsetroundjoin%
\definecolor{currentfill}{rgb}{0.000000,0.000000,0.000000}%
\pgfsetfillcolor{currentfill}%
\pgfsetlinewidth{0.803000pt}%
\definecolor{currentstroke}{rgb}{0.000000,0.000000,0.000000}%
\pgfsetstrokecolor{currentstroke}%
\pgfsetdash{}{0pt}%
\pgfsys@defobject{currentmarker}{\pgfqpoint{0.000000in}{-0.048611in}}{\pgfqpoint{0.000000in}{0.000000in}}{%
\pgfpathmoveto{\pgfqpoint{0.000000in}{0.000000in}}%
\pgfpathlineto{\pgfqpoint{0.000000in}{-0.048611in}}%
\pgfusepath{stroke,fill}%
}%
\begin{pgfscope}%
\pgfsys@transformshift{1.696981in}{6.110189in}%
\pgfsys@useobject{currentmarker}{}%
\end{pgfscope}%
\end{pgfscope}%
\begin{pgfscope}%
\definecolor{textcolor}{rgb}{0.000000,0.000000,0.000000}%
\pgfsetstrokecolor{textcolor}%
\pgfsetfillcolor{textcolor}%
\pgftext[x=1.696981in,y=6.012967in,,top]{\color{textcolor}\rmfamily\fontsize{8.000000}{9.600000}\selectfont Apr}%
\end{pgfscope}%
\begin{pgfscope}%
\pgfsetbuttcap%
\pgfsetroundjoin%
\definecolor{currentfill}{rgb}{0.000000,0.000000,0.000000}%
\pgfsetfillcolor{currentfill}%
\pgfsetlinewidth{0.803000pt}%
\definecolor{currentstroke}{rgb}{0.000000,0.000000,0.000000}%
\pgfsetstrokecolor{currentstroke}%
\pgfsetdash{}{0pt}%
\pgfsys@defobject{currentmarker}{\pgfqpoint{0.000000in}{-0.048611in}}{\pgfqpoint{0.000000in}{0.000000in}}{%
\pgfpathmoveto{\pgfqpoint{0.000000in}{0.000000in}}%
\pgfpathlineto{\pgfqpoint{0.000000in}{-0.048611in}}%
\pgfusepath{stroke,fill}%
}%
\begin{pgfscope}%
\pgfsys@transformshift{2.091792in}{6.110189in}%
\pgfsys@useobject{currentmarker}{}%
\end{pgfscope}%
\end{pgfscope}%
\begin{pgfscope}%
\definecolor{textcolor}{rgb}{0.000000,0.000000,0.000000}%
\pgfsetstrokecolor{textcolor}%
\pgfsetfillcolor{textcolor}%
\pgftext[x=2.091792in,y=6.012967in,,top]{\color{textcolor}\rmfamily\fontsize{8.000000}{9.600000}\selectfont May}%
\end{pgfscope}%
\begin{pgfscope}%
\pgfsetbuttcap%
\pgfsetroundjoin%
\definecolor{currentfill}{rgb}{0.000000,0.000000,0.000000}%
\pgfsetfillcolor{currentfill}%
\pgfsetlinewidth{0.803000pt}%
\definecolor{currentstroke}{rgb}{0.000000,0.000000,0.000000}%
\pgfsetstrokecolor{currentstroke}%
\pgfsetdash{}{0pt}%
\pgfsys@defobject{currentmarker}{\pgfqpoint{0.000000in}{-0.048611in}}{\pgfqpoint{0.000000in}{0.000000in}}{%
\pgfpathmoveto{\pgfqpoint{0.000000in}{0.000000in}}%
\pgfpathlineto{\pgfqpoint{0.000000in}{-0.048611in}}%
\pgfusepath{stroke,fill}%
}%
\begin{pgfscope}%
\pgfsys@transformshift{2.442736in}{6.110189in}%
\pgfsys@useobject{currentmarker}{}%
\end{pgfscope}%
\end{pgfscope}%
\begin{pgfscope}%
\definecolor{textcolor}{rgb}{0.000000,0.000000,0.000000}%
\pgfsetstrokecolor{textcolor}%
\pgfsetfillcolor{textcolor}%
\pgftext[x=2.442736in,y=6.012967in,,top]{\color{textcolor}\rmfamily\fontsize{8.000000}{9.600000}\selectfont Jun}%
\end{pgfscope}%
\begin{pgfscope}%
\pgfsetbuttcap%
\pgfsetroundjoin%
\definecolor{currentfill}{rgb}{0.000000,0.000000,0.000000}%
\pgfsetfillcolor{currentfill}%
\pgfsetlinewidth{0.803000pt}%
\definecolor{currentstroke}{rgb}{0.000000,0.000000,0.000000}%
\pgfsetstrokecolor{currentstroke}%
\pgfsetdash{}{0pt}%
\pgfsys@defobject{currentmarker}{\pgfqpoint{0.000000in}{-0.048611in}}{\pgfqpoint{0.000000in}{0.000000in}}{%
\pgfpathmoveto{\pgfqpoint{0.000000in}{0.000000in}}%
\pgfpathlineto{\pgfqpoint{0.000000in}{-0.048611in}}%
\pgfusepath{stroke,fill}%
}%
\begin{pgfscope}%
\pgfsys@transformshift{2.837547in}{6.110189in}%
\pgfsys@useobject{currentmarker}{}%
\end{pgfscope}%
\end{pgfscope}%
\begin{pgfscope}%
\definecolor{textcolor}{rgb}{0.000000,0.000000,0.000000}%
\pgfsetstrokecolor{textcolor}%
\pgfsetfillcolor{textcolor}%
\pgftext[x=2.837547in,y=6.012967in,,top]{\color{textcolor}\rmfamily\fontsize{8.000000}{9.600000}\selectfont Jul}%
\end{pgfscope}%
\begin{pgfscope}%
\pgfsetbuttcap%
\pgfsetroundjoin%
\definecolor{currentfill}{rgb}{0.000000,0.000000,0.000000}%
\pgfsetfillcolor{currentfill}%
\pgfsetlinewidth{0.803000pt}%
\definecolor{currentstroke}{rgb}{0.000000,0.000000,0.000000}%
\pgfsetstrokecolor{currentstroke}%
\pgfsetdash{}{0pt}%
\pgfsys@defobject{currentmarker}{\pgfqpoint{0.000000in}{-0.048611in}}{\pgfqpoint{0.000000in}{0.000000in}}{%
\pgfpathmoveto{\pgfqpoint{0.000000in}{0.000000in}}%
\pgfpathlineto{\pgfqpoint{0.000000in}{-0.048611in}}%
\pgfusepath{stroke,fill}%
}%
\begin{pgfscope}%
\pgfsys@transformshift{3.232358in}{6.110189in}%
\pgfsys@useobject{currentmarker}{}%
\end{pgfscope}%
\end{pgfscope}%
\begin{pgfscope}%
\definecolor{textcolor}{rgb}{0.000000,0.000000,0.000000}%
\pgfsetstrokecolor{textcolor}%
\pgfsetfillcolor{textcolor}%
\pgftext[x=3.232358in,y=6.012967in,,top]{\color{textcolor}\rmfamily\fontsize{8.000000}{9.600000}\selectfont Aug}%
\end{pgfscope}%
\begin{pgfscope}%
\pgfsetbuttcap%
\pgfsetroundjoin%
\definecolor{currentfill}{rgb}{0.000000,0.000000,0.000000}%
\pgfsetfillcolor{currentfill}%
\pgfsetlinewidth{0.803000pt}%
\definecolor{currentstroke}{rgb}{0.000000,0.000000,0.000000}%
\pgfsetstrokecolor{currentstroke}%
\pgfsetdash{}{0pt}%
\pgfsys@defobject{currentmarker}{\pgfqpoint{0.000000in}{-0.048611in}}{\pgfqpoint{0.000000in}{0.000000in}}{%
\pgfpathmoveto{\pgfqpoint{0.000000in}{0.000000in}}%
\pgfpathlineto{\pgfqpoint{0.000000in}{-0.048611in}}%
\pgfusepath{stroke,fill}%
}%
\begin{pgfscope}%
\pgfsys@transformshift{3.583302in}{6.110189in}%
\pgfsys@useobject{currentmarker}{}%
\end{pgfscope}%
\end{pgfscope}%
\begin{pgfscope}%
\definecolor{textcolor}{rgb}{0.000000,0.000000,0.000000}%
\pgfsetstrokecolor{textcolor}%
\pgfsetfillcolor{textcolor}%
\pgftext[x=3.583302in,y=6.012967in,,top]{\color{textcolor}\rmfamily\fontsize{8.000000}{9.600000}\selectfont Sep}%
\end{pgfscope}%
\begin{pgfscope}%
\pgfsetbuttcap%
\pgfsetroundjoin%
\definecolor{currentfill}{rgb}{0.000000,0.000000,0.000000}%
\pgfsetfillcolor{currentfill}%
\pgfsetlinewidth{0.803000pt}%
\definecolor{currentstroke}{rgb}{0.000000,0.000000,0.000000}%
\pgfsetstrokecolor{currentstroke}%
\pgfsetdash{}{0pt}%
\pgfsys@defobject{currentmarker}{\pgfqpoint{0.000000in}{-0.048611in}}{\pgfqpoint{0.000000in}{0.000000in}}{%
\pgfpathmoveto{\pgfqpoint{0.000000in}{0.000000in}}%
\pgfpathlineto{\pgfqpoint{0.000000in}{-0.048611in}}%
\pgfusepath{stroke,fill}%
}%
\begin{pgfscope}%
\pgfsys@transformshift{4.021981in}{6.110189in}%
\pgfsys@useobject{currentmarker}{}%
\end{pgfscope}%
\end{pgfscope}%
\begin{pgfscope}%
\definecolor{textcolor}{rgb}{0.000000,0.000000,0.000000}%
\pgfsetstrokecolor{textcolor}%
\pgfsetfillcolor{textcolor}%
\pgftext[x=4.021981in,y=6.012967in,,top]{\color{textcolor}\rmfamily\fontsize{8.000000}{9.600000}\selectfont Oct}%
\end{pgfscope}%
\begin{pgfscope}%
\pgfsetbuttcap%
\pgfsetroundjoin%
\definecolor{currentfill}{rgb}{0.000000,0.000000,0.000000}%
\pgfsetfillcolor{currentfill}%
\pgfsetlinewidth{0.803000pt}%
\definecolor{currentstroke}{rgb}{0.000000,0.000000,0.000000}%
\pgfsetstrokecolor{currentstroke}%
\pgfsetdash{}{0pt}%
\pgfsys@defobject{currentmarker}{\pgfqpoint{0.000000in}{-0.048611in}}{\pgfqpoint{0.000000in}{0.000000in}}{%
\pgfpathmoveto{\pgfqpoint{0.000000in}{0.000000in}}%
\pgfpathlineto{\pgfqpoint{0.000000in}{-0.048611in}}%
\pgfusepath{stroke,fill}%
}%
\begin{pgfscope}%
\pgfsys@transformshift{4.372924in}{6.110189in}%
\pgfsys@useobject{currentmarker}{}%
\end{pgfscope}%
\end{pgfscope}%
\begin{pgfscope}%
\definecolor{textcolor}{rgb}{0.000000,0.000000,0.000000}%
\pgfsetstrokecolor{textcolor}%
\pgfsetfillcolor{textcolor}%
\pgftext[x=4.372924in,y=6.012967in,,top]{\color{textcolor}\rmfamily\fontsize{8.000000}{9.600000}\selectfont Nov}%
\end{pgfscope}%
\begin{pgfscope}%
\pgfsetbuttcap%
\pgfsetroundjoin%
\definecolor{currentfill}{rgb}{0.000000,0.000000,0.000000}%
\pgfsetfillcolor{currentfill}%
\pgfsetlinewidth{0.803000pt}%
\definecolor{currentstroke}{rgb}{0.000000,0.000000,0.000000}%
\pgfsetstrokecolor{currentstroke}%
\pgfsetdash{}{0pt}%
\pgfsys@defobject{currentmarker}{\pgfqpoint{0.000000in}{-0.048611in}}{\pgfqpoint{0.000000in}{0.000000in}}{%
\pgfpathmoveto{\pgfqpoint{0.000000in}{0.000000in}}%
\pgfpathlineto{\pgfqpoint{0.000000in}{-0.048611in}}%
\pgfusepath{stroke,fill}%
}%
\begin{pgfscope}%
\pgfsys@transformshift{4.767736in}{6.110189in}%
\pgfsys@useobject{currentmarker}{}%
\end{pgfscope}%
\end{pgfscope}%
\begin{pgfscope}%
\definecolor{textcolor}{rgb}{0.000000,0.000000,0.000000}%
\pgfsetstrokecolor{textcolor}%
\pgfsetfillcolor{textcolor}%
\pgftext[x=4.767736in,y=6.012967in,,top]{\color{textcolor}\rmfamily\fontsize{8.000000}{9.600000}\selectfont Dec}%
\end{pgfscope}%
\begin{pgfscope}%
\pgfsetbuttcap%
\pgfsetroundjoin%
\definecolor{currentfill}{rgb}{0.000000,0.000000,0.000000}%
\pgfsetfillcolor{currentfill}%
\pgfsetlinewidth{0.803000pt}%
\definecolor{currentstroke}{rgb}{0.000000,0.000000,0.000000}%
\pgfsetstrokecolor{currentstroke}%
\pgfsetdash{}{0pt}%
\pgfsys@defobject{currentmarker}{\pgfqpoint{-0.048611in}{0.000000in}}{\pgfqpoint{-0.000000in}{0.000000in}}{%
\pgfpathmoveto{\pgfqpoint{-0.000000in}{0.000000in}}%
\pgfpathlineto{\pgfqpoint{-0.048611in}{0.000000in}}%
\pgfusepath{stroke,fill}%
}%
\begin{pgfscope}%
\pgfsys@transformshift{0.380943in}{6.680472in}%
\pgfsys@useobject{currentmarker}{}%
\end{pgfscope}%
\end{pgfscope}%
\begin{pgfscope}%
\definecolor{textcolor}{rgb}{0.000000,0.000000,0.000000}%
\pgfsetstrokecolor{textcolor}%
\pgfsetfillcolor{textcolor}%
\pgftext[x=0.113117in, y=6.641892in, left, base]{\color{textcolor}\rmfamily\fontsize{8.000000}{9.600000}\selectfont M}%
\end{pgfscope}%
\begin{pgfscope}%
\pgfsetbuttcap%
\pgfsetroundjoin%
\definecolor{currentfill}{rgb}{0.000000,0.000000,0.000000}%
\pgfsetfillcolor{currentfill}%
\pgfsetlinewidth{0.803000pt}%
\definecolor{currentstroke}{rgb}{0.000000,0.000000,0.000000}%
\pgfsetstrokecolor{currentstroke}%
\pgfsetdash{}{0pt}%
\pgfsys@defobject{currentmarker}{\pgfqpoint{-0.048611in}{0.000000in}}{\pgfqpoint{-0.000000in}{0.000000in}}{%
\pgfpathmoveto{\pgfqpoint{-0.000000in}{0.000000in}}%
\pgfpathlineto{\pgfqpoint{-0.048611in}{0.000000in}}%
\pgfusepath{stroke,fill}%
}%
\begin{pgfscope}%
\pgfsys@transformshift{0.380943in}{6.592736in}%
\pgfsys@useobject{currentmarker}{}%
\end{pgfscope}%
\end{pgfscope}%
\begin{pgfscope}%
\definecolor{textcolor}{rgb}{0.000000,0.000000,0.000000}%
\pgfsetstrokecolor{textcolor}%
\pgfsetfillcolor{textcolor}%
\pgftext[x=0.135957in, y=6.554156in, left, base]{\color{textcolor}\rmfamily\fontsize{8.000000}{9.600000}\selectfont T}%
\end{pgfscope}%
\begin{pgfscope}%
\pgfsetbuttcap%
\pgfsetroundjoin%
\definecolor{currentfill}{rgb}{0.000000,0.000000,0.000000}%
\pgfsetfillcolor{currentfill}%
\pgfsetlinewidth{0.803000pt}%
\definecolor{currentstroke}{rgb}{0.000000,0.000000,0.000000}%
\pgfsetstrokecolor{currentstroke}%
\pgfsetdash{}{0pt}%
\pgfsys@defobject{currentmarker}{\pgfqpoint{-0.048611in}{0.000000in}}{\pgfqpoint{-0.000000in}{0.000000in}}{%
\pgfpathmoveto{\pgfqpoint{-0.000000in}{0.000000in}}%
\pgfpathlineto{\pgfqpoint{-0.048611in}{0.000000in}}%
\pgfusepath{stroke,fill}%
}%
\begin{pgfscope}%
\pgfsys@transformshift{0.380943in}{6.505000in}%
\pgfsys@useobject{currentmarker}{}%
\end{pgfscope}%
\end{pgfscope}%
\begin{pgfscope}%
\definecolor{textcolor}{rgb}{0.000000,0.000000,0.000000}%
\pgfsetstrokecolor{textcolor}%
\pgfsetfillcolor{textcolor}%
\pgftext[x=0.100000in, y=6.466420in, left, base]{\color{textcolor}\rmfamily\fontsize{8.000000}{9.600000}\selectfont W}%
\end{pgfscope}%
\begin{pgfscope}%
\pgfsetbuttcap%
\pgfsetroundjoin%
\definecolor{currentfill}{rgb}{0.000000,0.000000,0.000000}%
\pgfsetfillcolor{currentfill}%
\pgfsetlinewidth{0.803000pt}%
\definecolor{currentstroke}{rgb}{0.000000,0.000000,0.000000}%
\pgfsetstrokecolor{currentstroke}%
\pgfsetdash{}{0pt}%
\pgfsys@defobject{currentmarker}{\pgfqpoint{-0.048611in}{0.000000in}}{\pgfqpoint{-0.000000in}{0.000000in}}{%
\pgfpathmoveto{\pgfqpoint{-0.000000in}{0.000000in}}%
\pgfpathlineto{\pgfqpoint{-0.048611in}{0.000000in}}%
\pgfusepath{stroke,fill}%
}%
\begin{pgfscope}%
\pgfsys@transformshift{0.380943in}{6.417264in}%
\pgfsys@useobject{currentmarker}{}%
\end{pgfscope}%
\end{pgfscope}%
\begin{pgfscope}%
\definecolor{textcolor}{rgb}{0.000000,0.000000,0.000000}%
\pgfsetstrokecolor{textcolor}%
\pgfsetfillcolor{textcolor}%
\pgftext[x=0.135957in, y=6.378684in, left, base]{\color{textcolor}\rmfamily\fontsize{8.000000}{9.600000}\selectfont T}%
\end{pgfscope}%
\begin{pgfscope}%
\pgfsetbuttcap%
\pgfsetroundjoin%
\definecolor{currentfill}{rgb}{0.000000,0.000000,0.000000}%
\pgfsetfillcolor{currentfill}%
\pgfsetlinewidth{0.803000pt}%
\definecolor{currentstroke}{rgb}{0.000000,0.000000,0.000000}%
\pgfsetstrokecolor{currentstroke}%
\pgfsetdash{}{0pt}%
\pgfsys@defobject{currentmarker}{\pgfqpoint{-0.048611in}{0.000000in}}{\pgfqpoint{-0.000000in}{0.000000in}}{%
\pgfpathmoveto{\pgfqpoint{-0.000000in}{0.000000in}}%
\pgfpathlineto{\pgfqpoint{-0.048611in}{0.000000in}}%
\pgfusepath{stroke,fill}%
}%
\begin{pgfscope}%
\pgfsys@transformshift{0.380943in}{6.329529in}%
\pgfsys@useobject{currentmarker}{}%
\end{pgfscope}%
\end{pgfscope}%
\begin{pgfscope}%
\definecolor{textcolor}{rgb}{0.000000,0.000000,0.000000}%
\pgfsetstrokecolor{textcolor}%
\pgfsetfillcolor{textcolor}%
\pgftext[x=0.144213in, y=6.290948in, left, base]{\color{textcolor}\rmfamily\fontsize{8.000000}{9.600000}\selectfont F}%
\end{pgfscope}%
\begin{pgfscope}%
\pgfsetbuttcap%
\pgfsetroundjoin%
\definecolor{currentfill}{rgb}{0.000000,0.000000,0.000000}%
\pgfsetfillcolor{currentfill}%
\pgfsetlinewidth{0.803000pt}%
\definecolor{currentstroke}{rgb}{0.000000,0.000000,0.000000}%
\pgfsetstrokecolor{currentstroke}%
\pgfsetdash{}{0pt}%
\pgfsys@defobject{currentmarker}{\pgfqpoint{-0.048611in}{0.000000in}}{\pgfqpoint{-0.000000in}{0.000000in}}{%
\pgfpathmoveto{\pgfqpoint{-0.000000in}{0.000000in}}%
\pgfpathlineto{\pgfqpoint{-0.048611in}{0.000000in}}%
\pgfusepath{stroke,fill}%
}%
\begin{pgfscope}%
\pgfsys@transformshift{0.380943in}{6.241793in}%
\pgfsys@useobject{currentmarker}{}%
\end{pgfscope}%
\end{pgfscope}%
\begin{pgfscope}%
\definecolor{textcolor}{rgb}{0.000000,0.000000,0.000000}%
\pgfsetstrokecolor{textcolor}%
\pgfsetfillcolor{textcolor}%
\pgftext[x=0.155633in, y=6.203212in, left, base]{\color{textcolor}\rmfamily\fontsize{8.000000}{9.600000}\selectfont S}%
\end{pgfscope}%
\begin{pgfscope}%
\pgfsetbuttcap%
\pgfsetroundjoin%
\definecolor{currentfill}{rgb}{0.000000,0.000000,0.000000}%
\pgfsetfillcolor{currentfill}%
\pgfsetlinewidth{0.803000pt}%
\definecolor{currentstroke}{rgb}{0.000000,0.000000,0.000000}%
\pgfsetstrokecolor{currentstroke}%
\pgfsetdash{}{0pt}%
\pgfsys@defobject{currentmarker}{\pgfqpoint{-0.048611in}{0.000000in}}{\pgfqpoint{-0.000000in}{0.000000in}}{%
\pgfpathmoveto{\pgfqpoint{-0.000000in}{0.000000in}}%
\pgfpathlineto{\pgfqpoint{-0.048611in}{0.000000in}}%
\pgfusepath{stroke,fill}%
}%
\begin{pgfscope}%
\pgfsys@transformshift{0.380943in}{6.154057in}%
\pgfsys@useobject{currentmarker}{}%
\end{pgfscope}%
\end{pgfscope}%
\begin{pgfscope}%
\definecolor{textcolor}{rgb}{0.000000,0.000000,0.000000}%
\pgfsetstrokecolor{textcolor}%
\pgfsetfillcolor{textcolor}%
\pgftext[x=0.155633in, y=6.115477in, left, base]{\color{textcolor}\rmfamily\fontsize{8.000000}{9.600000}\selectfont S}%
\end{pgfscope}%
\begin{pgfscope}%
\definecolor{textcolor}{rgb}{0.000000,0.000000,0.000000}%
\pgfsetstrokecolor{textcolor}%
\pgfsetfillcolor{textcolor}%
\pgftext[x=2.705943in,y=6.891007in,,]{\color{textcolor}\ttfamily\fontsize{14.400000}{17.280000}\selectfont 2018}%
\end{pgfscope}%
\begin{pgfscope}%
\pgfpathrectangle{\pgfqpoint{0.380943in}{4.185189in}}{\pgfqpoint{4.650000in}{0.614151in}}%
\pgfusepath{clip}%
\pgfsetbuttcap%
\pgfsetroundjoin%
\pgfsetlinewidth{0.250937pt}%
\definecolor{currentstroke}{rgb}{1.000000,1.000000,1.000000}%
\pgfsetstrokecolor{currentstroke}%
\pgfsetdash{}{0pt}%
\pgfpathmoveto{\pgfqpoint{0.380943in}{4.799340in}}%
\pgfpathlineto{\pgfqpoint{0.468679in}{4.799340in}}%
\pgfpathlineto{\pgfqpoint{0.468679in}{4.711604in}}%
\pgfpathlineto{\pgfqpoint{0.380943in}{4.711604in}}%
\pgfpathlineto{\pgfqpoint{0.380943in}{4.799340in}}%
\pgfusepath{stroke}%
\end{pgfscope}%
\begin{pgfscope}%
\pgfpathrectangle{\pgfqpoint{0.380943in}{4.185189in}}{\pgfqpoint{4.650000in}{0.614151in}}%
\pgfusepath{clip}%
\pgfsetbuttcap%
\pgfsetroundjoin%
\definecolor{currentfill}{rgb}{0.979654,0.837186,0.669619}%
\pgfsetfillcolor{currentfill}%
\pgfsetlinewidth{0.250937pt}%
\definecolor{currentstroke}{rgb}{1.000000,1.000000,1.000000}%
\pgfsetstrokecolor{currentstroke}%
\pgfsetdash{}{0pt}%
\pgfpathmoveto{\pgfqpoint{0.468679in}{4.799340in}}%
\pgfpathlineto{\pgfqpoint{0.556415in}{4.799340in}}%
\pgfpathlineto{\pgfqpoint{0.556415in}{4.711604in}}%
\pgfpathlineto{\pgfqpoint{0.468679in}{4.711604in}}%
\pgfpathlineto{\pgfqpoint{0.468679in}{4.799340in}}%
\pgfusepath{stroke,fill}%
\end{pgfscope}%
\begin{pgfscope}%
\pgfpathrectangle{\pgfqpoint{0.380943in}{4.185189in}}{\pgfqpoint{4.650000in}{0.614151in}}%
\pgfusepath{clip}%
\pgfsetbuttcap%
\pgfsetroundjoin%
\definecolor{currentfill}{rgb}{0.968166,0.945882,0.748604}%
\pgfsetfillcolor{currentfill}%
\pgfsetlinewidth{0.250937pt}%
\definecolor{currentstroke}{rgb}{1.000000,1.000000,1.000000}%
\pgfsetstrokecolor{currentstroke}%
\pgfsetdash{}{0pt}%
\pgfpathmoveto{\pgfqpoint{0.556415in}{4.799340in}}%
\pgfpathlineto{\pgfqpoint{0.644151in}{4.799340in}}%
\pgfpathlineto{\pgfqpoint{0.644151in}{4.711604in}}%
\pgfpathlineto{\pgfqpoint{0.556415in}{4.711604in}}%
\pgfpathlineto{\pgfqpoint{0.556415in}{4.799340in}}%
\pgfusepath{stroke,fill}%
\end{pgfscope}%
\begin{pgfscope}%
\pgfpathrectangle{\pgfqpoint{0.380943in}{4.185189in}}{\pgfqpoint{4.650000in}{0.614151in}}%
\pgfusepath{clip}%
\pgfsetbuttcap%
\pgfsetroundjoin%
\definecolor{currentfill}{rgb}{1.000000,0.557862,0.511772}%
\pgfsetfillcolor{currentfill}%
\pgfsetlinewidth{0.250937pt}%
\definecolor{currentstroke}{rgb}{1.000000,1.000000,1.000000}%
\pgfsetstrokecolor{currentstroke}%
\pgfsetdash{}{0pt}%
\pgfpathmoveto{\pgfqpoint{0.644151in}{4.799340in}}%
\pgfpathlineto{\pgfqpoint{0.731886in}{4.799340in}}%
\pgfpathlineto{\pgfqpoint{0.731886in}{4.711604in}}%
\pgfpathlineto{\pgfqpoint{0.644151in}{4.711604in}}%
\pgfpathlineto{\pgfqpoint{0.644151in}{4.799340in}}%
\pgfusepath{stroke,fill}%
\end{pgfscope}%
\begin{pgfscope}%
\pgfpathrectangle{\pgfqpoint{0.380943in}{4.185189in}}{\pgfqpoint{4.650000in}{0.614151in}}%
\pgfusepath{clip}%
\pgfsetbuttcap%
\pgfsetroundjoin%
\definecolor{currentfill}{rgb}{0.962414,0.923552,0.722891}%
\pgfsetfillcolor{currentfill}%
\pgfsetlinewidth{0.250937pt}%
\definecolor{currentstroke}{rgb}{1.000000,1.000000,1.000000}%
\pgfsetstrokecolor{currentstroke}%
\pgfsetdash{}{0pt}%
\pgfpathmoveto{\pgfqpoint{0.731886in}{4.799340in}}%
\pgfpathlineto{\pgfqpoint{0.819622in}{4.799340in}}%
\pgfpathlineto{\pgfqpoint{0.819622in}{4.711604in}}%
\pgfpathlineto{\pgfqpoint{0.731886in}{4.711604in}}%
\pgfpathlineto{\pgfqpoint{0.731886in}{4.799340in}}%
\pgfusepath{stroke,fill}%
\end{pgfscope}%
\begin{pgfscope}%
\pgfpathrectangle{\pgfqpoint{0.380943in}{4.185189in}}{\pgfqpoint{4.650000in}{0.614151in}}%
\pgfusepath{clip}%
\pgfsetbuttcap%
\pgfsetroundjoin%
\definecolor{currentfill}{rgb}{1.000000,0.557862,0.511772}%
\pgfsetfillcolor{currentfill}%
\pgfsetlinewidth{0.250937pt}%
\definecolor{currentstroke}{rgb}{1.000000,1.000000,1.000000}%
\pgfsetstrokecolor{currentstroke}%
\pgfsetdash{}{0pt}%
\pgfpathmoveto{\pgfqpoint{0.819622in}{4.799340in}}%
\pgfpathlineto{\pgfqpoint{0.907358in}{4.799340in}}%
\pgfpathlineto{\pgfqpoint{0.907358in}{4.711604in}}%
\pgfpathlineto{\pgfqpoint{0.819622in}{4.711604in}}%
\pgfpathlineto{\pgfqpoint{0.819622in}{4.799340in}}%
\pgfusepath{stroke,fill}%
\end{pgfscope}%
\begin{pgfscope}%
\pgfpathrectangle{\pgfqpoint{0.380943in}{4.185189in}}{\pgfqpoint{4.650000in}{0.614151in}}%
\pgfusepath{clip}%
\pgfsetbuttcap%
\pgfsetroundjoin%
\definecolor{currentfill}{rgb}{0.996571,0.720538,0.589189}%
\pgfsetfillcolor{currentfill}%
\pgfsetlinewidth{0.250937pt}%
\definecolor{currentstroke}{rgb}{1.000000,1.000000,1.000000}%
\pgfsetstrokecolor{currentstroke}%
\pgfsetdash{}{0pt}%
\pgfpathmoveto{\pgfqpoint{0.907358in}{4.799340in}}%
\pgfpathlineto{\pgfqpoint{0.995094in}{4.799340in}}%
\pgfpathlineto{\pgfqpoint{0.995094in}{4.711604in}}%
\pgfpathlineto{\pgfqpoint{0.907358in}{4.711604in}}%
\pgfpathlineto{\pgfqpoint{0.907358in}{4.799340in}}%
\pgfusepath{stroke,fill}%
\end{pgfscope}%
\begin{pgfscope}%
\pgfpathrectangle{\pgfqpoint{0.380943in}{4.185189in}}{\pgfqpoint{4.650000in}{0.614151in}}%
\pgfusepath{clip}%
\pgfsetbuttcap%
\pgfsetroundjoin%
\definecolor{currentfill}{rgb}{0.800000,0.278431,0.278431}%
\pgfsetfillcolor{currentfill}%
\pgfsetlinewidth{0.250937pt}%
\definecolor{currentstroke}{rgb}{1.000000,1.000000,1.000000}%
\pgfsetstrokecolor{currentstroke}%
\pgfsetdash{}{0pt}%
\pgfpathmoveto{\pgfqpoint{0.995094in}{4.799340in}}%
\pgfpathlineto{\pgfqpoint{1.082830in}{4.799340in}}%
\pgfpathlineto{\pgfqpoint{1.082830in}{4.711604in}}%
\pgfpathlineto{\pgfqpoint{0.995094in}{4.711604in}}%
\pgfpathlineto{\pgfqpoint{0.995094in}{4.799340in}}%
\pgfusepath{stroke,fill}%
\end{pgfscope}%
\begin{pgfscope}%
\pgfpathrectangle{\pgfqpoint{0.380943in}{4.185189in}}{\pgfqpoint{4.650000in}{0.614151in}}%
\pgfusepath{clip}%
\pgfsetbuttcap%
\pgfsetroundjoin%
\definecolor{currentfill}{rgb}{1.000000,0.605229,0.530719}%
\pgfsetfillcolor{currentfill}%
\pgfsetlinewidth{0.250937pt}%
\definecolor{currentstroke}{rgb}{1.000000,1.000000,1.000000}%
\pgfsetstrokecolor{currentstroke}%
\pgfsetdash{}{0pt}%
\pgfpathmoveto{\pgfqpoint{1.082830in}{4.799340in}}%
\pgfpathlineto{\pgfqpoint{1.170566in}{4.799340in}}%
\pgfpathlineto{\pgfqpoint{1.170566in}{4.711604in}}%
\pgfpathlineto{\pgfqpoint{1.082830in}{4.711604in}}%
\pgfpathlineto{\pgfqpoint{1.082830in}{4.799340in}}%
\pgfusepath{stroke,fill}%
\end{pgfscope}%
\begin{pgfscope}%
\pgfpathrectangle{\pgfqpoint{0.380943in}{4.185189in}}{\pgfqpoint{4.650000in}{0.614151in}}%
\pgfusepath{clip}%
\pgfsetbuttcap%
\pgfsetroundjoin%
\definecolor{currentfill}{rgb}{0.981546,0.459977,0.459977}%
\pgfsetfillcolor{currentfill}%
\pgfsetlinewidth{0.250937pt}%
\definecolor{currentstroke}{rgb}{1.000000,1.000000,1.000000}%
\pgfsetstrokecolor{currentstroke}%
\pgfsetdash{}{0pt}%
\pgfpathmoveto{\pgfqpoint{1.170566in}{4.799340in}}%
\pgfpathlineto{\pgfqpoint{1.258302in}{4.799340in}}%
\pgfpathlineto{\pgfqpoint{1.258302in}{4.711604in}}%
\pgfpathlineto{\pgfqpoint{1.170566in}{4.711604in}}%
\pgfpathlineto{\pgfqpoint{1.170566in}{4.799340in}}%
\pgfusepath{stroke,fill}%
\end{pgfscope}%
\begin{pgfscope}%
\pgfpathrectangle{\pgfqpoint{0.380943in}{4.185189in}}{\pgfqpoint{4.650000in}{0.614151in}}%
\pgfusepath{clip}%
\pgfsetbuttcap%
\pgfsetroundjoin%
\definecolor{currentfill}{rgb}{0.979654,0.837186,0.669619}%
\pgfsetfillcolor{currentfill}%
\pgfsetlinewidth{0.250937pt}%
\definecolor{currentstroke}{rgb}{1.000000,1.000000,1.000000}%
\pgfsetstrokecolor{currentstroke}%
\pgfsetdash{}{0pt}%
\pgfpathmoveto{\pgfqpoint{1.258302in}{4.799340in}}%
\pgfpathlineto{\pgfqpoint{1.346037in}{4.799340in}}%
\pgfpathlineto{\pgfqpoint{1.346037in}{4.711604in}}%
\pgfpathlineto{\pgfqpoint{1.258302in}{4.711604in}}%
\pgfpathlineto{\pgfqpoint{1.258302in}{4.799340in}}%
\pgfusepath{stroke,fill}%
\end{pgfscope}%
\begin{pgfscope}%
\pgfpathrectangle{\pgfqpoint{0.380943in}{4.185189in}}{\pgfqpoint{4.650000in}{0.614151in}}%
\pgfusepath{clip}%
\pgfsetbuttcap%
\pgfsetroundjoin%
\definecolor{currentfill}{rgb}{0.992326,0.765229,0.614840}%
\pgfsetfillcolor{currentfill}%
\pgfsetlinewidth{0.250937pt}%
\definecolor{currentstroke}{rgb}{1.000000,1.000000,1.000000}%
\pgfsetstrokecolor{currentstroke}%
\pgfsetdash{}{0pt}%
\pgfpathmoveto{\pgfqpoint{1.346037in}{4.799340in}}%
\pgfpathlineto{\pgfqpoint{1.433773in}{4.799340in}}%
\pgfpathlineto{\pgfqpoint{1.433773in}{4.711604in}}%
\pgfpathlineto{\pgfqpoint{1.346037in}{4.711604in}}%
\pgfpathlineto{\pgfqpoint{1.346037in}{4.799340in}}%
\pgfusepath{stroke,fill}%
\end{pgfscope}%
\begin{pgfscope}%
\pgfpathrectangle{\pgfqpoint{0.380943in}{4.185189in}}{\pgfqpoint{4.650000in}{0.614151in}}%
\pgfusepath{clip}%
\pgfsetbuttcap%
\pgfsetroundjoin%
\definecolor{currentfill}{rgb}{0.998939,0.658962,0.556032}%
\pgfsetfillcolor{currentfill}%
\pgfsetlinewidth{0.250937pt}%
\definecolor{currentstroke}{rgb}{1.000000,1.000000,1.000000}%
\pgfsetstrokecolor{currentstroke}%
\pgfsetdash{}{0pt}%
\pgfpathmoveto{\pgfqpoint{1.433773in}{4.799340in}}%
\pgfpathlineto{\pgfqpoint{1.521509in}{4.799340in}}%
\pgfpathlineto{\pgfqpoint{1.521509in}{4.711604in}}%
\pgfpathlineto{\pgfqpoint{1.433773in}{4.711604in}}%
\pgfpathlineto{\pgfqpoint{1.433773in}{4.799340in}}%
\pgfusepath{stroke,fill}%
\end{pgfscope}%
\begin{pgfscope}%
\pgfpathrectangle{\pgfqpoint{0.380943in}{4.185189in}}{\pgfqpoint{4.650000in}{0.614151in}}%
\pgfusepath{clip}%
\pgfsetbuttcap%
\pgfsetroundjoin%
\definecolor{currentfill}{rgb}{0.979654,0.837186,0.669619}%
\pgfsetfillcolor{currentfill}%
\pgfsetlinewidth{0.250937pt}%
\definecolor{currentstroke}{rgb}{1.000000,1.000000,1.000000}%
\pgfsetstrokecolor{currentstroke}%
\pgfsetdash{}{0pt}%
\pgfpathmoveto{\pgfqpoint{1.521509in}{4.799340in}}%
\pgfpathlineto{\pgfqpoint{1.609245in}{4.799340in}}%
\pgfpathlineto{\pgfqpoint{1.609245in}{4.711604in}}%
\pgfpathlineto{\pgfqpoint{1.521509in}{4.711604in}}%
\pgfpathlineto{\pgfqpoint{1.521509in}{4.799340in}}%
\pgfusepath{stroke,fill}%
\end{pgfscope}%
\begin{pgfscope}%
\pgfpathrectangle{\pgfqpoint{0.380943in}{4.185189in}}{\pgfqpoint{4.650000in}{0.614151in}}%
\pgfusepath{clip}%
\pgfsetbuttcap%
\pgfsetroundjoin%
\definecolor{currentfill}{rgb}{0.979654,0.837186,0.669619}%
\pgfsetfillcolor{currentfill}%
\pgfsetlinewidth{0.250937pt}%
\definecolor{currentstroke}{rgb}{1.000000,1.000000,1.000000}%
\pgfsetstrokecolor{currentstroke}%
\pgfsetdash{}{0pt}%
\pgfpathmoveto{\pgfqpoint{1.609245in}{4.799340in}}%
\pgfpathlineto{\pgfqpoint{1.696981in}{4.799340in}}%
\pgfpathlineto{\pgfqpoint{1.696981in}{4.711604in}}%
\pgfpathlineto{\pgfqpoint{1.609245in}{4.711604in}}%
\pgfpathlineto{\pgfqpoint{1.609245in}{4.799340in}}%
\pgfusepath{stroke,fill}%
\end{pgfscope}%
\begin{pgfscope}%
\pgfpathrectangle{\pgfqpoint{0.380943in}{4.185189in}}{\pgfqpoint{4.650000in}{0.614151in}}%
\pgfusepath{clip}%
\pgfsetbuttcap%
\pgfsetroundjoin%
\definecolor{currentfill}{rgb}{0.996571,0.720538,0.589189}%
\pgfsetfillcolor{currentfill}%
\pgfsetlinewidth{0.250937pt}%
\definecolor{currentstroke}{rgb}{1.000000,1.000000,1.000000}%
\pgfsetstrokecolor{currentstroke}%
\pgfsetdash{}{0pt}%
\pgfpathmoveto{\pgfqpoint{1.696981in}{4.799340in}}%
\pgfpathlineto{\pgfqpoint{1.784717in}{4.799340in}}%
\pgfpathlineto{\pgfqpoint{1.784717in}{4.711604in}}%
\pgfpathlineto{\pgfqpoint{1.696981in}{4.711604in}}%
\pgfpathlineto{\pgfqpoint{1.696981in}{4.799340in}}%
\pgfusepath{stroke,fill}%
\end{pgfscope}%
\begin{pgfscope}%
\pgfpathrectangle{\pgfqpoint{0.380943in}{4.185189in}}{\pgfqpoint{4.650000in}{0.614151in}}%
\pgfusepath{clip}%
\pgfsetbuttcap%
\pgfsetroundjoin%
\definecolor{currentfill}{rgb}{0.965444,0.906113,0.711757}%
\pgfsetfillcolor{currentfill}%
\pgfsetlinewidth{0.250937pt}%
\definecolor{currentstroke}{rgb}{1.000000,1.000000,1.000000}%
\pgfsetstrokecolor{currentstroke}%
\pgfsetdash{}{0pt}%
\pgfpathmoveto{\pgfqpoint{1.784717in}{4.799340in}}%
\pgfpathlineto{\pgfqpoint{1.872452in}{4.799340in}}%
\pgfpathlineto{\pgfqpoint{1.872452in}{4.711604in}}%
\pgfpathlineto{\pgfqpoint{1.784717in}{4.711604in}}%
\pgfpathlineto{\pgfqpoint{1.784717in}{4.799340in}}%
\pgfusepath{stroke,fill}%
\end{pgfscope}%
\begin{pgfscope}%
\pgfpathrectangle{\pgfqpoint{0.380943in}{4.185189in}}{\pgfqpoint{4.650000in}{0.614151in}}%
\pgfusepath{clip}%
\pgfsetbuttcap%
\pgfsetroundjoin%
\definecolor{currentfill}{rgb}{0.996571,0.720538,0.589189}%
\pgfsetfillcolor{currentfill}%
\pgfsetlinewidth{0.250937pt}%
\definecolor{currentstroke}{rgb}{1.000000,1.000000,1.000000}%
\pgfsetstrokecolor{currentstroke}%
\pgfsetdash{}{0pt}%
\pgfpathmoveto{\pgfqpoint{1.872452in}{4.799340in}}%
\pgfpathlineto{\pgfqpoint{1.960188in}{4.799340in}}%
\pgfpathlineto{\pgfqpoint{1.960188in}{4.711604in}}%
\pgfpathlineto{\pgfqpoint{1.872452in}{4.711604in}}%
\pgfpathlineto{\pgfqpoint{1.872452in}{4.799340in}}%
\pgfusepath{stroke,fill}%
\end{pgfscope}%
\begin{pgfscope}%
\pgfpathrectangle{\pgfqpoint{0.380943in}{4.185189in}}{\pgfqpoint{4.650000in}{0.614151in}}%
\pgfusepath{clip}%
\pgfsetbuttcap%
\pgfsetroundjoin%
\definecolor{currentfill}{rgb}{0.992326,0.765229,0.614840}%
\pgfsetfillcolor{currentfill}%
\pgfsetlinewidth{0.250937pt}%
\definecolor{currentstroke}{rgb}{1.000000,1.000000,1.000000}%
\pgfsetstrokecolor{currentstroke}%
\pgfsetdash{}{0pt}%
\pgfpathmoveto{\pgfqpoint{1.960188in}{4.799340in}}%
\pgfpathlineto{\pgfqpoint{2.047924in}{4.799340in}}%
\pgfpathlineto{\pgfqpoint{2.047924in}{4.711604in}}%
\pgfpathlineto{\pgfqpoint{1.960188in}{4.711604in}}%
\pgfpathlineto{\pgfqpoint{1.960188in}{4.799340in}}%
\pgfusepath{stroke,fill}%
\end{pgfscope}%
\begin{pgfscope}%
\pgfpathrectangle{\pgfqpoint{0.380943in}{4.185189in}}{\pgfqpoint{4.650000in}{0.614151in}}%
\pgfusepath{clip}%
\pgfsetbuttcap%
\pgfsetroundjoin%
\definecolor{currentfill}{rgb}{0.998939,0.658962,0.556032}%
\pgfsetfillcolor{currentfill}%
\pgfsetlinewidth{0.250937pt}%
\definecolor{currentstroke}{rgb}{1.000000,1.000000,1.000000}%
\pgfsetstrokecolor{currentstroke}%
\pgfsetdash{}{0pt}%
\pgfpathmoveto{\pgfqpoint{2.047924in}{4.799340in}}%
\pgfpathlineto{\pgfqpoint{2.135660in}{4.799340in}}%
\pgfpathlineto{\pgfqpoint{2.135660in}{4.711604in}}%
\pgfpathlineto{\pgfqpoint{2.047924in}{4.711604in}}%
\pgfpathlineto{\pgfqpoint{2.047924in}{4.799340in}}%
\pgfusepath{stroke,fill}%
\end{pgfscope}%
\begin{pgfscope}%
\pgfpathrectangle{\pgfqpoint{0.380943in}{4.185189in}}{\pgfqpoint{4.650000in}{0.614151in}}%
\pgfusepath{clip}%
\pgfsetbuttcap%
\pgfsetroundjoin%
\definecolor{currentfill}{rgb}{0.979654,0.837186,0.669619}%
\pgfsetfillcolor{currentfill}%
\pgfsetlinewidth{0.250937pt}%
\definecolor{currentstroke}{rgb}{1.000000,1.000000,1.000000}%
\pgfsetstrokecolor{currentstroke}%
\pgfsetdash{}{0pt}%
\pgfpathmoveto{\pgfqpoint{2.135660in}{4.799340in}}%
\pgfpathlineto{\pgfqpoint{2.223396in}{4.799340in}}%
\pgfpathlineto{\pgfqpoint{2.223396in}{4.711604in}}%
\pgfpathlineto{\pgfqpoint{2.135660in}{4.711604in}}%
\pgfpathlineto{\pgfqpoint{2.135660in}{4.799340in}}%
\pgfusepath{stroke,fill}%
\end{pgfscope}%
\begin{pgfscope}%
\pgfpathrectangle{\pgfqpoint{0.380943in}{4.185189in}}{\pgfqpoint{4.650000in}{0.614151in}}%
\pgfusepath{clip}%
\pgfsetbuttcap%
\pgfsetroundjoin%
\definecolor{currentfill}{rgb}{0.972549,0.870588,0.692810}%
\pgfsetfillcolor{currentfill}%
\pgfsetlinewidth{0.250937pt}%
\definecolor{currentstroke}{rgb}{1.000000,1.000000,1.000000}%
\pgfsetstrokecolor{currentstroke}%
\pgfsetdash{}{0pt}%
\pgfpathmoveto{\pgfqpoint{2.223396in}{4.799340in}}%
\pgfpathlineto{\pgfqpoint{2.311132in}{4.799340in}}%
\pgfpathlineto{\pgfqpoint{2.311132in}{4.711604in}}%
\pgfpathlineto{\pgfqpoint{2.223396in}{4.711604in}}%
\pgfpathlineto{\pgfqpoint{2.223396in}{4.799340in}}%
\pgfusepath{stroke,fill}%
\end{pgfscope}%
\begin{pgfscope}%
\pgfpathrectangle{\pgfqpoint{0.380943in}{4.185189in}}{\pgfqpoint{4.650000in}{0.614151in}}%
\pgfusepath{clip}%
\pgfsetbuttcap%
\pgfsetroundjoin%
\definecolor{currentfill}{rgb}{0.992326,0.765229,0.614840}%
\pgfsetfillcolor{currentfill}%
\pgfsetlinewidth{0.250937pt}%
\definecolor{currentstroke}{rgb}{1.000000,1.000000,1.000000}%
\pgfsetstrokecolor{currentstroke}%
\pgfsetdash{}{0pt}%
\pgfpathmoveto{\pgfqpoint{2.311132in}{4.799340in}}%
\pgfpathlineto{\pgfqpoint{2.398868in}{4.799340in}}%
\pgfpathlineto{\pgfqpoint{2.398868in}{4.711604in}}%
\pgfpathlineto{\pgfqpoint{2.311132in}{4.711604in}}%
\pgfpathlineto{\pgfqpoint{2.311132in}{4.799340in}}%
\pgfusepath{stroke,fill}%
\end{pgfscope}%
\begin{pgfscope}%
\pgfpathrectangle{\pgfqpoint{0.380943in}{4.185189in}}{\pgfqpoint{4.650000in}{0.614151in}}%
\pgfusepath{clip}%
\pgfsetbuttcap%
\pgfsetroundjoin%
\definecolor{currentfill}{rgb}{0.968166,0.945882,0.748604}%
\pgfsetfillcolor{currentfill}%
\pgfsetlinewidth{0.250937pt}%
\definecolor{currentstroke}{rgb}{1.000000,1.000000,1.000000}%
\pgfsetstrokecolor{currentstroke}%
\pgfsetdash{}{0pt}%
\pgfpathmoveto{\pgfqpoint{2.398868in}{4.799340in}}%
\pgfpathlineto{\pgfqpoint{2.486603in}{4.799340in}}%
\pgfpathlineto{\pgfqpoint{2.486603in}{4.711604in}}%
\pgfpathlineto{\pgfqpoint{2.398868in}{4.711604in}}%
\pgfpathlineto{\pgfqpoint{2.398868in}{4.799340in}}%
\pgfusepath{stroke,fill}%
\end{pgfscope}%
\begin{pgfscope}%
\pgfpathrectangle{\pgfqpoint{0.380943in}{4.185189in}}{\pgfqpoint{4.650000in}{0.614151in}}%
\pgfusepath{clip}%
\pgfsetbuttcap%
\pgfsetroundjoin%
\definecolor{currentfill}{rgb}{0.979654,0.837186,0.669619}%
\pgfsetfillcolor{currentfill}%
\pgfsetlinewidth{0.250937pt}%
\definecolor{currentstroke}{rgb}{1.000000,1.000000,1.000000}%
\pgfsetstrokecolor{currentstroke}%
\pgfsetdash{}{0pt}%
\pgfpathmoveto{\pgfqpoint{2.486603in}{4.799340in}}%
\pgfpathlineto{\pgfqpoint{2.574339in}{4.799340in}}%
\pgfpathlineto{\pgfqpoint{2.574339in}{4.711604in}}%
\pgfpathlineto{\pgfqpoint{2.486603in}{4.711604in}}%
\pgfpathlineto{\pgfqpoint{2.486603in}{4.799340in}}%
\pgfusepath{stroke,fill}%
\end{pgfscope}%
\begin{pgfscope}%
\pgfpathrectangle{\pgfqpoint{0.380943in}{4.185189in}}{\pgfqpoint{4.650000in}{0.614151in}}%
\pgfusepath{clip}%
\pgfsetbuttcap%
\pgfsetroundjoin%
\definecolor{currentfill}{rgb}{0.979654,0.837186,0.669619}%
\pgfsetfillcolor{currentfill}%
\pgfsetlinewidth{0.250937pt}%
\definecolor{currentstroke}{rgb}{1.000000,1.000000,1.000000}%
\pgfsetstrokecolor{currentstroke}%
\pgfsetdash{}{0pt}%
\pgfpathmoveto{\pgfqpoint{2.574339in}{4.799340in}}%
\pgfpathlineto{\pgfqpoint{2.662075in}{4.799340in}}%
\pgfpathlineto{\pgfqpoint{2.662075in}{4.711604in}}%
\pgfpathlineto{\pgfqpoint{2.574339in}{4.711604in}}%
\pgfpathlineto{\pgfqpoint{2.574339in}{4.799340in}}%
\pgfusepath{stroke,fill}%
\end{pgfscope}%
\begin{pgfscope}%
\pgfpathrectangle{\pgfqpoint{0.380943in}{4.185189in}}{\pgfqpoint{4.650000in}{0.614151in}}%
\pgfusepath{clip}%
\pgfsetbuttcap%
\pgfsetroundjoin%
\definecolor{currentfill}{rgb}{0.965444,0.906113,0.711757}%
\pgfsetfillcolor{currentfill}%
\pgfsetlinewidth{0.250937pt}%
\definecolor{currentstroke}{rgb}{1.000000,1.000000,1.000000}%
\pgfsetstrokecolor{currentstroke}%
\pgfsetdash{}{0pt}%
\pgfpathmoveto{\pgfqpoint{2.662075in}{4.799340in}}%
\pgfpathlineto{\pgfqpoint{2.749811in}{4.799340in}}%
\pgfpathlineto{\pgfqpoint{2.749811in}{4.711604in}}%
\pgfpathlineto{\pgfqpoint{2.662075in}{4.711604in}}%
\pgfpathlineto{\pgfqpoint{2.662075in}{4.799340in}}%
\pgfusepath{stroke,fill}%
\end{pgfscope}%
\begin{pgfscope}%
\pgfpathrectangle{\pgfqpoint{0.380943in}{4.185189in}}{\pgfqpoint{4.650000in}{0.614151in}}%
\pgfusepath{clip}%
\pgfsetbuttcap%
\pgfsetroundjoin%
\definecolor{currentfill}{rgb}{0.996571,0.720538,0.589189}%
\pgfsetfillcolor{currentfill}%
\pgfsetlinewidth{0.250937pt}%
\definecolor{currentstroke}{rgb}{1.000000,1.000000,1.000000}%
\pgfsetstrokecolor{currentstroke}%
\pgfsetdash{}{0pt}%
\pgfpathmoveto{\pgfqpoint{2.749811in}{4.799340in}}%
\pgfpathlineto{\pgfqpoint{2.837547in}{4.799340in}}%
\pgfpathlineto{\pgfqpoint{2.837547in}{4.711604in}}%
\pgfpathlineto{\pgfqpoint{2.749811in}{4.711604in}}%
\pgfpathlineto{\pgfqpoint{2.749811in}{4.799340in}}%
\pgfusepath{stroke,fill}%
\end{pgfscope}%
\begin{pgfscope}%
\pgfpathrectangle{\pgfqpoint{0.380943in}{4.185189in}}{\pgfqpoint{4.650000in}{0.614151in}}%
\pgfusepath{clip}%
\pgfsetbuttcap%
\pgfsetroundjoin%
\definecolor{currentfill}{rgb}{0.996571,0.720538,0.589189}%
\pgfsetfillcolor{currentfill}%
\pgfsetlinewidth{0.250937pt}%
\definecolor{currentstroke}{rgb}{1.000000,1.000000,1.000000}%
\pgfsetstrokecolor{currentstroke}%
\pgfsetdash{}{0pt}%
\pgfpathmoveto{\pgfqpoint{2.837547in}{4.799340in}}%
\pgfpathlineto{\pgfqpoint{2.925283in}{4.799340in}}%
\pgfpathlineto{\pgfqpoint{2.925283in}{4.711604in}}%
\pgfpathlineto{\pgfqpoint{2.837547in}{4.711604in}}%
\pgfpathlineto{\pgfqpoint{2.837547in}{4.799340in}}%
\pgfusepath{stroke,fill}%
\end{pgfscope}%
\begin{pgfscope}%
\pgfpathrectangle{\pgfqpoint{0.380943in}{4.185189in}}{\pgfqpoint{4.650000in}{0.614151in}}%
\pgfusepath{clip}%
\pgfsetbuttcap%
\pgfsetroundjoin%
\definecolor{currentfill}{rgb}{0.922338,0.400769,0.400769}%
\pgfsetfillcolor{currentfill}%
\pgfsetlinewidth{0.250937pt}%
\definecolor{currentstroke}{rgb}{1.000000,1.000000,1.000000}%
\pgfsetstrokecolor{currentstroke}%
\pgfsetdash{}{0pt}%
\pgfpathmoveto{\pgfqpoint{2.925283in}{4.799340in}}%
\pgfpathlineto{\pgfqpoint{3.013019in}{4.799340in}}%
\pgfpathlineto{\pgfqpoint{3.013019in}{4.711604in}}%
\pgfpathlineto{\pgfqpoint{2.925283in}{4.711604in}}%
\pgfpathlineto{\pgfqpoint{2.925283in}{4.799340in}}%
\pgfusepath{stroke,fill}%
\end{pgfscope}%
\begin{pgfscope}%
\pgfpathrectangle{\pgfqpoint{0.380943in}{4.185189in}}{\pgfqpoint{4.650000in}{0.614151in}}%
\pgfusepath{clip}%
\pgfsetbuttcap%
\pgfsetroundjoin%
\definecolor{currentfill}{rgb}{0.979654,0.837186,0.669619}%
\pgfsetfillcolor{currentfill}%
\pgfsetlinewidth{0.250937pt}%
\definecolor{currentstroke}{rgb}{1.000000,1.000000,1.000000}%
\pgfsetstrokecolor{currentstroke}%
\pgfsetdash{}{0pt}%
\pgfpathmoveto{\pgfqpoint{3.013019in}{4.799340in}}%
\pgfpathlineto{\pgfqpoint{3.100754in}{4.799340in}}%
\pgfpathlineto{\pgfqpoint{3.100754in}{4.711604in}}%
\pgfpathlineto{\pgfqpoint{3.013019in}{4.711604in}}%
\pgfpathlineto{\pgfqpoint{3.013019in}{4.799340in}}%
\pgfusepath{stroke,fill}%
\end{pgfscope}%
\begin{pgfscope}%
\pgfpathrectangle{\pgfqpoint{0.380943in}{4.185189in}}{\pgfqpoint{4.650000in}{0.614151in}}%
\pgfusepath{clip}%
\pgfsetbuttcap%
\pgfsetroundjoin%
\definecolor{currentfill}{rgb}{0.986759,0.806398,0.641200}%
\pgfsetfillcolor{currentfill}%
\pgfsetlinewidth{0.250937pt}%
\definecolor{currentstroke}{rgb}{1.000000,1.000000,1.000000}%
\pgfsetstrokecolor{currentstroke}%
\pgfsetdash{}{0pt}%
\pgfpathmoveto{\pgfqpoint{3.100754in}{4.799340in}}%
\pgfpathlineto{\pgfqpoint{3.188490in}{4.799340in}}%
\pgfpathlineto{\pgfqpoint{3.188490in}{4.711604in}}%
\pgfpathlineto{\pgfqpoint{3.100754in}{4.711604in}}%
\pgfpathlineto{\pgfqpoint{3.100754in}{4.799340in}}%
\pgfusepath{stroke,fill}%
\end{pgfscope}%
\begin{pgfscope}%
\pgfpathrectangle{\pgfqpoint{0.380943in}{4.185189in}}{\pgfqpoint{4.650000in}{0.614151in}}%
\pgfusepath{clip}%
\pgfsetbuttcap%
\pgfsetroundjoin%
\definecolor{currentfill}{rgb}{0.972549,0.870588,0.692810}%
\pgfsetfillcolor{currentfill}%
\pgfsetlinewidth{0.250937pt}%
\definecolor{currentstroke}{rgb}{1.000000,1.000000,1.000000}%
\pgfsetstrokecolor{currentstroke}%
\pgfsetdash{}{0pt}%
\pgfpathmoveto{\pgfqpoint{3.188490in}{4.799340in}}%
\pgfpathlineto{\pgfqpoint{3.276226in}{4.799340in}}%
\pgfpathlineto{\pgfqpoint{3.276226in}{4.711604in}}%
\pgfpathlineto{\pgfqpoint{3.188490in}{4.711604in}}%
\pgfpathlineto{\pgfqpoint{3.188490in}{4.799340in}}%
\pgfusepath{stroke,fill}%
\end{pgfscope}%
\begin{pgfscope}%
\pgfpathrectangle{\pgfqpoint{0.380943in}{4.185189in}}{\pgfqpoint{4.650000in}{0.614151in}}%
\pgfusepath{clip}%
\pgfsetbuttcap%
\pgfsetroundjoin%
\definecolor{currentfill}{rgb}{1.000000,0.605229,0.530719}%
\pgfsetfillcolor{currentfill}%
\pgfsetlinewidth{0.250937pt}%
\definecolor{currentstroke}{rgb}{1.000000,1.000000,1.000000}%
\pgfsetstrokecolor{currentstroke}%
\pgfsetdash{}{0pt}%
\pgfpathmoveto{\pgfqpoint{3.276226in}{4.799340in}}%
\pgfpathlineto{\pgfqpoint{3.363962in}{4.799340in}}%
\pgfpathlineto{\pgfqpoint{3.363962in}{4.711604in}}%
\pgfpathlineto{\pgfqpoint{3.276226in}{4.711604in}}%
\pgfpathlineto{\pgfqpoint{3.276226in}{4.799340in}}%
\pgfusepath{stroke,fill}%
\end{pgfscope}%
\begin{pgfscope}%
\pgfpathrectangle{\pgfqpoint{0.380943in}{4.185189in}}{\pgfqpoint{4.650000in}{0.614151in}}%
\pgfusepath{clip}%
\pgfsetbuttcap%
\pgfsetroundjoin%
\definecolor{currentfill}{rgb}{0.986759,0.806398,0.641200}%
\pgfsetfillcolor{currentfill}%
\pgfsetlinewidth{0.250937pt}%
\definecolor{currentstroke}{rgb}{1.000000,1.000000,1.000000}%
\pgfsetstrokecolor{currentstroke}%
\pgfsetdash{}{0pt}%
\pgfpathmoveto{\pgfqpoint{3.363962in}{4.799340in}}%
\pgfpathlineto{\pgfqpoint{3.451698in}{4.799340in}}%
\pgfpathlineto{\pgfqpoint{3.451698in}{4.711604in}}%
\pgfpathlineto{\pgfqpoint{3.363962in}{4.711604in}}%
\pgfpathlineto{\pgfqpoint{3.363962in}{4.799340in}}%
\pgfusepath{stroke,fill}%
\end{pgfscope}%
\begin{pgfscope}%
\pgfpathrectangle{\pgfqpoint{0.380943in}{4.185189in}}{\pgfqpoint{4.650000in}{0.614151in}}%
\pgfusepath{clip}%
\pgfsetbuttcap%
\pgfsetroundjoin%
\definecolor{currentfill}{rgb}{0.965444,0.906113,0.711757}%
\pgfsetfillcolor{currentfill}%
\pgfsetlinewidth{0.250937pt}%
\definecolor{currentstroke}{rgb}{1.000000,1.000000,1.000000}%
\pgfsetstrokecolor{currentstroke}%
\pgfsetdash{}{0pt}%
\pgfpathmoveto{\pgfqpoint{3.451698in}{4.799340in}}%
\pgfpathlineto{\pgfqpoint{3.539434in}{4.799340in}}%
\pgfpathlineto{\pgfqpoint{3.539434in}{4.711604in}}%
\pgfpathlineto{\pgfqpoint{3.451698in}{4.711604in}}%
\pgfpathlineto{\pgfqpoint{3.451698in}{4.799340in}}%
\pgfusepath{stroke,fill}%
\end{pgfscope}%
\begin{pgfscope}%
\pgfpathrectangle{\pgfqpoint{0.380943in}{4.185189in}}{\pgfqpoint{4.650000in}{0.614151in}}%
\pgfusepath{clip}%
\pgfsetbuttcap%
\pgfsetroundjoin%
\definecolor{currentfill}{rgb}{0.968166,0.945882,0.748604}%
\pgfsetfillcolor{currentfill}%
\pgfsetlinewidth{0.250937pt}%
\definecolor{currentstroke}{rgb}{1.000000,1.000000,1.000000}%
\pgfsetstrokecolor{currentstroke}%
\pgfsetdash{}{0pt}%
\pgfpathmoveto{\pgfqpoint{3.539434in}{4.799340in}}%
\pgfpathlineto{\pgfqpoint{3.627169in}{4.799340in}}%
\pgfpathlineto{\pgfqpoint{3.627169in}{4.711604in}}%
\pgfpathlineto{\pgfqpoint{3.539434in}{4.711604in}}%
\pgfpathlineto{\pgfqpoint{3.539434in}{4.799340in}}%
\pgfusepath{stroke,fill}%
\end{pgfscope}%
\begin{pgfscope}%
\pgfpathrectangle{\pgfqpoint{0.380943in}{4.185189in}}{\pgfqpoint{4.650000in}{0.614151in}}%
\pgfusepath{clip}%
\pgfsetbuttcap%
\pgfsetroundjoin%
\definecolor{currentfill}{rgb}{0.965444,0.906113,0.711757}%
\pgfsetfillcolor{currentfill}%
\pgfsetlinewidth{0.250937pt}%
\definecolor{currentstroke}{rgb}{1.000000,1.000000,1.000000}%
\pgfsetstrokecolor{currentstroke}%
\pgfsetdash{}{0pt}%
\pgfpathmoveto{\pgfqpoint{3.627169in}{4.799340in}}%
\pgfpathlineto{\pgfqpoint{3.714905in}{4.799340in}}%
\pgfpathlineto{\pgfqpoint{3.714905in}{4.711604in}}%
\pgfpathlineto{\pgfqpoint{3.627169in}{4.711604in}}%
\pgfpathlineto{\pgfqpoint{3.627169in}{4.799340in}}%
\pgfusepath{stroke,fill}%
\end{pgfscope}%
\begin{pgfscope}%
\pgfpathrectangle{\pgfqpoint{0.380943in}{4.185189in}}{\pgfqpoint{4.650000in}{0.614151in}}%
\pgfusepath{clip}%
\pgfsetbuttcap%
\pgfsetroundjoin%
\definecolor{currentfill}{rgb}{0.972549,0.870588,0.692810}%
\pgfsetfillcolor{currentfill}%
\pgfsetlinewidth{0.250937pt}%
\definecolor{currentstroke}{rgb}{1.000000,1.000000,1.000000}%
\pgfsetstrokecolor{currentstroke}%
\pgfsetdash{}{0pt}%
\pgfpathmoveto{\pgfqpoint{3.714905in}{4.799340in}}%
\pgfpathlineto{\pgfqpoint{3.802641in}{4.799340in}}%
\pgfpathlineto{\pgfqpoint{3.802641in}{4.711604in}}%
\pgfpathlineto{\pgfqpoint{3.714905in}{4.711604in}}%
\pgfpathlineto{\pgfqpoint{3.714905in}{4.799340in}}%
\pgfusepath{stroke,fill}%
\end{pgfscope}%
\begin{pgfscope}%
\pgfpathrectangle{\pgfqpoint{0.380943in}{4.185189in}}{\pgfqpoint{4.650000in}{0.614151in}}%
\pgfusepath{clip}%
\pgfsetbuttcap%
\pgfsetroundjoin%
\definecolor{currentfill}{rgb}{0.979654,0.837186,0.669619}%
\pgfsetfillcolor{currentfill}%
\pgfsetlinewidth{0.250937pt}%
\definecolor{currentstroke}{rgb}{1.000000,1.000000,1.000000}%
\pgfsetstrokecolor{currentstroke}%
\pgfsetdash{}{0pt}%
\pgfpathmoveto{\pgfqpoint{3.802641in}{4.799340in}}%
\pgfpathlineto{\pgfqpoint{3.890377in}{4.799340in}}%
\pgfpathlineto{\pgfqpoint{3.890377in}{4.711604in}}%
\pgfpathlineto{\pgfqpoint{3.802641in}{4.711604in}}%
\pgfpathlineto{\pgfqpoint{3.802641in}{4.799340in}}%
\pgfusepath{stroke,fill}%
\end{pgfscope}%
\begin{pgfscope}%
\pgfpathrectangle{\pgfqpoint{0.380943in}{4.185189in}}{\pgfqpoint{4.650000in}{0.614151in}}%
\pgfusepath{clip}%
\pgfsetbuttcap%
\pgfsetroundjoin%
\definecolor{currentfill}{rgb}{0.972549,0.870588,0.692810}%
\pgfsetfillcolor{currentfill}%
\pgfsetlinewidth{0.250937pt}%
\definecolor{currentstroke}{rgb}{1.000000,1.000000,1.000000}%
\pgfsetstrokecolor{currentstroke}%
\pgfsetdash{}{0pt}%
\pgfpathmoveto{\pgfqpoint{3.890377in}{4.799340in}}%
\pgfpathlineto{\pgfqpoint{3.978113in}{4.799340in}}%
\pgfpathlineto{\pgfqpoint{3.978113in}{4.711604in}}%
\pgfpathlineto{\pgfqpoint{3.890377in}{4.711604in}}%
\pgfpathlineto{\pgfqpoint{3.890377in}{4.799340in}}%
\pgfusepath{stroke,fill}%
\end{pgfscope}%
\begin{pgfscope}%
\pgfpathrectangle{\pgfqpoint{0.380943in}{4.185189in}}{\pgfqpoint{4.650000in}{0.614151in}}%
\pgfusepath{clip}%
\pgfsetbuttcap%
\pgfsetroundjoin%
\definecolor{currentfill}{rgb}{0.861576,0.340008,0.340008}%
\pgfsetfillcolor{currentfill}%
\pgfsetlinewidth{0.250937pt}%
\definecolor{currentstroke}{rgb}{1.000000,1.000000,1.000000}%
\pgfsetstrokecolor{currentstroke}%
\pgfsetdash{}{0pt}%
\pgfpathmoveto{\pgfqpoint{3.978113in}{4.799340in}}%
\pgfpathlineto{\pgfqpoint{4.065849in}{4.799340in}}%
\pgfpathlineto{\pgfqpoint{4.065849in}{4.711604in}}%
\pgfpathlineto{\pgfqpoint{3.978113in}{4.711604in}}%
\pgfpathlineto{\pgfqpoint{3.978113in}{4.799340in}}%
\pgfusepath{stroke,fill}%
\end{pgfscope}%
\begin{pgfscope}%
\pgfpathrectangle{\pgfqpoint{0.380943in}{4.185189in}}{\pgfqpoint{4.650000in}{0.614151in}}%
\pgfusepath{clip}%
\pgfsetbuttcap%
\pgfsetroundjoin%
\definecolor{currentfill}{rgb}{0.986759,0.806398,0.641200}%
\pgfsetfillcolor{currentfill}%
\pgfsetlinewidth{0.250937pt}%
\definecolor{currentstroke}{rgb}{1.000000,1.000000,1.000000}%
\pgfsetstrokecolor{currentstroke}%
\pgfsetdash{}{0pt}%
\pgfpathmoveto{\pgfqpoint{4.065849in}{4.799340in}}%
\pgfpathlineto{\pgfqpoint{4.153585in}{4.799340in}}%
\pgfpathlineto{\pgfqpoint{4.153585in}{4.711604in}}%
\pgfpathlineto{\pgfqpoint{4.065849in}{4.711604in}}%
\pgfpathlineto{\pgfqpoint{4.065849in}{4.799340in}}%
\pgfusepath{stroke,fill}%
\end{pgfscope}%
\begin{pgfscope}%
\pgfpathrectangle{\pgfqpoint{0.380943in}{4.185189in}}{\pgfqpoint{4.650000in}{0.614151in}}%
\pgfusepath{clip}%
\pgfsetbuttcap%
\pgfsetroundjoin%
\definecolor{currentfill}{rgb}{0.981546,0.459977,0.459977}%
\pgfsetfillcolor{currentfill}%
\pgfsetlinewidth{0.250937pt}%
\definecolor{currentstroke}{rgb}{1.000000,1.000000,1.000000}%
\pgfsetstrokecolor{currentstroke}%
\pgfsetdash{}{0pt}%
\pgfpathmoveto{\pgfqpoint{4.153585in}{4.799340in}}%
\pgfpathlineto{\pgfqpoint{4.241320in}{4.799340in}}%
\pgfpathlineto{\pgfqpoint{4.241320in}{4.711604in}}%
\pgfpathlineto{\pgfqpoint{4.153585in}{4.711604in}}%
\pgfpathlineto{\pgfqpoint{4.153585in}{4.799340in}}%
\pgfusepath{stroke,fill}%
\end{pgfscope}%
\begin{pgfscope}%
\pgfpathrectangle{\pgfqpoint{0.380943in}{4.185189in}}{\pgfqpoint{4.650000in}{0.614151in}}%
\pgfusepath{clip}%
\pgfsetbuttcap%
\pgfsetroundjoin%
\definecolor{currentfill}{rgb}{0.996571,0.720538,0.589189}%
\pgfsetfillcolor{currentfill}%
\pgfsetlinewidth{0.250937pt}%
\definecolor{currentstroke}{rgb}{1.000000,1.000000,1.000000}%
\pgfsetstrokecolor{currentstroke}%
\pgfsetdash{}{0pt}%
\pgfpathmoveto{\pgfqpoint{4.241320in}{4.799340in}}%
\pgfpathlineto{\pgfqpoint{4.329056in}{4.799340in}}%
\pgfpathlineto{\pgfqpoint{4.329056in}{4.711604in}}%
\pgfpathlineto{\pgfqpoint{4.241320in}{4.711604in}}%
\pgfpathlineto{\pgfqpoint{4.241320in}{4.799340in}}%
\pgfusepath{stroke,fill}%
\end{pgfscope}%
\begin{pgfscope}%
\pgfpathrectangle{\pgfqpoint{0.380943in}{4.185189in}}{\pgfqpoint{4.650000in}{0.614151in}}%
\pgfusepath{clip}%
\pgfsetbuttcap%
\pgfsetroundjoin%
\definecolor{currentfill}{rgb}{0.962414,0.923552,0.722891}%
\pgfsetfillcolor{currentfill}%
\pgfsetlinewidth{0.250937pt}%
\definecolor{currentstroke}{rgb}{1.000000,1.000000,1.000000}%
\pgfsetstrokecolor{currentstroke}%
\pgfsetdash{}{0pt}%
\pgfpathmoveto{\pgfqpoint{4.329056in}{4.799340in}}%
\pgfpathlineto{\pgfqpoint{4.416792in}{4.799340in}}%
\pgfpathlineto{\pgfqpoint{4.416792in}{4.711604in}}%
\pgfpathlineto{\pgfqpoint{4.329056in}{4.711604in}}%
\pgfpathlineto{\pgfqpoint{4.329056in}{4.799340in}}%
\pgfusepath{stroke,fill}%
\end{pgfscope}%
\begin{pgfscope}%
\pgfpathrectangle{\pgfqpoint{0.380943in}{4.185189in}}{\pgfqpoint{4.650000in}{0.614151in}}%
\pgfusepath{clip}%
\pgfsetbuttcap%
\pgfsetroundjoin%
\definecolor{currentfill}{rgb}{0.996571,0.720538,0.589189}%
\pgfsetfillcolor{currentfill}%
\pgfsetlinewidth{0.250937pt}%
\definecolor{currentstroke}{rgb}{1.000000,1.000000,1.000000}%
\pgfsetstrokecolor{currentstroke}%
\pgfsetdash{}{0pt}%
\pgfpathmoveto{\pgfqpoint{4.416792in}{4.799340in}}%
\pgfpathlineto{\pgfqpoint{4.504528in}{4.799340in}}%
\pgfpathlineto{\pgfqpoint{4.504528in}{4.711604in}}%
\pgfpathlineto{\pgfqpoint{4.416792in}{4.711604in}}%
\pgfpathlineto{\pgfqpoint{4.416792in}{4.799340in}}%
\pgfusepath{stroke,fill}%
\end{pgfscope}%
\begin{pgfscope}%
\pgfpathrectangle{\pgfqpoint{0.380943in}{4.185189in}}{\pgfqpoint{4.650000in}{0.614151in}}%
\pgfusepath{clip}%
\pgfsetbuttcap%
\pgfsetroundjoin%
\definecolor{currentfill}{rgb}{0.972549,0.870588,0.692810}%
\pgfsetfillcolor{currentfill}%
\pgfsetlinewidth{0.250937pt}%
\definecolor{currentstroke}{rgb}{1.000000,1.000000,1.000000}%
\pgfsetstrokecolor{currentstroke}%
\pgfsetdash{}{0pt}%
\pgfpathmoveto{\pgfqpoint{4.504528in}{4.799340in}}%
\pgfpathlineto{\pgfqpoint{4.592264in}{4.799340in}}%
\pgfpathlineto{\pgfqpoint{4.592264in}{4.711604in}}%
\pgfpathlineto{\pgfqpoint{4.504528in}{4.711604in}}%
\pgfpathlineto{\pgfqpoint{4.504528in}{4.799340in}}%
\pgfusepath{stroke,fill}%
\end{pgfscope}%
\begin{pgfscope}%
\pgfpathrectangle{\pgfqpoint{0.380943in}{4.185189in}}{\pgfqpoint{4.650000in}{0.614151in}}%
\pgfusepath{clip}%
\pgfsetbuttcap%
\pgfsetroundjoin%
\definecolor{currentfill}{rgb}{0.996571,0.720538,0.589189}%
\pgfsetfillcolor{currentfill}%
\pgfsetlinewidth{0.250937pt}%
\definecolor{currentstroke}{rgb}{1.000000,1.000000,1.000000}%
\pgfsetstrokecolor{currentstroke}%
\pgfsetdash{}{0pt}%
\pgfpathmoveto{\pgfqpoint{4.592264in}{4.799340in}}%
\pgfpathlineto{\pgfqpoint{4.680000in}{4.799340in}}%
\pgfpathlineto{\pgfqpoint{4.680000in}{4.711604in}}%
\pgfpathlineto{\pgfqpoint{4.592264in}{4.711604in}}%
\pgfpathlineto{\pgfqpoint{4.592264in}{4.799340in}}%
\pgfusepath{stroke,fill}%
\end{pgfscope}%
\begin{pgfscope}%
\pgfpathrectangle{\pgfqpoint{0.380943in}{4.185189in}}{\pgfqpoint{4.650000in}{0.614151in}}%
\pgfusepath{clip}%
\pgfsetbuttcap%
\pgfsetroundjoin%
\definecolor{currentfill}{rgb}{0.968166,0.945882,0.748604}%
\pgfsetfillcolor{currentfill}%
\pgfsetlinewidth{0.250937pt}%
\definecolor{currentstroke}{rgb}{1.000000,1.000000,1.000000}%
\pgfsetstrokecolor{currentstroke}%
\pgfsetdash{}{0pt}%
\pgfpathmoveto{\pgfqpoint{4.680000in}{4.799340in}}%
\pgfpathlineto{\pgfqpoint{4.767736in}{4.799340in}}%
\pgfpathlineto{\pgfqpoint{4.767736in}{4.711604in}}%
\pgfpathlineto{\pgfqpoint{4.680000in}{4.711604in}}%
\pgfpathlineto{\pgfqpoint{4.680000in}{4.799340in}}%
\pgfusepath{stroke,fill}%
\end{pgfscope}%
\begin{pgfscope}%
\pgfpathrectangle{\pgfqpoint{0.380943in}{4.185189in}}{\pgfqpoint{4.650000in}{0.614151in}}%
\pgfusepath{clip}%
\pgfsetbuttcap%
\pgfsetroundjoin%
\definecolor{currentfill}{rgb}{1.000000,0.605229,0.530719}%
\pgfsetfillcolor{currentfill}%
\pgfsetlinewidth{0.250937pt}%
\definecolor{currentstroke}{rgb}{1.000000,1.000000,1.000000}%
\pgfsetstrokecolor{currentstroke}%
\pgfsetdash{}{0pt}%
\pgfpathmoveto{\pgfqpoint{4.767736in}{4.799340in}}%
\pgfpathlineto{\pgfqpoint{4.855471in}{4.799340in}}%
\pgfpathlineto{\pgfqpoint{4.855471in}{4.711604in}}%
\pgfpathlineto{\pgfqpoint{4.767736in}{4.711604in}}%
\pgfpathlineto{\pgfqpoint{4.767736in}{4.799340in}}%
\pgfusepath{stroke,fill}%
\end{pgfscope}%
\begin{pgfscope}%
\pgfpathrectangle{\pgfqpoint{0.380943in}{4.185189in}}{\pgfqpoint{4.650000in}{0.614151in}}%
\pgfusepath{clip}%
\pgfsetbuttcap%
\pgfsetroundjoin%
\definecolor{currentfill}{rgb}{0.996571,0.720538,0.589189}%
\pgfsetfillcolor{currentfill}%
\pgfsetlinewidth{0.250937pt}%
\definecolor{currentstroke}{rgb}{1.000000,1.000000,1.000000}%
\pgfsetstrokecolor{currentstroke}%
\pgfsetdash{}{0pt}%
\pgfpathmoveto{\pgfqpoint{4.855471in}{4.799340in}}%
\pgfpathlineto{\pgfqpoint{4.943207in}{4.799340in}}%
\pgfpathlineto{\pgfqpoint{4.943207in}{4.711604in}}%
\pgfpathlineto{\pgfqpoint{4.855471in}{4.711604in}}%
\pgfpathlineto{\pgfqpoint{4.855471in}{4.799340in}}%
\pgfusepath{stroke,fill}%
\end{pgfscope}%
\begin{pgfscope}%
\pgfpathrectangle{\pgfqpoint{0.380943in}{4.185189in}}{\pgfqpoint{4.650000in}{0.614151in}}%
\pgfusepath{clip}%
\pgfsetbuttcap%
\pgfsetroundjoin%
\definecolor{currentfill}{rgb}{0.986759,0.806398,0.641200}%
\pgfsetfillcolor{currentfill}%
\pgfsetlinewidth{0.250937pt}%
\definecolor{currentstroke}{rgb}{1.000000,1.000000,1.000000}%
\pgfsetstrokecolor{currentstroke}%
\pgfsetdash{}{0pt}%
\pgfpathmoveto{\pgfqpoint{4.943207in}{4.799340in}}%
\pgfpathlineto{\pgfqpoint{5.030943in}{4.799340in}}%
\pgfpathlineto{\pgfqpoint{5.030943in}{4.711604in}}%
\pgfpathlineto{\pgfqpoint{4.943207in}{4.711604in}}%
\pgfpathlineto{\pgfqpoint{4.943207in}{4.799340in}}%
\pgfusepath{stroke,fill}%
\end{pgfscope}%
\begin{pgfscope}%
\pgfpathrectangle{\pgfqpoint{0.380943in}{4.185189in}}{\pgfqpoint{4.650000in}{0.614151in}}%
\pgfusepath{clip}%
\pgfsetbuttcap%
\pgfsetroundjoin%
\definecolor{currentfill}{rgb}{0.972549,0.870588,0.692810}%
\pgfsetfillcolor{currentfill}%
\pgfsetlinewidth{0.250937pt}%
\definecolor{currentstroke}{rgb}{1.000000,1.000000,1.000000}%
\pgfsetstrokecolor{currentstroke}%
\pgfsetdash{}{0pt}%
\pgfpathmoveto{\pgfqpoint{0.380943in}{4.711604in}}%
\pgfpathlineto{\pgfqpoint{0.468679in}{4.711604in}}%
\pgfpathlineto{\pgfqpoint{0.468679in}{4.623868in}}%
\pgfpathlineto{\pgfqpoint{0.380943in}{4.623868in}}%
\pgfpathlineto{\pgfqpoint{0.380943in}{4.711604in}}%
\pgfusepath{stroke,fill}%
\end{pgfscope}%
\begin{pgfscope}%
\pgfpathrectangle{\pgfqpoint{0.380943in}{4.185189in}}{\pgfqpoint{4.650000in}{0.614151in}}%
\pgfusepath{clip}%
\pgfsetbuttcap%
\pgfsetroundjoin%
\definecolor{currentfill}{rgb}{0.979654,0.837186,0.669619}%
\pgfsetfillcolor{currentfill}%
\pgfsetlinewidth{0.250937pt}%
\definecolor{currentstroke}{rgb}{1.000000,1.000000,1.000000}%
\pgfsetstrokecolor{currentstroke}%
\pgfsetdash{}{0pt}%
\pgfpathmoveto{\pgfqpoint{0.468679in}{4.711604in}}%
\pgfpathlineto{\pgfqpoint{0.556415in}{4.711604in}}%
\pgfpathlineto{\pgfqpoint{0.556415in}{4.623868in}}%
\pgfpathlineto{\pgfqpoint{0.468679in}{4.623868in}}%
\pgfpathlineto{\pgfqpoint{0.468679in}{4.711604in}}%
\pgfusepath{stroke,fill}%
\end{pgfscope}%
\begin{pgfscope}%
\pgfpathrectangle{\pgfqpoint{0.380943in}{4.185189in}}{\pgfqpoint{4.650000in}{0.614151in}}%
\pgfusepath{clip}%
\pgfsetbuttcap%
\pgfsetroundjoin%
\definecolor{currentfill}{rgb}{0.998939,0.658962,0.556032}%
\pgfsetfillcolor{currentfill}%
\pgfsetlinewidth{0.250937pt}%
\definecolor{currentstroke}{rgb}{1.000000,1.000000,1.000000}%
\pgfsetstrokecolor{currentstroke}%
\pgfsetdash{}{0pt}%
\pgfpathmoveto{\pgfqpoint{0.556415in}{4.711604in}}%
\pgfpathlineto{\pgfqpoint{0.644151in}{4.711604in}}%
\pgfpathlineto{\pgfqpoint{0.644151in}{4.623868in}}%
\pgfpathlineto{\pgfqpoint{0.556415in}{4.623868in}}%
\pgfpathlineto{\pgfqpoint{0.556415in}{4.711604in}}%
\pgfusepath{stroke,fill}%
\end{pgfscope}%
\begin{pgfscope}%
\pgfpathrectangle{\pgfqpoint{0.380943in}{4.185189in}}{\pgfqpoint{4.650000in}{0.614151in}}%
\pgfusepath{clip}%
\pgfsetbuttcap%
\pgfsetroundjoin%
\definecolor{currentfill}{rgb}{0.979654,0.837186,0.669619}%
\pgfsetfillcolor{currentfill}%
\pgfsetlinewidth{0.250937pt}%
\definecolor{currentstroke}{rgb}{1.000000,1.000000,1.000000}%
\pgfsetstrokecolor{currentstroke}%
\pgfsetdash{}{0pt}%
\pgfpathmoveto{\pgfqpoint{0.644151in}{4.711604in}}%
\pgfpathlineto{\pgfqpoint{0.731886in}{4.711604in}}%
\pgfpathlineto{\pgfqpoint{0.731886in}{4.623868in}}%
\pgfpathlineto{\pgfqpoint{0.644151in}{4.623868in}}%
\pgfpathlineto{\pgfqpoint{0.644151in}{4.711604in}}%
\pgfusepath{stroke,fill}%
\end{pgfscope}%
\begin{pgfscope}%
\pgfpathrectangle{\pgfqpoint{0.380943in}{4.185189in}}{\pgfqpoint{4.650000in}{0.614151in}}%
\pgfusepath{clip}%
\pgfsetbuttcap%
\pgfsetroundjoin%
\definecolor{currentfill}{rgb}{0.992326,0.765229,0.614840}%
\pgfsetfillcolor{currentfill}%
\pgfsetlinewidth{0.250937pt}%
\definecolor{currentstroke}{rgb}{1.000000,1.000000,1.000000}%
\pgfsetstrokecolor{currentstroke}%
\pgfsetdash{}{0pt}%
\pgfpathmoveto{\pgfqpoint{0.731886in}{4.711604in}}%
\pgfpathlineto{\pgfqpoint{0.819622in}{4.711604in}}%
\pgfpathlineto{\pgfqpoint{0.819622in}{4.623868in}}%
\pgfpathlineto{\pgfqpoint{0.731886in}{4.623868in}}%
\pgfpathlineto{\pgfqpoint{0.731886in}{4.711604in}}%
\pgfusepath{stroke,fill}%
\end{pgfscope}%
\begin{pgfscope}%
\pgfpathrectangle{\pgfqpoint{0.380943in}{4.185189in}}{\pgfqpoint{4.650000in}{0.614151in}}%
\pgfusepath{clip}%
\pgfsetbuttcap%
\pgfsetroundjoin%
\definecolor{currentfill}{rgb}{0.996571,0.720538,0.589189}%
\pgfsetfillcolor{currentfill}%
\pgfsetlinewidth{0.250937pt}%
\definecolor{currentstroke}{rgb}{1.000000,1.000000,1.000000}%
\pgfsetstrokecolor{currentstroke}%
\pgfsetdash{}{0pt}%
\pgfpathmoveto{\pgfqpoint{0.819622in}{4.711604in}}%
\pgfpathlineto{\pgfqpoint{0.907358in}{4.711604in}}%
\pgfpathlineto{\pgfqpoint{0.907358in}{4.623868in}}%
\pgfpathlineto{\pgfqpoint{0.819622in}{4.623868in}}%
\pgfpathlineto{\pgfqpoint{0.819622in}{4.711604in}}%
\pgfusepath{stroke,fill}%
\end{pgfscope}%
\begin{pgfscope}%
\pgfpathrectangle{\pgfqpoint{0.380943in}{4.185189in}}{\pgfqpoint{4.650000in}{0.614151in}}%
\pgfusepath{clip}%
\pgfsetbuttcap%
\pgfsetroundjoin%
\definecolor{currentfill}{rgb}{0.996571,0.720538,0.589189}%
\pgfsetfillcolor{currentfill}%
\pgfsetlinewidth{0.250937pt}%
\definecolor{currentstroke}{rgb}{1.000000,1.000000,1.000000}%
\pgfsetstrokecolor{currentstroke}%
\pgfsetdash{}{0pt}%
\pgfpathmoveto{\pgfqpoint{0.907358in}{4.711604in}}%
\pgfpathlineto{\pgfqpoint{0.995094in}{4.711604in}}%
\pgfpathlineto{\pgfqpoint{0.995094in}{4.623868in}}%
\pgfpathlineto{\pgfqpoint{0.907358in}{4.623868in}}%
\pgfpathlineto{\pgfqpoint{0.907358in}{4.711604in}}%
\pgfusepath{stroke,fill}%
\end{pgfscope}%
\begin{pgfscope}%
\pgfpathrectangle{\pgfqpoint{0.380943in}{4.185189in}}{\pgfqpoint{4.650000in}{0.614151in}}%
\pgfusepath{clip}%
\pgfsetbuttcap%
\pgfsetroundjoin%
\definecolor{currentfill}{rgb}{0.996571,0.720538,0.589189}%
\pgfsetfillcolor{currentfill}%
\pgfsetlinewidth{0.250937pt}%
\definecolor{currentstroke}{rgb}{1.000000,1.000000,1.000000}%
\pgfsetstrokecolor{currentstroke}%
\pgfsetdash{}{0pt}%
\pgfpathmoveto{\pgfqpoint{0.995094in}{4.711604in}}%
\pgfpathlineto{\pgfqpoint{1.082830in}{4.711604in}}%
\pgfpathlineto{\pgfqpoint{1.082830in}{4.623868in}}%
\pgfpathlineto{\pgfqpoint{0.995094in}{4.623868in}}%
\pgfpathlineto{\pgfqpoint{0.995094in}{4.711604in}}%
\pgfusepath{stroke,fill}%
\end{pgfscope}%
\begin{pgfscope}%
\pgfpathrectangle{\pgfqpoint{0.380943in}{4.185189in}}{\pgfqpoint{4.650000in}{0.614151in}}%
\pgfusepath{clip}%
\pgfsetbuttcap%
\pgfsetroundjoin%
\definecolor{currentfill}{rgb}{0.986759,0.806398,0.641200}%
\pgfsetfillcolor{currentfill}%
\pgfsetlinewidth{0.250937pt}%
\definecolor{currentstroke}{rgb}{1.000000,1.000000,1.000000}%
\pgfsetstrokecolor{currentstroke}%
\pgfsetdash{}{0pt}%
\pgfpathmoveto{\pgfqpoint{1.082830in}{4.711604in}}%
\pgfpathlineto{\pgfqpoint{1.170566in}{4.711604in}}%
\pgfpathlineto{\pgfqpoint{1.170566in}{4.623868in}}%
\pgfpathlineto{\pgfqpoint{1.082830in}{4.623868in}}%
\pgfpathlineto{\pgfqpoint{1.082830in}{4.711604in}}%
\pgfusepath{stroke,fill}%
\end{pgfscope}%
\begin{pgfscope}%
\pgfpathrectangle{\pgfqpoint{0.380943in}{4.185189in}}{\pgfqpoint{4.650000in}{0.614151in}}%
\pgfusepath{clip}%
\pgfsetbuttcap%
\pgfsetroundjoin%
\definecolor{currentfill}{rgb}{0.986759,0.806398,0.641200}%
\pgfsetfillcolor{currentfill}%
\pgfsetlinewidth{0.250937pt}%
\definecolor{currentstroke}{rgb}{1.000000,1.000000,1.000000}%
\pgfsetstrokecolor{currentstroke}%
\pgfsetdash{}{0pt}%
\pgfpathmoveto{\pgfqpoint{1.170566in}{4.711604in}}%
\pgfpathlineto{\pgfqpoint{1.258302in}{4.711604in}}%
\pgfpathlineto{\pgfqpoint{1.258302in}{4.623868in}}%
\pgfpathlineto{\pgfqpoint{1.170566in}{4.623868in}}%
\pgfpathlineto{\pgfqpoint{1.170566in}{4.711604in}}%
\pgfusepath{stroke,fill}%
\end{pgfscope}%
\begin{pgfscope}%
\pgfpathrectangle{\pgfqpoint{0.380943in}{4.185189in}}{\pgfqpoint{4.650000in}{0.614151in}}%
\pgfusepath{clip}%
\pgfsetbuttcap%
\pgfsetroundjoin%
\definecolor{currentfill}{rgb}{0.992326,0.765229,0.614840}%
\pgfsetfillcolor{currentfill}%
\pgfsetlinewidth{0.250937pt}%
\definecolor{currentstroke}{rgb}{1.000000,1.000000,1.000000}%
\pgfsetstrokecolor{currentstroke}%
\pgfsetdash{}{0pt}%
\pgfpathmoveto{\pgfqpoint{1.258302in}{4.711604in}}%
\pgfpathlineto{\pgfqpoint{1.346037in}{4.711604in}}%
\pgfpathlineto{\pgfqpoint{1.346037in}{4.623868in}}%
\pgfpathlineto{\pgfqpoint{1.258302in}{4.623868in}}%
\pgfpathlineto{\pgfqpoint{1.258302in}{4.711604in}}%
\pgfusepath{stroke,fill}%
\end{pgfscope}%
\begin{pgfscope}%
\pgfpathrectangle{\pgfqpoint{0.380943in}{4.185189in}}{\pgfqpoint{4.650000in}{0.614151in}}%
\pgfusepath{clip}%
\pgfsetbuttcap%
\pgfsetroundjoin%
\definecolor{currentfill}{rgb}{0.992326,0.765229,0.614840}%
\pgfsetfillcolor{currentfill}%
\pgfsetlinewidth{0.250937pt}%
\definecolor{currentstroke}{rgb}{1.000000,1.000000,1.000000}%
\pgfsetstrokecolor{currentstroke}%
\pgfsetdash{}{0pt}%
\pgfpathmoveto{\pgfqpoint{1.346037in}{4.711604in}}%
\pgfpathlineto{\pgfqpoint{1.433773in}{4.711604in}}%
\pgfpathlineto{\pgfqpoint{1.433773in}{4.623868in}}%
\pgfpathlineto{\pgfqpoint{1.346037in}{4.623868in}}%
\pgfpathlineto{\pgfqpoint{1.346037in}{4.711604in}}%
\pgfusepath{stroke,fill}%
\end{pgfscope}%
\begin{pgfscope}%
\pgfpathrectangle{\pgfqpoint{0.380943in}{4.185189in}}{\pgfqpoint{4.650000in}{0.614151in}}%
\pgfusepath{clip}%
\pgfsetbuttcap%
\pgfsetroundjoin%
\definecolor{currentfill}{rgb}{0.992326,0.765229,0.614840}%
\pgfsetfillcolor{currentfill}%
\pgfsetlinewidth{0.250937pt}%
\definecolor{currentstroke}{rgb}{1.000000,1.000000,1.000000}%
\pgfsetstrokecolor{currentstroke}%
\pgfsetdash{}{0pt}%
\pgfpathmoveto{\pgfqpoint{1.433773in}{4.711604in}}%
\pgfpathlineto{\pgfqpoint{1.521509in}{4.711604in}}%
\pgfpathlineto{\pgfqpoint{1.521509in}{4.623868in}}%
\pgfpathlineto{\pgfqpoint{1.433773in}{4.623868in}}%
\pgfpathlineto{\pgfqpoint{1.433773in}{4.711604in}}%
\pgfusepath{stroke,fill}%
\end{pgfscope}%
\begin{pgfscope}%
\pgfpathrectangle{\pgfqpoint{0.380943in}{4.185189in}}{\pgfqpoint{4.650000in}{0.614151in}}%
\pgfusepath{clip}%
\pgfsetbuttcap%
\pgfsetroundjoin%
\definecolor{currentfill}{rgb}{0.979654,0.837186,0.669619}%
\pgfsetfillcolor{currentfill}%
\pgfsetlinewidth{0.250937pt}%
\definecolor{currentstroke}{rgb}{1.000000,1.000000,1.000000}%
\pgfsetstrokecolor{currentstroke}%
\pgfsetdash{}{0pt}%
\pgfpathmoveto{\pgfqpoint{1.521509in}{4.711604in}}%
\pgfpathlineto{\pgfqpoint{1.609245in}{4.711604in}}%
\pgfpathlineto{\pgfqpoint{1.609245in}{4.623868in}}%
\pgfpathlineto{\pgfqpoint{1.521509in}{4.623868in}}%
\pgfpathlineto{\pgfqpoint{1.521509in}{4.711604in}}%
\pgfusepath{stroke,fill}%
\end{pgfscope}%
\begin{pgfscope}%
\pgfpathrectangle{\pgfqpoint{0.380943in}{4.185189in}}{\pgfqpoint{4.650000in}{0.614151in}}%
\pgfusepath{clip}%
\pgfsetbuttcap%
\pgfsetroundjoin%
\definecolor{currentfill}{rgb}{1.000000,0.557862,0.511772}%
\pgfsetfillcolor{currentfill}%
\pgfsetlinewidth{0.250937pt}%
\definecolor{currentstroke}{rgb}{1.000000,1.000000,1.000000}%
\pgfsetstrokecolor{currentstroke}%
\pgfsetdash{}{0pt}%
\pgfpathmoveto{\pgfqpoint{1.609245in}{4.711604in}}%
\pgfpathlineto{\pgfqpoint{1.696981in}{4.711604in}}%
\pgfpathlineto{\pgfqpoint{1.696981in}{4.623868in}}%
\pgfpathlineto{\pgfqpoint{1.609245in}{4.623868in}}%
\pgfpathlineto{\pgfqpoint{1.609245in}{4.711604in}}%
\pgfusepath{stroke,fill}%
\end{pgfscope}%
\begin{pgfscope}%
\pgfpathrectangle{\pgfqpoint{0.380943in}{4.185189in}}{\pgfqpoint{4.650000in}{0.614151in}}%
\pgfusepath{clip}%
\pgfsetbuttcap%
\pgfsetroundjoin%
\definecolor{currentfill}{rgb}{0.998939,0.658962,0.556032}%
\pgfsetfillcolor{currentfill}%
\pgfsetlinewidth{0.250937pt}%
\definecolor{currentstroke}{rgb}{1.000000,1.000000,1.000000}%
\pgfsetstrokecolor{currentstroke}%
\pgfsetdash{}{0pt}%
\pgfpathmoveto{\pgfqpoint{1.696981in}{4.711604in}}%
\pgfpathlineto{\pgfqpoint{1.784717in}{4.711604in}}%
\pgfpathlineto{\pgfqpoint{1.784717in}{4.623868in}}%
\pgfpathlineto{\pgfqpoint{1.696981in}{4.623868in}}%
\pgfpathlineto{\pgfqpoint{1.696981in}{4.711604in}}%
\pgfusepath{stroke,fill}%
\end{pgfscope}%
\begin{pgfscope}%
\pgfpathrectangle{\pgfqpoint{0.380943in}{4.185189in}}{\pgfqpoint{4.650000in}{0.614151in}}%
\pgfusepath{clip}%
\pgfsetbuttcap%
\pgfsetroundjoin%
\definecolor{currentfill}{rgb}{0.992326,0.765229,0.614840}%
\pgfsetfillcolor{currentfill}%
\pgfsetlinewidth{0.250937pt}%
\definecolor{currentstroke}{rgb}{1.000000,1.000000,1.000000}%
\pgfsetstrokecolor{currentstroke}%
\pgfsetdash{}{0pt}%
\pgfpathmoveto{\pgfqpoint{1.784717in}{4.711604in}}%
\pgfpathlineto{\pgfqpoint{1.872452in}{4.711604in}}%
\pgfpathlineto{\pgfqpoint{1.872452in}{4.623868in}}%
\pgfpathlineto{\pgfqpoint{1.784717in}{4.623868in}}%
\pgfpathlineto{\pgfqpoint{1.784717in}{4.711604in}}%
\pgfusepath{stroke,fill}%
\end{pgfscope}%
\begin{pgfscope}%
\pgfpathrectangle{\pgfqpoint{0.380943in}{4.185189in}}{\pgfqpoint{4.650000in}{0.614151in}}%
\pgfusepath{clip}%
\pgfsetbuttcap%
\pgfsetroundjoin%
\definecolor{currentfill}{rgb}{0.998939,0.658962,0.556032}%
\pgfsetfillcolor{currentfill}%
\pgfsetlinewidth{0.250937pt}%
\definecolor{currentstroke}{rgb}{1.000000,1.000000,1.000000}%
\pgfsetstrokecolor{currentstroke}%
\pgfsetdash{}{0pt}%
\pgfpathmoveto{\pgfqpoint{1.872452in}{4.711604in}}%
\pgfpathlineto{\pgfqpoint{1.960188in}{4.711604in}}%
\pgfpathlineto{\pgfqpoint{1.960188in}{4.623868in}}%
\pgfpathlineto{\pgfqpoint{1.872452in}{4.623868in}}%
\pgfpathlineto{\pgfqpoint{1.872452in}{4.711604in}}%
\pgfusepath{stroke,fill}%
\end{pgfscope}%
\begin{pgfscope}%
\pgfpathrectangle{\pgfqpoint{0.380943in}{4.185189in}}{\pgfqpoint{4.650000in}{0.614151in}}%
\pgfusepath{clip}%
\pgfsetbuttcap%
\pgfsetroundjoin%
\definecolor{currentfill}{rgb}{0.998939,0.658962,0.556032}%
\pgfsetfillcolor{currentfill}%
\pgfsetlinewidth{0.250937pt}%
\definecolor{currentstroke}{rgb}{1.000000,1.000000,1.000000}%
\pgfsetstrokecolor{currentstroke}%
\pgfsetdash{}{0pt}%
\pgfpathmoveto{\pgfqpoint{1.960188in}{4.711604in}}%
\pgfpathlineto{\pgfqpoint{2.047924in}{4.711604in}}%
\pgfpathlineto{\pgfqpoint{2.047924in}{4.623868in}}%
\pgfpathlineto{\pgfqpoint{1.960188in}{4.623868in}}%
\pgfpathlineto{\pgfqpoint{1.960188in}{4.711604in}}%
\pgfusepath{stroke,fill}%
\end{pgfscope}%
\begin{pgfscope}%
\pgfpathrectangle{\pgfqpoint{0.380943in}{4.185189in}}{\pgfqpoint{4.650000in}{0.614151in}}%
\pgfusepath{clip}%
\pgfsetbuttcap%
\pgfsetroundjoin%
\definecolor{currentfill}{rgb}{1.000000,0.509404,0.491473}%
\pgfsetfillcolor{currentfill}%
\pgfsetlinewidth{0.250937pt}%
\definecolor{currentstroke}{rgb}{1.000000,1.000000,1.000000}%
\pgfsetstrokecolor{currentstroke}%
\pgfsetdash{}{0pt}%
\pgfpathmoveto{\pgfqpoint{2.047924in}{4.711604in}}%
\pgfpathlineto{\pgfqpoint{2.135660in}{4.711604in}}%
\pgfpathlineto{\pgfqpoint{2.135660in}{4.623868in}}%
\pgfpathlineto{\pgfqpoint{2.047924in}{4.623868in}}%
\pgfpathlineto{\pgfqpoint{2.047924in}{4.711604in}}%
\pgfusepath{stroke,fill}%
\end{pgfscope}%
\begin{pgfscope}%
\pgfpathrectangle{\pgfqpoint{0.380943in}{4.185189in}}{\pgfqpoint{4.650000in}{0.614151in}}%
\pgfusepath{clip}%
\pgfsetbuttcap%
\pgfsetroundjoin%
\definecolor{currentfill}{rgb}{1.000000,0.509404,0.491473}%
\pgfsetfillcolor{currentfill}%
\pgfsetlinewidth{0.250937pt}%
\definecolor{currentstroke}{rgb}{1.000000,1.000000,1.000000}%
\pgfsetstrokecolor{currentstroke}%
\pgfsetdash{}{0pt}%
\pgfpathmoveto{\pgfqpoint{2.135660in}{4.711604in}}%
\pgfpathlineto{\pgfqpoint{2.223396in}{4.711604in}}%
\pgfpathlineto{\pgfqpoint{2.223396in}{4.623868in}}%
\pgfpathlineto{\pgfqpoint{2.135660in}{4.623868in}}%
\pgfpathlineto{\pgfqpoint{2.135660in}{4.711604in}}%
\pgfusepath{stroke,fill}%
\end{pgfscope}%
\begin{pgfscope}%
\pgfpathrectangle{\pgfqpoint{0.380943in}{4.185189in}}{\pgfqpoint{4.650000in}{0.614151in}}%
\pgfusepath{clip}%
\pgfsetbuttcap%
\pgfsetroundjoin%
\definecolor{currentfill}{rgb}{0.979654,0.837186,0.669619}%
\pgfsetfillcolor{currentfill}%
\pgfsetlinewidth{0.250937pt}%
\definecolor{currentstroke}{rgb}{1.000000,1.000000,1.000000}%
\pgfsetstrokecolor{currentstroke}%
\pgfsetdash{}{0pt}%
\pgfpathmoveto{\pgfqpoint{2.223396in}{4.711604in}}%
\pgfpathlineto{\pgfqpoint{2.311132in}{4.711604in}}%
\pgfpathlineto{\pgfqpoint{2.311132in}{4.623868in}}%
\pgfpathlineto{\pgfqpoint{2.223396in}{4.623868in}}%
\pgfpathlineto{\pgfqpoint{2.223396in}{4.711604in}}%
\pgfusepath{stroke,fill}%
\end{pgfscope}%
\begin{pgfscope}%
\pgfpathrectangle{\pgfqpoint{0.380943in}{4.185189in}}{\pgfqpoint{4.650000in}{0.614151in}}%
\pgfusepath{clip}%
\pgfsetbuttcap%
\pgfsetroundjoin%
\definecolor{currentfill}{rgb}{0.996571,0.720538,0.589189}%
\pgfsetfillcolor{currentfill}%
\pgfsetlinewidth{0.250937pt}%
\definecolor{currentstroke}{rgb}{1.000000,1.000000,1.000000}%
\pgfsetstrokecolor{currentstroke}%
\pgfsetdash{}{0pt}%
\pgfpathmoveto{\pgfqpoint{2.311132in}{4.711604in}}%
\pgfpathlineto{\pgfqpoint{2.398868in}{4.711604in}}%
\pgfpathlineto{\pgfqpoint{2.398868in}{4.623868in}}%
\pgfpathlineto{\pgfqpoint{2.311132in}{4.623868in}}%
\pgfpathlineto{\pgfqpoint{2.311132in}{4.711604in}}%
\pgfusepath{stroke,fill}%
\end{pgfscope}%
\begin{pgfscope}%
\pgfpathrectangle{\pgfqpoint{0.380943in}{4.185189in}}{\pgfqpoint{4.650000in}{0.614151in}}%
\pgfusepath{clip}%
\pgfsetbuttcap%
\pgfsetroundjoin%
\definecolor{currentfill}{rgb}{0.992326,0.765229,0.614840}%
\pgfsetfillcolor{currentfill}%
\pgfsetlinewidth{0.250937pt}%
\definecolor{currentstroke}{rgb}{1.000000,1.000000,1.000000}%
\pgfsetstrokecolor{currentstroke}%
\pgfsetdash{}{0pt}%
\pgfpathmoveto{\pgfqpoint{2.398868in}{4.711604in}}%
\pgfpathlineto{\pgfqpoint{2.486603in}{4.711604in}}%
\pgfpathlineto{\pgfqpoint{2.486603in}{4.623868in}}%
\pgfpathlineto{\pgfqpoint{2.398868in}{4.623868in}}%
\pgfpathlineto{\pgfqpoint{2.398868in}{4.711604in}}%
\pgfusepath{stroke,fill}%
\end{pgfscope}%
\begin{pgfscope}%
\pgfpathrectangle{\pgfqpoint{0.380943in}{4.185189in}}{\pgfqpoint{4.650000in}{0.614151in}}%
\pgfusepath{clip}%
\pgfsetbuttcap%
\pgfsetroundjoin%
\definecolor{currentfill}{rgb}{0.981546,0.459977,0.459977}%
\pgfsetfillcolor{currentfill}%
\pgfsetlinewidth{0.250937pt}%
\definecolor{currentstroke}{rgb}{1.000000,1.000000,1.000000}%
\pgfsetstrokecolor{currentstroke}%
\pgfsetdash{}{0pt}%
\pgfpathmoveto{\pgfqpoint{2.486603in}{4.711604in}}%
\pgfpathlineto{\pgfqpoint{2.574339in}{4.711604in}}%
\pgfpathlineto{\pgfqpoint{2.574339in}{4.623868in}}%
\pgfpathlineto{\pgfqpoint{2.486603in}{4.623868in}}%
\pgfpathlineto{\pgfqpoint{2.486603in}{4.711604in}}%
\pgfusepath{stroke,fill}%
\end{pgfscope}%
\begin{pgfscope}%
\pgfpathrectangle{\pgfqpoint{0.380943in}{4.185189in}}{\pgfqpoint{4.650000in}{0.614151in}}%
\pgfusepath{clip}%
\pgfsetbuttcap%
\pgfsetroundjoin%
\definecolor{currentfill}{rgb}{0.979654,0.837186,0.669619}%
\pgfsetfillcolor{currentfill}%
\pgfsetlinewidth{0.250937pt}%
\definecolor{currentstroke}{rgb}{1.000000,1.000000,1.000000}%
\pgfsetstrokecolor{currentstroke}%
\pgfsetdash{}{0pt}%
\pgfpathmoveto{\pgfqpoint{2.574339in}{4.711604in}}%
\pgfpathlineto{\pgfqpoint{2.662075in}{4.711604in}}%
\pgfpathlineto{\pgfqpoint{2.662075in}{4.623868in}}%
\pgfpathlineto{\pgfqpoint{2.574339in}{4.623868in}}%
\pgfpathlineto{\pgfqpoint{2.574339in}{4.711604in}}%
\pgfusepath{stroke,fill}%
\end{pgfscope}%
\begin{pgfscope}%
\pgfpathrectangle{\pgfqpoint{0.380943in}{4.185189in}}{\pgfqpoint{4.650000in}{0.614151in}}%
\pgfusepath{clip}%
\pgfsetbuttcap%
\pgfsetroundjoin%
\definecolor{currentfill}{rgb}{1.000000,0.557862,0.511772}%
\pgfsetfillcolor{currentfill}%
\pgfsetlinewidth{0.250937pt}%
\definecolor{currentstroke}{rgb}{1.000000,1.000000,1.000000}%
\pgfsetstrokecolor{currentstroke}%
\pgfsetdash{}{0pt}%
\pgfpathmoveto{\pgfqpoint{2.662075in}{4.711604in}}%
\pgfpathlineto{\pgfqpoint{2.749811in}{4.711604in}}%
\pgfpathlineto{\pgfqpoint{2.749811in}{4.623868in}}%
\pgfpathlineto{\pgfqpoint{2.662075in}{4.623868in}}%
\pgfpathlineto{\pgfqpoint{2.662075in}{4.711604in}}%
\pgfusepath{stroke,fill}%
\end{pgfscope}%
\begin{pgfscope}%
\pgfpathrectangle{\pgfqpoint{0.380943in}{4.185189in}}{\pgfqpoint{4.650000in}{0.614151in}}%
\pgfusepath{clip}%
\pgfsetbuttcap%
\pgfsetroundjoin%
\definecolor{currentfill}{rgb}{0.996571,0.720538,0.589189}%
\pgfsetfillcolor{currentfill}%
\pgfsetlinewidth{0.250937pt}%
\definecolor{currentstroke}{rgb}{1.000000,1.000000,1.000000}%
\pgfsetstrokecolor{currentstroke}%
\pgfsetdash{}{0pt}%
\pgfpathmoveto{\pgfqpoint{2.749811in}{4.711604in}}%
\pgfpathlineto{\pgfqpoint{2.837547in}{4.711604in}}%
\pgfpathlineto{\pgfqpoint{2.837547in}{4.623868in}}%
\pgfpathlineto{\pgfqpoint{2.749811in}{4.623868in}}%
\pgfpathlineto{\pgfqpoint{2.749811in}{4.711604in}}%
\pgfusepath{stroke,fill}%
\end{pgfscope}%
\begin{pgfscope}%
\pgfpathrectangle{\pgfqpoint{0.380943in}{4.185189in}}{\pgfqpoint{4.650000in}{0.614151in}}%
\pgfusepath{clip}%
\pgfsetbuttcap%
\pgfsetroundjoin%
\definecolor{currentfill}{rgb}{0.965444,0.906113,0.711757}%
\pgfsetfillcolor{currentfill}%
\pgfsetlinewidth{0.250937pt}%
\definecolor{currentstroke}{rgb}{1.000000,1.000000,1.000000}%
\pgfsetstrokecolor{currentstroke}%
\pgfsetdash{}{0pt}%
\pgfpathmoveto{\pgfqpoint{2.837547in}{4.711604in}}%
\pgfpathlineto{\pgfqpoint{2.925283in}{4.711604in}}%
\pgfpathlineto{\pgfqpoint{2.925283in}{4.623868in}}%
\pgfpathlineto{\pgfqpoint{2.837547in}{4.623868in}}%
\pgfpathlineto{\pgfqpoint{2.837547in}{4.711604in}}%
\pgfusepath{stroke,fill}%
\end{pgfscope}%
\begin{pgfscope}%
\pgfpathrectangle{\pgfqpoint{0.380943in}{4.185189in}}{\pgfqpoint{4.650000in}{0.614151in}}%
\pgfusepath{clip}%
\pgfsetbuttcap%
\pgfsetroundjoin%
\definecolor{currentfill}{rgb}{0.962414,0.923552,0.722891}%
\pgfsetfillcolor{currentfill}%
\pgfsetlinewidth{0.250937pt}%
\definecolor{currentstroke}{rgb}{1.000000,1.000000,1.000000}%
\pgfsetstrokecolor{currentstroke}%
\pgfsetdash{}{0pt}%
\pgfpathmoveto{\pgfqpoint{2.925283in}{4.711604in}}%
\pgfpathlineto{\pgfqpoint{3.013019in}{4.711604in}}%
\pgfpathlineto{\pgfqpoint{3.013019in}{4.623868in}}%
\pgfpathlineto{\pgfqpoint{2.925283in}{4.623868in}}%
\pgfpathlineto{\pgfqpoint{2.925283in}{4.711604in}}%
\pgfusepath{stroke,fill}%
\end{pgfscope}%
\begin{pgfscope}%
\pgfpathrectangle{\pgfqpoint{0.380943in}{4.185189in}}{\pgfqpoint{4.650000in}{0.614151in}}%
\pgfusepath{clip}%
\pgfsetbuttcap%
\pgfsetroundjoin%
\definecolor{currentfill}{rgb}{0.996571,0.720538,0.589189}%
\pgfsetfillcolor{currentfill}%
\pgfsetlinewidth{0.250937pt}%
\definecolor{currentstroke}{rgb}{1.000000,1.000000,1.000000}%
\pgfsetstrokecolor{currentstroke}%
\pgfsetdash{}{0pt}%
\pgfpathmoveto{\pgfqpoint{3.013019in}{4.711604in}}%
\pgfpathlineto{\pgfqpoint{3.100754in}{4.711604in}}%
\pgfpathlineto{\pgfqpoint{3.100754in}{4.623868in}}%
\pgfpathlineto{\pgfqpoint{3.013019in}{4.623868in}}%
\pgfpathlineto{\pgfqpoint{3.013019in}{4.711604in}}%
\pgfusepath{stroke,fill}%
\end{pgfscope}%
\begin{pgfscope}%
\pgfpathrectangle{\pgfqpoint{0.380943in}{4.185189in}}{\pgfqpoint{4.650000in}{0.614151in}}%
\pgfusepath{clip}%
\pgfsetbuttcap%
\pgfsetroundjoin%
\definecolor{currentfill}{rgb}{0.972549,0.870588,0.692810}%
\pgfsetfillcolor{currentfill}%
\pgfsetlinewidth{0.250937pt}%
\definecolor{currentstroke}{rgb}{1.000000,1.000000,1.000000}%
\pgfsetstrokecolor{currentstroke}%
\pgfsetdash{}{0pt}%
\pgfpathmoveto{\pgfqpoint{3.100754in}{4.711604in}}%
\pgfpathlineto{\pgfqpoint{3.188490in}{4.711604in}}%
\pgfpathlineto{\pgfqpoint{3.188490in}{4.623868in}}%
\pgfpathlineto{\pgfqpoint{3.100754in}{4.623868in}}%
\pgfpathlineto{\pgfqpoint{3.100754in}{4.711604in}}%
\pgfusepath{stroke,fill}%
\end{pgfscope}%
\begin{pgfscope}%
\pgfpathrectangle{\pgfqpoint{0.380943in}{4.185189in}}{\pgfqpoint{4.650000in}{0.614151in}}%
\pgfusepath{clip}%
\pgfsetbuttcap%
\pgfsetroundjoin%
\definecolor{currentfill}{rgb}{0.965444,0.906113,0.711757}%
\pgfsetfillcolor{currentfill}%
\pgfsetlinewidth{0.250937pt}%
\definecolor{currentstroke}{rgb}{1.000000,1.000000,1.000000}%
\pgfsetstrokecolor{currentstroke}%
\pgfsetdash{}{0pt}%
\pgfpathmoveto{\pgfqpoint{3.188490in}{4.711604in}}%
\pgfpathlineto{\pgfqpoint{3.276226in}{4.711604in}}%
\pgfpathlineto{\pgfqpoint{3.276226in}{4.623868in}}%
\pgfpathlineto{\pgfqpoint{3.188490in}{4.623868in}}%
\pgfpathlineto{\pgfqpoint{3.188490in}{4.711604in}}%
\pgfusepath{stroke,fill}%
\end{pgfscope}%
\begin{pgfscope}%
\pgfpathrectangle{\pgfqpoint{0.380943in}{4.185189in}}{\pgfqpoint{4.650000in}{0.614151in}}%
\pgfusepath{clip}%
\pgfsetbuttcap%
\pgfsetroundjoin%
\definecolor{currentfill}{rgb}{0.968166,0.945882,0.748604}%
\pgfsetfillcolor{currentfill}%
\pgfsetlinewidth{0.250937pt}%
\definecolor{currentstroke}{rgb}{1.000000,1.000000,1.000000}%
\pgfsetstrokecolor{currentstroke}%
\pgfsetdash{}{0pt}%
\pgfpathmoveto{\pgfqpoint{3.276226in}{4.711604in}}%
\pgfpathlineto{\pgfqpoint{3.363962in}{4.711604in}}%
\pgfpathlineto{\pgfqpoint{3.363962in}{4.623868in}}%
\pgfpathlineto{\pgfqpoint{3.276226in}{4.623868in}}%
\pgfpathlineto{\pgfqpoint{3.276226in}{4.711604in}}%
\pgfusepath{stroke,fill}%
\end{pgfscope}%
\begin{pgfscope}%
\pgfpathrectangle{\pgfqpoint{0.380943in}{4.185189in}}{\pgfqpoint{4.650000in}{0.614151in}}%
\pgfusepath{clip}%
\pgfsetbuttcap%
\pgfsetroundjoin%
\definecolor{currentfill}{rgb}{0.998939,0.658962,0.556032}%
\pgfsetfillcolor{currentfill}%
\pgfsetlinewidth{0.250937pt}%
\definecolor{currentstroke}{rgb}{1.000000,1.000000,1.000000}%
\pgfsetstrokecolor{currentstroke}%
\pgfsetdash{}{0pt}%
\pgfpathmoveto{\pgfqpoint{3.363962in}{4.711604in}}%
\pgfpathlineto{\pgfqpoint{3.451698in}{4.711604in}}%
\pgfpathlineto{\pgfqpoint{3.451698in}{4.623868in}}%
\pgfpathlineto{\pgfqpoint{3.363962in}{4.623868in}}%
\pgfpathlineto{\pgfqpoint{3.363962in}{4.711604in}}%
\pgfusepath{stroke,fill}%
\end{pgfscope}%
\begin{pgfscope}%
\pgfpathrectangle{\pgfqpoint{0.380943in}{4.185189in}}{\pgfqpoint{4.650000in}{0.614151in}}%
\pgfusepath{clip}%
\pgfsetbuttcap%
\pgfsetroundjoin%
\definecolor{currentfill}{rgb}{0.979654,0.837186,0.669619}%
\pgfsetfillcolor{currentfill}%
\pgfsetlinewidth{0.250937pt}%
\definecolor{currentstroke}{rgb}{1.000000,1.000000,1.000000}%
\pgfsetstrokecolor{currentstroke}%
\pgfsetdash{}{0pt}%
\pgfpathmoveto{\pgfqpoint{3.451698in}{4.711604in}}%
\pgfpathlineto{\pgfqpoint{3.539434in}{4.711604in}}%
\pgfpathlineto{\pgfqpoint{3.539434in}{4.623868in}}%
\pgfpathlineto{\pgfqpoint{3.451698in}{4.623868in}}%
\pgfpathlineto{\pgfqpoint{3.451698in}{4.711604in}}%
\pgfusepath{stroke,fill}%
\end{pgfscope}%
\begin{pgfscope}%
\pgfpathrectangle{\pgfqpoint{0.380943in}{4.185189in}}{\pgfqpoint{4.650000in}{0.614151in}}%
\pgfusepath{clip}%
\pgfsetbuttcap%
\pgfsetroundjoin%
\definecolor{currentfill}{rgb}{0.965444,0.906113,0.711757}%
\pgfsetfillcolor{currentfill}%
\pgfsetlinewidth{0.250937pt}%
\definecolor{currentstroke}{rgb}{1.000000,1.000000,1.000000}%
\pgfsetstrokecolor{currentstroke}%
\pgfsetdash{}{0pt}%
\pgfpathmoveto{\pgfqpoint{3.539434in}{4.711604in}}%
\pgfpathlineto{\pgfqpoint{3.627169in}{4.711604in}}%
\pgfpathlineto{\pgfqpoint{3.627169in}{4.623868in}}%
\pgfpathlineto{\pgfqpoint{3.539434in}{4.623868in}}%
\pgfpathlineto{\pgfqpoint{3.539434in}{4.711604in}}%
\pgfusepath{stroke,fill}%
\end{pgfscope}%
\begin{pgfscope}%
\pgfpathrectangle{\pgfqpoint{0.380943in}{4.185189in}}{\pgfqpoint{4.650000in}{0.614151in}}%
\pgfusepath{clip}%
\pgfsetbuttcap%
\pgfsetroundjoin%
\definecolor{currentfill}{rgb}{0.986759,0.806398,0.641200}%
\pgfsetfillcolor{currentfill}%
\pgfsetlinewidth{0.250937pt}%
\definecolor{currentstroke}{rgb}{1.000000,1.000000,1.000000}%
\pgfsetstrokecolor{currentstroke}%
\pgfsetdash{}{0pt}%
\pgfpathmoveto{\pgfqpoint{3.627169in}{4.711604in}}%
\pgfpathlineto{\pgfqpoint{3.714905in}{4.711604in}}%
\pgfpathlineto{\pgfqpoint{3.714905in}{4.623868in}}%
\pgfpathlineto{\pgfqpoint{3.627169in}{4.623868in}}%
\pgfpathlineto{\pgfqpoint{3.627169in}{4.711604in}}%
\pgfusepath{stroke,fill}%
\end{pgfscope}%
\begin{pgfscope}%
\pgfpathrectangle{\pgfqpoint{0.380943in}{4.185189in}}{\pgfqpoint{4.650000in}{0.614151in}}%
\pgfusepath{clip}%
\pgfsetbuttcap%
\pgfsetroundjoin%
\definecolor{currentfill}{rgb}{0.979654,0.837186,0.669619}%
\pgfsetfillcolor{currentfill}%
\pgfsetlinewidth{0.250937pt}%
\definecolor{currentstroke}{rgb}{1.000000,1.000000,1.000000}%
\pgfsetstrokecolor{currentstroke}%
\pgfsetdash{}{0pt}%
\pgfpathmoveto{\pgfqpoint{3.714905in}{4.711604in}}%
\pgfpathlineto{\pgfqpoint{3.802641in}{4.711604in}}%
\pgfpathlineto{\pgfqpoint{3.802641in}{4.623868in}}%
\pgfpathlineto{\pgfqpoint{3.714905in}{4.623868in}}%
\pgfpathlineto{\pgfqpoint{3.714905in}{4.711604in}}%
\pgfusepath{stroke,fill}%
\end{pgfscope}%
\begin{pgfscope}%
\pgfpathrectangle{\pgfqpoint{0.380943in}{4.185189in}}{\pgfqpoint{4.650000in}{0.614151in}}%
\pgfusepath{clip}%
\pgfsetbuttcap%
\pgfsetroundjoin%
\definecolor{currentfill}{rgb}{1.000000,0.605229,0.530719}%
\pgfsetfillcolor{currentfill}%
\pgfsetlinewidth{0.250937pt}%
\definecolor{currentstroke}{rgb}{1.000000,1.000000,1.000000}%
\pgfsetstrokecolor{currentstroke}%
\pgfsetdash{}{0pt}%
\pgfpathmoveto{\pgfqpoint{3.802641in}{4.711604in}}%
\pgfpathlineto{\pgfqpoint{3.890377in}{4.711604in}}%
\pgfpathlineto{\pgfqpoint{3.890377in}{4.623868in}}%
\pgfpathlineto{\pgfqpoint{3.802641in}{4.623868in}}%
\pgfpathlineto{\pgfqpoint{3.802641in}{4.711604in}}%
\pgfusepath{stroke,fill}%
\end{pgfscope}%
\begin{pgfscope}%
\pgfpathrectangle{\pgfqpoint{0.380943in}{4.185189in}}{\pgfqpoint{4.650000in}{0.614151in}}%
\pgfusepath{clip}%
\pgfsetbuttcap%
\pgfsetroundjoin%
\definecolor{currentfill}{rgb}{0.986759,0.806398,0.641200}%
\pgfsetfillcolor{currentfill}%
\pgfsetlinewidth{0.250937pt}%
\definecolor{currentstroke}{rgb}{1.000000,1.000000,1.000000}%
\pgfsetstrokecolor{currentstroke}%
\pgfsetdash{}{0pt}%
\pgfpathmoveto{\pgfqpoint{3.890377in}{4.711604in}}%
\pgfpathlineto{\pgfqpoint{3.978113in}{4.711604in}}%
\pgfpathlineto{\pgfqpoint{3.978113in}{4.623868in}}%
\pgfpathlineto{\pgfqpoint{3.890377in}{4.623868in}}%
\pgfpathlineto{\pgfqpoint{3.890377in}{4.711604in}}%
\pgfusepath{stroke,fill}%
\end{pgfscope}%
\begin{pgfscope}%
\pgfpathrectangle{\pgfqpoint{0.380943in}{4.185189in}}{\pgfqpoint{4.650000in}{0.614151in}}%
\pgfusepath{clip}%
\pgfsetbuttcap%
\pgfsetroundjoin%
\definecolor{currentfill}{rgb}{1.000000,0.557862,0.511772}%
\pgfsetfillcolor{currentfill}%
\pgfsetlinewidth{0.250937pt}%
\definecolor{currentstroke}{rgb}{1.000000,1.000000,1.000000}%
\pgfsetstrokecolor{currentstroke}%
\pgfsetdash{}{0pt}%
\pgfpathmoveto{\pgfqpoint{3.978113in}{4.711604in}}%
\pgfpathlineto{\pgfqpoint{4.065849in}{4.711604in}}%
\pgfpathlineto{\pgfqpoint{4.065849in}{4.623868in}}%
\pgfpathlineto{\pgfqpoint{3.978113in}{4.623868in}}%
\pgfpathlineto{\pgfqpoint{3.978113in}{4.711604in}}%
\pgfusepath{stroke,fill}%
\end{pgfscope}%
\begin{pgfscope}%
\pgfpathrectangle{\pgfqpoint{0.380943in}{4.185189in}}{\pgfqpoint{4.650000in}{0.614151in}}%
\pgfusepath{clip}%
\pgfsetbuttcap%
\pgfsetroundjoin%
\definecolor{currentfill}{rgb}{0.979654,0.837186,0.669619}%
\pgfsetfillcolor{currentfill}%
\pgfsetlinewidth{0.250937pt}%
\definecolor{currentstroke}{rgb}{1.000000,1.000000,1.000000}%
\pgfsetstrokecolor{currentstroke}%
\pgfsetdash{}{0pt}%
\pgfpathmoveto{\pgfqpoint{4.065849in}{4.711604in}}%
\pgfpathlineto{\pgfqpoint{4.153585in}{4.711604in}}%
\pgfpathlineto{\pgfqpoint{4.153585in}{4.623868in}}%
\pgfpathlineto{\pgfqpoint{4.065849in}{4.623868in}}%
\pgfpathlineto{\pgfqpoint{4.065849in}{4.711604in}}%
\pgfusepath{stroke,fill}%
\end{pgfscope}%
\begin{pgfscope}%
\pgfpathrectangle{\pgfqpoint{0.380943in}{4.185189in}}{\pgfqpoint{4.650000in}{0.614151in}}%
\pgfusepath{clip}%
\pgfsetbuttcap%
\pgfsetroundjoin%
\definecolor{currentfill}{rgb}{0.979654,0.837186,0.669619}%
\pgfsetfillcolor{currentfill}%
\pgfsetlinewidth{0.250937pt}%
\definecolor{currentstroke}{rgb}{1.000000,1.000000,1.000000}%
\pgfsetstrokecolor{currentstroke}%
\pgfsetdash{}{0pt}%
\pgfpathmoveto{\pgfqpoint{4.153585in}{4.711604in}}%
\pgfpathlineto{\pgfqpoint{4.241320in}{4.711604in}}%
\pgfpathlineto{\pgfqpoint{4.241320in}{4.623868in}}%
\pgfpathlineto{\pgfqpoint{4.153585in}{4.623868in}}%
\pgfpathlineto{\pgfqpoint{4.153585in}{4.711604in}}%
\pgfusepath{stroke,fill}%
\end{pgfscope}%
\begin{pgfscope}%
\pgfpathrectangle{\pgfqpoint{0.380943in}{4.185189in}}{\pgfqpoint{4.650000in}{0.614151in}}%
\pgfusepath{clip}%
\pgfsetbuttcap%
\pgfsetroundjoin%
\definecolor{currentfill}{rgb}{0.962414,0.923552,0.722891}%
\pgfsetfillcolor{currentfill}%
\pgfsetlinewidth{0.250937pt}%
\definecolor{currentstroke}{rgb}{1.000000,1.000000,1.000000}%
\pgfsetstrokecolor{currentstroke}%
\pgfsetdash{}{0pt}%
\pgfpathmoveto{\pgfqpoint{4.241320in}{4.711604in}}%
\pgfpathlineto{\pgfqpoint{4.329056in}{4.711604in}}%
\pgfpathlineto{\pgfqpoint{4.329056in}{4.623868in}}%
\pgfpathlineto{\pgfqpoint{4.241320in}{4.623868in}}%
\pgfpathlineto{\pgfqpoint{4.241320in}{4.711604in}}%
\pgfusepath{stroke,fill}%
\end{pgfscope}%
\begin{pgfscope}%
\pgfpathrectangle{\pgfqpoint{0.380943in}{4.185189in}}{\pgfqpoint{4.650000in}{0.614151in}}%
\pgfusepath{clip}%
\pgfsetbuttcap%
\pgfsetroundjoin%
\definecolor{currentfill}{rgb}{1.000000,0.509404,0.491473}%
\pgfsetfillcolor{currentfill}%
\pgfsetlinewidth{0.250937pt}%
\definecolor{currentstroke}{rgb}{1.000000,1.000000,1.000000}%
\pgfsetstrokecolor{currentstroke}%
\pgfsetdash{}{0pt}%
\pgfpathmoveto{\pgfqpoint{4.329056in}{4.711604in}}%
\pgfpathlineto{\pgfqpoint{4.416792in}{4.711604in}}%
\pgfpathlineto{\pgfqpoint{4.416792in}{4.623868in}}%
\pgfpathlineto{\pgfqpoint{4.329056in}{4.623868in}}%
\pgfpathlineto{\pgfqpoint{4.329056in}{4.711604in}}%
\pgfusepath{stroke,fill}%
\end{pgfscope}%
\begin{pgfscope}%
\pgfpathrectangle{\pgfqpoint{0.380943in}{4.185189in}}{\pgfqpoint{4.650000in}{0.614151in}}%
\pgfusepath{clip}%
\pgfsetbuttcap%
\pgfsetroundjoin%
\definecolor{currentfill}{rgb}{0.992326,0.765229,0.614840}%
\pgfsetfillcolor{currentfill}%
\pgfsetlinewidth{0.250937pt}%
\definecolor{currentstroke}{rgb}{1.000000,1.000000,1.000000}%
\pgfsetstrokecolor{currentstroke}%
\pgfsetdash{}{0pt}%
\pgfpathmoveto{\pgfqpoint{4.416792in}{4.711604in}}%
\pgfpathlineto{\pgfqpoint{4.504528in}{4.711604in}}%
\pgfpathlineto{\pgfqpoint{4.504528in}{4.623868in}}%
\pgfpathlineto{\pgfqpoint{4.416792in}{4.623868in}}%
\pgfpathlineto{\pgfqpoint{4.416792in}{4.711604in}}%
\pgfusepath{stroke,fill}%
\end{pgfscope}%
\begin{pgfscope}%
\pgfpathrectangle{\pgfqpoint{0.380943in}{4.185189in}}{\pgfqpoint{4.650000in}{0.614151in}}%
\pgfusepath{clip}%
\pgfsetbuttcap%
\pgfsetroundjoin%
\definecolor{currentfill}{rgb}{0.986759,0.806398,0.641200}%
\pgfsetfillcolor{currentfill}%
\pgfsetlinewidth{0.250937pt}%
\definecolor{currentstroke}{rgb}{1.000000,1.000000,1.000000}%
\pgfsetstrokecolor{currentstroke}%
\pgfsetdash{}{0pt}%
\pgfpathmoveto{\pgfqpoint{4.504528in}{4.711604in}}%
\pgfpathlineto{\pgfqpoint{4.592264in}{4.711604in}}%
\pgfpathlineto{\pgfqpoint{4.592264in}{4.623868in}}%
\pgfpathlineto{\pgfqpoint{4.504528in}{4.623868in}}%
\pgfpathlineto{\pgfqpoint{4.504528in}{4.711604in}}%
\pgfusepath{stroke,fill}%
\end{pgfscope}%
\begin{pgfscope}%
\pgfpathrectangle{\pgfqpoint{0.380943in}{4.185189in}}{\pgfqpoint{4.650000in}{0.614151in}}%
\pgfusepath{clip}%
\pgfsetbuttcap%
\pgfsetroundjoin%
\definecolor{currentfill}{rgb}{0.996571,0.720538,0.589189}%
\pgfsetfillcolor{currentfill}%
\pgfsetlinewidth{0.250937pt}%
\definecolor{currentstroke}{rgb}{1.000000,1.000000,1.000000}%
\pgfsetstrokecolor{currentstroke}%
\pgfsetdash{}{0pt}%
\pgfpathmoveto{\pgfqpoint{4.592264in}{4.711604in}}%
\pgfpathlineto{\pgfqpoint{4.680000in}{4.711604in}}%
\pgfpathlineto{\pgfqpoint{4.680000in}{4.623868in}}%
\pgfpathlineto{\pgfqpoint{4.592264in}{4.623868in}}%
\pgfpathlineto{\pgfqpoint{4.592264in}{4.711604in}}%
\pgfusepath{stroke,fill}%
\end{pgfscope}%
\begin{pgfscope}%
\pgfpathrectangle{\pgfqpoint{0.380943in}{4.185189in}}{\pgfqpoint{4.650000in}{0.614151in}}%
\pgfusepath{clip}%
\pgfsetbuttcap%
\pgfsetroundjoin%
\definecolor{currentfill}{rgb}{0.979654,0.837186,0.669619}%
\pgfsetfillcolor{currentfill}%
\pgfsetlinewidth{0.250937pt}%
\definecolor{currentstroke}{rgb}{1.000000,1.000000,1.000000}%
\pgfsetstrokecolor{currentstroke}%
\pgfsetdash{}{0pt}%
\pgfpathmoveto{\pgfqpoint{4.680000in}{4.711604in}}%
\pgfpathlineto{\pgfqpoint{4.767736in}{4.711604in}}%
\pgfpathlineto{\pgfqpoint{4.767736in}{4.623868in}}%
\pgfpathlineto{\pgfqpoint{4.680000in}{4.623868in}}%
\pgfpathlineto{\pgfqpoint{4.680000in}{4.711604in}}%
\pgfusepath{stroke,fill}%
\end{pgfscope}%
\begin{pgfscope}%
\pgfpathrectangle{\pgfqpoint{0.380943in}{4.185189in}}{\pgfqpoint{4.650000in}{0.614151in}}%
\pgfusepath{clip}%
\pgfsetbuttcap%
\pgfsetroundjoin%
\definecolor{currentfill}{rgb}{0.996571,0.720538,0.589189}%
\pgfsetfillcolor{currentfill}%
\pgfsetlinewidth{0.250937pt}%
\definecolor{currentstroke}{rgb}{1.000000,1.000000,1.000000}%
\pgfsetstrokecolor{currentstroke}%
\pgfsetdash{}{0pt}%
\pgfpathmoveto{\pgfqpoint{4.767736in}{4.711604in}}%
\pgfpathlineto{\pgfqpoint{4.855471in}{4.711604in}}%
\pgfpathlineto{\pgfqpoint{4.855471in}{4.623868in}}%
\pgfpathlineto{\pgfqpoint{4.767736in}{4.623868in}}%
\pgfpathlineto{\pgfqpoint{4.767736in}{4.711604in}}%
\pgfusepath{stroke,fill}%
\end{pgfscope}%
\begin{pgfscope}%
\pgfpathrectangle{\pgfqpoint{0.380943in}{4.185189in}}{\pgfqpoint{4.650000in}{0.614151in}}%
\pgfusepath{clip}%
\pgfsetbuttcap%
\pgfsetroundjoin%
\definecolor{currentfill}{rgb}{1.000000,1.000000,0.929412}%
\pgfsetfillcolor{currentfill}%
\pgfsetlinewidth{0.250937pt}%
\definecolor{currentstroke}{rgb}{1.000000,1.000000,1.000000}%
\pgfsetstrokecolor{currentstroke}%
\pgfsetdash{}{0pt}%
\pgfpathmoveto{\pgfqpoint{4.855471in}{4.711604in}}%
\pgfpathlineto{\pgfqpoint{4.943207in}{4.711604in}}%
\pgfpathlineto{\pgfqpoint{4.943207in}{4.623868in}}%
\pgfpathlineto{\pgfqpoint{4.855471in}{4.623868in}}%
\pgfpathlineto{\pgfqpoint{4.855471in}{4.711604in}}%
\pgfusepath{stroke,fill}%
\end{pgfscope}%
\begin{pgfscope}%
\pgfpathrectangle{\pgfqpoint{0.380943in}{4.185189in}}{\pgfqpoint{4.650000in}{0.614151in}}%
\pgfusepath{clip}%
\pgfsetbuttcap%
\pgfsetroundjoin%
\definecolor{currentfill}{rgb}{0.968166,0.945882,0.748604}%
\pgfsetfillcolor{currentfill}%
\pgfsetlinewidth{0.250937pt}%
\definecolor{currentstroke}{rgb}{1.000000,1.000000,1.000000}%
\pgfsetstrokecolor{currentstroke}%
\pgfsetdash{}{0pt}%
\pgfpathmoveto{\pgfqpoint{4.943207in}{4.711604in}}%
\pgfpathlineto{\pgfqpoint{5.030943in}{4.711604in}}%
\pgfpathlineto{\pgfqpoint{5.030943in}{4.623868in}}%
\pgfpathlineto{\pgfqpoint{4.943207in}{4.623868in}}%
\pgfpathlineto{\pgfqpoint{4.943207in}{4.711604in}}%
\pgfusepath{stroke,fill}%
\end{pgfscope}%
\begin{pgfscope}%
\pgfpathrectangle{\pgfqpoint{0.380943in}{4.185189in}}{\pgfqpoint{4.650000in}{0.614151in}}%
\pgfusepath{clip}%
\pgfsetbuttcap%
\pgfsetroundjoin%
\definecolor{currentfill}{rgb}{0.986759,0.806398,0.641200}%
\pgfsetfillcolor{currentfill}%
\pgfsetlinewidth{0.250937pt}%
\definecolor{currentstroke}{rgb}{1.000000,1.000000,1.000000}%
\pgfsetstrokecolor{currentstroke}%
\pgfsetdash{}{0pt}%
\pgfpathmoveto{\pgfqpoint{0.380943in}{4.623868in}}%
\pgfpathlineto{\pgfqpoint{0.468679in}{4.623868in}}%
\pgfpathlineto{\pgfqpoint{0.468679in}{4.536132in}}%
\pgfpathlineto{\pgfqpoint{0.380943in}{4.536132in}}%
\pgfpathlineto{\pgfqpoint{0.380943in}{4.623868in}}%
\pgfusepath{stroke,fill}%
\end{pgfscope}%
\begin{pgfscope}%
\pgfpathrectangle{\pgfqpoint{0.380943in}{4.185189in}}{\pgfqpoint{4.650000in}{0.614151in}}%
\pgfusepath{clip}%
\pgfsetbuttcap%
\pgfsetroundjoin%
\definecolor{currentfill}{rgb}{0.972549,0.870588,0.692810}%
\pgfsetfillcolor{currentfill}%
\pgfsetlinewidth{0.250937pt}%
\definecolor{currentstroke}{rgb}{1.000000,1.000000,1.000000}%
\pgfsetstrokecolor{currentstroke}%
\pgfsetdash{}{0pt}%
\pgfpathmoveto{\pgfqpoint{0.468679in}{4.623868in}}%
\pgfpathlineto{\pgfqpoint{0.556415in}{4.623868in}}%
\pgfpathlineto{\pgfqpoint{0.556415in}{4.536132in}}%
\pgfpathlineto{\pgfqpoint{0.468679in}{4.536132in}}%
\pgfpathlineto{\pgfqpoint{0.468679in}{4.623868in}}%
\pgfusepath{stroke,fill}%
\end{pgfscope}%
\begin{pgfscope}%
\pgfpathrectangle{\pgfqpoint{0.380943in}{4.185189in}}{\pgfqpoint{4.650000in}{0.614151in}}%
\pgfusepath{clip}%
\pgfsetbuttcap%
\pgfsetroundjoin%
\definecolor{currentfill}{rgb}{0.972549,0.870588,0.692810}%
\pgfsetfillcolor{currentfill}%
\pgfsetlinewidth{0.250937pt}%
\definecolor{currentstroke}{rgb}{1.000000,1.000000,1.000000}%
\pgfsetstrokecolor{currentstroke}%
\pgfsetdash{}{0pt}%
\pgfpathmoveto{\pgfqpoint{0.556415in}{4.623868in}}%
\pgfpathlineto{\pgfqpoint{0.644151in}{4.623868in}}%
\pgfpathlineto{\pgfqpoint{0.644151in}{4.536132in}}%
\pgfpathlineto{\pgfqpoint{0.556415in}{4.536132in}}%
\pgfpathlineto{\pgfqpoint{0.556415in}{4.623868in}}%
\pgfusepath{stroke,fill}%
\end{pgfscope}%
\begin{pgfscope}%
\pgfpathrectangle{\pgfqpoint{0.380943in}{4.185189in}}{\pgfqpoint{4.650000in}{0.614151in}}%
\pgfusepath{clip}%
\pgfsetbuttcap%
\pgfsetroundjoin%
\definecolor{currentfill}{rgb}{0.996571,0.720538,0.589189}%
\pgfsetfillcolor{currentfill}%
\pgfsetlinewidth{0.250937pt}%
\definecolor{currentstroke}{rgb}{1.000000,1.000000,1.000000}%
\pgfsetstrokecolor{currentstroke}%
\pgfsetdash{}{0pt}%
\pgfpathmoveto{\pgfqpoint{0.644151in}{4.623868in}}%
\pgfpathlineto{\pgfqpoint{0.731886in}{4.623868in}}%
\pgfpathlineto{\pgfqpoint{0.731886in}{4.536132in}}%
\pgfpathlineto{\pgfqpoint{0.644151in}{4.536132in}}%
\pgfpathlineto{\pgfqpoint{0.644151in}{4.623868in}}%
\pgfusepath{stroke,fill}%
\end{pgfscope}%
\begin{pgfscope}%
\pgfpathrectangle{\pgfqpoint{0.380943in}{4.185189in}}{\pgfqpoint{4.650000in}{0.614151in}}%
\pgfusepath{clip}%
\pgfsetbuttcap%
\pgfsetroundjoin%
\definecolor{currentfill}{rgb}{1.000000,0.509404,0.491473}%
\pgfsetfillcolor{currentfill}%
\pgfsetlinewidth{0.250937pt}%
\definecolor{currentstroke}{rgb}{1.000000,1.000000,1.000000}%
\pgfsetstrokecolor{currentstroke}%
\pgfsetdash{}{0pt}%
\pgfpathmoveto{\pgfqpoint{0.731886in}{4.623868in}}%
\pgfpathlineto{\pgfqpoint{0.819622in}{4.623868in}}%
\pgfpathlineto{\pgfqpoint{0.819622in}{4.536132in}}%
\pgfpathlineto{\pgfqpoint{0.731886in}{4.536132in}}%
\pgfpathlineto{\pgfqpoint{0.731886in}{4.623868in}}%
\pgfusepath{stroke,fill}%
\end{pgfscope}%
\begin{pgfscope}%
\pgfpathrectangle{\pgfqpoint{0.380943in}{4.185189in}}{\pgfqpoint{4.650000in}{0.614151in}}%
\pgfusepath{clip}%
\pgfsetbuttcap%
\pgfsetroundjoin%
\definecolor{currentfill}{rgb}{0.996571,0.720538,0.589189}%
\pgfsetfillcolor{currentfill}%
\pgfsetlinewidth{0.250937pt}%
\definecolor{currentstroke}{rgb}{1.000000,1.000000,1.000000}%
\pgfsetstrokecolor{currentstroke}%
\pgfsetdash{}{0pt}%
\pgfpathmoveto{\pgfqpoint{0.819622in}{4.623868in}}%
\pgfpathlineto{\pgfqpoint{0.907358in}{4.623868in}}%
\pgfpathlineto{\pgfqpoint{0.907358in}{4.536132in}}%
\pgfpathlineto{\pgfqpoint{0.819622in}{4.536132in}}%
\pgfpathlineto{\pgfqpoint{0.819622in}{4.623868in}}%
\pgfusepath{stroke,fill}%
\end{pgfscope}%
\begin{pgfscope}%
\pgfpathrectangle{\pgfqpoint{0.380943in}{4.185189in}}{\pgfqpoint{4.650000in}{0.614151in}}%
\pgfusepath{clip}%
\pgfsetbuttcap%
\pgfsetroundjoin%
\definecolor{currentfill}{rgb}{0.972549,0.870588,0.692810}%
\pgfsetfillcolor{currentfill}%
\pgfsetlinewidth{0.250937pt}%
\definecolor{currentstroke}{rgb}{1.000000,1.000000,1.000000}%
\pgfsetstrokecolor{currentstroke}%
\pgfsetdash{}{0pt}%
\pgfpathmoveto{\pgfqpoint{0.907358in}{4.623868in}}%
\pgfpathlineto{\pgfqpoint{0.995094in}{4.623868in}}%
\pgfpathlineto{\pgfqpoint{0.995094in}{4.536132in}}%
\pgfpathlineto{\pgfqpoint{0.907358in}{4.536132in}}%
\pgfpathlineto{\pgfqpoint{0.907358in}{4.623868in}}%
\pgfusepath{stroke,fill}%
\end{pgfscope}%
\begin{pgfscope}%
\pgfpathrectangle{\pgfqpoint{0.380943in}{4.185189in}}{\pgfqpoint{4.650000in}{0.614151in}}%
\pgfusepath{clip}%
\pgfsetbuttcap%
\pgfsetroundjoin%
\definecolor{currentfill}{rgb}{0.998939,0.658962,0.556032}%
\pgfsetfillcolor{currentfill}%
\pgfsetlinewidth{0.250937pt}%
\definecolor{currentstroke}{rgb}{1.000000,1.000000,1.000000}%
\pgfsetstrokecolor{currentstroke}%
\pgfsetdash{}{0pt}%
\pgfpathmoveto{\pgfqpoint{0.995094in}{4.623868in}}%
\pgfpathlineto{\pgfqpoint{1.082830in}{4.623868in}}%
\pgfpathlineto{\pgfqpoint{1.082830in}{4.536132in}}%
\pgfpathlineto{\pgfqpoint{0.995094in}{4.536132in}}%
\pgfpathlineto{\pgfqpoint{0.995094in}{4.623868in}}%
\pgfusepath{stroke,fill}%
\end{pgfscope}%
\begin{pgfscope}%
\pgfpathrectangle{\pgfqpoint{0.380943in}{4.185189in}}{\pgfqpoint{4.650000in}{0.614151in}}%
\pgfusepath{clip}%
\pgfsetbuttcap%
\pgfsetroundjoin%
\definecolor{currentfill}{rgb}{0.998939,0.658962,0.556032}%
\pgfsetfillcolor{currentfill}%
\pgfsetlinewidth{0.250937pt}%
\definecolor{currentstroke}{rgb}{1.000000,1.000000,1.000000}%
\pgfsetstrokecolor{currentstroke}%
\pgfsetdash{}{0pt}%
\pgfpathmoveto{\pgfqpoint{1.082830in}{4.623868in}}%
\pgfpathlineto{\pgfqpoint{1.170566in}{4.623868in}}%
\pgfpathlineto{\pgfqpoint{1.170566in}{4.536132in}}%
\pgfpathlineto{\pgfqpoint{1.082830in}{4.536132in}}%
\pgfpathlineto{\pgfqpoint{1.082830in}{4.623868in}}%
\pgfusepath{stroke,fill}%
\end{pgfscope}%
\begin{pgfscope}%
\pgfpathrectangle{\pgfqpoint{0.380943in}{4.185189in}}{\pgfqpoint{4.650000in}{0.614151in}}%
\pgfusepath{clip}%
\pgfsetbuttcap%
\pgfsetroundjoin%
\definecolor{currentfill}{rgb}{0.986759,0.806398,0.641200}%
\pgfsetfillcolor{currentfill}%
\pgfsetlinewidth{0.250937pt}%
\definecolor{currentstroke}{rgb}{1.000000,1.000000,1.000000}%
\pgfsetstrokecolor{currentstroke}%
\pgfsetdash{}{0pt}%
\pgfpathmoveto{\pgfqpoint{1.170566in}{4.623868in}}%
\pgfpathlineto{\pgfqpoint{1.258302in}{4.623868in}}%
\pgfpathlineto{\pgfqpoint{1.258302in}{4.536132in}}%
\pgfpathlineto{\pgfqpoint{1.170566in}{4.536132in}}%
\pgfpathlineto{\pgfqpoint{1.170566in}{4.623868in}}%
\pgfusepath{stroke,fill}%
\end{pgfscope}%
\begin{pgfscope}%
\pgfpathrectangle{\pgfqpoint{0.380943in}{4.185189in}}{\pgfqpoint{4.650000in}{0.614151in}}%
\pgfusepath{clip}%
\pgfsetbuttcap%
\pgfsetroundjoin%
\definecolor{currentfill}{rgb}{1.000000,0.605229,0.530719}%
\pgfsetfillcolor{currentfill}%
\pgfsetlinewidth{0.250937pt}%
\definecolor{currentstroke}{rgb}{1.000000,1.000000,1.000000}%
\pgfsetstrokecolor{currentstroke}%
\pgfsetdash{}{0pt}%
\pgfpathmoveto{\pgfqpoint{1.258302in}{4.623868in}}%
\pgfpathlineto{\pgfqpoint{1.346037in}{4.623868in}}%
\pgfpathlineto{\pgfqpoint{1.346037in}{4.536132in}}%
\pgfpathlineto{\pgfqpoint{1.258302in}{4.536132in}}%
\pgfpathlineto{\pgfqpoint{1.258302in}{4.623868in}}%
\pgfusepath{stroke,fill}%
\end{pgfscope}%
\begin{pgfscope}%
\pgfpathrectangle{\pgfqpoint{0.380943in}{4.185189in}}{\pgfqpoint{4.650000in}{0.614151in}}%
\pgfusepath{clip}%
\pgfsetbuttcap%
\pgfsetroundjoin%
\definecolor{currentfill}{rgb}{0.996571,0.720538,0.589189}%
\pgfsetfillcolor{currentfill}%
\pgfsetlinewidth{0.250937pt}%
\definecolor{currentstroke}{rgb}{1.000000,1.000000,1.000000}%
\pgfsetstrokecolor{currentstroke}%
\pgfsetdash{}{0pt}%
\pgfpathmoveto{\pgfqpoint{1.346037in}{4.623868in}}%
\pgfpathlineto{\pgfqpoint{1.433773in}{4.623868in}}%
\pgfpathlineto{\pgfqpoint{1.433773in}{4.536132in}}%
\pgfpathlineto{\pgfqpoint{1.346037in}{4.536132in}}%
\pgfpathlineto{\pgfqpoint{1.346037in}{4.623868in}}%
\pgfusepath{stroke,fill}%
\end{pgfscope}%
\begin{pgfscope}%
\pgfpathrectangle{\pgfqpoint{0.380943in}{4.185189in}}{\pgfqpoint{4.650000in}{0.614151in}}%
\pgfusepath{clip}%
\pgfsetbuttcap%
\pgfsetroundjoin%
\definecolor{currentfill}{rgb}{0.965444,0.906113,0.711757}%
\pgfsetfillcolor{currentfill}%
\pgfsetlinewidth{0.250937pt}%
\definecolor{currentstroke}{rgb}{1.000000,1.000000,1.000000}%
\pgfsetstrokecolor{currentstroke}%
\pgfsetdash{}{0pt}%
\pgfpathmoveto{\pgfqpoint{1.433773in}{4.623868in}}%
\pgfpathlineto{\pgfqpoint{1.521509in}{4.623868in}}%
\pgfpathlineto{\pgfqpoint{1.521509in}{4.536132in}}%
\pgfpathlineto{\pgfqpoint{1.433773in}{4.536132in}}%
\pgfpathlineto{\pgfqpoint{1.433773in}{4.623868in}}%
\pgfusepath{stroke,fill}%
\end{pgfscope}%
\begin{pgfscope}%
\pgfpathrectangle{\pgfqpoint{0.380943in}{4.185189in}}{\pgfqpoint{4.650000in}{0.614151in}}%
\pgfusepath{clip}%
\pgfsetbuttcap%
\pgfsetroundjoin%
\definecolor{currentfill}{rgb}{0.992326,0.765229,0.614840}%
\pgfsetfillcolor{currentfill}%
\pgfsetlinewidth{0.250937pt}%
\definecolor{currentstroke}{rgb}{1.000000,1.000000,1.000000}%
\pgfsetstrokecolor{currentstroke}%
\pgfsetdash{}{0pt}%
\pgfpathmoveto{\pgfqpoint{1.521509in}{4.623868in}}%
\pgfpathlineto{\pgfqpoint{1.609245in}{4.623868in}}%
\pgfpathlineto{\pgfqpoint{1.609245in}{4.536132in}}%
\pgfpathlineto{\pgfqpoint{1.521509in}{4.536132in}}%
\pgfpathlineto{\pgfqpoint{1.521509in}{4.623868in}}%
\pgfusepath{stroke,fill}%
\end{pgfscope}%
\begin{pgfscope}%
\pgfpathrectangle{\pgfqpoint{0.380943in}{4.185189in}}{\pgfqpoint{4.650000in}{0.614151in}}%
\pgfusepath{clip}%
\pgfsetbuttcap%
\pgfsetroundjoin%
\definecolor{currentfill}{rgb}{0.986759,0.806398,0.641200}%
\pgfsetfillcolor{currentfill}%
\pgfsetlinewidth{0.250937pt}%
\definecolor{currentstroke}{rgb}{1.000000,1.000000,1.000000}%
\pgfsetstrokecolor{currentstroke}%
\pgfsetdash{}{0pt}%
\pgfpathmoveto{\pgfqpoint{1.609245in}{4.623868in}}%
\pgfpathlineto{\pgfqpoint{1.696981in}{4.623868in}}%
\pgfpathlineto{\pgfqpoint{1.696981in}{4.536132in}}%
\pgfpathlineto{\pgfqpoint{1.609245in}{4.536132in}}%
\pgfpathlineto{\pgfqpoint{1.609245in}{4.623868in}}%
\pgfusepath{stroke,fill}%
\end{pgfscope}%
\begin{pgfscope}%
\pgfpathrectangle{\pgfqpoint{0.380943in}{4.185189in}}{\pgfqpoint{4.650000in}{0.614151in}}%
\pgfusepath{clip}%
\pgfsetbuttcap%
\pgfsetroundjoin%
\definecolor{currentfill}{rgb}{0.992326,0.765229,0.614840}%
\pgfsetfillcolor{currentfill}%
\pgfsetlinewidth{0.250937pt}%
\definecolor{currentstroke}{rgb}{1.000000,1.000000,1.000000}%
\pgfsetstrokecolor{currentstroke}%
\pgfsetdash{}{0pt}%
\pgfpathmoveto{\pgfqpoint{1.696981in}{4.623868in}}%
\pgfpathlineto{\pgfqpoint{1.784717in}{4.623868in}}%
\pgfpathlineto{\pgfqpoint{1.784717in}{4.536132in}}%
\pgfpathlineto{\pgfqpoint{1.696981in}{4.536132in}}%
\pgfpathlineto{\pgfqpoint{1.696981in}{4.623868in}}%
\pgfusepath{stroke,fill}%
\end{pgfscope}%
\begin{pgfscope}%
\pgfpathrectangle{\pgfqpoint{0.380943in}{4.185189in}}{\pgfqpoint{4.650000in}{0.614151in}}%
\pgfusepath{clip}%
\pgfsetbuttcap%
\pgfsetroundjoin%
\definecolor{currentfill}{rgb}{0.979654,0.837186,0.669619}%
\pgfsetfillcolor{currentfill}%
\pgfsetlinewidth{0.250937pt}%
\definecolor{currentstroke}{rgb}{1.000000,1.000000,1.000000}%
\pgfsetstrokecolor{currentstroke}%
\pgfsetdash{}{0pt}%
\pgfpathmoveto{\pgfqpoint{1.784717in}{4.623868in}}%
\pgfpathlineto{\pgfqpoint{1.872452in}{4.623868in}}%
\pgfpathlineto{\pgfqpoint{1.872452in}{4.536132in}}%
\pgfpathlineto{\pgfqpoint{1.784717in}{4.536132in}}%
\pgfpathlineto{\pgfqpoint{1.784717in}{4.623868in}}%
\pgfusepath{stroke,fill}%
\end{pgfscope}%
\begin{pgfscope}%
\pgfpathrectangle{\pgfqpoint{0.380943in}{4.185189in}}{\pgfqpoint{4.650000in}{0.614151in}}%
\pgfusepath{clip}%
\pgfsetbuttcap%
\pgfsetroundjoin%
\definecolor{currentfill}{rgb}{0.965444,0.906113,0.711757}%
\pgfsetfillcolor{currentfill}%
\pgfsetlinewidth{0.250937pt}%
\definecolor{currentstroke}{rgb}{1.000000,1.000000,1.000000}%
\pgfsetstrokecolor{currentstroke}%
\pgfsetdash{}{0pt}%
\pgfpathmoveto{\pgfqpoint{1.872452in}{4.623868in}}%
\pgfpathlineto{\pgfqpoint{1.960188in}{4.623868in}}%
\pgfpathlineto{\pgfqpoint{1.960188in}{4.536132in}}%
\pgfpathlineto{\pgfqpoint{1.872452in}{4.536132in}}%
\pgfpathlineto{\pgfqpoint{1.872452in}{4.623868in}}%
\pgfusepath{stroke,fill}%
\end{pgfscope}%
\begin{pgfscope}%
\pgfpathrectangle{\pgfqpoint{0.380943in}{4.185189in}}{\pgfqpoint{4.650000in}{0.614151in}}%
\pgfusepath{clip}%
\pgfsetbuttcap%
\pgfsetroundjoin%
\definecolor{currentfill}{rgb}{0.962414,0.923552,0.722891}%
\pgfsetfillcolor{currentfill}%
\pgfsetlinewidth{0.250937pt}%
\definecolor{currentstroke}{rgb}{1.000000,1.000000,1.000000}%
\pgfsetstrokecolor{currentstroke}%
\pgfsetdash{}{0pt}%
\pgfpathmoveto{\pgfqpoint{1.960188in}{4.623868in}}%
\pgfpathlineto{\pgfqpoint{2.047924in}{4.623868in}}%
\pgfpathlineto{\pgfqpoint{2.047924in}{4.536132in}}%
\pgfpathlineto{\pgfqpoint{1.960188in}{4.536132in}}%
\pgfpathlineto{\pgfqpoint{1.960188in}{4.623868in}}%
\pgfusepath{stroke,fill}%
\end{pgfscope}%
\begin{pgfscope}%
\pgfpathrectangle{\pgfqpoint{0.380943in}{4.185189in}}{\pgfqpoint{4.650000in}{0.614151in}}%
\pgfusepath{clip}%
\pgfsetbuttcap%
\pgfsetroundjoin%
\definecolor{currentfill}{rgb}{0.986759,0.806398,0.641200}%
\pgfsetfillcolor{currentfill}%
\pgfsetlinewidth{0.250937pt}%
\definecolor{currentstroke}{rgb}{1.000000,1.000000,1.000000}%
\pgfsetstrokecolor{currentstroke}%
\pgfsetdash{}{0pt}%
\pgfpathmoveto{\pgfqpoint{2.047924in}{4.623868in}}%
\pgfpathlineto{\pgfqpoint{2.135660in}{4.623868in}}%
\pgfpathlineto{\pgfqpoint{2.135660in}{4.536132in}}%
\pgfpathlineto{\pgfqpoint{2.047924in}{4.536132in}}%
\pgfpathlineto{\pgfqpoint{2.047924in}{4.623868in}}%
\pgfusepath{stroke,fill}%
\end{pgfscope}%
\begin{pgfscope}%
\pgfpathrectangle{\pgfqpoint{0.380943in}{4.185189in}}{\pgfqpoint{4.650000in}{0.614151in}}%
\pgfusepath{clip}%
\pgfsetbuttcap%
\pgfsetroundjoin%
\definecolor{currentfill}{rgb}{0.962414,0.923552,0.722891}%
\pgfsetfillcolor{currentfill}%
\pgfsetlinewidth{0.250937pt}%
\definecolor{currentstroke}{rgb}{1.000000,1.000000,1.000000}%
\pgfsetstrokecolor{currentstroke}%
\pgfsetdash{}{0pt}%
\pgfpathmoveto{\pgfqpoint{2.135660in}{4.623868in}}%
\pgfpathlineto{\pgfqpoint{2.223396in}{4.623868in}}%
\pgfpathlineto{\pgfqpoint{2.223396in}{4.536132in}}%
\pgfpathlineto{\pgfqpoint{2.135660in}{4.536132in}}%
\pgfpathlineto{\pgfqpoint{2.135660in}{4.623868in}}%
\pgfusepath{stroke,fill}%
\end{pgfscope}%
\begin{pgfscope}%
\pgfpathrectangle{\pgfqpoint{0.380943in}{4.185189in}}{\pgfqpoint{4.650000in}{0.614151in}}%
\pgfusepath{clip}%
\pgfsetbuttcap%
\pgfsetroundjoin%
\definecolor{currentfill}{rgb}{0.992326,0.765229,0.614840}%
\pgfsetfillcolor{currentfill}%
\pgfsetlinewidth{0.250937pt}%
\definecolor{currentstroke}{rgb}{1.000000,1.000000,1.000000}%
\pgfsetstrokecolor{currentstroke}%
\pgfsetdash{}{0pt}%
\pgfpathmoveto{\pgfqpoint{2.223396in}{4.623868in}}%
\pgfpathlineto{\pgfqpoint{2.311132in}{4.623868in}}%
\pgfpathlineto{\pgfqpoint{2.311132in}{4.536132in}}%
\pgfpathlineto{\pgfqpoint{2.223396in}{4.536132in}}%
\pgfpathlineto{\pgfqpoint{2.223396in}{4.623868in}}%
\pgfusepath{stroke,fill}%
\end{pgfscope}%
\begin{pgfscope}%
\pgfpathrectangle{\pgfqpoint{0.380943in}{4.185189in}}{\pgfqpoint{4.650000in}{0.614151in}}%
\pgfusepath{clip}%
\pgfsetbuttcap%
\pgfsetroundjoin%
\definecolor{currentfill}{rgb}{0.996571,0.720538,0.589189}%
\pgfsetfillcolor{currentfill}%
\pgfsetlinewidth{0.250937pt}%
\definecolor{currentstroke}{rgb}{1.000000,1.000000,1.000000}%
\pgfsetstrokecolor{currentstroke}%
\pgfsetdash{}{0pt}%
\pgfpathmoveto{\pgfqpoint{2.311132in}{4.623868in}}%
\pgfpathlineto{\pgfqpoint{2.398868in}{4.623868in}}%
\pgfpathlineto{\pgfqpoint{2.398868in}{4.536132in}}%
\pgfpathlineto{\pgfqpoint{2.311132in}{4.536132in}}%
\pgfpathlineto{\pgfqpoint{2.311132in}{4.623868in}}%
\pgfusepath{stroke,fill}%
\end{pgfscope}%
\begin{pgfscope}%
\pgfpathrectangle{\pgfqpoint{0.380943in}{4.185189in}}{\pgfqpoint{4.650000in}{0.614151in}}%
\pgfusepath{clip}%
\pgfsetbuttcap%
\pgfsetroundjoin%
\definecolor{currentfill}{rgb}{1.000000,0.557862,0.511772}%
\pgfsetfillcolor{currentfill}%
\pgfsetlinewidth{0.250937pt}%
\definecolor{currentstroke}{rgb}{1.000000,1.000000,1.000000}%
\pgfsetstrokecolor{currentstroke}%
\pgfsetdash{}{0pt}%
\pgfpathmoveto{\pgfqpoint{2.398868in}{4.623868in}}%
\pgfpathlineto{\pgfqpoint{2.486603in}{4.623868in}}%
\pgfpathlineto{\pgfqpoint{2.486603in}{4.536132in}}%
\pgfpathlineto{\pgfqpoint{2.398868in}{4.536132in}}%
\pgfpathlineto{\pgfqpoint{2.398868in}{4.623868in}}%
\pgfusepath{stroke,fill}%
\end{pgfscope}%
\begin{pgfscope}%
\pgfpathrectangle{\pgfqpoint{0.380943in}{4.185189in}}{\pgfqpoint{4.650000in}{0.614151in}}%
\pgfusepath{clip}%
\pgfsetbuttcap%
\pgfsetroundjoin%
\definecolor{currentfill}{rgb}{0.986759,0.806398,0.641200}%
\pgfsetfillcolor{currentfill}%
\pgfsetlinewidth{0.250937pt}%
\definecolor{currentstroke}{rgb}{1.000000,1.000000,1.000000}%
\pgfsetstrokecolor{currentstroke}%
\pgfsetdash{}{0pt}%
\pgfpathmoveto{\pgfqpoint{2.486603in}{4.623868in}}%
\pgfpathlineto{\pgfqpoint{2.574339in}{4.623868in}}%
\pgfpathlineto{\pgfqpoint{2.574339in}{4.536132in}}%
\pgfpathlineto{\pgfqpoint{2.486603in}{4.536132in}}%
\pgfpathlineto{\pgfqpoint{2.486603in}{4.623868in}}%
\pgfusepath{stroke,fill}%
\end{pgfscope}%
\begin{pgfscope}%
\pgfpathrectangle{\pgfqpoint{0.380943in}{4.185189in}}{\pgfqpoint{4.650000in}{0.614151in}}%
\pgfusepath{clip}%
\pgfsetbuttcap%
\pgfsetroundjoin%
\definecolor{currentfill}{rgb}{0.986759,0.806398,0.641200}%
\pgfsetfillcolor{currentfill}%
\pgfsetlinewidth{0.250937pt}%
\definecolor{currentstroke}{rgb}{1.000000,1.000000,1.000000}%
\pgfsetstrokecolor{currentstroke}%
\pgfsetdash{}{0pt}%
\pgfpathmoveto{\pgfqpoint{2.574339in}{4.623868in}}%
\pgfpathlineto{\pgfqpoint{2.662075in}{4.623868in}}%
\pgfpathlineto{\pgfqpoint{2.662075in}{4.536132in}}%
\pgfpathlineto{\pgfqpoint{2.574339in}{4.536132in}}%
\pgfpathlineto{\pgfqpoint{2.574339in}{4.623868in}}%
\pgfusepath{stroke,fill}%
\end{pgfscope}%
\begin{pgfscope}%
\pgfpathrectangle{\pgfqpoint{0.380943in}{4.185189in}}{\pgfqpoint{4.650000in}{0.614151in}}%
\pgfusepath{clip}%
\pgfsetbuttcap%
\pgfsetroundjoin%
\definecolor{currentfill}{rgb}{0.991849,0.986144,0.810181}%
\pgfsetfillcolor{currentfill}%
\pgfsetlinewidth{0.250937pt}%
\definecolor{currentstroke}{rgb}{1.000000,1.000000,1.000000}%
\pgfsetstrokecolor{currentstroke}%
\pgfsetdash{}{0pt}%
\pgfpathmoveto{\pgfqpoint{2.662075in}{4.623868in}}%
\pgfpathlineto{\pgfqpoint{2.749811in}{4.623868in}}%
\pgfpathlineto{\pgfqpoint{2.749811in}{4.536132in}}%
\pgfpathlineto{\pgfqpoint{2.662075in}{4.536132in}}%
\pgfpathlineto{\pgfqpoint{2.662075in}{4.623868in}}%
\pgfusepath{stroke,fill}%
\end{pgfscope}%
\begin{pgfscope}%
\pgfpathrectangle{\pgfqpoint{0.380943in}{4.185189in}}{\pgfqpoint{4.650000in}{0.614151in}}%
\pgfusepath{clip}%
\pgfsetbuttcap%
\pgfsetroundjoin%
\definecolor{currentfill}{rgb}{0.922338,0.400769,0.400769}%
\pgfsetfillcolor{currentfill}%
\pgfsetlinewidth{0.250937pt}%
\definecolor{currentstroke}{rgb}{1.000000,1.000000,1.000000}%
\pgfsetstrokecolor{currentstroke}%
\pgfsetdash{}{0pt}%
\pgfpathmoveto{\pgfqpoint{2.749811in}{4.623868in}}%
\pgfpathlineto{\pgfqpoint{2.837547in}{4.623868in}}%
\pgfpathlineto{\pgfqpoint{2.837547in}{4.536132in}}%
\pgfpathlineto{\pgfqpoint{2.749811in}{4.536132in}}%
\pgfpathlineto{\pgfqpoint{2.749811in}{4.623868in}}%
\pgfusepath{stroke,fill}%
\end{pgfscope}%
\begin{pgfscope}%
\pgfpathrectangle{\pgfqpoint{0.380943in}{4.185189in}}{\pgfqpoint{4.650000in}{0.614151in}}%
\pgfusepath{clip}%
\pgfsetbuttcap%
\pgfsetroundjoin%
\definecolor{currentfill}{rgb}{0.996571,0.720538,0.589189}%
\pgfsetfillcolor{currentfill}%
\pgfsetlinewidth{0.250937pt}%
\definecolor{currentstroke}{rgb}{1.000000,1.000000,1.000000}%
\pgfsetstrokecolor{currentstroke}%
\pgfsetdash{}{0pt}%
\pgfpathmoveto{\pgfqpoint{2.837547in}{4.623868in}}%
\pgfpathlineto{\pgfqpoint{2.925283in}{4.623868in}}%
\pgfpathlineto{\pgfqpoint{2.925283in}{4.536132in}}%
\pgfpathlineto{\pgfqpoint{2.837547in}{4.536132in}}%
\pgfpathlineto{\pgfqpoint{2.837547in}{4.623868in}}%
\pgfusepath{stroke,fill}%
\end{pgfscope}%
\begin{pgfscope}%
\pgfpathrectangle{\pgfqpoint{0.380943in}{4.185189in}}{\pgfqpoint{4.650000in}{0.614151in}}%
\pgfusepath{clip}%
\pgfsetbuttcap%
\pgfsetroundjoin%
\definecolor{currentfill}{rgb}{0.998939,0.658962,0.556032}%
\pgfsetfillcolor{currentfill}%
\pgfsetlinewidth{0.250937pt}%
\definecolor{currentstroke}{rgb}{1.000000,1.000000,1.000000}%
\pgfsetstrokecolor{currentstroke}%
\pgfsetdash{}{0pt}%
\pgfpathmoveto{\pgfqpoint{2.925283in}{4.623868in}}%
\pgfpathlineto{\pgfqpoint{3.013019in}{4.623868in}}%
\pgfpathlineto{\pgfqpoint{3.013019in}{4.536132in}}%
\pgfpathlineto{\pgfqpoint{2.925283in}{4.536132in}}%
\pgfpathlineto{\pgfqpoint{2.925283in}{4.623868in}}%
\pgfusepath{stroke,fill}%
\end{pgfscope}%
\begin{pgfscope}%
\pgfpathrectangle{\pgfqpoint{0.380943in}{4.185189in}}{\pgfqpoint{4.650000in}{0.614151in}}%
\pgfusepath{clip}%
\pgfsetbuttcap%
\pgfsetroundjoin%
\definecolor{currentfill}{rgb}{0.992326,0.765229,0.614840}%
\pgfsetfillcolor{currentfill}%
\pgfsetlinewidth{0.250937pt}%
\definecolor{currentstroke}{rgb}{1.000000,1.000000,1.000000}%
\pgfsetstrokecolor{currentstroke}%
\pgfsetdash{}{0pt}%
\pgfpathmoveto{\pgfqpoint{3.013019in}{4.623868in}}%
\pgfpathlineto{\pgfqpoint{3.100754in}{4.623868in}}%
\pgfpathlineto{\pgfqpoint{3.100754in}{4.536132in}}%
\pgfpathlineto{\pgfqpoint{3.013019in}{4.536132in}}%
\pgfpathlineto{\pgfqpoint{3.013019in}{4.623868in}}%
\pgfusepath{stroke,fill}%
\end{pgfscope}%
\begin{pgfscope}%
\pgfpathrectangle{\pgfqpoint{0.380943in}{4.185189in}}{\pgfqpoint{4.650000in}{0.614151in}}%
\pgfusepath{clip}%
\pgfsetbuttcap%
\pgfsetroundjoin%
\definecolor{currentfill}{rgb}{0.992326,0.765229,0.614840}%
\pgfsetfillcolor{currentfill}%
\pgfsetlinewidth{0.250937pt}%
\definecolor{currentstroke}{rgb}{1.000000,1.000000,1.000000}%
\pgfsetstrokecolor{currentstroke}%
\pgfsetdash{}{0pt}%
\pgfpathmoveto{\pgfqpoint{3.100754in}{4.623868in}}%
\pgfpathlineto{\pgfqpoint{3.188490in}{4.623868in}}%
\pgfpathlineto{\pgfqpoint{3.188490in}{4.536132in}}%
\pgfpathlineto{\pgfqpoint{3.100754in}{4.536132in}}%
\pgfpathlineto{\pgfqpoint{3.100754in}{4.623868in}}%
\pgfusepath{stroke,fill}%
\end{pgfscope}%
\begin{pgfscope}%
\pgfpathrectangle{\pgfqpoint{0.380943in}{4.185189in}}{\pgfqpoint{4.650000in}{0.614151in}}%
\pgfusepath{clip}%
\pgfsetbuttcap%
\pgfsetroundjoin%
\definecolor{currentfill}{rgb}{0.972549,0.870588,0.692810}%
\pgfsetfillcolor{currentfill}%
\pgfsetlinewidth{0.250937pt}%
\definecolor{currentstroke}{rgb}{1.000000,1.000000,1.000000}%
\pgfsetstrokecolor{currentstroke}%
\pgfsetdash{}{0pt}%
\pgfpathmoveto{\pgfqpoint{3.188490in}{4.623868in}}%
\pgfpathlineto{\pgfqpoint{3.276226in}{4.623868in}}%
\pgfpathlineto{\pgfqpoint{3.276226in}{4.536132in}}%
\pgfpathlineto{\pgfqpoint{3.188490in}{4.536132in}}%
\pgfpathlineto{\pgfqpoint{3.188490in}{4.623868in}}%
\pgfusepath{stroke,fill}%
\end{pgfscope}%
\begin{pgfscope}%
\pgfpathrectangle{\pgfqpoint{0.380943in}{4.185189in}}{\pgfqpoint{4.650000in}{0.614151in}}%
\pgfusepath{clip}%
\pgfsetbuttcap%
\pgfsetroundjoin%
\definecolor{currentfill}{rgb}{0.965444,0.906113,0.711757}%
\pgfsetfillcolor{currentfill}%
\pgfsetlinewidth{0.250937pt}%
\definecolor{currentstroke}{rgb}{1.000000,1.000000,1.000000}%
\pgfsetstrokecolor{currentstroke}%
\pgfsetdash{}{0pt}%
\pgfpathmoveto{\pgfqpoint{3.276226in}{4.623868in}}%
\pgfpathlineto{\pgfqpoint{3.363962in}{4.623868in}}%
\pgfpathlineto{\pgfqpoint{3.363962in}{4.536132in}}%
\pgfpathlineto{\pgfqpoint{3.276226in}{4.536132in}}%
\pgfpathlineto{\pgfqpoint{3.276226in}{4.623868in}}%
\pgfusepath{stroke,fill}%
\end{pgfscope}%
\begin{pgfscope}%
\pgfpathrectangle{\pgfqpoint{0.380943in}{4.185189in}}{\pgfqpoint{4.650000in}{0.614151in}}%
\pgfusepath{clip}%
\pgfsetbuttcap%
\pgfsetroundjoin%
\definecolor{currentfill}{rgb}{0.996571,0.720538,0.589189}%
\pgfsetfillcolor{currentfill}%
\pgfsetlinewidth{0.250937pt}%
\definecolor{currentstroke}{rgb}{1.000000,1.000000,1.000000}%
\pgfsetstrokecolor{currentstroke}%
\pgfsetdash{}{0pt}%
\pgfpathmoveto{\pgfqpoint{3.363962in}{4.623868in}}%
\pgfpathlineto{\pgfqpoint{3.451698in}{4.623868in}}%
\pgfpathlineto{\pgfqpoint{3.451698in}{4.536132in}}%
\pgfpathlineto{\pgfqpoint{3.363962in}{4.536132in}}%
\pgfpathlineto{\pgfqpoint{3.363962in}{4.623868in}}%
\pgfusepath{stroke,fill}%
\end{pgfscope}%
\begin{pgfscope}%
\pgfpathrectangle{\pgfqpoint{0.380943in}{4.185189in}}{\pgfqpoint{4.650000in}{0.614151in}}%
\pgfusepath{clip}%
\pgfsetbuttcap%
\pgfsetroundjoin%
\definecolor{currentfill}{rgb}{0.979654,0.837186,0.669619}%
\pgfsetfillcolor{currentfill}%
\pgfsetlinewidth{0.250937pt}%
\definecolor{currentstroke}{rgb}{1.000000,1.000000,1.000000}%
\pgfsetstrokecolor{currentstroke}%
\pgfsetdash{}{0pt}%
\pgfpathmoveto{\pgfqpoint{3.451698in}{4.623868in}}%
\pgfpathlineto{\pgfqpoint{3.539434in}{4.623868in}}%
\pgfpathlineto{\pgfqpoint{3.539434in}{4.536132in}}%
\pgfpathlineto{\pgfqpoint{3.451698in}{4.536132in}}%
\pgfpathlineto{\pgfqpoint{3.451698in}{4.623868in}}%
\pgfusepath{stroke,fill}%
\end{pgfscope}%
\begin{pgfscope}%
\pgfpathrectangle{\pgfqpoint{0.380943in}{4.185189in}}{\pgfqpoint{4.650000in}{0.614151in}}%
\pgfusepath{clip}%
\pgfsetbuttcap%
\pgfsetroundjoin%
\definecolor{currentfill}{rgb}{0.992326,0.765229,0.614840}%
\pgfsetfillcolor{currentfill}%
\pgfsetlinewidth{0.250937pt}%
\definecolor{currentstroke}{rgb}{1.000000,1.000000,1.000000}%
\pgfsetstrokecolor{currentstroke}%
\pgfsetdash{}{0pt}%
\pgfpathmoveto{\pgfqpoint{3.539434in}{4.623868in}}%
\pgfpathlineto{\pgfqpoint{3.627169in}{4.623868in}}%
\pgfpathlineto{\pgfqpoint{3.627169in}{4.536132in}}%
\pgfpathlineto{\pgfqpoint{3.539434in}{4.536132in}}%
\pgfpathlineto{\pgfqpoint{3.539434in}{4.623868in}}%
\pgfusepath{stroke,fill}%
\end{pgfscope}%
\begin{pgfscope}%
\pgfpathrectangle{\pgfqpoint{0.380943in}{4.185189in}}{\pgfqpoint{4.650000in}{0.614151in}}%
\pgfusepath{clip}%
\pgfsetbuttcap%
\pgfsetroundjoin%
\definecolor{currentfill}{rgb}{0.996571,0.720538,0.589189}%
\pgfsetfillcolor{currentfill}%
\pgfsetlinewidth{0.250937pt}%
\definecolor{currentstroke}{rgb}{1.000000,1.000000,1.000000}%
\pgfsetstrokecolor{currentstroke}%
\pgfsetdash{}{0pt}%
\pgfpathmoveto{\pgfqpoint{3.627169in}{4.623868in}}%
\pgfpathlineto{\pgfqpoint{3.714905in}{4.623868in}}%
\pgfpathlineto{\pgfqpoint{3.714905in}{4.536132in}}%
\pgfpathlineto{\pgfqpoint{3.627169in}{4.536132in}}%
\pgfpathlineto{\pgfqpoint{3.627169in}{4.623868in}}%
\pgfusepath{stroke,fill}%
\end{pgfscope}%
\begin{pgfscope}%
\pgfpathrectangle{\pgfqpoint{0.380943in}{4.185189in}}{\pgfqpoint{4.650000in}{0.614151in}}%
\pgfusepath{clip}%
\pgfsetbuttcap%
\pgfsetroundjoin%
\definecolor{currentfill}{rgb}{0.979654,0.837186,0.669619}%
\pgfsetfillcolor{currentfill}%
\pgfsetlinewidth{0.250937pt}%
\definecolor{currentstroke}{rgb}{1.000000,1.000000,1.000000}%
\pgfsetstrokecolor{currentstroke}%
\pgfsetdash{}{0pt}%
\pgfpathmoveto{\pgfqpoint{3.714905in}{4.623868in}}%
\pgfpathlineto{\pgfqpoint{3.802641in}{4.623868in}}%
\pgfpathlineto{\pgfqpoint{3.802641in}{4.536132in}}%
\pgfpathlineto{\pgfqpoint{3.714905in}{4.536132in}}%
\pgfpathlineto{\pgfqpoint{3.714905in}{4.623868in}}%
\pgfusepath{stroke,fill}%
\end{pgfscope}%
\begin{pgfscope}%
\pgfpathrectangle{\pgfqpoint{0.380943in}{4.185189in}}{\pgfqpoint{4.650000in}{0.614151in}}%
\pgfusepath{clip}%
\pgfsetbuttcap%
\pgfsetroundjoin%
\definecolor{currentfill}{rgb}{0.998939,0.658962,0.556032}%
\pgfsetfillcolor{currentfill}%
\pgfsetlinewidth{0.250937pt}%
\definecolor{currentstroke}{rgb}{1.000000,1.000000,1.000000}%
\pgfsetstrokecolor{currentstroke}%
\pgfsetdash{}{0pt}%
\pgfpathmoveto{\pgfqpoint{3.802641in}{4.623868in}}%
\pgfpathlineto{\pgfqpoint{3.890377in}{4.623868in}}%
\pgfpathlineto{\pgfqpoint{3.890377in}{4.536132in}}%
\pgfpathlineto{\pgfqpoint{3.802641in}{4.536132in}}%
\pgfpathlineto{\pgfqpoint{3.802641in}{4.623868in}}%
\pgfusepath{stroke,fill}%
\end{pgfscope}%
\begin{pgfscope}%
\pgfpathrectangle{\pgfqpoint{0.380943in}{4.185189in}}{\pgfqpoint{4.650000in}{0.614151in}}%
\pgfusepath{clip}%
\pgfsetbuttcap%
\pgfsetroundjoin%
\definecolor{currentfill}{rgb}{1.000000,0.509404,0.491473}%
\pgfsetfillcolor{currentfill}%
\pgfsetlinewidth{0.250937pt}%
\definecolor{currentstroke}{rgb}{1.000000,1.000000,1.000000}%
\pgfsetstrokecolor{currentstroke}%
\pgfsetdash{}{0pt}%
\pgfpathmoveto{\pgfqpoint{3.890377in}{4.623868in}}%
\pgfpathlineto{\pgfqpoint{3.978113in}{4.623868in}}%
\pgfpathlineto{\pgfqpoint{3.978113in}{4.536132in}}%
\pgfpathlineto{\pgfqpoint{3.890377in}{4.536132in}}%
\pgfpathlineto{\pgfqpoint{3.890377in}{4.623868in}}%
\pgfusepath{stroke,fill}%
\end{pgfscope}%
\begin{pgfscope}%
\pgfpathrectangle{\pgfqpoint{0.380943in}{4.185189in}}{\pgfqpoint{4.650000in}{0.614151in}}%
\pgfusepath{clip}%
\pgfsetbuttcap%
\pgfsetroundjoin%
\definecolor{currentfill}{rgb}{0.922338,0.400769,0.400769}%
\pgfsetfillcolor{currentfill}%
\pgfsetlinewidth{0.250937pt}%
\definecolor{currentstroke}{rgb}{1.000000,1.000000,1.000000}%
\pgfsetstrokecolor{currentstroke}%
\pgfsetdash{}{0pt}%
\pgfpathmoveto{\pgfqpoint{3.978113in}{4.623868in}}%
\pgfpathlineto{\pgfqpoint{4.065849in}{4.623868in}}%
\pgfpathlineto{\pgfqpoint{4.065849in}{4.536132in}}%
\pgfpathlineto{\pgfqpoint{3.978113in}{4.536132in}}%
\pgfpathlineto{\pgfqpoint{3.978113in}{4.623868in}}%
\pgfusepath{stroke,fill}%
\end{pgfscope}%
\begin{pgfscope}%
\pgfpathrectangle{\pgfqpoint{0.380943in}{4.185189in}}{\pgfqpoint{4.650000in}{0.614151in}}%
\pgfusepath{clip}%
\pgfsetbuttcap%
\pgfsetroundjoin%
\definecolor{currentfill}{rgb}{0.992326,0.765229,0.614840}%
\pgfsetfillcolor{currentfill}%
\pgfsetlinewidth{0.250937pt}%
\definecolor{currentstroke}{rgb}{1.000000,1.000000,1.000000}%
\pgfsetstrokecolor{currentstroke}%
\pgfsetdash{}{0pt}%
\pgfpathmoveto{\pgfqpoint{4.065849in}{4.623868in}}%
\pgfpathlineto{\pgfqpoint{4.153585in}{4.623868in}}%
\pgfpathlineto{\pgfqpoint{4.153585in}{4.536132in}}%
\pgfpathlineto{\pgfqpoint{4.065849in}{4.536132in}}%
\pgfpathlineto{\pgfqpoint{4.065849in}{4.623868in}}%
\pgfusepath{stroke,fill}%
\end{pgfscope}%
\begin{pgfscope}%
\pgfpathrectangle{\pgfqpoint{0.380943in}{4.185189in}}{\pgfqpoint{4.650000in}{0.614151in}}%
\pgfusepath{clip}%
\pgfsetbuttcap%
\pgfsetroundjoin%
\definecolor{currentfill}{rgb}{0.998939,0.658962,0.556032}%
\pgfsetfillcolor{currentfill}%
\pgfsetlinewidth{0.250937pt}%
\definecolor{currentstroke}{rgb}{1.000000,1.000000,1.000000}%
\pgfsetstrokecolor{currentstroke}%
\pgfsetdash{}{0pt}%
\pgfpathmoveto{\pgfqpoint{4.153585in}{4.623868in}}%
\pgfpathlineto{\pgfqpoint{4.241320in}{4.623868in}}%
\pgfpathlineto{\pgfqpoint{4.241320in}{4.536132in}}%
\pgfpathlineto{\pgfqpoint{4.153585in}{4.536132in}}%
\pgfpathlineto{\pgfqpoint{4.153585in}{4.623868in}}%
\pgfusepath{stroke,fill}%
\end{pgfscope}%
\begin{pgfscope}%
\pgfpathrectangle{\pgfqpoint{0.380943in}{4.185189in}}{\pgfqpoint{4.650000in}{0.614151in}}%
\pgfusepath{clip}%
\pgfsetbuttcap%
\pgfsetroundjoin%
\definecolor{currentfill}{rgb}{0.986759,0.806398,0.641200}%
\pgfsetfillcolor{currentfill}%
\pgfsetlinewidth{0.250937pt}%
\definecolor{currentstroke}{rgb}{1.000000,1.000000,1.000000}%
\pgfsetstrokecolor{currentstroke}%
\pgfsetdash{}{0pt}%
\pgfpathmoveto{\pgfqpoint{4.241320in}{4.623868in}}%
\pgfpathlineto{\pgfqpoint{4.329056in}{4.623868in}}%
\pgfpathlineto{\pgfqpoint{4.329056in}{4.536132in}}%
\pgfpathlineto{\pgfqpoint{4.241320in}{4.536132in}}%
\pgfpathlineto{\pgfqpoint{4.241320in}{4.623868in}}%
\pgfusepath{stroke,fill}%
\end{pgfscope}%
\begin{pgfscope}%
\pgfpathrectangle{\pgfqpoint{0.380943in}{4.185189in}}{\pgfqpoint{4.650000in}{0.614151in}}%
\pgfusepath{clip}%
\pgfsetbuttcap%
\pgfsetroundjoin%
\definecolor{currentfill}{rgb}{0.965444,0.906113,0.711757}%
\pgfsetfillcolor{currentfill}%
\pgfsetlinewidth{0.250937pt}%
\definecolor{currentstroke}{rgb}{1.000000,1.000000,1.000000}%
\pgfsetstrokecolor{currentstroke}%
\pgfsetdash{}{0pt}%
\pgfpathmoveto{\pgfqpoint{4.329056in}{4.623868in}}%
\pgfpathlineto{\pgfqpoint{4.416792in}{4.623868in}}%
\pgfpathlineto{\pgfqpoint{4.416792in}{4.536132in}}%
\pgfpathlineto{\pgfqpoint{4.329056in}{4.536132in}}%
\pgfpathlineto{\pgfqpoint{4.329056in}{4.623868in}}%
\pgfusepath{stroke,fill}%
\end{pgfscope}%
\begin{pgfscope}%
\pgfpathrectangle{\pgfqpoint{0.380943in}{4.185189in}}{\pgfqpoint{4.650000in}{0.614151in}}%
\pgfusepath{clip}%
\pgfsetbuttcap%
\pgfsetroundjoin%
\definecolor{currentfill}{rgb}{1.000000,0.605229,0.530719}%
\pgfsetfillcolor{currentfill}%
\pgfsetlinewidth{0.250937pt}%
\definecolor{currentstroke}{rgb}{1.000000,1.000000,1.000000}%
\pgfsetstrokecolor{currentstroke}%
\pgfsetdash{}{0pt}%
\pgfpathmoveto{\pgfqpoint{4.416792in}{4.623868in}}%
\pgfpathlineto{\pgfqpoint{4.504528in}{4.623868in}}%
\pgfpathlineto{\pgfqpoint{4.504528in}{4.536132in}}%
\pgfpathlineto{\pgfqpoint{4.416792in}{4.536132in}}%
\pgfpathlineto{\pgfqpoint{4.416792in}{4.623868in}}%
\pgfusepath{stroke,fill}%
\end{pgfscope}%
\begin{pgfscope}%
\pgfpathrectangle{\pgfqpoint{0.380943in}{4.185189in}}{\pgfqpoint{4.650000in}{0.614151in}}%
\pgfusepath{clip}%
\pgfsetbuttcap%
\pgfsetroundjoin%
\definecolor{currentfill}{rgb}{0.992326,0.765229,0.614840}%
\pgfsetfillcolor{currentfill}%
\pgfsetlinewidth{0.250937pt}%
\definecolor{currentstroke}{rgb}{1.000000,1.000000,1.000000}%
\pgfsetstrokecolor{currentstroke}%
\pgfsetdash{}{0pt}%
\pgfpathmoveto{\pgfqpoint{4.504528in}{4.623868in}}%
\pgfpathlineto{\pgfqpoint{4.592264in}{4.623868in}}%
\pgfpathlineto{\pgfqpoint{4.592264in}{4.536132in}}%
\pgfpathlineto{\pgfqpoint{4.504528in}{4.536132in}}%
\pgfpathlineto{\pgfqpoint{4.504528in}{4.623868in}}%
\pgfusepath{stroke,fill}%
\end{pgfscope}%
\begin{pgfscope}%
\pgfpathrectangle{\pgfqpoint{0.380943in}{4.185189in}}{\pgfqpoint{4.650000in}{0.614151in}}%
\pgfusepath{clip}%
\pgfsetbuttcap%
\pgfsetroundjoin%
\definecolor{currentfill}{rgb}{0.979654,0.837186,0.669619}%
\pgfsetfillcolor{currentfill}%
\pgfsetlinewidth{0.250937pt}%
\definecolor{currentstroke}{rgb}{1.000000,1.000000,1.000000}%
\pgfsetstrokecolor{currentstroke}%
\pgfsetdash{}{0pt}%
\pgfpathmoveto{\pgfqpoint{4.592264in}{4.623868in}}%
\pgfpathlineto{\pgfqpoint{4.680000in}{4.623868in}}%
\pgfpathlineto{\pgfqpoint{4.680000in}{4.536132in}}%
\pgfpathlineto{\pgfqpoint{4.592264in}{4.536132in}}%
\pgfpathlineto{\pgfqpoint{4.592264in}{4.623868in}}%
\pgfusepath{stroke,fill}%
\end{pgfscope}%
\begin{pgfscope}%
\pgfpathrectangle{\pgfqpoint{0.380943in}{4.185189in}}{\pgfqpoint{4.650000in}{0.614151in}}%
\pgfusepath{clip}%
\pgfsetbuttcap%
\pgfsetroundjoin%
\definecolor{currentfill}{rgb}{0.962414,0.923552,0.722891}%
\pgfsetfillcolor{currentfill}%
\pgfsetlinewidth{0.250937pt}%
\definecolor{currentstroke}{rgb}{1.000000,1.000000,1.000000}%
\pgfsetstrokecolor{currentstroke}%
\pgfsetdash{}{0pt}%
\pgfpathmoveto{\pgfqpoint{4.680000in}{4.623868in}}%
\pgfpathlineto{\pgfqpoint{4.767736in}{4.623868in}}%
\pgfpathlineto{\pgfqpoint{4.767736in}{4.536132in}}%
\pgfpathlineto{\pgfqpoint{4.680000in}{4.536132in}}%
\pgfpathlineto{\pgfqpoint{4.680000in}{4.623868in}}%
\pgfusepath{stroke,fill}%
\end{pgfscope}%
\begin{pgfscope}%
\pgfpathrectangle{\pgfqpoint{0.380943in}{4.185189in}}{\pgfqpoint{4.650000in}{0.614151in}}%
\pgfusepath{clip}%
\pgfsetbuttcap%
\pgfsetroundjoin%
\definecolor{currentfill}{rgb}{0.979654,0.837186,0.669619}%
\pgfsetfillcolor{currentfill}%
\pgfsetlinewidth{0.250937pt}%
\definecolor{currentstroke}{rgb}{1.000000,1.000000,1.000000}%
\pgfsetstrokecolor{currentstroke}%
\pgfsetdash{}{0pt}%
\pgfpathmoveto{\pgfqpoint{4.767736in}{4.623868in}}%
\pgfpathlineto{\pgfqpoint{4.855471in}{4.623868in}}%
\pgfpathlineto{\pgfqpoint{4.855471in}{4.536132in}}%
\pgfpathlineto{\pgfqpoint{4.767736in}{4.536132in}}%
\pgfpathlineto{\pgfqpoint{4.767736in}{4.623868in}}%
\pgfusepath{stroke,fill}%
\end{pgfscope}%
\begin{pgfscope}%
\pgfpathrectangle{\pgfqpoint{0.380943in}{4.185189in}}{\pgfqpoint{4.650000in}{0.614151in}}%
\pgfusepath{clip}%
\pgfsetbuttcap%
\pgfsetroundjoin%
\definecolor{currentfill}{rgb}{1.000000,1.000000,0.929412}%
\pgfsetfillcolor{currentfill}%
\pgfsetlinewidth{0.250937pt}%
\definecolor{currentstroke}{rgb}{1.000000,1.000000,1.000000}%
\pgfsetstrokecolor{currentstroke}%
\pgfsetdash{}{0pt}%
\pgfpathmoveto{\pgfqpoint{4.855471in}{4.623868in}}%
\pgfpathlineto{\pgfqpoint{4.943207in}{4.623868in}}%
\pgfpathlineto{\pgfqpoint{4.943207in}{4.536132in}}%
\pgfpathlineto{\pgfqpoint{4.855471in}{4.536132in}}%
\pgfpathlineto{\pgfqpoint{4.855471in}{4.623868in}}%
\pgfusepath{stroke,fill}%
\end{pgfscope}%
\begin{pgfscope}%
\pgfpathrectangle{\pgfqpoint{0.380943in}{4.185189in}}{\pgfqpoint{4.650000in}{0.614151in}}%
\pgfusepath{clip}%
\pgfsetbuttcap%
\pgfsetroundjoin%
\pgfsetlinewidth{0.250937pt}%
\definecolor{currentstroke}{rgb}{1.000000,1.000000,1.000000}%
\pgfsetstrokecolor{currentstroke}%
\pgfsetdash{}{0pt}%
\pgfpathmoveto{\pgfqpoint{4.943207in}{4.623868in}}%
\pgfpathlineto{\pgfqpoint{5.030943in}{4.623868in}}%
\pgfpathlineto{\pgfqpoint{5.030943in}{4.536132in}}%
\pgfpathlineto{\pgfqpoint{4.943207in}{4.536132in}}%
\pgfpathlineto{\pgfqpoint{4.943207in}{4.623868in}}%
\pgfusepath{stroke}%
\end{pgfscope}%
\begin{pgfscope}%
\pgfpathrectangle{\pgfqpoint{0.380943in}{4.185189in}}{\pgfqpoint{4.650000in}{0.614151in}}%
\pgfusepath{clip}%
\pgfsetbuttcap%
\pgfsetroundjoin%
\definecolor{currentfill}{rgb}{0.996571,0.720538,0.589189}%
\pgfsetfillcolor{currentfill}%
\pgfsetlinewidth{0.250937pt}%
\definecolor{currentstroke}{rgb}{1.000000,1.000000,1.000000}%
\pgfsetstrokecolor{currentstroke}%
\pgfsetdash{}{0pt}%
\pgfpathmoveto{\pgfqpoint{0.380943in}{4.536132in}}%
\pgfpathlineto{\pgfqpoint{0.468679in}{4.536132in}}%
\pgfpathlineto{\pgfqpoint{0.468679in}{4.448396in}}%
\pgfpathlineto{\pgfqpoint{0.380943in}{4.448396in}}%
\pgfpathlineto{\pgfqpoint{0.380943in}{4.536132in}}%
\pgfusepath{stroke,fill}%
\end{pgfscope}%
\begin{pgfscope}%
\pgfpathrectangle{\pgfqpoint{0.380943in}{4.185189in}}{\pgfqpoint{4.650000in}{0.614151in}}%
\pgfusepath{clip}%
\pgfsetbuttcap%
\pgfsetroundjoin%
\definecolor{currentfill}{rgb}{0.972549,0.870588,0.692810}%
\pgfsetfillcolor{currentfill}%
\pgfsetlinewidth{0.250937pt}%
\definecolor{currentstroke}{rgb}{1.000000,1.000000,1.000000}%
\pgfsetstrokecolor{currentstroke}%
\pgfsetdash{}{0pt}%
\pgfpathmoveto{\pgfqpoint{0.468679in}{4.536132in}}%
\pgfpathlineto{\pgfqpoint{0.556415in}{4.536132in}}%
\pgfpathlineto{\pgfqpoint{0.556415in}{4.448396in}}%
\pgfpathlineto{\pgfqpoint{0.468679in}{4.448396in}}%
\pgfpathlineto{\pgfqpoint{0.468679in}{4.536132in}}%
\pgfusepath{stroke,fill}%
\end{pgfscope}%
\begin{pgfscope}%
\pgfpathrectangle{\pgfqpoint{0.380943in}{4.185189in}}{\pgfqpoint{4.650000in}{0.614151in}}%
\pgfusepath{clip}%
\pgfsetbuttcap%
\pgfsetroundjoin%
\definecolor{currentfill}{rgb}{0.996571,0.720538,0.589189}%
\pgfsetfillcolor{currentfill}%
\pgfsetlinewidth{0.250937pt}%
\definecolor{currentstroke}{rgb}{1.000000,1.000000,1.000000}%
\pgfsetstrokecolor{currentstroke}%
\pgfsetdash{}{0pt}%
\pgfpathmoveto{\pgfqpoint{0.556415in}{4.536132in}}%
\pgfpathlineto{\pgfqpoint{0.644151in}{4.536132in}}%
\pgfpathlineto{\pgfqpoint{0.644151in}{4.448396in}}%
\pgfpathlineto{\pgfqpoint{0.556415in}{4.448396in}}%
\pgfpathlineto{\pgfqpoint{0.556415in}{4.536132in}}%
\pgfusepath{stroke,fill}%
\end{pgfscope}%
\begin{pgfscope}%
\pgfpathrectangle{\pgfqpoint{0.380943in}{4.185189in}}{\pgfqpoint{4.650000in}{0.614151in}}%
\pgfusepath{clip}%
\pgfsetbuttcap%
\pgfsetroundjoin%
\definecolor{currentfill}{rgb}{0.992326,0.765229,0.614840}%
\pgfsetfillcolor{currentfill}%
\pgfsetlinewidth{0.250937pt}%
\definecolor{currentstroke}{rgb}{1.000000,1.000000,1.000000}%
\pgfsetstrokecolor{currentstroke}%
\pgfsetdash{}{0pt}%
\pgfpathmoveto{\pgfqpoint{0.644151in}{4.536132in}}%
\pgfpathlineto{\pgfqpoint{0.731886in}{4.536132in}}%
\pgfpathlineto{\pgfqpoint{0.731886in}{4.448396in}}%
\pgfpathlineto{\pgfqpoint{0.644151in}{4.448396in}}%
\pgfpathlineto{\pgfqpoint{0.644151in}{4.536132in}}%
\pgfusepath{stroke,fill}%
\end{pgfscope}%
\begin{pgfscope}%
\pgfpathrectangle{\pgfqpoint{0.380943in}{4.185189in}}{\pgfqpoint{4.650000in}{0.614151in}}%
\pgfusepath{clip}%
\pgfsetbuttcap%
\pgfsetroundjoin%
\definecolor{currentfill}{rgb}{0.986759,0.806398,0.641200}%
\pgfsetfillcolor{currentfill}%
\pgfsetlinewidth{0.250937pt}%
\definecolor{currentstroke}{rgb}{1.000000,1.000000,1.000000}%
\pgfsetstrokecolor{currentstroke}%
\pgfsetdash{}{0pt}%
\pgfpathmoveto{\pgfqpoint{0.731886in}{4.536132in}}%
\pgfpathlineto{\pgfqpoint{0.819622in}{4.536132in}}%
\pgfpathlineto{\pgfqpoint{0.819622in}{4.448396in}}%
\pgfpathlineto{\pgfqpoint{0.731886in}{4.448396in}}%
\pgfpathlineto{\pgfqpoint{0.731886in}{4.536132in}}%
\pgfusepath{stroke,fill}%
\end{pgfscope}%
\begin{pgfscope}%
\pgfpathrectangle{\pgfqpoint{0.380943in}{4.185189in}}{\pgfqpoint{4.650000in}{0.614151in}}%
\pgfusepath{clip}%
\pgfsetbuttcap%
\pgfsetroundjoin%
\definecolor{currentfill}{rgb}{1.000000,0.557862,0.511772}%
\pgfsetfillcolor{currentfill}%
\pgfsetlinewidth{0.250937pt}%
\definecolor{currentstroke}{rgb}{1.000000,1.000000,1.000000}%
\pgfsetstrokecolor{currentstroke}%
\pgfsetdash{}{0pt}%
\pgfpathmoveto{\pgfqpoint{0.819622in}{4.536132in}}%
\pgfpathlineto{\pgfqpoint{0.907358in}{4.536132in}}%
\pgfpathlineto{\pgfqpoint{0.907358in}{4.448396in}}%
\pgfpathlineto{\pgfqpoint{0.819622in}{4.448396in}}%
\pgfpathlineto{\pgfqpoint{0.819622in}{4.536132in}}%
\pgfusepath{stroke,fill}%
\end{pgfscope}%
\begin{pgfscope}%
\pgfpathrectangle{\pgfqpoint{0.380943in}{4.185189in}}{\pgfqpoint{4.650000in}{0.614151in}}%
\pgfusepath{clip}%
\pgfsetbuttcap%
\pgfsetroundjoin%
\definecolor{currentfill}{rgb}{0.986759,0.806398,0.641200}%
\pgfsetfillcolor{currentfill}%
\pgfsetlinewidth{0.250937pt}%
\definecolor{currentstroke}{rgb}{1.000000,1.000000,1.000000}%
\pgfsetstrokecolor{currentstroke}%
\pgfsetdash{}{0pt}%
\pgfpathmoveto{\pgfqpoint{0.907358in}{4.536132in}}%
\pgfpathlineto{\pgfqpoint{0.995094in}{4.536132in}}%
\pgfpathlineto{\pgfqpoint{0.995094in}{4.448396in}}%
\pgfpathlineto{\pgfqpoint{0.907358in}{4.448396in}}%
\pgfpathlineto{\pgfqpoint{0.907358in}{4.536132in}}%
\pgfusepath{stroke,fill}%
\end{pgfscope}%
\begin{pgfscope}%
\pgfpathrectangle{\pgfqpoint{0.380943in}{4.185189in}}{\pgfqpoint{4.650000in}{0.614151in}}%
\pgfusepath{clip}%
\pgfsetbuttcap%
\pgfsetroundjoin%
\definecolor{currentfill}{rgb}{0.996571,0.720538,0.589189}%
\pgfsetfillcolor{currentfill}%
\pgfsetlinewidth{0.250937pt}%
\definecolor{currentstroke}{rgb}{1.000000,1.000000,1.000000}%
\pgfsetstrokecolor{currentstroke}%
\pgfsetdash{}{0pt}%
\pgfpathmoveto{\pgfqpoint{0.995094in}{4.536132in}}%
\pgfpathlineto{\pgfqpoint{1.082830in}{4.536132in}}%
\pgfpathlineto{\pgfqpoint{1.082830in}{4.448396in}}%
\pgfpathlineto{\pgfqpoint{0.995094in}{4.448396in}}%
\pgfpathlineto{\pgfqpoint{0.995094in}{4.536132in}}%
\pgfusepath{stroke,fill}%
\end{pgfscope}%
\begin{pgfscope}%
\pgfpathrectangle{\pgfqpoint{0.380943in}{4.185189in}}{\pgfqpoint{4.650000in}{0.614151in}}%
\pgfusepath{clip}%
\pgfsetbuttcap%
\pgfsetroundjoin%
\definecolor{currentfill}{rgb}{0.979654,0.837186,0.669619}%
\pgfsetfillcolor{currentfill}%
\pgfsetlinewidth{0.250937pt}%
\definecolor{currentstroke}{rgb}{1.000000,1.000000,1.000000}%
\pgfsetstrokecolor{currentstroke}%
\pgfsetdash{}{0pt}%
\pgfpathmoveto{\pgfqpoint{1.082830in}{4.536132in}}%
\pgfpathlineto{\pgfqpoint{1.170566in}{4.536132in}}%
\pgfpathlineto{\pgfqpoint{1.170566in}{4.448396in}}%
\pgfpathlineto{\pgfqpoint{1.082830in}{4.448396in}}%
\pgfpathlineto{\pgfqpoint{1.082830in}{4.536132in}}%
\pgfusepath{stroke,fill}%
\end{pgfscope}%
\begin{pgfscope}%
\pgfpathrectangle{\pgfqpoint{0.380943in}{4.185189in}}{\pgfqpoint{4.650000in}{0.614151in}}%
\pgfusepath{clip}%
\pgfsetbuttcap%
\pgfsetroundjoin%
\definecolor{currentfill}{rgb}{0.979654,0.837186,0.669619}%
\pgfsetfillcolor{currentfill}%
\pgfsetlinewidth{0.250937pt}%
\definecolor{currentstroke}{rgb}{1.000000,1.000000,1.000000}%
\pgfsetstrokecolor{currentstroke}%
\pgfsetdash{}{0pt}%
\pgfpathmoveto{\pgfqpoint{1.170566in}{4.536132in}}%
\pgfpathlineto{\pgfqpoint{1.258302in}{4.536132in}}%
\pgfpathlineto{\pgfqpoint{1.258302in}{4.448396in}}%
\pgfpathlineto{\pgfqpoint{1.170566in}{4.448396in}}%
\pgfpathlineto{\pgfqpoint{1.170566in}{4.536132in}}%
\pgfusepath{stroke,fill}%
\end{pgfscope}%
\begin{pgfscope}%
\pgfpathrectangle{\pgfqpoint{0.380943in}{4.185189in}}{\pgfqpoint{4.650000in}{0.614151in}}%
\pgfusepath{clip}%
\pgfsetbuttcap%
\pgfsetroundjoin%
\definecolor{currentfill}{rgb}{1.000000,0.509404,0.491473}%
\pgfsetfillcolor{currentfill}%
\pgfsetlinewidth{0.250937pt}%
\definecolor{currentstroke}{rgb}{1.000000,1.000000,1.000000}%
\pgfsetstrokecolor{currentstroke}%
\pgfsetdash{}{0pt}%
\pgfpathmoveto{\pgfqpoint{1.258302in}{4.536132in}}%
\pgfpathlineto{\pgfqpoint{1.346037in}{4.536132in}}%
\pgfpathlineto{\pgfqpoint{1.346037in}{4.448396in}}%
\pgfpathlineto{\pgfqpoint{1.258302in}{4.448396in}}%
\pgfpathlineto{\pgfqpoint{1.258302in}{4.536132in}}%
\pgfusepath{stroke,fill}%
\end{pgfscope}%
\begin{pgfscope}%
\pgfpathrectangle{\pgfqpoint{0.380943in}{4.185189in}}{\pgfqpoint{4.650000in}{0.614151in}}%
\pgfusepath{clip}%
\pgfsetbuttcap%
\pgfsetroundjoin%
\definecolor{currentfill}{rgb}{0.986759,0.806398,0.641200}%
\pgfsetfillcolor{currentfill}%
\pgfsetlinewidth{0.250937pt}%
\definecolor{currentstroke}{rgb}{1.000000,1.000000,1.000000}%
\pgfsetstrokecolor{currentstroke}%
\pgfsetdash{}{0pt}%
\pgfpathmoveto{\pgfqpoint{1.346037in}{4.536132in}}%
\pgfpathlineto{\pgfqpoint{1.433773in}{4.536132in}}%
\pgfpathlineto{\pgfqpoint{1.433773in}{4.448396in}}%
\pgfpathlineto{\pgfqpoint{1.346037in}{4.448396in}}%
\pgfpathlineto{\pgfqpoint{1.346037in}{4.536132in}}%
\pgfusepath{stroke,fill}%
\end{pgfscope}%
\begin{pgfscope}%
\pgfpathrectangle{\pgfqpoint{0.380943in}{4.185189in}}{\pgfqpoint{4.650000in}{0.614151in}}%
\pgfusepath{clip}%
\pgfsetbuttcap%
\pgfsetroundjoin%
\definecolor{currentfill}{rgb}{0.972549,0.870588,0.692810}%
\pgfsetfillcolor{currentfill}%
\pgfsetlinewidth{0.250937pt}%
\definecolor{currentstroke}{rgb}{1.000000,1.000000,1.000000}%
\pgfsetstrokecolor{currentstroke}%
\pgfsetdash{}{0pt}%
\pgfpathmoveto{\pgfqpoint{1.433773in}{4.536132in}}%
\pgfpathlineto{\pgfqpoint{1.521509in}{4.536132in}}%
\pgfpathlineto{\pgfqpoint{1.521509in}{4.448396in}}%
\pgfpathlineto{\pgfqpoint{1.433773in}{4.448396in}}%
\pgfpathlineto{\pgfqpoint{1.433773in}{4.536132in}}%
\pgfusepath{stroke,fill}%
\end{pgfscope}%
\begin{pgfscope}%
\pgfpathrectangle{\pgfqpoint{0.380943in}{4.185189in}}{\pgfqpoint{4.650000in}{0.614151in}}%
\pgfusepath{clip}%
\pgfsetbuttcap%
\pgfsetroundjoin%
\definecolor{currentfill}{rgb}{0.965444,0.906113,0.711757}%
\pgfsetfillcolor{currentfill}%
\pgfsetlinewidth{0.250937pt}%
\definecolor{currentstroke}{rgb}{1.000000,1.000000,1.000000}%
\pgfsetstrokecolor{currentstroke}%
\pgfsetdash{}{0pt}%
\pgfpathmoveto{\pgfqpoint{1.521509in}{4.536132in}}%
\pgfpathlineto{\pgfqpoint{1.609245in}{4.536132in}}%
\pgfpathlineto{\pgfqpoint{1.609245in}{4.448396in}}%
\pgfpathlineto{\pgfqpoint{1.521509in}{4.448396in}}%
\pgfpathlineto{\pgfqpoint{1.521509in}{4.536132in}}%
\pgfusepath{stroke,fill}%
\end{pgfscope}%
\begin{pgfscope}%
\pgfpathrectangle{\pgfqpoint{0.380943in}{4.185189in}}{\pgfqpoint{4.650000in}{0.614151in}}%
\pgfusepath{clip}%
\pgfsetbuttcap%
\pgfsetroundjoin%
\definecolor{currentfill}{rgb}{0.992326,0.765229,0.614840}%
\pgfsetfillcolor{currentfill}%
\pgfsetlinewidth{0.250937pt}%
\definecolor{currentstroke}{rgb}{1.000000,1.000000,1.000000}%
\pgfsetstrokecolor{currentstroke}%
\pgfsetdash{}{0pt}%
\pgfpathmoveto{\pgfqpoint{1.609245in}{4.536132in}}%
\pgfpathlineto{\pgfqpoint{1.696981in}{4.536132in}}%
\pgfpathlineto{\pgfqpoint{1.696981in}{4.448396in}}%
\pgfpathlineto{\pgfqpoint{1.609245in}{4.448396in}}%
\pgfpathlineto{\pgfqpoint{1.609245in}{4.536132in}}%
\pgfusepath{stroke,fill}%
\end{pgfscope}%
\begin{pgfscope}%
\pgfpathrectangle{\pgfqpoint{0.380943in}{4.185189in}}{\pgfqpoint{4.650000in}{0.614151in}}%
\pgfusepath{clip}%
\pgfsetbuttcap%
\pgfsetroundjoin%
\definecolor{currentfill}{rgb}{0.986759,0.806398,0.641200}%
\pgfsetfillcolor{currentfill}%
\pgfsetlinewidth{0.250937pt}%
\definecolor{currentstroke}{rgb}{1.000000,1.000000,1.000000}%
\pgfsetstrokecolor{currentstroke}%
\pgfsetdash{}{0pt}%
\pgfpathmoveto{\pgfqpoint{1.696981in}{4.536132in}}%
\pgfpathlineto{\pgfqpoint{1.784717in}{4.536132in}}%
\pgfpathlineto{\pgfqpoint{1.784717in}{4.448396in}}%
\pgfpathlineto{\pgfqpoint{1.696981in}{4.448396in}}%
\pgfpathlineto{\pgfqpoint{1.696981in}{4.536132in}}%
\pgfusepath{stroke,fill}%
\end{pgfscope}%
\begin{pgfscope}%
\pgfpathrectangle{\pgfqpoint{0.380943in}{4.185189in}}{\pgfqpoint{4.650000in}{0.614151in}}%
\pgfusepath{clip}%
\pgfsetbuttcap%
\pgfsetroundjoin%
\definecolor{currentfill}{rgb}{1.000000,0.605229,0.530719}%
\pgfsetfillcolor{currentfill}%
\pgfsetlinewidth{0.250937pt}%
\definecolor{currentstroke}{rgb}{1.000000,1.000000,1.000000}%
\pgfsetstrokecolor{currentstroke}%
\pgfsetdash{}{0pt}%
\pgfpathmoveto{\pgfqpoint{1.784717in}{4.536132in}}%
\pgfpathlineto{\pgfqpoint{1.872452in}{4.536132in}}%
\pgfpathlineto{\pgfqpoint{1.872452in}{4.448396in}}%
\pgfpathlineto{\pgfqpoint{1.784717in}{4.448396in}}%
\pgfpathlineto{\pgfqpoint{1.784717in}{4.536132in}}%
\pgfusepath{stroke,fill}%
\end{pgfscope}%
\begin{pgfscope}%
\pgfpathrectangle{\pgfqpoint{0.380943in}{4.185189in}}{\pgfqpoint{4.650000in}{0.614151in}}%
\pgfusepath{clip}%
\pgfsetbuttcap%
\pgfsetroundjoin%
\definecolor{currentfill}{rgb}{0.972549,0.870588,0.692810}%
\pgfsetfillcolor{currentfill}%
\pgfsetlinewidth{0.250937pt}%
\definecolor{currentstroke}{rgb}{1.000000,1.000000,1.000000}%
\pgfsetstrokecolor{currentstroke}%
\pgfsetdash{}{0pt}%
\pgfpathmoveto{\pgfqpoint{1.872452in}{4.536132in}}%
\pgfpathlineto{\pgfqpoint{1.960188in}{4.536132in}}%
\pgfpathlineto{\pgfqpoint{1.960188in}{4.448396in}}%
\pgfpathlineto{\pgfqpoint{1.872452in}{4.448396in}}%
\pgfpathlineto{\pgfqpoint{1.872452in}{4.536132in}}%
\pgfusepath{stroke,fill}%
\end{pgfscope}%
\begin{pgfscope}%
\pgfpathrectangle{\pgfqpoint{0.380943in}{4.185189in}}{\pgfqpoint{4.650000in}{0.614151in}}%
\pgfusepath{clip}%
\pgfsetbuttcap%
\pgfsetroundjoin%
\definecolor{currentfill}{rgb}{0.996571,0.720538,0.589189}%
\pgfsetfillcolor{currentfill}%
\pgfsetlinewidth{0.250937pt}%
\definecolor{currentstroke}{rgb}{1.000000,1.000000,1.000000}%
\pgfsetstrokecolor{currentstroke}%
\pgfsetdash{}{0pt}%
\pgfpathmoveto{\pgfqpoint{1.960188in}{4.536132in}}%
\pgfpathlineto{\pgfqpoint{2.047924in}{4.536132in}}%
\pgfpathlineto{\pgfqpoint{2.047924in}{4.448396in}}%
\pgfpathlineto{\pgfqpoint{1.960188in}{4.448396in}}%
\pgfpathlineto{\pgfqpoint{1.960188in}{4.536132in}}%
\pgfusepath{stroke,fill}%
\end{pgfscope}%
\begin{pgfscope}%
\pgfpathrectangle{\pgfqpoint{0.380943in}{4.185189in}}{\pgfqpoint{4.650000in}{0.614151in}}%
\pgfusepath{clip}%
\pgfsetbuttcap%
\pgfsetroundjoin%
\definecolor{currentfill}{rgb}{1.000000,0.509404,0.491473}%
\pgfsetfillcolor{currentfill}%
\pgfsetlinewidth{0.250937pt}%
\definecolor{currentstroke}{rgb}{1.000000,1.000000,1.000000}%
\pgfsetstrokecolor{currentstroke}%
\pgfsetdash{}{0pt}%
\pgfpathmoveto{\pgfqpoint{2.047924in}{4.536132in}}%
\pgfpathlineto{\pgfqpoint{2.135660in}{4.536132in}}%
\pgfpathlineto{\pgfqpoint{2.135660in}{4.448396in}}%
\pgfpathlineto{\pgfqpoint{2.047924in}{4.448396in}}%
\pgfpathlineto{\pgfqpoint{2.047924in}{4.536132in}}%
\pgfusepath{stroke,fill}%
\end{pgfscope}%
\begin{pgfscope}%
\pgfpathrectangle{\pgfqpoint{0.380943in}{4.185189in}}{\pgfqpoint{4.650000in}{0.614151in}}%
\pgfusepath{clip}%
\pgfsetbuttcap%
\pgfsetroundjoin%
\definecolor{currentfill}{rgb}{0.965444,0.906113,0.711757}%
\pgfsetfillcolor{currentfill}%
\pgfsetlinewidth{0.250937pt}%
\definecolor{currentstroke}{rgb}{1.000000,1.000000,1.000000}%
\pgfsetstrokecolor{currentstroke}%
\pgfsetdash{}{0pt}%
\pgfpathmoveto{\pgfqpoint{2.135660in}{4.536132in}}%
\pgfpathlineto{\pgfqpoint{2.223396in}{4.536132in}}%
\pgfpathlineto{\pgfqpoint{2.223396in}{4.448396in}}%
\pgfpathlineto{\pgfqpoint{2.135660in}{4.448396in}}%
\pgfpathlineto{\pgfqpoint{2.135660in}{4.536132in}}%
\pgfusepath{stroke,fill}%
\end{pgfscope}%
\begin{pgfscope}%
\pgfpathrectangle{\pgfqpoint{0.380943in}{4.185189in}}{\pgfqpoint{4.650000in}{0.614151in}}%
\pgfusepath{clip}%
\pgfsetbuttcap%
\pgfsetroundjoin%
\definecolor{currentfill}{rgb}{0.968166,0.945882,0.748604}%
\pgfsetfillcolor{currentfill}%
\pgfsetlinewidth{0.250937pt}%
\definecolor{currentstroke}{rgb}{1.000000,1.000000,1.000000}%
\pgfsetstrokecolor{currentstroke}%
\pgfsetdash{}{0pt}%
\pgfpathmoveto{\pgfqpoint{2.223396in}{4.536132in}}%
\pgfpathlineto{\pgfqpoint{2.311132in}{4.536132in}}%
\pgfpathlineto{\pgfqpoint{2.311132in}{4.448396in}}%
\pgfpathlineto{\pgfqpoint{2.223396in}{4.448396in}}%
\pgfpathlineto{\pgfqpoint{2.223396in}{4.536132in}}%
\pgfusepath{stroke,fill}%
\end{pgfscope}%
\begin{pgfscope}%
\pgfpathrectangle{\pgfqpoint{0.380943in}{4.185189in}}{\pgfqpoint{4.650000in}{0.614151in}}%
\pgfusepath{clip}%
\pgfsetbuttcap%
\pgfsetroundjoin%
\definecolor{currentfill}{rgb}{0.965444,0.906113,0.711757}%
\pgfsetfillcolor{currentfill}%
\pgfsetlinewidth{0.250937pt}%
\definecolor{currentstroke}{rgb}{1.000000,1.000000,1.000000}%
\pgfsetstrokecolor{currentstroke}%
\pgfsetdash{}{0pt}%
\pgfpathmoveto{\pgfqpoint{2.311132in}{4.536132in}}%
\pgfpathlineto{\pgfqpoint{2.398868in}{4.536132in}}%
\pgfpathlineto{\pgfqpoint{2.398868in}{4.448396in}}%
\pgfpathlineto{\pgfqpoint{2.311132in}{4.448396in}}%
\pgfpathlineto{\pgfqpoint{2.311132in}{4.536132in}}%
\pgfusepath{stroke,fill}%
\end{pgfscope}%
\begin{pgfscope}%
\pgfpathrectangle{\pgfqpoint{0.380943in}{4.185189in}}{\pgfqpoint{4.650000in}{0.614151in}}%
\pgfusepath{clip}%
\pgfsetbuttcap%
\pgfsetroundjoin%
\definecolor{currentfill}{rgb}{0.992326,0.765229,0.614840}%
\pgfsetfillcolor{currentfill}%
\pgfsetlinewidth{0.250937pt}%
\definecolor{currentstroke}{rgb}{1.000000,1.000000,1.000000}%
\pgfsetstrokecolor{currentstroke}%
\pgfsetdash{}{0pt}%
\pgfpathmoveto{\pgfqpoint{2.398868in}{4.536132in}}%
\pgfpathlineto{\pgfqpoint{2.486603in}{4.536132in}}%
\pgfpathlineto{\pgfqpoint{2.486603in}{4.448396in}}%
\pgfpathlineto{\pgfqpoint{2.398868in}{4.448396in}}%
\pgfpathlineto{\pgfqpoint{2.398868in}{4.536132in}}%
\pgfusepath{stroke,fill}%
\end{pgfscope}%
\begin{pgfscope}%
\pgfpathrectangle{\pgfqpoint{0.380943in}{4.185189in}}{\pgfqpoint{4.650000in}{0.614151in}}%
\pgfusepath{clip}%
\pgfsetbuttcap%
\pgfsetroundjoin%
\definecolor{currentfill}{rgb}{0.998939,0.658962,0.556032}%
\pgfsetfillcolor{currentfill}%
\pgfsetlinewidth{0.250937pt}%
\definecolor{currentstroke}{rgb}{1.000000,1.000000,1.000000}%
\pgfsetstrokecolor{currentstroke}%
\pgfsetdash{}{0pt}%
\pgfpathmoveto{\pgfqpoint{2.486603in}{4.536132in}}%
\pgfpathlineto{\pgfqpoint{2.574339in}{4.536132in}}%
\pgfpathlineto{\pgfqpoint{2.574339in}{4.448396in}}%
\pgfpathlineto{\pgfqpoint{2.486603in}{4.448396in}}%
\pgfpathlineto{\pgfqpoint{2.486603in}{4.536132in}}%
\pgfusepath{stroke,fill}%
\end{pgfscope}%
\begin{pgfscope}%
\pgfpathrectangle{\pgfqpoint{0.380943in}{4.185189in}}{\pgfqpoint{4.650000in}{0.614151in}}%
\pgfusepath{clip}%
\pgfsetbuttcap%
\pgfsetroundjoin%
\definecolor{currentfill}{rgb}{0.979654,0.837186,0.669619}%
\pgfsetfillcolor{currentfill}%
\pgfsetlinewidth{0.250937pt}%
\definecolor{currentstroke}{rgb}{1.000000,1.000000,1.000000}%
\pgfsetstrokecolor{currentstroke}%
\pgfsetdash{}{0pt}%
\pgfpathmoveto{\pgfqpoint{2.574339in}{4.536132in}}%
\pgfpathlineto{\pgfqpoint{2.662075in}{4.536132in}}%
\pgfpathlineto{\pgfqpoint{2.662075in}{4.448396in}}%
\pgfpathlineto{\pgfqpoint{2.574339in}{4.448396in}}%
\pgfpathlineto{\pgfqpoint{2.574339in}{4.536132in}}%
\pgfusepath{stroke,fill}%
\end{pgfscope}%
\begin{pgfscope}%
\pgfpathrectangle{\pgfqpoint{0.380943in}{4.185189in}}{\pgfqpoint{4.650000in}{0.614151in}}%
\pgfusepath{clip}%
\pgfsetbuttcap%
\pgfsetroundjoin%
\definecolor{currentfill}{rgb}{0.992326,0.765229,0.614840}%
\pgfsetfillcolor{currentfill}%
\pgfsetlinewidth{0.250937pt}%
\definecolor{currentstroke}{rgb}{1.000000,1.000000,1.000000}%
\pgfsetstrokecolor{currentstroke}%
\pgfsetdash{}{0pt}%
\pgfpathmoveto{\pgfqpoint{2.662075in}{4.536132in}}%
\pgfpathlineto{\pgfqpoint{2.749811in}{4.536132in}}%
\pgfpathlineto{\pgfqpoint{2.749811in}{4.448396in}}%
\pgfpathlineto{\pgfqpoint{2.662075in}{4.448396in}}%
\pgfpathlineto{\pgfqpoint{2.662075in}{4.536132in}}%
\pgfusepath{stroke,fill}%
\end{pgfscope}%
\begin{pgfscope}%
\pgfpathrectangle{\pgfqpoint{0.380943in}{4.185189in}}{\pgfqpoint{4.650000in}{0.614151in}}%
\pgfusepath{clip}%
\pgfsetbuttcap%
\pgfsetroundjoin%
\definecolor{currentfill}{rgb}{0.979654,0.837186,0.669619}%
\pgfsetfillcolor{currentfill}%
\pgfsetlinewidth{0.250937pt}%
\definecolor{currentstroke}{rgb}{1.000000,1.000000,1.000000}%
\pgfsetstrokecolor{currentstroke}%
\pgfsetdash{}{0pt}%
\pgfpathmoveto{\pgfqpoint{2.749811in}{4.536132in}}%
\pgfpathlineto{\pgfqpoint{2.837547in}{4.536132in}}%
\pgfpathlineto{\pgfqpoint{2.837547in}{4.448396in}}%
\pgfpathlineto{\pgfqpoint{2.749811in}{4.448396in}}%
\pgfpathlineto{\pgfqpoint{2.749811in}{4.536132in}}%
\pgfusepath{stroke,fill}%
\end{pgfscope}%
\begin{pgfscope}%
\pgfpathrectangle{\pgfqpoint{0.380943in}{4.185189in}}{\pgfqpoint{4.650000in}{0.614151in}}%
\pgfusepath{clip}%
\pgfsetbuttcap%
\pgfsetroundjoin%
\definecolor{currentfill}{rgb}{0.992326,0.765229,0.614840}%
\pgfsetfillcolor{currentfill}%
\pgfsetlinewidth{0.250937pt}%
\definecolor{currentstroke}{rgb}{1.000000,1.000000,1.000000}%
\pgfsetstrokecolor{currentstroke}%
\pgfsetdash{}{0pt}%
\pgfpathmoveto{\pgfqpoint{2.837547in}{4.536132in}}%
\pgfpathlineto{\pgfqpoint{2.925283in}{4.536132in}}%
\pgfpathlineto{\pgfqpoint{2.925283in}{4.448396in}}%
\pgfpathlineto{\pgfqpoint{2.837547in}{4.448396in}}%
\pgfpathlineto{\pgfqpoint{2.837547in}{4.536132in}}%
\pgfusepath{stroke,fill}%
\end{pgfscope}%
\begin{pgfscope}%
\pgfpathrectangle{\pgfqpoint{0.380943in}{4.185189in}}{\pgfqpoint{4.650000in}{0.614151in}}%
\pgfusepath{clip}%
\pgfsetbuttcap%
\pgfsetroundjoin%
\definecolor{currentfill}{rgb}{0.972549,0.870588,0.692810}%
\pgfsetfillcolor{currentfill}%
\pgfsetlinewidth{0.250937pt}%
\definecolor{currentstroke}{rgb}{1.000000,1.000000,1.000000}%
\pgfsetstrokecolor{currentstroke}%
\pgfsetdash{}{0pt}%
\pgfpathmoveto{\pgfqpoint{2.925283in}{4.536132in}}%
\pgfpathlineto{\pgfqpoint{3.013019in}{4.536132in}}%
\pgfpathlineto{\pgfqpoint{3.013019in}{4.448396in}}%
\pgfpathlineto{\pgfqpoint{2.925283in}{4.448396in}}%
\pgfpathlineto{\pgfqpoint{2.925283in}{4.536132in}}%
\pgfusepath{stroke,fill}%
\end{pgfscope}%
\begin{pgfscope}%
\pgfpathrectangle{\pgfqpoint{0.380943in}{4.185189in}}{\pgfqpoint{4.650000in}{0.614151in}}%
\pgfusepath{clip}%
\pgfsetbuttcap%
\pgfsetroundjoin%
\definecolor{currentfill}{rgb}{0.979654,0.837186,0.669619}%
\pgfsetfillcolor{currentfill}%
\pgfsetlinewidth{0.250937pt}%
\definecolor{currentstroke}{rgb}{1.000000,1.000000,1.000000}%
\pgfsetstrokecolor{currentstroke}%
\pgfsetdash{}{0pt}%
\pgfpathmoveto{\pgfqpoint{3.013019in}{4.536132in}}%
\pgfpathlineto{\pgfqpoint{3.100754in}{4.536132in}}%
\pgfpathlineto{\pgfqpoint{3.100754in}{4.448396in}}%
\pgfpathlineto{\pgfqpoint{3.013019in}{4.448396in}}%
\pgfpathlineto{\pgfqpoint{3.013019in}{4.536132in}}%
\pgfusepath{stroke,fill}%
\end{pgfscope}%
\begin{pgfscope}%
\pgfpathrectangle{\pgfqpoint{0.380943in}{4.185189in}}{\pgfqpoint{4.650000in}{0.614151in}}%
\pgfusepath{clip}%
\pgfsetbuttcap%
\pgfsetroundjoin%
\definecolor{currentfill}{rgb}{0.996571,0.720538,0.589189}%
\pgfsetfillcolor{currentfill}%
\pgfsetlinewidth{0.250937pt}%
\definecolor{currentstroke}{rgb}{1.000000,1.000000,1.000000}%
\pgfsetstrokecolor{currentstroke}%
\pgfsetdash{}{0pt}%
\pgfpathmoveto{\pgfqpoint{3.100754in}{4.536132in}}%
\pgfpathlineto{\pgfqpoint{3.188490in}{4.536132in}}%
\pgfpathlineto{\pgfqpoint{3.188490in}{4.448396in}}%
\pgfpathlineto{\pgfqpoint{3.100754in}{4.448396in}}%
\pgfpathlineto{\pgfqpoint{3.100754in}{4.536132in}}%
\pgfusepath{stroke,fill}%
\end{pgfscope}%
\begin{pgfscope}%
\pgfpathrectangle{\pgfqpoint{0.380943in}{4.185189in}}{\pgfqpoint{4.650000in}{0.614151in}}%
\pgfusepath{clip}%
\pgfsetbuttcap%
\pgfsetroundjoin%
\definecolor{currentfill}{rgb}{0.968166,0.945882,0.748604}%
\pgfsetfillcolor{currentfill}%
\pgfsetlinewidth{0.250937pt}%
\definecolor{currentstroke}{rgb}{1.000000,1.000000,1.000000}%
\pgfsetstrokecolor{currentstroke}%
\pgfsetdash{}{0pt}%
\pgfpathmoveto{\pgfqpoint{3.188490in}{4.536132in}}%
\pgfpathlineto{\pgfqpoint{3.276226in}{4.536132in}}%
\pgfpathlineto{\pgfqpoint{3.276226in}{4.448396in}}%
\pgfpathlineto{\pgfqpoint{3.188490in}{4.448396in}}%
\pgfpathlineto{\pgfqpoint{3.188490in}{4.536132in}}%
\pgfusepath{stroke,fill}%
\end{pgfscope}%
\begin{pgfscope}%
\pgfpathrectangle{\pgfqpoint{0.380943in}{4.185189in}}{\pgfqpoint{4.650000in}{0.614151in}}%
\pgfusepath{clip}%
\pgfsetbuttcap%
\pgfsetroundjoin%
\definecolor{currentfill}{rgb}{0.968166,0.945882,0.748604}%
\pgfsetfillcolor{currentfill}%
\pgfsetlinewidth{0.250937pt}%
\definecolor{currentstroke}{rgb}{1.000000,1.000000,1.000000}%
\pgfsetstrokecolor{currentstroke}%
\pgfsetdash{}{0pt}%
\pgfpathmoveto{\pgfqpoint{3.276226in}{4.536132in}}%
\pgfpathlineto{\pgfqpoint{3.363962in}{4.536132in}}%
\pgfpathlineto{\pgfqpoint{3.363962in}{4.448396in}}%
\pgfpathlineto{\pgfqpoint{3.276226in}{4.448396in}}%
\pgfpathlineto{\pgfqpoint{3.276226in}{4.536132in}}%
\pgfusepath{stroke,fill}%
\end{pgfscope}%
\begin{pgfscope}%
\pgfpathrectangle{\pgfqpoint{0.380943in}{4.185189in}}{\pgfqpoint{4.650000in}{0.614151in}}%
\pgfusepath{clip}%
\pgfsetbuttcap%
\pgfsetroundjoin%
\definecolor{currentfill}{rgb}{0.972549,0.870588,0.692810}%
\pgfsetfillcolor{currentfill}%
\pgfsetlinewidth{0.250937pt}%
\definecolor{currentstroke}{rgb}{1.000000,1.000000,1.000000}%
\pgfsetstrokecolor{currentstroke}%
\pgfsetdash{}{0pt}%
\pgfpathmoveto{\pgfqpoint{3.363962in}{4.536132in}}%
\pgfpathlineto{\pgfqpoint{3.451698in}{4.536132in}}%
\pgfpathlineto{\pgfqpoint{3.451698in}{4.448396in}}%
\pgfpathlineto{\pgfqpoint{3.363962in}{4.448396in}}%
\pgfpathlineto{\pgfqpoint{3.363962in}{4.536132in}}%
\pgfusepath{stroke,fill}%
\end{pgfscope}%
\begin{pgfscope}%
\pgfpathrectangle{\pgfqpoint{0.380943in}{4.185189in}}{\pgfqpoint{4.650000in}{0.614151in}}%
\pgfusepath{clip}%
\pgfsetbuttcap%
\pgfsetroundjoin%
\definecolor{currentfill}{rgb}{0.968166,0.945882,0.748604}%
\pgfsetfillcolor{currentfill}%
\pgfsetlinewidth{0.250937pt}%
\definecolor{currentstroke}{rgb}{1.000000,1.000000,1.000000}%
\pgfsetstrokecolor{currentstroke}%
\pgfsetdash{}{0pt}%
\pgfpathmoveto{\pgfqpoint{3.451698in}{4.536132in}}%
\pgfpathlineto{\pgfqpoint{3.539434in}{4.536132in}}%
\pgfpathlineto{\pgfqpoint{3.539434in}{4.448396in}}%
\pgfpathlineto{\pgfqpoint{3.451698in}{4.448396in}}%
\pgfpathlineto{\pgfqpoint{3.451698in}{4.536132in}}%
\pgfusepath{stroke,fill}%
\end{pgfscope}%
\begin{pgfscope}%
\pgfpathrectangle{\pgfqpoint{0.380943in}{4.185189in}}{\pgfqpoint{4.650000in}{0.614151in}}%
\pgfusepath{clip}%
\pgfsetbuttcap%
\pgfsetroundjoin%
\definecolor{currentfill}{rgb}{0.965444,0.906113,0.711757}%
\pgfsetfillcolor{currentfill}%
\pgfsetlinewidth{0.250937pt}%
\definecolor{currentstroke}{rgb}{1.000000,1.000000,1.000000}%
\pgfsetstrokecolor{currentstroke}%
\pgfsetdash{}{0pt}%
\pgfpathmoveto{\pgfqpoint{3.539434in}{4.536132in}}%
\pgfpathlineto{\pgfqpoint{3.627169in}{4.536132in}}%
\pgfpathlineto{\pgfqpoint{3.627169in}{4.448396in}}%
\pgfpathlineto{\pgfqpoint{3.539434in}{4.448396in}}%
\pgfpathlineto{\pgfqpoint{3.539434in}{4.536132in}}%
\pgfusepath{stroke,fill}%
\end{pgfscope}%
\begin{pgfscope}%
\pgfpathrectangle{\pgfqpoint{0.380943in}{4.185189in}}{\pgfqpoint{4.650000in}{0.614151in}}%
\pgfusepath{clip}%
\pgfsetbuttcap%
\pgfsetroundjoin%
\definecolor{currentfill}{rgb}{0.962414,0.923552,0.722891}%
\pgfsetfillcolor{currentfill}%
\pgfsetlinewidth{0.250937pt}%
\definecolor{currentstroke}{rgb}{1.000000,1.000000,1.000000}%
\pgfsetstrokecolor{currentstroke}%
\pgfsetdash{}{0pt}%
\pgfpathmoveto{\pgfqpoint{3.627169in}{4.536132in}}%
\pgfpathlineto{\pgfqpoint{3.714905in}{4.536132in}}%
\pgfpathlineto{\pgfqpoint{3.714905in}{4.448396in}}%
\pgfpathlineto{\pgfqpoint{3.627169in}{4.448396in}}%
\pgfpathlineto{\pgfqpoint{3.627169in}{4.536132in}}%
\pgfusepath{stroke,fill}%
\end{pgfscope}%
\begin{pgfscope}%
\pgfpathrectangle{\pgfqpoint{0.380943in}{4.185189in}}{\pgfqpoint{4.650000in}{0.614151in}}%
\pgfusepath{clip}%
\pgfsetbuttcap%
\pgfsetroundjoin%
\definecolor{currentfill}{rgb}{0.979654,0.837186,0.669619}%
\pgfsetfillcolor{currentfill}%
\pgfsetlinewidth{0.250937pt}%
\definecolor{currentstroke}{rgb}{1.000000,1.000000,1.000000}%
\pgfsetstrokecolor{currentstroke}%
\pgfsetdash{}{0pt}%
\pgfpathmoveto{\pgfqpoint{3.714905in}{4.536132in}}%
\pgfpathlineto{\pgfqpoint{3.802641in}{4.536132in}}%
\pgfpathlineto{\pgfqpoint{3.802641in}{4.448396in}}%
\pgfpathlineto{\pgfqpoint{3.714905in}{4.448396in}}%
\pgfpathlineto{\pgfqpoint{3.714905in}{4.536132in}}%
\pgfusepath{stroke,fill}%
\end{pgfscope}%
\begin{pgfscope}%
\pgfpathrectangle{\pgfqpoint{0.380943in}{4.185189in}}{\pgfqpoint{4.650000in}{0.614151in}}%
\pgfusepath{clip}%
\pgfsetbuttcap%
\pgfsetroundjoin%
\definecolor{currentfill}{rgb}{0.972549,0.870588,0.692810}%
\pgfsetfillcolor{currentfill}%
\pgfsetlinewidth{0.250937pt}%
\definecolor{currentstroke}{rgb}{1.000000,1.000000,1.000000}%
\pgfsetstrokecolor{currentstroke}%
\pgfsetdash{}{0pt}%
\pgfpathmoveto{\pgfqpoint{3.802641in}{4.536132in}}%
\pgfpathlineto{\pgfqpoint{3.890377in}{4.536132in}}%
\pgfpathlineto{\pgfqpoint{3.890377in}{4.448396in}}%
\pgfpathlineto{\pgfqpoint{3.802641in}{4.448396in}}%
\pgfpathlineto{\pgfqpoint{3.802641in}{4.536132in}}%
\pgfusepath{stroke,fill}%
\end{pgfscope}%
\begin{pgfscope}%
\pgfpathrectangle{\pgfqpoint{0.380943in}{4.185189in}}{\pgfqpoint{4.650000in}{0.614151in}}%
\pgfusepath{clip}%
\pgfsetbuttcap%
\pgfsetroundjoin%
\definecolor{currentfill}{rgb}{0.979654,0.837186,0.669619}%
\pgfsetfillcolor{currentfill}%
\pgfsetlinewidth{0.250937pt}%
\definecolor{currentstroke}{rgb}{1.000000,1.000000,1.000000}%
\pgfsetstrokecolor{currentstroke}%
\pgfsetdash{}{0pt}%
\pgfpathmoveto{\pgfqpoint{3.890377in}{4.536132in}}%
\pgfpathlineto{\pgfqpoint{3.978113in}{4.536132in}}%
\pgfpathlineto{\pgfqpoint{3.978113in}{4.448396in}}%
\pgfpathlineto{\pgfqpoint{3.890377in}{4.448396in}}%
\pgfpathlineto{\pgfqpoint{3.890377in}{4.536132in}}%
\pgfusepath{stroke,fill}%
\end{pgfscope}%
\begin{pgfscope}%
\pgfpathrectangle{\pgfqpoint{0.380943in}{4.185189in}}{\pgfqpoint{4.650000in}{0.614151in}}%
\pgfusepath{clip}%
\pgfsetbuttcap%
\pgfsetroundjoin%
\definecolor{currentfill}{rgb}{1.000000,0.557862,0.511772}%
\pgfsetfillcolor{currentfill}%
\pgfsetlinewidth{0.250937pt}%
\definecolor{currentstroke}{rgb}{1.000000,1.000000,1.000000}%
\pgfsetstrokecolor{currentstroke}%
\pgfsetdash{}{0pt}%
\pgfpathmoveto{\pgfqpoint{3.978113in}{4.536132in}}%
\pgfpathlineto{\pgfqpoint{4.065849in}{4.536132in}}%
\pgfpathlineto{\pgfqpoint{4.065849in}{4.448396in}}%
\pgfpathlineto{\pgfqpoint{3.978113in}{4.448396in}}%
\pgfpathlineto{\pgfqpoint{3.978113in}{4.536132in}}%
\pgfusepath{stroke,fill}%
\end{pgfscope}%
\begin{pgfscope}%
\pgfpathrectangle{\pgfqpoint{0.380943in}{4.185189in}}{\pgfqpoint{4.650000in}{0.614151in}}%
\pgfusepath{clip}%
\pgfsetbuttcap%
\pgfsetroundjoin%
\definecolor{currentfill}{rgb}{0.992326,0.765229,0.614840}%
\pgfsetfillcolor{currentfill}%
\pgfsetlinewidth{0.250937pt}%
\definecolor{currentstroke}{rgb}{1.000000,1.000000,1.000000}%
\pgfsetstrokecolor{currentstroke}%
\pgfsetdash{}{0pt}%
\pgfpathmoveto{\pgfqpoint{4.065849in}{4.536132in}}%
\pgfpathlineto{\pgfqpoint{4.153585in}{4.536132in}}%
\pgfpathlineto{\pgfqpoint{4.153585in}{4.448396in}}%
\pgfpathlineto{\pgfqpoint{4.065849in}{4.448396in}}%
\pgfpathlineto{\pgfqpoint{4.065849in}{4.536132in}}%
\pgfusepath{stroke,fill}%
\end{pgfscope}%
\begin{pgfscope}%
\pgfpathrectangle{\pgfqpoint{0.380943in}{4.185189in}}{\pgfqpoint{4.650000in}{0.614151in}}%
\pgfusepath{clip}%
\pgfsetbuttcap%
\pgfsetroundjoin%
\definecolor{currentfill}{rgb}{0.996571,0.720538,0.589189}%
\pgfsetfillcolor{currentfill}%
\pgfsetlinewidth{0.250937pt}%
\definecolor{currentstroke}{rgb}{1.000000,1.000000,1.000000}%
\pgfsetstrokecolor{currentstroke}%
\pgfsetdash{}{0pt}%
\pgfpathmoveto{\pgfqpoint{4.153585in}{4.536132in}}%
\pgfpathlineto{\pgfqpoint{4.241320in}{4.536132in}}%
\pgfpathlineto{\pgfqpoint{4.241320in}{4.448396in}}%
\pgfpathlineto{\pgfqpoint{4.153585in}{4.448396in}}%
\pgfpathlineto{\pgfqpoint{4.153585in}{4.536132in}}%
\pgfusepath{stroke,fill}%
\end{pgfscope}%
\begin{pgfscope}%
\pgfpathrectangle{\pgfqpoint{0.380943in}{4.185189in}}{\pgfqpoint{4.650000in}{0.614151in}}%
\pgfusepath{clip}%
\pgfsetbuttcap%
\pgfsetroundjoin%
\definecolor{currentfill}{rgb}{0.998939,0.658962,0.556032}%
\pgfsetfillcolor{currentfill}%
\pgfsetlinewidth{0.250937pt}%
\definecolor{currentstroke}{rgb}{1.000000,1.000000,1.000000}%
\pgfsetstrokecolor{currentstroke}%
\pgfsetdash{}{0pt}%
\pgfpathmoveto{\pgfqpoint{4.241320in}{4.536132in}}%
\pgfpathlineto{\pgfqpoint{4.329056in}{4.536132in}}%
\pgfpathlineto{\pgfqpoint{4.329056in}{4.448396in}}%
\pgfpathlineto{\pgfqpoint{4.241320in}{4.448396in}}%
\pgfpathlineto{\pgfqpoint{4.241320in}{4.536132in}}%
\pgfusepath{stroke,fill}%
\end{pgfscope}%
\begin{pgfscope}%
\pgfpathrectangle{\pgfqpoint{0.380943in}{4.185189in}}{\pgfqpoint{4.650000in}{0.614151in}}%
\pgfusepath{clip}%
\pgfsetbuttcap%
\pgfsetroundjoin%
\definecolor{currentfill}{rgb}{0.979654,0.837186,0.669619}%
\pgfsetfillcolor{currentfill}%
\pgfsetlinewidth{0.250937pt}%
\definecolor{currentstroke}{rgb}{1.000000,1.000000,1.000000}%
\pgfsetstrokecolor{currentstroke}%
\pgfsetdash{}{0pt}%
\pgfpathmoveto{\pgfqpoint{4.329056in}{4.536132in}}%
\pgfpathlineto{\pgfqpoint{4.416792in}{4.536132in}}%
\pgfpathlineto{\pgfqpoint{4.416792in}{4.448396in}}%
\pgfpathlineto{\pgfqpoint{4.329056in}{4.448396in}}%
\pgfpathlineto{\pgfqpoint{4.329056in}{4.536132in}}%
\pgfusepath{stroke,fill}%
\end{pgfscope}%
\begin{pgfscope}%
\pgfpathrectangle{\pgfqpoint{0.380943in}{4.185189in}}{\pgfqpoint{4.650000in}{0.614151in}}%
\pgfusepath{clip}%
\pgfsetbuttcap%
\pgfsetroundjoin%
\definecolor{currentfill}{rgb}{0.979654,0.837186,0.669619}%
\pgfsetfillcolor{currentfill}%
\pgfsetlinewidth{0.250937pt}%
\definecolor{currentstroke}{rgb}{1.000000,1.000000,1.000000}%
\pgfsetstrokecolor{currentstroke}%
\pgfsetdash{}{0pt}%
\pgfpathmoveto{\pgfqpoint{4.416792in}{4.536132in}}%
\pgfpathlineto{\pgfqpoint{4.504528in}{4.536132in}}%
\pgfpathlineto{\pgfqpoint{4.504528in}{4.448396in}}%
\pgfpathlineto{\pgfqpoint{4.416792in}{4.448396in}}%
\pgfpathlineto{\pgfqpoint{4.416792in}{4.536132in}}%
\pgfusepath{stroke,fill}%
\end{pgfscope}%
\begin{pgfscope}%
\pgfpathrectangle{\pgfqpoint{0.380943in}{4.185189in}}{\pgfqpoint{4.650000in}{0.614151in}}%
\pgfusepath{clip}%
\pgfsetbuttcap%
\pgfsetroundjoin%
\definecolor{currentfill}{rgb}{0.986759,0.806398,0.641200}%
\pgfsetfillcolor{currentfill}%
\pgfsetlinewidth{0.250937pt}%
\definecolor{currentstroke}{rgb}{1.000000,1.000000,1.000000}%
\pgfsetstrokecolor{currentstroke}%
\pgfsetdash{}{0pt}%
\pgfpathmoveto{\pgfqpoint{4.504528in}{4.536132in}}%
\pgfpathlineto{\pgfqpoint{4.592264in}{4.536132in}}%
\pgfpathlineto{\pgfqpoint{4.592264in}{4.448396in}}%
\pgfpathlineto{\pgfqpoint{4.504528in}{4.448396in}}%
\pgfpathlineto{\pgfqpoint{4.504528in}{4.536132in}}%
\pgfusepath{stroke,fill}%
\end{pgfscope}%
\begin{pgfscope}%
\pgfpathrectangle{\pgfqpoint{0.380943in}{4.185189in}}{\pgfqpoint{4.650000in}{0.614151in}}%
\pgfusepath{clip}%
\pgfsetbuttcap%
\pgfsetroundjoin%
\definecolor{currentfill}{rgb}{0.800000,0.278431,0.278431}%
\pgfsetfillcolor{currentfill}%
\pgfsetlinewidth{0.250937pt}%
\definecolor{currentstroke}{rgb}{1.000000,1.000000,1.000000}%
\pgfsetstrokecolor{currentstroke}%
\pgfsetdash{}{0pt}%
\pgfpathmoveto{\pgfqpoint{4.592264in}{4.536132in}}%
\pgfpathlineto{\pgfqpoint{4.680000in}{4.536132in}}%
\pgfpathlineto{\pgfqpoint{4.680000in}{4.448396in}}%
\pgfpathlineto{\pgfqpoint{4.592264in}{4.448396in}}%
\pgfpathlineto{\pgfqpoint{4.592264in}{4.536132in}}%
\pgfusepath{stroke,fill}%
\end{pgfscope}%
\begin{pgfscope}%
\pgfpathrectangle{\pgfqpoint{0.380943in}{4.185189in}}{\pgfqpoint{4.650000in}{0.614151in}}%
\pgfusepath{clip}%
\pgfsetbuttcap%
\pgfsetroundjoin%
\definecolor{currentfill}{rgb}{0.986759,0.806398,0.641200}%
\pgfsetfillcolor{currentfill}%
\pgfsetlinewidth{0.250937pt}%
\definecolor{currentstroke}{rgb}{1.000000,1.000000,1.000000}%
\pgfsetstrokecolor{currentstroke}%
\pgfsetdash{}{0pt}%
\pgfpathmoveto{\pgfqpoint{4.680000in}{4.536132in}}%
\pgfpathlineto{\pgfqpoint{4.767736in}{4.536132in}}%
\pgfpathlineto{\pgfqpoint{4.767736in}{4.448396in}}%
\pgfpathlineto{\pgfqpoint{4.680000in}{4.448396in}}%
\pgfpathlineto{\pgfqpoint{4.680000in}{4.536132in}}%
\pgfusepath{stroke,fill}%
\end{pgfscope}%
\begin{pgfscope}%
\pgfpathrectangle{\pgfqpoint{0.380943in}{4.185189in}}{\pgfqpoint{4.650000in}{0.614151in}}%
\pgfusepath{clip}%
\pgfsetbuttcap%
\pgfsetroundjoin%
\definecolor{currentfill}{rgb}{0.965444,0.906113,0.711757}%
\pgfsetfillcolor{currentfill}%
\pgfsetlinewidth{0.250937pt}%
\definecolor{currentstroke}{rgb}{1.000000,1.000000,1.000000}%
\pgfsetstrokecolor{currentstroke}%
\pgfsetdash{}{0pt}%
\pgfpathmoveto{\pgfqpoint{4.767736in}{4.536132in}}%
\pgfpathlineto{\pgfqpoint{4.855471in}{4.536132in}}%
\pgfpathlineto{\pgfqpoint{4.855471in}{4.448396in}}%
\pgfpathlineto{\pgfqpoint{4.767736in}{4.448396in}}%
\pgfpathlineto{\pgfqpoint{4.767736in}{4.536132in}}%
\pgfusepath{stroke,fill}%
\end{pgfscope}%
\begin{pgfscope}%
\pgfpathrectangle{\pgfqpoint{0.380943in}{4.185189in}}{\pgfqpoint{4.650000in}{0.614151in}}%
\pgfusepath{clip}%
\pgfsetbuttcap%
\pgfsetroundjoin%
\definecolor{currentfill}{rgb}{0.968166,0.945882,0.748604}%
\pgfsetfillcolor{currentfill}%
\pgfsetlinewidth{0.250937pt}%
\definecolor{currentstroke}{rgb}{1.000000,1.000000,1.000000}%
\pgfsetstrokecolor{currentstroke}%
\pgfsetdash{}{0pt}%
\pgfpathmoveto{\pgfqpoint{4.855471in}{4.536132in}}%
\pgfpathlineto{\pgfqpoint{4.943207in}{4.536132in}}%
\pgfpathlineto{\pgfqpoint{4.943207in}{4.448396in}}%
\pgfpathlineto{\pgfqpoint{4.855471in}{4.448396in}}%
\pgfpathlineto{\pgfqpoint{4.855471in}{4.536132in}}%
\pgfusepath{stroke,fill}%
\end{pgfscope}%
\begin{pgfscope}%
\pgfpathrectangle{\pgfqpoint{0.380943in}{4.185189in}}{\pgfqpoint{4.650000in}{0.614151in}}%
\pgfusepath{clip}%
\pgfsetbuttcap%
\pgfsetroundjoin%
\pgfsetlinewidth{0.250937pt}%
\definecolor{currentstroke}{rgb}{1.000000,1.000000,1.000000}%
\pgfsetstrokecolor{currentstroke}%
\pgfsetdash{}{0pt}%
\pgfpathmoveto{\pgfqpoint{4.943207in}{4.536132in}}%
\pgfpathlineto{\pgfqpoint{5.030943in}{4.536132in}}%
\pgfpathlineto{\pgfqpoint{5.030943in}{4.448396in}}%
\pgfpathlineto{\pgfqpoint{4.943207in}{4.448396in}}%
\pgfpathlineto{\pgfqpoint{4.943207in}{4.536132in}}%
\pgfusepath{stroke}%
\end{pgfscope}%
\begin{pgfscope}%
\pgfpathrectangle{\pgfqpoint{0.380943in}{4.185189in}}{\pgfqpoint{4.650000in}{0.614151in}}%
\pgfusepath{clip}%
\pgfsetbuttcap%
\pgfsetroundjoin%
\definecolor{currentfill}{rgb}{0.992326,0.765229,0.614840}%
\pgfsetfillcolor{currentfill}%
\pgfsetlinewidth{0.250937pt}%
\definecolor{currentstroke}{rgb}{1.000000,1.000000,1.000000}%
\pgfsetstrokecolor{currentstroke}%
\pgfsetdash{}{0pt}%
\pgfpathmoveto{\pgfqpoint{0.380943in}{4.448396in}}%
\pgfpathlineto{\pgfqpoint{0.468679in}{4.448396in}}%
\pgfpathlineto{\pgfqpoint{0.468679in}{4.360661in}}%
\pgfpathlineto{\pgfqpoint{0.380943in}{4.360661in}}%
\pgfpathlineto{\pgfqpoint{0.380943in}{4.448396in}}%
\pgfusepath{stroke,fill}%
\end{pgfscope}%
\begin{pgfscope}%
\pgfpathrectangle{\pgfqpoint{0.380943in}{4.185189in}}{\pgfqpoint{4.650000in}{0.614151in}}%
\pgfusepath{clip}%
\pgfsetbuttcap%
\pgfsetroundjoin%
\definecolor{currentfill}{rgb}{0.992326,0.765229,0.614840}%
\pgfsetfillcolor{currentfill}%
\pgfsetlinewidth{0.250937pt}%
\definecolor{currentstroke}{rgb}{1.000000,1.000000,1.000000}%
\pgfsetstrokecolor{currentstroke}%
\pgfsetdash{}{0pt}%
\pgfpathmoveto{\pgfqpoint{0.468679in}{4.448396in}}%
\pgfpathlineto{\pgfqpoint{0.556415in}{4.448396in}}%
\pgfpathlineto{\pgfqpoint{0.556415in}{4.360661in}}%
\pgfpathlineto{\pgfqpoint{0.468679in}{4.360661in}}%
\pgfpathlineto{\pgfqpoint{0.468679in}{4.448396in}}%
\pgfusepath{stroke,fill}%
\end{pgfscope}%
\begin{pgfscope}%
\pgfpathrectangle{\pgfqpoint{0.380943in}{4.185189in}}{\pgfqpoint{4.650000in}{0.614151in}}%
\pgfusepath{clip}%
\pgfsetbuttcap%
\pgfsetroundjoin%
\definecolor{currentfill}{rgb}{0.996571,0.720538,0.589189}%
\pgfsetfillcolor{currentfill}%
\pgfsetlinewidth{0.250937pt}%
\definecolor{currentstroke}{rgb}{1.000000,1.000000,1.000000}%
\pgfsetstrokecolor{currentstroke}%
\pgfsetdash{}{0pt}%
\pgfpathmoveto{\pgfqpoint{0.556415in}{4.448396in}}%
\pgfpathlineto{\pgfqpoint{0.644151in}{4.448396in}}%
\pgfpathlineto{\pgfqpoint{0.644151in}{4.360661in}}%
\pgfpathlineto{\pgfqpoint{0.556415in}{4.360661in}}%
\pgfpathlineto{\pgfqpoint{0.556415in}{4.448396in}}%
\pgfusepath{stroke,fill}%
\end{pgfscope}%
\begin{pgfscope}%
\pgfpathrectangle{\pgfqpoint{0.380943in}{4.185189in}}{\pgfqpoint{4.650000in}{0.614151in}}%
\pgfusepath{clip}%
\pgfsetbuttcap%
\pgfsetroundjoin%
\definecolor{currentfill}{rgb}{0.992326,0.765229,0.614840}%
\pgfsetfillcolor{currentfill}%
\pgfsetlinewidth{0.250937pt}%
\definecolor{currentstroke}{rgb}{1.000000,1.000000,1.000000}%
\pgfsetstrokecolor{currentstroke}%
\pgfsetdash{}{0pt}%
\pgfpathmoveto{\pgfqpoint{0.644151in}{4.448396in}}%
\pgfpathlineto{\pgfqpoint{0.731886in}{4.448396in}}%
\pgfpathlineto{\pgfqpoint{0.731886in}{4.360661in}}%
\pgfpathlineto{\pgfqpoint{0.644151in}{4.360661in}}%
\pgfpathlineto{\pgfqpoint{0.644151in}{4.448396in}}%
\pgfusepath{stroke,fill}%
\end{pgfscope}%
\begin{pgfscope}%
\pgfpathrectangle{\pgfqpoint{0.380943in}{4.185189in}}{\pgfqpoint{4.650000in}{0.614151in}}%
\pgfusepath{clip}%
\pgfsetbuttcap%
\pgfsetroundjoin%
\definecolor{currentfill}{rgb}{0.972549,0.870588,0.692810}%
\pgfsetfillcolor{currentfill}%
\pgfsetlinewidth{0.250937pt}%
\definecolor{currentstroke}{rgb}{1.000000,1.000000,1.000000}%
\pgfsetstrokecolor{currentstroke}%
\pgfsetdash{}{0pt}%
\pgfpathmoveto{\pgfqpoint{0.731886in}{4.448396in}}%
\pgfpathlineto{\pgfqpoint{0.819622in}{4.448396in}}%
\pgfpathlineto{\pgfqpoint{0.819622in}{4.360661in}}%
\pgfpathlineto{\pgfqpoint{0.731886in}{4.360661in}}%
\pgfpathlineto{\pgfqpoint{0.731886in}{4.448396in}}%
\pgfusepath{stroke,fill}%
\end{pgfscope}%
\begin{pgfscope}%
\pgfpathrectangle{\pgfqpoint{0.380943in}{4.185189in}}{\pgfqpoint{4.650000in}{0.614151in}}%
\pgfusepath{clip}%
\pgfsetbuttcap%
\pgfsetroundjoin%
\definecolor{currentfill}{rgb}{0.922338,0.400769,0.400769}%
\pgfsetfillcolor{currentfill}%
\pgfsetlinewidth{0.250937pt}%
\definecolor{currentstroke}{rgb}{1.000000,1.000000,1.000000}%
\pgfsetstrokecolor{currentstroke}%
\pgfsetdash{}{0pt}%
\pgfpathmoveto{\pgfqpoint{0.819622in}{4.448396in}}%
\pgfpathlineto{\pgfqpoint{0.907358in}{4.448396in}}%
\pgfpathlineto{\pgfqpoint{0.907358in}{4.360661in}}%
\pgfpathlineto{\pgfqpoint{0.819622in}{4.360661in}}%
\pgfpathlineto{\pgfqpoint{0.819622in}{4.448396in}}%
\pgfusepath{stroke,fill}%
\end{pgfscope}%
\begin{pgfscope}%
\pgfpathrectangle{\pgfqpoint{0.380943in}{4.185189in}}{\pgfqpoint{4.650000in}{0.614151in}}%
\pgfusepath{clip}%
\pgfsetbuttcap%
\pgfsetroundjoin%
\definecolor{currentfill}{rgb}{0.986759,0.806398,0.641200}%
\pgfsetfillcolor{currentfill}%
\pgfsetlinewidth{0.250937pt}%
\definecolor{currentstroke}{rgb}{1.000000,1.000000,1.000000}%
\pgfsetstrokecolor{currentstroke}%
\pgfsetdash{}{0pt}%
\pgfpathmoveto{\pgfqpoint{0.907358in}{4.448396in}}%
\pgfpathlineto{\pgfqpoint{0.995094in}{4.448396in}}%
\pgfpathlineto{\pgfqpoint{0.995094in}{4.360661in}}%
\pgfpathlineto{\pgfqpoint{0.907358in}{4.360661in}}%
\pgfpathlineto{\pgfqpoint{0.907358in}{4.448396in}}%
\pgfusepath{stroke,fill}%
\end{pgfscope}%
\begin{pgfscope}%
\pgfpathrectangle{\pgfqpoint{0.380943in}{4.185189in}}{\pgfqpoint{4.650000in}{0.614151in}}%
\pgfusepath{clip}%
\pgfsetbuttcap%
\pgfsetroundjoin%
\definecolor{currentfill}{rgb}{0.992326,0.765229,0.614840}%
\pgfsetfillcolor{currentfill}%
\pgfsetlinewidth{0.250937pt}%
\definecolor{currentstroke}{rgb}{1.000000,1.000000,1.000000}%
\pgfsetstrokecolor{currentstroke}%
\pgfsetdash{}{0pt}%
\pgfpathmoveto{\pgfqpoint{0.995094in}{4.448396in}}%
\pgfpathlineto{\pgfqpoint{1.082830in}{4.448396in}}%
\pgfpathlineto{\pgfqpoint{1.082830in}{4.360661in}}%
\pgfpathlineto{\pgfqpoint{0.995094in}{4.360661in}}%
\pgfpathlineto{\pgfqpoint{0.995094in}{4.448396in}}%
\pgfusepath{stroke,fill}%
\end{pgfscope}%
\begin{pgfscope}%
\pgfpathrectangle{\pgfqpoint{0.380943in}{4.185189in}}{\pgfqpoint{4.650000in}{0.614151in}}%
\pgfusepath{clip}%
\pgfsetbuttcap%
\pgfsetroundjoin%
\definecolor{currentfill}{rgb}{0.979654,0.837186,0.669619}%
\pgfsetfillcolor{currentfill}%
\pgfsetlinewidth{0.250937pt}%
\definecolor{currentstroke}{rgb}{1.000000,1.000000,1.000000}%
\pgfsetstrokecolor{currentstroke}%
\pgfsetdash{}{0pt}%
\pgfpathmoveto{\pgfqpoint{1.082830in}{4.448396in}}%
\pgfpathlineto{\pgfqpoint{1.170566in}{4.448396in}}%
\pgfpathlineto{\pgfqpoint{1.170566in}{4.360661in}}%
\pgfpathlineto{\pgfqpoint{1.082830in}{4.360661in}}%
\pgfpathlineto{\pgfqpoint{1.082830in}{4.448396in}}%
\pgfusepath{stroke,fill}%
\end{pgfscope}%
\begin{pgfscope}%
\pgfpathrectangle{\pgfqpoint{0.380943in}{4.185189in}}{\pgfqpoint{4.650000in}{0.614151in}}%
\pgfusepath{clip}%
\pgfsetbuttcap%
\pgfsetroundjoin%
\definecolor{currentfill}{rgb}{0.996571,0.720538,0.589189}%
\pgfsetfillcolor{currentfill}%
\pgfsetlinewidth{0.250937pt}%
\definecolor{currentstroke}{rgb}{1.000000,1.000000,1.000000}%
\pgfsetstrokecolor{currentstroke}%
\pgfsetdash{}{0pt}%
\pgfpathmoveto{\pgfqpoint{1.170566in}{4.448396in}}%
\pgfpathlineto{\pgfqpoint{1.258302in}{4.448396in}}%
\pgfpathlineto{\pgfqpoint{1.258302in}{4.360661in}}%
\pgfpathlineto{\pgfqpoint{1.170566in}{4.360661in}}%
\pgfpathlineto{\pgfqpoint{1.170566in}{4.448396in}}%
\pgfusepath{stroke,fill}%
\end{pgfscope}%
\begin{pgfscope}%
\pgfpathrectangle{\pgfqpoint{0.380943in}{4.185189in}}{\pgfqpoint{4.650000in}{0.614151in}}%
\pgfusepath{clip}%
\pgfsetbuttcap%
\pgfsetroundjoin%
\definecolor{currentfill}{rgb}{0.965444,0.906113,0.711757}%
\pgfsetfillcolor{currentfill}%
\pgfsetlinewidth{0.250937pt}%
\definecolor{currentstroke}{rgb}{1.000000,1.000000,1.000000}%
\pgfsetstrokecolor{currentstroke}%
\pgfsetdash{}{0pt}%
\pgfpathmoveto{\pgfqpoint{1.258302in}{4.448396in}}%
\pgfpathlineto{\pgfqpoint{1.346037in}{4.448396in}}%
\pgfpathlineto{\pgfqpoint{1.346037in}{4.360661in}}%
\pgfpathlineto{\pgfqpoint{1.258302in}{4.360661in}}%
\pgfpathlineto{\pgfqpoint{1.258302in}{4.448396in}}%
\pgfusepath{stroke,fill}%
\end{pgfscope}%
\begin{pgfscope}%
\pgfpathrectangle{\pgfqpoint{0.380943in}{4.185189in}}{\pgfqpoint{4.650000in}{0.614151in}}%
\pgfusepath{clip}%
\pgfsetbuttcap%
\pgfsetroundjoin%
\definecolor{currentfill}{rgb}{0.979654,0.837186,0.669619}%
\pgfsetfillcolor{currentfill}%
\pgfsetlinewidth{0.250937pt}%
\definecolor{currentstroke}{rgb}{1.000000,1.000000,1.000000}%
\pgfsetstrokecolor{currentstroke}%
\pgfsetdash{}{0pt}%
\pgfpathmoveto{\pgfqpoint{1.346037in}{4.448396in}}%
\pgfpathlineto{\pgfqpoint{1.433773in}{4.448396in}}%
\pgfpathlineto{\pgfqpoint{1.433773in}{4.360661in}}%
\pgfpathlineto{\pgfqpoint{1.346037in}{4.360661in}}%
\pgfpathlineto{\pgfqpoint{1.346037in}{4.448396in}}%
\pgfusepath{stroke,fill}%
\end{pgfscope}%
\begin{pgfscope}%
\pgfpathrectangle{\pgfqpoint{0.380943in}{4.185189in}}{\pgfqpoint{4.650000in}{0.614151in}}%
\pgfusepath{clip}%
\pgfsetbuttcap%
\pgfsetroundjoin%
\definecolor{currentfill}{rgb}{0.986759,0.806398,0.641200}%
\pgfsetfillcolor{currentfill}%
\pgfsetlinewidth{0.250937pt}%
\definecolor{currentstroke}{rgb}{1.000000,1.000000,1.000000}%
\pgfsetstrokecolor{currentstroke}%
\pgfsetdash{}{0pt}%
\pgfpathmoveto{\pgfqpoint{1.433773in}{4.448396in}}%
\pgfpathlineto{\pgfqpoint{1.521509in}{4.448396in}}%
\pgfpathlineto{\pgfqpoint{1.521509in}{4.360661in}}%
\pgfpathlineto{\pgfqpoint{1.433773in}{4.360661in}}%
\pgfpathlineto{\pgfqpoint{1.433773in}{4.448396in}}%
\pgfusepath{stroke,fill}%
\end{pgfscope}%
\begin{pgfscope}%
\pgfpathrectangle{\pgfqpoint{0.380943in}{4.185189in}}{\pgfqpoint{4.650000in}{0.614151in}}%
\pgfusepath{clip}%
\pgfsetbuttcap%
\pgfsetroundjoin%
\definecolor{currentfill}{rgb}{0.986759,0.806398,0.641200}%
\pgfsetfillcolor{currentfill}%
\pgfsetlinewidth{0.250937pt}%
\definecolor{currentstroke}{rgb}{1.000000,1.000000,1.000000}%
\pgfsetstrokecolor{currentstroke}%
\pgfsetdash{}{0pt}%
\pgfpathmoveto{\pgfqpoint{1.521509in}{4.448396in}}%
\pgfpathlineto{\pgfqpoint{1.609245in}{4.448396in}}%
\pgfpathlineto{\pgfqpoint{1.609245in}{4.360661in}}%
\pgfpathlineto{\pgfqpoint{1.521509in}{4.360661in}}%
\pgfpathlineto{\pgfqpoint{1.521509in}{4.448396in}}%
\pgfusepath{stroke,fill}%
\end{pgfscope}%
\begin{pgfscope}%
\pgfpathrectangle{\pgfqpoint{0.380943in}{4.185189in}}{\pgfqpoint{4.650000in}{0.614151in}}%
\pgfusepath{clip}%
\pgfsetbuttcap%
\pgfsetroundjoin%
\definecolor{currentfill}{rgb}{0.972549,0.870588,0.692810}%
\pgfsetfillcolor{currentfill}%
\pgfsetlinewidth{0.250937pt}%
\definecolor{currentstroke}{rgb}{1.000000,1.000000,1.000000}%
\pgfsetstrokecolor{currentstroke}%
\pgfsetdash{}{0pt}%
\pgfpathmoveto{\pgfqpoint{1.609245in}{4.448396in}}%
\pgfpathlineto{\pgfqpoint{1.696981in}{4.448396in}}%
\pgfpathlineto{\pgfqpoint{1.696981in}{4.360661in}}%
\pgfpathlineto{\pgfqpoint{1.609245in}{4.360661in}}%
\pgfpathlineto{\pgfqpoint{1.609245in}{4.448396in}}%
\pgfusepath{stroke,fill}%
\end{pgfscope}%
\begin{pgfscope}%
\pgfpathrectangle{\pgfqpoint{0.380943in}{4.185189in}}{\pgfqpoint{4.650000in}{0.614151in}}%
\pgfusepath{clip}%
\pgfsetbuttcap%
\pgfsetroundjoin%
\definecolor{currentfill}{rgb}{0.996571,0.720538,0.589189}%
\pgfsetfillcolor{currentfill}%
\pgfsetlinewidth{0.250937pt}%
\definecolor{currentstroke}{rgb}{1.000000,1.000000,1.000000}%
\pgfsetstrokecolor{currentstroke}%
\pgfsetdash{}{0pt}%
\pgfpathmoveto{\pgfqpoint{1.696981in}{4.448396in}}%
\pgfpathlineto{\pgfqpoint{1.784717in}{4.448396in}}%
\pgfpathlineto{\pgfqpoint{1.784717in}{4.360661in}}%
\pgfpathlineto{\pgfqpoint{1.696981in}{4.360661in}}%
\pgfpathlineto{\pgfqpoint{1.696981in}{4.448396in}}%
\pgfusepath{stroke,fill}%
\end{pgfscope}%
\begin{pgfscope}%
\pgfpathrectangle{\pgfqpoint{0.380943in}{4.185189in}}{\pgfqpoint{4.650000in}{0.614151in}}%
\pgfusepath{clip}%
\pgfsetbuttcap%
\pgfsetroundjoin%
\definecolor{currentfill}{rgb}{1.000000,0.605229,0.530719}%
\pgfsetfillcolor{currentfill}%
\pgfsetlinewidth{0.250937pt}%
\definecolor{currentstroke}{rgb}{1.000000,1.000000,1.000000}%
\pgfsetstrokecolor{currentstroke}%
\pgfsetdash{}{0pt}%
\pgfpathmoveto{\pgfqpoint{1.784717in}{4.448396in}}%
\pgfpathlineto{\pgfqpoint{1.872452in}{4.448396in}}%
\pgfpathlineto{\pgfqpoint{1.872452in}{4.360661in}}%
\pgfpathlineto{\pgfqpoint{1.784717in}{4.360661in}}%
\pgfpathlineto{\pgfqpoint{1.784717in}{4.448396in}}%
\pgfusepath{stroke,fill}%
\end{pgfscope}%
\begin{pgfscope}%
\pgfpathrectangle{\pgfqpoint{0.380943in}{4.185189in}}{\pgfqpoint{4.650000in}{0.614151in}}%
\pgfusepath{clip}%
\pgfsetbuttcap%
\pgfsetroundjoin%
\definecolor{currentfill}{rgb}{0.979654,0.837186,0.669619}%
\pgfsetfillcolor{currentfill}%
\pgfsetlinewidth{0.250937pt}%
\definecolor{currentstroke}{rgb}{1.000000,1.000000,1.000000}%
\pgfsetstrokecolor{currentstroke}%
\pgfsetdash{}{0pt}%
\pgfpathmoveto{\pgfqpoint{1.872452in}{4.448396in}}%
\pgfpathlineto{\pgfqpoint{1.960188in}{4.448396in}}%
\pgfpathlineto{\pgfqpoint{1.960188in}{4.360661in}}%
\pgfpathlineto{\pgfqpoint{1.872452in}{4.360661in}}%
\pgfpathlineto{\pgfqpoint{1.872452in}{4.448396in}}%
\pgfusepath{stroke,fill}%
\end{pgfscope}%
\begin{pgfscope}%
\pgfpathrectangle{\pgfqpoint{0.380943in}{4.185189in}}{\pgfqpoint{4.650000in}{0.614151in}}%
\pgfusepath{clip}%
\pgfsetbuttcap%
\pgfsetroundjoin%
\definecolor{currentfill}{rgb}{0.986759,0.806398,0.641200}%
\pgfsetfillcolor{currentfill}%
\pgfsetlinewidth{0.250937pt}%
\definecolor{currentstroke}{rgb}{1.000000,1.000000,1.000000}%
\pgfsetstrokecolor{currentstroke}%
\pgfsetdash{}{0pt}%
\pgfpathmoveto{\pgfqpoint{1.960188in}{4.448396in}}%
\pgfpathlineto{\pgfqpoint{2.047924in}{4.448396in}}%
\pgfpathlineto{\pgfqpoint{2.047924in}{4.360661in}}%
\pgfpathlineto{\pgfqpoint{1.960188in}{4.360661in}}%
\pgfpathlineto{\pgfqpoint{1.960188in}{4.448396in}}%
\pgfusepath{stroke,fill}%
\end{pgfscope}%
\begin{pgfscope}%
\pgfpathrectangle{\pgfqpoint{0.380943in}{4.185189in}}{\pgfqpoint{4.650000in}{0.614151in}}%
\pgfusepath{clip}%
\pgfsetbuttcap%
\pgfsetroundjoin%
\definecolor{currentfill}{rgb}{0.996571,0.720538,0.589189}%
\pgfsetfillcolor{currentfill}%
\pgfsetlinewidth{0.250937pt}%
\definecolor{currentstroke}{rgb}{1.000000,1.000000,1.000000}%
\pgfsetstrokecolor{currentstroke}%
\pgfsetdash{}{0pt}%
\pgfpathmoveto{\pgfqpoint{2.047924in}{4.448396in}}%
\pgfpathlineto{\pgfqpoint{2.135660in}{4.448396in}}%
\pgfpathlineto{\pgfqpoint{2.135660in}{4.360661in}}%
\pgfpathlineto{\pgfqpoint{2.047924in}{4.360661in}}%
\pgfpathlineto{\pgfqpoint{2.047924in}{4.448396in}}%
\pgfusepath{stroke,fill}%
\end{pgfscope}%
\begin{pgfscope}%
\pgfpathrectangle{\pgfqpoint{0.380943in}{4.185189in}}{\pgfqpoint{4.650000in}{0.614151in}}%
\pgfusepath{clip}%
\pgfsetbuttcap%
\pgfsetroundjoin%
\definecolor{currentfill}{rgb}{0.962414,0.923552,0.722891}%
\pgfsetfillcolor{currentfill}%
\pgfsetlinewidth{0.250937pt}%
\definecolor{currentstroke}{rgb}{1.000000,1.000000,1.000000}%
\pgfsetstrokecolor{currentstroke}%
\pgfsetdash{}{0pt}%
\pgfpathmoveto{\pgfqpoint{2.135660in}{4.448396in}}%
\pgfpathlineto{\pgfqpoint{2.223396in}{4.448396in}}%
\pgfpathlineto{\pgfqpoint{2.223396in}{4.360661in}}%
\pgfpathlineto{\pgfqpoint{2.135660in}{4.360661in}}%
\pgfpathlineto{\pgfqpoint{2.135660in}{4.448396in}}%
\pgfusepath{stroke,fill}%
\end{pgfscope}%
\begin{pgfscope}%
\pgfpathrectangle{\pgfqpoint{0.380943in}{4.185189in}}{\pgfqpoint{4.650000in}{0.614151in}}%
\pgfusepath{clip}%
\pgfsetbuttcap%
\pgfsetroundjoin%
\definecolor{currentfill}{rgb}{0.972549,0.870588,0.692810}%
\pgfsetfillcolor{currentfill}%
\pgfsetlinewidth{0.250937pt}%
\definecolor{currentstroke}{rgb}{1.000000,1.000000,1.000000}%
\pgfsetstrokecolor{currentstroke}%
\pgfsetdash{}{0pt}%
\pgfpathmoveto{\pgfqpoint{2.223396in}{4.448396in}}%
\pgfpathlineto{\pgfqpoint{2.311132in}{4.448396in}}%
\pgfpathlineto{\pgfqpoint{2.311132in}{4.360661in}}%
\pgfpathlineto{\pgfqpoint{2.223396in}{4.360661in}}%
\pgfpathlineto{\pgfqpoint{2.223396in}{4.448396in}}%
\pgfusepath{stroke,fill}%
\end{pgfscope}%
\begin{pgfscope}%
\pgfpathrectangle{\pgfqpoint{0.380943in}{4.185189in}}{\pgfqpoint{4.650000in}{0.614151in}}%
\pgfusepath{clip}%
\pgfsetbuttcap%
\pgfsetroundjoin%
\definecolor{currentfill}{rgb}{0.986759,0.806398,0.641200}%
\pgfsetfillcolor{currentfill}%
\pgfsetlinewidth{0.250937pt}%
\definecolor{currentstroke}{rgb}{1.000000,1.000000,1.000000}%
\pgfsetstrokecolor{currentstroke}%
\pgfsetdash{}{0pt}%
\pgfpathmoveto{\pgfqpoint{2.311132in}{4.448396in}}%
\pgfpathlineto{\pgfqpoint{2.398868in}{4.448396in}}%
\pgfpathlineto{\pgfqpoint{2.398868in}{4.360661in}}%
\pgfpathlineto{\pgfqpoint{2.311132in}{4.360661in}}%
\pgfpathlineto{\pgfqpoint{2.311132in}{4.448396in}}%
\pgfusepath{stroke,fill}%
\end{pgfscope}%
\begin{pgfscope}%
\pgfpathrectangle{\pgfqpoint{0.380943in}{4.185189in}}{\pgfqpoint{4.650000in}{0.614151in}}%
\pgfusepath{clip}%
\pgfsetbuttcap%
\pgfsetroundjoin%
\definecolor{currentfill}{rgb}{0.965444,0.906113,0.711757}%
\pgfsetfillcolor{currentfill}%
\pgfsetlinewidth{0.250937pt}%
\definecolor{currentstroke}{rgb}{1.000000,1.000000,1.000000}%
\pgfsetstrokecolor{currentstroke}%
\pgfsetdash{}{0pt}%
\pgfpathmoveto{\pgfqpoint{2.398868in}{4.448396in}}%
\pgfpathlineto{\pgfqpoint{2.486603in}{4.448396in}}%
\pgfpathlineto{\pgfqpoint{2.486603in}{4.360661in}}%
\pgfpathlineto{\pgfqpoint{2.398868in}{4.360661in}}%
\pgfpathlineto{\pgfqpoint{2.398868in}{4.448396in}}%
\pgfusepath{stroke,fill}%
\end{pgfscope}%
\begin{pgfscope}%
\pgfpathrectangle{\pgfqpoint{0.380943in}{4.185189in}}{\pgfqpoint{4.650000in}{0.614151in}}%
\pgfusepath{clip}%
\pgfsetbuttcap%
\pgfsetroundjoin%
\definecolor{currentfill}{rgb}{0.979654,0.837186,0.669619}%
\pgfsetfillcolor{currentfill}%
\pgfsetlinewidth{0.250937pt}%
\definecolor{currentstroke}{rgb}{1.000000,1.000000,1.000000}%
\pgfsetstrokecolor{currentstroke}%
\pgfsetdash{}{0pt}%
\pgfpathmoveto{\pgfqpoint{2.486603in}{4.448396in}}%
\pgfpathlineto{\pgfqpoint{2.574339in}{4.448396in}}%
\pgfpathlineto{\pgfqpoint{2.574339in}{4.360661in}}%
\pgfpathlineto{\pgfqpoint{2.486603in}{4.360661in}}%
\pgfpathlineto{\pgfqpoint{2.486603in}{4.448396in}}%
\pgfusepath{stroke,fill}%
\end{pgfscope}%
\begin{pgfscope}%
\pgfpathrectangle{\pgfqpoint{0.380943in}{4.185189in}}{\pgfqpoint{4.650000in}{0.614151in}}%
\pgfusepath{clip}%
\pgfsetbuttcap%
\pgfsetroundjoin%
\definecolor{currentfill}{rgb}{0.972549,0.870588,0.692810}%
\pgfsetfillcolor{currentfill}%
\pgfsetlinewidth{0.250937pt}%
\definecolor{currentstroke}{rgb}{1.000000,1.000000,1.000000}%
\pgfsetstrokecolor{currentstroke}%
\pgfsetdash{}{0pt}%
\pgfpathmoveto{\pgfqpoint{2.574339in}{4.448396in}}%
\pgfpathlineto{\pgfqpoint{2.662075in}{4.448396in}}%
\pgfpathlineto{\pgfqpoint{2.662075in}{4.360661in}}%
\pgfpathlineto{\pgfqpoint{2.574339in}{4.360661in}}%
\pgfpathlineto{\pgfqpoint{2.574339in}{4.448396in}}%
\pgfusepath{stroke,fill}%
\end{pgfscope}%
\begin{pgfscope}%
\pgfpathrectangle{\pgfqpoint{0.380943in}{4.185189in}}{\pgfqpoint{4.650000in}{0.614151in}}%
\pgfusepath{clip}%
\pgfsetbuttcap%
\pgfsetroundjoin%
\definecolor{currentfill}{rgb}{0.962414,0.923552,0.722891}%
\pgfsetfillcolor{currentfill}%
\pgfsetlinewidth{0.250937pt}%
\definecolor{currentstroke}{rgb}{1.000000,1.000000,1.000000}%
\pgfsetstrokecolor{currentstroke}%
\pgfsetdash{}{0pt}%
\pgfpathmoveto{\pgfqpoint{2.662075in}{4.448396in}}%
\pgfpathlineto{\pgfqpoint{2.749811in}{4.448396in}}%
\pgfpathlineto{\pgfqpoint{2.749811in}{4.360661in}}%
\pgfpathlineto{\pgfqpoint{2.662075in}{4.360661in}}%
\pgfpathlineto{\pgfqpoint{2.662075in}{4.448396in}}%
\pgfusepath{stroke,fill}%
\end{pgfscope}%
\begin{pgfscope}%
\pgfpathrectangle{\pgfqpoint{0.380943in}{4.185189in}}{\pgfqpoint{4.650000in}{0.614151in}}%
\pgfusepath{clip}%
\pgfsetbuttcap%
\pgfsetroundjoin%
\definecolor{currentfill}{rgb}{0.996571,0.720538,0.589189}%
\pgfsetfillcolor{currentfill}%
\pgfsetlinewidth{0.250937pt}%
\definecolor{currentstroke}{rgb}{1.000000,1.000000,1.000000}%
\pgfsetstrokecolor{currentstroke}%
\pgfsetdash{}{0pt}%
\pgfpathmoveto{\pgfqpoint{2.749811in}{4.448396in}}%
\pgfpathlineto{\pgfqpoint{2.837547in}{4.448396in}}%
\pgfpathlineto{\pgfqpoint{2.837547in}{4.360661in}}%
\pgfpathlineto{\pgfqpoint{2.749811in}{4.360661in}}%
\pgfpathlineto{\pgfqpoint{2.749811in}{4.448396in}}%
\pgfusepath{stroke,fill}%
\end{pgfscope}%
\begin{pgfscope}%
\pgfpathrectangle{\pgfqpoint{0.380943in}{4.185189in}}{\pgfqpoint{4.650000in}{0.614151in}}%
\pgfusepath{clip}%
\pgfsetbuttcap%
\pgfsetroundjoin%
\definecolor{currentfill}{rgb}{0.962414,0.923552,0.722891}%
\pgfsetfillcolor{currentfill}%
\pgfsetlinewidth{0.250937pt}%
\definecolor{currentstroke}{rgb}{1.000000,1.000000,1.000000}%
\pgfsetstrokecolor{currentstroke}%
\pgfsetdash{}{0pt}%
\pgfpathmoveto{\pgfqpoint{2.837547in}{4.448396in}}%
\pgfpathlineto{\pgfqpoint{2.925283in}{4.448396in}}%
\pgfpathlineto{\pgfqpoint{2.925283in}{4.360661in}}%
\pgfpathlineto{\pgfqpoint{2.837547in}{4.360661in}}%
\pgfpathlineto{\pgfqpoint{2.837547in}{4.448396in}}%
\pgfusepath{stroke,fill}%
\end{pgfscope}%
\begin{pgfscope}%
\pgfpathrectangle{\pgfqpoint{0.380943in}{4.185189in}}{\pgfqpoint{4.650000in}{0.614151in}}%
\pgfusepath{clip}%
\pgfsetbuttcap%
\pgfsetroundjoin%
\definecolor{currentfill}{rgb}{0.965444,0.906113,0.711757}%
\pgfsetfillcolor{currentfill}%
\pgfsetlinewidth{0.250937pt}%
\definecolor{currentstroke}{rgb}{1.000000,1.000000,1.000000}%
\pgfsetstrokecolor{currentstroke}%
\pgfsetdash{}{0pt}%
\pgfpathmoveto{\pgfqpoint{2.925283in}{4.448396in}}%
\pgfpathlineto{\pgfqpoint{3.013019in}{4.448396in}}%
\pgfpathlineto{\pgfqpoint{3.013019in}{4.360661in}}%
\pgfpathlineto{\pgfqpoint{2.925283in}{4.360661in}}%
\pgfpathlineto{\pgfqpoint{2.925283in}{4.448396in}}%
\pgfusepath{stroke,fill}%
\end{pgfscope}%
\begin{pgfscope}%
\pgfpathrectangle{\pgfqpoint{0.380943in}{4.185189in}}{\pgfqpoint{4.650000in}{0.614151in}}%
\pgfusepath{clip}%
\pgfsetbuttcap%
\pgfsetroundjoin%
\definecolor{currentfill}{rgb}{0.965444,0.906113,0.711757}%
\pgfsetfillcolor{currentfill}%
\pgfsetlinewidth{0.250937pt}%
\definecolor{currentstroke}{rgb}{1.000000,1.000000,1.000000}%
\pgfsetstrokecolor{currentstroke}%
\pgfsetdash{}{0pt}%
\pgfpathmoveto{\pgfqpoint{3.013019in}{4.448396in}}%
\pgfpathlineto{\pgfqpoint{3.100754in}{4.448396in}}%
\pgfpathlineto{\pgfqpoint{3.100754in}{4.360661in}}%
\pgfpathlineto{\pgfqpoint{3.013019in}{4.360661in}}%
\pgfpathlineto{\pgfqpoint{3.013019in}{4.448396in}}%
\pgfusepath{stroke,fill}%
\end{pgfscope}%
\begin{pgfscope}%
\pgfpathrectangle{\pgfqpoint{0.380943in}{4.185189in}}{\pgfqpoint{4.650000in}{0.614151in}}%
\pgfusepath{clip}%
\pgfsetbuttcap%
\pgfsetroundjoin%
\definecolor{currentfill}{rgb}{0.996571,0.720538,0.589189}%
\pgfsetfillcolor{currentfill}%
\pgfsetlinewidth{0.250937pt}%
\definecolor{currentstroke}{rgb}{1.000000,1.000000,1.000000}%
\pgfsetstrokecolor{currentstroke}%
\pgfsetdash{}{0pt}%
\pgfpathmoveto{\pgfqpoint{3.100754in}{4.448396in}}%
\pgfpathlineto{\pgfqpoint{3.188490in}{4.448396in}}%
\pgfpathlineto{\pgfqpoint{3.188490in}{4.360661in}}%
\pgfpathlineto{\pgfqpoint{3.100754in}{4.360661in}}%
\pgfpathlineto{\pgfqpoint{3.100754in}{4.448396in}}%
\pgfusepath{stroke,fill}%
\end{pgfscope}%
\begin{pgfscope}%
\pgfpathrectangle{\pgfqpoint{0.380943in}{4.185189in}}{\pgfqpoint{4.650000in}{0.614151in}}%
\pgfusepath{clip}%
\pgfsetbuttcap%
\pgfsetroundjoin%
\definecolor{currentfill}{rgb}{0.979654,0.837186,0.669619}%
\pgfsetfillcolor{currentfill}%
\pgfsetlinewidth{0.250937pt}%
\definecolor{currentstroke}{rgb}{1.000000,1.000000,1.000000}%
\pgfsetstrokecolor{currentstroke}%
\pgfsetdash{}{0pt}%
\pgfpathmoveto{\pgfqpoint{3.188490in}{4.448396in}}%
\pgfpathlineto{\pgfqpoint{3.276226in}{4.448396in}}%
\pgfpathlineto{\pgfqpoint{3.276226in}{4.360661in}}%
\pgfpathlineto{\pgfqpoint{3.188490in}{4.360661in}}%
\pgfpathlineto{\pgfqpoint{3.188490in}{4.448396in}}%
\pgfusepath{stroke,fill}%
\end{pgfscope}%
\begin{pgfscope}%
\pgfpathrectangle{\pgfqpoint{0.380943in}{4.185189in}}{\pgfqpoint{4.650000in}{0.614151in}}%
\pgfusepath{clip}%
\pgfsetbuttcap%
\pgfsetroundjoin%
\definecolor{currentfill}{rgb}{0.979654,0.837186,0.669619}%
\pgfsetfillcolor{currentfill}%
\pgfsetlinewidth{0.250937pt}%
\definecolor{currentstroke}{rgb}{1.000000,1.000000,1.000000}%
\pgfsetstrokecolor{currentstroke}%
\pgfsetdash{}{0pt}%
\pgfpathmoveto{\pgfqpoint{3.276226in}{4.448396in}}%
\pgfpathlineto{\pgfqpoint{3.363962in}{4.448396in}}%
\pgfpathlineto{\pgfqpoint{3.363962in}{4.360661in}}%
\pgfpathlineto{\pgfqpoint{3.276226in}{4.360661in}}%
\pgfpathlineto{\pgfqpoint{3.276226in}{4.448396in}}%
\pgfusepath{stroke,fill}%
\end{pgfscope}%
\begin{pgfscope}%
\pgfpathrectangle{\pgfqpoint{0.380943in}{4.185189in}}{\pgfqpoint{4.650000in}{0.614151in}}%
\pgfusepath{clip}%
\pgfsetbuttcap%
\pgfsetroundjoin%
\definecolor{currentfill}{rgb}{0.992326,0.765229,0.614840}%
\pgfsetfillcolor{currentfill}%
\pgfsetlinewidth{0.250937pt}%
\definecolor{currentstroke}{rgb}{1.000000,1.000000,1.000000}%
\pgfsetstrokecolor{currentstroke}%
\pgfsetdash{}{0pt}%
\pgfpathmoveto{\pgfqpoint{3.363962in}{4.448396in}}%
\pgfpathlineto{\pgfqpoint{3.451698in}{4.448396in}}%
\pgfpathlineto{\pgfqpoint{3.451698in}{4.360661in}}%
\pgfpathlineto{\pgfqpoint{3.363962in}{4.360661in}}%
\pgfpathlineto{\pgfqpoint{3.363962in}{4.448396in}}%
\pgfusepath{stroke,fill}%
\end{pgfscope}%
\begin{pgfscope}%
\pgfpathrectangle{\pgfqpoint{0.380943in}{4.185189in}}{\pgfqpoint{4.650000in}{0.614151in}}%
\pgfusepath{clip}%
\pgfsetbuttcap%
\pgfsetroundjoin%
\definecolor{currentfill}{rgb}{0.996571,0.720538,0.589189}%
\pgfsetfillcolor{currentfill}%
\pgfsetlinewidth{0.250937pt}%
\definecolor{currentstroke}{rgb}{1.000000,1.000000,1.000000}%
\pgfsetstrokecolor{currentstroke}%
\pgfsetdash{}{0pt}%
\pgfpathmoveto{\pgfqpoint{3.451698in}{4.448396in}}%
\pgfpathlineto{\pgfqpoint{3.539434in}{4.448396in}}%
\pgfpathlineto{\pgfqpoint{3.539434in}{4.360661in}}%
\pgfpathlineto{\pgfqpoint{3.451698in}{4.360661in}}%
\pgfpathlineto{\pgfqpoint{3.451698in}{4.448396in}}%
\pgfusepath{stroke,fill}%
\end{pgfscope}%
\begin{pgfscope}%
\pgfpathrectangle{\pgfqpoint{0.380943in}{4.185189in}}{\pgfqpoint{4.650000in}{0.614151in}}%
\pgfusepath{clip}%
\pgfsetbuttcap%
\pgfsetroundjoin%
\definecolor{currentfill}{rgb}{0.972549,0.870588,0.692810}%
\pgfsetfillcolor{currentfill}%
\pgfsetlinewidth{0.250937pt}%
\definecolor{currentstroke}{rgb}{1.000000,1.000000,1.000000}%
\pgfsetstrokecolor{currentstroke}%
\pgfsetdash{}{0pt}%
\pgfpathmoveto{\pgfqpoint{3.539434in}{4.448396in}}%
\pgfpathlineto{\pgfqpoint{3.627169in}{4.448396in}}%
\pgfpathlineto{\pgfqpoint{3.627169in}{4.360661in}}%
\pgfpathlineto{\pgfqpoint{3.539434in}{4.360661in}}%
\pgfpathlineto{\pgfqpoint{3.539434in}{4.448396in}}%
\pgfusepath{stroke,fill}%
\end{pgfscope}%
\begin{pgfscope}%
\pgfpathrectangle{\pgfqpoint{0.380943in}{4.185189in}}{\pgfqpoint{4.650000in}{0.614151in}}%
\pgfusepath{clip}%
\pgfsetbuttcap%
\pgfsetroundjoin%
\definecolor{currentfill}{rgb}{0.972549,0.870588,0.692810}%
\pgfsetfillcolor{currentfill}%
\pgfsetlinewidth{0.250937pt}%
\definecolor{currentstroke}{rgb}{1.000000,1.000000,1.000000}%
\pgfsetstrokecolor{currentstroke}%
\pgfsetdash{}{0pt}%
\pgfpathmoveto{\pgfqpoint{3.627169in}{4.448396in}}%
\pgfpathlineto{\pgfqpoint{3.714905in}{4.448396in}}%
\pgfpathlineto{\pgfqpoint{3.714905in}{4.360661in}}%
\pgfpathlineto{\pgfqpoint{3.627169in}{4.360661in}}%
\pgfpathlineto{\pgfqpoint{3.627169in}{4.448396in}}%
\pgfusepath{stroke,fill}%
\end{pgfscope}%
\begin{pgfscope}%
\pgfpathrectangle{\pgfqpoint{0.380943in}{4.185189in}}{\pgfqpoint{4.650000in}{0.614151in}}%
\pgfusepath{clip}%
\pgfsetbuttcap%
\pgfsetroundjoin%
\definecolor{currentfill}{rgb}{1.000000,0.509404,0.491473}%
\pgfsetfillcolor{currentfill}%
\pgfsetlinewidth{0.250937pt}%
\definecolor{currentstroke}{rgb}{1.000000,1.000000,1.000000}%
\pgfsetstrokecolor{currentstroke}%
\pgfsetdash{}{0pt}%
\pgfpathmoveto{\pgfqpoint{3.714905in}{4.448396in}}%
\pgfpathlineto{\pgfqpoint{3.802641in}{4.448396in}}%
\pgfpathlineto{\pgfqpoint{3.802641in}{4.360661in}}%
\pgfpathlineto{\pgfqpoint{3.714905in}{4.360661in}}%
\pgfpathlineto{\pgfqpoint{3.714905in}{4.448396in}}%
\pgfusepath{stroke,fill}%
\end{pgfscope}%
\begin{pgfscope}%
\pgfpathrectangle{\pgfqpoint{0.380943in}{4.185189in}}{\pgfqpoint{4.650000in}{0.614151in}}%
\pgfusepath{clip}%
\pgfsetbuttcap%
\pgfsetroundjoin%
\definecolor{currentfill}{rgb}{0.972549,0.870588,0.692810}%
\pgfsetfillcolor{currentfill}%
\pgfsetlinewidth{0.250937pt}%
\definecolor{currentstroke}{rgb}{1.000000,1.000000,1.000000}%
\pgfsetstrokecolor{currentstroke}%
\pgfsetdash{}{0pt}%
\pgfpathmoveto{\pgfqpoint{3.802641in}{4.448396in}}%
\pgfpathlineto{\pgfqpoint{3.890377in}{4.448396in}}%
\pgfpathlineto{\pgfqpoint{3.890377in}{4.360661in}}%
\pgfpathlineto{\pgfqpoint{3.802641in}{4.360661in}}%
\pgfpathlineto{\pgfqpoint{3.802641in}{4.448396in}}%
\pgfusepath{stroke,fill}%
\end{pgfscope}%
\begin{pgfscope}%
\pgfpathrectangle{\pgfqpoint{0.380943in}{4.185189in}}{\pgfqpoint{4.650000in}{0.614151in}}%
\pgfusepath{clip}%
\pgfsetbuttcap%
\pgfsetroundjoin%
\definecolor{currentfill}{rgb}{0.972549,0.870588,0.692810}%
\pgfsetfillcolor{currentfill}%
\pgfsetlinewidth{0.250937pt}%
\definecolor{currentstroke}{rgb}{1.000000,1.000000,1.000000}%
\pgfsetstrokecolor{currentstroke}%
\pgfsetdash{}{0pt}%
\pgfpathmoveto{\pgfqpoint{3.890377in}{4.448396in}}%
\pgfpathlineto{\pgfqpoint{3.978113in}{4.448396in}}%
\pgfpathlineto{\pgfqpoint{3.978113in}{4.360661in}}%
\pgfpathlineto{\pgfqpoint{3.890377in}{4.360661in}}%
\pgfpathlineto{\pgfqpoint{3.890377in}{4.448396in}}%
\pgfusepath{stroke,fill}%
\end{pgfscope}%
\begin{pgfscope}%
\pgfpathrectangle{\pgfqpoint{0.380943in}{4.185189in}}{\pgfqpoint{4.650000in}{0.614151in}}%
\pgfusepath{clip}%
\pgfsetbuttcap%
\pgfsetroundjoin%
\definecolor{currentfill}{rgb}{0.861576,0.340008,0.340008}%
\pgfsetfillcolor{currentfill}%
\pgfsetlinewidth{0.250937pt}%
\definecolor{currentstroke}{rgb}{1.000000,1.000000,1.000000}%
\pgfsetstrokecolor{currentstroke}%
\pgfsetdash{}{0pt}%
\pgfpathmoveto{\pgfqpoint{3.978113in}{4.448396in}}%
\pgfpathlineto{\pgfqpoint{4.065849in}{4.448396in}}%
\pgfpathlineto{\pgfqpoint{4.065849in}{4.360661in}}%
\pgfpathlineto{\pgfqpoint{3.978113in}{4.360661in}}%
\pgfpathlineto{\pgfqpoint{3.978113in}{4.448396in}}%
\pgfusepath{stroke,fill}%
\end{pgfscope}%
\begin{pgfscope}%
\pgfpathrectangle{\pgfqpoint{0.380943in}{4.185189in}}{\pgfqpoint{4.650000in}{0.614151in}}%
\pgfusepath{clip}%
\pgfsetbuttcap%
\pgfsetroundjoin%
\definecolor{currentfill}{rgb}{0.962414,0.923552,0.722891}%
\pgfsetfillcolor{currentfill}%
\pgfsetlinewidth{0.250937pt}%
\definecolor{currentstroke}{rgb}{1.000000,1.000000,1.000000}%
\pgfsetstrokecolor{currentstroke}%
\pgfsetdash{}{0pt}%
\pgfpathmoveto{\pgfqpoint{4.065849in}{4.448396in}}%
\pgfpathlineto{\pgfqpoint{4.153585in}{4.448396in}}%
\pgfpathlineto{\pgfqpoint{4.153585in}{4.360661in}}%
\pgfpathlineto{\pgfqpoint{4.065849in}{4.360661in}}%
\pgfpathlineto{\pgfqpoint{4.065849in}{4.448396in}}%
\pgfusepath{stroke,fill}%
\end{pgfscope}%
\begin{pgfscope}%
\pgfpathrectangle{\pgfqpoint{0.380943in}{4.185189in}}{\pgfqpoint{4.650000in}{0.614151in}}%
\pgfusepath{clip}%
\pgfsetbuttcap%
\pgfsetroundjoin%
\definecolor{currentfill}{rgb}{0.968166,0.945882,0.748604}%
\pgfsetfillcolor{currentfill}%
\pgfsetlinewidth{0.250937pt}%
\definecolor{currentstroke}{rgb}{1.000000,1.000000,1.000000}%
\pgfsetstrokecolor{currentstroke}%
\pgfsetdash{}{0pt}%
\pgfpathmoveto{\pgfqpoint{4.153585in}{4.448396in}}%
\pgfpathlineto{\pgfqpoint{4.241320in}{4.448396in}}%
\pgfpathlineto{\pgfqpoint{4.241320in}{4.360661in}}%
\pgfpathlineto{\pgfqpoint{4.153585in}{4.360661in}}%
\pgfpathlineto{\pgfqpoint{4.153585in}{4.448396in}}%
\pgfusepath{stroke,fill}%
\end{pgfscope}%
\begin{pgfscope}%
\pgfpathrectangle{\pgfqpoint{0.380943in}{4.185189in}}{\pgfqpoint{4.650000in}{0.614151in}}%
\pgfusepath{clip}%
\pgfsetbuttcap%
\pgfsetroundjoin%
\definecolor{currentfill}{rgb}{0.968166,0.945882,0.748604}%
\pgfsetfillcolor{currentfill}%
\pgfsetlinewidth{0.250937pt}%
\definecolor{currentstroke}{rgb}{1.000000,1.000000,1.000000}%
\pgfsetstrokecolor{currentstroke}%
\pgfsetdash{}{0pt}%
\pgfpathmoveto{\pgfqpoint{4.241320in}{4.448396in}}%
\pgfpathlineto{\pgfqpoint{4.329056in}{4.448396in}}%
\pgfpathlineto{\pgfqpoint{4.329056in}{4.360661in}}%
\pgfpathlineto{\pgfqpoint{4.241320in}{4.360661in}}%
\pgfpathlineto{\pgfqpoint{4.241320in}{4.448396in}}%
\pgfusepath{stroke,fill}%
\end{pgfscope}%
\begin{pgfscope}%
\pgfpathrectangle{\pgfqpoint{0.380943in}{4.185189in}}{\pgfqpoint{4.650000in}{0.614151in}}%
\pgfusepath{clip}%
\pgfsetbuttcap%
\pgfsetroundjoin%
\definecolor{currentfill}{rgb}{0.962414,0.923552,0.722891}%
\pgfsetfillcolor{currentfill}%
\pgfsetlinewidth{0.250937pt}%
\definecolor{currentstroke}{rgb}{1.000000,1.000000,1.000000}%
\pgfsetstrokecolor{currentstroke}%
\pgfsetdash{}{0pt}%
\pgfpathmoveto{\pgfqpoint{4.329056in}{4.448396in}}%
\pgfpathlineto{\pgfqpoint{4.416792in}{4.448396in}}%
\pgfpathlineto{\pgfqpoint{4.416792in}{4.360661in}}%
\pgfpathlineto{\pgfqpoint{4.329056in}{4.360661in}}%
\pgfpathlineto{\pgfqpoint{4.329056in}{4.448396in}}%
\pgfusepath{stroke,fill}%
\end{pgfscope}%
\begin{pgfscope}%
\pgfpathrectangle{\pgfqpoint{0.380943in}{4.185189in}}{\pgfqpoint{4.650000in}{0.614151in}}%
\pgfusepath{clip}%
\pgfsetbuttcap%
\pgfsetroundjoin%
\definecolor{currentfill}{rgb}{0.979654,0.837186,0.669619}%
\pgfsetfillcolor{currentfill}%
\pgfsetlinewidth{0.250937pt}%
\definecolor{currentstroke}{rgb}{1.000000,1.000000,1.000000}%
\pgfsetstrokecolor{currentstroke}%
\pgfsetdash{}{0pt}%
\pgfpathmoveto{\pgfqpoint{4.416792in}{4.448396in}}%
\pgfpathlineto{\pgfqpoint{4.504528in}{4.448396in}}%
\pgfpathlineto{\pgfqpoint{4.504528in}{4.360661in}}%
\pgfpathlineto{\pgfqpoint{4.416792in}{4.360661in}}%
\pgfpathlineto{\pgfqpoint{4.416792in}{4.448396in}}%
\pgfusepath{stroke,fill}%
\end{pgfscope}%
\begin{pgfscope}%
\pgfpathrectangle{\pgfqpoint{0.380943in}{4.185189in}}{\pgfqpoint{4.650000in}{0.614151in}}%
\pgfusepath{clip}%
\pgfsetbuttcap%
\pgfsetroundjoin%
\definecolor{currentfill}{rgb}{0.972549,0.870588,0.692810}%
\pgfsetfillcolor{currentfill}%
\pgfsetlinewidth{0.250937pt}%
\definecolor{currentstroke}{rgb}{1.000000,1.000000,1.000000}%
\pgfsetstrokecolor{currentstroke}%
\pgfsetdash{}{0pt}%
\pgfpathmoveto{\pgfqpoint{4.504528in}{4.448396in}}%
\pgfpathlineto{\pgfqpoint{4.592264in}{4.448396in}}%
\pgfpathlineto{\pgfqpoint{4.592264in}{4.360661in}}%
\pgfpathlineto{\pgfqpoint{4.504528in}{4.360661in}}%
\pgfpathlineto{\pgfqpoint{4.504528in}{4.448396in}}%
\pgfusepath{stroke,fill}%
\end{pgfscope}%
\begin{pgfscope}%
\pgfpathrectangle{\pgfqpoint{0.380943in}{4.185189in}}{\pgfqpoint{4.650000in}{0.614151in}}%
\pgfusepath{clip}%
\pgfsetbuttcap%
\pgfsetroundjoin%
\definecolor{currentfill}{rgb}{1.000000,0.605229,0.530719}%
\pgfsetfillcolor{currentfill}%
\pgfsetlinewidth{0.250937pt}%
\definecolor{currentstroke}{rgb}{1.000000,1.000000,1.000000}%
\pgfsetstrokecolor{currentstroke}%
\pgfsetdash{}{0pt}%
\pgfpathmoveto{\pgfqpoint{4.592264in}{4.448396in}}%
\pgfpathlineto{\pgfqpoint{4.680000in}{4.448396in}}%
\pgfpathlineto{\pgfqpoint{4.680000in}{4.360661in}}%
\pgfpathlineto{\pgfqpoint{4.592264in}{4.360661in}}%
\pgfpathlineto{\pgfqpoint{4.592264in}{4.448396in}}%
\pgfusepath{stroke,fill}%
\end{pgfscope}%
\begin{pgfscope}%
\pgfpathrectangle{\pgfqpoint{0.380943in}{4.185189in}}{\pgfqpoint{4.650000in}{0.614151in}}%
\pgfusepath{clip}%
\pgfsetbuttcap%
\pgfsetroundjoin%
\definecolor{currentfill}{rgb}{0.962414,0.923552,0.722891}%
\pgfsetfillcolor{currentfill}%
\pgfsetlinewidth{0.250937pt}%
\definecolor{currentstroke}{rgb}{1.000000,1.000000,1.000000}%
\pgfsetstrokecolor{currentstroke}%
\pgfsetdash{}{0pt}%
\pgfpathmoveto{\pgfqpoint{4.680000in}{4.448396in}}%
\pgfpathlineto{\pgfqpoint{4.767736in}{4.448396in}}%
\pgfpathlineto{\pgfqpoint{4.767736in}{4.360661in}}%
\pgfpathlineto{\pgfqpoint{4.680000in}{4.360661in}}%
\pgfpathlineto{\pgfqpoint{4.680000in}{4.448396in}}%
\pgfusepath{stroke,fill}%
\end{pgfscope}%
\begin{pgfscope}%
\pgfpathrectangle{\pgfqpoint{0.380943in}{4.185189in}}{\pgfqpoint{4.650000in}{0.614151in}}%
\pgfusepath{clip}%
\pgfsetbuttcap%
\pgfsetroundjoin%
\definecolor{currentfill}{rgb}{0.992326,0.765229,0.614840}%
\pgfsetfillcolor{currentfill}%
\pgfsetlinewidth{0.250937pt}%
\definecolor{currentstroke}{rgb}{1.000000,1.000000,1.000000}%
\pgfsetstrokecolor{currentstroke}%
\pgfsetdash{}{0pt}%
\pgfpathmoveto{\pgfqpoint{4.767736in}{4.448396in}}%
\pgfpathlineto{\pgfqpoint{4.855471in}{4.448396in}}%
\pgfpathlineto{\pgfqpoint{4.855471in}{4.360661in}}%
\pgfpathlineto{\pgfqpoint{4.767736in}{4.360661in}}%
\pgfpathlineto{\pgfqpoint{4.767736in}{4.448396in}}%
\pgfusepath{stroke,fill}%
\end{pgfscope}%
\begin{pgfscope}%
\pgfpathrectangle{\pgfqpoint{0.380943in}{4.185189in}}{\pgfqpoint{4.650000in}{0.614151in}}%
\pgfusepath{clip}%
\pgfsetbuttcap%
\pgfsetroundjoin%
\definecolor{currentfill}{rgb}{0.962414,0.923552,0.722891}%
\pgfsetfillcolor{currentfill}%
\pgfsetlinewidth{0.250937pt}%
\definecolor{currentstroke}{rgb}{1.000000,1.000000,1.000000}%
\pgfsetstrokecolor{currentstroke}%
\pgfsetdash{}{0pt}%
\pgfpathmoveto{\pgfqpoint{4.855471in}{4.448396in}}%
\pgfpathlineto{\pgfqpoint{4.943207in}{4.448396in}}%
\pgfpathlineto{\pgfqpoint{4.943207in}{4.360661in}}%
\pgfpathlineto{\pgfqpoint{4.855471in}{4.360661in}}%
\pgfpathlineto{\pgfqpoint{4.855471in}{4.448396in}}%
\pgfusepath{stroke,fill}%
\end{pgfscope}%
\begin{pgfscope}%
\pgfpathrectangle{\pgfqpoint{0.380943in}{4.185189in}}{\pgfqpoint{4.650000in}{0.614151in}}%
\pgfusepath{clip}%
\pgfsetbuttcap%
\pgfsetroundjoin%
\pgfsetlinewidth{0.250937pt}%
\definecolor{currentstroke}{rgb}{1.000000,1.000000,1.000000}%
\pgfsetstrokecolor{currentstroke}%
\pgfsetdash{}{0pt}%
\pgfpathmoveto{\pgfqpoint{4.943207in}{4.448396in}}%
\pgfpathlineto{\pgfqpoint{5.030943in}{4.448396in}}%
\pgfpathlineto{\pgfqpoint{5.030943in}{4.360661in}}%
\pgfpathlineto{\pgfqpoint{4.943207in}{4.360661in}}%
\pgfpathlineto{\pgfqpoint{4.943207in}{4.448396in}}%
\pgfusepath{stroke}%
\end{pgfscope}%
\begin{pgfscope}%
\pgfpathrectangle{\pgfqpoint{0.380943in}{4.185189in}}{\pgfqpoint{4.650000in}{0.614151in}}%
\pgfusepath{clip}%
\pgfsetbuttcap%
\pgfsetroundjoin%
\definecolor{currentfill}{rgb}{0.965444,0.906113,0.711757}%
\pgfsetfillcolor{currentfill}%
\pgfsetlinewidth{0.250937pt}%
\definecolor{currentstroke}{rgb}{1.000000,1.000000,1.000000}%
\pgfsetstrokecolor{currentstroke}%
\pgfsetdash{}{0pt}%
\pgfpathmoveto{\pgfqpoint{0.380943in}{4.360661in}}%
\pgfpathlineto{\pgfqpoint{0.468679in}{4.360661in}}%
\pgfpathlineto{\pgfqpoint{0.468679in}{4.272925in}}%
\pgfpathlineto{\pgfqpoint{0.380943in}{4.272925in}}%
\pgfpathlineto{\pgfqpoint{0.380943in}{4.360661in}}%
\pgfusepath{stroke,fill}%
\end{pgfscope}%
\begin{pgfscope}%
\pgfpathrectangle{\pgfqpoint{0.380943in}{4.185189in}}{\pgfqpoint{4.650000in}{0.614151in}}%
\pgfusepath{clip}%
\pgfsetbuttcap%
\pgfsetroundjoin%
\definecolor{currentfill}{rgb}{0.968166,0.945882,0.748604}%
\pgfsetfillcolor{currentfill}%
\pgfsetlinewidth{0.250937pt}%
\definecolor{currentstroke}{rgb}{1.000000,1.000000,1.000000}%
\pgfsetstrokecolor{currentstroke}%
\pgfsetdash{}{0pt}%
\pgfpathmoveto{\pgfqpoint{0.468679in}{4.360661in}}%
\pgfpathlineto{\pgfqpoint{0.556415in}{4.360661in}}%
\pgfpathlineto{\pgfqpoint{0.556415in}{4.272925in}}%
\pgfpathlineto{\pgfqpoint{0.468679in}{4.272925in}}%
\pgfpathlineto{\pgfqpoint{0.468679in}{4.360661in}}%
\pgfusepath{stroke,fill}%
\end{pgfscope}%
\begin{pgfscope}%
\pgfpathrectangle{\pgfqpoint{0.380943in}{4.185189in}}{\pgfqpoint{4.650000in}{0.614151in}}%
\pgfusepath{clip}%
\pgfsetbuttcap%
\pgfsetroundjoin%
\definecolor{currentfill}{rgb}{0.968166,0.945882,0.748604}%
\pgfsetfillcolor{currentfill}%
\pgfsetlinewidth{0.250937pt}%
\definecolor{currentstroke}{rgb}{1.000000,1.000000,1.000000}%
\pgfsetstrokecolor{currentstroke}%
\pgfsetdash{}{0pt}%
\pgfpathmoveto{\pgfqpoint{0.556415in}{4.360661in}}%
\pgfpathlineto{\pgfqpoint{0.644151in}{4.360661in}}%
\pgfpathlineto{\pgfqpoint{0.644151in}{4.272925in}}%
\pgfpathlineto{\pgfqpoint{0.556415in}{4.272925in}}%
\pgfpathlineto{\pgfqpoint{0.556415in}{4.360661in}}%
\pgfusepath{stroke,fill}%
\end{pgfscope}%
\begin{pgfscope}%
\pgfpathrectangle{\pgfqpoint{0.380943in}{4.185189in}}{\pgfqpoint{4.650000in}{0.614151in}}%
\pgfusepath{clip}%
\pgfsetbuttcap%
\pgfsetroundjoin%
\definecolor{currentfill}{rgb}{1.000000,0.605229,0.530719}%
\pgfsetfillcolor{currentfill}%
\pgfsetlinewidth{0.250937pt}%
\definecolor{currentstroke}{rgb}{1.000000,1.000000,1.000000}%
\pgfsetstrokecolor{currentstroke}%
\pgfsetdash{}{0pt}%
\pgfpathmoveto{\pgfqpoint{0.644151in}{4.360661in}}%
\pgfpathlineto{\pgfqpoint{0.731886in}{4.360661in}}%
\pgfpathlineto{\pgfqpoint{0.731886in}{4.272925in}}%
\pgfpathlineto{\pgfqpoint{0.644151in}{4.272925in}}%
\pgfpathlineto{\pgfqpoint{0.644151in}{4.360661in}}%
\pgfusepath{stroke,fill}%
\end{pgfscope}%
\begin{pgfscope}%
\pgfpathrectangle{\pgfqpoint{0.380943in}{4.185189in}}{\pgfqpoint{4.650000in}{0.614151in}}%
\pgfusepath{clip}%
\pgfsetbuttcap%
\pgfsetroundjoin%
\definecolor{currentfill}{rgb}{0.979654,0.837186,0.669619}%
\pgfsetfillcolor{currentfill}%
\pgfsetlinewidth{0.250937pt}%
\definecolor{currentstroke}{rgb}{1.000000,1.000000,1.000000}%
\pgfsetstrokecolor{currentstroke}%
\pgfsetdash{}{0pt}%
\pgfpathmoveto{\pgfqpoint{0.731886in}{4.360661in}}%
\pgfpathlineto{\pgfqpoint{0.819622in}{4.360661in}}%
\pgfpathlineto{\pgfqpoint{0.819622in}{4.272925in}}%
\pgfpathlineto{\pgfqpoint{0.731886in}{4.272925in}}%
\pgfpathlineto{\pgfqpoint{0.731886in}{4.360661in}}%
\pgfusepath{stroke,fill}%
\end{pgfscope}%
\begin{pgfscope}%
\pgfpathrectangle{\pgfqpoint{0.380943in}{4.185189in}}{\pgfqpoint{4.650000in}{0.614151in}}%
\pgfusepath{clip}%
\pgfsetbuttcap%
\pgfsetroundjoin%
\definecolor{currentfill}{rgb}{0.986759,0.806398,0.641200}%
\pgfsetfillcolor{currentfill}%
\pgfsetlinewidth{0.250937pt}%
\definecolor{currentstroke}{rgb}{1.000000,1.000000,1.000000}%
\pgfsetstrokecolor{currentstroke}%
\pgfsetdash{}{0pt}%
\pgfpathmoveto{\pgfqpoint{0.819622in}{4.360661in}}%
\pgfpathlineto{\pgfqpoint{0.907358in}{4.360661in}}%
\pgfpathlineto{\pgfqpoint{0.907358in}{4.272925in}}%
\pgfpathlineto{\pgfqpoint{0.819622in}{4.272925in}}%
\pgfpathlineto{\pgfqpoint{0.819622in}{4.360661in}}%
\pgfusepath{stroke,fill}%
\end{pgfscope}%
\begin{pgfscope}%
\pgfpathrectangle{\pgfqpoint{0.380943in}{4.185189in}}{\pgfqpoint{4.650000in}{0.614151in}}%
\pgfusepath{clip}%
\pgfsetbuttcap%
\pgfsetroundjoin%
\definecolor{currentfill}{rgb}{1.000000,1.000000,0.870204}%
\pgfsetfillcolor{currentfill}%
\pgfsetlinewidth{0.250937pt}%
\definecolor{currentstroke}{rgb}{1.000000,1.000000,1.000000}%
\pgfsetstrokecolor{currentstroke}%
\pgfsetdash{}{0pt}%
\pgfpathmoveto{\pgfqpoint{0.907358in}{4.360661in}}%
\pgfpathlineto{\pgfqpoint{0.995094in}{4.360661in}}%
\pgfpathlineto{\pgfqpoint{0.995094in}{4.272925in}}%
\pgfpathlineto{\pgfqpoint{0.907358in}{4.272925in}}%
\pgfpathlineto{\pgfqpoint{0.907358in}{4.360661in}}%
\pgfusepath{stroke,fill}%
\end{pgfscope}%
\begin{pgfscope}%
\pgfpathrectangle{\pgfqpoint{0.380943in}{4.185189in}}{\pgfqpoint{4.650000in}{0.614151in}}%
\pgfusepath{clip}%
\pgfsetbuttcap%
\pgfsetroundjoin%
\definecolor{currentfill}{rgb}{0.972549,0.870588,0.692810}%
\pgfsetfillcolor{currentfill}%
\pgfsetlinewidth{0.250937pt}%
\definecolor{currentstroke}{rgb}{1.000000,1.000000,1.000000}%
\pgfsetstrokecolor{currentstroke}%
\pgfsetdash{}{0pt}%
\pgfpathmoveto{\pgfqpoint{0.995094in}{4.360661in}}%
\pgfpathlineto{\pgfqpoint{1.082830in}{4.360661in}}%
\pgfpathlineto{\pgfqpoint{1.082830in}{4.272925in}}%
\pgfpathlineto{\pgfqpoint{0.995094in}{4.272925in}}%
\pgfpathlineto{\pgfqpoint{0.995094in}{4.360661in}}%
\pgfusepath{stroke,fill}%
\end{pgfscope}%
\begin{pgfscope}%
\pgfpathrectangle{\pgfqpoint{0.380943in}{4.185189in}}{\pgfqpoint{4.650000in}{0.614151in}}%
\pgfusepath{clip}%
\pgfsetbuttcap%
\pgfsetroundjoin%
\definecolor{currentfill}{rgb}{0.962414,0.923552,0.722891}%
\pgfsetfillcolor{currentfill}%
\pgfsetlinewidth{0.250937pt}%
\definecolor{currentstroke}{rgb}{1.000000,1.000000,1.000000}%
\pgfsetstrokecolor{currentstroke}%
\pgfsetdash{}{0pt}%
\pgfpathmoveto{\pgfqpoint{1.082830in}{4.360661in}}%
\pgfpathlineto{\pgfqpoint{1.170566in}{4.360661in}}%
\pgfpathlineto{\pgfqpoint{1.170566in}{4.272925in}}%
\pgfpathlineto{\pgfqpoint{1.082830in}{4.272925in}}%
\pgfpathlineto{\pgfqpoint{1.082830in}{4.360661in}}%
\pgfusepath{stroke,fill}%
\end{pgfscope}%
\begin{pgfscope}%
\pgfpathrectangle{\pgfqpoint{0.380943in}{4.185189in}}{\pgfqpoint{4.650000in}{0.614151in}}%
\pgfusepath{clip}%
\pgfsetbuttcap%
\pgfsetroundjoin%
\definecolor{currentfill}{rgb}{0.965444,0.906113,0.711757}%
\pgfsetfillcolor{currentfill}%
\pgfsetlinewidth{0.250937pt}%
\definecolor{currentstroke}{rgb}{1.000000,1.000000,1.000000}%
\pgfsetstrokecolor{currentstroke}%
\pgfsetdash{}{0pt}%
\pgfpathmoveto{\pgfqpoint{1.170566in}{4.360661in}}%
\pgfpathlineto{\pgfqpoint{1.258302in}{4.360661in}}%
\pgfpathlineto{\pgfqpoint{1.258302in}{4.272925in}}%
\pgfpathlineto{\pgfqpoint{1.170566in}{4.272925in}}%
\pgfpathlineto{\pgfqpoint{1.170566in}{4.360661in}}%
\pgfusepath{stroke,fill}%
\end{pgfscope}%
\begin{pgfscope}%
\pgfpathrectangle{\pgfqpoint{0.380943in}{4.185189in}}{\pgfqpoint{4.650000in}{0.614151in}}%
\pgfusepath{clip}%
\pgfsetbuttcap%
\pgfsetroundjoin%
\definecolor{currentfill}{rgb}{0.968166,0.945882,0.748604}%
\pgfsetfillcolor{currentfill}%
\pgfsetlinewidth{0.250937pt}%
\definecolor{currentstroke}{rgb}{1.000000,1.000000,1.000000}%
\pgfsetstrokecolor{currentstroke}%
\pgfsetdash{}{0pt}%
\pgfpathmoveto{\pgfqpoint{1.258302in}{4.360661in}}%
\pgfpathlineto{\pgfqpoint{1.346037in}{4.360661in}}%
\pgfpathlineto{\pgfqpoint{1.346037in}{4.272925in}}%
\pgfpathlineto{\pgfqpoint{1.258302in}{4.272925in}}%
\pgfpathlineto{\pgfqpoint{1.258302in}{4.360661in}}%
\pgfusepath{stroke,fill}%
\end{pgfscope}%
\begin{pgfscope}%
\pgfpathrectangle{\pgfqpoint{0.380943in}{4.185189in}}{\pgfqpoint{4.650000in}{0.614151in}}%
\pgfusepath{clip}%
\pgfsetbuttcap%
\pgfsetroundjoin%
\definecolor{currentfill}{rgb}{0.962414,0.923552,0.722891}%
\pgfsetfillcolor{currentfill}%
\pgfsetlinewidth{0.250937pt}%
\definecolor{currentstroke}{rgb}{1.000000,1.000000,1.000000}%
\pgfsetstrokecolor{currentstroke}%
\pgfsetdash{}{0pt}%
\pgfpathmoveto{\pgfqpoint{1.346037in}{4.360661in}}%
\pgfpathlineto{\pgfqpoint{1.433773in}{4.360661in}}%
\pgfpathlineto{\pgfqpoint{1.433773in}{4.272925in}}%
\pgfpathlineto{\pgfqpoint{1.346037in}{4.272925in}}%
\pgfpathlineto{\pgfqpoint{1.346037in}{4.360661in}}%
\pgfusepath{stroke,fill}%
\end{pgfscope}%
\begin{pgfscope}%
\pgfpathrectangle{\pgfqpoint{0.380943in}{4.185189in}}{\pgfqpoint{4.650000in}{0.614151in}}%
\pgfusepath{clip}%
\pgfsetbuttcap%
\pgfsetroundjoin%
\definecolor{currentfill}{rgb}{0.992326,0.765229,0.614840}%
\pgfsetfillcolor{currentfill}%
\pgfsetlinewidth{0.250937pt}%
\definecolor{currentstroke}{rgb}{1.000000,1.000000,1.000000}%
\pgfsetstrokecolor{currentstroke}%
\pgfsetdash{}{0pt}%
\pgfpathmoveto{\pgfqpoint{1.433773in}{4.360661in}}%
\pgfpathlineto{\pgfqpoint{1.521509in}{4.360661in}}%
\pgfpathlineto{\pgfqpoint{1.521509in}{4.272925in}}%
\pgfpathlineto{\pgfqpoint{1.433773in}{4.272925in}}%
\pgfpathlineto{\pgfqpoint{1.433773in}{4.360661in}}%
\pgfusepath{stroke,fill}%
\end{pgfscope}%
\begin{pgfscope}%
\pgfpathrectangle{\pgfqpoint{0.380943in}{4.185189in}}{\pgfqpoint{4.650000in}{0.614151in}}%
\pgfusepath{clip}%
\pgfsetbuttcap%
\pgfsetroundjoin%
\definecolor{currentfill}{rgb}{0.965444,0.906113,0.711757}%
\pgfsetfillcolor{currentfill}%
\pgfsetlinewidth{0.250937pt}%
\definecolor{currentstroke}{rgb}{1.000000,1.000000,1.000000}%
\pgfsetstrokecolor{currentstroke}%
\pgfsetdash{}{0pt}%
\pgfpathmoveto{\pgfqpoint{1.521509in}{4.360661in}}%
\pgfpathlineto{\pgfqpoint{1.609245in}{4.360661in}}%
\pgfpathlineto{\pgfqpoint{1.609245in}{4.272925in}}%
\pgfpathlineto{\pgfqpoint{1.521509in}{4.272925in}}%
\pgfpathlineto{\pgfqpoint{1.521509in}{4.360661in}}%
\pgfusepath{stroke,fill}%
\end{pgfscope}%
\begin{pgfscope}%
\pgfpathrectangle{\pgfqpoint{0.380943in}{4.185189in}}{\pgfqpoint{4.650000in}{0.614151in}}%
\pgfusepath{clip}%
\pgfsetbuttcap%
\pgfsetroundjoin%
\definecolor{currentfill}{rgb}{0.979654,0.837186,0.669619}%
\pgfsetfillcolor{currentfill}%
\pgfsetlinewidth{0.250937pt}%
\definecolor{currentstroke}{rgb}{1.000000,1.000000,1.000000}%
\pgfsetstrokecolor{currentstroke}%
\pgfsetdash{}{0pt}%
\pgfpathmoveto{\pgfqpoint{1.609245in}{4.360661in}}%
\pgfpathlineto{\pgfqpoint{1.696981in}{4.360661in}}%
\pgfpathlineto{\pgfqpoint{1.696981in}{4.272925in}}%
\pgfpathlineto{\pgfqpoint{1.609245in}{4.272925in}}%
\pgfpathlineto{\pgfqpoint{1.609245in}{4.360661in}}%
\pgfusepath{stroke,fill}%
\end{pgfscope}%
\begin{pgfscope}%
\pgfpathrectangle{\pgfqpoint{0.380943in}{4.185189in}}{\pgfqpoint{4.650000in}{0.614151in}}%
\pgfusepath{clip}%
\pgfsetbuttcap%
\pgfsetroundjoin%
\definecolor{currentfill}{rgb}{1.000000,1.000000,0.870204}%
\pgfsetfillcolor{currentfill}%
\pgfsetlinewidth{0.250937pt}%
\definecolor{currentstroke}{rgb}{1.000000,1.000000,1.000000}%
\pgfsetstrokecolor{currentstroke}%
\pgfsetdash{}{0pt}%
\pgfpathmoveto{\pgfqpoint{1.696981in}{4.360661in}}%
\pgfpathlineto{\pgfqpoint{1.784717in}{4.360661in}}%
\pgfpathlineto{\pgfqpoint{1.784717in}{4.272925in}}%
\pgfpathlineto{\pgfqpoint{1.696981in}{4.272925in}}%
\pgfpathlineto{\pgfqpoint{1.696981in}{4.360661in}}%
\pgfusepath{stroke,fill}%
\end{pgfscope}%
\begin{pgfscope}%
\pgfpathrectangle{\pgfqpoint{0.380943in}{4.185189in}}{\pgfqpoint{4.650000in}{0.614151in}}%
\pgfusepath{clip}%
\pgfsetbuttcap%
\pgfsetroundjoin%
\definecolor{currentfill}{rgb}{0.972549,0.870588,0.692810}%
\pgfsetfillcolor{currentfill}%
\pgfsetlinewidth{0.250937pt}%
\definecolor{currentstroke}{rgb}{1.000000,1.000000,1.000000}%
\pgfsetstrokecolor{currentstroke}%
\pgfsetdash{}{0pt}%
\pgfpathmoveto{\pgfqpoint{1.784717in}{4.360661in}}%
\pgfpathlineto{\pgfqpoint{1.872452in}{4.360661in}}%
\pgfpathlineto{\pgfqpoint{1.872452in}{4.272925in}}%
\pgfpathlineto{\pgfqpoint{1.784717in}{4.272925in}}%
\pgfpathlineto{\pgfqpoint{1.784717in}{4.360661in}}%
\pgfusepath{stroke,fill}%
\end{pgfscope}%
\begin{pgfscope}%
\pgfpathrectangle{\pgfqpoint{0.380943in}{4.185189in}}{\pgfqpoint{4.650000in}{0.614151in}}%
\pgfusepath{clip}%
\pgfsetbuttcap%
\pgfsetroundjoin%
\definecolor{currentfill}{rgb}{0.968166,0.945882,0.748604}%
\pgfsetfillcolor{currentfill}%
\pgfsetlinewidth{0.250937pt}%
\definecolor{currentstroke}{rgb}{1.000000,1.000000,1.000000}%
\pgfsetstrokecolor{currentstroke}%
\pgfsetdash{}{0pt}%
\pgfpathmoveto{\pgfqpoint{1.872452in}{4.360661in}}%
\pgfpathlineto{\pgfqpoint{1.960188in}{4.360661in}}%
\pgfpathlineto{\pgfqpoint{1.960188in}{4.272925in}}%
\pgfpathlineto{\pgfqpoint{1.872452in}{4.272925in}}%
\pgfpathlineto{\pgfqpoint{1.872452in}{4.360661in}}%
\pgfusepath{stroke,fill}%
\end{pgfscope}%
\begin{pgfscope}%
\pgfpathrectangle{\pgfqpoint{0.380943in}{4.185189in}}{\pgfqpoint{4.650000in}{0.614151in}}%
\pgfusepath{clip}%
\pgfsetbuttcap%
\pgfsetroundjoin%
\definecolor{currentfill}{rgb}{0.962414,0.923552,0.722891}%
\pgfsetfillcolor{currentfill}%
\pgfsetlinewidth{0.250937pt}%
\definecolor{currentstroke}{rgb}{1.000000,1.000000,1.000000}%
\pgfsetstrokecolor{currentstroke}%
\pgfsetdash{}{0pt}%
\pgfpathmoveto{\pgfqpoint{1.960188in}{4.360661in}}%
\pgfpathlineto{\pgfqpoint{2.047924in}{4.360661in}}%
\pgfpathlineto{\pgfqpoint{2.047924in}{4.272925in}}%
\pgfpathlineto{\pgfqpoint{1.960188in}{4.272925in}}%
\pgfpathlineto{\pgfqpoint{1.960188in}{4.360661in}}%
\pgfusepath{stroke,fill}%
\end{pgfscope}%
\begin{pgfscope}%
\pgfpathrectangle{\pgfqpoint{0.380943in}{4.185189in}}{\pgfqpoint{4.650000in}{0.614151in}}%
\pgfusepath{clip}%
\pgfsetbuttcap%
\pgfsetroundjoin%
\definecolor{currentfill}{rgb}{0.979654,0.837186,0.669619}%
\pgfsetfillcolor{currentfill}%
\pgfsetlinewidth{0.250937pt}%
\definecolor{currentstroke}{rgb}{1.000000,1.000000,1.000000}%
\pgfsetstrokecolor{currentstroke}%
\pgfsetdash{}{0pt}%
\pgfpathmoveto{\pgfqpoint{2.047924in}{4.360661in}}%
\pgfpathlineto{\pgfqpoint{2.135660in}{4.360661in}}%
\pgfpathlineto{\pgfqpoint{2.135660in}{4.272925in}}%
\pgfpathlineto{\pgfqpoint{2.047924in}{4.272925in}}%
\pgfpathlineto{\pgfqpoint{2.047924in}{4.360661in}}%
\pgfusepath{stroke,fill}%
\end{pgfscope}%
\begin{pgfscope}%
\pgfpathrectangle{\pgfqpoint{0.380943in}{4.185189in}}{\pgfqpoint{4.650000in}{0.614151in}}%
\pgfusepath{clip}%
\pgfsetbuttcap%
\pgfsetroundjoin%
\definecolor{currentfill}{rgb}{1.000000,1.000000,0.929412}%
\pgfsetfillcolor{currentfill}%
\pgfsetlinewidth{0.250937pt}%
\definecolor{currentstroke}{rgb}{1.000000,1.000000,1.000000}%
\pgfsetstrokecolor{currentstroke}%
\pgfsetdash{}{0pt}%
\pgfpathmoveto{\pgfqpoint{2.135660in}{4.360661in}}%
\pgfpathlineto{\pgfqpoint{2.223396in}{4.360661in}}%
\pgfpathlineto{\pgfqpoint{2.223396in}{4.272925in}}%
\pgfpathlineto{\pgfqpoint{2.135660in}{4.272925in}}%
\pgfpathlineto{\pgfqpoint{2.135660in}{4.360661in}}%
\pgfusepath{stroke,fill}%
\end{pgfscope}%
\begin{pgfscope}%
\pgfpathrectangle{\pgfqpoint{0.380943in}{4.185189in}}{\pgfqpoint{4.650000in}{0.614151in}}%
\pgfusepath{clip}%
\pgfsetbuttcap%
\pgfsetroundjoin%
\definecolor{currentfill}{rgb}{0.968166,0.945882,0.748604}%
\pgfsetfillcolor{currentfill}%
\pgfsetlinewidth{0.250937pt}%
\definecolor{currentstroke}{rgb}{1.000000,1.000000,1.000000}%
\pgfsetstrokecolor{currentstroke}%
\pgfsetdash{}{0pt}%
\pgfpathmoveto{\pgfqpoint{2.223396in}{4.360661in}}%
\pgfpathlineto{\pgfqpoint{2.311132in}{4.360661in}}%
\pgfpathlineto{\pgfqpoint{2.311132in}{4.272925in}}%
\pgfpathlineto{\pgfqpoint{2.223396in}{4.272925in}}%
\pgfpathlineto{\pgfqpoint{2.223396in}{4.360661in}}%
\pgfusepath{stroke,fill}%
\end{pgfscope}%
\begin{pgfscope}%
\pgfpathrectangle{\pgfqpoint{0.380943in}{4.185189in}}{\pgfqpoint{4.650000in}{0.614151in}}%
\pgfusepath{clip}%
\pgfsetbuttcap%
\pgfsetroundjoin%
\definecolor{currentfill}{rgb}{0.968166,0.945882,0.748604}%
\pgfsetfillcolor{currentfill}%
\pgfsetlinewidth{0.250937pt}%
\definecolor{currentstroke}{rgb}{1.000000,1.000000,1.000000}%
\pgfsetstrokecolor{currentstroke}%
\pgfsetdash{}{0pt}%
\pgfpathmoveto{\pgfqpoint{2.311132in}{4.360661in}}%
\pgfpathlineto{\pgfqpoint{2.398868in}{4.360661in}}%
\pgfpathlineto{\pgfqpoint{2.398868in}{4.272925in}}%
\pgfpathlineto{\pgfqpoint{2.311132in}{4.272925in}}%
\pgfpathlineto{\pgfqpoint{2.311132in}{4.360661in}}%
\pgfusepath{stroke,fill}%
\end{pgfscope}%
\begin{pgfscope}%
\pgfpathrectangle{\pgfqpoint{0.380943in}{4.185189in}}{\pgfqpoint{4.650000in}{0.614151in}}%
\pgfusepath{clip}%
\pgfsetbuttcap%
\pgfsetroundjoin%
\definecolor{currentfill}{rgb}{0.979654,0.837186,0.669619}%
\pgfsetfillcolor{currentfill}%
\pgfsetlinewidth{0.250937pt}%
\definecolor{currentstroke}{rgb}{1.000000,1.000000,1.000000}%
\pgfsetstrokecolor{currentstroke}%
\pgfsetdash{}{0pt}%
\pgfpathmoveto{\pgfqpoint{2.398868in}{4.360661in}}%
\pgfpathlineto{\pgfqpoint{2.486603in}{4.360661in}}%
\pgfpathlineto{\pgfqpoint{2.486603in}{4.272925in}}%
\pgfpathlineto{\pgfqpoint{2.398868in}{4.272925in}}%
\pgfpathlineto{\pgfqpoint{2.398868in}{4.360661in}}%
\pgfusepath{stroke,fill}%
\end{pgfscope}%
\begin{pgfscope}%
\pgfpathrectangle{\pgfqpoint{0.380943in}{4.185189in}}{\pgfqpoint{4.650000in}{0.614151in}}%
\pgfusepath{clip}%
\pgfsetbuttcap%
\pgfsetroundjoin%
\definecolor{currentfill}{rgb}{0.979654,0.837186,0.669619}%
\pgfsetfillcolor{currentfill}%
\pgfsetlinewidth{0.250937pt}%
\definecolor{currentstroke}{rgb}{1.000000,1.000000,1.000000}%
\pgfsetstrokecolor{currentstroke}%
\pgfsetdash{}{0pt}%
\pgfpathmoveto{\pgfqpoint{2.486603in}{4.360661in}}%
\pgfpathlineto{\pgfqpoint{2.574339in}{4.360661in}}%
\pgfpathlineto{\pgfqpoint{2.574339in}{4.272925in}}%
\pgfpathlineto{\pgfqpoint{2.486603in}{4.272925in}}%
\pgfpathlineto{\pgfqpoint{2.486603in}{4.360661in}}%
\pgfusepath{stroke,fill}%
\end{pgfscope}%
\begin{pgfscope}%
\pgfpathrectangle{\pgfqpoint{0.380943in}{4.185189in}}{\pgfqpoint{4.650000in}{0.614151in}}%
\pgfusepath{clip}%
\pgfsetbuttcap%
\pgfsetroundjoin%
\definecolor{currentfill}{rgb}{0.991849,0.986144,0.810181}%
\pgfsetfillcolor{currentfill}%
\pgfsetlinewidth{0.250937pt}%
\definecolor{currentstroke}{rgb}{1.000000,1.000000,1.000000}%
\pgfsetstrokecolor{currentstroke}%
\pgfsetdash{}{0pt}%
\pgfpathmoveto{\pgfqpoint{2.574339in}{4.360661in}}%
\pgfpathlineto{\pgfqpoint{2.662075in}{4.360661in}}%
\pgfpathlineto{\pgfqpoint{2.662075in}{4.272925in}}%
\pgfpathlineto{\pgfqpoint{2.574339in}{4.272925in}}%
\pgfpathlineto{\pgfqpoint{2.574339in}{4.360661in}}%
\pgfusepath{stroke,fill}%
\end{pgfscope}%
\begin{pgfscope}%
\pgfpathrectangle{\pgfqpoint{0.380943in}{4.185189in}}{\pgfqpoint{4.650000in}{0.614151in}}%
\pgfusepath{clip}%
\pgfsetbuttcap%
\pgfsetroundjoin%
\definecolor{currentfill}{rgb}{0.991849,0.986144,0.810181}%
\pgfsetfillcolor{currentfill}%
\pgfsetlinewidth{0.250937pt}%
\definecolor{currentstroke}{rgb}{1.000000,1.000000,1.000000}%
\pgfsetstrokecolor{currentstroke}%
\pgfsetdash{}{0pt}%
\pgfpathmoveto{\pgfqpoint{2.662075in}{4.360661in}}%
\pgfpathlineto{\pgfqpoint{2.749811in}{4.360661in}}%
\pgfpathlineto{\pgfqpoint{2.749811in}{4.272925in}}%
\pgfpathlineto{\pgfqpoint{2.662075in}{4.272925in}}%
\pgfpathlineto{\pgfqpoint{2.662075in}{4.360661in}}%
\pgfusepath{stroke,fill}%
\end{pgfscope}%
\begin{pgfscope}%
\pgfpathrectangle{\pgfqpoint{0.380943in}{4.185189in}}{\pgfqpoint{4.650000in}{0.614151in}}%
\pgfusepath{clip}%
\pgfsetbuttcap%
\pgfsetroundjoin%
\definecolor{currentfill}{rgb}{0.991849,0.986144,0.810181}%
\pgfsetfillcolor{currentfill}%
\pgfsetlinewidth{0.250937pt}%
\definecolor{currentstroke}{rgb}{1.000000,1.000000,1.000000}%
\pgfsetstrokecolor{currentstroke}%
\pgfsetdash{}{0pt}%
\pgfpathmoveto{\pgfqpoint{2.749811in}{4.360661in}}%
\pgfpathlineto{\pgfqpoint{2.837547in}{4.360661in}}%
\pgfpathlineto{\pgfqpoint{2.837547in}{4.272925in}}%
\pgfpathlineto{\pgfqpoint{2.749811in}{4.272925in}}%
\pgfpathlineto{\pgfqpoint{2.749811in}{4.360661in}}%
\pgfusepath{stroke,fill}%
\end{pgfscope}%
\begin{pgfscope}%
\pgfpathrectangle{\pgfqpoint{0.380943in}{4.185189in}}{\pgfqpoint{4.650000in}{0.614151in}}%
\pgfusepath{clip}%
\pgfsetbuttcap%
\pgfsetroundjoin%
\definecolor{currentfill}{rgb}{0.965444,0.906113,0.711757}%
\pgfsetfillcolor{currentfill}%
\pgfsetlinewidth{0.250937pt}%
\definecolor{currentstroke}{rgb}{1.000000,1.000000,1.000000}%
\pgfsetstrokecolor{currentstroke}%
\pgfsetdash{}{0pt}%
\pgfpathmoveto{\pgfqpoint{2.837547in}{4.360661in}}%
\pgfpathlineto{\pgfqpoint{2.925283in}{4.360661in}}%
\pgfpathlineto{\pgfqpoint{2.925283in}{4.272925in}}%
\pgfpathlineto{\pgfqpoint{2.837547in}{4.272925in}}%
\pgfpathlineto{\pgfqpoint{2.837547in}{4.360661in}}%
\pgfusepath{stroke,fill}%
\end{pgfscope}%
\begin{pgfscope}%
\pgfpathrectangle{\pgfqpoint{0.380943in}{4.185189in}}{\pgfqpoint{4.650000in}{0.614151in}}%
\pgfusepath{clip}%
\pgfsetbuttcap%
\pgfsetroundjoin%
\definecolor{currentfill}{rgb}{1.000000,1.000000,0.870204}%
\pgfsetfillcolor{currentfill}%
\pgfsetlinewidth{0.250937pt}%
\definecolor{currentstroke}{rgb}{1.000000,1.000000,1.000000}%
\pgfsetstrokecolor{currentstroke}%
\pgfsetdash{}{0pt}%
\pgfpathmoveto{\pgfqpoint{2.925283in}{4.360661in}}%
\pgfpathlineto{\pgfqpoint{3.013019in}{4.360661in}}%
\pgfpathlineto{\pgfqpoint{3.013019in}{4.272925in}}%
\pgfpathlineto{\pgfqpoint{2.925283in}{4.272925in}}%
\pgfpathlineto{\pgfqpoint{2.925283in}{4.360661in}}%
\pgfusepath{stroke,fill}%
\end{pgfscope}%
\begin{pgfscope}%
\pgfpathrectangle{\pgfqpoint{0.380943in}{4.185189in}}{\pgfqpoint{4.650000in}{0.614151in}}%
\pgfusepath{clip}%
\pgfsetbuttcap%
\pgfsetroundjoin%
\definecolor{currentfill}{rgb}{0.965444,0.906113,0.711757}%
\pgfsetfillcolor{currentfill}%
\pgfsetlinewidth{0.250937pt}%
\definecolor{currentstroke}{rgb}{1.000000,1.000000,1.000000}%
\pgfsetstrokecolor{currentstroke}%
\pgfsetdash{}{0pt}%
\pgfpathmoveto{\pgfqpoint{3.013019in}{4.360661in}}%
\pgfpathlineto{\pgfqpoint{3.100754in}{4.360661in}}%
\pgfpathlineto{\pgfqpoint{3.100754in}{4.272925in}}%
\pgfpathlineto{\pgfqpoint{3.013019in}{4.272925in}}%
\pgfpathlineto{\pgfqpoint{3.013019in}{4.360661in}}%
\pgfusepath{stroke,fill}%
\end{pgfscope}%
\begin{pgfscope}%
\pgfpathrectangle{\pgfqpoint{0.380943in}{4.185189in}}{\pgfqpoint{4.650000in}{0.614151in}}%
\pgfusepath{clip}%
\pgfsetbuttcap%
\pgfsetroundjoin%
\definecolor{currentfill}{rgb}{0.968166,0.945882,0.748604}%
\pgfsetfillcolor{currentfill}%
\pgfsetlinewidth{0.250937pt}%
\definecolor{currentstroke}{rgb}{1.000000,1.000000,1.000000}%
\pgfsetstrokecolor{currentstroke}%
\pgfsetdash{}{0pt}%
\pgfpathmoveto{\pgfqpoint{3.100754in}{4.360661in}}%
\pgfpathlineto{\pgfqpoint{3.188490in}{4.360661in}}%
\pgfpathlineto{\pgfqpoint{3.188490in}{4.272925in}}%
\pgfpathlineto{\pgfqpoint{3.100754in}{4.272925in}}%
\pgfpathlineto{\pgfqpoint{3.100754in}{4.360661in}}%
\pgfusepath{stroke,fill}%
\end{pgfscope}%
\begin{pgfscope}%
\pgfpathrectangle{\pgfqpoint{0.380943in}{4.185189in}}{\pgfqpoint{4.650000in}{0.614151in}}%
\pgfusepath{clip}%
\pgfsetbuttcap%
\pgfsetroundjoin%
\definecolor{currentfill}{rgb}{1.000000,1.000000,0.870204}%
\pgfsetfillcolor{currentfill}%
\pgfsetlinewidth{0.250937pt}%
\definecolor{currentstroke}{rgb}{1.000000,1.000000,1.000000}%
\pgfsetstrokecolor{currentstroke}%
\pgfsetdash{}{0pt}%
\pgfpathmoveto{\pgfqpoint{3.188490in}{4.360661in}}%
\pgfpathlineto{\pgfqpoint{3.276226in}{4.360661in}}%
\pgfpathlineto{\pgfqpoint{3.276226in}{4.272925in}}%
\pgfpathlineto{\pgfqpoint{3.188490in}{4.272925in}}%
\pgfpathlineto{\pgfqpoint{3.188490in}{4.360661in}}%
\pgfusepath{stroke,fill}%
\end{pgfscope}%
\begin{pgfscope}%
\pgfpathrectangle{\pgfqpoint{0.380943in}{4.185189in}}{\pgfqpoint{4.650000in}{0.614151in}}%
\pgfusepath{clip}%
\pgfsetbuttcap%
\pgfsetroundjoin%
\definecolor{currentfill}{rgb}{0.968166,0.945882,0.748604}%
\pgfsetfillcolor{currentfill}%
\pgfsetlinewidth{0.250937pt}%
\definecolor{currentstroke}{rgb}{1.000000,1.000000,1.000000}%
\pgfsetstrokecolor{currentstroke}%
\pgfsetdash{}{0pt}%
\pgfpathmoveto{\pgfqpoint{3.276226in}{4.360661in}}%
\pgfpathlineto{\pgfqpoint{3.363962in}{4.360661in}}%
\pgfpathlineto{\pgfqpoint{3.363962in}{4.272925in}}%
\pgfpathlineto{\pgfqpoint{3.276226in}{4.272925in}}%
\pgfpathlineto{\pgfqpoint{3.276226in}{4.360661in}}%
\pgfusepath{stroke,fill}%
\end{pgfscope}%
\begin{pgfscope}%
\pgfpathrectangle{\pgfqpoint{0.380943in}{4.185189in}}{\pgfqpoint{4.650000in}{0.614151in}}%
\pgfusepath{clip}%
\pgfsetbuttcap%
\pgfsetroundjoin%
\definecolor{currentfill}{rgb}{0.962414,0.923552,0.722891}%
\pgfsetfillcolor{currentfill}%
\pgfsetlinewidth{0.250937pt}%
\definecolor{currentstroke}{rgb}{1.000000,1.000000,1.000000}%
\pgfsetstrokecolor{currentstroke}%
\pgfsetdash{}{0pt}%
\pgfpathmoveto{\pgfqpoint{3.363962in}{4.360661in}}%
\pgfpathlineto{\pgfqpoint{3.451698in}{4.360661in}}%
\pgfpathlineto{\pgfqpoint{3.451698in}{4.272925in}}%
\pgfpathlineto{\pgfqpoint{3.363962in}{4.272925in}}%
\pgfpathlineto{\pgfqpoint{3.363962in}{4.360661in}}%
\pgfusepath{stroke,fill}%
\end{pgfscope}%
\begin{pgfscope}%
\pgfpathrectangle{\pgfqpoint{0.380943in}{4.185189in}}{\pgfqpoint{4.650000in}{0.614151in}}%
\pgfusepath{clip}%
\pgfsetbuttcap%
\pgfsetroundjoin%
\definecolor{currentfill}{rgb}{0.962414,0.923552,0.722891}%
\pgfsetfillcolor{currentfill}%
\pgfsetlinewidth{0.250937pt}%
\definecolor{currentstroke}{rgb}{1.000000,1.000000,1.000000}%
\pgfsetstrokecolor{currentstroke}%
\pgfsetdash{}{0pt}%
\pgfpathmoveto{\pgfqpoint{3.451698in}{4.360661in}}%
\pgfpathlineto{\pgfqpoint{3.539434in}{4.360661in}}%
\pgfpathlineto{\pgfqpoint{3.539434in}{4.272925in}}%
\pgfpathlineto{\pgfqpoint{3.451698in}{4.272925in}}%
\pgfpathlineto{\pgfqpoint{3.451698in}{4.360661in}}%
\pgfusepath{stroke,fill}%
\end{pgfscope}%
\begin{pgfscope}%
\pgfpathrectangle{\pgfqpoint{0.380943in}{4.185189in}}{\pgfqpoint{4.650000in}{0.614151in}}%
\pgfusepath{clip}%
\pgfsetbuttcap%
\pgfsetroundjoin%
\definecolor{currentfill}{rgb}{0.968166,0.945882,0.748604}%
\pgfsetfillcolor{currentfill}%
\pgfsetlinewidth{0.250937pt}%
\definecolor{currentstroke}{rgb}{1.000000,1.000000,1.000000}%
\pgfsetstrokecolor{currentstroke}%
\pgfsetdash{}{0pt}%
\pgfpathmoveto{\pgfqpoint{3.539434in}{4.360661in}}%
\pgfpathlineto{\pgfqpoint{3.627169in}{4.360661in}}%
\pgfpathlineto{\pgfqpoint{3.627169in}{4.272925in}}%
\pgfpathlineto{\pgfqpoint{3.539434in}{4.272925in}}%
\pgfpathlineto{\pgfqpoint{3.539434in}{4.360661in}}%
\pgfusepath{stroke,fill}%
\end{pgfscope}%
\begin{pgfscope}%
\pgfpathrectangle{\pgfqpoint{0.380943in}{4.185189in}}{\pgfqpoint{4.650000in}{0.614151in}}%
\pgfusepath{clip}%
\pgfsetbuttcap%
\pgfsetroundjoin%
\definecolor{currentfill}{rgb}{0.979654,0.837186,0.669619}%
\pgfsetfillcolor{currentfill}%
\pgfsetlinewidth{0.250937pt}%
\definecolor{currentstroke}{rgb}{1.000000,1.000000,1.000000}%
\pgfsetstrokecolor{currentstroke}%
\pgfsetdash{}{0pt}%
\pgfpathmoveto{\pgfqpoint{3.627169in}{4.360661in}}%
\pgfpathlineto{\pgfqpoint{3.714905in}{4.360661in}}%
\pgfpathlineto{\pgfqpoint{3.714905in}{4.272925in}}%
\pgfpathlineto{\pgfqpoint{3.627169in}{4.272925in}}%
\pgfpathlineto{\pgfqpoint{3.627169in}{4.360661in}}%
\pgfusepath{stroke,fill}%
\end{pgfscope}%
\begin{pgfscope}%
\pgfpathrectangle{\pgfqpoint{0.380943in}{4.185189in}}{\pgfqpoint{4.650000in}{0.614151in}}%
\pgfusepath{clip}%
\pgfsetbuttcap%
\pgfsetroundjoin%
\definecolor{currentfill}{rgb}{0.991849,0.986144,0.810181}%
\pgfsetfillcolor{currentfill}%
\pgfsetlinewidth{0.250937pt}%
\definecolor{currentstroke}{rgb}{1.000000,1.000000,1.000000}%
\pgfsetstrokecolor{currentstroke}%
\pgfsetdash{}{0pt}%
\pgfpathmoveto{\pgfqpoint{3.714905in}{4.360661in}}%
\pgfpathlineto{\pgfqpoint{3.802641in}{4.360661in}}%
\pgfpathlineto{\pgfqpoint{3.802641in}{4.272925in}}%
\pgfpathlineto{\pgfqpoint{3.714905in}{4.272925in}}%
\pgfpathlineto{\pgfqpoint{3.714905in}{4.360661in}}%
\pgfusepath{stroke,fill}%
\end{pgfscope}%
\begin{pgfscope}%
\pgfpathrectangle{\pgfqpoint{0.380943in}{4.185189in}}{\pgfqpoint{4.650000in}{0.614151in}}%
\pgfusepath{clip}%
\pgfsetbuttcap%
\pgfsetroundjoin%
\definecolor{currentfill}{rgb}{0.962414,0.923552,0.722891}%
\pgfsetfillcolor{currentfill}%
\pgfsetlinewidth{0.250937pt}%
\definecolor{currentstroke}{rgb}{1.000000,1.000000,1.000000}%
\pgfsetstrokecolor{currentstroke}%
\pgfsetdash{}{0pt}%
\pgfpathmoveto{\pgfqpoint{3.802641in}{4.360661in}}%
\pgfpathlineto{\pgfqpoint{3.890377in}{4.360661in}}%
\pgfpathlineto{\pgfqpoint{3.890377in}{4.272925in}}%
\pgfpathlineto{\pgfqpoint{3.802641in}{4.272925in}}%
\pgfpathlineto{\pgfqpoint{3.802641in}{4.360661in}}%
\pgfusepath{stroke,fill}%
\end{pgfscope}%
\begin{pgfscope}%
\pgfpathrectangle{\pgfqpoint{0.380943in}{4.185189in}}{\pgfqpoint{4.650000in}{0.614151in}}%
\pgfusepath{clip}%
\pgfsetbuttcap%
\pgfsetroundjoin%
\definecolor{currentfill}{rgb}{0.992326,0.765229,0.614840}%
\pgfsetfillcolor{currentfill}%
\pgfsetlinewidth{0.250937pt}%
\definecolor{currentstroke}{rgb}{1.000000,1.000000,1.000000}%
\pgfsetstrokecolor{currentstroke}%
\pgfsetdash{}{0pt}%
\pgfpathmoveto{\pgfqpoint{3.890377in}{4.360661in}}%
\pgfpathlineto{\pgfqpoint{3.978113in}{4.360661in}}%
\pgfpathlineto{\pgfqpoint{3.978113in}{4.272925in}}%
\pgfpathlineto{\pgfqpoint{3.890377in}{4.272925in}}%
\pgfpathlineto{\pgfqpoint{3.890377in}{4.360661in}}%
\pgfusepath{stroke,fill}%
\end{pgfscope}%
\begin{pgfscope}%
\pgfpathrectangle{\pgfqpoint{0.380943in}{4.185189in}}{\pgfqpoint{4.650000in}{0.614151in}}%
\pgfusepath{clip}%
\pgfsetbuttcap%
\pgfsetroundjoin%
\definecolor{currentfill}{rgb}{0.986759,0.806398,0.641200}%
\pgfsetfillcolor{currentfill}%
\pgfsetlinewidth{0.250937pt}%
\definecolor{currentstroke}{rgb}{1.000000,1.000000,1.000000}%
\pgfsetstrokecolor{currentstroke}%
\pgfsetdash{}{0pt}%
\pgfpathmoveto{\pgfqpoint{3.978113in}{4.360661in}}%
\pgfpathlineto{\pgfqpoint{4.065849in}{4.360661in}}%
\pgfpathlineto{\pgfqpoint{4.065849in}{4.272925in}}%
\pgfpathlineto{\pgfqpoint{3.978113in}{4.272925in}}%
\pgfpathlineto{\pgfqpoint{3.978113in}{4.360661in}}%
\pgfusepath{stroke,fill}%
\end{pgfscope}%
\begin{pgfscope}%
\pgfpathrectangle{\pgfqpoint{0.380943in}{4.185189in}}{\pgfqpoint{4.650000in}{0.614151in}}%
\pgfusepath{clip}%
\pgfsetbuttcap%
\pgfsetroundjoin%
\definecolor{currentfill}{rgb}{0.962414,0.923552,0.722891}%
\pgfsetfillcolor{currentfill}%
\pgfsetlinewidth{0.250937pt}%
\definecolor{currentstroke}{rgb}{1.000000,1.000000,1.000000}%
\pgfsetstrokecolor{currentstroke}%
\pgfsetdash{}{0pt}%
\pgfpathmoveto{\pgfqpoint{4.065849in}{4.360661in}}%
\pgfpathlineto{\pgfqpoint{4.153585in}{4.360661in}}%
\pgfpathlineto{\pgfqpoint{4.153585in}{4.272925in}}%
\pgfpathlineto{\pgfqpoint{4.065849in}{4.272925in}}%
\pgfpathlineto{\pgfqpoint{4.065849in}{4.360661in}}%
\pgfusepath{stroke,fill}%
\end{pgfscope}%
\begin{pgfscope}%
\pgfpathrectangle{\pgfqpoint{0.380943in}{4.185189in}}{\pgfqpoint{4.650000in}{0.614151in}}%
\pgfusepath{clip}%
\pgfsetbuttcap%
\pgfsetroundjoin%
\definecolor{currentfill}{rgb}{0.965444,0.906113,0.711757}%
\pgfsetfillcolor{currentfill}%
\pgfsetlinewidth{0.250937pt}%
\definecolor{currentstroke}{rgb}{1.000000,1.000000,1.000000}%
\pgfsetstrokecolor{currentstroke}%
\pgfsetdash{}{0pt}%
\pgfpathmoveto{\pgfqpoint{4.153585in}{4.360661in}}%
\pgfpathlineto{\pgfqpoint{4.241320in}{4.360661in}}%
\pgfpathlineto{\pgfqpoint{4.241320in}{4.272925in}}%
\pgfpathlineto{\pgfqpoint{4.153585in}{4.272925in}}%
\pgfpathlineto{\pgfqpoint{4.153585in}{4.360661in}}%
\pgfusepath{stroke,fill}%
\end{pgfscope}%
\begin{pgfscope}%
\pgfpathrectangle{\pgfqpoint{0.380943in}{4.185189in}}{\pgfqpoint{4.650000in}{0.614151in}}%
\pgfusepath{clip}%
\pgfsetbuttcap%
\pgfsetroundjoin%
\definecolor{currentfill}{rgb}{0.965444,0.906113,0.711757}%
\pgfsetfillcolor{currentfill}%
\pgfsetlinewidth{0.250937pt}%
\definecolor{currentstroke}{rgb}{1.000000,1.000000,1.000000}%
\pgfsetstrokecolor{currentstroke}%
\pgfsetdash{}{0pt}%
\pgfpathmoveto{\pgfqpoint{4.241320in}{4.360661in}}%
\pgfpathlineto{\pgfqpoint{4.329056in}{4.360661in}}%
\pgfpathlineto{\pgfqpoint{4.329056in}{4.272925in}}%
\pgfpathlineto{\pgfqpoint{4.241320in}{4.272925in}}%
\pgfpathlineto{\pgfqpoint{4.241320in}{4.360661in}}%
\pgfusepath{stroke,fill}%
\end{pgfscope}%
\begin{pgfscope}%
\pgfpathrectangle{\pgfqpoint{0.380943in}{4.185189in}}{\pgfqpoint{4.650000in}{0.614151in}}%
\pgfusepath{clip}%
\pgfsetbuttcap%
\pgfsetroundjoin%
\definecolor{currentfill}{rgb}{0.965444,0.906113,0.711757}%
\pgfsetfillcolor{currentfill}%
\pgfsetlinewidth{0.250937pt}%
\definecolor{currentstroke}{rgb}{1.000000,1.000000,1.000000}%
\pgfsetstrokecolor{currentstroke}%
\pgfsetdash{}{0pt}%
\pgfpathmoveto{\pgfqpoint{4.329056in}{4.360661in}}%
\pgfpathlineto{\pgfqpoint{4.416792in}{4.360661in}}%
\pgfpathlineto{\pgfqpoint{4.416792in}{4.272925in}}%
\pgfpathlineto{\pgfqpoint{4.329056in}{4.272925in}}%
\pgfpathlineto{\pgfqpoint{4.329056in}{4.360661in}}%
\pgfusepath{stroke,fill}%
\end{pgfscope}%
\begin{pgfscope}%
\pgfpathrectangle{\pgfqpoint{0.380943in}{4.185189in}}{\pgfqpoint{4.650000in}{0.614151in}}%
\pgfusepath{clip}%
\pgfsetbuttcap%
\pgfsetroundjoin%
\definecolor{currentfill}{rgb}{0.965444,0.906113,0.711757}%
\pgfsetfillcolor{currentfill}%
\pgfsetlinewidth{0.250937pt}%
\definecolor{currentstroke}{rgb}{1.000000,1.000000,1.000000}%
\pgfsetstrokecolor{currentstroke}%
\pgfsetdash{}{0pt}%
\pgfpathmoveto{\pgfqpoint{4.416792in}{4.360661in}}%
\pgfpathlineto{\pgfqpoint{4.504528in}{4.360661in}}%
\pgfpathlineto{\pgfqpoint{4.504528in}{4.272925in}}%
\pgfpathlineto{\pgfqpoint{4.416792in}{4.272925in}}%
\pgfpathlineto{\pgfqpoint{4.416792in}{4.360661in}}%
\pgfusepath{stroke,fill}%
\end{pgfscope}%
\begin{pgfscope}%
\pgfpathrectangle{\pgfqpoint{0.380943in}{4.185189in}}{\pgfqpoint{4.650000in}{0.614151in}}%
\pgfusepath{clip}%
\pgfsetbuttcap%
\pgfsetroundjoin%
\definecolor{currentfill}{rgb}{0.965444,0.906113,0.711757}%
\pgfsetfillcolor{currentfill}%
\pgfsetlinewidth{0.250937pt}%
\definecolor{currentstroke}{rgb}{1.000000,1.000000,1.000000}%
\pgfsetstrokecolor{currentstroke}%
\pgfsetdash{}{0pt}%
\pgfpathmoveto{\pgfqpoint{4.504528in}{4.360661in}}%
\pgfpathlineto{\pgfqpoint{4.592264in}{4.360661in}}%
\pgfpathlineto{\pgfqpoint{4.592264in}{4.272925in}}%
\pgfpathlineto{\pgfqpoint{4.504528in}{4.272925in}}%
\pgfpathlineto{\pgfqpoint{4.504528in}{4.360661in}}%
\pgfusepath{stroke,fill}%
\end{pgfscope}%
\begin{pgfscope}%
\pgfpathrectangle{\pgfqpoint{0.380943in}{4.185189in}}{\pgfqpoint{4.650000in}{0.614151in}}%
\pgfusepath{clip}%
\pgfsetbuttcap%
\pgfsetroundjoin%
\definecolor{currentfill}{rgb}{0.972549,0.870588,0.692810}%
\pgfsetfillcolor{currentfill}%
\pgfsetlinewidth{0.250937pt}%
\definecolor{currentstroke}{rgb}{1.000000,1.000000,1.000000}%
\pgfsetstrokecolor{currentstroke}%
\pgfsetdash{}{0pt}%
\pgfpathmoveto{\pgfqpoint{4.592264in}{4.360661in}}%
\pgfpathlineto{\pgfqpoint{4.680000in}{4.360661in}}%
\pgfpathlineto{\pgfqpoint{4.680000in}{4.272925in}}%
\pgfpathlineto{\pgfqpoint{4.592264in}{4.272925in}}%
\pgfpathlineto{\pgfqpoint{4.592264in}{4.360661in}}%
\pgfusepath{stroke,fill}%
\end{pgfscope}%
\begin{pgfscope}%
\pgfpathrectangle{\pgfqpoint{0.380943in}{4.185189in}}{\pgfqpoint{4.650000in}{0.614151in}}%
\pgfusepath{clip}%
\pgfsetbuttcap%
\pgfsetroundjoin%
\definecolor{currentfill}{rgb}{0.972549,0.870588,0.692810}%
\pgfsetfillcolor{currentfill}%
\pgfsetlinewidth{0.250937pt}%
\definecolor{currentstroke}{rgb}{1.000000,1.000000,1.000000}%
\pgfsetstrokecolor{currentstroke}%
\pgfsetdash{}{0pt}%
\pgfpathmoveto{\pgfqpoint{4.680000in}{4.360661in}}%
\pgfpathlineto{\pgfqpoint{4.767736in}{4.360661in}}%
\pgfpathlineto{\pgfqpoint{4.767736in}{4.272925in}}%
\pgfpathlineto{\pgfqpoint{4.680000in}{4.272925in}}%
\pgfpathlineto{\pgfqpoint{4.680000in}{4.360661in}}%
\pgfusepath{stroke,fill}%
\end{pgfscope}%
\begin{pgfscope}%
\pgfpathrectangle{\pgfqpoint{0.380943in}{4.185189in}}{\pgfqpoint{4.650000in}{0.614151in}}%
\pgfusepath{clip}%
\pgfsetbuttcap%
\pgfsetroundjoin%
\definecolor{currentfill}{rgb}{0.979654,0.837186,0.669619}%
\pgfsetfillcolor{currentfill}%
\pgfsetlinewidth{0.250937pt}%
\definecolor{currentstroke}{rgb}{1.000000,1.000000,1.000000}%
\pgfsetstrokecolor{currentstroke}%
\pgfsetdash{}{0pt}%
\pgfpathmoveto{\pgfqpoint{4.767736in}{4.360661in}}%
\pgfpathlineto{\pgfqpoint{4.855471in}{4.360661in}}%
\pgfpathlineto{\pgfqpoint{4.855471in}{4.272925in}}%
\pgfpathlineto{\pgfqpoint{4.767736in}{4.272925in}}%
\pgfpathlineto{\pgfqpoint{4.767736in}{4.360661in}}%
\pgfusepath{stroke,fill}%
\end{pgfscope}%
\begin{pgfscope}%
\pgfpathrectangle{\pgfqpoint{0.380943in}{4.185189in}}{\pgfqpoint{4.650000in}{0.614151in}}%
\pgfusepath{clip}%
\pgfsetbuttcap%
\pgfsetroundjoin%
\definecolor{currentfill}{rgb}{0.962414,0.923552,0.722891}%
\pgfsetfillcolor{currentfill}%
\pgfsetlinewidth{0.250937pt}%
\definecolor{currentstroke}{rgb}{1.000000,1.000000,1.000000}%
\pgfsetstrokecolor{currentstroke}%
\pgfsetdash{}{0pt}%
\pgfpathmoveto{\pgfqpoint{4.855471in}{4.360661in}}%
\pgfpathlineto{\pgfqpoint{4.943207in}{4.360661in}}%
\pgfpathlineto{\pgfqpoint{4.943207in}{4.272925in}}%
\pgfpathlineto{\pgfqpoint{4.855471in}{4.272925in}}%
\pgfpathlineto{\pgfqpoint{4.855471in}{4.360661in}}%
\pgfusepath{stroke,fill}%
\end{pgfscope}%
\begin{pgfscope}%
\pgfpathrectangle{\pgfqpoint{0.380943in}{4.185189in}}{\pgfqpoint{4.650000in}{0.614151in}}%
\pgfusepath{clip}%
\pgfsetbuttcap%
\pgfsetroundjoin%
\pgfsetlinewidth{0.250937pt}%
\definecolor{currentstroke}{rgb}{1.000000,1.000000,1.000000}%
\pgfsetstrokecolor{currentstroke}%
\pgfsetdash{}{0pt}%
\pgfpathmoveto{\pgfqpoint{4.943207in}{4.360661in}}%
\pgfpathlineto{\pgfqpoint{5.030943in}{4.360661in}}%
\pgfpathlineto{\pgfqpoint{5.030943in}{4.272925in}}%
\pgfpathlineto{\pgfqpoint{4.943207in}{4.272925in}}%
\pgfpathlineto{\pgfqpoint{4.943207in}{4.360661in}}%
\pgfusepath{stroke}%
\end{pgfscope}%
\begin{pgfscope}%
\pgfpathrectangle{\pgfqpoint{0.380943in}{4.185189in}}{\pgfqpoint{4.650000in}{0.614151in}}%
\pgfusepath{clip}%
\pgfsetbuttcap%
\pgfsetroundjoin%
\definecolor{currentfill}{rgb}{0.965444,0.906113,0.711757}%
\pgfsetfillcolor{currentfill}%
\pgfsetlinewidth{0.250937pt}%
\definecolor{currentstroke}{rgb}{1.000000,1.000000,1.000000}%
\pgfsetstrokecolor{currentstroke}%
\pgfsetdash{}{0pt}%
\pgfpathmoveto{\pgfqpoint{0.380943in}{4.272925in}}%
\pgfpathlineto{\pgfqpoint{0.468679in}{4.272925in}}%
\pgfpathlineto{\pgfqpoint{0.468679in}{4.185189in}}%
\pgfpathlineto{\pgfqpoint{0.380943in}{4.185189in}}%
\pgfpathlineto{\pgfqpoint{0.380943in}{4.272925in}}%
\pgfusepath{stroke,fill}%
\end{pgfscope}%
\begin{pgfscope}%
\pgfpathrectangle{\pgfqpoint{0.380943in}{4.185189in}}{\pgfqpoint{4.650000in}{0.614151in}}%
\pgfusepath{clip}%
\pgfsetbuttcap%
\pgfsetroundjoin%
\definecolor{currentfill}{rgb}{0.968166,0.945882,0.748604}%
\pgfsetfillcolor{currentfill}%
\pgfsetlinewidth{0.250937pt}%
\definecolor{currentstroke}{rgb}{1.000000,1.000000,1.000000}%
\pgfsetstrokecolor{currentstroke}%
\pgfsetdash{}{0pt}%
\pgfpathmoveto{\pgfqpoint{0.468679in}{4.272925in}}%
\pgfpathlineto{\pgfqpoint{0.556415in}{4.272925in}}%
\pgfpathlineto{\pgfqpoint{0.556415in}{4.185189in}}%
\pgfpathlineto{\pgfqpoint{0.468679in}{4.185189in}}%
\pgfpathlineto{\pgfqpoint{0.468679in}{4.272925in}}%
\pgfusepath{stroke,fill}%
\end{pgfscope}%
\begin{pgfscope}%
\pgfpathrectangle{\pgfqpoint{0.380943in}{4.185189in}}{\pgfqpoint{4.650000in}{0.614151in}}%
\pgfusepath{clip}%
\pgfsetbuttcap%
\pgfsetroundjoin%
\definecolor{currentfill}{rgb}{0.965444,0.906113,0.711757}%
\pgfsetfillcolor{currentfill}%
\pgfsetlinewidth{0.250937pt}%
\definecolor{currentstroke}{rgb}{1.000000,1.000000,1.000000}%
\pgfsetstrokecolor{currentstroke}%
\pgfsetdash{}{0pt}%
\pgfpathmoveto{\pgfqpoint{0.556415in}{4.272925in}}%
\pgfpathlineto{\pgfqpoint{0.644151in}{4.272925in}}%
\pgfpathlineto{\pgfqpoint{0.644151in}{4.185189in}}%
\pgfpathlineto{\pgfqpoint{0.556415in}{4.185189in}}%
\pgfpathlineto{\pgfqpoint{0.556415in}{4.272925in}}%
\pgfusepath{stroke,fill}%
\end{pgfscope}%
\begin{pgfscope}%
\pgfpathrectangle{\pgfqpoint{0.380943in}{4.185189in}}{\pgfqpoint{4.650000in}{0.614151in}}%
\pgfusepath{clip}%
\pgfsetbuttcap%
\pgfsetroundjoin%
\definecolor{currentfill}{rgb}{0.962414,0.923552,0.722891}%
\pgfsetfillcolor{currentfill}%
\pgfsetlinewidth{0.250937pt}%
\definecolor{currentstroke}{rgb}{1.000000,1.000000,1.000000}%
\pgfsetstrokecolor{currentstroke}%
\pgfsetdash{}{0pt}%
\pgfpathmoveto{\pgfqpoint{0.644151in}{4.272925in}}%
\pgfpathlineto{\pgfqpoint{0.731886in}{4.272925in}}%
\pgfpathlineto{\pgfqpoint{0.731886in}{4.185189in}}%
\pgfpathlineto{\pgfqpoint{0.644151in}{4.185189in}}%
\pgfpathlineto{\pgfqpoint{0.644151in}{4.272925in}}%
\pgfusepath{stroke,fill}%
\end{pgfscope}%
\begin{pgfscope}%
\pgfpathrectangle{\pgfqpoint{0.380943in}{4.185189in}}{\pgfqpoint{4.650000in}{0.614151in}}%
\pgfusepath{clip}%
\pgfsetbuttcap%
\pgfsetroundjoin%
\definecolor{currentfill}{rgb}{0.979654,0.837186,0.669619}%
\pgfsetfillcolor{currentfill}%
\pgfsetlinewidth{0.250937pt}%
\definecolor{currentstroke}{rgb}{1.000000,1.000000,1.000000}%
\pgfsetstrokecolor{currentstroke}%
\pgfsetdash{}{0pt}%
\pgfpathmoveto{\pgfqpoint{0.731886in}{4.272925in}}%
\pgfpathlineto{\pgfqpoint{0.819622in}{4.272925in}}%
\pgfpathlineto{\pgfqpoint{0.819622in}{4.185189in}}%
\pgfpathlineto{\pgfqpoint{0.731886in}{4.185189in}}%
\pgfpathlineto{\pgfqpoint{0.731886in}{4.272925in}}%
\pgfusepath{stroke,fill}%
\end{pgfscope}%
\begin{pgfscope}%
\pgfpathrectangle{\pgfqpoint{0.380943in}{4.185189in}}{\pgfqpoint{4.650000in}{0.614151in}}%
\pgfusepath{clip}%
\pgfsetbuttcap%
\pgfsetroundjoin%
\definecolor{currentfill}{rgb}{0.968166,0.945882,0.748604}%
\pgfsetfillcolor{currentfill}%
\pgfsetlinewidth{0.250937pt}%
\definecolor{currentstroke}{rgb}{1.000000,1.000000,1.000000}%
\pgfsetstrokecolor{currentstroke}%
\pgfsetdash{}{0pt}%
\pgfpathmoveto{\pgfqpoint{0.819622in}{4.272925in}}%
\pgfpathlineto{\pgfqpoint{0.907358in}{4.272925in}}%
\pgfpathlineto{\pgfqpoint{0.907358in}{4.185189in}}%
\pgfpathlineto{\pgfqpoint{0.819622in}{4.185189in}}%
\pgfpathlineto{\pgfqpoint{0.819622in}{4.272925in}}%
\pgfusepath{stroke,fill}%
\end{pgfscope}%
\begin{pgfscope}%
\pgfpathrectangle{\pgfqpoint{0.380943in}{4.185189in}}{\pgfqpoint{4.650000in}{0.614151in}}%
\pgfusepath{clip}%
\pgfsetbuttcap%
\pgfsetroundjoin%
\definecolor{currentfill}{rgb}{0.991849,0.986144,0.810181}%
\pgfsetfillcolor{currentfill}%
\pgfsetlinewidth{0.250937pt}%
\definecolor{currentstroke}{rgb}{1.000000,1.000000,1.000000}%
\pgfsetstrokecolor{currentstroke}%
\pgfsetdash{}{0pt}%
\pgfpathmoveto{\pgfqpoint{0.907358in}{4.272925in}}%
\pgfpathlineto{\pgfqpoint{0.995094in}{4.272925in}}%
\pgfpathlineto{\pgfqpoint{0.995094in}{4.185189in}}%
\pgfpathlineto{\pgfqpoint{0.907358in}{4.185189in}}%
\pgfpathlineto{\pgfqpoint{0.907358in}{4.272925in}}%
\pgfusepath{stroke,fill}%
\end{pgfscope}%
\begin{pgfscope}%
\pgfpathrectangle{\pgfqpoint{0.380943in}{4.185189in}}{\pgfqpoint{4.650000in}{0.614151in}}%
\pgfusepath{clip}%
\pgfsetbuttcap%
\pgfsetroundjoin%
\definecolor{currentfill}{rgb}{0.991849,0.986144,0.810181}%
\pgfsetfillcolor{currentfill}%
\pgfsetlinewidth{0.250937pt}%
\definecolor{currentstroke}{rgb}{1.000000,1.000000,1.000000}%
\pgfsetstrokecolor{currentstroke}%
\pgfsetdash{}{0pt}%
\pgfpathmoveto{\pgfqpoint{0.995094in}{4.272925in}}%
\pgfpathlineto{\pgfqpoint{1.082830in}{4.272925in}}%
\pgfpathlineto{\pgfqpoint{1.082830in}{4.185189in}}%
\pgfpathlineto{\pgfqpoint{0.995094in}{4.185189in}}%
\pgfpathlineto{\pgfqpoint{0.995094in}{4.272925in}}%
\pgfusepath{stroke,fill}%
\end{pgfscope}%
\begin{pgfscope}%
\pgfpathrectangle{\pgfqpoint{0.380943in}{4.185189in}}{\pgfqpoint{4.650000in}{0.614151in}}%
\pgfusepath{clip}%
\pgfsetbuttcap%
\pgfsetroundjoin%
\definecolor{currentfill}{rgb}{1.000000,1.000000,0.870204}%
\pgfsetfillcolor{currentfill}%
\pgfsetlinewidth{0.250937pt}%
\definecolor{currentstroke}{rgb}{1.000000,1.000000,1.000000}%
\pgfsetstrokecolor{currentstroke}%
\pgfsetdash{}{0pt}%
\pgfpathmoveto{\pgfqpoint{1.082830in}{4.272925in}}%
\pgfpathlineto{\pgfqpoint{1.170566in}{4.272925in}}%
\pgfpathlineto{\pgfqpoint{1.170566in}{4.185189in}}%
\pgfpathlineto{\pgfqpoint{1.082830in}{4.185189in}}%
\pgfpathlineto{\pgfqpoint{1.082830in}{4.272925in}}%
\pgfusepath{stroke,fill}%
\end{pgfscope}%
\begin{pgfscope}%
\pgfpathrectangle{\pgfqpoint{0.380943in}{4.185189in}}{\pgfqpoint{4.650000in}{0.614151in}}%
\pgfusepath{clip}%
\pgfsetbuttcap%
\pgfsetroundjoin%
\definecolor{currentfill}{rgb}{0.962414,0.923552,0.722891}%
\pgfsetfillcolor{currentfill}%
\pgfsetlinewidth{0.250937pt}%
\definecolor{currentstroke}{rgb}{1.000000,1.000000,1.000000}%
\pgfsetstrokecolor{currentstroke}%
\pgfsetdash{}{0pt}%
\pgfpathmoveto{\pgfqpoint{1.170566in}{4.272925in}}%
\pgfpathlineto{\pgfqpoint{1.258302in}{4.272925in}}%
\pgfpathlineto{\pgfqpoint{1.258302in}{4.185189in}}%
\pgfpathlineto{\pgfqpoint{1.170566in}{4.185189in}}%
\pgfpathlineto{\pgfqpoint{1.170566in}{4.272925in}}%
\pgfusepath{stroke,fill}%
\end{pgfscope}%
\begin{pgfscope}%
\pgfpathrectangle{\pgfqpoint{0.380943in}{4.185189in}}{\pgfqpoint{4.650000in}{0.614151in}}%
\pgfusepath{clip}%
\pgfsetbuttcap%
\pgfsetroundjoin%
\definecolor{currentfill}{rgb}{0.991849,0.986144,0.810181}%
\pgfsetfillcolor{currentfill}%
\pgfsetlinewidth{0.250937pt}%
\definecolor{currentstroke}{rgb}{1.000000,1.000000,1.000000}%
\pgfsetstrokecolor{currentstroke}%
\pgfsetdash{}{0pt}%
\pgfpathmoveto{\pgfqpoint{1.258302in}{4.272925in}}%
\pgfpathlineto{\pgfqpoint{1.346037in}{4.272925in}}%
\pgfpathlineto{\pgfqpoint{1.346037in}{4.185189in}}%
\pgfpathlineto{\pgfqpoint{1.258302in}{4.185189in}}%
\pgfpathlineto{\pgfqpoint{1.258302in}{4.272925in}}%
\pgfusepath{stroke,fill}%
\end{pgfscope}%
\begin{pgfscope}%
\pgfpathrectangle{\pgfqpoint{0.380943in}{4.185189in}}{\pgfqpoint{4.650000in}{0.614151in}}%
\pgfusepath{clip}%
\pgfsetbuttcap%
\pgfsetroundjoin%
\definecolor{currentfill}{rgb}{0.962414,0.923552,0.722891}%
\pgfsetfillcolor{currentfill}%
\pgfsetlinewidth{0.250937pt}%
\definecolor{currentstroke}{rgb}{1.000000,1.000000,1.000000}%
\pgfsetstrokecolor{currentstroke}%
\pgfsetdash{}{0pt}%
\pgfpathmoveto{\pgfqpoint{1.346037in}{4.272925in}}%
\pgfpathlineto{\pgfqpoint{1.433773in}{4.272925in}}%
\pgfpathlineto{\pgfqpoint{1.433773in}{4.185189in}}%
\pgfpathlineto{\pgfqpoint{1.346037in}{4.185189in}}%
\pgfpathlineto{\pgfqpoint{1.346037in}{4.272925in}}%
\pgfusepath{stroke,fill}%
\end{pgfscope}%
\begin{pgfscope}%
\pgfpathrectangle{\pgfqpoint{0.380943in}{4.185189in}}{\pgfqpoint{4.650000in}{0.614151in}}%
\pgfusepath{clip}%
\pgfsetbuttcap%
\pgfsetroundjoin%
\definecolor{currentfill}{rgb}{0.991849,0.986144,0.810181}%
\pgfsetfillcolor{currentfill}%
\pgfsetlinewidth{0.250937pt}%
\definecolor{currentstroke}{rgb}{1.000000,1.000000,1.000000}%
\pgfsetstrokecolor{currentstroke}%
\pgfsetdash{}{0pt}%
\pgfpathmoveto{\pgfqpoint{1.433773in}{4.272925in}}%
\pgfpathlineto{\pgfqpoint{1.521509in}{4.272925in}}%
\pgfpathlineto{\pgfqpoint{1.521509in}{4.185189in}}%
\pgfpathlineto{\pgfqpoint{1.433773in}{4.185189in}}%
\pgfpathlineto{\pgfqpoint{1.433773in}{4.272925in}}%
\pgfusepath{stroke,fill}%
\end{pgfscope}%
\begin{pgfscope}%
\pgfpathrectangle{\pgfqpoint{0.380943in}{4.185189in}}{\pgfqpoint{4.650000in}{0.614151in}}%
\pgfusepath{clip}%
\pgfsetbuttcap%
\pgfsetroundjoin%
\definecolor{currentfill}{rgb}{0.991849,0.986144,0.810181}%
\pgfsetfillcolor{currentfill}%
\pgfsetlinewidth{0.250937pt}%
\definecolor{currentstroke}{rgb}{1.000000,1.000000,1.000000}%
\pgfsetstrokecolor{currentstroke}%
\pgfsetdash{}{0pt}%
\pgfpathmoveto{\pgfqpoint{1.521509in}{4.272925in}}%
\pgfpathlineto{\pgfqpoint{1.609245in}{4.272925in}}%
\pgfpathlineto{\pgfqpoint{1.609245in}{4.185189in}}%
\pgfpathlineto{\pgfqpoint{1.521509in}{4.185189in}}%
\pgfpathlineto{\pgfqpoint{1.521509in}{4.272925in}}%
\pgfusepath{stroke,fill}%
\end{pgfscope}%
\begin{pgfscope}%
\pgfpathrectangle{\pgfqpoint{0.380943in}{4.185189in}}{\pgfqpoint{4.650000in}{0.614151in}}%
\pgfusepath{clip}%
\pgfsetbuttcap%
\pgfsetroundjoin%
\definecolor{currentfill}{rgb}{0.972549,0.870588,0.692810}%
\pgfsetfillcolor{currentfill}%
\pgfsetlinewidth{0.250937pt}%
\definecolor{currentstroke}{rgb}{1.000000,1.000000,1.000000}%
\pgfsetstrokecolor{currentstroke}%
\pgfsetdash{}{0pt}%
\pgfpathmoveto{\pgfqpoint{1.609245in}{4.272925in}}%
\pgfpathlineto{\pgfqpoint{1.696981in}{4.272925in}}%
\pgfpathlineto{\pgfqpoint{1.696981in}{4.185189in}}%
\pgfpathlineto{\pgfqpoint{1.609245in}{4.185189in}}%
\pgfpathlineto{\pgfqpoint{1.609245in}{4.272925in}}%
\pgfusepath{stroke,fill}%
\end{pgfscope}%
\begin{pgfscope}%
\pgfpathrectangle{\pgfqpoint{0.380943in}{4.185189in}}{\pgfqpoint{4.650000in}{0.614151in}}%
\pgfusepath{clip}%
\pgfsetbuttcap%
\pgfsetroundjoin%
\definecolor{currentfill}{rgb}{1.000000,1.000000,0.929412}%
\pgfsetfillcolor{currentfill}%
\pgfsetlinewidth{0.250937pt}%
\definecolor{currentstroke}{rgb}{1.000000,1.000000,1.000000}%
\pgfsetstrokecolor{currentstroke}%
\pgfsetdash{}{0pt}%
\pgfpathmoveto{\pgfqpoint{1.696981in}{4.272925in}}%
\pgfpathlineto{\pgfqpoint{1.784717in}{4.272925in}}%
\pgfpathlineto{\pgfqpoint{1.784717in}{4.185189in}}%
\pgfpathlineto{\pgfqpoint{1.696981in}{4.185189in}}%
\pgfpathlineto{\pgfqpoint{1.696981in}{4.272925in}}%
\pgfusepath{stroke,fill}%
\end{pgfscope}%
\begin{pgfscope}%
\pgfpathrectangle{\pgfqpoint{0.380943in}{4.185189in}}{\pgfqpoint{4.650000in}{0.614151in}}%
\pgfusepath{clip}%
\pgfsetbuttcap%
\pgfsetroundjoin%
\definecolor{currentfill}{rgb}{0.991849,0.986144,0.810181}%
\pgfsetfillcolor{currentfill}%
\pgfsetlinewidth{0.250937pt}%
\definecolor{currentstroke}{rgb}{1.000000,1.000000,1.000000}%
\pgfsetstrokecolor{currentstroke}%
\pgfsetdash{}{0pt}%
\pgfpathmoveto{\pgfqpoint{1.784717in}{4.272925in}}%
\pgfpathlineto{\pgfqpoint{1.872452in}{4.272925in}}%
\pgfpathlineto{\pgfqpoint{1.872452in}{4.185189in}}%
\pgfpathlineto{\pgfqpoint{1.784717in}{4.185189in}}%
\pgfpathlineto{\pgfqpoint{1.784717in}{4.272925in}}%
\pgfusepath{stroke,fill}%
\end{pgfscope}%
\begin{pgfscope}%
\pgfpathrectangle{\pgfqpoint{0.380943in}{4.185189in}}{\pgfqpoint{4.650000in}{0.614151in}}%
\pgfusepath{clip}%
\pgfsetbuttcap%
\pgfsetroundjoin%
\definecolor{currentfill}{rgb}{0.962414,0.923552,0.722891}%
\pgfsetfillcolor{currentfill}%
\pgfsetlinewidth{0.250937pt}%
\definecolor{currentstroke}{rgb}{1.000000,1.000000,1.000000}%
\pgfsetstrokecolor{currentstroke}%
\pgfsetdash{}{0pt}%
\pgfpathmoveto{\pgfqpoint{1.872452in}{4.272925in}}%
\pgfpathlineto{\pgfqpoint{1.960188in}{4.272925in}}%
\pgfpathlineto{\pgfqpoint{1.960188in}{4.185189in}}%
\pgfpathlineto{\pgfqpoint{1.872452in}{4.185189in}}%
\pgfpathlineto{\pgfqpoint{1.872452in}{4.272925in}}%
\pgfusepath{stroke,fill}%
\end{pgfscope}%
\begin{pgfscope}%
\pgfpathrectangle{\pgfqpoint{0.380943in}{4.185189in}}{\pgfqpoint{4.650000in}{0.614151in}}%
\pgfusepath{clip}%
\pgfsetbuttcap%
\pgfsetroundjoin%
\definecolor{currentfill}{rgb}{0.968166,0.945882,0.748604}%
\pgfsetfillcolor{currentfill}%
\pgfsetlinewidth{0.250937pt}%
\definecolor{currentstroke}{rgb}{1.000000,1.000000,1.000000}%
\pgfsetstrokecolor{currentstroke}%
\pgfsetdash{}{0pt}%
\pgfpathmoveto{\pgfqpoint{1.960188in}{4.272925in}}%
\pgfpathlineto{\pgfqpoint{2.047924in}{4.272925in}}%
\pgfpathlineto{\pgfqpoint{2.047924in}{4.185189in}}%
\pgfpathlineto{\pgfqpoint{1.960188in}{4.185189in}}%
\pgfpathlineto{\pgfqpoint{1.960188in}{4.272925in}}%
\pgfusepath{stroke,fill}%
\end{pgfscope}%
\begin{pgfscope}%
\pgfpathrectangle{\pgfqpoint{0.380943in}{4.185189in}}{\pgfqpoint{4.650000in}{0.614151in}}%
\pgfusepath{clip}%
\pgfsetbuttcap%
\pgfsetroundjoin%
\definecolor{currentfill}{rgb}{1.000000,1.000000,0.870204}%
\pgfsetfillcolor{currentfill}%
\pgfsetlinewidth{0.250937pt}%
\definecolor{currentstroke}{rgb}{1.000000,1.000000,1.000000}%
\pgfsetstrokecolor{currentstroke}%
\pgfsetdash{}{0pt}%
\pgfpathmoveto{\pgfqpoint{2.047924in}{4.272925in}}%
\pgfpathlineto{\pgfqpoint{2.135660in}{4.272925in}}%
\pgfpathlineto{\pgfqpoint{2.135660in}{4.185189in}}%
\pgfpathlineto{\pgfqpoint{2.047924in}{4.185189in}}%
\pgfpathlineto{\pgfqpoint{2.047924in}{4.272925in}}%
\pgfusepath{stroke,fill}%
\end{pgfscope}%
\begin{pgfscope}%
\pgfpathrectangle{\pgfqpoint{0.380943in}{4.185189in}}{\pgfqpoint{4.650000in}{0.614151in}}%
\pgfusepath{clip}%
\pgfsetbuttcap%
\pgfsetroundjoin%
\definecolor{currentfill}{rgb}{1.000000,1.000000,0.870204}%
\pgfsetfillcolor{currentfill}%
\pgfsetlinewidth{0.250937pt}%
\definecolor{currentstroke}{rgb}{1.000000,1.000000,1.000000}%
\pgfsetstrokecolor{currentstroke}%
\pgfsetdash{}{0pt}%
\pgfpathmoveto{\pgfqpoint{2.135660in}{4.272925in}}%
\pgfpathlineto{\pgfqpoint{2.223396in}{4.272925in}}%
\pgfpathlineto{\pgfqpoint{2.223396in}{4.185189in}}%
\pgfpathlineto{\pgfqpoint{2.135660in}{4.185189in}}%
\pgfpathlineto{\pgfqpoint{2.135660in}{4.272925in}}%
\pgfusepath{stroke,fill}%
\end{pgfscope}%
\begin{pgfscope}%
\pgfpathrectangle{\pgfqpoint{0.380943in}{4.185189in}}{\pgfqpoint{4.650000in}{0.614151in}}%
\pgfusepath{clip}%
\pgfsetbuttcap%
\pgfsetroundjoin%
\definecolor{currentfill}{rgb}{0.991849,0.986144,0.810181}%
\pgfsetfillcolor{currentfill}%
\pgfsetlinewidth{0.250937pt}%
\definecolor{currentstroke}{rgb}{1.000000,1.000000,1.000000}%
\pgfsetstrokecolor{currentstroke}%
\pgfsetdash{}{0pt}%
\pgfpathmoveto{\pgfqpoint{2.223396in}{4.272925in}}%
\pgfpathlineto{\pgfqpoint{2.311132in}{4.272925in}}%
\pgfpathlineto{\pgfqpoint{2.311132in}{4.185189in}}%
\pgfpathlineto{\pgfqpoint{2.223396in}{4.185189in}}%
\pgfpathlineto{\pgfqpoint{2.223396in}{4.272925in}}%
\pgfusepath{stroke,fill}%
\end{pgfscope}%
\begin{pgfscope}%
\pgfpathrectangle{\pgfqpoint{0.380943in}{4.185189in}}{\pgfqpoint{4.650000in}{0.614151in}}%
\pgfusepath{clip}%
\pgfsetbuttcap%
\pgfsetroundjoin%
\definecolor{currentfill}{rgb}{0.991849,0.986144,0.810181}%
\pgfsetfillcolor{currentfill}%
\pgfsetlinewidth{0.250937pt}%
\definecolor{currentstroke}{rgb}{1.000000,1.000000,1.000000}%
\pgfsetstrokecolor{currentstroke}%
\pgfsetdash{}{0pt}%
\pgfpathmoveto{\pgfqpoint{2.311132in}{4.272925in}}%
\pgfpathlineto{\pgfqpoint{2.398868in}{4.272925in}}%
\pgfpathlineto{\pgfqpoint{2.398868in}{4.185189in}}%
\pgfpathlineto{\pgfqpoint{2.311132in}{4.185189in}}%
\pgfpathlineto{\pgfqpoint{2.311132in}{4.272925in}}%
\pgfusepath{stroke,fill}%
\end{pgfscope}%
\begin{pgfscope}%
\pgfpathrectangle{\pgfqpoint{0.380943in}{4.185189in}}{\pgfqpoint{4.650000in}{0.614151in}}%
\pgfusepath{clip}%
\pgfsetbuttcap%
\pgfsetroundjoin%
\definecolor{currentfill}{rgb}{0.991849,0.986144,0.810181}%
\pgfsetfillcolor{currentfill}%
\pgfsetlinewidth{0.250937pt}%
\definecolor{currentstroke}{rgb}{1.000000,1.000000,1.000000}%
\pgfsetstrokecolor{currentstroke}%
\pgfsetdash{}{0pt}%
\pgfpathmoveto{\pgfqpoint{2.398868in}{4.272925in}}%
\pgfpathlineto{\pgfqpoint{2.486603in}{4.272925in}}%
\pgfpathlineto{\pgfqpoint{2.486603in}{4.185189in}}%
\pgfpathlineto{\pgfqpoint{2.398868in}{4.185189in}}%
\pgfpathlineto{\pgfqpoint{2.398868in}{4.272925in}}%
\pgfusepath{stroke,fill}%
\end{pgfscope}%
\begin{pgfscope}%
\pgfpathrectangle{\pgfqpoint{0.380943in}{4.185189in}}{\pgfqpoint{4.650000in}{0.614151in}}%
\pgfusepath{clip}%
\pgfsetbuttcap%
\pgfsetroundjoin%
\definecolor{currentfill}{rgb}{1.000000,1.000000,0.929412}%
\pgfsetfillcolor{currentfill}%
\pgfsetlinewidth{0.250937pt}%
\definecolor{currentstroke}{rgb}{1.000000,1.000000,1.000000}%
\pgfsetstrokecolor{currentstroke}%
\pgfsetdash{}{0pt}%
\pgfpathmoveto{\pgfqpoint{2.486603in}{4.272925in}}%
\pgfpathlineto{\pgfqpoint{2.574339in}{4.272925in}}%
\pgfpathlineto{\pgfqpoint{2.574339in}{4.185189in}}%
\pgfpathlineto{\pgfqpoint{2.486603in}{4.185189in}}%
\pgfpathlineto{\pgfqpoint{2.486603in}{4.272925in}}%
\pgfusepath{stroke,fill}%
\end{pgfscope}%
\begin{pgfscope}%
\pgfpathrectangle{\pgfqpoint{0.380943in}{4.185189in}}{\pgfqpoint{4.650000in}{0.614151in}}%
\pgfusepath{clip}%
\pgfsetbuttcap%
\pgfsetroundjoin%
\definecolor{currentfill}{rgb}{0.991849,0.986144,0.810181}%
\pgfsetfillcolor{currentfill}%
\pgfsetlinewidth{0.250937pt}%
\definecolor{currentstroke}{rgb}{1.000000,1.000000,1.000000}%
\pgfsetstrokecolor{currentstroke}%
\pgfsetdash{}{0pt}%
\pgfpathmoveto{\pgfqpoint{2.574339in}{4.272925in}}%
\pgfpathlineto{\pgfqpoint{2.662075in}{4.272925in}}%
\pgfpathlineto{\pgfqpoint{2.662075in}{4.185189in}}%
\pgfpathlineto{\pgfqpoint{2.574339in}{4.185189in}}%
\pgfpathlineto{\pgfqpoint{2.574339in}{4.272925in}}%
\pgfusepath{stroke,fill}%
\end{pgfscope}%
\begin{pgfscope}%
\pgfpathrectangle{\pgfqpoint{0.380943in}{4.185189in}}{\pgfqpoint{4.650000in}{0.614151in}}%
\pgfusepath{clip}%
\pgfsetbuttcap%
\pgfsetroundjoin%
\definecolor{currentfill}{rgb}{1.000000,1.000000,0.870204}%
\pgfsetfillcolor{currentfill}%
\pgfsetlinewidth{0.250937pt}%
\definecolor{currentstroke}{rgb}{1.000000,1.000000,1.000000}%
\pgfsetstrokecolor{currentstroke}%
\pgfsetdash{}{0pt}%
\pgfpathmoveto{\pgfqpoint{2.662075in}{4.272925in}}%
\pgfpathlineto{\pgfqpoint{2.749811in}{4.272925in}}%
\pgfpathlineto{\pgfqpoint{2.749811in}{4.185189in}}%
\pgfpathlineto{\pgfqpoint{2.662075in}{4.185189in}}%
\pgfpathlineto{\pgfqpoint{2.662075in}{4.272925in}}%
\pgfusepath{stroke,fill}%
\end{pgfscope}%
\begin{pgfscope}%
\pgfpathrectangle{\pgfqpoint{0.380943in}{4.185189in}}{\pgfqpoint{4.650000in}{0.614151in}}%
\pgfusepath{clip}%
\pgfsetbuttcap%
\pgfsetroundjoin%
\definecolor{currentfill}{rgb}{1.000000,1.000000,0.870204}%
\pgfsetfillcolor{currentfill}%
\pgfsetlinewidth{0.250937pt}%
\definecolor{currentstroke}{rgb}{1.000000,1.000000,1.000000}%
\pgfsetstrokecolor{currentstroke}%
\pgfsetdash{}{0pt}%
\pgfpathmoveto{\pgfqpoint{2.749811in}{4.272925in}}%
\pgfpathlineto{\pgfqpoint{2.837547in}{4.272925in}}%
\pgfpathlineto{\pgfqpoint{2.837547in}{4.185189in}}%
\pgfpathlineto{\pgfqpoint{2.749811in}{4.185189in}}%
\pgfpathlineto{\pgfqpoint{2.749811in}{4.272925in}}%
\pgfusepath{stroke,fill}%
\end{pgfscope}%
\begin{pgfscope}%
\pgfpathrectangle{\pgfqpoint{0.380943in}{4.185189in}}{\pgfqpoint{4.650000in}{0.614151in}}%
\pgfusepath{clip}%
\pgfsetbuttcap%
\pgfsetroundjoin%
\definecolor{currentfill}{rgb}{0.991849,0.986144,0.810181}%
\pgfsetfillcolor{currentfill}%
\pgfsetlinewidth{0.250937pt}%
\definecolor{currentstroke}{rgb}{1.000000,1.000000,1.000000}%
\pgfsetstrokecolor{currentstroke}%
\pgfsetdash{}{0pt}%
\pgfpathmoveto{\pgfqpoint{2.837547in}{4.272925in}}%
\pgfpathlineto{\pgfqpoint{2.925283in}{4.272925in}}%
\pgfpathlineto{\pgfqpoint{2.925283in}{4.185189in}}%
\pgfpathlineto{\pgfqpoint{2.837547in}{4.185189in}}%
\pgfpathlineto{\pgfqpoint{2.837547in}{4.272925in}}%
\pgfusepath{stroke,fill}%
\end{pgfscope}%
\begin{pgfscope}%
\pgfpathrectangle{\pgfqpoint{0.380943in}{4.185189in}}{\pgfqpoint{4.650000in}{0.614151in}}%
\pgfusepath{clip}%
\pgfsetbuttcap%
\pgfsetroundjoin%
\definecolor{currentfill}{rgb}{1.000000,1.000000,0.870204}%
\pgfsetfillcolor{currentfill}%
\pgfsetlinewidth{0.250937pt}%
\definecolor{currentstroke}{rgb}{1.000000,1.000000,1.000000}%
\pgfsetstrokecolor{currentstroke}%
\pgfsetdash{}{0pt}%
\pgfpathmoveto{\pgfqpoint{2.925283in}{4.272925in}}%
\pgfpathlineto{\pgfqpoint{3.013019in}{4.272925in}}%
\pgfpathlineto{\pgfqpoint{3.013019in}{4.185189in}}%
\pgfpathlineto{\pgfqpoint{2.925283in}{4.185189in}}%
\pgfpathlineto{\pgfqpoint{2.925283in}{4.272925in}}%
\pgfusepath{stroke,fill}%
\end{pgfscope}%
\begin{pgfscope}%
\pgfpathrectangle{\pgfqpoint{0.380943in}{4.185189in}}{\pgfqpoint{4.650000in}{0.614151in}}%
\pgfusepath{clip}%
\pgfsetbuttcap%
\pgfsetroundjoin%
\definecolor{currentfill}{rgb}{1.000000,1.000000,0.870204}%
\pgfsetfillcolor{currentfill}%
\pgfsetlinewidth{0.250937pt}%
\definecolor{currentstroke}{rgb}{1.000000,1.000000,1.000000}%
\pgfsetstrokecolor{currentstroke}%
\pgfsetdash{}{0pt}%
\pgfpathmoveto{\pgfqpoint{3.013019in}{4.272925in}}%
\pgfpathlineto{\pgfqpoint{3.100754in}{4.272925in}}%
\pgfpathlineto{\pgfqpoint{3.100754in}{4.185189in}}%
\pgfpathlineto{\pgfqpoint{3.013019in}{4.185189in}}%
\pgfpathlineto{\pgfqpoint{3.013019in}{4.272925in}}%
\pgfusepath{stroke,fill}%
\end{pgfscope}%
\begin{pgfscope}%
\pgfpathrectangle{\pgfqpoint{0.380943in}{4.185189in}}{\pgfqpoint{4.650000in}{0.614151in}}%
\pgfusepath{clip}%
\pgfsetbuttcap%
\pgfsetroundjoin%
\definecolor{currentfill}{rgb}{1.000000,1.000000,0.870204}%
\pgfsetfillcolor{currentfill}%
\pgfsetlinewidth{0.250937pt}%
\definecolor{currentstroke}{rgb}{1.000000,1.000000,1.000000}%
\pgfsetstrokecolor{currentstroke}%
\pgfsetdash{}{0pt}%
\pgfpathmoveto{\pgfqpoint{3.100754in}{4.272925in}}%
\pgfpathlineto{\pgfqpoint{3.188490in}{4.272925in}}%
\pgfpathlineto{\pgfqpoint{3.188490in}{4.185189in}}%
\pgfpathlineto{\pgfqpoint{3.100754in}{4.185189in}}%
\pgfpathlineto{\pgfqpoint{3.100754in}{4.272925in}}%
\pgfusepath{stroke,fill}%
\end{pgfscope}%
\begin{pgfscope}%
\pgfpathrectangle{\pgfqpoint{0.380943in}{4.185189in}}{\pgfqpoint{4.650000in}{0.614151in}}%
\pgfusepath{clip}%
\pgfsetbuttcap%
\pgfsetroundjoin%
\definecolor{currentfill}{rgb}{0.991849,0.986144,0.810181}%
\pgfsetfillcolor{currentfill}%
\pgfsetlinewidth{0.250937pt}%
\definecolor{currentstroke}{rgb}{1.000000,1.000000,1.000000}%
\pgfsetstrokecolor{currentstroke}%
\pgfsetdash{}{0pt}%
\pgfpathmoveto{\pgfqpoint{3.188490in}{4.272925in}}%
\pgfpathlineto{\pgfqpoint{3.276226in}{4.272925in}}%
\pgfpathlineto{\pgfqpoint{3.276226in}{4.185189in}}%
\pgfpathlineto{\pgfqpoint{3.188490in}{4.185189in}}%
\pgfpathlineto{\pgfqpoint{3.188490in}{4.272925in}}%
\pgfusepath{stroke,fill}%
\end{pgfscope}%
\begin{pgfscope}%
\pgfpathrectangle{\pgfqpoint{0.380943in}{4.185189in}}{\pgfqpoint{4.650000in}{0.614151in}}%
\pgfusepath{clip}%
\pgfsetbuttcap%
\pgfsetroundjoin%
\definecolor{currentfill}{rgb}{1.000000,1.000000,0.929412}%
\pgfsetfillcolor{currentfill}%
\pgfsetlinewidth{0.250937pt}%
\definecolor{currentstroke}{rgb}{1.000000,1.000000,1.000000}%
\pgfsetstrokecolor{currentstroke}%
\pgfsetdash{}{0pt}%
\pgfpathmoveto{\pgfqpoint{3.276226in}{4.272925in}}%
\pgfpathlineto{\pgfqpoint{3.363962in}{4.272925in}}%
\pgfpathlineto{\pgfqpoint{3.363962in}{4.185189in}}%
\pgfpathlineto{\pgfqpoint{3.276226in}{4.185189in}}%
\pgfpathlineto{\pgfqpoint{3.276226in}{4.272925in}}%
\pgfusepath{stroke,fill}%
\end{pgfscope}%
\begin{pgfscope}%
\pgfpathrectangle{\pgfqpoint{0.380943in}{4.185189in}}{\pgfqpoint{4.650000in}{0.614151in}}%
\pgfusepath{clip}%
\pgfsetbuttcap%
\pgfsetroundjoin%
\definecolor{currentfill}{rgb}{0.968166,0.945882,0.748604}%
\pgfsetfillcolor{currentfill}%
\pgfsetlinewidth{0.250937pt}%
\definecolor{currentstroke}{rgb}{1.000000,1.000000,1.000000}%
\pgfsetstrokecolor{currentstroke}%
\pgfsetdash{}{0pt}%
\pgfpathmoveto{\pgfqpoint{3.363962in}{4.272925in}}%
\pgfpathlineto{\pgfqpoint{3.451698in}{4.272925in}}%
\pgfpathlineto{\pgfqpoint{3.451698in}{4.185189in}}%
\pgfpathlineto{\pgfqpoint{3.363962in}{4.185189in}}%
\pgfpathlineto{\pgfqpoint{3.363962in}{4.272925in}}%
\pgfusepath{stroke,fill}%
\end{pgfscope}%
\begin{pgfscope}%
\pgfpathrectangle{\pgfqpoint{0.380943in}{4.185189in}}{\pgfqpoint{4.650000in}{0.614151in}}%
\pgfusepath{clip}%
\pgfsetbuttcap%
\pgfsetroundjoin%
\definecolor{currentfill}{rgb}{0.962414,0.923552,0.722891}%
\pgfsetfillcolor{currentfill}%
\pgfsetlinewidth{0.250937pt}%
\definecolor{currentstroke}{rgb}{1.000000,1.000000,1.000000}%
\pgfsetstrokecolor{currentstroke}%
\pgfsetdash{}{0pt}%
\pgfpathmoveto{\pgfqpoint{3.451698in}{4.272925in}}%
\pgfpathlineto{\pgfqpoint{3.539434in}{4.272925in}}%
\pgfpathlineto{\pgfqpoint{3.539434in}{4.185189in}}%
\pgfpathlineto{\pgfqpoint{3.451698in}{4.185189in}}%
\pgfpathlineto{\pgfqpoint{3.451698in}{4.272925in}}%
\pgfusepath{stroke,fill}%
\end{pgfscope}%
\begin{pgfscope}%
\pgfpathrectangle{\pgfqpoint{0.380943in}{4.185189in}}{\pgfqpoint{4.650000in}{0.614151in}}%
\pgfusepath{clip}%
\pgfsetbuttcap%
\pgfsetroundjoin%
\definecolor{currentfill}{rgb}{0.962414,0.923552,0.722891}%
\pgfsetfillcolor{currentfill}%
\pgfsetlinewidth{0.250937pt}%
\definecolor{currentstroke}{rgb}{1.000000,1.000000,1.000000}%
\pgfsetstrokecolor{currentstroke}%
\pgfsetdash{}{0pt}%
\pgfpathmoveto{\pgfqpoint{3.539434in}{4.272925in}}%
\pgfpathlineto{\pgfqpoint{3.627169in}{4.272925in}}%
\pgfpathlineto{\pgfqpoint{3.627169in}{4.185189in}}%
\pgfpathlineto{\pgfqpoint{3.539434in}{4.185189in}}%
\pgfpathlineto{\pgfqpoint{3.539434in}{4.272925in}}%
\pgfusepath{stroke,fill}%
\end{pgfscope}%
\begin{pgfscope}%
\pgfpathrectangle{\pgfqpoint{0.380943in}{4.185189in}}{\pgfqpoint{4.650000in}{0.614151in}}%
\pgfusepath{clip}%
\pgfsetbuttcap%
\pgfsetroundjoin%
\definecolor{currentfill}{rgb}{0.962414,0.923552,0.722891}%
\pgfsetfillcolor{currentfill}%
\pgfsetlinewidth{0.250937pt}%
\definecolor{currentstroke}{rgb}{1.000000,1.000000,1.000000}%
\pgfsetstrokecolor{currentstroke}%
\pgfsetdash{}{0pt}%
\pgfpathmoveto{\pgfqpoint{3.627169in}{4.272925in}}%
\pgfpathlineto{\pgfqpoint{3.714905in}{4.272925in}}%
\pgfpathlineto{\pgfqpoint{3.714905in}{4.185189in}}%
\pgfpathlineto{\pgfqpoint{3.627169in}{4.185189in}}%
\pgfpathlineto{\pgfqpoint{3.627169in}{4.272925in}}%
\pgfusepath{stroke,fill}%
\end{pgfscope}%
\begin{pgfscope}%
\pgfpathrectangle{\pgfqpoint{0.380943in}{4.185189in}}{\pgfqpoint{4.650000in}{0.614151in}}%
\pgfusepath{clip}%
\pgfsetbuttcap%
\pgfsetroundjoin%
\definecolor{currentfill}{rgb}{0.991849,0.986144,0.810181}%
\pgfsetfillcolor{currentfill}%
\pgfsetlinewidth{0.250937pt}%
\definecolor{currentstroke}{rgb}{1.000000,1.000000,1.000000}%
\pgfsetstrokecolor{currentstroke}%
\pgfsetdash{}{0pt}%
\pgfpathmoveto{\pgfqpoint{3.714905in}{4.272925in}}%
\pgfpathlineto{\pgfqpoint{3.802641in}{4.272925in}}%
\pgfpathlineto{\pgfqpoint{3.802641in}{4.185189in}}%
\pgfpathlineto{\pgfqpoint{3.714905in}{4.185189in}}%
\pgfpathlineto{\pgfqpoint{3.714905in}{4.272925in}}%
\pgfusepath{stroke,fill}%
\end{pgfscope}%
\begin{pgfscope}%
\pgfpathrectangle{\pgfqpoint{0.380943in}{4.185189in}}{\pgfqpoint{4.650000in}{0.614151in}}%
\pgfusepath{clip}%
\pgfsetbuttcap%
\pgfsetroundjoin%
\definecolor{currentfill}{rgb}{0.991849,0.986144,0.810181}%
\pgfsetfillcolor{currentfill}%
\pgfsetlinewidth{0.250937pt}%
\definecolor{currentstroke}{rgb}{1.000000,1.000000,1.000000}%
\pgfsetstrokecolor{currentstroke}%
\pgfsetdash{}{0pt}%
\pgfpathmoveto{\pgfqpoint{3.802641in}{4.272925in}}%
\pgfpathlineto{\pgfqpoint{3.890377in}{4.272925in}}%
\pgfpathlineto{\pgfqpoint{3.890377in}{4.185189in}}%
\pgfpathlineto{\pgfqpoint{3.802641in}{4.185189in}}%
\pgfpathlineto{\pgfqpoint{3.802641in}{4.272925in}}%
\pgfusepath{stroke,fill}%
\end{pgfscope}%
\begin{pgfscope}%
\pgfpathrectangle{\pgfqpoint{0.380943in}{4.185189in}}{\pgfqpoint{4.650000in}{0.614151in}}%
\pgfusepath{clip}%
\pgfsetbuttcap%
\pgfsetroundjoin%
\definecolor{currentfill}{rgb}{0.968166,0.945882,0.748604}%
\pgfsetfillcolor{currentfill}%
\pgfsetlinewidth{0.250937pt}%
\definecolor{currentstroke}{rgb}{1.000000,1.000000,1.000000}%
\pgfsetstrokecolor{currentstroke}%
\pgfsetdash{}{0pt}%
\pgfpathmoveto{\pgfqpoint{3.890377in}{4.272925in}}%
\pgfpathlineto{\pgfqpoint{3.978113in}{4.272925in}}%
\pgfpathlineto{\pgfqpoint{3.978113in}{4.185189in}}%
\pgfpathlineto{\pgfqpoint{3.890377in}{4.185189in}}%
\pgfpathlineto{\pgfqpoint{3.890377in}{4.272925in}}%
\pgfusepath{stroke,fill}%
\end{pgfscope}%
\begin{pgfscope}%
\pgfpathrectangle{\pgfqpoint{0.380943in}{4.185189in}}{\pgfqpoint{4.650000in}{0.614151in}}%
\pgfusepath{clip}%
\pgfsetbuttcap%
\pgfsetroundjoin%
\definecolor{currentfill}{rgb}{0.991849,0.986144,0.810181}%
\pgfsetfillcolor{currentfill}%
\pgfsetlinewidth{0.250937pt}%
\definecolor{currentstroke}{rgb}{1.000000,1.000000,1.000000}%
\pgfsetstrokecolor{currentstroke}%
\pgfsetdash{}{0pt}%
\pgfpathmoveto{\pgfqpoint{3.978113in}{4.272925in}}%
\pgfpathlineto{\pgfqpoint{4.065849in}{4.272925in}}%
\pgfpathlineto{\pgfqpoint{4.065849in}{4.185189in}}%
\pgfpathlineto{\pgfqpoint{3.978113in}{4.185189in}}%
\pgfpathlineto{\pgfqpoint{3.978113in}{4.272925in}}%
\pgfusepath{stroke,fill}%
\end{pgfscope}%
\begin{pgfscope}%
\pgfpathrectangle{\pgfqpoint{0.380943in}{4.185189in}}{\pgfqpoint{4.650000in}{0.614151in}}%
\pgfusepath{clip}%
\pgfsetbuttcap%
\pgfsetroundjoin%
\definecolor{currentfill}{rgb}{0.965444,0.906113,0.711757}%
\pgfsetfillcolor{currentfill}%
\pgfsetlinewidth{0.250937pt}%
\definecolor{currentstroke}{rgb}{1.000000,1.000000,1.000000}%
\pgfsetstrokecolor{currentstroke}%
\pgfsetdash{}{0pt}%
\pgfpathmoveto{\pgfqpoint{4.065849in}{4.272925in}}%
\pgfpathlineto{\pgfqpoint{4.153585in}{4.272925in}}%
\pgfpathlineto{\pgfqpoint{4.153585in}{4.185189in}}%
\pgfpathlineto{\pgfqpoint{4.065849in}{4.185189in}}%
\pgfpathlineto{\pgfqpoint{4.065849in}{4.272925in}}%
\pgfusepath{stroke,fill}%
\end{pgfscope}%
\begin{pgfscope}%
\pgfpathrectangle{\pgfqpoint{0.380943in}{4.185189in}}{\pgfqpoint{4.650000in}{0.614151in}}%
\pgfusepath{clip}%
\pgfsetbuttcap%
\pgfsetroundjoin%
\definecolor{currentfill}{rgb}{1.000000,1.000000,0.870204}%
\pgfsetfillcolor{currentfill}%
\pgfsetlinewidth{0.250937pt}%
\definecolor{currentstroke}{rgb}{1.000000,1.000000,1.000000}%
\pgfsetstrokecolor{currentstroke}%
\pgfsetdash{}{0pt}%
\pgfpathmoveto{\pgfqpoint{4.153585in}{4.272925in}}%
\pgfpathlineto{\pgfqpoint{4.241320in}{4.272925in}}%
\pgfpathlineto{\pgfqpoint{4.241320in}{4.185189in}}%
\pgfpathlineto{\pgfqpoint{4.153585in}{4.185189in}}%
\pgfpathlineto{\pgfqpoint{4.153585in}{4.272925in}}%
\pgfusepath{stroke,fill}%
\end{pgfscope}%
\begin{pgfscope}%
\pgfpathrectangle{\pgfqpoint{0.380943in}{4.185189in}}{\pgfqpoint{4.650000in}{0.614151in}}%
\pgfusepath{clip}%
\pgfsetbuttcap%
\pgfsetroundjoin%
\definecolor{currentfill}{rgb}{0.991849,0.986144,0.810181}%
\pgfsetfillcolor{currentfill}%
\pgfsetlinewidth{0.250937pt}%
\definecolor{currentstroke}{rgb}{1.000000,1.000000,1.000000}%
\pgfsetstrokecolor{currentstroke}%
\pgfsetdash{}{0pt}%
\pgfpathmoveto{\pgfqpoint{4.241320in}{4.272925in}}%
\pgfpathlineto{\pgfqpoint{4.329056in}{4.272925in}}%
\pgfpathlineto{\pgfqpoint{4.329056in}{4.185189in}}%
\pgfpathlineto{\pgfqpoint{4.241320in}{4.185189in}}%
\pgfpathlineto{\pgfqpoint{4.241320in}{4.272925in}}%
\pgfusepath{stroke,fill}%
\end{pgfscope}%
\begin{pgfscope}%
\pgfpathrectangle{\pgfqpoint{0.380943in}{4.185189in}}{\pgfqpoint{4.650000in}{0.614151in}}%
\pgfusepath{clip}%
\pgfsetbuttcap%
\pgfsetroundjoin%
\definecolor{currentfill}{rgb}{0.991849,0.986144,0.810181}%
\pgfsetfillcolor{currentfill}%
\pgfsetlinewidth{0.250937pt}%
\definecolor{currentstroke}{rgb}{1.000000,1.000000,1.000000}%
\pgfsetstrokecolor{currentstroke}%
\pgfsetdash{}{0pt}%
\pgfpathmoveto{\pgfqpoint{4.329056in}{4.272925in}}%
\pgfpathlineto{\pgfqpoint{4.416792in}{4.272925in}}%
\pgfpathlineto{\pgfqpoint{4.416792in}{4.185189in}}%
\pgfpathlineto{\pgfqpoint{4.329056in}{4.185189in}}%
\pgfpathlineto{\pgfqpoint{4.329056in}{4.272925in}}%
\pgfusepath{stroke,fill}%
\end{pgfscope}%
\begin{pgfscope}%
\pgfpathrectangle{\pgfqpoint{0.380943in}{4.185189in}}{\pgfqpoint{4.650000in}{0.614151in}}%
\pgfusepath{clip}%
\pgfsetbuttcap%
\pgfsetroundjoin%
\definecolor{currentfill}{rgb}{1.000000,1.000000,0.929412}%
\pgfsetfillcolor{currentfill}%
\pgfsetlinewidth{0.250937pt}%
\definecolor{currentstroke}{rgb}{1.000000,1.000000,1.000000}%
\pgfsetstrokecolor{currentstroke}%
\pgfsetdash{}{0pt}%
\pgfpathmoveto{\pgfqpoint{4.416792in}{4.272925in}}%
\pgfpathlineto{\pgfqpoint{4.504528in}{4.272925in}}%
\pgfpathlineto{\pgfqpoint{4.504528in}{4.185189in}}%
\pgfpathlineto{\pgfqpoint{4.416792in}{4.185189in}}%
\pgfpathlineto{\pgfqpoint{4.416792in}{4.272925in}}%
\pgfusepath{stroke,fill}%
\end{pgfscope}%
\begin{pgfscope}%
\pgfpathrectangle{\pgfqpoint{0.380943in}{4.185189in}}{\pgfqpoint{4.650000in}{0.614151in}}%
\pgfusepath{clip}%
\pgfsetbuttcap%
\pgfsetroundjoin%
\definecolor{currentfill}{rgb}{0.991849,0.986144,0.810181}%
\pgfsetfillcolor{currentfill}%
\pgfsetlinewidth{0.250937pt}%
\definecolor{currentstroke}{rgb}{1.000000,1.000000,1.000000}%
\pgfsetstrokecolor{currentstroke}%
\pgfsetdash{}{0pt}%
\pgfpathmoveto{\pgfqpoint{4.504528in}{4.272925in}}%
\pgfpathlineto{\pgfqpoint{4.592264in}{4.272925in}}%
\pgfpathlineto{\pgfqpoint{4.592264in}{4.185189in}}%
\pgfpathlineto{\pgfqpoint{4.504528in}{4.185189in}}%
\pgfpathlineto{\pgfqpoint{4.504528in}{4.272925in}}%
\pgfusepath{stroke,fill}%
\end{pgfscope}%
\begin{pgfscope}%
\pgfpathrectangle{\pgfqpoint{0.380943in}{4.185189in}}{\pgfqpoint{4.650000in}{0.614151in}}%
\pgfusepath{clip}%
\pgfsetbuttcap%
\pgfsetroundjoin%
\definecolor{currentfill}{rgb}{1.000000,1.000000,0.929412}%
\pgfsetfillcolor{currentfill}%
\pgfsetlinewidth{0.250937pt}%
\definecolor{currentstroke}{rgb}{1.000000,1.000000,1.000000}%
\pgfsetstrokecolor{currentstroke}%
\pgfsetdash{}{0pt}%
\pgfpathmoveto{\pgfqpoint{4.592264in}{4.272925in}}%
\pgfpathlineto{\pgfqpoint{4.680000in}{4.272925in}}%
\pgfpathlineto{\pgfqpoint{4.680000in}{4.185189in}}%
\pgfpathlineto{\pgfqpoint{4.592264in}{4.185189in}}%
\pgfpathlineto{\pgfqpoint{4.592264in}{4.272925in}}%
\pgfusepath{stroke,fill}%
\end{pgfscope}%
\begin{pgfscope}%
\pgfpathrectangle{\pgfqpoint{0.380943in}{4.185189in}}{\pgfqpoint{4.650000in}{0.614151in}}%
\pgfusepath{clip}%
\pgfsetbuttcap%
\pgfsetroundjoin%
\definecolor{currentfill}{rgb}{0.962414,0.923552,0.722891}%
\pgfsetfillcolor{currentfill}%
\pgfsetlinewidth{0.250937pt}%
\definecolor{currentstroke}{rgb}{1.000000,1.000000,1.000000}%
\pgfsetstrokecolor{currentstroke}%
\pgfsetdash{}{0pt}%
\pgfpathmoveto{\pgfqpoint{4.680000in}{4.272925in}}%
\pgfpathlineto{\pgfqpoint{4.767736in}{4.272925in}}%
\pgfpathlineto{\pgfqpoint{4.767736in}{4.185189in}}%
\pgfpathlineto{\pgfqpoint{4.680000in}{4.185189in}}%
\pgfpathlineto{\pgfqpoint{4.680000in}{4.272925in}}%
\pgfusepath{stroke,fill}%
\end{pgfscope}%
\begin{pgfscope}%
\pgfpathrectangle{\pgfqpoint{0.380943in}{4.185189in}}{\pgfqpoint{4.650000in}{0.614151in}}%
\pgfusepath{clip}%
\pgfsetbuttcap%
\pgfsetroundjoin%
\definecolor{currentfill}{rgb}{0.962414,0.923552,0.722891}%
\pgfsetfillcolor{currentfill}%
\pgfsetlinewidth{0.250937pt}%
\definecolor{currentstroke}{rgb}{1.000000,1.000000,1.000000}%
\pgfsetstrokecolor{currentstroke}%
\pgfsetdash{}{0pt}%
\pgfpathmoveto{\pgfqpoint{4.767736in}{4.272925in}}%
\pgfpathlineto{\pgfqpoint{4.855471in}{4.272925in}}%
\pgfpathlineto{\pgfqpoint{4.855471in}{4.185189in}}%
\pgfpathlineto{\pgfqpoint{4.767736in}{4.185189in}}%
\pgfpathlineto{\pgfqpoint{4.767736in}{4.272925in}}%
\pgfusepath{stroke,fill}%
\end{pgfscope}%
\begin{pgfscope}%
\pgfpathrectangle{\pgfqpoint{0.380943in}{4.185189in}}{\pgfqpoint{4.650000in}{0.614151in}}%
\pgfusepath{clip}%
\pgfsetbuttcap%
\pgfsetroundjoin%
\definecolor{currentfill}{rgb}{0.991849,0.986144,0.810181}%
\pgfsetfillcolor{currentfill}%
\pgfsetlinewidth{0.250937pt}%
\definecolor{currentstroke}{rgb}{1.000000,1.000000,1.000000}%
\pgfsetstrokecolor{currentstroke}%
\pgfsetdash{}{0pt}%
\pgfpathmoveto{\pgfqpoint{4.855471in}{4.272925in}}%
\pgfpathlineto{\pgfqpoint{4.943207in}{4.272925in}}%
\pgfpathlineto{\pgfqpoint{4.943207in}{4.185189in}}%
\pgfpathlineto{\pgfqpoint{4.855471in}{4.185189in}}%
\pgfpathlineto{\pgfqpoint{4.855471in}{4.272925in}}%
\pgfusepath{stroke,fill}%
\end{pgfscope}%
\begin{pgfscope}%
\pgfpathrectangle{\pgfqpoint{0.380943in}{4.185189in}}{\pgfqpoint{4.650000in}{0.614151in}}%
\pgfusepath{clip}%
\pgfsetbuttcap%
\pgfsetroundjoin%
\pgfsetlinewidth{0.250937pt}%
\definecolor{currentstroke}{rgb}{1.000000,1.000000,1.000000}%
\pgfsetstrokecolor{currentstroke}%
\pgfsetdash{}{0pt}%
\pgfpathmoveto{\pgfqpoint{4.943207in}{4.272925in}}%
\pgfpathlineto{\pgfqpoint{5.030943in}{4.272925in}}%
\pgfpathlineto{\pgfqpoint{5.030943in}{4.185189in}}%
\pgfpathlineto{\pgfqpoint{4.943207in}{4.185189in}}%
\pgfpathlineto{\pgfqpoint{4.943207in}{4.272925in}}%
\pgfusepath{stroke}%
\end{pgfscope}%
\begin{pgfscope}%
\pgfsetbuttcap%
\pgfsetroundjoin%
\definecolor{currentfill}{rgb}{0.000000,0.000000,0.000000}%
\pgfsetfillcolor{currentfill}%
\pgfsetlinewidth{0.803000pt}%
\definecolor{currentstroke}{rgb}{0.000000,0.000000,0.000000}%
\pgfsetstrokecolor{currentstroke}%
\pgfsetdash{}{0pt}%
\pgfsys@defobject{currentmarker}{\pgfqpoint{0.000000in}{-0.048611in}}{\pgfqpoint{0.000000in}{0.000000in}}{%
\pgfpathmoveto{\pgfqpoint{0.000000in}{0.000000in}}%
\pgfpathlineto{\pgfqpoint{0.000000in}{-0.048611in}}%
\pgfusepath{stroke,fill}%
}%
\begin{pgfscope}%
\pgfsys@transformshift{0.600283in}{4.185189in}%
\pgfsys@useobject{currentmarker}{}%
\end{pgfscope}%
\end{pgfscope}%
\begin{pgfscope}%
\definecolor{textcolor}{rgb}{0.000000,0.000000,0.000000}%
\pgfsetstrokecolor{textcolor}%
\pgfsetfillcolor{textcolor}%
\pgftext[x=0.600283in,y=4.087967in,,top]{\color{textcolor}\rmfamily\fontsize{8.000000}{9.600000}\selectfont Jan}%
\end{pgfscope}%
\begin{pgfscope}%
\pgfsetbuttcap%
\pgfsetroundjoin%
\definecolor{currentfill}{rgb}{0.000000,0.000000,0.000000}%
\pgfsetfillcolor{currentfill}%
\pgfsetlinewidth{0.803000pt}%
\definecolor{currentstroke}{rgb}{0.000000,0.000000,0.000000}%
\pgfsetstrokecolor{currentstroke}%
\pgfsetdash{}{0pt}%
\pgfsys@defobject{currentmarker}{\pgfqpoint{0.000000in}{-0.048611in}}{\pgfqpoint{0.000000in}{0.000000in}}{%
\pgfpathmoveto{\pgfqpoint{0.000000in}{0.000000in}}%
\pgfpathlineto{\pgfqpoint{0.000000in}{-0.048611in}}%
\pgfusepath{stroke,fill}%
}%
\begin{pgfscope}%
\pgfsys@transformshift{0.951226in}{4.185189in}%
\pgfsys@useobject{currentmarker}{}%
\end{pgfscope}%
\end{pgfscope}%
\begin{pgfscope}%
\definecolor{textcolor}{rgb}{0.000000,0.000000,0.000000}%
\pgfsetstrokecolor{textcolor}%
\pgfsetfillcolor{textcolor}%
\pgftext[x=0.951226in,y=4.087967in,,top]{\color{textcolor}\rmfamily\fontsize{8.000000}{9.600000}\selectfont Feb}%
\end{pgfscope}%
\begin{pgfscope}%
\pgfsetbuttcap%
\pgfsetroundjoin%
\definecolor{currentfill}{rgb}{0.000000,0.000000,0.000000}%
\pgfsetfillcolor{currentfill}%
\pgfsetlinewidth{0.803000pt}%
\definecolor{currentstroke}{rgb}{0.000000,0.000000,0.000000}%
\pgfsetstrokecolor{currentstroke}%
\pgfsetdash{}{0pt}%
\pgfsys@defobject{currentmarker}{\pgfqpoint{0.000000in}{-0.048611in}}{\pgfqpoint{0.000000in}{0.000000in}}{%
\pgfpathmoveto{\pgfqpoint{0.000000in}{0.000000in}}%
\pgfpathlineto{\pgfqpoint{0.000000in}{-0.048611in}}%
\pgfusepath{stroke,fill}%
}%
\begin{pgfscope}%
\pgfsys@transformshift{1.302169in}{4.185189in}%
\pgfsys@useobject{currentmarker}{}%
\end{pgfscope}%
\end{pgfscope}%
\begin{pgfscope}%
\definecolor{textcolor}{rgb}{0.000000,0.000000,0.000000}%
\pgfsetstrokecolor{textcolor}%
\pgfsetfillcolor{textcolor}%
\pgftext[x=1.302169in,y=4.087967in,,top]{\color{textcolor}\rmfamily\fontsize{8.000000}{9.600000}\selectfont Mar}%
\end{pgfscope}%
\begin{pgfscope}%
\pgfsetbuttcap%
\pgfsetroundjoin%
\definecolor{currentfill}{rgb}{0.000000,0.000000,0.000000}%
\pgfsetfillcolor{currentfill}%
\pgfsetlinewidth{0.803000pt}%
\definecolor{currentstroke}{rgb}{0.000000,0.000000,0.000000}%
\pgfsetstrokecolor{currentstroke}%
\pgfsetdash{}{0pt}%
\pgfsys@defobject{currentmarker}{\pgfqpoint{0.000000in}{-0.048611in}}{\pgfqpoint{0.000000in}{0.000000in}}{%
\pgfpathmoveto{\pgfqpoint{0.000000in}{0.000000in}}%
\pgfpathlineto{\pgfqpoint{0.000000in}{-0.048611in}}%
\pgfusepath{stroke,fill}%
}%
\begin{pgfscope}%
\pgfsys@transformshift{1.740849in}{4.185189in}%
\pgfsys@useobject{currentmarker}{}%
\end{pgfscope}%
\end{pgfscope}%
\begin{pgfscope}%
\definecolor{textcolor}{rgb}{0.000000,0.000000,0.000000}%
\pgfsetstrokecolor{textcolor}%
\pgfsetfillcolor{textcolor}%
\pgftext[x=1.740849in,y=4.087967in,,top]{\color{textcolor}\rmfamily\fontsize{8.000000}{9.600000}\selectfont Apr}%
\end{pgfscope}%
\begin{pgfscope}%
\pgfsetbuttcap%
\pgfsetroundjoin%
\definecolor{currentfill}{rgb}{0.000000,0.000000,0.000000}%
\pgfsetfillcolor{currentfill}%
\pgfsetlinewidth{0.803000pt}%
\definecolor{currentstroke}{rgb}{0.000000,0.000000,0.000000}%
\pgfsetstrokecolor{currentstroke}%
\pgfsetdash{}{0pt}%
\pgfsys@defobject{currentmarker}{\pgfqpoint{0.000000in}{-0.048611in}}{\pgfqpoint{0.000000in}{0.000000in}}{%
\pgfpathmoveto{\pgfqpoint{0.000000in}{0.000000in}}%
\pgfpathlineto{\pgfqpoint{0.000000in}{-0.048611in}}%
\pgfusepath{stroke,fill}%
}%
\begin{pgfscope}%
\pgfsys@transformshift{2.091792in}{4.185189in}%
\pgfsys@useobject{currentmarker}{}%
\end{pgfscope}%
\end{pgfscope}%
\begin{pgfscope}%
\definecolor{textcolor}{rgb}{0.000000,0.000000,0.000000}%
\pgfsetstrokecolor{textcolor}%
\pgfsetfillcolor{textcolor}%
\pgftext[x=2.091792in,y=4.087967in,,top]{\color{textcolor}\rmfamily\fontsize{8.000000}{9.600000}\selectfont May}%
\end{pgfscope}%
\begin{pgfscope}%
\pgfsetbuttcap%
\pgfsetroundjoin%
\definecolor{currentfill}{rgb}{0.000000,0.000000,0.000000}%
\pgfsetfillcolor{currentfill}%
\pgfsetlinewidth{0.803000pt}%
\definecolor{currentstroke}{rgb}{0.000000,0.000000,0.000000}%
\pgfsetstrokecolor{currentstroke}%
\pgfsetdash{}{0pt}%
\pgfsys@defobject{currentmarker}{\pgfqpoint{0.000000in}{-0.048611in}}{\pgfqpoint{0.000000in}{0.000000in}}{%
\pgfpathmoveto{\pgfqpoint{0.000000in}{0.000000in}}%
\pgfpathlineto{\pgfqpoint{0.000000in}{-0.048611in}}%
\pgfusepath{stroke,fill}%
}%
\begin{pgfscope}%
\pgfsys@transformshift{2.442736in}{4.185189in}%
\pgfsys@useobject{currentmarker}{}%
\end{pgfscope}%
\end{pgfscope}%
\begin{pgfscope}%
\definecolor{textcolor}{rgb}{0.000000,0.000000,0.000000}%
\pgfsetstrokecolor{textcolor}%
\pgfsetfillcolor{textcolor}%
\pgftext[x=2.442736in,y=4.087967in,,top]{\color{textcolor}\rmfamily\fontsize{8.000000}{9.600000}\selectfont Jun}%
\end{pgfscope}%
\begin{pgfscope}%
\pgfsetbuttcap%
\pgfsetroundjoin%
\definecolor{currentfill}{rgb}{0.000000,0.000000,0.000000}%
\pgfsetfillcolor{currentfill}%
\pgfsetlinewidth{0.803000pt}%
\definecolor{currentstroke}{rgb}{0.000000,0.000000,0.000000}%
\pgfsetstrokecolor{currentstroke}%
\pgfsetdash{}{0pt}%
\pgfsys@defobject{currentmarker}{\pgfqpoint{0.000000in}{-0.048611in}}{\pgfqpoint{0.000000in}{0.000000in}}{%
\pgfpathmoveto{\pgfqpoint{0.000000in}{0.000000in}}%
\pgfpathlineto{\pgfqpoint{0.000000in}{-0.048611in}}%
\pgfusepath{stroke,fill}%
}%
\begin{pgfscope}%
\pgfsys@transformshift{2.881415in}{4.185189in}%
\pgfsys@useobject{currentmarker}{}%
\end{pgfscope}%
\end{pgfscope}%
\begin{pgfscope}%
\definecolor{textcolor}{rgb}{0.000000,0.000000,0.000000}%
\pgfsetstrokecolor{textcolor}%
\pgfsetfillcolor{textcolor}%
\pgftext[x=2.881415in,y=4.087967in,,top]{\color{textcolor}\rmfamily\fontsize{8.000000}{9.600000}\selectfont Jul}%
\end{pgfscope}%
\begin{pgfscope}%
\pgfsetbuttcap%
\pgfsetroundjoin%
\definecolor{currentfill}{rgb}{0.000000,0.000000,0.000000}%
\pgfsetfillcolor{currentfill}%
\pgfsetlinewidth{0.803000pt}%
\definecolor{currentstroke}{rgb}{0.000000,0.000000,0.000000}%
\pgfsetstrokecolor{currentstroke}%
\pgfsetdash{}{0pt}%
\pgfsys@defobject{currentmarker}{\pgfqpoint{0.000000in}{-0.048611in}}{\pgfqpoint{0.000000in}{0.000000in}}{%
\pgfpathmoveto{\pgfqpoint{0.000000in}{0.000000in}}%
\pgfpathlineto{\pgfqpoint{0.000000in}{-0.048611in}}%
\pgfusepath{stroke,fill}%
}%
\begin{pgfscope}%
\pgfsys@transformshift{3.232358in}{4.185189in}%
\pgfsys@useobject{currentmarker}{}%
\end{pgfscope}%
\end{pgfscope}%
\begin{pgfscope}%
\definecolor{textcolor}{rgb}{0.000000,0.000000,0.000000}%
\pgfsetstrokecolor{textcolor}%
\pgfsetfillcolor{textcolor}%
\pgftext[x=3.232358in,y=4.087967in,,top]{\color{textcolor}\rmfamily\fontsize{8.000000}{9.600000}\selectfont Aug}%
\end{pgfscope}%
\begin{pgfscope}%
\pgfsetbuttcap%
\pgfsetroundjoin%
\definecolor{currentfill}{rgb}{0.000000,0.000000,0.000000}%
\pgfsetfillcolor{currentfill}%
\pgfsetlinewidth{0.803000pt}%
\definecolor{currentstroke}{rgb}{0.000000,0.000000,0.000000}%
\pgfsetstrokecolor{currentstroke}%
\pgfsetdash{}{0pt}%
\pgfsys@defobject{currentmarker}{\pgfqpoint{0.000000in}{-0.048611in}}{\pgfqpoint{0.000000in}{0.000000in}}{%
\pgfpathmoveto{\pgfqpoint{0.000000in}{0.000000in}}%
\pgfpathlineto{\pgfqpoint{0.000000in}{-0.048611in}}%
\pgfusepath{stroke,fill}%
}%
\begin{pgfscope}%
\pgfsys@transformshift{3.627169in}{4.185189in}%
\pgfsys@useobject{currentmarker}{}%
\end{pgfscope}%
\end{pgfscope}%
\begin{pgfscope}%
\definecolor{textcolor}{rgb}{0.000000,0.000000,0.000000}%
\pgfsetstrokecolor{textcolor}%
\pgfsetfillcolor{textcolor}%
\pgftext[x=3.627169in,y=4.087967in,,top]{\color{textcolor}\rmfamily\fontsize{8.000000}{9.600000}\selectfont Sep}%
\end{pgfscope}%
\begin{pgfscope}%
\pgfsetbuttcap%
\pgfsetroundjoin%
\definecolor{currentfill}{rgb}{0.000000,0.000000,0.000000}%
\pgfsetfillcolor{currentfill}%
\pgfsetlinewidth{0.803000pt}%
\definecolor{currentstroke}{rgb}{0.000000,0.000000,0.000000}%
\pgfsetstrokecolor{currentstroke}%
\pgfsetdash{}{0pt}%
\pgfsys@defobject{currentmarker}{\pgfqpoint{0.000000in}{-0.048611in}}{\pgfqpoint{0.000000in}{0.000000in}}{%
\pgfpathmoveto{\pgfqpoint{0.000000in}{0.000000in}}%
\pgfpathlineto{\pgfqpoint{0.000000in}{-0.048611in}}%
\pgfusepath{stroke,fill}%
}%
\begin{pgfscope}%
\pgfsys@transformshift{4.021981in}{4.185189in}%
\pgfsys@useobject{currentmarker}{}%
\end{pgfscope}%
\end{pgfscope}%
\begin{pgfscope}%
\definecolor{textcolor}{rgb}{0.000000,0.000000,0.000000}%
\pgfsetstrokecolor{textcolor}%
\pgfsetfillcolor{textcolor}%
\pgftext[x=4.021981in,y=4.087967in,,top]{\color{textcolor}\rmfamily\fontsize{8.000000}{9.600000}\selectfont Oct}%
\end{pgfscope}%
\begin{pgfscope}%
\pgfsetbuttcap%
\pgfsetroundjoin%
\definecolor{currentfill}{rgb}{0.000000,0.000000,0.000000}%
\pgfsetfillcolor{currentfill}%
\pgfsetlinewidth{0.803000pt}%
\definecolor{currentstroke}{rgb}{0.000000,0.000000,0.000000}%
\pgfsetstrokecolor{currentstroke}%
\pgfsetdash{}{0pt}%
\pgfsys@defobject{currentmarker}{\pgfqpoint{0.000000in}{-0.048611in}}{\pgfqpoint{0.000000in}{0.000000in}}{%
\pgfpathmoveto{\pgfqpoint{0.000000in}{0.000000in}}%
\pgfpathlineto{\pgfqpoint{0.000000in}{-0.048611in}}%
\pgfusepath{stroke,fill}%
}%
\begin{pgfscope}%
\pgfsys@transformshift{4.372924in}{4.185189in}%
\pgfsys@useobject{currentmarker}{}%
\end{pgfscope}%
\end{pgfscope}%
\begin{pgfscope}%
\definecolor{textcolor}{rgb}{0.000000,0.000000,0.000000}%
\pgfsetstrokecolor{textcolor}%
\pgfsetfillcolor{textcolor}%
\pgftext[x=4.372924in,y=4.087967in,,top]{\color{textcolor}\rmfamily\fontsize{8.000000}{9.600000}\selectfont Nov}%
\end{pgfscope}%
\begin{pgfscope}%
\pgfsetbuttcap%
\pgfsetroundjoin%
\definecolor{currentfill}{rgb}{0.000000,0.000000,0.000000}%
\pgfsetfillcolor{currentfill}%
\pgfsetlinewidth{0.803000pt}%
\definecolor{currentstroke}{rgb}{0.000000,0.000000,0.000000}%
\pgfsetstrokecolor{currentstroke}%
\pgfsetdash{}{0pt}%
\pgfsys@defobject{currentmarker}{\pgfqpoint{0.000000in}{-0.048611in}}{\pgfqpoint{0.000000in}{0.000000in}}{%
\pgfpathmoveto{\pgfqpoint{0.000000in}{0.000000in}}%
\pgfpathlineto{\pgfqpoint{0.000000in}{-0.048611in}}%
\pgfusepath{stroke,fill}%
}%
\begin{pgfscope}%
\pgfsys@transformshift{4.767736in}{4.185189in}%
\pgfsys@useobject{currentmarker}{}%
\end{pgfscope}%
\end{pgfscope}%
\begin{pgfscope}%
\definecolor{textcolor}{rgb}{0.000000,0.000000,0.000000}%
\pgfsetstrokecolor{textcolor}%
\pgfsetfillcolor{textcolor}%
\pgftext[x=4.767736in,y=4.087967in,,top]{\color{textcolor}\rmfamily\fontsize{8.000000}{9.600000}\selectfont Dec}%
\end{pgfscope}%
\begin{pgfscope}%
\pgfsetbuttcap%
\pgfsetroundjoin%
\definecolor{currentfill}{rgb}{0.000000,0.000000,0.000000}%
\pgfsetfillcolor{currentfill}%
\pgfsetlinewidth{0.803000pt}%
\definecolor{currentstroke}{rgb}{0.000000,0.000000,0.000000}%
\pgfsetstrokecolor{currentstroke}%
\pgfsetdash{}{0pt}%
\pgfsys@defobject{currentmarker}{\pgfqpoint{-0.048611in}{0.000000in}}{\pgfqpoint{-0.000000in}{0.000000in}}{%
\pgfpathmoveto{\pgfqpoint{-0.000000in}{0.000000in}}%
\pgfpathlineto{\pgfqpoint{-0.048611in}{0.000000in}}%
\pgfusepath{stroke,fill}%
}%
\begin{pgfscope}%
\pgfsys@transformshift{0.380943in}{4.755472in}%
\pgfsys@useobject{currentmarker}{}%
\end{pgfscope}%
\end{pgfscope}%
\begin{pgfscope}%
\definecolor{textcolor}{rgb}{0.000000,0.000000,0.000000}%
\pgfsetstrokecolor{textcolor}%
\pgfsetfillcolor{textcolor}%
\pgftext[x=0.113117in, y=4.716892in, left, base]{\color{textcolor}\rmfamily\fontsize{8.000000}{9.600000}\selectfont M}%
\end{pgfscope}%
\begin{pgfscope}%
\pgfsetbuttcap%
\pgfsetroundjoin%
\definecolor{currentfill}{rgb}{0.000000,0.000000,0.000000}%
\pgfsetfillcolor{currentfill}%
\pgfsetlinewidth{0.803000pt}%
\definecolor{currentstroke}{rgb}{0.000000,0.000000,0.000000}%
\pgfsetstrokecolor{currentstroke}%
\pgfsetdash{}{0pt}%
\pgfsys@defobject{currentmarker}{\pgfqpoint{-0.048611in}{0.000000in}}{\pgfqpoint{-0.000000in}{0.000000in}}{%
\pgfpathmoveto{\pgfqpoint{-0.000000in}{0.000000in}}%
\pgfpathlineto{\pgfqpoint{-0.048611in}{0.000000in}}%
\pgfusepath{stroke,fill}%
}%
\begin{pgfscope}%
\pgfsys@transformshift{0.380943in}{4.667736in}%
\pgfsys@useobject{currentmarker}{}%
\end{pgfscope}%
\end{pgfscope}%
\begin{pgfscope}%
\definecolor{textcolor}{rgb}{0.000000,0.000000,0.000000}%
\pgfsetstrokecolor{textcolor}%
\pgfsetfillcolor{textcolor}%
\pgftext[x=0.135957in, y=4.629156in, left, base]{\color{textcolor}\rmfamily\fontsize{8.000000}{9.600000}\selectfont T}%
\end{pgfscope}%
\begin{pgfscope}%
\pgfsetbuttcap%
\pgfsetroundjoin%
\definecolor{currentfill}{rgb}{0.000000,0.000000,0.000000}%
\pgfsetfillcolor{currentfill}%
\pgfsetlinewidth{0.803000pt}%
\definecolor{currentstroke}{rgb}{0.000000,0.000000,0.000000}%
\pgfsetstrokecolor{currentstroke}%
\pgfsetdash{}{0pt}%
\pgfsys@defobject{currentmarker}{\pgfqpoint{-0.048611in}{0.000000in}}{\pgfqpoint{-0.000000in}{0.000000in}}{%
\pgfpathmoveto{\pgfqpoint{-0.000000in}{0.000000in}}%
\pgfpathlineto{\pgfqpoint{-0.048611in}{0.000000in}}%
\pgfusepath{stroke,fill}%
}%
\begin{pgfscope}%
\pgfsys@transformshift{0.380943in}{4.580000in}%
\pgfsys@useobject{currentmarker}{}%
\end{pgfscope}%
\end{pgfscope}%
\begin{pgfscope}%
\definecolor{textcolor}{rgb}{0.000000,0.000000,0.000000}%
\pgfsetstrokecolor{textcolor}%
\pgfsetfillcolor{textcolor}%
\pgftext[x=0.100000in, y=4.541420in, left, base]{\color{textcolor}\rmfamily\fontsize{8.000000}{9.600000}\selectfont W}%
\end{pgfscope}%
\begin{pgfscope}%
\pgfsetbuttcap%
\pgfsetroundjoin%
\definecolor{currentfill}{rgb}{0.000000,0.000000,0.000000}%
\pgfsetfillcolor{currentfill}%
\pgfsetlinewidth{0.803000pt}%
\definecolor{currentstroke}{rgb}{0.000000,0.000000,0.000000}%
\pgfsetstrokecolor{currentstroke}%
\pgfsetdash{}{0pt}%
\pgfsys@defobject{currentmarker}{\pgfqpoint{-0.048611in}{0.000000in}}{\pgfqpoint{-0.000000in}{0.000000in}}{%
\pgfpathmoveto{\pgfqpoint{-0.000000in}{0.000000in}}%
\pgfpathlineto{\pgfqpoint{-0.048611in}{0.000000in}}%
\pgfusepath{stroke,fill}%
}%
\begin{pgfscope}%
\pgfsys@transformshift{0.380943in}{4.492264in}%
\pgfsys@useobject{currentmarker}{}%
\end{pgfscope}%
\end{pgfscope}%
\begin{pgfscope}%
\definecolor{textcolor}{rgb}{0.000000,0.000000,0.000000}%
\pgfsetstrokecolor{textcolor}%
\pgfsetfillcolor{textcolor}%
\pgftext[x=0.135957in, y=4.453684in, left, base]{\color{textcolor}\rmfamily\fontsize{8.000000}{9.600000}\selectfont T}%
\end{pgfscope}%
\begin{pgfscope}%
\pgfsetbuttcap%
\pgfsetroundjoin%
\definecolor{currentfill}{rgb}{0.000000,0.000000,0.000000}%
\pgfsetfillcolor{currentfill}%
\pgfsetlinewidth{0.803000pt}%
\definecolor{currentstroke}{rgb}{0.000000,0.000000,0.000000}%
\pgfsetstrokecolor{currentstroke}%
\pgfsetdash{}{0pt}%
\pgfsys@defobject{currentmarker}{\pgfqpoint{-0.048611in}{0.000000in}}{\pgfqpoint{-0.000000in}{0.000000in}}{%
\pgfpathmoveto{\pgfqpoint{-0.000000in}{0.000000in}}%
\pgfpathlineto{\pgfqpoint{-0.048611in}{0.000000in}}%
\pgfusepath{stroke,fill}%
}%
\begin{pgfscope}%
\pgfsys@transformshift{0.380943in}{4.404529in}%
\pgfsys@useobject{currentmarker}{}%
\end{pgfscope}%
\end{pgfscope}%
\begin{pgfscope}%
\definecolor{textcolor}{rgb}{0.000000,0.000000,0.000000}%
\pgfsetstrokecolor{textcolor}%
\pgfsetfillcolor{textcolor}%
\pgftext[x=0.144213in, y=4.365948in, left, base]{\color{textcolor}\rmfamily\fontsize{8.000000}{9.600000}\selectfont F}%
\end{pgfscope}%
\begin{pgfscope}%
\pgfsetbuttcap%
\pgfsetroundjoin%
\definecolor{currentfill}{rgb}{0.000000,0.000000,0.000000}%
\pgfsetfillcolor{currentfill}%
\pgfsetlinewidth{0.803000pt}%
\definecolor{currentstroke}{rgb}{0.000000,0.000000,0.000000}%
\pgfsetstrokecolor{currentstroke}%
\pgfsetdash{}{0pt}%
\pgfsys@defobject{currentmarker}{\pgfqpoint{-0.048611in}{0.000000in}}{\pgfqpoint{-0.000000in}{0.000000in}}{%
\pgfpathmoveto{\pgfqpoint{-0.000000in}{0.000000in}}%
\pgfpathlineto{\pgfqpoint{-0.048611in}{0.000000in}}%
\pgfusepath{stroke,fill}%
}%
\begin{pgfscope}%
\pgfsys@transformshift{0.380943in}{4.316793in}%
\pgfsys@useobject{currentmarker}{}%
\end{pgfscope}%
\end{pgfscope}%
\begin{pgfscope}%
\definecolor{textcolor}{rgb}{0.000000,0.000000,0.000000}%
\pgfsetstrokecolor{textcolor}%
\pgfsetfillcolor{textcolor}%
\pgftext[x=0.155633in, y=4.278212in, left, base]{\color{textcolor}\rmfamily\fontsize{8.000000}{9.600000}\selectfont S}%
\end{pgfscope}%
\begin{pgfscope}%
\pgfsetbuttcap%
\pgfsetroundjoin%
\definecolor{currentfill}{rgb}{0.000000,0.000000,0.000000}%
\pgfsetfillcolor{currentfill}%
\pgfsetlinewidth{0.803000pt}%
\definecolor{currentstroke}{rgb}{0.000000,0.000000,0.000000}%
\pgfsetstrokecolor{currentstroke}%
\pgfsetdash{}{0pt}%
\pgfsys@defobject{currentmarker}{\pgfqpoint{-0.048611in}{0.000000in}}{\pgfqpoint{-0.000000in}{0.000000in}}{%
\pgfpathmoveto{\pgfqpoint{-0.000000in}{0.000000in}}%
\pgfpathlineto{\pgfqpoint{-0.048611in}{0.000000in}}%
\pgfusepath{stroke,fill}%
}%
\begin{pgfscope}%
\pgfsys@transformshift{0.380943in}{4.229057in}%
\pgfsys@useobject{currentmarker}{}%
\end{pgfscope}%
\end{pgfscope}%
\begin{pgfscope}%
\definecolor{textcolor}{rgb}{0.000000,0.000000,0.000000}%
\pgfsetstrokecolor{textcolor}%
\pgfsetfillcolor{textcolor}%
\pgftext[x=0.155633in, y=4.190477in, left, base]{\color{textcolor}\rmfamily\fontsize{8.000000}{9.600000}\selectfont S}%
\end{pgfscope}%
\begin{pgfscope}%
\definecolor{textcolor}{rgb}{0.000000,0.000000,0.000000}%
\pgfsetstrokecolor{textcolor}%
\pgfsetfillcolor{textcolor}%
\pgftext[x=2.705943in,y=4.966007in,,]{\color{textcolor}\ttfamily\fontsize{14.400000}{17.280000}\selectfont 2019}%
\end{pgfscope}%
\begin{pgfscope}%
\pgfpathrectangle{\pgfqpoint{0.380943in}{2.260189in}}{\pgfqpoint{4.650000in}{0.614151in}}%
\pgfusepath{clip}%
\pgfsetbuttcap%
\pgfsetroundjoin%
\pgfsetlinewidth{0.250937pt}%
\definecolor{currentstroke}{rgb}{1.000000,1.000000,1.000000}%
\pgfsetstrokecolor{currentstroke}%
\pgfsetdash{}{0pt}%
\pgfpathmoveto{\pgfqpoint{0.380943in}{2.874340in}}%
\pgfpathlineto{\pgfqpoint{0.468679in}{2.874340in}}%
\pgfpathlineto{\pgfqpoint{0.468679in}{2.786604in}}%
\pgfpathlineto{\pgfqpoint{0.380943in}{2.786604in}}%
\pgfpathlineto{\pgfqpoint{0.380943in}{2.874340in}}%
\pgfusepath{stroke}%
\end{pgfscope}%
\begin{pgfscope}%
\pgfpathrectangle{\pgfqpoint{0.380943in}{2.260189in}}{\pgfqpoint{4.650000in}{0.614151in}}%
\pgfusepath{clip}%
\pgfsetbuttcap%
\pgfsetroundjoin%
\definecolor{currentfill}{rgb}{0.974072,0.862976,0.688750}%
\pgfsetfillcolor{currentfill}%
\pgfsetlinewidth{0.250937pt}%
\definecolor{currentstroke}{rgb}{1.000000,1.000000,1.000000}%
\pgfsetstrokecolor{currentstroke}%
\pgfsetdash{}{0pt}%
\pgfpathmoveto{\pgfqpoint{0.468679in}{2.874340in}}%
\pgfpathlineto{\pgfqpoint{0.556415in}{2.874340in}}%
\pgfpathlineto{\pgfqpoint{0.556415in}{2.786604in}}%
\pgfpathlineto{\pgfqpoint{0.468679in}{2.786604in}}%
\pgfpathlineto{\pgfqpoint{0.468679in}{2.874340in}}%
\pgfusepath{stroke,fill}%
\end{pgfscope}%
\begin{pgfscope}%
\pgfpathrectangle{\pgfqpoint{0.380943in}{2.260189in}}{\pgfqpoint{4.650000in}{0.614151in}}%
\pgfusepath{clip}%
\pgfsetbuttcap%
\pgfsetroundjoin%
\definecolor{currentfill}{rgb}{1.000000,0.615379,0.534779}%
\pgfsetfillcolor{currentfill}%
\pgfsetlinewidth{0.250937pt}%
\definecolor{currentstroke}{rgb}{1.000000,1.000000,1.000000}%
\pgfsetstrokecolor{currentstroke}%
\pgfsetdash{}{0pt}%
\pgfpathmoveto{\pgfqpoint{0.556415in}{2.874340in}}%
\pgfpathlineto{\pgfqpoint{0.644151in}{2.874340in}}%
\pgfpathlineto{\pgfqpoint{0.644151in}{2.786604in}}%
\pgfpathlineto{\pgfqpoint{0.556415in}{2.786604in}}%
\pgfpathlineto{\pgfqpoint{0.556415in}{2.874340in}}%
\pgfusepath{stroke,fill}%
\end{pgfscope}%
\begin{pgfscope}%
\pgfpathrectangle{\pgfqpoint{0.380943in}{2.260189in}}{\pgfqpoint{4.650000in}{0.614151in}}%
\pgfusepath{clip}%
\pgfsetbuttcap%
\pgfsetroundjoin%
\definecolor{currentfill}{rgb}{0.974072,0.862976,0.688750}%
\pgfsetfillcolor{currentfill}%
\pgfsetlinewidth{0.250937pt}%
\definecolor{currentstroke}{rgb}{1.000000,1.000000,1.000000}%
\pgfsetstrokecolor{currentstroke}%
\pgfsetdash{}{0pt}%
\pgfpathmoveto{\pgfqpoint{0.644151in}{2.874340in}}%
\pgfpathlineto{\pgfqpoint{0.731886in}{2.874340in}}%
\pgfpathlineto{\pgfqpoint{0.731886in}{2.786604in}}%
\pgfpathlineto{\pgfqpoint{0.644151in}{2.786604in}}%
\pgfpathlineto{\pgfqpoint{0.644151in}{2.874340in}}%
\pgfusepath{stroke,fill}%
\end{pgfscope}%
\begin{pgfscope}%
\pgfpathrectangle{\pgfqpoint{0.380943in}{2.260189in}}{\pgfqpoint{4.650000in}{0.614151in}}%
\pgfusepath{clip}%
\pgfsetbuttcap%
\pgfsetroundjoin%
\definecolor{currentfill}{rgb}{0.964937,0.908651,0.713110}%
\pgfsetfillcolor{currentfill}%
\pgfsetlinewidth{0.250937pt}%
\definecolor{currentstroke}{rgb}{1.000000,1.000000,1.000000}%
\pgfsetstrokecolor{currentstroke}%
\pgfsetdash{}{0pt}%
\pgfpathmoveto{\pgfqpoint{0.731886in}{2.874340in}}%
\pgfpathlineto{\pgfqpoint{0.819622in}{2.874340in}}%
\pgfpathlineto{\pgfqpoint{0.819622in}{2.786604in}}%
\pgfpathlineto{\pgfqpoint{0.731886in}{2.786604in}}%
\pgfpathlineto{\pgfqpoint{0.731886in}{2.874340in}}%
\pgfusepath{stroke,fill}%
\end{pgfscope}%
\begin{pgfscope}%
\pgfpathrectangle{\pgfqpoint{0.380943in}{2.260189in}}{\pgfqpoint{4.650000in}{0.614151in}}%
\pgfusepath{clip}%
\pgfsetbuttcap%
\pgfsetroundjoin%
\definecolor{currentfill}{rgb}{0.990634,0.779608,0.623299}%
\pgfsetfillcolor{currentfill}%
\pgfsetlinewidth{0.250937pt}%
\definecolor{currentstroke}{rgb}{1.000000,1.000000,1.000000}%
\pgfsetstrokecolor{currentstroke}%
\pgfsetdash{}{0pt}%
\pgfpathmoveto{\pgfqpoint{0.819622in}{2.874340in}}%
\pgfpathlineto{\pgfqpoint{0.907358in}{2.874340in}}%
\pgfpathlineto{\pgfqpoint{0.907358in}{2.786604in}}%
\pgfpathlineto{\pgfqpoint{0.819622in}{2.786604in}}%
\pgfpathlineto{\pgfqpoint{0.819622in}{2.874340in}}%
\pgfusepath{stroke,fill}%
\end{pgfscope}%
\begin{pgfscope}%
\pgfpathrectangle{\pgfqpoint{0.380943in}{2.260189in}}{\pgfqpoint{4.650000in}{0.614151in}}%
\pgfusepath{clip}%
\pgfsetbuttcap%
\pgfsetroundjoin%
\definecolor{currentfill}{rgb}{0.987266,0.804198,0.639170}%
\pgfsetfillcolor{currentfill}%
\pgfsetlinewidth{0.250937pt}%
\definecolor{currentstroke}{rgb}{1.000000,1.000000,1.000000}%
\pgfsetstrokecolor{currentstroke}%
\pgfsetdash{}{0pt}%
\pgfpathmoveto{\pgfqpoint{0.907358in}{2.874340in}}%
\pgfpathlineto{\pgfqpoint{0.995094in}{2.874340in}}%
\pgfpathlineto{\pgfqpoint{0.995094in}{2.786604in}}%
\pgfpathlineto{\pgfqpoint{0.907358in}{2.786604in}}%
\pgfpathlineto{\pgfqpoint{0.907358in}{2.874340in}}%
\pgfusepath{stroke,fill}%
\end{pgfscope}%
\begin{pgfscope}%
\pgfpathrectangle{\pgfqpoint{0.380943in}{2.260189in}}{\pgfqpoint{4.650000in}{0.614151in}}%
\pgfusepath{clip}%
\pgfsetbuttcap%
\pgfsetroundjoin%
\definecolor{currentfill}{rgb}{0.964937,0.908651,0.713110}%
\pgfsetfillcolor{currentfill}%
\pgfsetlinewidth{0.250937pt}%
\definecolor{currentstroke}{rgb}{1.000000,1.000000,1.000000}%
\pgfsetstrokecolor{currentstroke}%
\pgfsetdash{}{0pt}%
\pgfpathmoveto{\pgfqpoint{0.995094in}{2.874340in}}%
\pgfpathlineto{\pgfqpoint{1.082830in}{2.874340in}}%
\pgfpathlineto{\pgfqpoint{1.082830in}{2.786604in}}%
\pgfpathlineto{\pgfqpoint{0.995094in}{2.786604in}}%
\pgfpathlineto{\pgfqpoint{0.995094in}{2.874340in}}%
\pgfusepath{stroke,fill}%
\end{pgfscope}%
\begin{pgfscope}%
\pgfpathrectangle{\pgfqpoint{0.380943in}{2.260189in}}{\pgfqpoint{4.650000in}{0.614151in}}%
\pgfusepath{clip}%
\pgfsetbuttcap%
\pgfsetroundjoin%
\definecolor{currentfill}{rgb}{0.978639,0.841584,0.673679}%
\pgfsetfillcolor{currentfill}%
\pgfsetlinewidth{0.250937pt}%
\definecolor{currentstroke}{rgb}{1.000000,1.000000,1.000000}%
\pgfsetstrokecolor{currentstroke}%
\pgfsetdash{}{0pt}%
\pgfpathmoveto{\pgfqpoint{1.082830in}{2.874340in}}%
\pgfpathlineto{\pgfqpoint{1.170566in}{2.874340in}}%
\pgfpathlineto{\pgfqpoint{1.170566in}{2.786604in}}%
\pgfpathlineto{\pgfqpoint{1.082830in}{2.786604in}}%
\pgfpathlineto{\pgfqpoint{1.082830in}{2.874340in}}%
\pgfusepath{stroke,fill}%
\end{pgfscope}%
\begin{pgfscope}%
\pgfpathrectangle{\pgfqpoint{0.380943in}{2.260189in}}{\pgfqpoint{4.650000in}{0.614151in}}%
\pgfusepath{clip}%
\pgfsetbuttcap%
\pgfsetroundjoin%
\definecolor{currentfill}{rgb}{0.964937,0.908651,0.713110}%
\pgfsetfillcolor{currentfill}%
\pgfsetlinewidth{0.250937pt}%
\definecolor{currentstroke}{rgb}{1.000000,1.000000,1.000000}%
\pgfsetstrokecolor{currentstroke}%
\pgfsetdash{}{0pt}%
\pgfpathmoveto{\pgfqpoint{1.170566in}{2.874340in}}%
\pgfpathlineto{\pgfqpoint{1.258302in}{2.874340in}}%
\pgfpathlineto{\pgfqpoint{1.258302in}{2.786604in}}%
\pgfpathlineto{\pgfqpoint{1.170566in}{2.786604in}}%
\pgfpathlineto{\pgfqpoint{1.170566in}{2.874340in}}%
\pgfusepath{stroke,fill}%
\end{pgfscope}%
\begin{pgfscope}%
\pgfpathrectangle{\pgfqpoint{0.380943in}{2.260189in}}{\pgfqpoint{4.650000in}{0.614151in}}%
\pgfusepath{clip}%
\pgfsetbuttcap%
\pgfsetroundjoin%
\definecolor{currentfill}{rgb}{0.961738,0.927612,0.725598}%
\pgfsetfillcolor{currentfill}%
\pgfsetlinewidth{0.250937pt}%
\definecolor{currentstroke}{rgb}{1.000000,1.000000,1.000000}%
\pgfsetstrokecolor{currentstroke}%
\pgfsetdash{}{0pt}%
\pgfpathmoveto{\pgfqpoint{1.258302in}{2.874340in}}%
\pgfpathlineto{\pgfqpoint{1.346037in}{2.874340in}}%
\pgfpathlineto{\pgfqpoint{1.346037in}{2.786604in}}%
\pgfpathlineto{\pgfqpoint{1.258302in}{2.786604in}}%
\pgfpathlineto{\pgfqpoint{1.258302in}{2.874340in}}%
\pgfusepath{stroke,fill}%
\end{pgfscope}%
\begin{pgfscope}%
\pgfpathrectangle{\pgfqpoint{0.380943in}{2.260189in}}{\pgfqpoint{4.650000in}{0.614151in}}%
\pgfusepath{clip}%
\pgfsetbuttcap%
\pgfsetroundjoin%
\definecolor{currentfill}{rgb}{1.000000,1.000000,0.895579}%
\pgfsetfillcolor{currentfill}%
\pgfsetlinewidth{0.250937pt}%
\definecolor{currentstroke}{rgb}{1.000000,1.000000,1.000000}%
\pgfsetstrokecolor{currentstroke}%
\pgfsetdash{}{0pt}%
\pgfpathmoveto{\pgfqpoint{1.346037in}{2.874340in}}%
\pgfpathlineto{\pgfqpoint{1.433773in}{2.874340in}}%
\pgfpathlineto{\pgfqpoint{1.433773in}{2.786604in}}%
\pgfpathlineto{\pgfqpoint{1.346037in}{2.786604in}}%
\pgfpathlineto{\pgfqpoint{1.346037in}{2.874340in}}%
\pgfusepath{stroke,fill}%
\end{pgfscope}%
\begin{pgfscope}%
\pgfpathrectangle{\pgfqpoint{0.380943in}{2.260189in}}{\pgfqpoint{4.650000in}{0.614151in}}%
\pgfusepath{clip}%
\pgfsetbuttcap%
\pgfsetroundjoin%
\definecolor{currentfill}{rgb}{0.995233,0.991895,0.818977}%
\pgfsetfillcolor{currentfill}%
\pgfsetlinewidth{0.250937pt}%
\definecolor{currentstroke}{rgb}{1.000000,1.000000,1.000000}%
\pgfsetstrokecolor{currentstroke}%
\pgfsetdash{}{0pt}%
\pgfpathmoveto{\pgfqpoint{1.433773in}{2.874340in}}%
\pgfpathlineto{\pgfqpoint{1.521509in}{2.874340in}}%
\pgfpathlineto{\pgfqpoint{1.521509in}{2.786604in}}%
\pgfpathlineto{\pgfqpoint{1.433773in}{2.786604in}}%
\pgfpathlineto{\pgfqpoint{1.433773in}{2.874340in}}%
\pgfusepath{stroke,fill}%
\end{pgfscope}%
\begin{pgfscope}%
\pgfpathrectangle{\pgfqpoint{0.380943in}{2.260189in}}{\pgfqpoint{4.650000in}{0.614151in}}%
\pgfusepath{clip}%
\pgfsetbuttcap%
\pgfsetroundjoin%
\definecolor{currentfill}{rgb}{0.995233,0.991895,0.818977}%
\pgfsetfillcolor{currentfill}%
\pgfsetlinewidth{0.250937pt}%
\definecolor{currentstroke}{rgb}{1.000000,1.000000,1.000000}%
\pgfsetstrokecolor{currentstroke}%
\pgfsetdash{}{0pt}%
\pgfpathmoveto{\pgfqpoint{1.521509in}{2.874340in}}%
\pgfpathlineto{\pgfqpoint{1.609245in}{2.874340in}}%
\pgfpathlineto{\pgfqpoint{1.609245in}{2.786604in}}%
\pgfpathlineto{\pgfqpoint{1.521509in}{2.786604in}}%
\pgfpathlineto{\pgfqpoint{1.521509in}{2.874340in}}%
\pgfusepath{stroke,fill}%
\end{pgfscope}%
\begin{pgfscope}%
\pgfpathrectangle{\pgfqpoint{0.380943in}{2.260189in}}{\pgfqpoint{4.650000in}{0.614151in}}%
\pgfusepath{clip}%
\pgfsetbuttcap%
\pgfsetroundjoin%
\definecolor{currentfill}{rgb}{0.961738,0.927612,0.725598}%
\pgfsetfillcolor{currentfill}%
\pgfsetlinewidth{0.250937pt}%
\definecolor{currentstroke}{rgb}{1.000000,1.000000,1.000000}%
\pgfsetstrokecolor{currentstroke}%
\pgfsetdash{}{0pt}%
\pgfpathmoveto{\pgfqpoint{1.609245in}{2.874340in}}%
\pgfpathlineto{\pgfqpoint{1.696981in}{2.874340in}}%
\pgfpathlineto{\pgfqpoint{1.696981in}{2.786604in}}%
\pgfpathlineto{\pgfqpoint{1.609245in}{2.786604in}}%
\pgfpathlineto{\pgfqpoint{1.609245in}{2.874340in}}%
\pgfusepath{stroke,fill}%
\end{pgfscope}%
\begin{pgfscope}%
\pgfpathrectangle{\pgfqpoint{0.380943in}{2.260189in}}{\pgfqpoint{4.650000in}{0.614151in}}%
\pgfusepath{clip}%
\pgfsetbuttcap%
\pgfsetroundjoin%
\definecolor{currentfill}{rgb}{1.000000,1.000000,0.895579}%
\pgfsetfillcolor{currentfill}%
\pgfsetlinewidth{0.250937pt}%
\definecolor{currentstroke}{rgb}{1.000000,1.000000,1.000000}%
\pgfsetstrokecolor{currentstroke}%
\pgfsetdash{}{0pt}%
\pgfpathmoveto{\pgfqpoint{1.696981in}{2.874340in}}%
\pgfpathlineto{\pgfqpoint{1.784717in}{2.874340in}}%
\pgfpathlineto{\pgfqpoint{1.784717in}{2.786604in}}%
\pgfpathlineto{\pgfqpoint{1.696981in}{2.786604in}}%
\pgfpathlineto{\pgfqpoint{1.696981in}{2.874340in}}%
\pgfusepath{stroke,fill}%
\end{pgfscope}%
\begin{pgfscope}%
\pgfpathrectangle{\pgfqpoint{0.380943in}{2.260189in}}{\pgfqpoint{4.650000in}{0.614151in}}%
\pgfusepath{clip}%
\pgfsetbuttcap%
\pgfsetroundjoin%
\definecolor{currentfill}{rgb}{0.980008,0.966013,0.779393}%
\pgfsetfillcolor{currentfill}%
\pgfsetlinewidth{0.250937pt}%
\definecolor{currentstroke}{rgb}{1.000000,1.000000,1.000000}%
\pgfsetstrokecolor{currentstroke}%
\pgfsetdash{}{0pt}%
\pgfpathmoveto{\pgfqpoint{1.784717in}{2.874340in}}%
\pgfpathlineto{\pgfqpoint{1.872452in}{2.874340in}}%
\pgfpathlineto{\pgfqpoint{1.872452in}{2.786604in}}%
\pgfpathlineto{\pgfqpoint{1.784717in}{2.786604in}}%
\pgfpathlineto{\pgfqpoint{1.784717in}{2.874340in}}%
\pgfusepath{stroke,fill}%
\end{pgfscope}%
\begin{pgfscope}%
\pgfpathrectangle{\pgfqpoint{0.380943in}{2.260189in}}{\pgfqpoint{4.650000in}{0.614151in}}%
\pgfusepath{clip}%
\pgfsetbuttcap%
\pgfsetroundjoin%
\definecolor{currentfill}{rgb}{0.961738,0.927612,0.725598}%
\pgfsetfillcolor{currentfill}%
\pgfsetlinewidth{0.250937pt}%
\definecolor{currentstroke}{rgb}{1.000000,1.000000,1.000000}%
\pgfsetstrokecolor{currentstroke}%
\pgfsetdash{}{0pt}%
\pgfpathmoveto{\pgfqpoint{1.872452in}{2.874340in}}%
\pgfpathlineto{\pgfqpoint{1.960188in}{2.874340in}}%
\pgfpathlineto{\pgfqpoint{1.960188in}{2.786604in}}%
\pgfpathlineto{\pgfqpoint{1.872452in}{2.786604in}}%
\pgfpathlineto{\pgfqpoint{1.872452in}{2.874340in}}%
\pgfusepath{stroke,fill}%
\end{pgfscope}%
\begin{pgfscope}%
\pgfpathrectangle{\pgfqpoint{0.380943in}{2.260189in}}{\pgfqpoint{4.650000in}{0.614151in}}%
\pgfusepath{clip}%
\pgfsetbuttcap%
\pgfsetroundjoin%
\definecolor{currentfill}{rgb}{0.961738,0.927612,0.725598}%
\pgfsetfillcolor{currentfill}%
\pgfsetlinewidth{0.250937pt}%
\definecolor{currentstroke}{rgb}{1.000000,1.000000,1.000000}%
\pgfsetstrokecolor{currentstroke}%
\pgfsetdash{}{0pt}%
\pgfpathmoveto{\pgfqpoint{1.960188in}{2.874340in}}%
\pgfpathlineto{\pgfqpoint{2.047924in}{2.874340in}}%
\pgfpathlineto{\pgfqpoint{2.047924in}{2.786604in}}%
\pgfpathlineto{\pgfqpoint{1.960188in}{2.786604in}}%
\pgfpathlineto{\pgfqpoint{1.960188in}{2.874340in}}%
\pgfusepath{stroke,fill}%
\end{pgfscope}%
\begin{pgfscope}%
\pgfpathrectangle{\pgfqpoint{0.380943in}{2.260189in}}{\pgfqpoint{4.650000in}{0.614151in}}%
\pgfusepath{clip}%
\pgfsetbuttcap%
\pgfsetroundjoin%
\definecolor{currentfill}{rgb}{0.961738,0.927612,0.725598}%
\pgfsetfillcolor{currentfill}%
\pgfsetlinewidth{0.250937pt}%
\definecolor{currentstroke}{rgb}{1.000000,1.000000,1.000000}%
\pgfsetstrokecolor{currentstroke}%
\pgfsetdash{}{0pt}%
\pgfpathmoveto{\pgfqpoint{2.047924in}{2.874340in}}%
\pgfpathlineto{\pgfqpoint{2.135660in}{2.874340in}}%
\pgfpathlineto{\pgfqpoint{2.135660in}{2.786604in}}%
\pgfpathlineto{\pgfqpoint{2.047924in}{2.786604in}}%
\pgfpathlineto{\pgfqpoint{2.047924in}{2.874340in}}%
\pgfusepath{stroke,fill}%
\end{pgfscope}%
\begin{pgfscope}%
\pgfpathrectangle{\pgfqpoint{0.380943in}{2.260189in}}{\pgfqpoint{4.650000in}{0.614151in}}%
\pgfusepath{clip}%
\pgfsetbuttcap%
\pgfsetroundjoin%
\definecolor{currentfill}{rgb}{0.964783,0.940131,0.739808}%
\pgfsetfillcolor{currentfill}%
\pgfsetlinewidth{0.250937pt}%
\definecolor{currentstroke}{rgb}{1.000000,1.000000,1.000000}%
\pgfsetstrokecolor{currentstroke}%
\pgfsetdash{}{0pt}%
\pgfpathmoveto{\pgfqpoint{2.135660in}{2.874340in}}%
\pgfpathlineto{\pgfqpoint{2.223396in}{2.874340in}}%
\pgfpathlineto{\pgfqpoint{2.223396in}{2.786604in}}%
\pgfpathlineto{\pgfqpoint{2.135660in}{2.786604in}}%
\pgfpathlineto{\pgfqpoint{2.135660in}{2.874340in}}%
\pgfusepath{stroke,fill}%
\end{pgfscope}%
\begin{pgfscope}%
\pgfpathrectangle{\pgfqpoint{0.380943in}{2.260189in}}{\pgfqpoint{4.650000in}{0.614151in}}%
\pgfusepath{clip}%
\pgfsetbuttcap%
\pgfsetroundjoin%
\definecolor{currentfill}{rgb}{0.964937,0.908651,0.713110}%
\pgfsetfillcolor{currentfill}%
\pgfsetlinewidth{0.250937pt}%
\definecolor{currentstroke}{rgb}{1.000000,1.000000,1.000000}%
\pgfsetstrokecolor{currentstroke}%
\pgfsetdash{}{0pt}%
\pgfpathmoveto{\pgfqpoint{2.223396in}{2.874340in}}%
\pgfpathlineto{\pgfqpoint{2.311132in}{2.874340in}}%
\pgfpathlineto{\pgfqpoint{2.311132in}{2.786604in}}%
\pgfpathlineto{\pgfqpoint{2.223396in}{2.786604in}}%
\pgfpathlineto{\pgfqpoint{2.223396in}{2.874340in}}%
\pgfusepath{stroke,fill}%
\end{pgfscope}%
\begin{pgfscope}%
\pgfpathrectangle{\pgfqpoint{0.380943in}{2.260189in}}{\pgfqpoint{4.650000in}{0.614151in}}%
\pgfusepath{clip}%
\pgfsetbuttcap%
\pgfsetroundjoin%
\definecolor{currentfill}{rgb}{1.000000,1.000000,0.857516}%
\pgfsetfillcolor{currentfill}%
\pgfsetlinewidth{0.250937pt}%
\definecolor{currentstroke}{rgb}{1.000000,1.000000,1.000000}%
\pgfsetstrokecolor{currentstroke}%
\pgfsetdash{}{0pt}%
\pgfpathmoveto{\pgfqpoint{2.311132in}{2.874340in}}%
\pgfpathlineto{\pgfqpoint{2.398868in}{2.874340in}}%
\pgfpathlineto{\pgfqpoint{2.398868in}{2.786604in}}%
\pgfpathlineto{\pgfqpoint{2.311132in}{2.786604in}}%
\pgfpathlineto{\pgfqpoint{2.311132in}{2.874340in}}%
\pgfusepath{stroke,fill}%
\end{pgfscope}%
\begin{pgfscope}%
\pgfpathrectangle{\pgfqpoint{0.380943in}{2.260189in}}{\pgfqpoint{4.650000in}{0.614151in}}%
\pgfusepath{clip}%
\pgfsetbuttcap%
\pgfsetroundjoin%
\definecolor{currentfill}{rgb}{0.995233,0.991895,0.818977}%
\pgfsetfillcolor{currentfill}%
\pgfsetlinewidth{0.250937pt}%
\definecolor{currentstroke}{rgb}{1.000000,1.000000,1.000000}%
\pgfsetstrokecolor{currentstroke}%
\pgfsetdash{}{0pt}%
\pgfpathmoveto{\pgfqpoint{2.398868in}{2.874340in}}%
\pgfpathlineto{\pgfqpoint{2.486603in}{2.874340in}}%
\pgfpathlineto{\pgfqpoint{2.486603in}{2.786604in}}%
\pgfpathlineto{\pgfqpoint{2.398868in}{2.786604in}}%
\pgfpathlineto{\pgfqpoint{2.398868in}{2.874340in}}%
\pgfusepath{stroke,fill}%
\end{pgfscope}%
\begin{pgfscope}%
\pgfpathrectangle{\pgfqpoint{0.380943in}{2.260189in}}{\pgfqpoint{4.650000in}{0.614151in}}%
\pgfusepath{clip}%
\pgfsetbuttcap%
\pgfsetroundjoin%
\definecolor{currentfill}{rgb}{0.990004,0.468435,0.468435}%
\pgfsetfillcolor{currentfill}%
\pgfsetlinewidth{0.250937pt}%
\definecolor{currentstroke}{rgb}{1.000000,1.000000,1.000000}%
\pgfsetstrokecolor{currentstroke}%
\pgfsetdash{}{0pt}%
\pgfpathmoveto{\pgfqpoint{2.486603in}{2.874340in}}%
\pgfpathlineto{\pgfqpoint{2.574339in}{2.874340in}}%
\pgfpathlineto{\pgfqpoint{2.574339in}{2.786604in}}%
\pgfpathlineto{\pgfqpoint{2.486603in}{2.786604in}}%
\pgfpathlineto{\pgfqpoint{2.486603in}{2.874340in}}%
\pgfusepath{stroke,fill}%
\end{pgfscope}%
\begin{pgfscope}%
\pgfpathrectangle{\pgfqpoint{0.380943in}{2.260189in}}{\pgfqpoint{4.650000in}{0.614151in}}%
\pgfusepath{clip}%
\pgfsetbuttcap%
\pgfsetroundjoin%
\definecolor{currentfill}{rgb}{0.990634,0.779608,0.623299}%
\pgfsetfillcolor{currentfill}%
\pgfsetlinewidth{0.250937pt}%
\definecolor{currentstroke}{rgb}{1.000000,1.000000,1.000000}%
\pgfsetstrokecolor{currentstroke}%
\pgfsetdash{}{0pt}%
\pgfpathmoveto{\pgfqpoint{2.574339in}{2.874340in}}%
\pgfpathlineto{\pgfqpoint{2.662075in}{2.874340in}}%
\pgfpathlineto{\pgfqpoint{2.662075in}{2.786604in}}%
\pgfpathlineto{\pgfqpoint{2.574339in}{2.786604in}}%
\pgfpathlineto{\pgfqpoint{2.574339in}{2.874340in}}%
\pgfusepath{stroke,fill}%
\end{pgfscope}%
\begin{pgfscope}%
\pgfpathrectangle{\pgfqpoint{0.380943in}{2.260189in}}{\pgfqpoint{4.650000in}{0.614151in}}%
\pgfusepath{clip}%
\pgfsetbuttcap%
\pgfsetroundjoin%
\definecolor{currentfill}{rgb}{1.000000,0.554479,0.510419}%
\pgfsetfillcolor{currentfill}%
\pgfsetlinewidth{0.250937pt}%
\definecolor{currentstroke}{rgb}{1.000000,1.000000,1.000000}%
\pgfsetstrokecolor{currentstroke}%
\pgfsetdash{}{0pt}%
\pgfpathmoveto{\pgfqpoint{2.662075in}{2.874340in}}%
\pgfpathlineto{\pgfqpoint{2.749811in}{2.874340in}}%
\pgfpathlineto{\pgfqpoint{2.749811in}{2.786604in}}%
\pgfpathlineto{\pgfqpoint{2.662075in}{2.786604in}}%
\pgfpathlineto{\pgfqpoint{2.662075in}{2.874340in}}%
\pgfusepath{stroke,fill}%
\end{pgfscope}%
\begin{pgfscope}%
\pgfpathrectangle{\pgfqpoint{0.380943in}{2.260189in}}{\pgfqpoint{4.650000in}{0.614151in}}%
\pgfusepath{clip}%
\pgfsetbuttcap%
\pgfsetroundjoin%
\definecolor{currentfill}{rgb}{0.987266,0.804198,0.639170}%
\pgfsetfillcolor{currentfill}%
\pgfsetlinewidth{0.250937pt}%
\definecolor{currentstroke}{rgb}{1.000000,1.000000,1.000000}%
\pgfsetstrokecolor{currentstroke}%
\pgfsetdash{}{0pt}%
\pgfpathmoveto{\pgfqpoint{2.749811in}{2.874340in}}%
\pgfpathlineto{\pgfqpoint{2.837547in}{2.874340in}}%
\pgfpathlineto{\pgfqpoint{2.837547in}{2.786604in}}%
\pgfpathlineto{\pgfqpoint{2.749811in}{2.786604in}}%
\pgfpathlineto{\pgfqpoint{2.749811in}{2.874340in}}%
\pgfusepath{stroke,fill}%
\end{pgfscope}%
\begin{pgfscope}%
\pgfpathrectangle{\pgfqpoint{0.380943in}{2.260189in}}{\pgfqpoint{4.650000in}{0.614151in}}%
\pgfusepath{clip}%
\pgfsetbuttcap%
\pgfsetroundjoin%
\definecolor{currentfill}{rgb}{1.000000,0.615379,0.534779}%
\pgfsetfillcolor{currentfill}%
\pgfsetlinewidth{0.250937pt}%
\definecolor{currentstroke}{rgb}{1.000000,1.000000,1.000000}%
\pgfsetstrokecolor{currentstroke}%
\pgfsetdash{}{0pt}%
\pgfpathmoveto{\pgfqpoint{2.837547in}{2.874340in}}%
\pgfpathlineto{\pgfqpoint{2.925283in}{2.874340in}}%
\pgfpathlineto{\pgfqpoint{2.925283in}{2.786604in}}%
\pgfpathlineto{\pgfqpoint{2.837547in}{2.786604in}}%
\pgfpathlineto{\pgfqpoint{2.837547in}{2.874340in}}%
\pgfusepath{stroke,fill}%
\end{pgfscope}%
\begin{pgfscope}%
\pgfpathrectangle{\pgfqpoint{0.380943in}{2.260189in}}{\pgfqpoint{4.650000in}{0.614151in}}%
\pgfusepath{clip}%
\pgfsetbuttcap%
\pgfsetroundjoin%
\definecolor{currentfill}{rgb}{0.993679,0.753725,0.608074}%
\pgfsetfillcolor{currentfill}%
\pgfsetlinewidth{0.250937pt}%
\definecolor{currentstroke}{rgb}{1.000000,1.000000,1.000000}%
\pgfsetstrokecolor{currentstroke}%
\pgfsetdash{}{0pt}%
\pgfpathmoveto{\pgfqpoint{2.925283in}{2.874340in}}%
\pgfpathlineto{\pgfqpoint{3.013019in}{2.874340in}}%
\pgfpathlineto{\pgfqpoint{3.013019in}{2.786604in}}%
\pgfpathlineto{\pgfqpoint{2.925283in}{2.786604in}}%
\pgfpathlineto{\pgfqpoint{2.925283in}{2.874340in}}%
\pgfusepath{stroke,fill}%
\end{pgfscope}%
\begin{pgfscope}%
\pgfpathrectangle{\pgfqpoint{0.380943in}{2.260189in}}{\pgfqpoint{4.650000in}{0.614151in}}%
\pgfusepath{clip}%
\pgfsetbuttcap%
\pgfsetroundjoin%
\definecolor{currentfill}{rgb}{0.969504,0.885813,0.700930}%
\pgfsetfillcolor{currentfill}%
\pgfsetlinewidth{0.250937pt}%
\definecolor{currentstroke}{rgb}{1.000000,1.000000,1.000000}%
\pgfsetstrokecolor{currentstroke}%
\pgfsetdash{}{0pt}%
\pgfpathmoveto{\pgfqpoint{3.013019in}{2.874340in}}%
\pgfpathlineto{\pgfqpoint{3.100754in}{2.874340in}}%
\pgfpathlineto{\pgfqpoint{3.100754in}{2.786604in}}%
\pgfpathlineto{\pgfqpoint{3.013019in}{2.786604in}}%
\pgfpathlineto{\pgfqpoint{3.013019in}{2.874340in}}%
\pgfusepath{stroke,fill}%
\end{pgfscope}%
\begin{pgfscope}%
\pgfpathrectangle{\pgfqpoint{0.380943in}{2.260189in}}{\pgfqpoint{4.650000in}{0.614151in}}%
\pgfusepath{clip}%
\pgfsetbuttcap%
\pgfsetroundjoin%
\definecolor{currentfill}{rgb}{0.963260,0.918478,0.719508}%
\pgfsetfillcolor{currentfill}%
\pgfsetlinewidth{0.250937pt}%
\definecolor{currentstroke}{rgb}{1.000000,1.000000,1.000000}%
\pgfsetstrokecolor{currentstroke}%
\pgfsetdash{}{0pt}%
\pgfpathmoveto{\pgfqpoint{3.100754in}{2.874340in}}%
\pgfpathlineto{\pgfqpoint{3.188490in}{2.874340in}}%
\pgfpathlineto{\pgfqpoint{3.188490in}{2.786604in}}%
\pgfpathlineto{\pgfqpoint{3.100754in}{2.786604in}}%
\pgfpathlineto{\pgfqpoint{3.100754in}{2.874340in}}%
\pgfusepath{stroke,fill}%
\end{pgfscope}%
\begin{pgfscope}%
\pgfpathrectangle{\pgfqpoint{0.380943in}{2.260189in}}{\pgfqpoint{4.650000in}{0.614151in}}%
\pgfusepath{clip}%
\pgfsetbuttcap%
\pgfsetroundjoin%
\definecolor{currentfill}{rgb}{0.974072,0.862976,0.688750}%
\pgfsetfillcolor{currentfill}%
\pgfsetlinewidth{0.250937pt}%
\definecolor{currentstroke}{rgb}{1.000000,1.000000,1.000000}%
\pgfsetstrokecolor{currentstroke}%
\pgfsetdash{}{0pt}%
\pgfpathmoveto{\pgfqpoint{3.188490in}{2.874340in}}%
\pgfpathlineto{\pgfqpoint{3.276226in}{2.874340in}}%
\pgfpathlineto{\pgfqpoint{3.276226in}{2.786604in}}%
\pgfpathlineto{\pgfqpoint{3.188490in}{2.786604in}}%
\pgfpathlineto{\pgfqpoint{3.188490in}{2.874340in}}%
\pgfusepath{stroke,fill}%
\end{pgfscope}%
\begin{pgfscope}%
\pgfpathrectangle{\pgfqpoint{0.380943in}{2.260189in}}{\pgfqpoint{4.650000in}{0.614151in}}%
\pgfusepath{clip}%
\pgfsetbuttcap%
\pgfsetroundjoin%
\definecolor{currentfill}{rgb}{0.993679,0.753725,0.608074}%
\pgfsetfillcolor{currentfill}%
\pgfsetlinewidth{0.250937pt}%
\definecolor{currentstroke}{rgb}{1.000000,1.000000,1.000000}%
\pgfsetstrokecolor{currentstroke}%
\pgfsetdash{}{0pt}%
\pgfpathmoveto{\pgfqpoint{3.276226in}{2.874340in}}%
\pgfpathlineto{\pgfqpoint{3.363962in}{2.874340in}}%
\pgfpathlineto{\pgfqpoint{3.363962in}{2.786604in}}%
\pgfpathlineto{\pgfqpoint{3.276226in}{2.786604in}}%
\pgfpathlineto{\pgfqpoint{3.276226in}{2.874340in}}%
\pgfusepath{stroke,fill}%
\end{pgfscope}%
\begin{pgfscope}%
\pgfpathrectangle{\pgfqpoint{0.380943in}{2.260189in}}{\pgfqpoint{4.650000in}{0.614151in}}%
\pgfusepath{clip}%
\pgfsetbuttcap%
\pgfsetroundjoin%
\definecolor{currentfill}{rgb}{0.969504,0.885813,0.700930}%
\pgfsetfillcolor{currentfill}%
\pgfsetlinewidth{0.250937pt}%
\definecolor{currentstroke}{rgb}{1.000000,1.000000,1.000000}%
\pgfsetstrokecolor{currentstroke}%
\pgfsetdash{}{0pt}%
\pgfpathmoveto{\pgfqpoint{3.363962in}{2.874340in}}%
\pgfpathlineto{\pgfqpoint{3.451698in}{2.874340in}}%
\pgfpathlineto{\pgfqpoint{3.451698in}{2.786604in}}%
\pgfpathlineto{\pgfqpoint{3.363962in}{2.786604in}}%
\pgfpathlineto{\pgfqpoint{3.363962in}{2.874340in}}%
\pgfusepath{stroke,fill}%
\end{pgfscope}%
\begin{pgfscope}%
\pgfpathrectangle{\pgfqpoint{0.380943in}{2.260189in}}{\pgfqpoint{4.650000in}{0.614151in}}%
\pgfusepath{clip}%
\pgfsetbuttcap%
\pgfsetroundjoin%
\definecolor{currentfill}{rgb}{1.000000,0.584929,0.522599}%
\pgfsetfillcolor{currentfill}%
\pgfsetlinewidth{0.250937pt}%
\definecolor{currentstroke}{rgb}{1.000000,1.000000,1.000000}%
\pgfsetstrokecolor{currentstroke}%
\pgfsetdash{}{0pt}%
\pgfpathmoveto{\pgfqpoint{3.451698in}{2.874340in}}%
\pgfpathlineto{\pgfqpoint{3.539434in}{2.874340in}}%
\pgfpathlineto{\pgfqpoint{3.539434in}{2.786604in}}%
\pgfpathlineto{\pgfqpoint{3.451698in}{2.786604in}}%
\pgfpathlineto{\pgfqpoint{3.451698in}{2.874340in}}%
\pgfusepath{stroke,fill}%
\end{pgfscope}%
\begin{pgfscope}%
\pgfpathrectangle{\pgfqpoint{0.380943in}{2.260189in}}{\pgfqpoint{4.650000in}{0.614151in}}%
\pgfusepath{clip}%
\pgfsetbuttcap%
\pgfsetroundjoin%
\definecolor{currentfill}{rgb}{0.982699,0.823991,0.657439}%
\pgfsetfillcolor{currentfill}%
\pgfsetlinewidth{0.250937pt}%
\definecolor{currentstroke}{rgb}{1.000000,1.000000,1.000000}%
\pgfsetstrokecolor{currentstroke}%
\pgfsetdash{}{0pt}%
\pgfpathmoveto{\pgfqpoint{3.539434in}{2.874340in}}%
\pgfpathlineto{\pgfqpoint{3.627169in}{2.874340in}}%
\pgfpathlineto{\pgfqpoint{3.627169in}{2.786604in}}%
\pgfpathlineto{\pgfqpoint{3.539434in}{2.786604in}}%
\pgfpathlineto{\pgfqpoint{3.539434in}{2.874340in}}%
\pgfusepath{stroke,fill}%
\end{pgfscope}%
\begin{pgfscope}%
\pgfpathrectangle{\pgfqpoint{0.380943in}{2.260189in}}{\pgfqpoint{4.650000in}{0.614151in}}%
\pgfusepath{clip}%
\pgfsetbuttcap%
\pgfsetroundjoin%
\definecolor{currentfill}{rgb}{0.961738,0.927612,0.725598}%
\pgfsetfillcolor{currentfill}%
\pgfsetlinewidth{0.250937pt}%
\definecolor{currentstroke}{rgb}{1.000000,1.000000,1.000000}%
\pgfsetstrokecolor{currentstroke}%
\pgfsetdash{}{0pt}%
\pgfpathmoveto{\pgfqpoint{3.627169in}{2.874340in}}%
\pgfpathlineto{\pgfqpoint{3.714905in}{2.874340in}}%
\pgfpathlineto{\pgfqpoint{3.714905in}{2.786604in}}%
\pgfpathlineto{\pgfqpoint{3.627169in}{2.786604in}}%
\pgfpathlineto{\pgfqpoint{3.627169in}{2.874340in}}%
\pgfusepath{stroke,fill}%
\end{pgfscope}%
\begin{pgfscope}%
\pgfpathrectangle{\pgfqpoint{0.380943in}{2.260189in}}{\pgfqpoint{4.650000in}{0.614151in}}%
\pgfusepath{clip}%
\pgfsetbuttcap%
\pgfsetroundjoin%
\definecolor{currentfill}{rgb}{0.990634,0.779608,0.623299}%
\pgfsetfillcolor{currentfill}%
\pgfsetlinewidth{0.250937pt}%
\definecolor{currentstroke}{rgb}{1.000000,1.000000,1.000000}%
\pgfsetstrokecolor{currentstroke}%
\pgfsetdash{}{0pt}%
\pgfpathmoveto{\pgfqpoint{3.714905in}{2.874340in}}%
\pgfpathlineto{\pgfqpoint{3.802641in}{2.874340in}}%
\pgfpathlineto{\pgfqpoint{3.802641in}{2.786604in}}%
\pgfpathlineto{\pgfqpoint{3.714905in}{2.786604in}}%
\pgfpathlineto{\pgfqpoint{3.714905in}{2.874340in}}%
\pgfusepath{stroke,fill}%
\end{pgfscope}%
\begin{pgfscope}%
\pgfpathrectangle{\pgfqpoint{0.380943in}{2.260189in}}{\pgfqpoint{4.650000in}{0.614151in}}%
\pgfusepath{clip}%
\pgfsetbuttcap%
\pgfsetroundjoin%
\definecolor{currentfill}{rgb}{0.982699,0.823991,0.657439}%
\pgfsetfillcolor{currentfill}%
\pgfsetlinewidth{0.250937pt}%
\definecolor{currentstroke}{rgb}{1.000000,1.000000,1.000000}%
\pgfsetstrokecolor{currentstroke}%
\pgfsetdash{}{0pt}%
\pgfpathmoveto{\pgfqpoint{3.802641in}{2.874340in}}%
\pgfpathlineto{\pgfqpoint{3.890377in}{2.874340in}}%
\pgfpathlineto{\pgfqpoint{3.890377in}{2.786604in}}%
\pgfpathlineto{\pgfqpoint{3.802641in}{2.786604in}}%
\pgfpathlineto{\pgfqpoint{3.802641in}{2.874340in}}%
\pgfusepath{stroke,fill}%
\end{pgfscope}%
\begin{pgfscope}%
\pgfpathrectangle{\pgfqpoint{0.380943in}{2.260189in}}{\pgfqpoint{4.650000in}{0.614151in}}%
\pgfusepath{clip}%
\pgfsetbuttcap%
\pgfsetroundjoin%
\definecolor{currentfill}{rgb}{0.987266,0.804198,0.639170}%
\pgfsetfillcolor{currentfill}%
\pgfsetlinewidth{0.250937pt}%
\definecolor{currentstroke}{rgb}{1.000000,1.000000,1.000000}%
\pgfsetstrokecolor{currentstroke}%
\pgfsetdash{}{0pt}%
\pgfpathmoveto{\pgfqpoint{3.890377in}{2.874340in}}%
\pgfpathlineto{\pgfqpoint{3.978113in}{2.874340in}}%
\pgfpathlineto{\pgfqpoint{3.978113in}{2.786604in}}%
\pgfpathlineto{\pgfqpoint{3.890377in}{2.786604in}}%
\pgfpathlineto{\pgfqpoint{3.890377in}{2.874340in}}%
\pgfusepath{stroke,fill}%
\end{pgfscope}%
\begin{pgfscope}%
\pgfpathrectangle{\pgfqpoint{0.380943in}{2.260189in}}{\pgfqpoint{4.650000in}{0.614151in}}%
\pgfusepath{clip}%
\pgfsetbuttcap%
\pgfsetroundjoin%
\definecolor{currentfill}{rgb}{0.982699,0.823991,0.657439}%
\pgfsetfillcolor{currentfill}%
\pgfsetlinewidth{0.250937pt}%
\definecolor{currentstroke}{rgb}{1.000000,1.000000,1.000000}%
\pgfsetstrokecolor{currentstroke}%
\pgfsetdash{}{0pt}%
\pgfpathmoveto{\pgfqpoint{3.978113in}{2.874340in}}%
\pgfpathlineto{\pgfqpoint{4.065849in}{2.874340in}}%
\pgfpathlineto{\pgfqpoint{4.065849in}{2.786604in}}%
\pgfpathlineto{\pgfqpoint{3.978113in}{2.786604in}}%
\pgfpathlineto{\pgfqpoint{3.978113in}{2.874340in}}%
\pgfusepath{stroke,fill}%
\end{pgfscope}%
\begin{pgfscope}%
\pgfpathrectangle{\pgfqpoint{0.380943in}{2.260189in}}{\pgfqpoint{4.650000in}{0.614151in}}%
\pgfusepath{clip}%
\pgfsetbuttcap%
\pgfsetroundjoin%
\definecolor{currentfill}{rgb}{0.978639,0.841584,0.673679}%
\pgfsetfillcolor{currentfill}%
\pgfsetlinewidth{0.250937pt}%
\definecolor{currentstroke}{rgb}{1.000000,1.000000,1.000000}%
\pgfsetstrokecolor{currentstroke}%
\pgfsetdash{}{0pt}%
\pgfpathmoveto{\pgfqpoint{4.065849in}{2.874340in}}%
\pgfpathlineto{\pgfqpoint{4.153585in}{2.874340in}}%
\pgfpathlineto{\pgfqpoint{4.153585in}{2.786604in}}%
\pgfpathlineto{\pgfqpoint{4.065849in}{2.786604in}}%
\pgfpathlineto{\pgfqpoint{4.065849in}{2.874340in}}%
\pgfusepath{stroke,fill}%
\end{pgfscope}%
\begin{pgfscope}%
\pgfpathrectangle{\pgfqpoint{0.380943in}{2.260189in}}{\pgfqpoint{4.650000in}{0.614151in}}%
\pgfusepath{clip}%
\pgfsetbuttcap%
\pgfsetroundjoin%
\definecolor{currentfill}{rgb}{0.978639,0.841584,0.673679}%
\pgfsetfillcolor{currentfill}%
\pgfsetlinewidth{0.250937pt}%
\definecolor{currentstroke}{rgb}{1.000000,1.000000,1.000000}%
\pgfsetstrokecolor{currentstroke}%
\pgfsetdash{}{0pt}%
\pgfpathmoveto{\pgfqpoint{4.153585in}{2.874340in}}%
\pgfpathlineto{\pgfqpoint{4.241320in}{2.874340in}}%
\pgfpathlineto{\pgfqpoint{4.241320in}{2.786604in}}%
\pgfpathlineto{\pgfqpoint{4.153585in}{2.786604in}}%
\pgfpathlineto{\pgfqpoint{4.153585in}{2.874340in}}%
\pgfusepath{stroke,fill}%
\end{pgfscope}%
\begin{pgfscope}%
\pgfpathrectangle{\pgfqpoint{0.380943in}{2.260189in}}{\pgfqpoint{4.650000in}{0.614151in}}%
\pgfusepath{clip}%
\pgfsetbuttcap%
\pgfsetroundjoin%
\definecolor{currentfill}{rgb}{0.997924,0.685352,0.570242}%
\pgfsetfillcolor{currentfill}%
\pgfsetlinewidth{0.250937pt}%
\definecolor{currentstroke}{rgb}{1.000000,1.000000,1.000000}%
\pgfsetstrokecolor{currentstroke}%
\pgfsetdash{}{0pt}%
\pgfpathmoveto{\pgfqpoint{4.241320in}{2.874340in}}%
\pgfpathlineto{\pgfqpoint{4.329056in}{2.874340in}}%
\pgfpathlineto{\pgfqpoint{4.329056in}{2.786604in}}%
\pgfpathlineto{\pgfqpoint{4.241320in}{2.786604in}}%
\pgfpathlineto{\pgfqpoint{4.241320in}{2.874340in}}%
\pgfusepath{stroke,fill}%
\end{pgfscope}%
\begin{pgfscope}%
\pgfpathrectangle{\pgfqpoint{0.380943in}{2.260189in}}{\pgfqpoint{4.650000in}{0.614151in}}%
\pgfusepath{clip}%
\pgfsetbuttcap%
\pgfsetroundjoin%
\definecolor{currentfill}{rgb}{0.987266,0.804198,0.639170}%
\pgfsetfillcolor{currentfill}%
\pgfsetlinewidth{0.250937pt}%
\definecolor{currentstroke}{rgb}{1.000000,1.000000,1.000000}%
\pgfsetstrokecolor{currentstroke}%
\pgfsetdash{}{0pt}%
\pgfpathmoveto{\pgfqpoint{4.329056in}{2.874340in}}%
\pgfpathlineto{\pgfqpoint{4.416792in}{2.874340in}}%
\pgfpathlineto{\pgfqpoint{4.416792in}{2.786604in}}%
\pgfpathlineto{\pgfqpoint{4.329056in}{2.786604in}}%
\pgfpathlineto{\pgfqpoint{4.329056in}{2.874340in}}%
\pgfusepath{stroke,fill}%
\end{pgfscope}%
\begin{pgfscope}%
\pgfpathrectangle{\pgfqpoint{0.380943in}{2.260189in}}{\pgfqpoint{4.650000in}{0.614151in}}%
\pgfusepath{clip}%
\pgfsetbuttcap%
\pgfsetroundjoin%
\definecolor{currentfill}{rgb}{0.993679,0.753725,0.608074}%
\pgfsetfillcolor{currentfill}%
\pgfsetlinewidth{0.250937pt}%
\definecolor{currentstroke}{rgb}{1.000000,1.000000,1.000000}%
\pgfsetstrokecolor{currentstroke}%
\pgfsetdash{}{0pt}%
\pgfpathmoveto{\pgfqpoint{4.416792in}{2.874340in}}%
\pgfpathlineto{\pgfqpoint{4.504528in}{2.874340in}}%
\pgfpathlineto{\pgfqpoint{4.504528in}{2.786604in}}%
\pgfpathlineto{\pgfqpoint{4.416792in}{2.786604in}}%
\pgfpathlineto{\pgfqpoint{4.416792in}{2.874340in}}%
\pgfusepath{stroke,fill}%
\end{pgfscope}%
\begin{pgfscope}%
\pgfpathrectangle{\pgfqpoint{0.380943in}{2.260189in}}{\pgfqpoint{4.650000in}{0.614151in}}%
\pgfusepath{clip}%
\pgfsetbuttcap%
\pgfsetroundjoin%
\definecolor{currentfill}{rgb}{0.987266,0.804198,0.639170}%
\pgfsetfillcolor{currentfill}%
\pgfsetlinewidth{0.250937pt}%
\definecolor{currentstroke}{rgb}{1.000000,1.000000,1.000000}%
\pgfsetstrokecolor{currentstroke}%
\pgfsetdash{}{0pt}%
\pgfpathmoveto{\pgfqpoint{4.504528in}{2.874340in}}%
\pgfpathlineto{\pgfqpoint{4.592264in}{2.874340in}}%
\pgfpathlineto{\pgfqpoint{4.592264in}{2.786604in}}%
\pgfpathlineto{\pgfqpoint{4.504528in}{2.786604in}}%
\pgfpathlineto{\pgfqpoint{4.504528in}{2.874340in}}%
\pgfusepath{stroke,fill}%
\end{pgfscope}%
\begin{pgfscope}%
\pgfpathrectangle{\pgfqpoint{0.380943in}{2.260189in}}{\pgfqpoint{4.650000in}{0.614151in}}%
\pgfusepath{clip}%
\pgfsetbuttcap%
\pgfsetroundjoin%
\definecolor{currentfill}{rgb}{0.964937,0.908651,0.713110}%
\pgfsetfillcolor{currentfill}%
\pgfsetlinewidth{0.250937pt}%
\definecolor{currentstroke}{rgb}{1.000000,1.000000,1.000000}%
\pgfsetstrokecolor{currentstroke}%
\pgfsetdash{}{0pt}%
\pgfpathmoveto{\pgfqpoint{4.592264in}{2.874340in}}%
\pgfpathlineto{\pgfqpoint{4.680000in}{2.874340in}}%
\pgfpathlineto{\pgfqpoint{4.680000in}{2.786604in}}%
\pgfpathlineto{\pgfqpoint{4.592264in}{2.786604in}}%
\pgfpathlineto{\pgfqpoint{4.592264in}{2.874340in}}%
\pgfusepath{stroke,fill}%
\end{pgfscope}%
\begin{pgfscope}%
\pgfpathrectangle{\pgfqpoint{0.380943in}{2.260189in}}{\pgfqpoint{4.650000in}{0.614151in}}%
\pgfusepath{clip}%
\pgfsetbuttcap%
\pgfsetroundjoin%
\definecolor{currentfill}{rgb}{0.987266,0.804198,0.639170}%
\pgfsetfillcolor{currentfill}%
\pgfsetlinewidth{0.250937pt}%
\definecolor{currentstroke}{rgb}{1.000000,1.000000,1.000000}%
\pgfsetstrokecolor{currentstroke}%
\pgfsetdash{}{0pt}%
\pgfpathmoveto{\pgfqpoint{4.680000in}{2.874340in}}%
\pgfpathlineto{\pgfqpoint{4.767736in}{2.874340in}}%
\pgfpathlineto{\pgfqpoint{4.767736in}{2.786604in}}%
\pgfpathlineto{\pgfqpoint{4.680000in}{2.786604in}}%
\pgfpathlineto{\pgfqpoint{4.680000in}{2.874340in}}%
\pgfusepath{stroke,fill}%
\end{pgfscope}%
\begin{pgfscope}%
\pgfpathrectangle{\pgfqpoint{0.380943in}{2.260189in}}{\pgfqpoint{4.650000in}{0.614151in}}%
\pgfusepath{clip}%
\pgfsetbuttcap%
\pgfsetroundjoin%
\definecolor{currentfill}{rgb}{0.978639,0.841584,0.673679}%
\pgfsetfillcolor{currentfill}%
\pgfsetlinewidth{0.250937pt}%
\definecolor{currentstroke}{rgb}{1.000000,1.000000,1.000000}%
\pgfsetstrokecolor{currentstroke}%
\pgfsetdash{}{0pt}%
\pgfpathmoveto{\pgfqpoint{4.767736in}{2.874340in}}%
\pgfpathlineto{\pgfqpoint{4.855471in}{2.874340in}}%
\pgfpathlineto{\pgfqpoint{4.855471in}{2.786604in}}%
\pgfpathlineto{\pgfqpoint{4.767736in}{2.786604in}}%
\pgfpathlineto{\pgfqpoint{4.767736in}{2.874340in}}%
\pgfusepath{stroke,fill}%
\end{pgfscope}%
\begin{pgfscope}%
\pgfpathrectangle{\pgfqpoint{0.380943in}{2.260189in}}{\pgfqpoint{4.650000in}{0.614151in}}%
\pgfusepath{clip}%
\pgfsetbuttcap%
\pgfsetroundjoin%
\definecolor{currentfill}{rgb}{0.969504,0.885813,0.700930}%
\pgfsetfillcolor{currentfill}%
\pgfsetlinewidth{0.250937pt}%
\definecolor{currentstroke}{rgb}{1.000000,1.000000,1.000000}%
\pgfsetstrokecolor{currentstroke}%
\pgfsetdash{}{0pt}%
\pgfpathmoveto{\pgfqpoint{4.855471in}{2.874340in}}%
\pgfpathlineto{\pgfqpoint{4.943207in}{2.874340in}}%
\pgfpathlineto{\pgfqpoint{4.943207in}{2.786604in}}%
\pgfpathlineto{\pgfqpoint{4.855471in}{2.786604in}}%
\pgfpathlineto{\pgfqpoint{4.855471in}{2.874340in}}%
\pgfusepath{stroke,fill}%
\end{pgfscope}%
\begin{pgfscope}%
\pgfpathrectangle{\pgfqpoint{0.380943in}{2.260189in}}{\pgfqpoint{4.650000in}{0.614151in}}%
\pgfusepath{clip}%
\pgfsetbuttcap%
\pgfsetroundjoin%
\definecolor{currentfill}{rgb}{0.987266,0.804198,0.639170}%
\pgfsetfillcolor{currentfill}%
\pgfsetlinewidth{0.250937pt}%
\definecolor{currentstroke}{rgb}{1.000000,1.000000,1.000000}%
\pgfsetstrokecolor{currentstroke}%
\pgfsetdash{}{0pt}%
\pgfpathmoveto{\pgfqpoint{4.943207in}{2.874340in}}%
\pgfpathlineto{\pgfqpoint{5.030943in}{2.874340in}}%
\pgfpathlineto{\pgfqpoint{5.030943in}{2.786604in}}%
\pgfpathlineto{\pgfqpoint{4.943207in}{2.786604in}}%
\pgfpathlineto{\pgfqpoint{4.943207in}{2.874340in}}%
\pgfusepath{stroke,fill}%
\end{pgfscope}%
\begin{pgfscope}%
\pgfpathrectangle{\pgfqpoint{0.380943in}{2.260189in}}{\pgfqpoint{4.650000in}{0.614151in}}%
\pgfusepath{clip}%
\pgfsetbuttcap%
\pgfsetroundjoin%
\pgfsetlinewidth{0.250937pt}%
\definecolor{currentstroke}{rgb}{1.000000,1.000000,1.000000}%
\pgfsetstrokecolor{currentstroke}%
\pgfsetdash{}{0pt}%
\pgfpathmoveto{\pgfqpoint{0.380943in}{2.786604in}}%
\pgfpathlineto{\pgfqpoint{0.468679in}{2.786604in}}%
\pgfpathlineto{\pgfqpoint{0.468679in}{2.698868in}}%
\pgfpathlineto{\pgfqpoint{0.380943in}{2.698868in}}%
\pgfpathlineto{\pgfqpoint{0.380943in}{2.786604in}}%
\pgfusepath{stroke}%
\end{pgfscope}%
\begin{pgfscope}%
\pgfpathrectangle{\pgfqpoint{0.380943in}{2.260189in}}{\pgfqpoint{4.650000in}{0.614151in}}%
\pgfusepath{clip}%
\pgfsetbuttcap%
\pgfsetroundjoin%
\definecolor{currentfill}{rgb}{0.996401,0.724937,0.591557}%
\pgfsetfillcolor{currentfill}%
\pgfsetlinewidth{0.250937pt}%
\definecolor{currentstroke}{rgb}{1.000000,1.000000,1.000000}%
\pgfsetstrokecolor{currentstroke}%
\pgfsetdash{}{0pt}%
\pgfpathmoveto{\pgfqpoint{0.468679in}{2.786604in}}%
\pgfpathlineto{\pgfqpoint{0.556415in}{2.786604in}}%
\pgfpathlineto{\pgfqpoint{0.556415in}{2.698868in}}%
\pgfpathlineto{\pgfqpoint{0.468679in}{2.698868in}}%
\pgfpathlineto{\pgfqpoint{0.468679in}{2.786604in}}%
\pgfusepath{stroke,fill}%
\end{pgfscope}%
\begin{pgfscope}%
\pgfpathrectangle{\pgfqpoint{0.380943in}{2.260189in}}{\pgfqpoint{4.650000in}{0.614151in}}%
\pgfusepath{clip}%
\pgfsetbuttcap%
\pgfsetroundjoin%
\definecolor{currentfill}{rgb}{0.978639,0.841584,0.673679}%
\pgfsetfillcolor{currentfill}%
\pgfsetlinewidth{0.250937pt}%
\definecolor{currentstroke}{rgb}{1.000000,1.000000,1.000000}%
\pgfsetstrokecolor{currentstroke}%
\pgfsetdash{}{0pt}%
\pgfpathmoveto{\pgfqpoint{0.556415in}{2.786604in}}%
\pgfpathlineto{\pgfqpoint{0.644151in}{2.786604in}}%
\pgfpathlineto{\pgfqpoint{0.644151in}{2.698868in}}%
\pgfpathlineto{\pgfqpoint{0.556415in}{2.698868in}}%
\pgfpathlineto{\pgfqpoint{0.556415in}{2.786604in}}%
\pgfusepath{stroke,fill}%
\end{pgfscope}%
\begin{pgfscope}%
\pgfpathrectangle{\pgfqpoint{0.380943in}{2.260189in}}{\pgfqpoint{4.650000in}{0.614151in}}%
\pgfusepath{clip}%
\pgfsetbuttcap%
\pgfsetroundjoin%
\definecolor{currentfill}{rgb}{0.964937,0.908651,0.713110}%
\pgfsetfillcolor{currentfill}%
\pgfsetlinewidth{0.250937pt}%
\definecolor{currentstroke}{rgb}{1.000000,1.000000,1.000000}%
\pgfsetstrokecolor{currentstroke}%
\pgfsetdash{}{0pt}%
\pgfpathmoveto{\pgfqpoint{0.644151in}{2.786604in}}%
\pgfpathlineto{\pgfqpoint{0.731886in}{2.786604in}}%
\pgfpathlineto{\pgfqpoint{0.731886in}{2.698868in}}%
\pgfpathlineto{\pgfqpoint{0.644151in}{2.698868in}}%
\pgfpathlineto{\pgfqpoint{0.644151in}{2.786604in}}%
\pgfusepath{stroke,fill}%
\end{pgfscope}%
\begin{pgfscope}%
\pgfpathrectangle{\pgfqpoint{0.380943in}{2.260189in}}{\pgfqpoint{4.650000in}{0.614151in}}%
\pgfusepath{clip}%
\pgfsetbuttcap%
\pgfsetroundjoin%
\definecolor{currentfill}{rgb}{0.963260,0.918478,0.719508}%
\pgfsetfillcolor{currentfill}%
\pgfsetlinewidth{0.250937pt}%
\definecolor{currentstroke}{rgb}{1.000000,1.000000,1.000000}%
\pgfsetstrokecolor{currentstroke}%
\pgfsetdash{}{0pt}%
\pgfpathmoveto{\pgfqpoint{0.731886in}{2.786604in}}%
\pgfpathlineto{\pgfqpoint{0.819622in}{2.786604in}}%
\pgfpathlineto{\pgfqpoint{0.819622in}{2.698868in}}%
\pgfpathlineto{\pgfqpoint{0.731886in}{2.698868in}}%
\pgfpathlineto{\pgfqpoint{0.731886in}{2.786604in}}%
\pgfusepath{stroke,fill}%
\end{pgfscope}%
\begin{pgfscope}%
\pgfpathrectangle{\pgfqpoint{0.380943in}{2.260189in}}{\pgfqpoint{4.650000in}{0.614151in}}%
\pgfusepath{clip}%
\pgfsetbuttcap%
\pgfsetroundjoin%
\definecolor{currentfill}{rgb}{0.964937,0.908651,0.713110}%
\pgfsetfillcolor{currentfill}%
\pgfsetlinewidth{0.250937pt}%
\definecolor{currentstroke}{rgb}{1.000000,1.000000,1.000000}%
\pgfsetstrokecolor{currentstroke}%
\pgfsetdash{}{0pt}%
\pgfpathmoveto{\pgfqpoint{0.819622in}{2.786604in}}%
\pgfpathlineto{\pgfqpoint{0.907358in}{2.786604in}}%
\pgfpathlineto{\pgfqpoint{0.907358in}{2.698868in}}%
\pgfpathlineto{\pgfqpoint{0.819622in}{2.698868in}}%
\pgfpathlineto{\pgfqpoint{0.819622in}{2.786604in}}%
\pgfusepath{stroke,fill}%
\end{pgfscope}%
\begin{pgfscope}%
\pgfpathrectangle{\pgfqpoint{0.380943in}{2.260189in}}{\pgfqpoint{4.650000in}{0.614151in}}%
\pgfusepath{clip}%
\pgfsetbuttcap%
\pgfsetroundjoin%
\definecolor{currentfill}{rgb}{0.996401,0.724937,0.591557}%
\pgfsetfillcolor{currentfill}%
\pgfsetlinewidth{0.250937pt}%
\definecolor{currentstroke}{rgb}{1.000000,1.000000,1.000000}%
\pgfsetstrokecolor{currentstroke}%
\pgfsetdash{}{0pt}%
\pgfpathmoveto{\pgfqpoint{0.907358in}{2.786604in}}%
\pgfpathlineto{\pgfqpoint{0.995094in}{2.786604in}}%
\pgfpathlineto{\pgfqpoint{0.995094in}{2.698868in}}%
\pgfpathlineto{\pgfqpoint{0.907358in}{2.698868in}}%
\pgfpathlineto{\pgfqpoint{0.907358in}{2.786604in}}%
\pgfusepath{stroke,fill}%
\end{pgfscope}%
\begin{pgfscope}%
\pgfpathrectangle{\pgfqpoint{0.380943in}{2.260189in}}{\pgfqpoint{4.650000in}{0.614151in}}%
\pgfusepath{clip}%
\pgfsetbuttcap%
\pgfsetroundjoin%
\definecolor{currentfill}{rgb}{0.987266,0.804198,0.639170}%
\pgfsetfillcolor{currentfill}%
\pgfsetlinewidth{0.250937pt}%
\definecolor{currentstroke}{rgb}{1.000000,1.000000,1.000000}%
\pgfsetstrokecolor{currentstroke}%
\pgfsetdash{}{0pt}%
\pgfpathmoveto{\pgfqpoint{0.995094in}{2.786604in}}%
\pgfpathlineto{\pgfqpoint{1.082830in}{2.786604in}}%
\pgfpathlineto{\pgfqpoint{1.082830in}{2.698868in}}%
\pgfpathlineto{\pgfqpoint{0.995094in}{2.698868in}}%
\pgfpathlineto{\pgfqpoint{0.995094in}{2.786604in}}%
\pgfusepath{stroke,fill}%
\end{pgfscope}%
\begin{pgfscope}%
\pgfpathrectangle{\pgfqpoint{0.380943in}{2.260189in}}{\pgfqpoint{4.650000in}{0.614151in}}%
\pgfusepath{clip}%
\pgfsetbuttcap%
\pgfsetroundjoin%
\definecolor{currentfill}{rgb}{0.978639,0.841584,0.673679}%
\pgfsetfillcolor{currentfill}%
\pgfsetlinewidth{0.250937pt}%
\definecolor{currentstroke}{rgb}{1.000000,1.000000,1.000000}%
\pgfsetstrokecolor{currentstroke}%
\pgfsetdash{}{0pt}%
\pgfpathmoveto{\pgfqpoint{1.082830in}{2.786604in}}%
\pgfpathlineto{\pgfqpoint{1.170566in}{2.786604in}}%
\pgfpathlineto{\pgfqpoint{1.170566in}{2.698868in}}%
\pgfpathlineto{\pgfqpoint{1.082830in}{2.698868in}}%
\pgfpathlineto{\pgfqpoint{1.082830in}{2.786604in}}%
\pgfusepath{stroke,fill}%
\end{pgfscope}%
\begin{pgfscope}%
\pgfpathrectangle{\pgfqpoint{0.380943in}{2.260189in}}{\pgfqpoint{4.650000in}{0.614151in}}%
\pgfusepath{clip}%
\pgfsetbuttcap%
\pgfsetroundjoin%
\definecolor{currentfill}{rgb}{0.974072,0.862976,0.688750}%
\pgfsetfillcolor{currentfill}%
\pgfsetlinewidth{0.250937pt}%
\definecolor{currentstroke}{rgb}{1.000000,1.000000,1.000000}%
\pgfsetstrokecolor{currentstroke}%
\pgfsetdash{}{0pt}%
\pgfpathmoveto{\pgfqpoint{1.170566in}{2.786604in}}%
\pgfpathlineto{\pgfqpoint{1.258302in}{2.786604in}}%
\pgfpathlineto{\pgfqpoint{1.258302in}{2.698868in}}%
\pgfpathlineto{\pgfqpoint{1.170566in}{2.698868in}}%
\pgfpathlineto{\pgfqpoint{1.170566in}{2.786604in}}%
\pgfusepath{stroke,fill}%
\end{pgfscope}%
\begin{pgfscope}%
\pgfpathrectangle{\pgfqpoint{0.380943in}{2.260189in}}{\pgfqpoint{4.650000in}{0.614151in}}%
\pgfusepath{clip}%
\pgfsetbuttcap%
\pgfsetroundjoin%
\definecolor{currentfill}{rgb}{0.964937,0.908651,0.713110}%
\pgfsetfillcolor{currentfill}%
\pgfsetlinewidth{0.250937pt}%
\definecolor{currentstroke}{rgb}{1.000000,1.000000,1.000000}%
\pgfsetstrokecolor{currentstroke}%
\pgfsetdash{}{0pt}%
\pgfpathmoveto{\pgfqpoint{1.258302in}{2.786604in}}%
\pgfpathlineto{\pgfqpoint{1.346037in}{2.786604in}}%
\pgfpathlineto{\pgfqpoint{1.346037in}{2.698868in}}%
\pgfpathlineto{\pgfqpoint{1.258302in}{2.698868in}}%
\pgfpathlineto{\pgfqpoint{1.258302in}{2.786604in}}%
\pgfusepath{stroke,fill}%
\end{pgfscope}%
\begin{pgfscope}%
\pgfpathrectangle{\pgfqpoint{0.380943in}{2.260189in}}{\pgfqpoint{4.650000in}{0.614151in}}%
\pgfusepath{clip}%
\pgfsetbuttcap%
\pgfsetroundjoin%
\definecolor{currentfill}{rgb}{1.000000,1.000000,0.857516}%
\pgfsetfillcolor{currentfill}%
\pgfsetlinewidth{0.250937pt}%
\definecolor{currentstroke}{rgb}{1.000000,1.000000,1.000000}%
\pgfsetstrokecolor{currentstroke}%
\pgfsetdash{}{0pt}%
\pgfpathmoveto{\pgfqpoint{1.346037in}{2.786604in}}%
\pgfpathlineto{\pgfqpoint{1.433773in}{2.786604in}}%
\pgfpathlineto{\pgfqpoint{1.433773in}{2.698868in}}%
\pgfpathlineto{\pgfqpoint{1.346037in}{2.698868in}}%
\pgfpathlineto{\pgfqpoint{1.346037in}{2.786604in}}%
\pgfusepath{stroke,fill}%
\end{pgfscope}%
\begin{pgfscope}%
\pgfpathrectangle{\pgfqpoint{0.380943in}{2.260189in}}{\pgfqpoint{4.650000in}{0.614151in}}%
\pgfusepath{clip}%
\pgfsetbuttcap%
\pgfsetroundjoin%
\definecolor{currentfill}{rgb}{0.964783,0.940131,0.739808}%
\pgfsetfillcolor{currentfill}%
\pgfsetlinewidth{0.250937pt}%
\definecolor{currentstroke}{rgb}{1.000000,1.000000,1.000000}%
\pgfsetstrokecolor{currentstroke}%
\pgfsetdash{}{0pt}%
\pgfpathmoveto{\pgfqpoint{1.433773in}{2.786604in}}%
\pgfpathlineto{\pgfqpoint{1.521509in}{2.786604in}}%
\pgfpathlineto{\pgfqpoint{1.521509in}{2.698868in}}%
\pgfpathlineto{\pgfqpoint{1.433773in}{2.698868in}}%
\pgfpathlineto{\pgfqpoint{1.433773in}{2.786604in}}%
\pgfusepath{stroke,fill}%
\end{pgfscope}%
\begin{pgfscope}%
\pgfpathrectangle{\pgfqpoint{0.380943in}{2.260189in}}{\pgfqpoint{4.650000in}{0.614151in}}%
\pgfusepath{clip}%
\pgfsetbuttcap%
\pgfsetroundjoin%
\definecolor{currentfill}{rgb}{0.980008,0.966013,0.779393}%
\pgfsetfillcolor{currentfill}%
\pgfsetlinewidth{0.250937pt}%
\definecolor{currentstroke}{rgb}{1.000000,1.000000,1.000000}%
\pgfsetstrokecolor{currentstroke}%
\pgfsetdash{}{0pt}%
\pgfpathmoveto{\pgfqpoint{1.521509in}{2.786604in}}%
\pgfpathlineto{\pgfqpoint{1.609245in}{2.786604in}}%
\pgfpathlineto{\pgfqpoint{1.609245in}{2.698868in}}%
\pgfpathlineto{\pgfqpoint{1.521509in}{2.698868in}}%
\pgfpathlineto{\pgfqpoint{1.521509in}{2.786604in}}%
\pgfusepath{stroke,fill}%
\end{pgfscope}%
\begin{pgfscope}%
\pgfpathrectangle{\pgfqpoint{0.380943in}{2.260189in}}{\pgfqpoint{4.650000in}{0.614151in}}%
\pgfusepath{clip}%
\pgfsetbuttcap%
\pgfsetroundjoin%
\definecolor{currentfill}{rgb}{0.964783,0.940131,0.739808}%
\pgfsetfillcolor{currentfill}%
\pgfsetlinewidth{0.250937pt}%
\definecolor{currentstroke}{rgb}{1.000000,1.000000,1.000000}%
\pgfsetstrokecolor{currentstroke}%
\pgfsetdash{}{0pt}%
\pgfpathmoveto{\pgfqpoint{1.609245in}{2.786604in}}%
\pgfpathlineto{\pgfqpoint{1.696981in}{2.786604in}}%
\pgfpathlineto{\pgfqpoint{1.696981in}{2.698868in}}%
\pgfpathlineto{\pgfqpoint{1.609245in}{2.698868in}}%
\pgfpathlineto{\pgfqpoint{1.609245in}{2.786604in}}%
\pgfusepath{stroke,fill}%
\end{pgfscope}%
\begin{pgfscope}%
\pgfpathrectangle{\pgfqpoint{0.380943in}{2.260189in}}{\pgfqpoint{4.650000in}{0.614151in}}%
\pgfusepath{clip}%
\pgfsetbuttcap%
\pgfsetroundjoin%
\definecolor{currentfill}{rgb}{0.995233,0.991895,0.818977}%
\pgfsetfillcolor{currentfill}%
\pgfsetlinewidth{0.250937pt}%
\definecolor{currentstroke}{rgb}{1.000000,1.000000,1.000000}%
\pgfsetstrokecolor{currentstroke}%
\pgfsetdash{}{0pt}%
\pgfpathmoveto{\pgfqpoint{1.696981in}{2.786604in}}%
\pgfpathlineto{\pgfqpoint{1.784717in}{2.786604in}}%
\pgfpathlineto{\pgfqpoint{1.784717in}{2.698868in}}%
\pgfpathlineto{\pgfqpoint{1.696981in}{2.698868in}}%
\pgfpathlineto{\pgfqpoint{1.696981in}{2.786604in}}%
\pgfusepath{stroke,fill}%
\end{pgfscope}%
\begin{pgfscope}%
\pgfpathrectangle{\pgfqpoint{0.380943in}{2.260189in}}{\pgfqpoint{4.650000in}{0.614151in}}%
\pgfusepath{clip}%
\pgfsetbuttcap%
\pgfsetroundjoin%
\definecolor{currentfill}{rgb}{0.964783,0.940131,0.739808}%
\pgfsetfillcolor{currentfill}%
\pgfsetlinewidth{0.250937pt}%
\definecolor{currentstroke}{rgb}{1.000000,1.000000,1.000000}%
\pgfsetstrokecolor{currentstroke}%
\pgfsetdash{}{0pt}%
\pgfpathmoveto{\pgfqpoint{1.784717in}{2.786604in}}%
\pgfpathlineto{\pgfqpoint{1.872452in}{2.786604in}}%
\pgfpathlineto{\pgfqpoint{1.872452in}{2.698868in}}%
\pgfpathlineto{\pgfqpoint{1.784717in}{2.698868in}}%
\pgfpathlineto{\pgfqpoint{1.784717in}{2.786604in}}%
\pgfusepath{stroke,fill}%
\end{pgfscope}%
\begin{pgfscope}%
\pgfpathrectangle{\pgfqpoint{0.380943in}{2.260189in}}{\pgfqpoint{4.650000in}{0.614151in}}%
\pgfusepath{clip}%
\pgfsetbuttcap%
\pgfsetroundjoin%
\definecolor{currentfill}{rgb}{0.964937,0.908651,0.713110}%
\pgfsetfillcolor{currentfill}%
\pgfsetlinewidth{0.250937pt}%
\definecolor{currentstroke}{rgb}{1.000000,1.000000,1.000000}%
\pgfsetstrokecolor{currentstroke}%
\pgfsetdash{}{0pt}%
\pgfpathmoveto{\pgfqpoint{1.872452in}{2.786604in}}%
\pgfpathlineto{\pgfqpoint{1.960188in}{2.786604in}}%
\pgfpathlineto{\pgfqpoint{1.960188in}{2.698868in}}%
\pgfpathlineto{\pgfqpoint{1.872452in}{2.698868in}}%
\pgfpathlineto{\pgfqpoint{1.872452in}{2.786604in}}%
\pgfusepath{stroke,fill}%
\end{pgfscope}%
\begin{pgfscope}%
\pgfpathrectangle{\pgfqpoint{0.380943in}{2.260189in}}{\pgfqpoint{4.650000in}{0.614151in}}%
\pgfusepath{clip}%
\pgfsetbuttcap%
\pgfsetroundjoin%
\definecolor{currentfill}{rgb}{0.963260,0.918478,0.719508}%
\pgfsetfillcolor{currentfill}%
\pgfsetlinewidth{0.250937pt}%
\definecolor{currentstroke}{rgb}{1.000000,1.000000,1.000000}%
\pgfsetstrokecolor{currentstroke}%
\pgfsetdash{}{0pt}%
\pgfpathmoveto{\pgfqpoint{1.960188in}{2.786604in}}%
\pgfpathlineto{\pgfqpoint{2.047924in}{2.786604in}}%
\pgfpathlineto{\pgfqpoint{2.047924in}{2.698868in}}%
\pgfpathlineto{\pgfqpoint{1.960188in}{2.698868in}}%
\pgfpathlineto{\pgfqpoint{1.960188in}{2.786604in}}%
\pgfusepath{stroke,fill}%
\end{pgfscope}%
\begin{pgfscope}%
\pgfpathrectangle{\pgfqpoint{0.380943in}{2.260189in}}{\pgfqpoint{4.650000in}{0.614151in}}%
\pgfusepath{clip}%
\pgfsetbuttcap%
\pgfsetroundjoin%
\definecolor{currentfill}{rgb}{0.961738,0.927612,0.725598}%
\pgfsetfillcolor{currentfill}%
\pgfsetlinewidth{0.250937pt}%
\definecolor{currentstroke}{rgb}{1.000000,1.000000,1.000000}%
\pgfsetstrokecolor{currentstroke}%
\pgfsetdash{}{0pt}%
\pgfpathmoveto{\pgfqpoint{2.047924in}{2.786604in}}%
\pgfpathlineto{\pgfqpoint{2.135660in}{2.786604in}}%
\pgfpathlineto{\pgfqpoint{2.135660in}{2.698868in}}%
\pgfpathlineto{\pgfqpoint{2.047924in}{2.698868in}}%
\pgfpathlineto{\pgfqpoint{2.047924in}{2.786604in}}%
\pgfusepath{stroke,fill}%
\end{pgfscope}%
\begin{pgfscope}%
\pgfpathrectangle{\pgfqpoint{0.380943in}{2.260189in}}{\pgfqpoint{4.650000in}{0.614151in}}%
\pgfusepath{clip}%
\pgfsetbuttcap%
\pgfsetroundjoin%
\definecolor{currentfill}{rgb}{0.978639,0.841584,0.673679}%
\pgfsetfillcolor{currentfill}%
\pgfsetlinewidth{0.250937pt}%
\definecolor{currentstroke}{rgb}{1.000000,1.000000,1.000000}%
\pgfsetstrokecolor{currentstroke}%
\pgfsetdash{}{0pt}%
\pgfpathmoveto{\pgfqpoint{2.135660in}{2.786604in}}%
\pgfpathlineto{\pgfqpoint{2.223396in}{2.786604in}}%
\pgfpathlineto{\pgfqpoint{2.223396in}{2.698868in}}%
\pgfpathlineto{\pgfqpoint{2.135660in}{2.698868in}}%
\pgfpathlineto{\pgfqpoint{2.135660in}{2.786604in}}%
\pgfusepath{stroke,fill}%
\end{pgfscope}%
\begin{pgfscope}%
\pgfpathrectangle{\pgfqpoint{0.380943in}{2.260189in}}{\pgfqpoint{4.650000in}{0.614151in}}%
\pgfusepath{clip}%
\pgfsetbuttcap%
\pgfsetroundjoin%
\definecolor{currentfill}{rgb}{0.963260,0.918478,0.719508}%
\pgfsetfillcolor{currentfill}%
\pgfsetlinewidth{0.250937pt}%
\definecolor{currentstroke}{rgb}{1.000000,1.000000,1.000000}%
\pgfsetstrokecolor{currentstroke}%
\pgfsetdash{}{0pt}%
\pgfpathmoveto{\pgfqpoint{2.223396in}{2.786604in}}%
\pgfpathlineto{\pgfqpoint{2.311132in}{2.786604in}}%
\pgfpathlineto{\pgfqpoint{2.311132in}{2.698868in}}%
\pgfpathlineto{\pgfqpoint{2.223396in}{2.698868in}}%
\pgfpathlineto{\pgfqpoint{2.223396in}{2.786604in}}%
\pgfusepath{stroke,fill}%
\end{pgfscope}%
\begin{pgfscope}%
\pgfpathrectangle{\pgfqpoint{0.380943in}{2.260189in}}{\pgfqpoint{4.650000in}{0.614151in}}%
\pgfusepath{clip}%
\pgfsetbuttcap%
\pgfsetroundjoin%
\definecolor{currentfill}{rgb}{0.974072,0.862976,0.688750}%
\pgfsetfillcolor{currentfill}%
\pgfsetlinewidth{0.250937pt}%
\definecolor{currentstroke}{rgb}{1.000000,1.000000,1.000000}%
\pgfsetstrokecolor{currentstroke}%
\pgfsetdash{}{0pt}%
\pgfpathmoveto{\pgfqpoint{2.311132in}{2.786604in}}%
\pgfpathlineto{\pgfqpoint{2.398868in}{2.786604in}}%
\pgfpathlineto{\pgfqpoint{2.398868in}{2.698868in}}%
\pgfpathlineto{\pgfqpoint{2.311132in}{2.698868in}}%
\pgfpathlineto{\pgfqpoint{2.311132in}{2.786604in}}%
\pgfusepath{stroke,fill}%
\end{pgfscope}%
\begin{pgfscope}%
\pgfpathrectangle{\pgfqpoint{0.380943in}{2.260189in}}{\pgfqpoint{4.650000in}{0.614151in}}%
\pgfusepath{clip}%
\pgfsetbuttcap%
\pgfsetroundjoin%
\definecolor{currentfill}{rgb}{0.969504,0.885813,0.700930}%
\pgfsetfillcolor{currentfill}%
\pgfsetlinewidth{0.250937pt}%
\definecolor{currentstroke}{rgb}{1.000000,1.000000,1.000000}%
\pgfsetstrokecolor{currentstroke}%
\pgfsetdash{}{0pt}%
\pgfpathmoveto{\pgfqpoint{2.398868in}{2.786604in}}%
\pgfpathlineto{\pgfqpoint{2.486603in}{2.786604in}}%
\pgfpathlineto{\pgfqpoint{2.486603in}{2.698868in}}%
\pgfpathlineto{\pgfqpoint{2.398868in}{2.698868in}}%
\pgfpathlineto{\pgfqpoint{2.398868in}{2.786604in}}%
\pgfusepath{stroke,fill}%
\end{pgfscope}%
\begin{pgfscope}%
\pgfpathrectangle{\pgfqpoint{0.380943in}{2.260189in}}{\pgfqpoint{4.650000in}{0.614151in}}%
\pgfusepath{clip}%
\pgfsetbuttcap%
\pgfsetroundjoin%
\definecolor{currentfill}{rgb}{0.990004,0.468435,0.468435}%
\pgfsetfillcolor{currentfill}%
\pgfsetlinewidth{0.250937pt}%
\definecolor{currentstroke}{rgb}{1.000000,1.000000,1.000000}%
\pgfsetstrokecolor{currentstroke}%
\pgfsetdash{}{0pt}%
\pgfpathmoveto{\pgfqpoint{2.486603in}{2.786604in}}%
\pgfpathlineto{\pgfqpoint{2.574339in}{2.786604in}}%
\pgfpathlineto{\pgfqpoint{2.574339in}{2.698868in}}%
\pgfpathlineto{\pgfqpoint{2.486603in}{2.698868in}}%
\pgfpathlineto{\pgfqpoint{2.486603in}{2.786604in}}%
\pgfusepath{stroke,fill}%
\end{pgfscope}%
\begin{pgfscope}%
\pgfpathrectangle{\pgfqpoint{0.380943in}{2.260189in}}{\pgfqpoint{4.650000in}{0.614151in}}%
\pgfusepath{clip}%
\pgfsetbuttcap%
\pgfsetroundjoin%
\definecolor{currentfill}{rgb}{1.000000,0.615379,0.534779}%
\pgfsetfillcolor{currentfill}%
\pgfsetlinewidth{0.250937pt}%
\definecolor{currentstroke}{rgb}{1.000000,1.000000,1.000000}%
\pgfsetstrokecolor{currentstroke}%
\pgfsetdash{}{0pt}%
\pgfpathmoveto{\pgfqpoint{2.574339in}{2.786604in}}%
\pgfpathlineto{\pgfqpoint{2.662075in}{2.786604in}}%
\pgfpathlineto{\pgfqpoint{2.662075in}{2.698868in}}%
\pgfpathlineto{\pgfqpoint{2.574339in}{2.698868in}}%
\pgfpathlineto{\pgfqpoint{2.574339in}{2.786604in}}%
\pgfusepath{stroke,fill}%
\end{pgfscope}%
\begin{pgfscope}%
\pgfpathrectangle{\pgfqpoint{0.380943in}{2.260189in}}{\pgfqpoint{4.650000in}{0.614151in}}%
\pgfusepath{clip}%
\pgfsetbuttcap%
\pgfsetroundjoin%
\definecolor{currentfill}{rgb}{1.000000,0.584929,0.522599}%
\pgfsetfillcolor{currentfill}%
\pgfsetlinewidth{0.250937pt}%
\definecolor{currentstroke}{rgb}{1.000000,1.000000,1.000000}%
\pgfsetstrokecolor{currentstroke}%
\pgfsetdash{}{0pt}%
\pgfpathmoveto{\pgfqpoint{2.662075in}{2.786604in}}%
\pgfpathlineto{\pgfqpoint{2.749811in}{2.786604in}}%
\pgfpathlineto{\pgfqpoint{2.749811in}{2.698868in}}%
\pgfpathlineto{\pgfqpoint{2.662075in}{2.698868in}}%
\pgfpathlineto{\pgfqpoint{2.662075in}{2.786604in}}%
\pgfusepath{stroke,fill}%
\end{pgfscope}%
\begin{pgfscope}%
\pgfpathrectangle{\pgfqpoint{0.380943in}{2.260189in}}{\pgfqpoint{4.650000in}{0.614151in}}%
\pgfusepath{clip}%
\pgfsetbuttcap%
\pgfsetroundjoin%
\definecolor{currentfill}{rgb}{0.987266,0.804198,0.639170}%
\pgfsetfillcolor{currentfill}%
\pgfsetlinewidth{0.250937pt}%
\definecolor{currentstroke}{rgb}{1.000000,1.000000,1.000000}%
\pgfsetstrokecolor{currentstroke}%
\pgfsetdash{}{0pt}%
\pgfpathmoveto{\pgfqpoint{2.749811in}{2.786604in}}%
\pgfpathlineto{\pgfqpoint{2.837547in}{2.786604in}}%
\pgfpathlineto{\pgfqpoint{2.837547in}{2.698868in}}%
\pgfpathlineto{\pgfqpoint{2.749811in}{2.698868in}}%
\pgfpathlineto{\pgfqpoint{2.749811in}{2.786604in}}%
\pgfusepath{stroke,fill}%
\end{pgfscope}%
\begin{pgfscope}%
\pgfpathrectangle{\pgfqpoint{0.380943in}{2.260189in}}{\pgfqpoint{4.650000in}{0.614151in}}%
\pgfusepath{clip}%
\pgfsetbuttcap%
\pgfsetroundjoin%
\definecolor{currentfill}{rgb}{0.987266,0.804198,0.639170}%
\pgfsetfillcolor{currentfill}%
\pgfsetlinewidth{0.250937pt}%
\definecolor{currentstroke}{rgb}{1.000000,1.000000,1.000000}%
\pgfsetstrokecolor{currentstroke}%
\pgfsetdash{}{0pt}%
\pgfpathmoveto{\pgfqpoint{2.837547in}{2.786604in}}%
\pgfpathlineto{\pgfqpoint{2.925283in}{2.786604in}}%
\pgfpathlineto{\pgfqpoint{2.925283in}{2.698868in}}%
\pgfpathlineto{\pgfqpoint{2.837547in}{2.698868in}}%
\pgfpathlineto{\pgfqpoint{2.837547in}{2.786604in}}%
\pgfusepath{stroke,fill}%
\end{pgfscope}%
\begin{pgfscope}%
\pgfpathrectangle{\pgfqpoint{0.380943in}{2.260189in}}{\pgfqpoint{4.650000in}{0.614151in}}%
\pgfusepath{clip}%
\pgfsetbuttcap%
\pgfsetroundjoin%
\definecolor{currentfill}{rgb}{0.996401,0.724937,0.591557}%
\pgfsetfillcolor{currentfill}%
\pgfsetlinewidth{0.250937pt}%
\definecolor{currentstroke}{rgb}{1.000000,1.000000,1.000000}%
\pgfsetstrokecolor{currentstroke}%
\pgfsetdash{}{0pt}%
\pgfpathmoveto{\pgfqpoint{2.925283in}{2.786604in}}%
\pgfpathlineto{\pgfqpoint{3.013019in}{2.786604in}}%
\pgfpathlineto{\pgfqpoint{3.013019in}{2.698868in}}%
\pgfpathlineto{\pgfqpoint{2.925283in}{2.698868in}}%
\pgfpathlineto{\pgfqpoint{2.925283in}{2.786604in}}%
\pgfusepath{stroke,fill}%
\end{pgfscope}%
\begin{pgfscope}%
\pgfpathrectangle{\pgfqpoint{0.380943in}{2.260189in}}{\pgfqpoint{4.650000in}{0.614151in}}%
\pgfusepath{clip}%
\pgfsetbuttcap%
\pgfsetroundjoin%
\definecolor{currentfill}{rgb}{0.993679,0.753725,0.608074}%
\pgfsetfillcolor{currentfill}%
\pgfsetlinewidth{0.250937pt}%
\definecolor{currentstroke}{rgb}{1.000000,1.000000,1.000000}%
\pgfsetstrokecolor{currentstroke}%
\pgfsetdash{}{0pt}%
\pgfpathmoveto{\pgfqpoint{3.013019in}{2.786604in}}%
\pgfpathlineto{\pgfqpoint{3.100754in}{2.786604in}}%
\pgfpathlineto{\pgfqpoint{3.100754in}{2.698868in}}%
\pgfpathlineto{\pgfqpoint{3.013019in}{2.698868in}}%
\pgfpathlineto{\pgfqpoint{3.013019in}{2.786604in}}%
\pgfusepath{stroke,fill}%
\end{pgfscope}%
\begin{pgfscope}%
\pgfpathrectangle{\pgfqpoint{0.380943in}{2.260189in}}{\pgfqpoint{4.650000in}{0.614151in}}%
\pgfusepath{clip}%
\pgfsetbuttcap%
\pgfsetroundjoin%
\definecolor{currentfill}{rgb}{0.961738,0.927612,0.725598}%
\pgfsetfillcolor{currentfill}%
\pgfsetlinewidth{0.250937pt}%
\definecolor{currentstroke}{rgb}{1.000000,1.000000,1.000000}%
\pgfsetstrokecolor{currentstroke}%
\pgfsetdash{}{0pt}%
\pgfpathmoveto{\pgfqpoint{3.100754in}{2.786604in}}%
\pgfpathlineto{\pgfqpoint{3.188490in}{2.786604in}}%
\pgfpathlineto{\pgfqpoint{3.188490in}{2.698868in}}%
\pgfpathlineto{\pgfqpoint{3.100754in}{2.698868in}}%
\pgfpathlineto{\pgfqpoint{3.100754in}{2.786604in}}%
\pgfusepath{stroke,fill}%
\end{pgfscope}%
\begin{pgfscope}%
\pgfpathrectangle{\pgfqpoint{0.380943in}{2.260189in}}{\pgfqpoint{4.650000in}{0.614151in}}%
\pgfusepath{clip}%
\pgfsetbuttcap%
\pgfsetroundjoin%
\definecolor{currentfill}{rgb}{0.987266,0.804198,0.639170}%
\pgfsetfillcolor{currentfill}%
\pgfsetlinewidth{0.250937pt}%
\definecolor{currentstroke}{rgb}{1.000000,1.000000,1.000000}%
\pgfsetstrokecolor{currentstroke}%
\pgfsetdash{}{0pt}%
\pgfpathmoveto{\pgfqpoint{3.188490in}{2.786604in}}%
\pgfpathlineto{\pgfqpoint{3.276226in}{2.786604in}}%
\pgfpathlineto{\pgfqpoint{3.276226in}{2.698868in}}%
\pgfpathlineto{\pgfqpoint{3.188490in}{2.698868in}}%
\pgfpathlineto{\pgfqpoint{3.188490in}{2.786604in}}%
\pgfusepath{stroke,fill}%
\end{pgfscope}%
\begin{pgfscope}%
\pgfpathrectangle{\pgfqpoint{0.380943in}{2.260189in}}{\pgfqpoint{4.650000in}{0.614151in}}%
\pgfusepath{clip}%
\pgfsetbuttcap%
\pgfsetroundjoin%
\definecolor{currentfill}{rgb}{0.969504,0.885813,0.700930}%
\pgfsetfillcolor{currentfill}%
\pgfsetlinewidth{0.250937pt}%
\definecolor{currentstroke}{rgb}{1.000000,1.000000,1.000000}%
\pgfsetstrokecolor{currentstroke}%
\pgfsetdash{}{0pt}%
\pgfpathmoveto{\pgfqpoint{3.276226in}{2.786604in}}%
\pgfpathlineto{\pgfqpoint{3.363962in}{2.786604in}}%
\pgfpathlineto{\pgfqpoint{3.363962in}{2.698868in}}%
\pgfpathlineto{\pgfqpoint{3.276226in}{2.698868in}}%
\pgfpathlineto{\pgfqpoint{3.276226in}{2.786604in}}%
\pgfusepath{stroke,fill}%
\end{pgfscope}%
\begin{pgfscope}%
\pgfpathrectangle{\pgfqpoint{0.380943in}{2.260189in}}{\pgfqpoint{4.650000in}{0.614151in}}%
\pgfusepath{clip}%
\pgfsetbuttcap%
\pgfsetroundjoin%
\definecolor{currentfill}{rgb}{0.999277,0.650165,0.551296}%
\pgfsetfillcolor{currentfill}%
\pgfsetlinewidth{0.250937pt}%
\definecolor{currentstroke}{rgb}{1.000000,1.000000,1.000000}%
\pgfsetstrokecolor{currentstroke}%
\pgfsetdash{}{0pt}%
\pgfpathmoveto{\pgfqpoint{3.363962in}{2.786604in}}%
\pgfpathlineto{\pgfqpoint{3.451698in}{2.786604in}}%
\pgfpathlineto{\pgfqpoint{3.451698in}{2.698868in}}%
\pgfpathlineto{\pgfqpoint{3.363962in}{2.698868in}}%
\pgfpathlineto{\pgfqpoint{3.363962in}{2.786604in}}%
\pgfusepath{stroke,fill}%
\end{pgfscope}%
\begin{pgfscope}%
\pgfpathrectangle{\pgfqpoint{0.380943in}{2.260189in}}{\pgfqpoint{4.650000in}{0.614151in}}%
\pgfusepath{clip}%
\pgfsetbuttcap%
\pgfsetroundjoin%
\definecolor{currentfill}{rgb}{0.974072,0.862976,0.688750}%
\pgfsetfillcolor{currentfill}%
\pgfsetlinewidth{0.250937pt}%
\definecolor{currentstroke}{rgb}{1.000000,1.000000,1.000000}%
\pgfsetstrokecolor{currentstroke}%
\pgfsetdash{}{0pt}%
\pgfpathmoveto{\pgfqpoint{3.451698in}{2.786604in}}%
\pgfpathlineto{\pgfqpoint{3.539434in}{2.786604in}}%
\pgfpathlineto{\pgfqpoint{3.539434in}{2.698868in}}%
\pgfpathlineto{\pgfqpoint{3.451698in}{2.698868in}}%
\pgfpathlineto{\pgfqpoint{3.451698in}{2.786604in}}%
\pgfusepath{stroke,fill}%
\end{pgfscope}%
\begin{pgfscope}%
\pgfpathrectangle{\pgfqpoint{0.380943in}{2.260189in}}{\pgfqpoint{4.650000in}{0.614151in}}%
\pgfusepath{clip}%
\pgfsetbuttcap%
\pgfsetroundjoin%
\definecolor{currentfill}{rgb}{0.982699,0.823991,0.657439}%
\pgfsetfillcolor{currentfill}%
\pgfsetlinewidth{0.250937pt}%
\definecolor{currentstroke}{rgb}{1.000000,1.000000,1.000000}%
\pgfsetstrokecolor{currentstroke}%
\pgfsetdash{}{0pt}%
\pgfpathmoveto{\pgfqpoint{3.539434in}{2.786604in}}%
\pgfpathlineto{\pgfqpoint{3.627169in}{2.786604in}}%
\pgfpathlineto{\pgfqpoint{3.627169in}{2.698868in}}%
\pgfpathlineto{\pgfqpoint{3.539434in}{2.698868in}}%
\pgfpathlineto{\pgfqpoint{3.539434in}{2.786604in}}%
\pgfusepath{stroke,fill}%
\end{pgfscope}%
\begin{pgfscope}%
\pgfpathrectangle{\pgfqpoint{0.380943in}{2.260189in}}{\pgfqpoint{4.650000in}{0.614151in}}%
\pgfusepath{clip}%
\pgfsetbuttcap%
\pgfsetroundjoin%
\definecolor{currentfill}{rgb}{0.990634,0.779608,0.623299}%
\pgfsetfillcolor{currentfill}%
\pgfsetlinewidth{0.250937pt}%
\definecolor{currentstroke}{rgb}{1.000000,1.000000,1.000000}%
\pgfsetstrokecolor{currentstroke}%
\pgfsetdash{}{0pt}%
\pgfpathmoveto{\pgfqpoint{3.627169in}{2.786604in}}%
\pgfpathlineto{\pgfqpoint{3.714905in}{2.786604in}}%
\pgfpathlineto{\pgfqpoint{3.714905in}{2.698868in}}%
\pgfpathlineto{\pgfqpoint{3.627169in}{2.698868in}}%
\pgfpathlineto{\pgfqpoint{3.627169in}{2.786604in}}%
\pgfusepath{stroke,fill}%
\end{pgfscope}%
\begin{pgfscope}%
\pgfpathrectangle{\pgfqpoint{0.380943in}{2.260189in}}{\pgfqpoint{4.650000in}{0.614151in}}%
\pgfusepath{clip}%
\pgfsetbuttcap%
\pgfsetroundjoin%
\definecolor{currentfill}{rgb}{0.978639,0.841584,0.673679}%
\pgfsetfillcolor{currentfill}%
\pgfsetlinewidth{0.250937pt}%
\definecolor{currentstroke}{rgb}{1.000000,1.000000,1.000000}%
\pgfsetstrokecolor{currentstroke}%
\pgfsetdash{}{0pt}%
\pgfpathmoveto{\pgfqpoint{3.714905in}{2.786604in}}%
\pgfpathlineto{\pgfqpoint{3.802641in}{2.786604in}}%
\pgfpathlineto{\pgfqpoint{3.802641in}{2.698868in}}%
\pgfpathlineto{\pgfqpoint{3.714905in}{2.698868in}}%
\pgfpathlineto{\pgfqpoint{3.714905in}{2.786604in}}%
\pgfusepath{stroke,fill}%
\end{pgfscope}%
\begin{pgfscope}%
\pgfpathrectangle{\pgfqpoint{0.380943in}{2.260189in}}{\pgfqpoint{4.650000in}{0.614151in}}%
\pgfusepath{clip}%
\pgfsetbuttcap%
\pgfsetroundjoin%
\definecolor{currentfill}{rgb}{1.000000,0.615379,0.534779}%
\pgfsetfillcolor{currentfill}%
\pgfsetlinewidth{0.250937pt}%
\definecolor{currentstroke}{rgb}{1.000000,1.000000,1.000000}%
\pgfsetstrokecolor{currentstroke}%
\pgfsetdash{}{0pt}%
\pgfpathmoveto{\pgfqpoint{3.802641in}{2.786604in}}%
\pgfpathlineto{\pgfqpoint{3.890377in}{2.786604in}}%
\pgfpathlineto{\pgfqpoint{3.890377in}{2.698868in}}%
\pgfpathlineto{\pgfqpoint{3.802641in}{2.698868in}}%
\pgfpathlineto{\pgfqpoint{3.802641in}{2.786604in}}%
\pgfusepath{stroke,fill}%
\end{pgfscope}%
\begin{pgfscope}%
\pgfpathrectangle{\pgfqpoint{0.380943in}{2.260189in}}{\pgfqpoint{4.650000in}{0.614151in}}%
\pgfusepath{clip}%
\pgfsetbuttcap%
\pgfsetroundjoin%
\definecolor{currentfill}{rgb}{0.987266,0.804198,0.639170}%
\pgfsetfillcolor{currentfill}%
\pgfsetlinewidth{0.250937pt}%
\definecolor{currentstroke}{rgb}{1.000000,1.000000,1.000000}%
\pgfsetstrokecolor{currentstroke}%
\pgfsetdash{}{0pt}%
\pgfpathmoveto{\pgfqpoint{3.890377in}{2.786604in}}%
\pgfpathlineto{\pgfqpoint{3.978113in}{2.786604in}}%
\pgfpathlineto{\pgfqpoint{3.978113in}{2.698868in}}%
\pgfpathlineto{\pgfqpoint{3.890377in}{2.698868in}}%
\pgfpathlineto{\pgfqpoint{3.890377in}{2.786604in}}%
\pgfusepath{stroke,fill}%
\end{pgfscope}%
\begin{pgfscope}%
\pgfpathrectangle{\pgfqpoint{0.380943in}{2.260189in}}{\pgfqpoint{4.650000in}{0.614151in}}%
\pgfusepath{clip}%
\pgfsetbuttcap%
\pgfsetroundjoin%
\definecolor{currentfill}{rgb}{0.990634,0.779608,0.623299}%
\pgfsetfillcolor{currentfill}%
\pgfsetlinewidth{0.250937pt}%
\definecolor{currentstroke}{rgb}{1.000000,1.000000,1.000000}%
\pgfsetstrokecolor{currentstroke}%
\pgfsetdash{}{0pt}%
\pgfpathmoveto{\pgfqpoint{3.978113in}{2.786604in}}%
\pgfpathlineto{\pgfqpoint{4.065849in}{2.786604in}}%
\pgfpathlineto{\pgfqpoint{4.065849in}{2.698868in}}%
\pgfpathlineto{\pgfqpoint{3.978113in}{2.698868in}}%
\pgfpathlineto{\pgfqpoint{3.978113in}{2.786604in}}%
\pgfusepath{stroke,fill}%
\end{pgfscope}%
\begin{pgfscope}%
\pgfpathrectangle{\pgfqpoint{0.380943in}{2.260189in}}{\pgfqpoint{4.650000in}{0.614151in}}%
\pgfusepath{clip}%
\pgfsetbuttcap%
\pgfsetroundjoin%
\definecolor{currentfill}{rgb}{0.997924,0.685352,0.570242}%
\pgfsetfillcolor{currentfill}%
\pgfsetlinewidth{0.250937pt}%
\definecolor{currentstroke}{rgb}{1.000000,1.000000,1.000000}%
\pgfsetstrokecolor{currentstroke}%
\pgfsetdash{}{0pt}%
\pgfpathmoveto{\pgfqpoint{4.065849in}{2.786604in}}%
\pgfpathlineto{\pgfqpoint{4.153585in}{2.786604in}}%
\pgfpathlineto{\pgfqpoint{4.153585in}{2.698868in}}%
\pgfpathlineto{\pgfqpoint{4.065849in}{2.698868in}}%
\pgfpathlineto{\pgfqpoint{4.065849in}{2.786604in}}%
\pgfusepath{stroke,fill}%
\end{pgfscope}%
\begin{pgfscope}%
\pgfpathrectangle{\pgfqpoint{0.380943in}{2.260189in}}{\pgfqpoint{4.650000in}{0.614151in}}%
\pgfusepath{clip}%
\pgfsetbuttcap%
\pgfsetroundjoin%
\definecolor{currentfill}{rgb}{0.999277,0.650165,0.551296}%
\pgfsetfillcolor{currentfill}%
\pgfsetlinewidth{0.250937pt}%
\definecolor{currentstroke}{rgb}{1.000000,1.000000,1.000000}%
\pgfsetstrokecolor{currentstroke}%
\pgfsetdash{}{0pt}%
\pgfpathmoveto{\pgfqpoint{4.153585in}{2.786604in}}%
\pgfpathlineto{\pgfqpoint{4.241320in}{2.786604in}}%
\pgfpathlineto{\pgfqpoint{4.241320in}{2.698868in}}%
\pgfpathlineto{\pgfqpoint{4.153585in}{2.698868in}}%
\pgfpathlineto{\pgfqpoint{4.153585in}{2.786604in}}%
\pgfusepath{stroke,fill}%
\end{pgfscope}%
\begin{pgfscope}%
\pgfpathrectangle{\pgfqpoint{0.380943in}{2.260189in}}{\pgfqpoint{4.650000in}{0.614151in}}%
\pgfusepath{clip}%
\pgfsetbuttcap%
\pgfsetroundjoin%
\definecolor{currentfill}{rgb}{0.997924,0.685352,0.570242}%
\pgfsetfillcolor{currentfill}%
\pgfsetlinewidth{0.250937pt}%
\definecolor{currentstroke}{rgb}{1.000000,1.000000,1.000000}%
\pgfsetstrokecolor{currentstroke}%
\pgfsetdash{}{0pt}%
\pgfpathmoveto{\pgfqpoint{4.241320in}{2.786604in}}%
\pgfpathlineto{\pgfqpoint{4.329056in}{2.786604in}}%
\pgfpathlineto{\pgfqpoint{4.329056in}{2.698868in}}%
\pgfpathlineto{\pgfqpoint{4.241320in}{2.698868in}}%
\pgfpathlineto{\pgfqpoint{4.241320in}{2.786604in}}%
\pgfusepath{stroke,fill}%
\end{pgfscope}%
\begin{pgfscope}%
\pgfpathrectangle{\pgfqpoint{0.380943in}{2.260189in}}{\pgfqpoint{4.650000in}{0.614151in}}%
\pgfusepath{clip}%
\pgfsetbuttcap%
\pgfsetroundjoin%
\definecolor{currentfill}{rgb}{0.990634,0.779608,0.623299}%
\pgfsetfillcolor{currentfill}%
\pgfsetlinewidth{0.250937pt}%
\definecolor{currentstroke}{rgb}{1.000000,1.000000,1.000000}%
\pgfsetstrokecolor{currentstroke}%
\pgfsetdash{}{0pt}%
\pgfpathmoveto{\pgfqpoint{4.329056in}{2.786604in}}%
\pgfpathlineto{\pgfqpoint{4.416792in}{2.786604in}}%
\pgfpathlineto{\pgfqpoint{4.416792in}{2.698868in}}%
\pgfpathlineto{\pgfqpoint{4.329056in}{2.698868in}}%
\pgfpathlineto{\pgfqpoint{4.329056in}{2.786604in}}%
\pgfusepath{stroke,fill}%
\end{pgfscope}%
\begin{pgfscope}%
\pgfpathrectangle{\pgfqpoint{0.380943in}{2.260189in}}{\pgfqpoint{4.650000in}{0.614151in}}%
\pgfusepath{clip}%
\pgfsetbuttcap%
\pgfsetroundjoin%
\definecolor{currentfill}{rgb}{0.982699,0.823991,0.657439}%
\pgfsetfillcolor{currentfill}%
\pgfsetlinewidth{0.250937pt}%
\definecolor{currentstroke}{rgb}{1.000000,1.000000,1.000000}%
\pgfsetstrokecolor{currentstroke}%
\pgfsetdash{}{0pt}%
\pgfpathmoveto{\pgfqpoint{4.416792in}{2.786604in}}%
\pgfpathlineto{\pgfqpoint{4.504528in}{2.786604in}}%
\pgfpathlineto{\pgfqpoint{4.504528in}{2.698868in}}%
\pgfpathlineto{\pgfqpoint{4.416792in}{2.698868in}}%
\pgfpathlineto{\pgfqpoint{4.416792in}{2.786604in}}%
\pgfusepath{stroke,fill}%
\end{pgfscope}%
\begin{pgfscope}%
\pgfpathrectangle{\pgfqpoint{0.380943in}{2.260189in}}{\pgfqpoint{4.650000in}{0.614151in}}%
\pgfusepath{clip}%
\pgfsetbuttcap%
\pgfsetroundjoin%
\definecolor{currentfill}{rgb}{0.974072,0.862976,0.688750}%
\pgfsetfillcolor{currentfill}%
\pgfsetlinewidth{0.250937pt}%
\definecolor{currentstroke}{rgb}{1.000000,1.000000,1.000000}%
\pgfsetstrokecolor{currentstroke}%
\pgfsetdash{}{0pt}%
\pgfpathmoveto{\pgfqpoint{4.504528in}{2.786604in}}%
\pgfpathlineto{\pgfqpoint{4.592264in}{2.786604in}}%
\pgfpathlineto{\pgfqpoint{4.592264in}{2.698868in}}%
\pgfpathlineto{\pgfqpoint{4.504528in}{2.698868in}}%
\pgfpathlineto{\pgfqpoint{4.504528in}{2.786604in}}%
\pgfusepath{stroke,fill}%
\end{pgfscope}%
\begin{pgfscope}%
\pgfpathrectangle{\pgfqpoint{0.380943in}{2.260189in}}{\pgfqpoint{4.650000in}{0.614151in}}%
\pgfusepath{clip}%
\pgfsetbuttcap%
\pgfsetroundjoin%
\definecolor{currentfill}{rgb}{0.963260,0.918478,0.719508}%
\pgfsetfillcolor{currentfill}%
\pgfsetlinewidth{0.250937pt}%
\definecolor{currentstroke}{rgb}{1.000000,1.000000,1.000000}%
\pgfsetstrokecolor{currentstroke}%
\pgfsetdash{}{0pt}%
\pgfpathmoveto{\pgfqpoint{4.592264in}{2.786604in}}%
\pgfpathlineto{\pgfqpoint{4.680000in}{2.786604in}}%
\pgfpathlineto{\pgfqpoint{4.680000in}{2.698868in}}%
\pgfpathlineto{\pgfqpoint{4.592264in}{2.698868in}}%
\pgfpathlineto{\pgfqpoint{4.592264in}{2.786604in}}%
\pgfusepath{stroke,fill}%
\end{pgfscope}%
\begin{pgfscope}%
\pgfpathrectangle{\pgfqpoint{0.380943in}{2.260189in}}{\pgfqpoint{4.650000in}{0.614151in}}%
\pgfusepath{clip}%
\pgfsetbuttcap%
\pgfsetroundjoin%
\definecolor{currentfill}{rgb}{0.997924,0.685352,0.570242}%
\pgfsetfillcolor{currentfill}%
\pgfsetlinewidth{0.250937pt}%
\definecolor{currentstroke}{rgb}{1.000000,1.000000,1.000000}%
\pgfsetstrokecolor{currentstroke}%
\pgfsetdash{}{0pt}%
\pgfpathmoveto{\pgfqpoint{4.680000in}{2.786604in}}%
\pgfpathlineto{\pgfqpoint{4.767736in}{2.786604in}}%
\pgfpathlineto{\pgfqpoint{4.767736in}{2.698868in}}%
\pgfpathlineto{\pgfqpoint{4.680000in}{2.698868in}}%
\pgfpathlineto{\pgfqpoint{4.680000in}{2.786604in}}%
\pgfusepath{stroke,fill}%
\end{pgfscope}%
\begin{pgfscope}%
\pgfpathrectangle{\pgfqpoint{0.380943in}{2.260189in}}{\pgfqpoint{4.650000in}{0.614151in}}%
\pgfusepath{clip}%
\pgfsetbuttcap%
\pgfsetroundjoin%
\definecolor{currentfill}{rgb}{0.978639,0.841584,0.673679}%
\pgfsetfillcolor{currentfill}%
\pgfsetlinewidth{0.250937pt}%
\definecolor{currentstroke}{rgb}{1.000000,1.000000,1.000000}%
\pgfsetstrokecolor{currentstroke}%
\pgfsetdash{}{0pt}%
\pgfpathmoveto{\pgfqpoint{4.767736in}{2.786604in}}%
\pgfpathlineto{\pgfqpoint{4.855471in}{2.786604in}}%
\pgfpathlineto{\pgfqpoint{4.855471in}{2.698868in}}%
\pgfpathlineto{\pgfqpoint{4.767736in}{2.698868in}}%
\pgfpathlineto{\pgfqpoint{4.767736in}{2.786604in}}%
\pgfusepath{stroke,fill}%
\end{pgfscope}%
\begin{pgfscope}%
\pgfpathrectangle{\pgfqpoint{0.380943in}{2.260189in}}{\pgfqpoint{4.650000in}{0.614151in}}%
\pgfusepath{clip}%
\pgfsetbuttcap%
\pgfsetroundjoin%
\definecolor{currentfill}{rgb}{0.974072,0.862976,0.688750}%
\pgfsetfillcolor{currentfill}%
\pgfsetlinewidth{0.250937pt}%
\definecolor{currentstroke}{rgb}{1.000000,1.000000,1.000000}%
\pgfsetstrokecolor{currentstroke}%
\pgfsetdash{}{0pt}%
\pgfpathmoveto{\pgfqpoint{4.855471in}{2.786604in}}%
\pgfpathlineto{\pgfqpoint{4.943207in}{2.786604in}}%
\pgfpathlineto{\pgfqpoint{4.943207in}{2.698868in}}%
\pgfpathlineto{\pgfqpoint{4.855471in}{2.698868in}}%
\pgfpathlineto{\pgfqpoint{4.855471in}{2.786604in}}%
\pgfusepath{stroke,fill}%
\end{pgfscope}%
\begin{pgfscope}%
\pgfpathrectangle{\pgfqpoint{0.380943in}{2.260189in}}{\pgfqpoint{4.650000in}{0.614151in}}%
\pgfusepath{clip}%
\pgfsetbuttcap%
\pgfsetroundjoin%
\definecolor{currentfill}{rgb}{0.990634,0.779608,0.623299}%
\pgfsetfillcolor{currentfill}%
\pgfsetlinewidth{0.250937pt}%
\definecolor{currentstroke}{rgb}{1.000000,1.000000,1.000000}%
\pgfsetstrokecolor{currentstroke}%
\pgfsetdash{}{0pt}%
\pgfpathmoveto{\pgfqpoint{4.943207in}{2.786604in}}%
\pgfpathlineto{\pgfqpoint{5.030943in}{2.786604in}}%
\pgfpathlineto{\pgfqpoint{5.030943in}{2.698868in}}%
\pgfpathlineto{\pgfqpoint{4.943207in}{2.698868in}}%
\pgfpathlineto{\pgfqpoint{4.943207in}{2.786604in}}%
\pgfusepath{stroke,fill}%
\end{pgfscope}%
\begin{pgfscope}%
\pgfpathrectangle{\pgfqpoint{0.380943in}{2.260189in}}{\pgfqpoint{4.650000in}{0.614151in}}%
\pgfusepath{clip}%
\pgfsetbuttcap%
\pgfsetroundjoin%
\definecolor{currentfill}{rgb}{0.995233,0.991895,0.818977}%
\pgfsetfillcolor{currentfill}%
\pgfsetlinewidth{0.250937pt}%
\definecolor{currentstroke}{rgb}{1.000000,1.000000,1.000000}%
\pgfsetstrokecolor{currentstroke}%
\pgfsetdash{}{0pt}%
\pgfpathmoveto{\pgfqpoint{0.380943in}{2.698868in}}%
\pgfpathlineto{\pgfqpoint{0.468679in}{2.698868in}}%
\pgfpathlineto{\pgfqpoint{0.468679in}{2.611132in}}%
\pgfpathlineto{\pgfqpoint{0.380943in}{2.611132in}}%
\pgfpathlineto{\pgfqpoint{0.380943in}{2.698868in}}%
\pgfusepath{stroke,fill}%
\end{pgfscope}%
\begin{pgfscope}%
\pgfpathrectangle{\pgfqpoint{0.380943in}{2.260189in}}{\pgfqpoint{4.650000in}{0.614151in}}%
\pgfusepath{clip}%
\pgfsetbuttcap%
\pgfsetroundjoin%
\definecolor{currentfill}{rgb}{0.990634,0.779608,0.623299}%
\pgfsetfillcolor{currentfill}%
\pgfsetlinewidth{0.250937pt}%
\definecolor{currentstroke}{rgb}{1.000000,1.000000,1.000000}%
\pgfsetstrokecolor{currentstroke}%
\pgfsetdash{}{0pt}%
\pgfpathmoveto{\pgfqpoint{0.468679in}{2.698868in}}%
\pgfpathlineto{\pgfqpoint{0.556415in}{2.698868in}}%
\pgfpathlineto{\pgfqpoint{0.556415in}{2.611132in}}%
\pgfpathlineto{\pgfqpoint{0.468679in}{2.611132in}}%
\pgfpathlineto{\pgfqpoint{0.468679in}{2.698868in}}%
\pgfusepath{stroke,fill}%
\end{pgfscope}%
\begin{pgfscope}%
\pgfpathrectangle{\pgfqpoint{0.380943in}{2.260189in}}{\pgfqpoint{4.650000in}{0.614151in}}%
\pgfusepath{clip}%
\pgfsetbuttcap%
\pgfsetroundjoin%
\definecolor{currentfill}{rgb}{0.969504,0.885813,0.700930}%
\pgfsetfillcolor{currentfill}%
\pgfsetlinewidth{0.250937pt}%
\definecolor{currentstroke}{rgb}{1.000000,1.000000,1.000000}%
\pgfsetstrokecolor{currentstroke}%
\pgfsetdash{}{0pt}%
\pgfpathmoveto{\pgfqpoint{0.556415in}{2.698868in}}%
\pgfpathlineto{\pgfqpoint{0.644151in}{2.698868in}}%
\pgfpathlineto{\pgfqpoint{0.644151in}{2.611132in}}%
\pgfpathlineto{\pgfqpoint{0.556415in}{2.611132in}}%
\pgfpathlineto{\pgfqpoint{0.556415in}{2.698868in}}%
\pgfusepath{stroke,fill}%
\end{pgfscope}%
\begin{pgfscope}%
\pgfpathrectangle{\pgfqpoint{0.380943in}{2.260189in}}{\pgfqpoint{4.650000in}{0.614151in}}%
\pgfusepath{clip}%
\pgfsetbuttcap%
\pgfsetroundjoin%
\definecolor{currentfill}{rgb}{0.969504,0.885813,0.700930}%
\pgfsetfillcolor{currentfill}%
\pgfsetlinewidth{0.250937pt}%
\definecolor{currentstroke}{rgb}{1.000000,1.000000,1.000000}%
\pgfsetstrokecolor{currentstroke}%
\pgfsetdash{}{0pt}%
\pgfpathmoveto{\pgfqpoint{0.644151in}{2.698868in}}%
\pgfpathlineto{\pgfqpoint{0.731886in}{2.698868in}}%
\pgfpathlineto{\pgfqpoint{0.731886in}{2.611132in}}%
\pgfpathlineto{\pgfqpoint{0.644151in}{2.611132in}}%
\pgfpathlineto{\pgfqpoint{0.644151in}{2.698868in}}%
\pgfusepath{stroke,fill}%
\end{pgfscope}%
\begin{pgfscope}%
\pgfpathrectangle{\pgfqpoint{0.380943in}{2.260189in}}{\pgfqpoint{4.650000in}{0.614151in}}%
\pgfusepath{clip}%
\pgfsetbuttcap%
\pgfsetroundjoin%
\definecolor{currentfill}{rgb}{0.964783,0.940131,0.739808}%
\pgfsetfillcolor{currentfill}%
\pgfsetlinewidth{0.250937pt}%
\definecolor{currentstroke}{rgb}{1.000000,1.000000,1.000000}%
\pgfsetstrokecolor{currentstroke}%
\pgfsetdash{}{0pt}%
\pgfpathmoveto{\pgfqpoint{0.731886in}{2.698868in}}%
\pgfpathlineto{\pgfqpoint{0.819622in}{2.698868in}}%
\pgfpathlineto{\pgfqpoint{0.819622in}{2.611132in}}%
\pgfpathlineto{\pgfqpoint{0.731886in}{2.611132in}}%
\pgfpathlineto{\pgfqpoint{0.731886in}{2.698868in}}%
\pgfusepath{stroke,fill}%
\end{pgfscope}%
\begin{pgfscope}%
\pgfpathrectangle{\pgfqpoint{0.380943in}{2.260189in}}{\pgfqpoint{4.650000in}{0.614151in}}%
\pgfusepath{clip}%
\pgfsetbuttcap%
\pgfsetroundjoin%
\definecolor{currentfill}{rgb}{0.974072,0.862976,0.688750}%
\pgfsetfillcolor{currentfill}%
\pgfsetlinewidth{0.250937pt}%
\definecolor{currentstroke}{rgb}{1.000000,1.000000,1.000000}%
\pgfsetstrokecolor{currentstroke}%
\pgfsetdash{}{0pt}%
\pgfpathmoveto{\pgfqpoint{0.819622in}{2.698868in}}%
\pgfpathlineto{\pgfqpoint{0.907358in}{2.698868in}}%
\pgfpathlineto{\pgfqpoint{0.907358in}{2.611132in}}%
\pgfpathlineto{\pgfqpoint{0.819622in}{2.611132in}}%
\pgfpathlineto{\pgfqpoint{0.819622in}{2.698868in}}%
\pgfusepath{stroke,fill}%
\end{pgfscope}%
\begin{pgfscope}%
\pgfpathrectangle{\pgfqpoint{0.380943in}{2.260189in}}{\pgfqpoint{4.650000in}{0.614151in}}%
\pgfusepath{clip}%
\pgfsetbuttcap%
\pgfsetroundjoin%
\definecolor{currentfill}{rgb}{0.963260,0.918478,0.719508}%
\pgfsetfillcolor{currentfill}%
\pgfsetlinewidth{0.250937pt}%
\definecolor{currentstroke}{rgb}{1.000000,1.000000,1.000000}%
\pgfsetstrokecolor{currentstroke}%
\pgfsetdash{}{0pt}%
\pgfpathmoveto{\pgfqpoint{0.907358in}{2.698868in}}%
\pgfpathlineto{\pgfqpoint{0.995094in}{2.698868in}}%
\pgfpathlineto{\pgfqpoint{0.995094in}{2.611132in}}%
\pgfpathlineto{\pgfqpoint{0.907358in}{2.611132in}}%
\pgfpathlineto{\pgfqpoint{0.907358in}{2.698868in}}%
\pgfusepath{stroke,fill}%
\end{pgfscope}%
\begin{pgfscope}%
\pgfpathrectangle{\pgfqpoint{0.380943in}{2.260189in}}{\pgfqpoint{4.650000in}{0.614151in}}%
\pgfusepath{clip}%
\pgfsetbuttcap%
\pgfsetroundjoin%
\definecolor{currentfill}{rgb}{0.993679,0.753725,0.608074}%
\pgfsetfillcolor{currentfill}%
\pgfsetlinewidth{0.250937pt}%
\definecolor{currentstroke}{rgb}{1.000000,1.000000,1.000000}%
\pgfsetstrokecolor{currentstroke}%
\pgfsetdash{}{0pt}%
\pgfpathmoveto{\pgfqpoint{0.995094in}{2.698868in}}%
\pgfpathlineto{\pgfqpoint{1.082830in}{2.698868in}}%
\pgfpathlineto{\pgfqpoint{1.082830in}{2.611132in}}%
\pgfpathlineto{\pgfqpoint{0.995094in}{2.611132in}}%
\pgfpathlineto{\pgfqpoint{0.995094in}{2.698868in}}%
\pgfusepath{stroke,fill}%
\end{pgfscope}%
\begin{pgfscope}%
\pgfpathrectangle{\pgfqpoint{0.380943in}{2.260189in}}{\pgfqpoint{4.650000in}{0.614151in}}%
\pgfusepath{clip}%
\pgfsetbuttcap%
\pgfsetroundjoin%
\definecolor{currentfill}{rgb}{0.969504,0.885813,0.700930}%
\pgfsetfillcolor{currentfill}%
\pgfsetlinewidth{0.250937pt}%
\definecolor{currentstroke}{rgb}{1.000000,1.000000,1.000000}%
\pgfsetstrokecolor{currentstroke}%
\pgfsetdash{}{0pt}%
\pgfpathmoveto{\pgfqpoint{1.082830in}{2.698868in}}%
\pgfpathlineto{\pgfqpoint{1.170566in}{2.698868in}}%
\pgfpathlineto{\pgfqpoint{1.170566in}{2.611132in}}%
\pgfpathlineto{\pgfqpoint{1.082830in}{2.611132in}}%
\pgfpathlineto{\pgfqpoint{1.082830in}{2.698868in}}%
\pgfusepath{stroke,fill}%
\end{pgfscope}%
\begin{pgfscope}%
\pgfpathrectangle{\pgfqpoint{0.380943in}{2.260189in}}{\pgfqpoint{4.650000in}{0.614151in}}%
\pgfusepath{clip}%
\pgfsetbuttcap%
\pgfsetroundjoin%
\definecolor{currentfill}{rgb}{0.978639,0.841584,0.673679}%
\pgfsetfillcolor{currentfill}%
\pgfsetlinewidth{0.250937pt}%
\definecolor{currentstroke}{rgb}{1.000000,1.000000,1.000000}%
\pgfsetstrokecolor{currentstroke}%
\pgfsetdash{}{0pt}%
\pgfpathmoveto{\pgfqpoint{1.170566in}{2.698868in}}%
\pgfpathlineto{\pgfqpoint{1.258302in}{2.698868in}}%
\pgfpathlineto{\pgfqpoint{1.258302in}{2.611132in}}%
\pgfpathlineto{\pgfqpoint{1.170566in}{2.611132in}}%
\pgfpathlineto{\pgfqpoint{1.170566in}{2.698868in}}%
\pgfusepath{stroke,fill}%
\end{pgfscope}%
\begin{pgfscope}%
\pgfpathrectangle{\pgfqpoint{0.380943in}{2.260189in}}{\pgfqpoint{4.650000in}{0.614151in}}%
\pgfusepath{clip}%
\pgfsetbuttcap%
\pgfsetroundjoin%
\definecolor{currentfill}{rgb}{0.964937,0.908651,0.713110}%
\pgfsetfillcolor{currentfill}%
\pgfsetlinewidth{0.250937pt}%
\definecolor{currentstroke}{rgb}{1.000000,1.000000,1.000000}%
\pgfsetstrokecolor{currentstroke}%
\pgfsetdash{}{0pt}%
\pgfpathmoveto{\pgfqpoint{1.258302in}{2.698868in}}%
\pgfpathlineto{\pgfqpoint{1.346037in}{2.698868in}}%
\pgfpathlineto{\pgfqpoint{1.346037in}{2.611132in}}%
\pgfpathlineto{\pgfqpoint{1.258302in}{2.611132in}}%
\pgfpathlineto{\pgfqpoint{1.258302in}{2.698868in}}%
\pgfusepath{stroke,fill}%
\end{pgfscope}%
\begin{pgfscope}%
\pgfpathrectangle{\pgfqpoint{0.380943in}{2.260189in}}{\pgfqpoint{4.650000in}{0.614151in}}%
\pgfusepath{clip}%
\pgfsetbuttcap%
\pgfsetroundjoin%
\definecolor{currentfill}{rgb}{0.969504,0.885813,0.700930}%
\pgfsetfillcolor{currentfill}%
\pgfsetlinewidth{0.250937pt}%
\definecolor{currentstroke}{rgb}{1.000000,1.000000,1.000000}%
\pgfsetstrokecolor{currentstroke}%
\pgfsetdash{}{0pt}%
\pgfpathmoveto{\pgfqpoint{1.346037in}{2.698868in}}%
\pgfpathlineto{\pgfqpoint{1.433773in}{2.698868in}}%
\pgfpathlineto{\pgfqpoint{1.433773in}{2.611132in}}%
\pgfpathlineto{\pgfqpoint{1.346037in}{2.611132in}}%
\pgfpathlineto{\pgfqpoint{1.346037in}{2.698868in}}%
\pgfusepath{stroke,fill}%
\end{pgfscope}%
\begin{pgfscope}%
\pgfpathrectangle{\pgfqpoint{0.380943in}{2.260189in}}{\pgfqpoint{4.650000in}{0.614151in}}%
\pgfusepath{clip}%
\pgfsetbuttcap%
\pgfsetroundjoin%
\definecolor{currentfill}{rgb}{0.995233,0.991895,0.818977}%
\pgfsetfillcolor{currentfill}%
\pgfsetlinewidth{0.250937pt}%
\definecolor{currentstroke}{rgb}{1.000000,1.000000,1.000000}%
\pgfsetstrokecolor{currentstroke}%
\pgfsetdash{}{0pt}%
\pgfpathmoveto{\pgfqpoint{1.433773in}{2.698868in}}%
\pgfpathlineto{\pgfqpoint{1.521509in}{2.698868in}}%
\pgfpathlineto{\pgfqpoint{1.521509in}{2.611132in}}%
\pgfpathlineto{\pgfqpoint{1.433773in}{2.611132in}}%
\pgfpathlineto{\pgfqpoint{1.433773in}{2.698868in}}%
\pgfusepath{stroke,fill}%
\end{pgfscope}%
\begin{pgfscope}%
\pgfpathrectangle{\pgfqpoint{0.380943in}{2.260189in}}{\pgfqpoint{4.650000in}{0.614151in}}%
\pgfusepath{clip}%
\pgfsetbuttcap%
\pgfsetroundjoin%
\definecolor{currentfill}{rgb}{0.995233,0.991895,0.818977}%
\pgfsetfillcolor{currentfill}%
\pgfsetlinewidth{0.250937pt}%
\definecolor{currentstroke}{rgb}{1.000000,1.000000,1.000000}%
\pgfsetstrokecolor{currentstroke}%
\pgfsetdash{}{0pt}%
\pgfpathmoveto{\pgfqpoint{1.521509in}{2.698868in}}%
\pgfpathlineto{\pgfqpoint{1.609245in}{2.698868in}}%
\pgfpathlineto{\pgfqpoint{1.609245in}{2.611132in}}%
\pgfpathlineto{\pgfqpoint{1.521509in}{2.611132in}}%
\pgfpathlineto{\pgfqpoint{1.521509in}{2.698868in}}%
\pgfusepath{stroke,fill}%
\end{pgfscope}%
\begin{pgfscope}%
\pgfpathrectangle{\pgfqpoint{0.380943in}{2.260189in}}{\pgfqpoint{4.650000in}{0.614151in}}%
\pgfusepath{clip}%
\pgfsetbuttcap%
\pgfsetroundjoin%
\definecolor{currentfill}{rgb}{0.964783,0.940131,0.739808}%
\pgfsetfillcolor{currentfill}%
\pgfsetlinewidth{0.250937pt}%
\definecolor{currentstroke}{rgb}{1.000000,1.000000,1.000000}%
\pgfsetstrokecolor{currentstroke}%
\pgfsetdash{}{0pt}%
\pgfpathmoveto{\pgfqpoint{1.609245in}{2.698868in}}%
\pgfpathlineto{\pgfqpoint{1.696981in}{2.698868in}}%
\pgfpathlineto{\pgfqpoint{1.696981in}{2.611132in}}%
\pgfpathlineto{\pgfqpoint{1.609245in}{2.611132in}}%
\pgfpathlineto{\pgfqpoint{1.609245in}{2.698868in}}%
\pgfusepath{stroke,fill}%
\end{pgfscope}%
\begin{pgfscope}%
\pgfpathrectangle{\pgfqpoint{0.380943in}{2.260189in}}{\pgfqpoint{4.650000in}{0.614151in}}%
\pgfusepath{clip}%
\pgfsetbuttcap%
\pgfsetroundjoin%
\definecolor{currentfill}{rgb}{0.961738,0.927612,0.725598}%
\pgfsetfillcolor{currentfill}%
\pgfsetlinewidth{0.250937pt}%
\definecolor{currentstroke}{rgb}{1.000000,1.000000,1.000000}%
\pgfsetstrokecolor{currentstroke}%
\pgfsetdash{}{0pt}%
\pgfpathmoveto{\pgfqpoint{1.696981in}{2.698868in}}%
\pgfpathlineto{\pgfqpoint{1.784717in}{2.698868in}}%
\pgfpathlineto{\pgfqpoint{1.784717in}{2.611132in}}%
\pgfpathlineto{\pgfqpoint{1.696981in}{2.611132in}}%
\pgfpathlineto{\pgfqpoint{1.696981in}{2.698868in}}%
\pgfusepath{stroke,fill}%
\end{pgfscope}%
\begin{pgfscope}%
\pgfpathrectangle{\pgfqpoint{0.380943in}{2.260189in}}{\pgfqpoint{4.650000in}{0.614151in}}%
\pgfusepath{clip}%
\pgfsetbuttcap%
\pgfsetroundjoin%
\definecolor{currentfill}{rgb}{0.963260,0.918478,0.719508}%
\pgfsetfillcolor{currentfill}%
\pgfsetlinewidth{0.250937pt}%
\definecolor{currentstroke}{rgb}{1.000000,1.000000,1.000000}%
\pgfsetstrokecolor{currentstroke}%
\pgfsetdash{}{0pt}%
\pgfpathmoveto{\pgfqpoint{1.784717in}{2.698868in}}%
\pgfpathlineto{\pgfqpoint{1.872452in}{2.698868in}}%
\pgfpathlineto{\pgfqpoint{1.872452in}{2.611132in}}%
\pgfpathlineto{\pgfqpoint{1.784717in}{2.611132in}}%
\pgfpathlineto{\pgfqpoint{1.784717in}{2.698868in}}%
\pgfusepath{stroke,fill}%
\end{pgfscope}%
\begin{pgfscope}%
\pgfpathrectangle{\pgfqpoint{0.380943in}{2.260189in}}{\pgfqpoint{4.650000in}{0.614151in}}%
\pgfusepath{clip}%
\pgfsetbuttcap%
\pgfsetroundjoin%
\definecolor{currentfill}{rgb}{0.963260,0.918478,0.719508}%
\pgfsetfillcolor{currentfill}%
\pgfsetlinewidth{0.250937pt}%
\definecolor{currentstroke}{rgb}{1.000000,1.000000,1.000000}%
\pgfsetstrokecolor{currentstroke}%
\pgfsetdash{}{0pt}%
\pgfpathmoveto{\pgfqpoint{1.872452in}{2.698868in}}%
\pgfpathlineto{\pgfqpoint{1.960188in}{2.698868in}}%
\pgfpathlineto{\pgfqpoint{1.960188in}{2.611132in}}%
\pgfpathlineto{\pgfqpoint{1.872452in}{2.611132in}}%
\pgfpathlineto{\pgfqpoint{1.872452in}{2.698868in}}%
\pgfusepath{stroke,fill}%
\end{pgfscope}%
\begin{pgfscope}%
\pgfpathrectangle{\pgfqpoint{0.380943in}{2.260189in}}{\pgfqpoint{4.650000in}{0.614151in}}%
\pgfusepath{clip}%
\pgfsetbuttcap%
\pgfsetroundjoin%
\definecolor{currentfill}{rgb}{0.974072,0.862976,0.688750}%
\pgfsetfillcolor{currentfill}%
\pgfsetlinewidth{0.250937pt}%
\definecolor{currentstroke}{rgb}{1.000000,1.000000,1.000000}%
\pgfsetstrokecolor{currentstroke}%
\pgfsetdash{}{0pt}%
\pgfpathmoveto{\pgfqpoint{1.960188in}{2.698868in}}%
\pgfpathlineto{\pgfqpoint{2.047924in}{2.698868in}}%
\pgfpathlineto{\pgfqpoint{2.047924in}{2.611132in}}%
\pgfpathlineto{\pgfqpoint{1.960188in}{2.611132in}}%
\pgfpathlineto{\pgfqpoint{1.960188in}{2.698868in}}%
\pgfusepath{stroke,fill}%
\end{pgfscope}%
\begin{pgfscope}%
\pgfpathrectangle{\pgfqpoint{0.380943in}{2.260189in}}{\pgfqpoint{4.650000in}{0.614151in}}%
\pgfusepath{clip}%
\pgfsetbuttcap%
\pgfsetroundjoin%
\definecolor{currentfill}{rgb}{0.995233,0.991895,0.818977}%
\pgfsetfillcolor{currentfill}%
\pgfsetlinewidth{0.250937pt}%
\definecolor{currentstroke}{rgb}{1.000000,1.000000,1.000000}%
\pgfsetstrokecolor{currentstroke}%
\pgfsetdash{}{0pt}%
\pgfpathmoveto{\pgfqpoint{2.047924in}{2.698868in}}%
\pgfpathlineto{\pgfqpoint{2.135660in}{2.698868in}}%
\pgfpathlineto{\pgfqpoint{2.135660in}{2.611132in}}%
\pgfpathlineto{\pgfqpoint{2.047924in}{2.611132in}}%
\pgfpathlineto{\pgfqpoint{2.047924in}{2.698868in}}%
\pgfusepath{stroke,fill}%
\end{pgfscope}%
\begin{pgfscope}%
\pgfpathrectangle{\pgfqpoint{0.380943in}{2.260189in}}{\pgfqpoint{4.650000in}{0.614151in}}%
\pgfusepath{clip}%
\pgfsetbuttcap%
\pgfsetroundjoin%
\definecolor{currentfill}{rgb}{0.974072,0.862976,0.688750}%
\pgfsetfillcolor{currentfill}%
\pgfsetlinewidth{0.250937pt}%
\definecolor{currentstroke}{rgb}{1.000000,1.000000,1.000000}%
\pgfsetstrokecolor{currentstroke}%
\pgfsetdash{}{0pt}%
\pgfpathmoveto{\pgfqpoint{2.135660in}{2.698868in}}%
\pgfpathlineto{\pgfqpoint{2.223396in}{2.698868in}}%
\pgfpathlineto{\pgfqpoint{2.223396in}{2.611132in}}%
\pgfpathlineto{\pgfqpoint{2.135660in}{2.611132in}}%
\pgfpathlineto{\pgfqpoint{2.135660in}{2.698868in}}%
\pgfusepath{stroke,fill}%
\end{pgfscope}%
\begin{pgfscope}%
\pgfpathrectangle{\pgfqpoint{0.380943in}{2.260189in}}{\pgfqpoint{4.650000in}{0.614151in}}%
\pgfusepath{clip}%
\pgfsetbuttcap%
\pgfsetroundjoin%
\definecolor{currentfill}{rgb}{0.978639,0.841584,0.673679}%
\pgfsetfillcolor{currentfill}%
\pgfsetlinewidth{0.250937pt}%
\definecolor{currentstroke}{rgb}{1.000000,1.000000,1.000000}%
\pgfsetstrokecolor{currentstroke}%
\pgfsetdash{}{0pt}%
\pgfpathmoveto{\pgfqpoint{2.223396in}{2.698868in}}%
\pgfpathlineto{\pgfqpoint{2.311132in}{2.698868in}}%
\pgfpathlineto{\pgfqpoint{2.311132in}{2.611132in}}%
\pgfpathlineto{\pgfqpoint{2.223396in}{2.611132in}}%
\pgfpathlineto{\pgfqpoint{2.223396in}{2.698868in}}%
\pgfusepath{stroke,fill}%
\end{pgfscope}%
\begin{pgfscope}%
\pgfpathrectangle{\pgfqpoint{0.380943in}{2.260189in}}{\pgfqpoint{4.650000in}{0.614151in}}%
\pgfusepath{clip}%
\pgfsetbuttcap%
\pgfsetroundjoin%
\definecolor{currentfill}{rgb}{0.969504,0.885813,0.700930}%
\pgfsetfillcolor{currentfill}%
\pgfsetlinewidth{0.250937pt}%
\definecolor{currentstroke}{rgb}{1.000000,1.000000,1.000000}%
\pgfsetstrokecolor{currentstroke}%
\pgfsetdash{}{0pt}%
\pgfpathmoveto{\pgfqpoint{2.311132in}{2.698868in}}%
\pgfpathlineto{\pgfqpoint{2.398868in}{2.698868in}}%
\pgfpathlineto{\pgfqpoint{2.398868in}{2.611132in}}%
\pgfpathlineto{\pgfqpoint{2.311132in}{2.611132in}}%
\pgfpathlineto{\pgfqpoint{2.311132in}{2.698868in}}%
\pgfusepath{stroke,fill}%
\end{pgfscope}%
\begin{pgfscope}%
\pgfpathrectangle{\pgfqpoint{0.380943in}{2.260189in}}{\pgfqpoint{4.650000in}{0.614151in}}%
\pgfusepath{clip}%
\pgfsetbuttcap%
\pgfsetroundjoin%
\definecolor{currentfill}{rgb}{0.978639,0.841584,0.673679}%
\pgfsetfillcolor{currentfill}%
\pgfsetlinewidth{0.250937pt}%
\definecolor{currentstroke}{rgb}{1.000000,1.000000,1.000000}%
\pgfsetstrokecolor{currentstroke}%
\pgfsetdash{}{0pt}%
\pgfpathmoveto{\pgfqpoint{2.398868in}{2.698868in}}%
\pgfpathlineto{\pgfqpoint{2.486603in}{2.698868in}}%
\pgfpathlineto{\pgfqpoint{2.486603in}{2.611132in}}%
\pgfpathlineto{\pgfqpoint{2.398868in}{2.611132in}}%
\pgfpathlineto{\pgfqpoint{2.398868in}{2.698868in}}%
\pgfusepath{stroke,fill}%
\end{pgfscope}%
\begin{pgfscope}%
\pgfpathrectangle{\pgfqpoint{0.380943in}{2.260189in}}{\pgfqpoint{4.650000in}{0.614151in}}%
\pgfusepath{clip}%
\pgfsetbuttcap%
\pgfsetroundjoin%
\definecolor{currentfill}{rgb}{0.913879,0.392311,0.392311}%
\pgfsetfillcolor{currentfill}%
\pgfsetlinewidth{0.250937pt}%
\definecolor{currentstroke}{rgb}{1.000000,1.000000,1.000000}%
\pgfsetstrokecolor{currentstroke}%
\pgfsetdash{}{0pt}%
\pgfpathmoveto{\pgfqpoint{2.486603in}{2.698868in}}%
\pgfpathlineto{\pgfqpoint{2.574339in}{2.698868in}}%
\pgfpathlineto{\pgfqpoint{2.574339in}{2.611132in}}%
\pgfpathlineto{\pgfqpoint{2.486603in}{2.611132in}}%
\pgfpathlineto{\pgfqpoint{2.486603in}{2.698868in}}%
\pgfusepath{stroke,fill}%
\end{pgfscope}%
\begin{pgfscope}%
\pgfpathrectangle{\pgfqpoint{0.380943in}{2.260189in}}{\pgfqpoint{4.650000in}{0.614151in}}%
\pgfusepath{clip}%
\pgfsetbuttcap%
\pgfsetroundjoin%
\definecolor{currentfill}{rgb}{0.996401,0.724937,0.591557}%
\pgfsetfillcolor{currentfill}%
\pgfsetlinewidth{0.250937pt}%
\definecolor{currentstroke}{rgb}{1.000000,1.000000,1.000000}%
\pgfsetstrokecolor{currentstroke}%
\pgfsetdash{}{0pt}%
\pgfpathmoveto{\pgfqpoint{2.574339in}{2.698868in}}%
\pgfpathlineto{\pgfqpoint{2.662075in}{2.698868in}}%
\pgfpathlineto{\pgfqpoint{2.662075in}{2.611132in}}%
\pgfpathlineto{\pgfqpoint{2.574339in}{2.611132in}}%
\pgfpathlineto{\pgfqpoint{2.574339in}{2.698868in}}%
\pgfusepath{stroke,fill}%
\end{pgfscope}%
\begin{pgfscope}%
\pgfpathrectangle{\pgfqpoint{0.380943in}{2.260189in}}{\pgfqpoint{4.650000in}{0.614151in}}%
\pgfusepath{clip}%
\pgfsetbuttcap%
\pgfsetroundjoin%
\definecolor{currentfill}{rgb}{0.987266,0.804198,0.639170}%
\pgfsetfillcolor{currentfill}%
\pgfsetlinewidth{0.250937pt}%
\definecolor{currentstroke}{rgb}{1.000000,1.000000,1.000000}%
\pgfsetstrokecolor{currentstroke}%
\pgfsetdash{}{0pt}%
\pgfpathmoveto{\pgfqpoint{2.662075in}{2.698868in}}%
\pgfpathlineto{\pgfqpoint{2.749811in}{2.698868in}}%
\pgfpathlineto{\pgfqpoint{2.749811in}{2.611132in}}%
\pgfpathlineto{\pgfqpoint{2.662075in}{2.611132in}}%
\pgfpathlineto{\pgfqpoint{2.662075in}{2.698868in}}%
\pgfusepath{stroke,fill}%
\end{pgfscope}%
\begin{pgfscope}%
\pgfpathrectangle{\pgfqpoint{0.380943in}{2.260189in}}{\pgfqpoint{4.650000in}{0.614151in}}%
\pgfusepath{clip}%
\pgfsetbuttcap%
\pgfsetroundjoin%
\definecolor{currentfill}{rgb}{0.987266,0.804198,0.639170}%
\pgfsetfillcolor{currentfill}%
\pgfsetlinewidth{0.250937pt}%
\definecolor{currentstroke}{rgb}{1.000000,1.000000,1.000000}%
\pgfsetstrokecolor{currentstroke}%
\pgfsetdash{}{0pt}%
\pgfpathmoveto{\pgfqpoint{2.749811in}{2.698868in}}%
\pgfpathlineto{\pgfqpoint{2.837547in}{2.698868in}}%
\pgfpathlineto{\pgfqpoint{2.837547in}{2.611132in}}%
\pgfpathlineto{\pgfqpoint{2.749811in}{2.611132in}}%
\pgfpathlineto{\pgfqpoint{2.749811in}{2.698868in}}%
\pgfusepath{stroke,fill}%
\end{pgfscope}%
\begin{pgfscope}%
\pgfpathrectangle{\pgfqpoint{0.380943in}{2.260189in}}{\pgfqpoint{4.650000in}{0.614151in}}%
\pgfusepath{clip}%
\pgfsetbuttcap%
\pgfsetroundjoin%
\definecolor{currentfill}{rgb}{0.997924,0.685352,0.570242}%
\pgfsetfillcolor{currentfill}%
\pgfsetlinewidth{0.250937pt}%
\definecolor{currentstroke}{rgb}{1.000000,1.000000,1.000000}%
\pgfsetstrokecolor{currentstroke}%
\pgfsetdash{}{0pt}%
\pgfpathmoveto{\pgfqpoint{2.837547in}{2.698868in}}%
\pgfpathlineto{\pgfqpoint{2.925283in}{2.698868in}}%
\pgfpathlineto{\pgfqpoint{2.925283in}{2.611132in}}%
\pgfpathlineto{\pgfqpoint{2.837547in}{2.611132in}}%
\pgfpathlineto{\pgfqpoint{2.837547in}{2.698868in}}%
\pgfusepath{stroke,fill}%
\end{pgfscope}%
\begin{pgfscope}%
\pgfpathrectangle{\pgfqpoint{0.380943in}{2.260189in}}{\pgfqpoint{4.650000in}{0.614151in}}%
\pgfusepath{clip}%
\pgfsetbuttcap%
\pgfsetroundjoin%
\definecolor{currentfill}{rgb}{0.996401,0.724937,0.591557}%
\pgfsetfillcolor{currentfill}%
\pgfsetlinewidth{0.250937pt}%
\definecolor{currentstroke}{rgb}{1.000000,1.000000,1.000000}%
\pgfsetstrokecolor{currentstroke}%
\pgfsetdash{}{0pt}%
\pgfpathmoveto{\pgfqpoint{2.925283in}{2.698868in}}%
\pgfpathlineto{\pgfqpoint{3.013019in}{2.698868in}}%
\pgfpathlineto{\pgfqpoint{3.013019in}{2.611132in}}%
\pgfpathlineto{\pgfqpoint{2.925283in}{2.611132in}}%
\pgfpathlineto{\pgfqpoint{2.925283in}{2.698868in}}%
\pgfusepath{stroke,fill}%
\end{pgfscope}%
\begin{pgfscope}%
\pgfpathrectangle{\pgfqpoint{0.380943in}{2.260189in}}{\pgfqpoint{4.650000in}{0.614151in}}%
\pgfusepath{clip}%
\pgfsetbuttcap%
\pgfsetroundjoin%
\definecolor{currentfill}{rgb}{0.961738,0.927612,0.725598}%
\pgfsetfillcolor{currentfill}%
\pgfsetlinewidth{0.250937pt}%
\definecolor{currentstroke}{rgb}{1.000000,1.000000,1.000000}%
\pgfsetstrokecolor{currentstroke}%
\pgfsetdash{}{0pt}%
\pgfpathmoveto{\pgfqpoint{3.013019in}{2.698868in}}%
\pgfpathlineto{\pgfqpoint{3.100754in}{2.698868in}}%
\pgfpathlineto{\pgfqpoint{3.100754in}{2.611132in}}%
\pgfpathlineto{\pgfqpoint{3.013019in}{2.611132in}}%
\pgfpathlineto{\pgfqpoint{3.013019in}{2.698868in}}%
\pgfusepath{stroke,fill}%
\end{pgfscope}%
\begin{pgfscope}%
\pgfpathrectangle{\pgfqpoint{0.380943in}{2.260189in}}{\pgfqpoint{4.650000in}{0.614151in}}%
\pgfusepath{clip}%
\pgfsetbuttcap%
\pgfsetroundjoin%
\definecolor{currentfill}{rgb}{0.982699,0.823991,0.657439}%
\pgfsetfillcolor{currentfill}%
\pgfsetlinewidth{0.250937pt}%
\definecolor{currentstroke}{rgb}{1.000000,1.000000,1.000000}%
\pgfsetstrokecolor{currentstroke}%
\pgfsetdash{}{0pt}%
\pgfpathmoveto{\pgfqpoint{3.100754in}{2.698868in}}%
\pgfpathlineto{\pgfqpoint{3.188490in}{2.698868in}}%
\pgfpathlineto{\pgfqpoint{3.188490in}{2.611132in}}%
\pgfpathlineto{\pgfqpoint{3.100754in}{2.611132in}}%
\pgfpathlineto{\pgfqpoint{3.100754in}{2.698868in}}%
\pgfusepath{stroke,fill}%
\end{pgfscope}%
\begin{pgfscope}%
\pgfpathrectangle{\pgfqpoint{0.380943in}{2.260189in}}{\pgfqpoint{4.650000in}{0.614151in}}%
\pgfusepath{clip}%
\pgfsetbuttcap%
\pgfsetroundjoin%
\definecolor{currentfill}{rgb}{0.978639,0.841584,0.673679}%
\pgfsetfillcolor{currentfill}%
\pgfsetlinewidth{0.250937pt}%
\definecolor{currentstroke}{rgb}{1.000000,1.000000,1.000000}%
\pgfsetstrokecolor{currentstroke}%
\pgfsetdash{}{0pt}%
\pgfpathmoveto{\pgfqpoint{3.188490in}{2.698868in}}%
\pgfpathlineto{\pgfqpoint{3.276226in}{2.698868in}}%
\pgfpathlineto{\pgfqpoint{3.276226in}{2.611132in}}%
\pgfpathlineto{\pgfqpoint{3.188490in}{2.611132in}}%
\pgfpathlineto{\pgfqpoint{3.188490in}{2.698868in}}%
\pgfusepath{stroke,fill}%
\end{pgfscope}%
\begin{pgfscope}%
\pgfpathrectangle{\pgfqpoint{0.380943in}{2.260189in}}{\pgfqpoint{4.650000in}{0.614151in}}%
\pgfusepath{clip}%
\pgfsetbuttcap%
\pgfsetroundjoin%
\definecolor{currentfill}{rgb}{0.996401,0.724937,0.591557}%
\pgfsetfillcolor{currentfill}%
\pgfsetlinewidth{0.250937pt}%
\definecolor{currentstroke}{rgb}{1.000000,1.000000,1.000000}%
\pgfsetstrokecolor{currentstroke}%
\pgfsetdash{}{0pt}%
\pgfpathmoveto{\pgfqpoint{3.276226in}{2.698868in}}%
\pgfpathlineto{\pgfqpoint{3.363962in}{2.698868in}}%
\pgfpathlineto{\pgfqpoint{3.363962in}{2.611132in}}%
\pgfpathlineto{\pgfqpoint{3.276226in}{2.611132in}}%
\pgfpathlineto{\pgfqpoint{3.276226in}{2.698868in}}%
\pgfusepath{stroke,fill}%
\end{pgfscope}%
\begin{pgfscope}%
\pgfpathrectangle{\pgfqpoint{0.380943in}{2.260189in}}{\pgfqpoint{4.650000in}{0.614151in}}%
\pgfusepath{clip}%
\pgfsetbuttcap%
\pgfsetroundjoin%
\definecolor{currentfill}{rgb}{0.964937,0.908651,0.713110}%
\pgfsetfillcolor{currentfill}%
\pgfsetlinewidth{0.250937pt}%
\definecolor{currentstroke}{rgb}{1.000000,1.000000,1.000000}%
\pgfsetstrokecolor{currentstroke}%
\pgfsetdash{}{0pt}%
\pgfpathmoveto{\pgfqpoint{3.363962in}{2.698868in}}%
\pgfpathlineto{\pgfqpoint{3.451698in}{2.698868in}}%
\pgfpathlineto{\pgfqpoint{3.451698in}{2.611132in}}%
\pgfpathlineto{\pgfqpoint{3.363962in}{2.611132in}}%
\pgfpathlineto{\pgfqpoint{3.363962in}{2.698868in}}%
\pgfusepath{stroke,fill}%
\end{pgfscope}%
\begin{pgfscope}%
\pgfpathrectangle{\pgfqpoint{0.380943in}{2.260189in}}{\pgfqpoint{4.650000in}{0.614151in}}%
\pgfusepath{clip}%
\pgfsetbuttcap%
\pgfsetroundjoin%
\definecolor{currentfill}{rgb}{0.999277,0.650165,0.551296}%
\pgfsetfillcolor{currentfill}%
\pgfsetlinewidth{0.250937pt}%
\definecolor{currentstroke}{rgb}{1.000000,1.000000,1.000000}%
\pgfsetstrokecolor{currentstroke}%
\pgfsetdash{}{0pt}%
\pgfpathmoveto{\pgfqpoint{3.451698in}{2.698868in}}%
\pgfpathlineto{\pgfqpoint{3.539434in}{2.698868in}}%
\pgfpathlineto{\pgfqpoint{3.539434in}{2.611132in}}%
\pgfpathlineto{\pgfqpoint{3.451698in}{2.611132in}}%
\pgfpathlineto{\pgfqpoint{3.451698in}{2.698868in}}%
\pgfusepath{stroke,fill}%
\end{pgfscope}%
\begin{pgfscope}%
\pgfpathrectangle{\pgfqpoint{0.380943in}{2.260189in}}{\pgfqpoint{4.650000in}{0.614151in}}%
\pgfusepath{clip}%
\pgfsetbuttcap%
\pgfsetroundjoin%
\definecolor{currentfill}{rgb}{0.982699,0.823991,0.657439}%
\pgfsetfillcolor{currentfill}%
\pgfsetlinewidth{0.250937pt}%
\definecolor{currentstroke}{rgb}{1.000000,1.000000,1.000000}%
\pgfsetstrokecolor{currentstroke}%
\pgfsetdash{}{0pt}%
\pgfpathmoveto{\pgfqpoint{3.539434in}{2.698868in}}%
\pgfpathlineto{\pgfqpoint{3.627169in}{2.698868in}}%
\pgfpathlineto{\pgfqpoint{3.627169in}{2.611132in}}%
\pgfpathlineto{\pgfqpoint{3.539434in}{2.611132in}}%
\pgfpathlineto{\pgfqpoint{3.539434in}{2.698868in}}%
\pgfusepath{stroke,fill}%
\end{pgfscope}%
\begin{pgfscope}%
\pgfpathrectangle{\pgfqpoint{0.380943in}{2.260189in}}{\pgfqpoint{4.650000in}{0.614151in}}%
\pgfusepath{clip}%
\pgfsetbuttcap%
\pgfsetroundjoin%
\definecolor{currentfill}{rgb}{0.990634,0.779608,0.623299}%
\pgfsetfillcolor{currentfill}%
\pgfsetlinewidth{0.250937pt}%
\definecolor{currentstroke}{rgb}{1.000000,1.000000,1.000000}%
\pgfsetstrokecolor{currentstroke}%
\pgfsetdash{}{0pt}%
\pgfpathmoveto{\pgfqpoint{3.627169in}{2.698868in}}%
\pgfpathlineto{\pgfqpoint{3.714905in}{2.698868in}}%
\pgfpathlineto{\pgfqpoint{3.714905in}{2.611132in}}%
\pgfpathlineto{\pgfqpoint{3.627169in}{2.611132in}}%
\pgfpathlineto{\pgfqpoint{3.627169in}{2.698868in}}%
\pgfusepath{stroke,fill}%
\end{pgfscope}%
\begin{pgfscope}%
\pgfpathrectangle{\pgfqpoint{0.380943in}{2.260189in}}{\pgfqpoint{4.650000in}{0.614151in}}%
\pgfusepath{clip}%
\pgfsetbuttcap%
\pgfsetroundjoin%
\definecolor{currentfill}{rgb}{0.996401,0.724937,0.591557}%
\pgfsetfillcolor{currentfill}%
\pgfsetlinewidth{0.250937pt}%
\definecolor{currentstroke}{rgb}{1.000000,1.000000,1.000000}%
\pgfsetstrokecolor{currentstroke}%
\pgfsetdash{}{0pt}%
\pgfpathmoveto{\pgfqpoint{3.714905in}{2.698868in}}%
\pgfpathlineto{\pgfqpoint{3.802641in}{2.698868in}}%
\pgfpathlineto{\pgfqpoint{3.802641in}{2.611132in}}%
\pgfpathlineto{\pgfqpoint{3.714905in}{2.611132in}}%
\pgfpathlineto{\pgfqpoint{3.714905in}{2.698868in}}%
\pgfusepath{stroke,fill}%
\end{pgfscope}%
\begin{pgfscope}%
\pgfpathrectangle{\pgfqpoint{0.380943in}{2.260189in}}{\pgfqpoint{4.650000in}{0.614151in}}%
\pgfusepath{clip}%
\pgfsetbuttcap%
\pgfsetroundjoin%
\definecolor{currentfill}{rgb}{1.000000,0.525475,0.498239}%
\pgfsetfillcolor{currentfill}%
\pgfsetlinewidth{0.250937pt}%
\definecolor{currentstroke}{rgb}{1.000000,1.000000,1.000000}%
\pgfsetstrokecolor{currentstroke}%
\pgfsetdash{}{0pt}%
\pgfpathmoveto{\pgfqpoint{3.802641in}{2.698868in}}%
\pgfpathlineto{\pgfqpoint{3.890377in}{2.698868in}}%
\pgfpathlineto{\pgfqpoint{3.890377in}{2.611132in}}%
\pgfpathlineto{\pgfqpoint{3.802641in}{2.611132in}}%
\pgfpathlineto{\pgfqpoint{3.802641in}{2.698868in}}%
\pgfusepath{stroke,fill}%
\end{pgfscope}%
\begin{pgfscope}%
\pgfpathrectangle{\pgfqpoint{0.380943in}{2.260189in}}{\pgfqpoint{4.650000in}{0.614151in}}%
\pgfusepath{clip}%
\pgfsetbuttcap%
\pgfsetroundjoin%
\definecolor{currentfill}{rgb}{0.982699,0.823991,0.657439}%
\pgfsetfillcolor{currentfill}%
\pgfsetlinewidth{0.250937pt}%
\definecolor{currentstroke}{rgb}{1.000000,1.000000,1.000000}%
\pgfsetstrokecolor{currentstroke}%
\pgfsetdash{}{0pt}%
\pgfpathmoveto{\pgfqpoint{3.890377in}{2.698868in}}%
\pgfpathlineto{\pgfqpoint{3.978113in}{2.698868in}}%
\pgfpathlineto{\pgfqpoint{3.978113in}{2.611132in}}%
\pgfpathlineto{\pgfqpoint{3.890377in}{2.611132in}}%
\pgfpathlineto{\pgfqpoint{3.890377in}{2.698868in}}%
\pgfusepath{stroke,fill}%
\end{pgfscope}%
\begin{pgfscope}%
\pgfpathrectangle{\pgfqpoint{0.380943in}{2.260189in}}{\pgfqpoint{4.650000in}{0.614151in}}%
\pgfusepath{clip}%
\pgfsetbuttcap%
\pgfsetroundjoin%
\definecolor{currentfill}{rgb}{0.999277,0.650165,0.551296}%
\pgfsetfillcolor{currentfill}%
\pgfsetlinewidth{0.250937pt}%
\definecolor{currentstroke}{rgb}{1.000000,1.000000,1.000000}%
\pgfsetstrokecolor{currentstroke}%
\pgfsetdash{}{0pt}%
\pgfpathmoveto{\pgfqpoint{3.978113in}{2.698868in}}%
\pgfpathlineto{\pgfqpoint{4.065849in}{2.698868in}}%
\pgfpathlineto{\pgfqpoint{4.065849in}{2.611132in}}%
\pgfpathlineto{\pgfqpoint{3.978113in}{2.611132in}}%
\pgfpathlineto{\pgfqpoint{3.978113in}{2.698868in}}%
\pgfusepath{stroke,fill}%
\end{pgfscope}%
\begin{pgfscope}%
\pgfpathrectangle{\pgfqpoint{0.380943in}{2.260189in}}{\pgfqpoint{4.650000in}{0.614151in}}%
\pgfusepath{clip}%
\pgfsetbuttcap%
\pgfsetroundjoin%
\definecolor{currentfill}{rgb}{1.000000,0.554479,0.510419}%
\pgfsetfillcolor{currentfill}%
\pgfsetlinewidth{0.250937pt}%
\definecolor{currentstroke}{rgb}{1.000000,1.000000,1.000000}%
\pgfsetstrokecolor{currentstroke}%
\pgfsetdash{}{0pt}%
\pgfpathmoveto{\pgfqpoint{4.065849in}{2.698868in}}%
\pgfpathlineto{\pgfqpoint{4.153585in}{2.698868in}}%
\pgfpathlineto{\pgfqpoint{4.153585in}{2.611132in}}%
\pgfpathlineto{\pgfqpoint{4.065849in}{2.611132in}}%
\pgfpathlineto{\pgfqpoint{4.065849in}{2.698868in}}%
\pgfusepath{stroke,fill}%
\end{pgfscope}%
\begin{pgfscope}%
\pgfpathrectangle{\pgfqpoint{0.380943in}{2.260189in}}{\pgfqpoint{4.650000in}{0.614151in}}%
\pgfusepath{clip}%
\pgfsetbuttcap%
\pgfsetroundjoin%
\definecolor{currentfill}{rgb}{0.961738,0.927612,0.725598}%
\pgfsetfillcolor{currentfill}%
\pgfsetlinewidth{0.250937pt}%
\definecolor{currentstroke}{rgb}{1.000000,1.000000,1.000000}%
\pgfsetstrokecolor{currentstroke}%
\pgfsetdash{}{0pt}%
\pgfpathmoveto{\pgfqpoint{4.153585in}{2.698868in}}%
\pgfpathlineto{\pgfqpoint{4.241320in}{2.698868in}}%
\pgfpathlineto{\pgfqpoint{4.241320in}{2.611132in}}%
\pgfpathlineto{\pgfqpoint{4.153585in}{2.611132in}}%
\pgfpathlineto{\pgfqpoint{4.153585in}{2.698868in}}%
\pgfusepath{stroke,fill}%
\end{pgfscope}%
\begin{pgfscope}%
\pgfpathrectangle{\pgfqpoint{0.380943in}{2.260189in}}{\pgfqpoint{4.650000in}{0.614151in}}%
\pgfusepath{clip}%
\pgfsetbuttcap%
\pgfsetroundjoin%
\definecolor{currentfill}{rgb}{0.969504,0.885813,0.700930}%
\pgfsetfillcolor{currentfill}%
\pgfsetlinewidth{0.250937pt}%
\definecolor{currentstroke}{rgb}{1.000000,1.000000,1.000000}%
\pgfsetstrokecolor{currentstroke}%
\pgfsetdash{}{0pt}%
\pgfpathmoveto{\pgfqpoint{4.241320in}{2.698868in}}%
\pgfpathlineto{\pgfqpoint{4.329056in}{2.698868in}}%
\pgfpathlineto{\pgfqpoint{4.329056in}{2.611132in}}%
\pgfpathlineto{\pgfqpoint{4.241320in}{2.611132in}}%
\pgfpathlineto{\pgfqpoint{4.241320in}{2.698868in}}%
\pgfusepath{stroke,fill}%
\end{pgfscope}%
\begin{pgfscope}%
\pgfpathrectangle{\pgfqpoint{0.380943in}{2.260189in}}{\pgfqpoint{4.650000in}{0.614151in}}%
\pgfusepath{clip}%
\pgfsetbuttcap%
\pgfsetroundjoin%
\definecolor{currentfill}{rgb}{0.963260,0.918478,0.719508}%
\pgfsetfillcolor{currentfill}%
\pgfsetlinewidth{0.250937pt}%
\definecolor{currentstroke}{rgb}{1.000000,1.000000,1.000000}%
\pgfsetstrokecolor{currentstroke}%
\pgfsetdash{}{0pt}%
\pgfpathmoveto{\pgfqpoint{4.329056in}{2.698868in}}%
\pgfpathlineto{\pgfqpoint{4.416792in}{2.698868in}}%
\pgfpathlineto{\pgfqpoint{4.416792in}{2.611132in}}%
\pgfpathlineto{\pgfqpoint{4.329056in}{2.611132in}}%
\pgfpathlineto{\pgfqpoint{4.329056in}{2.698868in}}%
\pgfusepath{stroke,fill}%
\end{pgfscope}%
\begin{pgfscope}%
\pgfpathrectangle{\pgfqpoint{0.380943in}{2.260189in}}{\pgfqpoint{4.650000in}{0.614151in}}%
\pgfusepath{clip}%
\pgfsetbuttcap%
\pgfsetroundjoin%
\definecolor{currentfill}{rgb}{0.987266,0.804198,0.639170}%
\pgfsetfillcolor{currentfill}%
\pgfsetlinewidth{0.250937pt}%
\definecolor{currentstroke}{rgb}{1.000000,1.000000,1.000000}%
\pgfsetstrokecolor{currentstroke}%
\pgfsetdash{}{0pt}%
\pgfpathmoveto{\pgfqpoint{4.416792in}{2.698868in}}%
\pgfpathlineto{\pgfqpoint{4.504528in}{2.698868in}}%
\pgfpathlineto{\pgfqpoint{4.504528in}{2.611132in}}%
\pgfpathlineto{\pgfqpoint{4.416792in}{2.611132in}}%
\pgfpathlineto{\pgfqpoint{4.416792in}{2.698868in}}%
\pgfusepath{stroke,fill}%
\end{pgfscope}%
\begin{pgfscope}%
\pgfpathrectangle{\pgfqpoint{0.380943in}{2.260189in}}{\pgfqpoint{4.650000in}{0.614151in}}%
\pgfusepath{clip}%
\pgfsetbuttcap%
\pgfsetroundjoin%
\definecolor{currentfill}{rgb}{0.978639,0.841584,0.673679}%
\pgfsetfillcolor{currentfill}%
\pgfsetlinewidth{0.250937pt}%
\definecolor{currentstroke}{rgb}{1.000000,1.000000,1.000000}%
\pgfsetstrokecolor{currentstroke}%
\pgfsetdash{}{0pt}%
\pgfpathmoveto{\pgfqpoint{4.504528in}{2.698868in}}%
\pgfpathlineto{\pgfqpoint{4.592264in}{2.698868in}}%
\pgfpathlineto{\pgfqpoint{4.592264in}{2.611132in}}%
\pgfpathlineto{\pgfqpoint{4.504528in}{2.611132in}}%
\pgfpathlineto{\pgfqpoint{4.504528in}{2.698868in}}%
\pgfusepath{stroke,fill}%
\end{pgfscope}%
\begin{pgfscope}%
\pgfpathrectangle{\pgfqpoint{0.380943in}{2.260189in}}{\pgfqpoint{4.650000in}{0.614151in}}%
\pgfusepath{clip}%
\pgfsetbuttcap%
\pgfsetroundjoin%
\definecolor{currentfill}{rgb}{0.997924,0.685352,0.570242}%
\pgfsetfillcolor{currentfill}%
\pgfsetlinewidth{0.250937pt}%
\definecolor{currentstroke}{rgb}{1.000000,1.000000,1.000000}%
\pgfsetstrokecolor{currentstroke}%
\pgfsetdash{}{0pt}%
\pgfpathmoveto{\pgfqpoint{4.592264in}{2.698868in}}%
\pgfpathlineto{\pgfqpoint{4.680000in}{2.698868in}}%
\pgfpathlineto{\pgfqpoint{4.680000in}{2.611132in}}%
\pgfpathlineto{\pgfqpoint{4.592264in}{2.611132in}}%
\pgfpathlineto{\pgfqpoint{4.592264in}{2.698868in}}%
\pgfusepath{stroke,fill}%
\end{pgfscope}%
\begin{pgfscope}%
\pgfpathrectangle{\pgfqpoint{0.380943in}{2.260189in}}{\pgfqpoint{4.650000in}{0.614151in}}%
\pgfusepath{clip}%
\pgfsetbuttcap%
\pgfsetroundjoin%
\definecolor{currentfill}{rgb}{0.997924,0.685352,0.570242}%
\pgfsetfillcolor{currentfill}%
\pgfsetlinewidth{0.250937pt}%
\definecolor{currentstroke}{rgb}{1.000000,1.000000,1.000000}%
\pgfsetstrokecolor{currentstroke}%
\pgfsetdash{}{0pt}%
\pgfpathmoveto{\pgfqpoint{4.680000in}{2.698868in}}%
\pgfpathlineto{\pgfqpoint{4.767736in}{2.698868in}}%
\pgfpathlineto{\pgfqpoint{4.767736in}{2.611132in}}%
\pgfpathlineto{\pgfqpoint{4.680000in}{2.611132in}}%
\pgfpathlineto{\pgfqpoint{4.680000in}{2.698868in}}%
\pgfusepath{stroke,fill}%
\end{pgfscope}%
\begin{pgfscope}%
\pgfpathrectangle{\pgfqpoint{0.380943in}{2.260189in}}{\pgfqpoint{4.650000in}{0.614151in}}%
\pgfusepath{clip}%
\pgfsetbuttcap%
\pgfsetroundjoin%
\definecolor{currentfill}{rgb}{0.993679,0.753725,0.608074}%
\pgfsetfillcolor{currentfill}%
\pgfsetlinewidth{0.250937pt}%
\definecolor{currentstroke}{rgb}{1.000000,1.000000,1.000000}%
\pgfsetstrokecolor{currentstroke}%
\pgfsetdash{}{0pt}%
\pgfpathmoveto{\pgfqpoint{4.767736in}{2.698868in}}%
\pgfpathlineto{\pgfqpoint{4.855471in}{2.698868in}}%
\pgfpathlineto{\pgfqpoint{4.855471in}{2.611132in}}%
\pgfpathlineto{\pgfqpoint{4.767736in}{2.611132in}}%
\pgfpathlineto{\pgfqpoint{4.767736in}{2.698868in}}%
\pgfusepath{stroke,fill}%
\end{pgfscope}%
\begin{pgfscope}%
\pgfpathrectangle{\pgfqpoint{0.380943in}{2.260189in}}{\pgfqpoint{4.650000in}{0.614151in}}%
\pgfusepath{clip}%
\pgfsetbuttcap%
\pgfsetroundjoin%
\definecolor{currentfill}{rgb}{0.990634,0.779608,0.623299}%
\pgfsetfillcolor{currentfill}%
\pgfsetlinewidth{0.250937pt}%
\definecolor{currentstroke}{rgb}{1.000000,1.000000,1.000000}%
\pgfsetstrokecolor{currentstroke}%
\pgfsetdash{}{0pt}%
\pgfpathmoveto{\pgfqpoint{4.855471in}{2.698868in}}%
\pgfpathlineto{\pgfqpoint{4.943207in}{2.698868in}}%
\pgfpathlineto{\pgfqpoint{4.943207in}{2.611132in}}%
\pgfpathlineto{\pgfqpoint{4.855471in}{2.611132in}}%
\pgfpathlineto{\pgfqpoint{4.855471in}{2.698868in}}%
\pgfusepath{stroke,fill}%
\end{pgfscope}%
\begin{pgfscope}%
\pgfpathrectangle{\pgfqpoint{0.380943in}{2.260189in}}{\pgfqpoint{4.650000in}{0.614151in}}%
\pgfusepath{clip}%
\pgfsetbuttcap%
\pgfsetroundjoin%
\definecolor{currentfill}{rgb}{0.987266,0.804198,0.639170}%
\pgfsetfillcolor{currentfill}%
\pgfsetlinewidth{0.250937pt}%
\definecolor{currentstroke}{rgb}{1.000000,1.000000,1.000000}%
\pgfsetstrokecolor{currentstroke}%
\pgfsetdash{}{0pt}%
\pgfpathmoveto{\pgfqpoint{4.943207in}{2.698868in}}%
\pgfpathlineto{\pgfqpoint{5.030943in}{2.698868in}}%
\pgfpathlineto{\pgfqpoint{5.030943in}{2.611132in}}%
\pgfpathlineto{\pgfqpoint{4.943207in}{2.611132in}}%
\pgfpathlineto{\pgfqpoint{4.943207in}{2.698868in}}%
\pgfusepath{stroke,fill}%
\end{pgfscope}%
\begin{pgfscope}%
\pgfpathrectangle{\pgfqpoint{0.380943in}{2.260189in}}{\pgfqpoint{4.650000in}{0.614151in}}%
\pgfusepath{clip}%
\pgfsetbuttcap%
\pgfsetroundjoin%
\definecolor{currentfill}{rgb}{0.961738,0.927612,0.725598}%
\pgfsetfillcolor{currentfill}%
\pgfsetlinewidth{0.250937pt}%
\definecolor{currentstroke}{rgb}{1.000000,1.000000,1.000000}%
\pgfsetstrokecolor{currentstroke}%
\pgfsetdash{}{0pt}%
\pgfpathmoveto{\pgfqpoint{0.380943in}{2.611132in}}%
\pgfpathlineto{\pgfqpoint{0.468679in}{2.611132in}}%
\pgfpathlineto{\pgfqpoint{0.468679in}{2.523396in}}%
\pgfpathlineto{\pgfqpoint{0.380943in}{2.523396in}}%
\pgfpathlineto{\pgfqpoint{0.380943in}{2.611132in}}%
\pgfusepath{stroke,fill}%
\end{pgfscope}%
\begin{pgfscope}%
\pgfpathrectangle{\pgfqpoint{0.380943in}{2.260189in}}{\pgfqpoint{4.650000in}{0.614151in}}%
\pgfusepath{clip}%
\pgfsetbuttcap%
\pgfsetroundjoin%
\definecolor{currentfill}{rgb}{0.987266,0.804198,0.639170}%
\pgfsetfillcolor{currentfill}%
\pgfsetlinewidth{0.250937pt}%
\definecolor{currentstroke}{rgb}{1.000000,1.000000,1.000000}%
\pgfsetstrokecolor{currentstroke}%
\pgfsetdash{}{0pt}%
\pgfpathmoveto{\pgfqpoint{0.468679in}{2.611132in}}%
\pgfpathlineto{\pgfqpoint{0.556415in}{2.611132in}}%
\pgfpathlineto{\pgfqpoint{0.556415in}{2.523396in}}%
\pgfpathlineto{\pgfqpoint{0.468679in}{2.523396in}}%
\pgfpathlineto{\pgfqpoint{0.468679in}{2.611132in}}%
\pgfusepath{stroke,fill}%
\end{pgfscope}%
\begin{pgfscope}%
\pgfpathrectangle{\pgfqpoint{0.380943in}{2.260189in}}{\pgfqpoint{4.650000in}{0.614151in}}%
\pgfusepath{clip}%
\pgfsetbuttcap%
\pgfsetroundjoin%
\definecolor{currentfill}{rgb}{0.974072,0.862976,0.688750}%
\pgfsetfillcolor{currentfill}%
\pgfsetlinewidth{0.250937pt}%
\definecolor{currentstroke}{rgb}{1.000000,1.000000,1.000000}%
\pgfsetstrokecolor{currentstroke}%
\pgfsetdash{}{0pt}%
\pgfpathmoveto{\pgfqpoint{0.556415in}{2.611132in}}%
\pgfpathlineto{\pgfqpoint{0.644151in}{2.611132in}}%
\pgfpathlineto{\pgfqpoint{0.644151in}{2.523396in}}%
\pgfpathlineto{\pgfqpoint{0.556415in}{2.523396in}}%
\pgfpathlineto{\pgfqpoint{0.556415in}{2.611132in}}%
\pgfusepath{stroke,fill}%
\end{pgfscope}%
\begin{pgfscope}%
\pgfpathrectangle{\pgfqpoint{0.380943in}{2.260189in}}{\pgfqpoint{4.650000in}{0.614151in}}%
\pgfusepath{clip}%
\pgfsetbuttcap%
\pgfsetroundjoin%
\definecolor{currentfill}{rgb}{0.987266,0.804198,0.639170}%
\pgfsetfillcolor{currentfill}%
\pgfsetlinewidth{0.250937pt}%
\definecolor{currentstroke}{rgb}{1.000000,1.000000,1.000000}%
\pgfsetstrokecolor{currentstroke}%
\pgfsetdash{}{0pt}%
\pgfpathmoveto{\pgfqpoint{0.644151in}{2.611132in}}%
\pgfpathlineto{\pgfqpoint{0.731886in}{2.611132in}}%
\pgfpathlineto{\pgfqpoint{0.731886in}{2.523396in}}%
\pgfpathlineto{\pgfqpoint{0.644151in}{2.523396in}}%
\pgfpathlineto{\pgfqpoint{0.644151in}{2.611132in}}%
\pgfusepath{stroke,fill}%
\end{pgfscope}%
\begin{pgfscope}%
\pgfpathrectangle{\pgfqpoint{0.380943in}{2.260189in}}{\pgfqpoint{4.650000in}{0.614151in}}%
\pgfusepath{clip}%
\pgfsetbuttcap%
\pgfsetroundjoin%
\definecolor{currentfill}{rgb}{0.982699,0.823991,0.657439}%
\pgfsetfillcolor{currentfill}%
\pgfsetlinewidth{0.250937pt}%
\definecolor{currentstroke}{rgb}{1.000000,1.000000,1.000000}%
\pgfsetstrokecolor{currentstroke}%
\pgfsetdash{}{0pt}%
\pgfpathmoveto{\pgfqpoint{0.731886in}{2.611132in}}%
\pgfpathlineto{\pgfqpoint{0.819622in}{2.611132in}}%
\pgfpathlineto{\pgfqpoint{0.819622in}{2.523396in}}%
\pgfpathlineto{\pgfqpoint{0.731886in}{2.523396in}}%
\pgfpathlineto{\pgfqpoint{0.731886in}{2.611132in}}%
\pgfusepath{stroke,fill}%
\end{pgfscope}%
\begin{pgfscope}%
\pgfpathrectangle{\pgfqpoint{0.380943in}{2.260189in}}{\pgfqpoint{4.650000in}{0.614151in}}%
\pgfusepath{clip}%
\pgfsetbuttcap%
\pgfsetroundjoin%
\definecolor{currentfill}{rgb}{0.987266,0.804198,0.639170}%
\pgfsetfillcolor{currentfill}%
\pgfsetlinewidth{0.250937pt}%
\definecolor{currentstroke}{rgb}{1.000000,1.000000,1.000000}%
\pgfsetstrokecolor{currentstroke}%
\pgfsetdash{}{0pt}%
\pgfpathmoveto{\pgfqpoint{0.819622in}{2.611132in}}%
\pgfpathlineto{\pgfqpoint{0.907358in}{2.611132in}}%
\pgfpathlineto{\pgfqpoint{0.907358in}{2.523396in}}%
\pgfpathlineto{\pgfqpoint{0.819622in}{2.523396in}}%
\pgfpathlineto{\pgfqpoint{0.819622in}{2.611132in}}%
\pgfusepath{stroke,fill}%
\end{pgfscope}%
\begin{pgfscope}%
\pgfpathrectangle{\pgfqpoint{0.380943in}{2.260189in}}{\pgfqpoint{4.650000in}{0.614151in}}%
\pgfusepath{clip}%
\pgfsetbuttcap%
\pgfsetroundjoin%
\definecolor{currentfill}{rgb}{0.987266,0.804198,0.639170}%
\pgfsetfillcolor{currentfill}%
\pgfsetlinewidth{0.250937pt}%
\definecolor{currentstroke}{rgb}{1.000000,1.000000,1.000000}%
\pgfsetstrokecolor{currentstroke}%
\pgfsetdash{}{0pt}%
\pgfpathmoveto{\pgfqpoint{0.907358in}{2.611132in}}%
\pgfpathlineto{\pgfqpoint{0.995094in}{2.611132in}}%
\pgfpathlineto{\pgfqpoint{0.995094in}{2.523396in}}%
\pgfpathlineto{\pgfqpoint{0.907358in}{2.523396in}}%
\pgfpathlineto{\pgfqpoint{0.907358in}{2.611132in}}%
\pgfusepath{stroke,fill}%
\end{pgfscope}%
\begin{pgfscope}%
\pgfpathrectangle{\pgfqpoint{0.380943in}{2.260189in}}{\pgfqpoint{4.650000in}{0.614151in}}%
\pgfusepath{clip}%
\pgfsetbuttcap%
\pgfsetroundjoin%
\definecolor{currentfill}{rgb}{0.982699,0.823991,0.657439}%
\pgfsetfillcolor{currentfill}%
\pgfsetlinewidth{0.250937pt}%
\definecolor{currentstroke}{rgb}{1.000000,1.000000,1.000000}%
\pgfsetstrokecolor{currentstroke}%
\pgfsetdash{}{0pt}%
\pgfpathmoveto{\pgfqpoint{0.995094in}{2.611132in}}%
\pgfpathlineto{\pgfqpoint{1.082830in}{2.611132in}}%
\pgfpathlineto{\pgfqpoint{1.082830in}{2.523396in}}%
\pgfpathlineto{\pgfqpoint{0.995094in}{2.523396in}}%
\pgfpathlineto{\pgfqpoint{0.995094in}{2.611132in}}%
\pgfusepath{stroke,fill}%
\end{pgfscope}%
\begin{pgfscope}%
\pgfpathrectangle{\pgfqpoint{0.380943in}{2.260189in}}{\pgfqpoint{4.650000in}{0.614151in}}%
\pgfusepath{clip}%
\pgfsetbuttcap%
\pgfsetroundjoin%
\definecolor{currentfill}{rgb}{0.978639,0.841584,0.673679}%
\pgfsetfillcolor{currentfill}%
\pgfsetlinewidth{0.250937pt}%
\definecolor{currentstroke}{rgb}{1.000000,1.000000,1.000000}%
\pgfsetstrokecolor{currentstroke}%
\pgfsetdash{}{0pt}%
\pgfpathmoveto{\pgfqpoint{1.082830in}{2.611132in}}%
\pgfpathlineto{\pgfqpoint{1.170566in}{2.611132in}}%
\pgfpathlineto{\pgfqpoint{1.170566in}{2.523396in}}%
\pgfpathlineto{\pgfqpoint{1.082830in}{2.523396in}}%
\pgfpathlineto{\pgfqpoint{1.082830in}{2.611132in}}%
\pgfusepath{stroke,fill}%
\end{pgfscope}%
\begin{pgfscope}%
\pgfpathrectangle{\pgfqpoint{0.380943in}{2.260189in}}{\pgfqpoint{4.650000in}{0.614151in}}%
\pgfusepath{clip}%
\pgfsetbuttcap%
\pgfsetroundjoin%
\definecolor{currentfill}{rgb}{0.961738,0.927612,0.725598}%
\pgfsetfillcolor{currentfill}%
\pgfsetlinewidth{0.250937pt}%
\definecolor{currentstroke}{rgb}{1.000000,1.000000,1.000000}%
\pgfsetstrokecolor{currentstroke}%
\pgfsetdash{}{0pt}%
\pgfpathmoveto{\pgfqpoint{1.170566in}{2.611132in}}%
\pgfpathlineto{\pgfqpoint{1.258302in}{2.611132in}}%
\pgfpathlineto{\pgfqpoint{1.258302in}{2.523396in}}%
\pgfpathlineto{\pgfqpoint{1.170566in}{2.523396in}}%
\pgfpathlineto{\pgfqpoint{1.170566in}{2.611132in}}%
\pgfusepath{stroke,fill}%
\end{pgfscope}%
\begin{pgfscope}%
\pgfpathrectangle{\pgfqpoint{0.380943in}{2.260189in}}{\pgfqpoint{4.650000in}{0.614151in}}%
\pgfusepath{clip}%
\pgfsetbuttcap%
\pgfsetroundjoin%
\definecolor{currentfill}{rgb}{1.000000,1.000000,0.857516}%
\pgfsetfillcolor{currentfill}%
\pgfsetlinewidth{0.250937pt}%
\definecolor{currentstroke}{rgb}{1.000000,1.000000,1.000000}%
\pgfsetstrokecolor{currentstroke}%
\pgfsetdash{}{0pt}%
\pgfpathmoveto{\pgfqpoint{1.258302in}{2.611132in}}%
\pgfpathlineto{\pgfqpoint{1.346037in}{2.611132in}}%
\pgfpathlineto{\pgfqpoint{1.346037in}{2.523396in}}%
\pgfpathlineto{\pgfqpoint{1.258302in}{2.523396in}}%
\pgfpathlineto{\pgfqpoint{1.258302in}{2.611132in}}%
\pgfusepath{stroke,fill}%
\end{pgfscope}%
\begin{pgfscope}%
\pgfpathrectangle{\pgfqpoint{0.380943in}{2.260189in}}{\pgfqpoint{4.650000in}{0.614151in}}%
\pgfusepath{clip}%
\pgfsetbuttcap%
\pgfsetroundjoin%
\definecolor{currentfill}{rgb}{0.980008,0.966013,0.779393}%
\pgfsetfillcolor{currentfill}%
\pgfsetlinewidth{0.250937pt}%
\definecolor{currentstroke}{rgb}{1.000000,1.000000,1.000000}%
\pgfsetstrokecolor{currentstroke}%
\pgfsetdash{}{0pt}%
\pgfpathmoveto{\pgfqpoint{1.346037in}{2.611132in}}%
\pgfpathlineto{\pgfqpoint{1.433773in}{2.611132in}}%
\pgfpathlineto{\pgfqpoint{1.433773in}{2.523396in}}%
\pgfpathlineto{\pgfqpoint{1.346037in}{2.523396in}}%
\pgfpathlineto{\pgfqpoint{1.346037in}{2.611132in}}%
\pgfusepath{stroke,fill}%
\end{pgfscope}%
\begin{pgfscope}%
\pgfpathrectangle{\pgfqpoint{0.380943in}{2.260189in}}{\pgfqpoint{4.650000in}{0.614151in}}%
\pgfusepath{clip}%
\pgfsetbuttcap%
\pgfsetroundjoin%
\definecolor{currentfill}{rgb}{1.000000,1.000000,0.929412}%
\pgfsetfillcolor{currentfill}%
\pgfsetlinewidth{0.250937pt}%
\definecolor{currentstroke}{rgb}{1.000000,1.000000,1.000000}%
\pgfsetstrokecolor{currentstroke}%
\pgfsetdash{}{0pt}%
\pgfpathmoveto{\pgfqpoint{1.433773in}{2.611132in}}%
\pgfpathlineto{\pgfqpoint{1.521509in}{2.611132in}}%
\pgfpathlineto{\pgfqpoint{1.521509in}{2.523396in}}%
\pgfpathlineto{\pgfqpoint{1.433773in}{2.523396in}}%
\pgfpathlineto{\pgfqpoint{1.433773in}{2.611132in}}%
\pgfusepath{stroke,fill}%
\end{pgfscope}%
\begin{pgfscope}%
\pgfpathrectangle{\pgfqpoint{0.380943in}{2.260189in}}{\pgfqpoint{4.650000in}{0.614151in}}%
\pgfusepath{clip}%
\pgfsetbuttcap%
\pgfsetroundjoin%
\definecolor{currentfill}{rgb}{0.961738,0.927612,0.725598}%
\pgfsetfillcolor{currentfill}%
\pgfsetlinewidth{0.250937pt}%
\definecolor{currentstroke}{rgb}{1.000000,1.000000,1.000000}%
\pgfsetstrokecolor{currentstroke}%
\pgfsetdash{}{0pt}%
\pgfpathmoveto{\pgfqpoint{1.521509in}{2.611132in}}%
\pgfpathlineto{\pgfqpoint{1.609245in}{2.611132in}}%
\pgfpathlineto{\pgfqpoint{1.609245in}{2.523396in}}%
\pgfpathlineto{\pgfqpoint{1.521509in}{2.523396in}}%
\pgfpathlineto{\pgfqpoint{1.521509in}{2.611132in}}%
\pgfusepath{stroke,fill}%
\end{pgfscope}%
\begin{pgfscope}%
\pgfpathrectangle{\pgfqpoint{0.380943in}{2.260189in}}{\pgfqpoint{4.650000in}{0.614151in}}%
\pgfusepath{clip}%
\pgfsetbuttcap%
\pgfsetroundjoin%
\definecolor{currentfill}{rgb}{0.964783,0.940131,0.739808}%
\pgfsetfillcolor{currentfill}%
\pgfsetlinewidth{0.250937pt}%
\definecolor{currentstroke}{rgb}{1.000000,1.000000,1.000000}%
\pgfsetstrokecolor{currentstroke}%
\pgfsetdash{}{0pt}%
\pgfpathmoveto{\pgfqpoint{1.609245in}{2.611132in}}%
\pgfpathlineto{\pgfqpoint{1.696981in}{2.611132in}}%
\pgfpathlineto{\pgfqpoint{1.696981in}{2.523396in}}%
\pgfpathlineto{\pgfqpoint{1.609245in}{2.523396in}}%
\pgfpathlineto{\pgfqpoint{1.609245in}{2.611132in}}%
\pgfusepath{stroke,fill}%
\end{pgfscope}%
\begin{pgfscope}%
\pgfpathrectangle{\pgfqpoint{0.380943in}{2.260189in}}{\pgfqpoint{4.650000in}{0.614151in}}%
\pgfusepath{clip}%
\pgfsetbuttcap%
\pgfsetroundjoin%
\definecolor{currentfill}{rgb}{1.000000,1.000000,0.895579}%
\pgfsetfillcolor{currentfill}%
\pgfsetlinewidth{0.250937pt}%
\definecolor{currentstroke}{rgb}{1.000000,1.000000,1.000000}%
\pgfsetstrokecolor{currentstroke}%
\pgfsetdash{}{0pt}%
\pgfpathmoveto{\pgfqpoint{1.696981in}{2.611132in}}%
\pgfpathlineto{\pgfqpoint{1.784717in}{2.611132in}}%
\pgfpathlineto{\pgfqpoint{1.784717in}{2.523396in}}%
\pgfpathlineto{\pgfqpoint{1.696981in}{2.523396in}}%
\pgfpathlineto{\pgfqpoint{1.696981in}{2.611132in}}%
\pgfusepath{stroke,fill}%
\end{pgfscope}%
\begin{pgfscope}%
\pgfpathrectangle{\pgfqpoint{0.380943in}{2.260189in}}{\pgfqpoint{4.650000in}{0.614151in}}%
\pgfusepath{clip}%
\pgfsetbuttcap%
\pgfsetroundjoin%
\definecolor{currentfill}{rgb}{0.964937,0.908651,0.713110}%
\pgfsetfillcolor{currentfill}%
\pgfsetlinewidth{0.250937pt}%
\definecolor{currentstroke}{rgb}{1.000000,1.000000,1.000000}%
\pgfsetstrokecolor{currentstroke}%
\pgfsetdash{}{0pt}%
\pgfpathmoveto{\pgfqpoint{1.784717in}{2.611132in}}%
\pgfpathlineto{\pgfqpoint{1.872452in}{2.611132in}}%
\pgfpathlineto{\pgfqpoint{1.872452in}{2.523396in}}%
\pgfpathlineto{\pgfqpoint{1.784717in}{2.523396in}}%
\pgfpathlineto{\pgfqpoint{1.784717in}{2.611132in}}%
\pgfusepath{stroke,fill}%
\end{pgfscope}%
\begin{pgfscope}%
\pgfpathrectangle{\pgfqpoint{0.380943in}{2.260189in}}{\pgfqpoint{4.650000in}{0.614151in}}%
\pgfusepath{clip}%
\pgfsetbuttcap%
\pgfsetroundjoin%
\definecolor{currentfill}{rgb}{0.995233,0.991895,0.818977}%
\pgfsetfillcolor{currentfill}%
\pgfsetlinewidth{0.250937pt}%
\definecolor{currentstroke}{rgb}{1.000000,1.000000,1.000000}%
\pgfsetstrokecolor{currentstroke}%
\pgfsetdash{}{0pt}%
\pgfpathmoveto{\pgfqpoint{1.872452in}{2.611132in}}%
\pgfpathlineto{\pgfqpoint{1.960188in}{2.611132in}}%
\pgfpathlineto{\pgfqpoint{1.960188in}{2.523396in}}%
\pgfpathlineto{\pgfqpoint{1.872452in}{2.523396in}}%
\pgfpathlineto{\pgfqpoint{1.872452in}{2.611132in}}%
\pgfusepath{stroke,fill}%
\end{pgfscope}%
\begin{pgfscope}%
\pgfpathrectangle{\pgfqpoint{0.380943in}{2.260189in}}{\pgfqpoint{4.650000in}{0.614151in}}%
\pgfusepath{clip}%
\pgfsetbuttcap%
\pgfsetroundjoin%
\definecolor{currentfill}{rgb}{0.961738,0.927612,0.725598}%
\pgfsetfillcolor{currentfill}%
\pgfsetlinewidth{0.250937pt}%
\definecolor{currentstroke}{rgb}{1.000000,1.000000,1.000000}%
\pgfsetstrokecolor{currentstroke}%
\pgfsetdash{}{0pt}%
\pgfpathmoveto{\pgfqpoint{1.960188in}{2.611132in}}%
\pgfpathlineto{\pgfqpoint{2.047924in}{2.611132in}}%
\pgfpathlineto{\pgfqpoint{2.047924in}{2.523396in}}%
\pgfpathlineto{\pgfqpoint{1.960188in}{2.523396in}}%
\pgfpathlineto{\pgfqpoint{1.960188in}{2.611132in}}%
\pgfusepath{stroke,fill}%
\end{pgfscope}%
\begin{pgfscope}%
\pgfpathrectangle{\pgfqpoint{0.380943in}{2.260189in}}{\pgfqpoint{4.650000in}{0.614151in}}%
\pgfusepath{clip}%
\pgfsetbuttcap%
\pgfsetroundjoin%
\definecolor{currentfill}{rgb}{0.995233,0.991895,0.818977}%
\pgfsetfillcolor{currentfill}%
\pgfsetlinewidth{0.250937pt}%
\definecolor{currentstroke}{rgb}{1.000000,1.000000,1.000000}%
\pgfsetstrokecolor{currentstroke}%
\pgfsetdash{}{0pt}%
\pgfpathmoveto{\pgfqpoint{2.047924in}{2.611132in}}%
\pgfpathlineto{\pgfqpoint{2.135660in}{2.611132in}}%
\pgfpathlineto{\pgfqpoint{2.135660in}{2.523396in}}%
\pgfpathlineto{\pgfqpoint{2.047924in}{2.523396in}}%
\pgfpathlineto{\pgfqpoint{2.047924in}{2.611132in}}%
\pgfusepath{stroke,fill}%
\end{pgfscope}%
\begin{pgfscope}%
\pgfpathrectangle{\pgfqpoint{0.380943in}{2.260189in}}{\pgfqpoint{4.650000in}{0.614151in}}%
\pgfusepath{clip}%
\pgfsetbuttcap%
\pgfsetroundjoin%
\definecolor{currentfill}{rgb}{0.963260,0.918478,0.719508}%
\pgfsetfillcolor{currentfill}%
\pgfsetlinewidth{0.250937pt}%
\definecolor{currentstroke}{rgb}{1.000000,1.000000,1.000000}%
\pgfsetstrokecolor{currentstroke}%
\pgfsetdash{}{0pt}%
\pgfpathmoveto{\pgfqpoint{2.135660in}{2.611132in}}%
\pgfpathlineto{\pgfqpoint{2.223396in}{2.611132in}}%
\pgfpathlineto{\pgfqpoint{2.223396in}{2.523396in}}%
\pgfpathlineto{\pgfqpoint{2.135660in}{2.523396in}}%
\pgfpathlineto{\pgfqpoint{2.135660in}{2.611132in}}%
\pgfusepath{stroke,fill}%
\end{pgfscope}%
\begin{pgfscope}%
\pgfpathrectangle{\pgfqpoint{0.380943in}{2.260189in}}{\pgfqpoint{4.650000in}{0.614151in}}%
\pgfusepath{clip}%
\pgfsetbuttcap%
\pgfsetroundjoin%
\definecolor{currentfill}{rgb}{0.963260,0.918478,0.719508}%
\pgfsetfillcolor{currentfill}%
\pgfsetlinewidth{0.250937pt}%
\definecolor{currentstroke}{rgb}{1.000000,1.000000,1.000000}%
\pgfsetstrokecolor{currentstroke}%
\pgfsetdash{}{0pt}%
\pgfpathmoveto{\pgfqpoint{2.223396in}{2.611132in}}%
\pgfpathlineto{\pgfqpoint{2.311132in}{2.611132in}}%
\pgfpathlineto{\pgfqpoint{2.311132in}{2.523396in}}%
\pgfpathlineto{\pgfqpoint{2.223396in}{2.523396in}}%
\pgfpathlineto{\pgfqpoint{2.223396in}{2.611132in}}%
\pgfusepath{stroke,fill}%
\end{pgfscope}%
\begin{pgfscope}%
\pgfpathrectangle{\pgfqpoint{0.380943in}{2.260189in}}{\pgfqpoint{4.650000in}{0.614151in}}%
\pgfusepath{clip}%
\pgfsetbuttcap%
\pgfsetroundjoin%
\definecolor{currentfill}{rgb}{0.978639,0.841584,0.673679}%
\pgfsetfillcolor{currentfill}%
\pgfsetlinewidth{0.250937pt}%
\definecolor{currentstroke}{rgb}{1.000000,1.000000,1.000000}%
\pgfsetstrokecolor{currentstroke}%
\pgfsetdash{}{0pt}%
\pgfpathmoveto{\pgfqpoint{2.311132in}{2.611132in}}%
\pgfpathlineto{\pgfqpoint{2.398868in}{2.611132in}}%
\pgfpathlineto{\pgfqpoint{2.398868in}{2.523396in}}%
\pgfpathlineto{\pgfqpoint{2.311132in}{2.523396in}}%
\pgfpathlineto{\pgfqpoint{2.311132in}{2.611132in}}%
\pgfusepath{stroke,fill}%
\end{pgfscope}%
\begin{pgfscope}%
\pgfpathrectangle{\pgfqpoint{0.380943in}{2.260189in}}{\pgfqpoint{4.650000in}{0.614151in}}%
\pgfusepath{clip}%
\pgfsetbuttcap%
\pgfsetroundjoin%
\definecolor{currentfill}{rgb}{0.974072,0.862976,0.688750}%
\pgfsetfillcolor{currentfill}%
\pgfsetlinewidth{0.250937pt}%
\definecolor{currentstroke}{rgb}{1.000000,1.000000,1.000000}%
\pgfsetstrokecolor{currentstroke}%
\pgfsetdash{}{0pt}%
\pgfpathmoveto{\pgfqpoint{2.398868in}{2.611132in}}%
\pgfpathlineto{\pgfqpoint{2.486603in}{2.611132in}}%
\pgfpathlineto{\pgfqpoint{2.486603in}{2.523396in}}%
\pgfpathlineto{\pgfqpoint{2.398868in}{2.523396in}}%
\pgfpathlineto{\pgfqpoint{2.398868in}{2.611132in}}%
\pgfusepath{stroke,fill}%
\end{pgfscope}%
\begin{pgfscope}%
\pgfpathrectangle{\pgfqpoint{0.380943in}{2.260189in}}{\pgfqpoint{4.650000in}{0.614151in}}%
\pgfusepath{clip}%
\pgfsetbuttcap%
\pgfsetroundjoin%
\definecolor{currentfill}{rgb}{0.996401,0.724937,0.591557}%
\pgfsetfillcolor{currentfill}%
\pgfsetlinewidth{0.250937pt}%
\definecolor{currentstroke}{rgb}{1.000000,1.000000,1.000000}%
\pgfsetstrokecolor{currentstroke}%
\pgfsetdash{}{0pt}%
\pgfpathmoveto{\pgfqpoint{2.486603in}{2.611132in}}%
\pgfpathlineto{\pgfqpoint{2.574339in}{2.611132in}}%
\pgfpathlineto{\pgfqpoint{2.574339in}{2.523396in}}%
\pgfpathlineto{\pgfqpoint{2.486603in}{2.523396in}}%
\pgfpathlineto{\pgfqpoint{2.486603in}{2.611132in}}%
\pgfusepath{stroke,fill}%
\end{pgfscope}%
\begin{pgfscope}%
\pgfpathrectangle{\pgfqpoint{0.380943in}{2.260189in}}{\pgfqpoint{4.650000in}{0.614151in}}%
\pgfusepath{clip}%
\pgfsetbuttcap%
\pgfsetroundjoin%
\definecolor{currentfill}{rgb}{0.997924,0.685352,0.570242}%
\pgfsetfillcolor{currentfill}%
\pgfsetlinewidth{0.250937pt}%
\definecolor{currentstroke}{rgb}{1.000000,1.000000,1.000000}%
\pgfsetstrokecolor{currentstroke}%
\pgfsetdash{}{0pt}%
\pgfpathmoveto{\pgfqpoint{2.574339in}{2.611132in}}%
\pgfpathlineto{\pgfqpoint{2.662075in}{2.611132in}}%
\pgfpathlineto{\pgfqpoint{2.662075in}{2.523396in}}%
\pgfpathlineto{\pgfqpoint{2.574339in}{2.523396in}}%
\pgfpathlineto{\pgfqpoint{2.574339in}{2.611132in}}%
\pgfusepath{stroke,fill}%
\end{pgfscope}%
\begin{pgfscope}%
\pgfpathrectangle{\pgfqpoint{0.380943in}{2.260189in}}{\pgfqpoint{4.650000in}{0.614151in}}%
\pgfusepath{clip}%
\pgfsetbuttcap%
\pgfsetroundjoin%
\definecolor{currentfill}{rgb}{0.987266,0.804198,0.639170}%
\pgfsetfillcolor{currentfill}%
\pgfsetlinewidth{0.250937pt}%
\definecolor{currentstroke}{rgb}{1.000000,1.000000,1.000000}%
\pgfsetstrokecolor{currentstroke}%
\pgfsetdash{}{0pt}%
\pgfpathmoveto{\pgfqpoint{2.662075in}{2.611132in}}%
\pgfpathlineto{\pgfqpoint{2.749811in}{2.611132in}}%
\pgfpathlineto{\pgfqpoint{2.749811in}{2.523396in}}%
\pgfpathlineto{\pgfqpoint{2.662075in}{2.523396in}}%
\pgfpathlineto{\pgfqpoint{2.662075in}{2.611132in}}%
\pgfusepath{stroke,fill}%
\end{pgfscope}%
\begin{pgfscope}%
\pgfpathrectangle{\pgfqpoint{0.380943in}{2.260189in}}{\pgfqpoint{4.650000in}{0.614151in}}%
\pgfusepath{clip}%
\pgfsetbuttcap%
\pgfsetroundjoin%
\definecolor{currentfill}{rgb}{0.990634,0.779608,0.623299}%
\pgfsetfillcolor{currentfill}%
\pgfsetlinewidth{0.250937pt}%
\definecolor{currentstroke}{rgb}{1.000000,1.000000,1.000000}%
\pgfsetstrokecolor{currentstroke}%
\pgfsetdash{}{0pt}%
\pgfpathmoveto{\pgfqpoint{2.749811in}{2.611132in}}%
\pgfpathlineto{\pgfqpoint{2.837547in}{2.611132in}}%
\pgfpathlineto{\pgfqpoint{2.837547in}{2.523396in}}%
\pgfpathlineto{\pgfqpoint{2.749811in}{2.523396in}}%
\pgfpathlineto{\pgfqpoint{2.749811in}{2.611132in}}%
\pgfusepath{stroke,fill}%
\end{pgfscope}%
\begin{pgfscope}%
\pgfpathrectangle{\pgfqpoint{0.380943in}{2.260189in}}{\pgfqpoint{4.650000in}{0.614151in}}%
\pgfusepath{clip}%
\pgfsetbuttcap%
\pgfsetroundjoin%
\definecolor{currentfill}{rgb}{0.993679,0.753725,0.608074}%
\pgfsetfillcolor{currentfill}%
\pgfsetlinewidth{0.250937pt}%
\definecolor{currentstroke}{rgb}{1.000000,1.000000,1.000000}%
\pgfsetstrokecolor{currentstroke}%
\pgfsetdash{}{0pt}%
\pgfpathmoveto{\pgfqpoint{2.837547in}{2.611132in}}%
\pgfpathlineto{\pgfqpoint{2.925283in}{2.611132in}}%
\pgfpathlineto{\pgfqpoint{2.925283in}{2.523396in}}%
\pgfpathlineto{\pgfqpoint{2.837547in}{2.523396in}}%
\pgfpathlineto{\pgfqpoint{2.837547in}{2.611132in}}%
\pgfusepath{stroke,fill}%
\end{pgfscope}%
\begin{pgfscope}%
\pgfpathrectangle{\pgfqpoint{0.380943in}{2.260189in}}{\pgfqpoint{4.650000in}{0.614151in}}%
\pgfusepath{clip}%
\pgfsetbuttcap%
\pgfsetroundjoin%
\definecolor{currentfill}{rgb}{0.999277,0.650165,0.551296}%
\pgfsetfillcolor{currentfill}%
\pgfsetlinewidth{0.250937pt}%
\definecolor{currentstroke}{rgb}{1.000000,1.000000,1.000000}%
\pgfsetstrokecolor{currentstroke}%
\pgfsetdash{}{0pt}%
\pgfpathmoveto{\pgfqpoint{2.925283in}{2.611132in}}%
\pgfpathlineto{\pgfqpoint{3.013019in}{2.611132in}}%
\pgfpathlineto{\pgfqpoint{3.013019in}{2.523396in}}%
\pgfpathlineto{\pgfqpoint{2.925283in}{2.523396in}}%
\pgfpathlineto{\pgfqpoint{2.925283in}{2.611132in}}%
\pgfusepath{stroke,fill}%
\end{pgfscope}%
\begin{pgfscope}%
\pgfpathrectangle{\pgfqpoint{0.380943in}{2.260189in}}{\pgfqpoint{4.650000in}{0.614151in}}%
\pgfusepath{clip}%
\pgfsetbuttcap%
\pgfsetroundjoin%
\definecolor{currentfill}{rgb}{1.000000,0.615379,0.534779}%
\pgfsetfillcolor{currentfill}%
\pgfsetlinewidth{0.250937pt}%
\definecolor{currentstroke}{rgb}{1.000000,1.000000,1.000000}%
\pgfsetstrokecolor{currentstroke}%
\pgfsetdash{}{0pt}%
\pgfpathmoveto{\pgfqpoint{3.013019in}{2.611132in}}%
\pgfpathlineto{\pgfqpoint{3.100754in}{2.611132in}}%
\pgfpathlineto{\pgfqpoint{3.100754in}{2.523396in}}%
\pgfpathlineto{\pgfqpoint{3.013019in}{2.523396in}}%
\pgfpathlineto{\pgfqpoint{3.013019in}{2.611132in}}%
\pgfusepath{stroke,fill}%
\end{pgfscope}%
\begin{pgfscope}%
\pgfpathrectangle{\pgfqpoint{0.380943in}{2.260189in}}{\pgfqpoint{4.650000in}{0.614151in}}%
\pgfusepath{clip}%
\pgfsetbuttcap%
\pgfsetroundjoin%
\definecolor{currentfill}{rgb}{0.993679,0.753725,0.608074}%
\pgfsetfillcolor{currentfill}%
\pgfsetlinewidth{0.250937pt}%
\definecolor{currentstroke}{rgb}{1.000000,1.000000,1.000000}%
\pgfsetstrokecolor{currentstroke}%
\pgfsetdash{}{0pt}%
\pgfpathmoveto{\pgfqpoint{3.100754in}{2.611132in}}%
\pgfpathlineto{\pgfqpoint{3.188490in}{2.611132in}}%
\pgfpathlineto{\pgfqpoint{3.188490in}{2.523396in}}%
\pgfpathlineto{\pgfqpoint{3.100754in}{2.523396in}}%
\pgfpathlineto{\pgfqpoint{3.100754in}{2.611132in}}%
\pgfusepath{stroke,fill}%
\end{pgfscope}%
\begin{pgfscope}%
\pgfpathrectangle{\pgfqpoint{0.380943in}{2.260189in}}{\pgfqpoint{4.650000in}{0.614151in}}%
\pgfusepath{clip}%
\pgfsetbuttcap%
\pgfsetroundjoin%
\definecolor{currentfill}{rgb}{0.990634,0.779608,0.623299}%
\pgfsetfillcolor{currentfill}%
\pgfsetlinewidth{0.250937pt}%
\definecolor{currentstroke}{rgb}{1.000000,1.000000,1.000000}%
\pgfsetstrokecolor{currentstroke}%
\pgfsetdash{}{0pt}%
\pgfpathmoveto{\pgfqpoint{3.188490in}{2.611132in}}%
\pgfpathlineto{\pgfqpoint{3.276226in}{2.611132in}}%
\pgfpathlineto{\pgfqpoint{3.276226in}{2.523396in}}%
\pgfpathlineto{\pgfqpoint{3.188490in}{2.523396in}}%
\pgfpathlineto{\pgfqpoint{3.188490in}{2.611132in}}%
\pgfusepath{stroke,fill}%
\end{pgfscope}%
\begin{pgfscope}%
\pgfpathrectangle{\pgfqpoint{0.380943in}{2.260189in}}{\pgfqpoint{4.650000in}{0.614151in}}%
\pgfusepath{clip}%
\pgfsetbuttcap%
\pgfsetroundjoin%
\definecolor{currentfill}{rgb}{0.990634,0.779608,0.623299}%
\pgfsetfillcolor{currentfill}%
\pgfsetlinewidth{0.250937pt}%
\definecolor{currentstroke}{rgb}{1.000000,1.000000,1.000000}%
\pgfsetstrokecolor{currentstroke}%
\pgfsetdash{}{0pt}%
\pgfpathmoveto{\pgfqpoint{3.276226in}{2.611132in}}%
\pgfpathlineto{\pgfqpoint{3.363962in}{2.611132in}}%
\pgfpathlineto{\pgfqpoint{3.363962in}{2.523396in}}%
\pgfpathlineto{\pgfqpoint{3.276226in}{2.523396in}}%
\pgfpathlineto{\pgfqpoint{3.276226in}{2.611132in}}%
\pgfusepath{stroke,fill}%
\end{pgfscope}%
\begin{pgfscope}%
\pgfpathrectangle{\pgfqpoint{0.380943in}{2.260189in}}{\pgfqpoint{4.650000in}{0.614151in}}%
\pgfusepath{clip}%
\pgfsetbuttcap%
\pgfsetroundjoin%
\definecolor{currentfill}{rgb}{0.978639,0.841584,0.673679}%
\pgfsetfillcolor{currentfill}%
\pgfsetlinewidth{0.250937pt}%
\definecolor{currentstroke}{rgb}{1.000000,1.000000,1.000000}%
\pgfsetstrokecolor{currentstroke}%
\pgfsetdash{}{0pt}%
\pgfpathmoveto{\pgfqpoint{3.363962in}{2.611132in}}%
\pgfpathlineto{\pgfqpoint{3.451698in}{2.611132in}}%
\pgfpathlineto{\pgfqpoint{3.451698in}{2.523396in}}%
\pgfpathlineto{\pgfqpoint{3.363962in}{2.523396in}}%
\pgfpathlineto{\pgfqpoint{3.363962in}{2.611132in}}%
\pgfusepath{stroke,fill}%
\end{pgfscope}%
\begin{pgfscope}%
\pgfpathrectangle{\pgfqpoint{0.380943in}{2.260189in}}{\pgfqpoint{4.650000in}{0.614151in}}%
\pgfusepath{clip}%
\pgfsetbuttcap%
\pgfsetroundjoin%
\definecolor{currentfill}{rgb}{0.978639,0.841584,0.673679}%
\pgfsetfillcolor{currentfill}%
\pgfsetlinewidth{0.250937pt}%
\definecolor{currentstroke}{rgb}{1.000000,1.000000,1.000000}%
\pgfsetstrokecolor{currentstroke}%
\pgfsetdash{}{0pt}%
\pgfpathmoveto{\pgfqpoint{3.451698in}{2.611132in}}%
\pgfpathlineto{\pgfqpoint{3.539434in}{2.611132in}}%
\pgfpathlineto{\pgfqpoint{3.539434in}{2.523396in}}%
\pgfpathlineto{\pgfqpoint{3.451698in}{2.523396in}}%
\pgfpathlineto{\pgfqpoint{3.451698in}{2.611132in}}%
\pgfusepath{stroke,fill}%
\end{pgfscope}%
\begin{pgfscope}%
\pgfpathrectangle{\pgfqpoint{0.380943in}{2.260189in}}{\pgfqpoint{4.650000in}{0.614151in}}%
\pgfusepath{clip}%
\pgfsetbuttcap%
\pgfsetroundjoin%
\definecolor{currentfill}{rgb}{0.996401,0.724937,0.591557}%
\pgfsetfillcolor{currentfill}%
\pgfsetlinewidth{0.250937pt}%
\definecolor{currentstroke}{rgb}{1.000000,1.000000,1.000000}%
\pgfsetstrokecolor{currentstroke}%
\pgfsetdash{}{0pt}%
\pgfpathmoveto{\pgfqpoint{3.539434in}{2.611132in}}%
\pgfpathlineto{\pgfqpoint{3.627169in}{2.611132in}}%
\pgfpathlineto{\pgfqpoint{3.627169in}{2.523396in}}%
\pgfpathlineto{\pgfqpoint{3.539434in}{2.523396in}}%
\pgfpathlineto{\pgfqpoint{3.539434in}{2.611132in}}%
\pgfusepath{stroke,fill}%
\end{pgfscope}%
\begin{pgfscope}%
\pgfpathrectangle{\pgfqpoint{0.380943in}{2.260189in}}{\pgfqpoint{4.650000in}{0.614151in}}%
\pgfusepath{clip}%
\pgfsetbuttcap%
\pgfsetroundjoin%
\definecolor{currentfill}{rgb}{0.987266,0.804198,0.639170}%
\pgfsetfillcolor{currentfill}%
\pgfsetlinewidth{0.250937pt}%
\definecolor{currentstroke}{rgb}{1.000000,1.000000,1.000000}%
\pgfsetstrokecolor{currentstroke}%
\pgfsetdash{}{0pt}%
\pgfpathmoveto{\pgfqpoint{3.627169in}{2.611132in}}%
\pgfpathlineto{\pgfqpoint{3.714905in}{2.611132in}}%
\pgfpathlineto{\pgfqpoint{3.714905in}{2.523396in}}%
\pgfpathlineto{\pgfqpoint{3.627169in}{2.523396in}}%
\pgfpathlineto{\pgfqpoint{3.627169in}{2.611132in}}%
\pgfusepath{stroke,fill}%
\end{pgfscope}%
\begin{pgfscope}%
\pgfpathrectangle{\pgfqpoint{0.380943in}{2.260189in}}{\pgfqpoint{4.650000in}{0.614151in}}%
\pgfusepath{clip}%
\pgfsetbuttcap%
\pgfsetroundjoin%
\definecolor{currentfill}{rgb}{1.000000,0.525475,0.498239}%
\pgfsetfillcolor{currentfill}%
\pgfsetlinewidth{0.250937pt}%
\definecolor{currentstroke}{rgb}{1.000000,1.000000,1.000000}%
\pgfsetstrokecolor{currentstroke}%
\pgfsetdash{}{0pt}%
\pgfpathmoveto{\pgfqpoint{3.714905in}{2.611132in}}%
\pgfpathlineto{\pgfqpoint{3.802641in}{2.611132in}}%
\pgfpathlineto{\pgfqpoint{3.802641in}{2.523396in}}%
\pgfpathlineto{\pgfqpoint{3.714905in}{2.523396in}}%
\pgfpathlineto{\pgfqpoint{3.714905in}{2.611132in}}%
\pgfusepath{stroke,fill}%
\end{pgfscope}%
\begin{pgfscope}%
\pgfpathrectangle{\pgfqpoint{0.380943in}{2.260189in}}{\pgfqpoint{4.650000in}{0.614151in}}%
\pgfusepath{clip}%
\pgfsetbuttcap%
\pgfsetroundjoin%
\definecolor{currentfill}{rgb}{1.000000,0.525475,0.498239}%
\pgfsetfillcolor{currentfill}%
\pgfsetlinewidth{0.250937pt}%
\definecolor{currentstroke}{rgb}{1.000000,1.000000,1.000000}%
\pgfsetstrokecolor{currentstroke}%
\pgfsetdash{}{0pt}%
\pgfpathmoveto{\pgfqpoint{3.802641in}{2.611132in}}%
\pgfpathlineto{\pgfqpoint{3.890377in}{2.611132in}}%
\pgfpathlineto{\pgfqpoint{3.890377in}{2.523396in}}%
\pgfpathlineto{\pgfqpoint{3.802641in}{2.523396in}}%
\pgfpathlineto{\pgfqpoint{3.802641in}{2.611132in}}%
\pgfusepath{stroke,fill}%
\end{pgfscope}%
\begin{pgfscope}%
\pgfpathrectangle{\pgfqpoint{0.380943in}{2.260189in}}{\pgfqpoint{4.650000in}{0.614151in}}%
\pgfusepath{clip}%
\pgfsetbuttcap%
\pgfsetroundjoin%
\definecolor{currentfill}{rgb}{0.978639,0.841584,0.673679}%
\pgfsetfillcolor{currentfill}%
\pgfsetlinewidth{0.250937pt}%
\definecolor{currentstroke}{rgb}{1.000000,1.000000,1.000000}%
\pgfsetstrokecolor{currentstroke}%
\pgfsetdash{}{0pt}%
\pgfpathmoveto{\pgfqpoint{3.890377in}{2.611132in}}%
\pgfpathlineto{\pgfqpoint{3.978113in}{2.611132in}}%
\pgfpathlineto{\pgfqpoint{3.978113in}{2.523396in}}%
\pgfpathlineto{\pgfqpoint{3.890377in}{2.523396in}}%
\pgfpathlineto{\pgfqpoint{3.890377in}{2.611132in}}%
\pgfusepath{stroke,fill}%
\end{pgfscope}%
\begin{pgfscope}%
\pgfpathrectangle{\pgfqpoint{0.380943in}{2.260189in}}{\pgfqpoint{4.650000in}{0.614151in}}%
\pgfusepath{clip}%
\pgfsetbuttcap%
\pgfsetroundjoin%
\definecolor{currentfill}{rgb}{0.978639,0.841584,0.673679}%
\pgfsetfillcolor{currentfill}%
\pgfsetlinewidth{0.250937pt}%
\definecolor{currentstroke}{rgb}{1.000000,1.000000,1.000000}%
\pgfsetstrokecolor{currentstroke}%
\pgfsetdash{}{0pt}%
\pgfpathmoveto{\pgfqpoint{3.978113in}{2.611132in}}%
\pgfpathlineto{\pgfqpoint{4.065849in}{2.611132in}}%
\pgfpathlineto{\pgfqpoint{4.065849in}{2.523396in}}%
\pgfpathlineto{\pgfqpoint{3.978113in}{2.523396in}}%
\pgfpathlineto{\pgfqpoint{3.978113in}{2.611132in}}%
\pgfusepath{stroke,fill}%
\end{pgfscope}%
\begin{pgfscope}%
\pgfpathrectangle{\pgfqpoint{0.380943in}{2.260189in}}{\pgfqpoint{4.650000in}{0.614151in}}%
\pgfusepath{clip}%
\pgfsetbuttcap%
\pgfsetroundjoin%
\definecolor{currentfill}{rgb}{0.913879,0.392311,0.392311}%
\pgfsetfillcolor{currentfill}%
\pgfsetlinewidth{0.250937pt}%
\definecolor{currentstroke}{rgb}{1.000000,1.000000,1.000000}%
\pgfsetstrokecolor{currentstroke}%
\pgfsetdash{}{0pt}%
\pgfpathmoveto{\pgfqpoint{4.065849in}{2.611132in}}%
\pgfpathlineto{\pgfqpoint{4.153585in}{2.611132in}}%
\pgfpathlineto{\pgfqpoint{4.153585in}{2.523396in}}%
\pgfpathlineto{\pgfqpoint{4.065849in}{2.523396in}}%
\pgfpathlineto{\pgfqpoint{4.065849in}{2.611132in}}%
\pgfusepath{stroke,fill}%
\end{pgfscope}%
\begin{pgfscope}%
\pgfpathrectangle{\pgfqpoint{0.380943in}{2.260189in}}{\pgfqpoint{4.650000in}{0.614151in}}%
\pgfusepath{clip}%
\pgfsetbuttcap%
\pgfsetroundjoin%
\definecolor{currentfill}{rgb}{0.990634,0.779608,0.623299}%
\pgfsetfillcolor{currentfill}%
\pgfsetlinewidth{0.250937pt}%
\definecolor{currentstroke}{rgb}{1.000000,1.000000,1.000000}%
\pgfsetstrokecolor{currentstroke}%
\pgfsetdash{}{0pt}%
\pgfpathmoveto{\pgfqpoint{4.153585in}{2.611132in}}%
\pgfpathlineto{\pgfqpoint{4.241320in}{2.611132in}}%
\pgfpathlineto{\pgfqpoint{4.241320in}{2.523396in}}%
\pgfpathlineto{\pgfqpoint{4.153585in}{2.523396in}}%
\pgfpathlineto{\pgfqpoint{4.153585in}{2.611132in}}%
\pgfusepath{stroke,fill}%
\end{pgfscope}%
\begin{pgfscope}%
\pgfpathrectangle{\pgfqpoint{0.380943in}{2.260189in}}{\pgfqpoint{4.650000in}{0.614151in}}%
\pgfusepath{clip}%
\pgfsetbuttcap%
\pgfsetroundjoin%
\definecolor{currentfill}{rgb}{0.996401,0.724937,0.591557}%
\pgfsetfillcolor{currentfill}%
\pgfsetlinewidth{0.250937pt}%
\definecolor{currentstroke}{rgb}{1.000000,1.000000,1.000000}%
\pgfsetstrokecolor{currentstroke}%
\pgfsetdash{}{0pt}%
\pgfpathmoveto{\pgfqpoint{4.241320in}{2.611132in}}%
\pgfpathlineto{\pgfqpoint{4.329056in}{2.611132in}}%
\pgfpathlineto{\pgfqpoint{4.329056in}{2.523396in}}%
\pgfpathlineto{\pgfqpoint{4.241320in}{2.523396in}}%
\pgfpathlineto{\pgfqpoint{4.241320in}{2.611132in}}%
\pgfusepath{stroke,fill}%
\end{pgfscope}%
\begin{pgfscope}%
\pgfpathrectangle{\pgfqpoint{0.380943in}{2.260189in}}{\pgfqpoint{4.650000in}{0.614151in}}%
\pgfusepath{clip}%
\pgfsetbuttcap%
\pgfsetroundjoin%
\definecolor{currentfill}{rgb}{0.974072,0.862976,0.688750}%
\pgfsetfillcolor{currentfill}%
\pgfsetlinewidth{0.250937pt}%
\definecolor{currentstroke}{rgb}{1.000000,1.000000,1.000000}%
\pgfsetstrokecolor{currentstroke}%
\pgfsetdash{}{0pt}%
\pgfpathmoveto{\pgfqpoint{4.329056in}{2.611132in}}%
\pgfpathlineto{\pgfqpoint{4.416792in}{2.611132in}}%
\pgfpathlineto{\pgfqpoint{4.416792in}{2.523396in}}%
\pgfpathlineto{\pgfqpoint{4.329056in}{2.523396in}}%
\pgfpathlineto{\pgfqpoint{4.329056in}{2.611132in}}%
\pgfusepath{stroke,fill}%
\end{pgfscope}%
\begin{pgfscope}%
\pgfpathrectangle{\pgfqpoint{0.380943in}{2.260189in}}{\pgfqpoint{4.650000in}{0.614151in}}%
\pgfusepath{clip}%
\pgfsetbuttcap%
\pgfsetroundjoin%
\definecolor{currentfill}{rgb}{0.996401,0.724937,0.591557}%
\pgfsetfillcolor{currentfill}%
\pgfsetlinewidth{0.250937pt}%
\definecolor{currentstroke}{rgb}{1.000000,1.000000,1.000000}%
\pgfsetstrokecolor{currentstroke}%
\pgfsetdash{}{0pt}%
\pgfpathmoveto{\pgfqpoint{4.416792in}{2.611132in}}%
\pgfpathlineto{\pgfqpoint{4.504528in}{2.611132in}}%
\pgfpathlineto{\pgfqpoint{4.504528in}{2.523396in}}%
\pgfpathlineto{\pgfqpoint{4.416792in}{2.523396in}}%
\pgfpathlineto{\pgfqpoint{4.416792in}{2.611132in}}%
\pgfusepath{stroke,fill}%
\end{pgfscope}%
\begin{pgfscope}%
\pgfpathrectangle{\pgfqpoint{0.380943in}{2.260189in}}{\pgfqpoint{4.650000in}{0.614151in}}%
\pgfusepath{clip}%
\pgfsetbuttcap%
\pgfsetroundjoin%
\definecolor{currentfill}{rgb}{0.993679,0.753725,0.608074}%
\pgfsetfillcolor{currentfill}%
\pgfsetlinewidth{0.250937pt}%
\definecolor{currentstroke}{rgb}{1.000000,1.000000,1.000000}%
\pgfsetstrokecolor{currentstroke}%
\pgfsetdash{}{0pt}%
\pgfpathmoveto{\pgfqpoint{4.504528in}{2.611132in}}%
\pgfpathlineto{\pgfqpoint{4.592264in}{2.611132in}}%
\pgfpathlineto{\pgfqpoint{4.592264in}{2.523396in}}%
\pgfpathlineto{\pgfqpoint{4.504528in}{2.523396in}}%
\pgfpathlineto{\pgfqpoint{4.504528in}{2.611132in}}%
\pgfusepath{stroke,fill}%
\end{pgfscope}%
\begin{pgfscope}%
\pgfpathrectangle{\pgfqpoint{0.380943in}{2.260189in}}{\pgfqpoint{4.650000in}{0.614151in}}%
\pgfusepath{clip}%
\pgfsetbuttcap%
\pgfsetroundjoin%
\definecolor{currentfill}{rgb}{0.993679,0.753725,0.608074}%
\pgfsetfillcolor{currentfill}%
\pgfsetlinewidth{0.250937pt}%
\definecolor{currentstroke}{rgb}{1.000000,1.000000,1.000000}%
\pgfsetstrokecolor{currentstroke}%
\pgfsetdash{}{0pt}%
\pgfpathmoveto{\pgfqpoint{4.592264in}{2.611132in}}%
\pgfpathlineto{\pgfqpoint{4.680000in}{2.611132in}}%
\pgfpathlineto{\pgfqpoint{4.680000in}{2.523396in}}%
\pgfpathlineto{\pgfqpoint{4.592264in}{2.523396in}}%
\pgfpathlineto{\pgfqpoint{4.592264in}{2.611132in}}%
\pgfusepath{stroke,fill}%
\end{pgfscope}%
\begin{pgfscope}%
\pgfpathrectangle{\pgfqpoint{0.380943in}{2.260189in}}{\pgfqpoint{4.650000in}{0.614151in}}%
\pgfusepath{clip}%
\pgfsetbuttcap%
\pgfsetroundjoin%
\definecolor{currentfill}{rgb}{1.000000,0.584929,0.522599}%
\pgfsetfillcolor{currentfill}%
\pgfsetlinewidth{0.250937pt}%
\definecolor{currentstroke}{rgb}{1.000000,1.000000,1.000000}%
\pgfsetstrokecolor{currentstroke}%
\pgfsetdash{}{0pt}%
\pgfpathmoveto{\pgfqpoint{4.680000in}{2.611132in}}%
\pgfpathlineto{\pgfqpoint{4.767736in}{2.611132in}}%
\pgfpathlineto{\pgfqpoint{4.767736in}{2.523396in}}%
\pgfpathlineto{\pgfqpoint{4.680000in}{2.523396in}}%
\pgfpathlineto{\pgfqpoint{4.680000in}{2.611132in}}%
\pgfusepath{stroke,fill}%
\end{pgfscope}%
\begin{pgfscope}%
\pgfpathrectangle{\pgfqpoint{0.380943in}{2.260189in}}{\pgfqpoint{4.650000in}{0.614151in}}%
\pgfusepath{clip}%
\pgfsetbuttcap%
\pgfsetroundjoin%
\definecolor{currentfill}{rgb}{0.997924,0.685352,0.570242}%
\pgfsetfillcolor{currentfill}%
\pgfsetlinewidth{0.250937pt}%
\definecolor{currentstroke}{rgb}{1.000000,1.000000,1.000000}%
\pgfsetstrokecolor{currentstroke}%
\pgfsetdash{}{0pt}%
\pgfpathmoveto{\pgfqpoint{4.767736in}{2.611132in}}%
\pgfpathlineto{\pgfqpoint{4.855471in}{2.611132in}}%
\pgfpathlineto{\pgfqpoint{4.855471in}{2.523396in}}%
\pgfpathlineto{\pgfqpoint{4.767736in}{2.523396in}}%
\pgfpathlineto{\pgfqpoint{4.767736in}{2.611132in}}%
\pgfusepath{stroke,fill}%
\end{pgfscope}%
\begin{pgfscope}%
\pgfpathrectangle{\pgfqpoint{0.380943in}{2.260189in}}{\pgfqpoint{4.650000in}{0.614151in}}%
\pgfusepath{clip}%
\pgfsetbuttcap%
\pgfsetroundjoin%
\definecolor{currentfill}{rgb}{0.963260,0.918478,0.719508}%
\pgfsetfillcolor{currentfill}%
\pgfsetlinewidth{0.250937pt}%
\definecolor{currentstroke}{rgb}{1.000000,1.000000,1.000000}%
\pgfsetstrokecolor{currentstroke}%
\pgfsetdash{}{0pt}%
\pgfpathmoveto{\pgfqpoint{4.855471in}{2.611132in}}%
\pgfpathlineto{\pgfqpoint{4.943207in}{2.611132in}}%
\pgfpathlineto{\pgfqpoint{4.943207in}{2.523396in}}%
\pgfpathlineto{\pgfqpoint{4.855471in}{2.523396in}}%
\pgfpathlineto{\pgfqpoint{4.855471in}{2.611132in}}%
\pgfusepath{stroke,fill}%
\end{pgfscope}%
\begin{pgfscope}%
\pgfpathrectangle{\pgfqpoint{0.380943in}{2.260189in}}{\pgfqpoint{4.650000in}{0.614151in}}%
\pgfusepath{clip}%
\pgfsetbuttcap%
\pgfsetroundjoin%
\definecolor{currentfill}{rgb}{0.982699,0.823991,0.657439}%
\pgfsetfillcolor{currentfill}%
\pgfsetlinewidth{0.250937pt}%
\definecolor{currentstroke}{rgb}{1.000000,1.000000,1.000000}%
\pgfsetstrokecolor{currentstroke}%
\pgfsetdash{}{0pt}%
\pgfpathmoveto{\pgfqpoint{4.943207in}{2.611132in}}%
\pgfpathlineto{\pgfqpoint{5.030943in}{2.611132in}}%
\pgfpathlineto{\pgfqpoint{5.030943in}{2.523396in}}%
\pgfpathlineto{\pgfqpoint{4.943207in}{2.523396in}}%
\pgfpathlineto{\pgfqpoint{4.943207in}{2.611132in}}%
\pgfusepath{stroke,fill}%
\end{pgfscope}%
\begin{pgfscope}%
\pgfpathrectangle{\pgfqpoint{0.380943in}{2.260189in}}{\pgfqpoint{4.650000in}{0.614151in}}%
\pgfusepath{clip}%
\pgfsetbuttcap%
\pgfsetroundjoin%
\definecolor{currentfill}{rgb}{0.990634,0.779608,0.623299}%
\pgfsetfillcolor{currentfill}%
\pgfsetlinewidth{0.250937pt}%
\definecolor{currentstroke}{rgb}{1.000000,1.000000,1.000000}%
\pgfsetstrokecolor{currentstroke}%
\pgfsetdash{}{0pt}%
\pgfpathmoveto{\pgfqpoint{0.380943in}{2.523396in}}%
\pgfpathlineto{\pgfqpoint{0.468679in}{2.523396in}}%
\pgfpathlineto{\pgfqpoint{0.468679in}{2.435661in}}%
\pgfpathlineto{\pgfqpoint{0.380943in}{2.435661in}}%
\pgfpathlineto{\pgfqpoint{0.380943in}{2.523396in}}%
\pgfusepath{stroke,fill}%
\end{pgfscope}%
\begin{pgfscope}%
\pgfpathrectangle{\pgfqpoint{0.380943in}{2.260189in}}{\pgfqpoint{4.650000in}{0.614151in}}%
\pgfusepath{clip}%
\pgfsetbuttcap%
\pgfsetroundjoin%
\definecolor{currentfill}{rgb}{0.990634,0.779608,0.623299}%
\pgfsetfillcolor{currentfill}%
\pgfsetlinewidth{0.250937pt}%
\definecolor{currentstroke}{rgb}{1.000000,1.000000,1.000000}%
\pgfsetstrokecolor{currentstroke}%
\pgfsetdash{}{0pt}%
\pgfpathmoveto{\pgfqpoint{0.468679in}{2.523396in}}%
\pgfpathlineto{\pgfqpoint{0.556415in}{2.523396in}}%
\pgfpathlineto{\pgfqpoint{0.556415in}{2.435661in}}%
\pgfpathlineto{\pgfqpoint{0.468679in}{2.435661in}}%
\pgfpathlineto{\pgfqpoint{0.468679in}{2.523396in}}%
\pgfusepath{stroke,fill}%
\end{pgfscope}%
\begin{pgfscope}%
\pgfpathrectangle{\pgfqpoint{0.380943in}{2.260189in}}{\pgfqpoint{4.650000in}{0.614151in}}%
\pgfusepath{clip}%
\pgfsetbuttcap%
\pgfsetroundjoin%
\definecolor{currentfill}{rgb}{0.964937,0.908651,0.713110}%
\pgfsetfillcolor{currentfill}%
\pgfsetlinewidth{0.250937pt}%
\definecolor{currentstroke}{rgb}{1.000000,1.000000,1.000000}%
\pgfsetstrokecolor{currentstroke}%
\pgfsetdash{}{0pt}%
\pgfpathmoveto{\pgfqpoint{0.556415in}{2.523396in}}%
\pgfpathlineto{\pgfqpoint{0.644151in}{2.523396in}}%
\pgfpathlineto{\pgfqpoint{0.644151in}{2.435661in}}%
\pgfpathlineto{\pgfqpoint{0.556415in}{2.435661in}}%
\pgfpathlineto{\pgfqpoint{0.556415in}{2.523396in}}%
\pgfusepath{stroke,fill}%
\end{pgfscope}%
\begin{pgfscope}%
\pgfpathrectangle{\pgfqpoint{0.380943in}{2.260189in}}{\pgfqpoint{4.650000in}{0.614151in}}%
\pgfusepath{clip}%
\pgfsetbuttcap%
\pgfsetroundjoin%
\definecolor{currentfill}{rgb}{0.987266,0.804198,0.639170}%
\pgfsetfillcolor{currentfill}%
\pgfsetlinewidth{0.250937pt}%
\definecolor{currentstroke}{rgb}{1.000000,1.000000,1.000000}%
\pgfsetstrokecolor{currentstroke}%
\pgfsetdash{}{0pt}%
\pgfpathmoveto{\pgfqpoint{0.644151in}{2.523396in}}%
\pgfpathlineto{\pgfqpoint{0.731886in}{2.523396in}}%
\pgfpathlineto{\pgfqpoint{0.731886in}{2.435661in}}%
\pgfpathlineto{\pgfqpoint{0.644151in}{2.435661in}}%
\pgfpathlineto{\pgfqpoint{0.644151in}{2.523396in}}%
\pgfusepath{stroke,fill}%
\end{pgfscope}%
\begin{pgfscope}%
\pgfpathrectangle{\pgfqpoint{0.380943in}{2.260189in}}{\pgfqpoint{4.650000in}{0.614151in}}%
\pgfusepath{clip}%
\pgfsetbuttcap%
\pgfsetroundjoin%
\definecolor{currentfill}{rgb}{0.964783,0.940131,0.739808}%
\pgfsetfillcolor{currentfill}%
\pgfsetlinewidth{0.250937pt}%
\definecolor{currentstroke}{rgb}{1.000000,1.000000,1.000000}%
\pgfsetstrokecolor{currentstroke}%
\pgfsetdash{}{0pt}%
\pgfpathmoveto{\pgfqpoint{0.731886in}{2.523396in}}%
\pgfpathlineto{\pgfqpoint{0.819622in}{2.523396in}}%
\pgfpathlineto{\pgfqpoint{0.819622in}{2.435661in}}%
\pgfpathlineto{\pgfqpoint{0.731886in}{2.435661in}}%
\pgfpathlineto{\pgfqpoint{0.731886in}{2.523396in}}%
\pgfusepath{stroke,fill}%
\end{pgfscope}%
\begin{pgfscope}%
\pgfpathrectangle{\pgfqpoint{0.380943in}{2.260189in}}{\pgfqpoint{4.650000in}{0.614151in}}%
\pgfusepath{clip}%
\pgfsetbuttcap%
\pgfsetroundjoin%
\definecolor{currentfill}{rgb}{0.963260,0.918478,0.719508}%
\pgfsetfillcolor{currentfill}%
\pgfsetlinewidth{0.250937pt}%
\definecolor{currentstroke}{rgb}{1.000000,1.000000,1.000000}%
\pgfsetstrokecolor{currentstroke}%
\pgfsetdash{}{0pt}%
\pgfpathmoveto{\pgfqpoint{0.819622in}{2.523396in}}%
\pgfpathlineto{\pgfqpoint{0.907358in}{2.523396in}}%
\pgfpathlineto{\pgfqpoint{0.907358in}{2.435661in}}%
\pgfpathlineto{\pgfqpoint{0.819622in}{2.435661in}}%
\pgfpathlineto{\pgfqpoint{0.819622in}{2.523396in}}%
\pgfusepath{stroke,fill}%
\end{pgfscope}%
\begin{pgfscope}%
\pgfpathrectangle{\pgfqpoint{0.380943in}{2.260189in}}{\pgfqpoint{4.650000in}{0.614151in}}%
\pgfusepath{clip}%
\pgfsetbuttcap%
\pgfsetroundjoin%
\definecolor{currentfill}{rgb}{0.997924,0.685352,0.570242}%
\pgfsetfillcolor{currentfill}%
\pgfsetlinewidth{0.250937pt}%
\definecolor{currentstroke}{rgb}{1.000000,1.000000,1.000000}%
\pgfsetstrokecolor{currentstroke}%
\pgfsetdash{}{0pt}%
\pgfpathmoveto{\pgfqpoint{0.907358in}{2.523396in}}%
\pgfpathlineto{\pgfqpoint{0.995094in}{2.523396in}}%
\pgfpathlineto{\pgfqpoint{0.995094in}{2.435661in}}%
\pgfpathlineto{\pgfqpoint{0.907358in}{2.435661in}}%
\pgfpathlineto{\pgfqpoint{0.907358in}{2.523396in}}%
\pgfusepath{stroke,fill}%
\end{pgfscope}%
\begin{pgfscope}%
\pgfpathrectangle{\pgfqpoint{0.380943in}{2.260189in}}{\pgfqpoint{4.650000in}{0.614151in}}%
\pgfusepath{clip}%
\pgfsetbuttcap%
\pgfsetroundjoin%
\definecolor{currentfill}{rgb}{0.978639,0.841584,0.673679}%
\pgfsetfillcolor{currentfill}%
\pgfsetlinewidth{0.250937pt}%
\definecolor{currentstroke}{rgb}{1.000000,1.000000,1.000000}%
\pgfsetstrokecolor{currentstroke}%
\pgfsetdash{}{0pt}%
\pgfpathmoveto{\pgfqpoint{0.995094in}{2.523396in}}%
\pgfpathlineto{\pgfqpoint{1.082830in}{2.523396in}}%
\pgfpathlineto{\pgfqpoint{1.082830in}{2.435661in}}%
\pgfpathlineto{\pgfqpoint{0.995094in}{2.435661in}}%
\pgfpathlineto{\pgfqpoint{0.995094in}{2.523396in}}%
\pgfusepath{stroke,fill}%
\end{pgfscope}%
\begin{pgfscope}%
\pgfpathrectangle{\pgfqpoint{0.380943in}{2.260189in}}{\pgfqpoint{4.650000in}{0.614151in}}%
\pgfusepath{clip}%
\pgfsetbuttcap%
\pgfsetroundjoin%
\definecolor{currentfill}{rgb}{0.969504,0.885813,0.700930}%
\pgfsetfillcolor{currentfill}%
\pgfsetlinewidth{0.250937pt}%
\definecolor{currentstroke}{rgb}{1.000000,1.000000,1.000000}%
\pgfsetstrokecolor{currentstroke}%
\pgfsetdash{}{0pt}%
\pgfpathmoveto{\pgfqpoint{1.082830in}{2.523396in}}%
\pgfpathlineto{\pgfqpoint{1.170566in}{2.523396in}}%
\pgfpathlineto{\pgfqpoint{1.170566in}{2.435661in}}%
\pgfpathlineto{\pgfqpoint{1.082830in}{2.435661in}}%
\pgfpathlineto{\pgfqpoint{1.082830in}{2.523396in}}%
\pgfusepath{stroke,fill}%
\end{pgfscope}%
\begin{pgfscope}%
\pgfpathrectangle{\pgfqpoint{0.380943in}{2.260189in}}{\pgfqpoint{4.650000in}{0.614151in}}%
\pgfusepath{clip}%
\pgfsetbuttcap%
\pgfsetroundjoin%
\definecolor{currentfill}{rgb}{0.978639,0.841584,0.673679}%
\pgfsetfillcolor{currentfill}%
\pgfsetlinewidth{0.250937pt}%
\definecolor{currentstroke}{rgb}{1.000000,1.000000,1.000000}%
\pgfsetstrokecolor{currentstroke}%
\pgfsetdash{}{0pt}%
\pgfpathmoveto{\pgfqpoint{1.170566in}{2.523396in}}%
\pgfpathlineto{\pgfqpoint{1.258302in}{2.523396in}}%
\pgfpathlineto{\pgfqpoint{1.258302in}{2.435661in}}%
\pgfpathlineto{\pgfqpoint{1.170566in}{2.435661in}}%
\pgfpathlineto{\pgfqpoint{1.170566in}{2.523396in}}%
\pgfusepath{stroke,fill}%
\end{pgfscope}%
\begin{pgfscope}%
\pgfpathrectangle{\pgfqpoint{0.380943in}{2.260189in}}{\pgfqpoint{4.650000in}{0.614151in}}%
\pgfusepath{clip}%
\pgfsetbuttcap%
\pgfsetroundjoin%
\definecolor{currentfill}{rgb}{0.974072,0.862976,0.688750}%
\pgfsetfillcolor{currentfill}%
\pgfsetlinewidth{0.250937pt}%
\definecolor{currentstroke}{rgb}{1.000000,1.000000,1.000000}%
\pgfsetstrokecolor{currentstroke}%
\pgfsetdash{}{0pt}%
\pgfpathmoveto{\pgfqpoint{1.258302in}{2.523396in}}%
\pgfpathlineto{\pgfqpoint{1.346037in}{2.523396in}}%
\pgfpathlineto{\pgfqpoint{1.346037in}{2.435661in}}%
\pgfpathlineto{\pgfqpoint{1.258302in}{2.435661in}}%
\pgfpathlineto{\pgfqpoint{1.258302in}{2.523396in}}%
\pgfusepath{stroke,fill}%
\end{pgfscope}%
\begin{pgfscope}%
\pgfpathrectangle{\pgfqpoint{0.380943in}{2.260189in}}{\pgfqpoint{4.650000in}{0.614151in}}%
\pgfusepath{clip}%
\pgfsetbuttcap%
\pgfsetroundjoin%
\definecolor{currentfill}{rgb}{0.964783,0.940131,0.739808}%
\pgfsetfillcolor{currentfill}%
\pgfsetlinewidth{0.250937pt}%
\definecolor{currentstroke}{rgb}{1.000000,1.000000,1.000000}%
\pgfsetstrokecolor{currentstroke}%
\pgfsetdash{}{0pt}%
\pgfpathmoveto{\pgfqpoint{1.346037in}{2.523396in}}%
\pgfpathlineto{\pgfqpoint{1.433773in}{2.523396in}}%
\pgfpathlineto{\pgfqpoint{1.433773in}{2.435661in}}%
\pgfpathlineto{\pgfqpoint{1.346037in}{2.435661in}}%
\pgfpathlineto{\pgfqpoint{1.346037in}{2.523396in}}%
\pgfusepath{stroke,fill}%
\end{pgfscope}%
\begin{pgfscope}%
\pgfpathrectangle{\pgfqpoint{0.380943in}{2.260189in}}{\pgfqpoint{4.650000in}{0.614151in}}%
\pgfusepath{clip}%
\pgfsetbuttcap%
\pgfsetroundjoin%
\definecolor{currentfill}{rgb}{1.000000,1.000000,0.929412}%
\pgfsetfillcolor{currentfill}%
\pgfsetlinewidth{0.250937pt}%
\definecolor{currentstroke}{rgb}{1.000000,1.000000,1.000000}%
\pgfsetstrokecolor{currentstroke}%
\pgfsetdash{}{0pt}%
\pgfpathmoveto{\pgfqpoint{1.433773in}{2.523396in}}%
\pgfpathlineto{\pgfqpoint{1.521509in}{2.523396in}}%
\pgfpathlineto{\pgfqpoint{1.521509in}{2.435661in}}%
\pgfpathlineto{\pgfqpoint{1.433773in}{2.435661in}}%
\pgfpathlineto{\pgfqpoint{1.433773in}{2.523396in}}%
\pgfusepath{stroke,fill}%
\end{pgfscope}%
\begin{pgfscope}%
\pgfpathrectangle{\pgfqpoint{0.380943in}{2.260189in}}{\pgfqpoint{4.650000in}{0.614151in}}%
\pgfusepath{clip}%
\pgfsetbuttcap%
\pgfsetroundjoin%
\definecolor{currentfill}{rgb}{0.995233,0.991895,0.818977}%
\pgfsetfillcolor{currentfill}%
\pgfsetlinewidth{0.250937pt}%
\definecolor{currentstroke}{rgb}{1.000000,1.000000,1.000000}%
\pgfsetstrokecolor{currentstroke}%
\pgfsetdash{}{0pt}%
\pgfpathmoveto{\pgfqpoint{1.521509in}{2.523396in}}%
\pgfpathlineto{\pgfqpoint{1.609245in}{2.523396in}}%
\pgfpathlineto{\pgfqpoint{1.609245in}{2.435661in}}%
\pgfpathlineto{\pgfqpoint{1.521509in}{2.435661in}}%
\pgfpathlineto{\pgfqpoint{1.521509in}{2.523396in}}%
\pgfusepath{stroke,fill}%
\end{pgfscope}%
\begin{pgfscope}%
\pgfpathrectangle{\pgfqpoint{0.380943in}{2.260189in}}{\pgfqpoint{4.650000in}{0.614151in}}%
\pgfusepath{clip}%
\pgfsetbuttcap%
\pgfsetroundjoin%
\definecolor{currentfill}{rgb}{0.969504,0.885813,0.700930}%
\pgfsetfillcolor{currentfill}%
\pgfsetlinewidth{0.250937pt}%
\definecolor{currentstroke}{rgb}{1.000000,1.000000,1.000000}%
\pgfsetstrokecolor{currentstroke}%
\pgfsetdash{}{0pt}%
\pgfpathmoveto{\pgfqpoint{1.609245in}{2.523396in}}%
\pgfpathlineto{\pgfqpoint{1.696981in}{2.523396in}}%
\pgfpathlineto{\pgfqpoint{1.696981in}{2.435661in}}%
\pgfpathlineto{\pgfqpoint{1.609245in}{2.435661in}}%
\pgfpathlineto{\pgfqpoint{1.609245in}{2.523396in}}%
\pgfusepath{stroke,fill}%
\end{pgfscope}%
\begin{pgfscope}%
\pgfpathrectangle{\pgfqpoint{0.380943in}{2.260189in}}{\pgfqpoint{4.650000in}{0.614151in}}%
\pgfusepath{clip}%
\pgfsetbuttcap%
\pgfsetroundjoin%
\definecolor{currentfill}{rgb}{0.969504,0.885813,0.700930}%
\pgfsetfillcolor{currentfill}%
\pgfsetlinewidth{0.250937pt}%
\definecolor{currentstroke}{rgb}{1.000000,1.000000,1.000000}%
\pgfsetstrokecolor{currentstroke}%
\pgfsetdash{}{0pt}%
\pgfpathmoveto{\pgfqpoint{1.696981in}{2.523396in}}%
\pgfpathlineto{\pgfqpoint{1.784717in}{2.523396in}}%
\pgfpathlineto{\pgfqpoint{1.784717in}{2.435661in}}%
\pgfpathlineto{\pgfqpoint{1.696981in}{2.435661in}}%
\pgfpathlineto{\pgfqpoint{1.696981in}{2.523396in}}%
\pgfusepath{stroke,fill}%
\end{pgfscope}%
\begin{pgfscope}%
\pgfpathrectangle{\pgfqpoint{0.380943in}{2.260189in}}{\pgfqpoint{4.650000in}{0.614151in}}%
\pgfusepath{clip}%
\pgfsetbuttcap%
\pgfsetroundjoin%
\definecolor{currentfill}{rgb}{0.974072,0.862976,0.688750}%
\pgfsetfillcolor{currentfill}%
\pgfsetlinewidth{0.250937pt}%
\definecolor{currentstroke}{rgb}{1.000000,1.000000,1.000000}%
\pgfsetstrokecolor{currentstroke}%
\pgfsetdash{}{0pt}%
\pgfpathmoveto{\pgfqpoint{1.784717in}{2.523396in}}%
\pgfpathlineto{\pgfqpoint{1.872452in}{2.523396in}}%
\pgfpathlineto{\pgfqpoint{1.872452in}{2.435661in}}%
\pgfpathlineto{\pgfqpoint{1.784717in}{2.435661in}}%
\pgfpathlineto{\pgfqpoint{1.784717in}{2.523396in}}%
\pgfusepath{stroke,fill}%
\end{pgfscope}%
\begin{pgfscope}%
\pgfpathrectangle{\pgfqpoint{0.380943in}{2.260189in}}{\pgfqpoint{4.650000in}{0.614151in}}%
\pgfusepath{clip}%
\pgfsetbuttcap%
\pgfsetroundjoin%
\definecolor{currentfill}{rgb}{0.995233,0.991895,0.818977}%
\pgfsetfillcolor{currentfill}%
\pgfsetlinewidth{0.250937pt}%
\definecolor{currentstroke}{rgb}{1.000000,1.000000,1.000000}%
\pgfsetstrokecolor{currentstroke}%
\pgfsetdash{}{0pt}%
\pgfpathmoveto{\pgfqpoint{1.872452in}{2.523396in}}%
\pgfpathlineto{\pgfqpoint{1.960188in}{2.523396in}}%
\pgfpathlineto{\pgfqpoint{1.960188in}{2.435661in}}%
\pgfpathlineto{\pgfqpoint{1.872452in}{2.435661in}}%
\pgfpathlineto{\pgfqpoint{1.872452in}{2.523396in}}%
\pgfusepath{stroke,fill}%
\end{pgfscope}%
\begin{pgfscope}%
\pgfpathrectangle{\pgfqpoint{0.380943in}{2.260189in}}{\pgfqpoint{4.650000in}{0.614151in}}%
\pgfusepath{clip}%
\pgfsetbuttcap%
\pgfsetroundjoin%
\definecolor{currentfill}{rgb}{0.980008,0.966013,0.779393}%
\pgfsetfillcolor{currentfill}%
\pgfsetlinewidth{0.250937pt}%
\definecolor{currentstroke}{rgb}{1.000000,1.000000,1.000000}%
\pgfsetstrokecolor{currentstroke}%
\pgfsetdash{}{0pt}%
\pgfpathmoveto{\pgfqpoint{1.960188in}{2.523396in}}%
\pgfpathlineto{\pgfqpoint{2.047924in}{2.523396in}}%
\pgfpathlineto{\pgfqpoint{2.047924in}{2.435661in}}%
\pgfpathlineto{\pgfqpoint{1.960188in}{2.435661in}}%
\pgfpathlineto{\pgfqpoint{1.960188in}{2.523396in}}%
\pgfusepath{stroke,fill}%
\end{pgfscope}%
\begin{pgfscope}%
\pgfpathrectangle{\pgfqpoint{0.380943in}{2.260189in}}{\pgfqpoint{4.650000in}{0.614151in}}%
\pgfusepath{clip}%
\pgfsetbuttcap%
\pgfsetroundjoin%
\definecolor{currentfill}{rgb}{0.980008,0.966013,0.779393}%
\pgfsetfillcolor{currentfill}%
\pgfsetlinewidth{0.250937pt}%
\definecolor{currentstroke}{rgb}{1.000000,1.000000,1.000000}%
\pgfsetstrokecolor{currentstroke}%
\pgfsetdash{}{0pt}%
\pgfpathmoveto{\pgfqpoint{2.047924in}{2.523396in}}%
\pgfpathlineto{\pgfqpoint{2.135660in}{2.523396in}}%
\pgfpathlineto{\pgfqpoint{2.135660in}{2.435661in}}%
\pgfpathlineto{\pgfqpoint{2.047924in}{2.435661in}}%
\pgfpathlineto{\pgfqpoint{2.047924in}{2.523396in}}%
\pgfusepath{stroke,fill}%
\end{pgfscope}%
\begin{pgfscope}%
\pgfpathrectangle{\pgfqpoint{0.380943in}{2.260189in}}{\pgfqpoint{4.650000in}{0.614151in}}%
\pgfusepath{clip}%
\pgfsetbuttcap%
\pgfsetroundjoin%
\definecolor{currentfill}{rgb}{0.980008,0.966013,0.779393}%
\pgfsetfillcolor{currentfill}%
\pgfsetlinewidth{0.250937pt}%
\definecolor{currentstroke}{rgb}{1.000000,1.000000,1.000000}%
\pgfsetstrokecolor{currentstroke}%
\pgfsetdash{}{0pt}%
\pgfpathmoveto{\pgfqpoint{2.135660in}{2.523396in}}%
\pgfpathlineto{\pgfqpoint{2.223396in}{2.523396in}}%
\pgfpathlineto{\pgfqpoint{2.223396in}{2.435661in}}%
\pgfpathlineto{\pgfqpoint{2.135660in}{2.435661in}}%
\pgfpathlineto{\pgfqpoint{2.135660in}{2.523396in}}%
\pgfusepath{stroke,fill}%
\end{pgfscope}%
\begin{pgfscope}%
\pgfpathrectangle{\pgfqpoint{0.380943in}{2.260189in}}{\pgfqpoint{4.650000in}{0.614151in}}%
\pgfusepath{clip}%
\pgfsetbuttcap%
\pgfsetroundjoin%
\definecolor{currentfill}{rgb}{0.978639,0.841584,0.673679}%
\pgfsetfillcolor{currentfill}%
\pgfsetlinewidth{0.250937pt}%
\definecolor{currentstroke}{rgb}{1.000000,1.000000,1.000000}%
\pgfsetstrokecolor{currentstroke}%
\pgfsetdash{}{0pt}%
\pgfpathmoveto{\pgfqpoint{2.223396in}{2.523396in}}%
\pgfpathlineto{\pgfqpoint{2.311132in}{2.523396in}}%
\pgfpathlineto{\pgfqpoint{2.311132in}{2.435661in}}%
\pgfpathlineto{\pgfqpoint{2.223396in}{2.435661in}}%
\pgfpathlineto{\pgfqpoint{2.223396in}{2.523396in}}%
\pgfusepath{stroke,fill}%
\end{pgfscope}%
\begin{pgfscope}%
\pgfpathrectangle{\pgfqpoint{0.380943in}{2.260189in}}{\pgfqpoint{4.650000in}{0.614151in}}%
\pgfusepath{clip}%
\pgfsetbuttcap%
\pgfsetroundjoin%
\definecolor{currentfill}{rgb}{0.978639,0.841584,0.673679}%
\pgfsetfillcolor{currentfill}%
\pgfsetlinewidth{0.250937pt}%
\definecolor{currentstroke}{rgb}{1.000000,1.000000,1.000000}%
\pgfsetstrokecolor{currentstroke}%
\pgfsetdash{}{0pt}%
\pgfpathmoveto{\pgfqpoint{2.311132in}{2.523396in}}%
\pgfpathlineto{\pgfqpoint{2.398868in}{2.523396in}}%
\pgfpathlineto{\pgfqpoint{2.398868in}{2.435661in}}%
\pgfpathlineto{\pgfqpoint{2.311132in}{2.435661in}}%
\pgfpathlineto{\pgfqpoint{2.311132in}{2.523396in}}%
\pgfusepath{stroke,fill}%
\end{pgfscope}%
\begin{pgfscope}%
\pgfpathrectangle{\pgfqpoint{0.380943in}{2.260189in}}{\pgfqpoint{4.650000in}{0.614151in}}%
\pgfusepath{clip}%
\pgfsetbuttcap%
\pgfsetroundjoin%
\definecolor{currentfill}{rgb}{0.982699,0.823991,0.657439}%
\pgfsetfillcolor{currentfill}%
\pgfsetlinewidth{0.250937pt}%
\definecolor{currentstroke}{rgb}{1.000000,1.000000,1.000000}%
\pgfsetstrokecolor{currentstroke}%
\pgfsetdash{}{0pt}%
\pgfpathmoveto{\pgfqpoint{2.398868in}{2.523396in}}%
\pgfpathlineto{\pgfqpoint{2.486603in}{2.523396in}}%
\pgfpathlineto{\pgfqpoint{2.486603in}{2.435661in}}%
\pgfpathlineto{\pgfqpoint{2.398868in}{2.435661in}}%
\pgfpathlineto{\pgfqpoint{2.398868in}{2.523396in}}%
\pgfusepath{stroke,fill}%
\end{pgfscope}%
\begin{pgfscope}%
\pgfpathrectangle{\pgfqpoint{0.380943in}{2.260189in}}{\pgfqpoint{4.650000in}{0.614151in}}%
\pgfusepath{clip}%
\pgfsetbuttcap%
\pgfsetroundjoin%
\definecolor{currentfill}{rgb}{0.990634,0.779608,0.623299}%
\pgfsetfillcolor{currentfill}%
\pgfsetlinewidth{0.250937pt}%
\definecolor{currentstroke}{rgb}{1.000000,1.000000,1.000000}%
\pgfsetstrokecolor{currentstroke}%
\pgfsetdash{}{0pt}%
\pgfpathmoveto{\pgfqpoint{2.486603in}{2.523396in}}%
\pgfpathlineto{\pgfqpoint{2.574339in}{2.523396in}}%
\pgfpathlineto{\pgfqpoint{2.574339in}{2.435661in}}%
\pgfpathlineto{\pgfqpoint{2.486603in}{2.435661in}}%
\pgfpathlineto{\pgfqpoint{2.486603in}{2.523396in}}%
\pgfusepath{stroke,fill}%
\end{pgfscope}%
\begin{pgfscope}%
\pgfpathrectangle{\pgfqpoint{0.380943in}{2.260189in}}{\pgfqpoint{4.650000in}{0.614151in}}%
\pgfusepath{clip}%
\pgfsetbuttcap%
\pgfsetroundjoin%
\definecolor{currentfill}{rgb}{0.993679,0.753725,0.608074}%
\pgfsetfillcolor{currentfill}%
\pgfsetlinewidth{0.250937pt}%
\definecolor{currentstroke}{rgb}{1.000000,1.000000,1.000000}%
\pgfsetstrokecolor{currentstroke}%
\pgfsetdash{}{0pt}%
\pgfpathmoveto{\pgfqpoint{2.574339in}{2.523396in}}%
\pgfpathlineto{\pgfqpoint{2.662075in}{2.523396in}}%
\pgfpathlineto{\pgfqpoint{2.662075in}{2.435661in}}%
\pgfpathlineto{\pgfqpoint{2.574339in}{2.435661in}}%
\pgfpathlineto{\pgfqpoint{2.574339in}{2.523396in}}%
\pgfusepath{stroke,fill}%
\end{pgfscope}%
\begin{pgfscope}%
\pgfpathrectangle{\pgfqpoint{0.380943in}{2.260189in}}{\pgfqpoint{4.650000in}{0.614151in}}%
\pgfusepath{clip}%
\pgfsetbuttcap%
\pgfsetroundjoin%
\definecolor{currentfill}{rgb}{0.982699,0.823991,0.657439}%
\pgfsetfillcolor{currentfill}%
\pgfsetlinewidth{0.250937pt}%
\definecolor{currentstroke}{rgb}{1.000000,1.000000,1.000000}%
\pgfsetstrokecolor{currentstroke}%
\pgfsetdash{}{0pt}%
\pgfpathmoveto{\pgfqpoint{2.662075in}{2.523396in}}%
\pgfpathlineto{\pgfqpoint{2.749811in}{2.523396in}}%
\pgfpathlineto{\pgfqpoint{2.749811in}{2.435661in}}%
\pgfpathlineto{\pgfqpoint{2.662075in}{2.435661in}}%
\pgfpathlineto{\pgfqpoint{2.662075in}{2.523396in}}%
\pgfusepath{stroke,fill}%
\end{pgfscope}%
\begin{pgfscope}%
\pgfpathrectangle{\pgfqpoint{0.380943in}{2.260189in}}{\pgfqpoint{4.650000in}{0.614151in}}%
\pgfusepath{clip}%
\pgfsetbuttcap%
\pgfsetroundjoin%
\definecolor{currentfill}{rgb}{0.993679,0.753725,0.608074}%
\pgfsetfillcolor{currentfill}%
\pgfsetlinewidth{0.250937pt}%
\definecolor{currentstroke}{rgb}{1.000000,1.000000,1.000000}%
\pgfsetstrokecolor{currentstroke}%
\pgfsetdash{}{0pt}%
\pgfpathmoveto{\pgfqpoint{2.749811in}{2.523396in}}%
\pgfpathlineto{\pgfqpoint{2.837547in}{2.523396in}}%
\pgfpathlineto{\pgfqpoint{2.837547in}{2.435661in}}%
\pgfpathlineto{\pgfqpoint{2.749811in}{2.435661in}}%
\pgfpathlineto{\pgfqpoint{2.749811in}{2.523396in}}%
\pgfusepath{stroke,fill}%
\end{pgfscope}%
\begin{pgfscope}%
\pgfpathrectangle{\pgfqpoint{0.380943in}{2.260189in}}{\pgfqpoint{4.650000in}{0.614151in}}%
\pgfusepath{clip}%
\pgfsetbuttcap%
\pgfsetroundjoin%
\definecolor{currentfill}{rgb}{0.800000,0.278431,0.278431}%
\pgfsetfillcolor{currentfill}%
\pgfsetlinewidth{0.250937pt}%
\definecolor{currentstroke}{rgb}{1.000000,1.000000,1.000000}%
\pgfsetstrokecolor{currentstroke}%
\pgfsetdash{}{0pt}%
\pgfpathmoveto{\pgfqpoint{2.837547in}{2.523396in}}%
\pgfpathlineto{\pgfqpoint{2.925283in}{2.523396in}}%
\pgfpathlineto{\pgfqpoint{2.925283in}{2.435661in}}%
\pgfpathlineto{\pgfqpoint{2.837547in}{2.435661in}}%
\pgfpathlineto{\pgfqpoint{2.837547in}{2.523396in}}%
\pgfusepath{stroke,fill}%
\end{pgfscope}%
\begin{pgfscope}%
\pgfpathrectangle{\pgfqpoint{0.380943in}{2.260189in}}{\pgfqpoint{4.650000in}{0.614151in}}%
\pgfusepath{clip}%
\pgfsetbuttcap%
\pgfsetroundjoin%
\definecolor{currentfill}{rgb}{0.964937,0.908651,0.713110}%
\pgfsetfillcolor{currentfill}%
\pgfsetlinewidth{0.250937pt}%
\definecolor{currentstroke}{rgb}{1.000000,1.000000,1.000000}%
\pgfsetstrokecolor{currentstroke}%
\pgfsetdash{}{0pt}%
\pgfpathmoveto{\pgfqpoint{2.925283in}{2.523396in}}%
\pgfpathlineto{\pgfqpoint{3.013019in}{2.523396in}}%
\pgfpathlineto{\pgfqpoint{3.013019in}{2.435661in}}%
\pgfpathlineto{\pgfqpoint{2.925283in}{2.435661in}}%
\pgfpathlineto{\pgfqpoint{2.925283in}{2.523396in}}%
\pgfusepath{stroke,fill}%
\end{pgfscope}%
\begin{pgfscope}%
\pgfpathrectangle{\pgfqpoint{0.380943in}{2.260189in}}{\pgfqpoint{4.650000in}{0.614151in}}%
\pgfusepath{clip}%
\pgfsetbuttcap%
\pgfsetroundjoin%
\definecolor{currentfill}{rgb}{0.969504,0.885813,0.700930}%
\pgfsetfillcolor{currentfill}%
\pgfsetlinewidth{0.250937pt}%
\definecolor{currentstroke}{rgb}{1.000000,1.000000,1.000000}%
\pgfsetstrokecolor{currentstroke}%
\pgfsetdash{}{0pt}%
\pgfpathmoveto{\pgfqpoint{3.013019in}{2.523396in}}%
\pgfpathlineto{\pgfqpoint{3.100754in}{2.523396in}}%
\pgfpathlineto{\pgfqpoint{3.100754in}{2.435661in}}%
\pgfpathlineto{\pgfqpoint{3.013019in}{2.435661in}}%
\pgfpathlineto{\pgfqpoint{3.013019in}{2.523396in}}%
\pgfusepath{stroke,fill}%
\end{pgfscope}%
\begin{pgfscope}%
\pgfpathrectangle{\pgfqpoint{0.380943in}{2.260189in}}{\pgfqpoint{4.650000in}{0.614151in}}%
\pgfusepath{clip}%
\pgfsetbuttcap%
\pgfsetroundjoin%
\definecolor{currentfill}{rgb}{0.993679,0.753725,0.608074}%
\pgfsetfillcolor{currentfill}%
\pgfsetlinewidth{0.250937pt}%
\definecolor{currentstroke}{rgb}{1.000000,1.000000,1.000000}%
\pgfsetstrokecolor{currentstroke}%
\pgfsetdash{}{0pt}%
\pgfpathmoveto{\pgfqpoint{3.100754in}{2.523396in}}%
\pgfpathlineto{\pgfqpoint{3.188490in}{2.523396in}}%
\pgfpathlineto{\pgfqpoint{3.188490in}{2.435661in}}%
\pgfpathlineto{\pgfqpoint{3.100754in}{2.435661in}}%
\pgfpathlineto{\pgfqpoint{3.100754in}{2.523396in}}%
\pgfusepath{stroke,fill}%
\end{pgfscope}%
\begin{pgfscope}%
\pgfpathrectangle{\pgfqpoint{0.380943in}{2.260189in}}{\pgfqpoint{4.650000in}{0.614151in}}%
\pgfusepath{clip}%
\pgfsetbuttcap%
\pgfsetroundjoin%
\definecolor{currentfill}{rgb}{0.974072,0.862976,0.688750}%
\pgfsetfillcolor{currentfill}%
\pgfsetlinewidth{0.250937pt}%
\definecolor{currentstroke}{rgb}{1.000000,1.000000,1.000000}%
\pgfsetstrokecolor{currentstroke}%
\pgfsetdash{}{0pt}%
\pgfpathmoveto{\pgfqpoint{3.188490in}{2.523396in}}%
\pgfpathlineto{\pgfqpoint{3.276226in}{2.523396in}}%
\pgfpathlineto{\pgfqpoint{3.276226in}{2.435661in}}%
\pgfpathlineto{\pgfqpoint{3.188490in}{2.435661in}}%
\pgfpathlineto{\pgfqpoint{3.188490in}{2.523396in}}%
\pgfusepath{stroke,fill}%
\end{pgfscope}%
\begin{pgfscope}%
\pgfpathrectangle{\pgfqpoint{0.380943in}{2.260189in}}{\pgfqpoint{4.650000in}{0.614151in}}%
\pgfusepath{clip}%
\pgfsetbuttcap%
\pgfsetroundjoin%
\definecolor{currentfill}{rgb}{0.995233,0.991895,0.818977}%
\pgfsetfillcolor{currentfill}%
\pgfsetlinewidth{0.250937pt}%
\definecolor{currentstroke}{rgb}{1.000000,1.000000,1.000000}%
\pgfsetstrokecolor{currentstroke}%
\pgfsetdash{}{0pt}%
\pgfpathmoveto{\pgfqpoint{3.276226in}{2.523396in}}%
\pgfpathlineto{\pgfqpoint{3.363962in}{2.523396in}}%
\pgfpathlineto{\pgfqpoint{3.363962in}{2.435661in}}%
\pgfpathlineto{\pgfqpoint{3.276226in}{2.435661in}}%
\pgfpathlineto{\pgfqpoint{3.276226in}{2.523396in}}%
\pgfusepath{stroke,fill}%
\end{pgfscope}%
\begin{pgfscope}%
\pgfpathrectangle{\pgfqpoint{0.380943in}{2.260189in}}{\pgfqpoint{4.650000in}{0.614151in}}%
\pgfusepath{clip}%
\pgfsetbuttcap%
\pgfsetroundjoin%
\definecolor{currentfill}{rgb}{0.969504,0.885813,0.700930}%
\pgfsetfillcolor{currentfill}%
\pgfsetlinewidth{0.250937pt}%
\definecolor{currentstroke}{rgb}{1.000000,1.000000,1.000000}%
\pgfsetstrokecolor{currentstroke}%
\pgfsetdash{}{0pt}%
\pgfpathmoveto{\pgfqpoint{3.363962in}{2.523396in}}%
\pgfpathlineto{\pgfqpoint{3.451698in}{2.523396in}}%
\pgfpathlineto{\pgfqpoint{3.451698in}{2.435661in}}%
\pgfpathlineto{\pgfqpoint{3.363962in}{2.435661in}}%
\pgfpathlineto{\pgfqpoint{3.363962in}{2.523396in}}%
\pgfusepath{stroke,fill}%
\end{pgfscope}%
\begin{pgfscope}%
\pgfpathrectangle{\pgfqpoint{0.380943in}{2.260189in}}{\pgfqpoint{4.650000in}{0.614151in}}%
\pgfusepath{clip}%
\pgfsetbuttcap%
\pgfsetroundjoin%
\definecolor{currentfill}{rgb}{0.961738,0.927612,0.725598}%
\pgfsetfillcolor{currentfill}%
\pgfsetlinewidth{0.250937pt}%
\definecolor{currentstroke}{rgb}{1.000000,1.000000,1.000000}%
\pgfsetstrokecolor{currentstroke}%
\pgfsetdash{}{0pt}%
\pgfpathmoveto{\pgfqpoint{3.451698in}{2.523396in}}%
\pgfpathlineto{\pgfqpoint{3.539434in}{2.523396in}}%
\pgfpathlineto{\pgfqpoint{3.539434in}{2.435661in}}%
\pgfpathlineto{\pgfqpoint{3.451698in}{2.435661in}}%
\pgfpathlineto{\pgfqpoint{3.451698in}{2.523396in}}%
\pgfusepath{stroke,fill}%
\end{pgfscope}%
\begin{pgfscope}%
\pgfpathrectangle{\pgfqpoint{0.380943in}{2.260189in}}{\pgfqpoint{4.650000in}{0.614151in}}%
\pgfusepath{clip}%
\pgfsetbuttcap%
\pgfsetroundjoin%
\definecolor{currentfill}{rgb}{0.978639,0.841584,0.673679}%
\pgfsetfillcolor{currentfill}%
\pgfsetlinewidth{0.250937pt}%
\definecolor{currentstroke}{rgb}{1.000000,1.000000,1.000000}%
\pgfsetstrokecolor{currentstroke}%
\pgfsetdash{}{0pt}%
\pgfpathmoveto{\pgfqpoint{3.539434in}{2.523396in}}%
\pgfpathlineto{\pgfqpoint{3.627169in}{2.523396in}}%
\pgfpathlineto{\pgfqpoint{3.627169in}{2.435661in}}%
\pgfpathlineto{\pgfqpoint{3.539434in}{2.435661in}}%
\pgfpathlineto{\pgfqpoint{3.539434in}{2.523396in}}%
\pgfusepath{stroke,fill}%
\end{pgfscope}%
\begin{pgfscope}%
\pgfpathrectangle{\pgfqpoint{0.380943in}{2.260189in}}{\pgfqpoint{4.650000in}{0.614151in}}%
\pgfusepath{clip}%
\pgfsetbuttcap%
\pgfsetroundjoin%
\definecolor{currentfill}{rgb}{0.996401,0.724937,0.591557}%
\pgfsetfillcolor{currentfill}%
\pgfsetlinewidth{0.250937pt}%
\definecolor{currentstroke}{rgb}{1.000000,1.000000,1.000000}%
\pgfsetstrokecolor{currentstroke}%
\pgfsetdash{}{0pt}%
\pgfpathmoveto{\pgfqpoint{3.627169in}{2.523396in}}%
\pgfpathlineto{\pgfqpoint{3.714905in}{2.523396in}}%
\pgfpathlineto{\pgfqpoint{3.714905in}{2.435661in}}%
\pgfpathlineto{\pgfqpoint{3.627169in}{2.435661in}}%
\pgfpathlineto{\pgfqpoint{3.627169in}{2.523396in}}%
\pgfusepath{stroke,fill}%
\end{pgfscope}%
\begin{pgfscope}%
\pgfpathrectangle{\pgfqpoint{0.380943in}{2.260189in}}{\pgfqpoint{4.650000in}{0.614151in}}%
\pgfusepath{clip}%
\pgfsetbuttcap%
\pgfsetroundjoin%
\definecolor{currentfill}{rgb}{0.982699,0.823991,0.657439}%
\pgfsetfillcolor{currentfill}%
\pgfsetlinewidth{0.250937pt}%
\definecolor{currentstroke}{rgb}{1.000000,1.000000,1.000000}%
\pgfsetstrokecolor{currentstroke}%
\pgfsetdash{}{0pt}%
\pgfpathmoveto{\pgfqpoint{3.714905in}{2.523396in}}%
\pgfpathlineto{\pgfqpoint{3.802641in}{2.523396in}}%
\pgfpathlineto{\pgfqpoint{3.802641in}{2.435661in}}%
\pgfpathlineto{\pgfqpoint{3.714905in}{2.435661in}}%
\pgfpathlineto{\pgfqpoint{3.714905in}{2.523396in}}%
\pgfusepath{stroke,fill}%
\end{pgfscope}%
\begin{pgfscope}%
\pgfpathrectangle{\pgfqpoint{0.380943in}{2.260189in}}{\pgfqpoint{4.650000in}{0.614151in}}%
\pgfusepath{clip}%
\pgfsetbuttcap%
\pgfsetroundjoin%
\definecolor{currentfill}{rgb}{0.993679,0.753725,0.608074}%
\pgfsetfillcolor{currentfill}%
\pgfsetlinewidth{0.250937pt}%
\definecolor{currentstroke}{rgb}{1.000000,1.000000,1.000000}%
\pgfsetstrokecolor{currentstroke}%
\pgfsetdash{}{0pt}%
\pgfpathmoveto{\pgfqpoint{3.802641in}{2.523396in}}%
\pgfpathlineto{\pgfqpoint{3.890377in}{2.523396in}}%
\pgfpathlineto{\pgfqpoint{3.890377in}{2.435661in}}%
\pgfpathlineto{\pgfqpoint{3.802641in}{2.435661in}}%
\pgfpathlineto{\pgfqpoint{3.802641in}{2.523396in}}%
\pgfusepath{stroke,fill}%
\end{pgfscope}%
\begin{pgfscope}%
\pgfpathrectangle{\pgfqpoint{0.380943in}{2.260189in}}{\pgfqpoint{4.650000in}{0.614151in}}%
\pgfusepath{clip}%
\pgfsetbuttcap%
\pgfsetroundjoin%
\definecolor{currentfill}{rgb}{1.000000,0.615379,0.534779}%
\pgfsetfillcolor{currentfill}%
\pgfsetlinewidth{0.250937pt}%
\definecolor{currentstroke}{rgb}{1.000000,1.000000,1.000000}%
\pgfsetstrokecolor{currentstroke}%
\pgfsetdash{}{0pt}%
\pgfpathmoveto{\pgfqpoint{3.890377in}{2.523396in}}%
\pgfpathlineto{\pgfqpoint{3.978113in}{2.523396in}}%
\pgfpathlineto{\pgfqpoint{3.978113in}{2.435661in}}%
\pgfpathlineto{\pgfqpoint{3.890377in}{2.435661in}}%
\pgfpathlineto{\pgfqpoint{3.890377in}{2.523396in}}%
\pgfusepath{stroke,fill}%
\end{pgfscope}%
\begin{pgfscope}%
\pgfpathrectangle{\pgfqpoint{0.380943in}{2.260189in}}{\pgfqpoint{4.650000in}{0.614151in}}%
\pgfusepath{clip}%
\pgfsetbuttcap%
\pgfsetroundjoin%
\definecolor{currentfill}{rgb}{0.964937,0.908651,0.713110}%
\pgfsetfillcolor{currentfill}%
\pgfsetlinewidth{0.250937pt}%
\definecolor{currentstroke}{rgb}{1.000000,1.000000,1.000000}%
\pgfsetstrokecolor{currentstroke}%
\pgfsetdash{}{0pt}%
\pgfpathmoveto{\pgfqpoint{3.978113in}{2.523396in}}%
\pgfpathlineto{\pgfqpoint{4.065849in}{2.523396in}}%
\pgfpathlineto{\pgfqpoint{4.065849in}{2.435661in}}%
\pgfpathlineto{\pgfqpoint{3.978113in}{2.435661in}}%
\pgfpathlineto{\pgfqpoint{3.978113in}{2.523396in}}%
\pgfusepath{stroke,fill}%
\end{pgfscope}%
\begin{pgfscope}%
\pgfpathrectangle{\pgfqpoint{0.380943in}{2.260189in}}{\pgfqpoint{4.650000in}{0.614151in}}%
\pgfusepath{clip}%
\pgfsetbuttcap%
\pgfsetroundjoin%
\definecolor{currentfill}{rgb}{0.997924,0.685352,0.570242}%
\pgfsetfillcolor{currentfill}%
\pgfsetlinewidth{0.250937pt}%
\definecolor{currentstroke}{rgb}{1.000000,1.000000,1.000000}%
\pgfsetstrokecolor{currentstroke}%
\pgfsetdash{}{0pt}%
\pgfpathmoveto{\pgfqpoint{4.065849in}{2.523396in}}%
\pgfpathlineto{\pgfqpoint{4.153585in}{2.523396in}}%
\pgfpathlineto{\pgfqpoint{4.153585in}{2.435661in}}%
\pgfpathlineto{\pgfqpoint{4.065849in}{2.435661in}}%
\pgfpathlineto{\pgfqpoint{4.065849in}{2.523396in}}%
\pgfusepath{stroke,fill}%
\end{pgfscope}%
\begin{pgfscope}%
\pgfpathrectangle{\pgfqpoint{0.380943in}{2.260189in}}{\pgfqpoint{4.650000in}{0.614151in}}%
\pgfusepath{clip}%
\pgfsetbuttcap%
\pgfsetroundjoin%
\definecolor{currentfill}{rgb}{1.000000,0.584929,0.522599}%
\pgfsetfillcolor{currentfill}%
\pgfsetlinewidth{0.250937pt}%
\definecolor{currentstroke}{rgb}{1.000000,1.000000,1.000000}%
\pgfsetstrokecolor{currentstroke}%
\pgfsetdash{}{0pt}%
\pgfpathmoveto{\pgfqpoint{4.153585in}{2.523396in}}%
\pgfpathlineto{\pgfqpoint{4.241320in}{2.523396in}}%
\pgfpathlineto{\pgfqpoint{4.241320in}{2.435661in}}%
\pgfpathlineto{\pgfqpoint{4.153585in}{2.435661in}}%
\pgfpathlineto{\pgfqpoint{4.153585in}{2.523396in}}%
\pgfusepath{stroke,fill}%
\end{pgfscope}%
\begin{pgfscope}%
\pgfpathrectangle{\pgfqpoint{0.380943in}{2.260189in}}{\pgfqpoint{4.650000in}{0.614151in}}%
\pgfusepath{clip}%
\pgfsetbuttcap%
\pgfsetroundjoin%
\definecolor{currentfill}{rgb}{0.974072,0.862976,0.688750}%
\pgfsetfillcolor{currentfill}%
\pgfsetlinewidth{0.250937pt}%
\definecolor{currentstroke}{rgb}{1.000000,1.000000,1.000000}%
\pgfsetstrokecolor{currentstroke}%
\pgfsetdash{}{0pt}%
\pgfpathmoveto{\pgfqpoint{4.241320in}{2.523396in}}%
\pgfpathlineto{\pgfqpoint{4.329056in}{2.523396in}}%
\pgfpathlineto{\pgfqpoint{4.329056in}{2.435661in}}%
\pgfpathlineto{\pgfqpoint{4.241320in}{2.435661in}}%
\pgfpathlineto{\pgfqpoint{4.241320in}{2.523396in}}%
\pgfusepath{stroke,fill}%
\end{pgfscope}%
\begin{pgfscope}%
\pgfpathrectangle{\pgfqpoint{0.380943in}{2.260189in}}{\pgfqpoint{4.650000in}{0.614151in}}%
\pgfusepath{clip}%
\pgfsetbuttcap%
\pgfsetroundjoin%
\definecolor{currentfill}{rgb}{1.000000,0.496547,0.486059}%
\pgfsetfillcolor{currentfill}%
\pgfsetlinewidth{0.250937pt}%
\definecolor{currentstroke}{rgb}{1.000000,1.000000,1.000000}%
\pgfsetstrokecolor{currentstroke}%
\pgfsetdash{}{0pt}%
\pgfpathmoveto{\pgfqpoint{4.329056in}{2.523396in}}%
\pgfpathlineto{\pgfqpoint{4.416792in}{2.523396in}}%
\pgfpathlineto{\pgfqpoint{4.416792in}{2.435661in}}%
\pgfpathlineto{\pgfqpoint{4.329056in}{2.435661in}}%
\pgfpathlineto{\pgfqpoint{4.329056in}{2.523396in}}%
\pgfusepath{stroke,fill}%
\end{pgfscope}%
\begin{pgfscope}%
\pgfpathrectangle{\pgfqpoint{0.380943in}{2.260189in}}{\pgfqpoint{4.650000in}{0.614151in}}%
\pgfusepath{clip}%
\pgfsetbuttcap%
\pgfsetroundjoin%
\definecolor{currentfill}{rgb}{0.997924,0.685352,0.570242}%
\pgfsetfillcolor{currentfill}%
\pgfsetlinewidth{0.250937pt}%
\definecolor{currentstroke}{rgb}{1.000000,1.000000,1.000000}%
\pgfsetstrokecolor{currentstroke}%
\pgfsetdash{}{0pt}%
\pgfpathmoveto{\pgfqpoint{4.416792in}{2.523396in}}%
\pgfpathlineto{\pgfqpoint{4.504528in}{2.523396in}}%
\pgfpathlineto{\pgfqpoint{4.504528in}{2.435661in}}%
\pgfpathlineto{\pgfqpoint{4.416792in}{2.435661in}}%
\pgfpathlineto{\pgfqpoint{4.416792in}{2.523396in}}%
\pgfusepath{stroke,fill}%
\end{pgfscope}%
\begin{pgfscope}%
\pgfpathrectangle{\pgfqpoint{0.380943in}{2.260189in}}{\pgfqpoint{4.650000in}{0.614151in}}%
\pgfusepath{clip}%
\pgfsetbuttcap%
\pgfsetroundjoin%
\definecolor{currentfill}{rgb}{0.982699,0.823991,0.657439}%
\pgfsetfillcolor{currentfill}%
\pgfsetlinewidth{0.250937pt}%
\definecolor{currentstroke}{rgb}{1.000000,1.000000,1.000000}%
\pgfsetstrokecolor{currentstroke}%
\pgfsetdash{}{0pt}%
\pgfpathmoveto{\pgfqpoint{4.504528in}{2.523396in}}%
\pgfpathlineto{\pgfqpoint{4.592264in}{2.523396in}}%
\pgfpathlineto{\pgfqpoint{4.592264in}{2.435661in}}%
\pgfpathlineto{\pgfqpoint{4.504528in}{2.435661in}}%
\pgfpathlineto{\pgfqpoint{4.504528in}{2.523396in}}%
\pgfusepath{stroke,fill}%
\end{pgfscope}%
\begin{pgfscope}%
\pgfpathrectangle{\pgfqpoint{0.380943in}{2.260189in}}{\pgfqpoint{4.650000in}{0.614151in}}%
\pgfusepath{clip}%
\pgfsetbuttcap%
\pgfsetroundjoin%
\definecolor{currentfill}{rgb}{0.969504,0.885813,0.700930}%
\pgfsetfillcolor{currentfill}%
\pgfsetlinewidth{0.250937pt}%
\definecolor{currentstroke}{rgb}{1.000000,1.000000,1.000000}%
\pgfsetstrokecolor{currentstroke}%
\pgfsetdash{}{0pt}%
\pgfpathmoveto{\pgfqpoint{4.592264in}{2.523396in}}%
\pgfpathlineto{\pgfqpoint{4.680000in}{2.523396in}}%
\pgfpathlineto{\pgfqpoint{4.680000in}{2.435661in}}%
\pgfpathlineto{\pgfqpoint{4.592264in}{2.435661in}}%
\pgfpathlineto{\pgfqpoint{4.592264in}{2.523396in}}%
\pgfusepath{stroke,fill}%
\end{pgfscope}%
\begin{pgfscope}%
\pgfpathrectangle{\pgfqpoint{0.380943in}{2.260189in}}{\pgfqpoint{4.650000in}{0.614151in}}%
\pgfusepath{clip}%
\pgfsetbuttcap%
\pgfsetroundjoin%
\definecolor{currentfill}{rgb}{1.000000,0.554479,0.510419}%
\pgfsetfillcolor{currentfill}%
\pgfsetlinewidth{0.250937pt}%
\definecolor{currentstroke}{rgb}{1.000000,1.000000,1.000000}%
\pgfsetstrokecolor{currentstroke}%
\pgfsetdash{}{0pt}%
\pgfpathmoveto{\pgfqpoint{4.680000in}{2.523396in}}%
\pgfpathlineto{\pgfqpoint{4.767736in}{2.523396in}}%
\pgfpathlineto{\pgfqpoint{4.767736in}{2.435661in}}%
\pgfpathlineto{\pgfqpoint{4.680000in}{2.435661in}}%
\pgfpathlineto{\pgfqpoint{4.680000in}{2.523396in}}%
\pgfusepath{stroke,fill}%
\end{pgfscope}%
\begin{pgfscope}%
\pgfpathrectangle{\pgfqpoint{0.380943in}{2.260189in}}{\pgfqpoint{4.650000in}{0.614151in}}%
\pgfusepath{clip}%
\pgfsetbuttcap%
\pgfsetroundjoin%
\definecolor{currentfill}{rgb}{0.990634,0.779608,0.623299}%
\pgfsetfillcolor{currentfill}%
\pgfsetlinewidth{0.250937pt}%
\definecolor{currentstroke}{rgb}{1.000000,1.000000,1.000000}%
\pgfsetstrokecolor{currentstroke}%
\pgfsetdash{}{0pt}%
\pgfpathmoveto{\pgfqpoint{4.767736in}{2.523396in}}%
\pgfpathlineto{\pgfqpoint{4.855471in}{2.523396in}}%
\pgfpathlineto{\pgfqpoint{4.855471in}{2.435661in}}%
\pgfpathlineto{\pgfqpoint{4.767736in}{2.435661in}}%
\pgfpathlineto{\pgfqpoint{4.767736in}{2.523396in}}%
\pgfusepath{stroke,fill}%
\end{pgfscope}%
\begin{pgfscope}%
\pgfpathrectangle{\pgfqpoint{0.380943in}{2.260189in}}{\pgfqpoint{4.650000in}{0.614151in}}%
\pgfusepath{clip}%
\pgfsetbuttcap%
\pgfsetroundjoin%
\definecolor{currentfill}{rgb}{0.964783,0.940131,0.739808}%
\pgfsetfillcolor{currentfill}%
\pgfsetlinewidth{0.250937pt}%
\definecolor{currentstroke}{rgb}{1.000000,1.000000,1.000000}%
\pgfsetstrokecolor{currentstroke}%
\pgfsetdash{}{0pt}%
\pgfpathmoveto{\pgfqpoint{4.855471in}{2.523396in}}%
\pgfpathlineto{\pgfqpoint{4.943207in}{2.523396in}}%
\pgfpathlineto{\pgfqpoint{4.943207in}{2.435661in}}%
\pgfpathlineto{\pgfqpoint{4.855471in}{2.435661in}}%
\pgfpathlineto{\pgfqpoint{4.855471in}{2.523396in}}%
\pgfusepath{stroke,fill}%
\end{pgfscope}%
\begin{pgfscope}%
\pgfpathrectangle{\pgfqpoint{0.380943in}{2.260189in}}{\pgfqpoint{4.650000in}{0.614151in}}%
\pgfusepath{clip}%
\pgfsetbuttcap%
\pgfsetroundjoin%
\pgfsetlinewidth{0.250937pt}%
\definecolor{currentstroke}{rgb}{1.000000,1.000000,1.000000}%
\pgfsetstrokecolor{currentstroke}%
\pgfsetdash{}{0pt}%
\pgfpathmoveto{\pgfqpoint{4.943207in}{2.523396in}}%
\pgfpathlineto{\pgfqpoint{5.030943in}{2.523396in}}%
\pgfpathlineto{\pgfqpoint{5.030943in}{2.435661in}}%
\pgfpathlineto{\pgfqpoint{4.943207in}{2.435661in}}%
\pgfpathlineto{\pgfqpoint{4.943207in}{2.523396in}}%
\pgfusepath{stroke}%
\end{pgfscope}%
\begin{pgfscope}%
\pgfpathrectangle{\pgfqpoint{0.380943in}{2.260189in}}{\pgfqpoint{4.650000in}{0.614151in}}%
\pgfusepath{clip}%
\pgfsetbuttcap%
\pgfsetroundjoin%
\definecolor{currentfill}{rgb}{0.980008,0.966013,0.779393}%
\pgfsetfillcolor{currentfill}%
\pgfsetlinewidth{0.250937pt}%
\definecolor{currentstroke}{rgb}{1.000000,1.000000,1.000000}%
\pgfsetstrokecolor{currentstroke}%
\pgfsetdash{}{0pt}%
\pgfpathmoveto{\pgfqpoint{0.380943in}{2.435661in}}%
\pgfpathlineto{\pgfqpoint{0.468679in}{2.435661in}}%
\pgfpathlineto{\pgfqpoint{0.468679in}{2.347925in}}%
\pgfpathlineto{\pgfqpoint{0.380943in}{2.347925in}}%
\pgfpathlineto{\pgfqpoint{0.380943in}{2.435661in}}%
\pgfusepath{stroke,fill}%
\end{pgfscope}%
\begin{pgfscope}%
\pgfpathrectangle{\pgfqpoint{0.380943in}{2.260189in}}{\pgfqpoint{4.650000in}{0.614151in}}%
\pgfusepath{clip}%
\pgfsetbuttcap%
\pgfsetroundjoin%
\definecolor{currentfill}{rgb}{0.961738,0.927612,0.725598}%
\pgfsetfillcolor{currentfill}%
\pgfsetlinewidth{0.250937pt}%
\definecolor{currentstroke}{rgb}{1.000000,1.000000,1.000000}%
\pgfsetstrokecolor{currentstroke}%
\pgfsetdash{}{0pt}%
\pgfpathmoveto{\pgfqpoint{0.468679in}{2.435661in}}%
\pgfpathlineto{\pgfqpoint{0.556415in}{2.435661in}}%
\pgfpathlineto{\pgfqpoint{0.556415in}{2.347925in}}%
\pgfpathlineto{\pgfqpoint{0.468679in}{2.347925in}}%
\pgfpathlineto{\pgfqpoint{0.468679in}{2.435661in}}%
\pgfusepath{stroke,fill}%
\end{pgfscope}%
\begin{pgfscope}%
\pgfpathrectangle{\pgfqpoint{0.380943in}{2.260189in}}{\pgfqpoint{4.650000in}{0.614151in}}%
\pgfusepath{clip}%
\pgfsetbuttcap%
\pgfsetroundjoin%
\definecolor{currentfill}{rgb}{0.995233,0.991895,0.818977}%
\pgfsetfillcolor{currentfill}%
\pgfsetlinewidth{0.250937pt}%
\definecolor{currentstroke}{rgb}{1.000000,1.000000,1.000000}%
\pgfsetstrokecolor{currentstroke}%
\pgfsetdash{}{0pt}%
\pgfpathmoveto{\pgfqpoint{0.556415in}{2.435661in}}%
\pgfpathlineto{\pgfqpoint{0.644151in}{2.435661in}}%
\pgfpathlineto{\pgfqpoint{0.644151in}{2.347925in}}%
\pgfpathlineto{\pgfqpoint{0.556415in}{2.347925in}}%
\pgfpathlineto{\pgfqpoint{0.556415in}{2.435661in}}%
\pgfusepath{stroke,fill}%
\end{pgfscope}%
\begin{pgfscope}%
\pgfpathrectangle{\pgfqpoint{0.380943in}{2.260189in}}{\pgfqpoint{4.650000in}{0.614151in}}%
\pgfusepath{clip}%
\pgfsetbuttcap%
\pgfsetroundjoin%
\definecolor{currentfill}{rgb}{0.961738,0.927612,0.725598}%
\pgfsetfillcolor{currentfill}%
\pgfsetlinewidth{0.250937pt}%
\definecolor{currentstroke}{rgb}{1.000000,1.000000,1.000000}%
\pgfsetstrokecolor{currentstroke}%
\pgfsetdash{}{0pt}%
\pgfpathmoveto{\pgfqpoint{0.644151in}{2.435661in}}%
\pgfpathlineto{\pgfqpoint{0.731886in}{2.435661in}}%
\pgfpathlineto{\pgfqpoint{0.731886in}{2.347925in}}%
\pgfpathlineto{\pgfqpoint{0.644151in}{2.347925in}}%
\pgfpathlineto{\pgfqpoint{0.644151in}{2.435661in}}%
\pgfusepath{stroke,fill}%
\end{pgfscope}%
\begin{pgfscope}%
\pgfpathrectangle{\pgfqpoint{0.380943in}{2.260189in}}{\pgfqpoint{4.650000in}{0.614151in}}%
\pgfusepath{clip}%
\pgfsetbuttcap%
\pgfsetroundjoin%
\definecolor{currentfill}{rgb}{0.980008,0.966013,0.779393}%
\pgfsetfillcolor{currentfill}%
\pgfsetlinewidth{0.250937pt}%
\definecolor{currentstroke}{rgb}{1.000000,1.000000,1.000000}%
\pgfsetstrokecolor{currentstroke}%
\pgfsetdash{}{0pt}%
\pgfpathmoveto{\pgfqpoint{0.731886in}{2.435661in}}%
\pgfpathlineto{\pgfqpoint{0.819622in}{2.435661in}}%
\pgfpathlineto{\pgfqpoint{0.819622in}{2.347925in}}%
\pgfpathlineto{\pgfqpoint{0.731886in}{2.347925in}}%
\pgfpathlineto{\pgfqpoint{0.731886in}{2.435661in}}%
\pgfusepath{stroke,fill}%
\end{pgfscope}%
\begin{pgfscope}%
\pgfpathrectangle{\pgfqpoint{0.380943in}{2.260189in}}{\pgfqpoint{4.650000in}{0.614151in}}%
\pgfusepath{clip}%
\pgfsetbuttcap%
\pgfsetroundjoin%
\definecolor{currentfill}{rgb}{0.978639,0.841584,0.673679}%
\pgfsetfillcolor{currentfill}%
\pgfsetlinewidth{0.250937pt}%
\definecolor{currentstroke}{rgb}{1.000000,1.000000,1.000000}%
\pgfsetstrokecolor{currentstroke}%
\pgfsetdash{}{0pt}%
\pgfpathmoveto{\pgfqpoint{0.819622in}{2.435661in}}%
\pgfpathlineto{\pgfqpoint{0.907358in}{2.435661in}}%
\pgfpathlineto{\pgfqpoint{0.907358in}{2.347925in}}%
\pgfpathlineto{\pgfqpoint{0.819622in}{2.347925in}}%
\pgfpathlineto{\pgfqpoint{0.819622in}{2.435661in}}%
\pgfusepath{stroke,fill}%
\end{pgfscope}%
\begin{pgfscope}%
\pgfpathrectangle{\pgfqpoint{0.380943in}{2.260189in}}{\pgfqpoint{4.650000in}{0.614151in}}%
\pgfusepath{clip}%
\pgfsetbuttcap%
\pgfsetroundjoin%
\definecolor{currentfill}{rgb}{0.980008,0.966013,0.779393}%
\pgfsetfillcolor{currentfill}%
\pgfsetlinewidth{0.250937pt}%
\definecolor{currentstroke}{rgb}{1.000000,1.000000,1.000000}%
\pgfsetstrokecolor{currentstroke}%
\pgfsetdash{}{0pt}%
\pgfpathmoveto{\pgfqpoint{0.907358in}{2.435661in}}%
\pgfpathlineto{\pgfqpoint{0.995094in}{2.435661in}}%
\pgfpathlineto{\pgfqpoint{0.995094in}{2.347925in}}%
\pgfpathlineto{\pgfqpoint{0.907358in}{2.347925in}}%
\pgfpathlineto{\pgfqpoint{0.907358in}{2.435661in}}%
\pgfusepath{stroke,fill}%
\end{pgfscope}%
\begin{pgfscope}%
\pgfpathrectangle{\pgfqpoint{0.380943in}{2.260189in}}{\pgfqpoint{4.650000in}{0.614151in}}%
\pgfusepath{clip}%
\pgfsetbuttcap%
\pgfsetroundjoin%
\definecolor{currentfill}{rgb}{0.980008,0.966013,0.779393}%
\pgfsetfillcolor{currentfill}%
\pgfsetlinewidth{0.250937pt}%
\definecolor{currentstroke}{rgb}{1.000000,1.000000,1.000000}%
\pgfsetstrokecolor{currentstroke}%
\pgfsetdash{}{0pt}%
\pgfpathmoveto{\pgfqpoint{0.995094in}{2.435661in}}%
\pgfpathlineto{\pgfqpoint{1.082830in}{2.435661in}}%
\pgfpathlineto{\pgfqpoint{1.082830in}{2.347925in}}%
\pgfpathlineto{\pgfqpoint{0.995094in}{2.347925in}}%
\pgfpathlineto{\pgfqpoint{0.995094in}{2.435661in}}%
\pgfusepath{stroke,fill}%
\end{pgfscope}%
\begin{pgfscope}%
\pgfpathrectangle{\pgfqpoint{0.380943in}{2.260189in}}{\pgfqpoint{4.650000in}{0.614151in}}%
\pgfusepath{clip}%
\pgfsetbuttcap%
\pgfsetroundjoin%
\definecolor{currentfill}{rgb}{0.964783,0.940131,0.739808}%
\pgfsetfillcolor{currentfill}%
\pgfsetlinewidth{0.250937pt}%
\definecolor{currentstroke}{rgb}{1.000000,1.000000,1.000000}%
\pgfsetstrokecolor{currentstroke}%
\pgfsetdash{}{0pt}%
\pgfpathmoveto{\pgfqpoint{1.082830in}{2.435661in}}%
\pgfpathlineto{\pgfqpoint{1.170566in}{2.435661in}}%
\pgfpathlineto{\pgfqpoint{1.170566in}{2.347925in}}%
\pgfpathlineto{\pgfqpoint{1.082830in}{2.347925in}}%
\pgfpathlineto{\pgfqpoint{1.082830in}{2.435661in}}%
\pgfusepath{stroke,fill}%
\end{pgfscope}%
\begin{pgfscope}%
\pgfpathrectangle{\pgfqpoint{0.380943in}{2.260189in}}{\pgfqpoint{4.650000in}{0.614151in}}%
\pgfusepath{clip}%
\pgfsetbuttcap%
\pgfsetroundjoin%
\definecolor{currentfill}{rgb}{0.964783,0.940131,0.739808}%
\pgfsetfillcolor{currentfill}%
\pgfsetlinewidth{0.250937pt}%
\definecolor{currentstroke}{rgb}{1.000000,1.000000,1.000000}%
\pgfsetstrokecolor{currentstroke}%
\pgfsetdash{}{0pt}%
\pgfpathmoveto{\pgfqpoint{1.170566in}{2.435661in}}%
\pgfpathlineto{\pgfqpoint{1.258302in}{2.435661in}}%
\pgfpathlineto{\pgfqpoint{1.258302in}{2.347925in}}%
\pgfpathlineto{\pgfqpoint{1.170566in}{2.347925in}}%
\pgfpathlineto{\pgfqpoint{1.170566in}{2.435661in}}%
\pgfusepath{stroke,fill}%
\end{pgfscope}%
\begin{pgfscope}%
\pgfpathrectangle{\pgfqpoint{0.380943in}{2.260189in}}{\pgfqpoint{4.650000in}{0.614151in}}%
\pgfusepath{clip}%
\pgfsetbuttcap%
\pgfsetroundjoin%
\definecolor{currentfill}{rgb}{0.980008,0.966013,0.779393}%
\pgfsetfillcolor{currentfill}%
\pgfsetlinewidth{0.250937pt}%
\definecolor{currentstroke}{rgb}{1.000000,1.000000,1.000000}%
\pgfsetstrokecolor{currentstroke}%
\pgfsetdash{}{0pt}%
\pgfpathmoveto{\pgfqpoint{1.258302in}{2.435661in}}%
\pgfpathlineto{\pgfqpoint{1.346037in}{2.435661in}}%
\pgfpathlineto{\pgfqpoint{1.346037in}{2.347925in}}%
\pgfpathlineto{\pgfqpoint{1.258302in}{2.347925in}}%
\pgfpathlineto{\pgfqpoint{1.258302in}{2.435661in}}%
\pgfusepath{stroke,fill}%
\end{pgfscope}%
\begin{pgfscope}%
\pgfpathrectangle{\pgfqpoint{0.380943in}{2.260189in}}{\pgfqpoint{4.650000in}{0.614151in}}%
\pgfusepath{clip}%
\pgfsetbuttcap%
\pgfsetroundjoin%
\definecolor{currentfill}{rgb}{0.995233,0.991895,0.818977}%
\pgfsetfillcolor{currentfill}%
\pgfsetlinewidth{0.250937pt}%
\definecolor{currentstroke}{rgb}{1.000000,1.000000,1.000000}%
\pgfsetstrokecolor{currentstroke}%
\pgfsetdash{}{0pt}%
\pgfpathmoveto{\pgfqpoint{1.346037in}{2.435661in}}%
\pgfpathlineto{\pgfqpoint{1.433773in}{2.435661in}}%
\pgfpathlineto{\pgfqpoint{1.433773in}{2.347925in}}%
\pgfpathlineto{\pgfqpoint{1.346037in}{2.347925in}}%
\pgfpathlineto{\pgfqpoint{1.346037in}{2.435661in}}%
\pgfusepath{stroke,fill}%
\end{pgfscope}%
\begin{pgfscope}%
\pgfpathrectangle{\pgfqpoint{0.380943in}{2.260189in}}{\pgfqpoint{4.650000in}{0.614151in}}%
\pgfusepath{clip}%
\pgfsetbuttcap%
\pgfsetroundjoin%
\definecolor{currentfill}{rgb}{1.000000,1.000000,0.895579}%
\pgfsetfillcolor{currentfill}%
\pgfsetlinewidth{0.250937pt}%
\definecolor{currentstroke}{rgb}{1.000000,1.000000,1.000000}%
\pgfsetstrokecolor{currentstroke}%
\pgfsetdash{}{0pt}%
\pgfpathmoveto{\pgfqpoint{1.433773in}{2.435661in}}%
\pgfpathlineto{\pgfqpoint{1.521509in}{2.435661in}}%
\pgfpathlineto{\pgfqpoint{1.521509in}{2.347925in}}%
\pgfpathlineto{\pgfqpoint{1.433773in}{2.347925in}}%
\pgfpathlineto{\pgfqpoint{1.433773in}{2.435661in}}%
\pgfusepath{stroke,fill}%
\end{pgfscope}%
\begin{pgfscope}%
\pgfpathrectangle{\pgfqpoint{0.380943in}{2.260189in}}{\pgfqpoint{4.650000in}{0.614151in}}%
\pgfusepath{clip}%
\pgfsetbuttcap%
\pgfsetroundjoin%
\definecolor{currentfill}{rgb}{1.000000,1.000000,0.857516}%
\pgfsetfillcolor{currentfill}%
\pgfsetlinewidth{0.250937pt}%
\definecolor{currentstroke}{rgb}{1.000000,1.000000,1.000000}%
\pgfsetstrokecolor{currentstroke}%
\pgfsetdash{}{0pt}%
\pgfpathmoveto{\pgfqpoint{1.521509in}{2.435661in}}%
\pgfpathlineto{\pgfqpoint{1.609245in}{2.435661in}}%
\pgfpathlineto{\pgfqpoint{1.609245in}{2.347925in}}%
\pgfpathlineto{\pgfqpoint{1.521509in}{2.347925in}}%
\pgfpathlineto{\pgfqpoint{1.521509in}{2.435661in}}%
\pgfusepath{stroke,fill}%
\end{pgfscope}%
\begin{pgfscope}%
\pgfpathrectangle{\pgfqpoint{0.380943in}{2.260189in}}{\pgfqpoint{4.650000in}{0.614151in}}%
\pgfusepath{clip}%
\pgfsetbuttcap%
\pgfsetroundjoin%
\definecolor{currentfill}{rgb}{1.000000,1.000000,0.929412}%
\pgfsetfillcolor{currentfill}%
\pgfsetlinewidth{0.250937pt}%
\definecolor{currentstroke}{rgb}{1.000000,1.000000,1.000000}%
\pgfsetstrokecolor{currentstroke}%
\pgfsetdash{}{0pt}%
\pgfpathmoveto{\pgfqpoint{1.609245in}{2.435661in}}%
\pgfpathlineto{\pgfqpoint{1.696981in}{2.435661in}}%
\pgfpathlineto{\pgfqpoint{1.696981in}{2.347925in}}%
\pgfpathlineto{\pgfqpoint{1.609245in}{2.347925in}}%
\pgfpathlineto{\pgfqpoint{1.609245in}{2.435661in}}%
\pgfusepath{stroke,fill}%
\end{pgfscope}%
\begin{pgfscope}%
\pgfpathrectangle{\pgfqpoint{0.380943in}{2.260189in}}{\pgfqpoint{4.650000in}{0.614151in}}%
\pgfusepath{clip}%
\pgfsetbuttcap%
\pgfsetroundjoin%
\definecolor{currentfill}{rgb}{0.980008,0.966013,0.779393}%
\pgfsetfillcolor{currentfill}%
\pgfsetlinewidth{0.250937pt}%
\definecolor{currentstroke}{rgb}{1.000000,1.000000,1.000000}%
\pgfsetstrokecolor{currentstroke}%
\pgfsetdash{}{0pt}%
\pgfpathmoveto{\pgfqpoint{1.696981in}{2.435661in}}%
\pgfpathlineto{\pgfqpoint{1.784717in}{2.435661in}}%
\pgfpathlineto{\pgfqpoint{1.784717in}{2.347925in}}%
\pgfpathlineto{\pgfqpoint{1.696981in}{2.347925in}}%
\pgfpathlineto{\pgfqpoint{1.696981in}{2.435661in}}%
\pgfusepath{stroke,fill}%
\end{pgfscope}%
\begin{pgfscope}%
\pgfpathrectangle{\pgfqpoint{0.380943in}{2.260189in}}{\pgfqpoint{4.650000in}{0.614151in}}%
\pgfusepath{clip}%
\pgfsetbuttcap%
\pgfsetroundjoin%
\definecolor{currentfill}{rgb}{1.000000,1.000000,0.857516}%
\pgfsetfillcolor{currentfill}%
\pgfsetlinewidth{0.250937pt}%
\definecolor{currentstroke}{rgb}{1.000000,1.000000,1.000000}%
\pgfsetstrokecolor{currentstroke}%
\pgfsetdash{}{0pt}%
\pgfpathmoveto{\pgfqpoint{1.784717in}{2.435661in}}%
\pgfpathlineto{\pgfqpoint{1.872452in}{2.435661in}}%
\pgfpathlineto{\pgfqpoint{1.872452in}{2.347925in}}%
\pgfpathlineto{\pgfqpoint{1.784717in}{2.347925in}}%
\pgfpathlineto{\pgfqpoint{1.784717in}{2.435661in}}%
\pgfusepath{stroke,fill}%
\end{pgfscope}%
\begin{pgfscope}%
\pgfpathrectangle{\pgfqpoint{0.380943in}{2.260189in}}{\pgfqpoint{4.650000in}{0.614151in}}%
\pgfusepath{clip}%
\pgfsetbuttcap%
\pgfsetroundjoin%
\definecolor{currentfill}{rgb}{0.995233,0.991895,0.818977}%
\pgfsetfillcolor{currentfill}%
\pgfsetlinewidth{0.250937pt}%
\definecolor{currentstroke}{rgb}{1.000000,1.000000,1.000000}%
\pgfsetstrokecolor{currentstroke}%
\pgfsetdash{}{0pt}%
\pgfpathmoveto{\pgfqpoint{1.872452in}{2.435661in}}%
\pgfpathlineto{\pgfqpoint{1.960188in}{2.435661in}}%
\pgfpathlineto{\pgfqpoint{1.960188in}{2.347925in}}%
\pgfpathlineto{\pgfqpoint{1.872452in}{2.347925in}}%
\pgfpathlineto{\pgfqpoint{1.872452in}{2.435661in}}%
\pgfusepath{stroke,fill}%
\end{pgfscope}%
\begin{pgfscope}%
\pgfpathrectangle{\pgfqpoint{0.380943in}{2.260189in}}{\pgfqpoint{4.650000in}{0.614151in}}%
\pgfusepath{clip}%
\pgfsetbuttcap%
\pgfsetroundjoin%
\definecolor{currentfill}{rgb}{0.964783,0.940131,0.739808}%
\pgfsetfillcolor{currentfill}%
\pgfsetlinewidth{0.250937pt}%
\definecolor{currentstroke}{rgb}{1.000000,1.000000,1.000000}%
\pgfsetstrokecolor{currentstroke}%
\pgfsetdash{}{0pt}%
\pgfpathmoveto{\pgfqpoint{1.960188in}{2.435661in}}%
\pgfpathlineto{\pgfqpoint{2.047924in}{2.435661in}}%
\pgfpathlineto{\pgfqpoint{2.047924in}{2.347925in}}%
\pgfpathlineto{\pgfqpoint{1.960188in}{2.347925in}}%
\pgfpathlineto{\pgfqpoint{1.960188in}{2.435661in}}%
\pgfusepath{stroke,fill}%
\end{pgfscope}%
\begin{pgfscope}%
\pgfpathrectangle{\pgfqpoint{0.380943in}{2.260189in}}{\pgfqpoint{4.650000in}{0.614151in}}%
\pgfusepath{clip}%
\pgfsetbuttcap%
\pgfsetroundjoin%
\definecolor{currentfill}{rgb}{1.000000,1.000000,0.895579}%
\pgfsetfillcolor{currentfill}%
\pgfsetlinewidth{0.250937pt}%
\definecolor{currentstroke}{rgb}{1.000000,1.000000,1.000000}%
\pgfsetstrokecolor{currentstroke}%
\pgfsetdash{}{0pt}%
\pgfpathmoveto{\pgfqpoint{2.047924in}{2.435661in}}%
\pgfpathlineto{\pgfqpoint{2.135660in}{2.435661in}}%
\pgfpathlineto{\pgfqpoint{2.135660in}{2.347925in}}%
\pgfpathlineto{\pgfqpoint{2.047924in}{2.347925in}}%
\pgfpathlineto{\pgfqpoint{2.047924in}{2.435661in}}%
\pgfusepath{stroke,fill}%
\end{pgfscope}%
\begin{pgfscope}%
\pgfpathrectangle{\pgfqpoint{0.380943in}{2.260189in}}{\pgfqpoint{4.650000in}{0.614151in}}%
\pgfusepath{clip}%
\pgfsetbuttcap%
\pgfsetroundjoin%
\definecolor{currentfill}{rgb}{0.961738,0.927612,0.725598}%
\pgfsetfillcolor{currentfill}%
\pgfsetlinewidth{0.250937pt}%
\definecolor{currentstroke}{rgb}{1.000000,1.000000,1.000000}%
\pgfsetstrokecolor{currentstroke}%
\pgfsetdash{}{0pt}%
\pgfpathmoveto{\pgfqpoint{2.135660in}{2.435661in}}%
\pgfpathlineto{\pgfqpoint{2.223396in}{2.435661in}}%
\pgfpathlineto{\pgfqpoint{2.223396in}{2.347925in}}%
\pgfpathlineto{\pgfqpoint{2.135660in}{2.347925in}}%
\pgfpathlineto{\pgfqpoint{2.135660in}{2.435661in}}%
\pgfusepath{stroke,fill}%
\end{pgfscope}%
\begin{pgfscope}%
\pgfpathrectangle{\pgfqpoint{0.380943in}{2.260189in}}{\pgfqpoint{4.650000in}{0.614151in}}%
\pgfusepath{clip}%
\pgfsetbuttcap%
\pgfsetroundjoin%
\definecolor{currentfill}{rgb}{0.964783,0.940131,0.739808}%
\pgfsetfillcolor{currentfill}%
\pgfsetlinewidth{0.250937pt}%
\definecolor{currentstroke}{rgb}{1.000000,1.000000,1.000000}%
\pgfsetstrokecolor{currentstroke}%
\pgfsetdash{}{0pt}%
\pgfpathmoveto{\pgfqpoint{2.223396in}{2.435661in}}%
\pgfpathlineto{\pgfqpoint{2.311132in}{2.435661in}}%
\pgfpathlineto{\pgfqpoint{2.311132in}{2.347925in}}%
\pgfpathlineto{\pgfqpoint{2.223396in}{2.347925in}}%
\pgfpathlineto{\pgfqpoint{2.223396in}{2.435661in}}%
\pgfusepath{stroke,fill}%
\end{pgfscope}%
\begin{pgfscope}%
\pgfpathrectangle{\pgfqpoint{0.380943in}{2.260189in}}{\pgfqpoint{4.650000in}{0.614151in}}%
\pgfusepath{clip}%
\pgfsetbuttcap%
\pgfsetroundjoin%
\definecolor{currentfill}{rgb}{0.980008,0.966013,0.779393}%
\pgfsetfillcolor{currentfill}%
\pgfsetlinewidth{0.250937pt}%
\definecolor{currentstroke}{rgb}{1.000000,1.000000,1.000000}%
\pgfsetstrokecolor{currentstroke}%
\pgfsetdash{}{0pt}%
\pgfpathmoveto{\pgfqpoint{2.311132in}{2.435661in}}%
\pgfpathlineto{\pgfqpoint{2.398868in}{2.435661in}}%
\pgfpathlineto{\pgfqpoint{2.398868in}{2.347925in}}%
\pgfpathlineto{\pgfqpoint{2.311132in}{2.347925in}}%
\pgfpathlineto{\pgfqpoint{2.311132in}{2.435661in}}%
\pgfusepath{stroke,fill}%
\end{pgfscope}%
\begin{pgfscope}%
\pgfpathrectangle{\pgfqpoint{0.380943in}{2.260189in}}{\pgfqpoint{4.650000in}{0.614151in}}%
\pgfusepath{clip}%
\pgfsetbuttcap%
\pgfsetroundjoin%
\definecolor{currentfill}{rgb}{0.990634,0.779608,0.623299}%
\pgfsetfillcolor{currentfill}%
\pgfsetlinewidth{0.250937pt}%
\definecolor{currentstroke}{rgb}{1.000000,1.000000,1.000000}%
\pgfsetstrokecolor{currentstroke}%
\pgfsetdash{}{0pt}%
\pgfpathmoveto{\pgfqpoint{2.398868in}{2.435661in}}%
\pgfpathlineto{\pgfqpoint{2.486603in}{2.435661in}}%
\pgfpathlineto{\pgfqpoint{2.486603in}{2.347925in}}%
\pgfpathlineto{\pgfqpoint{2.398868in}{2.347925in}}%
\pgfpathlineto{\pgfqpoint{2.398868in}{2.435661in}}%
\pgfusepath{stroke,fill}%
\end{pgfscope}%
\begin{pgfscope}%
\pgfpathrectangle{\pgfqpoint{0.380943in}{2.260189in}}{\pgfqpoint{4.650000in}{0.614151in}}%
\pgfusepath{clip}%
\pgfsetbuttcap%
\pgfsetroundjoin%
\definecolor{currentfill}{rgb}{0.990634,0.779608,0.623299}%
\pgfsetfillcolor{currentfill}%
\pgfsetlinewidth{0.250937pt}%
\definecolor{currentstroke}{rgb}{1.000000,1.000000,1.000000}%
\pgfsetstrokecolor{currentstroke}%
\pgfsetdash{}{0pt}%
\pgfpathmoveto{\pgfqpoint{2.486603in}{2.435661in}}%
\pgfpathlineto{\pgfqpoint{2.574339in}{2.435661in}}%
\pgfpathlineto{\pgfqpoint{2.574339in}{2.347925in}}%
\pgfpathlineto{\pgfqpoint{2.486603in}{2.347925in}}%
\pgfpathlineto{\pgfqpoint{2.486603in}{2.435661in}}%
\pgfusepath{stroke,fill}%
\end{pgfscope}%
\begin{pgfscope}%
\pgfpathrectangle{\pgfqpoint{0.380943in}{2.260189in}}{\pgfqpoint{4.650000in}{0.614151in}}%
\pgfusepath{clip}%
\pgfsetbuttcap%
\pgfsetroundjoin%
\definecolor{currentfill}{rgb}{0.987266,0.804198,0.639170}%
\pgfsetfillcolor{currentfill}%
\pgfsetlinewidth{0.250937pt}%
\definecolor{currentstroke}{rgb}{1.000000,1.000000,1.000000}%
\pgfsetstrokecolor{currentstroke}%
\pgfsetdash{}{0pt}%
\pgfpathmoveto{\pgfqpoint{2.574339in}{2.435661in}}%
\pgfpathlineto{\pgfqpoint{2.662075in}{2.435661in}}%
\pgfpathlineto{\pgfqpoint{2.662075in}{2.347925in}}%
\pgfpathlineto{\pgfqpoint{2.574339in}{2.347925in}}%
\pgfpathlineto{\pgfqpoint{2.574339in}{2.435661in}}%
\pgfusepath{stroke,fill}%
\end{pgfscope}%
\begin{pgfscope}%
\pgfpathrectangle{\pgfqpoint{0.380943in}{2.260189in}}{\pgfqpoint{4.650000in}{0.614151in}}%
\pgfusepath{clip}%
\pgfsetbuttcap%
\pgfsetroundjoin%
\definecolor{currentfill}{rgb}{0.980008,0.966013,0.779393}%
\pgfsetfillcolor{currentfill}%
\pgfsetlinewidth{0.250937pt}%
\definecolor{currentstroke}{rgb}{1.000000,1.000000,1.000000}%
\pgfsetstrokecolor{currentstroke}%
\pgfsetdash{}{0pt}%
\pgfpathmoveto{\pgfqpoint{2.662075in}{2.435661in}}%
\pgfpathlineto{\pgfqpoint{2.749811in}{2.435661in}}%
\pgfpathlineto{\pgfqpoint{2.749811in}{2.347925in}}%
\pgfpathlineto{\pgfqpoint{2.662075in}{2.347925in}}%
\pgfpathlineto{\pgfqpoint{2.662075in}{2.435661in}}%
\pgfusepath{stroke,fill}%
\end{pgfscope}%
\begin{pgfscope}%
\pgfpathrectangle{\pgfqpoint{0.380943in}{2.260189in}}{\pgfqpoint{4.650000in}{0.614151in}}%
\pgfusepath{clip}%
\pgfsetbuttcap%
\pgfsetroundjoin%
\definecolor{currentfill}{rgb}{0.963260,0.918478,0.719508}%
\pgfsetfillcolor{currentfill}%
\pgfsetlinewidth{0.250937pt}%
\definecolor{currentstroke}{rgb}{1.000000,1.000000,1.000000}%
\pgfsetstrokecolor{currentstroke}%
\pgfsetdash{}{0pt}%
\pgfpathmoveto{\pgfqpoint{2.749811in}{2.435661in}}%
\pgfpathlineto{\pgfqpoint{2.837547in}{2.435661in}}%
\pgfpathlineto{\pgfqpoint{2.837547in}{2.347925in}}%
\pgfpathlineto{\pgfqpoint{2.749811in}{2.347925in}}%
\pgfpathlineto{\pgfqpoint{2.749811in}{2.435661in}}%
\pgfusepath{stroke,fill}%
\end{pgfscope}%
\begin{pgfscope}%
\pgfpathrectangle{\pgfqpoint{0.380943in}{2.260189in}}{\pgfqpoint{4.650000in}{0.614151in}}%
\pgfusepath{clip}%
\pgfsetbuttcap%
\pgfsetroundjoin%
\definecolor{currentfill}{rgb}{0.964937,0.908651,0.713110}%
\pgfsetfillcolor{currentfill}%
\pgfsetlinewidth{0.250937pt}%
\definecolor{currentstroke}{rgb}{1.000000,1.000000,1.000000}%
\pgfsetstrokecolor{currentstroke}%
\pgfsetdash{}{0pt}%
\pgfpathmoveto{\pgfqpoint{2.837547in}{2.435661in}}%
\pgfpathlineto{\pgfqpoint{2.925283in}{2.435661in}}%
\pgfpathlineto{\pgfqpoint{2.925283in}{2.347925in}}%
\pgfpathlineto{\pgfqpoint{2.837547in}{2.347925in}}%
\pgfpathlineto{\pgfqpoint{2.837547in}{2.435661in}}%
\pgfusepath{stroke,fill}%
\end{pgfscope}%
\begin{pgfscope}%
\pgfpathrectangle{\pgfqpoint{0.380943in}{2.260189in}}{\pgfqpoint{4.650000in}{0.614151in}}%
\pgfusepath{clip}%
\pgfsetbuttcap%
\pgfsetroundjoin%
\definecolor{currentfill}{rgb}{0.995233,0.991895,0.818977}%
\pgfsetfillcolor{currentfill}%
\pgfsetlinewidth{0.250937pt}%
\definecolor{currentstroke}{rgb}{1.000000,1.000000,1.000000}%
\pgfsetstrokecolor{currentstroke}%
\pgfsetdash{}{0pt}%
\pgfpathmoveto{\pgfqpoint{2.925283in}{2.435661in}}%
\pgfpathlineto{\pgfqpoint{3.013019in}{2.435661in}}%
\pgfpathlineto{\pgfqpoint{3.013019in}{2.347925in}}%
\pgfpathlineto{\pgfqpoint{2.925283in}{2.347925in}}%
\pgfpathlineto{\pgfqpoint{2.925283in}{2.435661in}}%
\pgfusepath{stroke,fill}%
\end{pgfscope}%
\begin{pgfscope}%
\pgfpathrectangle{\pgfqpoint{0.380943in}{2.260189in}}{\pgfqpoint{4.650000in}{0.614151in}}%
\pgfusepath{clip}%
\pgfsetbuttcap%
\pgfsetroundjoin%
\definecolor{currentfill}{rgb}{0.961738,0.927612,0.725598}%
\pgfsetfillcolor{currentfill}%
\pgfsetlinewidth{0.250937pt}%
\definecolor{currentstroke}{rgb}{1.000000,1.000000,1.000000}%
\pgfsetstrokecolor{currentstroke}%
\pgfsetdash{}{0pt}%
\pgfpathmoveto{\pgfqpoint{3.013019in}{2.435661in}}%
\pgfpathlineto{\pgfqpoint{3.100754in}{2.435661in}}%
\pgfpathlineto{\pgfqpoint{3.100754in}{2.347925in}}%
\pgfpathlineto{\pgfqpoint{3.013019in}{2.347925in}}%
\pgfpathlineto{\pgfqpoint{3.013019in}{2.435661in}}%
\pgfusepath{stroke,fill}%
\end{pgfscope}%
\begin{pgfscope}%
\pgfpathrectangle{\pgfqpoint{0.380943in}{2.260189in}}{\pgfqpoint{4.650000in}{0.614151in}}%
\pgfusepath{clip}%
\pgfsetbuttcap%
\pgfsetroundjoin%
\definecolor{currentfill}{rgb}{0.961738,0.927612,0.725598}%
\pgfsetfillcolor{currentfill}%
\pgfsetlinewidth{0.250937pt}%
\definecolor{currentstroke}{rgb}{1.000000,1.000000,1.000000}%
\pgfsetstrokecolor{currentstroke}%
\pgfsetdash{}{0pt}%
\pgfpathmoveto{\pgfqpoint{3.100754in}{2.435661in}}%
\pgfpathlineto{\pgfqpoint{3.188490in}{2.435661in}}%
\pgfpathlineto{\pgfqpoint{3.188490in}{2.347925in}}%
\pgfpathlineto{\pgfqpoint{3.100754in}{2.347925in}}%
\pgfpathlineto{\pgfqpoint{3.100754in}{2.435661in}}%
\pgfusepath{stroke,fill}%
\end{pgfscope}%
\begin{pgfscope}%
\pgfpathrectangle{\pgfqpoint{0.380943in}{2.260189in}}{\pgfqpoint{4.650000in}{0.614151in}}%
\pgfusepath{clip}%
\pgfsetbuttcap%
\pgfsetroundjoin%
\definecolor{currentfill}{rgb}{0.980008,0.966013,0.779393}%
\pgfsetfillcolor{currentfill}%
\pgfsetlinewidth{0.250937pt}%
\definecolor{currentstroke}{rgb}{1.000000,1.000000,1.000000}%
\pgfsetstrokecolor{currentstroke}%
\pgfsetdash{}{0pt}%
\pgfpathmoveto{\pgfqpoint{3.188490in}{2.435661in}}%
\pgfpathlineto{\pgfqpoint{3.276226in}{2.435661in}}%
\pgfpathlineto{\pgfqpoint{3.276226in}{2.347925in}}%
\pgfpathlineto{\pgfqpoint{3.188490in}{2.347925in}}%
\pgfpathlineto{\pgfqpoint{3.188490in}{2.435661in}}%
\pgfusepath{stroke,fill}%
\end{pgfscope}%
\begin{pgfscope}%
\pgfpathrectangle{\pgfqpoint{0.380943in}{2.260189in}}{\pgfqpoint{4.650000in}{0.614151in}}%
\pgfusepath{clip}%
\pgfsetbuttcap%
\pgfsetroundjoin%
\definecolor{currentfill}{rgb}{0.964783,0.940131,0.739808}%
\pgfsetfillcolor{currentfill}%
\pgfsetlinewidth{0.250937pt}%
\definecolor{currentstroke}{rgb}{1.000000,1.000000,1.000000}%
\pgfsetstrokecolor{currentstroke}%
\pgfsetdash{}{0pt}%
\pgfpathmoveto{\pgfqpoint{3.276226in}{2.435661in}}%
\pgfpathlineto{\pgfqpoint{3.363962in}{2.435661in}}%
\pgfpathlineto{\pgfqpoint{3.363962in}{2.347925in}}%
\pgfpathlineto{\pgfqpoint{3.276226in}{2.347925in}}%
\pgfpathlineto{\pgfqpoint{3.276226in}{2.435661in}}%
\pgfusepath{stroke,fill}%
\end{pgfscope}%
\begin{pgfscope}%
\pgfpathrectangle{\pgfqpoint{0.380943in}{2.260189in}}{\pgfqpoint{4.650000in}{0.614151in}}%
\pgfusepath{clip}%
\pgfsetbuttcap%
\pgfsetroundjoin%
\definecolor{currentfill}{rgb}{0.963260,0.918478,0.719508}%
\pgfsetfillcolor{currentfill}%
\pgfsetlinewidth{0.250937pt}%
\definecolor{currentstroke}{rgb}{1.000000,1.000000,1.000000}%
\pgfsetstrokecolor{currentstroke}%
\pgfsetdash{}{0pt}%
\pgfpathmoveto{\pgfqpoint{3.363962in}{2.435661in}}%
\pgfpathlineto{\pgfqpoint{3.451698in}{2.435661in}}%
\pgfpathlineto{\pgfqpoint{3.451698in}{2.347925in}}%
\pgfpathlineto{\pgfqpoint{3.363962in}{2.347925in}}%
\pgfpathlineto{\pgfqpoint{3.363962in}{2.435661in}}%
\pgfusepath{stroke,fill}%
\end{pgfscope}%
\begin{pgfscope}%
\pgfpathrectangle{\pgfqpoint{0.380943in}{2.260189in}}{\pgfqpoint{4.650000in}{0.614151in}}%
\pgfusepath{clip}%
\pgfsetbuttcap%
\pgfsetroundjoin%
\definecolor{currentfill}{rgb}{1.000000,1.000000,0.857516}%
\pgfsetfillcolor{currentfill}%
\pgfsetlinewidth{0.250937pt}%
\definecolor{currentstroke}{rgb}{1.000000,1.000000,1.000000}%
\pgfsetstrokecolor{currentstroke}%
\pgfsetdash{}{0pt}%
\pgfpathmoveto{\pgfqpoint{3.451698in}{2.435661in}}%
\pgfpathlineto{\pgfqpoint{3.539434in}{2.435661in}}%
\pgfpathlineto{\pgfqpoint{3.539434in}{2.347925in}}%
\pgfpathlineto{\pgfqpoint{3.451698in}{2.347925in}}%
\pgfpathlineto{\pgfqpoint{3.451698in}{2.435661in}}%
\pgfusepath{stroke,fill}%
\end{pgfscope}%
\begin{pgfscope}%
\pgfpathrectangle{\pgfqpoint{0.380943in}{2.260189in}}{\pgfqpoint{4.650000in}{0.614151in}}%
\pgfusepath{clip}%
\pgfsetbuttcap%
\pgfsetroundjoin%
\definecolor{currentfill}{rgb}{0.963260,0.918478,0.719508}%
\pgfsetfillcolor{currentfill}%
\pgfsetlinewidth{0.250937pt}%
\definecolor{currentstroke}{rgb}{1.000000,1.000000,1.000000}%
\pgfsetstrokecolor{currentstroke}%
\pgfsetdash{}{0pt}%
\pgfpathmoveto{\pgfqpoint{3.539434in}{2.435661in}}%
\pgfpathlineto{\pgfqpoint{3.627169in}{2.435661in}}%
\pgfpathlineto{\pgfqpoint{3.627169in}{2.347925in}}%
\pgfpathlineto{\pgfqpoint{3.539434in}{2.347925in}}%
\pgfpathlineto{\pgfqpoint{3.539434in}{2.435661in}}%
\pgfusepath{stroke,fill}%
\end{pgfscope}%
\begin{pgfscope}%
\pgfpathrectangle{\pgfqpoint{0.380943in}{2.260189in}}{\pgfqpoint{4.650000in}{0.614151in}}%
\pgfusepath{clip}%
\pgfsetbuttcap%
\pgfsetroundjoin%
\definecolor{currentfill}{rgb}{0.969504,0.885813,0.700930}%
\pgfsetfillcolor{currentfill}%
\pgfsetlinewidth{0.250937pt}%
\definecolor{currentstroke}{rgb}{1.000000,1.000000,1.000000}%
\pgfsetstrokecolor{currentstroke}%
\pgfsetdash{}{0pt}%
\pgfpathmoveto{\pgfqpoint{3.627169in}{2.435661in}}%
\pgfpathlineto{\pgfqpoint{3.714905in}{2.435661in}}%
\pgfpathlineto{\pgfqpoint{3.714905in}{2.347925in}}%
\pgfpathlineto{\pgfqpoint{3.627169in}{2.347925in}}%
\pgfpathlineto{\pgfqpoint{3.627169in}{2.435661in}}%
\pgfusepath{stroke,fill}%
\end{pgfscope}%
\begin{pgfscope}%
\pgfpathrectangle{\pgfqpoint{0.380943in}{2.260189in}}{\pgfqpoint{4.650000in}{0.614151in}}%
\pgfusepath{clip}%
\pgfsetbuttcap%
\pgfsetroundjoin%
\definecolor{currentfill}{rgb}{0.964937,0.908651,0.713110}%
\pgfsetfillcolor{currentfill}%
\pgfsetlinewidth{0.250937pt}%
\definecolor{currentstroke}{rgb}{1.000000,1.000000,1.000000}%
\pgfsetstrokecolor{currentstroke}%
\pgfsetdash{}{0pt}%
\pgfpathmoveto{\pgfqpoint{3.714905in}{2.435661in}}%
\pgfpathlineto{\pgfqpoint{3.802641in}{2.435661in}}%
\pgfpathlineto{\pgfqpoint{3.802641in}{2.347925in}}%
\pgfpathlineto{\pgfqpoint{3.714905in}{2.347925in}}%
\pgfpathlineto{\pgfqpoint{3.714905in}{2.435661in}}%
\pgfusepath{stroke,fill}%
\end{pgfscope}%
\begin{pgfscope}%
\pgfpathrectangle{\pgfqpoint{0.380943in}{2.260189in}}{\pgfqpoint{4.650000in}{0.614151in}}%
\pgfusepath{clip}%
\pgfsetbuttcap%
\pgfsetroundjoin%
\definecolor{currentfill}{rgb}{0.964783,0.940131,0.739808}%
\pgfsetfillcolor{currentfill}%
\pgfsetlinewidth{0.250937pt}%
\definecolor{currentstroke}{rgb}{1.000000,1.000000,1.000000}%
\pgfsetstrokecolor{currentstroke}%
\pgfsetdash{}{0pt}%
\pgfpathmoveto{\pgfqpoint{3.802641in}{2.435661in}}%
\pgfpathlineto{\pgfqpoint{3.890377in}{2.435661in}}%
\pgfpathlineto{\pgfqpoint{3.890377in}{2.347925in}}%
\pgfpathlineto{\pgfqpoint{3.802641in}{2.347925in}}%
\pgfpathlineto{\pgfqpoint{3.802641in}{2.435661in}}%
\pgfusepath{stroke,fill}%
\end{pgfscope}%
\begin{pgfscope}%
\pgfpathrectangle{\pgfqpoint{0.380943in}{2.260189in}}{\pgfqpoint{4.650000in}{0.614151in}}%
\pgfusepath{clip}%
\pgfsetbuttcap%
\pgfsetroundjoin%
\definecolor{currentfill}{rgb}{0.987266,0.804198,0.639170}%
\pgfsetfillcolor{currentfill}%
\pgfsetlinewidth{0.250937pt}%
\definecolor{currentstroke}{rgb}{1.000000,1.000000,1.000000}%
\pgfsetstrokecolor{currentstroke}%
\pgfsetdash{}{0pt}%
\pgfpathmoveto{\pgfqpoint{3.890377in}{2.435661in}}%
\pgfpathlineto{\pgfqpoint{3.978113in}{2.435661in}}%
\pgfpathlineto{\pgfqpoint{3.978113in}{2.347925in}}%
\pgfpathlineto{\pgfqpoint{3.890377in}{2.347925in}}%
\pgfpathlineto{\pgfqpoint{3.890377in}{2.435661in}}%
\pgfusepath{stroke,fill}%
\end{pgfscope}%
\begin{pgfscope}%
\pgfpathrectangle{\pgfqpoint{0.380943in}{2.260189in}}{\pgfqpoint{4.650000in}{0.614151in}}%
\pgfusepath{clip}%
\pgfsetbuttcap%
\pgfsetroundjoin%
\definecolor{currentfill}{rgb}{1.000000,1.000000,0.857516}%
\pgfsetfillcolor{currentfill}%
\pgfsetlinewidth{0.250937pt}%
\definecolor{currentstroke}{rgb}{1.000000,1.000000,1.000000}%
\pgfsetstrokecolor{currentstroke}%
\pgfsetdash{}{0pt}%
\pgfpathmoveto{\pgfqpoint{3.978113in}{2.435661in}}%
\pgfpathlineto{\pgfqpoint{4.065849in}{2.435661in}}%
\pgfpathlineto{\pgfqpoint{4.065849in}{2.347925in}}%
\pgfpathlineto{\pgfqpoint{3.978113in}{2.347925in}}%
\pgfpathlineto{\pgfqpoint{3.978113in}{2.435661in}}%
\pgfusepath{stroke,fill}%
\end{pgfscope}%
\begin{pgfscope}%
\pgfpathrectangle{\pgfqpoint{0.380943in}{2.260189in}}{\pgfqpoint{4.650000in}{0.614151in}}%
\pgfusepath{clip}%
\pgfsetbuttcap%
\pgfsetroundjoin%
\definecolor{currentfill}{rgb}{0.978639,0.841584,0.673679}%
\pgfsetfillcolor{currentfill}%
\pgfsetlinewidth{0.250937pt}%
\definecolor{currentstroke}{rgb}{1.000000,1.000000,1.000000}%
\pgfsetstrokecolor{currentstroke}%
\pgfsetdash{}{0pt}%
\pgfpathmoveto{\pgfqpoint{4.065849in}{2.435661in}}%
\pgfpathlineto{\pgfqpoint{4.153585in}{2.435661in}}%
\pgfpathlineto{\pgfqpoint{4.153585in}{2.347925in}}%
\pgfpathlineto{\pgfqpoint{4.065849in}{2.347925in}}%
\pgfpathlineto{\pgfqpoint{4.065849in}{2.435661in}}%
\pgfusepath{stroke,fill}%
\end{pgfscope}%
\begin{pgfscope}%
\pgfpathrectangle{\pgfqpoint{0.380943in}{2.260189in}}{\pgfqpoint{4.650000in}{0.614151in}}%
\pgfusepath{clip}%
\pgfsetbuttcap%
\pgfsetroundjoin%
\definecolor{currentfill}{rgb}{0.969504,0.885813,0.700930}%
\pgfsetfillcolor{currentfill}%
\pgfsetlinewidth{0.250937pt}%
\definecolor{currentstroke}{rgb}{1.000000,1.000000,1.000000}%
\pgfsetstrokecolor{currentstroke}%
\pgfsetdash{}{0pt}%
\pgfpathmoveto{\pgfqpoint{4.153585in}{2.435661in}}%
\pgfpathlineto{\pgfqpoint{4.241320in}{2.435661in}}%
\pgfpathlineto{\pgfqpoint{4.241320in}{2.347925in}}%
\pgfpathlineto{\pgfqpoint{4.153585in}{2.347925in}}%
\pgfpathlineto{\pgfqpoint{4.153585in}{2.435661in}}%
\pgfusepath{stroke,fill}%
\end{pgfscope}%
\begin{pgfscope}%
\pgfpathrectangle{\pgfqpoint{0.380943in}{2.260189in}}{\pgfqpoint{4.650000in}{0.614151in}}%
\pgfusepath{clip}%
\pgfsetbuttcap%
\pgfsetroundjoin%
\definecolor{currentfill}{rgb}{0.982699,0.823991,0.657439}%
\pgfsetfillcolor{currentfill}%
\pgfsetlinewidth{0.250937pt}%
\definecolor{currentstroke}{rgb}{1.000000,1.000000,1.000000}%
\pgfsetstrokecolor{currentstroke}%
\pgfsetdash{}{0pt}%
\pgfpathmoveto{\pgfqpoint{4.241320in}{2.435661in}}%
\pgfpathlineto{\pgfqpoint{4.329056in}{2.435661in}}%
\pgfpathlineto{\pgfqpoint{4.329056in}{2.347925in}}%
\pgfpathlineto{\pgfqpoint{4.241320in}{2.347925in}}%
\pgfpathlineto{\pgfqpoint{4.241320in}{2.435661in}}%
\pgfusepath{stroke,fill}%
\end{pgfscope}%
\begin{pgfscope}%
\pgfpathrectangle{\pgfqpoint{0.380943in}{2.260189in}}{\pgfqpoint{4.650000in}{0.614151in}}%
\pgfusepath{clip}%
\pgfsetbuttcap%
\pgfsetroundjoin%
\definecolor{currentfill}{rgb}{0.980008,0.966013,0.779393}%
\pgfsetfillcolor{currentfill}%
\pgfsetlinewidth{0.250937pt}%
\definecolor{currentstroke}{rgb}{1.000000,1.000000,1.000000}%
\pgfsetstrokecolor{currentstroke}%
\pgfsetdash{}{0pt}%
\pgfpathmoveto{\pgfqpoint{4.329056in}{2.435661in}}%
\pgfpathlineto{\pgfqpoint{4.416792in}{2.435661in}}%
\pgfpathlineto{\pgfqpoint{4.416792in}{2.347925in}}%
\pgfpathlineto{\pgfqpoint{4.329056in}{2.347925in}}%
\pgfpathlineto{\pgfqpoint{4.329056in}{2.435661in}}%
\pgfusepath{stroke,fill}%
\end{pgfscope}%
\begin{pgfscope}%
\pgfpathrectangle{\pgfqpoint{0.380943in}{2.260189in}}{\pgfqpoint{4.650000in}{0.614151in}}%
\pgfusepath{clip}%
\pgfsetbuttcap%
\pgfsetroundjoin%
\definecolor{currentfill}{rgb}{0.969504,0.885813,0.700930}%
\pgfsetfillcolor{currentfill}%
\pgfsetlinewidth{0.250937pt}%
\definecolor{currentstroke}{rgb}{1.000000,1.000000,1.000000}%
\pgfsetstrokecolor{currentstroke}%
\pgfsetdash{}{0pt}%
\pgfpathmoveto{\pgfqpoint{4.416792in}{2.435661in}}%
\pgfpathlineto{\pgfqpoint{4.504528in}{2.435661in}}%
\pgfpathlineto{\pgfqpoint{4.504528in}{2.347925in}}%
\pgfpathlineto{\pgfqpoint{4.416792in}{2.347925in}}%
\pgfpathlineto{\pgfqpoint{4.416792in}{2.435661in}}%
\pgfusepath{stroke,fill}%
\end{pgfscope}%
\begin{pgfscope}%
\pgfpathrectangle{\pgfqpoint{0.380943in}{2.260189in}}{\pgfqpoint{4.650000in}{0.614151in}}%
\pgfusepath{clip}%
\pgfsetbuttcap%
\pgfsetroundjoin%
\definecolor{currentfill}{rgb}{0.964783,0.940131,0.739808}%
\pgfsetfillcolor{currentfill}%
\pgfsetlinewidth{0.250937pt}%
\definecolor{currentstroke}{rgb}{1.000000,1.000000,1.000000}%
\pgfsetstrokecolor{currentstroke}%
\pgfsetdash{}{0pt}%
\pgfpathmoveto{\pgfqpoint{4.504528in}{2.435661in}}%
\pgfpathlineto{\pgfqpoint{4.592264in}{2.435661in}}%
\pgfpathlineto{\pgfqpoint{4.592264in}{2.347925in}}%
\pgfpathlineto{\pgfqpoint{4.504528in}{2.347925in}}%
\pgfpathlineto{\pgfqpoint{4.504528in}{2.435661in}}%
\pgfusepath{stroke,fill}%
\end{pgfscope}%
\begin{pgfscope}%
\pgfpathrectangle{\pgfqpoint{0.380943in}{2.260189in}}{\pgfqpoint{4.650000in}{0.614151in}}%
\pgfusepath{clip}%
\pgfsetbuttcap%
\pgfsetroundjoin%
\definecolor{currentfill}{rgb}{0.963260,0.918478,0.719508}%
\pgfsetfillcolor{currentfill}%
\pgfsetlinewidth{0.250937pt}%
\definecolor{currentstroke}{rgb}{1.000000,1.000000,1.000000}%
\pgfsetstrokecolor{currentstroke}%
\pgfsetdash{}{0pt}%
\pgfpathmoveto{\pgfqpoint{4.592264in}{2.435661in}}%
\pgfpathlineto{\pgfqpoint{4.680000in}{2.435661in}}%
\pgfpathlineto{\pgfqpoint{4.680000in}{2.347925in}}%
\pgfpathlineto{\pgfqpoint{4.592264in}{2.347925in}}%
\pgfpathlineto{\pgfqpoint{4.592264in}{2.435661in}}%
\pgfusepath{stroke,fill}%
\end{pgfscope}%
\begin{pgfscope}%
\pgfpathrectangle{\pgfqpoint{0.380943in}{2.260189in}}{\pgfqpoint{4.650000in}{0.614151in}}%
\pgfusepath{clip}%
\pgfsetbuttcap%
\pgfsetroundjoin%
\definecolor{currentfill}{rgb}{0.978639,0.841584,0.673679}%
\pgfsetfillcolor{currentfill}%
\pgfsetlinewidth{0.250937pt}%
\definecolor{currentstroke}{rgb}{1.000000,1.000000,1.000000}%
\pgfsetstrokecolor{currentstroke}%
\pgfsetdash{}{0pt}%
\pgfpathmoveto{\pgfqpoint{4.680000in}{2.435661in}}%
\pgfpathlineto{\pgfqpoint{4.767736in}{2.435661in}}%
\pgfpathlineto{\pgfqpoint{4.767736in}{2.347925in}}%
\pgfpathlineto{\pgfqpoint{4.680000in}{2.347925in}}%
\pgfpathlineto{\pgfqpoint{4.680000in}{2.435661in}}%
\pgfusepath{stroke,fill}%
\end{pgfscope}%
\begin{pgfscope}%
\pgfpathrectangle{\pgfqpoint{0.380943in}{2.260189in}}{\pgfqpoint{4.650000in}{0.614151in}}%
\pgfusepath{clip}%
\pgfsetbuttcap%
\pgfsetroundjoin%
\definecolor{currentfill}{rgb}{0.964937,0.908651,0.713110}%
\pgfsetfillcolor{currentfill}%
\pgfsetlinewidth{0.250937pt}%
\definecolor{currentstroke}{rgb}{1.000000,1.000000,1.000000}%
\pgfsetstrokecolor{currentstroke}%
\pgfsetdash{}{0pt}%
\pgfpathmoveto{\pgfqpoint{4.767736in}{2.435661in}}%
\pgfpathlineto{\pgfqpoint{4.855471in}{2.435661in}}%
\pgfpathlineto{\pgfqpoint{4.855471in}{2.347925in}}%
\pgfpathlineto{\pgfqpoint{4.767736in}{2.347925in}}%
\pgfpathlineto{\pgfqpoint{4.767736in}{2.435661in}}%
\pgfusepath{stroke,fill}%
\end{pgfscope}%
\begin{pgfscope}%
\pgfpathrectangle{\pgfqpoint{0.380943in}{2.260189in}}{\pgfqpoint{4.650000in}{0.614151in}}%
\pgfusepath{clip}%
\pgfsetbuttcap%
\pgfsetroundjoin%
\definecolor{currentfill}{rgb}{0.995233,0.991895,0.818977}%
\pgfsetfillcolor{currentfill}%
\pgfsetlinewidth{0.250937pt}%
\definecolor{currentstroke}{rgb}{1.000000,1.000000,1.000000}%
\pgfsetstrokecolor{currentstroke}%
\pgfsetdash{}{0pt}%
\pgfpathmoveto{\pgfqpoint{4.855471in}{2.435661in}}%
\pgfpathlineto{\pgfqpoint{4.943207in}{2.435661in}}%
\pgfpathlineto{\pgfqpoint{4.943207in}{2.347925in}}%
\pgfpathlineto{\pgfqpoint{4.855471in}{2.347925in}}%
\pgfpathlineto{\pgfqpoint{4.855471in}{2.435661in}}%
\pgfusepath{stroke,fill}%
\end{pgfscope}%
\begin{pgfscope}%
\pgfpathrectangle{\pgfqpoint{0.380943in}{2.260189in}}{\pgfqpoint{4.650000in}{0.614151in}}%
\pgfusepath{clip}%
\pgfsetbuttcap%
\pgfsetroundjoin%
\pgfsetlinewidth{0.250937pt}%
\definecolor{currentstroke}{rgb}{1.000000,1.000000,1.000000}%
\pgfsetstrokecolor{currentstroke}%
\pgfsetdash{}{0pt}%
\pgfpathmoveto{\pgfqpoint{4.943207in}{2.435661in}}%
\pgfpathlineto{\pgfqpoint{5.030943in}{2.435661in}}%
\pgfpathlineto{\pgfqpoint{5.030943in}{2.347925in}}%
\pgfpathlineto{\pgfqpoint{4.943207in}{2.347925in}}%
\pgfpathlineto{\pgfqpoint{4.943207in}{2.435661in}}%
\pgfusepath{stroke}%
\end{pgfscope}%
\begin{pgfscope}%
\pgfpathrectangle{\pgfqpoint{0.380943in}{2.260189in}}{\pgfqpoint{4.650000in}{0.614151in}}%
\pgfusepath{clip}%
\pgfsetbuttcap%
\pgfsetroundjoin%
\definecolor{currentfill}{rgb}{0.964783,0.940131,0.739808}%
\pgfsetfillcolor{currentfill}%
\pgfsetlinewidth{0.250937pt}%
\definecolor{currentstroke}{rgb}{1.000000,1.000000,1.000000}%
\pgfsetstrokecolor{currentstroke}%
\pgfsetdash{}{0pt}%
\pgfpathmoveto{\pgfqpoint{0.380943in}{2.347925in}}%
\pgfpathlineto{\pgfqpoint{0.468679in}{2.347925in}}%
\pgfpathlineto{\pgfqpoint{0.468679in}{2.260189in}}%
\pgfpathlineto{\pgfqpoint{0.380943in}{2.260189in}}%
\pgfpathlineto{\pgfqpoint{0.380943in}{2.347925in}}%
\pgfusepath{stroke,fill}%
\end{pgfscope}%
\begin{pgfscope}%
\pgfpathrectangle{\pgfqpoint{0.380943in}{2.260189in}}{\pgfqpoint{4.650000in}{0.614151in}}%
\pgfusepath{clip}%
\pgfsetbuttcap%
\pgfsetroundjoin%
\definecolor{currentfill}{rgb}{0.995233,0.991895,0.818977}%
\pgfsetfillcolor{currentfill}%
\pgfsetlinewidth{0.250937pt}%
\definecolor{currentstroke}{rgb}{1.000000,1.000000,1.000000}%
\pgfsetstrokecolor{currentstroke}%
\pgfsetdash{}{0pt}%
\pgfpathmoveto{\pgfqpoint{0.468679in}{2.347925in}}%
\pgfpathlineto{\pgfqpoint{0.556415in}{2.347925in}}%
\pgfpathlineto{\pgfqpoint{0.556415in}{2.260189in}}%
\pgfpathlineto{\pgfqpoint{0.468679in}{2.260189in}}%
\pgfpathlineto{\pgfqpoint{0.468679in}{2.347925in}}%
\pgfusepath{stroke,fill}%
\end{pgfscope}%
\begin{pgfscope}%
\pgfpathrectangle{\pgfqpoint{0.380943in}{2.260189in}}{\pgfqpoint{4.650000in}{0.614151in}}%
\pgfusepath{clip}%
\pgfsetbuttcap%
\pgfsetroundjoin%
\definecolor{currentfill}{rgb}{1.000000,1.000000,0.857516}%
\pgfsetfillcolor{currentfill}%
\pgfsetlinewidth{0.250937pt}%
\definecolor{currentstroke}{rgb}{1.000000,1.000000,1.000000}%
\pgfsetstrokecolor{currentstroke}%
\pgfsetdash{}{0pt}%
\pgfpathmoveto{\pgfqpoint{0.556415in}{2.347925in}}%
\pgfpathlineto{\pgfqpoint{0.644151in}{2.347925in}}%
\pgfpathlineto{\pgfqpoint{0.644151in}{2.260189in}}%
\pgfpathlineto{\pgfqpoint{0.556415in}{2.260189in}}%
\pgfpathlineto{\pgfqpoint{0.556415in}{2.347925in}}%
\pgfusepath{stroke,fill}%
\end{pgfscope}%
\begin{pgfscope}%
\pgfpathrectangle{\pgfqpoint{0.380943in}{2.260189in}}{\pgfqpoint{4.650000in}{0.614151in}}%
\pgfusepath{clip}%
\pgfsetbuttcap%
\pgfsetroundjoin%
\definecolor{currentfill}{rgb}{1.000000,1.000000,0.857516}%
\pgfsetfillcolor{currentfill}%
\pgfsetlinewidth{0.250937pt}%
\definecolor{currentstroke}{rgb}{1.000000,1.000000,1.000000}%
\pgfsetstrokecolor{currentstroke}%
\pgfsetdash{}{0pt}%
\pgfpathmoveto{\pgfqpoint{0.644151in}{2.347925in}}%
\pgfpathlineto{\pgfqpoint{0.731886in}{2.347925in}}%
\pgfpathlineto{\pgfqpoint{0.731886in}{2.260189in}}%
\pgfpathlineto{\pgfqpoint{0.644151in}{2.260189in}}%
\pgfpathlineto{\pgfqpoint{0.644151in}{2.347925in}}%
\pgfusepath{stroke,fill}%
\end{pgfscope}%
\begin{pgfscope}%
\pgfpathrectangle{\pgfqpoint{0.380943in}{2.260189in}}{\pgfqpoint{4.650000in}{0.614151in}}%
\pgfusepath{clip}%
\pgfsetbuttcap%
\pgfsetroundjoin%
\definecolor{currentfill}{rgb}{1.000000,1.000000,0.857516}%
\pgfsetfillcolor{currentfill}%
\pgfsetlinewidth{0.250937pt}%
\definecolor{currentstroke}{rgb}{1.000000,1.000000,1.000000}%
\pgfsetstrokecolor{currentstroke}%
\pgfsetdash{}{0pt}%
\pgfpathmoveto{\pgfqpoint{0.731886in}{2.347925in}}%
\pgfpathlineto{\pgfqpoint{0.819622in}{2.347925in}}%
\pgfpathlineto{\pgfqpoint{0.819622in}{2.260189in}}%
\pgfpathlineto{\pgfqpoint{0.731886in}{2.260189in}}%
\pgfpathlineto{\pgfqpoint{0.731886in}{2.347925in}}%
\pgfusepath{stroke,fill}%
\end{pgfscope}%
\begin{pgfscope}%
\pgfpathrectangle{\pgfqpoint{0.380943in}{2.260189in}}{\pgfqpoint{4.650000in}{0.614151in}}%
\pgfusepath{clip}%
\pgfsetbuttcap%
\pgfsetroundjoin%
\definecolor{currentfill}{rgb}{0.980008,0.966013,0.779393}%
\pgfsetfillcolor{currentfill}%
\pgfsetlinewidth{0.250937pt}%
\definecolor{currentstroke}{rgb}{1.000000,1.000000,1.000000}%
\pgfsetstrokecolor{currentstroke}%
\pgfsetdash{}{0pt}%
\pgfpathmoveto{\pgfqpoint{0.819622in}{2.347925in}}%
\pgfpathlineto{\pgfqpoint{0.907358in}{2.347925in}}%
\pgfpathlineto{\pgfqpoint{0.907358in}{2.260189in}}%
\pgfpathlineto{\pgfqpoint{0.819622in}{2.260189in}}%
\pgfpathlineto{\pgfqpoint{0.819622in}{2.347925in}}%
\pgfusepath{stroke,fill}%
\end{pgfscope}%
\begin{pgfscope}%
\pgfpathrectangle{\pgfqpoint{0.380943in}{2.260189in}}{\pgfqpoint{4.650000in}{0.614151in}}%
\pgfusepath{clip}%
\pgfsetbuttcap%
\pgfsetroundjoin%
\definecolor{currentfill}{rgb}{1.000000,1.000000,0.857516}%
\pgfsetfillcolor{currentfill}%
\pgfsetlinewidth{0.250937pt}%
\definecolor{currentstroke}{rgb}{1.000000,1.000000,1.000000}%
\pgfsetstrokecolor{currentstroke}%
\pgfsetdash{}{0pt}%
\pgfpathmoveto{\pgfqpoint{0.907358in}{2.347925in}}%
\pgfpathlineto{\pgfqpoint{0.995094in}{2.347925in}}%
\pgfpathlineto{\pgfqpoint{0.995094in}{2.260189in}}%
\pgfpathlineto{\pgfqpoint{0.907358in}{2.260189in}}%
\pgfpathlineto{\pgfqpoint{0.907358in}{2.347925in}}%
\pgfusepath{stroke,fill}%
\end{pgfscope}%
\begin{pgfscope}%
\pgfpathrectangle{\pgfqpoint{0.380943in}{2.260189in}}{\pgfqpoint{4.650000in}{0.614151in}}%
\pgfusepath{clip}%
\pgfsetbuttcap%
\pgfsetroundjoin%
\definecolor{currentfill}{rgb}{0.995233,0.991895,0.818977}%
\pgfsetfillcolor{currentfill}%
\pgfsetlinewidth{0.250937pt}%
\definecolor{currentstroke}{rgb}{1.000000,1.000000,1.000000}%
\pgfsetstrokecolor{currentstroke}%
\pgfsetdash{}{0pt}%
\pgfpathmoveto{\pgfqpoint{0.995094in}{2.347925in}}%
\pgfpathlineto{\pgfqpoint{1.082830in}{2.347925in}}%
\pgfpathlineto{\pgfqpoint{1.082830in}{2.260189in}}%
\pgfpathlineto{\pgfqpoint{0.995094in}{2.260189in}}%
\pgfpathlineto{\pgfqpoint{0.995094in}{2.347925in}}%
\pgfusepath{stroke,fill}%
\end{pgfscope}%
\begin{pgfscope}%
\pgfpathrectangle{\pgfqpoint{0.380943in}{2.260189in}}{\pgfqpoint{4.650000in}{0.614151in}}%
\pgfusepath{clip}%
\pgfsetbuttcap%
\pgfsetroundjoin%
\definecolor{currentfill}{rgb}{0.961738,0.927612,0.725598}%
\pgfsetfillcolor{currentfill}%
\pgfsetlinewidth{0.250937pt}%
\definecolor{currentstroke}{rgb}{1.000000,1.000000,1.000000}%
\pgfsetstrokecolor{currentstroke}%
\pgfsetdash{}{0pt}%
\pgfpathmoveto{\pgfqpoint{1.082830in}{2.347925in}}%
\pgfpathlineto{\pgfqpoint{1.170566in}{2.347925in}}%
\pgfpathlineto{\pgfqpoint{1.170566in}{2.260189in}}%
\pgfpathlineto{\pgfqpoint{1.082830in}{2.260189in}}%
\pgfpathlineto{\pgfqpoint{1.082830in}{2.347925in}}%
\pgfusepath{stroke,fill}%
\end{pgfscope}%
\begin{pgfscope}%
\pgfpathrectangle{\pgfqpoint{0.380943in}{2.260189in}}{\pgfqpoint{4.650000in}{0.614151in}}%
\pgfusepath{clip}%
\pgfsetbuttcap%
\pgfsetroundjoin%
\definecolor{currentfill}{rgb}{0.995233,0.991895,0.818977}%
\pgfsetfillcolor{currentfill}%
\pgfsetlinewidth{0.250937pt}%
\definecolor{currentstroke}{rgb}{1.000000,1.000000,1.000000}%
\pgfsetstrokecolor{currentstroke}%
\pgfsetdash{}{0pt}%
\pgfpathmoveto{\pgfqpoint{1.170566in}{2.347925in}}%
\pgfpathlineto{\pgfqpoint{1.258302in}{2.347925in}}%
\pgfpathlineto{\pgfqpoint{1.258302in}{2.260189in}}%
\pgfpathlineto{\pgfqpoint{1.170566in}{2.260189in}}%
\pgfpathlineto{\pgfqpoint{1.170566in}{2.347925in}}%
\pgfusepath{stroke,fill}%
\end{pgfscope}%
\begin{pgfscope}%
\pgfpathrectangle{\pgfqpoint{0.380943in}{2.260189in}}{\pgfqpoint{4.650000in}{0.614151in}}%
\pgfusepath{clip}%
\pgfsetbuttcap%
\pgfsetroundjoin%
\definecolor{currentfill}{rgb}{1.000000,1.000000,0.929412}%
\pgfsetfillcolor{currentfill}%
\pgfsetlinewidth{0.250937pt}%
\definecolor{currentstroke}{rgb}{1.000000,1.000000,1.000000}%
\pgfsetstrokecolor{currentstroke}%
\pgfsetdash{}{0pt}%
\pgfpathmoveto{\pgfqpoint{1.258302in}{2.347925in}}%
\pgfpathlineto{\pgfqpoint{1.346037in}{2.347925in}}%
\pgfpathlineto{\pgfqpoint{1.346037in}{2.260189in}}%
\pgfpathlineto{\pgfqpoint{1.258302in}{2.260189in}}%
\pgfpathlineto{\pgfqpoint{1.258302in}{2.347925in}}%
\pgfusepath{stroke,fill}%
\end{pgfscope}%
\begin{pgfscope}%
\pgfpathrectangle{\pgfqpoint{0.380943in}{2.260189in}}{\pgfqpoint{4.650000in}{0.614151in}}%
\pgfusepath{clip}%
\pgfsetbuttcap%
\pgfsetroundjoin%
\definecolor{currentfill}{rgb}{1.000000,1.000000,0.895579}%
\pgfsetfillcolor{currentfill}%
\pgfsetlinewidth{0.250937pt}%
\definecolor{currentstroke}{rgb}{1.000000,1.000000,1.000000}%
\pgfsetstrokecolor{currentstroke}%
\pgfsetdash{}{0pt}%
\pgfpathmoveto{\pgfqpoint{1.346037in}{2.347925in}}%
\pgfpathlineto{\pgfqpoint{1.433773in}{2.347925in}}%
\pgfpathlineto{\pgfqpoint{1.433773in}{2.260189in}}%
\pgfpathlineto{\pgfqpoint{1.346037in}{2.260189in}}%
\pgfpathlineto{\pgfqpoint{1.346037in}{2.347925in}}%
\pgfusepath{stroke,fill}%
\end{pgfscope}%
\begin{pgfscope}%
\pgfpathrectangle{\pgfqpoint{0.380943in}{2.260189in}}{\pgfqpoint{4.650000in}{0.614151in}}%
\pgfusepath{clip}%
\pgfsetbuttcap%
\pgfsetroundjoin%
\definecolor{currentfill}{rgb}{1.000000,1.000000,0.929412}%
\pgfsetfillcolor{currentfill}%
\pgfsetlinewidth{0.250937pt}%
\definecolor{currentstroke}{rgb}{1.000000,1.000000,1.000000}%
\pgfsetstrokecolor{currentstroke}%
\pgfsetdash{}{0pt}%
\pgfpathmoveto{\pgfqpoint{1.433773in}{2.347925in}}%
\pgfpathlineto{\pgfqpoint{1.521509in}{2.347925in}}%
\pgfpathlineto{\pgfqpoint{1.521509in}{2.260189in}}%
\pgfpathlineto{\pgfqpoint{1.433773in}{2.260189in}}%
\pgfpathlineto{\pgfqpoint{1.433773in}{2.347925in}}%
\pgfusepath{stroke,fill}%
\end{pgfscope}%
\begin{pgfscope}%
\pgfpathrectangle{\pgfqpoint{0.380943in}{2.260189in}}{\pgfqpoint{4.650000in}{0.614151in}}%
\pgfusepath{clip}%
\pgfsetbuttcap%
\pgfsetroundjoin%
\definecolor{currentfill}{rgb}{1.000000,1.000000,0.929412}%
\pgfsetfillcolor{currentfill}%
\pgfsetlinewidth{0.250937pt}%
\definecolor{currentstroke}{rgb}{1.000000,1.000000,1.000000}%
\pgfsetstrokecolor{currentstroke}%
\pgfsetdash{}{0pt}%
\pgfpathmoveto{\pgfqpoint{1.521509in}{2.347925in}}%
\pgfpathlineto{\pgfqpoint{1.609245in}{2.347925in}}%
\pgfpathlineto{\pgfqpoint{1.609245in}{2.260189in}}%
\pgfpathlineto{\pgfqpoint{1.521509in}{2.260189in}}%
\pgfpathlineto{\pgfqpoint{1.521509in}{2.347925in}}%
\pgfusepath{stroke,fill}%
\end{pgfscope}%
\begin{pgfscope}%
\pgfpathrectangle{\pgfqpoint{0.380943in}{2.260189in}}{\pgfqpoint{4.650000in}{0.614151in}}%
\pgfusepath{clip}%
\pgfsetbuttcap%
\pgfsetroundjoin%
\definecolor{currentfill}{rgb}{1.000000,1.000000,0.857516}%
\pgfsetfillcolor{currentfill}%
\pgfsetlinewidth{0.250937pt}%
\definecolor{currentstroke}{rgb}{1.000000,1.000000,1.000000}%
\pgfsetstrokecolor{currentstroke}%
\pgfsetdash{}{0pt}%
\pgfpathmoveto{\pgfqpoint{1.609245in}{2.347925in}}%
\pgfpathlineto{\pgfqpoint{1.696981in}{2.347925in}}%
\pgfpathlineto{\pgfqpoint{1.696981in}{2.260189in}}%
\pgfpathlineto{\pgfqpoint{1.609245in}{2.260189in}}%
\pgfpathlineto{\pgfqpoint{1.609245in}{2.347925in}}%
\pgfusepath{stroke,fill}%
\end{pgfscope}%
\begin{pgfscope}%
\pgfpathrectangle{\pgfqpoint{0.380943in}{2.260189in}}{\pgfqpoint{4.650000in}{0.614151in}}%
\pgfusepath{clip}%
\pgfsetbuttcap%
\pgfsetroundjoin%
\definecolor{currentfill}{rgb}{0.995233,0.991895,0.818977}%
\pgfsetfillcolor{currentfill}%
\pgfsetlinewidth{0.250937pt}%
\definecolor{currentstroke}{rgb}{1.000000,1.000000,1.000000}%
\pgfsetstrokecolor{currentstroke}%
\pgfsetdash{}{0pt}%
\pgfpathmoveto{\pgfqpoint{1.696981in}{2.347925in}}%
\pgfpathlineto{\pgfqpoint{1.784717in}{2.347925in}}%
\pgfpathlineto{\pgfqpoint{1.784717in}{2.260189in}}%
\pgfpathlineto{\pgfqpoint{1.696981in}{2.260189in}}%
\pgfpathlineto{\pgfqpoint{1.696981in}{2.347925in}}%
\pgfusepath{stroke,fill}%
\end{pgfscope}%
\begin{pgfscope}%
\pgfpathrectangle{\pgfqpoint{0.380943in}{2.260189in}}{\pgfqpoint{4.650000in}{0.614151in}}%
\pgfusepath{clip}%
\pgfsetbuttcap%
\pgfsetroundjoin%
\definecolor{currentfill}{rgb}{0.980008,0.966013,0.779393}%
\pgfsetfillcolor{currentfill}%
\pgfsetlinewidth{0.250937pt}%
\definecolor{currentstroke}{rgb}{1.000000,1.000000,1.000000}%
\pgfsetstrokecolor{currentstroke}%
\pgfsetdash{}{0pt}%
\pgfpathmoveto{\pgfqpoint{1.784717in}{2.347925in}}%
\pgfpathlineto{\pgfqpoint{1.872452in}{2.347925in}}%
\pgfpathlineto{\pgfqpoint{1.872452in}{2.260189in}}%
\pgfpathlineto{\pgfqpoint{1.784717in}{2.260189in}}%
\pgfpathlineto{\pgfqpoint{1.784717in}{2.347925in}}%
\pgfusepath{stroke,fill}%
\end{pgfscope}%
\begin{pgfscope}%
\pgfpathrectangle{\pgfqpoint{0.380943in}{2.260189in}}{\pgfqpoint{4.650000in}{0.614151in}}%
\pgfusepath{clip}%
\pgfsetbuttcap%
\pgfsetroundjoin%
\definecolor{currentfill}{rgb}{0.980008,0.966013,0.779393}%
\pgfsetfillcolor{currentfill}%
\pgfsetlinewidth{0.250937pt}%
\definecolor{currentstroke}{rgb}{1.000000,1.000000,1.000000}%
\pgfsetstrokecolor{currentstroke}%
\pgfsetdash{}{0pt}%
\pgfpathmoveto{\pgfqpoint{1.872452in}{2.347925in}}%
\pgfpathlineto{\pgfqpoint{1.960188in}{2.347925in}}%
\pgfpathlineto{\pgfqpoint{1.960188in}{2.260189in}}%
\pgfpathlineto{\pgfqpoint{1.872452in}{2.260189in}}%
\pgfpathlineto{\pgfqpoint{1.872452in}{2.347925in}}%
\pgfusepath{stroke,fill}%
\end{pgfscope}%
\begin{pgfscope}%
\pgfpathrectangle{\pgfqpoint{0.380943in}{2.260189in}}{\pgfqpoint{4.650000in}{0.614151in}}%
\pgfusepath{clip}%
\pgfsetbuttcap%
\pgfsetroundjoin%
\definecolor{currentfill}{rgb}{0.980008,0.966013,0.779393}%
\pgfsetfillcolor{currentfill}%
\pgfsetlinewidth{0.250937pt}%
\definecolor{currentstroke}{rgb}{1.000000,1.000000,1.000000}%
\pgfsetstrokecolor{currentstroke}%
\pgfsetdash{}{0pt}%
\pgfpathmoveto{\pgfqpoint{1.960188in}{2.347925in}}%
\pgfpathlineto{\pgfqpoint{2.047924in}{2.347925in}}%
\pgfpathlineto{\pgfqpoint{2.047924in}{2.260189in}}%
\pgfpathlineto{\pgfqpoint{1.960188in}{2.260189in}}%
\pgfpathlineto{\pgfqpoint{1.960188in}{2.347925in}}%
\pgfusepath{stroke,fill}%
\end{pgfscope}%
\begin{pgfscope}%
\pgfpathrectangle{\pgfqpoint{0.380943in}{2.260189in}}{\pgfqpoint{4.650000in}{0.614151in}}%
\pgfusepath{clip}%
\pgfsetbuttcap%
\pgfsetroundjoin%
\definecolor{currentfill}{rgb}{1.000000,1.000000,0.895579}%
\pgfsetfillcolor{currentfill}%
\pgfsetlinewidth{0.250937pt}%
\definecolor{currentstroke}{rgb}{1.000000,1.000000,1.000000}%
\pgfsetstrokecolor{currentstroke}%
\pgfsetdash{}{0pt}%
\pgfpathmoveto{\pgfqpoint{2.047924in}{2.347925in}}%
\pgfpathlineto{\pgfqpoint{2.135660in}{2.347925in}}%
\pgfpathlineto{\pgfqpoint{2.135660in}{2.260189in}}%
\pgfpathlineto{\pgfqpoint{2.047924in}{2.260189in}}%
\pgfpathlineto{\pgfqpoint{2.047924in}{2.347925in}}%
\pgfusepath{stroke,fill}%
\end{pgfscope}%
\begin{pgfscope}%
\pgfpathrectangle{\pgfqpoint{0.380943in}{2.260189in}}{\pgfqpoint{4.650000in}{0.614151in}}%
\pgfusepath{clip}%
\pgfsetbuttcap%
\pgfsetroundjoin%
\definecolor{currentfill}{rgb}{0.995233,0.991895,0.818977}%
\pgfsetfillcolor{currentfill}%
\pgfsetlinewidth{0.250937pt}%
\definecolor{currentstroke}{rgb}{1.000000,1.000000,1.000000}%
\pgfsetstrokecolor{currentstroke}%
\pgfsetdash{}{0pt}%
\pgfpathmoveto{\pgfqpoint{2.135660in}{2.347925in}}%
\pgfpathlineto{\pgfqpoint{2.223396in}{2.347925in}}%
\pgfpathlineto{\pgfqpoint{2.223396in}{2.260189in}}%
\pgfpathlineto{\pgfqpoint{2.135660in}{2.260189in}}%
\pgfpathlineto{\pgfqpoint{2.135660in}{2.347925in}}%
\pgfusepath{stroke,fill}%
\end{pgfscope}%
\begin{pgfscope}%
\pgfpathrectangle{\pgfqpoint{0.380943in}{2.260189in}}{\pgfqpoint{4.650000in}{0.614151in}}%
\pgfusepath{clip}%
\pgfsetbuttcap%
\pgfsetroundjoin%
\definecolor{currentfill}{rgb}{0.995233,0.991895,0.818977}%
\pgfsetfillcolor{currentfill}%
\pgfsetlinewidth{0.250937pt}%
\definecolor{currentstroke}{rgb}{1.000000,1.000000,1.000000}%
\pgfsetstrokecolor{currentstroke}%
\pgfsetdash{}{0pt}%
\pgfpathmoveto{\pgfqpoint{2.223396in}{2.347925in}}%
\pgfpathlineto{\pgfqpoint{2.311132in}{2.347925in}}%
\pgfpathlineto{\pgfqpoint{2.311132in}{2.260189in}}%
\pgfpathlineto{\pgfqpoint{2.223396in}{2.260189in}}%
\pgfpathlineto{\pgfqpoint{2.223396in}{2.347925in}}%
\pgfusepath{stroke,fill}%
\end{pgfscope}%
\begin{pgfscope}%
\pgfpathrectangle{\pgfqpoint{0.380943in}{2.260189in}}{\pgfqpoint{4.650000in}{0.614151in}}%
\pgfusepath{clip}%
\pgfsetbuttcap%
\pgfsetroundjoin%
\definecolor{currentfill}{rgb}{0.980008,0.966013,0.779393}%
\pgfsetfillcolor{currentfill}%
\pgfsetlinewidth{0.250937pt}%
\definecolor{currentstroke}{rgb}{1.000000,1.000000,1.000000}%
\pgfsetstrokecolor{currentstroke}%
\pgfsetdash{}{0pt}%
\pgfpathmoveto{\pgfqpoint{2.311132in}{2.347925in}}%
\pgfpathlineto{\pgfqpoint{2.398868in}{2.347925in}}%
\pgfpathlineto{\pgfqpoint{2.398868in}{2.260189in}}%
\pgfpathlineto{\pgfqpoint{2.311132in}{2.260189in}}%
\pgfpathlineto{\pgfqpoint{2.311132in}{2.347925in}}%
\pgfusepath{stroke,fill}%
\end{pgfscope}%
\begin{pgfscope}%
\pgfpathrectangle{\pgfqpoint{0.380943in}{2.260189in}}{\pgfqpoint{4.650000in}{0.614151in}}%
\pgfusepath{clip}%
\pgfsetbuttcap%
\pgfsetroundjoin%
\definecolor{currentfill}{rgb}{0.974072,0.862976,0.688750}%
\pgfsetfillcolor{currentfill}%
\pgfsetlinewidth{0.250937pt}%
\definecolor{currentstroke}{rgb}{1.000000,1.000000,1.000000}%
\pgfsetstrokecolor{currentstroke}%
\pgfsetdash{}{0pt}%
\pgfpathmoveto{\pgfqpoint{2.398868in}{2.347925in}}%
\pgfpathlineto{\pgfqpoint{2.486603in}{2.347925in}}%
\pgfpathlineto{\pgfqpoint{2.486603in}{2.260189in}}%
\pgfpathlineto{\pgfqpoint{2.398868in}{2.260189in}}%
\pgfpathlineto{\pgfqpoint{2.398868in}{2.347925in}}%
\pgfusepath{stroke,fill}%
\end{pgfscope}%
\begin{pgfscope}%
\pgfpathrectangle{\pgfqpoint{0.380943in}{2.260189in}}{\pgfqpoint{4.650000in}{0.614151in}}%
\pgfusepath{clip}%
\pgfsetbuttcap%
\pgfsetroundjoin%
\definecolor{currentfill}{rgb}{0.964937,0.908651,0.713110}%
\pgfsetfillcolor{currentfill}%
\pgfsetlinewidth{0.250937pt}%
\definecolor{currentstroke}{rgb}{1.000000,1.000000,1.000000}%
\pgfsetstrokecolor{currentstroke}%
\pgfsetdash{}{0pt}%
\pgfpathmoveto{\pgfqpoint{2.486603in}{2.347925in}}%
\pgfpathlineto{\pgfqpoint{2.574339in}{2.347925in}}%
\pgfpathlineto{\pgfqpoint{2.574339in}{2.260189in}}%
\pgfpathlineto{\pgfqpoint{2.486603in}{2.260189in}}%
\pgfpathlineto{\pgfqpoint{2.486603in}{2.347925in}}%
\pgfusepath{stroke,fill}%
\end{pgfscope}%
\begin{pgfscope}%
\pgfpathrectangle{\pgfqpoint{0.380943in}{2.260189in}}{\pgfqpoint{4.650000in}{0.614151in}}%
\pgfusepath{clip}%
\pgfsetbuttcap%
\pgfsetroundjoin%
\definecolor{currentfill}{rgb}{0.963260,0.918478,0.719508}%
\pgfsetfillcolor{currentfill}%
\pgfsetlinewidth{0.250937pt}%
\definecolor{currentstroke}{rgb}{1.000000,1.000000,1.000000}%
\pgfsetstrokecolor{currentstroke}%
\pgfsetdash{}{0pt}%
\pgfpathmoveto{\pgfqpoint{2.574339in}{2.347925in}}%
\pgfpathlineto{\pgfqpoint{2.662075in}{2.347925in}}%
\pgfpathlineto{\pgfqpoint{2.662075in}{2.260189in}}%
\pgfpathlineto{\pgfqpoint{2.574339in}{2.260189in}}%
\pgfpathlineto{\pgfqpoint{2.574339in}{2.347925in}}%
\pgfusepath{stroke,fill}%
\end{pgfscope}%
\begin{pgfscope}%
\pgfpathrectangle{\pgfqpoint{0.380943in}{2.260189in}}{\pgfqpoint{4.650000in}{0.614151in}}%
\pgfusepath{clip}%
\pgfsetbuttcap%
\pgfsetroundjoin%
\definecolor{currentfill}{rgb}{1.000000,1.000000,0.857516}%
\pgfsetfillcolor{currentfill}%
\pgfsetlinewidth{0.250937pt}%
\definecolor{currentstroke}{rgb}{1.000000,1.000000,1.000000}%
\pgfsetstrokecolor{currentstroke}%
\pgfsetdash{}{0pt}%
\pgfpathmoveto{\pgfqpoint{2.662075in}{2.347925in}}%
\pgfpathlineto{\pgfqpoint{2.749811in}{2.347925in}}%
\pgfpathlineto{\pgfqpoint{2.749811in}{2.260189in}}%
\pgfpathlineto{\pgfqpoint{2.662075in}{2.260189in}}%
\pgfpathlineto{\pgfqpoint{2.662075in}{2.347925in}}%
\pgfusepath{stroke,fill}%
\end{pgfscope}%
\begin{pgfscope}%
\pgfpathrectangle{\pgfqpoint{0.380943in}{2.260189in}}{\pgfqpoint{4.650000in}{0.614151in}}%
\pgfusepath{clip}%
\pgfsetbuttcap%
\pgfsetroundjoin%
\definecolor{currentfill}{rgb}{0.964937,0.908651,0.713110}%
\pgfsetfillcolor{currentfill}%
\pgfsetlinewidth{0.250937pt}%
\definecolor{currentstroke}{rgb}{1.000000,1.000000,1.000000}%
\pgfsetstrokecolor{currentstroke}%
\pgfsetdash{}{0pt}%
\pgfpathmoveto{\pgfqpoint{2.749811in}{2.347925in}}%
\pgfpathlineto{\pgfqpoint{2.837547in}{2.347925in}}%
\pgfpathlineto{\pgfqpoint{2.837547in}{2.260189in}}%
\pgfpathlineto{\pgfqpoint{2.749811in}{2.260189in}}%
\pgfpathlineto{\pgfqpoint{2.749811in}{2.347925in}}%
\pgfusepath{stroke,fill}%
\end{pgfscope}%
\begin{pgfscope}%
\pgfpathrectangle{\pgfqpoint{0.380943in}{2.260189in}}{\pgfqpoint{4.650000in}{0.614151in}}%
\pgfusepath{clip}%
\pgfsetbuttcap%
\pgfsetroundjoin%
\definecolor{currentfill}{rgb}{0.963260,0.918478,0.719508}%
\pgfsetfillcolor{currentfill}%
\pgfsetlinewidth{0.250937pt}%
\definecolor{currentstroke}{rgb}{1.000000,1.000000,1.000000}%
\pgfsetstrokecolor{currentstroke}%
\pgfsetdash{}{0pt}%
\pgfpathmoveto{\pgfqpoint{2.837547in}{2.347925in}}%
\pgfpathlineto{\pgfqpoint{2.925283in}{2.347925in}}%
\pgfpathlineto{\pgfqpoint{2.925283in}{2.260189in}}%
\pgfpathlineto{\pgfqpoint{2.837547in}{2.260189in}}%
\pgfpathlineto{\pgfqpoint{2.837547in}{2.347925in}}%
\pgfusepath{stroke,fill}%
\end{pgfscope}%
\begin{pgfscope}%
\pgfpathrectangle{\pgfqpoint{0.380943in}{2.260189in}}{\pgfqpoint{4.650000in}{0.614151in}}%
\pgfusepath{clip}%
\pgfsetbuttcap%
\pgfsetroundjoin%
\definecolor{currentfill}{rgb}{0.969504,0.885813,0.700930}%
\pgfsetfillcolor{currentfill}%
\pgfsetlinewidth{0.250937pt}%
\definecolor{currentstroke}{rgb}{1.000000,1.000000,1.000000}%
\pgfsetstrokecolor{currentstroke}%
\pgfsetdash{}{0pt}%
\pgfpathmoveto{\pgfqpoint{2.925283in}{2.347925in}}%
\pgfpathlineto{\pgfqpoint{3.013019in}{2.347925in}}%
\pgfpathlineto{\pgfqpoint{3.013019in}{2.260189in}}%
\pgfpathlineto{\pgfqpoint{2.925283in}{2.260189in}}%
\pgfpathlineto{\pgfqpoint{2.925283in}{2.347925in}}%
\pgfusepath{stroke,fill}%
\end{pgfscope}%
\begin{pgfscope}%
\pgfpathrectangle{\pgfqpoint{0.380943in}{2.260189in}}{\pgfqpoint{4.650000in}{0.614151in}}%
\pgfusepath{clip}%
\pgfsetbuttcap%
\pgfsetroundjoin%
\definecolor{currentfill}{rgb}{0.964783,0.940131,0.739808}%
\pgfsetfillcolor{currentfill}%
\pgfsetlinewidth{0.250937pt}%
\definecolor{currentstroke}{rgb}{1.000000,1.000000,1.000000}%
\pgfsetstrokecolor{currentstroke}%
\pgfsetdash{}{0pt}%
\pgfpathmoveto{\pgfqpoint{3.013019in}{2.347925in}}%
\pgfpathlineto{\pgfqpoint{3.100754in}{2.347925in}}%
\pgfpathlineto{\pgfqpoint{3.100754in}{2.260189in}}%
\pgfpathlineto{\pgfqpoint{3.013019in}{2.260189in}}%
\pgfpathlineto{\pgfqpoint{3.013019in}{2.347925in}}%
\pgfusepath{stroke,fill}%
\end{pgfscope}%
\begin{pgfscope}%
\pgfpathrectangle{\pgfqpoint{0.380943in}{2.260189in}}{\pgfqpoint{4.650000in}{0.614151in}}%
\pgfusepath{clip}%
\pgfsetbuttcap%
\pgfsetroundjoin%
\definecolor{currentfill}{rgb}{0.964937,0.908651,0.713110}%
\pgfsetfillcolor{currentfill}%
\pgfsetlinewidth{0.250937pt}%
\definecolor{currentstroke}{rgb}{1.000000,1.000000,1.000000}%
\pgfsetstrokecolor{currentstroke}%
\pgfsetdash{}{0pt}%
\pgfpathmoveto{\pgfqpoint{3.100754in}{2.347925in}}%
\pgfpathlineto{\pgfqpoint{3.188490in}{2.347925in}}%
\pgfpathlineto{\pgfqpoint{3.188490in}{2.260189in}}%
\pgfpathlineto{\pgfqpoint{3.100754in}{2.260189in}}%
\pgfpathlineto{\pgfqpoint{3.100754in}{2.347925in}}%
\pgfusepath{stroke,fill}%
\end{pgfscope}%
\begin{pgfscope}%
\pgfpathrectangle{\pgfqpoint{0.380943in}{2.260189in}}{\pgfqpoint{4.650000in}{0.614151in}}%
\pgfusepath{clip}%
\pgfsetbuttcap%
\pgfsetroundjoin%
\definecolor{currentfill}{rgb}{0.995233,0.991895,0.818977}%
\pgfsetfillcolor{currentfill}%
\pgfsetlinewidth{0.250937pt}%
\definecolor{currentstroke}{rgb}{1.000000,1.000000,1.000000}%
\pgfsetstrokecolor{currentstroke}%
\pgfsetdash{}{0pt}%
\pgfpathmoveto{\pgfqpoint{3.188490in}{2.347925in}}%
\pgfpathlineto{\pgfqpoint{3.276226in}{2.347925in}}%
\pgfpathlineto{\pgfqpoint{3.276226in}{2.260189in}}%
\pgfpathlineto{\pgfqpoint{3.188490in}{2.260189in}}%
\pgfpathlineto{\pgfqpoint{3.188490in}{2.347925in}}%
\pgfusepath{stroke,fill}%
\end{pgfscope}%
\begin{pgfscope}%
\pgfpathrectangle{\pgfqpoint{0.380943in}{2.260189in}}{\pgfqpoint{4.650000in}{0.614151in}}%
\pgfusepath{clip}%
\pgfsetbuttcap%
\pgfsetroundjoin%
\definecolor{currentfill}{rgb}{0.961738,0.927612,0.725598}%
\pgfsetfillcolor{currentfill}%
\pgfsetlinewidth{0.250937pt}%
\definecolor{currentstroke}{rgb}{1.000000,1.000000,1.000000}%
\pgfsetstrokecolor{currentstroke}%
\pgfsetdash{}{0pt}%
\pgfpathmoveto{\pgfqpoint{3.276226in}{2.347925in}}%
\pgfpathlineto{\pgfqpoint{3.363962in}{2.347925in}}%
\pgfpathlineto{\pgfqpoint{3.363962in}{2.260189in}}%
\pgfpathlineto{\pgfqpoint{3.276226in}{2.260189in}}%
\pgfpathlineto{\pgfqpoint{3.276226in}{2.347925in}}%
\pgfusepath{stroke,fill}%
\end{pgfscope}%
\begin{pgfscope}%
\pgfpathrectangle{\pgfqpoint{0.380943in}{2.260189in}}{\pgfqpoint{4.650000in}{0.614151in}}%
\pgfusepath{clip}%
\pgfsetbuttcap%
\pgfsetroundjoin%
\definecolor{currentfill}{rgb}{0.961738,0.927612,0.725598}%
\pgfsetfillcolor{currentfill}%
\pgfsetlinewidth{0.250937pt}%
\definecolor{currentstroke}{rgb}{1.000000,1.000000,1.000000}%
\pgfsetstrokecolor{currentstroke}%
\pgfsetdash{}{0pt}%
\pgfpathmoveto{\pgfqpoint{3.363962in}{2.347925in}}%
\pgfpathlineto{\pgfqpoint{3.451698in}{2.347925in}}%
\pgfpathlineto{\pgfqpoint{3.451698in}{2.260189in}}%
\pgfpathlineto{\pgfqpoint{3.363962in}{2.260189in}}%
\pgfpathlineto{\pgfqpoint{3.363962in}{2.347925in}}%
\pgfusepath{stroke,fill}%
\end{pgfscope}%
\begin{pgfscope}%
\pgfpathrectangle{\pgfqpoint{0.380943in}{2.260189in}}{\pgfqpoint{4.650000in}{0.614151in}}%
\pgfusepath{clip}%
\pgfsetbuttcap%
\pgfsetroundjoin%
\definecolor{currentfill}{rgb}{0.995233,0.991895,0.818977}%
\pgfsetfillcolor{currentfill}%
\pgfsetlinewidth{0.250937pt}%
\definecolor{currentstroke}{rgb}{1.000000,1.000000,1.000000}%
\pgfsetstrokecolor{currentstroke}%
\pgfsetdash{}{0pt}%
\pgfpathmoveto{\pgfqpoint{3.451698in}{2.347925in}}%
\pgfpathlineto{\pgfqpoint{3.539434in}{2.347925in}}%
\pgfpathlineto{\pgfqpoint{3.539434in}{2.260189in}}%
\pgfpathlineto{\pgfqpoint{3.451698in}{2.260189in}}%
\pgfpathlineto{\pgfqpoint{3.451698in}{2.347925in}}%
\pgfusepath{stroke,fill}%
\end{pgfscope}%
\begin{pgfscope}%
\pgfpathrectangle{\pgfqpoint{0.380943in}{2.260189in}}{\pgfqpoint{4.650000in}{0.614151in}}%
\pgfusepath{clip}%
\pgfsetbuttcap%
\pgfsetroundjoin%
\definecolor{currentfill}{rgb}{1.000000,1.000000,0.857516}%
\pgfsetfillcolor{currentfill}%
\pgfsetlinewidth{0.250937pt}%
\definecolor{currentstroke}{rgb}{1.000000,1.000000,1.000000}%
\pgfsetstrokecolor{currentstroke}%
\pgfsetdash{}{0pt}%
\pgfpathmoveto{\pgfqpoint{3.539434in}{2.347925in}}%
\pgfpathlineto{\pgfqpoint{3.627169in}{2.347925in}}%
\pgfpathlineto{\pgfqpoint{3.627169in}{2.260189in}}%
\pgfpathlineto{\pgfqpoint{3.539434in}{2.260189in}}%
\pgfpathlineto{\pgfqpoint{3.539434in}{2.347925in}}%
\pgfusepath{stroke,fill}%
\end{pgfscope}%
\begin{pgfscope}%
\pgfpathrectangle{\pgfqpoint{0.380943in}{2.260189in}}{\pgfqpoint{4.650000in}{0.614151in}}%
\pgfusepath{clip}%
\pgfsetbuttcap%
\pgfsetroundjoin%
\definecolor{currentfill}{rgb}{1.000000,1.000000,0.895579}%
\pgfsetfillcolor{currentfill}%
\pgfsetlinewidth{0.250937pt}%
\definecolor{currentstroke}{rgb}{1.000000,1.000000,1.000000}%
\pgfsetstrokecolor{currentstroke}%
\pgfsetdash{}{0pt}%
\pgfpathmoveto{\pgfqpoint{3.627169in}{2.347925in}}%
\pgfpathlineto{\pgfqpoint{3.714905in}{2.347925in}}%
\pgfpathlineto{\pgfqpoint{3.714905in}{2.260189in}}%
\pgfpathlineto{\pgfqpoint{3.627169in}{2.260189in}}%
\pgfpathlineto{\pgfqpoint{3.627169in}{2.347925in}}%
\pgfusepath{stroke,fill}%
\end{pgfscope}%
\begin{pgfscope}%
\pgfpathrectangle{\pgfqpoint{0.380943in}{2.260189in}}{\pgfqpoint{4.650000in}{0.614151in}}%
\pgfusepath{clip}%
\pgfsetbuttcap%
\pgfsetroundjoin%
\definecolor{currentfill}{rgb}{0.961738,0.927612,0.725598}%
\pgfsetfillcolor{currentfill}%
\pgfsetlinewidth{0.250937pt}%
\definecolor{currentstroke}{rgb}{1.000000,1.000000,1.000000}%
\pgfsetstrokecolor{currentstroke}%
\pgfsetdash{}{0pt}%
\pgfpathmoveto{\pgfqpoint{3.714905in}{2.347925in}}%
\pgfpathlineto{\pgfqpoint{3.802641in}{2.347925in}}%
\pgfpathlineto{\pgfqpoint{3.802641in}{2.260189in}}%
\pgfpathlineto{\pgfqpoint{3.714905in}{2.260189in}}%
\pgfpathlineto{\pgfqpoint{3.714905in}{2.347925in}}%
\pgfusepath{stroke,fill}%
\end{pgfscope}%
\begin{pgfscope}%
\pgfpathrectangle{\pgfqpoint{0.380943in}{2.260189in}}{\pgfqpoint{4.650000in}{0.614151in}}%
\pgfusepath{clip}%
\pgfsetbuttcap%
\pgfsetroundjoin%
\definecolor{currentfill}{rgb}{0.963260,0.918478,0.719508}%
\pgfsetfillcolor{currentfill}%
\pgfsetlinewidth{0.250937pt}%
\definecolor{currentstroke}{rgb}{1.000000,1.000000,1.000000}%
\pgfsetstrokecolor{currentstroke}%
\pgfsetdash{}{0pt}%
\pgfpathmoveto{\pgfqpoint{3.802641in}{2.347925in}}%
\pgfpathlineto{\pgfqpoint{3.890377in}{2.347925in}}%
\pgfpathlineto{\pgfqpoint{3.890377in}{2.260189in}}%
\pgfpathlineto{\pgfqpoint{3.802641in}{2.260189in}}%
\pgfpathlineto{\pgfqpoint{3.802641in}{2.347925in}}%
\pgfusepath{stroke,fill}%
\end{pgfscope}%
\begin{pgfscope}%
\pgfpathrectangle{\pgfqpoint{0.380943in}{2.260189in}}{\pgfqpoint{4.650000in}{0.614151in}}%
\pgfusepath{clip}%
\pgfsetbuttcap%
\pgfsetroundjoin%
\definecolor{currentfill}{rgb}{1.000000,1.000000,0.857516}%
\pgfsetfillcolor{currentfill}%
\pgfsetlinewidth{0.250937pt}%
\definecolor{currentstroke}{rgb}{1.000000,1.000000,1.000000}%
\pgfsetstrokecolor{currentstroke}%
\pgfsetdash{}{0pt}%
\pgfpathmoveto{\pgfqpoint{3.890377in}{2.347925in}}%
\pgfpathlineto{\pgfqpoint{3.978113in}{2.347925in}}%
\pgfpathlineto{\pgfqpoint{3.978113in}{2.260189in}}%
\pgfpathlineto{\pgfqpoint{3.890377in}{2.260189in}}%
\pgfpathlineto{\pgfqpoint{3.890377in}{2.347925in}}%
\pgfusepath{stroke,fill}%
\end{pgfscope}%
\begin{pgfscope}%
\pgfpathrectangle{\pgfqpoint{0.380943in}{2.260189in}}{\pgfqpoint{4.650000in}{0.614151in}}%
\pgfusepath{clip}%
\pgfsetbuttcap%
\pgfsetroundjoin%
\definecolor{currentfill}{rgb}{0.964783,0.940131,0.739808}%
\pgfsetfillcolor{currentfill}%
\pgfsetlinewidth{0.250937pt}%
\definecolor{currentstroke}{rgb}{1.000000,1.000000,1.000000}%
\pgfsetstrokecolor{currentstroke}%
\pgfsetdash{}{0pt}%
\pgfpathmoveto{\pgfqpoint{3.978113in}{2.347925in}}%
\pgfpathlineto{\pgfqpoint{4.065849in}{2.347925in}}%
\pgfpathlineto{\pgfqpoint{4.065849in}{2.260189in}}%
\pgfpathlineto{\pgfqpoint{3.978113in}{2.260189in}}%
\pgfpathlineto{\pgfqpoint{3.978113in}{2.347925in}}%
\pgfusepath{stroke,fill}%
\end{pgfscope}%
\begin{pgfscope}%
\pgfpathrectangle{\pgfqpoint{0.380943in}{2.260189in}}{\pgfqpoint{4.650000in}{0.614151in}}%
\pgfusepath{clip}%
\pgfsetbuttcap%
\pgfsetroundjoin%
\definecolor{currentfill}{rgb}{0.982699,0.823991,0.657439}%
\pgfsetfillcolor{currentfill}%
\pgfsetlinewidth{0.250937pt}%
\definecolor{currentstroke}{rgb}{1.000000,1.000000,1.000000}%
\pgfsetstrokecolor{currentstroke}%
\pgfsetdash{}{0pt}%
\pgfpathmoveto{\pgfqpoint{4.065849in}{2.347925in}}%
\pgfpathlineto{\pgfqpoint{4.153585in}{2.347925in}}%
\pgfpathlineto{\pgfqpoint{4.153585in}{2.260189in}}%
\pgfpathlineto{\pgfqpoint{4.065849in}{2.260189in}}%
\pgfpathlineto{\pgfqpoint{4.065849in}{2.347925in}}%
\pgfusepath{stroke,fill}%
\end{pgfscope}%
\begin{pgfscope}%
\pgfpathrectangle{\pgfqpoint{0.380943in}{2.260189in}}{\pgfqpoint{4.650000in}{0.614151in}}%
\pgfusepath{clip}%
\pgfsetbuttcap%
\pgfsetroundjoin%
\definecolor{currentfill}{rgb}{0.969504,0.885813,0.700930}%
\pgfsetfillcolor{currentfill}%
\pgfsetlinewidth{0.250937pt}%
\definecolor{currentstroke}{rgb}{1.000000,1.000000,1.000000}%
\pgfsetstrokecolor{currentstroke}%
\pgfsetdash{}{0pt}%
\pgfpathmoveto{\pgfqpoint{4.153585in}{2.347925in}}%
\pgfpathlineto{\pgfqpoint{4.241320in}{2.347925in}}%
\pgfpathlineto{\pgfqpoint{4.241320in}{2.260189in}}%
\pgfpathlineto{\pgfqpoint{4.153585in}{2.260189in}}%
\pgfpathlineto{\pgfqpoint{4.153585in}{2.347925in}}%
\pgfusepath{stroke,fill}%
\end{pgfscope}%
\begin{pgfscope}%
\pgfpathrectangle{\pgfqpoint{0.380943in}{2.260189in}}{\pgfqpoint{4.650000in}{0.614151in}}%
\pgfusepath{clip}%
\pgfsetbuttcap%
\pgfsetroundjoin%
\definecolor{currentfill}{rgb}{0.964937,0.908651,0.713110}%
\pgfsetfillcolor{currentfill}%
\pgfsetlinewidth{0.250937pt}%
\definecolor{currentstroke}{rgb}{1.000000,1.000000,1.000000}%
\pgfsetstrokecolor{currentstroke}%
\pgfsetdash{}{0pt}%
\pgfpathmoveto{\pgfqpoint{4.241320in}{2.347925in}}%
\pgfpathlineto{\pgfqpoint{4.329056in}{2.347925in}}%
\pgfpathlineto{\pgfqpoint{4.329056in}{2.260189in}}%
\pgfpathlineto{\pgfqpoint{4.241320in}{2.260189in}}%
\pgfpathlineto{\pgfqpoint{4.241320in}{2.347925in}}%
\pgfusepath{stroke,fill}%
\end{pgfscope}%
\begin{pgfscope}%
\pgfpathrectangle{\pgfqpoint{0.380943in}{2.260189in}}{\pgfqpoint{4.650000in}{0.614151in}}%
\pgfusepath{clip}%
\pgfsetbuttcap%
\pgfsetroundjoin%
\definecolor{currentfill}{rgb}{0.961738,0.927612,0.725598}%
\pgfsetfillcolor{currentfill}%
\pgfsetlinewidth{0.250937pt}%
\definecolor{currentstroke}{rgb}{1.000000,1.000000,1.000000}%
\pgfsetstrokecolor{currentstroke}%
\pgfsetdash{}{0pt}%
\pgfpathmoveto{\pgfqpoint{4.329056in}{2.347925in}}%
\pgfpathlineto{\pgfqpoint{4.416792in}{2.347925in}}%
\pgfpathlineto{\pgfqpoint{4.416792in}{2.260189in}}%
\pgfpathlineto{\pgfqpoint{4.329056in}{2.260189in}}%
\pgfpathlineto{\pgfqpoint{4.329056in}{2.347925in}}%
\pgfusepath{stroke,fill}%
\end{pgfscope}%
\begin{pgfscope}%
\pgfpathrectangle{\pgfqpoint{0.380943in}{2.260189in}}{\pgfqpoint{4.650000in}{0.614151in}}%
\pgfusepath{clip}%
\pgfsetbuttcap%
\pgfsetroundjoin%
\definecolor{currentfill}{rgb}{0.961738,0.927612,0.725598}%
\pgfsetfillcolor{currentfill}%
\pgfsetlinewidth{0.250937pt}%
\definecolor{currentstroke}{rgb}{1.000000,1.000000,1.000000}%
\pgfsetstrokecolor{currentstroke}%
\pgfsetdash{}{0pt}%
\pgfpathmoveto{\pgfqpoint{4.416792in}{2.347925in}}%
\pgfpathlineto{\pgfqpoint{4.504528in}{2.347925in}}%
\pgfpathlineto{\pgfqpoint{4.504528in}{2.260189in}}%
\pgfpathlineto{\pgfqpoint{4.416792in}{2.260189in}}%
\pgfpathlineto{\pgfqpoint{4.416792in}{2.347925in}}%
\pgfusepath{stroke,fill}%
\end{pgfscope}%
\begin{pgfscope}%
\pgfpathrectangle{\pgfqpoint{0.380943in}{2.260189in}}{\pgfqpoint{4.650000in}{0.614151in}}%
\pgfusepath{clip}%
\pgfsetbuttcap%
\pgfsetroundjoin%
\definecolor{currentfill}{rgb}{0.969504,0.885813,0.700930}%
\pgfsetfillcolor{currentfill}%
\pgfsetlinewidth{0.250937pt}%
\definecolor{currentstroke}{rgb}{1.000000,1.000000,1.000000}%
\pgfsetstrokecolor{currentstroke}%
\pgfsetdash{}{0pt}%
\pgfpathmoveto{\pgfqpoint{4.504528in}{2.347925in}}%
\pgfpathlineto{\pgfqpoint{4.592264in}{2.347925in}}%
\pgfpathlineto{\pgfqpoint{4.592264in}{2.260189in}}%
\pgfpathlineto{\pgfqpoint{4.504528in}{2.260189in}}%
\pgfpathlineto{\pgfqpoint{4.504528in}{2.347925in}}%
\pgfusepath{stroke,fill}%
\end{pgfscope}%
\begin{pgfscope}%
\pgfpathrectangle{\pgfqpoint{0.380943in}{2.260189in}}{\pgfqpoint{4.650000in}{0.614151in}}%
\pgfusepath{clip}%
\pgfsetbuttcap%
\pgfsetroundjoin%
\definecolor{currentfill}{rgb}{1.000000,1.000000,0.857516}%
\pgfsetfillcolor{currentfill}%
\pgfsetlinewidth{0.250937pt}%
\definecolor{currentstroke}{rgb}{1.000000,1.000000,1.000000}%
\pgfsetstrokecolor{currentstroke}%
\pgfsetdash{}{0pt}%
\pgfpathmoveto{\pgfqpoint{4.592264in}{2.347925in}}%
\pgfpathlineto{\pgfqpoint{4.680000in}{2.347925in}}%
\pgfpathlineto{\pgfqpoint{4.680000in}{2.260189in}}%
\pgfpathlineto{\pgfqpoint{4.592264in}{2.260189in}}%
\pgfpathlineto{\pgfqpoint{4.592264in}{2.347925in}}%
\pgfusepath{stroke,fill}%
\end{pgfscope}%
\begin{pgfscope}%
\pgfpathrectangle{\pgfqpoint{0.380943in}{2.260189in}}{\pgfqpoint{4.650000in}{0.614151in}}%
\pgfusepath{clip}%
\pgfsetbuttcap%
\pgfsetroundjoin%
\definecolor{currentfill}{rgb}{0.964783,0.940131,0.739808}%
\pgfsetfillcolor{currentfill}%
\pgfsetlinewidth{0.250937pt}%
\definecolor{currentstroke}{rgb}{1.000000,1.000000,1.000000}%
\pgfsetstrokecolor{currentstroke}%
\pgfsetdash{}{0pt}%
\pgfpathmoveto{\pgfqpoint{4.680000in}{2.347925in}}%
\pgfpathlineto{\pgfqpoint{4.767736in}{2.347925in}}%
\pgfpathlineto{\pgfqpoint{4.767736in}{2.260189in}}%
\pgfpathlineto{\pgfqpoint{4.680000in}{2.260189in}}%
\pgfpathlineto{\pgfqpoint{4.680000in}{2.347925in}}%
\pgfusepath{stroke,fill}%
\end{pgfscope}%
\begin{pgfscope}%
\pgfpathrectangle{\pgfqpoint{0.380943in}{2.260189in}}{\pgfqpoint{4.650000in}{0.614151in}}%
\pgfusepath{clip}%
\pgfsetbuttcap%
\pgfsetroundjoin%
\definecolor{currentfill}{rgb}{0.969504,0.885813,0.700930}%
\pgfsetfillcolor{currentfill}%
\pgfsetlinewidth{0.250937pt}%
\definecolor{currentstroke}{rgb}{1.000000,1.000000,1.000000}%
\pgfsetstrokecolor{currentstroke}%
\pgfsetdash{}{0pt}%
\pgfpathmoveto{\pgfqpoint{4.767736in}{2.347925in}}%
\pgfpathlineto{\pgfqpoint{4.855471in}{2.347925in}}%
\pgfpathlineto{\pgfqpoint{4.855471in}{2.260189in}}%
\pgfpathlineto{\pgfqpoint{4.767736in}{2.260189in}}%
\pgfpathlineto{\pgfqpoint{4.767736in}{2.347925in}}%
\pgfusepath{stroke,fill}%
\end{pgfscope}%
\begin{pgfscope}%
\pgfpathrectangle{\pgfqpoint{0.380943in}{2.260189in}}{\pgfqpoint{4.650000in}{0.614151in}}%
\pgfusepath{clip}%
\pgfsetbuttcap%
\pgfsetroundjoin%
\definecolor{currentfill}{rgb}{0.980008,0.966013,0.779393}%
\pgfsetfillcolor{currentfill}%
\pgfsetlinewidth{0.250937pt}%
\definecolor{currentstroke}{rgb}{1.000000,1.000000,1.000000}%
\pgfsetstrokecolor{currentstroke}%
\pgfsetdash{}{0pt}%
\pgfpathmoveto{\pgfqpoint{4.855471in}{2.347925in}}%
\pgfpathlineto{\pgfqpoint{4.943207in}{2.347925in}}%
\pgfpathlineto{\pgfqpoint{4.943207in}{2.260189in}}%
\pgfpathlineto{\pgfqpoint{4.855471in}{2.260189in}}%
\pgfpathlineto{\pgfqpoint{4.855471in}{2.347925in}}%
\pgfusepath{stroke,fill}%
\end{pgfscope}%
\begin{pgfscope}%
\pgfpathrectangle{\pgfqpoint{0.380943in}{2.260189in}}{\pgfqpoint{4.650000in}{0.614151in}}%
\pgfusepath{clip}%
\pgfsetbuttcap%
\pgfsetroundjoin%
\pgfsetlinewidth{0.250937pt}%
\definecolor{currentstroke}{rgb}{1.000000,1.000000,1.000000}%
\pgfsetstrokecolor{currentstroke}%
\pgfsetdash{}{0pt}%
\pgfpathmoveto{\pgfqpoint{4.943207in}{2.347925in}}%
\pgfpathlineto{\pgfqpoint{5.030943in}{2.347925in}}%
\pgfpathlineto{\pgfqpoint{5.030943in}{2.260189in}}%
\pgfpathlineto{\pgfqpoint{4.943207in}{2.260189in}}%
\pgfpathlineto{\pgfqpoint{4.943207in}{2.347925in}}%
\pgfusepath{stroke}%
\end{pgfscope}%
\begin{pgfscope}%
\pgfsetbuttcap%
\pgfsetroundjoin%
\definecolor{currentfill}{rgb}{0.000000,0.000000,0.000000}%
\pgfsetfillcolor{currentfill}%
\pgfsetlinewidth{0.803000pt}%
\definecolor{currentstroke}{rgb}{0.000000,0.000000,0.000000}%
\pgfsetstrokecolor{currentstroke}%
\pgfsetdash{}{0pt}%
\pgfsys@defobject{currentmarker}{\pgfqpoint{0.000000in}{-0.048611in}}{\pgfqpoint{0.000000in}{0.000000in}}{%
\pgfpathmoveto{\pgfqpoint{0.000000in}{0.000000in}}%
\pgfpathlineto{\pgfqpoint{0.000000in}{-0.048611in}}%
\pgfusepath{stroke,fill}%
}%
\begin{pgfscope}%
\pgfsys@transformshift{0.600283in}{2.260189in}%
\pgfsys@useobject{currentmarker}{}%
\end{pgfscope}%
\end{pgfscope}%
\begin{pgfscope}%
\definecolor{textcolor}{rgb}{0.000000,0.000000,0.000000}%
\pgfsetstrokecolor{textcolor}%
\pgfsetfillcolor{textcolor}%
\pgftext[x=0.600283in,y=2.162967in,,top]{\color{textcolor}\rmfamily\fontsize{8.000000}{9.600000}\selectfont Jan}%
\end{pgfscope}%
\begin{pgfscope}%
\pgfsetbuttcap%
\pgfsetroundjoin%
\definecolor{currentfill}{rgb}{0.000000,0.000000,0.000000}%
\pgfsetfillcolor{currentfill}%
\pgfsetlinewidth{0.803000pt}%
\definecolor{currentstroke}{rgb}{0.000000,0.000000,0.000000}%
\pgfsetstrokecolor{currentstroke}%
\pgfsetdash{}{0pt}%
\pgfsys@defobject{currentmarker}{\pgfqpoint{0.000000in}{-0.048611in}}{\pgfqpoint{0.000000in}{0.000000in}}{%
\pgfpathmoveto{\pgfqpoint{0.000000in}{0.000000in}}%
\pgfpathlineto{\pgfqpoint{0.000000in}{-0.048611in}}%
\pgfusepath{stroke,fill}%
}%
\begin{pgfscope}%
\pgfsys@transformshift{0.951226in}{2.260189in}%
\pgfsys@useobject{currentmarker}{}%
\end{pgfscope}%
\end{pgfscope}%
\begin{pgfscope}%
\definecolor{textcolor}{rgb}{0.000000,0.000000,0.000000}%
\pgfsetstrokecolor{textcolor}%
\pgfsetfillcolor{textcolor}%
\pgftext[x=0.951226in,y=2.162967in,,top]{\color{textcolor}\rmfamily\fontsize{8.000000}{9.600000}\selectfont Feb}%
\end{pgfscope}%
\begin{pgfscope}%
\pgfsetbuttcap%
\pgfsetroundjoin%
\definecolor{currentfill}{rgb}{0.000000,0.000000,0.000000}%
\pgfsetfillcolor{currentfill}%
\pgfsetlinewidth{0.803000pt}%
\definecolor{currentstroke}{rgb}{0.000000,0.000000,0.000000}%
\pgfsetstrokecolor{currentstroke}%
\pgfsetdash{}{0pt}%
\pgfsys@defobject{currentmarker}{\pgfqpoint{0.000000in}{-0.048611in}}{\pgfqpoint{0.000000in}{0.000000in}}{%
\pgfpathmoveto{\pgfqpoint{0.000000in}{0.000000in}}%
\pgfpathlineto{\pgfqpoint{0.000000in}{-0.048611in}}%
\pgfusepath{stroke,fill}%
}%
\begin{pgfscope}%
\pgfsys@transformshift{1.346037in}{2.260189in}%
\pgfsys@useobject{currentmarker}{}%
\end{pgfscope}%
\end{pgfscope}%
\begin{pgfscope}%
\definecolor{textcolor}{rgb}{0.000000,0.000000,0.000000}%
\pgfsetstrokecolor{textcolor}%
\pgfsetfillcolor{textcolor}%
\pgftext[x=1.346037in,y=2.162967in,,top]{\color{textcolor}\rmfamily\fontsize{8.000000}{9.600000}\selectfont Mar}%
\end{pgfscope}%
\begin{pgfscope}%
\pgfsetbuttcap%
\pgfsetroundjoin%
\definecolor{currentfill}{rgb}{0.000000,0.000000,0.000000}%
\pgfsetfillcolor{currentfill}%
\pgfsetlinewidth{0.803000pt}%
\definecolor{currentstroke}{rgb}{0.000000,0.000000,0.000000}%
\pgfsetstrokecolor{currentstroke}%
\pgfsetdash{}{0pt}%
\pgfsys@defobject{currentmarker}{\pgfqpoint{0.000000in}{-0.048611in}}{\pgfqpoint{0.000000in}{0.000000in}}{%
\pgfpathmoveto{\pgfqpoint{0.000000in}{0.000000in}}%
\pgfpathlineto{\pgfqpoint{0.000000in}{-0.048611in}}%
\pgfusepath{stroke,fill}%
}%
\begin{pgfscope}%
\pgfsys@transformshift{1.740849in}{2.260189in}%
\pgfsys@useobject{currentmarker}{}%
\end{pgfscope}%
\end{pgfscope}%
\begin{pgfscope}%
\definecolor{textcolor}{rgb}{0.000000,0.000000,0.000000}%
\pgfsetstrokecolor{textcolor}%
\pgfsetfillcolor{textcolor}%
\pgftext[x=1.740849in,y=2.162967in,,top]{\color{textcolor}\rmfamily\fontsize{8.000000}{9.600000}\selectfont Apr}%
\end{pgfscope}%
\begin{pgfscope}%
\pgfsetbuttcap%
\pgfsetroundjoin%
\definecolor{currentfill}{rgb}{0.000000,0.000000,0.000000}%
\pgfsetfillcolor{currentfill}%
\pgfsetlinewidth{0.803000pt}%
\definecolor{currentstroke}{rgb}{0.000000,0.000000,0.000000}%
\pgfsetstrokecolor{currentstroke}%
\pgfsetdash{}{0pt}%
\pgfsys@defobject{currentmarker}{\pgfqpoint{0.000000in}{-0.048611in}}{\pgfqpoint{0.000000in}{0.000000in}}{%
\pgfpathmoveto{\pgfqpoint{0.000000in}{0.000000in}}%
\pgfpathlineto{\pgfqpoint{0.000000in}{-0.048611in}}%
\pgfusepath{stroke,fill}%
}%
\begin{pgfscope}%
\pgfsys@transformshift{2.091792in}{2.260189in}%
\pgfsys@useobject{currentmarker}{}%
\end{pgfscope}%
\end{pgfscope}%
\begin{pgfscope}%
\definecolor{textcolor}{rgb}{0.000000,0.000000,0.000000}%
\pgfsetstrokecolor{textcolor}%
\pgfsetfillcolor{textcolor}%
\pgftext[x=2.091792in,y=2.162967in,,top]{\color{textcolor}\rmfamily\fontsize{8.000000}{9.600000}\selectfont May}%
\end{pgfscope}%
\begin{pgfscope}%
\pgfsetbuttcap%
\pgfsetroundjoin%
\definecolor{currentfill}{rgb}{0.000000,0.000000,0.000000}%
\pgfsetfillcolor{currentfill}%
\pgfsetlinewidth{0.803000pt}%
\definecolor{currentstroke}{rgb}{0.000000,0.000000,0.000000}%
\pgfsetstrokecolor{currentstroke}%
\pgfsetdash{}{0pt}%
\pgfsys@defobject{currentmarker}{\pgfqpoint{0.000000in}{-0.048611in}}{\pgfqpoint{0.000000in}{0.000000in}}{%
\pgfpathmoveto{\pgfqpoint{0.000000in}{0.000000in}}%
\pgfpathlineto{\pgfqpoint{0.000000in}{-0.048611in}}%
\pgfusepath{stroke,fill}%
}%
\begin{pgfscope}%
\pgfsys@transformshift{2.530471in}{2.260189in}%
\pgfsys@useobject{currentmarker}{}%
\end{pgfscope}%
\end{pgfscope}%
\begin{pgfscope}%
\definecolor{textcolor}{rgb}{0.000000,0.000000,0.000000}%
\pgfsetstrokecolor{textcolor}%
\pgfsetfillcolor{textcolor}%
\pgftext[x=2.530471in,y=2.162967in,,top]{\color{textcolor}\rmfamily\fontsize{8.000000}{9.600000}\selectfont Jun}%
\end{pgfscope}%
\begin{pgfscope}%
\pgfsetbuttcap%
\pgfsetroundjoin%
\definecolor{currentfill}{rgb}{0.000000,0.000000,0.000000}%
\pgfsetfillcolor{currentfill}%
\pgfsetlinewidth{0.803000pt}%
\definecolor{currentstroke}{rgb}{0.000000,0.000000,0.000000}%
\pgfsetstrokecolor{currentstroke}%
\pgfsetdash{}{0pt}%
\pgfsys@defobject{currentmarker}{\pgfqpoint{0.000000in}{-0.048611in}}{\pgfqpoint{0.000000in}{0.000000in}}{%
\pgfpathmoveto{\pgfqpoint{0.000000in}{0.000000in}}%
\pgfpathlineto{\pgfqpoint{0.000000in}{-0.048611in}}%
\pgfusepath{stroke,fill}%
}%
\begin{pgfscope}%
\pgfsys@transformshift{2.881415in}{2.260189in}%
\pgfsys@useobject{currentmarker}{}%
\end{pgfscope}%
\end{pgfscope}%
\begin{pgfscope}%
\definecolor{textcolor}{rgb}{0.000000,0.000000,0.000000}%
\pgfsetstrokecolor{textcolor}%
\pgfsetfillcolor{textcolor}%
\pgftext[x=2.881415in,y=2.162967in,,top]{\color{textcolor}\rmfamily\fontsize{8.000000}{9.600000}\selectfont Jul}%
\end{pgfscope}%
\begin{pgfscope}%
\pgfsetbuttcap%
\pgfsetroundjoin%
\definecolor{currentfill}{rgb}{0.000000,0.000000,0.000000}%
\pgfsetfillcolor{currentfill}%
\pgfsetlinewidth{0.803000pt}%
\definecolor{currentstroke}{rgb}{0.000000,0.000000,0.000000}%
\pgfsetstrokecolor{currentstroke}%
\pgfsetdash{}{0pt}%
\pgfsys@defobject{currentmarker}{\pgfqpoint{0.000000in}{-0.048611in}}{\pgfqpoint{0.000000in}{0.000000in}}{%
\pgfpathmoveto{\pgfqpoint{0.000000in}{0.000000in}}%
\pgfpathlineto{\pgfqpoint{0.000000in}{-0.048611in}}%
\pgfusepath{stroke,fill}%
}%
\begin{pgfscope}%
\pgfsys@transformshift{3.276226in}{2.260189in}%
\pgfsys@useobject{currentmarker}{}%
\end{pgfscope}%
\end{pgfscope}%
\begin{pgfscope}%
\definecolor{textcolor}{rgb}{0.000000,0.000000,0.000000}%
\pgfsetstrokecolor{textcolor}%
\pgfsetfillcolor{textcolor}%
\pgftext[x=3.276226in,y=2.162967in,,top]{\color{textcolor}\rmfamily\fontsize{8.000000}{9.600000}\selectfont Aug}%
\end{pgfscope}%
\begin{pgfscope}%
\pgfsetbuttcap%
\pgfsetroundjoin%
\definecolor{currentfill}{rgb}{0.000000,0.000000,0.000000}%
\pgfsetfillcolor{currentfill}%
\pgfsetlinewidth{0.803000pt}%
\definecolor{currentstroke}{rgb}{0.000000,0.000000,0.000000}%
\pgfsetstrokecolor{currentstroke}%
\pgfsetdash{}{0pt}%
\pgfsys@defobject{currentmarker}{\pgfqpoint{0.000000in}{-0.048611in}}{\pgfqpoint{0.000000in}{0.000000in}}{%
\pgfpathmoveto{\pgfqpoint{0.000000in}{0.000000in}}%
\pgfpathlineto{\pgfqpoint{0.000000in}{-0.048611in}}%
\pgfusepath{stroke,fill}%
}%
\begin{pgfscope}%
\pgfsys@transformshift{3.671037in}{2.260189in}%
\pgfsys@useobject{currentmarker}{}%
\end{pgfscope}%
\end{pgfscope}%
\begin{pgfscope}%
\definecolor{textcolor}{rgb}{0.000000,0.000000,0.000000}%
\pgfsetstrokecolor{textcolor}%
\pgfsetfillcolor{textcolor}%
\pgftext[x=3.671037in,y=2.162967in,,top]{\color{textcolor}\rmfamily\fontsize{8.000000}{9.600000}\selectfont Sep}%
\end{pgfscope}%
\begin{pgfscope}%
\pgfsetbuttcap%
\pgfsetroundjoin%
\definecolor{currentfill}{rgb}{0.000000,0.000000,0.000000}%
\pgfsetfillcolor{currentfill}%
\pgfsetlinewidth{0.803000pt}%
\definecolor{currentstroke}{rgb}{0.000000,0.000000,0.000000}%
\pgfsetstrokecolor{currentstroke}%
\pgfsetdash{}{0pt}%
\pgfsys@defobject{currentmarker}{\pgfqpoint{0.000000in}{-0.048611in}}{\pgfqpoint{0.000000in}{0.000000in}}{%
\pgfpathmoveto{\pgfqpoint{0.000000in}{0.000000in}}%
\pgfpathlineto{\pgfqpoint{0.000000in}{-0.048611in}}%
\pgfusepath{stroke,fill}%
}%
\begin{pgfscope}%
\pgfsys@transformshift{4.021981in}{2.260189in}%
\pgfsys@useobject{currentmarker}{}%
\end{pgfscope}%
\end{pgfscope}%
\begin{pgfscope}%
\definecolor{textcolor}{rgb}{0.000000,0.000000,0.000000}%
\pgfsetstrokecolor{textcolor}%
\pgfsetfillcolor{textcolor}%
\pgftext[x=4.021981in,y=2.162967in,,top]{\color{textcolor}\rmfamily\fontsize{8.000000}{9.600000}\selectfont Oct}%
\end{pgfscope}%
\begin{pgfscope}%
\pgfsetbuttcap%
\pgfsetroundjoin%
\definecolor{currentfill}{rgb}{0.000000,0.000000,0.000000}%
\pgfsetfillcolor{currentfill}%
\pgfsetlinewidth{0.803000pt}%
\definecolor{currentstroke}{rgb}{0.000000,0.000000,0.000000}%
\pgfsetstrokecolor{currentstroke}%
\pgfsetdash{}{0pt}%
\pgfsys@defobject{currentmarker}{\pgfqpoint{0.000000in}{-0.048611in}}{\pgfqpoint{0.000000in}{0.000000in}}{%
\pgfpathmoveto{\pgfqpoint{0.000000in}{0.000000in}}%
\pgfpathlineto{\pgfqpoint{0.000000in}{-0.048611in}}%
\pgfusepath{stroke,fill}%
}%
\begin{pgfscope}%
\pgfsys@transformshift{4.416792in}{2.260189in}%
\pgfsys@useobject{currentmarker}{}%
\end{pgfscope}%
\end{pgfscope}%
\begin{pgfscope}%
\definecolor{textcolor}{rgb}{0.000000,0.000000,0.000000}%
\pgfsetstrokecolor{textcolor}%
\pgfsetfillcolor{textcolor}%
\pgftext[x=4.416792in,y=2.162967in,,top]{\color{textcolor}\rmfamily\fontsize{8.000000}{9.600000}\selectfont Nov}%
\end{pgfscope}%
\begin{pgfscope}%
\pgfsetbuttcap%
\pgfsetroundjoin%
\definecolor{currentfill}{rgb}{0.000000,0.000000,0.000000}%
\pgfsetfillcolor{currentfill}%
\pgfsetlinewidth{0.803000pt}%
\definecolor{currentstroke}{rgb}{0.000000,0.000000,0.000000}%
\pgfsetstrokecolor{currentstroke}%
\pgfsetdash{}{0pt}%
\pgfsys@defobject{currentmarker}{\pgfqpoint{0.000000in}{-0.048611in}}{\pgfqpoint{0.000000in}{0.000000in}}{%
\pgfpathmoveto{\pgfqpoint{0.000000in}{0.000000in}}%
\pgfpathlineto{\pgfqpoint{0.000000in}{-0.048611in}}%
\pgfusepath{stroke,fill}%
}%
\begin{pgfscope}%
\pgfsys@transformshift{4.811603in}{2.260189in}%
\pgfsys@useobject{currentmarker}{}%
\end{pgfscope}%
\end{pgfscope}%
\begin{pgfscope}%
\definecolor{textcolor}{rgb}{0.000000,0.000000,0.000000}%
\pgfsetstrokecolor{textcolor}%
\pgfsetfillcolor{textcolor}%
\pgftext[x=4.811603in,y=2.162967in,,top]{\color{textcolor}\rmfamily\fontsize{8.000000}{9.600000}\selectfont Dec}%
\end{pgfscope}%
\begin{pgfscope}%
\pgfsetbuttcap%
\pgfsetroundjoin%
\definecolor{currentfill}{rgb}{0.000000,0.000000,0.000000}%
\pgfsetfillcolor{currentfill}%
\pgfsetlinewidth{0.803000pt}%
\definecolor{currentstroke}{rgb}{0.000000,0.000000,0.000000}%
\pgfsetstrokecolor{currentstroke}%
\pgfsetdash{}{0pt}%
\pgfsys@defobject{currentmarker}{\pgfqpoint{-0.048611in}{0.000000in}}{\pgfqpoint{-0.000000in}{0.000000in}}{%
\pgfpathmoveto{\pgfqpoint{-0.000000in}{0.000000in}}%
\pgfpathlineto{\pgfqpoint{-0.048611in}{0.000000in}}%
\pgfusepath{stroke,fill}%
}%
\begin{pgfscope}%
\pgfsys@transformshift{0.380943in}{2.830472in}%
\pgfsys@useobject{currentmarker}{}%
\end{pgfscope}%
\end{pgfscope}%
\begin{pgfscope}%
\definecolor{textcolor}{rgb}{0.000000,0.000000,0.000000}%
\pgfsetstrokecolor{textcolor}%
\pgfsetfillcolor{textcolor}%
\pgftext[x=0.113117in, y=2.791892in, left, base]{\color{textcolor}\rmfamily\fontsize{8.000000}{9.600000}\selectfont M}%
\end{pgfscope}%
\begin{pgfscope}%
\pgfsetbuttcap%
\pgfsetroundjoin%
\definecolor{currentfill}{rgb}{0.000000,0.000000,0.000000}%
\pgfsetfillcolor{currentfill}%
\pgfsetlinewidth{0.803000pt}%
\definecolor{currentstroke}{rgb}{0.000000,0.000000,0.000000}%
\pgfsetstrokecolor{currentstroke}%
\pgfsetdash{}{0pt}%
\pgfsys@defobject{currentmarker}{\pgfqpoint{-0.048611in}{0.000000in}}{\pgfqpoint{-0.000000in}{0.000000in}}{%
\pgfpathmoveto{\pgfqpoint{-0.000000in}{0.000000in}}%
\pgfpathlineto{\pgfqpoint{-0.048611in}{0.000000in}}%
\pgfusepath{stroke,fill}%
}%
\begin{pgfscope}%
\pgfsys@transformshift{0.380943in}{2.742736in}%
\pgfsys@useobject{currentmarker}{}%
\end{pgfscope}%
\end{pgfscope}%
\begin{pgfscope}%
\definecolor{textcolor}{rgb}{0.000000,0.000000,0.000000}%
\pgfsetstrokecolor{textcolor}%
\pgfsetfillcolor{textcolor}%
\pgftext[x=0.135957in, y=2.704156in, left, base]{\color{textcolor}\rmfamily\fontsize{8.000000}{9.600000}\selectfont T}%
\end{pgfscope}%
\begin{pgfscope}%
\pgfsetbuttcap%
\pgfsetroundjoin%
\definecolor{currentfill}{rgb}{0.000000,0.000000,0.000000}%
\pgfsetfillcolor{currentfill}%
\pgfsetlinewidth{0.803000pt}%
\definecolor{currentstroke}{rgb}{0.000000,0.000000,0.000000}%
\pgfsetstrokecolor{currentstroke}%
\pgfsetdash{}{0pt}%
\pgfsys@defobject{currentmarker}{\pgfqpoint{-0.048611in}{0.000000in}}{\pgfqpoint{-0.000000in}{0.000000in}}{%
\pgfpathmoveto{\pgfqpoint{-0.000000in}{0.000000in}}%
\pgfpathlineto{\pgfqpoint{-0.048611in}{0.000000in}}%
\pgfusepath{stroke,fill}%
}%
\begin{pgfscope}%
\pgfsys@transformshift{0.380943in}{2.655000in}%
\pgfsys@useobject{currentmarker}{}%
\end{pgfscope}%
\end{pgfscope}%
\begin{pgfscope}%
\definecolor{textcolor}{rgb}{0.000000,0.000000,0.000000}%
\pgfsetstrokecolor{textcolor}%
\pgfsetfillcolor{textcolor}%
\pgftext[x=0.100000in, y=2.616420in, left, base]{\color{textcolor}\rmfamily\fontsize{8.000000}{9.600000}\selectfont W}%
\end{pgfscope}%
\begin{pgfscope}%
\pgfsetbuttcap%
\pgfsetroundjoin%
\definecolor{currentfill}{rgb}{0.000000,0.000000,0.000000}%
\pgfsetfillcolor{currentfill}%
\pgfsetlinewidth{0.803000pt}%
\definecolor{currentstroke}{rgb}{0.000000,0.000000,0.000000}%
\pgfsetstrokecolor{currentstroke}%
\pgfsetdash{}{0pt}%
\pgfsys@defobject{currentmarker}{\pgfqpoint{-0.048611in}{0.000000in}}{\pgfqpoint{-0.000000in}{0.000000in}}{%
\pgfpathmoveto{\pgfqpoint{-0.000000in}{0.000000in}}%
\pgfpathlineto{\pgfqpoint{-0.048611in}{0.000000in}}%
\pgfusepath{stroke,fill}%
}%
\begin{pgfscope}%
\pgfsys@transformshift{0.380943in}{2.567264in}%
\pgfsys@useobject{currentmarker}{}%
\end{pgfscope}%
\end{pgfscope}%
\begin{pgfscope}%
\definecolor{textcolor}{rgb}{0.000000,0.000000,0.000000}%
\pgfsetstrokecolor{textcolor}%
\pgfsetfillcolor{textcolor}%
\pgftext[x=0.135957in, y=2.528684in, left, base]{\color{textcolor}\rmfamily\fontsize{8.000000}{9.600000}\selectfont T}%
\end{pgfscope}%
\begin{pgfscope}%
\pgfsetbuttcap%
\pgfsetroundjoin%
\definecolor{currentfill}{rgb}{0.000000,0.000000,0.000000}%
\pgfsetfillcolor{currentfill}%
\pgfsetlinewidth{0.803000pt}%
\definecolor{currentstroke}{rgb}{0.000000,0.000000,0.000000}%
\pgfsetstrokecolor{currentstroke}%
\pgfsetdash{}{0pt}%
\pgfsys@defobject{currentmarker}{\pgfqpoint{-0.048611in}{0.000000in}}{\pgfqpoint{-0.000000in}{0.000000in}}{%
\pgfpathmoveto{\pgfqpoint{-0.000000in}{0.000000in}}%
\pgfpathlineto{\pgfqpoint{-0.048611in}{0.000000in}}%
\pgfusepath{stroke,fill}%
}%
\begin{pgfscope}%
\pgfsys@transformshift{0.380943in}{2.479529in}%
\pgfsys@useobject{currentmarker}{}%
\end{pgfscope}%
\end{pgfscope}%
\begin{pgfscope}%
\definecolor{textcolor}{rgb}{0.000000,0.000000,0.000000}%
\pgfsetstrokecolor{textcolor}%
\pgfsetfillcolor{textcolor}%
\pgftext[x=0.144213in, y=2.440948in, left, base]{\color{textcolor}\rmfamily\fontsize{8.000000}{9.600000}\selectfont F}%
\end{pgfscope}%
\begin{pgfscope}%
\pgfsetbuttcap%
\pgfsetroundjoin%
\definecolor{currentfill}{rgb}{0.000000,0.000000,0.000000}%
\pgfsetfillcolor{currentfill}%
\pgfsetlinewidth{0.803000pt}%
\definecolor{currentstroke}{rgb}{0.000000,0.000000,0.000000}%
\pgfsetstrokecolor{currentstroke}%
\pgfsetdash{}{0pt}%
\pgfsys@defobject{currentmarker}{\pgfqpoint{-0.048611in}{0.000000in}}{\pgfqpoint{-0.000000in}{0.000000in}}{%
\pgfpathmoveto{\pgfqpoint{-0.000000in}{0.000000in}}%
\pgfpathlineto{\pgfqpoint{-0.048611in}{0.000000in}}%
\pgfusepath{stroke,fill}%
}%
\begin{pgfscope}%
\pgfsys@transformshift{0.380943in}{2.391793in}%
\pgfsys@useobject{currentmarker}{}%
\end{pgfscope}%
\end{pgfscope}%
\begin{pgfscope}%
\definecolor{textcolor}{rgb}{0.000000,0.000000,0.000000}%
\pgfsetstrokecolor{textcolor}%
\pgfsetfillcolor{textcolor}%
\pgftext[x=0.155633in, y=2.353212in, left, base]{\color{textcolor}\rmfamily\fontsize{8.000000}{9.600000}\selectfont S}%
\end{pgfscope}%
\begin{pgfscope}%
\pgfsetbuttcap%
\pgfsetroundjoin%
\definecolor{currentfill}{rgb}{0.000000,0.000000,0.000000}%
\pgfsetfillcolor{currentfill}%
\pgfsetlinewidth{0.803000pt}%
\definecolor{currentstroke}{rgb}{0.000000,0.000000,0.000000}%
\pgfsetstrokecolor{currentstroke}%
\pgfsetdash{}{0pt}%
\pgfsys@defobject{currentmarker}{\pgfqpoint{-0.048611in}{0.000000in}}{\pgfqpoint{-0.000000in}{0.000000in}}{%
\pgfpathmoveto{\pgfqpoint{-0.000000in}{0.000000in}}%
\pgfpathlineto{\pgfqpoint{-0.048611in}{0.000000in}}%
\pgfusepath{stroke,fill}%
}%
\begin{pgfscope}%
\pgfsys@transformshift{0.380943in}{2.304057in}%
\pgfsys@useobject{currentmarker}{}%
\end{pgfscope}%
\end{pgfscope}%
\begin{pgfscope}%
\definecolor{textcolor}{rgb}{0.000000,0.000000,0.000000}%
\pgfsetstrokecolor{textcolor}%
\pgfsetfillcolor{textcolor}%
\pgftext[x=0.155633in, y=2.265477in, left, base]{\color{textcolor}\rmfamily\fontsize{8.000000}{9.600000}\selectfont S}%
\end{pgfscope}%
\begin{pgfscope}%
\definecolor{textcolor}{rgb}{0.000000,0.000000,0.000000}%
\pgfsetstrokecolor{textcolor}%
\pgfsetfillcolor{textcolor}%
\pgftext[x=2.705943in,y=3.041007in,,]{\color{textcolor}\ttfamily\fontsize{14.400000}{17.280000}\selectfont 2020}%
\end{pgfscope}%
\begin{pgfscope}%
\pgfpathrectangle{\pgfqpoint{0.380943in}{0.295988in}}{\pgfqpoint{4.650000in}{0.692553in}}%
\pgfusepath{clip}%
\pgfsetbuttcap%
\pgfsetroundjoin%
\pgfsetlinewidth{0.250937pt}%
\definecolor{currentstroke}{rgb}{1.000000,1.000000,1.000000}%
\pgfsetstrokecolor{currentstroke}%
\pgfsetdash{}{0pt}%
\pgfpathmoveto{\pgfqpoint{0.380943in}{0.988541in}}%
\pgfpathlineto{\pgfqpoint{0.479879in}{0.988541in}}%
\pgfpathlineto{\pgfqpoint{0.479879in}{0.889605in}}%
\pgfpathlineto{\pgfqpoint{0.380943in}{0.889605in}}%
\pgfpathlineto{\pgfqpoint{0.380943in}{0.988541in}}%
\pgfusepath{stroke}%
\end{pgfscope}%
\begin{pgfscope}%
\pgfpathrectangle{\pgfqpoint{0.380943in}{0.295988in}}{\pgfqpoint{4.650000in}{0.692553in}}%
\pgfusepath{clip}%
\pgfsetbuttcap%
\pgfsetroundjoin%
\definecolor{currentfill}{rgb}{0.991849,0.986144,0.810181}%
\pgfsetfillcolor{currentfill}%
\pgfsetlinewidth{0.250937pt}%
\definecolor{currentstroke}{rgb}{1.000000,1.000000,1.000000}%
\pgfsetstrokecolor{currentstroke}%
\pgfsetdash{}{0pt}%
\pgfpathmoveto{\pgfqpoint{0.479879in}{0.988541in}}%
\pgfpathlineto{\pgfqpoint{0.578815in}{0.988541in}}%
\pgfpathlineto{\pgfqpoint{0.578815in}{0.889605in}}%
\pgfpathlineto{\pgfqpoint{0.479879in}{0.889605in}}%
\pgfpathlineto{\pgfqpoint{0.479879in}{0.988541in}}%
\pgfusepath{stroke,fill}%
\end{pgfscope}%
\begin{pgfscope}%
\pgfpathrectangle{\pgfqpoint{0.380943in}{0.295988in}}{\pgfqpoint{4.650000in}{0.692553in}}%
\pgfusepath{clip}%
\pgfsetbuttcap%
\pgfsetroundjoin%
\definecolor{currentfill}{rgb}{1.000000,1.000000,0.844829}%
\pgfsetfillcolor{currentfill}%
\pgfsetlinewidth{0.250937pt}%
\definecolor{currentstroke}{rgb}{1.000000,1.000000,1.000000}%
\pgfsetstrokecolor{currentstroke}%
\pgfsetdash{}{0pt}%
\pgfpathmoveto{\pgfqpoint{0.578815in}{0.988541in}}%
\pgfpathlineto{\pgfqpoint{0.677752in}{0.988541in}}%
\pgfpathlineto{\pgfqpoint{0.677752in}{0.889605in}}%
\pgfpathlineto{\pgfqpoint{0.578815in}{0.889605in}}%
\pgfpathlineto{\pgfqpoint{0.578815in}{0.988541in}}%
\pgfusepath{stroke,fill}%
\end{pgfscope}%
\begin{pgfscope}%
\pgfpathrectangle{\pgfqpoint{0.380943in}{0.295988in}}{\pgfqpoint{4.650000in}{0.692553in}}%
\pgfusepath{clip}%
\pgfsetbuttcap%
\pgfsetroundjoin%
\definecolor{currentfill}{rgb}{0.973241,0.954510,0.761799}%
\pgfsetfillcolor{currentfill}%
\pgfsetlinewidth{0.250937pt}%
\definecolor{currentstroke}{rgb}{1.000000,1.000000,1.000000}%
\pgfsetstrokecolor{currentstroke}%
\pgfsetdash{}{0pt}%
\pgfpathmoveto{\pgfqpoint{0.677752in}{0.988541in}}%
\pgfpathlineto{\pgfqpoint{0.776688in}{0.988541in}}%
\pgfpathlineto{\pgfqpoint{0.776688in}{0.889605in}}%
\pgfpathlineto{\pgfqpoint{0.677752in}{0.889605in}}%
\pgfpathlineto{\pgfqpoint{0.677752in}{0.988541in}}%
\pgfusepath{stroke,fill}%
\end{pgfscope}%
\begin{pgfscope}%
\pgfpathrectangle{\pgfqpoint{0.380943in}{0.295988in}}{\pgfqpoint{4.650000in}{0.692553in}}%
\pgfusepath{clip}%
\pgfsetbuttcap%
\pgfsetroundjoin%
\definecolor{currentfill}{rgb}{0.983391,0.971765,0.788189}%
\pgfsetfillcolor{currentfill}%
\pgfsetlinewidth{0.250937pt}%
\definecolor{currentstroke}{rgb}{1.000000,1.000000,1.000000}%
\pgfsetstrokecolor{currentstroke}%
\pgfsetdash{}{0pt}%
\pgfpathmoveto{\pgfqpoint{0.776688in}{0.988541in}}%
\pgfpathlineto{\pgfqpoint{0.875624in}{0.988541in}}%
\pgfpathlineto{\pgfqpoint{0.875624in}{0.889605in}}%
\pgfpathlineto{\pgfqpoint{0.776688in}{0.889605in}}%
\pgfpathlineto{\pgfqpoint{0.776688in}{0.988541in}}%
\pgfusepath{stroke,fill}%
\end{pgfscope}%
\begin{pgfscope}%
\pgfpathrectangle{\pgfqpoint{0.380943in}{0.295988in}}{\pgfqpoint{4.650000in}{0.692553in}}%
\pgfusepath{clip}%
\pgfsetbuttcap%
\pgfsetroundjoin%
\definecolor{currentfill}{rgb}{1.000000,1.000000,0.844829}%
\pgfsetfillcolor{currentfill}%
\pgfsetlinewidth{0.250937pt}%
\definecolor{currentstroke}{rgb}{1.000000,1.000000,1.000000}%
\pgfsetstrokecolor{currentstroke}%
\pgfsetdash{}{0pt}%
\pgfpathmoveto{\pgfqpoint{0.875624in}{0.988541in}}%
\pgfpathlineto{\pgfqpoint{0.974560in}{0.988541in}}%
\pgfpathlineto{\pgfqpoint{0.974560in}{0.889605in}}%
\pgfpathlineto{\pgfqpoint{0.875624in}{0.889605in}}%
\pgfpathlineto{\pgfqpoint{0.875624in}{0.988541in}}%
\pgfusepath{stroke,fill}%
\end{pgfscope}%
\begin{pgfscope}%
\pgfpathrectangle{\pgfqpoint{0.380943in}{0.295988in}}{\pgfqpoint{4.650000in}{0.692553in}}%
\pgfusepath{clip}%
\pgfsetbuttcap%
\pgfsetroundjoin%
\definecolor{currentfill}{rgb}{1.000000,1.000000,0.887120}%
\pgfsetfillcolor{currentfill}%
\pgfsetlinewidth{0.250937pt}%
\definecolor{currentstroke}{rgb}{1.000000,1.000000,1.000000}%
\pgfsetstrokecolor{currentstroke}%
\pgfsetdash{}{0pt}%
\pgfpathmoveto{\pgfqpoint{0.974560in}{0.988541in}}%
\pgfpathlineto{\pgfqpoint{1.073496in}{0.988541in}}%
\pgfpathlineto{\pgfqpoint{1.073496in}{0.889605in}}%
\pgfpathlineto{\pgfqpoint{0.974560in}{0.889605in}}%
\pgfpathlineto{\pgfqpoint{0.974560in}{0.988541in}}%
\pgfusepath{stroke,fill}%
\end{pgfscope}%
\begin{pgfscope}%
\pgfpathrectangle{\pgfqpoint{0.380943in}{0.295988in}}{\pgfqpoint{4.650000in}{0.692553in}}%
\pgfusepath{clip}%
\pgfsetbuttcap%
\pgfsetroundjoin%
\definecolor{currentfill}{rgb}{0.960892,0.932687,0.728981}%
\pgfsetfillcolor{currentfill}%
\pgfsetlinewidth{0.250937pt}%
\definecolor{currentstroke}{rgb}{1.000000,1.000000,1.000000}%
\pgfsetstrokecolor{currentstroke}%
\pgfsetdash{}{0pt}%
\pgfpathmoveto{\pgfqpoint{1.073496in}{0.988541in}}%
\pgfpathlineto{\pgfqpoint{1.172432in}{0.988541in}}%
\pgfpathlineto{\pgfqpoint{1.172432in}{0.889605in}}%
\pgfpathlineto{\pgfqpoint{1.073496in}{0.889605in}}%
\pgfpathlineto{\pgfqpoint{1.073496in}{0.988541in}}%
\pgfusepath{stroke,fill}%
\end{pgfscope}%
\begin{pgfscope}%
\pgfpathrectangle{\pgfqpoint{0.380943in}{0.295988in}}{\pgfqpoint{4.650000in}{0.692553in}}%
\pgfusepath{clip}%
\pgfsetbuttcap%
\pgfsetroundjoin%
\definecolor{currentfill}{rgb}{0.991849,0.986144,0.810181}%
\pgfsetfillcolor{currentfill}%
\pgfsetlinewidth{0.250937pt}%
\definecolor{currentstroke}{rgb}{1.000000,1.000000,1.000000}%
\pgfsetstrokecolor{currentstroke}%
\pgfsetdash{}{0pt}%
\pgfpathmoveto{\pgfqpoint{1.172432in}{0.988541in}}%
\pgfpathlineto{\pgfqpoint{1.271369in}{0.988541in}}%
\pgfpathlineto{\pgfqpoint{1.271369in}{0.889605in}}%
\pgfpathlineto{\pgfqpoint{1.172432in}{0.889605in}}%
\pgfpathlineto{\pgfqpoint{1.172432in}{0.988541in}}%
\pgfusepath{stroke,fill}%
\end{pgfscope}%
\begin{pgfscope}%
\pgfpathrectangle{\pgfqpoint{0.380943in}{0.295988in}}{\pgfqpoint{4.650000in}{0.692553in}}%
\pgfusepath{clip}%
\pgfsetbuttcap%
\pgfsetroundjoin%
\definecolor{currentfill}{rgb}{1.000000,0.564629,0.514479}%
\pgfsetfillcolor{currentfill}%
\pgfsetlinewidth{0.250937pt}%
\definecolor{currentstroke}{rgb}{1.000000,1.000000,1.000000}%
\pgfsetstrokecolor{currentstroke}%
\pgfsetdash{}{0pt}%
\pgfpathmoveto{\pgfqpoint{1.271369in}{0.988541in}}%
\pgfpathlineto{\pgfqpoint{1.370305in}{0.988541in}}%
\pgfpathlineto{\pgfqpoint{1.370305in}{0.889605in}}%
\pgfpathlineto{\pgfqpoint{1.271369in}{0.889605in}}%
\pgfpathlineto{\pgfqpoint{1.271369in}{0.988541in}}%
\pgfusepath{stroke,fill}%
\end{pgfscope}%
\begin{pgfscope}%
\pgfpathrectangle{\pgfqpoint{0.380943in}{0.295988in}}{\pgfqpoint{4.650000in}{0.692553in}}%
\pgfusepath{clip}%
\pgfsetbuttcap%
\pgfsetroundjoin%
\definecolor{currentfill}{rgb}{0.998939,0.658962,0.556032}%
\pgfsetfillcolor{currentfill}%
\pgfsetlinewidth{0.250937pt}%
\definecolor{currentstroke}{rgb}{1.000000,1.000000,1.000000}%
\pgfsetstrokecolor{currentstroke}%
\pgfsetdash{}{0pt}%
\pgfpathmoveto{\pgfqpoint{1.370305in}{0.988541in}}%
\pgfpathlineto{\pgfqpoint{1.469241in}{0.988541in}}%
\pgfpathlineto{\pgfqpoint{1.469241in}{0.889605in}}%
\pgfpathlineto{\pgfqpoint{1.370305in}{0.889605in}}%
\pgfpathlineto{\pgfqpoint{1.370305in}{0.988541in}}%
\pgfusepath{stroke,fill}%
\end{pgfscope}%
\begin{pgfscope}%
\pgfpathrectangle{\pgfqpoint{0.380943in}{0.295988in}}{\pgfqpoint{4.650000in}{0.692553in}}%
\pgfusepath{clip}%
\pgfsetbuttcap%
\pgfsetroundjoin%
\definecolor{currentfill}{rgb}{0.997247,0.702945,0.579715}%
\pgfsetfillcolor{currentfill}%
\pgfsetlinewidth{0.250937pt}%
\definecolor{currentstroke}{rgb}{1.000000,1.000000,1.000000}%
\pgfsetstrokecolor{currentstroke}%
\pgfsetdash{}{0pt}%
\pgfpathmoveto{\pgfqpoint{1.469241in}{0.988541in}}%
\pgfpathlineto{\pgfqpoint{1.568177in}{0.988541in}}%
\pgfpathlineto{\pgfqpoint{1.568177in}{0.889605in}}%
\pgfpathlineto{\pgfqpoint{1.469241in}{0.889605in}}%
\pgfpathlineto{\pgfqpoint{1.469241in}{0.988541in}}%
\pgfusepath{stroke,fill}%
\end{pgfscope}%
\begin{pgfscope}%
\pgfpathrectangle{\pgfqpoint{0.380943in}{0.295988in}}{\pgfqpoint{4.650000in}{0.692553in}}%
\pgfusepath{clip}%
\pgfsetbuttcap%
\pgfsetroundjoin%
\definecolor{currentfill}{rgb}{0.913879,0.392311,0.392311}%
\pgfsetfillcolor{currentfill}%
\pgfsetlinewidth{0.250937pt}%
\definecolor{currentstroke}{rgb}{1.000000,1.000000,1.000000}%
\pgfsetstrokecolor{currentstroke}%
\pgfsetdash{}{0pt}%
\pgfpathmoveto{\pgfqpoint{1.568177in}{0.988541in}}%
\pgfpathlineto{\pgfqpoint{1.667113in}{0.988541in}}%
\pgfpathlineto{\pgfqpoint{1.667113in}{0.889605in}}%
\pgfpathlineto{\pgfqpoint{1.568177in}{0.889605in}}%
\pgfpathlineto{\pgfqpoint{1.568177in}{0.988541in}}%
\pgfusepath{stroke,fill}%
\end{pgfscope}%
\begin{pgfscope}%
\pgfpathrectangle{\pgfqpoint{0.380943in}{0.295988in}}{\pgfqpoint{4.650000in}{0.692553in}}%
\pgfusepath{clip}%
\pgfsetbuttcap%
\pgfsetroundjoin%
\definecolor{currentfill}{rgb}{0.994018,0.750850,0.606382}%
\pgfsetfillcolor{currentfill}%
\pgfsetlinewidth{0.250937pt}%
\definecolor{currentstroke}{rgb}{1.000000,1.000000,1.000000}%
\pgfsetstrokecolor{currentstroke}%
\pgfsetdash{}{0pt}%
\pgfpathmoveto{\pgfqpoint{1.667113in}{0.988541in}}%
\pgfpathlineto{\pgfqpoint{1.766049in}{0.988541in}}%
\pgfpathlineto{\pgfqpoint{1.766049in}{0.889605in}}%
\pgfpathlineto{\pgfqpoint{1.667113in}{0.889605in}}%
\pgfpathlineto{\pgfqpoint{1.667113in}{0.988541in}}%
\pgfusepath{stroke,fill}%
\end{pgfscope}%
\begin{pgfscope}%
\pgfpathrectangle{\pgfqpoint{0.380943in}{0.295988in}}{\pgfqpoint{4.650000in}{0.692553in}}%
\pgfusepath{clip}%
\pgfsetbuttcap%
\pgfsetroundjoin%
\definecolor{currentfill}{rgb}{0.974072,0.862976,0.688750}%
\pgfsetfillcolor{currentfill}%
\pgfsetlinewidth{0.250937pt}%
\definecolor{currentstroke}{rgb}{1.000000,1.000000,1.000000}%
\pgfsetstrokecolor{currentstroke}%
\pgfsetdash{}{0pt}%
\pgfpathmoveto{\pgfqpoint{1.766049in}{0.988541in}}%
\pgfpathlineto{\pgfqpoint{1.864986in}{0.988541in}}%
\pgfpathlineto{\pgfqpoint{1.864986in}{0.889605in}}%
\pgfpathlineto{\pgfqpoint{1.766049in}{0.889605in}}%
\pgfpathlineto{\pgfqpoint{1.766049in}{0.988541in}}%
\pgfusepath{stroke,fill}%
\end{pgfscope}%
\begin{pgfscope}%
\pgfpathrectangle{\pgfqpoint{0.380943in}{0.295988in}}{\pgfqpoint{4.650000in}{0.692553in}}%
\pgfusepath{clip}%
\pgfsetbuttcap%
\pgfsetroundjoin%
\definecolor{currentfill}{rgb}{0.999785,0.636970,0.544191}%
\pgfsetfillcolor{currentfill}%
\pgfsetlinewidth{0.250937pt}%
\definecolor{currentstroke}{rgb}{1.000000,1.000000,1.000000}%
\pgfsetstrokecolor{currentstroke}%
\pgfsetdash{}{0pt}%
\pgfpathmoveto{\pgfqpoint{1.864986in}{0.988541in}}%
\pgfpathlineto{\pgfqpoint{1.963922in}{0.988541in}}%
\pgfpathlineto{\pgfqpoint{1.963922in}{0.889605in}}%
\pgfpathlineto{\pgfqpoint{1.864986in}{0.889605in}}%
\pgfpathlineto{\pgfqpoint{1.864986in}{0.988541in}}%
\pgfusepath{stroke,fill}%
\end{pgfscope}%
\begin{pgfscope}%
\pgfpathrectangle{\pgfqpoint{0.380943in}{0.295988in}}{\pgfqpoint{4.650000in}{0.692553in}}%
\pgfusepath{clip}%
\pgfsetbuttcap%
\pgfsetroundjoin%
\definecolor{currentfill}{rgb}{0.994694,0.745098,0.602999}%
\pgfsetfillcolor{currentfill}%
\pgfsetlinewidth{0.250937pt}%
\definecolor{currentstroke}{rgb}{1.000000,1.000000,1.000000}%
\pgfsetstrokecolor{currentstroke}%
\pgfsetdash{}{0pt}%
\pgfpathmoveto{\pgfqpoint{1.963922in}{0.988541in}}%
\pgfpathlineto{\pgfqpoint{2.062858in}{0.988541in}}%
\pgfpathlineto{\pgfqpoint{2.062858in}{0.889605in}}%
\pgfpathlineto{\pgfqpoint{1.963922in}{0.889605in}}%
\pgfpathlineto{\pgfqpoint{1.963922in}{0.988541in}}%
\pgfusepath{stroke,fill}%
\end{pgfscope}%
\begin{pgfscope}%
\pgfpathrectangle{\pgfqpoint{0.380943in}{0.295988in}}{\pgfqpoint{4.650000in}{0.692553in}}%
\pgfusepath{clip}%
\pgfsetbuttcap%
\pgfsetroundjoin%
\definecolor{currentfill}{rgb}{0.996401,0.724937,0.591557}%
\pgfsetfillcolor{currentfill}%
\pgfsetlinewidth{0.250937pt}%
\definecolor{currentstroke}{rgb}{1.000000,1.000000,1.000000}%
\pgfsetstrokecolor{currentstroke}%
\pgfsetdash{}{0pt}%
\pgfpathmoveto{\pgfqpoint{2.062858in}{0.988541in}}%
\pgfpathlineto{\pgfqpoint{2.161794in}{0.988541in}}%
\pgfpathlineto{\pgfqpoint{2.161794in}{0.889605in}}%
\pgfpathlineto{\pgfqpoint{2.062858in}{0.889605in}}%
\pgfpathlineto{\pgfqpoint{2.062858in}{0.988541in}}%
\pgfusepath{stroke,fill}%
\end{pgfscope}%
\begin{pgfscope}%
\pgfpathrectangle{\pgfqpoint{0.380943in}{0.295988in}}{\pgfqpoint{4.650000in}{0.692553in}}%
\pgfusepath{clip}%
\pgfsetbuttcap%
\pgfsetroundjoin%
\definecolor{currentfill}{rgb}{0.989619,0.788235,0.628374}%
\pgfsetfillcolor{currentfill}%
\pgfsetlinewidth{0.250937pt}%
\definecolor{currentstroke}{rgb}{1.000000,1.000000,1.000000}%
\pgfsetstrokecolor{currentstroke}%
\pgfsetdash{}{0pt}%
\pgfpathmoveto{\pgfqpoint{2.161794in}{0.988541in}}%
\pgfpathlineto{\pgfqpoint{2.260730in}{0.988541in}}%
\pgfpathlineto{\pgfqpoint{2.260730in}{0.889605in}}%
\pgfpathlineto{\pgfqpoint{2.161794in}{0.889605in}}%
\pgfpathlineto{\pgfqpoint{2.161794in}{0.988541in}}%
\pgfusepath{stroke,fill}%
\end{pgfscope}%
\begin{pgfscope}%
\pgfpathrectangle{\pgfqpoint{0.380943in}{0.295988in}}{\pgfqpoint{4.650000in}{0.692553in}}%
\pgfusepath{clip}%
\pgfsetbuttcap%
\pgfsetroundjoin%
\definecolor{currentfill}{rgb}{0.996401,0.724937,0.591557}%
\pgfsetfillcolor{currentfill}%
\pgfsetlinewidth{0.250937pt}%
\definecolor{currentstroke}{rgb}{1.000000,1.000000,1.000000}%
\pgfsetstrokecolor{currentstroke}%
\pgfsetdash{}{0pt}%
\pgfpathmoveto{\pgfqpoint{2.260730in}{0.988541in}}%
\pgfpathlineto{\pgfqpoint{2.359666in}{0.988541in}}%
\pgfpathlineto{\pgfqpoint{2.359666in}{0.889605in}}%
\pgfpathlineto{\pgfqpoint{2.260730in}{0.889605in}}%
\pgfpathlineto{\pgfqpoint{2.260730in}{0.988541in}}%
\pgfusepath{stroke,fill}%
\end{pgfscope}%
\begin{pgfscope}%
\pgfpathrectangle{\pgfqpoint{0.380943in}{0.295988in}}{\pgfqpoint{4.650000in}{0.692553in}}%
\pgfusepath{clip}%
\pgfsetbuttcap%
\pgfsetroundjoin%
\definecolor{currentfill}{rgb}{0.999446,0.645767,0.548927}%
\pgfsetfillcolor{currentfill}%
\pgfsetlinewidth{0.250937pt}%
\definecolor{currentstroke}{rgb}{1.000000,1.000000,1.000000}%
\pgfsetstrokecolor{currentstroke}%
\pgfsetdash{}{0pt}%
\pgfpathmoveto{\pgfqpoint{2.359666in}{0.988541in}}%
\pgfpathlineto{\pgfqpoint{2.458603in}{0.988541in}}%
\pgfpathlineto{\pgfqpoint{2.458603in}{0.889605in}}%
\pgfpathlineto{\pgfqpoint{2.359666in}{0.889605in}}%
\pgfpathlineto{\pgfqpoint{2.359666in}{0.988541in}}%
\pgfusepath{stroke,fill}%
\end{pgfscope}%
\begin{pgfscope}%
\pgfpathrectangle{\pgfqpoint{0.380943in}{0.295988in}}{\pgfqpoint{4.650000in}{0.692553in}}%
\pgfusepath{clip}%
\pgfsetbuttcap%
\pgfsetroundjoin%
\definecolor{currentfill}{rgb}{0.964783,0.940131,0.739808}%
\pgfsetfillcolor{currentfill}%
\pgfsetlinewidth{0.250937pt}%
\definecolor{currentstroke}{rgb}{1.000000,1.000000,1.000000}%
\pgfsetstrokecolor{currentstroke}%
\pgfsetdash{}{0pt}%
\pgfpathmoveto{\pgfqpoint{2.458603in}{0.988541in}}%
\pgfpathlineto{\pgfqpoint{2.557539in}{0.988541in}}%
\pgfpathlineto{\pgfqpoint{2.557539in}{0.889605in}}%
\pgfpathlineto{\pgfqpoint{2.458603in}{0.889605in}}%
\pgfpathlineto{\pgfqpoint{2.458603in}{0.988541in}}%
\pgfusepath{stroke,fill}%
\end{pgfscope}%
\begin{pgfscope}%
\pgfpathrectangle{\pgfqpoint{0.380943in}{0.295988in}}{\pgfqpoint{4.650000in}{0.692553in}}%
\pgfusepath{clip}%
\pgfsetbuttcap%
\pgfsetroundjoin%
\definecolor{currentfill}{rgb}{0.998939,0.658962,0.556032}%
\pgfsetfillcolor{currentfill}%
\pgfsetlinewidth{0.250937pt}%
\definecolor{currentstroke}{rgb}{1.000000,1.000000,1.000000}%
\pgfsetstrokecolor{currentstroke}%
\pgfsetdash{}{0pt}%
\pgfpathmoveto{\pgfqpoint{2.557539in}{0.988541in}}%
\pgfpathlineto{\pgfqpoint{2.656475in}{0.988541in}}%
\pgfpathlineto{\pgfqpoint{2.656475in}{0.889605in}}%
\pgfpathlineto{\pgfqpoint{2.557539in}{0.889605in}}%
\pgfpathlineto{\pgfqpoint{2.557539in}{0.988541in}}%
\pgfusepath{stroke,fill}%
\end{pgfscope}%
\begin{pgfscope}%
\pgfpathrectangle{\pgfqpoint{0.380943in}{0.295988in}}{\pgfqpoint{4.650000in}{0.692553in}}%
\pgfusepath{clip}%
\pgfsetbuttcap%
\pgfsetroundjoin%
\definecolor{currentfill}{rgb}{0.999446,0.645767,0.548927}%
\pgfsetfillcolor{currentfill}%
\pgfsetlinewidth{0.250937pt}%
\definecolor{currentstroke}{rgb}{1.000000,1.000000,1.000000}%
\pgfsetstrokecolor{currentstroke}%
\pgfsetdash{}{0pt}%
\pgfpathmoveto{\pgfqpoint{2.656475in}{0.988541in}}%
\pgfpathlineto{\pgfqpoint{2.755411in}{0.988541in}}%
\pgfpathlineto{\pgfqpoint{2.755411in}{0.889605in}}%
\pgfpathlineto{\pgfqpoint{2.656475in}{0.889605in}}%
\pgfpathlineto{\pgfqpoint{2.656475in}{0.988541in}}%
\pgfusepath{stroke,fill}%
\end{pgfscope}%
\begin{pgfscope}%
\pgfpathrectangle{\pgfqpoint{0.380943in}{0.295988in}}{\pgfqpoint{4.650000in}{0.692553in}}%
\pgfusepath{clip}%
\pgfsetbuttcap%
\pgfsetroundjoin%
\definecolor{currentfill}{rgb}{0.993003,0.759477,0.611457}%
\pgfsetfillcolor{currentfill}%
\pgfsetlinewidth{0.250937pt}%
\definecolor{currentstroke}{rgb}{1.000000,1.000000,1.000000}%
\pgfsetstrokecolor{currentstroke}%
\pgfsetdash{}{0pt}%
\pgfpathmoveto{\pgfqpoint{2.755411in}{0.988541in}}%
\pgfpathlineto{\pgfqpoint{2.854347in}{0.988541in}}%
\pgfpathlineto{\pgfqpoint{2.854347in}{0.889605in}}%
\pgfpathlineto{\pgfqpoint{2.755411in}{0.889605in}}%
\pgfpathlineto{\pgfqpoint{2.755411in}{0.988541in}}%
\pgfusepath{stroke,fill}%
\end{pgfscope}%
\begin{pgfscope}%
\pgfpathrectangle{\pgfqpoint{0.380943in}{0.295988in}}{\pgfqpoint{4.650000in}{0.692553in}}%
\pgfusepath{clip}%
\pgfsetbuttcap%
\pgfsetroundjoin%
\definecolor{currentfill}{rgb}{0.990296,0.782484,0.624990}%
\pgfsetfillcolor{currentfill}%
\pgfsetlinewidth{0.250937pt}%
\definecolor{currentstroke}{rgb}{1.000000,1.000000,1.000000}%
\pgfsetstrokecolor{currentstroke}%
\pgfsetdash{}{0pt}%
\pgfpathmoveto{\pgfqpoint{2.854347in}{0.988541in}}%
\pgfpathlineto{\pgfqpoint{2.953283in}{0.988541in}}%
\pgfpathlineto{\pgfqpoint{2.953283in}{0.889605in}}%
\pgfpathlineto{\pgfqpoint{2.854347in}{0.889605in}}%
\pgfpathlineto{\pgfqpoint{2.854347in}{0.988541in}}%
\pgfusepath{stroke,fill}%
\end{pgfscope}%
\begin{pgfscope}%
\pgfpathrectangle{\pgfqpoint{0.380943in}{0.295988in}}{\pgfqpoint{4.650000in}{0.692553in}}%
\pgfusepath{clip}%
\pgfsetbuttcap%
\pgfsetroundjoin%
\definecolor{currentfill}{rgb}{0.980669,0.832787,0.665559}%
\pgfsetfillcolor{currentfill}%
\pgfsetlinewidth{0.250937pt}%
\definecolor{currentstroke}{rgb}{1.000000,1.000000,1.000000}%
\pgfsetstrokecolor{currentstroke}%
\pgfsetdash{}{0pt}%
\pgfpathmoveto{\pgfqpoint{2.953283in}{0.988541in}}%
\pgfpathlineto{\pgfqpoint{3.052220in}{0.988541in}}%
\pgfpathlineto{\pgfqpoint{3.052220in}{0.889605in}}%
\pgfpathlineto{\pgfqpoint{2.953283in}{0.889605in}}%
\pgfpathlineto{\pgfqpoint{2.953283in}{0.988541in}}%
\pgfusepath{stroke,fill}%
\end{pgfscope}%
\begin{pgfscope}%
\pgfpathrectangle{\pgfqpoint{0.380943in}{0.295988in}}{\pgfqpoint{4.650000in}{0.692553in}}%
\pgfusepath{clip}%
\pgfsetbuttcap%
\pgfsetroundjoin%
\definecolor{currentfill}{rgb}{0.991311,0.773856,0.619915}%
\pgfsetfillcolor{currentfill}%
\pgfsetlinewidth{0.250937pt}%
\definecolor{currentstroke}{rgb}{1.000000,1.000000,1.000000}%
\pgfsetstrokecolor{currentstroke}%
\pgfsetdash{}{0pt}%
\pgfpathmoveto{\pgfqpoint{3.052220in}{0.988541in}}%
\pgfpathlineto{\pgfqpoint{3.151156in}{0.988541in}}%
\pgfpathlineto{\pgfqpoint{3.151156in}{0.889605in}}%
\pgfpathlineto{\pgfqpoint{3.052220in}{0.889605in}}%
\pgfpathlineto{\pgfqpoint{3.052220in}{0.988541in}}%
\pgfusepath{stroke,fill}%
\end{pgfscope}%
\begin{pgfscope}%
\pgfpathrectangle{\pgfqpoint{0.380943in}{0.295988in}}{\pgfqpoint{4.650000in}{0.692553in}}%
\pgfusepath{clip}%
\pgfsetbuttcap%
\pgfsetroundjoin%
\definecolor{currentfill}{rgb}{0.988604,0.796863,0.633449}%
\pgfsetfillcolor{currentfill}%
\pgfsetlinewidth{0.250937pt}%
\definecolor{currentstroke}{rgb}{1.000000,1.000000,1.000000}%
\pgfsetstrokecolor{currentstroke}%
\pgfsetdash{}{0pt}%
\pgfpathmoveto{\pgfqpoint{3.151156in}{0.988541in}}%
\pgfpathlineto{\pgfqpoint{3.250092in}{0.988541in}}%
\pgfpathlineto{\pgfqpoint{3.250092in}{0.889605in}}%
\pgfpathlineto{\pgfqpoint{3.151156in}{0.889605in}}%
\pgfpathlineto{\pgfqpoint{3.151156in}{0.988541in}}%
\pgfusepath{stroke,fill}%
\end{pgfscope}%
\begin{pgfscope}%
\pgfpathrectangle{\pgfqpoint{0.380943in}{0.295988in}}{\pgfqpoint{4.650000in}{0.692553in}}%
\pgfusepath{clip}%
\pgfsetbuttcap%
\pgfsetroundjoin%
\definecolor{currentfill}{rgb}{0.990296,0.782484,0.624990}%
\pgfsetfillcolor{currentfill}%
\pgfsetlinewidth{0.250937pt}%
\definecolor{currentstroke}{rgb}{1.000000,1.000000,1.000000}%
\pgfsetstrokecolor{currentstroke}%
\pgfsetdash{}{0pt}%
\pgfpathmoveto{\pgfqpoint{3.250092in}{0.988541in}}%
\pgfpathlineto{\pgfqpoint{3.349028in}{0.988541in}}%
\pgfpathlineto{\pgfqpoint{3.349028in}{0.889605in}}%
\pgfpathlineto{\pgfqpoint{3.250092in}{0.889605in}}%
\pgfpathlineto{\pgfqpoint{3.250092in}{0.988541in}}%
\pgfusepath{stroke,fill}%
\end{pgfscope}%
\begin{pgfscope}%
\pgfpathrectangle{\pgfqpoint{0.380943in}{0.295988in}}{\pgfqpoint{4.650000in}{0.692553in}}%
\pgfusepath{clip}%
\pgfsetbuttcap%
\pgfsetroundjoin%
\definecolor{currentfill}{rgb}{0.887966,0.366398,0.366398}%
\pgfsetfillcolor{currentfill}%
\pgfsetlinewidth{0.250937pt}%
\definecolor{currentstroke}{rgb}{1.000000,1.000000,1.000000}%
\pgfsetstrokecolor{currentstroke}%
\pgfsetdash{}{0pt}%
\pgfpathmoveto{\pgfqpoint{3.349028in}{0.988541in}}%
\pgfpathlineto{\pgfqpoint{3.447964in}{0.988541in}}%
\pgfpathlineto{\pgfqpoint{3.447964in}{0.889605in}}%
\pgfpathlineto{\pgfqpoint{3.349028in}{0.889605in}}%
\pgfpathlineto{\pgfqpoint{3.349028in}{0.988541in}}%
\pgfusepath{stroke,fill}%
\end{pgfscope}%
\begin{pgfscope}%
\pgfpathrectangle{\pgfqpoint{0.380943in}{0.295988in}}{\pgfqpoint{4.650000in}{0.692553in}}%
\pgfusepath{clip}%
\pgfsetbuttcap%
\pgfsetroundjoin%
\definecolor{currentfill}{rgb}{1.000000,0.625529,0.538839}%
\pgfsetfillcolor{currentfill}%
\pgfsetlinewidth{0.250937pt}%
\definecolor{currentstroke}{rgb}{1.000000,1.000000,1.000000}%
\pgfsetstrokecolor{currentstroke}%
\pgfsetdash{}{0pt}%
\pgfpathmoveto{\pgfqpoint{3.447964in}{0.988541in}}%
\pgfpathlineto{\pgfqpoint{3.546901in}{0.988541in}}%
\pgfpathlineto{\pgfqpoint{3.546901in}{0.889605in}}%
\pgfpathlineto{\pgfqpoint{3.447964in}{0.889605in}}%
\pgfpathlineto{\pgfqpoint{3.447964in}{0.988541in}}%
\pgfusepath{stroke,fill}%
\end{pgfscope}%
\begin{pgfscope}%
\pgfpathrectangle{\pgfqpoint{0.380943in}{0.295988in}}{\pgfqpoint{4.650000in}{0.692553in}}%
\pgfusepath{clip}%
\pgfsetbuttcap%
\pgfsetroundjoin%
\definecolor{currentfill}{rgb}{0.998093,0.680953,0.567874}%
\pgfsetfillcolor{currentfill}%
\pgfsetlinewidth{0.250937pt}%
\definecolor{currentstroke}{rgb}{1.000000,1.000000,1.000000}%
\pgfsetstrokecolor{currentstroke}%
\pgfsetdash{}{0pt}%
\pgfpathmoveto{\pgfqpoint{3.546901in}{0.988541in}}%
\pgfpathlineto{\pgfqpoint{3.645837in}{0.988541in}}%
\pgfpathlineto{\pgfqpoint{3.645837in}{0.889605in}}%
\pgfpathlineto{\pgfqpoint{3.546901in}{0.889605in}}%
\pgfpathlineto{\pgfqpoint{3.546901in}{0.988541in}}%
\pgfusepath{stroke,fill}%
\end{pgfscope}%
\begin{pgfscope}%
\pgfpathrectangle{\pgfqpoint{0.380943in}{0.295988in}}{\pgfqpoint{4.650000in}{0.692553in}}%
\pgfusepath{clip}%
\pgfsetbuttcap%
\pgfsetroundjoin%
\definecolor{currentfill}{rgb}{0.998939,0.658962,0.556032}%
\pgfsetfillcolor{currentfill}%
\pgfsetlinewidth{0.250937pt}%
\definecolor{currentstroke}{rgb}{1.000000,1.000000,1.000000}%
\pgfsetstrokecolor{currentstroke}%
\pgfsetdash{}{0pt}%
\pgfpathmoveto{\pgfqpoint{3.645837in}{0.988541in}}%
\pgfpathlineto{\pgfqpoint{3.744773in}{0.988541in}}%
\pgfpathlineto{\pgfqpoint{3.744773in}{0.889605in}}%
\pgfpathlineto{\pgfqpoint{3.645837in}{0.889605in}}%
\pgfpathlineto{\pgfqpoint{3.645837in}{0.988541in}}%
\pgfusepath{stroke,fill}%
\end{pgfscope}%
\begin{pgfscope}%
\pgfpathrectangle{\pgfqpoint{0.380943in}{0.295988in}}{\pgfqpoint{4.650000in}{0.692553in}}%
\pgfusepath{clip}%
\pgfsetbuttcap%
\pgfsetroundjoin%
\definecolor{currentfill}{rgb}{0.997247,0.702945,0.579715}%
\pgfsetfillcolor{currentfill}%
\pgfsetlinewidth{0.250937pt}%
\definecolor{currentstroke}{rgb}{1.000000,1.000000,1.000000}%
\pgfsetstrokecolor{currentstroke}%
\pgfsetdash{}{0pt}%
\pgfpathmoveto{\pgfqpoint{3.744773in}{0.988541in}}%
\pgfpathlineto{\pgfqpoint{3.843709in}{0.988541in}}%
\pgfpathlineto{\pgfqpoint{3.843709in}{0.889605in}}%
\pgfpathlineto{\pgfqpoint{3.744773in}{0.889605in}}%
\pgfpathlineto{\pgfqpoint{3.744773in}{0.988541in}}%
\pgfusepath{stroke,fill}%
\end{pgfscope}%
\begin{pgfscope}%
\pgfpathrectangle{\pgfqpoint{0.380943in}{0.295988in}}{\pgfqpoint{4.650000in}{0.692553in}}%
\pgfusepath{clip}%
\pgfsetbuttcap%
\pgfsetroundjoin%
\definecolor{currentfill}{rgb}{1.000000,0.538331,0.503652}%
\pgfsetfillcolor{currentfill}%
\pgfsetlinewidth{0.250937pt}%
\definecolor{currentstroke}{rgb}{1.000000,1.000000,1.000000}%
\pgfsetstrokecolor{currentstroke}%
\pgfsetdash{}{0pt}%
\pgfpathmoveto{\pgfqpoint{3.843709in}{0.988541in}}%
\pgfpathlineto{\pgfqpoint{3.942645in}{0.988541in}}%
\pgfpathlineto{\pgfqpoint{3.942645in}{0.889605in}}%
\pgfpathlineto{\pgfqpoint{3.843709in}{0.889605in}}%
\pgfpathlineto{\pgfqpoint{3.843709in}{0.988541in}}%
\pgfusepath{stroke,fill}%
\end{pgfscope}%
\begin{pgfscope}%
\pgfpathrectangle{\pgfqpoint{0.380943in}{0.295988in}}{\pgfqpoint{4.650000in}{0.692553in}}%
\pgfusepath{clip}%
\pgfsetbuttcap%
\pgfsetroundjoin%
\definecolor{currentfill}{rgb}{0.998939,0.658962,0.556032}%
\pgfsetfillcolor{currentfill}%
\pgfsetlinewidth{0.250937pt}%
\definecolor{currentstroke}{rgb}{1.000000,1.000000,1.000000}%
\pgfsetstrokecolor{currentstroke}%
\pgfsetdash{}{0pt}%
\pgfpathmoveto{\pgfqpoint{3.942645in}{0.988541in}}%
\pgfpathlineto{\pgfqpoint{4.041581in}{0.988541in}}%
\pgfpathlineto{\pgfqpoint{4.041581in}{0.889605in}}%
\pgfpathlineto{\pgfqpoint{3.942645in}{0.889605in}}%
\pgfpathlineto{\pgfqpoint{3.942645in}{0.988541in}}%
\pgfusepath{stroke,fill}%
\end{pgfscope}%
\begin{pgfscope}%
\pgfpathrectangle{\pgfqpoint{0.380943in}{0.295988in}}{\pgfqpoint{4.650000in}{0.692553in}}%
\pgfusepath{clip}%
\pgfsetbuttcap%
\pgfsetroundjoin%
\definecolor{currentfill}{rgb}{1.000000,0.564629,0.514479}%
\pgfsetfillcolor{currentfill}%
\pgfsetlinewidth{0.250937pt}%
\definecolor{currentstroke}{rgb}{1.000000,1.000000,1.000000}%
\pgfsetstrokecolor{currentstroke}%
\pgfsetdash{}{0pt}%
\pgfpathmoveto{\pgfqpoint{4.041581in}{0.988541in}}%
\pgfpathlineto{\pgfqpoint{4.140518in}{0.988541in}}%
\pgfpathlineto{\pgfqpoint{4.140518in}{0.889605in}}%
\pgfpathlineto{\pgfqpoint{4.041581in}{0.889605in}}%
\pgfpathlineto{\pgfqpoint{4.041581in}{0.988541in}}%
\pgfusepath{stroke,fill}%
\end{pgfscope}%
\begin{pgfscope}%
\pgfpathrectangle{\pgfqpoint{0.380943in}{0.295988in}}{\pgfqpoint{4.650000in}{0.692553in}}%
\pgfusepath{clip}%
\pgfsetbuttcap%
\pgfsetroundjoin%
\definecolor{currentfill}{rgb}{0.996401,0.724937,0.591557}%
\pgfsetfillcolor{currentfill}%
\pgfsetlinewidth{0.250937pt}%
\definecolor{currentstroke}{rgb}{1.000000,1.000000,1.000000}%
\pgfsetstrokecolor{currentstroke}%
\pgfsetdash{}{0pt}%
\pgfpathmoveto{\pgfqpoint{4.140518in}{0.988541in}}%
\pgfpathlineto{\pgfqpoint{4.239454in}{0.988541in}}%
\pgfpathlineto{\pgfqpoint{4.239454in}{0.889605in}}%
\pgfpathlineto{\pgfqpoint{4.140518in}{0.889605in}}%
\pgfpathlineto{\pgfqpoint{4.140518in}{0.988541in}}%
\pgfusepath{stroke,fill}%
\end{pgfscope}%
\begin{pgfscope}%
\pgfpathrectangle{\pgfqpoint{0.380943in}{0.295988in}}{\pgfqpoint{4.650000in}{0.692553in}}%
\pgfusepath{clip}%
\pgfsetbuttcap%
\pgfsetroundjoin%
\definecolor{currentfill}{rgb}{0.998093,0.680953,0.567874}%
\pgfsetfillcolor{currentfill}%
\pgfsetlinewidth{0.250937pt}%
\definecolor{currentstroke}{rgb}{1.000000,1.000000,1.000000}%
\pgfsetstrokecolor{currentstroke}%
\pgfsetdash{}{0pt}%
\pgfpathmoveto{\pgfqpoint{4.239454in}{0.988541in}}%
\pgfpathlineto{\pgfqpoint{4.338390in}{0.988541in}}%
\pgfpathlineto{\pgfqpoint{4.338390in}{0.889605in}}%
\pgfpathlineto{\pgfqpoint{4.239454in}{0.889605in}}%
\pgfpathlineto{\pgfqpoint{4.239454in}{0.988541in}}%
\pgfusepath{stroke,fill}%
\end{pgfscope}%
\begin{pgfscope}%
\pgfpathrectangle{\pgfqpoint{0.380943in}{0.295988in}}{\pgfqpoint{4.650000in}{0.692553in}}%
\pgfusepath{clip}%
\pgfsetbuttcap%
\pgfsetroundjoin%
\definecolor{currentfill}{rgb}{0.964275,0.912388,0.715448}%
\pgfsetfillcolor{currentfill}%
\pgfsetlinewidth{0.250937pt}%
\definecolor{currentstroke}{rgb}{1.000000,1.000000,1.000000}%
\pgfsetstrokecolor{currentstroke}%
\pgfsetdash{}{0pt}%
\pgfpathmoveto{\pgfqpoint{4.338390in}{0.988541in}}%
\pgfpathlineto{\pgfqpoint{4.437326in}{0.988541in}}%
\pgfpathlineto{\pgfqpoint{4.437326in}{0.889605in}}%
\pgfpathlineto{\pgfqpoint{4.338390in}{0.889605in}}%
\pgfpathlineto{\pgfqpoint{4.338390in}{0.988541in}}%
\pgfusepath{stroke,fill}%
\end{pgfscope}%
\begin{pgfscope}%
\pgfpathrectangle{\pgfqpoint{0.380943in}{0.295988in}}{\pgfqpoint{4.650000in}{0.692553in}}%
\pgfusepath{clip}%
\pgfsetbuttcap%
\pgfsetroundjoin%
\definecolor{currentfill}{rgb}{0.966459,0.901038,0.709050}%
\pgfsetfillcolor{currentfill}%
\pgfsetlinewidth{0.250937pt}%
\definecolor{currentstroke}{rgb}{1.000000,1.000000,1.000000}%
\pgfsetstrokecolor{currentstroke}%
\pgfsetdash{}{0pt}%
\pgfpathmoveto{\pgfqpoint{4.437326in}{0.988541in}}%
\pgfpathlineto{\pgfqpoint{4.536262in}{0.988541in}}%
\pgfpathlineto{\pgfqpoint{4.536262in}{0.889605in}}%
\pgfpathlineto{\pgfqpoint{4.437326in}{0.889605in}}%
\pgfpathlineto{\pgfqpoint{4.437326in}{0.988541in}}%
\pgfusepath{stroke,fill}%
\end{pgfscope}%
\begin{pgfscope}%
\pgfpathrectangle{\pgfqpoint{0.380943in}{0.295988in}}{\pgfqpoint{4.650000in}{0.692553in}}%
\pgfusepath{clip}%
\pgfsetbuttcap%
\pgfsetroundjoin%
\definecolor{currentfill}{rgb}{0.978131,0.843783,0.675709}%
\pgfsetfillcolor{currentfill}%
\pgfsetlinewidth{0.250937pt}%
\definecolor{currentstroke}{rgb}{1.000000,1.000000,1.000000}%
\pgfsetstrokecolor{currentstroke}%
\pgfsetdash{}{0pt}%
\pgfpathmoveto{\pgfqpoint{4.536262in}{0.988541in}}%
\pgfpathlineto{\pgfqpoint{4.635198in}{0.988541in}}%
\pgfpathlineto{\pgfqpoint{4.635198in}{0.889605in}}%
\pgfpathlineto{\pgfqpoint{4.536262in}{0.889605in}}%
\pgfpathlineto{\pgfqpoint{4.536262in}{0.988541in}}%
\pgfusepath{stroke,fill}%
\end{pgfscope}%
\begin{pgfscope}%
\pgfpathrectangle{\pgfqpoint{0.380943in}{0.295988in}}{\pgfqpoint{4.650000in}{0.692553in}}%
\pgfusepath{clip}%
\pgfsetbuttcap%
\pgfsetroundjoin%
\definecolor{currentfill}{rgb}{0.961738,0.927612,0.725598}%
\pgfsetfillcolor{currentfill}%
\pgfsetlinewidth{0.250937pt}%
\definecolor{currentstroke}{rgb}{1.000000,1.000000,1.000000}%
\pgfsetstrokecolor{currentstroke}%
\pgfsetdash{}{0pt}%
\pgfpathmoveto{\pgfqpoint{4.635198in}{0.988541in}}%
\pgfpathlineto{\pgfqpoint{4.734135in}{0.988541in}}%
\pgfpathlineto{\pgfqpoint{4.734135in}{0.889605in}}%
\pgfpathlineto{\pgfqpoint{4.635198in}{0.889605in}}%
\pgfpathlineto{\pgfqpoint{4.635198in}{0.988541in}}%
\pgfusepath{stroke,fill}%
\end{pgfscope}%
\begin{pgfscope}%
\pgfpathrectangle{\pgfqpoint{0.380943in}{0.295988in}}{\pgfqpoint{4.650000in}{0.692553in}}%
\pgfusepath{clip}%
\pgfsetbuttcap%
\pgfsetroundjoin%
\definecolor{currentfill}{rgb}{1.000000,1.000000,0.899808}%
\pgfsetfillcolor{currentfill}%
\pgfsetlinewidth{0.250937pt}%
\definecolor{currentstroke}{rgb}{1.000000,1.000000,1.000000}%
\pgfsetstrokecolor{currentstroke}%
\pgfsetdash{}{0pt}%
\pgfpathmoveto{\pgfqpoint{4.734135in}{0.988541in}}%
\pgfpathlineto{\pgfqpoint{4.833071in}{0.988541in}}%
\pgfpathlineto{\pgfqpoint{4.833071in}{0.889605in}}%
\pgfpathlineto{\pgfqpoint{4.734135in}{0.889605in}}%
\pgfpathlineto{\pgfqpoint{4.734135in}{0.988541in}}%
\pgfusepath{stroke,fill}%
\end{pgfscope}%
\begin{pgfscope}%
\pgfpathrectangle{\pgfqpoint{0.380943in}{0.295988in}}{\pgfqpoint{4.650000in}{0.692553in}}%
\pgfusepath{clip}%
\pgfsetbuttcap%
\pgfsetroundjoin%
\definecolor{currentfill}{rgb}{0.963429,0.917463,0.718831}%
\pgfsetfillcolor{currentfill}%
\pgfsetlinewidth{0.250937pt}%
\definecolor{currentstroke}{rgb}{1.000000,1.000000,1.000000}%
\pgfsetstrokecolor{currentstroke}%
\pgfsetdash{}{0pt}%
\pgfpathmoveto{\pgfqpoint{4.833071in}{0.988541in}}%
\pgfpathlineto{\pgfqpoint{4.932007in}{0.988541in}}%
\pgfpathlineto{\pgfqpoint{4.932007in}{0.889605in}}%
\pgfpathlineto{\pgfqpoint{4.833071in}{0.889605in}}%
\pgfpathlineto{\pgfqpoint{4.833071in}{0.988541in}}%
\pgfusepath{stroke,fill}%
\end{pgfscope}%
\begin{pgfscope}%
\pgfpathrectangle{\pgfqpoint{0.380943in}{0.295988in}}{\pgfqpoint{4.650000in}{0.692553in}}%
\pgfusepath{clip}%
\pgfsetbuttcap%
\pgfsetroundjoin%
\definecolor{currentfill}{rgb}{0.974072,0.862976,0.688750}%
\pgfsetfillcolor{currentfill}%
\pgfsetlinewidth{0.250937pt}%
\definecolor{currentstroke}{rgb}{1.000000,1.000000,1.000000}%
\pgfsetstrokecolor{currentstroke}%
\pgfsetdash{}{0pt}%
\pgfpathmoveto{\pgfqpoint{4.932007in}{0.988541in}}%
\pgfpathlineto{\pgfqpoint{5.030943in}{0.988541in}}%
\pgfpathlineto{\pgfqpoint{5.030943in}{0.889605in}}%
\pgfpathlineto{\pgfqpoint{4.932007in}{0.889605in}}%
\pgfpathlineto{\pgfqpoint{4.932007in}{0.988541in}}%
\pgfusepath{stroke,fill}%
\end{pgfscope}%
\begin{pgfscope}%
\pgfpathrectangle{\pgfqpoint{0.380943in}{0.295988in}}{\pgfqpoint{4.650000in}{0.692553in}}%
\pgfusepath{clip}%
\pgfsetbuttcap%
\pgfsetroundjoin%
\pgfsetlinewidth{0.250937pt}%
\definecolor{currentstroke}{rgb}{1.000000,1.000000,1.000000}%
\pgfsetstrokecolor{currentstroke}%
\pgfsetdash{}{0pt}%
\pgfpathmoveto{\pgfqpoint{0.380943in}{0.889605in}}%
\pgfpathlineto{\pgfqpoint{0.479879in}{0.889605in}}%
\pgfpathlineto{\pgfqpoint{0.479879in}{0.790669in}}%
\pgfpathlineto{\pgfqpoint{0.380943in}{0.790669in}}%
\pgfpathlineto{\pgfqpoint{0.380943in}{0.889605in}}%
\pgfusepath{stroke}%
\end{pgfscope}%
\begin{pgfscope}%
\pgfpathrectangle{\pgfqpoint{0.380943in}{0.295988in}}{\pgfqpoint{4.650000in}{0.692553in}}%
\pgfusepath{clip}%
\pgfsetbuttcap%
\pgfsetroundjoin%
\definecolor{currentfill}{rgb}{0.986774,0.977516,0.796986}%
\pgfsetfillcolor{currentfill}%
\pgfsetlinewidth{0.250937pt}%
\definecolor{currentstroke}{rgb}{1.000000,1.000000,1.000000}%
\pgfsetstrokecolor{currentstroke}%
\pgfsetdash{}{0pt}%
\pgfpathmoveto{\pgfqpoint{0.479879in}{0.889605in}}%
\pgfpathlineto{\pgfqpoint{0.578815in}{0.889605in}}%
\pgfpathlineto{\pgfqpoint{0.578815in}{0.790669in}}%
\pgfpathlineto{\pgfqpoint{0.479879in}{0.790669in}}%
\pgfpathlineto{\pgfqpoint{0.479879in}{0.889605in}}%
\pgfusepath{stroke,fill}%
\end{pgfscope}%
\begin{pgfscope}%
\pgfpathrectangle{\pgfqpoint{0.380943in}{0.295988in}}{\pgfqpoint{4.650000in}{0.692553in}}%
\pgfusepath{clip}%
\pgfsetbuttcap%
\pgfsetroundjoin%
\definecolor{currentfill}{rgb}{0.964275,0.912388,0.715448}%
\pgfsetfillcolor{currentfill}%
\pgfsetlinewidth{0.250937pt}%
\definecolor{currentstroke}{rgb}{1.000000,1.000000,1.000000}%
\pgfsetstrokecolor{currentstroke}%
\pgfsetdash{}{0pt}%
\pgfpathmoveto{\pgfqpoint{0.578815in}{0.889605in}}%
\pgfpathlineto{\pgfqpoint{0.677752in}{0.889605in}}%
\pgfpathlineto{\pgfqpoint{0.677752in}{0.790669in}}%
\pgfpathlineto{\pgfqpoint{0.578815in}{0.790669in}}%
\pgfpathlineto{\pgfqpoint{0.578815in}{0.889605in}}%
\pgfusepath{stroke,fill}%
\end{pgfscope}%
\begin{pgfscope}%
\pgfpathrectangle{\pgfqpoint{0.380943in}{0.295988in}}{\pgfqpoint{4.650000in}{0.692553in}}%
\pgfusepath{clip}%
\pgfsetbuttcap%
\pgfsetroundjoin%
\definecolor{currentfill}{rgb}{0.991849,0.986144,0.810181}%
\pgfsetfillcolor{currentfill}%
\pgfsetlinewidth{0.250937pt}%
\definecolor{currentstroke}{rgb}{1.000000,1.000000,1.000000}%
\pgfsetstrokecolor{currentstroke}%
\pgfsetdash{}{0pt}%
\pgfpathmoveto{\pgfqpoint{0.677752in}{0.889605in}}%
\pgfpathlineto{\pgfqpoint{0.776688in}{0.889605in}}%
\pgfpathlineto{\pgfqpoint{0.776688in}{0.790669in}}%
\pgfpathlineto{\pgfqpoint{0.677752in}{0.790669in}}%
\pgfpathlineto{\pgfqpoint{0.677752in}{0.889605in}}%
\pgfusepath{stroke,fill}%
\end{pgfscope}%
\begin{pgfscope}%
\pgfpathrectangle{\pgfqpoint{0.380943in}{0.295988in}}{\pgfqpoint{4.650000in}{0.692553in}}%
\pgfusepath{clip}%
\pgfsetbuttcap%
\pgfsetroundjoin%
\definecolor{currentfill}{rgb}{0.963937,0.914418,0.716801}%
\pgfsetfillcolor{currentfill}%
\pgfsetlinewidth{0.250937pt}%
\definecolor{currentstroke}{rgb}{1.000000,1.000000,1.000000}%
\pgfsetstrokecolor{currentstroke}%
\pgfsetdash{}{0pt}%
\pgfpathmoveto{\pgfqpoint{0.776688in}{0.889605in}}%
\pgfpathlineto{\pgfqpoint{0.875624in}{0.889605in}}%
\pgfpathlineto{\pgfqpoint{0.875624in}{0.790669in}}%
\pgfpathlineto{\pgfqpoint{0.776688in}{0.790669in}}%
\pgfpathlineto{\pgfqpoint{0.776688in}{0.889605in}}%
\pgfusepath{stroke,fill}%
\end{pgfscope}%
\begin{pgfscope}%
\pgfpathrectangle{\pgfqpoint{0.380943in}{0.295988in}}{\pgfqpoint{4.650000in}{0.692553in}}%
\pgfusepath{clip}%
\pgfsetbuttcap%
\pgfsetroundjoin%
\definecolor{currentfill}{rgb}{1.000000,1.000000,0.887120}%
\pgfsetfillcolor{currentfill}%
\pgfsetlinewidth{0.250937pt}%
\definecolor{currentstroke}{rgb}{1.000000,1.000000,1.000000}%
\pgfsetstrokecolor{currentstroke}%
\pgfsetdash{}{0pt}%
\pgfpathmoveto{\pgfqpoint{0.875624in}{0.889605in}}%
\pgfpathlineto{\pgfqpoint{0.974560in}{0.889605in}}%
\pgfpathlineto{\pgfqpoint{0.974560in}{0.790669in}}%
\pgfpathlineto{\pgfqpoint{0.875624in}{0.790669in}}%
\pgfpathlineto{\pgfqpoint{0.875624in}{0.889605in}}%
\pgfusepath{stroke,fill}%
\end{pgfscope}%
\begin{pgfscope}%
\pgfpathrectangle{\pgfqpoint{0.380943in}{0.295988in}}{\pgfqpoint{4.650000in}{0.692553in}}%
\pgfusepath{clip}%
\pgfsetbuttcap%
\pgfsetroundjoin%
\definecolor{currentfill}{rgb}{0.960892,0.932687,0.728981}%
\pgfsetfillcolor{currentfill}%
\pgfsetlinewidth{0.250937pt}%
\definecolor{currentstroke}{rgb}{1.000000,1.000000,1.000000}%
\pgfsetstrokecolor{currentstroke}%
\pgfsetdash{}{0pt}%
\pgfpathmoveto{\pgfqpoint{0.974560in}{0.889605in}}%
\pgfpathlineto{\pgfqpoint{1.073496in}{0.889605in}}%
\pgfpathlineto{\pgfqpoint{1.073496in}{0.790669in}}%
\pgfpathlineto{\pgfqpoint{0.974560in}{0.790669in}}%
\pgfpathlineto{\pgfqpoint{0.974560in}{0.889605in}}%
\pgfusepath{stroke,fill}%
\end{pgfscope}%
\begin{pgfscope}%
\pgfpathrectangle{\pgfqpoint{0.380943in}{0.295988in}}{\pgfqpoint{4.650000in}{0.692553in}}%
\pgfusepath{clip}%
\pgfsetbuttcap%
\pgfsetroundjoin%
\definecolor{currentfill}{rgb}{0.975594,0.855363,0.684691}%
\pgfsetfillcolor{currentfill}%
\pgfsetlinewidth{0.250937pt}%
\definecolor{currentstroke}{rgb}{1.000000,1.000000,1.000000}%
\pgfsetstrokecolor{currentstroke}%
\pgfsetdash{}{0pt}%
\pgfpathmoveto{\pgfqpoint{1.073496in}{0.889605in}}%
\pgfpathlineto{\pgfqpoint{1.172432in}{0.889605in}}%
\pgfpathlineto{\pgfqpoint{1.172432in}{0.790669in}}%
\pgfpathlineto{\pgfqpoint{1.073496in}{0.790669in}}%
\pgfpathlineto{\pgfqpoint{1.073496in}{0.889605in}}%
\pgfusepath{stroke,fill}%
\end{pgfscope}%
\begin{pgfscope}%
\pgfpathrectangle{\pgfqpoint{0.380943in}{0.295988in}}{\pgfqpoint{4.650000in}{0.692553in}}%
\pgfusepath{clip}%
\pgfsetbuttcap%
\pgfsetroundjoin%
\definecolor{currentfill}{rgb}{0.978316,0.963137,0.774994}%
\pgfsetfillcolor{currentfill}%
\pgfsetlinewidth{0.250937pt}%
\definecolor{currentstroke}{rgb}{1.000000,1.000000,1.000000}%
\pgfsetstrokecolor{currentstroke}%
\pgfsetdash{}{0pt}%
\pgfpathmoveto{\pgfqpoint{1.172432in}{0.889605in}}%
\pgfpathlineto{\pgfqpoint{1.271369in}{0.889605in}}%
\pgfpathlineto{\pgfqpoint{1.271369in}{0.790669in}}%
\pgfpathlineto{\pgfqpoint{1.172432in}{0.790669in}}%
\pgfpathlineto{\pgfqpoint{1.172432in}{0.889605in}}%
\pgfusepath{stroke,fill}%
\end{pgfscope}%
\begin{pgfscope}%
\pgfpathrectangle{\pgfqpoint{0.380943in}{0.295988in}}{\pgfqpoint{4.650000in}{0.692553in}}%
\pgfusepath{clip}%
\pgfsetbuttcap%
\pgfsetroundjoin%
\definecolor{currentfill}{rgb}{1.000000,0.512618,0.492826}%
\pgfsetfillcolor{currentfill}%
\pgfsetlinewidth{0.250937pt}%
\definecolor{currentstroke}{rgb}{1.000000,1.000000,1.000000}%
\pgfsetstrokecolor{currentstroke}%
\pgfsetdash{}{0pt}%
\pgfpathmoveto{\pgfqpoint{1.271369in}{0.889605in}}%
\pgfpathlineto{\pgfqpoint{1.370305in}{0.889605in}}%
\pgfpathlineto{\pgfqpoint{1.370305in}{0.790669in}}%
\pgfpathlineto{\pgfqpoint{1.271369in}{0.790669in}}%
\pgfpathlineto{\pgfqpoint{1.271369in}{0.889605in}}%
\pgfusepath{stroke,fill}%
\end{pgfscope}%
\begin{pgfscope}%
\pgfpathrectangle{\pgfqpoint{0.380943in}{0.295988in}}{\pgfqpoint{4.650000in}{0.692553in}}%
\pgfusepath{clip}%
\pgfsetbuttcap%
\pgfsetroundjoin%
\definecolor{currentfill}{rgb}{0.988604,0.796863,0.633449}%
\pgfsetfillcolor{currentfill}%
\pgfsetlinewidth{0.250937pt}%
\definecolor{currentstroke}{rgb}{1.000000,1.000000,1.000000}%
\pgfsetstrokecolor{currentstroke}%
\pgfsetdash{}{0pt}%
\pgfpathmoveto{\pgfqpoint{1.370305in}{0.889605in}}%
\pgfpathlineto{\pgfqpoint{1.469241in}{0.889605in}}%
\pgfpathlineto{\pgfqpoint{1.469241in}{0.790669in}}%
\pgfpathlineto{\pgfqpoint{1.370305in}{0.790669in}}%
\pgfpathlineto{\pgfqpoint{1.370305in}{0.889605in}}%
\pgfusepath{stroke,fill}%
\end{pgfscope}%
\begin{pgfscope}%
\pgfpathrectangle{\pgfqpoint{0.380943in}{0.295988in}}{\pgfqpoint{4.650000in}{0.692553in}}%
\pgfusepath{clip}%
\pgfsetbuttcap%
\pgfsetroundjoin%
\definecolor{currentfill}{rgb}{0.988604,0.796863,0.633449}%
\pgfsetfillcolor{currentfill}%
\pgfsetlinewidth{0.250937pt}%
\definecolor{currentstroke}{rgb}{1.000000,1.000000,1.000000}%
\pgfsetstrokecolor{currentstroke}%
\pgfsetdash{}{0pt}%
\pgfpathmoveto{\pgfqpoint{1.469241in}{0.889605in}}%
\pgfpathlineto{\pgfqpoint{1.568177in}{0.889605in}}%
\pgfpathlineto{\pgfqpoint{1.568177in}{0.790669in}}%
\pgfpathlineto{\pgfqpoint{1.469241in}{0.790669in}}%
\pgfpathlineto{\pgfqpoint{1.469241in}{0.889605in}}%
\pgfusepath{stroke,fill}%
\end{pgfscope}%
\begin{pgfscope}%
\pgfpathrectangle{\pgfqpoint{0.380943in}{0.295988in}}{\pgfqpoint{4.650000in}{0.692553in}}%
\pgfusepath{clip}%
\pgfsetbuttcap%
\pgfsetroundjoin%
\definecolor{currentfill}{rgb}{1.000000,0.588312,0.523952}%
\pgfsetfillcolor{currentfill}%
\pgfsetlinewidth{0.250937pt}%
\definecolor{currentstroke}{rgb}{1.000000,1.000000,1.000000}%
\pgfsetstrokecolor{currentstroke}%
\pgfsetdash{}{0pt}%
\pgfpathmoveto{\pgfqpoint{1.568177in}{0.889605in}}%
\pgfpathlineto{\pgfqpoint{1.667113in}{0.889605in}}%
\pgfpathlineto{\pgfqpoint{1.667113in}{0.790669in}}%
\pgfpathlineto{\pgfqpoint{1.568177in}{0.790669in}}%
\pgfpathlineto{\pgfqpoint{1.568177in}{0.889605in}}%
\pgfusepath{stroke,fill}%
\end{pgfscope}%
\begin{pgfscope}%
\pgfpathrectangle{\pgfqpoint{0.380943in}{0.295988in}}{\pgfqpoint{4.650000in}{0.692553in}}%
\pgfusepath{clip}%
\pgfsetbuttcap%
\pgfsetroundjoin%
\definecolor{currentfill}{rgb}{0.999446,0.645767,0.548927}%
\pgfsetfillcolor{currentfill}%
\pgfsetlinewidth{0.250937pt}%
\definecolor{currentstroke}{rgb}{1.000000,1.000000,1.000000}%
\pgfsetstrokecolor{currentstroke}%
\pgfsetdash{}{0pt}%
\pgfpathmoveto{\pgfqpoint{1.667113in}{0.889605in}}%
\pgfpathlineto{\pgfqpoint{1.766049in}{0.889605in}}%
\pgfpathlineto{\pgfqpoint{1.766049in}{0.790669in}}%
\pgfpathlineto{\pgfqpoint{1.667113in}{0.790669in}}%
\pgfpathlineto{\pgfqpoint{1.667113in}{0.889605in}}%
\pgfusepath{stroke,fill}%
\end{pgfscope}%
\begin{pgfscope}%
\pgfpathrectangle{\pgfqpoint{0.380943in}{0.295988in}}{\pgfqpoint{4.650000in}{0.692553in}}%
\pgfusepath{clip}%
\pgfsetbuttcap%
\pgfsetroundjoin%
\definecolor{currentfill}{rgb}{0.998601,0.667759,0.560769}%
\pgfsetfillcolor{currentfill}%
\pgfsetlinewidth{0.250937pt}%
\definecolor{currentstroke}{rgb}{1.000000,1.000000,1.000000}%
\pgfsetstrokecolor{currentstroke}%
\pgfsetdash{}{0pt}%
\pgfpathmoveto{\pgfqpoint{1.766049in}{0.889605in}}%
\pgfpathlineto{\pgfqpoint{1.864986in}{0.889605in}}%
\pgfpathlineto{\pgfqpoint{1.864986in}{0.790669in}}%
\pgfpathlineto{\pgfqpoint{1.766049in}{0.790669in}}%
\pgfpathlineto{\pgfqpoint{1.766049in}{0.889605in}}%
\pgfusepath{stroke,fill}%
\end{pgfscope}%
\begin{pgfscope}%
\pgfpathrectangle{\pgfqpoint{0.380943in}{0.295988in}}{\pgfqpoint{4.650000in}{0.692553in}}%
\pgfusepath{clip}%
\pgfsetbuttcap%
\pgfsetroundjoin%
\definecolor{currentfill}{rgb}{0.996401,0.724937,0.591557}%
\pgfsetfillcolor{currentfill}%
\pgfsetlinewidth{0.250937pt}%
\definecolor{currentstroke}{rgb}{1.000000,1.000000,1.000000}%
\pgfsetstrokecolor{currentstroke}%
\pgfsetdash{}{0pt}%
\pgfpathmoveto{\pgfqpoint{1.864986in}{0.889605in}}%
\pgfpathlineto{\pgfqpoint{1.963922in}{0.889605in}}%
\pgfpathlineto{\pgfqpoint{1.963922in}{0.790669in}}%
\pgfpathlineto{\pgfqpoint{1.864986in}{0.790669in}}%
\pgfpathlineto{\pgfqpoint{1.864986in}{0.889605in}}%
\pgfusepath{stroke,fill}%
\end{pgfscope}%
\begin{pgfscope}%
\pgfpathrectangle{\pgfqpoint{0.380943in}{0.295988in}}{\pgfqpoint{4.650000in}{0.692553in}}%
\pgfusepath{clip}%
\pgfsetbuttcap%
\pgfsetroundjoin%
\definecolor{currentfill}{rgb}{0.989619,0.788235,0.628374}%
\pgfsetfillcolor{currentfill}%
\pgfsetlinewidth{0.250937pt}%
\definecolor{currentstroke}{rgb}{1.000000,1.000000,1.000000}%
\pgfsetstrokecolor{currentstroke}%
\pgfsetdash{}{0pt}%
\pgfpathmoveto{\pgfqpoint{1.963922in}{0.889605in}}%
\pgfpathlineto{\pgfqpoint{2.062858in}{0.889605in}}%
\pgfpathlineto{\pgfqpoint{2.062858in}{0.790669in}}%
\pgfpathlineto{\pgfqpoint{1.963922in}{0.790669in}}%
\pgfpathlineto{\pgfqpoint{1.963922in}{0.889605in}}%
\pgfusepath{stroke,fill}%
\end{pgfscope}%
\begin{pgfscope}%
\pgfpathrectangle{\pgfqpoint{0.380943in}{0.295988in}}{\pgfqpoint{4.650000in}{0.692553in}}%
\pgfusepath{clip}%
\pgfsetbuttcap%
\pgfsetroundjoin%
\definecolor{currentfill}{rgb}{0.998939,0.658962,0.556032}%
\pgfsetfillcolor{currentfill}%
\pgfsetlinewidth{0.250937pt}%
\definecolor{currentstroke}{rgb}{1.000000,1.000000,1.000000}%
\pgfsetstrokecolor{currentstroke}%
\pgfsetdash{}{0pt}%
\pgfpathmoveto{\pgfqpoint{2.062858in}{0.889605in}}%
\pgfpathlineto{\pgfqpoint{2.161794in}{0.889605in}}%
\pgfpathlineto{\pgfqpoint{2.161794in}{0.790669in}}%
\pgfpathlineto{\pgfqpoint{2.062858in}{0.790669in}}%
\pgfpathlineto{\pgfqpoint{2.062858in}{0.889605in}}%
\pgfusepath{stroke,fill}%
\end{pgfscope}%
\begin{pgfscope}%
\pgfpathrectangle{\pgfqpoint{0.380943in}{0.295988in}}{\pgfqpoint{4.650000in}{0.692553in}}%
\pgfusepath{clip}%
\pgfsetbuttcap%
\pgfsetroundjoin%
\definecolor{currentfill}{rgb}{0.999785,0.636970,0.544191}%
\pgfsetfillcolor{currentfill}%
\pgfsetlinewidth{0.250937pt}%
\definecolor{currentstroke}{rgb}{1.000000,1.000000,1.000000}%
\pgfsetstrokecolor{currentstroke}%
\pgfsetdash{}{0pt}%
\pgfpathmoveto{\pgfqpoint{2.161794in}{0.889605in}}%
\pgfpathlineto{\pgfqpoint{2.260730in}{0.889605in}}%
\pgfpathlineto{\pgfqpoint{2.260730in}{0.790669in}}%
\pgfpathlineto{\pgfqpoint{2.161794in}{0.790669in}}%
\pgfpathlineto{\pgfqpoint{2.161794in}{0.889605in}}%
\pgfusepath{stroke,fill}%
\end{pgfscope}%
\begin{pgfscope}%
\pgfpathrectangle{\pgfqpoint{0.380943in}{0.295988in}}{\pgfqpoint{4.650000in}{0.692553in}}%
\pgfusepath{clip}%
\pgfsetbuttcap%
\pgfsetroundjoin%
\definecolor{currentfill}{rgb}{0.995709,0.736471,0.597924}%
\pgfsetfillcolor{currentfill}%
\pgfsetlinewidth{0.250937pt}%
\definecolor{currentstroke}{rgb}{1.000000,1.000000,1.000000}%
\pgfsetstrokecolor{currentstroke}%
\pgfsetdash{}{0pt}%
\pgfpathmoveto{\pgfqpoint{2.260730in}{0.889605in}}%
\pgfpathlineto{\pgfqpoint{2.359666in}{0.889605in}}%
\pgfpathlineto{\pgfqpoint{2.359666in}{0.790669in}}%
\pgfpathlineto{\pgfqpoint{2.260730in}{0.790669in}}%
\pgfpathlineto{\pgfqpoint{2.260730in}{0.889605in}}%
\pgfusepath{stroke,fill}%
\end{pgfscope}%
\begin{pgfscope}%
\pgfpathrectangle{\pgfqpoint{0.380943in}{0.295988in}}{\pgfqpoint{4.650000in}{0.692553in}}%
\pgfusepath{clip}%
\pgfsetbuttcap%
\pgfsetroundjoin%
\definecolor{currentfill}{rgb}{0.999785,0.636970,0.544191}%
\pgfsetfillcolor{currentfill}%
\pgfsetlinewidth{0.250937pt}%
\definecolor{currentstroke}{rgb}{1.000000,1.000000,1.000000}%
\pgfsetstrokecolor{currentstroke}%
\pgfsetdash{}{0pt}%
\pgfpathmoveto{\pgfqpoint{2.359666in}{0.889605in}}%
\pgfpathlineto{\pgfqpoint{2.458603in}{0.889605in}}%
\pgfpathlineto{\pgfqpoint{2.458603in}{0.790669in}}%
\pgfpathlineto{\pgfqpoint{2.359666in}{0.790669in}}%
\pgfpathlineto{\pgfqpoint{2.359666in}{0.889605in}}%
\pgfusepath{stroke,fill}%
\end{pgfscope}%
\begin{pgfscope}%
\pgfpathrectangle{\pgfqpoint{0.380943in}{0.295988in}}{\pgfqpoint{4.650000in}{0.692553in}}%
\pgfusepath{clip}%
\pgfsetbuttcap%
\pgfsetroundjoin%
\definecolor{currentfill}{rgb}{0.998093,0.680953,0.567874}%
\pgfsetfillcolor{currentfill}%
\pgfsetlinewidth{0.250937pt}%
\definecolor{currentstroke}{rgb}{1.000000,1.000000,1.000000}%
\pgfsetstrokecolor{currentstroke}%
\pgfsetdash{}{0pt}%
\pgfpathmoveto{\pgfqpoint{2.458603in}{0.889605in}}%
\pgfpathlineto{\pgfqpoint{2.557539in}{0.889605in}}%
\pgfpathlineto{\pgfqpoint{2.557539in}{0.790669in}}%
\pgfpathlineto{\pgfqpoint{2.458603in}{0.790669in}}%
\pgfpathlineto{\pgfqpoint{2.458603in}{0.889605in}}%
\pgfusepath{stroke,fill}%
\end{pgfscope}%
\begin{pgfscope}%
\pgfpathrectangle{\pgfqpoint{0.380943in}{0.295988in}}{\pgfqpoint{4.650000in}{0.692553in}}%
\pgfusepath{clip}%
\pgfsetbuttcap%
\pgfsetroundjoin%
\definecolor{currentfill}{rgb}{1.000000,0.581546,0.521246}%
\pgfsetfillcolor{currentfill}%
\pgfsetlinewidth{0.250937pt}%
\definecolor{currentstroke}{rgb}{1.000000,1.000000,1.000000}%
\pgfsetstrokecolor{currentstroke}%
\pgfsetdash{}{0pt}%
\pgfpathmoveto{\pgfqpoint{2.557539in}{0.889605in}}%
\pgfpathlineto{\pgfqpoint{2.656475in}{0.889605in}}%
\pgfpathlineto{\pgfqpoint{2.656475in}{0.790669in}}%
\pgfpathlineto{\pgfqpoint{2.557539in}{0.790669in}}%
\pgfpathlineto{\pgfqpoint{2.557539in}{0.889605in}}%
\pgfusepath{stroke,fill}%
\end{pgfscope}%
\begin{pgfscope}%
\pgfpathrectangle{\pgfqpoint{0.380943in}{0.295988in}}{\pgfqpoint{4.650000in}{0.692553in}}%
\pgfusepath{clip}%
\pgfsetbuttcap%
\pgfsetroundjoin%
\definecolor{currentfill}{rgb}{0.991311,0.773856,0.619915}%
\pgfsetfillcolor{currentfill}%
\pgfsetlinewidth{0.250937pt}%
\definecolor{currentstroke}{rgb}{1.000000,1.000000,1.000000}%
\pgfsetstrokecolor{currentstroke}%
\pgfsetdash{}{0pt}%
\pgfpathmoveto{\pgfqpoint{2.656475in}{0.889605in}}%
\pgfpathlineto{\pgfqpoint{2.755411in}{0.889605in}}%
\pgfpathlineto{\pgfqpoint{2.755411in}{0.790669in}}%
\pgfpathlineto{\pgfqpoint{2.656475in}{0.790669in}}%
\pgfpathlineto{\pgfqpoint{2.656475in}{0.889605in}}%
\pgfusepath{stroke,fill}%
\end{pgfscope}%
\begin{pgfscope}%
\pgfpathrectangle{\pgfqpoint{0.380943in}{0.295988in}}{\pgfqpoint{4.650000in}{0.692553in}}%
\pgfusepath{clip}%
\pgfsetbuttcap%
\pgfsetroundjoin%
\definecolor{currentfill}{rgb}{0.988604,0.796863,0.633449}%
\pgfsetfillcolor{currentfill}%
\pgfsetlinewidth{0.250937pt}%
\definecolor{currentstroke}{rgb}{1.000000,1.000000,1.000000}%
\pgfsetstrokecolor{currentstroke}%
\pgfsetdash{}{0pt}%
\pgfpathmoveto{\pgfqpoint{2.755411in}{0.889605in}}%
\pgfpathlineto{\pgfqpoint{2.854347in}{0.889605in}}%
\pgfpathlineto{\pgfqpoint{2.854347in}{0.790669in}}%
\pgfpathlineto{\pgfqpoint{2.755411in}{0.790669in}}%
\pgfpathlineto{\pgfqpoint{2.755411in}{0.889605in}}%
\pgfusepath{stroke,fill}%
\end{pgfscope}%
\begin{pgfscope}%
\pgfpathrectangle{\pgfqpoint{0.380943in}{0.295988in}}{\pgfqpoint{4.650000in}{0.692553in}}%
\pgfusepath{clip}%
\pgfsetbuttcap%
\pgfsetroundjoin%
\definecolor{currentfill}{rgb}{0.994694,0.745098,0.602999}%
\pgfsetfillcolor{currentfill}%
\pgfsetlinewidth{0.250937pt}%
\definecolor{currentstroke}{rgb}{1.000000,1.000000,1.000000}%
\pgfsetstrokecolor{currentstroke}%
\pgfsetdash{}{0pt}%
\pgfpathmoveto{\pgfqpoint{2.854347in}{0.889605in}}%
\pgfpathlineto{\pgfqpoint{2.953283in}{0.889605in}}%
\pgfpathlineto{\pgfqpoint{2.953283in}{0.790669in}}%
\pgfpathlineto{\pgfqpoint{2.854347in}{0.790669in}}%
\pgfpathlineto{\pgfqpoint{2.854347in}{0.889605in}}%
\pgfusepath{stroke,fill}%
\end{pgfscope}%
\begin{pgfscope}%
\pgfpathrectangle{\pgfqpoint{0.380943in}{0.295988in}}{\pgfqpoint{4.650000in}{0.692553in}}%
\pgfusepath{clip}%
\pgfsetbuttcap%
\pgfsetroundjoin%
\definecolor{currentfill}{rgb}{0.982191,0.826190,0.659469}%
\pgfsetfillcolor{currentfill}%
\pgfsetlinewidth{0.250937pt}%
\definecolor{currentstroke}{rgb}{1.000000,1.000000,1.000000}%
\pgfsetstrokecolor{currentstroke}%
\pgfsetdash{}{0pt}%
\pgfpathmoveto{\pgfqpoint{2.953283in}{0.889605in}}%
\pgfpathlineto{\pgfqpoint{3.052220in}{0.889605in}}%
\pgfpathlineto{\pgfqpoint{3.052220in}{0.790669in}}%
\pgfpathlineto{\pgfqpoint{2.953283in}{0.790669in}}%
\pgfpathlineto{\pgfqpoint{2.953283in}{0.889605in}}%
\pgfusepath{stroke,fill}%
\end{pgfscope}%
\begin{pgfscope}%
\pgfpathrectangle{\pgfqpoint{0.380943in}{0.295988in}}{\pgfqpoint{4.650000in}{0.692553in}}%
\pgfusepath{clip}%
\pgfsetbuttcap%
\pgfsetroundjoin%
\definecolor{currentfill}{rgb}{0.999446,0.645767,0.548927}%
\pgfsetfillcolor{currentfill}%
\pgfsetlinewidth{0.250937pt}%
\definecolor{currentstroke}{rgb}{1.000000,1.000000,1.000000}%
\pgfsetstrokecolor{currentstroke}%
\pgfsetdash{}{0pt}%
\pgfpathmoveto{\pgfqpoint{3.052220in}{0.889605in}}%
\pgfpathlineto{\pgfqpoint{3.151156in}{0.889605in}}%
\pgfpathlineto{\pgfqpoint{3.151156in}{0.790669in}}%
\pgfpathlineto{\pgfqpoint{3.052220in}{0.790669in}}%
\pgfpathlineto{\pgfqpoint{3.052220in}{0.889605in}}%
\pgfusepath{stroke,fill}%
\end{pgfscope}%
\begin{pgfscope}%
\pgfpathrectangle{\pgfqpoint{0.380943in}{0.295988in}}{\pgfqpoint{4.650000in}{0.692553in}}%
\pgfusepath{clip}%
\pgfsetbuttcap%
\pgfsetroundjoin%
\definecolor{currentfill}{rgb}{0.994694,0.745098,0.602999}%
\pgfsetfillcolor{currentfill}%
\pgfsetlinewidth{0.250937pt}%
\definecolor{currentstroke}{rgb}{1.000000,1.000000,1.000000}%
\pgfsetstrokecolor{currentstroke}%
\pgfsetdash{}{0pt}%
\pgfpathmoveto{\pgfqpoint{3.151156in}{0.889605in}}%
\pgfpathlineto{\pgfqpoint{3.250092in}{0.889605in}}%
\pgfpathlineto{\pgfqpoint{3.250092in}{0.790669in}}%
\pgfpathlineto{\pgfqpoint{3.151156in}{0.790669in}}%
\pgfpathlineto{\pgfqpoint{3.151156in}{0.889605in}}%
\pgfusepath{stroke,fill}%
\end{pgfscope}%
\begin{pgfscope}%
\pgfpathrectangle{\pgfqpoint{0.380943in}{0.295988in}}{\pgfqpoint{4.650000in}{0.692553in}}%
\pgfusepath{clip}%
\pgfsetbuttcap%
\pgfsetroundjoin%
\definecolor{currentfill}{rgb}{0.996401,0.724937,0.591557}%
\pgfsetfillcolor{currentfill}%
\pgfsetlinewidth{0.250937pt}%
\definecolor{currentstroke}{rgb}{1.000000,1.000000,1.000000}%
\pgfsetstrokecolor{currentstroke}%
\pgfsetdash{}{0pt}%
\pgfpathmoveto{\pgfqpoint{3.250092in}{0.889605in}}%
\pgfpathlineto{\pgfqpoint{3.349028in}{0.889605in}}%
\pgfpathlineto{\pgfqpoint{3.349028in}{0.790669in}}%
\pgfpathlineto{\pgfqpoint{3.250092in}{0.790669in}}%
\pgfpathlineto{\pgfqpoint{3.250092in}{0.889605in}}%
\pgfusepath{stroke,fill}%
\end{pgfscope}%
\begin{pgfscope}%
\pgfpathrectangle{\pgfqpoint{0.380943in}{0.295988in}}{\pgfqpoint{4.650000in}{0.692553in}}%
\pgfusepath{clip}%
\pgfsetbuttcap%
\pgfsetroundjoin%
\definecolor{currentfill}{rgb}{0.956171,0.434602,0.434602}%
\pgfsetfillcolor{currentfill}%
\pgfsetlinewidth{0.250937pt}%
\definecolor{currentstroke}{rgb}{1.000000,1.000000,1.000000}%
\pgfsetstrokecolor{currentstroke}%
\pgfsetdash{}{0pt}%
\pgfpathmoveto{\pgfqpoint{3.349028in}{0.889605in}}%
\pgfpathlineto{\pgfqpoint{3.447964in}{0.889605in}}%
\pgfpathlineto{\pgfqpoint{3.447964in}{0.790669in}}%
\pgfpathlineto{\pgfqpoint{3.349028in}{0.790669in}}%
\pgfpathlineto{\pgfqpoint{3.349028in}{0.889605in}}%
\pgfusepath{stroke,fill}%
\end{pgfscope}%
\begin{pgfscope}%
\pgfpathrectangle{\pgfqpoint{0.380943in}{0.295988in}}{\pgfqpoint{4.650000in}{0.692553in}}%
\pgfusepath{clip}%
\pgfsetbuttcap%
\pgfsetroundjoin%
\definecolor{currentfill}{rgb}{1.000000,0.528689,0.499592}%
\pgfsetfillcolor{currentfill}%
\pgfsetlinewidth{0.250937pt}%
\definecolor{currentstroke}{rgb}{1.000000,1.000000,1.000000}%
\pgfsetstrokecolor{currentstroke}%
\pgfsetdash{}{0pt}%
\pgfpathmoveto{\pgfqpoint{3.447964in}{0.889605in}}%
\pgfpathlineto{\pgfqpoint{3.546901in}{0.889605in}}%
\pgfpathlineto{\pgfqpoint{3.546901in}{0.790669in}}%
\pgfpathlineto{\pgfqpoint{3.447964in}{0.790669in}}%
\pgfpathlineto{\pgfqpoint{3.447964in}{0.889605in}}%
\pgfusepath{stroke,fill}%
\end{pgfscope}%
\begin{pgfscope}%
\pgfpathrectangle{\pgfqpoint{0.380943in}{0.295988in}}{\pgfqpoint{4.650000in}{0.692553in}}%
\pgfusepath{clip}%
\pgfsetbuttcap%
\pgfsetroundjoin%
\definecolor{currentfill}{rgb}{0.997247,0.702945,0.579715}%
\pgfsetfillcolor{currentfill}%
\pgfsetlinewidth{0.250937pt}%
\definecolor{currentstroke}{rgb}{1.000000,1.000000,1.000000}%
\pgfsetstrokecolor{currentstroke}%
\pgfsetdash{}{0pt}%
\pgfpathmoveto{\pgfqpoint{3.546901in}{0.889605in}}%
\pgfpathlineto{\pgfqpoint{3.645837in}{0.889605in}}%
\pgfpathlineto{\pgfqpoint{3.645837in}{0.790669in}}%
\pgfpathlineto{\pgfqpoint{3.546901in}{0.790669in}}%
\pgfpathlineto{\pgfqpoint{3.546901in}{0.889605in}}%
\pgfusepath{stroke,fill}%
\end{pgfscope}%
\begin{pgfscope}%
\pgfpathrectangle{\pgfqpoint{0.380943in}{0.295988in}}{\pgfqpoint{4.650000in}{0.692553in}}%
\pgfusepath{clip}%
\pgfsetbuttcap%
\pgfsetroundjoin%
\definecolor{currentfill}{rgb}{0.995709,0.736471,0.597924}%
\pgfsetfillcolor{currentfill}%
\pgfsetlinewidth{0.250937pt}%
\definecolor{currentstroke}{rgb}{1.000000,1.000000,1.000000}%
\pgfsetstrokecolor{currentstroke}%
\pgfsetdash{}{0pt}%
\pgfpathmoveto{\pgfqpoint{3.645837in}{0.889605in}}%
\pgfpathlineto{\pgfqpoint{3.744773in}{0.889605in}}%
\pgfpathlineto{\pgfqpoint{3.744773in}{0.790669in}}%
\pgfpathlineto{\pgfqpoint{3.645837in}{0.790669in}}%
\pgfpathlineto{\pgfqpoint{3.645837in}{0.889605in}}%
\pgfusepath{stroke,fill}%
\end{pgfscope}%
\begin{pgfscope}%
\pgfpathrectangle{\pgfqpoint{0.380943in}{0.295988in}}{\pgfqpoint{4.650000in}{0.692553in}}%
\pgfusepath{clip}%
\pgfsetbuttcap%
\pgfsetroundjoin%
\definecolor{currentfill}{rgb}{0.998939,0.658962,0.556032}%
\pgfsetfillcolor{currentfill}%
\pgfsetlinewidth{0.250937pt}%
\definecolor{currentstroke}{rgb}{1.000000,1.000000,1.000000}%
\pgfsetstrokecolor{currentstroke}%
\pgfsetdash{}{0pt}%
\pgfpathmoveto{\pgfqpoint{3.744773in}{0.889605in}}%
\pgfpathlineto{\pgfqpoint{3.843709in}{0.889605in}}%
\pgfpathlineto{\pgfqpoint{3.843709in}{0.790669in}}%
\pgfpathlineto{\pgfqpoint{3.744773in}{0.790669in}}%
\pgfpathlineto{\pgfqpoint{3.744773in}{0.889605in}}%
\pgfusepath{stroke,fill}%
\end{pgfscope}%
\begin{pgfscope}%
\pgfpathrectangle{\pgfqpoint{0.380943in}{0.295988in}}{\pgfqpoint{4.650000in}{0.692553in}}%
\pgfusepath{clip}%
\pgfsetbuttcap%
\pgfsetroundjoin%
\definecolor{currentfill}{rgb}{1.000000,0.522261,0.496886}%
\pgfsetfillcolor{currentfill}%
\pgfsetlinewidth{0.250937pt}%
\definecolor{currentstroke}{rgb}{1.000000,1.000000,1.000000}%
\pgfsetstrokecolor{currentstroke}%
\pgfsetdash{}{0pt}%
\pgfpathmoveto{\pgfqpoint{3.843709in}{0.889605in}}%
\pgfpathlineto{\pgfqpoint{3.942645in}{0.889605in}}%
\pgfpathlineto{\pgfqpoint{3.942645in}{0.790669in}}%
\pgfpathlineto{\pgfqpoint{3.843709in}{0.790669in}}%
\pgfpathlineto{\pgfqpoint{3.843709in}{0.889605in}}%
\pgfusepath{stroke,fill}%
\end{pgfscope}%
\begin{pgfscope}%
\pgfpathrectangle{\pgfqpoint{0.380943in}{0.295988in}}{\pgfqpoint{4.650000in}{0.692553in}}%
\pgfusepath{clip}%
\pgfsetbuttcap%
\pgfsetroundjoin%
\definecolor{currentfill}{rgb}{0.998939,0.658962,0.556032}%
\pgfsetfillcolor{currentfill}%
\pgfsetlinewidth{0.250937pt}%
\definecolor{currentstroke}{rgb}{1.000000,1.000000,1.000000}%
\pgfsetstrokecolor{currentstroke}%
\pgfsetdash{}{0pt}%
\pgfpathmoveto{\pgfqpoint{3.942645in}{0.889605in}}%
\pgfpathlineto{\pgfqpoint{4.041581in}{0.889605in}}%
\pgfpathlineto{\pgfqpoint{4.041581in}{0.790669in}}%
\pgfpathlineto{\pgfqpoint{3.942645in}{0.790669in}}%
\pgfpathlineto{\pgfqpoint{3.942645in}{0.889605in}}%
\pgfusepath{stroke,fill}%
\end{pgfscope}%
\begin{pgfscope}%
\pgfpathrectangle{\pgfqpoint{0.380943in}{0.295988in}}{\pgfqpoint{4.650000in}{0.692553in}}%
\pgfusepath{clip}%
\pgfsetbuttcap%
\pgfsetroundjoin%
\definecolor{currentfill}{rgb}{0.995709,0.736471,0.597924}%
\pgfsetfillcolor{currentfill}%
\pgfsetlinewidth{0.250937pt}%
\definecolor{currentstroke}{rgb}{1.000000,1.000000,1.000000}%
\pgfsetstrokecolor{currentstroke}%
\pgfsetdash{}{0pt}%
\pgfpathmoveto{\pgfqpoint{4.041581in}{0.889605in}}%
\pgfpathlineto{\pgfqpoint{4.140518in}{0.889605in}}%
\pgfpathlineto{\pgfqpoint{4.140518in}{0.790669in}}%
\pgfpathlineto{\pgfqpoint{4.041581in}{0.790669in}}%
\pgfpathlineto{\pgfqpoint{4.041581in}{0.889605in}}%
\pgfusepath{stroke,fill}%
\end{pgfscope}%
\begin{pgfscope}%
\pgfpathrectangle{\pgfqpoint{0.380943in}{0.295988in}}{\pgfqpoint{4.650000in}{0.692553in}}%
\pgfusepath{clip}%
\pgfsetbuttcap%
\pgfsetroundjoin%
\definecolor{currentfill}{rgb}{0.997586,0.694148,0.574979}%
\pgfsetfillcolor{currentfill}%
\pgfsetlinewidth{0.250937pt}%
\definecolor{currentstroke}{rgb}{1.000000,1.000000,1.000000}%
\pgfsetstrokecolor{currentstroke}%
\pgfsetdash{}{0pt}%
\pgfpathmoveto{\pgfqpoint{4.140518in}{0.889605in}}%
\pgfpathlineto{\pgfqpoint{4.239454in}{0.889605in}}%
\pgfpathlineto{\pgfqpoint{4.239454in}{0.790669in}}%
\pgfpathlineto{\pgfqpoint{4.140518in}{0.790669in}}%
\pgfpathlineto{\pgfqpoint{4.140518in}{0.889605in}}%
\pgfusepath{stroke,fill}%
\end{pgfscope}%
\begin{pgfscope}%
\pgfpathrectangle{\pgfqpoint{0.380943in}{0.295988in}}{\pgfqpoint{4.650000in}{0.692553in}}%
\pgfusepath{clip}%
\pgfsetbuttcap%
\pgfsetroundjoin%
\definecolor{currentfill}{rgb}{1.000000,0.571396,0.517186}%
\pgfsetfillcolor{currentfill}%
\pgfsetlinewidth{0.250937pt}%
\definecolor{currentstroke}{rgb}{1.000000,1.000000,1.000000}%
\pgfsetstrokecolor{currentstroke}%
\pgfsetdash{}{0pt}%
\pgfpathmoveto{\pgfqpoint{4.239454in}{0.889605in}}%
\pgfpathlineto{\pgfqpoint{4.338390in}{0.889605in}}%
\pgfpathlineto{\pgfqpoint{4.338390in}{0.790669in}}%
\pgfpathlineto{\pgfqpoint{4.239454in}{0.790669in}}%
\pgfpathlineto{\pgfqpoint{4.239454in}{0.889605in}}%
\pgfusepath{stroke,fill}%
\end{pgfscope}%
\begin{pgfscope}%
\pgfpathrectangle{\pgfqpoint{0.380943in}{0.295988in}}{\pgfqpoint{4.650000in}{0.692553in}}%
\pgfusepath{clip}%
\pgfsetbuttcap%
\pgfsetroundjoin%
\definecolor{currentfill}{rgb}{0.973057,0.868051,0.691457}%
\pgfsetfillcolor{currentfill}%
\pgfsetlinewidth{0.250937pt}%
\definecolor{currentstroke}{rgb}{1.000000,1.000000,1.000000}%
\pgfsetstrokecolor{currentstroke}%
\pgfsetdash{}{0pt}%
\pgfpathmoveto{\pgfqpoint{4.338390in}{0.889605in}}%
\pgfpathlineto{\pgfqpoint{4.437326in}{0.889605in}}%
\pgfpathlineto{\pgfqpoint{4.437326in}{0.790669in}}%
\pgfpathlineto{\pgfqpoint{4.338390in}{0.790669in}}%
\pgfpathlineto{\pgfqpoint{4.338390in}{0.889605in}}%
\pgfusepath{stroke,fill}%
\end{pgfscope}%
\begin{pgfscope}%
\pgfpathrectangle{\pgfqpoint{0.380943in}{0.295988in}}{\pgfqpoint{4.650000in}{0.692553in}}%
\pgfusepath{clip}%
\pgfsetbuttcap%
\pgfsetroundjoin%
\definecolor{currentfill}{rgb}{0.970012,0.883276,0.699577}%
\pgfsetfillcolor{currentfill}%
\pgfsetlinewidth{0.250937pt}%
\definecolor{currentstroke}{rgb}{1.000000,1.000000,1.000000}%
\pgfsetstrokecolor{currentstroke}%
\pgfsetdash{}{0pt}%
\pgfpathmoveto{\pgfqpoint{4.437326in}{0.889605in}}%
\pgfpathlineto{\pgfqpoint{4.536262in}{0.889605in}}%
\pgfpathlineto{\pgfqpoint{4.536262in}{0.790669in}}%
\pgfpathlineto{\pgfqpoint{4.437326in}{0.790669in}}%
\pgfpathlineto{\pgfqpoint{4.437326in}{0.889605in}}%
\pgfusepath{stroke,fill}%
\end{pgfscope}%
\begin{pgfscope}%
\pgfpathrectangle{\pgfqpoint{0.380943in}{0.295988in}}{\pgfqpoint{4.650000in}{0.692553in}}%
\pgfusepath{clip}%
\pgfsetbuttcap%
\pgfsetroundjoin%
\definecolor{currentfill}{rgb}{0.973057,0.868051,0.691457}%
\pgfsetfillcolor{currentfill}%
\pgfsetlinewidth{0.250937pt}%
\definecolor{currentstroke}{rgb}{1.000000,1.000000,1.000000}%
\pgfsetstrokecolor{currentstroke}%
\pgfsetdash{}{0pt}%
\pgfpathmoveto{\pgfqpoint{4.536262in}{0.889605in}}%
\pgfpathlineto{\pgfqpoint{4.635198in}{0.889605in}}%
\pgfpathlineto{\pgfqpoint{4.635198in}{0.790669in}}%
\pgfpathlineto{\pgfqpoint{4.536262in}{0.790669in}}%
\pgfpathlineto{\pgfqpoint{4.536262in}{0.889605in}}%
\pgfusepath{stroke,fill}%
\end{pgfscope}%
\begin{pgfscope}%
\pgfpathrectangle{\pgfqpoint{0.380943in}{0.295988in}}{\pgfqpoint{4.650000in}{0.692553in}}%
\pgfusepath{clip}%
\pgfsetbuttcap%
\pgfsetroundjoin%
\definecolor{currentfill}{rgb}{0.977116,0.848181,0.679769}%
\pgfsetfillcolor{currentfill}%
\pgfsetlinewidth{0.250937pt}%
\definecolor{currentstroke}{rgb}{1.000000,1.000000,1.000000}%
\pgfsetstrokecolor{currentstroke}%
\pgfsetdash{}{0pt}%
\pgfpathmoveto{\pgfqpoint{4.635198in}{0.889605in}}%
\pgfpathlineto{\pgfqpoint{4.734135in}{0.889605in}}%
\pgfpathlineto{\pgfqpoint{4.734135in}{0.790669in}}%
\pgfpathlineto{\pgfqpoint{4.635198in}{0.790669in}}%
\pgfpathlineto{\pgfqpoint{4.635198in}{0.889605in}}%
\pgfusepath{stroke,fill}%
\end{pgfscope}%
\begin{pgfscope}%
\pgfpathrectangle{\pgfqpoint{0.380943in}{0.295988in}}{\pgfqpoint{4.650000in}{0.692553in}}%
\pgfusepath{clip}%
\pgfsetbuttcap%
\pgfsetroundjoin%
\definecolor{currentfill}{rgb}{0.963091,0.919493,0.720185}%
\pgfsetfillcolor{currentfill}%
\pgfsetlinewidth{0.250937pt}%
\definecolor{currentstroke}{rgb}{1.000000,1.000000,1.000000}%
\pgfsetstrokecolor{currentstroke}%
\pgfsetdash{}{0pt}%
\pgfpathmoveto{\pgfqpoint{4.734135in}{0.889605in}}%
\pgfpathlineto{\pgfqpoint{4.833071in}{0.889605in}}%
\pgfpathlineto{\pgfqpoint{4.833071in}{0.790669in}}%
\pgfpathlineto{\pgfqpoint{4.734135in}{0.790669in}}%
\pgfpathlineto{\pgfqpoint{4.734135in}{0.889605in}}%
\pgfusepath{stroke,fill}%
\end{pgfscope}%
\begin{pgfscope}%
\pgfpathrectangle{\pgfqpoint{0.380943in}{0.295988in}}{\pgfqpoint{4.650000in}{0.692553in}}%
\pgfusepath{clip}%
\pgfsetbuttcap%
\pgfsetroundjoin%
\definecolor{currentfill}{rgb}{0.964275,0.912388,0.715448}%
\pgfsetfillcolor{currentfill}%
\pgfsetlinewidth{0.250937pt}%
\definecolor{currentstroke}{rgb}{1.000000,1.000000,1.000000}%
\pgfsetstrokecolor{currentstroke}%
\pgfsetdash{}{0pt}%
\pgfpathmoveto{\pgfqpoint{4.833071in}{0.889605in}}%
\pgfpathlineto{\pgfqpoint{4.932007in}{0.889605in}}%
\pgfpathlineto{\pgfqpoint{4.932007in}{0.790669in}}%
\pgfpathlineto{\pgfqpoint{4.833071in}{0.790669in}}%
\pgfpathlineto{\pgfqpoint{4.833071in}{0.889605in}}%
\pgfusepath{stroke,fill}%
\end{pgfscope}%
\begin{pgfscope}%
\pgfpathrectangle{\pgfqpoint{0.380943in}{0.295988in}}{\pgfqpoint{4.650000in}{0.692553in}}%
\pgfusepath{clip}%
\pgfsetbuttcap%
\pgfsetroundjoin%
\definecolor{currentfill}{rgb}{0.983391,0.971765,0.788189}%
\pgfsetfillcolor{currentfill}%
\pgfsetlinewidth{0.250937pt}%
\definecolor{currentstroke}{rgb}{1.000000,1.000000,1.000000}%
\pgfsetstrokecolor{currentstroke}%
\pgfsetdash{}{0pt}%
\pgfpathmoveto{\pgfqpoint{4.932007in}{0.889605in}}%
\pgfpathlineto{\pgfqpoint{5.030943in}{0.889605in}}%
\pgfpathlineto{\pgfqpoint{5.030943in}{0.790669in}}%
\pgfpathlineto{\pgfqpoint{4.932007in}{0.790669in}}%
\pgfpathlineto{\pgfqpoint{4.932007in}{0.889605in}}%
\pgfusepath{stroke,fill}%
\end{pgfscope}%
\begin{pgfscope}%
\pgfpathrectangle{\pgfqpoint{0.380943in}{0.295988in}}{\pgfqpoint{4.650000in}{0.692553in}}%
\pgfusepath{clip}%
\pgfsetbuttcap%
\pgfsetroundjoin%
\pgfsetlinewidth{0.250937pt}%
\definecolor{currentstroke}{rgb}{1.000000,1.000000,1.000000}%
\pgfsetstrokecolor{currentstroke}%
\pgfsetdash{}{0pt}%
\pgfpathmoveto{\pgfqpoint{0.380943in}{0.790669in}}%
\pgfpathlineto{\pgfqpoint{0.479879in}{0.790669in}}%
\pgfpathlineto{\pgfqpoint{0.479879in}{0.691732in}}%
\pgfpathlineto{\pgfqpoint{0.380943in}{0.691732in}}%
\pgfpathlineto{\pgfqpoint{0.380943in}{0.790669in}}%
\pgfusepath{stroke}%
\end{pgfscope}%
\begin{pgfscope}%
\pgfpathrectangle{\pgfqpoint{0.380943in}{0.295988in}}{\pgfqpoint{4.650000in}{0.692553in}}%
\pgfusepath{clip}%
\pgfsetbuttcap%
\pgfsetroundjoin%
\definecolor{currentfill}{rgb}{0.986774,0.977516,0.796986}%
\pgfsetfillcolor{currentfill}%
\pgfsetlinewidth{0.250937pt}%
\definecolor{currentstroke}{rgb}{1.000000,1.000000,1.000000}%
\pgfsetstrokecolor{currentstroke}%
\pgfsetdash{}{0pt}%
\pgfpathmoveto{\pgfqpoint{0.479879in}{0.790669in}}%
\pgfpathlineto{\pgfqpoint{0.578815in}{0.790669in}}%
\pgfpathlineto{\pgfqpoint{0.578815in}{0.691732in}}%
\pgfpathlineto{\pgfqpoint{0.479879in}{0.691732in}}%
\pgfpathlineto{\pgfqpoint{0.479879in}{0.790669in}}%
\pgfusepath{stroke,fill}%
\end{pgfscope}%
\begin{pgfscope}%
\pgfpathrectangle{\pgfqpoint{0.380943in}{0.295988in}}{\pgfqpoint{4.650000in}{0.692553in}}%
\pgfusepath{clip}%
\pgfsetbuttcap%
\pgfsetroundjoin%
\definecolor{currentfill}{rgb}{0.971534,0.875663,0.695517}%
\pgfsetfillcolor{currentfill}%
\pgfsetlinewidth{0.250937pt}%
\definecolor{currentstroke}{rgb}{1.000000,1.000000,1.000000}%
\pgfsetstrokecolor{currentstroke}%
\pgfsetdash{}{0pt}%
\pgfpathmoveto{\pgfqpoint{0.578815in}{0.790669in}}%
\pgfpathlineto{\pgfqpoint{0.677752in}{0.790669in}}%
\pgfpathlineto{\pgfqpoint{0.677752in}{0.691732in}}%
\pgfpathlineto{\pgfqpoint{0.578815in}{0.691732in}}%
\pgfpathlineto{\pgfqpoint{0.578815in}{0.790669in}}%
\pgfusepath{stroke,fill}%
\end{pgfscope}%
\begin{pgfscope}%
\pgfpathrectangle{\pgfqpoint{0.380943in}{0.295988in}}{\pgfqpoint{4.650000in}{0.692553in}}%
\pgfusepath{clip}%
\pgfsetbuttcap%
\pgfsetroundjoin%
\definecolor{currentfill}{rgb}{0.963937,0.914418,0.716801}%
\pgfsetfillcolor{currentfill}%
\pgfsetlinewidth{0.250937pt}%
\definecolor{currentstroke}{rgb}{1.000000,1.000000,1.000000}%
\pgfsetstrokecolor{currentstroke}%
\pgfsetdash{}{0pt}%
\pgfpathmoveto{\pgfqpoint{0.677752in}{0.790669in}}%
\pgfpathlineto{\pgfqpoint{0.776688in}{0.790669in}}%
\pgfpathlineto{\pgfqpoint{0.776688in}{0.691732in}}%
\pgfpathlineto{\pgfqpoint{0.677752in}{0.691732in}}%
\pgfpathlineto{\pgfqpoint{0.677752in}{0.790669in}}%
\pgfusepath{stroke,fill}%
\end{pgfscope}%
\begin{pgfscope}%
\pgfpathrectangle{\pgfqpoint{0.380943in}{0.295988in}}{\pgfqpoint{4.650000in}{0.692553in}}%
\pgfusepath{clip}%
\pgfsetbuttcap%
\pgfsetroundjoin%
\definecolor{currentfill}{rgb}{0.961738,0.927612,0.725598}%
\pgfsetfillcolor{currentfill}%
\pgfsetlinewidth{0.250937pt}%
\definecolor{currentstroke}{rgb}{1.000000,1.000000,1.000000}%
\pgfsetstrokecolor{currentstroke}%
\pgfsetdash{}{0pt}%
\pgfpathmoveto{\pgfqpoint{0.776688in}{0.790669in}}%
\pgfpathlineto{\pgfqpoint{0.875624in}{0.790669in}}%
\pgfpathlineto{\pgfqpoint{0.875624in}{0.691732in}}%
\pgfpathlineto{\pgfqpoint{0.776688in}{0.691732in}}%
\pgfpathlineto{\pgfqpoint{0.776688in}{0.790669in}}%
\pgfusepath{stroke,fill}%
\end{pgfscope}%
\begin{pgfscope}%
\pgfpathrectangle{\pgfqpoint{0.380943in}{0.295988in}}{\pgfqpoint{4.650000in}{0.692553in}}%
\pgfusepath{clip}%
\pgfsetbuttcap%
\pgfsetroundjoin%
\definecolor{currentfill}{rgb}{0.960892,0.932687,0.728981}%
\pgfsetfillcolor{currentfill}%
\pgfsetlinewidth{0.250937pt}%
\definecolor{currentstroke}{rgb}{1.000000,1.000000,1.000000}%
\pgfsetstrokecolor{currentstroke}%
\pgfsetdash{}{0pt}%
\pgfpathmoveto{\pgfqpoint{0.875624in}{0.790669in}}%
\pgfpathlineto{\pgfqpoint{0.974560in}{0.790669in}}%
\pgfpathlineto{\pgfqpoint{0.974560in}{0.691732in}}%
\pgfpathlineto{\pgfqpoint{0.875624in}{0.691732in}}%
\pgfpathlineto{\pgfqpoint{0.875624in}{0.790669in}}%
\pgfusepath{stroke,fill}%
\end{pgfscope}%
\begin{pgfscope}%
\pgfpathrectangle{\pgfqpoint{0.380943in}{0.295988in}}{\pgfqpoint{4.650000in}{0.692553in}}%
\pgfusepath{clip}%
\pgfsetbuttcap%
\pgfsetroundjoin%
\definecolor{currentfill}{rgb}{0.962076,0.925582,0.724245}%
\pgfsetfillcolor{currentfill}%
\pgfsetlinewidth{0.250937pt}%
\definecolor{currentstroke}{rgb}{1.000000,1.000000,1.000000}%
\pgfsetstrokecolor{currentstroke}%
\pgfsetdash{}{0pt}%
\pgfpathmoveto{\pgfqpoint{0.974560in}{0.790669in}}%
\pgfpathlineto{\pgfqpoint{1.073496in}{0.790669in}}%
\pgfpathlineto{\pgfqpoint{1.073496in}{0.691732in}}%
\pgfpathlineto{\pgfqpoint{0.974560in}{0.691732in}}%
\pgfpathlineto{\pgfqpoint{0.974560in}{0.790669in}}%
\pgfusepath{stroke,fill}%
\end{pgfscope}%
\begin{pgfscope}%
\pgfpathrectangle{\pgfqpoint{0.380943in}{0.295988in}}{\pgfqpoint{4.650000in}{0.692553in}}%
\pgfusepath{clip}%
\pgfsetbuttcap%
\pgfsetroundjoin%
\definecolor{currentfill}{rgb}{0.968997,0.888351,0.702284}%
\pgfsetfillcolor{currentfill}%
\pgfsetlinewidth{0.250937pt}%
\definecolor{currentstroke}{rgb}{1.000000,1.000000,1.000000}%
\pgfsetstrokecolor{currentstroke}%
\pgfsetdash{}{0pt}%
\pgfpathmoveto{\pgfqpoint{1.073496in}{0.790669in}}%
\pgfpathlineto{\pgfqpoint{1.172432in}{0.790669in}}%
\pgfpathlineto{\pgfqpoint{1.172432in}{0.691732in}}%
\pgfpathlineto{\pgfqpoint{1.073496in}{0.691732in}}%
\pgfpathlineto{\pgfqpoint{1.073496in}{0.790669in}}%
\pgfusepath{stroke,fill}%
\end{pgfscope}%
\begin{pgfscope}%
\pgfpathrectangle{\pgfqpoint{0.380943in}{0.295988in}}{\pgfqpoint{4.650000in}{0.692553in}}%
\pgfusepath{clip}%
\pgfsetbuttcap%
\pgfsetroundjoin%
\definecolor{currentfill}{rgb}{0.963091,0.919493,0.720185}%
\pgfsetfillcolor{currentfill}%
\pgfsetlinewidth{0.250937pt}%
\definecolor{currentstroke}{rgb}{1.000000,1.000000,1.000000}%
\pgfsetstrokecolor{currentstroke}%
\pgfsetdash{}{0pt}%
\pgfpathmoveto{\pgfqpoint{1.172432in}{0.790669in}}%
\pgfpathlineto{\pgfqpoint{1.271369in}{0.790669in}}%
\pgfpathlineto{\pgfqpoint{1.271369in}{0.691732in}}%
\pgfpathlineto{\pgfqpoint{1.172432in}{0.691732in}}%
\pgfpathlineto{\pgfqpoint{1.172432in}{0.790669in}}%
\pgfusepath{stroke,fill}%
\end{pgfscope}%
\begin{pgfscope}%
\pgfpathrectangle{\pgfqpoint{0.380943in}{0.295988in}}{\pgfqpoint{4.650000in}{0.692553in}}%
\pgfusepath{clip}%
\pgfsetbuttcap%
\pgfsetroundjoin%
\definecolor{currentfill}{rgb}{0.922338,0.400769,0.400769}%
\pgfsetfillcolor{currentfill}%
\pgfsetlinewidth{0.250937pt}%
\definecolor{currentstroke}{rgb}{1.000000,1.000000,1.000000}%
\pgfsetstrokecolor{currentstroke}%
\pgfsetdash{}{0pt}%
\pgfpathmoveto{\pgfqpoint{1.271369in}{0.790669in}}%
\pgfpathlineto{\pgfqpoint{1.370305in}{0.790669in}}%
\pgfpathlineto{\pgfqpoint{1.370305in}{0.691732in}}%
\pgfpathlineto{\pgfqpoint{1.271369in}{0.691732in}}%
\pgfpathlineto{\pgfqpoint{1.271369in}{0.790669in}}%
\pgfusepath{stroke,fill}%
\end{pgfscope}%
\begin{pgfscope}%
\pgfpathrectangle{\pgfqpoint{0.380943in}{0.295988in}}{\pgfqpoint{4.650000in}{0.692553in}}%
\pgfusepath{clip}%
\pgfsetbuttcap%
\pgfsetroundjoin%
\definecolor{currentfill}{rgb}{0.998939,0.658962,0.556032}%
\pgfsetfillcolor{currentfill}%
\pgfsetlinewidth{0.250937pt}%
\definecolor{currentstroke}{rgb}{1.000000,1.000000,1.000000}%
\pgfsetstrokecolor{currentstroke}%
\pgfsetdash{}{0pt}%
\pgfpathmoveto{\pgfqpoint{1.370305in}{0.790669in}}%
\pgfpathlineto{\pgfqpoint{1.469241in}{0.790669in}}%
\pgfpathlineto{\pgfqpoint{1.469241in}{0.691732in}}%
\pgfpathlineto{\pgfqpoint{1.370305in}{0.691732in}}%
\pgfpathlineto{\pgfqpoint{1.370305in}{0.790669in}}%
\pgfusepath{stroke,fill}%
\end{pgfscope}%
\begin{pgfscope}%
\pgfpathrectangle{\pgfqpoint{0.380943in}{0.295988in}}{\pgfqpoint{4.650000in}{0.692553in}}%
\pgfusepath{clip}%
\pgfsetbuttcap%
\pgfsetroundjoin%
\definecolor{currentfill}{rgb}{0.989619,0.788235,0.628374}%
\pgfsetfillcolor{currentfill}%
\pgfsetlinewidth{0.250937pt}%
\definecolor{currentstroke}{rgb}{1.000000,1.000000,1.000000}%
\pgfsetstrokecolor{currentstroke}%
\pgfsetdash{}{0pt}%
\pgfpathmoveto{\pgfqpoint{1.469241in}{0.790669in}}%
\pgfpathlineto{\pgfqpoint{1.568177in}{0.790669in}}%
\pgfpathlineto{\pgfqpoint{1.568177in}{0.691732in}}%
\pgfpathlineto{\pgfqpoint{1.469241in}{0.691732in}}%
\pgfpathlineto{\pgfqpoint{1.469241in}{0.790669in}}%
\pgfusepath{stroke,fill}%
\end{pgfscope}%
\begin{pgfscope}%
\pgfpathrectangle{\pgfqpoint{0.380943in}{0.295988in}}{\pgfqpoint{4.650000in}{0.692553in}}%
\pgfusepath{clip}%
\pgfsetbuttcap%
\pgfsetroundjoin%
\definecolor{currentfill}{rgb}{0.843983,0.322414,0.322414}%
\pgfsetfillcolor{currentfill}%
\pgfsetlinewidth{0.250937pt}%
\definecolor{currentstroke}{rgb}{1.000000,1.000000,1.000000}%
\pgfsetstrokecolor{currentstroke}%
\pgfsetdash{}{0pt}%
\pgfpathmoveto{\pgfqpoint{1.568177in}{0.790669in}}%
\pgfpathlineto{\pgfqpoint{1.667113in}{0.790669in}}%
\pgfpathlineto{\pgfqpoint{1.667113in}{0.691732in}}%
\pgfpathlineto{\pgfqpoint{1.568177in}{0.691732in}}%
\pgfpathlineto{\pgfqpoint{1.568177in}{0.790669in}}%
\pgfusepath{stroke,fill}%
\end{pgfscope}%
\begin{pgfscope}%
\pgfpathrectangle{\pgfqpoint{0.380943in}{0.295988in}}{\pgfqpoint{4.650000in}{0.692553in}}%
\pgfusepath{clip}%
\pgfsetbuttcap%
\pgfsetroundjoin%
\definecolor{currentfill}{rgb}{1.000000,0.608612,0.532072}%
\pgfsetfillcolor{currentfill}%
\pgfsetlinewidth{0.250937pt}%
\definecolor{currentstroke}{rgb}{1.000000,1.000000,1.000000}%
\pgfsetstrokecolor{currentstroke}%
\pgfsetdash{}{0pt}%
\pgfpathmoveto{\pgfqpoint{1.667113in}{0.790669in}}%
\pgfpathlineto{\pgfqpoint{1.766049in}{0.790669in}}%
\pgfpathlineto{\pgfqpoint{1.766049in}{0.691732in}}%
\pgfpathlineto{\pgfqpoint{1.667113in}{0.691732in}}%
\pgfpathlineto{\pgfqpoint{1.667113in}{0.790669in}}%
\pgfusepath{stroke,fill}%
\end{pgfscope}%
\begin{pgfscope}%
\pgfpathrectangle{\pgfqpoint{0.380943in}{0.295988in}}{\pgfqpoint{4.650000in}{0.692553in}}%
\pgfusepath{clip}%
\pgfsetbuttcap%
\pgfsetroundjoin%
\definecolor{currentfill}{rgb}{1.000000,0.608612,0.532072}%
\pgfsetfillcolor{currentfill}%
\pgfsetlinewidth{0.250937pt}%
\definecolor{currentstroke}{rgb}{1.000000,1.000000,1.000000}%
\pgfsetstrokecolor{currentstroke}%
\pgfsetdash{}{0pt}%
\pgfpathmoveto{\pgfqpoint{1.766049in}{0.790669in}}%
\pgfpathlineto{\pgfqpoint{1.864986in}{0.790669in}}%
\pgfpathlineto{\pgfqpoint{1.864986in}{0.691732in}}%
\pgfpathlineto{\pgfqpoint{1.766049in}{0.691732in}}%
\pgfpathlineto{\pgfqpoint{1.766049in}{0.790669in}}%
\pgfusepath{stroke,fill}%
\end{pgfscope}%
\begin{pgfscope}%
\pgfpathrectangle{\pgfqpoint{0.380943in}{0.295988in}}{\pgfqpoint{4.650000in}{0.692553in}}%
\pgfusepath{clip}%
\pgfsetbuttcap%
\pgfsetroundjoin%
\definecolor{currentfill}{rgb}{0.990296,0.782484,0.624990}%
\pgfsetfillcolor{currentfill}%
\pgfsetlinewidth{0.250937pt}%
\definecolor{currentstroke}{rgb}{1.000000,1.000000,1.000000}%
\pgfsetstrokecolor{currentstroke}%
\pgfsetdash{}{0pt}%
\pgfpathmoveto{\pgfqpoint{1.864986in}{0.790669in}}%
\pgfpathlineto{\pgfqpoint{1.963922in}{0.790669in}}%
\pgfpathlineto{\pgfqpoint{1.963922in}{0.691732in}}%
\pgfpathlineto{\pgfqpoint{1.864986in}{0.691732in}}%
\pgfpathlineto{\pgfqpoint{1.864986in}{0.790669in}}%
\pgfusepath{stroke,fill}%
\end{pgfscope}%
\begin{pgfscope}%
\pgfpathrectangle{\pgfqpoint{0.380943in}{0.295988in}}{\pgfqpoint{4.650000in}{0.692553in}}%
\pgfusepath{clip}%
\pgfsetbuttcap%
\pgfsetroundjoin%
\definecolor{currentfill}{rgb}{0.993003,0.759477,0.611457}%
\pgfsetfillcolor{currentfill}%
\pgfsetlinewidth{0.250937pt}%
\definecolor{currentstroke}{rgb}{1.000000,1.000000,1.000000}%
\pgfsetstrokecolor{currentstroke}%
\pgfsetdash{}{0pt}%
\pgfpathmoveto{\pgfqpoint{1.963922in}{0.790669in}}%
\pgfpathlineto{\pgfqpoint{2.062858in}{0.790669in}}%
\pgfpathlineto{\pgfqpoint{2.062858in}{0.691732in}}%
\pgfpathlineto{\pgfqpoint{1.963922in}{0.691732in}}%
\pgfpathlineto{\pgfqpoint{1.963922in}{0.790669in}}%
\pgfusepath{stroke,fill}%
\end{pgfscope}%
\begin{pgfscope}%
\pgfpathrectangle{\pgfqpoint{0.380943in}{0.295988in}}{\pgfqpoint{4.650000in}{0.692553in}}%
\pgfusepath{clip}%
\pgfsetbuttcap%
\pgfsetroundjoin%
\definecolor{currentfill}{rgb}{0.994018,0.750850,0.606382}%
\pgfsetfillcolor{currentfill}%
\pgfsetlinewidth{0.250937pt}%
\definecolor{currentstroke}{rgb}{1.000000,1.000000,1.000000}%
\pgfsetstrokecolor{currentstroke}%
\pgfsetdash{}{0pt}%
\pgfpathmoveto{\pgfqpoint{2.062858in}{0.790669in}}%
\pgfpathlineto{\pgfqpoint{2.161794in}{0.790669in}}%
\pgfpathlineto{\pgfqpoint{2.161794in}{0.691732in}}%
\pgfpathlineto{\pgfqpoint{2.062858in}{0.691732in}}%
\pgfpathlineto{\pgfqpoint{2.062858in}{0.790669in}}%
\pgfusepath{stroke,fill}%
\end{pgfscope}%
\begin{pgfscope}%
\pgfpathrectangle{\pgfqpoint{0.380943in}{0.295988in}}{\pgfqpoint{4.650000in}{0.692553in}}%
\pgfusepath{clip}%
\pgfsetbuttcap%
\pgfsetroundjoin%
\definecolor{currentfill}{rgb}{0.998093,0.680953,0.567874}%
\pgfsetfillcolor{currentfill}%
\pgfsetlinewidth{0.250937pt}%
\definecolor{currentstroke}{rgb}{1.000000,1.000000,1.000000}%
\pgfsetstrokecolor{currentstroke}%
\pgfsetdash{}{0pt}%
\pgfpathmoveto{\pgfqpoint{2.161794in}{0.790669in}}%
\pgfpathlineto{\pgfqpoint{2.260730in}{0.790669in}}%
\pgfpathlineto{\pgfqpoint{2.260730in}{0.691732in}}%
\pgfpathlineto{\pgfqpoint{2.161794in}{0.691732in}}%
\pgfpathlineto{\pgfqpoint{2.161794in}{0.790669in}}%
\pgfusepath{stroke,fill}%
\end{pgfscope}%
\begin{pgfscope}%
\pgfpathrectangle{\pgfqpoint{0.380943in}{0.295988in}}{\pgfqpoint{4.650000in}{0.692553in}}%
\pgfusepath{clip}%
\pgfsetbuttcap%
\pgfsetroundjoin%
\definecolor{currentfill}{rgb}{0.988604,0.796863,0.633449}%
\pgfsetfillcolor{currentfill}%
\pgfsetlinewidth{0.250937pt}%
\definecolor{currentstroke}{rgb}{1.000000,1.000000,1.000000}%
\pgfsetstrokecolor{currentstroke}%
\pgfsetdash{}{0pt}%
\pgfpathmoveto{\pgfqpoint{2.260730in}{0.790669in}}%
\pgfpathlineto{\pgfqpoint{2.359666in}{0.790669in}}%
\pgfpathlineto{\pgfqpoint{2.359666in}{0.691732in}}%
\pgfpathlineto{\pgfqpoint{2.260730in}{0.691732in}}%
\pgfpathlineto{\pgfqpoint{2.260730in}{0.790669in}}%
\pgfusepath{stroke,fill}%
\end{pgfscope}%
\begin{pgfscope}%
\pgfpathrectangle{\pgfqpoint{0.380943in}{0.295988in}}{\pgfqpoint{4.650000in}{0.692553in}}%
\pgfusepath{clip}%
\pgfsetbuttcap%
\pgfsetroundjoin%
\definecolor{currentfill}{rgb}{1.000000,0.588312,0.523952}%
\pgfsetfillcolor{currentfill}%
\pgfsetlinewidth{0.250937pt}%
\definecolor{currentstroke}{rgb}{1.000000,1.000000,1.000000}%
\pgfsetstrokecolor{currentstroke}%
\pgfsetdash{}{0pt}%
\pgfpathmoveto{\pgfqpoint{2.359666in}{0.790669in}}%
\pgfpathlineto{\pgfqpoint{2.458603in}{0.790669in}}%
\pgfpathlineto{\pgfqpoint{2.458603in}{0.691732in}}%
\pgfpathlineto{\pgfqpoint{2.359666in}{0.691732in}}%
\pgfpathlineto{\pgfqpoint{2.359666in}{0.790669in}}%
\pgfusepath{stroke,fill}%
\end{pgfscope}%
\begin{pgfscope}%
\pgfpathrectangle{\pgfqpoint{0.380943in}{0.295988in}}{\pgfqpoint{4.650000in}{0.692553in}}%
\pgfusepath{clip}%
\pgfsetbuttcap%
\pgfsetroundjoin%
\definecolor{currentfill}{rgb}{1.000000,0.554479,0.510419}%
\pgfsetfillcolor{currentfill}%
\pgfsetlinewidth{0.250937pt}%
\definecolor{currentstroke}{rgb}{1.000000,1.000000,1.000000}%
\pgfsetstrokecolor{currentstroke}%
\pgfsetdash{}{0pt}%
\pgfpathmoveto{\pgfqpoint{2.458603in}{0.790669in}}%
\pgfpathlineto{\pgfqpoint{2.557539in}{0.790669in}}%
\pgfpathlineto{\pgfqpoint{2.557539in}{0.691732in}}%
\pgfpathlineto{\pgfqpoint{2.458603in}{0.691732in}}%
\pgfpathlineto{\pgfqpoint{2.458603in}{0.790669in}}%
\pgfusepath{stroke,fill}%
\end{pgfscope}%
\begin{pgfscope}%
\pgfpathrectangle{\pgfqpoint{0.380943in}{0.295988in}}{\pgfqpoint{4.650000in}{0.692553in}}%
\pgfusepath{clip}%
\pgfsetbuttcap%
\pgfsetroundjoin%
\definecolor{currentfill}{rgb}{0.999785,0.636970,0.544191}%
\pgfsetfillcolor{currentfill}%
\pgfsetlinewidth{0.250937pt}%
\definecolor{currentstroke}{rgb}{1.000000,1.000000,1.000000}%
\pgfsetstrokecolor{currentstroke}%
\pgfsetdash{}{0pt}%
\pgfpathmoveto{\pgfqpoint{2.557539in}{0.790669in}}%
\pgfpathlineto{\pgfqpoint{2.656475in}{0.790669in}}%
\pgfpathlineto{\pgfqpoint{2.656475in}{0.691732in}}%
\pgfpathlineto{\pgfqpoint{2.557539in}{0.691732in}}%
\pgfpathlineto{\pgfqpoint{2.557539in}{0.790669in}}%
\pgfusepath{stroke,fill}%
\end{pgfscope}%
\begin{pgfscope}%
\pgfpathrectangle{\pgfqpoint{0.380943in}{0.295988in}}{\pgfqpoint{4.650000in}{0.692553in}}%
\pgfusepath{clip}%
\pgfsetbuttcap%
\pgfsetroundjoin%
\definecolor{currentfill}{rgb}{1.000000,0.571396,0.517186}%
\pgfsetfillcolor{currentfill}%
\pgfsetlinewidth{0.250937pt}%
\definecolor{currentstroke}{rgb}{1.000000,1.000000,1.000000}%
\pgfsetstrokecolor{currentstroke}%
\pgfsetdash{}{0pt}%
\pgfpathmoveto{\pgfqpoint{2.656475in}{0.790669in}}%
\pgfpathlineto{\pgfqpoint{2.755411in}{0.790669in}}%
\pgfpathlineto{\pgfqpoint{2.755411in}{0.691732in}}%
\pgfpathlineto{\pgfqpoint{2.656475in}{0.691732in}}%
\pgfpathlineto{\pgfqpoint{2.656475in}{0.790669in}}%
\pgfusepath{stroke,fill}%
\end{pgfscope}%
\begin{pgfscope}%
\pgfpathrectangle{\pgfqpoint{0.380943in}{0.295988in}}{\pgfqpoint{4.650000in}{0.692553in}}%
\pgfusepath{clip}%
\pgfsetbuttcap%
\pgfsetroundjoin%
\definecolor{currentfill}{rgb}{0.995709,0.736471,0.597924}%
\pgfsetfillcolor{currentfill}%
\pgfsetlinewidth{0.250937pt}%
\definecolor{currentstroke}{rgb}{1.000000,1.000000,1.000000}%
\pgfsetstrokecolor{currentstroke}%
\pgfsetdash{}{0pt}%
\pgfpathmoveto{\pgfqpoint{2.755411in}{0.790669in}}%
\pgfpathlineto{\pgfqpoint{2.854347in}{0.790669in}}%
\pgfpathlineto{\pgfqpoint{2.854347in}{0.691732in}}%
\pgfpathlineto{\pgfqpoint{2.755411in}{0.691732in}}%
\pgfpathlineto{\pgfqpoint{2.755411in}{0.790669in}}%
\pgfusepath{stroke,fill}%
\end{pgfscope}%
\begin{pgfscope}%
\pgfpathrectangle{\pgfqpoint{0.380943in}{0.295988in}}{\pgfqpoint{4.650000in}{0.692553in}}%
\pgfusepath{clip}%
\pgfsetbuttcap%
\pgfsetroundjoin%
\definecolor{currentfill}{rgb}{0.990296,0.782484,0.624990}%
\pgfsetfillcolor{currentfill}%
\pgfsetlinewidth{0.250937pt}%
\definecolor{currentstroke}{rgb}{1.000000,1.000000,1.000000}%
\pgfsetstrokecolor{currentstroke}%
\pgfsetdash{}{0pt}%
\pgfpathmoveto{\pgfqpoint{2.854347in}{0.790669in}}%
\pgfpathlineto{\pgfqpoint{2.953283in}{0.790669in}}%
\pgfpathlineto{\pgfqpoint{2.953283in}{0.691732in}}%
\pgfpathlineto{\pgfqpoint{2.854347in}{0.691732in}}%
\pgfpathlineto{\pgfqpoint{2.854347in}{0.790669in}}%
\pgfusepath{stroke,fill}%
\end{pgfscope}%
\begin{pgfscope}%
\pgfpathrectangle{\pgfqpoint{0.380943in}{0.295988in}}{\pgfqpoint{4.650000in}{0.692553in}}%
\pgfusepath{clip}%
\pgfsetbuttcap%
\pgfsetroundjoin%
\definecolor{currentfill}{rgb}{0.994018,0.750850,0.606382}%
\pgfsetfillcolor{currentfill}%
\pgfsetlinewidth{0.250937pt}%
\definecolor{currentstroke}{rgb}{1.000000,1.000000,1.000000}%
\pgfsetstrokecolor{currentstroke}%
\pgfsetdash{}{0pt}%
\pgfpathmoveto{\pgfqpoint{2.953283in}{0.790669in}}%
\pgfpathlineto{\pgfqpoint{3.052220in}{0.790669in}}%
\pgfpathlineto{\pgfqpoint{3.052220in}{0.691732in}}%
\pgfpathlineto{\pgfqpoint{2.953283in}{0.691732in}}%
\pgfpathlineto{\pgfqpoint{2.953283in}{0.790669in}}%
\pgfusepath{stroke,fill}%
\end{pgfscope}%
\begin{pgfscope}%
\pgfpathrectangle{\pgfqpoint{0.380943in}{0.295988in}}{\pgfqpoint{4.650000in}{0.692553in}}%
\pgfusepath{clip}%
\pgfsetbuttcap%
\pgfsetroundjoin%
\definecolor{currentfill}{rgb}{0.997586,0.694148,0.574979}%
\pgfsetfillcolor{currentfill}%
\pgfsetlinewidth{0.250937pt}%
\definecolor{currentstroke}{rgb}{1.000000,1.000000,1.000000}%
\pgfsetstrokecolor{currentstroke}%
\pgfsetdash{}{0pt}%
\pgfpathmoveto{\pgfqpoint{3.052220in}{0.790669in}}%
\pgfpathlineto{\pgfqpoint{3.151156in}{0.790669in}}%
\pgfpathlineto{\pgfqpoint{3.151156in}{0.691732in}}%
\pgfpathlineto{\pgfqpoint{3.052220in}{0.691732in}}%
\pgfpathlineto{\pgfqpoint{3.052220in}{0.790669in}}%
\pgfusepath{stroke,fill}%
\end{pgfscope}%
\begin{pgfscope}%
\pgfpathrectangle{\pgfqpoint{0.380943in}{0.295988in}}{\pgfqpoint{4.650000in}{0.692553in}}%
\pgfusepath{clip}%
\pgfsetbuttcap%
\pgfsetroundjoin%
\definecolor{currentfill}{rgb}{0.964937,0.908651,0.713110}%
\pgfsetfillcolor{currentfill}%
\pgfsetlinewidth{0.250937pt}%
\definecolor{currentstroke}{rgb}{1.000000,1.000000,1.000000}%
\pgfsetstrokecolor{currentstroke}%
\pgfsetdash{}{0pt}%
\pgfpathmoveto{\pgfqpoint{3.151156in}{0.790669in}}%
\pgfpathlineto{\pgfqpoint{3.250092in}{0.790669in}}%
\pgfpathlineto{\pgfqpoint{3.250092in}{0.691732in}}%
\pgfpathlineto{\pgfqpoint{3.151156in}{0.691732in}}%
\pgfpathlineto{\pgfqpoint{3.151156in}{0.790669in}}%
\pgfusepath{stroke,fill}%
\end{pgfscope}%
\begin{pgfscope}%
\pgfpathrectangle{\pgfqpoint{0.380943in}{0.295988in}}{\pgfqpoint{4.650000in}{0.692553in}}%
\pgfusepath{clip}%
\pgfsetbuttcap%
\pgfsetroundjoin%
\definecolor{currentfill}{rgb}{0.993003,0.759477,0.611457}%
\pgfsetfillcolor{currentfill}%
\pgfsetlinewidth{0.250937pt}%
\definecolor{currentstroke}{rgb}{1.000000,1.000000,1.000000}%
\pgfsetstrokecolor{currentstroke}%
\pgfsetdash{}{0pt}%
\pgfpathmoveto{\pgfqpoint{3.250092in}{0.790669in}}%
\pgfpathlineto{\pgfqpoint{3.349028in}{0.790669in}}%
\pgfpathlineto{\pgfqpoint{3.349028in}{0.691732in}}%
\pgfpathlineto{\pgfqpoint{3.250092in}{0.691732in}}%
\pgfpathlineto{\pgfqpoint{3.250092in}{0.790669in}}%
\pgfusepath{stroke,fill}%
\end{pgfscope}%
\begin{pgfscope}%
\pgfpathrectangle{\pgfqpoint{0.380943in}{0.295988in}}{\pgfqpoint{4.650000in}{0.692553in}}%
\pgfusepath{clip}%
\pgfsetbuttcap%
\pgfsetroundjoin%
\definecolor{currentfill}{rgb}{1.000000,0.538331,0.503652}%
\pgfsetfillcolor{currentfill}%
\pgfsetlinewidth{0.250937pt}%
\definecolor{currentstroke}{rgb}{1.000000,1.000000,1.000000}%
\pgfsetstrokecolor{currentstroke}%
\pgfsetdash{}{0pt}%
\pgfpathmoveto{\pgfqpoint{3.349028in}{0.790669in}}%
\pgfpathlineto{\pgfqpoint{3.447964in}{0.790669in}}%
\pgfpathlineto{\pgfqpoint{3.447964in}{0.691732in}}%
\pgfpathlineto{\pgfqpoint{3.349028in}{0.691732in}}%
\pgfpathlineto{\pgfqpoint{3.349028in}{0.790669in}}%
\pgfusepath{stroke,fill}%
\end{pgfscope}%
\begin{pgfscope}%
\pgfpathrectangle{\pgfqpoint{0.380943in}{0.295988in}}{\pgfqpoint{4.650000in}{0.692553in}}%
\pgfusepath{clip}%
\pgfsetbuttcap%
\pgfsetroundjoin%
\definecolor{currentfill}{rgb}{0.977316,0.455748,0.455748}%
\pgfsetfillcolor{currentfill}%
\pgfsetlinewidth{0.250937pt}%
\definecolor{currentstroke}{rgb}{1.000000,1.000000,1.000000}%
\pgfsetstrokecolor{currentstroke}%
\pgfsetdash{}{0pt}%
\pgfpathmoveto{\pgfqpoint{3.447964in}{0.790669in}}%
\pgfpathlineto{\pgfqpoint{3.546901in}{0.790669in}}%
\pgfpathlineto{\pgfqpoint{3.546901in}{0.691732in}}%
\pgfpathlineto{\pgfqpoint{3.447964in}{0.691732in}}%
\pgfpathlineto{\pgfqpoint{3.447964in}{0.790669in}}%
\pgfusepath{stroke,fill}%
\end{pgfscope}%
\begin{pgfscope}%
\pgfpathrectangle{\pgfqpoint{0.380943in}{0.295988in}}{\pgfqpoint{4.650000in}{0.692553in}}%
\pgfusepath{clip}%
\pgfsetbuttcap%
\pgfsetroundjoin%
\definecolor{currentfill}{rgb}{1.000000,0.615379,0.534779}%
\pgfsetfillcolor{currentfill}%
\pgfsetlinewidth{0.250937pt}%
\definecolor{currentstroke}{rgb}{1.000000,1.000000,1.000000}%
\pgfsetstrokecolor{currentstroke}%
\pgfsetdash{}{0pt}%
\pgfpathmoveto{\pgfqpoint{3.546901in}{0.790669in}}%
\pgfpathlineto{\pgfqpoint{3.645837in}{0.790669in}}%
\pgfpathlineto{\pgfqpoint{3.645837in}{0.691732in}}%
\pgfpathlineto{\pgfqpoint{3.546901in}{0.691732in}}%
\pgfpathlineto{\pgfqpoint{3.546901in}{0.790669in}}%
\pgfusepath{stroke,fill}%
\end{pgfscope}%
\begin{pgfscope}%
\pgfpathrectangle{\pgfqpoint{0.380943in}{0.295988in}}{\pgfqpoint{4.650000in}{0.692553in}}%
\pgfusepath{clip}%
\pgfsetbuttcap%
\pgfsetroundjoin%
\definecolor{currentfill}{rgb}{0.999785,0.636970,0.544191}%
\pgfsetfillcolor{currentfill}%
\pgfsetlinewidth{0.250937pt}%
\definecolor{currentstroke}{rgb}{1.000000,1.000000,1.000000}%
\pgfsetstrokecolor{currentstroke}%
\pgfsetdash{}{0pt}%
\pgfpathmoveto{\pgfqpoint{3.645837in}{0.790669in}}%
\pgfpathlineto{\pgfqpoint{3.744773in}{0.790669in}}%
\pgfpathlineto{\pgfqpoint{3.744773in}{0.691732in}}%
\pgfpathlineto{\pgfqpoint{3.645837in}{0.691732in}}%
\pgfpathlineto{\pgfqpoint{3.645837in}{0.790669in}}%
\pgfusepath{stroke,fill}%
\end{pgfscope}%
\begin{pgfscope}%
\pgfpathrectangle{\pgfqpoint{0.380943in}{0.295988in}}{\pgfqpoint{4.650000in}{0.692553in}}%
\pgfusepath{clip}%
\pgfsetbuttcap%
\pgfsetroundjoin%
\definecolor{currentfill}{rgb}{0.995709,0.736471,0.597924}%
\pgfsetfillcolor{currentfill}%
\pgfsetlinewidth{0.250937pt}%
\definecolor{currentstroke}{rgb}{1.000000,1.000000,1.000000}%
\pgfsetstrokecolor{currentstroke}%
\pgfsetdash{}{0pt}%
\pgfpathmoveto{\pgfqpoint{3.744773in}{0.790669in}}%
\pgfpathlineto{\pgfqpoint{3.843709in}{0.790669in}}%
\pgfpathlineto{\pgfqpoint{3.843709in}{0.691732in}}%
\pgfpathlineto{\pgfqpoint{3.744773in}{0.691732in}}%
\pgfpathlineto{\pgfqpoint{3.744773in}{0.790669in}}%
\pgfusepath{stroke,fill}%
\end{pgfscope}%
\begin{pgfscope}%
\pgfpathrectangle{\pgfqpoint{0.380943in}{0.295988in}}{\pgfqpoint{4.650000in}{0.692553in}}%
\pgfusepath{clip}%
\pgfsetbuttcap%
\pgfsetroundjoin%
\definecolor{currentfill}{rgb}{1.000000,0.480477,0.479293}%
\pgfsetfillcolor{currentfill}%
\pgfsetlinewidth{0.250937pt}%
\definecolor{currentstroke}{rgb}{1.000000,1.000000,1.000000}%
\pgfsetstrokecolor{currentstroke}%
\pgfsetdash{}{0pt}%
\pgfpathmoveto{\pgfqpoint{3.843709in}{0.790669in}}%
\pgfpathlineto{\pgfqpoint{3.942645in}{0.790669in}}%
\pgfpathlineto{\pgfqpoint{3.942645in}{0.691732in}}%
\pgfpathlineto{\pgfqpoint{3.843709in}{0.691732in}}%
\pgfpathlineto{\pgfqpoint{3.843709in}{0.790669in}}%
\pgfusepath{stroke,fill}%
\end{pgfscope}%
\begin{pgfscope}%
\pgfpathrectangle{\pgfqpoint{0.380943in}{0.295988in}}{\pgfqpoint{4.650000in}{0.692553in}}%
\pgfusepath{clip}%
\pgfsetbuttcap%
\pgfsetroundjoin%
\definecolor{currentfill}{rgb}{1.000000,0.522261,0.496886}%
\pgfsetfillcolor{currentfill}%
\pgfsetlinewidth{0.250937pt}%
\definecolor{currentstroke}{rgb}{1.000000,1.000000,1.000000}%
\pgfsetstrokecolor{currentstroke}%
\pgfsetdash{}{0pt}%
\pgfpathmoveto{\pgfqpoint{3.942645in}{0.790669in}}%
\pgfpathlineto{\pgfqpoint{4.041581in}{0.790669in}}%
\pgfpathlineto{\pgfqpoint{4.041581in}{0.691732in}}%
\pgfpathlineto{\pgfqpoint{3.942645in}{0.691732in}}%
\pgfpathlineto{\pgfqpoint{3.942645in}{0.790669in}}%
\pgfusepath{stroke,fill}%
\end{pgfscope}%
\begin{pgfscope}%
\pgfpathrectangle{\pgfqpoint{0.380943in}{0.295988in}}{\pgfqpoint{4.650000in}{0.692553in}}%
\pgfusepath{clip}%
\pgfsetbuttcap%
\pgfsetroundjoin%
\definecolor{currentfill}{rgb}{0.995709,0.736471,0.597924}%
\pgfsetfillcolor{currentfill}%
\pgfsetlinewidth{0.250937pt}%
\definecolor{currentstroke}{rgb}{1.000000,1.000000,1.000000}%
\pgfsetstrokecolor{currentstroke}%
\pgfsetdash{}{0pt}%
\pgfpathmoveto{\pgfqpoint{4.041581in}{0.790669in}}%
\pgfpathlineto{\pgfqpoint{4.140518in}{0.790669in}}%
\pgfpathlineto{\pgfqpoint{4.140518in}{0.691732in}}%
\pgfpathlineto{\pgfqpoint{4.041581in}{0.691732in}}%
\pgfpathlineto{\pgfqpoint{4.041581in}{0.790669in}}%
\pgfusepath{stroke,fill}%
\end{pgfscope}%
\begin{pgfscope}%
\pgfpathrectangle{\pgfqpoint{0.380943in}{0.295988in}}{\pgfqpoint{4.650000in}{0.692553in}}%
\pgfusepath{clip}%
\pgfsetbuttcap%
\pgfsetroundjoin%
\definecolor{currentfill}{rgb}{1.000000,0.554479,0.510419}%
\pgfsetfillcolor{currentfill}%
\pgfsetlinewidth{0.250937pt}%
\definecolor{currentstroke}{rgb}{1.000000,1.000000,1.000000}%
\pgfsetstrokecolor{currentstroke}%
\pgfsetdash{}{0pt}%
\pgfpathmoveto{\pgfqpoint{4.140518in}{0.790669in}}%
\pgfpathlineto{\pgfqpoint{4.239454in}{0.790669in}}%
\pgfpathlineto{\pgfqpoint{4.239454in}{0.691732in}}%
\pgfpathlineto{\pgfqpoint{4.140518in}{0.691732in}}%
\pgfpathlineto{\pgfqpoint{4.140518in}{0.790669in}}%
\pgfusepath{stroke,fill}%
\end{pgfscope}%
\begin{pgfscope}%
\pgfpathrectangle{\pgfqpoint{0.380943in}{0.295988in}}{\pgfqpoint{4.650000in}{0.692553in}}%
\pgfusepath{clip}%
\pgfsetbuttcap%
\pgfsetroundjoin%
\definecolor{currentfill}{rgb}{0.922338,0.400769,0.400769}%
\pgfsetfillcolor{currentfill}%
\pgfsetlinewidth{0.250937pt}%
\definecolor{currentstroke}{rgb}{1.000000,1.000000,1.000000}%
\pgfsetstrokecolor{currentstroke}%
\pgfsetdash{}{0pt}%
\pgfpathmoveto{\pgfqpoint{4.239454in}{0.790669in}}%
\pgfpathlineto{\pgfqpoint{4.338390in}{0.790669in}}%
\pgfpathlineto{\pgfqpoint{4.338390in}{0.691732in}}%
\pgfpathlineto{\pgfqpoint{4.239454in}{0.691732in}}%
\pgfpathlineto{\pgfqpoint{4.239454in}{0.790669in}}%
\pgfusepath{stroke,fill}%
\end{pgfscope}%
\begin{pgfscope}%
\pgfpathrectangle{\pgfqpoint{0.380943in}{0.295988in}}{\pgfqpoint{4.650000in}{0.692553in}}%
\pgfusepath{clip}%
\pgfsetbuttcap%
\pgfsetroundjoin%
\definecolor{currentfill}{rgb}{0.967474,0.895963,0.706344}%
\pgfsetfillcolor{currentfill}%
\pgfsetlinewidth{0.250937pt}%
\definecolor{currentstroke}{rgb}{1.000000,1.000000,1.000000}%
\pgfsetstrokecolor{currentstroke}%
\pgfsetdash{}{0pt}%
\pgfpathmoveto{\pgfqpoint{4.338390in}{0.790669in}}%
\pgfpathlineto{\pgfqpoint{4.437326in}{0.790669in}}%
\pgfpathlineto{\pgfqpoint{4.437326in}{0.691732in}}%
\pgfpathlineto{\pgfqpoint{4.338390in}{0.691732in}}%
\pgfpathlineto{\pgfqpoint{4.338390in}{0.790669in}}%
\pgfusepath{stroke,fill}%
\end{pgfscope}%
\begin{pgfscope}%
\pgfpathrectangle{\pgfqpoint{0.380943in}{0.295988in}}{\pgfqpoint{4.650000in}{0.692553in}}%
\pgfusepath{clip}%
\pgfsetbuttcap%
\pgfsetroundjoin%
\definecolor{currentfill}{rgb}{0.980669,0.832787,0.665559}%
\pgfsetfillcolor{currentfill}%
\pgfsetlinewidth{0.250937pt}%
\definecolor{currentstroke}{rgb}{1.000000,1.000000,1.000000}%
\pgfsetstrokecolor{currentstroke}%
\pgfsetdash{}{0pt}%
\pgfpathmoveto{\pgfqpoint{4.437326in}{0.790669in}}%
\pgfpathlineto{\pgfqpoint{4.536262in}{0.790669in}}%
\pgfpathlineto{\pgfqpoint{4.536262in}{0.691732in}}%
\pgfpathlineto{\pgfqpoint{4.437326in}{0.691732in}}%
\pgfpathlineto{\pgfqpoint{4.437326in}{0.790669in}}%
\pgfusepath{stroke,fill}%
\end{pgfscope}%
\begin{pgfscope}%
\pgfpathrectangle{\pgfqpoint{0.380943in}{0.295988in}}{\pgfqpoint{4.650000in}{0.692553in}}%
\pgfusepath{clip}%
\pgfsetbuttcap%
\pgfsetroundjoin%
\definecolor{currentfill}{rgb}{0.963429,0.917463,0.718831}%
\pgfsetfillcolor{currentfill}%
\pgfsetlinewidth{0.250937pt}%
\definecolor{currentstroke}{rgb}{1.000000,1.000000,1.000000}%
\pgfsetstrokecolor{currentstroke}%
\pgfsetdash{}{0pt}%
\pgfpathmoveto{\pgfqpoint{4.536262in}{0.790669in}}%
\pgfpathlineto{\pgfqpoint{4.635198in}{0.790669in}}%
\pgfpathlineto{\pgfqpoint{4.635198in}{0.691732in}}%
\pgfpathlineto{\pgfqpoint{4.536262in}{0.691732in}}%
\pgfpathlineto{\pgfqpoint{4.536262in}{0.790669in}}%
\pgfusepath{stroke,fill}%
\end{pgfscope}%
\begin{pgfscope}%
\pgfpathrectangle{\pgfqpoint{0.380943in}{0.295988in}}{\pgfqpoint{4.650000in}{0.692553in}}%
\pgfusepath{clip}%
\pgfsetbuttcap%
\pgfsetroundjoin%
\definecolor{currentfill}{rgb}{0.963429,0.917463,0.718831}%
\pgfsetfillcolor{currentfill}%
\pgfsetlinewidth{0.250937pt}%
\definecolor{currentstroke}{rgb}{1.000000,1.000000,1.000000}%
\pgfsetstrokecolor{currentstroke}%
\pgfsetdash{}{0pt}%
\pgfpathmoveto{\pgfqpoint{4.635198in}{0.790669in}}%
\pgfpathlineto{\pgfqpoint{4.734135in}{0.790669in}}%
\pgfpathlineto{\pgfqpoint{4.734135in}{0.691732in}}%
\pgfpathlineto{\pgfqpoint{4.635198in}{0.691732in}}%
\pgfpathlineto{\pgfqpoint{4.635198in}{0.790669in}}%
\pgfusepath{stroke,fill}%
\end{pgfscope}%
\begin{pgfscope}%
\pgfpathrectangle{\pgfqpoint{0.380943in}{0.295988in}}{\pgfqpoint{4.650000in}{0.692553in}}%
\pgfusepath{clip}%
\pgfsetbuttcap%
\pgfsetroundjoin%
\definecolor{currentfill}{rgb}{0.963429,0.917463,0.718831}%
\pgfsetfillcolor{currentfill}%
\pgfsetlinewidth{0.250937pt}%
\definecolor{currentstroke}{rgb}{1.000000,1.000000,1.000000}%
\pgfsetstrokecolor{currentstroke}%
\pgfsetdash{}{0pt}%
\pgfpathmoveto{\pgfqpoint{4.734135in}{0.790669in}}%
\pgfpathlineto{\pgfqpoint{4.833071in}{0.790669in}}%
\pgfpathlineto{\pgfqpoint{4.833071in}{0.691732in}}%
\pgfpathlineto{\pgfqpoint{4.734135in}{0.691732in}}%
\pgfpathlineto{\pgfqpoint{4.734135in}{0.790669in}}%
\pgfusepath{stroke,fill}%
\end{pgfscope}%
\begin{pgfscope}%
\pgfpathrectangle{\pgfqpoint{0.380943in}{0.295988in}}{\pgfqpoint{4.650000in}{0.692553in}}%
\pgfusepath{clip}%
\pgfsetbuttcap%
\pgfsetroundjoin%
\definecolor{currentfill}{rgb}{0.963091,0.919493,0.720185}%
\pgfsetfillcolor{currentfill}%
\pgfsetlinewidth{0.250937pt}%
\definecolor{currentstroke}{rgb}{1.000000,1.000000,1.000000}%
\pgfsetstrokecolor{currentstroke}%
\pgfsetdash{}{0pt}%
\pgfpathmoveto{\pgfqpoint{4.833071in}{0.790669in}}%
\pgfpathlineto{\pgfqpoint{4.932007in}{0.790669in}}%
\pgfpathlineto{\pgfqpoint{4.932007in}{0.691732in}}%
\pgfpathlineto{\pgfqpoint{4.833071in}{0.691732in}}%
\pgfpathlineto{\pgfqpoint{4.833071in}{0.790669in}}%
\pgfusepath{stroke,fill}%
\end{pgfscope}%
\begin{pgfscope}%
\pgfpathrectangle{\pgfqpoint{0.380943in}{0.295988in}}{\pgfqpoint{4.650000in}{0.692553in}}%
\pgfusepath{clip}%
\pgfsetbuttcap%
\pgfsetroundjoin%
\pgfsetlinewidth{0.250937pt}%
\definecolor{currentstroke}{rgb}{1.000000,1.000000,1.000000}%
\pgfsetstrokecolor{currentstroke}%
\pgfsetdash{}{0pt}%
\pgfpathmoveto{\pgfqpoint{4.932007in}{0.790669in}}%
\pgfpathlineto{\pgfqpoint{5.030943in}{0.790669in}}%
\pgfpathlineto{\pgfqpoint{5.030943in}{0.691732in}}%
\pgfpathlineto{\pgfqpoint{4.932007in}{0.691732in}}%
\pgfpathlineto{\pgfqpoint{4.932007in}{0.790669in}}%
\pgfusepath{stroke}%
\end{pgfscope}%
\begin{pgfscope}%
\pgfpathrectangle{\pgfqpoint{0.380943in}{0.295988in}}{\pgfqpoint{4.650000in}{0.692553in}}%
\pgfusepath{clip}%
\pgfsetbuttcap%
\pgfsetroundjoin%
\pgfsetlinewidth{0.250937pt}%
\definecolor{currentstroke}{rgb}{1.000000,1.000000,1.000000}%
\pgfsetstrokecolor{currentstroke}%
\pgfsetdash{}{0pt}%
\pgfpathmoveto{\pgfqpoint{0.380943in}{0.691732in}}%
\pgfpathlineto{\pgfqpoint{0.479879in}{0.691732in}}%
\pgfpathlineto{\pgfqpoint{0.479879in}{0.592796in}}%
\pgfpathlineto{\pgfqpoint{0.380943in}{0.592796in}}%
\pgfpathlineto{\pgfqpoint{0.380943in}{0.691732in}}%
\pgfusepath{stroke}%
\end{pgfscope}%
\begin{pgfscope}%
\pgfpathrectangle{\pgfqpoint{0.380943in}{0.295988in}}{\pgfqpoint{4.650000in}{0.692553in}}%
\pgfusepath{clip}%
\pgfsetbuttcap%
\pgfsetroundjoin%
\definecolor{currentfill}{rgb}{1.000000,1.000000,0.853287}%
\pgfsetfillcolor{currentfill}%
\pgfsetlinewidth{0.250937pt}%
\definecolor{currentstroke}{rgb}{1.000000,1.000000,1.000000}%
\pgfsetstrokecolor{currentstroke}%
\pgfsetdash{}{0pt}%
\pgfpathmoveto{\pgfqpoint{0.479879in}{0.691732in}}%
\pgfpathlineto{\pgfqpoint{0.578815in}{0.691732in}}%
\pgfpathlineto{\pgfqpoint{0.578815in}{0.592796in}}%
\pgfpathlineto{\pgfqpoint{0.479879in}{0.592796in}}%
\pgfpathlineto{\pgfqpoint{0.479879in}{0.691732in}}%
\pgfusepath{stroke,fill}%
\end{pgfscope}%
\begin{pgfscope}%
\pgfpathrectangle{\pgfqpoint{0.380943in}{0.295988in}}{\pgfqpoint{4.650000in}{0.692553in}}%
\pgfusepath{clip}%
\pgfsetbuttcap%
\pgfsetroundjoin%
\definecolor{currentfill}{rgb}{0.969858,0.948758,0.753003}%
\pgfsetfillcolor{currentfill}%
\pgfsetlinewidth{0.250937pt}%
\definecolor{currentstroke}{rgb}{1.000000,1.000000,1.000000}%
\pgfsetstrokecolor{currentstroke}%
\pgfsetdash{}{0pt}%
\pgfpathmoveto{\pgfqpoint{0.578815in}{0.691732in}}%
\pgfpathlineto{\pgfqpoint{0.677752in}{0.691732in}}%
\pgfpathlineto{\pgfqpoint{0.677752in}{0.592796in}}%
\pgfpathlineto{\pgfqpoint{0.578815in}{0.592796in}}%
\pgfpathlineto{\pgfqpoint{0.578815in}{0.691732in}}%
\pgfusepath{stroke,fill}%
\end{pgfscope}%
\begin{pgfscope}%
\pgfpathrectangle{\pgfqpoint{0.380943in}{0.295988in}}{\pgfqpoint{4.650000in}{0.692553in}}%
\pgfusepath{clip}%
\pgfsetbuttcap%
\pgfsetroundjoin%
\definecolor{currentfill}{rgb}{0.960892,0.932687,0.728981}%
\pgfsetfillcolor{currentfill}%
\pgfsetlinewidth{0.250937pt}%
\definecolor{currentstroke}{rgb}{1.000000,1.000000,1.000000}%
\pgfsetstrokecolor{currentstroke}%
\pgfsetdash{}{0pt}%
\pgfpathmoveto{\pgfqpoint{0.677752in}{0.691732in}}%
\pgfpathlineto{\pgfqpoint{0.776688in}{0.691732in}}%
\pgfpathlineto{\pgfqpoint{0.776688in}{0.592796in}}%
\pgfpathlineto{\pgfqpoint{0.677752in}{0.592796in}}%
\pgfpathlineto{\pgfqpoint{0.677752in}{0.691732in}}%
\pgfusepath{stroke,fill}%
\end{pgfscope}%
\begin{pgfscope}%
\pgfpathrectangle{\pgfqpoint{0.380943in}{0.295988in}}{\pgfqpoint{4.650000in}{0.692553in}}%
\pgfusepath{clip}%
\pgfsetbuttcap%
\pgfsetroundjoin%
\definecolor{currentfill}{rgb}{0.986774,0.977516,0.796986}%
\pgfsetfillcolor{currentfill}%
\pgfsetlinewidth{0.250937pt}%
\definecolor{currentstroke}{rgb}{1.000000,1.000000,1.000000}%
\pgfsetstrokecolor{currentstroke}%
\pgfsetdash{}{0pt}%
\pgfpathmoveto{\pgfqpoint{0.776688in}{0.691732in}}%
\pgfpathlineto{\pgfqpoint{0.875624in}{0.691732in}}%
\pgfpathlineto{\pgfqpoint{0.875624in}{0.592796in}}%
\pgfpathlineto{\pgfqpoint{0.776688in}{0.592796in}}%
\pgfpathlineto{\pgfqpoint{0.776688in}{0.691732in}}%
\pgfusepath{stroke,fill}%
\end{pgfscope}%
\begin{pgfscope}%
\pgfpathrectangle{\pgfqpoint{0.380943in}{0.295988in}}{\pgfqpoint{4.650000in}{0.692553in}}%
\pgfusepath{clip}%
\pgfsetbuttcap%
\pgfsetroundjoin%
\definecolor{currentfill}{rgb}{0.960892,0.932687,0.728981}%
\pgfsetfillcolor{currentfill}%
\pgfsetlinewidth{0.250937pt}%
\definecolor{currentstroke}{rgb}{1.000000,1.000000,1.000000}%
\pgfsetstrokecolor{currentstroke}%
\pgfsetdash{}{0pt}%
\pgfpathmoveto{\pgfqpoint{0.875624in}{0.691732in}}%
\pgfpathlineto{\pgfqpoint{0.974560in}{0.691732in}}%
\pgfpathlineto{\pgfqpoint{0.974560in}{0.592796in}}%
\pgfpathlineto{\pgfqpoint{0.875624in}{0.592796in}}%
\pgfpathlineto{\pgfqpoint{0.875624in}{0.691732in}}%
\pgfusepath{stroke,fill}%
\end{pgfscope}%
\begin{pgfscope}%
\pgfpathrectangle{\pgfqpoint{0.380943in}{0.295988in}}{\pgfqpoint{4.650000in}{0.692553in}}%
\pgfusepath{clip}%
\pgfsetbuttcap%
\pgfsetroundjoin%
\definecolor{currentfill}{rgb}{0.961230,0.930657,0.727628}%
\pgfsetfillcolor{currentfill}%
\pgfsetlinewidth{0.250937pt}%
\definecolor{currentstroke}{rgb}{1.000000,1.000000,1.000000}%
\pgfsetstrokecolor{currentstroke}%
\pgfsetdash{}{0pt}%
\pgfpathmoveto{\pgfqpoint{0.974560in}{0.691732in}}%
\pgfpathlineto{\pgfqpoint{1.073496in}{0.691732in}}%
\pgfpathlineto{\pgfqpoint{1.073496in}{0.592796in}}%
\pgfpathlineto{\pgfqpoint{0.974560in}{0.592796in}}%
\pgfpathlineto{\pgfqpoint{0.974560in}{0.691732in}}%
\pgfusepath{stroke,fill}%
\end{pgfscope}%
\begin{pgfscope}%
\pgfpathrectangle{\pgfqpoint{0.380943in}{0.295988in}}{\pgfqpoint{4.650000in}{0.692553in}}%
\pgfusepath{clip}%
\pgfsetbuttcap%
\pgfsetroundjoin%
\definecolor{currentfill}{rgb}{0.961738,0.927612,0.725598}%
\pgfsetfillcolor{currentfill}%
\pgfsetlinewidth{0.250937pt}%
\definecolor{currentstroke}{rgb}{1.000000,1.000000,1.000000}%
\pgfsetstrokecolor{currentstroke}%
\pgfsetdash{}{0pt}%
\pgfpathmoveto{\pgfqpoint{1.073496in}{0.691732in}}%
\pgfpathlineto{\pgfqpoint{1.172432in}{0.691732in}}%
\pgfpathlineto{\pgfqpoint{1.172432in}{0.592796in}}%
\pgfpathlineto{\pgfqpoint{1.073496in}{0.592796in}}%
\pgfpathlineto{\pgfqpoint{1.073496in}{0.691732in}}%
\pgfusepath{stroke,fill}%
\end{pgfscope}%
\begin{pgfscope}%
\pgfpathrectangle{\pgfqpoint{0.380943in}{0.295988in}}{\pgfqpoint{4.650000in}{0.692553in}}%
\pgfusepath{clip}%
\pgfsetbuttcap%
\pgfsetroundjoin%
\definecolor{currentfill}{rgb}{0.995233,0.991895,0.818977}%
\pgfsetfillcolor{currentfill}%
\pgfsetlinewidth{0.250937pt}%
\definecolor{currentstroke}{rgb}{1.000000,1.000000,1.000000}%
\pgfsetstrokecolor{currentstroke}%
\pgfsetdash{}{0pt}%
\pgfpathmoveto{\pgfqpoint{1.172432in}{0.691732in}}%
\pgfpathlineto{\pgfqpoint{1.271369in}{0.691732in}}%
\pgfpathlineto{\pgfqpoint{1.271369in}{0.592796in}}%
\pgfpathlineto{\pgfqpoint{1.172432in}{0.592796in}}%
\pgfpathlineto{\pgfqpoint{1.172432in}{0.691732in}}%
\pgfusepath{stroke,fill}%
\end{pgfscope}%
\begin{pgfscope}%
\pgfpathrectangle{\pgfqpoint{0.380943in}{0.295988in}}{\pgfqpoint{4.650000in}{0.692553in}}%
\pgfusepath{clip}%
\pgfsetbuttcap%
\pgfsetroundjoin%
\definecolor{currentfill}{rgb}{1.000000,0.598462,0.528012}%
\pgfsetfillcolor{currentfill}%
\pgfsetlinewidth{0.250937pt}%
\definecolor{currentstroke}{rgb}{1.000000,1.000000,1.000000}%
\pgfsetstrokecolor{currentstroke}%
\pgfsetdash{}{0pt}%
\pgfpathmoveto{\pgfqpoint{1.271369in}{0.691732in}}%
\pgfpathlineto{\pgfqpoint{1.370305in}{0.691732in}}%
\pgfpathlineto{\pgfqpoint{1.370305in}{0.592796in}}%
\pgfpathlineto{\pgfqpoint{1.271369in}{0.592796in}}%
\pgfpathlineto{\pgfqpoint{1.271369in}{0.691732in}}%
\pgfusepath{stroke,fill}%
\end{pgfscope}%
\begin{pgfscope}%
\pgfpathrectangle{\pgfqpoint{0.380943in}{0.295988in}}{\pgfqpoint{4.650000in}{0.692553in}}%
\pgfusepath{clip}%
\pgfsetbuttcap%
\pgfsetroundjoin%
\definecolor{currentfill}{rgb}{0.989619,0.788235,0.628374}%
\pgfsetfillcolor{currentfill}%
\pgfsetlinewidth{0.250937pt}%
\definecolor{currentstroke}{rgb}{1.000000,1.000000,1.000000}%
\pgfsetstrokecolor{currentstroke}%
\pgfsetdash{}{0pt}%
\pgfpathmoveto{\pgfqpoint{1.370305in}{0.691732in}}%
\pgfpathlineto{\pgfqpoint{1.469241in}{0.691732in}}%
\pgfpathlineto{\pgfqpoint{1.469241in}{0.592796in}}%
\pgfpathlineto{\pgfqpoint{1.370305in}{0.592796in}}%
\pgfpathlineto{\pgfqpoint{1.370305in}{0.691732in}}%
\pgfusepath{stroke,fill}%
\end{pgfscope}%
\begin{pgfscope}%
\pgfpathrectangle{\pgfqpoint{0.380943in}{0.295988in}}{\pgfqpoint{4.650000in}{0.692553in}}%
\pgfusepath{clip}%
\pgfsetbuttcap%
\pgfsetroundjoin%
\definecolor{currentfill}{rgb}{1.000000,0.598462,0.528012}%
\pgfsetfillcolor{currentfill}%
\pgfsetlinewidth{0.250937pt}%
\definecolor{currentstroke}{rgb}{1.000000,1.000000,1.000000}%
\pgfsetstrokecolor{currentstroke}%
\pgfsetdash{}{0pt}%
\pgfpathmoveto{\pgfqpoint{1.469241in}{0.691732in}}%
\pgfpathlineto{\pgfqpoint{1.568177in}{0.691732in}}%
\pgfpathlineto{\pgfqpoint{1.568177in}{0.592796in}}%
\pgfpathlineto{\pgfqpoint{1.469241in}{0.592796in}}%
\pgfpathlineto{\pgfqpoint{1.469241in}{0.691732in}}%
\pgfusepath{stroke,fill}%
\end{pgfscope}%
\begin{pgfscope}%
\pgfpathrectangle{\pgfqpoint{0.380943in}{0.295988in}}{\pgfqpoint{4.650000in}{0.692553in}}%
\pgfusepath{clip}%
\pgfsetbuttcap%
\pgfsetroundjoin%
\definecolor{currentfill}{rgb}{1.000000,0.581546,0.521246}%
\pgfsetfillcolor{currentfill}%
\pgfsetlinewidth{0.250937pt}%
\definecolor{currentstroke}{rgb}{1.000000,1.000000,1.000000}%
\pgfsetstrokecolor{currentstroke}%
\pgfsetdash{}{0pt}%
\pgfpathmoveto{\pgfqpoint{1.568177in}{0.691732in}}%
\pgfpathlineto{\pgfqpoint{1.667113in}{0.691732in}}%
\pgfpathlineto{\pgfqpoint{1.667113in}{0.592796in}}%
\pgfpathlineto{\pgfqpoint{1.568177in}{0.592796in}}%
\pgfpathlineto{\pgfqpoint{1.568177in}{0.691732in}}%
\pgfusepath{stroke,fill}%
\end{pgfscope}%
\begin{pgfscope}%
\pgfpathrectangle{\pgfqpoint{0.380943in}{0.295988in}}{\pgfqpoint{4.650000in}{0.692553in}}%
\pgfusepath{clip}%
\pgfsetbuttcap%
\pgfsetroundjoin%
\definecolor{currentfill}{rgb}{0.998601,0.667759,0.560769}%
\pgfsetfillcolor{currentfill}%
\pgfsetlinewidth{0.250937pt}%
\definecolor{currentstroke}{rgb}{1.000000,1.000000,1.000000}%
\pgfsetstrokecolor{currentstroke}%
\pgfsetdash{}{0pt}%
\pgfpathmoveto{\pgfqpoint{1.667113in}{0.691732in}}%
\pgfpathlineto{\pgfqpoint{1.766049in}{0.691732in}}%
\pgfpathlineto{\pgfqpoint{1.766049in}{0.592796in}}%
\pgfpathlineto{\pgfqpoint{1.667113in}{0.592796in}}%
\pgfpathlineto{\pgfqpoint{1.667113in}{0.691732in}}%
\pgfusepath{stroke,fill}%
\end{pgfscope}%
\begin{pgfscope}%
\pgfpathrectangle{\pgfqpoint{0.380943in}{0.295988in}}{\pgfqpoint{4.650000in}{0.692553in}}%
\pgfusepath{clip}%
\pgfsetbuttcap%
\pgfsetroundjoin%
\definecolor{currentfill}{rgb}{0.800000,0.278431,0.278431}%
\pgfsetfillcolor{currentfill}%
\pgfsetlinewidth{0.250937pt}%
\definecolor{currentstroke}{rgb}{1.000000,1.000000,1.000000}%
\pgfsetstrokecolor{currentstroke}%
\pgfsetdash{}{0pt}%
\pgfpathmoveto{\pgfqpoint{1.766049in}{0.691732in}}%
\pgfpathlineto{\pgfqpoint{1.864986in}{0.691732in}}%
\pgfpathlineto{\pgfqpoint{1.864986in}{0.592796in}}%
\pgfpathlineto{\pgfqpoint{1.766049in}{0.592796in}}%
\pgfpathlineto{\pgfqpoint{1.766049in}{0.691732in}}%
\pgfusepath{stroke,fill}%
\end{pgfscope}%
\begin{pgfscope}%
\pgfpathrectangle{\pgfqpoint{0.380943in}{0.295988in}}{\pgfqpoint{4.650000in}{0.692553in}}%
\pgfusepath{clip}%
\pgfsetbuttcap%
\pgfsetroundjoin%
\definecolor{currentfill}{rgb}{0.994694,0.745098,0.602999}%
\pgfsetfillcolor{currentfill}%
\pgfsetlinewidth{0.250937pt}%
\definecolor{currentstroke}{rgb}{1.000000,1.000000,1.000000}%
\pgfsetstrokecolor{currentstroke}%
\pgfsetdash{}{0pt}%
\pgfpathmoveto{\pgfqpoint{1.864986in}{0.691732in}}%
\pgfpathlineto{\pgfqpoint{1.963922in}{0.691732in}}%
\pgfpathlineto{\pgfqpoint{1.963922in}{0.592796in}}%
\pgfpathlineto{\pgfqpoint{1.864986in}{0.592796in}}%
\pgfpathlineto{\pgfqpoint{1.864986in}{0.691732in}}%
\pgfusepath{stroke,fill}%
\end{pgfscope}%
\begin{pgfscope}%
\pgfpathrectangle{\pgfqpoint{0.380943in}{0.295988in}}{\pgfqpoint{4.650000in}{0.692553in}}%
\pgfusepath{clip}%
\pgfsetbuttcap%
\pgfsetroundjoin%
\definecolor{currentfill}{rgb}{0.997586,0.694148,0.574979}%
\pgfsetfillcolor{currentfill}%
\pgfsetlinewidth{0.250937pt}%
\definecolor{currentstroke}{rgb}{1.000000,1.000000,1.000000}%
\pgfsetstrokecolor{currentstroke}%
\pgfsetdash{}{0pt}%
\pgfpathmoveto{\pgfqpoint{1.963922in}{0.691732in}}%
\pgfpathlineto{\pgfqpoint{2.062858in}{0.691732in}}%
\pgfpathlineto{\pgfqpoint{2.062858in}{0.592796in}}%
\pgfpathlineto{\pgfqpoint{1.963922in}{0.592796in}}%
\pgfpathlineto{\pgfqpoint{1.963922in}{0.691732in}}%
\pgfusepath{stroke,fill}%
\end{pgfscope}%
\begin{pgfscope}%
\pgfpathrectangle{\pgfqpoint{0.380943in}{0.295988in}}{\pgfqpoint{4.650000in}{0.692553in}}%
\pgfusepath{clip}%
\pgfsetbuttcap%
\pgfsetroundjoin%
\definecolor{currentfill}{rgb}{0.994018,0.750850,0.606382}%
\pgfsetfillcolor{currentfill}%
\pgfsetlinewidth{0.250937pt}%
\definecolor{currentstroke}{rgb}{1.000000,1.000000,1.000000}%
\pgfsetstrokecolor{currentstroke}%
\pgfsetdash{}{0pt}%
\pgfpathmoveto{\pgfqpoint{2.062858in}{0.691732in}}%
\pgfpathlineto{\pgfqpoint{2.161794in}{0.691732in}}%
\pgfpathlineto{\pgfqpoint{2.161794in}{0.592796in}}%
\pgfpathlineto{\pgfqpoint{2.062858in}{0.592796in}}%
\pgfpathlineto{\pgfqpoint{2.062858in}{0.691732in}}%
\pgfusepath{stroke,fill}%
\end{pgfscope}%
\begin{pgfscope}%
\pgfpathrectangle{\pgfqpoint{0.380943in}{0.295988in}}{\pgfqpoint{4.650000in}{0.692553in}}%
\pgfusepath{clip}%
\pgfsetbuttcap%
\pgfsetroundjoin%
\definecolor{currentfill}{rgb}{0.996401,0.724937,0.591557}%
\pgfsetfillcolor{currentfill}%
\pgfsetlinewidth{0.250937pt}%
\definecolor{currentstroke}{rgb}{1.000000,1.000000,1.000000}%
\pgfsetstrokecolor{currentstroke}%
\pgfsetdash{}{0pt}%
\pgfpathmoveto{\pgfqpoint{2.161794in}{0.691732in}}%
\pgfpathlineto{\pgfqpoint{2.260730in}{0.691732in}}%
\pgfpathlineto{\pgfqpoint{2.260730in}{0.592796in}}%
\pgfpathlineto{\pgfqpoint{2.161794in}{0.592796in}}%
\pgfpathlineto{\pgfqpoint{2.161794in}{0.691732in}}%
\pgfusepath{stroke,fill}%
\end{pgfscope}%
\begin{pgfscope}%
\pgfpathrectangle{\pgfqpoint{0.380943in}{0.295988in}}{\pgfqpoint{4.650000in}{0.692553in}}%
\pgfusepath{clip}%
\pgfsetbuttcap%
\pgfsetroundjoin%
\definecolor{currentfill}{rgb}{0.964937,0.908651,0.713110}%
\pgfsetfillcolor{currentfill}%
\pgfsetlinewidth{0.250937pt}%
\definecolor{currentstroke}{rgb}{1.000000,1.000000,1.000000}%
\pgfsetstrokecolor{currentstroke}%
\pgfsetdash{}{0pt}%
\pgfpathmoveto{\pgfqpoint{2.260730in}{0.691732in}}%
\pgfpathlineto{\pgfqpoint{2.359666in}{0.691732in}}%
\pgfpathlineto{\pgfqpoint{2.359666in}{0.592796in}}%
\pgfpathlineto{\pgfqpoint{2.260730in}{0.592796in}}%
\pgfpathlineto{\pgfqpoint{2.260730in}{0.691732in}}%
\pgfusepath{stroke,fill}%
\end{pgfscope}%
\begin{pgfscope}%
\pgfpathrectangle{\pgfqpoint{0.380943in}{0.295988in}}{\pgfqpoint{4.650000in}{0.692553in}}%
\pgfusepath{clip}%
\pgfsetbuttcap%
\pgfsetroundjoin%
\definecolor{currentfill}{rgb}{0.977116,0.848181,0.679769}%
\pgfsetfillcolor{currentfill}%
\pgfsetlinewidth{0.250937pt}%
\definecolor{currentstroke}{rgb}{1.000000,1.000000,1.000000}%
\pgfsetstrokecolor{currentstroke}%
\pgfsetdash{}{0pt}%
\pgfpathmoveto{\pgfqpoint{2.359666in}{0.691732in}}%
\pgfpathlineto{\pgfqpoint{2.458603in}{0.691732in}}%
\pgfpathlineto{\pgfqpoint{2.458603in}{0.592796in}}%
\pgfpathlineto{\pgfqpoint{2.359666in}{0.592796in}}%
\pgfpathlineto{\pgfqpoint{2.359666in}{0.691732in}}%
\pgfusepath{stroke,fill}%
\end{pgfscope}%
\begin{pgfscope}%
\pgfpathrectangle{\pgfqpoint{0.380943in}{0.295988in}}{\pgfqpoint{4.650000in}{0.692553in}}%
\pgfusepath{clip}%
\pgfsetbuttcap%
\pgfsetroundjoin%
\definecolor{currentfill}{rgb}{0.991311,0.773856,0.619915}%
\pgfsetfillcolor{currentfill}%
\pgfsetlinewidth{0.250937pt}%
\definecolor{currentstroke}{rgb}{1.000000,1.000000,1.000000}%
\pgfsetstrokecolor{currentstroke}%
\pgfsetdash{}{0pt}%
\pgfpathmoveto{\pgfqpoint{2.458603in}{0.691732in}}%
\pgfpathlineto{\pgfqpoint{2.557539in}{0.691732in}}%
\pgfpathlineto{\pgfqpoint{2.557539in}{0.592796in}}%
\pgfpathlineto{\pgfqpoint{2.458603in}{0.592796in}}%
\pgfpathlineto{\pgfqpoint{2.458603in}{0.691732in}}%
\pgfusepath{stroke,fill}%
\end{pgfscope}%
\begin{pgfscope}%
\pgfpathrectangle{\pgfqpoint{0.380943in}{0.295988in}}{\pgfqpoint{4.650000in}{0.692553in}}%
\pgfusepath{clip}%
\pgfsetbuttcap%
\pgfsetroundjoin%
\definecolor{currentfill}{rgb}{0.994018,0.750850,0.606382}%
\pgfsetfillcolor{currentfill}%
\pgfsetlinewidth{0.250937pt}%
\definecolor{currentstroke}{rgb}{1.000000,1.000000,1.000000}%
\pgfsetstrokecolor{currentstroke}%
\pgfsetdash{}{0pt}%
\pgfpathmoveto{\pgfqpoint{2.557539in}{0.691732in}}%
\pgfpathlineto{\pgfqpoint{2.656475in}{0.691732in}}%
\pgfpathlineto{\pgfqpoint{2.656475in}{0.592796in}}%
\pgfpathlineto{\pgfqpoint{2.557539in}{0.592796in}}%
\pgfpathlineto{\pgfqpoint{2.557539in}{0.691732in}}%
\pgfusepath{stroke,fill}%
\end{pgfscope}%
\begin{pgfscope}%
\pgfpathrectangle{\pgfqpoint{0.380943in}{0.295988in}}{\pgfqpoint{4.650000in}{0.692553in}}%
\pgfusepath{clip}%
\pgfsetbuttcap%
\pgfsetroundjoin%
\definecolor{currentfill}{rgb}{1.000000,0.598462,0.528012}%
\pgfsetfillcolor{currentfill}%
\pgfsetlinewidth{0.250937pt}%
\definecolor{currentstroke}{rgb}{1.000000,1.000000,1.000000}%
\pgfsetstrokecolor{currentstroke}%
\pgfsetdash{}{0pt}%
\pgfpathmoveto{\pgfqpoint{2.656475in}{0.691732in}}%
\pgfpathlineto{\pgfqpoint{2.755411in}{0.691732in}}%
\pgfpathlineto{\pgfqpoint{2.755411in}{0.592796in}}%
\pgfpathlineto{\pgfqpoint{2.656475in}{0.592796in}}%
\pgfpathlineto{\pgfqpoint{2.656475in}{0.691732in}}%
\pgfusepath{stroke,fill}%
\end{pgfscope}%
\begin{pgfscope}%
\pgfpathrectangle{\pgfqpoint{0.380943in}{0.295988in}}{\pgfqpoint{4.650000in}{0.692553in}}%
\pgfusepath{clip}%
\pgfsetbuttcap%
\pgfsetroundjoin%
\definecolor{currentfill}{rgb}{0.989619,0.788235,0.628374}%
\pgfsetfillcolor{currentfill}%
\pgfsetlinewidth{0.250937pt}%
\definecolor{currentstroke}{rgb}{1.000000,1.000000,1.000000}%
\pgfsetstrokecolor{currentstroke}%
\pgfsetdash{}{0pt}%
\pgfpathmoveto{\pgfqpoint{2.755411in}{0.691732in}}%
\pgfpathlineto{\pgfqpoint{2.854347in}{0.691732in}}%
\pgfpathlineto{\pgfqpoint{2.854347in}{0.592796in}}%
\pgfpathlineto{\pgfqpoint{2.755411in}{0.592796in}}%
\pgfpathlineto{\pgfqpoint{2.755411in}{0.691732in}}%
\pgfusepath{stroke,fill}%
\end{pgfscope}%
\begin{pgfscope}%
\pgfpathrectangle{\pgfqpoint{0.380943in}{0.295988in}}{\pgfqpoint{4.650000in}{0.692553in}}%
\pgfusepath{clip}%
\pgfsetbuttcap%
\pgfsetroundjoin%
\definecolor{currentfill}{rgb}{0.975594,0.855363,0.684691}%
\pgfsetfillcolor{currentfill}%
\pgfsetlinewidth{0.250937pt}%
\definecolor{currentstroke}{rgb}{1.000000,1.000000,1.000000}%
\pgfsetstrokecolor{currentstroke}%
\pgfsetdash{}{0pt}%
\pgfpathmoveto{\pgfqpoint{2.854347in}{0.691732in}}%
\pgfpathlineto{\pgfqpoint{2.953283in}{0.691732in}}%
\pgfpathlineto{\pgfqpoint{2.953283in}{0.592796in}}%
\pgfpathlineto{\pgfqpoint{2.854347in}{0.592796in}}%
\pgfpathlineto{\pgfqpoint{2.854347in}{0.691732in}}%
\pgfusepath{stroke,fill}%
\end{pgfscope}%
\begin{pgfscope}%
\pgfpathrectangle{\pgfqpoint{0.380943in}{0.295988in}}{\pgfqpoint{4.650000in}{0.692553in}}%
\pgfusepath{clip}%
\pgfsetbuttcap%
\pgfsetroundjoin%
\definecolor{currentfill}{rgb}{1.000000,0.554479,0.510419}%
\pgfsetfillcolor{currentfill}%
\pgfsetlinewidth{0.250937pt}%
\definecolor{currentstroke}{rgb}{1.000000,1.000000,1.000000}%
\pgfsetstrokecolor{currentstroke}%
\pgfsetdash{}{0pt}%
\pgfpathmoveto{\pgfqpoint{2.953283in}{0.691732in}}%
\pgfpathlineto{\pgfqpoint{3.052220in}{0.691732in}}%
\pgfpathlineto{\pgfqpoint{3.052220in}{0.592796in}}%
\pgfpathlineto{\pgfqpoint{2.953283in}{0.592796in}}%
\pgfpathlineto{\pgfqpoint{2.953283in}{0.691732in}}%
\pgfusepath{stroke,fill}%
\end{pgfscope}%
\begin{pgfscope}%
\pgfpathrectangle{\pgfqpoint{0.380943in}{0.295988in}}{\pgfqpoint{4.650000in}{0.692553in}}%
\pgfusepath{clip}%
\pgfsetbuttcap%
\pgfsetroundjoin%
\definecolor{currentfill}{rgb}{0.980669,0.832787,0.665559}%
\pgfsetfillcolor{currentfill}%
\pgfsetlinewidth{0.250937pt}%
\definecolor{currentstroke}{rgb}{1.000000,1.000000,1.000000}%
\pgfsetstrokecolor{currentstroke}%
\pgfsetdash{}{0pt}%
\pgfpathmoveto{\pgfqpoint{3.052220in}{0.691732in}}%
\pgfpathlineto{\pgfqpoint{3.151156in}{0.691732in}}%
\pgfpathlineto{\pgfqpoint{3.151156in}{0.592796in}}%
\pgfpathlineto{\pgfqpoint{3.052220in}{0.592796in}}%
\pgfpathlineto{\pgfqpoint{3.052220in}{0.691732in}}%
\pgfusepath{stroke,fill}%
\end{pgfscope}%
\begin{pgfscope}%
\pgfpathrectangle{\pgfqpoint{0.380943in}{0.295988in}}{\pgfqpoint{4.650000in}{0.692553in}}%
\pgfusepath{clip}%
\pgfsetbuttcap%
\pgfsetroundjoin%
\definecolor{currentfill}{rgb}{0.984729,0.815194,0.649319}%
\pgfsetfillcolor{currentfill}%
\pgfsetlinewidth{0.250937pt}%
\definecolor{currentstroke}{rgb}{1.000000,1.000000,1.000000}%
\pgfsetstrokecolor{currentstroke}%
\pgfsetdash{}{0pt}%
\pgfpathmoveto{\pgfqpoint{3.151156in}{0.691732in}}%
\pgfpathlineto{\pgfqpoint{3.250092in}{0.691732in}}%
\pgfpathlineto{\pgfqpoint{3.250092in}{0.592796in}}%
\pgfpathlineto{\pgfqpoint{3.151156in}{0.592796in}}%
\pgfpathlineto{\pgfqpoint{3.151156in}{0.691732in}}%
\pgfusepath{stroke,fill}%
\end{pgfscope}%
\begin{pgfscope}%
\pgfpathrectangle{\pgfqpoint{0.380943in}{0.295988in}}{\pgfqpoint{4.650000in}{0.692553in}}%
\pgfusepath{clip}%
\pgfsetbuttcap%
\pgfsetroundjoin%
\definecolor{currentfill}{rgb}{0.998601,0.667759,0.560769}%
\pgfsetfillcolor{currentfill}%
\pgfsetlinewidth{0.250937pt}%
\definecolor{currentstroke}{rgb}{1.000000,1.000000,1.000000}%
\pgfsetstrokecolor{currentstroke}%
\pgfsetdash{}{0pt}%
\pgfpathmoveto{\pgfqpoint{3.250092in}{0.691732in}}%
\pgfpathlineto{\pgfqpoint{3.349028in}{0.691732in}}%
\pgfpathlineto{\pgfqpoint{3.349028in}{0.592796in}}%
\pgfpathlineto{\pgfqpoint{3.250092in}{0.592796in}}%
\pgfpathlineto{\pgfqpoint{3.250092in}{0.691732in}}%
\pgfusepath{stroke,fill}%
\end{pgfscope}%
\begin{pgfscope}%
\pgfpathrectangle{\pgfqpoint{0.380943in}{0.295988in}}{\pgfqpoint{4.650000in}{0.692553in}}%
\pgfusepath{clip}%
\pgfsetbuttcap%
\pgfsetroundjoin%
\definecolor{currentfill}{rgb}{0.998093,0.680953,0.567874}%
\pgfsetfillcolor{currentfill}%
\pgfsetlinewidth{0.250937pt}%
\definecolor{currentstroke}{rgb}{1.000000,1.000000,1.000000}%
\pgfsetstrokecolor{currentstroke}%
\pgfsetdash{}{0pt}%
\pgfpathmoveto{\pgfqpoint{3.349028in}{0.691732in}}%
\pgfpathlineto{\pgfqpoint{3.447964in}{0.691732in}}%
\pgfpathlineto{\pgfqpoint{3.447964in}{0.592796in}}%
\pgfpathlineto{\pgfqpoint{3.349028in}{0.592796in}}%
\pgfpathlineto{\pgfqpoint{3.349028in}{0.691732in}}%
\pgfusepath{stroke,fill}%
\end{pgfscope}%
\begin{pgfscope}%
\pgfpathrectangle{\pgfqpoint{0.380943in}{0.295988in}}{\pgfqpoint{4.650000in}{0.692553in}}%
\pgfusepath{clip}%
\pgfsetbuttcap%
\pgfsetroundjoin%
\definecolor{currentfill}{rgb}{1.000000,0.581546,0.521246}%
\pgfsetfillcolor{currentfill}%
\pgfsetlinewidth{0.250937pt}%
\definecolor{currentstroke}{rgb}{1.000000,1.000000,1.000000}%
\pgfsetstrokecolor{currentstroke}%
\pgfsetdash{}{0pt}%
\pgfpathmoveto{\pgfqpoint{3.447964in}{0.691732in}}%
\pgfpathlineto{\pgfqpoint{3.546901in}{0.691732in}}%
\pgfpathlineto{\pgfqpoint{3.546901in}{0.592796in}}%
\pgfpathlineto{\pgfqpoint{3.447964in}{0.592796in}}%
\pgfpathlineto{\pgfqpoint{3.447964in}{0.691732in}}%
\pgfusepath{stroke,fill}%
\end{pgfscope}%
\begin{pgfscope}%
\pgfpathrectangle{\pgfqpoint{0.380943in}{0.295988in}}{\pgfqpoint{4.650000in}{0.692553in}}%
\pgfusepath{clip}%
\pgfsetbuttcap%
\pgfsetroundjoin%
\definecolor{currentfill}{rgb}{0.995709,0.736471,0.597924}%
\pgfsetfillcolor{currentfill}%
\pgfsetlinewidth{0.250937pt}%
\definecolor{currentstroke}{rgb}{1.000000,1.000000,1.000000}%
\pgfsetstrokecolor{currentstroke}%
\pgfsetdash{}{0pt}%
\pgfpathmoveto{\pgfqpoint{3.546901in}{0.691732in}}%
\pgfpathlineto{\pgfqpoint{3.645837in}{0.691732in}}%
\pgfpathlineto{\pgfqpoint{3.645837in}{0.592796in}}%
\pgfpathlineto{\pgfqpoint{3.546901in}{0.592796in}}%
\pgfpathlineto{\pgfqpoint{3.546901in}{0.691732in}}%
\pgfusepath{stroke,fill}%
\end{pgfscope}%
\begin{pgfscope}%
\pgfpathrectangle{\pgfqpoint{0.380943in}{0.295988in}}{\pgfqpoint{4.650000in}{0.692553in}}%
\pgfusepath{clip}%
\pgfsetbuttcap%
\pgfsetroundjoin%
\definecolor{currentfill}{rgb}{0.998939,0.658962,0.556032}%
\pgfsetfillcolor{currentfill}%
\pgfsetlinewidth{0.250937pt}%
\definecolor{currentstroke}{rgb}{1.000000,1.000000,1.000000}%
\pgfsetstrokecolor{currentstroke}%
\pgfsetdash{}{0pt}%
\pgfpathmoveto{\pgfqpoint{3.645837in}{0.691732in}}%
\pgfpathlineto{\pgfqpoint{3.744773in}{0.691732in}}%
\pgfpathlineto{\pgfqpoint{3.744773in}{0.592796in}}%
\pgfpathlineto{\pgfqpoint{3.645837in}{0.592796in}}%
\pgfpathlineto{\pgfqpoint{3.645837in}{0.691732in}}%
\pgfusepath{stroke,fill}%
\end{pgfscope}%
\begin{pgfscope}%
\pgfpathrectangle{\pgfqpoint{0.380943in}{0.295988in}}{\pgfqpoint{4.650000in}{0.692553in}}%
\pgfusepath{clip}%
\pgfsetbuttcap%
\pgfsetroundjoin%
\definecolor{currentfill}{rgb}{0.998939,0.658962,0.556032}%
\pgfsetfillcolor{currentfill}%
\pgfsetlinewidth{0.250937pt}%
\definecolor{currentstroke}{rgb}{1.000000,1.000000,1.000000}%
\pgfsetstrokecolor{currentstroke}%
\pgfsetdash{}{0pt}%
\pgfpathmoveto{\pgfqpoint{3.744773in}{0.691732in}}%
\pgfpathlineto{\pgfqpoint{3.843709in}{0.691732in}}%
\pgfpathlineto{\pgfqpoint{3.843709in}{0.592796in}}%
\pgfpathlineto{\pgfqpoint{3.744773in}{0.592796in}}%
\pgfpathlineto{\pgfqpoint{3.744773in}{0.691732in}}%
\pgfusepath{stroke,fill}%
\end{pgfscope}%
\begin{pgfscope}%
\pgfpathrectangle{\pgfqpoint{0.380943in}{0.295988in}}{\pgfqpoint{4.650000in}{0.692553in}}%
\pgfusepath{clip}%
\pgfsetbuttcap%
\pgfsetroundjoin%
\definecolor{currentfill}{rgb}{0.808797,0.287228,0.287228}%
\pgfsetfillcolor{currentfill}%
\pgfsetlinewidth{0.250937pt}%
\definecolor{currentstroke}{rgb}{1.000000,1.000000,1.000000}%
\pgfsetstrokecolor{currentstroke}%
\pgfsetdash{}{0pt}%
\pgfpathmoveto{\pgfqpoint{3.843709in}{0.691732in}}%
\pgfpathlineto{\pgfqpoint{3.942645in}{0.691732in}}%
\pgfpathlineto{\pgfqpoint{3.942645in}{0.592796in}}%
\pgfpathlineto{\pgfqpoint{3.843709in}{0.592796in}}%
\pgfpathlineto{\pgfqpoint{3.843709in}{0.691732in}}%
\pgfusepath{stroke,fill}%
\end{pgfscope}%
\begin{pgfscope}%
\pgfpathrectangle{\pgfqpoint{0.380943in}{0.295988in}}{\pgfqpoint{4.650000in}{0.692553in}}%
\pgfusepath{clip}%
\pgfsetbuttcap%
\pgfsetroundjoin%
\definecolor{currentfill}{rgb}{1.000000,0.615379,0.534779}%
\pgfsetfillcolor{currentfill}%
\pgfsetlinewidth{0.250937pt}%
\definecolor{currentstroke}{rgb}{1.000000,1.000000,1.000000}%
\pgfsetstrokecolor{currentstroke}%
\pgfsetdash{}{0pt}%
\pgfpathmoveto{\pgfqpoint{3.942645in}{0.691732in}}%
\pgfpathlineto{\pgfqpoint{4.041581in}{0.691732in}}%
\pgfpathlineto{\pgfqpoint{4.041581in}{0.592796in}}%
\pgfpathlineto{\pgfqpoint{3.942645in}{0.592796in}}%
\pgfpathlineto{\pgfqpoint{3.942645in}{0.691732in}}%
\pgfusepath{stroke,fill}%
\end{pgfscope}%
\begin{pgfscope}%
\pgfpathrectangle{\pgfqpoint{0.380943in}{0.295988in}}{\pgfqpoint{4.650000in}{0.692553in}}%
\pgfusepath{clip}%
\pgfsetbuttcap%
\pgfsetroundjoin%
\definecolor{currentfill}{rgb}{0.970012,0.883276,0.699577}%
\pgfsetfillcolor{currentfill}%
\pgfsetlinewidth{0.250937pt}%
\definecolor{currentstroke}{rgb}{1.000000,1.000000,1.000000}%
\pgfsetstrokecolor{currentstroke}%
\pgfsetdash{}{0pt}%
\pgfpathmoveto{\pgfqpoint{4.041581in}{0.691732in}}%
\pgfpathlineto{\pgfqpoint{4.140518in}{0.691732in}}%
\pgfpathlineto{\pgfqpoint{4.140518in}{0.592796in}}%
\pgfpathlineto{\pgfqpoint{4.041581in}{0.592796in}}%
\pgfpathlineto{\pgfqpoint{4.041581in}{0.691732in}}%
\pgfusepath{stroke,fill}%
\end{pgfscope}%
\begin{pgfscope}%
\pgfpathrectangle{\pgfqpoint{0.380943in}{0.295988in}}{\pgfqpoint{4.650000in}{0.692553in}}%
\pgfusepath{clip}%
\pgfsetbuttcap%
\pgfsetroundjoin%
\definecolor{currentfill}{rgb}{0.995709,0.736471,0.597924}%
\pgfsetfillcolor{currentfill}%
\pgfsetlinewidth{0.250937pt}%
\definecolor{currentstroke}{rgb}{1.000000,1.000000,1.000000}%
\pgfsetstrokecolor{currentstroke}%
\pgfsetdash{}{0pt}%
\pgfpathmoveto{\pgfqpoint{4.140518in}{0.691732in}}%
\pgfpathlineto{\pgfqpoint{4.239454in}{0.691732in}}%
\pgfpathlineto{\pgfqpoint{4.239454in}{0.592796in}}%
\pgfpathlineto{\pgfqpoint{4.140518in}{0.592796in}}%
\pgfpathlineto{\pgfqpoint{4.140518in}{0.691732in}}%
\pgfusepath{stroke,fill}%
\end{pgfscope}%
\begin{pgfscope}%
\pgfpathrectangle{\pgfqpoint{0.380943in}{0.295988in}}{\pgfqpoint{4.650000in}{0.692553in}}%
\pgfusepath{clip}%
\pgfsetbuttcap%
\pgfsetroundjoin%
\definecolor{currentfill}{rgb}{1.000000,0.554479,0.510419}%
\pgfsetfillcolor{currentfill}%
\pgfsetlinewidth{0.250937pt}%
\definecolor{currentstroke}{rgb}{1.000000,1.000000,1.000000}%
\pgfsetstrokecolor{currentstroke}%
\pgfsetdash{}{0pt}%
\pgfpathmoveto{\pgfqpoint{4.239454in}{0.691732in}}%
\pgfpathlineto{\pgfqpoint{4.338390in}{0.691732in}}%
\pgfpathlineto{\pgfqpoint{4.338390in}{0.592796in}}%
\pgfpathlineto{\pgfqpoint{4.239454in}{0.592796in}}%
\pgfpathlineto{\pgfqpoint{4.239454in}{0.691732in}}%
\pgfusepath{stroke,fill}%
\end{pgfscope}%
\begin{pgfscope}%
\pgfpathrectangle{\pgfqpoint{0.380943in}{0.295988in}}{\pgfqpoint{4.650000in}{0.692553in}}%
\pgfusepath{clip}%
\pgfsetbuttcap%
\pgfsetroundjoin%
\definecolor{currentfill}{rgb}{0.974072,0.862976,0.688750}%
\pgfsetfillcolor{currentfill}%
\pgfsetlinewidth{0.250937pt}%
\definecolor{currentstroke}{rgb}{1.000000,1.000000,1.000000}%
\pgfsetstrokecolor{currentstroke}%
\pgfsetdash{}{0pt}%
\pgfpathmoveto{\pgfqpoint{4.338390in}{0.691732in}}%
\pgfpathlineto{\pgfqpoint{4.437326in}{0.691732in}}%
\pgfpathlineto{\pgfqpoint{4.437326in}{0.592796in}}%
\pgfpathlineto{\pgfqpoint{4.338390in}{0.592796in}}%
\pgfpathlineto{\pgfqpoint{4.338390in}{0.691732in}}%
\pgfusepath{stroke,fill}%
\end{pgfscope}%
\begin{pgfscope}%
\pgfpathrectangle{\pgfqpoint{0.380943in}{0.295988in}}{\pgfqpoint{4.650000in}{0.692553in}}%
\pgfusepath{clip}%
\pgfsetbuttcap%
\pgfsetroundjoin%
\definecolor{currentfill}{rgb}{0.975594,0.855363,0.684691}%
\pgfsetfillcolor{currentfill}%
\pgfsetlinewidth{0.250937pt}%
\definecolor{currentstroke}{rgb}{1.000000,1.000000,1.000000}%
\pgfsetstrokecolor{currentstroke}%
\pgfsetdash{}{0pt}%
\pgfpathmoveto{\pgfqpoint{4.437326in}{0.691732in}}%
\pgfpathlineto{\pgfqpoint{4.536262in}{0.691732in}}%
\pgfpathlineto{\pgfqpoint{4.536262in}{0.592796in}}%
\pgfpathlineto{\pgfqpoint{4.437326in}{0.592796in}}%
\pgfpathlineto{\pgfqpoint{4.437326in}{0.691732in}}%
\pgfusepath{stroke,fill}%
\end{pgfscope}%
\begin{pgfscope}%
\pgfpathrectangle{\pgfqpoint{0.380943in}{0.295988in}}{\pgfqpoint{4.650000in}{0.692553in}}%
\pgfusepath{clip}%
\pgfsetbuttcap%
\pgfsetroundjoin%
\definecolor{currentfill}{rgb}{0.983714,0.819592,0.653379}%
\pgfsetfillcolor{currentfill}%
\pgfsetlinewidth{0.250937pt}%
\definecolor{currentstroke}{rgb}{1.000000,1.000000,1.000000}%
\pgfsetstrokecolor{currentstroke}%
\pgfsetdash{}{0pt}%
\pgfpathmoveto{\pgfqpoint{4.536262in}{0.691732in}}%
\pgfpathlineto{\pgfqpoint{4.635198in}{0.691732in}}%
\pgfpathlineto{\pgfqpoint{4.635198in}{0.592796in}}%
\pgfpathlineto{\pgfqpoint{4.536262in}{0.592796in}}%
\pgfpathlineto{\pgfqpoint{4.536262in}{0.691732in}}%
\pgfusepath{stroke,fill}%
\end{pgfscope}%
\begin{pgfscope}%
\pgfpathrectangle{\pgfqpoint{0.380943in}{0.295988in}}{\pgfqpoint{4.650000in}{0.692553in}}%
\pgfusepath{clip}%
\pgfsetbuttcap%
\pgfsetroundjoin%
\definecolor{currentfill}{rgb}{0.966459,0.901038,0.709050}%
\pgfsetfillcolor{currentfill}%
\pgfsetlinewidth{0.250937pt}%
\definecolor{currentstroke}{rgb}{1.000000,1.000000,1.000000}%
\pgfsetstrokecolor{currentstroke}%
\pgfsetdash{}{0pt}%
\pgfpathmoveto{\pgfqpoint{4.635198in}{0.691732in}}%
\pgfpathlineto{\pgfqpoint{4.734135in}{0.691732in}}%
\pgfpathlineto{\pgfqpoint{4.734135in}{0.592796in}}%
\pgfpathlineto{\pgfqpoint{4.635198in}{0.592796in}}%
\pgfpathlineto{\pgfqpoint{4.635198in}{0.691732in}}%
\pgfusepath{stroke,fill}%
\end{pgfscope}%
\begin{pgfscope}%
\pgfpathrectangle{\pgfqpoint{0.380943in}{0.295988in}}{\pgfqpoint{4.650000in}{0.692553in}}%
\pgfusepath{clip}%
\pgfsetbuttcap%
\pgfsetroundjoin%
\definecolor{currentfill}{rgb}{0.973057,0.868051,0.691457}%
\pgfsetfillcolor{currentfill}%
\pgfsetlinewidth{0.250937pt}%
\definecolor{currentstroke}{rgb}{1.000000,1.000000,1.000000}%
\pgfsetstrokecolor{currentstroke}%
\pgfsetdash{}{0pt}%
\pgfpathmoveto{\pgfqpoint{4.734135in}{0.691732in}}%
\pgfpathlineto{\pgfqpoint{4.833071in}{0.691732in}}%
\pgfpathlineto{\pgfqpoint{4.833071in}{0.592796in}}%
\pgfpathlineto{\pgfqpoint{4.734135in}{0.592796in}}%
\pgfpathlineto{\pgfqpoint{4.734135in}{0.691732in}}%
\pgfusepath{stroke,fill}%
\end{pgfscope}%
\begin{pgfscope}%
\pgfpathrectangle{\pgfqpoint{0.380943in}{0.295988in}}{\pgfqpoint{4.650000in}{0.692553in}}%
\pgfusepath{clip}%
\pgfsetbuttcap%
\pgfsetroundjoin%
\definecolor{currentfill}{rgb}{0.983391,0.971765,0.788189}%
\pgfsetfillcolor{currentfill}%
\pgfsetlinewidth{0.250937pt}%
\definecolor{currentstroke}{rgb}{1.000000,1.000000,1.000000}%
\pgfsetstrokecolor{currentstroke}%
\pgfsetdash{}{0pt}%
\pgfpathmoveto{\pgfqpoint{4.833071in}{0.691732in}}%
\pgfpathlineto{\pgfqpoint{4.932007in}{0.691732in}}%
\pgfpathlineto{\pgfqpoint{4.932007in}{0.592796in}}%
\pgfpathlineto{\pgfqpoint{4.833071in}{0.592796in}}%
\pgfpathlineto{\pgfqpoint{4.833071in}{0.691732in}}%
\pgfusepath{stroke,fill}%
\end{pgfscope}%
\begin{pgfscope}%
\pgfpathrectangle{\pgfqpoint{0.380943in}{0.295988in}}{\pgfqpoint{4.650000in}{0.692553in}}%
\pgfusepath{clip}%
\pgfsetbuttcap%
\pgfsetroundjoin%
\pgfsetlinewidth{0.250937pt}%
\definecolor{currentstroke}{rgb}{1.000000,1.000000,1.000000}%
\pgfsetstrokecolor{currentstroke}%
\pgfsetdash{}{0pt}%
\pgfpathmoveto{\pgfqpoint{4.932007in}{0.691732in}}%
\pgfpathlineto{\pgfqpoint{5.030943in}{0.691732in}}%
\pgfpathlineto{\pgfqpoint{5.030943in}{0.592796in}}%
\pgfpathlineto{\pgfqpoint{4.932007in}{0.592796in}}%
\pgfpathlineto{\pgfqpoint{4.932007in}{0.691732in}}%
\pgfusepath{stroke}%
\end{pgfscope}%
\begin{pgfscope}%
\pgfpathrectangle{\pgfqpoint{0.380943in}{0.295988in}}{\pgfqpoint{4.650000in}{0.692553in}}%
\pgfusepath{clip}%
\pgfsetbuttcap%
\pgfsetroundjoin%
\definecolor{currentfill}{rgb}{1.000000,1.000000,0.908266}%
\pgfsetfillcolor{currentfill}%
\pgfsetlinewidth{0.250937pt}%
\definecolor{currentstroke}{rgb}{1.000000,1.000000,1.000000}%
\pgfsetstrokecolor{currentstroke}%
\pgfsetdash{}{0pt}%
\pgfpathmoveto{\pgfqpoint{0.380943in}{0.592796in}}%
\pgfpathlineto{\pgfqpoint{0.479879in}{0.592796in}}%
\pgfpathlineto{\pgfqpoint{0.479879in}{0.493860in}}%
\pgfpathlineto{\pgfqpoint{0.380943in}{0.493860in}}%
\pgfpathlineto{\pgfqpoint{0.380943in}{0.592796in}}%
\pgfusepath{stroke,fill}%
\end{pgfscope}%
\begin{pgfscope}%
\pgfpathrectangle{\pgfqpoint{0.380943in}{0.295988in}}{\pgfqpoint{4.650000in}{0.692553in}}%
\pgfusepath{clip}%
\pgfsetbuttcap%
\pgfsetroundjoin%
\definecolor{currentfill}{rgb}{1.000000,1.000000,0.865975}%
\pgfsetfillcolor{currentfill}%
\pgfsetlinewidth{0.250937pt}%
\definecolor{currentstroke}{rgb}{1.000000,1.000000,1.000000}%
\pgfsetstrokecolor{currentstroke}%
\pgfsetdash{}{0pt}%
\pgfpathmoveto{\pgfqpoint{0.479879in}{0.592796in}}%
\pgfpathlineto{\pgfqpoint{0.578815in}{0.592796in}}%
\pgfpathlineto{\pgfqpoint{0.578815in}{0.493860in}}%
\pgfpathlineto{\pgfqpoint{0.479879in}{0.493860in}}%
\pgfpathlineto{\pgfqpoint{0.479879in}{0.592796in}}%
\pgfusepath{stroke,fill}%
\end{pgfscope}%
\begin{pgfscope}%
\pgfpathrectangle{\pgfqpoint{0.380943in}{0.295988in}}{\pgfqpoint{4.650000in}{0.692553in}}%
\pgfusepath{clip}%
\pgfsetbuttcap%
\pgfsetroundjoin%
\definecolor{currentfill}{rgb}{0.973241,0.954510,0.761799}%
\pgfsetfillcolor{currentfill}%
\pgfsetlinewidth{0.250937pt}%
\definecolor{currentstroke}{rgb}{1.000000,1.000000,1.000000}%
\pgfsetstrokecolor{currentstroke}%
\pgfsetdash{}{0pt}%
\pgfpathmoveto{\pgfqpoint{0.578815in}{0.592796in}}%
\pgfpathlineto{\pgfqpoint{0.677752in}{0.592796in}}%
\pgfpathlineto{\pgfqpoint{0.677752in}{0.493860in}}%
\pgfpathlineto{\pgfqpoint{0.578815in}{0.493860in}}%
\pgfpathlineto{\pgfqpoint{0.578815in}{0.592796in}}%
\pgfusepath{stroke,fill}%
\end{pgfscope}%
\begin{pgfscope}%
\pgfpathrectangle{\pgfqpoint{0.380943in}{0.295988in}}{\pgfqpoint{4.650000in}{0.692553in}}%
\pgfusepath{clip}%
\pgfsetbuttcap%
\pgfsetroundjoin%
\definecolor{currentfill}{rgb}{0.978316,0.963137,0.774994}%
\pgfsetfillcolor{currentfill}%
\pgfsetlinewidth{0.250937pt}%
\definecolor{currentstroke}{rgb}{1.000000,1.000000,1.000000}%
\pgfsetstrokecolor{currentstroke}%
\pgfsetdash{}{0pt}%
\pgfpathmoveto{\pgfqpoint{0.677752in}{0.592796in}}%
\pgfpathlineto{\pgfqpoint{0.776688in}{0.592796in}}%
\pgfpathlineto{\pgfqpoint{0.776688in}{0.493860in}}%
\pgfpathlineto{\pgfqpoint{0.677752in}{0.493860in}}%
\pgfpathlineto{\pgfqpoint{0.677752in}{0.592796in}}%
\pgfusepath{stroke,fill}%
\end{pgfscope}%
\begin{pgfscope}%
\pgfpathrectangle{\pgfqpoint{0.380943in}{0.295988in}}{\pgfqpoint{4.650000in}{0.692553in}}%
\pgfusepath{clip}%
\pgfsetbuttcap%
\pgfsetroundjoin%
\definecolor{currentfill}{rgb}{0.986774,0.977516,0.796986}%
\pgfsetfillcolor{currentfill}%
\pgfsetlinewidth{0.250937pt}%
\definecolor{currentstroke}{rgb}{1.000000,1.000000,1.000000}%
\pgfsetstrokecolor{currentstroke}%
\pgfsetdash{}{0pt}%
\pgfpathmoveto{\pgfqpoint{0.776688in}{0.592796in}}%
\pgfpathlineto{\pgfqpoint{0.875624in}{0.592796in}}%
\pgfpathlineto{\pgfqpoint{0.875624in}{0.493860in}}%
\pgfpathlineto{\pgfqpoint{0.776688in}{0.493860in}}%
\pgfpathlineto{\pgfqpoint{0.776688in}{0.592796in}}%
\pgfusepath{stroke,fill}%
\end{pgfscope}%
\begin{pgfscope}%
\pgfpathrectangle{\pgfqpoint{0.380943in}{0.295988in}}{\pgfqpoint{4.650000in}{0.692553in}}%
\pgfusepath{clip}%
\pgfsetbuttcap%
\pgfsetroundjoin%
\definecolor{currentfill}{rgb}{0.986774,0.977516,0.796986}%
\pgfsetfillcolor{currentfill}%
\pgfsetlinewidth{0.250937pt}%
\definecolor{currentstroke}{rgb}{1.000000,1.000000,1.000000}%
\pgfsetstrokecolor{currentstroke}%
\pgfsetdash{}{0pt}%
\pgfpathmoveto{\pgfqpoint{0.875624in}{0.592796in}}%
\pgfpathlineto{\pgfqpoint{0.974560in}{0.592796in}}%
\pgfpathlineto{\pgfqpoint{0.974560in}{0.493860in}}%
\pgfpathlineto{\pgfqpoint{0.875624in}{0.493860in}}%
\pgfpathlineto{\pgfqpoint{0.875624in}{0.592796in}}%
\pgfusepath{stroke,fill}%
\end{pgfscope}%
\begin{pgfscope}%
\pgfpathrectangle{\pgfqpoint{0.380943in}{0.295988in}}{\pgfqpoint{4.650000in}{0.692553in}}%
\pgfusepath{clip}%
\pgfsetbuttcap%
\pgfsetroundjoin%
\definecolor{currentfill}{rgb}{0.986774,0.977516,0.796986}%
\pgfsetfillcolor{currentfill}%
\pgfsetlinewidth{0.250937pt}%
\definecolor{currentstroke}{rgb}{1.000000,1.000000,1.000000}%
\pgfsetstrokecolor{currentstroke}%
\pgfsetdash{}{0pt}%
\pgfpathmoveto{\pgfqpoint{0.974560in}{0.592796in}}%
\pgfpathlineto{\pgfqpoint{1.073496in}{0.592796in}}%
\pgfpathlineto{\pgfqpoint{1.073496in}{0.493860in}}%
\pgfpathlineto{\pgfqpoint{0.974560in}{0.493860in}}%
\pgfpathlineto{\pgfqpoint{0.974560in}{0.592796in}}%
\pgfusepath{stroke,fill}%
\end{pgfscope}%
\begin{pgfscope}%
\pgfpathrectangle{\pgfqpoint{0.380943in}{0.295988in}}{\pgfqpoint{4.650000in}{0.692553in}}%
\pgfusepath{clip}%
\pgfsetbuttcap%
\pgfsetroundjoin%
\definecolor{currentfill}{rgb}{0.973241,0.954510,0.761799}%
\pgfsetfillcolor{currentfill}%
\pgfsetlinewidth{0.250937pt}%
\definecolor{currentstroke}{rgb}{1.000000,1.000000,1.000000}%
\pgfsetstrokecolor{currentstroke}%
\pgfsetdash{}{0pt}%
\pgfpathmoveto{\pgfqpoint{1.073496in}{0.592796in}}%
\pgfpathlineto{\pgfqpoint{1.172432in}{0.592796in}}%
\pgfpathlineto{\pgfqpoint{1.172432in}{0.493860in}}%
\pgfpathlineto{\pgfqpoint{1.073496in}{0.493860in}}%
\pgfpathlineto{\pgfqpoint{1.073496in}{0.592796in}}%
\pgfusepath{stroke,fill}%
\end{pgfscope}%
\begin{pgfscope}%
\pgfpathrectangle{\pgfqpoint{0.380943in}{0.295988in}}{\pgfqpoint{4.650000in}{0.692553in}}%
\pgfusepath{clip}%
\pgfsetbuttcap%
\pgfsetroundjoin%
\definecolor{currentfill}{rgb}{0.978316,0.963137,0.774994}%
\pgfsetfillcolor{currentfill}%
\pgfsetlinewidth{0.250937pt}%
\definecolor{currentstroke}{rgb}{1.000000,1.000000,1.000000}%
\pgfsetstrokecolor{currentstroke}%
\pgfsetdash{}{0pt}%
\pgfpathmoveto{\pgfqpoint{1.172432in}{0.592796in}}%
\pgfpathlineto{\pgfqpoint{1.271369in}{0.592796in}}%
\pgfpathlineto{\pgfqpoint{1.271369in}{0.493860in}}%
\pgfpathlineto{\pgfqpoint{1.172432in}{0.493860in}}%
\pgfpathlineto{\pgfqpoint{1.172432in}{0.592796in}}%
\pgfusepath{stroke,fill}%
\end{pgfscope}%
\begin{pgfscope}%
\pgfpathrectangle{\pgfqpoint{0.380943in}{0.295988in}}{\pgfqpoint{4.650000in}{0.692553in}}%
\pgfusepath{clip}%
\pgfsetbuttcap%
\pgfsetroundjoin%
\definecolor{currentfill}{rgb}{0.996740,0.716140,0.586820}%
\pgfsetfillcolor{currentfill}%
\pgfsetlinewidth{0.250937pt}%
\definecolor{currentstroke}{rgb}{1.000000,1.000000,1.000000}%
\pgfsetstrokecolor{currentstroke}%
\pgfsetdash{}{0pt}%
\pgfpathmoveto{\pgfqpoint{1.271369in}{0.592796in}}%
\pgfpathlineto{\pgfqpoint{1.370305in}{0.592796in}}%
\pgfpathlineto{\pgfqpoint{1.370305in}{0.493860in}}%
\pgfpathlineto{\pgfqpoint{1.271369in}{0.493860in}}%
\pgfpathlineto{\pgfqpoint{1.271369in}{0.592796in}}%
\pgfusepath{stroke,fill}%
\end{pgfscope}%
\begin{pgfscope}%
\pgfpathrectangle{\pgfqpoint{0.380943in}{0.295988in}}{\pgfqpoint{4.650000in}{0.692553in}}%
\pgfusepath{clip}%
\pgfsetbuttcap%
\pgfsetroundjoin%
\definecolor{currentfill}{rgb}{0.986251,0.808597,0.643230}%
\pgfsetfillcolor{currentfill}%
\pgfsetlinewidth{0.250937pt}%
\definecolor{currentstroke}{rgb}{1.000000,1.000000,1.000000}%
\pgfsetstrokecolor{currentstroke}%
\pgfsetdash{}{0pt}%
\pgfpathmoveto{\pgfqpoint{1.370305in}{0.592796in}}%
\pgfpathlineto{\pgfqpoint{1.469241in}{0.592796in}}%
\pgfpathlineto{\pgfqpoint{1.469241in}{0.493860in}}%
\pgfpathlineto{\pgfqpoint{1.370305in}{0.493860in}}%
\pgfpathlineto{\pgfqpoint{1.370305in}{0.592796in}}%
\pgfusepath{stroke,fill}%
\end{pgfscope}%
\begin{pgfscope}%
\pgfpathrectangle{\pgfqpoint{0.380943in}{0.295988in}}{\pgfqpoint{4.650000in}{0.692553in}}%
\pgfusepath{clip}%
\pgfsetbuttcap%
\pgfsetroundjoin%
\definecolor{currentfill}{rgb}{0.997247,0.702945,0.579715}%
\pgfsetfillcolor{currentfill}%
\pgfsetlinewidth{0.250937pt}%
\definecolor{currentstroke}{rgb}{1.000000,1.000000,1.000000}%
\pgfsetstrokecolor{currentstroke}%
\pgfsetdash{}{0pt}%
\pgfpathmoveto{\pgfqpoint{1.469241in}{0.592796in}}%
\pgfpathlineto{\pgfqpoint{1.568177in}{0.592796in}}%
\pgfpathlineto{\pgfqpoint{1.568177in}{0.493860in}}%
\pgfpathlineto{\pgfqpoint{1.469241in}{0.493860in}}%
\pgfpathlineto{\pgfqpoint{1.469241in}{0.592796in}}%
\pgfusepath{stroke,fill}%
\end{pgfscope}%
\begin{pgfscope}%
\pgfpathrectangle{\pgfqpoint{0.380943in}{0.295988in}}{\pgfqpoint{4.650000in}{0.692553in}}%
\pgfusepath{clip}%
\pgfsetbuttcap%
\pgfsetroundjoin%
\definecolor{currentfill}{rgb}{0.999446,0.645767,0.548927}%
\pgfsetfillcolor{currentfill}%
\pgfsetlinewidth{0.250937pt}%
\definecolor{currentstroke}{rgb}{1.000000,1.000000,1.000000}%
\pgfsetstrokecolor{currentstroke}%
\pgfsetdash{}{0pt}%
\pgfpathmoveto{\pgfqpoint{1.568177in}{0.592796in}}%
\pgfpathlineto{\pgfqpoint{1.667113in}{0.592796in}}%
\pgfpathlineto{\pgfqpoint{1.667113in}{0.493860in}}%
\pgfpathlineto{\pgfqpoint{1.568177in}{0.493860in}}%
\pgfpathlineto{\pgfqpoint{1.568177in}{0.592796in}}%
\pgfusepath{stroke,fill}%
\end{pgfscope}%
\begin{pgfscope}%
\pgfpathrectangle{\pgfqpoint{0.380943in}{0.295988in}}{\pgfqpoint{4.650000in}{0.692553in}}%
\pgfusepath{clip}%
\pgfsetbuttcap%
\pgfsetroundjoin%
\definecolor{currentfill}{rgb}{0.997247,0.702945,0.579715}%
\pgfsetfillcolor{currentfill}%
\pgfsetlinewidth{0.250937pt}%
\definecolor{currentstroke}{rgb}{1.000000,1.000000,1.000000}%
\pgfsetstrokecolor{currentstroke}%
\pgfsetdash{}{0pt}%
\pgfpathmoveto{\pgfqpoint{1.667113in}{0.592796in}}%
\pgfpathlineto{\pgfqpoint{1.766049in}{0.592796in}}%
\pgfpathlineto{\pgfqpoint{1.766049in}{0.493860in}}%
\pgfpathlineto{\pgfqpoint{1.667113in}{0.493860in}}%
\pgfpathlineto{\pgfqpoint{1.667113in}{0.592796in}}%
\pgfusepath{stroke,fill}%
\end{pgfscope}%
\begin{pgfscope}%
\pgfpathrectangle{\pgfqpoint{0.380943in}{0.295988in}}{\pgfqpoint{4.650000in}{0.692553in}}%
\pgfusepath{clip}%
\pgfsetbuttcap%
\pgfsetroundjoin%
\definecolor{currentfill}{rgb}{0.994694,0.745098,0.602999}%
\pgfsetfillcolor{currentfill}%
\pgfsetlinewidth{0.250937pt}%
\definecolor{currentstroke}{rgb}{1.000000,1.000000,1.000000}%
\pgfsetstrokecolor{currentstroke}%
\pgfsetdash{}{0pt}%
\pgfpathmoveto{\pgfqpoint{1.766049in}{0.592796in}}%
\pgfpathlineto{\pgfqpoint{1.864986in}{0.592796in}}%
\pgfpathlineto{\pgfqpoint{1.864986in}{0.493860in}}%
\pgfpathlineto{\pgfqpoint{1.766049in}{0.493860in}}%
\pgfpathlineto{\pgfqpoint{1.766049in}{0.592796in}}%
\pgfusepath{stroke,fill}%
\end{pgfscope}%
\begin{pgfscope}%
\pgfpathrectangle{\pgfqpoint{0.380943in}{0.295988in}}{\pgfqpoint{4.650000in}{0.692553in}}%
\pgfusepath{clip}%
\pgfsetbuttcap%
\pgfsetroundjoin%
\definecolor{currentfill}{rgb}{0.998939,0.658962,0.556032}%
\pgfsetfillcolor{currentfill}%
\pgfsetlinewidth{0.250937pt}%
\definecolor{currentstroke}{rgb}{1.000000,1.000000,1.000000}%
\pgfsetstrokecolor{currentstroke}%
\pgfsetdash{}{0pt}%
\pgfpathmoveto{\pgfqpoint{1.864986in}{0.592796in}}%
\pgfpathlineto{\pgfqpoint{1.963922in}{0.592796in}}%
\pgfpathlineto{\pgfqpoint{1.963922in}{0.493860in}}%
\pgfpathlineto{\pgfqpoint{1.864986in}{0.493860in}}%
\pgfpathlineto{\pgfqpoint{1.864986in}{0.592796in}}%
\pgfusepath{stroke,fill}%
\end{pgfscope}%
\begin{pgfscope}%
\pgfpathrectangle{\pgfqpoint{0.380943in}{0.295988in}}{\pgfqpoint{4.650000in}{0.692553in}}%
\pgfusepath{clip}%
\pgfsetbuttcap%
\pgfsetroundjoin%
\definecolor{currentfill}{rgb}{0.995709,0.736471,0.597924}%
\pgfsetfillcolor{currentfill}%
\pgfsetlinewidth{0.250937pt}%
\definecolor{currentstroke}{rgb}{1.000000,1.000000,1.000000}%
\pgfsetstrokecolor{currentstroke}%
\pgfsetdash{}{0pt}%
\pgfpathmoveto{\pgfqpoint{1.963922in}{0.592796in}}%
\pgfpathlineto{\pgfqpoint{2.062858in}{0.592796in}}%
\pgfpathlineto{\pgfqpoint{2.062858in}{0.493860in}}%
\pgfpathlineto{\pgfqpoint{1.963922in}{0.493860in}}%
\pgfpathlineto{\pgfqpoint{1.963922in}{0.592796in}}%
\pgfusepath{stroke,fill}%
\end{pgfscope}%
\begin{pgfscope}%
\pgfpathrectangle{\pgfqpoint{0.380943in}{0.295988in}}{\pgfqpoint{4.650000in}{0.692553in}}%
\pgfusepath{clip}%
\pgfsetbuttcap%
\pgfsetroundjoin%
\definecolor{currentfill}{rgb}{0.988604,0.796863,0.633449}%
\pgfsetfillcolor{currentfill}%
\pgfsetlinewidth{0.250937pt}%
\definecolor{currentstroke}{rgb}{1.000000,1.000000,1.000000}%
\pgfsetstrokecolor{currentstroke}%
\pgfsetdash{}{0pt}%
\pgfpathmoveto{\pgfqpoint{2.062858in}{0.592796in}}%
\pgfpathlineto{\pgfqpoint{2.161794in}{0.592796in}}%
\pgfpathlineto{\pgfqpoint{2.161794in}{0.493860in}}%
\pgfpathlineto{\pgfqpoint{2.062858in}{0.493860in}}%
\pgfpathlineto{\pgfqpoint{2.062858in}{0.592796in}}%
\pgfusepath{stroke,fill}%
\end{pgfscope}%
\begin{pgfscope}%
\pgfpathrectangle{\pgfqpoint{0.380943in}{0.295988in}}{\pgfqpoint{4.650000in}{0.692553in}}%
\pgfusepath{clip}%
\pgfsetbuttcap%
\pgfsetroundjoin%
\definecolor{currentfill}{rgb}{0.983714,0.819592,0.653379}%
\pgfsetfillcolor{currentfill}%
\pgfsetlinewidth{0.250937pt}%
\definecolor{currentstroke}{rgb}{1.000000,1.000000,1.000000}%
\pgfsetstrokecolor{currentstroke}%
\pgfsetdash{}{0pt}%
\pgfpathmoveto{\pgfqpoint{2.161794in}{0.592796in}}%
\pgfpathlineto{\pgfqpoint{2.260730in}{0.592796in}}%
\pgfpathlineto{\pgfqpoint{2.260730in}{0.493860in}}%
\pgfpathlineto{\pgfqpoint{2.161794in}{0.493860in}}%
\pgfpathlineto{\pgfqpoint{2.161794in}{0.592796in}}%
\pgfusepath{stroke,fill}%
\end{pgfscope}%
\begin{pgfscope}%
\pgfpathrectangle{\pgfqpoint{0.380943in}{0.295988in}}{\pgfqpoint{4.650000in}{0.692553in}}%
\pgfusepath{clip}%
\pgfsetbuttcap%
\pgfsetroundjoin%
\definecolor{currentfill}{rgb}{0.982191,0.826190,0.659469}%
\pgfsetfillcolor{currentfill}%
\pgfsetlinewidth{0.250937pt}%
\definecolor{currentstroke}{rgb}{1.000000,1.000000,1.000000}%
\pgfsetstrokecolor{currentstroke}%
\pgfsetdash{}{0pt}%
\pgfpathmoveto{\pgfqpoint{2.260730in}{0.592796in}}%
\pgfpathlineto{\pgfqpoint{2.359666in}{0.592796in}}%
\pgfpathlineto{\pgfqpoint{2.359666in}{0.493860in}}%
\pgfpathlineto{\pgfqpoint{2.260730in}{0.493860in}}%
\pgfpathlineto{\pgfqpoint{2.260730in}{0.592796in}}%
\pgfusepath{stroke,fill}%
\end{pgfscope}%
\begin{pgfscope}%
\pgfpathrectangle{\pgfqpoint{0.380943in}{0.295988in}}{\pgfqpoint{4.650000in}{0.692553in}}%
\pgfusepath{clip}%
\pgfsetbuttcap%
\pgfsetroundjoin%
\definecolor{currentfill}{rgb}{0.975594,0.855363,0.684691}%
\pgfsetfillcolor{currentfill}%
\pgfsetlinewidth{0.250937pt}%
\definecolor{currentstroke}{rgb}{1.000000,1.000000,1.000000}%
\pgfsetstrokecolor{currentstroke}%
\pgfsetdash{}{0pt}%
\pgfpathmoveto{\pgfqpoint{2.359666in}{0.592796in}}%
\pgfpathlineto{\pgfqpoint{2.458603in}{0.592796in}}%
\pgfpathlineto{\pgfqpoint{2.458603in}{0.493860in}}%
\pgfpathlineto{\pgfqpoint{2.359666in}{0.493860in}}%
\pgfpathlineto{\pgfqpoint{2.359666in}{0.592796in}}%
\pgfusepath{stroke,fill}%
\end{pgfscope}%
\begin{pgfscope}%
\pgfpathrectangle{\pgfqpoint{0.380943in}{0.295988in}}{\pgfqpoint{4.650000in}{0.692553in}}%
\pgfusepath{clip}%
\pgfsetbuttcap%
\pgfsetroundjoin%
\definecolor{currentfill}{rgb}{0.996740,0.716140,0.586820}%
\pgfsetfillcolor{currentfill}%
\pgfsetlinewidth{0.250937pt}%
\definecolor{currentstroke}{rgb}{1.000000,1.000000,1.000000}%
\pgfsetstrokecolor{currentstroke}%
\pgfsetdash{}{0pt}%
\pgfpathmoveto{\pgfqpoint{2.458603in}{0.592796in}}%
\pgfpathlineto{\pgfqpoint{2.557539in}{0.592796in}}%
\pgfpathlineto{\pgfqpoint{2.557539in}{0.493860in}}%
\pgfpathlineto{\pgfqpoint{2.458603in}{0.493860in}}%
\pgfpathlineto{\pgfqpoint{2.458603in}{0.592796in}}%
\pgfusepath{stroke,fill}%
\end{pgfscope}%
\begin{pgfscope}%
\pgfpathrectangle{\pgfqpoint{0.380943in}{0.295988in}}{\pgfqpoint{4.650000in}{0.692553in}}%
\pgfusepath{clip}%
\pgfsetbuttcap%
\pgfsetroundjoin%
\definecolor{currentfill}{rgb}{1.000000,0.554479,0.510419}%
\pgfsetfillcolor{currentfill}%
\pgfsetlinewidth{0.250937pt}%
\definecolor{currentstroke}{rgb}{1.000000,1.000000,1.000000}%
\pgfsetstrokecolor{currentstroke}%
\pgfsetdash{}{0pt}%
\pgfpathmoveto{\pgfqpoint{2.557539in}{0.592796in}}%
\pgfpathlineto{\pgfqpoint{2.656475in}{0.592796in}}%
\pgfpathlineto{\pgfqpoint{2.656475in}{0.493860in}}%
\pgfpathlineto{\pgfqpoint{2.557539in}{0.493860in}}%
\pgfpathlineto{\pgfqpoint{2.557539in}{0.592796in}}%
\pgfusepath{stroke,fill}%
\end{pgfscope}%
\begin{pgfscope}%
\pgfpathrectangle{\pgfqpoint{0.380943in}{0.295988in}}{\pgfqpoint{4.650000in}{0.692553in}}%
\pgfusepath{clip}%
\pgfsetbuttcap%
\pgfsetroundjoin%
\definecolor{currentfill}{rgb}{0.989619,0.788235,0.628374}%
\pgfsetfillcolor{currentfill}%
\pgfsetlinewidth{0.250937pt}%
\definecolor{currentstroke}{rgb}{1.000000,1.000000,1.000000}%
\pgfsetstrokecolor{currentstroke}%
\pgfsetdash{}{0pt}%
\pgfpathmoveto{\pgfqpoint{2.656475in}{0.592796in}}%
\pgfpathlineto{\pgfqpoint{2.755411in}{0.592796in}}%
\pgfpathlineto{\pgfqpoint{2.755411in}{0.493860in}}%
\pgfpathlineto{\pgfqpoint{2.656475in}{0.493860in}}%
\pgfpathlineto{\pgfqpoint{2.656475in}{0.592796in}}%
\pgfusepath{stroke,fill}%
\end{pgfscope}%
\begin{pgfscope}%
\pgfpathrectangle{\pgfqpoint{0.380943in}{0.295988in}}{\pgfqpoint{4.650000in}{0.692553in}}%
\pgfusepath{clip}%
\pgfsetbuttcap%
\pgfsetroundjoin%
\definecolor{currentfill}{rgb}{0.995709,0.736471,0.597924}%
\pgfsetfillcolor{currentfill}%
\pgfsetlinewidth{0.250937pt}%
\definecolor{currentstroke}{rgb}{1.000000,1.000000,1.000000}%
\pgfsetstrokecolor{currentstroke}%
\pgfsetdash{}{0pt}%
\pgfpathmoveto{\pgfqpoint{2.755411in}{0.592796in}}%
\pgfpathlineto{\pgfqpoint{2.854347in}{0.592796in}}%
\pgfpathlineto{\pgfqpoint{2.854347in}{0.493860in}}%
\pgfpathlineto{\pgfqpoint{2.755411in}{0.493860in}}%
\pgfpathlineto{\pgfqpoint{2.755411in}{0.592796in}}%
\pgfusepath{stroke,fill}%
\end{pgfscope}%
\begin{pgfscope}%
\pgfpathrectangle{\pgfqpoint{0.380943in}{0.295988in}}{\pgfqpoint{4.650000in}{0.692553in}}%
\pgfusepath{clip}%
\pgfsetbuttcap%
\pgfsetroundjoin%
\definecolor{currentfill}{rgb}{0.993003,0.759477,0.611457}%
\pgfsetfillcolor{currentfill}%
\pgfsetlinewidth{0.250937pt}%
\definecolor{currentstroke}{rgb}{1.000000,1.000000,1.000000}%
\pgfsetstrokecolor{currentstroke}%
\pgfsetdash{}{0pt}%
\pgfpathmoveto{\pgfqpoint{2.854347in}{0.592796in}}%
\pgfpathlineto{\pgfqpoint{2.953283in}{0.592796in}}%
\pgfpathlineto{\pgfqpoint{2.953283in}{0.493860in}}%
\pgfpathlineto{\pgfqpoint{2.854347in}{0.493860in}}%
\pgfpathlineto{\pgfqpoint{2.854347in}{0.592796in}}%
\pgfusepath{stroke,fill}%
\end{pgfscope}%
\begin{pgfscope}%
\pgfpathrectangle{\pgfqpoint{0.380943in}{0.295988in}}{\pgfqpoint{4.650000in}{0.692553in}}%
\pgfusepath{clip}%
\pgfsetbuttcap%
\pgfsetroundjoin%
\definecolor{currentfill}{rgb}{0.994018,0.750850,0.606382}%
\pgfsetfillcolor{currentfill}%
\pgfsetlinewidth{0.250937pt}%
\definecolor{currentstroke}{rgb}{1.000000,1.000000,1.000000}%
\pgfsetstrokecolor{currentstroke}%
\pgfsetdash{}{0pt}%
\pgfpathmoveto{\pgfqpoint{2.953283in}{0.592796in}}%
\pgfpathlineto{\pgfqpoint{3.052220in}{0.592796in}}%
\pgfpathlineto{\pgfqpoint{3.052220in}{0.493860in}}%
\pgfpathlineto{\pgfqpoint{2.953283in}{0.493860in}}%
\pgfpathlineto{\pgfqpoint{2.953283in}{0.592796in}}%
\pgfusepath{stroke,fill}%
\end{pgfscope}%
\begin{pgfscope}%
\pgfpathrectangle{\pgfqpoint{0.380943in}{0.295988in}}{\pgfqpoint{4.650000in}{0.692553in}}%
\pgfusepath{clip}%
\pgfsetbuttcap%
\pgfsetroundjoin%
\definecolor{currentfill}{rgb}{0.990296,0.782484,0.624990}%
\pgfsetfillcolor{currentfill}%
\pgfsetlinewidth{0.250937pt}%
\definecolor{currentstroke}{rgb}{1.000000,1.000000,1.000000}%
\pgfsetstrokecolor{currentstroke}%
\pgfsetdash{}{0pt}%
\pgfpathmoveto{\pgfqpoint{3.052220in}{0.592796in}}%
\pgfpathlineto{\pgfqpoint{3.151156in}{0.592796in}}%
\pgfpathlineto{\pgfqpoint{3.151156in}{0.493860in}}%
\pgfpathlineto{\pgfqpoint{3.052220in}{0.493860in}}%
\pgfpathlineto{\pgfqpoint{3.052220in}{0.592796in}}%
\pgfusepath{stroke,fill}%
\end{pgfscope}%
\begin{pgfscope}%
\pgfpathrectangle{\pgfqpoint{0.380943in}{0.295988in}}{\pgfqpoint{4.650000in}{0.692553in}}%
\pgfusepath{clip}%
\pgfsetbuttcap%
\pgfsetroundjoin%
\definecolor{currentfill}{rgb}{0.990296,0.782484,0.624990}%
\pgfsetfillcolor{currentfill}%
\pgfsetlinewidth{0.250937pt}%
\definecolor{currentstroke}{rgb}{1.000000,1.000000,1.000000}%
\pgfsetstrokecolor{currentstroke}%
\pgfsetdash{}{0pt}%
\pgfpathmoveto{\pgfqpoint{3.151156in}{0.592796in}}%
\pgfpathlineto{\pgfqpoint{3.250092in}{0.592796in}}%
\pgfpathlineto{\pgfqpoint{3.250092in}{0.493860in}}%
\pgfpathlineto{\pgfqpoint{3.151156in}{0.493860in}}%
\pgfpathlineto{\pgfqpoint{3.151156in}{0.592796in}}%
\pgfusepath{stroke,fill}%
\end{pgfscope}%
\begin{pgfscope}%
\pgfpathrectangle{\pgfqpoint{0.380943in}{0.295988in}}{\pgfqpoint{4.650000in}{0.692553in}}%
\pgfusepath{clip}%
\pgfsetbuttcap%
\pgfsetroundjoin%
\definecolor{currentfill}{rgb}{1.000000,0.571396,0.517186}%
\pgfsetfillcolor{currentfill}%
\pgfsetlinewidth{0.250937pt}%
\definecolor{currentstroke}{rgb}{1.000000,1.000000,1.000000}%
\pgfsetstrokecolor{currentstroke}%
\pgfsetdash{}{0pt}%
\pgfpathmoveto{\pgfqpoint{3.250092in}{0.592796in}}%
\pgfpathlineto{\pgfqpoint{3.349028in}{0.592796in}}%
\pgfpathlineto{\pgfqpoint{3.349028in}{0.493860in}}%
\pgfpathlineto{\pgfqpoint{3.250092in}{0.493860in}}%
\pgfpathlineto{\pgfqpoint{3.250092in}{0.592796in}}%
\pgfusepath{stroke,fill}%
\end{pgfscope}%
\begin{pgfscope}%
\pgfpathrectangle{\pgfqpoint{0.380943in}{0.295988in}}{\pgfqpoint{4.650000in}{0.692553in}}%
\pgfusepath{clip}%
\pgfsetbuttcap%
\pgfsetroundjoin%
\definecolor{currentfill}{rgb}{1.000000,0.554479,0.510419}%
\pgfsetfillcolor{currentfill}%
\pgfsetlinewidth{0.250937pt}%
\definecolor{currentstroke}{rgb}{1.000000,1.000000,1.000000}%
\pgfsetstrokecolor{currentstroke}%
\pgfsetdash{}{0pt}%
\pgfpathmoveto{\pgfqpoint{3.349028in}{0.592796in}}%
\pgfpathlineto{\pgfqpoint{3.447964in}{0.592796in}}%
\pgfpathlineto{\pgfqpoint{3.447964in}{0.493860in}}%
\pgfpathlineto{\pgfqpoint{3.349028in}{0.493860in}}%
\pgfpathlineto{\pgfqpoint{3.349028in}{0.592796in}}%
\pgfusepath{stroke,fill}%
\end{pgfscope}%
\begin{pgfscope}%
\pgfpathrectangle{\pgfqpoint{0.380943in}{0.295988in}}{\pgfqpoint{4.650000in}{0.692553in}}%
\pgfusepath{clip}%
\pgfsetbuttcap%
\pgfsetroundjoin%
\definecolor{currentfill}{rgb}{1.000000,0.512618,0.492826}%
\pgfsetfillcolor{currentfill}%
\pgfsetlinewidth{0.250937pt}%
\definecolor{currentstroke}{rgb}{1.000000,1.000000,1.000000}%
\pgfsetstrokecolor{currentstroke}%
\pgfsetdash{}{0pt}%
\pgfpathmoveto{\pgfqpoint{3.447964in}{0.592796in}}%
\pgfpathlineto{\pgfqpoint{3.546901in}{0.592796in}}%
\pgfpathlineto{\pgfqpoint{3.546901in}{0.493860in}}%
\pgfpathlineto{\pgfqpoint{3.447964in}{0.493860in}}%
\pgfpathlineto{\pgfqpoint{3.447964in}{0.592796in}}%
\pgfusepath{stroke,fill}%
\end{pgfscope}%
\begin{pgfscope}%
\pgfpathrectangle{\pgfqpoint{0.380943in}{0.295988in}}{\pgfqpoint{4.650000in}{0.692553in}}%
\pgfusepath{clip}%
\pgfsetbuttcap%
\pgfsetroundjoin%
\definecolor{currentfill}{rgb}{0.989619,0.788235,0.628374}%
\pgfsetfillcolor{currentfill}%
\pgfsetlinewidth{0.250937pt}%
\definecolor{currentstroke}{rgb}{1.000000,1.000000,1.000000}%
\pgfsetstrokecolor{currentstroke}%
\pgfsetdash{}{0pt}%
\pgfpathmoveto{\pgfqpoint{3.546901in}{0.592796in}}%
\pgfpathlineto{\pgfqpoint{3.645837in}{0.592796in}}%
\pgfpathlineto{\pgfqpoint{3.645837in}{0.493860in}}%
\pgfpathlineto{\pgfqpoint{3.546901in}{0.493860in}}%
\pgfpathlineto{\pgfqpoint{3.546901in}{0.592796in}}%
\pgfusepath{stroke,fill}%
\end{pgfscope}%
\begin{pgfscope}%
\pgfpathrectangle{\pgfqpoint{0.380943in}{0.295988in}}{\pgfqpoint{4.650000in}{0.692553in}}%
\pgfusepath{clip}%
\pgfsetbuttcap%
\pgfsetroundjoin%
\definecolor{currentfill}{rgb}{0.999785,0.636970,0.544191}%
\pgfsetfillcolor{currentfill}%
\pgfsetlinewidth{0.250937pt}%
\definecolor{currentstroke}{rgb}{1.000000,1.000000,1.000000}%
\pgfsetstrokecolor{currentstroke}%
\pgfsetdash{}{0pt}%
\pgfpathmoveto{\pgfqpoint{3.645837in}{0.592796in}}%
\pgfpathlineto{\pgfqpoint{3.744773in}{0.592796in}}%
\pgfpathlineto{\pgfqpoint{3.744773in}{0.493860in}}%
\pgfpathlineto{\pgfqpoint{3.645837in}{0.493860in}}%
\pgfpathlineto{\pgfqpoint{3.645837in}{0.592796in}}%
\pgfusepath{stroke,fill}%
\end{pgfscope}%
\begin{pgfscope}%
\pgfpathrectangle{\pgfqpoint{0.380943in}{0.295988in}}{\pgfqpoint{4.650000in}{0.692553in}}%
\pgfusepath{clip}%
\pgfsetbuttcap%
\pgfsetroundjoin%
\definecolor{currentfill}{rgb}{1.000000,0.625529,0.538839}%
\pgfsetfillcolor{currentfill}%
\pgfsetlinewidth{0.250937pt}%
\definecolor{currentstroke}{rgb}{1.000000,1.000000,1.000000}%
\pgfsetstrokecolor{currentstroke}%
\pgfsetdash{}{0pt}%
\pgfpathmoveto{\pgfqpoint{3.744773in}{0.592796in}}%
\pgfpathlineto{\pgfqpoint{3.843709in}{0.592796in}}%
\pgfpathlineto{\pgfqpoint{3.843709in}{0.493860in}}%
\pgfpathlineto{\pgfqpoint{3.744773in}{0.493860in}}%
\pgfpathlineto{\pgfqpoint{3.744773in}{0.592796in}}%
\pgfusepath{stroke,fill}%
\end{pgfscope}%
\begin{pgfscope}%
\pgfpathrectangle{\pgfqpoint{0.380943in}{0.295988in}}{\pgfqpoint{4.650000in}{0.692553in}}%
\pgfusepath{clip}%
\pgfsetbuttcap%
\pgfsetroundjoin%
\definecolor{currentfill}{rgb}{0.956171,0.434602,0.434602}%
\pgfsetfillcolor{currentfill}%
\pgfsetlinewidth{0.250937pt}%
\definecolor{currentstroke}{rgb}{1.000000,1.000000,1.000000}%
\pgfsetstrokecolor{currentstroke}%
\pgfsetdash{}{0pt}%
\pgfpathmoveto{\pgfqpoint{3.843709in}{0.592796in}}%
\pgfpathlineto{\pgfqpoint{3.942645in}{0.592796in}}%
\pgfpathlineto{\pgfqpoint{3.942645in}{0.493860in}}%
\pgfpathlineto{\pgfqpoint{3.843709in}{0.493860in}}%
\pgfpathlineto{\pgfqpoint{3.843709in}{0.592796in}}%
\pgfusepath{stroke,fill}%
\end{pgfscope}%
\begin{pgfscope}%
\pgfpathrectangle{\pgfqpoint{0.380943in}{0.295988in}}{\pgfqpoint{4.650000in}{0.692553in}}%
\pgfusepath{clip}%
\pgfsetbuttcap%
\pgfsetroundjoin%
\definecolor{currentfill}{rgb}{1.000000,0.598462,0.528012}%
\pgfsetfillcolor{currentfill}%
\pgfsetlinewidth{0.250937pt}%
\definecolor{currentstroke}{rgb}{1.000000,1.000000,1.000000}%
\pgfsetstrokecolor{currentstroke}%
\pgfsetdash{}{0pt}%
\pgfpathmoveto{\pgfqpoint{3.942645in}{0.592796in}}%
\pgfpathlineto{\pgfqpoint{4.041581in}{0.592796in}}%
\pgfpathlineto{\pgfqpoint{4.041581in}{0.493860in}}%
\pgfpathlineto{\pgfqpoint{3.942645in}{0.493860in}}%
\pgfpathlineto{\pgfqpoint{3.942645in}{0.592796in}}%
\pgfusepath{stroke,fill}%
\end{pgfscope}%
\begin{pgfscope}%
\pgfpathrectangle{\pgfqpoint{0.380943in}{0.295988in}}{\pgfqpoint{4.650000in}{0.692553in}}%
\pgfusepath{clip}%
\pgfsetbuttcap%
\pgfsetroundjoin%
\definecolor{currentfill}{rgb}{1.000000,0.522261,0.496886}%
\pgfsetfillcolor{currentfill}%
\pgfsetlinewidth{0.250937pt}%
\definecolor{currentstroke}{rgb}{1.000000,1.000000,1.000000}%
\pgfsetstrokecolor{currentstroke}%
\pgfsetdash{}{0pt}%
\pgfpathmoveto{\pgfqpoint{4.041581in}{0.592796in}}%
\pgfpathlineto{\pgfqpoint{4.140518in}{0.592796in}}%
\pgfpathlineto{\pgfqpoint{4.140518in}{0.493860in}}%
\pgfpathlineto{\pgfqpoint{4.041581in}{0.493860in}}%
\pgfpathlineto{\pgfqpoint{4.041581in}{0.592796in}}%
\pgfusepath{stroke,fill}%
\end{pgfscope}%
\begin{pgfscope}%
\pgfpathrectangle{\pgfqpoint{0.380943in}{0.295988in}}{\pgfqpoint{4.650000in}{0.692553in}}%
\pgfusepath{clip}%
\pgfsetbuttcap%
\pgfsetroundjoin%
\definecolor{currentfill}{rgb}{0.999446,0.645767,0.548927}%
\pgfsetfillcolor{currentfill}%
\pgfsetlinewidth{0.250937pt}%
\definecolor{currentstroke}{rgb}{1.000000,1.000000,1.000000}%
\pgfsetstrokecolor{currentstroke}%
\pgfsetdash{}{0pt}%
\pgfpathmoveto{\pgfqpoint{4.140518in}{0.592796in}}%
\pgfpathlineto{\pgfqpoint{4.239454in}{0.592796in}}%
\pgfpathlineto{\pgfqpoint{4.239454in}{0.493860in}}%
\pgfpathlineto{\pgfqpoint{4.140518in}{0.493860in}}%
\pgfpathlineto{\pgfqpoint{4.140518in}{0.592796in}}%
\pgfusepath{stroke,fill}%
\end{pgfscope}%
\begin{pgfscope}%
\pgfpathrectangle{\pgfqpoint{0.380943in}{0.295988in}}{\pgfqpoint{4.650000in}{0.692553in}}%
\pgfusepath{clip}%
\pgfsetbuttcap%
\pgfsetroundjoin%
\definecolor{currentfill}{rgb}{0.995709,0.736471,0.597924}%
\pgfsetfillcolor{currentfill}%
\pgfsetlinewidth{0.250937pt}%
\definecolor{currentstroke}{rgb}{1.000000,1.000000,1.000000}%
\pgfsetstrokecolor{currentstroke}%
\pgfsetdash{}{0pt}%
\pgfpathmoveto{\pgfqpoint{4.239454in}{0.592796in}}%
\pgfpathlineto{\pgfqpoint{4.338390in}{0.592796in}}%
\pgfpathlineto{\pgfqpoint{4.338390in}{0.493860in}}%
\pgfpathlineto{\pgfqpoint{4.239454in}{0.493860in}}%
\pgfpathlineto{\pgfqpoint{4.239454in}{0.592796in}}%
\pgfusepath{stroke,fill}%
\end{pgfscope}%
\begin{pgfscope}%
\pgfpathrectangle{\pgfqpoint{0.380943in}{0.295988in}}{\pgfqpoint{4.650000in}{0.692553in}}%
\pgfusepath{clip}%
\pgfsetbuttcap%
\pgfsetroundjoin%
\definecolor{currentfill}{rgb}{0.963937,0.914418,0.716801}%
\pgfsetfillcolor{currentfill}%
\pgfsetlinewidth{0.250937pt}%
\definecolor{currentstroke}{rgb}{1.000000,1.000000,1.000000}%
\pgfsetstrokecolor{currentstroke}%
\pgfsetdash{}{0pt}%
\pgfpathmoveto{\pgfqpoint{4.338390in}{0.592796in}}%
\pgfpathlineto{\pgfqpoint{4.437326in}{0.592796in}}%
\pgfpathlineto{\pgfqpoint{4.437326in}{0.493860in}}%
\pgfpathlineto{\pgfqpoint{4.338390in}{0.493860in}}%
\pgfpathlineto{\pgfqpoint{4.338390in}{0.592796in}}%
\pgfusepath{stroke,fill}%
\end{pgfscope}%
\begin{pgfscope}%
\pgfpathrectangle{\pgfqpoint{0.380943in}{0.295988in}}{\pgfqpoint{4.650000in}{0.692553in}}%
\pgfusepath{clip}%
\pgfsetbuttcap%
\pgfsetroundjoin%
\definecolor{currentfill}{rgb}{0.963091,0.919493,0.720185}%
\pgfsetfillcolor{currentfill}%
\pgfsetlinewidth{0.250937pt}%
\definecolor{currentstroke}{rgb}{1.000000,1.000000,1.000000}%
\pgfsetstrokecolor{currentstroke}%
\pgfsetdash{}{0pt}%
\pgfpathmoveto{\pgfqpoint{4.437326in}{0.592796in}}%
\pgfpathlineto{\pgfqpoint{4.536262in}{0.592796in}}%
\pgfpathlineto{\pgfqpoint{4.536262in}{0.493860in}}%
\pgfpathlineto{\pgfqpoint{4.437326in}{0.493860in}}%
\pgfpathlineto{\pgfqpoint{4.437326in}{0.592796in}}%
\pgfusepath{stroke,fill}%
\end{pgfscope}%
\begin{pgfscope}%
\pgfpathrectangle{\pgfqpoint{0.380943in}{0.295988in}}{\pgfqpoint{4.650000in}{0.692553in}}%
\pgfusepath{clip}%
\pgfsetbuttcap%
\pgfsetroundjoin%
\definecolor{currentfill}{rgb}{0.961230,0.930657,0.727628}%
\pgfsetfillcolor{currentfill}%
\pgfsetlinewidth{0.250937pt}%
\definecolor{currentstroke}{rgb}{1.000000,1.000000,1.000000}%
\pgfsetstrokecolor{currentstroke}%
\pgfsetdash{}{0pt}%
\pgfpathmoveto{\pgfqpoint{4.536262in}{0.592796in}}%
\pgfpathlineto{\pgfqpoint{4.635198in}{0.592796in}}%
\pgfpathlineto{\pgfqpoint{4.635198in}{0.493860in}}%
\pgfpathlineto{\pgfqpoint{4.536262in}{0.493860in}}%
\pgfpathlineto{\pgfqpoint{4.536262in}{0.592796in}}%
\pgfusepath{stroke,fill}%
\end{pgfscope}%
\begin{pgfscope}%
\pgfpathrectangle{\pgfqpoint{0.380943in}{0.295988in}}{\pgfqpoint{4.650000in}{0.692553in}}%
\pgfusepath{clip}%
\pgfsetbuttcap%
\pgfsetroundjoin%
\definecolor{currentfill}{rgb}{0.962584,0.922537,0.722215}%
\pgfsetfillcolor{currentfill}%
\pgfsetlinewidth{0.250937pt}%
\definecolor{currentstroke}{rgb}{1.000000,1.000000,1.000000}%
\pgfsetstrokecolor{currentstroke}%
\pgfsetdash{}{0pt}%
\pgfpathmoveto{\pgfqpoint{4.635198in}{0.592796in}}%
\pgfpathlineto{\pgfqpoint{4.734135in}{0.592796in}}%
\pgfpathlineto{\pgfqpoint{4.734135in}{0.493860in}}%
\pgfpathlineto{\pgfqpoint{4.635198in}{0.493860in}}%
\pgfpathlineto{\pgfqpoint{4.635198in}{0.592796in}}%
\pgfusepath{stroke,fill}%
\end{pgfscope}%
\begin{pgfscope}%
\pgfpathrectangle{\pgfqpoint{0.380943in}{0.295988in}}{\pgfqpoint{4.650000in}{0.692553in}}%
\pgfusepath{clip}%
\pgfsetbuttcap%
\pgfsetroundjoin%
\definecolor{currentfill}{rgb}{0.964275,0.912388,0.715448}%
\pgfsetfillcolor{currentfill}%
\pgfsetlinewidth{0.250937pt}%
\definecolor{currentstroke}{rgb}{1.000000,1.000000,1.000000}%
\pgfsetstrokecolor{currentstroke}%
\pgfsetdash{}{0pt}%
\pgfpathmoveto{\pgfqpoint{4.734135in}{0.592796in}}%
\pgfpathlineto{\pgfqpoint{4.833071in}{0.592796in}}%
\pgfpathlineto{\pgfqpoint{4.833071in}{0.493860in}}%
\pgfpathlineto{\pgfqpoint{4.734135in}{0.493860in}}%
\pgfpathlineto{\pgfqpoint{4.734135in}{0.592796in}}%
\pgfusepath{stroke,fill}%
\end{pgfscope}%
\begin{pgfscope}%
\pgfpathrectangle{\pgfqpoint{0.380943in}{0.295988in}}{\pgfqpoint{4.650000in}{0.692553in}}%
\pgfusepath{clip}%
\pgfsetbuttcap%
\pgfsetroundjoin%
\definecolor{currentfill}{rgb}{0.964783,0.940131,0.739808}%
\pgfsetfillcolor{currentfill}%
\pgfsetlinewidth{0.250937pt}%
\definecolor{currentstroke}{rgb}{1.000000,1.000000,1.000000}%
\pgfsetstrokecolor{currentstroke}%
\pgfsetdash{}{0pt}%
\pgfpathmoveto{\pgfqpoint{4.833071in}{0.592796in}}%
\pgfpathlineto{\pgfqpoint{4.932007in}{0.592796in}}%
\pgfpathlineto{\pgfqpoint{4.932007in}{0.493860in}}%
\pgfpathlineto{\pgfqpoint{4.833071in}{0.493860in}}%
\pgfpathlineto{\pgfqpoint{4.833071in}{0.592796in}}%
\pgfusepath{stroke,fill}%
\end{pgfscope}%
\begin{pgfscope}%
\pgfpathrectangle{\pgfqpoint{0.380943in}{0.295988in}}{\pgfqpoint{4.650000in}{0.692553in}}%
\pgfusepath{clip}%
\pgfsetbuttcap%
\pgfsetroundjoin%
\pgfsetlinewidth{0.250937pt}%
\definecolor{currentstroke}{rgb}{1.000000,1.000000,1.000000}%
\pgfsetstrokecolor{currentstroke}%
\pgfsetdash{}{0pt}%
\pgfpathmoveto{\pgfqpoint{4.932007in}{0.592796in}}%
\pgfpathlineto{\pgfqpoint{5.030943in}{0.592796in}}%
\pgfpathlineto{\pgfqpoint{5.030943in}{0.493860in}}%
\pgfpathlineto{\pgfqpoint{4.932007in}{0.493860in}}%
\pgfpathlineto{\pgfqpoint{4.932007in}{0.592796in}}%
\pgfusepath{stroke}%
\end{pgfscope}%
\begin{pgfscope}%
\pgfpathrectangle{\pgfqpoint{0.380943in}{0.295988in}}{\pgfqpoint{4.650000in}{0.692553in}}%
\pgfusepath{clip}%
\pgfsetbuttcap%
\pgfsetroundjoin%
\definecolor{currentfill}{rgb}{1.000000,1.000000,0.899808}%
\pgfsetfillcolor{currentfill}%
\pgfsetlinewidth{0.250937pt}%
\definecolor{currentstroke}{rgb}{1.000000,1.000000,1.000000}%
\pgfsetstrokecolor{currentstroke}%
\pgfsetdash{}{0pt}%
\pgfpathmoveto{\pgfqpoint{0.380943in}{0.493860in}}%
\pgfpathlineto{\pgfqpoint{0.479879in}{0.493860in}}%
\pgfpathlineto{\pgfqpoint{0.479879in}{0.394924in}}%
\pgfpathlineto{\pgfqpoint{0.380943in}{0.394924in}}%
\pgfpathlineto{\pgfqpoint{0.380943in}{0.493860in}}%
\pgfusepath{stroke,fill}%
\end{pgfscope}%
\begin{pgfscope}%
\pgfpathrectangle{\pgfqpoint{0.380943in}{0.295988in}}{\pgfqpoint{4.650000in}{0.692553in}}%
\pgfusepath{clip}%
\pgfsetbuttcap%
\pgfsetroundjoin%
\definecolor{currentfill}{rgb}{1.000000,1.000000,0.908266}%
\pgfsetfillcolor{currentfill}%
\pgfsetlinewidth{0.250937pt}%
\definecolor{currentstroke}{rgb}{1.000000,1.000000,1.000000}%
\pgfsetstrokecolor{currentstroke}%
\pgfsetdash{}{0pt}%
\pgfpathmoveto{\pgfqpoint{0.479879in}{0.493860in}}%
\pgfpathlineto{\pgfqpoint{0.578815in}{0.493860in}}%
\pgfpathlineto{\pgfqpoint{0.578815in}{0.394924in}}%
\pgfpathlineto{\pgfqpoint{0.479879in}{0.394924in}}%
\pgfpathlineto{\pgfqpoint{0.479879in}{0.493860in}}%
\pgfusepath{stroke,fill}%
\end{pgfscope}%
\begin{pgfscope}%
\pgfpathrectangle{\pgfqpoint{0.380943in}{0.295988in}}{\pgfqpoint{4.650000in}{0.692553in}}%
\pgfusepath{clip}%
\pgfsetbuttcap%
\pgfsetroundjoin%
\definecolor{currentfill}{rgb}{1.000000,1.000000,0.920953}%
\pgfsetfillcolor{currentfill}%
\pgfsetlinewidth{0.250937pt}%
\definecolor{currentstroke}{rgb}{1.000000,1.000000,1.000000}%
\pgfsetstrokecolor{currentstroke}%
\pgfsetdash{}{0pt}%
\pgfpathmoveto{\pgfqpoint{0.578815in}{0.493860in}}%
\pgfpathlineto{\pgfqpoint{0.677752in}{0.493860in}}%
\pgfpathlineto{\pgfqpoint{0.677752in}{0.394924in}}%
\pgfpathlineto{\pgfqpoint{0.578815in}{0.394924in}}%
\pgfpathlineto{\pgfqpoint{0.578815in}{0.493860in}}%
\pgfusepath{stroke,fill}%
\end{pgfscope}%
\begin{pgfscope}%
\pgfpathrectangle{\pgfqpoint{0.380943in}{0.295988in}}{\pgfqpoint{4.650000in}{0.692553in}}%
\pgfusepath{clip}%
\pgfsetbuttcap%
\pgfsetroundjoin%
\definecolor{currentfill}{rgb}{1.000000,1.000000,0.899808}%
\pgfsetfillcolor{currentfill}%
\pgfsetlinewidth{0.250937pt}%
\definecolor{currentstroke}{rgb}{1.000000,1.000000,1.000000}%
\pgfsetstrokecolor{currentstroke}%
\pgfsetdash{}{0pt}%
\pgfpathmoveto{\pgfqpoint{0.677752in}{0.493860in}}%
\pgfpathlineto{\pgfqpoint{0.776688in}{0.493860in}}%
\pgfpathlineto{\pgfqpoint{0.776688in}{0.394924in}}%
\pgfpathlineto{\pgfqpoint{0.677752in}{0.394924in}}%
\pgfpathlineto{\pgfqpoint{0.677752in}{0.493860in}}%
\pgfusepath{stroke,fill}%
\end{pgfscope}%
\begin{pgfscope}%
\pgfpathrectangle{\pgfqpoint{0.380943in}{0.295988in}}{\pgfqpoint{4.650000in}{0.692553in}}%
\pgfusepath{clip}%
\pgfsetbuttcap%
\pgfsetroundjoin%
\definecolor{currentfill}{rgb}{1.000000,1.000000,0.899808}%
\pgfsetfillcolor{currentfill}%
\pgfsetlinewidth{0.250937pt}%
\definecolor{currentstroke}{rgb}{1.000000,1.000000,1.000000}%
\pgfsetstrokecolor{currentstroke}%
\pgfsetdash{}{0pt}%
\pgfpathmoveto{\pgfqpoint{0.776688in}{0.493860in}}%
\pgfpathlineto{\pgfqpoint{0.875624in}{0.493860in}}%
\pgfpathlineto{\pgfqpoint{0.875624in}{0.394924in}}%
\pgfpathlineto{\pgfqpoint{0.776688in}{0.394924in}}%
\pgfpathlineto{\pgfqpoint{0.776688in}{0.493860in}}%
\pgfusepath{stroke,fill}%
\end{pgfscope}%
\begin{pgfscope}%
\pgfpathrectangle{\pgfqpoint{0.380943in}{0.295988in}}{\pgfqpoint{4.650000in}{0.692553in}}%
\pgfusepath{clip}%
\pgfsetbuttcap%
\pgfsetroundjoin%
\definecolor{currentfill}{rgb}{1.000000,1.000000,0.865975}%
\pgfsetfillcolor{currentfill}%
\pgfsetlinewidth{0.250937pt}%
\definecolor{currentstroke}{rgb}{1.000000,1.000000,1.000000}%
\pgfsetstrokecolor{currentstroke}%
\pgfsetdash{}{0pt}%
\pgfpathmoveto{\pgfqpoint{0.875624in}{0.493860in}}%
\pgfpathlineto{\pgfqpoint{0.974560in}{0.493860in}}%
\pgfpathlineto{\pgfqpoint{0.974560in}{0.394924in}}%
\pgfpathlineto{\pgfqpoint{0.875624in}{0.394924in}}%
\pgfpathlineto{\pgfqpoint{0.875624in}{0.493860in}}%
\pgfusepath{stroke,fill}%
\end{pgfscope}%
\begin{pgfscope}%
\pgfpathrectangle{\pgfqpoint{0.380943in}{0.295988in}}{\pgfqpoint{4.650000in}{0.692553in}}%
\pgfusepath{clip}%
\pgfsetbuttcap%
\pgfsetroundjoin%
\definecolor{currentfill}{rgb}{1.000000,1.000000,0.832141}%
\pgfsetfillcolor{currentfill}%
\pgfsetlinewidth{0.250937pt}%
\definecolor{currentstroke}{rgb}{1.000000,1.000000,1.000000}%
\pgfsetstrokecolor{currentstroke}%
\pgfsetdash{}{0pt}%
\pgfpathmoveto{\pgfqpoint{0.974560in}{0.493860in}}%
\pgfpathlineto{\pgfqpoint{1.073496in}{0.493860in}}%
\pgfpathlineto{\pgfqpoint{1.073496in}{0.394924in}}%
\pgfpathlineto{\pgfqpoint{0.974560in}{0.394924in}}%
\pgfpathlineto{\pgfqpoint{0.974560in}{0.493860in}}%
\pgfusepath{stroke,fill}%
\end{pgfscope}%
\begin{pgfscope}%
\pgfpathrectangle{\pgfqpoint{0.380943in}{0.295988in}}{\pgfqpoint{4.650000in}{0.692553in}}%
\pgfusepath{clip}%
\pgfsetbuttcap%
\pgfsetroundjoin%
\definecolor{currentfill}{rgb}{1.000000,1.000000,0.920953}%
\pgfsetfillcolor{currentfill}%
\pgfsetlinewidth{0.250937pt}%
\definecolor{currentstroke}{rgb}{1.000000,1.000000,1.000000}%
\pgfsetstrokecolor{currentstroke}%
\pgfsetdash{}{0pt}%
\pgfpathmoveto{\pgfqpoint{1.073496in}{0.493860in}}%
\pgfpathlineto{\pgfqpoint{1.172432in}{0.493860in}}%
\pgfpathlineto{\pgfqpoint{1.172432in}{0.394924in}}%
\pgfpathlineto{\pgfqpoint{1.073496in}{0.394924in}}%
\pgfpathlineto{\pgfqpoint{1.073496in}{0.493860in}}%
\pgfusepath{stroke,fill}%
\end{pgfscope}%
\begin{pgfscope}%
\pgfpathrectangle{\pgfqpoint{0.380943in}{0.295988in}}{\pgfqpoint{4.650000in}{0.692553in}}%
\pgfusepath{clip}%
\pgfsetbuttcap%
\pgfsetroundjoin%
\definecolor{currentfill}{rgb}{1.000000,1.000000,0.887120}%
\pgfsetfillcolor{currentfill}%
\pgfsetlinewidth{0.250937pt}%
\definecolor{currentstroke}{rgb}{1.000000,1.000000,1.000000}%
\pgfsetstrokecolor{currentstroke}%
\pgfsetdash{}{0pt}%
\pgfpathmoveto{\pgfqpoint{1.172432in}{0.493860in}}%
\pgfpathlineto{\pgfqpoint{1.271369in}{0.493860in}}%
\pgfpathlineto{\pgfqpoint{1.271369in}{0.394924in}}%
\pgfpathlineto{\pgfqpoint{1.172432in}{0.394924in}}%
\pgfpathlineto{\pgfqpoint{1.172432in}{0.493860in}}%
\pgfusepath{stroke,fill}%
\end{pgfscope}%
\begin{pgfscope}%
\pgfpathrectangle{\pgfqpoint{0.380943in}{0.295988in}}{\pgfqpoint{4.650000in}{0.692553in}}%
\pgfusepath{clip}%
\pgfsetbuttcap%
\pgfsetroundjoin%
\definecolor{currentfill}{rgb}{0.973057,0.868051,0.691457}%
\pgfsetfillcolor{currentfill}%
\pgfsetlinewidth{0.250937pt}%
\definecolor{currentstroke}{rgb}{1.000000,1.000000,1.000000}%
\pgfsetstrokecolor{currentstroke}%
\pgfsetdash{}{0pt}%
\pgfpathmoveto{\pgfqpoint{1.271369in}{0.493860in}}%
\pgfpathlineto{\pgfqpoint{1.370305in}{0.493860in}}%
\pgfpathlineto{\pgfqpoint{1.370305in}{0.394924in}}%
\pgfpathlineto{\pgfqpoint{1.271369in}{0.394924in}}%
\pgfpathlineto{\pgfqpoint{1.271369in}{0.493860in}}%
\pgfusepath{stroke,fill}%
\end{pgfscope}%
\begin{pgfscope}%
\pgfpathrectangle{\pgfqpoint{0.380943in}{0.295988in}}{\pgfqpoint{4.650000in}{0.692553in}}%
\pgfusepath{clip}%
\pgfsetbuttcap%
\pgfsetroundjoin%
\definecolor{currentfill}{rgb}{0.964275,0.912388,0.715448}%
\pgfsetfillcolor{currentfill}%
\pgfsetlinewidth{0.250937pt}%
\definecolor{currentstroke}{rgb}{1.000000,1.000000,1.000000}%
\pgfsetstrokecolor{currentstroke}%
\pgfsetdash{}{0pt}%
\pgfpathmoveto{\pgfqpoint{1.370305in}{0.493860in}}%
\pgfpathlineto{\pgfqpoint{1.469241in}{0.493860in}}%
\pgfpathlineto{\pgfqpoint{1.469241in}{0.394924in}}%
\pgfpathlineto{\pgfqpoint{1.370305in}{0.394924in}}%
\pgfpathlineto{\pgfqpoint{1.370305in}{0.493860in}}%
\pgfusepath{stroke,fill}%
\end{pgfscope}%
\begin{pgfscope}%
\pgfpathrectangle{\pgfqpoint{0.380943in}{0.295988in}}{\pgfqpoint{4.650000in}{0.692553in}}%
\pgfusepath{clip}%
\pgfsetbuttcap%
\pgfsetroundjoin%
\definecolor{currentfill}{rgb}{0.978131,0.843783,0.675709}%
\pgfsetfillcolor{currentfill}%
\pgfsetlinewidth{0.250937pt}%
\definecolor{currentstroke}{rgb}{1.000000,1.000000,1.000000}%
\pgfsetstrokecolor{currentstroke}%
\pgfsetdash{}{0pt}%
\pgfpathmoveto{\pgfqpoint{1.469241in}{0.493860in}}%
\pgfpathlineto{\pgfqpoint{1.568177in}{0.493860in}}%
\pgfpathlineto{\pgfqpoint{1.568177in}{0.394924in}}%
\pgfpathlineto{\pgfqpoint{1.469241in}{0.394924in}}%
\pgfpathlineto{\pgfqpoint{1.469241in}{0.493860in}}%
\pgfusepath{stroke,fill}%
\end{pgfscope}%
\begin{pgfscope}%
\pgfpathrectangle{\pgfqpoint{0.380943in}{0.295988in}}{\pgfqpoint{4.650000in}{0.692553in}}%
\pgfusepath{clip}%
\pgfsetbuttcap%
\pgfsetroundjoin%
\definecolor{currentfill}{rgb}{0.975594,0.855363,0.684691}%
\pgfsetfillcolor{currentfill}%
\pgfsetlinewidth{0.250937pt}%
\definecolor{currentstroke}{rgb}{1.000000,1.000000,1.000000}%
\pgfsetstrokecolor{currentstroke}%
\pgfsetdash{}{0pt}%
\pgfpathmoveto{\pgfqpoint{1.568177in}{0.493860in}}%
\pgfpathlineto{\pgfqpoint{1.667113in}{0.493860in}}%
\pgfpathlineto{\pgfqpoint{1.667113in}{0.394924in}}%
\pgfpathlineto{\pgfqpoint{1.568177in}{0.394924in}}%
\pgfpathlineto{\pgfqpoint{1.568177in}{0.493860in}}%
\pgfusepath{stroke,fill}%
\end{pgfscope}%
\begin{pgfscope}%
\pgfpathrectangle{\pgfqpoint{0.380943in}{0.295988in}}{\pgfqpoint{4.650000in}{0.692553in}}%
\pgfusepath{clip}%
\pgfsetbuttcap%
\pgfsetroundjoin%
\definecolor{currentfill}{rgb}{0.974072,0.862976,0.688750}%
\pgfsetfillcolor{currentfill}%
\pgfsetlinewidth{0.250937pt}%
\definecolor{currentstroke}{rgb}{1.000000,1.000000,1.000000}%
\pgfsetstrokecolor{currentstroke}%
\pgfsetdash{}{0pt}%
\pgfpathmoveto{\pgfqpoint{1.667113in}{0.493860in}}%
\pgfpathlineto{\pgfqpoint{1.766049in}{0.493860in}}%
\pgfpathlineto{\pgfqpoint{1.766049in}{0.394924in}}%
\pgfpathlineto{\pgfqpoint{1.667113in}{0.394924in}}%
\pgfpathlineto{\pgfqpoint{1.667113in}{0.493860in}}%
\pgfusepath{stroke,fill}%
\end{pgfscope}%
\begin{pgfscope}%
\pgfpathrectangle{\pgfqpoint{0.380943in}{0.295988in}}{\pgfqpoint{4.650000in}{0.692553in}}%
\pgfusepath{clip}%
\pgfsetbuttcap%
\pgfsetroundjoin%
\definecolor{currentfill}{rgb}{0.978131,0.843783,0.675709}%
\pgfsetfillcolor{currentfill}%
\pgfsetlinewidth{0.250937pt}%
\definecolor{currentstroke}{rgb}{1.000000,1.000000,1.000000}%
\pgfsetstrokecolor{currentstroke}%
\pgfsetdash{}{0pt}%
\pgfpathmoveto{\pgfqpoint{1.766049in}{0.493860in}}%
\pgfpathlineto{\pgfqpoint{1.864986in}{0.493860in}}%
\pgfpathlineto{\pgfqpoint{1.864986in}{0.394924in}}%
\pgfpathlineto{\pgfqpoint{1.766049in}{0.394924in}}%
\pgfpathlineto{\pgfqpoint{1.766049in}{0.493860in}}%
\pgfusepath{stroke,fill}%
\end{pgfscope}%
\begin{pgfscope}%
\pgfpathrectangle{\pgfqpoint{0.380943in}{0.295988in}}{\pgfqpoint{4.650000in}{0.692553in}}%
\pgfusepath{clip}%
\pgfsetbuttcap%
\pgfsetroundjoin%
\definecolor{currentfill}{rgb}{0.968997,0.888351,0.702284}%
\pgfsetfillcolor{currentfill}%
\pgfsetlinewidth{0.250937pt}%
\definecolor{currentstroke}{rgb}{1.000000,1.000000,1.000000}%
\pgfsetstrokecolor{currentstroke}%
\pgfsetdash{}{0pt}%
\pgfpathmoveto{\pgfqpoint{1.864986in}{0.493860in}}%
\pgfpathlineto{\pgfqpoint{1.963922in}{0.493860in}}%
\pgfpathlineto{\pgfqpoint{1.963922in}{0.394924in}}%
\pgfpathlineto{\pgfqpoint{1.864986in}{0.394924in}}%
\pgfpathlineto{\pgfqpoint{1.864986in}{0.493860in}}%
\pgfusepath{stroke,fill}%
\end{pgfscope}%
\begin{pgfscope}%
\pgfpathrectangle{\pgfqpoint{0.380943in}{0.295988in}}{\pgfqpoint{4.650000in}{0.692553in}}%
\pgfusepath{clip}%
\pgfsetbuttcap%
\pgfsetroundjoin%
\definecolor{currentfill}{rgb}{0.961738,0.927612,0.725598}%
\pgfsetfillcolor{currentfill}%
\pgfsetlinewidth{0.250937pt}%
\definecolor{currentstroke}{rgb}{1.000000,1.000000,1.000000}%
\pgfsetstrokecolor{currentstroke}%
\pgfsetdash{}{0pt}%
\pgfpathmoveto{\pgfqpoint{1.963922in}{0.493860in}}%
\pgfpathlineto{\pgfqpoint{2.062858in}{0.493860in}}%
\pgfpathlineto{\pgfqpoint{2.062858in}{0.394924in}}%
\pgfpathlineto{\pgfqpoint{1.963922in}{0.394924in}}%
\pgfpathlineto{\pgfqpoint{1.963922in}{0.493860in}}%
\pgfusepath{stroke,fill}%
\end{pgfscope}%
\begin{pgfscope}%
\pgfpathrectangle{\pgfqpoint{0.380943in}{0.295988in}}{\pgfqpoint{4.650000in}{0.692553in}}%
\pgfusepath{clip}%
\pgfsetbuttcap%
\pgfsetroundjoin%
\definecolor{currentfill}{rgb}{0.971534,0.875663,0.695517}%
\pgfsetfillcolor{currentfill}%
\pgfsetlinewidth{0.250937pt}%
\definecolor{currentstroke}{rgb}{1.000000,1.000000,1.000000}%
\pgfsetstrokecolor{currentstroke}%
\pgfsetdash{}{0pt}%
\pgfpathmoveto{\pgfqpoint{2.062858in}{0.493860in}}%
\pgfpathlineto{\pgfqpoint{2.161794in}{0.493860in}}%
\pgfpathlineto{\pgfqpoint{2.161794in}{0.394924in}}%
\pgfpathlineto{\pgfqpoint{2.062858in}{0.394924in}}%
\pgfpathlineto{\pgfqpoint{2.062858in}{0.493860in}}%
\pgfusepath{stroke,fill}%
\end{pgfscope}%
\begin{pgfscope}%
\pgfpathrectangle{\pgfqpoint{0.380943in}{0.295988in}}{\pgfqpoint{4.650000in}{0.692553in}}%
\pgfusepath{clip}%
\pgfsetbuttcap%
\pgfsetroundjoin%
\definecolor{currentfill}{rgb}{0.962584,0.922537,0.722215}%
\pgfsetfillcolor{currentfill}%
\pgfsetlinewidth{0.250937pt}%
\definecolor{currentstroke}{rgb}{1.000000,1.000000,1.000000}%
\pgfsetstrokecolor{currentstroke}%
\pgfsetdash{}{0pt}%
\pgfpathmoveto{\pgfqpoint{2.161794in}{0.493860in}}%
\pgfpathlineto{\pgfqpoint{2.260730in}{0.493860in}}%
\pgfpathlineto{\pgfqpoint{2.260730in}{0.394924in}}%
\pgfpathlineto{\pgfqpoint{2.161794in}{0.394924in}}%
\pgfpathlineto{\pgfqpoint{2.161794in}{0.493860in}}%
\pgfusepath{stroke,fill}%
\end{pgfscope}%
\begin{pgfscope}%
\pgfpathrectangle{\pgfqpoint{0.380943in}{0.295988in}}{\pgfqpoint{4.650000in}{0.692553in}}%
\pgfusepath{clip}%
\pgfsetbuttcap%
\pgfsetroundjoin%
\definecolor{currentfill}{rgb}{0.963937,0.914418,0.716801}%
\pgfsetfillcolor{currentfill}%
\pgfsetlinewidth{0.250937pt}%
\definecolor{currentstroke}{rgb}{1.000000,1.000000,1.000000}%
\pgfsetstrokecolor{currentstroke}%
\pgfsetdash{}{0pt}%
\pgfpathmoveto{\pgfqpoint{2.260730in}{0.493860in}}%
\pgfpathlineto{\pgfqpoint{2.359666in}{0.493860in}}%
\pgfpathlineto{\pgfqpoint{2.359666in}{0.394924in}}%
\pgfpathlineto{\pgfqpoint{2.260730in}{0.394924in}}%
\pgfpathlineto{\pgfqpoint{2.260730in}{0.493860in}}%
\pgfusepath{stroke,fill}%
\end{pgfscope}%
\begin{pgfscope}%
\pgfpathrectangle{\pgfqpoint{0.380943in}{0.295988in}}{\pgfqpoint{4.650000in}{0.692553in}}%
\pgfusepath{clip}%
\pgfsetbuttcap%
\pgfsetroundjoin%
\definecolor{currentfill}{rgb}{0.970012,0.883276,0.699577}%
\pgfsetfillcolor{currentfill}%
\pgfsetlinewidth{0.250937pt}%
\definecolor{currentstroke}{rgb}{1.000000,1.000000,1.000000}%
\pgfsetstrokecolor{currentstroke}%
\pgfsetdash{}{0pt}%
\pgfpathmoveto{\pgfqpoint{2.359666in}{0.493860in}}%
\pgfpathlineto{\pgfqpoint{2.458603in}{0.493860in}}%
\pgfpathlineto{\pgfqpoint{2.458603in}{0.394924in}}%
\pgfpathlineto{\pgfqpoint{2.359666in}{0.394924in}}%
\pgfpathlineto{\pgfqpoint{2.359666in}{0.493860in}}%
\pgfusepath{stroke,fill}%
\end{pgfscope}%
\begin{pgfscope}%
\pgfpathrectangle{\pgfqpoint{0.380943in}{0.295988in}}{\pgfqpoint{4.650000in}{0.692553in}}%
\pgfusepath{clip}%
\pgfsetbuttcap%
\pgfsetroundjoin%
\definecolor{currentfill}{rgb}{0.970012,0.883276,0.699577}%
\pgfsetfillcolor{currentfill}%
\pgfsetlinewidth{0.250937pt}%
\definecolor{currentstroke}{rgb}{1.000000,1.000000,1.000000}%
\pgfsetstrokecolor{currentstroke}%
\pgfsetdash{}{0pt}%
\pgfpathmoveto{\pgfqpoint{2.458603in}{0.493860in}}%
\pgfpathlineto{\pgfqpoint{2.557539in}{0.493860in}}%
\pgfpathlineto{\pgfqpoint{2.557539in}{0.394924in}}%
\pgfpathlineto{\pgfqpoint{2.458603in}{0.394924in}}%
\pgfpathlineto{\pgfqpoint{2.458603in}{0.493860in}}%
\pgfusepath{stroke,fill}%
\end{pgfscope}%
\begin{pgfscope}%
\pgfpathrectangle{\pgfqpoint{0.380943in}{0.295988in}}{\pgfqpoint{4.650000in}{0.692553in}}%
\pgfusepath{clip}%
\pgfsetbuttcap%
\pgfsetroundjoin%
\definecolor{currentfill}{rgb}{0.968997,0.888351,0.702284}%
\pgfsetfillcolor{currentfill}%
\pgfsetlinewidth{0.250937pt}%
\definecolor{currentstroke}{rgb}{1.000000,1.000000,1.000000}%
\pgfsetstrokecolor{currentstroke}%
\pgfsetdash{}{0pt}%
\pgfpathmoveto{\pgfqpoint{2.557539in}{0.493860in}}%
\pgfpathlineto{\pgfqpoint{2.656475in}{0.493860in}}%
\pgfpathlineto{\pgfqpoint{2.656475in}{0.394924in}}%
\pgfpathlineto{\pgfqpoint{2.557539in}{0.394924in}}%
\pgfpathlineto{\pgfqpoint{2.557539in}{0.493860in}}%
\pgfusepath{stroke,fill}%
\end{pgfscope}%
\begin{pgfscope}%
\pgfpathrectangle{\pgfqpoint{0.380943in}{0.295988in}}{\pgfqpoint{4.650000in}{0.692553in}}%
\pgfusepath{clip}%
\pgfsetbuttcap%
\pgfsetroundjoin%
\definecolor{currentfill}{rgb}{0.962584,0.922537,0.722215}%
\pgfsetfillcolor{currentfill}%
\pgfsetlinewidth{0.250937pt}%
\definecolor{currentstroke}{rgb}{1.000000,1.000000,1.000000}%
\pgfsetstrokecolor{currentstroke}%
\pgfsetdash{}{0pt}%
\pgfpathmoveto{\pgfqpoint{2.656475in}{0.493860in}}%
\pgfpathlineto{\pgfqpoint{2.755411in}{0.493860in}}%
\pgfpathlineto{\pgfqpoint{2.755411in}{0.394924in}}%
\pgfpathlineto{\pgfqpoint{2.656475in}{0.394924in}}%
\pgfpathlineto{\pgfqpoint{2.656475in}{0.493860in}}%
\pgfusepath{stroke,fill}%
\end{pgfscope}%
\begin{pgfscope}%
\pgfpathrectangle{\pgfqpoint{0.380943in}{0.295988in}}{\pgfqpoint{4.650000in}{0.692553in}}%
\pgfusepath{clip}%
\pgfsetbuttcap%
\pgfsetroundjoin%
\definecolor{currentfill}{rgb}{0.962584,0.922537,0.722215}%
\pgfsetfillcolor{currentfill}%
\pgfsetlinewidth{0.250937pt}%
\definecolor{currentstroke}{rgb}{1.000000,1.000000,1.000000}%
\pgfsetstrokecolor{currentstroke}%
\pgfsetdash{}{0pt}%
\pgfpathmoveto{\pgfqpoint{2.755411in}{0.493860in}}%
\pgfpathlineto{\pgfqpoint{2.854347in}{0.493860in}}%
\pgfpathlineto{\pgfqpoint{2.854347in}{0.394924in}}%
\pgfpathlineto{\pgfqpoint{2.755411in}{0.394924in}}%
\pgfpathlineto{\pgfqpoint{2.755411in}{0.493860in}}%
\pgfusepath{stroke,fill}%
\end{pgfscope}%
\begin{pgfscope}%
\pgfpathrectangle{\pgfqpoint{0.380943in}{0.295988in}}{\pgfqpoint{4.650000in}{0.692553in}}%
\pgfusepath{clip}%
\pgfsetbuttcap%
\pgfsetroundjoin%
\definecolor{currentfill}{rgb}{0.962584,0.922537,0.722215}%
\pgfsetfillcolor{currentfill}%
\pgfsetlinewidth{0.250937pt}%
\definecolor{currentstroke}{rgb}{1.000000,1.000000,1.000000}%
\pgfsetstrokecolor{currentstroke}%
\pgfsetdash{}{0pt}%
\pgfpathmoveto{\pgfqpoint{2.854347in}{0.493860in}}%
\pgfpathlineto{\pgfqpoint{2.953283in}{0.493860in}}%
\pgfpathlineto{\pgfqpoint{2.953283in}{0.394924in}}%
\pgfpathlineto{\pgfqpoint{2.854347in}{0.394924in}}%
\pgfpathlineto{\pgfqpoint{2.854347in}{0.493860in}}%
\pgfusepath{stroke,fill}%
\end{pgfscope}%
\begin{pgfscope}%
\pgfpathrectangle{\pgfqpoint{0.380943in}{0.295988in}}{\pgfqpoint{4.650000in}{0.692553in}}%
\pgfusepath{clip}%
\pgfsetbuttcap%
\pgfsetroundjoin%
\definecolor{currentfill}{rgb}{0.963937,0.914418,0.716801}%
\pgfsetfillcolor{currentfill}%
\pgfsetlinewidth{0.250937pt}%
\definecolor{currentstroke}{rgb}{1.000000,1.000000,1.000000}%
\pgfsetstrokecolor{currentstroke}%
\pgfsetdash{}{0pt}%
\pgfpathmoveto{\pgfqpoint{2.953283in}{0.493860in}}%
\pgfpathlineto{\pgfqpoint{3.052220in}{0.493860in}}%
\pgfpathlineto{\pgfqpoint{3.052220in}{0.394924in}}%
\pgfpathlineto{\pgfqpoint{2.953283in}{0.394924in}}%
\pgfpathlineto{\pgfqpoint{2.953283in}{0.493860in}}%
\pgfusepath{stroke,fill}%
\end{pgfscope}%
\begin{pgfscope}%
\pgfpathrectangle{\pgfqpoint{0.380943in}{0.295988in}}{\pgfqpoint{4.650000in}{0.692553in}}%
\pgfusepath{clip}%
\pgfsetbuttcap%
\pgfsetroundjoin%
\definecolor{currentfill}{rgb}{0.971534,0.875663,0.695517}%
\pgfsetfillcolor{currentfill}%
\pgfsetlinewidth{0.250937pt}%
\definecolor{currentstroke}{rgb}{1.000000,1.000000,1.000000}%
\pgfsetstrokecolor{currentstroke}%
\pgfsetdash{}{0pt}%
\pgfpathmoveto{\pgfqpoint{3.052220in}{0.493860in}}%
\pgfpathlineto{\pgfqpoint{3.151156in}{0.493860in}}%
\pgfpathlineto{\pgfqpoint{3.151156in}{0.394924in}}%
\pgfpathlineto{\pgfqpoint{3.052220in}{0.394924in}}%
\pgfpathlineto{\pgfqpoint{3.052220in}{0.493860in}}%
\pgfusepath{stroke,fill}%
\end{pgfscope}%
\begin{pgfscope}%
\pgfpathrectangle{\pgfqpoint{0.380943in}{0.295988in}}{\pgfqpoint{4.650000in}{0.692553in}}%
\pgfusepath{clip}%
\pgfsetbuttcap%
\pgfsetroundjoin%
\definecolor{currentfill}{rgb}{0.962584,0.922537,0.722215}%
\pgfsetfillcolor{currentfill}%
\pgfsetlinewidth{0.250937pt}%
\definecolor{currentstroke}{rgb}{1.000000,1.000000,1.000000}%
\pgfsetstrokecolor{currentstroke}%
\pgfsetdash{}{0pt}%
\pgfpathmoveto{\pgfqpoint{3.151156in}{0.493860in}}%
\pgfpathlineto{\pgfqpoint{3.250092in}{0.493860in}}%
\pgfpathlineto{\pgfqpoint{3.250092in}{0.394924in}}%
\pgfpathlineto{\pgfqpoint{3.151156in}{0.394924in}}%
\pgfpathlineto{\pgfqpoint{3.151156in}{0.493860in}}%
\pgfusepath{stroke,fill}%
\end{pgfscope}%
\begin{pgfscope}%
\pgfpathrectangle{\pgfqpoint{0.380943in}{0.295988in}}{\pgfqpoint{4.650000in}{0.692553in}}%
\pgfusepath{clip}%
\pgfsetbuttcap%
\pgfsetroundjoin%
\definecolor{currentfill}{rgb}{0.993003,0.759477,0.611457}%
\pgfsetfillcolor{currentfill}%
\pgfsetlinewidth{0.250937pt}%
\definecolor{currentstroke}{rgb}{1.000000,1.000000,1.000000}%
\pgfsetstrokecolor{currentstroke}%
\pgfsetdash{}{0pt}%
\pgfpathmoveto{\pgfqpoint{3.250092in}{0.493860in}}%
\pgfpathlineto{\pgfqpoint{3.349028in}{0.493860in}}%
\pgfpathlineto{\pgfqpoint{3.349028in}{0.394924in}}%
\pgfpathlineto{\pgfqpoint{3.250092in}{0.394924in}}%
\pgfpathlineto{\pgfqpoint{3.250092in}{0.493860in}}%
\pgfusepath{stroke,fill}%
\end{pgfscope}%
\begin{pgfscope}%
\pgfpathrectangle{\pgfqpoint{0.380943in}{0.295988in}}{\pgfqpoint{4.650000in}{0.692553in}}%
\pgfusepath{clip}%
\pgfsetbuttcap%
\pgfsetroundjoin%
\definecolor{currentfill}{rgb}{0.977116,0.848181,0.679769}%
\pgfsetfillcolor{currentfill}%
\pgfsetlinewidth{0.250937pt}%
\definecolor{currentstroke}{rgb}{1.000000,1.000000,1.000000}%
\pgfsetstrokecolor{currentstroke}%
\pgfsetdash{}{0pt}%
\pgfpathmoveto{\pgfqpoint{3.349028in}{0.493860in}}%
\pgfpathlineto{\pgfqpoint{3.447964in}{0.493860in}}%
\pgfpathlineto{\pgfqpoint{3.447964in}{0.394924in}}%
\pgfpathlineto{\pgfqpoint{3.349028in}{0.394924in}}%
\pgfpathlineto{\pgfqpoint{3.349028in}{0.493860in}}%
\pgfusepath{stroke,fill}%
\end{pgfscope}%
\begin{pgfscope}%
\pgfpathrectangle{\pgfqpoint{0.380943in}{0.295988in}}{\pgfqpoint{4.650000in}{0.692553in}}%
\pgfusepath{clip}%
\pgfsetbuttcap%
\pgfsetroundjoin%
\definecolor{currentfill}{rgb}{0.980669,0.832787,0.665559}%
\pgfsetfillcolor{currentfill}%
\pgfsetlinewidth{0.250937pt}%
\definecolor{currentstroke}{rgb}{1.000000,1.000000,1.000000}%
\pgfsetstrokecolor{currentstroke}%
\pgfsetdash{}{0pt}%
\pgfpathmoveto{\pgfqpoint{3.447964in}{0.493860in}}%
\pgfpathlineto{\pgfqpoint{3.546901in}{0.493860in}}%
\pgfpathlineto{\pgfqpoint{3.546901in}{0.394924in}}%
\pgfpathlineto{\pgfqpoint{3.447964in}{0.394924in}}%
\pgfpathlineto{\pgfqpoint{3.447964in}{0.493860in}}%
\pgfusepath{stroke,fill}%
\end{pgfscope}%
\begin{pgfscope}%
\pgfpathrectangle{\pgfqpoint{0.380943in}{0.295988in}}{\pgfqpoint{4.650000in}{0.692553in}}%
\pgfusepath{clip}%
\pgfsetbuttcap%
\pgfsetroundjoin%
\definecolor{currentfill}{rgb}{0.978131,0.843783,0.675709}%
\pgfsetfillcolor{currentfill}%
\pgfsetlinewidth{0.250937pt}%
\definecolor{currentstroke}{rgb}{1.000000,1.000000,1.000000}%
\pgfsetstrokecolor{currentstroke}%
\pgfsetdash{}{0pt}%
\pgfpathmoveto{\pgfqpoint{3.546901in}{0.493860in}}%
\pgfpathlineto{\pgfqpoint{3.645837in}{0.493860in}}%
\pgfpathlineto{\pgfqpoint{3.645837in}{0.394924in}}%
\pgfpathlineto{\pgfqpoint{3.546901in}{0.394924in}}%
\pgfpathlineto{\pgfqpoint{3.546901in}{0.493860in}}%
\pgfusepath{stroke,fill}%
\end{pgfscope}%
\begin{pgfscope}%
\pgfpathrectangle{\pgfqpoint{0.380943in}{0.295988in}}{\pgfqpoint{4.650000in}{0.692553in}}%
\pgfusepath{clip}%
\pgfsetbuttcap%
\pgfsetroundjoin%
\definecolor{currentfill}{rgb}{0.967474,0.895963,0.706344}%
\pgfsetfillcolor{currentfill}%
\pgfsetlinewidth{0.250937pt}%
\definecolor{currentstroke}{rgb}{1.000000,1.000000,1.000000}%
\pgfsetstrokecolor{currentstroke}%
\pgfsetdash{}{0pt}%
\pgfpathmoveto{\pgfqpoint{3.645837in}{0.493860in}}%
\pgfpathlineto{\pgfqpoint{3.744773in}{0.493860in}}%
\pgfpathlineto{\pgfqpoint{3.744773in}{0.394924in}}%
\pgfpathlineto{\pgfqpoint{3.645837in}{0.394924in}}%
\pgfpathlineto{\pgfqpoint{3.645837in}{0.493860in}}%
\pgfusepath{stroke,fill}%
\end{pgfscope}%
\begin{pgfscope}%
\pgfpathrectangle{\pgfqpoint{0.380943in}{0.295988in}}{\pgfqpoint{4.650000in}{0.692553in}}%
\pgfusepath{clip}%
\pgfsetbuttcap%
\pgfsetroundjoin%
\definecolor{currentfill}{rgb}{0.986251,0.808597,0.643230}%
\pgfsetfillcolor{currentfill}%
\pgfsetlinewidth{0.250937pt}%
\definecolor{currentstroke}{rgb}{1.000000,1.000000,1.000000}%
\pgfsetstrokecolor{currentstroke}%
\pgfsetdash{}{0pt}%
\pgfpathmoveto{\pgfqpoint{3.744773in}{0.493860in}}%
\pgfpathlineto{\pgfqpoint{3.843709in}{0.493860in}}%
\pgfpathlineto{\pgfqpoint{3.843709in}{0.394924in}}%
\pgfpathlineto{\pgfqpoint{3.744773in}{0.394924in}}%
\pgfpathlineto{\pgfqpoint{3.744773in}{0.493860in}}%
\pgfusepath{stroke,fill}%
\end{pgfscope}%
\begin{pgfscope}%
\pgfpathrectangle{\pgfqpoint{0.380943in}{0.295988in}}{\pgfqpoint{4.650000in}{0.692553in}}%
\pgfusepath{clip}%
\pgfsetbuttcap%
\pgfsetroundjoin%
\definecolor{currentfill}{rgb}{0.999446,0.645767,0.548927}%
\pgfsetfillcolor{currentfill}%
\pgfsetlinewidth{0.250937pt}%
\definecolor{currentstroke}{rgb}{1.000000,1.000000,1.000000}%
\pgfsetstrokecolor{currentstroke}%
\pgfsetdash{}{0pt}%
\pgfpathmoveto{\pgfqpoint{3.843709in}{0.493860in}}%
\pgfpathlineto{\pgfqpoint{3.942645in}{0.493860in}}%
\pgfpathlineto{\pgfqpoint{3.942645in}{0.394924in}}%
\pgfpathlineto{\pgfqpoint{3.843709in}{0.394924in}}%
\pgfpathlineto{\pgfqpoint{3.843709in}{0.493860in}}%
\pgfusepath{stroke,fill}%
\end{pgfscope}%
\begin{pgfscope}%
\pgfpathrectangle{\pgfqpoint{0.380943in}{0.295988in}}{\pgfqpoint{4.650000in}{0.692553in}}%
\pgfusepath{clip}%
\pgfsetbuttcap%
\pgfsetroundjoin%
\definecolor{currentfill}{rgb}{0.980669,0.832787,0.665559}%
\pgfsetfillcolor{currentfill}%
\pgfsetlinewidth{0.250937pt}%
\definecolor{currentstroke}{rgb}{1.000000,1.000000,1.000000}%
\pgfsetstrokecolor{currentstroke}%
\pgfsetdash{}{0pt}%
\pgfpathmoveto{\pgfqpoint{3.942645in}{0.493860in}}%
\pgfpathlineto{\pgfqpoint{4.041581in}{0.493860in}}%
\pgfpathlineto{\pgfqpoint{4.041581in}{0.394924in}}%
\pgfpathlineto{\pgfqpoint{3.942645in}{0.394924in}}%
\pgfpathlineto{\pgfqpoint{3.942645in}{0.493860in}}%
\pgfusepath{stroke,fill}%
\end{pgfscope}%
\begin{pgfscope}%
\pgfpathrectangle{\pgfqpoint{0.380943in}{0.295988in}}{\pgfqpoint{4.650000in}{0.692553in}}%
\pgfusepath{clip}%
\pgfsetbuttcap%
\pgfsetroundjoin%
\definecolor{currentfill}{rgb}{0.997586,0.694148,0.574979}%
\pgfsetfillcolor{currentfill}%
\pgfsetlinewidth{0.250937pt}%
\definecolor{currentstroke}{rgb}{1.000000,1.000000,1.000000}%
\pgfsetstrokecolor{currentstroke}%
\pgfsetdash{}{0pt}%
\pgfpathmoveto{\pgfqpoint{4.041581in}{0.493860in}}%
\pgfpathlineto{\pgfqpoint{4.140518in}{0.493860in}}%
\pgfpathlineto{\pgfqpoint{4.140518in}{0.394924in}}%
\pgfpathlineto{\pgfqpoint{4.041581in}{0.394924in}}%
\pgfpathlineto{\pgfqpoint{4.041581in}{0.493860in}}%
\pgfusepath{stroke,fill}%
\end{pgfscope}%
\begin{pgfscope}%
\pgfpathrectangle{\pgfqpoint{0.380943in}{0.295988in}}{\pgfqpoint{4.650000in}{0.692553in}}%
\pgfusepath{clip}%
\pgfsetbuttcap%
\pgfsetroundjoin%
\definecolor{currentfill}{rgb}{0.991311,0.773856,0.619915}%
\pgfsetfillcolor{currentfill}%
\pgfsetlinewidth{0.250937pt}%
\definecolor{currentstroke}{rgb}{1.000000,1.000000,1.000000}%
\pgfsetstrokecolor{currentstroke}%
\pgfsetdash{}{0pt}%
\pgfpathmoveto{\pgfqpoint{4.140518in}{0.493860in}}%
\pgfpathlineto{\pgfqpoint{4.239454in}{0.493860in}}%
\pgfpathlineto{\pgfqpoint{4.239454in}{0.394924in}}%
\pgfpathlineto{\pgfqpoint{4.140518in}{0.394924in}}%
\pgfpathlineto{\pgfqpoint{4.140518in}{0.493860in}}%
\pgfusepath{stroke,fill}%
\end{pgfscope}%
\begin{pgfscope}%
\pgfpathrectangle{\pgfqpoint{0.380943in}{0.295988in}}{\pgfqpoint{4.650000in}{0.692553in}}%
\pgfusepath{clip}%
\pgfsetbuttcap%
\pgfsetroundjoin%
\definecolor{currentfill}{rgb}{0.963429,0.917463,0.718831}%
\pgfsetfillcolor{currentfill}%
\pgfsetlinewidth{0.250937pt}%
\definecolor{currentstroke}{rgb}{1.000000,1.000000,1.000000}%
\pgfsetstrokecolor{currentstroke}%
\pgfsetdash{}{0pt}%
\pgfpathmoveto{\pgfqpoint{4.239454in}{0.493860in}}%
\pgfpathlineto{\pgfqpoint{4.338390in}{0.493860in}}%
\pgfpathlineto{\pgfqpoint{4.338390in}{0.394924in}}%
\pgfpathlineto{\pgfqpoint{4.239454in}{0.394924in}}%
\pgfpathlineto{\pgfqpoint{4.239454in}{0.493860in}}%
\pgfusepath{stroke,fill}%
\end{pgfscope}%
\begin{pgfscope}%
\pgfpathrectangle{\pgfqpoint{0.380943in}{0.295988in}}{\pgfqpoint{4.650000in}{0.692553in}}%
\pgfusepath{clip}%
\pgfsetbuttcap%
\pgfsetroundjoin%
\definecolor{currentfill}{rgb}{0.960892,0.932687,0.728981}%
\pgfsetfillcolor{currentfill}%
\pgfsetlinewidth{0.250937pt}%
\definecolor{currentstroke}{rgb}{1.000000,1.000000,1.000000}%
\pgfsetstrokecolor{currentstroke}%
\pgfsetdash{}{0pt}%
\pgfpathmoveto{\pgfqpoint{4.338390in}{0.493860in}}%
\pgfpathlineto{\pgfqpoint{4.437326in}{0.493860in}}%
\pgfpathlineto{\pgfqpoint{4.437326in}{0.394924in}}%
\pgfpathlineto{\pgfqpoint{4.338390in}{0.394924in}}%
\pgfpathlineto{\pgfqpoint{4.338390in}{0.493860in}}%
\pgfusepath{stroke,fill}%
\end{pgfscope}%
\begin{pgfscope}%
\pgfpathrectangle{\pgfqpoint{0.380943in}{0.295988in}}{\pgfqpoint{4.650000in}{0.692553in}}%
\pgfusepath{clip}%
\pgfsetbuttcap%
\pgfsetroundjoin%
\definecolor{currentfill}{rgb}{0.964783,0.940131,0.739808}%
\pgfsetfillcolor{currentfill}%
\pgfsetlinewidth{0.250937pt}%
\definecolor{currentstroke}{rgb}{1.000000,1.000000,1.000000}%
\pgfsetstrokecolor{currentstroke}%
\pgfsetdash{}{0pt}%
\pgfpathmoveto{\pgfqpoint{4.437326in}{0.493860in}}%
\pgfpathlineto{\pgfqpoint{4.536262in}{0.493860in}}%
\pgfpathlineto{\pgfqpoint{4.536262in}{0.394924in}}%
\pgfpathlineto{\pgfqpoint{4.437326in}{0.394924in}}%
\pgfpathlineto{\pgfqpoint{4.437326in}{0.493860in}}%
\pgfusepath{stroke,fill}%
\end{pgfscope}%
\begin{pgfscope}%
\pgfpathrectangle{\pgfqpoint{0.380943in}{0.295988in}}{\pgfqpoint{4.650000in}{0.692553in}}%
\pgfusepath{clip}%
\pgfsetbuttcap%
\pgfsetroundjoin%
\definecolor{currentfill}{rgb}{0.962584,0.922537,0.722215}%
\pgfsetfillcolor{currentfill}%
\pgfsetlinewidth{0.250937pt}%
\definecolor{currentstroke}{rgb}{1.000000,1.000000,1.000000}%
\pgfsetstrokecolor{currentstroke}%
\pgfsetdash{}{0pt}%
\pgfpathmoveto{\pgfqpoint{4.536262in}{0.493860in}}%
\pgfpathlineto{\pgfqpoint{4.635198in}{0.493860in}}%
\pgfpathlineto{\pgfqpoint{4.635198in}{0.394924in}}%
\pgfpathlineto{\pgfqpoint{4.536262in}{0.394924in}}%
\pgfpathlineto{\pgfqpoint{4.536262in}{0.493860in}}%
\pgfusepath{stroke,fill}%
\end{pgfscope}%
\begin{pgfscope}%
\pgfpathrectangle{\pgfqpoint{0.380943in}{0.295988in}}{\pgfqpoint{4.650000in}{0.692553in}}%
\pgfusepath{clip}%
\pgfsetbuttcap%
\pgfsetroundjoin%
\definecolor{currentfill}{rgb}{0.964783,0.940131,0.739808}%
\pgfsetfillcolor{currentfill}%
\pgfsetlinewidth{0.250937pt}%
\definecolor{currentstroke}{rgb}{1.000000,1.000000,1.000000}%
\pgfsetstrokecolor{currentstroke}%
\pgfsetdash{}{0pt}%
\pgfpathmoveto{\pgfqpoint{4.635198in}{0.493860in}}%
\pgfpathlineto{\pgfqpoint{4.734135in}{0.493860in}}%
\pgfpathlineto{\pgfqpoint{4.734135in}{0.394924in}}%
\pgfpathlineto{\pgfqpoint{4.635198in}{0.394924in}}%
\pgfpathlineto{\pgfqpoint{4.635198in}{0.493860in}}%
\pgfusepath{stroke,fill}%
\end{pgfscope}%
\begin{pgfscope}%
\pgfpathrectangle{\pgfqpoint{0.380943in}{0.295988in}}{\pgfqpoint{4.650000in}{0.692553in}}%
\pgfusepath{clip}%
\pgfsetbuttcap%
\pgfsetroundjoin%
\definecolor{currentfill}{rgb}{0.960892,0.932687,0.728981}%
\pgfsetfillcolor{currentfill}%
\pgfsetlinewidth{0.250937pt}%
\definecolor{currentstroke}{rgb}{1.000000,1.000000,1.000000}%
\pgfsetstrokecolor{currentstroke}%
\pgfsetdash{}{0pt}%
\pgfpathmoveto{\pgfqpoint{4.734135in}{0.493860in}}%
\pgfpathlineto{\pgfqpoint{4.833071in}{0.493860in}}%
\pgfpathlineto{\pgfqpoint{4.833071in}{0.394924in}}%
\pgfpathlineto{\pgfqpoint{4.734135in}{0.394924in}}%
\pgfpathlineto{\pgfqpoint{4.734135in}{0.493860in}}%
\pgfusepath{stroke,fill}%
\end{pgfscope}%
\begin{pgfscope}%
\pgfpathrectangle{\pgfqpoint{0.380943in}{0.295988in}}{\pgfqpoint{4.650000in}{0.692553in}}%
\pgfusepath{clip}%
\pgfsetbuttcap%
\pgfsetroundjoin%
\definecolor{currentfill}{rgb}{0.960892,0.932687,0.728981}%
\pgfsetfillcolor{currentfill}%
\pgfsetlinewidth{0.250937pt}%
\definecolor{currentstroke}{rgb}{1.000000,1.000000,1.000000}%
\pgfsetstrokecolor{currentstroke}%
\pgfsetdash{}{0pt}%
\pgfpathmoveto{\pgfqpoint{4.833071in}{0.493860in}}%
\pgfpathlineto{\pgfqpoint{4.932007in}{0.493860in}}%
\pgfpathlineto{\pgfqpoint{4.932007in}{0.394924in}}%
\pgfpathlineto{\pgfqpoint{4.833071in}{0.394924in}}%
\pgfpathlineto{\pgfqpoint{4.833071in}{0.493860in}}%
\pgfusepath{stroke,fill}%
\end{pgfscope}%
\begin{pgfscope}%
\pgfpathrectangle{\pgfqpoint{0.380943in}{0.295988in}}{\pgfqpoint{4.650000in}{0.692553in}}%
\pgfusepath{clip}%
\pgfsetbuttcap%
\pgfsetroundjoin%
\pgfsetlinewidth{0.250937pt}%
\definecolor{currentstroke}{rgb}{1.000000,1.000000,1.000000}%
\pgfsetstrokecolor{currentstroke}%
\pgfsetdash{}{0pt}%
\pgfpathmoveto{\pgfqpoint{4.932007in}{0.493860in}}%
\pgfpathlineto{\pgfqpoint{5.030943in}{0.493860in}}%
\pgfpathlineto{\pgfqpoint{5.030943in}{0.394924in}}%
\pgfpathlineto{\pgfqpoint{4.932007in}{0.394924in}}%
\pgfpathlineto{\pgfqpoint{4.932007in}{0.493860in}}%
\pgfusepath{stroke}%
\end{pgfscope}%
\begin{pgfscope}%
\pgfpathrectangle{\pgfqpoint{0.380943in}{0.295988in}}{\pgfqpoint{4.650000in}{0.692553in}}%
\pgfusepath{clip}%
\pgfsetbuttcap%
\pgfsetroundjoin%
\definecolor{currentfill}{rgb}{1.000000,1.000000,0.929412}%
\pgfsetfillcolor{currentfill}%
\pgfsetlinewidth{0.250937pt}%
\definecolor{currentstroke}{rgb}{1.000000,1.000000,1.000000}%
\pgfsetstrokecolor{currentstroke}%
\pgfsetdash{}{0pt}%
\pgfpathmoveto{\pgfqpoint{0.380943in}{0.394924in}}%
\pgfpathlineto{\pgfqpoint{0.479879in}{0.394924in}}%
\pgfpathlineto{\pgfqpoint{0.479879in}{0.295988in}}%
\pgfpathlineto{\pgfqpoint{0.380943in}{0.295988in}}%
\pgfpathlineto{\pgfqpoint{0.380943in}{0.394924in}}%
\pgfusepath{stroke,fill}%
\end{pgfscope}%
\begin{pgfscope}%
\pgfpathrectangle{\pgfqpoint{0.380943in}{0.295988in}}{\pgfqpoint{4.650000in}{0.692553in}}%
\pgfusepath{clip}%
\pgfsetbuttcap%
\pgfsetroundjoin%
\definecolor{currentfill}{rgb}{1.000000,1.000000,0.874433}%
\pgfsetfillcolor{currentfill}%
\pgfsetlinewidth{0.250937pt}%
\definecolor{currentstroke}{rgb}{1.000000,1.000000,1.000000}%
\pgfsetstrokecolor{currentstroke}%
\pgfsetdash{}{0pt}%
\pgfpathmoveto{\pgfqpoint{0.479879in}{0.394924in}}%
\pgfpathlineto{\pgfqpoint{0.578815in}{0.394924in}}%
\pgfpathlineto{\pgfqpoint{0.578815in}{0.295988in}}%
\pgfpathlineto{\pgfqpoint{0.479879in}{0.295988in}}%
\pgfpathlineto{\pgfqpoint{0.479879in}{0.394924in}}%
\pgfusepath{stroke,fill}%
\end{pgfscope}%
\begin{pgfscope}%
\pgfpathrectangle{\pgfqpoint{0.380943in}{0.295988in}}{\pgfqpoint{4.650000in}{0.692553in}}%
\pgfusepath{clip}%
\pgfsetbuttcap%
\pgfsetroundjoin%
\definecolor{currentfill}{rgb}{1.000000,1.000000,0.865975}%
\pgfsetfillcolor{currentfill}%
\pgfsetlinewidth{0.250937pt}%
\definecolor{currentstroke}{rgb}{1.000000,1.000000,1.000000}%
\pgfsetstrokecolor{currentstroke}%
\pgfsetdash{}{0pt}%
\pgfpathmoveto{\pgfqpoint{0.578815in}{0.394924in}}%
\pgfpathlineto{\pgfqpoint{0.677752in}{0.394924in}}%
\pgfpathlineto{\pgfqpoint{0.677752in}{0.295988in}}%
\pgfpathlineto{\pgfqpoint{0.578815in}{0.295988in}}%
\pgfpathlineto{\pgfqpoint{0.578815in}{0.394924in}}%
\pgfusepath{stroke,fill}%
\end{pgfscope}%
\begin{pgfscope}%
\pgfpathrectangle{\pgfqpoint{0.380943in}{0.295988in}}{\pgfqpoint{4.650000in}{0.692553in}}%
\pgfusepath{clip}%
\pgfsetbuttcap%
\pgfsetroundjoin%
\definecolor{currentfill}{rgb}{1.000000,1.000000,0.920953}%
\pgfsetfillcolor{currentfill}%
\pgfsetlinewidth{0.250937pt}%
\definecolor{currentstroke}{rgb}{1.000000,1.000000,1.000000}%
\pgfsetstrokecolor{currentstroke}%
\pgfsetdash{}{0pt}%
\pgfpathmoveto{\pgfqpoint{0.677752in}{0.394924in}}%
\pgfpathlineto{\pgfqpoint{0.776688in}{0.394924in}}%
\pgfpathlineto{\pgfqpoint{0.776688in}{0.295988in}}%
\pgfpathlineto{\pgfqpoint{0.677752in}{0.295988in}}%
\pgfpathlineto{\pgfqpoint{0.677752in}{0.394924in}}%
\pgfusepath{stroke,fill}%
\end{pgfscope}%
\begin{pgfscope}%
\pgfpathrectangle{\pgfqpoint{0.380943in}{0.295988in}}{\pgfqpoint{4.650000in}{0.692553in}}%
\pgfusepath{clip}%
\pgfsetbuttcap%
\pgfsetroundjoin%
\definecolor{currentfill}{rgb}{1.000000,1.000000,0.865975}%
\pgfsetfillcolor{currentfill}%
\pgfsetlinewidth{0.250937pt}%
\definecolor{currentstroke}{rgb}{1.000000,1.000000,1.000000}%
\pgfsetstrokecolor{currentstroke}%
\pgfsetdash{}{0pt}%
\pgfpathmoveto{\pgfqpoint{0.776688in}{0.394924in}}%
\pgfpathlineto{\pgfqpoint{0.875624in}{0.394924in}}%
\pgfpathlineto{\pgfqpoint{0.875624in}{0.295988in}}%
\pgfpathlineto{\pgfqpoint{0.776688in}{0.295988in}}%
\pgfpathlineto{\pgfqpoint{0.776688in}{0.394924in}}%
\pgfusepath{stroke,fill}%
\end{pgfscope}%
\begin{pgfscope}%
\pgfpathrectangle{\pgfqpoint{0.380943in}{0.295988in}}{\pgfqpoint{4.650000in}{0.692553in}}%
\pgfusepath{clip}%
\pgfsetbuttcap%
\pgfsetroundjoin%
\definecolor{currentfill}{rgb}{1.000000,1.000000,0.920953}%
\pgfsetfillcolor{currentfill}%
\pgfsetlinewidth{0.250937pt}%
\definecolor{currentstroke}{rgb}{1.000000,1.000000,1.000000}%
\pgfsetstrokecolor{currentstroke}%
\pgfsetdash{}{0pt}%
\pgfpathmoveto{\pgfqpoint{0.875624in}{0.394924in}}%
\pgfpathlineto{\pgfqpoint{0.974560in}{0.394924in}}%
\pgfpathlineto{\pgfqpoint{0.974560in}{0.295988in}}%
\pgfpathlineto{\pgfqpoint{0.875624in}{0.295988in}}%
\pgfpathlineto{\pgfqpoint{0.875624in}{0.394924in}}%
\pgfusepath{stroke,fill}%
\end{pgfscope}%
\begin{pgfscope}%
\pgfpathrectangle{\pgfqpoint{0.380943in}{0.295988in}}{\pgfqpoint{4.650000in}{0.692553in}}%
\pgfusepath{clip}%
\pgfsetbuttcap%
\pgfsetroundjoin%
\definecolor{currentfill}{rgb}{1.000000,1.000000,0.908266}%
\pgfsetfillcolor{currentfill}%
\pgfsetlinewidth{0.250937pt}%
\definecolor{currentstroke}{rgb}{1.000000,1.000000,1.000000}%
\pgfsetstrokecolor{currentstroke}%
\pgfsetdash{}{0pt}%
\pgfpathmoveto{\pgfqpoint{0.974560in}{0.394924in}}%
\pgfpathlineto{\pgfqpoint{1.073496in}{0.394924in}}%
\pgfpathlineto{\pgfqpoint{1.073496in}{0.295988in}}%
\pgfpathlineto{\pgfqpoint{0.974560in}{0.295988in}}%
\pgfpathlineto{\pgfqpoint{0.974560in}{0.394924in}}%
\pgfusepath{stroke,fill}%
\end{pgfscope}%
\begin{pgfscope}%
\pgfpathrectangle{\pgfqpoint{0.380943in}{0.295988in}}{\pgfqpoint{4.650000in}{0.692553in}}%
\pgfusepath{clip}%
\pgfsetbuttcap%
\pgfsetroundjoin%
\definecolor{currentfill}{rgb}{1.000000,1.000000,0.887120}%
\pgfsetfillcolor{currentfill}%
\pgfsetlinewidth{0.250937pt}%
\definecolor{currentstroke}{rgb}{1.000000,1.000000,1.000000}%
\pgfsetstrokecolor{currentstroke}%
\pgfsetdash{}{0pt}%
\pgfpathmoveto{\pgfqpoint{1.073496in}{0.394924in}}%
\pgfpathlineto{\pgfqpoint{1.172432in}{0.394924in}}%
\pgfpathlineto{\pgfqpoint{1.172432in}{0.295988in}}%
\pgfpathlineto{\pgfqpoint{1.073496in}{0.295988in}}%
\pgfpathlineto{\pgfqpoint{1.073496in}{0.394924in}}%
\pgfusepath{stroke,fill}%
\end{pgfscope}%
\begin{pgfscope}%
\pgfpathrectangle{\pgfqpoint{0.380943in}{0.295988in}}{\pgfqpoint{4.650000in}{0.692553in}}%
\pgfusepath{clip}%
\pgfsetbuttcap%
\pgfsetroundjoin%
\definecolor{currentfill}{rgb}{1.000000,1.000000,0.908266}%
\pgfsetfillcolor{currentfill}%
\pgfsetlinewidth{0.250937pt}%
\definecolor{currentstroke}{rgb}{1.000000,1.000000,1.000000}%
\pgfsetstrokecolor{currentstroke}%
\pgfsetdash{}{0pt}%
\pgfpathmoveto{\pgfqpoint{1.172432in}{0.394924in}}%
\pgfpathlineto{\pgfqpoint{1.271369in}{0.394924in}}%
\pgfpathlineto{\pgfqpoint{1.271369in}{0.295988in}}%
\pgfpathlineto{\pgfqpoint{1.172432in}{0.295988in}}%
\pgfpathlineto{\pgfqpoint{1.172432in}{0.394924in}}%
\pgfusepath{stroke,fill}%
\end{pgfscope}%
\begin{pgfscope}%
\pgfpathrectangle{\pgfqpoint{0.380943in}{0.295988in}}{\pgfqpoint{4.650000in}{0.692553in}}%
\pgfusepath{clip}%
\pgfsetbuttcap%
\pgfsetroundjoin%
\definecolor{currentfill}{rgb}{0.966459,0.901038,0.709050}%
\pgfsetfillcolor{currentfill}%
\pgfsetlinewidth{0.250937pt}%
\definecolor{currentstroke}{rgb}{1.000000,1.000000,1.000000}%
\pgfsetstrokecolor{currentstroke}%
\pgfsetdash{}{0pt}%
\pgfpathmoveto{\pgfqpoint{1.271369in}{0.394924in}}%
\pgfpathlineto{\pgfqpoint{1.370305in}{0.394924in}}%
\pgfpathlineto{\pgfqpoint{1.370305in}{0.295988in}}%
\pgfpathlineto{\pgfqpoint{1.271369in}{0.295988in}}%
\pgfpathlineto{\pgfqpoint{1.271369in}{0.394924in}}%
\pgfusepath{stroke,fill}%
\end{pgfscope}%
\begin{pgfscope}%
\pgfpathrectangle{\pgfqpoint{0.380943in}{0.295988in}}{\pgfqpoint{4.650000in}{0.692553in}}%
\pgfusepath{clip}%
\pgfsetbuttcap%
\pgfsetroundjoin%
\definecolor{currentfill}{rgb}{0.963429,0.917463,0.718831}%
\pgfsetfillcolor{currentfill}%
\pgfsetlinewidth{0.250937pt}%
\definecolor{currentstroke}{rgb}{1.000000,1.000000,1.000000}%
\pgfsetstrokecolor{currentstroke}%
\pgfsetdash{}{0pt}%
\pgfpathmoveto{\pgfqpoint{1.370305in}{0.394924in}}%
\pgfpathlineto{\pgfqpoint{1.469241in}{0.394924in}}%
\pgfpathlineto{\pgfqpoint{1.469241in}{0.295988in}}%
\pgfpathlineto{\pgfqpoint{1.370305in}{0.295988in}}%
\pgfpathlineto{\pgfqpoint{1.370305in}{0.394924in}}%
\pgfusepath{stroke,fill}%
\end{pgfscope}%
\begin{pgfscope}%
\pgfpathrectangle{\pgfqpoint{0.380943in}{0.295988in}}{\pgfqpoint{4.650000in}{0.692553in}}%
\pgfusepath{clip}%
\pgfsetbuttcap%
\pgfsetroundjoin%
\definecolor{currentfill}{rgb}{0.964937,0.908651,0.713110}%
\pgfsetfillcolor{currentfill}%
\pgfsetlinewidth{0.250937pt}%
\definecolor{currentstroke}{rgb}{1.000000,1.000000,1.000000}%
\pgfsetstrokecolor{currentstroke}%
\pgfsetdash{}{0pt}%
\pgfpathmoveto{\pgfqpoint{1.469241in}{0.394924in}}%
\pgfpathlineto{\pgfqpoint{1.568177in}{0.394924in}}%
\pgfpathlineto{\pgfqpoint{1.568177in}{0.295988in}}%
\pgfpathlineto{\pgfqpoint{1.469241in}{0.295988in}}%
\pgfpathlineto{\pgfqpoint{1.469241in}{0.394924in}}%
\pgfusepath{stroke,fill}%
\end{pgfscope}%
\begin{pgfscope}%
\pgfpathrectangle{\pgfqpoint{0.380943in}{0.295988in}}{\pgfqpoint{4.650000in}{0.692553in}}%
\pgfusepath{clip}%
\pgfsetbuttcap%
\pgfsetroundjoin%
\definecolor{currentfill}{rgb}{0.961738,0.927612,0.725598}%
\pgfsetfillcolor{currentfill}%
\pgfsetlinewidth{0.250937pt}%
\definecolor{currentstroke}{rgb}{1.000000,1.000000,1.000000}%
\pgfsetstrokecolor{currentstroke}%
\pgfsetdash{}{0pt}%
\pgfpathmoveto{\pgfqpoint{1.568177in}{0.394924in}}%
\pgfpathlineto{\pgfqpoint{1.667113in}{0.394924in}}%
\pgfpathlineto{\pgfqpoint{1.667113in}{0.295988in}}%
\pgfpathlineto{\pgfqpoint{1.568177in}{0.295988in}}%
\pgfpathlineto{\pgfqpoint{1.568177in}{0.394924in}}%
\pgfusepath{stroke,fill}%
\end{pgfscope}%
\begin{pgfscope}%
\pgfpathrectangle{\pgfqpoint{0.380943in}{0.295988in}}{\pgfqpoint{4.650000in}{0.692553in}}%
\pgfusepath{clip}%
\pgfsetbuttcap%
\pgfsetroundjoin%
\definecolor{currentfill}{rgb}{0.961230,0.930657,0.727628}%
\pgfsetfillcolor{currentfill}%
\pgfsetlinewidth{0.250937pt}%
\definecolor{currentstroke}{rgb}{1.000000,1.000000,1.000000}%
\pgfsetstrokecolor{currentstroke}%
\pgfsetdash{}{0pt}%
\pgfpathmoveto{\pgfqpoint{1.667113in}{0.394924in}}%
\pgfpathlineto{\pgfqpoint{1.766049in}{0.394924in}}%
\pgfpathlineto{\pgfqpoint{1.766049in}{0.295988in}}%
\pgfpathlineto{\pgfqpoint{1.667113in}{0.295988in}}%
\pgfpathlineto{\pgfqpoint{1.667113in}{0.394924in}}%
\pgfusepath{stroke,fill}%
\end{pgfscope}%
\begin{pgfscope}%
\pgfpathrectangle{\pgfqpoint{0.380943in}{0.295988in}}{\pgfqpoint{4.650000in}{0.692553in}}%
\pgfusepath{clip}%
\pgfsetbuttcap%
\pgfsetroundjoin%
\definecolor{currentfill}{rgb}{0.967474,0.895963,0.706344}%
\pgfsetfillcolor{currentfill}%
\pgfsetlinewidth{0.250937pt}%
\definecolor{currentstroke}{rgb}{1.000000,1.000000,1.000000}%
\pgfsetstrokecolor{currentstroke}%
\pgfsetdash{}{0pt}%
\pgfpathmoveto{\pgfqpoint{1.766049in}{0.394924in}}%
\pgfpathlineto{\pgfqpoint{1.864986in}{0.394924in}}%
\pgfpathlineto{\pgfqpoint{1.864986in}{0.295988in}}%
\pgfpathlineto{\pgfqpoint{1.766049in}{0.295988in}}%
\pgfpathlineto{\pgfqpoint{1.766049in}{0.394924in}}%
\pgfusepath{stroke,fill}%
\end{pgfscope}%
\begin{pgfscope}%
\pgfpathrectangle{\pgfqpoint{0.380943in}{0.295988in}}{\pgfqpoint{4.650000in}{0.692553in}}%
\pgfusepath{clip}%
\pgfsetbuttcap%
\pgfsetroundjoin%
\definecolor{currentfill}{rgb}{0.964783,0.940131,0.739808}%
\pgfsetfillcolor{currentfill}%
\pgfsetlinewidth{0.250937pt}%
\definecolor{currentstroke}{rgb}{1.000000,1.000000,1.000000}%
\pgfsetstrokecolor{currentstroke}%
\pgfsetdash{}{0pt}%
\pgfpathmoveto{\pgfqpoint{1.864986in}{0.394924in}}%
\pgfpathlineto{\pgfqpoint{1.963922in}{0.394924in}}%
\pgfpathlineto{\pgfqpoint{1.963922in}{0.295988in}}%
\pgfpathlineto{\pgfqpoint{1.864986in}{0.295988in}}%
\pgfpathlineto{\pgfqpoint{1.864986in}{0.394924in}}%
\pgfusepath{stroke,fill}%
\end{pgfscope}%
\begin{pgfscope}%
\pgfpathrectangle{\pgfqpoint{0.380943in}{0.295988in}}{\pgfqpoint{4.650000in}{0.692553in}}%
\pgfusepath{clip}%
\pgfsetbuttcap%
\pgfsetroundjoin%
\definecolor{currentfill}{rgb}{0.978316,0.963137,0.774994}%
\pgfsetfillcolor{currentfill}%
\pgfsetlinewidth{0.250937pt}%
\definecolor{currentstroke}{rgb}{1.000000,1.000000,1.000000}%
\pgfsetstrokecolor{currentstroke}%
\pgfsetdash{}{0pt}%
\pgfpathmoveto{\pgfqpoint{1.963922in}{0.394924in}}%
\pgfpathlineto{\pgfqpoint{2.062858in}{0.394924in}}%
\pgfpathlineto{\pgfqpoint{2.062858in}{0.295988in}}%
\pgfpathlineto{\pgfqpoint{1.963922in}{0.295988in}}%
\pgfpathlineto{\pgfqpoint{1.963922in}{0.394924in}}%
\pgfusepath{stroke,fill}%
\end{pgfscope}%
\begin{pgfscope}%
\pgfpathrectangle{\pgfqpoint{0.380943in}{0.295988in}}{\pgfqpoint{4.650000in}{0.692553in}}%
\pgfusepath{clip}%
\pgfsetbuttcap%
\pgfsetroundjoin%
\definecolor{currentfill}{rgb}{0.963091,0.919493,0.720185}%
\pgfsetfillcolor{currentfill}%
\pgfsetlinewidth{0.250937pt}%
\definecolor{currentstroke}{rgb}{1.000000,1.000000,1.000000}%
\pgfsetstrokecolor{currentstroke}%
\pgfsetdash{}{0pt}%
\pgfpathmoveto{\pgfqpoint{2.062858in}{0.394924in}}%
\pgfpathlineto{\pgfqpoint{2.161794in}{0.394924in}}%
\pgfpathlineto{\pgfqpoint{2.161794in}{0.295988in}}%
\pgfpathlineto{\pgfqpoint{2.062858in}{0.295988in}}%
\pgfpathlineto{\pgfqpoint{2.062858in}{0.394924in}}%
\pgfusepath{stroke,fill}%
\end{pgfscope}%
\begin{pgfscope}%
\pgfpathrectangle{\pgfqpoint{0.380943in}{0.295988in}}{\pgfqpoint{4.650000in}{0.692553in}}%
\pgfusepath{clip}%
\pgfsetbuttcap%
\pgfsetroundjoin%
\definecolor{currentfill}{rgb}{0.964275,0.912388,0.715448}%
\pgfsetfillcolor{currentfill}%
\pgfsetlinewidth{0.250937pt}%
\definecolor{currentstroke}{rgb}{1.000000,1.000000,1.000000}%
\pgfsetstrokecolor{currentstroke}%
\pgfsetdash{}{0pt}%
\pgfpathmoveto{\pgfqpoint{2.161794in}{0.394924in}}%
\pgfpathlineto{\pgfqpoint{2.260730in}{0.394924in}}%
\pgfpathlineto{\pgfqpoint{2.260730in}{0.295988in}}%
\pgfpathlineto{\pgfqpoint{2.161794in}{0.295988in}}%
\pgfpathlineto{\pgfqpoint{2.161794in}{0.394924in}}%
\pgfusepath{stroke,fill}%
\end{pgfscope}%
\begin{pgfscope}%
\pgfpathrectangle{\pgfqpoint{0.380943in}{0.295988in}}{\pgfqpoint{4.650000in}{0.692553in}}%
\pgfusepath{clip}%
\pgfsetbuttcap%
\pgfsetroundjoin%
\definecolor{currentfill}{rgb}{0.961230,0.930657,0.727628}%
\pgfsetfillcolor{currentfill}%
\pgfsetlinewidth{0.250937pt}%
\definecolor{currentstroke}{rgb}{1.000000,1.000000,1.000000}%
\pgfsetstrokecolor{currentstroke}%
\pgfsetdash{}{0pt}%
\pgfpathmoveto{\pgfqpoint{2.260730in}{0.394924in}}%
\pgfpathlineto{\pgfqpoint{2.359666in}{0.394924in}}%
\pgfpathlineto{\pgfqpoint{2.359666in}{0.295988in}}%
\pgfpathlineto{\pgfqpoint{2.260730in}{0.295988in}}%
\pgfpathlineto{\pgfqpoint{2.260730in}{0.394924in}}%
\pgfusepath{stroke,fill}%
\end{pgfscope}%
\begin{pgfscope}%
\pgfpathrectangle{\pgfqpoint{0.380943in}{0.295988in}}{\pgfqpoint{4.650000in}{0.692553in}}%
\pgfusepath{clip}%
\pgfsetbuttcap%
\pgfsetroundjoin%
\definecolor{currentfill}{rgb}{0.983391,0.971765,0.788189}%
\pgfsetfillcolor{currentfill}%
\pgfsetlinewidth{0.250937pt}%
\definecolor{currentstroke}{rgb}{1.000000,1.000000,1.000000}%
\pgfsetstrokecolor{currentstroke}%
\pgfsetdash{}{0pt}%
\pgfpathmoveto{\pgfqpoint{2.359666in}{0.394924in}}%
\pgfpathlineto{\pgfqpoint{2.458603in}{0.394924in}}%
\pgfpathlineto{\pgfqpoint{2.458603in}{0.295988in}}%
\pgfpathlineto{\pgfqpoint{2.359666in}{0.295988in}}%
\pgfpathlineto{\pgfqpoint{2.359666in}{0.394924in}}%
\pgfusepath{stroke,fill}%
\end{pgfscope}%
\begin{pgfscope}%
\pgfpathrectangle{\pgfqpoint{0.380943in}{0.295988in}}{\pgfqpoint{4.650000in}{0.692553in}}%
\pgfusepath{clip}%
\pgfsetbuttcap%
\pgfsetroundjoin%
\definecolor{currentfill}{rgb}{0.973241,0.954510,0.761799}%
\pgfsetfillcolor{currentfill}%
\pgfsetlinewidth{0.250937pt}%
\definecolor{currentstroke}{rgb}{1.000000,1.000000,1.000000}%
\pgfsetstrokecolor{currentstroke}%
\pgfsetdash{}{0pt}%
\pgfpathmoveto{\pgfqpoint{2.458603in}{0.394924in}}%
\pgfpathlineto{\pgfqpoint{2.557539in}{0.394924in}}%
\pgfpathlineto{\pgfqpoint{2.557539in}{0.295988in}}%
\pgfpathlineto{\pgfqpoint{2.458603in}{0.295988in}}%
\pgfpathlineto{\pgfqpoint{2.458603in}{0.394924in}}%
\pgfusepath{stroke,fill}%
\end{pgfscope}%
\begin{pgfscope}%
\pgfpathrectangle{\pgfqpoint{0.380943in}{0.295988in}}{\pgfqpoint{4.650000in}{0.692553in}}%
\pgfusepath{clip}%
\pgfsetbuttcap%
\pgfsetroundjoin%
\definecolor{currentfill}{rgb}{0.962584,0.922537,0.722215}%
\pgfsetfillcolor{currentfill}%
\pgfsetlinewidth{0.250937pt}%
\definecolor{currentstroke}{rgb}{1.000000,1.000000,1.000000}%
\pgfsetstrokecolor{currentstroke}%
\pgfsetdash{}{0pt}%
\pgfpathmoveto{\pgfqpoint{2.557539in}{0.394924in}}%
\pgfpathlineto{\pgfqpoint{2.656475in}{0.394924in}}%
\pgfpathlineto{\pgfqpoint{2.656475in}{0.295988in}}%
\pgfpathlineto{\pgfqpoint{2.557539in}{0.295988in}}%
\pgfpathlineto{\pgfqpoint{2.557539in}{0.394924in}}%
\pgfusepath{stroke,fill}%
\end{pgfscope}%
\begin{pgfscope}%
\pgfpathrectangle{\pgfqpoint{0.380943in}{0.295988in}}{\pgfqpoint{4.650000in}{0.692553in}}%
\pgfusepath{clip}%
\pgfsetbuttcap%
\pgfsetroundjoin%
\definecolor{currentfill}{rgb}{0.973241,0.954510,0.761799}%
\pgfsetfillcolor{currentfill}%
\pgfsetlinewidth{0.250937pt}%
\definecolor{currentstroke}{rgb}{1.000000,1.000000,1.000000}%
\pgfsetstrokecolor{currentstroke}%
\pgfsetdash{}{0pt}%
\pgfpathmoveto{\pgfqpoint{2.656475in}{0.394924in}}%
\pgfpathlineto{\pgfqpoint{2.755411in}{0.394924in}}%
\pgfpathlineto{\pgfqpoint{2.755411in}{0.295988in}}%
\pgfpathlineto{\pgfqpoint{2.656475in}{0.295988in}}%
\pgfpathlineto{\pgfqpoint{2.656475in}{0.394924in}}%
\pgfusepath{stroke,fill}%
\end{pgfscope}%
\begin{pgfscope}%
\pgfpathrectangle{\pgfqpoint{0.380943in}{0.295988in}}{\pgfqpoint{4.650000in}{0.692553in}}%
\pgfusepath{clip}%
\pgfsetbuttcap%
\pgfsetroundjoin%
\definecolor{currentfill}{rgb}{0.978316,0.963137,0.774994}%
\pgfsetfillcolor{currentfill}%
\pgfsetlinewidth{0.250937pt}%
\definecolor{currentstroke}{rgb}{1.000000,1.000000,1.000000}%
\pgfsetstrokecolor{currentstroke}%
\pgfsetdash{}{0pt}%
\pgfpathmoveto{\pgfqpoint{2.755411in}{0.394924in}}%
\pgfpathlineto{\pgfqpoint{2.854347in}{0.394924in}}%
\pgfpathlineto{\pgfqpoint{2.854347in}{0.295988in}}%
\pgfpathlineto{\pgfqpoint{2.755411in}{0.295988in}}%
\pgfpathlineto{\pgfqpoint{2.755411in}{0.394924in}}%
\pgfusepath{stroke,fill}%
\end{pgfscope}%
\begin{pgfscope}%
\pgfpathrectangle{\pgfqpoint{0.380943in}{0.295988in}}{\pgfqpoint{4.650000in}{0.692553in}}%
\pgfusepath{clip}%
\pgfsetbuttcap%
\pgfsetroundjoin%
\definecolor{currentfill}{rgb}{1.000000,1.000000,0.832141}%
\pgfsetfillcolor{currentfill}%
\pgfsetlinewidth{0.250937pt}%
\definecolor{currentstroke}{rgb}{1.000000,1.000000,1.000000}%
\pgfsetstrokecolor{currentstroke}%
\pgfsetdash{}{0pt}%
\pgfpathmoveto{\pgfqpoint{2.854347in}{0.394924in}}%
\pgfpathlineto{\pgfqpoint{2.953283in}{0.394924in}}%
\pgfpathlineto{\pgfqpoint{2.953283in}{0.295988in}}%
\pgfpathlineto{\pgfqpoint{2.854347in}{0.295988in}}%
\pgfpathlineto{\pgfqpoint{2.854347in}{0.394924in}}%
\pgfusepath{stroke,fill}%
\end{pgfscope}%
\begin{pgfscope}%
\pgfpathrectangle{\pgfqpoint{0.380943in}{0.295988in}}{\pgfqpoint{4.650000in}{0.692553in}}%
\pgfusepath{clip}%
\pgfsetbuttcap%
\pgfsetroundjoin%
\definecolor{currentfill}{rgb}{0.966459,0.901038,0.709050}%
\pgfsetfillcolor{currentfill}%
\pgfsetlinewidth{0.250937pt}%
\definecolor{currentstroke}{rgb}{1.000000,1.000000,1.000000}%
\pgfsetstrokecolor{currentstroke}%
\pgfsetdash{}{0pt}%
\pgfpathmoveto{\pgfqpoint{2.953283in}{0.394924in}}%
\pgfpathlineto{\pgfqpoint{3.052220in}{0.394924in}}%
\pgfpathlineto{\pgfqpoint{3.052220in}{0.295988in}}%
\pgfpathlineto{\pgfqpoint{2.953283in}{0.295988in}}%
\pgfpathlineto{\pgfqpoint{2.953283in}{0.394924in}}%
\pgfusepath{stroke,fill}%
\end{pgfscope}%
\begin{pgfscope}%
\pgfpathrectangle{\pgfqpoint{0.380943in}{0.295988in}}{\pgfqpoint{4.650000in}{0.692553in}}%
\pgfusepath{clip}%
\pgfsetbuttcap%
\pgfsetroundjoin%
\definecolor{currentfill}{rgb}{0.964783,0.940131,0.739808}%
\pgfsetfillcolor{currentfill}%
\pgfsetlinewidth{0.250937pt}%
\definecolor{currentstroke}{rgb}{1.000000,1.000000,1.000000}%
\pgfsetstrokecolor{currentstroke}%
\pgfsetdash{}{0pt}%
\pgfpathmoveto{\pgfqpoint{3.052220in}{0.394924in}}%
\pgfpathlineto{\pgfqpoint{3.151156in}{0.394924in}}%
\pgfpathlineto{\pgfqpoint{3.151156in}{0.295988in}}%
\pgfpathlineto{\pgfqpoint{3.052220in}{0.295988in}}%
\pgfpathlineto{\pgfqpoint{3.052220in}{0.394924in}}%
\pgfusepath{stroke,fill}%
\end{pgfscope}%
\begin{pgfscope}%
\pgfpathrectangle{\pgfqpoint{0.380943in}{0.295988in}}{\pgfqpoint{4.650000in}{0.692553in}}%
\pgfusepath{clip}%
\pgfsetbuttcap%
\pgfsetroundjoin%
\definecolor{currentfill}{rgb}{0.967474,0.895963,0.706344}%
\pgfsetfillcolor{currentfill}%
\pgfsetlinewidth{0.250937pt}%
\definecolor{currentstroke}{rgb}{1.000000,1.000000,1.000000}%
\pgfsetstrokecolor{currentstroke}%
\pgfsetdash{}{0pt}%
\pgfpathmoveto{\pgfqpoint{3.151156in}{0.394924in}}%
\pgfpathlineto{\pgfqpoint{3.250092in}{0.394924in}}%
\pgfpathlineto{\pgfqpoint{3.250092in}{0.295988in}}%
\pgfpathlineto{\pgfqpoint{3.151156in}{0.295988in}}%
\pgfpathlineto{\pgfqpoint{3.151156in}{0.394924in}}%
\pgfusepath{stroke,fill}%
\end{pgfscope}%
\begin{pgfscope}%
\pgfpathrectangle{\pgfqpoint{0.380943in}{0.295988in}}{\pgfqpoint{4.650000in}{0.692553in}}%
\pgfusepath{clip}%
\pgfsetbuttcap%
\pgfsetroundjoin%
\definecolor{currentfill}{rgb}{0.974072,0.862976,0.688750}%
\pgfsetfillcolor{currentfill}%
\pgfsetlinewidth{0.250937pt}%
\definecolor{currentstroke}{rgb}{1.000000,1.000000,1.000000}%
\pgfsetstrokecolor{currentstroke}%
\pgfsetdash{}{0pt}%
\pgfpathmoveto{\pgfqpoint{3.250092in}{0.394924in}}%
\pgfpathlineto{\pgfqpoint{3.349028in}{0.394924in}}%
\pgfpathlineto{\pgfqpoint{3.349028in}{0.295988in}}%
\pgfpathlineto{\pgfqpoint{3.250092in}{0.295988in}}%
\pgfpathlineto{\pgfqpoint{3.250092in}{0.394924in}}%
\pgfusepath{stroke,fill}%
\end{pgfscope}%
\begin{pgfscope}%
\pgfpathrectangle{\pgfqpoint{0.380943in}{0.295988in}}{\pgfqpoint{4.650000in}{0.692553in}}%
\pgfusepath{clip}%
\pgfsetbuttcap%
\pgfsetroundjoin%
\definecolor{currentfill}{rgb}{0.983714,0.819592,0.653379}%
\pgfsetfillcolor{currentfill}%
\pgfsetlinewidth{0.250937pt}%
\definecolor{currentstroke}{rgb}{1.000000,1.000000,1.000000}%
\pgfsetstrokecolor{currentstroke}%
\pgfsetdash{}{0pt}%
\pgfpathmoveto{\pgfqpoint{3.349028in}{0.394924in}}%
\pgfpathlineto{\pgfqpoint{3.447964in}{0.394924in}}%
\pgfpathlineto{\pgfqpoint{3.447964in}{0.295988in}}%
\pgfpathlineto{\pgfqpoint{3.349028in}{0.295988in}}%
\pgfpathlineto{\pgfqpoint{3.349028in}{0.394924in}}%
\pgfusepath{stroke,fill}%
\end{pgfscope}%
\begin{pgfscope}%
\pgfpathrectangle{\pgfqpoint{0.380943in}{0.295988in}}{\pgfqpoint{4.650000in}{0.692553in}}%
\pgfusepath{clip}%
\pgfsetbuttcap%
\pgfsetroundjoin%
\definecolor{currentfill}{rgb}{0.964275,0.912388,0.715448}%
\pgfsetfillcolor{currentfill}%
\pgfsetlinewidth{0.250937pt}%
\definecolor{currentstroke}{rgb}{1.000000,1.000000,1.000000}%
\pgfsetstrokecolor{currentstroke}%
\pgfsetdash{}{0pt}%
\pgfpathmoveto{\pgfqpoint{3.447964in}{0.394924in}}%
\pgfpathlineto{\pgfqpoint{3.546901in}{0.394924in}}%
\pgfpathlineto{\pgfqpoint{3.546901in}{0.295988in}}%
\pgfpathlineto{\pgfqpoint{3.447964in}{0.295988in}}%
\pgfpathlineto{\pgfqpoint{3.447964in}{0.394924in}}%
\pgfusepath{stroke,fill}%
\end{pgfscope}%
\begin{pgfscope}%
\pgfpathrectangle{\pgfqpoint{0.380943in}{0.295988in}}{\pgfqpoint{4.650000in}{0.692553in}}%
\pgfusepath{clip}%
\pgfsetbuttcap%
\pgfsetroundjoin%
\definecolor{currentfill}{rgb}{0.964783,0.940131,0.739808}%
\pgfsetfillcolor{currentfill}%
\pgfsetlinewidth{0.250937pt}%
\definecolor{currentstroke}{rgb}{1.000000,1.000000,1.000000}%
\pgfsetstrokecolor{currentstroke}%
\pgfsetdash{}{0pt}%
\pgfpathmoveto{\pgfqpoint{3.546901in}{0.394924in}}%
\pgfpathlineto{\pgfqpoint{3.645837in}{0.394924in}}%
\pgfpathlineto{\pgfqpoint{3.645837in}{0.295988in}}%
\pgfpathlineto{\pgfqpoint{3.546901in}{0.295988in}}%
\pgfpathlineto{\pgfqpoint{3.546901in}{0.394924in}}%
\pgfusepath{stroke,fill}%
\end{pgfscope}%
\begin{pgfscope}%
\pgfpathrectangle{\pgfqpoint{0.380943in}{0.295988in}}{\pgfqpoint{4.650000in}{0.692553in}}%
\pgfusepath{clip}%
\pgfsetbuttcap%
\pgfsetroundjoin%
\definecolor{currentfill}{rgb}{0.969858,0.948758,0.753003}%
\pgfsetfillcolor{currentfill}%
\pgfsetlinewidth{0.250937pt}%
\definecolor{currentstroke}{rgb}{1.000000,1.000000,1.000000}%
\pgfsetstrokecolor{currentstroke}%
\pgfsetdash{}{0pt}%
\pgfpathmoveto{\pgfqpoint{3.645837in}{0.394924in}}%
\pgfpathlineto{\pgfqpoint{3.744773in}{0.394924in}}%
\pgfpathlineto{\pgfqpoint{3.744773in}{0.295988in}}%
\pgfpathlineto{\pgfqpoint{3.645837in}{0.295988in}}%
\pgfpathlineto{\pgfqpoint{3.645837in}{0.394924in}}%
\pgfusepath{stroke,fill}%
\end{pgfscope}%
\begin{pgfscope}%
\pgfpathrectangle{\pgfqpoint{0.380943in}{0.295988in}}{\pgfqpoint{4.650000in}{0.692553in}}%
\pgfusepath{clip}%
\pgfsetbuttcap%
\pgfsetroundjoin%
\definecolor{currentfill}{rgb}{0.988604,0.796863,0.633449}%
\pgfsetfillcolor{currentfill}%
\pgfsetlinewidth{0.250937pt}%
\definecolor{currentstroke}{rgb}{1.000000,1.000000,1.000000}%
\pgfsetstrokecolor{currentstroke}%
\pgfsetdash{}{0pt}%
\pgfpathmoveto{\pgfqpoint{3.744773in}{0.394924in}}%
\pgfpathlineto{\pgfqpoint{3.843709in}{0.394924in}}%
\pgfpathlineto{\pgfqpoint{3.843709in}{0.295988in}}%
\pgfpathlineto{\pgfqpoint{3.744773in}{0.295988in}}%
\pgfpathlineto{\pgfqpoint{3.744773in}{0.394924in}}%
\pgfusepath{stroke,fill}%
\end{pgfscope}%
\begin{pgfscope}%
\pgfpathrectangle{\pgfqpoint{0.380943in}{0.295988in}}{\pgfqpoint{4.650000in}{0.692553in}}%
\pgfusepath{clip}%
\pgfsetbuttcap%
\pgfsetroundjoin%
\definecolor{currentfill}{rgb}{0.980669,0.832787,0.665559}%
\pgfsetfillcolor{currentfill}%
\pgfsetlinewidth{0.250937pt}%
\definecolor{currentstroke}{rgb}{1.000000,1.000000,1.000000}%
\pgfsetstrokecolor{currentstroke}%
\pgfsetdash{}{0pt}%
\pgfpathmoveto{\pgfqpoint{3.843709in}{0.394924in}}%
\pgfpathlineto{\pgfqpoint{3.942645in}{0.394924in}}%
\pgfpathlineto{\pgfqpoint{3.942645in}{0.295988in}}%
\pgfpathlineto{\pgfqpoint{3.843709in}{0.295988in}}%
\pgfpathlineto{\pgfqpoint{3.843709in}{0.394924in}}%
\pgfusepath{stroke,fill}%
\end{pgfscope}%
\begin{pgfscope}%
\pgfpathrectangle{\pgfqpoint{0.380943in}{0.295988in}}{\pgfqpoint{4.650000in}{0.692553in}}%
\pgfusepath{clip}%
\pgfsetbuttcap%
\pgfsetroundjoin%
\definecolor{currentfill}{rgb}{0.979654,0.837186,0.669619}%
\pgfsetfillcolor{currentfill}%
\pgfsetlinewidth{0.250937pt}%
\definecolor{currentstroke}{rgb}{1.000000,1.000000,1.000000}%
\pgfsetstrokecolor{currentstroke}%
\pgfsetdash{}{0pt}%
\pgfpathmoveto{\pgfqpoint{3.942645in}{0.394924in}}%
\pgfpathlineto{\pgfqpoint{4.041581in}{0.394924in}}%
\pgfpathlineto{\pgfqpoint{4.041581in}{0.295988in}}%
\pgfpathlineto{\pgfqpoint{3.942645in}{0.295988in}}%
\pgfpathlineto{\pgfqpoint{3.942645in}{0.394924in}}%
\pgfusepath{stroke,fill}%
\end{pgfscope}%
\begin{pgfscope}%
\pgfpathrectangle{\pgfqpoint{0.380943in}{0.295988in}}{\pgfqpoint{4.650000in}{0.692553in}}%
\pgfusepath{clip}%
\pgfsetbuttcap%
\pgfsetroundjoin%
\definecolor{currentfill}{rgb}{0.963937,0.914418,0.716801}%
\pgfsetfillcolor{currentfill}%
\pgfsetlinewidth{0.250937pt}%
\definecolor{currentstroke}{rgb}{1.000000,1.000000,1.000000}%
\pgfsetstrokecolor{currentstroke}%
\pgfsetdash{}{0pt}%
\pgfpathmoveto{\pgfqpoint{4.041581in}{0.394924in}}%
\pgfpathlineto{\pgfqpoint{4.140518in}{0.394924in}}%
\pgfpathlineto{\pgfqpoint{4.140518in}{0.295988in}}%
\pgfpathlineto{\pgfqpoint{4.041581in}{0.295988in}}%
\pgfpathlineto{\pgfqpoint{4.041581in}{0.394924in}}%
\pgfusepath{stroke,fill}%
\end{pgfscope}%
\begin{pgfscope}%
\pgfpathrectangle{\pgfqpoint{0.380943in}{0.295988in}}{\pgfqpoint{4.650000in}{0.692553in}}%
\pgfusepath{clip}%
\pgfsetbuttcap%
\pgfsetroundjoin%
\definecolor{currentfill}{rgb}{0.963429,0.917463,0.718831}%
\pgfsetfillcolor{currentfill}%
\pgfsetlinewidth{0.250937pt}%
\definecolor{currentstroke}{rgb}{1.000000,1.000000,1.000000}%
\pgfsetstrokecolor{currentstroke}%
\pgfsetdash{}{0pt}%
\pgfpathmoveto{\pgfqpoint{4.140518in}{0.394924in}}%
\pgfpathlineto{\pgfqpoint{4.239454in}{0.394924in}}%
\pgfpathlineto{\pgfqpoint{4.239454in}{0.295988in}}%
\pgfpathlineto{\pgfqpoint{4.140518in}{0.295988in}}%
\pgfpathlineto{\pgfqpoint{4.140518in}{0.394924in}}%
\pgfusepath{stroke,fill}%
\end{pgfscope}%
\begin{pgfscope}%
\pgfpathrectangle{\pgfqpoint{0.380943in}{0.295988in}}{\pgfqpoint{4.650000in}{0.692553in}}%
\pgfusepath{clip}%
\pgfsetbuttcap%
\pgfsetroundjoin%
\definecolor{currentfill}{rgb}{0.978316,0.963137,0.774994}%
\pgfsetfillcolor{currentfill}%
\pgfsetlinewidth{0.250937pt}%
\definecolor{currentstroke}{rgb}{1.000000,1.000000,1.000000}%
\pgfsetstrokecolor{currentstroke}%
\pgfsetdash{}{0pt}%
\pgfpathmoveto{\pgfqpoint{4.239454in}{0.394924in}}%
\pgfpathlineto{\pgfqpoint{4.338390in}{0.394924in}}%
\pgfpathlineto{\pgfqpoint{4.338390in}{0.295988in}}%
\pgfpathlineto{\pgfqpoint{4.239454in}{0.295988in}}%
\pgfpathlineto{\pgfqpoint{4.239454in}{0.394924in}}%
\pgfusepath{stroke,fill}%
\end{pgfscope}%
\begin{pgfscope}%
\pgfpathrectangle{\pgfqpoint{0.380943in}{0.295988in}}{\pgfqpoint{4.650000in}{0.692553in}}%
\pgfusepath{clip}%
\pgfsetbuttcap%
\pgfsetroundjoin%
\definecolor{currentfill}{rgb}{0.991849,0.986144,0.810181}%
\pgfsetfillcolor{currentfill}%
\pgfsetlinewidth{0.250937pt}%
\definecolor{currentstroke}{rgb}{1.000000,1.000000,1.000000}%
\pgfsetstrokecolor{currentstroke}%
\pgfsetdash{}{0pt}%
\pgfpathmoveto{\pgfqpoint{4.338390in}{0.394924in}}%
\pgfpathlineto{\pgfqpoint{4.437326in}{0.394924in}}%
\pgfpathlineto{\pgfqpoint{4.437326in}{0.295988in}}%
\pgfpathlineto{\pgfqpoint{4.338390in}{0.295988in}}%
\pgfpathlineto{\pgfqpoint{4.338390in}{0.394924in}}%
\pgfusepath{stroke,fill}%
\end{pgfscope}%
\begin{pgfscope}%
\pgfpathrectangle{\pgfqpoint{0.380943in}{0.295988in}}{\pgfqpoint{4.650000in}{0.692553in}}%
\pgfusepath{clip}%
\pgfsetbuttcap%
\pgfsetroundjoin%
\definecolor{currentfill}{rgb}{1.000000,1.000000,0.865975}%
\pgfsetfillcolor{currentfill}%
\pgfsetlinewidth{0.250937pt}%
\definecolor{currentstroke}{rgb}{1.000000,1.000000,1.000000}%
\pgfsetstrokecolor{currentstroke}%
\pgfsetdash{}{0pt}%
\pgfpathmoveto{\pgfqpoint{4.437326in}{0.394924in}}%
\pgfpathlineto{\pgfqpoint{4.536262in}{0.394924in}}%
\pgfpathlineto{\pgfqpoint{4.536262in}{0.295988in}}%
\pgfpathlineto{\pgfqpoint{4.437326in}{0.295988in}}%
\pgfpathlineto{\pgfqpoint{4.437326in}{0.394924in}}%
\pgfusepath{stroke,fill}%
\end{pgfscope}%
\begin{pgfscope}%
\pgfpathrectangle{\pgfqpoint{0.380943in}{0.295988in}}{\pgfqpoint{4.650000in}{0.692553in}}%
\pgfusepath{clip}%
\pgfsetbuttcap%
\pgfsetroundjoin%
\definecolor{currentfill}{rgb}{1.000000,1.000000,0.844829}%
\pgfsetfillcolor{currentfill}%
\pgfsetlinewidth{0.250937pt}%
\definecolor{currentstroke}{rgb}{1.000000,1.000000,1.000000}%
\pgfsetstrokecolor{currentstroke}%
\pgfsetdash{}{0pt}%
\pgfpathmoveto{\pgfqpoint{4.536262in}{0.394924in}}%
\pgfpathlineto{\pgfqpoint{4.635198in}{0.394924in}}%
\pgfpathlineto{\pgfqpoint{4.635198in}{0.295988in}}%
\pgfpathlineto{\pgfqpoint{4.536262in}{0.295988in}}%
\pgfpathlineto{\pgfqpoint{4.536262in}{0.394924in}}%
\pgfusepath{stroke,fill}%
\end{pgfscope}%
\begin{pgfscope}%
\pgfpathrectangle{\pgfqpoint{0.380943in}{0.295988in}}{\pgfqpoint{4.650000in}{0.692553in}}%
\pgfusepath{clip}%
\pgfsetbuttcap%
\pgfsetroundjoin%
\definecolor{currentfill}{rgb}{1.000000,1.000000,0.853287}%
\pgfsetfillcolor{currentfill}%
\pgfsetlinewidth{0.250937pt}%
\definecolor{currentstroke}{rgb}{1.000000,1.000000,1.000000}%
\pgfsetstrokecolor{currentstroke}%
\pgfsetdash{}{0pt}%
\pgfpathmoveto{\pgfqpoint{4.635198in}{0.394924in}}%
\pgfpathlineto{\pgfqpoint{4.734135in}{0.394924in}}%
\pgfpathlineto{\pgfqpoint{4.734135in}{0.295988in}}%
\pgfpathlineto{\pgfqpoint{4.635198in}{0.295988in}}%
\pgfpathlineto{\pgfqpoint{4.635198in}{0.394924in}}%
\pgfusepath{stroke,fill}%
\end{pgfscope}%
\begin{pgfscope}%
\pgfpathrectangle{\pgfqpoint{0.380943in}{0.295988in}}{\pgfqpoint{4.650000in}{0.692553in}}%
\pgfusepath{clip}%
\pgfsetbuttcap%
\pgfsetroundjoin%
\definecolor{currentfill}{rgb}{1.000000,1.000000,0.920953}%
\pgfsetfillcolor{currentfill}%
\pgfsetlinewidth{0.250937pt}%
\definecolor{currentstroke}{rgb}{1.000000,1.000000,1.000000}%
\pgfsetstrokecolor{currentstroke}%
\pgfsetdash{}{0pt}%
\pgfpathmoveto{\pgfqpoint{4.734135in}{0.394924in}}%
\pgfpathlineto{\pgfqpoint{4.833071in}{0.394924in}}%
\pgfpathlineto{\pgfqpoint{4.833071in}{0.295988in}}%
\pgfpathlineto{\pgfqpoint{4.734135in}{0.295988in}}%
\pgfpathlineto{\pgfqpoint{4.734135in}{0.394924in}}%
\pgfusepath{stroke,fill}%
\end{pgfscope}%
\begin{pgfscope}%
\pgfpathrectangle{\pgfqpoint{0.380943in}{0.295988in}}{\pgfqpoint{4.650000in}{0.692553in}}%
\pgfusepath{clip}%
\pgfsetbuttcap%
\pgfsetroundjoin%
\definecolor{currentfill}{rgb}{0.991849,0.986144,0.810181}%
\pgfsetfillcolor{currentfill}%
\pgfsetlinewidth{0.250937pt}%
\definecolor{currentstroke}{rgb}{1.000000,1.000000,1.000000}%
\pgfsetstrokecolor{currentstroke}%
\pgfsetdash{}{0pt}%
\pgfpathmoveto{\pgfqpoint{4.833071in}{0.394924in}}%
\pgfpathlineto{\pgfqpoint{4.932007in}{0.394924in}}%
\pgfpathlineto{\pgfqpoint{4.932007in}{0.295988in}}%
\pgfpathlineto{\pgfqpoint{4.833071in}{0.295988in}}%
\pgfpathlineto{\pgfqpoint{4.833071in}{0.394924in}}%
\pgfusepath{stroke,fill}%
\end{pgfscope}%
\begin{pgfscope}%
\pgfpathrectangle{\pgfqpoint{0.380943in}{0.295988in}}{\pgfqpoint{4.650000in}{0.692553in}}%
\pgfusepath{clip}%
\pgfsetbuttcap%
\pgfsetroundjoin%
\pgfsetlinewidth{0.250937pt}%
\definecolor{currentstroke}{rgb}{1.000000,1.000000,1.000000}%
\pgfsetstrokecolor{currentstroke}%
\pgfsetdash{}{0pt}%
\pgfpathmoveto{\pgfqpoint{4.932007in}{0.394924in}}%
\pgfpathlineto{\pgfqpoint{5.030943in}{0.394924in}}%
\pgfpathlineto{\pgfqpoint{5.030943in}{0.295988in}}%
\pgfpathlineto{\pgfqpoint{4.932007in}{0.295988in}}%
\pgfpathlineto{\pgfqpoint{4.932007in}{0.394924in}}%
\pgfusepath{stroke}%
\end{pgfscope}%
\begin{pgfscope}%
\pgfsetbuttcap%
\pgfsetroundjoin%
\definecolor{currentfill}{rgb}{0.000000,0.000000,0.000000}%
\pgfsetfillcolor{currentfill}%
\pgfsetlinewidth{0.803000pt}%
\definecolor{currentstroke}{rgb}{0.000000,0.000000,0.000000}%
\pgfsetstrokecolor{currentstroke}%
\pgfsetdash{}{0pt}%
\pgfsys@defobject{currentmarker}{\pgfqpoint{0.000000in}{-0.048611in}}{\pgfqpoint{0.000000in}{0.000000in}}{%
\pgfpathmoveto{\pgfqpoint{0.000000in}{0.000000in}}%
\pgfpathlineto{\pgfqpoint{0.000000in}{-0.048611in}}%
\pgfusepath{stroke,fill}%
}%
\begin{pgfscope}%
\pgfsys@transformshift{0.628283in}{0.295988in}%
\pgfsys@useobject{currentmarker}{}%
\end{pgfscope}%
\end{pgfscope}%
\begin{pgfscope}%
\definecolor{textcolor}{rgb}{0.000000,0.000000,0.000000}%
\pgfsetstrokecolor{textcolor}%
\pgfsetfillcolor{textcolor}%
\pgftext[x=0.628283in,y=0.198766in,,top]{\color{textcolor}\rmfamily\fontsize{8.000000}{9.600000}\selectfont Jan}%
\end{pgfscope}%
\begin{pgfscope}%
\pgfsetbuttcap%
\pgfsetroundjoin%
\definecolor{currentfill}{rgb}{0.000000,0.000000,0.000000}%
\pgfsetfillcolor{currentfill}%
\pgfsetlinewidth{0.803000pt}%
\definecolor{currentstroke}{rgb}{0.000000,0.000000,0.000000}%
\pgfsetstrokecolor{currentstroke}%
\pgfsetdash{}{0pt}%
\pgfsys@defobject{currentmarker}{\pgfqpoint{0.000000in}{-0.048611in}}{\pgfqpoint{0.000000in}{0.000000in}}{%
\pgfpathmoveto{\pgfqpoint{0.000000in}{0.000000in}}%
\pgfpathlineto{\pgfqpoint{0.000000in}{-0.048611in}}%
\pgfusepath{stroke,fill}%
}%
\begin{pgfscope}%
\pgfsys@transformshift{1.073496in}{0.295988in}%
\pgfsys@useobject{currentmarker}{}%
\end{pgfscope}%
\end{pgfscope}%
\begin{pgfscope}%
\definecolor{textcolor}{rgb}{0.000000,0.000000,0.000000}%
\pgfsetstrokecolor{textcolor}%
\pgfsetfillcolor{textcolor}%
\pgftext[x=1.073496in,y=0.198766in,,top]{\color{textcolor}\rmfamily\fontsize{8.000000}{9.600000}\selectfont Feb}%
\end{pgfscope}%
\begin{pgfscope}%
\pgfsetbuttcap%
\pgfsetroundjoin%
\definecolor{currentfill}{rgb}{0.000000,0.000000,0.000000}%
\pgfsetfillcolor{currentfill}%
\pgfsetlinewidth{0.803000pt}%
\definecolor{currentstroke}{rgb}{0.000000,0.000000,0.000000}%
\pgfsetstrokecolor{currentstroke}%
\pgfsetdash{}{0pt}%
\pgfsys@defobject{currentmarker}{\pgfqpoint{0.000000in}{-0.048611in}}{\pgfqpoint{0.000000in}{0.000000in}}{%
\pgfpathmoveto{\pgfqpoint{0.000000in}{0.000000in}}%
\pgfpathlineto{\pgfqpoint{0.000000in}{-0.048611in}}%
\pgfusepath{stroke,fill}%
}%
\begin{pgfscope}%
\pgfsys@transformshift{1.518709in}{0.295988in}%
\pgfsys@useobject{currentmarker}{}%
\end{pgfscope}%
\end{pgfscope}%
\begin{pgfscope}%
\definecolor{textcolor}{rgb}{0.000000,0.000000,0.000000}%
\pgfsetstrokecolor{textcolor}%
\pgfsetfillcolor{textcolor}%
\pgftext[x=1.518709in,y=0.198766in,,top]{\color{textcolor}\rmfamily\fontsize{8.000000}{9.600000}\selectfont Mar}%
\end{pgfscope}%
\begin{pgfscope}%
\pgfsetbuttcap%
\pgfsetroundjoin%
\definecolor{currentfill}{rgb}{0.000000,0.000000,0.000000}%
\pgfsetfillcolor{currentfill}%
\pgfsetlinewidth{0.803000pt}%
\definecolor{currentstroke}{rgb}{0.000000,0.000000,0.000000}%
\pgfsetstrokecolor{currentstroke}%
\pgfsetdash{}{0pt}%
\pgfsys@defobject{currentmarker}{\pgfqpoint{0.000000in}{-0.048611in}}{\pgfqpoint{0.000000in}{0.000000in}}{%
\pgfpathmoveto{\pgfqpoint{0.000000in}{0.000000in}}%
\pgfpathlineto{\pgfqpoint{0.000000in}{-0.048611in}}%
\pgfusepath{stroke,fill}%
}%
\begin{pgfscope}%
\pgfsys@transformshift{1.914454in}{0.295988in}%
\pgfsys@useobject{currentmarker}{}%
\end{pgfscope}%
\end{pgfscope}%
\begin{pgfscope}%
\definecolor{textcolor}{rgb}{0.000000,0.000000,0.000000}%
\pgfsetstrokecolor{textcolor}%
\pgfsetfillcolor{textcolor}%
\pgftext[x=1.914454in,y=0.198766in,,top]{\color{textcolor}\rmfamily\fontsize{8.000000}{9.600000}\selectfont Apr}%
\end{pgfscope}%
\begin{pgfscope}%
\pgfsetbuttcap%
\pgfsetroundjoin%
\definecolor{currentfill}{rgb}{0.000000,0.000000,0.000000}%
\pgfsetfillcolor{currentfill}%
\pgfsetlinewidth{0.803000pt}%
\definecolor{currentstroke}{rgb}{0.000000,0.000000,0.000000}%
\pgfsetstrokecolor{currentstroke}%
\pgfsetdash{}{0pt}%
\pgfsys@defobject{currentmarker}{\pgfqpoint{0.000000in}{-0.048611in}}{\pgfqpoint{0.000000in}{0.000000in}}{%
\pgfpathmoveto{\pgfqpoint{0.000000in}{0.000000in}}%
\pgfpathlineto{\pgfqpoint{0.000000in}{-0.048611in}}%
\pgfusepath{stroke,fill}%
}%
\begin{pgfscope}%
\pgfsys@transformshift{2.359666in}{0.295988in}%
\pgfsys@useobject{currentmarker}{}%
\end{pgfscope}%
\end{pgfscope}%
\begin{pgfscope}%
\definecolor{textcolor}{rgb}{0.000000,0.000000,0.000000}%
\pgfsetstrokecolor{textcolor}%
\pgfsetfillcolor{textcolor}%
\pgftext[x=2.359666in,y=0.198766in,,top]{\color{textcolor}\rmfamily\fontsize{8.000000}{9.600000}\selectfont May}%
\end{pgfscope}%
\begin{pgfscope}%
\pgfsetbuttcap%
\pgfsetroundjoin%
\definecolor{currentfill}{rgb}{0.000000,0.000000,0.000000}%
\pgfsetfillcolor{currentfill}%
\pgfsetlinewidth{0.803000pt}%
\definecolor{currentstroke}{rgb}{0.000000,0.000000,0.000000}%
\pgfsetstrokecolor{currentstroke}%
\pgfsetdash{}{0pt}%
\pgfsys@defobject{currentmarker}{\pgfqpoint{0.000000in}{-0.048611in}}{\pgfqpoint{0.000000in}{0.000000in}}{%
\pgfpathmoveto{\pgfqpoint{0.000000in}{0.000000in}}%
\pgfpathlineto{\pgfqpoint{0.000000in}{-0.048611in}}%
\pgfusepath{stroke,fill}%
}%
\begin{pgfscope}%
\pgfsys@transformshift{2.804879in}{0.295988in}%
\pgfsys@useobject{currentmarker}{}%
\end{pgfscope}%
\end{pgfscope}%
\begin{pgfscope}%
\definecolor{textcolor}{rgb}{0.000000,0.000000,0.000000}%
\pgfsetstrokecolor{textcolor}%
\pgfsetfillcolor{textcolor}%
\pgftext[x=2.804879in,y=0.198766in,,top]{\color{textcolor}\rmfamily\fontsize{8.000000}{9.600000}\selectfont Jun}%
\end{pgfscope}%
\begin{pgfscope}%
\pgfsetbuttcap%
\pgfsetroundjoin%
\definecolor{currentfill}{rgb}{0.000000,0.000000,0.000000}%
\pgfsetfillcolor{currentfill}%
\pgfsetlinewidth{0.803000pt}%
\definecolor{currentstroke}{rgb}{0.000000,0.000000,0.000000}%
\pgfsetstrokecolor{currentstroke}%
\pgfsetdash{}{0pt}%
\pgfsys@defobject{currentmarker}{\pgfqpoint{0.000000in}{-0.048611in}}{\pgfqpoint{0.000000in}{0.000000in}}{%
\pgfpathmoveto{\pgfqpoint{0.000000in}{0.000000in}}%
\pgfpathlineto{\pgfqpoint{0.000000in}{-0.048611in}}%
\pgfusepath{stroke,fill}%
}%
\begin{pgfscope}%
\pgfsys@transformshift{3.200624in}{0.295988in}%
\pgfsys@useobject{currentmarker}{}%
\end{pgfscope}%
\end{pgfscope}%
\begin{pgfscope}%
\definecolor{textcolor}{rgb}{0.000000,0.000000,0.000000}%
\pgfsetstrokecolor{textcolor}%
\pgfsetfillcolor{textcolor}%
\pgftext[x=3.200624in,y=0.198766in,,top]{\color{textcolor}\rmfamily\fontsize{8.000000}{9.600000}\selectfont Jul}%
\end{pgfscope}%
\begin{pgfscope}%
\pgfsetbuttcap%
\pgfsetroundjoin%
\definecolor{currentfill}{rgb}{0.000000,0.000000,0.000000}%
\pgfsetfillcolor{currentfill}%
\pgfsetlinewidth{0.803000pt}%
\definecolor{currentstroke}{rgb}{0.000000,0.000000,0.000000}%
\pgfsetstrokecolor{currentstroke}%
\pgfsetdash{}{0pt}%
\pgfsys@defobject{currentmarker}{\pgfqpoint{0.000000in}{-0.048611in}}{\pgfqpoint{0.000000in}{0.000000in}}{%
\pgfpathmoveto{\pgfqpoint{0.000000in}{0.000000in}}%
\pgfpathlineto{\pgfqpoint{0.000000in}{-0.048611in}}%
\pgfusepath{stroke,fill}%
}%
\begin{pgfscope}%
\pgfsys@transformshift{3.645837in}{0.295988in}%
\pgfsys@useobject{currentmarker}{}%
\end{pgfscope}%
\end{pgfscope}%
\begin{pgfscope}%
\definecolor{textcolor}{rgb}{0.000000,0.000000,0.000000}%
\pgfsetstrokecolor{textcolor}%
\pgfsetfillcolor{textcolor}%
\pgftext[x=3.645837in,y=0.198766in,,top]{\color{textcolor}\rmfamily\fontsize{8.000000}{9.600000}\selectfont Aug}%
\end{pgfscope}%
\begin{pgfscope}%
\pgfsetbuttcap%
\pgfsetroundjoin%
\definecolor{currentfill}{rgb}{0.000000,0.000000,0.000000}%
\pgfsetfillcolor{currentfill}%
\pgfsetlinewidth{0.803000pt}%
\definecolor{currentstroke}{rgb}{0.000000,0.000000,0.000000}%
\pgfsetstrokecolor{currentstroke}%
\pgfsetdash{}{0pt}%
\pgfsys@defobject{currentmarker}{\pgfqpoint{0.000000in}{-0.048611in}}{\pgfqpoint{0.000000in}{0.000000in}}{%
\pgfpathmoveto{\pgfqpoint{0.000000in}{0.000000in}}%
\pgfpathlineto{\pgfqpoint{0.000000in}{-0.048611in}}%
\pgfusepath{stroke,fill}%
}%
\begin{pgfscope}%
\pgfsys@transformshift{4.091049in}{0.295988in}%
\pgfsys@useobject{currentmarker}{}%
\end{pgfscope}%
\end{pgfscope}%
\begin{pgfscope}%
\definecolor{textcolor}{rgb}{0.000000,0.000000,0.000000}%
\pgfsetstrokecolor{textcolor}%
\pgfsetfillcolor{textcolor}%
\pgftext[x=4.091049in,y=0.198766in,,top]{\color{textcolor}\rmfamily\fontsize{8.000000}{9.600000}\selectfont Sep}%
\end{pgfscope}%
\begin{pgfscope}%
\pgfsetbuttcap%
\pgfsetroundjoin%
\definecolor{currentfill}{rgb}{0.000000,0.000000,0.000000}%
\pgfsetfillcolor{currentfill}%
\pgfsetlinewidth{0.803000pt}%
\definecolor{currentstroke}{rgb}{0.000000,0.000000,0.000000}%
\pgfsetstrokecolor{currentstroke}%
\pgfsetdash{}{0pt}%
\pgfsys@defobject{currentmarker}{\pgfqpoint{0.000000in}{-0.048611in}}{\pgfqpoint{0.000000in}{0.000000in}}{%
\pgfpathmoveto{\pgfqpoint{0.000000in}{0.000000in}}%
\pgfpathlineto{\pgfqpoint{0.000000in}{-0.048611in}}%
\pgfusepath{stroke,fill}%
}%
\begin{pgfscope}%
\pgfsys@transformshift{4.486794in}{0.295988in}%
\pgfsys@useobject{currentmarker}{}%
\end{pgfscope}%
\end{pgfscope}%
\begin{pgfscope}%
\definecolor{textcolor}{rgb}{0.000000,0.000000,0.000000}%
\pgfsetstrokecolor{textcolor}%
\pgfsetfillcolor{textcolor}%
\pgftext[x=4.486794in,y=0.198766in,,top]{\color{textcolor}\rmfamily\fontsize{8.000000}{9.600000}\selectfont Oct}%
\end{pgfscope}%
\begin{pgfscope}%
\pgfsetbuttcap%
\pgfsetroundjoin%
\definecolor{currentfill}{rgb}{0.000000,0.000000,0.000000}%
\pgfsetfillcolor{currentfill}%
\pgfsetlinewidth{0.803000pt}%
\definecolor{currentstroke}{rgb}{0.000000,0.000000,0.000000}%
\pgfsetstrokecolor{currentstroke}%
\pgfsetdash{}{0pt}%
\pgfsys@defobject{currentmarker}{\pgfqpoint{0.000000in}{-0.048611in}}{\pgfqpoint{0.000000in}{0.000000in}}{%
\pgfpathmoveto{\pgfqpoint{0.000000in}{0.000000in}}%
\pgfpathlineto{\pgfqpoint{0.000000in}{-0.048611in}}%
\pgfusepath{stroke,fill}%
}%
\begin{pgfscope}%
\pgfsys@transformshift{4.882539in}{0.295988in}%
\pgfsys@useobject{currentmarker}{}%
\end{pgfscope}%
\end{pgfscope}%
\begin{pgfscope}%
\definecolor{textcolor}{rgb}{0.000000,0.000000,0.000000}%
\pgfsetstrokecolor{textcolor}%
\pgfsetfillcolor{textcolor}%
\pgftext[x=4.882539in,y=0.198766in,,top]{\color{textcolor}\rmfamily\fontsize{8.000000}{9.600000}\selectfont Nov}%
\end{pgfscope}%
\begin{pgfscope}%
\pgfsetbuttcap%
\pgfsetroundjoin%
\definecolor{currentfill}{rgb}{0.000000,0.000000,0.000000}%
\pgfsetfillcolor{currentfill}%
\pgfsetlinewidth{0.803000pt}%
\definecolor{currentstroke}{rgb}{0.000000,0.000000,0.000000}%
\pgfsetstrokecolor{currentstroke}%
\pgfsetdash{}{0pt}%
\pgfsys@defobject{currentmarker}{\pgfqpoint{-0.048611in}{0.000000in}}{\pgfqpoint{-0.000000in}{0.000000in}}{%
\pgfpathmoveto{\pgfqpoint{-0.000000in}{0.000000in}}%
\pgfpathlineto{\pgfqpoint{-0.048611in}{0.000000in}}%
\pgfusepath{stroke,fill}%
}%
\begin{pgfscope}%
\pgfsys@transformshift{0.380943in}{0.939073in}%
\pgfsys@useobject{currentmarker}{}%
\end{pgfscope}%
\end{pgfscope}%
\begin{pgfscope}%
\definecolor{textcolor}{rgb}{0.000000,0.000000,0.000000}%
\pgfsetstrokecolor{textcolor}%
\pgfsetfillcolor{textcolor}%
\pgftext[x=0.113117in, y=0.900493in, left, base]{\color{textcolor}\rmfamily\fontsize{8.000000}{9.600000}\selectfont M}%
\end{pgfscope}%
\begin{pgfscope}%
\pgfsetbuttcap%
\pgfsetroundjoin%
\definecolor{currentfill}{rgb}{0.000000,0.000000,0.000000}%
\pgfsetfillcolor{currentfill}%
\pgfsetlinewidth{0.803000pt}%
\definecolor{currentstroke}{rgb}{0.000000,0.000000,0.000000}%
\pgfsetstrokecolor{currentstroke}%
\pgfsetdash{}{0pt}%
\pgfsys@defobject{currentmarker}{\pgfqpoint{-0.048611in}{0.000000in}}{\pgfqpoint{-0.000000in}{0.000000in}}{%
\pgfpathmoveto{\pgfqpoint{-0.000000in}{0.000000in}}%
\pgfpathlineto{\pgfqpoint{-0.048611in}{0.000000in}}%
\pgfusepath{stroke,fill}%
}%
\begin{pgfscope}%
\pgfsys@transformshift{0.380943in}{0.840137in}%
\pgfsys@useobject{currentmarker}{}%
\end{pgfscope}%
\end{pgfscope}%
\begin{pgfscope}%
\definecolor{textcolor}{rgb}{0.000000,0.000000,0.000000}%
\pgfsetstrokecolor{textcolor}%
\pgfsetfillcolor{textcolor}%
\pgftext[x=0.135957in, y=0.801556in, left, base]{\color{textcolor}\rmfamily\fontsize{8.000000}{9.600000}\selectfont T}%
\end{pgfscope}%
\begin{pgfscope}%
\pgfsetbuttcap%
\pgfsetroundjoin%
\definecolor{currentfill}{rgb}{0.000000,0.000000,0.000000}%
\pgfsetfillcolor{currentfill}%
\pgfsetlinewidth{0.803000pt}%
\definecolor{currentstroke}{rgb}{0.000000,0.000000,0.000000}%
\pgfsetstrokecolor{currentstroke}%
\pgfsetdash{}{0pt}%
\pgfsys@defobject{currentmarker}{\pgfqpoint{-0.048611in}{0.000000in}}{\pgfqpoint{-0.000000in}{0.000000in}}{%
\pgfpathmoveto{\pgfqpoint{-0.000000in}{0.000000in}}%
\pgfpathlineto{\pgfqpoint{-0.048611in}{0.000000in}}%
\pgfusepath{stroke,fill}%
}%
\begin{pgfscope}%
\pgfsys@transformshift{0.380943in}{0.741201in}%
\pgfsys@useobject{currentmarker}{}%
\end{pgfscope}%
\end{pgfscope}%
\begin{pgfscope}%
\definecolor{textcolor}{rgb}{0.000000,0.000000,0.000000}%
\pgfsetstrokecolor{textcolor}%
\pgfsetfillcolor{textcolor}%
\pgftext[x=0.100000in, y=0.702620in, left, base]{\color{textcolor}\rmfamily\fontsize{8.000000}{9.600000}\selectfont W}%
\end{pgfscope}%
\begin{pgfscope}%
\pgfsetbuttcap%
\pgfsetroundjoin%
\definecolor{currentfill}{rgb}{0.000000,0.000000,0.000000}%
\pgfsetfillcolor{currentfill}%
\pgfsetlinewidth{0.803000pt}%
\definecolor{currentstroke}{rgb}{0.000000,0.000000,0.000000}%
\pgfsetstrokecolor{currentstroke}%
\pgfsetdash{}{0pt}%
\pgfsys@defobject{currentmarker}{\pgfqpoint{-0.048611in}{0.000000in}}{\pgfqpoint{-0.000000in}{0.000000in}}{%
\pgfpathmoveto{\pgfqpoint{-0.000000in}{0.000000in}}%
\pgfpathlineto{\pgfqpoint{-0.048611in}{0.000000in}}%
\pgfusepath{stroke,fill}%
}%
\begin{pgfscope}%
\pgfsys@transformshift{0.380943in}{0.642264in}%
\pgfsys@useobject{currentmarker}{}%
\end{pgfscope}%
\end{pgfscope}%
\begin{pgfscope}%
\definecolor{textcolor}{rgb}{0.000000,0.000000,0.000000}%
\pgfsetstrokecolor{textcolor}%
\pgfsetfillcolor{textcolor}%
\pgftext[x=0.135957in, y=0.603684in, left, base]{\color{textcolor}\rmfamily\fontsize{8.000000}{9.600000}\selectfont T}%
\end{pgfscope}%
\begin{pgfscope}%
\pgfsetbuttcap%
\pgfsetroundjoin%
\definecolor{currentfill}{rgb}{0.000000,0.000000,0.000000}%
\pgfsetfillcolor{currentfill}%
\pgfsetlinewidth{0.803000pt}%
\definecolor{currentstroke}{rgb}{0.000000,0.000000,0.000000}%
\pgfsetstrokecolor{currentstroke}%
\pgfsetdash{}{0pt}%
\pgfsys@defobject{currentmarker}{\pgfqpoint{-0.048611in}{0.000000in}}{\pgfqpoint{-0.000000in}{0.000000in}}{%
\pgfpathmoveto{\pgfqpoint{-0.000000in}{0.000000in}}%
\pgfpathlineto{\pgfqpoint{-0.048611in}{0.000000in}}%
\pgfusepath{stroke,fill}%
}%
\begin{pgfscope}%
\pgfsys@transformshift{0.380943in}{0.543328in}%
\pgfsys@useobject{currentmarker}{}%
\end{pgfscope}%
\end{pgfscope}%
\begin{pgfscope}%
\definecolor{textcolor}{rgb}{0.000000,0.000000,0.000000}%
\pgfsetstrokecolor{textcolor}%
\pgfsetfillcolor{textcolor}%
\pgftext[x=0.144213in, y=0.504748in, left, base]{\color{textcolor}\rmfamily\fontsize{8.000000}{9.600000}\selectfont F}%
\end{pgfscope}%
\begin{pgfscope}%
\pgfsetbuttcap%
\pgfsetroundjoin%
\definecolor{currentfill}{rgb}{0.000000,0.000000,0.000000}%
\pgfsetfillcolor{currentfill}%
\pgfsetlinewidth{0.803000pt}%
\definecolor{currentstroke}{rgb}{0.000000,0.000000,0.000000}%
\pgfsetstrokecolor{currentstroke}%
\pgfsetdash{}{0pt}%
\pgfsys@defobject{currentmarker}{\pgfqpoint{-0.048611in}{0.000000in}}{\pgfqpoint{-0.000000in}{0.000000in}}{%
\pgfpathmoveto{\pgfqpoint{-0.000000in}{0.000000in}}%
\pgfpathlineto{\pgfqpoint{-0.048611in}{0.000000in}}%
\pgfusepath{stroke,fill}%
}%
\begin{pgfscope}%
\pgfsys@transformshift{0.380943in}{0.444392in}%
\pgfsys@useobject{currentmarker}{}%
\end{pgfscope}%
\end{pgfscope}%
\begin{pgfscope}%
\definecolor{textcolor}{rgb}{0.000000,0.000000,0.000000}%
\pgfsetstrokecolor{textcolor}%
\pgfsetfillcolor{textcolor}%
\pgftext[x=0.155633in, y=0.405812in, left, base]{\color{textcolor}\rmfamily\fontsize{8.000000}{9.600000}\selectfont S}%
\end{pgfscope}%
\begin{pgfscope}%
\pgfsetbuttcap%
\pgfsetroundjoin%
\definecolor{currentfill}{rgb}{0.000000,0.000000,0.000000}%
\pgfsetfillcolor{currentfill}%
\pgfsetlinewidth{0.803000pt}%
\definecolor{currentstroke}{rgb}{0.000000,0.000000,0.000000}%
\pgfsetstrokecolor{currentstroke}%
\pgfsetdash{}{0pt}%
\pgfsys@defobject{currentmarker}{\pgfqpoint{-0.048611in}{0.000000in}}{\pgfqpoint{-0.000000in}{0.000000in}}{%
\pgfpathmoveto{\pgfqpoint{-0.000000in}{0.000000in}}%
\pgfpathlineto{\pgfqpoint{-0.048611in}{0.000000in}}%
\pgfusepath{stroke,fill}%
}%
\begin{pgfscope}%
\pgfsys@transformshift{0.380943in}{0.345456in}%
\pgfsys@useobject{currentmarker}{}%
\end{pgfscope}%
\end{pgfscope}%
\begin{pgfscope}%
\definecolor{textcolor}{rgb}{0.000000,0.000000,0.000000}%
\pgfsetstrokecolor{textcolor}%
\pgfsetfillcolor{textcolor}%
\pgftext[x=0.155633in, y=0.306876in, left, base]{\color{textcolor}\rmfamily\fontsize{8.000000}{9.600000}\selectfont S}%
\end{pgfscope}%
\begin{pgfscope}%
\definecolor{textcolor}{rgb}{0.000000,0.000000,0.000000}%
\pgfsetstrokecolor{textcolor}%
\pgfsetfillcolor{textcolor}%
\pgftext[x=2.705943in,y=1.155208in,,]{\color{textcolor}\ttfamily\fontsize{14.400000}{17.280000}\selectfont 2021}%
\end{pgfscope}%
\end{pgfpicture}%
\makeatother%
\endgroup%

    \caption{Number of review per day}
    \label{fig:count_calendar}
\end{figure}
\restoregeometry

\section{Unsupervised}

We chose to use the Latent Dirichlet Allocation model to extract the main topics of the \lstinline{avis} column.
We trained 6 models in totals, one on all the \lstinline{avis} concatenated, and one for each \lstinline{avis} concatenated by their respective \lstinline{note}.

\subsection{Data processing}

We apply the following processing for the LDA models:
\begin{itemize}
    \item Word-tokenize the \lstinline{avis} using \lstinline{spacy}
    \item Remove punctuation, stop words and words under 3 characters
    \item Strip and lower the words
    \item Lemmatize
    \item Stemmatize
\end{itemize}

This processing applied to the first \lstinline{avis} gives us the following results:

\begin{table}[H]
\centering
\begin{tabular}{|p{7cm}|p{7cm}|}
\hline
\textbf{Original} &
  \textbf{Processed} \\ \hline
Meilleurs assurances, prix, solutions, écoute, rapidité, et je recommande cette compagnie pour vous \textbackslash{}r\textbackslash{}nDes prix attractif et services de qualité et rapidité  &
  meilleur assur prix solut écout rapid recommand compagn prix attract servic qualit rapid \\ \hline
\end{tabular}
\end{table}

\newpage
\subsection{Model}

The \cref{tab:unsupervised} shows the result for the LDA model trained on all the \lstinline{avis} and the \cref{tab:unsupervised_split} shows the results for the LDA models trained on the \lstinline{avis} grouped by their \lstinline{note}.

\begin{table}[H]
\centering
\begin{tabular}{|p{1cm}|p{13cm}|}
\hline
 \textbf{Topic} &                                                                  \textbf{Words} \\
\hline
     0 & [assur, servic, prix, contrat, être, bien, client, demand, fair, mois] \\
\hline
\end{tabular}
\label{tab:unsupervised}
\caption{LDA result}
\end{table}

\begin{table}[H]
\centering
\begin{tabular}{|p{1cm}|p{13cm}|}
\hline
\textbf{Note} & \textbf{Words} \\
\hline
    0 &    [assur, contrat, mois, aucun, être, demand, fair, rembours, pai, client] \\
    2 & [assur, être, contrat, sinistr, demand, mois, aucun, fair, servic, véhicul] \\
    3 &     [assur, servic, prix, contrat, bien, être, satisf, fair, client, rapid] \\
    4 &       [assur, prix, servic, satisf, rapid, bon, bien, conseil, être, tarif] \\
    5 &  [assur, prix, servic, satisf, rapid, recommand, bon, bien, conseil, tarif] \\
\hline
\end{tabular}
\label{tab:unsupervised_split}
\caption{LDA result on grouped avis}
\end{table}

\newpage
\section{Supervised}

For the supervised task we use word embedding to vectorize the \lstinline{avis} column, then we use this embedding in addition to other features to train multiple models.

\subsection{Data processing}

We use \lstinline{spacy} and \lstinline{fasttext} word embedding to vectorize the \lstinline{avis}, both give 300 size long vectors.
We converted the column \lstinline{assureur} and \lstinline{produit} into categorical variables.
Finally we extracted the \lstinline{dayofweek, day, month} and \lstinline{year} from the \lstinline{date} column, then we droped the \lstinline{auteur} and \lstinline{date} columns.

\subsection{Models}

\subsubsection{Random Forest}


\begin{table}[H]
    \centering
    \begin{tabular}{l|l|l|l|l|l|}
         True\backslash^{\textstyle{\textrm{Predicted}}} & 1 & 2 & 3 & 4 & 5\\ \hline
         1 & 5818 & 0 & 0 & 0 & 0 \\ \hline
         2 & 0 & 2999 & 0 & 0 & 0 \\ \hline
         3 & 0 & 0 & 2745 & 0 & 0 \\ \hline
         4 & 0 & 0 & 0 & 3860 & 0 \\ \hline
         5 & 0 & 0 & 0 & 0 & 3861  \\ \hline
    \end{tabular}
    \caption{Train confusion matrix}
    \label{tab:rfs_train_confusion_matrix}
\end{table}



\begin{table}[H]
    \centering
    \begin{tabular}{l|l|l|l|l|l|}
         True\backslash^{\textstyle{\textrm{Predicted}}} & 1 & 2 & 3 & 4 & 5\\ \hline
         1 & 1319 & 38 & 23 & 47 & 26 \\ \hline
         2 & 592 & 26 & 19 & 52 & 28 \\ \hline
         3 & 285 & 31 & 33 & 178 & 110 \\ \hline
         4 & 168 & 11 & 50 & 412 & 384 \\ \hline
         5 & 99 & 8 & 31 & 318 & 533 \\ \hline
    \end{tabular}
    \caption{Val confusion matrix}
    \label{tab:rfs_val_confusion_matrix}
\end{table}

Validation set accuracy: 0.48185023853972203

\subsubsection{Feed Forward Neural Network}

 
\begin{figure}[H]
    \centering
    %% Creator: Matplotlib, PGF backend
%%
%% To include the figure in your LaTeX document, write
%%   \input{<filename>.pgf}
%%
%% Make sure the required packages are loaded in your preamble
%%   \usepackage{pgf}
%%
%% Also ensure that all the required font packages are loaded; for instance,
%% the lmodern package is sometimes necessary when using math font.
%%   \usepackage{lmodern}
%%
%% Figures using additional raster images can only be included by \input if
%% they are in the same directory as the main LaTeX file. For loading figures
%% from other directories you can use the `import` package
%%   \usepackage{import}
%%
%% and then include the figures with
%%   \import{<path to file>}{<filename>.pgf}
%%
%% Matplotlib used the following preamble
%%
\begingroup%
\makeatletter%
\begin{pgfpicture}%
\pgfpathrectangle{\pgfpointorigin}{\pgfqpoint{5.426985in}{4.270679in}}%
\pgfusepath{use as bounding box, clip}%
\begin{pgfscope}%
\pgfsetbuttcap%
\pgfsetmiterjoin%
\definecolor{currentfill}{rgb}{1.000000,1.000000,1.000000}%
\pgfsetfillcolor{currentfill}%
\pgfsetlinewidth{0.000000pt}%
\definecolor{currentstroke}{rgb}{1.000000,1.000000,1.000000}%
\pgfsetstrokecolor{currentstroke}%
\pgfsetdash{}{0pt}%
\pgfpathmoveto{\pgfqpoint{0.000000in}{0.000000in}}%
\pgfpathlineto{\pgfqpoint{5.426985in}{0.000000in}}%
\pgfpathlineto{\pgfqpoint{5.426985in}{4.270679in}}%
\pgfpathlineto{\pgfqpoint{0.000000in}{4.270679in}}%
\pgfpathlineto{\pgfqpoint{0.000000in}{0.000000in}}%
\pgfpathclose%
\pgfusepath{fill}%
\end{pgfscope}%
\begin{pgfscope}%
\pgfsetbuttcap%
\pgfsetmiterjoin%
\definecolor{currentfill}{rgb}{1.000000,1.000000,1.000000}%
\pgfsetfillcolor{currentfill}%
\pgfsetlinewidth{0.000000pt}%
\definecolor{currentstroke}{rgb}{0.000000,0.000000,0.000000}%
\pgfsetstrokecolor{currentstroke}%
\pgfsetstrokeopacity{0.000000}%
\pgfsetdash{}{0pt}%
\pgfpathmoveto{\pgfqpoint{0.444137in}{0.320679in}}%
\pgfpathlineto{\pgfqpoint{5.326985in}{0.320679in}}%
\pgfpathlineto{\pgfqpoint{5.326985in}{4.170679in}}%
\pgfpathlineto{\pgfqpoint{0.444137in}{4.170679in}}%
\pgfpathlineto{\pgfqpoint{0.444137in}{0.320679in}}%
\pgfpathclose%
\pgfusepath{fill}%
\end{pgfscope}%
\begin{pgfscope}%
\pgfsetbuttcap%
\pgfsetroundjoin%
\definecolor{currentfill}{rgb}{0.000000,0.000000,0.000000}%
\pgfsetfillcolor{currentfill}%
\pgfsetlinewidth{0.803000pt}%
\definecolor{currentstroke}{rgb}{0.000000,0.000000,0.000000}%
\pgfsetstrokecolor{currentstroke}%
\pgfsetdash{}{0pt}%
\pgfsys@defobject{currentmarker}{\pgfqpoint{0.000000in}{-0.048611in}}{\pgfqpoint{0.000000in}{0.000000in}}{%
\pgfpathmoveto{\pgfqpoint{0.000000in}{0.000000in}}%
\pgfpathlineto{\pgfqpoint{0.000000in}{-0.048611in}}%
\pgfusepath{stroke,fill}%
}%
\begin{pgfscope}%
\pgfsys@transformshift{0.666084in}{0.320679in}%
\pgfsys@useobject{currentmarker}{}%
\end{pgfscope}%
\end{pgfscope}%
\begin{pgfscope}%
\definecolor{textcolor}{rgb}{0.000000,0.000000,0.000000}%
\pgfsetstrokecolor{textcolor}%
\pgfsetfillcolor{textcolor}%
\pgftext[x=0.666084in,y=0.223457in,,top]{\color{textcolor}\rmfamily\fontsize{10.000000}{12.000000}\selectfont \(\displaystyle {0}\)}%
\end{pgfscope}%
\begin{pgfscope}%
\pgfsetbuttcap%
\pgfsetroundjoin%
\definecolor{currentfill}{rgb}{0.000000,0.000000,0.000000}%
\pgfsetfillcolor{currentfill}%
\pgfsetlinewidth{0.803000pt}%
\definecolor{currentstroke}{rgb}{0.000000,0.000000,0.000000}%
\pgfsetstrokecolor{currentstroke}%
\pgfsetdash{}{0pt}%
\pgfsys@defobject{currentmarker}{\pgfqpoint{0.000000in}{-0.048611in}}{\pgfqpoint{0.000000in}{0.000000in}}{%
\pgfpathmoveto{\pgfqpoint{0.000000in}{0.000000in}}%
\pgfpathlineto{\pgfqpoint{0.000000in}{-0.048611in}}%
\pgfusepath{stroke,fill}%
}%
\begin{pgfscope}%
\pgfsys@transformshift{1.258735in}{0.320679in}%
\pgfsys@useobject{currentmarker}{}%
\end{pgfscope}%
\end{pgfscope}%
\begin{pgfscope}%
\definecolor{textcolor}{rgb}{0.000000,0.000000,0.000000}%
\pgfsetstrokecolor{textcolor}%
\pgfsetfillcolor{textcolor}%
\pgftext[x=1.258735in,y=0.223457in,,top]{\color{textcolor}\rmfamily\fontsize{10.000000}{12.000000}\selectfont \(\displaystyle {200}\)}%
\end{pgfscope}%
\begin{pgfscope}%
\pgfsetbuttcap%
\pgfsetroundjoin%
\definecolor{currentfill}{rgb}{0.000000,0.000000,0.000000}%
\pgfsetfillcolor{currentfill}%
\pgfsetlinewidth{0.803000pt}%
\definecolor{currentstroke}{rgb}{0.000000,0.000000,0.000000}%
\pgfsetstrokecolor{currentstroke}%
\pgfsetdash{}{0pt}%
\pgfsys@defobject{currentmarker}{\pgfqpoint{0.000000in}{-0.048611in}}{\pgfqpoint{0.000000in}{0.000000in}}{%
\pgfpathmoveto{\pgfqpoint{0.000000in}{0.000000in}}%
\pgfpathlineto{\pgfqpoint{0.000000in}{-0.048611in}}%
\pgfusepath{stroke,fill}%
}%
\begin{pgfscope}%
\pgfsys@transformshift{1.851386in}{0.320679in}%
\pgfsys@useobject{currentmarker}{}%
\end{pgfscope}%
\end{pgfscope}%
\begin{pgfscope}%
\definecolor{textcolor}{rgb}{0.000000,0.000000,0.000000}%
\pgfsetstrokecolor{textcolor}%
\pgfsetfillcolor{textcolor}%
\pgftext[x=1.851386in,y=0.223457in,,top]{\color{textcolor}\rmfamily\fontsize{10.000000}{12.000000}\selectfont \(\displaystyle {400}\)}%
\end{pgfscope}%
\begin{pgfscope}%
\pgfsetbuttcap%
\pgfsetroundjoin%
\definecolor{currentfill}{rgb}{0.000000,0.000000,0.000000}%
\pgfsetfillcolor{currentfill}%
\pgfsetlinewidth{0.803000pt}%
\definecolor{currentstroke}{rgb}{0.000000,0.000000,0.000000}%
\pgfsetstrokecolor{currentstroke}%
\pgfsetdash{}{0pt}%
\pgfsys@defobject{currentmarker}{\pgfqpoint{0.000000in}{-0.048611in}}{\pgfqpoint{0.000000in}{0.000000in}}{%
\pgfpathmoveto{\pgfqpoint{0.000000in}{0.000000in}}%
\pgfpathlineto{\pgfqpoint{0.000000in}{-0.048611in}}%
\pgfusepath{stroke,fill}%
}%
\begin{pgfscope}%
\pgfsys@transformshift{2.444036in}{0.320679in}%
\pgfsys@useobject{currentmarker}{}%
\end{pgfscope}%
\end{pgfscope}%
\begin{pgfscope}%
\definecolor{textcolor}{rgb}{0.000000,0.000000,0.000000}%
\pgfsetstrokecolor{textcolor}%
\pgfsetfillcolor{textcolor}%
\pgftext[x=2.444036in,y=0.223457in,,top]{\color{textcolor}\rmfamily\fontsize{10.000000}{12.000000}\selectfont \(\displaystyle {600}\)}%
\end{pgfscope}%
\begin{pgfscope}%
\pgfsetbuttcap%
\pgfsetroundjoin%
\definecolor{currentfill}{rgb}{0.000000,0.000000,0.000000}%
\pgfsetfillcolor{currentfill}%
\pgfsetlinewidth{0.803000pt}%
\definecolor{currentstroke}{rgb}{0.000000,0.000000,0.000000}%
\pgfsetstrokecolor{currentstroke}%
\pgfsetdash{}{0pt}%
\pgfsys@defobject{currentmarker}{\pgfqpoint{0.000000in}{-0.048611in}}{\pgfqpoint{0.000000in}{0.000000in}}{%
\pgfpathmoveto{\pgfqpoint{0.000000in}{0.000000in}}%
\pgfpathlineto{\pgfqpoint{0.000000in}{-0.048611in}}%
\pgfusepath{stroke,fill}%
}%
\begin{pgfscope}%
\pgfsys@transformshift{3.036687in}{0.320679in}%
\pgfsys@useobject{currentmarker}{}%
\end{pgfscope}%
\end{pgfscope}%
\begin{pgfscope}%
\definecolor{textcolor}{rgb}{0.000000,0.000000,0.000000}%
\pgfsetstrokecolor{textcolor}%
\pgfsetfillcolor{textcolor}%
\pgftext[x=3.036687in,y=0.223457in,,top]{\color{textcolor}\rmfamily\fontsize{10.000000}{12.000000}\selectfont \(\displaystyle {800}\)}%
\end{pgfscope}%
\begin{pgfscope}%
\pgfsetbuttcap%
\pgfsetroundjoin%
\definecolor{currentfill}{rgb}{0.000000,0.000000,0.000000}%
\pgfsetfillcolor{currentfill}%
\pgfsetlinewidth{0.803000pt}%
\definecolor{currentstroke}{rgb}{0.000000,0.000000,0.000000}%
\pgfsetstrokecolor{currentstroke}%
\pgfsetdash{}{0pt}%
\pgfsys@defobject{currentmarker}{\pgfqpoint{0.000000in}{-0.048611in}}{\pgfqpoint{0.000000in}{0.000000in}}{%
\pgfpathmoveto{\pgfqpoint{0.000000in}{0.000000in}}%
\pgfpathlineto{\pgfqpoint{0.000000in}{-0.048611in}}%
\pgfusepath{stroke,fill}%
}%
\begin{pgfscope}%
\pgfsys@transformshift{3.629337in}{0.320679in}%
\pgfsys@useobject{currentmarker}{}%
\end{pgfscope}%
\end{pgfscope}%
\begin{pgfscope}%
\definecolor{textcolor}{rgb}{0.000000,0.000000,0.000000}%
\pgfsetstrokecolor{textcolor}%
\pgfsetfillcolor{textcolor}%
\pgftext[x=3.629337in,y=0.223457in,,top]{\color{textcolor}\rmfamily\fontsize{10.000000}{12.000000}\selectfont \(\displaystyle {1000}\)}%
\end{pgfscope}%
\begin{pgfscope}%
\pgfsetbuttcap%
\pgfsetroundjoin%
\definecolor{currentfill}{rgb}{0.000000,0.000000,0.000000}%
\pgfsetfillcolor{currentfill}%
\pgfsetlinewidth{0.803000pt}%
\definecolor{currentstroke}{rgb}{0.000000,0.000000,0.000000}%
\pgfsetstrokecolor{currentstroke}%
\pgfsetdash{}{0pt}%
\pgfsys@defobject{currentmarker}{\pgfqpoint{0.000000in}{-0.048611in}}{\pgfqpoint{0.000000in}{0.000000in}}{%
\pgfpathmoveto{\pgfqpoint{0.000000in}{0.000000in}}%
\pgfpathlineto{\pgfqpoint{0.000000in}{-0.048611in}}%
\pgfusepath{stroke,fill}%
}%
\begin{pgfscope}%
\pgfsys@transformshift{4.221988in}{0.320679in}%
\pgfsys@useobject{currentmarker}{}%
\end{pgfscope}%
\end{pgfscope}%
\begin{pgfscope}%
\definecolor{textcolor}{rgb}{0.000000,0.000000,0.000000}%
\pgfsetstrokecolor{textcolor}%
\pgfsetfillcolor{textcolor}%
\pgftext[x=4.221988in,y=0.223457in,,top]{\color{textcolor}\rmfamily\fontsize{10.000000}{12.000000}\selectfont \(\displaystyle {1200}\)}%
\end{pgfscope}%
\begin{pgfscope}%
\pgfsetbuttcap%
\pgfsetroundjoin%
\definecolor{currentfill}{rgb}{0.000000,0.000000,0.000000}%
\pgfsetfillcolor{currentfill}%
\pgfsetlinewidth{0.803000pt}%
\definecolor{currentstroke}{rgb}{0.000000,0.000000,0.000000}%
\pgfsetstrokecolor{currentstroke}%
\pgfsetdash{}{0pt}%
\pgfsys@defobject{currentmarker}{\pgfqpoint{0.000000in}{-0.048611in}}{\pgfqpoint{0.000000in}{0.000000in}}{%
\pgfpathmoveto{\pgfqpoint{0.000000in}{0.000000in}}%
\pgfpathlineto{\pgfqpoint{0.000000in}{-0.048611in}}%
\pgfusepath{stroke,fill}%
}%
\begin{pgfscope}%
\pgfsys@transformshift{4.814639in}{0.320679in}%
\pgfsys@useobject{currentmarker}{}%
\end{pgfscope}%
\end{pgfscope}%
\begin{pgfscope}%
\definecolor{textcolor}{rgb}{0.000000,0.000000,0.000000}%
\pgfsetstrokecolor{textcolor}%
\pgfsetfillcolor{textcolor}%
\pgftext[x=4.814639in,y=0.223457in,,top]{\color{textcolor}\rmfamily\fontsize{10.000000}{12.000000}\selectfont \(\displaystyle {1400}\)}%
\end{pgfscope}%
\begin{pgfscope}%
\pgfsetbuttcap%
\pgfsetroundjoin%
\definecolor{currentfill}{rgb}{0.000000,0.000000,0.000000}%
\pgfsetfillcolor{currentfill}%
\pgfsetlinewidth{0.803000pt}%
\definecolor{currentstroke}{rgb}{0.000000,0.000000,0.000000}%
\pgfsetstrokecolor{currentstroke}%
\pgfsetdash{}{0pt}%
\pgfsys@defobject{currentmarker}{\pgfqpoint{-0.048611in}{0.000000in}}{\pgfqpoint{-0.000000in}{0.000000in}}{%
\pgfpathmoveto{\pgfqpoint{-0.000000in}{0.000000in}}%
\pgfpathlineto{\pgfqpoint{-0.048611in}{0.000000in}}%
\pgfusepath{stroke,fill}%
}%
\begin{pgfscope}%
\pgfsys@transformshift{0.444137in}{0.428917in}%
\pgfsys@useobject{currentmarker}{}%
\end{pgfscope}%
\end{pgfscope}%
\begin{pgfscope}%
\definecolor{textcolor}{rgb}{0.000000,0.000000,0.000000}%
\pgfsetstrokecolor{textcolor}%
\pgfsetfillcolor{textcolor}%
\pgftext[x=0.100000in, y=0.380692in, left, base]{\color{textcolor}\rmfamily\fontsize{10.000000}{12.000000}\selectfont \(\displaystyle {1.35}\)}%
\end{pgfscope}%
\begin{pgfscope}%
\pgfsetbuttcap%
\pgfsetroundjoin%
\definecolor{currentfill}{rgb}{0.000000,0.000000,0.000000}%
\pgfsetfillcolor{currentfill}%
\pgfsetlinewidth{0.803000pt}%
\definecolor{currentstroke}{rgb}{0.000000,0.000000,0.000000}%
\pgfsetstrokecolor{currentstroke}%
\pgfsetdash{}{0pt}%
\pgfsys@defobject{currentmarker}{\pgfqpoint{-0.048611in}{0.000000in}}{\pgfqpoint{-0.000000in}{0.000000in}}{%
\pgfpathmoveto{\pgfqpoint{-0.000000in}{0.000000in}}%
\pgfpathlineto{\pgfqpoint{-0.048611in}{0.000000in}}%
\pgfusepath{stroke,fill}%
}%
\begin{pgfscope}%
\pgfsys@transformshift{0.444137in}{1.132583in}%
\pgfsys@useobject{currentmarker}{}%
\end{pgfscope}%
\end{pgfscope}%
\begin{pgfscope}%
\definecolor{textcolor}{rgb}{0.000000,0.000000,0.000000}%
\pgfsetstrokecolor{textcolor}%
\pgfsetfillcolor{textcolor}%
\pgftext[x=0.100000in, y=1.084358in, left, base]{\color{textcolor}\rmfamily\fontsize{10.000000}{12.000000}\selectfont \(\displaystyle {1.40}\)}%
\end{pgfscope}%
\begin{pgfscope}%
\pgfsetbuttcap%
\pgfsetroundjoin%
\definecolor{currentfill}{rgb}{0.000000,0.000000,0.000000}%
\pgfsetfillcolor{currentfill}%
\pgfsetlinewidth{0.803000pt}%
\definecolor{currentstroke}{rgb}{0.000000,0.000000,0.000000}%
\pgfsetstrokecolor{currentstroke}%
\pgfsetdash{}{0pt}%
\pgfsys@defobject{currentmarker}{\pgfqpoint{-0.048611in}{0.000000in}}{\pgfqpoint{-0.000000in}{0.000000in}}{%
\pgfpathmoveto{\pgfqpoint{-0.000000in}{0.000000in}}%
\pgfpathlineto{\pgfqpoint{-0.048611in}{0.000000in}}%
\pgfusepath{stroke,fill}%
}%
\begin{pgfscope}%
\pgfsys@transformshift{0.444137in}{1.836249in}%
\pgfsys@useobject{currentmarker}{}%
\end{pgfscope}%
\end{pgfscope}%
\begin{pgfscope}%
\definecolor{textcolor}{rgb}{0.000000,0.000000,0.000000}%
\pgfsetstrokecolor{textcolor}%
\pgfsetfillcolor{textcolor}%
\pgftext[x=0.100000in, y=1.788024in, left, base]{\color{textcolor}\rmfamily\fontsize{10.000000}{12.000000}\selectfont \(\displaystyle {1.45}\)}%
\end{pgfscope}%
\begin{pgfscope}%
\pgfsetbuttcap%
\pgfsetroundjoin%
\definecolor{currentfill}{rgb}{0.000000,0.000000,0.000000}%
\pgfsetfillcolor{currentfill}%
\pgfsetlinewidth{0.803000pt}%
\definecolor{currentstroke}{rgb}{0.000000,0.000000,0.000000}%
\pgfsetstrokecolor{currentstroke}%
\pgfsetdash{}{0pt}%
\pgfsys@defobject{currentmarker}{\pgfqpoint{-0.048611in}{0.000000in}}{\pgfqpoint{-0.000000in}{0.000000in}}{%
\pgfpathmoveto{\pgfqpoint{-0.000000in}{0.000000in}}%
\pgfpathlineto{\pgfqpoint{-0.048611in}{0.000000in}}%
\pgfusepath{stroke,fill}%
}%
\begin{pgfscope}%
\pgfsys@transformshift{0.444137in}{2.539915in}%
\pgfsys@useobject{currentmarker}{}%
\end{pgfscope}%
\end{pgfscope}%
\begin{pgfscope}%
\definecolor{textcolor}{rgb}{0.000000,0.000000,0.000000}%
\pgfsetstrokecolor{textcolor}%
\pgfsetfillcolor{textcolor}%
\pgftext[x=0.100000in, y=2.491690in, left, base]{\color{textcolor}\rmfamily\fontsize{10.000000}{12.000000}\selectfont \(\displaystyle {1.50}\)}%
\end{pgfscope}%
\begin{pgfscope}%
\pgfsetbuttcap%
\pgfsetroundjoin%
\definecolor{currentfill}{rgb}{0.000000,0.000000,0.000000}%
\pgfsetfillcolor{currentfill}%
\pgfsetlinewidth{0.803000pt}%
\definecolor{currentstroke}{rgb}{0.000000,0.000000,0.000000}%
\pgfsetstrokecolor{currentstroke}%
\pgfsetdash{}{0pt}%
\pgfsys@defobject{currentmarker}{\pgfqpoint{-0.048611in}{0.000000in}}{\pgfqpoint{-0.000000in}{0.000000in}}{%
\pgfpathmoveto{\pgfqpoint{-0.000000in}{0.000000in}}%
\pgfpathlineto{\pgfqpoint{-0.048611in}{0.000000in}}%
\pgfusepath{stroke,fill}%
}%
\begin{pgfscope}%
\pgfsys@transformshift{0.444137in}{3.243582in}%
\pgfsys@useobject{currentmarker}{}%
\end{pgfscope}%
\end{pgfscope}%
\begin{pgfscope}%
\definecolor{textcolor}{rgb}{0.000000,0.000000,0.000000}%
\pgfsetstrokecolor{textcolor}%
\pgfsetfillcolor{textcolor}%
\pgftext[x=0.100000in, y=3.195356in, left, base]{\color{textcolor}\rmfamily\fontsize{10.000000}{12.000000}\selectfont \(\displaystyle {1.55}\)}%
\end{pgfscope}%
\begin{pgfscope}%
\pgfsetbuttcap%
\pgfsetroundjoin%
\definecolor{currentfill}{rgb}{0.000000,0.000000,0.000000}%
\pgfsetfillcolor{currentfill}%
\pgfsetlinewidth{0.803000pt}%
\definecolor{currentstroke}{rgb}{0.000000,0.000000,0.000000}%
\pgfsetstrokecolor{currentstroke}%
\pgfsetdash{}{0pt}%
\pgfsys@defobject{currentmarker}{\pgfqpoint{-0.048611in}{0.000000in}}{\pgfqpoint{-0.000000in}{0.000000in}}{%
\pgfpathmoveto{\pgfqpoint{-0.000000in}{0.000000in}}%
\pgfpathlineto{\pgfqpoint{-0.048611in}{0.000000in}}%
\pgfusepath{stroke,fill}%
}%
\begin{pgfscope}%
\pgfsys@transformshift{0.444137in}{3.947248in}%
\pgfsys@useobject{currentmarker}{}%
\end{pgfscope}%
\end{pgfscope}%
\begin{pgfscope}%
\definecolor{textcolor}{rgb}{0.000000,0.000000,0.000000}%
\pgfsetstrokecolor{textcolor}%
\pgfsetfillcolor{textcolor}%
\pgftext[x=0.100000in, y=3.899022in, left, base]{\color{textcolor}\rmfamily\fontsize{10.000000}{12.000000}\selectfont \(\displaystyle {1.60}\)}%
\end{pgfscope}%
\begin{pgfscope}%
\pgfpathrectangle{\pgfqpoint{0.444137in}{0.320679in}}{\pgfqpoint{4.882849in}{3.850000in}}%
\pgfusepath{clip}%
\pgfsetrectcap%
\pgfsetroundjoin%
\pgfsetlinewidth{1.505625pt}%
\definecolor{currentstroke}{rgb}{0.121569,0.466667,0.705882}%
\pgfsetstrokecolor{currentstroke}%
\pgfsetdash{}{0pt}%
\pgfpathmoveto{\pgfqpoint{0.666084in}{3.990980in}}%
\pgfpathlineto{\pgfqpoint{0.680900in}{3.991235in}}%
\pgfpathlineto{\pgfqpoint{0.728313in}{3.990559in}}%
\pgfpathlineto{\pgfqpoint{0.731276in}{3.989067in}}%
\pgfpathlineto{\pgfqpoint{0.734239in}{3.977825in}}%
\pgfpathlineto{\pgfqpoint{0.737202in}{3.818827in}}%
\pgfpathlineto{\pgfqpoint{0.746092in}{3.149802in}}%
\pgfpathlineto{\pgfqpoint{0.749055in}{3.061168in}}%
\pgfpathlineto{\pgfqpoint{0.752019in}{2.906941in}}%
\pgfpathlineto{\pgfqpoint{0.754982in}{2.929905in}}%
\pgfpathlineto{\pgfqpoint{0.757945in}{2.596502in}}%
\pgfpathlineto{\pgfqpoint{0.763872in}{2.256481in}}%
\pgfpathlineto{\pgfqpoint{0.766835in}{2.228156in}}%
\pgfpathlineto{\pgfqpoint{0.769798in}{2.064368in}}%
\pgfpathlineto{\pgfqpoint{0.772761in}{2.118629in}}%
\pgfpathlineto{\pgfqpoint{0.775725in}{1.897576in}}%
\pgfpathlineto{\pgfqpoint{0.778688in}{1.857946in}}%
\pgfpathlineto{\pgfqpoint{0.781651in}{1.776401in}}%
\pgfpathlineto{\pgfqpoint{0.784614in}{1.813066in}}%
\pgfpathlineto{\pgfqpoint{0.787578in}{1.740307in}}%
\pgfpathlineto{\pgfqpoint{0.790541in}{1.705175in}}%
\pgfpathlineto{\pgfqpoint{0.793504in}{1.736779in}}%
\pgfpathlineto{\pgfqpoint{0.796467in}{1.681594in}}%
\pgfpathlineto{\pgfqpoint{0.799431in}{1.587690in}}%
\pgfpathlineto{\pgfqpoint{0.802394in}{1.650258in}}%
\pgfpathlineto{\pgfqpoint{0.805357in}{1.578114in}}%
\pgfpathlineto{\pgfqpoint{0.808320in}{1.619883in}}%
\pgfpathlineto{\pgfqpoint{0.811284in}{1.562303in}}%
\pgfpathlineto{\pgfqpoint{0.814247in}{1.633846in}}%
\pgfpathlineto{\pgfqpoint{0.817210in}{1.738997in}}%
\pgfpathlineto{\pgfqpoint{0.820173in}{1.718867in}}%
\pgfpathlineto{\pgfqpoint{0.823137in}{1.567727in}}%
\pgfpathlineto{\pgfqpoint{0.826100in}{1.679822in}}%
\pgfpathlineto{\pgfqpoint{0.829063in}{1.507831in}}%
\pgfpathlineto{\pgfqpoint{0.832026in}{1.459158in}}%
\pgfpathlineto{\pgfqpoint{0.834990in}{1.505325in}}%
\pgfpathlineto{\pgfqpoint{0.837953in}{1.859484in}}%
\pgfpathlineto{\pgfqpoint{0.840916in}{1.517273in}}%
\pgfpathlineto{\pgfqpoint{0.843879in}{1.742957in}}%
\pgfpathlineto{\pgfqpoint{0.846843in}{1.772458in}}%
\pgfpathlineto{\pgfqpoint{0.855732in}{1.416457in}}%
\pgfpathlineto{\pgfqpoint{0.858696in}{1.411437in}}%
\pgfpathlineto{\pgfqpoint{0.861659in}{1.540383in}}%
\pgfpathlineto{\pgfqpoint{0.864622in}{1.453424in}}%
\pgfpathlineto{\pgfqpoint{0.867585in}{1.410159in}}%
\pgfpathlineto{\pgfqpoint{0.870549in}{1.463915in}}%
\pgfpathlineto{\pgfqpoint{0.873512in}{1.385440in}}%
\pgfpathlineto{\pgfqpoint{0.879438in}{1.475561in}}%
\pgfpathlineto{\pgfqpoint{0.882402in}{1.453290in}}%
\pgfpathlineto{\pgfqpoint{0.885365in}{1.381142in}}%
\pgfpathlineto{\pgfqpoint{0.888328in}{1.409674in}}%
\pgfpathlineto{\pgfqpoint{0.891291in}{1.556267in}}%
\pgfpathlineto{\pgfqpoint{0.894255in}{1.374196in}}%
\pgfpathlineto{\pgfqpoint{0.897218in}{1.422035in}}%
\pgfpathlineto{\pgfqpoint{0.900181in}{1.432485in}}%
\pgfpathlineto{\pgfqpoint{0.903144in}{1.495836in}}%
\pgfpathlineto{\pgfqpoint{0.906108in}{1.370633in}}%
\pgfpathlineto{\pgfqpoint{0.909071in}{1.595018in}}%
\pgfpathlineto{\pgfqpoint{0.912034in}{1.634254in}}%
\pgfpathlineto{\pgfqpoint{0.914997in}{1.446849in}}%
\pgfpathlineto{\pgfqpoint{0.917961in}{1.401514in}}%
\pgfpathlineto{\pgfqpoint{0.920924in}{1.404357in}}%
\pgfpathlineto{\pgfqpoint{0.923887in}{1.404517in}}%
\pgfpathlineto{\pgfqpoint{0.926850in}{1.348800in}}%
\pgfpathlineto{\pgfqpoint{0.929814in}{1.502951in}}%
\pgfpathlineto{\pgfqpoint{0.932777in}{1.745532in}}%
\pgfpathlineto{\pgfqpoint{0.935740in}{1.321360in}}%
\pgfpathlineto{\pgfqpoint{0.938703in}{1.556170in}}%
\pgfpathlineto{\pgfqpoint{0.941667in}{1.596808in}}%
\pgfpathlineto{\pgfqpoint{0.944630in}{1.361075in}}%
\pgfpathlineto{\pgfqpoint{0.947593in}{1.540853in}}%
\pgfpathlineto{\pgfqpoint{0.950557in}{1.466231in}}%
\pgfpathlineto{\pgfqpoint{0.953520in}{1.323072in}}%
\pgfpathlineto{\pgfqpoint{0.956483in}{1.471384in}}%
\pgfpathlineto{\pgfqpoint{0.959446in}{1.372215in}}%
\pgfpathlineto{\pgfqpoint{0.962410in}{1.513050in}}%
\pgfpathlineto{\pgfqpoint{0.965373in}{1.332315in}}%
\pgfpathlineto{\pgfqpoint{0.968336in}{1.409994in}}%
\pgfpathlineto{\pgfqpoint{0.971299in}{1.344206in}}%
\pgfpathlineto{\pgfqpoint{0.974263in}{1.395034in}}%
\pgfpathlineto{\pgfqpoint{0.977226in}{1.667919in}}%
\pgfpathlineto{\pgfqpoint{0.980189in}{1.413873in}}%
\pgfpathlineto{\pgfqpoint{0.983152in}{1.295148in}}%
\pgfpathlineto{\pgfqpoint{0.986116in}{1.357763in}}%
\pgfpathlineto{\pgfqpoint{0.989079in}{1.328894in}}%
\pgfpathlineto{\pgfqpoint{0.992042in}{1.424466in}}%
\pgfpathlineto{\pgfqpoint{0.995005in}{1.341237in}}%
\pgfpathlineto{\pgfqpoint{0.997969in}{1.499991in}}%
\pgfpathlineto{\pgfqpoint{1.000932in}{1.403339in}}%
\pgfpathlineto{\pgfqpoint{1.003895in}{1.352766in}}%
\pgfpathlineto{\pgfqpoint{1.006858in}{1.462287in}}%
\pgfpathlineto{\pgfqpoint{1.009822in}{1.623986in}}%
\pgfpathlineto{\pgfqpoint{1.012785in}{1.458197in}}%
\pgfpathlineto{\pgfqpoint{1.015748in}{1.811853in}}%
\pgfpathlineto{\pgfqpoint{1.018711in}{1.361369in}}%
\pgfpathlineto{\pgfqpoint{1.021675in}{1.292700in}}%
\pgfpathlineto{\pgfqpoint{1.024638in}{1.345208in}}%
\pgfpathlineto{\pgfqpoint{1.027601in}{1.683963in}}%
\pgfpathlineto{\pgfqpoint{1.030564in}{1.298196in}}%
\pgfpathlineto{\pgfqpoint{1.033528in}{1.285016in}}%
\pgfpathlineto{\pgfqpoint{1.036491in}{1.310753in}}%
\pgfpathlineto{\pgfqpoint{1.039454in}{1.618319in}}%
\pgfpathlineto{\pgfqpoint{1.042417in}{1.530386in}}%
\pgfpathlineto{\pgfqpoint{1.045381in}{1.265948in}}%
\pgfpathlineto{\pgfqpoint{1.048344in}{1.532565in}}%
\pgfpathlineto{\pgfqpoint{1.051307in}{1.342869in}}%
\pgfpathlineto{\pgfqpoint{1.054270in}{1.457164in}}%
\pgfpathlineto{\pgfqpoint{1.057234in}{1.225043in}}%
\pgfpathlineto{\pgfqpoint{1.060197in}{1.384687in}}%
\pgfpathlineto{\pgfqpoint{1.063160in}{1.433995in}}%
\pgfpathlineto{\pgfqpoint{1.066123in}{1.354066in}}%
\pgfpathlineto{\pgfqpoint{1.069087in}{1.505897in}}%
\pgfpathlineto{\pgfqpoint{1.072050in}{1.330040in}}%
\pgfpathlineto{\pgfqpoint{1.075013in}{1.264389in}}%
\pgfpathlineto{\pgfqpoint{1.077976in}{1.630244in}}%
\pgfpathlineto{\pgfqpoint{1.080940in}{1.385655in}}%
\pgfpathlineto{\pgfqpoint{1.083903in}{1.255657in}}%
\pgfpathlineto{\pgfqpoint{1.086866in}{1.341000in}}%
\pgfpathlineto{\pgfqpoint{1.089829in}{1.222417in}}%
\pgfpathlineto{\pgfqpoint{1.092793in}{1.661843in}}%
\pgfpathlineto{\pgfqpoint{1.095756in}{1.236813in}}%
\pgfpathlineto{\pgfqpoint{1.101682in}{1.302746in}}%
\pgfpathlineto{\pgfqpoint{1.104646in}{1.541346in}}%
\pgfpathlineto{\pgfqpoint{1.107609in}{1.456650in}}%
\pgfpathlineto{\pgfqpoint{1.110572in}{1.437889in}}%
\pgfpathlineto{\pgfqpoint{1.113535in}{1.268785in}}%
\pgfpathlineto{\pgfqpoint{1.116499in}{1.298092in}}%
\pgfpathlineto{\pgfqpoint{1.119462in}{1.265891in}}%
\pgfpathlineto{\pgfqpoint{1.122425in}{1.333358in}}%
\pgfpathlineto{\pgfqpoint{1.128352in}{1.235298in}}%
\pgfpathlineto{\pgfqpoint{1.131315in}{1.240831in}}%
\pgfpathlineto{\pgfqpoint{1.134278in}{1.288689in}}%
\pgfpathlineto{\pgfqpoint{1.137241in}{1.363645in}}%
\pgfpathlineto{\pgfqpoint{1.140205in}{1.242214in}}%
\pgfpathlineto{\pgfqpoint{1.143168in}{1.302613in}}%
\pgfpathlineto{\pgfqpoint{1.146131in}{1.168326in}}%
\pgfpathlineto{\pgfqpoint{1.152058in}{1.298785in}}%
\pgfpathlineto{\pgfqpoint{1.155021in}{1.178789in}}%
\pgfpathlineto{\pgfqpoint{1.157984in}{1.189857in}}%
\pgfpathlineto{\pgfqpoint{1.160947in}{1.160951in}}%
\pgfpathlineto{\pgfqpoint{1.163911in}{1.357862in}}%
\pgfpathlineto{\pgfqpoint{1.166874in}{1.219191in}}%
\pgfpathlineto{\pgfqpoint{1.169837in}{1.234636in}}%
\pgfpathlineto{\pgfqpoint{1.172801in}{1.191721in}}%
\pgfpathlineto{\pgfqpoint{1.175764in}{1.204773in}}%
\pgfpathlineto{\pgfqpoint{1.178727in}{1.225726in}}%
\pgfpathlineto{\pgfqpoint{1.181690in}{1.472613in}}%
\pgfpathlineto{\pgfqpoint{1.184654in}{1.249309in}}%
\pgfpathlineto{\pgfqpoint{1.187617in}{1.493343in}}%
\pgfpathlineto{\pgfqpoint{1.190580in}{1.287318in}}%
\pgfpathlineto{\pgfqpoint{1.193543in}{1.265186in}}%
\pgfpathlineto{\pgfqpoint{1.196507in}{1.145956in}}%
\pgfpathlineto{\pgfqpoint{1.199470in}{1.221293in}}%
\pgfpathlineto{\pgfqpoint{1.202433in}{1.128714in}}%
\pgfpathlineto{\pgfqpoint{1.205396in}{1.166084in}}%
\pgfpathlineto{\pgfqpoint{1.208360in}{1.145405in}}%
\pgfpathlineto{\pgfqpoint{1.211323in}{1.146655in}}%
\pgfpathlineto{\pgfqpoint{1.214286in}{1.387146in}}%
\pgfpathlineto{\pgfqpoint{1.217249in}{1.139166in}}%
\pgfpathlineto{\pgfqpoint{1.220213in}{1.174015in}}%
\pgfpathlineto{\pgfqpoint{1.223176in}{1.329677in}}%
\pgfpathlineto{\pgfqpoint{1.226139in}{1.170171in}}%
\pgfpathlineto{\pgfqpoint{1.229102in}{1.366106in}}%
\pgfpathlineto{\pgfqpoint{1.232066in}{1.185408in}}%
\pgfpathlineto{\pgfqpoint{1.235029in}{1.329481in}}%
\pgfpathlineto{\pgfqpoint{1.237992in}{1.253588in}}%
\pgfpathlineto{\pgfqpoint{1.240955in}{1.110835in}}%
\pgfpathlineto{\pgfqpoint{1.243919in}{1.135521in}}%
\pgfpathlineto{\pgfqpoint{1.246882in}{1.106104in}}%
\pgfpathlineto{\pgfqpoint{1.249845in}{1.173768in}}%
\pgfpathlineto{\pgfqpoint{1.252808in}{1.096926in}}%
\pgfpathlineto{\pgfqpoint{1.255772in}{1.126403in}}%
\pgfpathlineto{\pgfqpoint{1.258735in}{1.136928in}}%
\pgfpathlineto{\pgfqpoint{1.261698in}{1.123688in}}%
\pgfpathlineto{\pgfqpoint{1.264661in}{1.155841in}}%
\pgfpathlineto{\pgfqpoint{1.267625in}{1.374008in}}%
\pgfpathlineto{\pgfqpoint{1.270588in}{1.444158in}}%
\pgfpathlineto{\pgfqpoint{1.273551in}{1.131925in}}%
\pgfpathlineto{\pgfqpoint{1.276514in}{1.235547in}}%
\pgfpathlineto{\pgfqpoint{1.279478in}{1.102529in}}%
\pgfpathlineto{\pgfqpoint{1.282441in}{1.279869in}}%
\pgfpathlineto{\pgfqpoint{1.285404in}{1.155322in}}%
\pgfpathlineto{\pgfqpoint{1.288367in}{1.103251in}}%
\pgfpathlineto{\pgfqpoint{1.291331in}{1.087959in}}%
\pgfpathlineto{\pgfqpoint{1.294294in}{1.094354in}}%
\pgfpathlineto{\pgfqpoint{1.297257in}{1.138020in}}%
\pgfpathlineto{\pgfqpoint{1.300220in}{1.085204in}}%
\pgfpathlineto{\pgfqpoint{1.303184in}{1.163909in}}%
\pgfpathlineto{\pgfqpoint{1.306147in}{1.156466in}}%
\pgfpathlineto{\pgfqpoint{1.312073in}{1.205000in}}%
\pgfpathlineto{\pgfqpoint{1.315037in}{1.073120in}}%
\pgfpathlineto{\pgfqpoint{1.318000in}{1.099238in}}%
\pgfpathlineto{\pgfqpoint{1.320963in}{1.073886in}}%
\pgfpathlineto{\pgfqpoint{1.323926in}{1.074499in}}%
\pgfpathlineto{\pgfqpoint{1.326890in}{1.070524in}}%
\pgfpathlineto{\pgfqpoint{1.332816in}{1.113085in}}%
\pgfpathlineto{\pgfqpoint{1.335779in}{1.079775in}}%
\pgfpathlineto{\pgfqpoint{1.338743in}{1.132714in}}%
\pgfpathlineto{\pgfqpoint{1.341706in}{1.075125in}}%
\pgfpathlineto{\pgfqpoint{1.344669in}{1.066570in}}%
\pgfpathlineto{\pgfqpoint{1.347632in}{1.105712in}}%
\pgfpathlineto{\pgfqpoint{1.350596in}{1.117662in}}%
\pgfpathlineto{\pgfqpoint{1.353559in}{1.070211in}}%
\pgfpathlineto{\pgfqpoint{1.359485in}{1.052590in}}%
\pgfpathlineto{\pgfqpoint{1.362449in}{1.078394in}}%
\pgfpathlineto{\pgfqpoint{1.365412in}{1.074042in}}%
\pgfpathlineto{\pgfqpoint{1.368375in}{1.148286in}}%
\pgfpathlineto{\pgfqpoint{1.371338in}{1.094330in}}%
\pgfpathlineto{\pgfqpoint{1.374302in}{1.107839in}}%
\pgfpathlineto{\pgfqpoint{1.377265in}{1.062171in}}%
\pgfpathlineto{\pgfqpoint{1.380228in}{1.057266in}}%
\pgfpathlineto{\pgfqpoint{1.383191in}{1.054652in}}%
\pgfpathlineto{\pgfqpoint{1.386155in}{1.058625in}}%
\pgfpathlineto{\pgfqpoint{1.389118in}{1.049632in}}%
\pgfpathlineto{\pgfqpoint{1.392081in}{1.098523in}}%
\pgfpathlineto{\pgfqpoint{1.395045in}{1.195940in}}%
\pgfpathlineto{\pgfqpoint{1.398008in}{1.051865in}}%
\pgfpathlineto{\pgfqpoint{1.400971in}{1.159666in}}%
\pgfpathlineto{\pgfqpoint{1.403934in}{1.133095in}}%
\pgfpathlineto{\pgfqpoint{1.406898in}{1.149012in}}%
\pgfpathlineto{\pgfqpoint{1.409861in}{1.230692in}}%
\pgfpathlineto{\pgfqpoint{1.412824in}{1.075037in}}%
\pgfpathlineto{\pgfqpoint{1.415787in}{1.069961in}}%
\pgfpathlineto{\pgfqpoint{1.418751in}{1.093317in}}%
\pgfpathlineto{\pgfqpoint{1.421714in}{1.157439in}}%
\pgfpathlineto{\pgfqpoint{1.427640in}{1.043821in}}%
\pgfpathlineto{\pgfqpoint{1.430604in}{1.215749in}}%
\pgfpathlineto{\pgfqpoint{1.433567in}{1.074873in}}%
\pgfpathlineto{\pgfqpoint{1.436530in}{1.098134in}}%
\pgfpathlineto{\pgfqpoint{1.439493in}{1.030614in}}%
\pgfpathlineto{\pgfqpoint{1.442457in}{1.042687in}}%
\pgfpathlineto{\pgfqpoint{1.445420in}{1.145387in}}%
\pgfpathlineto{\pgfqpoint{1.448383in}{1.024924in}}%
\pgfpathlineto{\pgfqpoint{1.451346in}{1.081906in}}%
\pgfpathlineto{\pgfqpoint{1.454310in}{1.035505in}}%
\pgfpathlineto{\pgfqpoint{1.457273in}{1.101127in}}%
\pgfpathlineto{\pgfqpoint{1.460236in}{1.043881in}}%
\pgfpathlineto{\pgfqpoint{1.463199in}{1.043442in}}%
\pgfpathlineto{\pgfqpoint{1.466163in}{1.162313in}}%
\pgfpathlineto{\pgfqpoint{1.469126in}{1.022355in}}%
\pgfpathlineto{\pgfqpoint{1.472089in}{1.149415in}}%
\pgfpathlineto{\pgfqpoint{1.475052in}{1.138693in}}%
\pgfpathlineto{\pgfqpoint{1.478016in}{1.015389in}}%
\pgfpathlineto{\pgfqpoint{1.480979in}{1.074730in}}%
\pgfpathlineto{\pgfqpoint{1.486905in}{1.028740in}}%
\pgfpathlineto{\pgfqpoint{1.489869in}{1.038924in}}%
\pgfpathlineto{\pgfqpoint{1.492832in}{1.018465in}}%
\pgfpathlineto{\pgfqpoint{1.495795in}{1.057905in}}%
\pgfpathlineto{\pgfqpoint{1.498758in}{1.051731in}}%
\pgfpathlineto{\pgfqpoint{1.501722in}{1.074477in}}%
\pgfpathlineto{\pgfqpoint{1.504685in}{1.058947in}}%
\pgfpathlineto{\pgfqpoint{1.507648in}{1.065053in}}%
\pgfpathlineto{\pgfqpoint{1.513575in}{1.004656in}}%
\pgfpathlineto{\pgfqpoint{1.516538in}{1.059512in}}%
\pgfpathlineto{\pgfqpoint{1.519501in}{1.013571in}}%
\pgfpathlineto{\pgfqpoint{1.522464in}{1.059393in}}%
\pgfpathlineto{\pgfqpoint{1.525428in}{1.032456in}}%
\pgfpathlineto{\pgfqpoint{1.528391in}{1.044901in}}%
\pgfpathlineto{\pgfqpoint{1.531354in}{1.065852in}}%
\pgfpathlineto{\pgfqpoint{1.534317in}{1.114400in}}%
\pgfpathlineto{\pgfqpoint{1.537281in}{1.031074in}}%
\pgfpathlineto{\pgfqpoint{1.540244in}{1.059992in}}%
\pgfpathlineto{\pgfqpoint{1.543207in}{1.160481in}}%
\pgfpathlineto{\pgfqpoint{1.546170in}{1.114353in}}%
\pgfpathlineto{\pgfqpoint{1.549134in}{0.999569in}}%
\pgfpathlineto{\pgfqpoint{1.552097in}{1.052676in}}%
\pgfpathlineto{\pgfqpoint{1.555060in}{1.007840in}}%
\pgfpathlineto{\pgfqpoint{1.558023in}{0.984002in}}%
\pgfpathlineto{\pgfqpoint{1.560987in}{0.996598in}}%
\pgfpathlineto{\pgfqpoint{1.563950in}{0.990419in}}%
\pgfpathlineto{\pgfqpoint{1.566913in}{1.006426in}}%
\pgfpathlineto{\pgfqpoint{1.569876in}{1.070617in}}%
\pgfpathlineto{\pgfqpoint{1.572840in}{1.012230in}}%
\pgfpathlineto{\pgfqpoint{1.575803in}{1.002324in}}%
\pgfpathlineto{\pgfqpoint{1.578766in}{1.003245in}}%
\pgfpathlineto{\pgfqpoint{1.581729in}{1.037934in}}%
\pgfpathlineto{\pgfqpoint{1.584693in}{0.991761in}}%
\pgfpathlineto{\pgfqpoint{1.587656in}{0.991530in}}%
\pgfpathlineto{\pgfqpoint{1.590619in}{0.987146in}}%
\pgfpathlineto{\pgfqpoint{1.593582in}{0.984980in}}%
\pgfpathlineto{\pgfqpoint{1.596546in}{0.967663in}}%
\pgfpathlineto{\pgfqpoint{1.599509in}{1.032948in}}%
\pgfpathlineto{\pgfqpoint{1.602472in}{1.040222in}}%
\pgfpathlineto{\pgfqpoint{1.605435in}{0.976177in}}%
\pgfpathlineto{\pgfqpoint{1.608399in}{1.000745in}}%
\pgfpathlineto{\pgfqpoint{1.611362in}{0.996110in}}%
\pgfpathlineto{\pgfqpoint{1.614325in}{0.986099in}}%
\pgfpathlineto{\pgfqpoint{1.617289in}{0.982104in}}%
\pgfpathlineto{\pgfqpoint{1.620252in}{0.972921in}}%
\pgfpathlineto{\pgfqpoint{1.623215in}{0.986904in}}%
\pgfpathlineto{\pgfqpoint{1.626178in}{1.019518in}}%
\pgfpathlineto{\pgfqpoint{1.629142in}{0.974679in}}%
\pgfpathlineto{\pgfqpoint{1.632105in}{1.018424in}}%
\pgfpathlineto{\pgfqpoint{1.635068in}{1.040291in}}%
\pgfpathlineto{\pgfqpoint{1.638031in}{0.986109in}}%
\pgfpathlineto{\pgfqpoint{1.640995in}{0.994588in}}%
\pgfpathlineto{\pgfqpoint{1.643958in}{0.986007in}}%
\pgfpathlineto{\pgfqpoint{1.646921in}{0.970540in}}%
\pgfpathlineto{\pgfqpoint{1.649884in}{0.969191in}}%
\pgfpathlineto{\pgfqpoint{1.652848in}{0.971468in}}%
\pgfpathlineto{\pgfqpoint{1.655811in}{0.967522in}}%
\pgfpathlineto{\pgfqpoint{1.658774in}{0.997519in}}%
\pgfpathlineto{\pgfqpoint{1.661737in}{1.093690in}}%
\pgfpathlineto{\pgfqpoint{1.664701in}{0.982531in}}%
\pgfpathlineto{\pgfqpoint{1.667664in}{0.998391in}}%
\pgfpathlineto{\pgfqpoint{1.673590in}{0.958436in}}%
\pgfpathlineto{\pgfqpoint{1.676554in}{0.993501in}}%
\pgfpathlineto{\pgfqpoint{1.679517in}{0.978013in}}%
\pgfpathlineto{\pgfqpoint{1.682480in}{0.989073in}}%
\pgfpathlineto{\pgfqpoint{1.685443in}{1.113755in}}%
\pgfpathlineto{\pgfqpoint{1.688407in}{1.096914in}}%
\pgfpathlineto{\pgfqpoint{1.691370in}{1.093549in}}%
\pgfpathlineto{\pgfqpoint{1.694333in}{0.962308in}}%
\pgfpathlineto{\pgfqpoint{1.697296in}{0.964898in}}%
\pgfpathlineto{\pgfqpoint{1.700260in}{0.962080in}}%
\pgfpathlineto{\pgfqpoint{1.703223in}{1.035439in}}%
\pgfpathlineto{\pgfqpoint{1.706186in}{0.965323in}}%
\pgfpathlineto{\pgfqpoint{1.709149in}{1.017542in}}%
\pgfpathlineto{\pgfqpoint{1.712113in}{1.006340in}}%
\pgfpathlineto{\pgfqpoint{1.715076in}{0.973711in}}%
\pgfpathlineto{\pgfqpoint{1.718039in}{0.997500in}}%
\pgfpathlineto{\pgfqpoint{1.721002in}{0.977930in}}%
\pgfpathlineto{\pgfqpoint{1.723966in}{0.940882in}}%
\pgfpathlineto{\pgfqpoint{1.726929in}{0.954493in}}%
\pgfpathlineto{\pgfqpoint{1.729892in}{0.990835in}}%
\pgfpathlineto{\pgfqpoint{1.732855in}{0.945330in}}%
\pgfpathlineto{\pgfqpoint{1.735819in}{1.068847in}}%
\pgfpathlineto{\pgfqpoint{1.738782in}{1.249002in}}%
\pgfpathlineto{\pgfqpoint{1.741745in}{0.949947in}}%
\pgfpathlineto{\pgfqpoint{1.744708in}{0.957070in}}%
\pgfpathlineto{\pgfqpoint{1.747672in}{0.938664in}}%
\pgfpathlineto{\pgfqpoint{1.750635in}{0.968134in}}%
\pgfpathlineto{\pgfqpoint{1.753598in}{0.932541in}}%
\pgfpathlineto{\pgfqpoint{1.756561in}{1.012794in}}%
\pgfpathlineto{\pgfqpoint{1.759525in}{0.940106in}}%
\pgfpathlineto{\pgfqpoint{1.762488in}{0.956186in}}%
\pgfpathlineto{\pgfqpoint{1.765451in}{0.960536in}}%
\pgfpathlineto{\pgfqpoint{1.768414in}{1.020634in}}%
\pgfpathlineto{\pgfqpoint{1.774341in}{0.933903in}}%
\pgfpathlineto{\pgfqpoint{1.777304in}{0.931284in}}%
\pgfpathlineto{\pgfqpoint{1.780267in}{0.937040in}}%
\pgfpathlineto{\pgfqpoint{1.783231in}{0.938201in}}%
\pgfpathlineto{\pgfqpoint{1.786194in}{0.923616in}}%
\pgfpathlineto{\pgfqpoint{1.792120in}{1.042826in}}%
\pgfpathlineto{\pgfqpoint{1.795084in}{0.966859in}}%
\pgfpathlineto{\pgfqpoint{1.798047in}{0.944615in}}%
\pgfpathlineto{\pgfqpoint{1.801010in}{0.977890in}}%
\pgfpathlineto{\pgfqpoint{1.803973in}{0.917771in}}%
\pgfpathlineto{\pgfqpoint{1.806937in}{0.985327in}}%
\pgfpathlineto{\pgfqpoint{1.809900in}{0.922104in}}%
\pgfpathlineto{\pgfqpoint{1.812863in}{1.106827in}}%
\pgfpathlineto{\pgfqpoint{1.815826in}{1.068929in}}%
\pgfpathlineto{\pgfqpoint{1.818790in}{0.932667in}}%
\pgfpathlineto{\pgfqpoint{1.821753in}{1.088640in}}%
\pgfpathlineto{\pgfqpoint{1.827679in}{0.984446in}}%
\pgfpathlineto{\pgfqpoint{1.830643in}{0.978731in}}%
\pgfpathlineto{\pgfqpoint{1.833606in}{0.922795in}}%
\pgfpathlineto{\pgfqpoint{1.836569in}{1.042304in}}%
\pgfpathlineto{\pgfqpoint{1.839532in}{0.943199in}}%
\pgfpathlineto{\pgfqpoint{1.842496in}{0.957728in}}%
\pgfpathlineto{\pgfqpoint{1.845459in}{1.042262in}}%
\pgfpathlineto{\pgfqpoint{1.848422in}{0.918200in}}%
\pgfpathlineto{\pgfqpoint{1.851386in}{1.051265in}}%
\pgfpathlineto{\pgfqpoint{1.854349in}{0.948717in}}%
\pgfpathlineto{\pgfqpoint{1.857312in}{0.998774in}}%
\pgfpathlineto{\pgfqpoint{1.860275in}{0.912840in}}%
\pgfpathlineto{\pgfqpoint{1.863239in}{0.966700in}}%
\pgfpathlineto{\pgfqpoint{1.866202in}{0.921936in}}%
\pgfpathlineto{\pgfqpoint{1.869165in}{0.983827in}}%
\pgfpathlineto{\pgfqpoint{1.872128in}{1.086952in}}%
\pgfpathlineto{\pgfqpoint{1.875092in}{0.918712in}}%
\pgfpathlineto{\pgfqpoint{1.878055in}{0.936116in}}%
\pgfpathlineto{\pgfqpoint{1.881018in}{0.910453in}}%
\pgfpathlineto{\pgfqpoint{1.883981in}{0.943360in}}%
\pgfpathlineto{\pgfqpoint{1.886945in}{1.060551in}}%
\pgfpathlineto{\pgfqpoint{1.889908in}{0.928474in}}%
\pgfpathlineto{\pgfqpoint{1.892871in}{0.893478in}}%
\pgfpathlineto{\pgfqpoint{1.895834in}{0.978444in}}%
\pgfpathlineto{\pgfqpoint{1.898798in}{0.909527in}}%
\pgfpathlineto{\pgfqpoint{1.901761in}{0.919391in}}%
\pgfpathlineto{\pgfqpoint{1.904724in}{0.911550in}}%
\pgfpathlineto{\pgfqpoint{1.907687in}{0.917616in}}%
\pgfpathlineto{\pgfqpoint{1.910651in}{0.998205in}}%
\pgfpathlineto{\pgfqpoint{1.913614in}{0.907577in}}%
\pgfpathlineto{\pgfqpoint{1.916577in}{1.005296in}}%
\pgfpathlineto{\pgfqpoint{1.919540in}{0.921462in}}%
\pgfpathlineto{\pgfqpoint{1.922504in}{0.895768in}}%
\pgfpathlineto{\pgfqpoint{1.925467in}{0.926495in}}%
\pgfpathlineto{\pgfqpoint{1.928430in}{0.897278in}}%
\pgfpathlineto{\pgfqpoint{1.931393in}{0.906416in}}%
\pgfpathlineto{\pgfqpoint{1.934357in}{0.911849in}}%
\pgfpathlineto{\pgfqpoint{1.937320in}{0.904968in}}%
\pgfpathlineto{\pgfqpoint{1.940283in}{0.992333in}}%
\pgfpathlineto{\pgfqpoint{1.943246in}{0.983650in}}%
\pgfpathlineto{\pgfqpoint{1.946210in}{0.906670in}}%
\pgfpathlineto{\pgfqpoint{1.949173in}{0.940132in}}%
\pgfpathlineto{\pgfqpoint{1.952136in}{0.892591in}}%
\pgfpathlineto{\pgfqpoint{1.955099in}{0.892882in}}%
\pgfpathlineto{\pgfqpoint{1.958063in}{0.925087in}}%
\pgfpathlineto{\pgfqpoint{1.963989in}{0.885308in}}%
\pgfpathlineto{\pgfqpoint{1.966952in}{0.898055in}}%
\pgfpathlineto{\pgfqpoint{1.969916in}{0.966740in}}%
\pgfpathlineto{\pgfqpoint{1.972879in}{0.923841in}}%
\pgfpathlineto{\pgfqpoint{1.975842in}{1.021394in}}%
\pgfpathlineto{\pgfqpoint{1.978805in}{0.931008in}}%
\pgfpathlineto{\pgfqpoint{1.981769in}{0.982787in}}%
\pgfpathlineto{\pgfqpoint{1.984732in}{0.900271in}}%
\pgfpathlineto{\pgfqpoint{1.987695in}{0.882253in}}%
\pgfpathlineto{\pgfqpoint{1.993622in}{0.962840in}}%
\pgfpathlineto{\pgfqpoint{1.996585in}{0.896357in}}%
\pgfpathlineto{\pgfqpoint{1.999548in}{0.889133in}}%
\pgfpathlineto{\pgfqpoint{2.002511in}{1.010893in}}%
\pgfpathlineto{\pgfqpoint{2.005475in}{0.933388in}}%
\pgfpathlineto{\pgfqpoint{2.008438in}{1.050983in}}%
\pgfpathlineto{\pgfqpoint{2.011401in}{0.881486in}}%
\pgfpathlineto{\pgfqpoint{2.014364in}{0.923022in}}%
\pgfpathlineto{\pgfqpoint{2.017328in}{0.907995in}}%
\pgfpathlineto{\pgfqpoint{2.020291in}{0.946466in}}%
\pgfpathlineto{\pgfqpoint{2.023254in}{0.879431in}}%
\pgfpathlineto{\pgfqpoint{2.026217in}{0.869459in}}%
\pgfpathlineto{\pgfqpoint{2.029181in}{0.946175in}}%
\pgfpathlineto{\pgfqpoint{2.032144in}{0.879382in}}%
\pgfpathlineto{\pgfqpoint{2.035107in}{0.929815in}}%
\pgfpathlineto{\pgfqpoint{2.038070in}{0.861792in}}%
\pgfpathlineto{\pgfqpoint{2.041034in}{0.887809in}}%
\pgfpathlineto{\pgfqpoint{2.043997in}{0.892022in}}%
\pgfpathlineto{\pgfqpoint{2.046960in}{0.869333in}}%
\pgfpathlineto{\pgfqpoint{2.049923in}{0.891893in}}%
\pgfpathlineto{\pgfqpoint{2.052887in}{0.879924in}}%
\pgfpathlineto{\pgfqpoint{2.055850in}{0.895604in}}%
\pgfpathlineto{\pgfqpoint{2.058813in}{0.884774in}}%
\pgfpathlineto{\pgfqpoint{2.061776in}{0.883441in}}%
\pgfpathlineto{\pgfqpoint{2.064740in}{0.877904in}}%
\pgfpathlineto{\pgfqpoint{2.067703in}{0.882758in}}%
\pgfpathlineto{\pgfqpoint{2.070666in}{0.899013in}}%
\pgfpathlineto{\pgfqpoint{2.073630in}{0.862653in}}%
\pgfpathlineto{\pgfqpoint{2.076593in}{0.890730in}}%
\pgfpathlineto{\pgfqpoint{2.079556in}{0.932868in}}%
\pgfpathlineto{\pgfqpoint{2.082519in}{0.920433in}}%
\pgfpathlineto{\pgfqpoint{2.085483in}{0.976962in}}%
\pgfpathlineto{\pgfqpoint{2.088446in}{0.906623in}}%
\pgfpathlineto{\pgfqpoint{2.091409in}{1.005442in}}%
\pgfpathlineto{\pgfqpoint{2.094372in}{0.861883in}}%
\pgfpathlineto{\pgfqpoint{2.100299in}{0.895162in}}%
\pgfpathlineto{\pgfqpoint{2.103262in}{1.127515in}}%
\pgfpathlineto{\pgfqpoint{2.109189in}{0.923029in}}%
\pgfpathlineto{\pgfqpoint{2.112152in}{1.023014in}}%
\pgfpathlineto{\pgfqpoint{2.115115in}{0.853858in}}%
\pgfpathlineto{\pgfqpoint{2.118078in}{0.892451in}}%
\pgfpathlineto{\pgfqpoint{2.124005in}{0.870033in}}%
\pgfpathlineto{\pgfqpoint{2.126968in}{0.859661in}}%
\pgfpathlineto{\pgfqpoint{2.129931in}{0.896052in}}%
\pgfpathlineto{\pgfqpoint{2.132895in}{0.861522in}}%
\pgfpathlineto{\pgfqpoint{2.135858in}{0.868909in}}%
\pgfpathlineto{\pgfqpoint{2.138821in}{0.934943in}}%
\pgfpathlineto{\pgfqpoint{2.141784in}{0.841952in}}%
\pgfpathlineto{\pgfqpoint{2.144748in}{0.946778in}}%
\pgfpathlineto{\pgfqpoint{2.147711in}{0.843975in}}%
\pgfpathlineto{\pgfqpoint{2.150674in}{0.852890in}}%
\pgfpathlineto{\pgfqpoint{2.153637in}{0.849984in}}%
\pgfpathlineto{\pgfqpoint{2.156601in}{0.869729in}}%
\pgfpathlineto{\pgfqpoint{2.159564in}{0.900583in}}%
\pgfpathlineto{\pgfqpoint{2.162527in}{0.975045in}}%
\pgfpathlineto{\pgfqpoint{2.165490in}{0.849936in}}%
\pgfpathlineto{\pgfqpoint{2.168454in}{0.922024in}}%
\pgfpathlineto{\pgfqpoint{2.171417in}{0.906589in}}%
\pgfpathlineto{\pgfqpoint{2.174380in}{0.847666in}}%
\pgfpathlineto{\pgfqpoint{2.177343in}{0.843445in}}%
\pgfpathlineto{\pgfqpoint{2.180307in}{0.849315in}}%
\pgfpathlineto{\pgfqpoint{2.183270in}{0.842361in}}%
\pgfpathlineto{\pgfqpoint{2.186233in}{0.888011in}}%
\pgfpathlineto{\pgfqpoint{2.189196in}{0.914211in}}%
\pgfpathlineto{\pgfqpoint{2.195123in}{0.859884in}}%
\pgfpathlineto{\pgfqpoint{2.198086in}{0.916710in}}%
\pgfpathlineto{\pgfqpoint{2.201049in}{0.857406in}}%
\pgfpathlineto{\pgfqpoint{2.204013in}{0.839340in}}%
\pgfpathlineto{\pgfqpoint{2.206976in}{0.865911in}}%
\pgfpathlineto{\pgfqpoint{2.209939in}{0.880501in}}%
\pgfpathlineto{\pgfqpoint{2.212902in}{0.836432in}}%
\pgfpathlineto{\pgfqpoint{2.215866in}{0.845220in}}%
\pgfpathlineto{\pgfqpoint{2.218829in}{0.892705in}}%
\pgfpathlineto{\pgfqpoint{2.221792in}{0.894042in}}%
\pgfpathlineto{\pgfqpoint{2.224755in}{0.938899in}}%
\pgfpathlineto{\pgfqpoint{2.227719in}{0.844378in}}%
\pgfpathlineto{\pgfqpoint{2.230682in}{0.842396in}}%
\pgfpathlineto{\pgfqpoint{2.233645in}{0.996610in}}%
\pgfpathlineto{\pgfqpoint{2.236608in}{0.935806in}}%
\pgfpathlineto{\pgfqpoint{2.239572in}{0.819961in}}%
\pgfpathlineto{\pgfqpoint{2.242535in}{0.834307in}}%
\pgfpathlineto{\pgfqpoint{2.245498in}{0.836879in}}%
\pgfpathlineto{\pgfqpoint{2.248461in}{0.874824in}}%
\pgfpathlineto{\pgfqpoint{2.251425in}{0.833463in}}%
\pgfpathlineto{\pgfqpoint{2.254388in}{0.829698in}}%
\pgfpathlineto{\pgfqpoint{2.257351in}{0.853803in}}%
\pgfpathlineto{\pgfqpoint{2.260314in}{0.841482in}}%
\pgfpathlineto{\pgfqpoint{2.263278in}{0.878329in}}%
\pgfpathlineto{\pgfqpoint{2.266241in}{0.838581in}}%
\pgfpathlineto{\pgfqpoint{2.269204in}{0.837095in}}%
\pgfpathlineto{\pgfqpoint{2.272167in}{0.911644in}}%
\pgfpathlineto{\pgfqpoint{2.275131in}{0.902972in}}%
\pgfpathlineto{\pgfqpoint{2.278094in}{0.861654in}}%
\pgfpathlineto{\pgfqpoint{2.281057in}{0.856190in}}%
\pgfpathlineto{\pgfqpoint{2.284020in}{0.825831in}}%
\pgfpathlineto{\pgfqpoint{2.286984in}{0.833782in}}%
\pgfpathlineto{\pgfqpoint{2.289947in}{0.867264in}}%
\pgfpathlineto{\pgfqpoint{2.292910in}{0.980512in}}%
\pgfpathlineto{\pgfqpoint{2.295874in}{0.831649in}}%
\pgfpathlineto{\pgfqpoint{2.298837in}{0.832713in}}%
\pgfpathlineto{\pgfqpoint{2.301800in}{0.876478in}}%
\pgfpathlineto{\pgfqpoint{2.304763in}{0.839616in}}%
\pgfpathlineto{\pgfqpoint{2.310690in}{0.892309in}}%
\pgfpathlineto{\pgfqpoint{2.313653in}{0.827765in}}%
\pgfpathlineto{\pgfqpoint{2.316616in}{0.852503in}}%
\pgfpathlineto{\pgfqpoint{2.319580in}{0.857502in}}%
\pgfpathlineto{\pgfqpoint{2.322543in}{0.844238in}}%
\pgfpathlineto{\pgfqpoint{2.325506in}{0.872007in}}%
\pgfpathlineto{\pgfqpoint{2.328469in}{0.883479in}}%
\pgfpathlineto{\pgfqpoint{2.331433in}{0.824868in}}%
\pgfpathlineto{\pgfqpoint{2.334396in}{0.871324in}}%
\pgfpathlineto{\pgfqpoint{2.337359in}{0.840355in}}%
\pgfpathlineto{\pgfqpoint{2.340322in}{0.826618in}}%
\pgfpathlineto{\pgfqpoint{2.343286in}{0.882577in}}%
\pgfpathlineto{\pgfqpoint{2.346249in}{0.904532in}}%
\pgfpathlineto{\pgfqpoint{2.349212in}{0.842307in}}%
\pgfpathlineto{\pgfqpoint{2.352175in}{0.944078in}}%
\pgfpathlineto{\pgfqpoint{2.355139in}{0.858146in}}%
\pgfpathlineto{\pgfqpoint{2.358102in}{0.875049in}}%
\pgfpathlineto{\pgfqpoint{2.361065in}{0.830899in}}%
\pgfpathlineto{\pgfqpoint{2.364028in}{0.907658in}}%
\pgfpathlineto{\pgfqpoint{2.366992in}{0.841014in}}%
\pgfpathlineto{\pgfqpoint{2.369955in}{0.861653in}}%
\pgfpathlineto{\pgfqpoint{2.372918in}{0.816033in}}%
\pgfpathlineto{\pgfqpoint{2.375881in}{0.836667in}}%
\pgfpathlineto{\pgfqpoint{2.378845in}{0.881050in}}%
\pgfpathlineto{\pgfqpoint{2.381808in}{0.821748in}}%
\pgfpathlineto{\pgfqpoint{2.384771in}{0.950930in}}%
\pgfpathlineto{\pgfqpoint{2.387734in}{0.891347in}}%
\pgfpathlineto{\pgfqpoint{2.390698in}{0.896768in}}%
\pgfpathlineto{\pgfqpoint{2.393661in}{0.850219in}}%
\pgfpathlineto{\pgfqpoint{2.396624in}{0.838034in}}%
\pgfpathlineto{\pgfqpoint{2.399587in}{0.900377in}}%
\pgfpathlineto{\pgfqpoint{2.402551in}{0.810360in}}%
\pgfpathlineto{\pgfqpoint{2.405514in}{0.885912in}}%
\pgfpathlineto{\pgfqpoint{2.408477in}{0.801483in}}%
\pgfpathlineto{\pgfqpoint{2.411440in}{0.799461in}}%
\pgfpathlineto{\pgfqpoint{2.414404in}{0.868333in}}%
\pgfpathlineto{\pgfqpoint{2.417367in}{0.833584in}}%
\pgfpathlineto{\pgfqpoint{2.420330in}{0.921262in}}%
\pgfpathlineto{\pgfqpoint{2.423293in}{0.835028in}}%
\pgfpathlineto{\pgfqpoint{2.426257in}{0.829398in}}%
\pgfpathlineto{\pgfqpoint{2.429220in}{0.832597in}}%
\pgfpathlineto{\pgfqpoint{2.435146in}{0.801038in}}%
\pgfpathlineto{\pgfqpoint{2.438110in}{0.948898in}}%
\pgfpathlineto{\pgfqpoint{2.441073in}{0.834850in}}%
\pgfpathlineto{\pgfqpoint{2.444036in}{0.913835in}}%
\pgfpathlineto{\pgfqpoint{2.446999in}{0.797849in}}%
\pgfpathlineto{\pgfqpoint{2.449963in}{0.825110in}}%
\pgfpathlineto{\pgfqpoint{2.452926in}{0.904849in}}%
\pgfpathlineto{\pgfqpoint{2.455889in}{0.838127in}}%
\pgfpathlineto{\pgfqpoint{2.458852in}{0.819021in}}%
\pgfpathlineto{\pgfqpoint{2.461816in}{0.892193in}}%
\pgfpathlineto{\pgfqpoint{2.467742in}{0.797614in}}%
\pgfpathlineto{\pgfqpoint{2.470705in}{0.855912in}}%
\pgfpathlineto{\pgfqpoint{2.473669in}{0.846542in}}%
\pgfpathlineto{\pgfqpoint{2.476632in}{0.933058in}}%
\pgfpathlineto{\pgfqpoint{2.479595in}{0.804531in}}%
\pgfpathlineto{\pgfqpoint{2.482558in}{0.800666in}}%
\pgfpathlineto{\pgfqpoint{2.485522in}{0.884276in}}%
\pgfpathlineto{\pgfqpoint{2.488485in}{0.799664in}}%
\pgfpathlineto{\pgfqpoint{2.491448in}{0.803661in}}%
\pgfpathlineto{\pgfqpoint{2.494411in}{0.819699in}}%
\pgfpathlineto{\pgfqpoint{2.500338in}{0.878263in}}%
\pgfpathlineto{\pgfqpoint{2.503301in}{0.796682in}}%
\pgfpathlineto{\pgfqpoint{2.506264in}{0.794339in}}%
\pgfpathlineto{\pgfqpoint{2.509228in}{0.806318in}}%
\pgfpathlineto{\pgfqpoint{2.512191in}{0.919784in}}%
\pgfpathlineto{\pgfqpoint{2.515154in}{0.819620in}}%
\pgfpathlineto{\pgfqpoint{2.518117in}{0.898229in}}%
\pgfpathlineto{\pgfqpoint{2.521081in}{0.798836in}}%
\pgfpathlineto{\pgfqpoint{2.524044in}{0.811391in}}%
\pgfpathlineto{\pgfqpoint{2.527007in}{0.803753in}}%
\pgfpathlineto{\pgfqpoint{2.529971in}{0.826247in}}%
\pgfpathlineto{\pgfqpoint{2.532934in}{0.877295in}}%
\pgfpathlineto{\pgfqpoint{2.535897in}{0.802312in}}%
\pgfpathlineto{\pgfqpoint{2.538860in}{0.795524in}}%
\pgfpathlineto{\pgfqpoint{2.541824in}{0.840432in}}%
\pgfpathlineto{\pgfqpoint{2.544787in}{0.816486in}}%
\pgfpathlineto{\pgfqpoint{2.547750in}{0.865634in}}%
\pgfpathlineto{\pgfqpoint{2.550713in}{0.948945in}}%
\pgfpathlineto{\pgfqpoint{2.553677in}{0.811450in}}%
\pgfpathlineto{\pgfqpoint{2.556640in}{0.803218in}}%
\pgfpathlineto{\pgfqpoint{2.559603in}{0.862706in}}%
\pgfpathlineto{\pgfqpoint{2.562566in}{0.841049in}}%
\pgfpathlineto{\pgfqpoint{2.565530in}{0.843286in}}%
\pgfpathlineto{\pgfqpoint{2.568493in}{0.788392in}}%
\pgfpathlineto{\pgfqpoint{2.571456in}{0.784396in}}%
\pgfpathlineto{\pgfqpoint{2.574419in}{0.902053in}}%
\pgfpathlineto{\pgfqpoint{2.577383in}{0.813457in}}%
\pgfpathlineto{\pgfqpoint{2.580346in}{1.032413in}}%
\pgfpathlineto{\pgfqpoint{2.583309in}{0.778214in}}%
\pgfpathlineto{\pgfqpoint{2.586272in}{0.803672in}}%
\pgfpathlineto{\pgfqpoint{2.592199in}{0.787736in}}%
\pgfpathlineto{\pgfqpoint{2.598125in}{0.896979in}}%
\pgfpathlineto{\pgfqpoint{2.601089in}{0.911298in}}%
\pgfpathlineto{\pgfqpoint{2.604052in}{0.793776in}}%
\pgfpathlineto{\pgfqpoint{2.607015in}{0.794071in}}%
\pgfpathlineto{\pgfqpoint{2.609978in}{0.776194in}}%
\pgfpathlineto{\pgfqpoint{2.612942in}{0.924414in}}%
\pgfpathlineto{\pgfqpoint{2.615905in}{0.837150in}}%
\pgfpathlineto{\pgfqpoint{2.618868in}{0.856804in}}%
\pgfpathlineto{\pgfqpoint{2.621831in}{0.852880in}}%
\pgfpathlineto{\pgfqpoint{2.624795in}{0.861542in}}%
\pgfpathlineto{\pgfqpoint{2.627758in}{0.864733in}}%
\pgfpathlineto{\pgfqpoint{2.630721in}{0.772426in}}%
\pgfpathlineto{\pgfqpoint{2.633684in}{0.815976in}}%
\pgfpathlineto{\pgfqpoint{2.636648in}{0.775486in}}%
\pgfpathlineto{\pgfqpoint{2.642574in}{0.807162in}}%
\pgfpathlineto{\pgfqpoint{2.645537in}{0.772823in}}%
\pgfpathlineto{\pgfqpoint{2.648501in}{0.825635in}}%
\pgfpathlineto{\pgfqpoint{2.651464in}{0.828379in}}%
\pgfpathlineto{\pgfqpoint{2.654427in}{0.766670in}}%
\pgfpathlineto{\pgfqpoint{2.657390in}{0.874982in}}%
\pgfpathlineto{\pgfqpoint{2.660354in}{0.856316in}}%
\pgfpathlineto{\pgfqpoint{2.663317in}{0.772417in}}%
\pgfpathlineto{\pgfqpoint{2.666280in}{0.791284in}}%
\pgfpathlineto{\pgfqpoint{2.669243in}{0.771813in}}%
\pgfpathlineto{\pgfqpoint{2.675170in}{0.841167in}}%
\pgfpathlineto{\pgfqpoint{2.678133in}{0.786226in}}%
\pgfpathlineto{\pgfqpoint{2.681096in}{0.861029in}}%
\pgfpathlineto{\pgfqpoint{2.684060in}{0.801548in}}%
\pgfpathlineto{\pgfqpoint{2.687023in}{0.772739in}}%
\pgfpathlineto{\pgfqpoint{2.692949in}{0.852947in}}%
\pgfpathlineto{\pgfqpoint{2.695913in}{0.867587in}}%
\pgfpathlineto{\pgfqpoint{2.698876in}{0.792070in}}%
\pgfpathlineto{\pgfqpoint{2.701839in}{0.867696in}}%
\pgfpathlineto{\pgfqpoint{2.704802in}{0.774268in}}%
\pgfpathlineto{\pgfqpoint{2.710729in}{0.762470in}}%
\pgfpathlineto{\pgfqpoint{2.713692in}{0.776454in}}%
\pgfpathlineto{\pgfqpoint{2.716655in}{0.774780in}}%
\pgfpathlineto{\pgfqpoint{2.719619in}{0.861718in}}%
\pgfpathlineto{\pgfqpoint{2.722582in}{0.823346in}}%
\pgfpathlineto{\pgfqpoint{2.725545in}{0.864783in}}%
\pgfpathlineto{\pgfqpoint{2.728508in}{0.782435in}}%
\pgfpathlineto{\pgfqpoint{2.731472in}{0.924746in}}%
\pgfpathlineto{\pgfqpoint{2.734435in}{0.807115in}}%
\pgfpathlineto{\pgfqpoint{2.737398in}{0.775835in}}%
\pgfpathlineto{\pgfqpoint{2.740361in}{0.812036in}}%
\pgfpathlineto{\pgfqpoint{2.743325in}{0.786183in}}%
\pgfpathlineto{\pgfqpoint{2.746288in}{0.923854in}}%
\pgfpathlineto{\pgfqpoint{2.749251in}{0.853360in}}%
\pgfpathlineto{\pgfqpoint{2.752215in}{0.865681in}}%
\pgfpathlineto{\pgfqpoint{2.755178in}{0.782789in}}%
\pgfpathlineto{\pgfqpoint{2.758141in}{0.807395in}}%
\pgfpathlineto{\pgfqpoint{2.764068in}{0.783096in}}%
\pgfpathlineto{\pgfqpoint{2.767031in}{0.798173in}}%
\pgfpathlineto{\pgfqpoint{2.769994in}{0.915065in}}%
\pgfpathlineto{\pgfqpoint{2.772957in}{0.901851in}}%
\pgfpathlineto{\pgfqpoint{2.775921in}{0.999567in}}%
\pgfpathlineto{\pgfqpoint{2.778884in}{0.765978in}}%
\pgfpathlineto{\pgfqpoint{2.781847in}{0.843510in}}%
\pgfpathlineto{\pgfqpoint{2.784810in}{0.760872in}}%
\pgfpathlineto{\pgfqpoint{2.787774in}{0.781554in}}%
\pgfpathlineto{\pgfqpoint{2.790737in}{0.853566in}}%
\pgfpathlineto{\pgfqpoint{2.793700in}{0.875789in}}%
\pgfpathlineto{\pgfqpoint{2.796663in}{0.799398in}}%
\pgfpathlineto{\pgfqpoint{2.799627in}{0.783787in}}%
\pgfpathlineto{\pgfqpoint{2.802590in}{0.754136in}}%
\pgfpathlineto{\pgfqpoint{2.805553in}{0.749713in}}%
\pgfpathlineto{\pgfqpoint{2.811480in}{0.841068in}}%
\pgfpathlineto{\pgfqpoint{2.814443in}{0.826804in}}%
\pgfpathlineto{\pgfqpoint{2.817406in}{0.772302in}}%
\pgfpathlineto{\pgfqpoint{2.820369in}{0.855413in}}%
\pgfpathlineto{\pgfqpoint{2.823333in}{0.791531in}}%
\pgfpathlineto{\pgfqpoint{2.826296in}{0.762184in}}%
\pgfpathlineto{\pgfqpoint{2.829259in}{0.759075in}}%
\pgfpathlineto{\pgfqpoint{2.832222in}{0.761018in}}%
\pgfpathlineto{\pgfqpoint{2.835186in}{0.870194in}}%
\pgfpathlineto{\pgfqpoint{2.841112in}{0.780982in}}%
\pgfpathlineto{\pgfqpoint{2.844075in}{0.830752in}}%
\pgfpathlineto{\pgfqpoint{2.847039in}{0.762541in}}%
\pgfpathlineto{\pgfqpoint{2.850002in}{0.755374in}}%
\pgfpathlineto{\pgfqpoint{2.852965in}{0.792254in}}%
\pgfpathlineto{\pgfqpoint{2.855928in}{0.770117in}}%
\pgfpathlineto{\pgfqpoint{2.858892in}{0.795796in}}%
\pgfpathlineto{\pgfqpoint{2.861855in}{0.961613in}}%
\pgfpathlineto{\pgfqpoint{2.864818in}{0.826806in}}%
\pgfpathlineto{\pgfqpoint{2.867781in}{0.780361in}}%
\pgfpathlineto{\pgfqpoint{2.870745in}{0.831616in}}%
\pgfpathlineto{\pgfqpoint{2.873708in}{0.972560in}}%
\pgfpathlineto{\pgfqpoint{2.876671in}{0.751564in}}%
\pgfpathlineto{\pgfqpoint{2.882598in}{0.746873in}}%
\pgfpathlineto{\pgfqpoint{2.885561in}{0.785721in}}%
\pgfpathlineto{\pgfqpoint{2.888524in}{0.760085in}}%
\pgfpathlineto{\pgfqpoint{2.891487in}{0.831560in}}%
\pgfpathlineto{\pgfqpoint{2.894451in}{0.784502in}}%
\pgfpathlineto{\pgfqpoint{2.897414in}{0.806685in}}%
\pgfpathlineto{\pgfqpoint{2.900377in}{0.766106in}}%
\pgfpathlineto{\pgfqpoint{2.903340in}{0.750279in}}%
\pgfpathlineto{\pgfqpoint{2.906304in}{0.789409in}}%
\pgfpathlineto{\pgfqpoint{2.909267in}{0.807659in}}%
\pgfpathlineto{\pgfqpoint{2.912230in}{0.884469in}}%
\pgfpathlineto{\pgfqpoint{2.915193in}{0.803889in}}%
\pgfpathlineto{\pgfqpoint{2.921120in}{0.724587in}}%
\pgfpathlineto{\pgfqpoint{2.924083in}{0.747957in}}%
\pgfpathlineto{\pgfqpoint{2.927046in}{0.739292in}}%
\pgfpathlineto{\pgfqpoint{2.932973in}{0.794012in}}%
\pgfpathlineto{\pgfqpoint{2.935936in}{0.750705in}}%
\pgfpathlineto{\pgfqpoint{2.938899in}{0.769423in}}%
\pgfpathlineto{\pgfqpoint{2.941863in}{0.739652in}}%
\pgfpathlineto{\pgfqpoint{2.944826in}{0.821028in}}%
\pgfpathlineto{\pgfqpoint{2.947789in}{0.859205in}}%
\pgfpathlineto{\pgfqpoint{2.950752in}{0.777113in}}%
\pgfpathlineto{\pgfqpoint{2.953716in}{0.765859in}}%
\pgfpathlineto{\pgfqpoint{2.956679in}{0.769044in}}%
\pgfpathlineto{\pgfqpoint{2.959642in}{0.743030in}}%
\pgfpathlineto{\pgfqpoint{2.962605in}{0.826982in}}%
\pgfpathlineto{\pgfqpoint{2.965569in}{0.735661in}}%
\pgfpathlineto{\pgfqpoint{2.968532in}{0.759340in}}%
\pgfpathlineto{\pgfqpoint{2.971495in}{0.807333in}}%
\pgfpathlineto{\pgfqpoint{2.974459in}{0.782254in}}%
\pgfpathlineto{\pgfqpoint{2.977422in}{0.739595in}}%
\pgfpathlineto{\pgfqpoint{2.980385in}{0.730952in}}%
\pgfpathlineto{\pgfqpoint{2.983348in}{0.752678in}}%
\pgfpathlineto{\pgfqpoint{2.986312in}{0.742307in}}%
\pgfpathlineto{\pgfqpoint{2.989275in}{0.722331in}}%
\pgfpathlineto{\pgfqpoint{2.992238in}{0.756392in}}%
\pgfpathlineto{\pgfqpoint{2.995201in}{0.879371in}}%
\pgfpathlineto{\pgfqpoint{2.998165in}{0.752019in}}%
\pgfpathlineto{\pgfqpoint{3.001128in}{0.796623in}}%
\pgfpathlineto{\pgfqpoint{3.004091in}{0.940820in}}%
\pgfpathlineto{\pgfqpoint{3.010018in}{0.805352in}}%
\pgfpathlineto{\pgfqpoint{3.012981in}{0.789119in}}%
\pgfpathlineto{\pgfqpoint{3.015944in}{0.895978in}}%
\pgfpathlineto{\pgfqpoint{3.021871in}{0.735041in}}%
\pgfpathlineto{\pgfqpoint{3.024834in}{0.732811in}}%
\pgfpathlineto{\pgfqpoint{3.027797in}{0.728566in}}%
\pgfpathlineto{\pgfqpoint{3.030760in}{0.805934in}}%
\pgfpathlineto{\pgfqpoint{3.033724in}{0.721067in}}%
\pgfpathlineto{\pgfqpoint{3.036687in}{0.766114in}}%
\pgfpathlineto{\pgfqpoint{3.039650in}{0.790726in}}%
\pgfpathlineto{\pgfqpoint{3.042613in}{0.752414in}}%
\pgfpathlineto{\pgfqpoint{3.045577in}{0.770149in}}%
\pgfpathlineto{\pgfqpoint{3.048540in}{0.753856in}}%
\pgfpathlineto{\pgfqpoint{3.054466in}{0.765846in}}%
\pgfpathlineto{\pgfqpoint{3.057430in}{0.792548in}}%
\pgfpathlineto{\pgfqpoint{3.060393in}{0.744660in}}%
\pgfpathlineto{\pgfqpoint{3.063356in}{0.751795in}}%
\pgfpathlineto{\pgfqpoint{3.066319in}{0.868073in}}%
\pgfpathlineto{\pgfqpoint{3.069283in}{0.715424in}}%
\pgfpathlineto{\pgfqpoint{3.072246in}{0.754901in}}%
\pgfpathlineto{\pgfqpoint{3.075209in}{0.748366in}}%
\pgfpathlineto{\pgfqpoint{3.078172in}{0.755352in}}%
\pgfpathlineto{\pgfqpoint{3.081136in}{0.976880in}}%
\pgfpathlineto{\pgfqpoint{3.084099in}{0.761887in}}%
\pgfpathlineto{\pgfqpoint{3.087062in}{0.740414in}}%
\pgfpathlineto{\pgfqpoint{3.090025in}{0.742364in}}%
\pgfpathlineto{\pgfqpoint{3.092989in}{0.750291in}}%
\pgfpathlineto{\pgfqpoint{3.095952in}{0.724523in}}%
\pgfpathlineto{\pgfqpoint{3.098915in}{0.818523in}}%
\pgfpathlineto{\pgfqpoint{3.101878in}{0.765859in}}%
\pgfpathlineto{\pgfqpoint{3.104842in}{0.785609in}}%
\pgfpathlineto{\pgfqpoint{3.107805in}{0.751275in}}%
\pgfpathlineto{\pgfqpoint{3.110768in}{0.772144in}}%
\pgfpathlineto{\pgfqpoint{3.113731in}{0.887256in}}%
\pgfpathlineto{\pgfqpoint{3.116695in}{0.730034in}}%
\pgfpathlineto{\pgfqpoint{3.119658in}{0.723731in}}%
\pgfpathlineto{\pgfqpoint{3.122621in}{0.720446in}}%
\pgfpathlineto{\pgfqpoint{3.125584in}{0.707706in}}%
\pgfpathlineto{\pgfqpoint{3.128548in}{0.740600in}}%
\pgfpathlineto{\pgfqpoint{3.131511in}{0.814131in}}%
\pgfpathlineto{\pgfqpoint{3.134474in}{0.730875in}}%
\pgfpathlineto{\pgfqpoint{3.137437in}{0.840373in}}%
\pgfpathlineto{\pgfqpoint{3.143364in}{0.724555in}}%
\pgfpathlineto{\pgfqpoint{3.146327in}{0.705072in}}%
\pgfpathlineto{\pgfqpoint{3.149290in}{0.713377in}}%
\pgfpathlineto{\pgfqpoint{3.152254in}{0.714100in}}%
\pgfpathlineto{\pgfqpoint{3.155217in}{0.746298in}}%
\pgfpathlineto{\pgfqpoint{3.158180in}{0.720249in}}%
\pgfpathlineto{\pgfqpoint{3.161143in}{0.719121in}}%
\pgfpathlineto{\pgfqpoint{3.164107in}{0.758608in}}%
\pgfpathlineto{\pgfqpoint{3.167070in}{0.749724in}}%
\pgfpathlineto{\pgfqpoint{3.170033in}{0.747692in}}%
\pgfpathlineto{\pgfqpoint{3.172996in}{0.721448in}}%
\pgfpathlineto{\pgfqpoint{3.175960in}{0.720908in}}%
\pgfpathlineto{\pgfqpoint{3.178923in}{0.706383in}}%
\pgfpathlineto{\pgfqpoint{3.181886in}{0.718499in}}%
\pgfpathlineto{\pgfqpoint{3.184849in}{0.717133in}}%
\pgfpathlineto{\pgfqpoint{3.187813in}{0.717245in}}%
\pgfpathlineto{\pgfqpoint{3.190776in}{0.696169in}}%
\pgfpathlineto{\pgfqpoint{3.193739in}{0.720869in}}%
\pgfpathlineto{\pgfqpoint{3.196702in}{0.731861in}}%
\pgfpathlineto{\pgfqpoint{3.199666in}{0.764563in}}%
\pgfpathlineto{\pgfqpoint{3.202629in}{0.716566in}}%
\pgfpathlineto{\pgfqpoint{3.205592in}{0.707916in}}%
\pgfpathlineto{\pgfqpoint{3.208556in}{0.854308in}}%
\pgfpathlineto{\pgfqpoint{3.211519in}{0.698197in}}%
\pgfpathlineto{\pgfqpoint{3.214482in}{0.712894in}}%
\pgfpathlineto{\pgfqpoint{3.217445in}{0.735287in}}%
\pgfpathlineto{\pgfqpoint{3.220409in}{0.727545in}}%
\pgfpathlineto{\pgfqpoint{3.223372in}{0.826542in}}%
\pgfpathlineto{\pgfqpoint{3.226335in}{0.696602in}}%
\pgfpathlineto{\pgfqpoint{3.229298in}{0.754409in}}%
\pgfpathlineto{\pgfqpoint{3.232262in}{0.749341in}}%
\pgfpathlineto{\pgfqpoint{3.235225in}{0.716142in}}%
\pgfpathlineto{\pgfqpoint{3.238188in}{0.844701in}}%
\pgfpathlineto{\pgfqpoint{3.241151in}{0.707659in}}%
\pgfpathlineto{\pgfqpoint{3.247078in}{0.828581in}}%
\pgfpathlineto{\pgfqpoint{3.250041in}{0.757984in}}%
\pgfpathlineto{\pgfqpoint{3.255968in}{0.726815in}}%
\pgfpathlineto{\pgfqpoint{3.258931in}{0.719527in}}%
\pgfpathlineto{\pgfqpoint{3.261894in}{0.708871in}}%
\pgfpathlineto{\pgfqpoint{3.264857in}{0.755505in}}%
\pgfpathlineto{\pgfqpoint{3.267821in}{0.689537in}}%
\pgfpathlineto{\pgfqpoint{3.270784in}{0.707159in}}%
\pgfpathlineto{\pgfqpoint{3.273747in}{0.824965in}}%
\pgfpathlineto{\pgfqpoint{3.276710in}{0.713501in}}%
\pgfpathlineto{\pgfqpoint{3.279674in}{0.720420in}}%
\pgfpathlineto{\pgfqpoint{3.282637in}{0.835778in}}%
\pgfpathlineto{\pgfqpoint{3.285600in}{0.728519in}}%
\pgfpathlineto{\pgfqpoint{3.288563in}{0.716110in}}%
\pgfpathlineto{\pgfqpoint{3.291527in}{0.796970in}}%
\pgfpathlineto{\pgfqpoint{3.297453in}{0.694131in}}%
\pgfpathlineto{\pgfqpoint{3.300416in}{0.720785in}}%
\pgfpathlineto{\pgfqpoint{3.303380in}{0.707283in}}%
\pgfpathlineto{\pgfqpoint{3.306343in}{0.788843in}}%
\pgfpathlineto{\pgfqpoint{3.309306in}{0.743687in}}%
\pgfpathlineto{\pgfqpoint{3.312269in}{0.678560in}}%
\pgfpathlineto{\pgfqpoint{3.315233in}{0.691705in}}%
\pgfpathlineto{\pgfqpoint{3.318196in}{0.737289in}}%
\pgfpathlineto{\pgfqpoint{3.321159in}{0.743053in}}%
\pgfpathlineto{\pgfqpoint{3.324122in}{0.734779in}}%
\pgfpathlineto{\pgfqpoint{3.327086in}{0.734997in}}%
\pgfpathlineto{\pgfqpoint{3.330049in}{0.731746in}}%
\pgfpathlineto{\pgfqpoint{3.333012in}{0.709220in}}%
\pgfpathlineto{\pgfqpoint{3.335975in}{0.696838in}}%
\pgfpathlineto{\pgfqpoint{3.338939in}{0.742907in}}%
\pgfpathlineto{\pgfqpoint{3.341902in}{0.719854in}}%
\pgfpathlineto{\pgfqpoint{3.344865in}{0.818048in}}%
\pgfpathlineto{\pgfqpoint{3.347828in}{0.711233in}}%
\pgfpathlineto{\pgfqpoint{3.350792in}{0.786206in}}%
\pgfpathlineto{\pgfqpoint{3.353755in}{0.676681in}}%
\pgfpathlineto{\pgfqpoint{3.356718in}{0.741604in}}%
\pgfpathlineto{\pgfqpoint{3.359681in}{0.699861in}}%
\pgfpathlineto{\pgfqpoint{3.362645in}{0.686774in}}%
\pgfpathlineto{\pgfqpoint{3.365608in}{0.710159in}}%
\pgfpathlineto{\pgfqpoint{3.371534in}{0.708797in}}%
\pgfpathlineto{\pgfqpoint{3.374498in}{0.692215in}}%
\pgfpathlineto{\pgfqpoint{3.377461in}{0.694491in}}%
\pgfpathlineto{\pgfqpoint{3.380424in}{0.738993in}}%
\pgfpathlineto{\pgfqpoint{3.383387in}{0.732581in}}%
\pgfpathlineto{\pgfqpoint{3.386351in}{0.746734in}}%
\pgfpathlineto{\pgfqpoint{3.389314in}{0.692748in}}%
\pgfpathlineto{\pgfqpoint{3.392277in}{0.675716in}}%
\pgfpathlineto{\pgfqpoint{3.395240in}{0.811777in}}%
\pgfpathlineto{\pgfqpoint{3.398204in}{0.806774in}}%
\pgfpathlineto{\pgfqpoint{3.401167in}{0.710902in}}%
\pgfpathlineto{\pgfqpoint{3.404130in}{0.683102in}}%
\pgfpathlineto{\pgfqpoint{3.407093in}{0.701855in}}%
\pgfpathlineto{\pgfqpoint{3.410057in}{0.745937in}}%
\pgfpathlineto{\pgfqpoint{3.413020in}{0.710010in}}%
\pgfpathlineto{\pgfqpoint{3.415983in}{0.812218in}}%
\pgfpathlineto{\pgfqpoint{3.418946in}{0.828269in}}%
\pgfpathlineto{\pgfqpoint{3.421910in}{0.777722in}}%
\pgfpathlineto{\pgfqpoint{3.424873in}{0.688831in}}%
\pgfpathlineto{\pgfqpoint{3.427836in}{0.691691in}}%
\pgfpathlineto{\pgfqpoint{3.430800in}{0.735740in}}%
\pgfpathlineto{\pgfqpoint{3.433763in}{0.804310in}}%
\pgfpathlineto{\pgfqpoint{3.436726in}{0.780284in}}%
\pgfpathlineto{\pgfqpoint{3.439689in}{0.685517in}}%
\pgfpathlineto{\pgfqpoint{3.442653in}{0.682295in}}%
\pgfpathlineto{\pgfqpoint{3.445616in}{0.740127in}}%
\pgfpathlineto{\pgfqpoint{3.448579in}{0.735512in}}%
\pgfpathlineto{\pgfqpoint{3.451542in}{0.718140in}}%
\pgfpathlineto{\pgfqpoint{3.454506in}{0.795116in}}%
\pgfpathlineto{\pgfqpoint{3.457469in}{0.678490in}}%
\pgfpathlineto{\pgfqpoint{3.460432in}{0.677753in}}%
\pgfpathlineto{\pgfqpoint{3.463395in}{0.658980in}}%
\pgfpathlineto{\pgfqpoint{3.466359in}{0.809281in}}%
\pgfpathlineto{\pgfqpoint{3.469322in}{0.666813in}}%
\pgfpathlineto{\pgfqpoint{3.472285in}{0.734182in}}%
\pgfpathlineto{\pgfqpoint{3.475248in}{0.737139in}}%
\pgfpathlineto{\pgfqpoint{3.478212in}{0.704871in}}%
\pgfpathlineto{\pgfqpoint{3.481175in}{0.770174in}}%
\pgfpathlineto{\pgfqpoint{3.484138in}{0.675351in}}%
\pgfpathlineto{\pgfqpoint{3.487101in}{0.726350in}}%
\pgfpathlineto{\pgfqpoint{3.490065in}{0.728327in}}%
\pgfpathlineto{\pgfqpoint{3.493028in}{0.700106in}}%
\pgfpathlineto{\pgfqpoint{3.495991in}{0.691191in}}%
\pgfpathlineto{\pgfqpoint{3.498954in}{0.686251in}}%
\pgfpathlineto{\pgfqpoint{3.501918in}{0.678805in}}%
\pgfpathlineto{\pgfqpoint{3.504881in}{0.779180in}}%
\pgfpathlineto{\pgfqpoint{3.507844in}{0.728124in}}%
\pgfpathlineto{\pgfqpoint{3.510807in}{0.721255in}}%
\pgfpathlineto{\pgfqpoint{3.513771in}{0.708793in}}%
\pgfpathlineto{\pgfqpoint{3.516734in}{0.680211in}}%
\pgfpathlineto{\pgfqpoint{3.519697in}{0.774674in}}%
\pgfpathlineto{\pgfqpoint{3.522660in}{0.724882in}}%
\pgfpathlineto{\pgfqpoint{3.525624in}{0.811257in}}%
\pgfpathlineto{\pgfqpoint{3.528587in}{0.687618in}}%
\pgfpathlineto{\pgfqpoint{3.531550in}{0.689910in}}%
\pgfpathlineto{\pgfqpoint{3.534513in}{0.694748in}}%
\pgfpathlineto{\pgfqpoint{3.537477in}{0.678906in}}%
\pgfpathlineto{\pgfqpoint{3.540440in}{0.732339in}}%
\pgfpathlineto{\pgfqpoint{3.543403in}{0.708107in}}%
\pgfpathlineto{\pgfqpoint{3.546366in}{0.846777in}}%
\pgfpathlineto{\pgfqpoint{3.549330in}{0.668798in}}%
\pgfpathlineto{\pgfqpoint{3.552293in}{0.754584in}}%
\pgfpathlineto{\pgfqpoint{3.555256in}{0.648552in}}%
\pgfpathlineto{\pgfqpoint{3.558219in}{0.715523in}}%
\pgfpathlineto{\pgfqpoint{3.561183in}{0.630988in}}%
\pgfpathlineto{\pgfqpoint{3.564146in}{0.870502in}}%
\pgfpathlineto{\pgfqpoint{3.567109in}{0.813143in}}%
\pgfpathlineto{\pgfqpoint{3.570072in}{0.911183in}}%
\pgfpathlineto{\pgfqpoint{3.573036in}{0.706532in}}%
\pgfpathlineto{\pgfqpoint{3.575999in}{0.700115in}}%
\pgfpathlineto{\pgfqpoint{3.578962in}{0.666172in}}%
\pgfpathlineto{\pgfqpoint{3.581925in}{0.695711in}}%
\pgfpathlineto{\pgfqpoint{3.584889in}{0.701519in}}%
\pgfpathlineto{\pgfqpoint{3.587852in}{0.711016in}}%
\pgfpathlineto{\pgfqpoint{3.590815in}{0.740293in}}%
\pgfpathlineto{\pgfqpoint{3.593778in}{0.733952in}}%
\pgfpathlineto{\pgfqpoint{3.596742in}{0.685086in}}%
\pgfpathlineto{\pgfqpoint{3.599705in}{0.684182in}}%
\pgfpathlineto{\pgfqpoint{3.602668in}{0.696172in}}%
\pgfpathlineto{\pgfqpoint{3.605631in}{0.732922in}}%
\pgfpathlineto{\pgfqpoint{3.608595in}{0.792315in}}%
\pgfpathlineto{\pgfqpoint{3.611558in}{0.696978in}}%
\pgfpathlineto{\pgfqpoint{3.614521in}{0.795110in}}%
\pgfpathlineto{\pgfqpoint{3.617484in}{0.685001in}}%
\pgfpathlineto{\pgfqpoint{3.620448in}{0.778249in}}%
\pgfpathlineto{\pgfqpoint{3.623411in}{0.692862in}}%
\pgfpathlineto{\pgfqpoint{3.626374in}{0.655789in}}%
\pgfpathlineto{\pgfqpoint{3.629337in}{0.658686in}}%
\pgfpathlineto{\pgfqpoint{3.632301in}{0.655953in}}%
\pgfpathlineto{\pgfqpoint{3.635264in}{0.867592in}}%
\pgfpathlineto{\pgfqpoint{3.638227in}{0.659425in}}%
\pgfpathlineto{\pgfqpoint{3.641190in}{0.804540in}}%
\pgfpathlineto{\pgfqpoint{3.644154in}{0.661020in}}%
\pgfpathlineto{\pgfqpoint{3.647117in}{0.659894in}}%
\pgfpathlineto{\pgfqpoint{3.653044in}{0.810460in}}%
\pgfpathlineto{\pgfqpoint{3.656007in}{0.900556in}}%
\pgfpathlineto{\pgfqpoint{3.658970in}{0.656477in}}%
\pgfpathlineto{\pgfqpoint{3.661933in}{0.703911in}}%
\pgfpathlineto{\pgfqpoint{3.664897in}{0.668652in}}%
\pgfpathlineto{\pgfqpoint{3.667860in}{0.670853in}}%
\pgfpathlineto{\pgfqpoint{3.670823in}{0.756427in}}%
\pgfpathlineto{\pgfqpoint{3.673786in}{0.695736in}}%
\pgfpathlineto{\pgfqpoint{3.676750in}{0.689967in}}%
\pgfpathlineto{\pgfqpoint{3.679713in}{0.648641in}}%
\pgfpathlineto{\pgfqpoint{3.682676in}{0.729798in}}%
\pgfpathlineto{\pgfqpoint{3.685639in}{0.726152in}}%
\pgfpathlineto{\pgfqpoint{3.688603in}{0.797447in}}%
\pgfpathlineto{\pgfqpoint{3.691566in}{0.653945in}}%
\pgfpathlineto{\pgfqpoint{3.694529in}{0.634921in}}%
\pgfpathlineto{\pgfqpoint{3.697492in}{0.664944in}}%
\pgfpathlineto{\pgfqpoint{3.700456in}{0.650791in}}%
\pgfpathlineto{\pgfqpoint{3.703419in}{0.654452in}}%
\pgfpathlineto{\pgfqpoint{3.709345in}{0.753841in}}%
\pgfpathlineto{\pgfqpoint{3.712309in}{0.667566in}}%
\pgfpathlineto{\pgfqpoint{3.715272in}{0.688289in}}%
\pgfpathlineto{\pgfqpoint{3.718235in}{0.658243in}}%
\pgfpathlineto{\pgfqpoint{3.721198in}{0.653187in}}%
\pgfpathlineto{\pgfqpoint{3.724162in}{0.664979in}}%
\pgfpathlineto{\pgfqpoint{3.727125in}{0.658091in}}%
\pgfpathlineto{\pgfqpoint{3.730088in}{0.681998in}}%
\pgfpathlineto{\pgfqpoint{3.733051in}{0.664773in}}%
\pgfpathlineto{\pgfqpoint{3.736015in}{0.695431in}}%
\pgfpathlineto{\pgfqpoint{3.738978in}{0.686529in}}%
\pgfpathlineto{\pgfqpoint{3.741941in}{0.686896in}}%
\pgfpathlineto{\pgfqpoint{3.744904in}{0.656717in}}%
\pgfpathlineto{\pgfqpoint{3.747868in}{0.735832in}}%
\pgfpathlineto{\pgfqpoint{3.750831in}{0.751190in}}%
\pgfpathlineto{\pgfqpoint{3.753794in}{0.658965in}}%
\pgfpathlineto{\pgfqpoint{3.756757in}{0.832295in}}%
\pgfpathlineto{\pgfqpoint{3.759721in}{0.686922in}}%
\pgfpathlineto{\pgfqpoint{3.765647in}{0.657190in}}%
\pgfpathlineto{\pgfqpoint{3.768610in}{0.651763in}}%
\pgfpathlineto{\pgfqpoint{3.771574in}{0.639084in}}%
\pgfpathlineto{\pgfqpoint{3.774537in}{0.681481in}}%
\pgfpathlineto{\pgfqpoint{3.777500in}{0.683392in}}%
\pgfpathlineto{\pgfqpoint{3.780463in}{0.724508in}}%
\pgfpathlineto{\pgfqpoint{3.783427in}{0.665974in}}%
\pgfpathlineto{\pgfqpoint{3.786390in}{0.771805in}}%
\pgfpathlineto{\pgfqpoint{3.792316in}{0.622788in}}%
\pgfpathlineto{\pgfqpoint{3.795280in}{0.612440in}}%
\pgfpathlineto{\pgfqpoint{3.798243in}{0.751478in}}%
\pgfpathlineto{\pgfqpoint{3.804169in}{0.642851in}}%
\pgfpathlineto{\pgfqpoint{3.807133in}{0.681437in}}%
\pgfpathlineto{\pgfqpoint{3.810096in}{0.626928in}}%
\pgfpathlineto{\pgfqpoint{3.813059in}{0.763210in}}%
\pgfpathlineto{\pgfqpoint{3.816022in}{0.797181in}}%
\pgfpathlineto{\pgfqpoint{3.818986in}{0.682992in}}%
\pgfpathlineto{\pgfqpoint{3.821949in}{0.740278in}}%
\pgfpathlineto{\pgfqpoint{3.824912in}{0.629998in}}%
\pgfpathlineto{\pgfqpoint{3.827875in}{0.657579in}}%
\pgfpathlineto{\pgfqpoint{3.830839in}{0.997562in}}%
\pgfpathlineto{\pgfqpoint{3.833802in}{0.665598in}}%
\pgfpathlineto{\pgfqpoint{3.836765in}{0.726676in}}%
\pgfpathlineto{\pgfqpoint{3.839728in}{0.656645in}}%
\pgfpathlineto{\pgfqpoint{3.842692in}{0.615751in}}%
\pgfpathlineto{\pgfqpoint{3.845655in}{0.649748in}}%
\pgfpathlineto{\pgfqpoint{3.848618in}{0.632914in}}%
\pgfpathlineto{\pgfqpoint{3.851581in}{0.850795in}}%
\pgfpathlineto{\pgfqpoint{3.854545in}{0.760419in}}%
\pgfpathlineto{\pgfqpoint{3.857508in}{0.820454in}}%
\pgfpathlineto{\pgfqpoint{3.860471in}{0.664310in}}%
\pgfpathlineto{\pgfqpoint{3.863434in}{0.728343in}}%
\pgfpathlineto{\pgfqpoint{3.866398in}{0.763229in}}%
\pgfpathlineto{\pgfqpoint{3.869361in}{0.676735in}}%
\pgfpathlineto{\pgfqpoint{3.872324in}{0.656549in}}%
\pgfpathlineto{\pgfqpoint{3.875287in}{0.609047in}}%
\pgfpathlineto{\pgfqpoint{3.878251in}{0.731685in}}%
\pgfpathlineto{\pgfqpoint{3.881214in}{0.652501in}}%
\pgfpathlineto{\pgfqpoint{3.884177in}{0.662683in}}%
\pgfpathlineto{\pgfqpoint{3.887141in}{0.690196in}}%
\pgfpathlineto{\pgfqpoint{3.890104in}{0.607182in}}%
\pgfpathlineto{\pgfqpoint{3.893067in}{0.681744in}}%
\pgfpathlineto{\pgfqpoint{3.896030in}{0.639128in}}%
\pgfpathlineto{\pgfqpoint{3.898994in}{0.696677in}}%
\pgfpathlineto{\pgfqpoint{3.901957in}{0.635429in}}%
\pgfpathlineto{\pgfqpoint{3.904920in}{0.670855in}}%
\pgfpathlineto{\pgfqpoint{3.910847in}{0.627881in}}%
\pgfpathlineto{\pgfqpoint{3.913810in}{0.742620in}}%
\pgfpathlineto{\pgfqpoint{3.916773in}{0.614869in}}%
\pgfpathlineto{\pgfqpoint{3.919736in}{0.702910in}}%
\pgfpathlineto{\pgfqpoint{3.922700in}{0.650254in}}%
\pgfpathlineto{\pgfqpoint{3.925663in}{0.628628in}}%
\pgfpathlineto{\pgfqpoint{3.928626in}{0.695104in}}%
\pgfpathlineto{\pgfqpoint{3.931589in}{0.697771in}}%
\pgfpathlineto{\pgfqpoint{3.934553in}{0.623074in}}%
\pgfpathlineto{\pgfqpoint{3.937516in}{0.634291in}}%
\pgfpathlineto{\pgfqpoint{3.940479in}{0.653479in}}%
\pgfpathlineto{\pgfqpoint{3.943442in}{0.620623in}}%
\pgfpathlineto{\pgfqpoint{3.946406in}{0.737022in}}%
\pgfpathlineto{\pgfqpoint{3.949369in}{0.804431in}}%
\pgfpathlineto{\pgfqpoint{3.952332in}{0.702111in}}%
\pgfpathlineto{\pgfqpoint{3.955295in}{0.669590in}}%
\pgfpathlineto{\pgfqpoint{3.958259in}{0.863552in}}%
\pgfpathlineto{\pgfqpoint{3.961222in}{0.672378in}}%
\pgfpathlineto{\pgfqpoint{3.964185in}{0.601577in}}%
\pgfpathlineto{\pgfqpoint{3.967148in}{0.659767in}}%
\pgfpathlineto{\pgfqpoint{3.970112in}{0.756600in}}%
\pgfpathlineto{\pgfqpoint{3.973075in}{0.707421in}}%
\pgfpathlineto{\pgfqpoint{3.976038in}{0.603144in}}%
\pgfpathlineto{\pgfqpoint{3.979001in}{0.650773in}}%
\pgfpathlineto{\pgfqpoint{3.981965in}{0.766190in}}%
\pgfpathlineto{\pgfqpoint{3.984928in}{0.621296in}}%
\pgfpathlineto{\pgfqpoint{3.987891in}{0.620048in}}%
\pgfpathlineto{\pgfqpoint{3.993818in}{0.674925in}}%
\pgfpathlineto{\pgfqpoint{3.996781in}{0.851088in}}%
\pgfpathlineto{\pgfqpoint{3.999744in}{0.627780in}}%
\pgfpathlineto{\pgfqpoint{4.002707in}{0.614346in}}%
\pgfpathlineto{\pgfqpoint{4.005671in}{0.833481in}}%
\pgfpathlineto{\pgfqpoint{4.011597in}{0.629487in}}%
\pgfpathlineto{\pgfqpoint{4.014560in}{0.647808in}}%
\pgfpathlineto{\pgfqpoint{4.017524in}{0.677750in}}%
\pgfpathlineto{\pgfqpoint{4.020487in}{0.835862in}}%
\pgfpathlineto{\pgfqpoint{4.023450in}{0.620331in}}%
\pgfpathlineto{\pgfqpoint{4.026413in}{0.673341in}}%
\pgfpathlineto{\pgfqpoint{4.029377in}{0.615860in}}%
\pgfpathlineto{\pgfqpoint{4.032340in}{0.595244in}}%
\pgfpathlineto{\pgfqpoint{4.035303in}{0.631938in}}%
\pgfpathlineto{\pgfqpoint{4.038266in}{0.639764in}}%
\pgfpathlineto{\pgfqpoint{4.041230in}{0.608234in}}%
\pgfpathlineto{\pgfqpoint{4.044193in}{0.634672in}}%
\pgfpathlineto{\pgfqpoint{4.047156in}{0.636248in}}%
\pgfpathlineto{\pgfqpoint{4.050119in}{0.786251in}}%
\pgfpathlineto{\pgfqpoint{4.053083in}{0.646312in}}%
\pgfpathlineto{\pgfqpoint{4.056046in}{0.650895in}}%
\pgfpathlineto{\pgfqpoint{4.059009in}{0.666249in}}%
\pgfpathlineto{\pgfqpoint{4.061972in}{0.593623in}}%
\pgfpathlineto{\pgfqpoint{4.064936in}{0.624724in}}%
\pgfpathlineto{\pgfqpoint{4.067899in}{0.613539in}}%
\pgfpathlineto{\pgfqpoint{4.070862in}{0.702269in}}%
\pgfpathlineto{\pgfqpoint{4.073825in}{0.749703in}}%
\pgfpathlineto{\pgfqpoint{4.076789in}{0.638801in}}%
\pgfpathlineto{\pgfqpoint{4.079752in}{0.793162in}}%
\pgfpathlineto{\pgfqpoint{4.082715in}{0.655702in}}%
\pgfpathlineto{\pgfqpoint{4.085678in}{0.602682in}}%
\pgfpathlineto{\pgfqpoint{4.088642in}{0.658760in}}%
\pgfpathlineto{\pgfqpoint{4.091605in}{0.638064in}}%
\pgfpathlineto{\pgfqpoint{4.094568in}{0.627611in}}%
\pgfpathlineto{\pgfqpoint{4.097531in}{0.641994in}}%
\pgfpathlineto{\pgfqpoint{4.100495in}{0.728266in}}%
\pgfpathlineto{\pgfqpoint{4.103458in}{0.665962in}}%
\pgfpathlineto{\pgfqpoint{4.106421in}{0.679414in}}%
\pgfpathlineto{\pgfqpoint{4.109385in}{0.609158in}}%
\pgfpathlineto{\pgfqpoint{4.115311in}{0.694179in}}%
\pgfpathlineto{\pgfqpoint{4.118274in}{0.632688in}}%
\pgfpathlineto{\pgfqpoint{4.121238in}{0.629384in}}%
\pgfpathlineto{\pgfqpoint{4.124201in}{0.595286in}}%
\pgfpathlineto{\pgfqpoint{4.127164in}{0.659087in}}%
\pgfpathlineto{\pgfqpoint{4.130127in}{0.654573in}}%
\pgfpathlineto{\pgfqpoint{4.133091in}{0.633832in}}%
\pgfpathlineto{\pgfqpoint{4.136054in}{0.722471in}}%
\pgfpathlineto{\pgfqpoint{4.139017in}{0.589202in}}%
\pgfpathlineto{\pgfqpoint{4.141980in}{0.660273in}}%
\pgfpathlineto{\pgfqpoint{4.144944in}{0.586659in}}%
\pgfpathlineto{\pgfqpoint{4.147907in}{0.706914in}}%
\pgfpathlineto{\pgfqpoint{4.150870in}{0.683467in}}%
\pgfpathlineto{\pgfqpoint{4.153833in}{0.625442in}}%
\pgfpathlineto{\pgfqpoint{4.156797in}{0.610811in}}%
\pgfpathlineto{\pgfqpoint{4.159760in}{0.655883in}}%
\pgfpathlineto{\pgfqpoint{4.162723in}{0.640571in}}%
\pgfpathlineto{\pgfqpoint{4.168650in}{0.623120in}}%
\pgfpathlineto{\pgfqpoint{4.171613in}{0.610026in}}%
\pgfpathlineto{\pgfqpoint{4.174576in}{0.579015in}}%
\pgfpathlineto{\pgfqpoint{4.177539in}{0.699370in}}%
\pgfpathlineto{\pgfqpoint{4.180503in}{0.656782in}}%
\pgfpathlineto{\pgfqpoint{4.183466in}{0.640385in}}%
\pgfpathlineto{\pgfqpoint{4.186429in}{0.600604in}}%
\pgfpathlineto{\pgfqpoint{4.189392in}{0.663199in}}%
\pgfpathlineto{\pgfqpoint{4.192356in}{0.667971in}}%
\pgfpathlineto{\pgfqpoint{4.195319in}{0.589798in}}%
\pgfpathlineto{\pgfqpoint{4.198282in}{0.602187in}}%
\pgfpathlineto{\pgfqpoint{4.201245in}{0.621311in}}%
\pgfpathlineto{\pgfqpoint{4.204209in}{0.728645in}}%
\pgfpathlineto{\pgfqpoint{4.207172in}{0.620972in}}%
\pgfpathlineto{\pgfqpoint{4.210135in}{0.637281in}}%
\pgfpathlineto{\pgfqpoint{4.213098in}{0.593059in}}%
\pgfpathlineto{\pgfqpoint{4.216062in}{0.627836in}}%
\pgfpathlineto{\pgfqpoint{4.219025in}{0.703641in}}%
\pgfpathlineto{\pgfqpoint{4.221988in}{0.645731in}}%
\pgfpathlineto{\pgfqpoint{4.224951in}{0.861025in}}%
\pgfpathlineto{\pgfqpoint{4.227915in}{0.592944in}}%
\pgfpathlineto{\pgfqpoint{4.230878in}{0.752265in}}%
\pgfpathlineto{\pgfqpoint{4.233841in}{0.620821in}}%
\pgfpathlineto{\pgfqpoint{4.236804in}{0.607395in}}%
\pgfpathlineto{\pgfqpoint{4.239768in}{0.668408in}}%
\pgfpathlineto{\pgfqpoint{4.245694in}{1.143110in}}%
\pgfpathlineto{\pgfqpoint{4.248657in}{0.574942in}}%
\pgfpathlineto{\pgfqpoint{4.251621in}{0.613659in}}%
\pgfpathlineto{\pgfqpoint{4.254584in}{0.749829in}}%
\pgfpathlineto{\pgfqpoint{4.257547in}{0.586649in}}%
\pgfpathlineto{\pgfqpoint{4.260510in}{0.586308in}}%
\pgfpathlineto{\pgfqpoint{4.263474in}{0.624514in}}%
\pgfpathlineto{\pgfqpoint{4.266437in}{0.677708in}}%
\pgfpathlineto{\pgfqpoint{4.269400in}{0.592660in}}%
\pgfpathlineto{\pgfqpoint{4.272363in}{0.592642in}}%
\pgfpathlineto{\pgfqpoint{4.275327in}{0.574888in}}%
\pgfpathlineto{\pgfqpoint{4.278290in}{0.765584in}}%
\pgfpathlineto{\pgfqpoint{4.287180in}{0.583084in}}%
\pgfpathlineto{\pgfqpoint{4.290143in}{0.594073in}}%
\pgfpathlineto{\pgfqpoint{4.293106in}{0.947581in}}%
\pgfpathlineto{\pgfqpoint{4.296069in}{0.650157in}}%
\pgfpathlineto{\pgfqpoint{4.301996in}{0.586186in}}%
\pgfpathlineto{\pgfqpoint{4.304959in}{0.749817in}}%
\pgfpathlineto{\pgfqpoint{4.307922in}{0.643128in}}%
\pgfpathlineto{\pgfqpoint{4.310886in}{0.597505in}}%
\pgfpathlineto{\pgfqpoint{4.313849in}{0.622613in}}%
\pgfpathlineto{\pgfqpoint{4.316812in}{0.583330in}}%
\pgfpathlineto{\pgfqpoint{4.319775in}{0.627344in}}%
\pgfpathlineto{\pgfqpoint{4.322739in}{0.574783in}}%
\pgfpathlineto{\pgfqpoint{4.325702in}{0.746583in}}%
\pgfpathlineto{\pgfqpoint{4.328665in}{0.762707in}}%
\pgfpathlineto{\pgfqpoint{4.331629in}{0.714472in}}%
\pgfpathlineto{\pgfqpoint{4.334592in}{0.965328in}}%
\pgfpathlineto{\pgfqpoint{4.337555in}{0.661897in}}%
\pgfpathlineto{\pgfqpoint{4.340518in}{0.603432in}}%
\pgfpathlineto{\pgfqpoint{4.343482in}{0.631837in}}%
\pgfpathlineto{\pgfqpoint{4.346445in}{0.618374in}}%
\pgfpathlineto{\pgfqpoint{4.349408in}{0.675569in}}%
\pgfpathlineto{\pgfqpoint{4.352371in}{0.575808in}}%
\pgfpathlineto{\pgfqpoint{4.355335in}{0.591977in}}%
\pgfpathlineto{\pgfqpoint{4.358298in}{0.579202in}}%
\pgfpathlineto{\pgfqpoint{4.361261in}{0.653814in}}%
\pgfpathlineto{\pgfqpoint{4.364224in}{0.607459in}}%
\pgfpathlineto{\pgfqpoint{4.367188in}{0.581977in}}%
\pgfpathlineto{\pgfqpoint{4.370151in}{0.593388in}}%
\pgfpathlineto{\pgfqpoint{4.373114in}{0.896818in}}%
\pgfpathlineto{\pgfqpoint{4.376077in}{0.783854in}}%
\pgfpathlineto{\pgfqpoint{4.379041in}{0.606939in}}%
\pgfpathlineto{\pgfqpoint{4.382004in}{0.617453in}}%
\pgfpathlineto{\pgfqpoint{4.384967in}{0.600265in}}%
\pgfpathlineto{\pgfqpoint{4.387930in}{0.563494in}}%
\pgfpathlineto{\pgfqpoint{4.390894in}{0.569359in}}%
\pgfpathlineto{\pgfqpoint{4.393857in}{0.600644in}}%
\pgfpathlineto{\pgfqpoint{4.396820in}{0.654905in}}%
\pgfpathlineto{\pgfqpoint{4.399783in}{0.671019in}}%
\pgfpathlineto{\pgfqpoint{4.402747in}{0.642324in}}%
\pgfpathlineto{\pgfqpoint{4.405710in}{0.592662in}}%
\pgfpathlineto{\pgfqpoint{4.408673in}{0.703462in}}%
\pgfpathlineto{\pgfqpoint{4.411636in}{0.676955in}}%
\pgfpathlineto{\pgfqpoint{4.414600in}{0.599701in}}%
\pgfpathlineto{\pgfqpoint{4.417563in}{0.671859in}}%
\pgfpathlineto{\pgfqpoint{4.420526in}{0.593093in}}%
\pgfpathlineto{\pgfqpoint{4.423489in}{0.589147in}}%
\pgfpathlineto{\pgfqpoint{4.426453in}{0.570332in}}%
\pgfpathlineto{\pgfqpoint{4.429416in}{0.850147in}}%
\pgfpathlineto{\pgfqpoint{4.432379in}{0.662515in}}%
\pgfpathlineto{\pgfqpoint{4.435342in}{0.721126in}}%
\pgfpathlineto{\pgfqpoint{4.438306in}{0.690450in}}%
\pgfpathlineto{\pgfqpoint{4.444232in}{0.762309in}}%
\pgfpathlineto{\pgfqpoint{4.447195in}{0.595692in}}%
\pgfpathlineto{\pgfqpoint{4.450159in}{0.558491in}}%
\pgfpathlineto{\pgfqpoint{4.453122in}{0.601317in}}%
\pgfpathlineto{\pgfqpoint{4.456085in}{0.684682in}}%
\pgfpathlineto{\pgfqpoint{4.459048in}{0.583931in}}%
\pgfpathlineto{\pgfqpoint{4.462012in}{0.582616in}}%
\pgfpathlineto{\pgfqpoint{4.464975in}{0.646084in}}%
\pgfpathlineto{\pgfqpoint{4.467938in}{0.627562in}}%
\pgfpathlineto{\pgfqpoint{4.470901in}{0.628931in}}%
\pgfpathlineto{\pgfqpoint{4.473865in}{0.721190in}}%
\pgfpathlineto{\pgfqpoint{4.476828in}{1.041992in}}%
\pgfpathlineto{\pgfqpoint{4.479791in}{0.601110in}}%
\pgfpathlineto{\pgfqpoint{4.482754in}{0.570120in}}%
\pgfpathlineto{\pgfqpoint{4.485718in}{0.652439in}}%
\pgfpathlineto{\pgfqpoint{4.488681in}{0.573488in}}%
\pgfpathlineto{\pgfqpoint{4.491644in}{0.602563in}}%
\pgfpathlineto{\pgfqpoint{4.494607in}{0.732207in}}%
\pgfpathlineto{\pgfqpoint{4.497571in}{0.618934in}}%
\pgfpathlineto{\pgfqpoint{4.500534in}{0.613668in}}%
\pgfpathlineto{\pgfqpoint{4.503497in}{0.581784in}}%
\pgfpathlineto{\pgfqpoint{4.506460in}{0.571609in}}%
\pgfpathlineto{\pgfqpoint{4.509424in}{0.654521in}}%
\pgfpathlineto{\pgfqpoint{4.515350in}{0.554365in}}%
\pgfpathlineto{\pgfqpoint{4.518313in}{0.550202in}}%
\pgfpathlineto{\pgfqpoint{4.521277in}{0.569877in}}%
\pgfpathlineto{\pgfqpoint{4.524240in}{0.733521in}}%
\pgfpathlineto{\pgfqpoint{4.527203in}{0.614772in}}%
\pgfpathlineto{\pgfqpoint{4.530166in}{0.548581in}}%
\pgfpathlineto{\pgfqpoint{4.533130in}{0.588422in}}%
\pgfpathlineto{\pgfqpoint{4.536093in}{0.780462in}}%
\pgfpathlineto{\pgfqpoint{4.539056in}{0.649635in}}%
\pgfpathlineto{\pgfqpoint{4.544983in}{0.570884in}}%
\pgfpathlineto{\pgfqpoint{4.550909in}{0.619020in}}%
\pgfpathlineto{\pgfqpoint{4.553872in}{0.606013in}}%
\pgfpathlineto{\pgfqpoint{4.556836in}{0.626557in}}%
\pgfpathlineto{\pgfqpoint{4.559799in}{0.572739in}}%
\pgfpathlineto{\pgfqpoint{4.562762in}{0.645856in}}%
\pgfpathlineto{\pgfqpoint{4.565726in}{0.584876in}}%
\pgfpathlineto{\pgfqpoint{4.568689in}{0.575356in}}%
\pgfpathlineto{\pgfqpoint{4.571652in}{0.561410in}}%
\pgfpathlineto{\pgfqpoint{4.574615in}{0.618058in}}%
\pgfpathlineto{\pgfqpoint{4.577579in}{0.581539in}}%
\pgfpathlineto{\pgfqpoint{4.580542in}{0.628512in}}%
\pgfpathlineto{\pgfqpoint{4.583505in}{0.568656in}}%
\pgfpathlineto{\pgfqpoint{4.586468in}{0.578385in}}%
\pgfpathlineto{\pgfqpoint{4.592395in}{0.690898in}}%
\pgfpathlineto{\pgfqpoint{4.595358in}{0.581810in}}%
\pgfpathlineto{\pgfqpoint{4.598321in}{0.664045in}}%
\pgfpathlineto{\pgfqpoint{4.601285in}{0.700437in}}%
\pgfpathlineto{\pgfqpoint{4.604248in}{0.610636in}}%
\pgfpathlineto{\pgfqpoint{4.607211in}{0.564675in}}%
\pgfpathlineto{\pgfqpoint{4.610174in}{0.563517in}}%
\pgfpathlineto{\pgfqpoint{4.613138in}{0.594064in}}%
\pgfpathlineto{\pgfqpoint{4.616101in}{0.593081in}}%
\pgfpathlineto{\pgfqpoint{4.619064in}{0.790323in}}%
\pgfpathlineto{\pgfqpoint{4.622027in}{0.575484in}}%
\pgfpathlineto{\pgfqpoint{4.624991in}{0.547489in}}%
\pgfpathlineto{\pgfqpoint{4.627954in}{0.692171in}}%
\pgfpathlineto{\pgfqpoint{4.630917in}{0.557986in}}%
\pgfpathlineto{\pgfqpoint{4.633880in}{0.555238in}}%
\pgfpathlineto{\pgfqpoint{4.636844in}{0.541686in}}%
\pgfpathlineto{\pgfqpoint{4.639807in}{0.560717in}}%
\pgfpathlineto{\pgfqpoint{4.642770in}{0.776107in}}%
\pgfpathlineto{\pgfqpoint{4.645733in}{0.549113in}}%
\pgfpathlineto{\pgfqpoint{4.648697in}{0.531391in}}%
\pgfpathlineto{\pgfqpoint{4.651660in}{0.555793in}}%
\pgfpathlineto{\pgfqpoint{4.654623in}{0.593247in}}%
\pgfpathlineto{\pgfqpoint{4.657586in}{0.768599in}}%
\pgfpathlineto{\pgfqpoint{4.660550in}{0.610557in}}%
\pgfpathlineto{\pgfqpoint{4.666476in}{0.536757in}}%
\pgfpathlineto{\pgfqpoint{4.669439in}{0.599211in}}%
\pgfpathlineto{\pgfqpoint{4.672403in}{0.622457in}}%
\pgfpathlineto{\pgfqpoint{4.675366in}{0.687215in}}%
\pgfpathlineto{\pgfqpoint{4.678329in}{0.675757in}}%
\pgfpathlineto{\pgfqpoint{4.681292in}{0.514950in}}%
\pgfpathlineto{\pgfqpoint{4.684256in}{0.578861in}}%
\pgfpathlineto{\pgfqpoint{4.687219in}{0.539127in}}%
\pgfpathlineto{\pgfqpoint{4.690182in}{0.532633in}}%
\pgfpathlineto{\pgfqpoint{4.693145in}{0.556605in}}%
\pgfpathlineto{\pgfqpoint{4.696109in}{0.554528in}}%
\pgfpathlineto{\pgfqpoint{4.699072in}{0.556169in}}%
\pgfpathlineto{\pgfqpoint{4.702035in}{0.550230in}}%
\pgfpathlineto{\pgfqpoint{4.704998in}{0.604332in}}%
\pgfpathlineto{\pgfqpoint{4.707962in}{0.599686in}}%
\pgfpathlineto{\pgfqpoint{4.710925in}{0.585359in}}%
\pgfpathlineto{\pgfqpoint{4.713888in}{0.602404in}}%
\pgfpathlineto{\pgfqpoint{4.716851in}{0.536310in}}%
\pgfpathlineto{\pgfqpoint{4.719815in}{0.647451in}}%
\pgfpathlineto{\pgfqpoint{4.722778in}{0.548913in}}%
\pgfpathlineto{\pgfqpoint{4.725741in}{0.708703in}}%
\pgfpathlineto{\pgfqpoint{4.728704in}{0.542484in}}%
\pgfpathlineto{\pgfqpoint{4.731668in}{0.561418in}}%
\pgfpathlineto{\pgfqpoint{4.734631in}{0.587791in}}%
\pgfpathlineto{\pgfqpoint{4.740557in}{0.613542in}}%
\pgfpathlineto{\pgfqpoint{4.743521in}{0.543642in}}%
\pgfpathlineto{\pgfqpoint{4.746484in}{0.590702in}}%
\pgfpathlineto{\pgfqpoint{4.749447in}{0.565111in}}%
\pgfpathlineto{\pgfqpoint{4.752410in}{0.557577in}}%
\pgfpathlineto{\pgfqpoint{4.755374in}{0.644393in}}%
\pgfpathlineto{\pgfqpoint{4.758337in}{0.537783in}}%
\pgfpathlineto{\pgfqpoint{4.761300in}{0.633996in}}%
\pgfpathlineto{\pgfqpoint{4.764263in}{0.542803in}}%
\pgfpathlineto{\pgfqpoint{4.767227in}{0.773637in}}%
\pgfpathlineto{\pgfqpoint{4.770190in}{0.546422in}}%
\pgfpathlineto{\pgfqpoint{4.773153in}{0.690285in}}%
\pgfpathlineto{\pgfqpoint{4.776116in}{0.716730in}}%
\pgfpathlineto{\pgfqpoint{4.779080in}{0.588325in}}%
\pgfpathlineto{\pgfqpoint{4.782043in}{0.581426in}}%
\pgfpathlineto{\pgfqpoint{4.785006in}{0.672358in}}%
\pgfpathlineto{\pgfqpoint{4.787970in}{0.539156in}}%
\pgfpathlineto{\pgfqpoint{4.790933in}{0.595086in}}%
\pgfpathlineto{\pgfqpoint{4.793896in}{0.539375in}}%
\pgfpathlineto{\pgfqpoint{4.796859in}{0.681207in}}%
\pgfpathlineto{\pgfqpoint{4.799823in}{0.579312in}}%
\pgfpathlineto{\pgfqpoint{4.802786in}{0.625880in}}%
\pgfpathlineto{\pgfqpoint{4.805749in}{0.544122in}}%
\pgfpathlineto{\pgfqpoint{4.808712in}{0.690815in}}%
\pgfpathlineto{\pgfqpoint{4.811676in}{0.563660in}}%
\pgfpathlineto{\pgfqpoint{4.814639in}{0.895869in}}%
\pgfpathlineto{\pgfqpoint{4.817602in}{0.609351in}}%
\pgfpathlineto{\pgfqpoint{4.820565in}{0.526894in}}%
\pgfpathlineto{\pgfqpoint{4.823529in}{0.827051in}}%
\pgfpathlineto{\pgfqpoint{4.826492in}{0.562667in}}%
\pgfpathlineto{\pgfqpoint{4.829455in}{0.609125in}}%
\pgfpathlineto{\pgfqpoint{4.832418in}{0.578249in}}%
\pgfpathlineto{\pgfqpoint{4.838345in}{0.667066in}}%
\pgfpathlineto{\pgfqpoint{4.841308in}{0.627480in}}%
\pgfpathlineto{\pgfqpoint{4.844271in}{0.655467in}}%
\pgfpathlineto{\pgfqpoint{4.847235in}{0.609829in}}%
\pgfpathlineto{\pgfqpoint{4.850198in}{0.584755in}}%
\pgfpathlineto{\pgfqpoint{4.853161in}{0.619835in}}%
\pgfpathlineto{\pgfqpoint{4.856124in}{0.966908in}}%
\pgfpathlineto{\pgfqpoint{4.859088in}{0.680120in}}%
\pgfpathlineto{\pgfqpoint{4.862051in}{0.553919in}}%
\pgfpathlineto{\pgfqpoint{4.865014in}{0.540281in}}%
\pgfpathlineto{\pgfqpoint{4.867977in}{0.613832in}}%
\pgfpathlineto{\pgfqpoint{4.870941in}{0.585193in}}%
\pgfpathlineto{\pgfqpoint{4.873904in}{0.626390in}}%
\pgfpathlineto{\pgfqpoint{4.876867in}{0.513321in}}%
\pgfpathlineto{\pgfqpoint{4.879830in}{0.568985in}}%
\pgfpathlineto{\pgfqpoint{4.882794in}{0.569168in}}%
\pgfpathlineto{\pgfqpoint{4.885757in}{0.628178in}}%
\pgfpathlineto{\pgfqpoint{4.888720in}{0.522414in}}%
\pgfpathlineto{\pgfqpoint{4.891683in}{0.602483in}}%
\pgfpathlineto{\pgfqpoint{4.894647in}{0.604273in}}%
\pgfpathlineto{\pgfqpoint{4.897610in}{0.721913in}}%
\pgfpathlineto{\pgfqpoint{4.900573in}{0.639631in}}%
\pgfpathlineto{\pgfqpoint{4.903536in}{0.654657in}}%
\pgfpathlineto{\pgfqpoint{4.906500in}{0.717603in}}%
\pgfpathlineto{\pgfqpoint{4.912426in}{0.515219in}}%
\pgfpathlineto{\pgfqpoint{4.915389in}{0.542509in}}%
\pgfpathlineto{\pgfqpoint{4.918353in}{0.632443in}}%
\pgfpathlineto{\pgfqpoint{4.921316in}{0.540199in}}%
\pgfpathlineto{\pgfqpoint{4.924279in}{0.618219in}}%
\pgfpathlineto{\pgfqpoint{4.927242in}{0.527083in}}%
\pgfpathlineto{\pgfqpoint{4.930206in}{0.627238in}}%
\pgfpathlineto{\pgfqpoint{4.933169in}{0.639591in}}%
\pgfpathlineto{\pgfqpoint{4.936132in}{0.502096in}}%
\pgfpathlineto{\pgfqpoint{4.942059in}{0.651051in}}%
\pgfpathlineto{\pgfqpoint{4.945022in}{0.581847in}}%
\pgfpathlineto{\pgfqpoint{4.947985in}{0.546903in}}%
\pgfpathlineto{\pgfqpoint{4.950948in}{0.495679in}}%
\pgfpathlineto{\pgfqpoint{4.953912in}{0.574694in}}%
\pgfpathlineto{\pgfqpoint{4.956875in}{0.512875in}}%
\pgfpathlineto{\pgfqpoint{4.959838in}{0.678612in}}%
\pgfpathlineto{\pgfqpoint{4.962801in}{0.526097in}}%
\pgfpathlineto{\pgfqpoint{4.965765in}{0.616736in}}%
\pgfpathlineto{\pgfqpoint{4.968728in}{0.585481in}}%
\pgfpathlineto{\pgfqpoint{4.971691in}{0.729095in}}%
\pgfpathlineto{\pgfqpoint{4.974654in}{0.528528in}}%
\pgfpathlineto{\pgfqpoint{4.977618in}{0.540944in}}%
\pgfpathlineto{\pgfqpoint{4.980581in}{0.512731in}}%
\pgfpathlineto{\pgfqpoint{4.983544in}{0.559039in}}%
\pgfpathlineto{\pgfqpoint{4.986507in}{0.749324in}}%
\pgfpathlineto{\pgfqpoint{4.989471in}{0.552696in}}%
\pgfpathlineto{\pgfqpoint{4.992434in}{0.568002in}}%
\pgfpathlineto{\pgfqpoint{4.995397in}{0.555948in}}%
\pgfpathlineto{\pgfqpoint{4.998360in}{0.578237in}}%
\pgfpathlineto{\pgfqpoint{5.001324in}{0.665011in}}%
\pgfpathlineto{\pgfqpoint{5.004287in}{0.530563in}}%
\pgfpathlineto{\pgfqpoint{5.007250in}{0.505329in}}%
\pgfpathlineto{\pgfqpoint{5.010214in}{0.600577in}}%
\pgfpathlineto{\pgfqpoint{5.013177in}{0.764724in}}%
\pgfpathlineto{\pgfqpoint{5.016140in}{0.645206in}}%
\pgfpathlineto{\pgfqpoint{5.019103in}{0.617191in}}%
\pgfpathlineto{\pgfqpoint{5.022067in}{0.531096in}}%
\pgfpathlineto{\pgfqpoint{5.025030in}{0.516937in}}%
\pgfpathlineto{\pgfqpoint{5.027993in}{0.520540in}}%
\pgfpathlineto{\pgfqpoint{5.030956in}{0.598384in}}%
\pgfpathlineto{\pgfqpoint{5.033920in}{0.535108in}}%
\pgfpathlineto{\pgfqpoint{5.036883in}{0.544088in}}%
\pgfpathlineto{\pgfqpoint{5.039846in}{0.524542in}}%
\pgfpathlineto{\pgfqpoint{5.042809in}{0.515784in}}%
\pgfpathlineto{\pgfqpoint{5.045773in}{0.564448in}}%
\pgfpathlineto{\pgfqpoint{5.048736in}{0.502333in}}%
\pgfpathlineto{\pgfqpoint{5.051699in}{0.527675in}}%
\pgfpathlineto{\pgfqpoint{5.054662in}{0.501304in}}%
\pgfpathlineto{\pgfqpoint{5.057626in}{0.524436in}}%
\pgfpathlineto{\pgfqpoint{5.060589in}{0.519967in}}%
\pgfpathlineto{\pgfqpoint{5.063552in}{0.733408in}}%
\pgfpathlineto{\pgfqpoint{5.066515in}{0.785278in}}%
\pgfpathlineto{\pgfqpoint{5.069479in}{0.692928in}}%
\pgfpathlineto{\pgfqpoint{5.072442in}{0.522802in}}%
\pgfpathlineto{\pgfqpoint{5.078368in}{0.554414in}}%
\pgfpathlineto{\pgfqpoint{5.081332in}{0.531365in}}%
\pgfpathlineto{\pgfqpoint{5.084295in}{0.529145in}}%
\pgfpathlineto{\pgfqpoint{5.087258in}{0.516734in}}%
\pgfpathlineto{\pgfqpoint{5.090221in}{0.509402in}}%
\pgfpathlineto{\pgfqpoint{5.093185in}{0.518931in}}%
\pgfpathlineto{\pgfqpoint{5.096148in}{0.498523in}}%
\pgfpathlineto{\pgfqpoint{5.099111in}{0.521950in}}%
\pgfpathlineto{\pgfqpoint{5.102074in}{0.508666in}}%
\pgfpathlineto{\pgfqpoint{5.105038in}{0.539696in}}%
\pgfpathlineto{\pgfqpoint{5.105038in}{0.539696in}}%
\pgfusepath{stroke}%
\end{pgfscope}%
\begin{pgfscope}%
\pgfpathrectangle{\pgfqpoint{0.444137in}{0.320679in}}{\pgfqpoint{4.882849in}{3.850000in}}%
\pgfusepath{clip}%
\pgfsetrectcap%
\pgfsetroundjoin%
\pgfsetlinewidth{1.505625pt}%
\definecolor{currentstroke}{rgb}{1.000000,0.498039,0.054902}%
\pgfsetstrokecolor{currentstroke}%
\pgfsetdash{}{0pt}%
\pgfpathmoveto{\pgfqpoint{0.666084in}{3.995585in}}%
\pgfpathlineto{\pgfqpoint{0.704606in}{3.995674in}}%
\pgfpathlineto{\pgfqpoint{0.728313in}{3.995122in}}%
\pgfpathlineto{\pgfqpoint{0.731276in}{3.993602in}}%
\pgfpathlineto{\pgfqpoint{0.734239in}{3.981967in}}%
\pgfpathlineto{\pgfqpoint{0.737202in}{3.816925in}}%
\pgfpathlineto{\pgfqpoint{0.746092in}{3.122190in}}%
\pgfpathlineto{\pgfqpoint{0.749055in}{3.034562in}}%
\pgfpathlineto{\pgfqpoint{0.752019in}{2.876936in}}%
\pgfpathlineto{\pgfqpoint{0.754982in}{2.900529in}}%
\pgfpathlineto{\pgfqpoint{0.757945in}{2.554469in}}%
\pgfpathlineto{\pgfqpoint{0.763872in}{2.199332in}}%
\pgfpathlineto{\pgfqpoint{0.766835in}{2.171957in}}%
\pgfpathlineto{\pgfqpoint{0.769798in}{2.002908in}}%
\pgfpathlineto{\pgfqpoint{0.772761in}{2.057578in}}%
\pgfpathlineto{\pgfqpoint{0.775725in}{1.828019in}}%
\pgfpathlineto{\pgfqpoint{0.778688in}{1.783736in}}%
\pgfpathlineto{\pgfqpoint{0.781651in}{1.694856in}}%
\pgfpathlineto{\pgfqpoint{0.784614in}{1.736618in}}%
\pgfpathlineto{\pgfqpoint{0.787578in}{1.658155in}}%
\pgfpathlineto{\pgfqpoint{0.790541in}{1.618894in}}%
\pgfpathlineto{\pgfqpoint{0.793504in}{1.652230in}}%
\pgfpathlineto{\pgfqpoint{0.796467in}{1.594174in}}%
\pgfpathlineto{\pgfqpoint{0.799431in}{1.485250in}}%
\pgfpathlineto{\pgfqpoint{0.802394in}{1.558480in}}%
\pgfpathlineto{\pgfqpoint{0.805357in}{1.480371in}}%
\pgfpathlineto{\pgfqpoint{0.808320in}{1.528231in}}%
\pgfpathlineto{\pgfqpoint{0.811284in}{1.463217in}}%
\pgfpathlineto{\pgfqpoint{0.814247in}{1.545100in}}%
\pgfpathlineto{\pgfqpoint{0.817210in}{1.663732in}}%
\pgfpathlineto{\pgfqpoint{0.820173in}{1.643061in}}%
\pgfpathlineto{\pgfqpoint{0.823137in}{1.473315in}}%
\pgfpathlineto{\pgfqpoint{0.826100in}{1.600202in}}%
\pgfpathlineto{\pgfqpoint{0.829063in}{1.404933in}}%
\pgfpathlineto{\pgfqpoint{0.832026in}{1.334932in}}%
\pgfpathlineto{\pgfqpoint{0.834990in}{1.399412in}}%
\pgfpathlineto{\pgfqpoint{0.837953in}{1.801667in}}%
\pgfpathlineto{\pgfqpoint{0.840916in}{1.416668in}}%
\pgfpathlineto{\pgfqpoint{0.843879in}{1.670729in}}%
\pgfpathlineto{\pgfqpoint{0.846843in}{1.707959in}}%
\pgfpathlineto{\pgfqpoint{0.855732in}{1.291594in}}%
\pgfpathlineto{\pgfqpoint{0.858696in}{1.283837in}}%
\pgfpathlineto{\pgfqpoint{0.861659in}{1.442053in}}%
\pgfpathlineto{\pgfqpoint{0.864622in}{1.341128in}}%
\pgfpathlineto{\pgfqpoint{0.867585in}{1.287633in}}%
\pgfpathlineto{\pgfqpoint{0.870549in}{1.353328in}}%
\pgfpathlineto{\pgfqpoint{0.873512in}{1.258459in}}%
\pgfpathlineto{\pgfqpoint{0.879438in}{1.366776in}}%
\pgfpathlineto{\pgfqpoint{0.882402in}{1.338019in}}%
\pgfpathlineto{\pgfqpoint{0.885365in}{1.256684in}}%
\pgfpathlineto{\pgfqpoint{0.888328in}{1.291923in}}%
\pgfpathlineto{\pgfqpoint{0.891291in}{1.459430in}}%
\pgfpathlineto{\pgfqpoint{0.894255in}{1.245606in}}%
\pgfpathlineto{\pgfqpoint{0.897218in}{1.302961in}}%
\pgfpathlineto{\pgfqpoint{0.900181in}{1.311663in}}%
\pgfpathlineto{\pgfqpoint{0.903144in}{1.386681in}}%
\pgfpathlineto{\pgfqpoint{0.906108in}{1.246202in}}%
\pgfpathlineto{\pgfqpoint{0.909071in}{1.504061in}}%
\pgfpathlineto{\pgfqpoint{0.912034in}{1.546337in}}%
\pgfpathlineto{\pgfqpoint{0.914997in}{1.336093in}}%
\pgfpathlineto{\pgfqpoint{0.917961in}{1.281196in}}%
\pgfpathlineto{\pgfqpoint{0.920924in}{1.289093in}}%
\pgfpathlineto{\pgfqpoint{0.923887in}{1.287271in}}%
\pgfpathlineto{\pgfqpoint{0.926850in}{1.223608in}}%
\pgfpathlineto{\pgfqpoint{0.929814in}{1.393969in}}%
\pgfpathlineto{\pgfqpoint{0.932777in}{1.671142in}}%
\pgfpathlineto{\pgfqpoint{0.935740in}{1.190869in}}%
\pgfpathlineto{\pgfqpoint{0.938703in}{1.452285in}}%
\pgfpathlineto{\pgfqpoint{0.941667in}{1.496601in}}%
\pgfpathlineto{\pgfqpoint{0.944630in}{1.245437in}}%
\pgfpathlineto{\pgfqpoint{0.947593in}{1.438068in}}%
\pgfpathlineto{\pgfqpoint{0.950557in}{1.360453in}}%
\pgfpathlineto{\pgfqpoint{0.953520in}{1.197212in}}%
\pgfpathlineto{\pgfqpoint{0.956483in}{1.363179in}}%
\pgfpathlineto{\pgfqpoint{0.959446in}{1.255456in}}%
\pgfpathlineto{\pgfqpoint{0.962410in}{1.407380in}}%
\pgfpathlineto{\pgfqpoint{0.965373in}{1.201455in}}%
\pgfpathlineto{\pgfqpoint{0.968336in}{1.289514in}}%
\pgfpathlineto{\pgfqpoint{0.971299in}{1.218429in}}%
\pgfpathlineto{\pgfqpoint{0.974263in}{1.279936in}}%
\pgfpathlineto{\pgfqpoint{0.977226in}{1.577184in}}%
\pgfpathlineto{\pgfqpoint{0.980189in}{1.299359in}}%
\pgfpathlineto{\pgfqpoint{0.983152in}{1.166378in}}%
\pgfpathlineto{\pgfqpoint{0.986116in}{1.228826in}}%
\pgfpathlineto{\pgfqpoint{0.989079in}{1.198888in}}%
\pgfpathlineto{\pgfqpoint{0.992042in}{1.307442in}}%
\pgfpathlineto{\pgfqpoint{0.995005in}{1.218683in}}%
\pgfpathlineto{\pgfqpoint{0.997969in}{1.390928in}}%
\pgfpathlineto{\pgfqpoint{1.000932in}{1.287467in}}%
\pgfpathlineto{\pgfqpoint{1.003895in}{1.227479in}}%
\pgfpathlineto{\pgfqpoint{1.006858in}{1.347248in}}%
\pgfpathlineto{\pgfqpoint{1.009822in}{1.523447in}}%
\pgfpathlineto{\pgfqpoint{1.012785in}{1.339765in}}%
\pgfpathlineto{\pgfqpoint{1.015748in}{1.742874in}}%
\pgfpathlineto{\pgfqpoint{1.018711in}{1.235010in}}%
\pgfpathlineto{\pgfqpoint{1.021675in}{1.160991in}}%
\pgfpathlineto{\pgfqpoint{1.024638in}{1.222296in}}%
\pgfpathlineto{\pgfqpoint{1.027601in}{1.589723in}}%
\pgfpathlineto{\pgfqpoint{1.030564in}{1.164464in}}%
\pgfpathlineto{\pgfqpoint{1.033528in}{1.151584in}}%
\pgfpathlineto{\pgfqpoint{1.036491in}{1.175471in}}%
\pgfpathlineto{\pgfqpoint{1.039454in}{1.524257in}}%
\pgfpathlineto{\pgfqpoint{1.042417in}{1.424117in}}%
\pgfpathlineto{\pgfqpoint{1.045381in}{1.133410in}}%
\pgfpathlineto{\pgfqpoint{1.048344in}{1.423701in}}%
\pgfpathlineto{\pgfqpoint{1.051307in}{1.215200in}}%
\pgfpathlineto{\pgfqpoint{1.054270in}{1.335935in}}%
\pgfpathlineto{\pgfqpoint{1.057234in}{1.085640in}}%
\pgfpathlineto{\pgfqpoint{1.060197in}{1.259480in}}%
\pgfpathlineto{\pgfqpoint{1.063160in}{1.317424in}}%
\pgfpathlineto{\pgfqpoint{1.066123in}{1.223624in}}%
\pgfpathlineto{\pgfqpoint{1.069087in}{1.400195in}}%
\pgfpathlineto{\pgfqpoint{1.072050in}{1.199997in}}%
\pgfpathlineto{\pgfqpoint{1.075013in}{1.137281in}}%
\pgfpathlineto{\pgfqpoint{1.077976in}{1.530115in}}%
\pgfpathlineto{\pgfqpoint{1.080940in}{1.261940in}}%
\pgfpathlineto{\pgfqpoint{1.083903in}{1.114387in}}%
\pgfpathlineto{\pgfqpoint{1.086866in}{1.206862in}}%
\pgfpathlineto{\pgfqpoint{1.089829in}{1.086397in}}%
\pgfpathlineto{\pgfqpoint{1.092793in}{1.568881in}}%
\pgfpathlineto{\pgfqpoint{1.095756in}{1.094050in}}%
\pgfpathlineto{\pgfqpoint{1.101682in}{1.169935in}}%
\pgfpathlineto{\pgfqpoint{1.104646in}{1.433317in}}%
\pgfpathlineto{\pgfqpoint{1.107609in}{1.333662in}}%
\pgfpathlineto{\pgfqpoint{1.110572in}{1.317566in}}%
\pgfpathlineto{\pgfqpoint{1.113535in}{1.133722in}}%
\pgfpathlineto{\pgfqpoint{1.116499in}{1.167428in}}%
\pgfpathlineto{\pgfqpoint{1.119462in}{1.128741in}}%
\pgfpathlineto{\pgfqpoint{1.122425in}{1.200869in}}%
\pgfpathlineto{\pgfqpoint{1.128352in}{1.095448in}}%
\pgfpathlineto{\pgfqpoint{1.131315in}{1.108594in}}%
\pgfpathlineto{\pgfqpoint{1.134278in}{1.148152in}}%
\pgfpathlineto{\pgfqpoint{1.137241in}{1.227522in}}%
\pgfpathlineto{\pgfqpoint{1.140205in}{1.098100in}}%
\pgfpathlineto{\pgfqpoint{1.143168in}{1.164054in}}%
\pgfpathlineto{\pgfqpoint{1.146131in}{1.035635in}}%
\pgfpathlineto{\pgfqpoint{1.152058in}{1.169025in}}%
\pgfpathlineto{\pgfqpoint{1.155021in}{1.035783in}}%
\pgfpathlineto{\pgfqpoint{1.157984in}{1.052051in}}%
\pgfpathlineto{\pgfqpoint{1.160947in}{1.019532in}}%
\pgfpathlineto{\pgfqpoint{1.163911in}{1.225732in}}%
\pgfpathlineto{\pgfqpoint{1.166874in}{1.084488in}}%
\pgfpathlineto{\pgfqpoint{1.169837in}{1.086850in}}%
\pgfpathlineto{\pgfqpoint{1.172801in}{1.053118in}}%
\pgfpathlineto{\pgfqpoint{1.175764in}{1.066694in}}%
\pgfpathlineto{\pgfqpoint{1.178727in}{1.088444in}}%
\pgfpathlineto{\pgfqpoint{1.181690in}{1.358540in}}%
\pgfpathlineto{\pgfqpoint{1.184654in}{1.115734in}}%
\pgfpathlineto{\pgfqpoint{1.187617in}{1.375253in}}%
\pgfpathlineto{\pgfqpoint{1.190580in}{1.156560in}}%
\pgfpathlineto{\pgfqpoint{1.193543in}{1.132142in}}%
\pgfpathlineto{\pgfqpoint{1.196507in}{1.009053in}}%
\pgfpathlineto{\pgfqpoint{1.199470in}{1.083177in}}%
\pgfpathlineto{\pgfqpoint{1.202433in}{0.990124in}}%
\pgfpathlineto{\pgfqpoint{1.205396in}{1.022986in}}%
\pgfpathlineto{\pgfqpoint{1.208360in}{1.003919in}}%
\pgfpathlineto{\pgfqpoint{1.211323in}{1.007434in}}%
\pgfpathlineto{\pgfqpoint{1.214286in}{1.251178in}}%
\pgfpathlineto{\pgfqpoint{1.217249in}{1.006338in}}%
\pgfpathlineto{\pgfqpoint{1.220213in}{1.033283in}}%
\pgfpathlineto{\pgfqpoint{1.223176in}{1.193048in}}%
\pgfpathlineto{\pgfqpoint{1.226139in}{1.028700in}}%
\pgfpathlineto{\pgfqpoint{1.229102in}{1.231415in}}%
\pgfpathlineto{\pgfqpoint{1.232066in}{1.038328in}}%
\pgfpathlineto{\pgfqpoint{1.235029in}{1.189278in}}%
\pgfpathlineto{\pgfqpoint{1.237992in}{1.115286in}}%
\pgfpathlineto{\pgfqpoint{1.240955in}{0.980579in}}%
\pgfpathlineto{\pgfqpoint{1.243919in}{0.996348in}}%
\pgfpathlineto{\pgfqpoint{1.246882in}{0.968787in}}%
\pgfpathlineto{\pgfqpoint{1.249845in}{1.032060in}}%
\pgfpathlineto{\pgfqpoint{1.252808in}{0.962999in}}%
\pgfpathlineto{\pgfqpoint{1.255772in}{0.997881in}}%
\pgfpathlineto{\pgfqpoint{1.258735in}{1.008959in}}%
\pgfpathlineto{\pgfqpoint{1.261698in}{0.994647in}}%
\pgfpathlineto{\pgfqpoint{1.264661in}{1.019901in}}%
\pgfpathlineto{\pgfqpoint{1.267625in}{1.234062in}}%
\pgfpathlineto{\pgfqpoint{1.270588in}{1.322532in}}%
\pgfpathlineto{\pgfqpoint{1.273551in}{1.001305in}}%
\pgfpathlineto{\pgfqpoint{1.276514in}{1.088108in}}%
\pgfpathlineto{\pgfqpoint{1.279478in}{0.964131in}}%
\pgfpathlineto{\pgfqpoint{1.282441in}{1.138493in}}%
\pgfpathlineto{\pgfqpoint{1.285404in}{1.024820in}}%
\pgfpathlineto{\pgfqpoint{1.288367in}{0.969307in}}%
\pgfpathlineto{\pgfqpoint{1.291331in}{0.952183in}}%
\pgfpathlineto{\pgfqpoint{1.294294in}{0.955518in}}%
\pgfpathlineto{\pgfqpoint{1.297257in}{1.004090in}}%
\pgfpathlineto{\pgfqpoint{1.300220in}{0.943119in}}%
\pgfpathlineto{\pgfqpoint{1.303184in}{1.029681in}}%
\pgfpathlineto{\pgfqpoint{1.306147in}{1.022436in}}%
\pgfpathlineto{\pgfqpoint{1.309110in}{1.038333in}}%
\pgfpathlineto{\pgfqpoint{1.312073in}{1.066778in}}%
\pgfpathlineto{\pgfqpoint{1.315037in}{0.940317in}}%
\pgfpathlineto{\pgfqpoint{1.318000in}{0.968993in}}%
\pgfpathlineto{\pgfqpoint{1.320963in}{0.936426in}}%
\pgfpathlineto{\pgfqpoint{1.323926in}{0.936088in}}%
\pgfpathlineto{\pgfqpoint{1.326890in}{0.942946in}}%
\pgfpathlineto{\pgfqpoint{1.332816in}{0.976758in}}%
\pgfpathlineto{\pgfqpoint{1.335779in}{0.938951in}}%
\pgfpathlineto{\pgfqpoint{1.338743in}{1.001758in}}%
\pgfpathlineto{\pgfqpoint{1.341706in}{0.947987in}}%
\pgfpathlineto{\pgfqpoint{1.344669in}{0.934878in}}%
\pgfpathlineto{\pgfqpoint{1.347632in}{0.962232in}}%
\pgfpathlineto{\pgfqpoint{1.350596in}{0.978727in}}%
\pgfpathlineto{\pgfqpoint{1.353559in}{0.935224in}}%
\pgfpathlineto{\pgfqpoint{1.356522in}{0.930303in}}%
\pgfpathlineto{\pgfqpoint{1.359485in}{0.917630in}}%
\pgfpathlineto{\pgfqpoint{1.362449in}{0.955535in}}%
\pgfpathlineto{\pgfqpoint{1.365412in}{0.942177in}}%
\pgfpathlineto{\pgfqpoint{1.368375in}{1.013611in}}%
\pgfpathlineto{\pgfqpoint{1.371338in}{0.952282in}}%
\pgfpathlineto{\pgfqpoint{1.374302in}{0.975088in}}%
\pgfpathlineto{\pgfqpoint{1.377265in}{0.927501in}}%
\pgfpathlineto{\pgfqpoint{1.380228in}{0.924309in}}%
\pgfpathlineto{\pgfqpoint{1.383191in}{0.918772in}}%
\pgfpathlineto{\pgfqpoint{1.386155in}{0.930365in}}%
\pgfpathlineto{\pgfqpoint{1.389118in}{0.916737in}}%
\pgfpathlineto{\pgfqpoint{1.392081in}{0.960519in}}%
\pgfpathlineto{\pgfqpoint{1.395045in}{1.052852in}}%
\pgfpathlineto{\pgfqpoint{1.398008in}{0.916222in}}%
\pgfpathlineto{\pgfqpoint{1.400971in}{1.025160in}}%
\pgfpathlineto{\pgfqpoint{1.403934in}{0.998804in}}%
\pgfpathlineto{\pgfqpoint{1.406898in}{1.010346in}}%
\pgfpathlineto{\pgfqpoint{1.409861in}{1.100325in}}%
\pgfpathlineto{\pgfqpoint{1.412824in}{0.941327in}}%
\pgfpathlineto{\pgfqpoint{1.415787in}{0.933012in}}%
\pgfpathlineto{\pgfqpoint{1.418751in}{0.958288in}}%
\pgfpathlineto{\pgfqpoint{1.421714in}{1.019208in}}%
\pgfpathlineto{\pgfqpoint{1.427640in}{0.918002in}}%
\pgfpathlineto{\pgfqpoint{1.430604in}{1.078564in}}%
\pgfpathlineto{\pgfqpoint{1.433567in}{0.945417in}}%
\pgfpathlineto{\pgfqpoint{1.436530in}{0.968274in}}%
\pgfpathlineto{\pgfqpoint{1.439493in}{0.898310in}}%
\pgfpathlineto{\pgfqpoint{1.442457in}{0.912650in}}%
\pgfpathlineto{\pgfqpoint{1.445420in}{1.006855in}}%
\pgfpathlineto{\pgfqpoint{1.448383in}{0.887910in}}%
\pgfpathlineto{\pgfqpoint{1.451346in}{0.944709in}}%
\pgfpathlineto{\pgfqpoint{1.454310in}{0.907280in}}%
\pgfpathlineto{\pgfqpoint{1.457273in}{0.969872in}}%
\pgfpathlineto{\pgfqpoint{1.460236in}{0.915397in}}%
\pgfpathlineto{\pgfqpoint{1.463199in}{0.917257in}}%
\pgfpathlineto{\pgfqpoint{1.466163in}{1.021181in}}%
\pgfpathlineto{\pgfqpoint{1.469126in}{0.894136in}}%
\pgfpathlineto{\pgfqpoint{1.472089in}{1.010720in}}%
\pgfpathlineto{\pgfqpoint{1.475052in}{1.000002in}}%
\pgfpathlineto{\pgfqpoint{1.478016in}{0.887375in}}%
\pgfpathlineto{\pgfqpoint{1.480979in}{0.945738in}}%
\pgfpathlineto{\pgfqpoint{1.486905in}{0.902891in}}%
\pgfpathlineto{\pgfqpoint{1.489869in}{0.924648in}}%
\pgfpathlineto{\pgfqpoint{1.492832in}{0.897689in}}%
\pgfpathlineto{\pgfqpoint{1.495795in}{0.927387in}}%
\pgfpathlineto{\pgfqpoint{1.498758in}{0.922777in}}%
\pgfpathlineto{\pgfqpoint{1.501722in}{0.947816in}}%
\pgfpathlineto{\pgfqpoint{1.504685in}{0.927692in}}%
\pgfpathlineto{\pgfqpoint{1.507648in}{0.941060in}}%
\pgfpathlineto{\pgfqpoint{1.513575in}{0.875290in}}%
\pgfpathlineto{\pgfqpoint{1.516538in}{0.929939in}}%
\pgfpathlineto{\pgfqpoint{1.519501in}{0.885001in}}%
\pgfpathlineto{\pgfqpoint{1.522464in}{0.934225in}}%
\pgfpathlineto{\pgfqpoint{1.525428in}{0.908134in}}%
\pgfpathlineto{\pgfqpoint{1.528391in}{0.917559in}}%
\pgfpathlineto{\pgfqpoint{1.531354in}{0.935237in}}%
\pgfpathlineto{\pgfqpoint{1.534317in}{0.985592in}}%
\pgfpathlineto{\pgfqpoint{1.537281in}{0.901808in}}%
\pgfpathlineto{\pgfqpoint{1.540244in}{0.925545in}}%
\pgfpathlineto{\pgfqpoint{1.543207in}{1.021045in}}%
\pgfpathlineto{\pgfqpoint{1.546170in}{0.978677in}}%
\pgfpathlineto{\pgfqpoint{1.549134in}{0.871002in}}%
\pgfpathlineto{\pgfqpoint{1.552097in}{0.924817in}}%
\pgfpathlineto{\pgfqpoint{1.558023in}{0.850154in}}%
\pgfpathlineto{\pgfqpoint{1.560987in}{0.871269in}}%
\pgfpathlineto{\pgfqpoint{1.563950in}{0.861064in}}%
\pgfpathlineto{\pgfqpoint{1.566913in}{0.881640in}}%
\pgfpathlineto{\pgfqpoint{1.569876in}{0.941681in}}%
\pgfpathlineto{\pgfqpoint{1.572840in}{0.886177in}}%
\pgfpathlineto{\pgfqpoint{1.578766in}{0.875556in}}%
\pgfpathlineto{\pgfqpoint{1.581729in}{0.909904in}}%
\pgfpathlineto{\pgfqpoint{1.584693in}{0.868756in}}%
\pgfpathlineto{\pgfqpoint{1.587656in}{0.873235in}}%
\pgfpathlineto{\pgfqpoint{1.593582in}{0.858873in}}%
\pgfpathlineto{\pgfqpoint{1.596546in}{0.841262in}}%
\pgfpathlineto{\pgfqpoint{1.599509in}{0.904631in}}%
\pgfpathlineto{\pgfqpoint{1.602472in}{0.911887in}}%
\pgfpathlineto{\pgfqpoint{1.605435in}{0.848679in}}%
\pgfpathlineto{\pgfqpoint{1.608399in}{0.871279in}}%
\pgfpathlineto{\pgfqpoint{1.611362in}{0.872705in}}%
\pgfpathlineto{\pgfqpoint{1.617289in}{0.850563in}}%
\pgfpathlineto{\pgfqpoint{1.620252in}{0.850209in}}%
\pgfpathlineto{\pgfqpoint{1.623215in}{0.862684in}}%
\pgfpathlineto{\pgfqpoint{1.626178in}{0.892560in}}%
\pgfpathlineto{\pgfqpoint{1.629142in}{0.842898in}}%
\pgfpathlineto{\pgfqpoint{1.632105in}{0.889386in}}%
\pgfpathlineto{\pgfqpoint{1.635068in}{0.916679in}}%
\pgfpathlineto{\pgfqpoint{1.638031in}{0.863837in}}%
\pgfpathlineto{\pgfqpoint{1.640995in}{0.878871in}}%
\pgfpathlineto{\pgfqpoint{1.643958in}{0.868971in}}%
\pgfpathlineto{\pgfqpoint{1.646921in}{0.848223in}}%
\pgfpathlineto{\pgfqpoint{1.649884in}{0.853227in}}%
\pgfpathlineto{\pgfqpoint{1.652848in}{0.850510in}}%
\pgfpathlineto{\pgfqpoint{1.655811in}{0.838056in}}%
\pgfpathlineto{\pgfqpoint{1.658774in}{0.881822in}}%
\pgfpathlineto{\pgfqpoint{1.661737in}{0.957909in}}%
\pgfpathlineto{\pgfqpoint{1.664701in}{0.867216in}}%
\pgfpathlineto{\pgfqpoint{1.667664in}{0.879095in}}%
\pgfpathlineto{\pgfqpoint{1.670627in}{0.869011in}}%
\pgfpathlineto{\pgfqpoint{1.673590in}{0.839843in}}%
\pgfpathlineto{\pgfqpoint{1.676554in}{0.870722in}}%
\pgfpathlineto{\pgfqpoint{1.679517in}{0.855585in}}%
\pgfpathlineto{\pgfqpoint{1.682480in}{0.874708in}}%
\pgfpathlineto{\pgfqpoint{1.685443in}{0.978444in}}%
\pgfpathlineto{\pgfqpoint{1.691370in}{0.960704in}}%
\pgfpathlineto{\pgfqpoint{1.694333in}{0.839783in}}%
\pgfpathlineto{\pgfqpoint{1.697296in}{0.843666in}}%
\pgfpathlineto{\pgfqpoint{1.700260in}{0.844030in}}%
\pgfpathlineto{\pgfqpoint{1.703223in}{0.916739in}}%
\pgfpathlineto{\pgfqpoint{1.706186in}{0.850942in}}%
\pgfpathlineto{\pgfqpoint{1.709149in}{0.895198in}}%
\pgfpathlineto{\pgfqpoint{1.712113in}{0.891399in}}%
\pgfpathlineto{\pgfqpoint{1.715076in}{0.851907in}}%
\pgfpathlineto{\pgfqpoint{1.718039in}{0.876326in}}%
\pgfpathlineto{\pgfqpoint{1.721002in}{0.858039in}}%
\pgfpathlineto{\pgfqpoint{1.723966in}{0.814309in}}%
\pgfpathlineto{\pgfqpoint{1.729892in}{0.873769in}}%
\pgfpathlineto{\pgfqpoint{1.732855in}{0.826616in}}%
\pgfpathlineto{\pgfqpoint{1.738782in}{1.111164in}}%
\pgfpathlineto{\pgfqpoint{1.741745in}{0.836392in}}%
\pgfpathlineto{\pgfqpoint{1.744708in}{0.839383in}}%
\pgfpathlineto{\pgfqpoint{1.747672in}{0.823982in}}%
\pgfpathlineto{\pgfqpoint{1.750635in}{0.855576in}}%
\pgfpathlineto{\pgfqpoint{1.753598in}{0.816762in}}%
\pgfpathlineto{\pgfqpoint{1.756561in}{0.890525in}}%
\pgfpathlineto{\pgfqpoint{1.759525in}{0.809452in}}%
\pgfpathlineto{\pgfqpoint{1.762488in}{0.829537in}}%
\pgfpathlineto{\pgfqpoint{1.765451in}{0.836103in}}%
\pgfpathlineto{\pgfqpoint{1.768414in}{0.895379in}}%
\pgfpathlineto{\pgfqpoint{1.774341in}{0.802461in}}%
\pgfpathlineto{\pgfqpoint{1.777304in}{0.815067in}}%
\pgfpathlineto{\pgfqpoint{1.780267in}{0.816231in}}%
\pgfpathlineto{\pgfqpoint{1.783231in}{0.815190in}}%
\pgfpathlineto{\pgfqpoint{1.786194in}{0.802745in}}%
\pgfpathlineto{\pgfqpoint{1.792120in}{0.920312in}}%
\pgfpathlineto{\pgfqpoint{1.795084in}{0.827863in}}%
\pgfpathlineto{\pgfqpoint{1.798047in}{0.828955in}}%
\pgfpathlineto{\pgfqpoint{1.801010in}{0.857918in}}%
\pgfpathlineto{\pgfqpoint{1.803973in}{0.801421in}}%
\pgfpathlineto{\pgfqpoint{1.806937in}{0.875381in}}%
\pgfpathlineto{\pgfqpoint{1.809900in}{0.807836in}}%
\pgfpathlineto{\pgfqpoint{1.812863in}{0.975615in}}%
\pgfpathlineto{\pgfqpoint{1.815826in}{0.939389in}}%
\pgfpathlineto{\pgfqpoint{1.818790in}{0.806222in}}%
\pgfpathlineto{\pgfqpoint{1.821753in}{0.965771in}}%
\pgfpathlineto{\pgfqpoint{1.827679in}{0.866954in}}%
\pgfpathlineto{\pgfqpoint{1.830643in}{0.866013in}}%
\pgfpathlineto{\pgfqpoint{1.833606in}{0.817374in}}%
\pgfpathlineto{\pgfqpoint{1.836569in}{0.916813in}}%
\pgfpathlineto{\pgfqpoint{1.839532in}{0.812688in}}%
\pgfpathlineto{\pgfqpoint{1.842496in}{0.846586in}}%
\pgfpathlineto{\pgfqpoint{1.845459in}{0.924745in}}%
\pgfpathlineto{\pgfqpoint{1.848422in}{0.803664in}}%
\pgfpathlineto{\pgfqpoint{1.851386in}{0.925374in}}%
\pgfpathlineto{\pgfqpoint{1.854349in}{0.829607in}}%
\pgfpathlineto{\pgfqpoint{1.857312in}{0.876728in}}%
\pgfpathlineto{\pgfqpoint{1.860275in}{0.800943in}}%
\pgfpathlineto{\pgfqpoint{1.863239in}{0.853437in}}%
\pgfpathlineto{\pgfqpoint{1.866202in}{0.800654in}}%
\pgfpathlineto{\pgfqpoint{1.872128in}{0.952760in}}%
\pgfpathlineto{\pgfqpoint{1.875092in}{0.809304in}}%
\pgfpathlineto{\pgfqpoint{1.878055in}{0.829769in}}%
\pgfpathlineto{\pgfqpoint{1.881018in}{0.789627in}}%
\pgfpathlineto{\pgfqpoint{1.883981in}{0.834048in}}%
\pgfpathlineto{\pgfqpoint{1.886945in}{0.933546in}}%
\pgfpathlineto{\pgfqpoint{1.889908in}{0.817605in}}%
\pgfpathlineto{\pgfqpoint{1.892871in}{0.775437in}}%
\pgfpathlineto{\pgfqpoint{1.895834in}{0.862689in}}%
\pgfpathlineto{\pgfqpoint{1.898798in}{0.796777in}}%
\pgfpathlineto{\pgfqpoint{1.901761in}{0.808091in}}%
\pgfpathlineto{\pgfqpoint{1.904724in}{0.796208in}}%
\pgfpathlineto{\pgfqpoint{1.907687in}{0.815243in}}%
\pgfpathlineto{\pgfqpoint{1.910651in}{0.878055in}}%
\pgfpathlineto{\pgfqpoint{1.913614in}{0.790236in}}%
\pgfpathlineto{\pgfqpoint{1.916577in}{0.894048in}}%
\pgfpathlineto{\pgfqpoint{1.919540in}{0.822962in}}%
\pgfpathlineto{\pgfqpoint{1.922504in}{0.778970in}}%
\pgfpathlineto{\pgfqpoint{1.925467in}{0.787870in}}%
\pgfpathlineto{\pgfqpoint{1.928430in}{0.791684in}}%
\pgfpathlineto{\pgfqpoint{1.931393in}{0.774696in}}%
\pgfpathlineto{\pgfqpoint{1.937320in}{0.805449in}}%
\pgfpathlineto{\pgfqpoint{1.940283in}{0.879555in}}%
\pgfpathlineto{\pgfqpoint{1.943246in}{0.868427in}}%
\pgfpathlineto{\pgfqpoint{1.946210in}{0.786726in}}%
\pgfpathlineto{\pgfqpoint{1.949173in}{0.825338in}}%
\pgfpathlineto{\pgfqpoint{1.952136in}{0.776657in}}%
\pgfpathlineto{\pgfqpoint{1.955099in}{0.784336in}}%
\pgfpathlineto{\pgfqpoint{1.958063in}{0.809063in}}%
\pgfpathlineto{\pgfqpoint{1.961026in}{0.789625in}}%
\pgfpathlineto{\pgfqpoint{1.963989in}{0.784398in}}%
\pgfpathlineto{\pgfqpoint{1.966952in}{0.772862in}}%
\pgfpathlineto{\pgfqpoint{1.969916in}{0.861222in}}%
\pgfpathlineto{\pgfqpoint{1.972879in}{0.815330in}}%
\pgfpathlineto{\pgfqpoint{1.975842in}{0.902652in}}%
\pgfpathlineto{\pgfqpoint{1.978805in}{0.815148in}}%
\pgfpathlineto{\pgfqpoint{1.981769in}{0.874893in}}%
\pgfpathlineto{\pgfqpoint{1.984732in}{0.779296in}}%
\pgfpathlineto{\pgfqpoint{1.987695in}{0.777695in}}%
\pgfpathlineto{\pgfqpoint{1.993622in}{0.849874in}}%
\pgfpathlineto{\pgfqpoint{1.996585in}{0.775580in}}%
\pgfpathlineto{\pgfqpoint{1.999548in}{0.784029in}}%
\pgfpathlineto{\pgfqpoint{2.002511in}{0.895845in}}%
\pgfpathlineto{\pgfqpoint{2.005475in}{0.833134in}}%
\pgfpathlineto{\pgfqpoint{2.008438in}{0.926474in}}%
\pgfpathlineto{\pgfqpoint{2.011401in}{0.764185in}}%
\pgfpathlineto{\pgfqpoint{2.014364in}{0.818280in}}%
\pgfpathlineto{\pgfqpoint{2.017328in}{0.793955in}}%
\pgfpathlineto{\pgfqpoint{2.020291in}{0.850619in}}%
\pgfpathlineto{\pgfqpoint{2.023254in}{0.765267in}}%
\pgfpathlineto{\pgfqpoint{2.026217in}{0.749447in}}%
\pgfpathlineto{\pgfqpoint{2.029181in}{0.846210in}}%
\pgfpathlineto{\pgfqpoint{2.032144in}{0.777378in}}%
\pgfpathlineto{\pgfqpoint{2.035107in}{0.831463in}}%
\pgfpathlineto{\pgfqpoint{2.038070in}{0.744632in}}%
\pgfpathlineto{\pgfqpoint{2.041034in}{0.791061in}}%
\pgfpathlineto{\pgfqpoint{2.043997in}{0.790219in}}%
\pgfpathlineto{\pgfqpoint{2.046960in}{0.765385in}}%
\pgfpathlineto{\pgfqpoint{2.049923in}{0.777471in}}%
\pgfpathlineto{\pgfqpoint{2.052887in}{0.773317in}}%
\pgfpathlineto{\pgfqpoint{2.058813in}{0.775140in}}%
\pgfpathlineto{\pgfqpoint{2.061776in}{0.760543in}}%
\pgfpathlineto{\pgfqpoint{2.064740in}{0.773790in}}%
\pgfpathlineto{\pgfqpoint{2.067703in}{0.780425in}}%
\pgfpathlineto{\pgfqpoint{2.070666in}{0.795995in}}%
\pgfpathlineto{\pgfqpoint{2.073630in}{0.756448in}}%
\pgfpathlineto{\pgfqpoint{2.076593in}{0.775545in}}%
\pgfpathlineto{\pgfqpoint{2.079556in}{0.836459in}}%
\pgfpathlineto{\pgfqpoint{2.082519in}{0.828336in}}%
\pgfpathlineto{\pgfqpoint{2.085483in}{0.869130in}}%
\pgfpathlineto{\pgfqpoint{2.088446in}{0.808887in}}%
\pgfpathlineto{\pgfqpoint{2.091409in}{0.889408in}}%
\pgfpathlineto{\pgfqpoint{2.094372in}{0.751572in}}%
\pgfpathlineto{\pgfqpoint{2.097336in}{0.787543in}}%
\pgfpathlineto{\pgfqpoint{2.100299in}{0.799146in}}%
\pgfpathlineto{\pgfqpoint{2.103262in}{0.996079in}}%
\pgfpathlineto{\pgfqpoint{2.109189in}{0.821848in}}%
\pgfpathlineto{\pgfqpoint{2.112152in}{0.914894in}}%
\pgfpathlineto{\pgfqpoint{2.115115in}{0.749908in}}%
\pgfpathlineto{\pgfqpoint{2.118078in}{0.797997in}}%
\pgfpathlineto{\pgfqpoint{2.121042in}{0.783609in}}%
\pgfpathlineto{\pgfqpoint{2.124005in}{0.776244in}}%
\pgfpathlineto{\pgfqpoint{2.126968in}{0.739570in}}%
\pgfpathlineto{\pgfqpoint{2.129931in}{0.801664in}}%
\pgfpathlineto{\pgfqpoint{2.132895in}{0.750292in}}%
\pgfpathlineto{\pgfqpoint{2.135858in}{0.765240in}}%
\pgfpathlineto{\pgfqpoint{2.138821in}{0.831018in}}%
\pgfpathlineto{\pgfqpoint{2.141784in}{0.743555in}}%
\pgfpathlineto{\pgfqpoint{2.144748in}{0.850716in}}%
\pgfpathlineto{\pgfqpoint{2.147711in}{0.734505in}}%
\pgfpathlineto{\pgfqpoint{2.150674in}{0.740035in}}%
\pgfpathlineto{\pgfqpoint{2.153637in}{0.737888in}}%
\pgfpathlineto{\pgfqpoint{2.156601in}{0.762845in}}%
\pgfpathlineto{\pgfqpoint{2.159564in}{0.807726in}}%
\pgfpathlineto{\pgfqpoint{2.162527in}{0.880379in}}%
\pgfpathlineto{\pgfqpoint{2.165490in}{0.745675in}}%
\pgfpathlineto{\pgfqpoint{2.168454in}{0.791788in}}%
\pgfpathlineto{\pgfqpoint{2.171417in}{0.811541in}}%
\pgfpathlineto{\pgfqpoint{2.174380in}{0.741661in}}%
\pgfpathlineto{\pgfqpoint{2.177343in}{0.741174in}}%
\pgfpathlineto{\pgfqpoint{2.180307in}{0.745293in}}%
\pgfpathlineto{\pgfqpoint{2.183270in}{0.737534in}}%
\pgfpathlineto{\pgfqpoint{2.189196in}{0.825259in}}%
\pgfpathlineto{\pgfqpoint{2.195123in}{0.771429in}}%
\pgfpathlineto{\pgfqpoint{2.198086in}{0.820910in}}%
\pgfpathlineto{\pgfqpoint{2.201049in}{0.757620in}}%
\pgfpathlineto{\pgfqpoint{2.204013in}{0.734606in}}%
\pgfpathlineto{\pgfqpoint{2.209939in}{0.780687in}}%
\pgfpathlineto{\pgfqpoint{2.212902in}{0.734509in}}%
\pgfpathlineto{\pgfqpoint{2.215866in}{0.738473in}}%
\pgfpathlineto{\pgfqpoint{2.221792in}{0.804590in}}%
\pgfpathlineto{\pgfqpoint{2.224755in}{0.831461in}}%
\pgfpathlineto{\pgfqpoint{2.227719in}{0.756661in}}%
\pgfpathlineto{\pgfqpoint{2.230682in}{0.748257in}}%
\pgfpathlineto{\pgfqpoint{2.233645in}{0.908025in}}%
\pgfpathlineto{\pgfqpoint{2.236608in}{0.839150in}}%
\pgfpathlineto{\pgfqpoint{2.239572in}{0.722200in}}%
\pgfpathlineto{\pgfqpoint{2.242535in}{0.741924in}}%
\pgfpathlineto{\pgfqpoint{2.245498in}{0.739686in}}%
\pgfpathlineto{\pgfqpoint{2.248461in}{0.772694in}}%
\pgfpathlineto{\pgfqpoint{2.251425in}{0.735730in}}%
\pgfpathlineto{\pgfqpoint{2.254388in}{0.720633in}}%
\pgfpathlineto{\pgfqpoint{2.257351in}{0.748371in}}%
\pgfpathlineto{\pgfqpoint{2.260314in}{0.716376in}}%
\pgfpathlineto{\pgfqpoint{2.263278in}{0.732685in}}%
\pgfpathlineto{\pgfqpoint{2.266241in}{0.732912in}}%
\pgfpathlineto{\pgfqpoint{2.269204in}{0.728100in}}%
\pgfpathlineto{\pgfqpoint{2.272167in}{0.811039in}}%
\pgfpathlineto{\pgfqpoint{2.275131in}{0.809824in}}%
\pgfpathlineto{\pgfqpoint{2.278094in}{0.762491in}}%
\pgfpathlineto{\pgfqpoint{2.281057in}{0.748254in}}%
\pgfpathlineto{\pgfqpoint{2.284020in}{0.742355in}}%
\pgfpathlineto{\pgfqpoint{2.286984in}{0.730469in}}%
\pgfpathlineto{\pgfqpoint{2.289947in}{0.787186in}}%
\pgfpathlineto{\pgfqpoint{2.292910in}{0.888631in}}%
\pgfpathlineto{\pgfqpoint{2.295874in}{0.719739in}}%
\pgfpathlineto{\pgfqpoint{2.298837in}{0.748121in}}%
\pgfpathlineto{\pgfqpoint{2.301800in}{0.790711in}}%
\pgfpathlineto{\pgfqpoint{2.304763in}{0.759920in}}%
\pgfpathlineto{\pgfqpoint{2.307727in}{0.753986in}}%
\pgfpathlineto{\pgfqpoint{2.310690in}{0.793603in}}%
\pgfpathlineto{\pgfqpoint{2.313653in}{0.719183in}}%
\pgfpathlineto{\pgfqpoint{2.316616in}{0.771075in}}%
\pgfpathlineto{\pgfqpoint{2.319580in}{0.769386in}}%
\pgfpathlineto{\pgfqpoint{2.322543in}{0.735873in}}%
\pgfpathlineto{\pgfqpoint{2.325506in}{0.775318in}}%
\pgfpathlineto{\pgfqpoint{2.328469in}{0.790013in}}%
\pgfpathlineto{\pgfqpoint{2.331433in}{0.709324in}}%
\pgfpathlineto{\pgfqpoint{2.334396in}{0.783146in}}%
\pgfpathlineto{\pgfqpoint{2.337359in}{0.718314in}}%
\pgfpathlineto{\pgfqpoint{2.340322in}{0.730251in}}%
\pgfpathlineto{\pgfqpoint{2.343286in}{0.788436in}}%
\pgfpathlineto{\pgfqpoint{2.346249in}{0.813811in}}%
\pgfpathlineto{\pgfqpoint{2.349212in}{0.759077in}}%
\pgfpathlineto{\pgfqpoint{2.352175in}{0.855355in}}%
\pgfpathlineto{\pgfqpoint{2.355139in}{0.779163in}}%
\pgfpathlineto{\pgfqpoint{2.361065in}{0.743050in}}%
\pgfpathlineto{\pgfqpoint{2.364028in}{0.811730in}}%
\pgfpathlineto{\pgfqpoint{2.366992in}{0.717732in}}%
\pgfpathlineto{\pgfqpoint{2.369955in}{0.762400in}}%
\pgfpathlineto{\pgfqpoint{2.372918in}{0.725016in}}%
\pgfpathlineto{\pgfqpoint{2.375881in}{0.725407in}}%
\pgfpathlineto{\pgfqpoint{2.378845in}{0.792710in}}%
\pgfpathlineto{\pgfqpoint{2.381808in}{0.716633in}}%
\pgfpathlineto{\pgfqpoint{2.384771in}{0.857698in}}%
\pgfpathlineto{\pgfqpoint{2.387734in}{0.819349in}}%
\pgfpathlineto{\pgfqpoint{2.390698in}{0.811021in}}%
\pgfpathlineto{\pgfqpoint{2.393661in}{0.768540in}}%
\pgfpathlineto{\pgfqpoint{2.396624in}{0.749690in}}%
\pgfpathlineto{\pgfqpoint{2.399587in}{0.812995in}}%
\pgfpathlineto{\pgfqpoint{2.402551in}{0.714112in}}%
\pgfpathlineto{\pgfqpoint{2.405514in}{0.789630in}}%
\pgfpathlineto{\pgfqpoint{2.408477in}{0.721621in}}%
\pgfpathlineto{\pgfqpoint{2.411440in}{0.696357in}}%
\pgfpathlineto{\pgfqpoint{2.414404in}{0.782958in}}%
\pgfpathlineto{\pgfqpoint{2.417367in}{0.748120in}}%
\pgfpathlineto{\pgfqpoint{2.420330in}{0.830458in}}%
\pgfpathlineto{\pgfqpoint{2.423293in}{0.753762in}}%
\pgfpathlineto{\pgfqpoint{2.426257in}{0.730640in}}%
\pgfpathlineto{\pgfqpoint{2.429220in}{0.755433in}}%
\pgfpathlineto{\pgfqpoint{2.432183in}{0.719881in}}%
\pgfpathlineto{\pgfqpoint{2.435146in}{0.709723in}}%
\pgfpathlineto{\pgfqpoint{2.438110in}{0.858334in}}%
\pgfpathlineto{\pgfqpoint{2.441073in}{0.740121in}}%
\pgfpathlineto{\pgfqpoint{2.444036in}{0.826502in}}%
\pgfpathlineto{\pgfqpoint{2.446999in}{0.720911in}}%
\pgfpathlineto{\pgfqpoint{2.449963in}{0.706149in}}%
\pgfpathlineto{\pgfqpoint{2.452926in}{0.827846in}}%
\pgfpathlineto{\pgfqpoint{2.455889in}{0.756706in}}%
\pgfpathlineto{\pgfqpoint{2.458852in}{0.720084in}}%
\pgfpathlineto{\pgfqpoint{2.461816in}{0.798683in}}%
\pgfpathlineto{\pgfqpoint{2.467742in}{0.723570in}}%
\pgfpathlineto{\pgfqpoint{2.470705in}{0.773790in}}%
\pgfpathlineto{\pgfqpoint{2.473669in}{0.758862in}}%
\pgfpathlineto{\pgfqpoint{2.476632in}{0.835496in}}%
\pgfpathlineto{\pgfqpoint{2.479595in}{0.721119in}}%
\pgfpathlineto{\pgfqpoint{2.482558in}{0.695731in}}%
\pgfpathlineto{\pgfqpoint{2.485522in}{0.805285in}}%
\pgfpathlineto{\pgfqpoint{2.488485in}{0.723762in}}%
\pgfpathlineto{\pgfqpoint{2.491448in}{0.702058in}}%
\pgfpathlineto{\pgfqpoint{2.497375in}{0.774627in}}%
\pgfpathlineto{\pgfqpoint{2.500338in}{0.802100in}}%
\pgfpathlineto{\pgfqpoint{2.503301in}{0.725231in}}%
\pgfpathlineto{\pgfqpoint{2.506264in}{0.707614in}}%
\pgfpathlineto{\pgfqpoint{2.509228in}{0.743234in}}%
\pgfpathlineto{\pgfqpoint{2.512191in}{0.836051in}}%
\pgfpathlineto{\pgfqpoint{2.515154in}{0.745700in}}%
\pgfpathlineto{\pgfqpoint{2.518117in}{0.816564in}}%
\pgfpathlineto{\pgfqpoint{2.521081in}{0.700652in}}%
\pgfpathlineto{\pgfqpoint{2.524044in}{0.713389in}}%
\pgfpathlineto{\pgfqpoint{2.527007in}{0.686631in}}%
\pgfpathlineto{\pgfqpoint{2.532934in}{0.790137in}}%
\pgfpathlineto{\pgfqpoint{2.535897in}{0.725258in}}%
\pgfpathlineto{\pgfqpoint{2.538860in}{0.717412in}}%
\pgfpathlineto{\pgfqpoint{2.541824in}{0.755939in}}%
\pgfpathlineto{\pgfqpoint{2.544787in}{0.739844in}}%
\pgfpathlineto{\pgfqpoint{2.547750in}{0.807607in}}%
\pgfpathlineto{\pgfqpoint{2.550713in}{0.803902in}}%
\pgfpathlineto{\pgfqpoint{2.553677in}{0.712405in}}%
\pgfpathlineto{\pgfqpoint{2.556640in}{0.690285in}}%
\pgfpathlineto{\pgfqpoint{2.559603in}{0.781839in}}%
\pgfpathlineto{\pgfqpoint{2.562566in}{0.755126in}}%
\pgfpathlineto{\pgfqpoint{2.565530in}{0.766200in}}%
\pgfpathlineto{\pgfqpoint{2.568493in}{0.713966in}}%
\pgfpathlineto{\pgfqpoint{2.571456in}{0.712914in}}%
\pgfpathlineto{\pgfqpoint{2.574419in}{0.805293in}}%
\pgfpathlineto{\pgfqpoint{2.577383in}{0.742914in}}%
\pgfpathlineto{\pgfqpoint{2.580346in}{0.951873in}}%
\pgfpathlineto{\pgfqpoint{2.583309in}{0.705292in}}%
\pgfpathlineto{\pgfqpoint{2.586272in}{0.684675in}}%
\pgfpathlineto{\pgfqpoint{2.589236in}{0.730009in}}%
\pgfpathlineto{\pgfqpoint{2.592199in}{0.693258in}}%
\pgfpathlineto{\pgfqpoint{2.595162in}{0.777855in}}%
\pgfpathlineto{\pgfqpoint{2.598125in}{0.819672in}}%
\pgfpathlineto{\pgfqpoint{2.601089in}{0.834964in}}%
\pgfpathlineto{\pgfqpoint{2.604052in}{0.693035in}}%
\pgfpathlineto{\pgfqpoint{2.607015in}{0.720806in}}%
\pgfpathlineto{\pgfqpoint{2.609978in}{0.681654in}}%
\pgfpathlineto{\pgfqpoint{2.612942in}{0.855808in}}%
\pgfpathlineto{\pgfqpoint{2.615905in}{0.770902in}}%
\pgfpathlineto{\pgfqpoint{2.618868in}{0.778358in}}%
\pgfpathlineto{\pgfqpoint{2.621831in}{0.768450in}}%
\pgfpathlineto{\pgfqpoint{2.624795in}{0.769411in}}%
\pgfpathlineto{\pgfqpoint{2.627758in}{0.790878in}}%
\pgfpathlineto{\pgfqpoint{2.630721in}{0.677872in}}%
\pgfpathlineto{\pgfqpoint{2.633684in}{0.745445in}}%
\pgfpathlineto{\pgfqpoint{2.636648in}{0.708349in}}%
\pgfpathlineto{\pgfqpoint{2.639611in}{0.714352in}}%
\pgfpathlineto{\pgfqpoint{2.642574in}{0.734767in}}%
\pgfpathlineto{\pgfqpoint{2.645537in}{0.703168in}}%
\pgfpathlineto{\pgfqpoint{2.648501in}{0.746172in}}%
\pgfpathlineto{\pgfqpoint{2.651464in}{0.758026in}}%
\pgfpathlineto{\pgfqpoint{2.654427in}{0.683175in}}%
\pgfpathlineto{\pgfqpoint{2.657390in}{0.787169in}}%
\pgfpathlineto{\pgfqpoint{2.660354in}{0.781827in}}%
\pgfpathlineto{\pgfqpoint{2.663317in}{0.712328in}}%
\pgfpathlineto{\pgfqpoint{2.666280in}{0.723411in}}%
\pgfpathlineto{\pgfqpoint{2.669243in}{0.706794in}}%
\pgfpathlineto{\pgfqpoint{2.675170in}{0.765306in}}%
\pgfpathlineto{\pgfqpoint{2.678133in}{0.688962in}}%
\pgfpathlineto{\pgfqpoint{2.681096in}{0.782267in}}%
\pgfpathlineto{\pgfqpoint{2.687023in}{0.687192in}}%
\pgfpathlineto{\pgfqpoint{2.689986in}{0.747392in}}%
\pgfpathlineto{\pgfqpoint{2.695913in}{0.803031in}}%
\pgfpathlineto{\pgfqpoint{2.698876in}{0.727172in}}%
\pgfpathlineto{\pgfqpoint{2.701839in}{0.792768in}}%
\pgfpathlineto{\pgfqpoint{2.704802in}{0.722799in}}%
\pgfpathlineto{\pgfqpoint{2.707766in}{0.693414in}}%
\pgfpathlineto{\pgfqpoint{2.710729in}{0.679572in}}%
\pgfpathlineto{\pgfqpoint{2.713692in}{0.707074in}}%
\pgfpathlineto{\pgfqpoint{2.716655in}{0.708142in}}%
\pgfpathlineto{\pgfqpoint{2.719619in}{0.780794in}}%
\pgfpathlineto{\pgfqpoint{2.722582in}{0.759325in}}%
\pgfpathlineto{\pgfqpoint{2.725545in}{0.805729in}}%
\pgfpathlineto{\pgfqpoint{2.728508in}{0.720317in}}%
\pgfpathlineto{\pgfqpoint{2.731472in}{0.858197in}}%
\pgfpathlineto{\pgfqpoint{2.734435in}{0.741531in}}%
\pgfpathlineto{\pgfqpoint{2.737398in}{0.690215in}}%
\pgfpathlineto{\pgfqpoint{2.740361in}{0.744590in}}%
\pgfpathlineto{\pgfqpoint{2.743325in}{0.736264in}}%
\pgfpathlineto{\pgfqpoint{2.746288in}{0.858574in}}%
\pgfpathlineto{\pgfqpoint{2.749251in}{0.728155in}}%
\pgfpathlineto{\pgfqpoint{2.752215in}{0.783834in}}%
\pgfpathlineto{\pgfqpoint{2.755178in}{0.674789in}}%
\pgfpathlineto{\pgfqpoint{2.758141in}{0.759608in}}%
\pgfpathlineto{\pgfqpoint{2.761104in}{0.677490in}}%
\pgfpathlineto{\pgfqpoint{2.764068in}{0.722399in}}%
\pgfpathlineto{\pgfqpoint{2.767031in}{0.726555in}}%
\pgfpathlineto{\pgfqpoint{2.772957in}{0.838576in}}%
\pgfpathlineto{\pgfqpoint{2.775921in}{0.941600in}}%
\pgfpathlineto{\pgfqpoint{2.778884in}{0.680318in}}%
\pgfpathlineto{\pgfqpoint{2.781847in}{0.784094in}}%
\pgfpathlineto{\pgfqpoint{2.784810in}{0.692644in}}%
\pgfpathlineto{\pgfqpoint{2.787774in}{0.721612in}}%
\pgfpathlineto{\pgfqpoint{2.790737in}{0.780740in}}%
\pgfpathlineto{\pgfqpoint{2.793700in}{0.738040in}}%
\pgfpathlineto{\pgfqpoint{2.796663in}{0.738775in}}%
\pgfpathlineto{\pgfqpoint{2.799627in}{0.728909in}}%
\pgfpathlineto{\pgfqpoint{2.802590in}{0.676676in}}%
\pgfpathlineto{\pgfqpoint{2.805553in}{0.672416in}}%
\pgfpathlineto{\pgfqpoint{2.808516in}{0.752074in}}%
\pgfpathlineto{\pgfqpoint{2.811480in}{0.775972in}}%
\pgfpathlineto{\pgfqpoint{2.814443in}{0.751557in}}%
\pgfpathlineto{\pgfqpoint{2.817406in}{0.664954in}}%
\pgfpathlineto{\pgfqpoint{2.820369in}{0.784403in}}%
\pgfpathlineto{\pgfqpoint{2.823333in}{0.708934in}}%
\pgfpathlineto{\pgfqpoint{2.826296in}{0.706480in}}%
\pgfpathlineto{\pgfqpoint{2.829259in}{0.697845in}}%
\pgfpathlineto{\pgfqpoint{2.832222in}{0.706485in}}%
\pgfpathlineto{\pgfqpoint{2.835186in}{0.800765in}}%
\pgfpathlineto{\pgfqpoint{2.841112in}{0.744491in}}%
\pgfpathlineto{\pgfqpoint{2.844075in}{0.769339in}}%
\pgfpathlineto{\pgfqpoint{2.847039in}{0.706059in}}%
\pgfpathlineto{\pgfqpoint{2.850002in}{0.688286in}}%
\pgfpathlineto{\pgfqpoint{2.852965in}{0.739122in}}%
\pgfpathlineto{\pgfqpoint{2.855928in}{0.713521in}}%
\pgfpathlineto{\pgfqpoint{2.858892in}{0.729311in}}%
\pgfpathlineto{\pgfqpoint{2.861855in}{0.895859in}}%
\pgfpathlineto{\pgfqpoint{2.867781in}{0.662684in}}%
\pgfpathlineto{\pgfqpoint{2.870745in}{0.703425in}}%
\pgfpathlineto{\pgfqpoint{2.873708in}{0.906712in}}%
\pgfpathlineto{\pgfqpoint{2.876671in}{0.655207in}}%
\pgfpathlineto{\pgfqpoint{2.879634in}{0.658042in}}%
\pgfpathlineto{\pgfqpoint{2.882598in}{0.681942in}}%
\pgfpathlineto{\pgfqpoint{2.885561in}{0.743421in}}%
\pgfpathlineto{\pgfqpoint{2.888524in}{0.724384in}}%
\pgfpathlineto{\pgfqpoint{2.891487in}{0.762942in}}%
\pgfpathlineto{\pgfqpoint{2.894451in}{0.726703in}}%
\pgfpathlineto{\pgfqpoint{2.897414in}{0.737664in}}%
\pgfpathlineto{\pgfqpoint{2.900377in}{0.692872in}}%
\pgfpathlineto{\pgfqpoint{2.903340in}{0.689916in}}%
\pgfpathlineto{\pgfqpoint{2.909267in}{0.746026in}}%
\pgfpathlineto{\pgfqpoint{2.912230in}{0.744590in}}%
\pgfpathlineto{\pgfqpoint{2.915193in}{0.738659in}}%
\pgfpathlineto{\pgfqpoint{2.918157in}{0.714239in}}%
\pgfpathlineto{\pgfqpoint{2.921120in}{0.668185in}}%
\pgfpathlineto{\pgfqpoint{2.924083in}{0.696035in}}%
\pgfpathlineto{\pgfqpoint{2.927046in}{0.667360in}}%
\pgfpathlineto{\pgfqpoint{2.930010in}{0.717162in}}%
\pgfpathlineto{\pgfqpoint{2.932973in}{0.688490in}}%
\pgfpathlineto{\pgfqpoint{2.935936in}{0.673702in}}%
\pgfpathlineto{\pgfqpoint{2.938899in}{0.718005in}}%
\pgfpathlineto{\pgfqpoint{2.941863in}{0.653501in}}%
\pgfpathlineto{\pgfqpoint{2.944826in}{0.768765in}}%
\pgfpathlineto{\pgfqpoint{2.947789in}{0.812267in}}%
\pgfpathlineto{\pgfqpoint{2.953716in}{0.663283in}}%
\pgfpathlineto{\pgfqpoint{2.959642in}{0.696659in}}%
\pgfpathlineto{\pgfqpoint{2.962605in}{0.762021in}}%
\pgfpathlineto{\pgfqpoint{2.965569in}{0.661715in}}%
\pgfpathlineto{\pgfqpoint{2.968532in}{0.693213in}}%
\pgfpathlineto{\pgfqpoint{2.971495in}{0.761008in}}%
\pgfpathlineto{\pgfqpoint{2.974459in}{0.745506in}}%
\pgfpathlineto{\pgfqpoint{2.977422in}{0.655418in}}%
\pgfpathlineto{\pgfqpoint{2.980385in}{0.665650in}}%
\pgfpathlineto{\pgfqpoint{2.983348in}{0.700516in}}%
\pgfpathlineto{\pgfqpoint{2.986312in}{0.695077in}}%
\pgfpathlineto{\pgfqpoint{2.989275in}{0.649689in}}%
\pgfpathlineto{\pgfqpoint{2.992238in}{0.653105in}}%
\pgfpathlineto{\pgfqpoint{2.995201in}{0.761622in}}%
\pgfpathlineto{\pgfqpoint{2.998165in}{0.710661in}}%
\pgfpathlineto{\pgfqpoint{3.001128in}{0.742253in}}%
\pgfpathlineto{\pgfqpoint{3.004091in}{0.901726in}}%
\pgfpathlineto{\pgfqpoint{3.007054in}{0.739288in}}%
\pgfpathlineto{\pgfqpoint{3.010018in}{0.752559in}}%
\pgfpathlineto{\pgfqpoint{3.012981in}{0.742909in}}%
\pgfpathlineto{\pgfqpoint{3.015944in}{0.842348in}}%
\pgfpathlineto{\pgfqpoint{3.018907in}{0.677386in}}%
\pgfpathlineto{\pgfqpoint{3.021871in}{0.700556in}}%
\pgfpathlineto{\pgfqpoint{3.024834in}{0.668211in}}%
\pgfpathlineto{\pgfqpoint{3.027797in}{0.668264in}}%
\pgfpathlineto{\pgfqpoint{3.030760in}{0.745667in}}%
\pgfpathlineto{\pgfqpoint{3.033724in}{0.658807in}}%
\pgfpathlineto{\pgfqpoint{3.039650in}{0.753331in}}%
\pgfpathlineto{\pgfqpoint{3.042613in}{0.656282in}}%
\pgfpathlineto{\pgfqpoint{3.045577in}{0.718834in}}%
\pgfpathlineto{\pgfqpoint{3.048540in}{0.721193in}}%
\pgfpathlineto{\pgfqpoint{3.051503in}{0.708236in}}%
\pgfpathlineto{\pgfqpoint{3.054466in}{0.721661in}}%
\pgfpathlineto{\pgfqpoint{3.057430in}{0.723478in}}%
\pgfpathlineto{\pgfqpoint{3.060393in}{0.670467in}}%
\pgfpathlineto{\pgfqpoint{3.063356in}{0.716709in}}%
\pgfpathlineto{\pgfqpoint{3.066319in}{0.732910in}}%
\pgfpathlineto{\pgfqpoint{3.069283in}{0.655616in}}%
\pgfpathlineto{\pgfqpoint{3.072246in}{0.668658in}}%
\pgfpathlineto{\pgfqpoint{3.075209in}{0.640960in}}%
\pgfpathlineto{\pgfqpoint{3.078172in}{0.708654in}}%
\pgfpathlineto{\pgfqpoint{3.081136in}{0.818659in}}%
\pgfpathlineto{\pgfqpoint{3.084099in}{0.722679in}}%
\pgfpathlineto{\pgfqpoint{3.087062in}{0.713778in}}%
\pgfpathlineto{\pgfqpoint{3.090025in}{0.637113in}}%
\pgfpathlineto{\pgfqpoint{3.095952in}{0.677092in}}%
\pgfpathlineto{\pgfqpoint{3.098915in}{0.776254in}}%
\pgfpathlineto{\pgfqpoint{3.101878in}{0.668237in}}%
\pgfpathlineto{\pgfqpoint{3.104842in}{0.675824in}}%
\pgfpathlineto{\pgfqpoint{3.107805in}{0.716350in}}%
\pgfpathlineto{\pgfqpoint{3.110768in}{0.722887in}}%
\pgfpathlineto{\pgfqpoint{3.113731in}{0.834191in}}%
\pgfpathlineto{\pgfqpoint{3.116695in}{0.710193in}}%
\pgfpathlineto{\pgfqpoint{3.122621in}{0.663542in}}%
\pgfpathlineto{\pgfqpoint{3.125584in}{0.664466in}}%
\pgfpathlineto{\pgfqpoint{3.128548in}{0.681650in}}%
\pgfpathlineto{\pgfqpoint{3.131511in}{0.777187in}}%
\pgfpathlineto{\pgfqpoint{3.134474in}{0.690604in}}%
\pgfpathlineto{\pgfqpoint{3.137437in}{0.807033in}}%
\pgfpathlineto{\pgfqpoint{3.140401in}{0.741401in}}%
\pgfpathlineto{\pgfqpoint{3.143364in}{0.637417in}}%
\pgfpathlineto{\pgfqpoint{3.146327in}{0.642923in}}%
\pgfpathlineto{\pgfqpoint{3.149290in}{0.690948in}}%
\pgfpathlineto{\pgfqpoint{3.152254in}{0.660334in}}%
\pgfpathlineto{\pgfqpoint{3.155217in}{0.708070in}}%
\pgfpathlineto{\pgfqpoint{3.158180in}{0.697971in}}%
\pgfpathlineto{\pgfqpoint{3.161143in}{0.701789in}}%
\pgfpathlineto{\pgfqpoint{3.164107in}{0.739716in}}%
\pgfpathlineto{\pgfqpoint{3.167070in}{0.736534in}}%
\pgfpathlineto{\pgfqpoint{3.172996in}{0.685029in}}%
\pgfpathlineto{\pgfqpoint{3.175960in}{0.681697in}}%
\pgfpathlineto{\pgfqpoint{3.181886in}{0.642269in}}%
\pgfpathlineto{\pgfqpoint{3.184849in}{0.677364in}}%
\pgfpathlineto{\pgfqpoint{3.187813in}{0.681961in}}%
\pgfpathlineto{\pgfqpoint{3.190776in}{0.670041in}}%
\pgfpathlineto{\pgfqpoint{3.193739in}{0.694988in}}%
\pgfpathlineto{\pgfqpoint{3.196702in}{0.684424in}}%
\pgfpathlineto{\pgfqpoint{3.199666in}{0.749220in}}%
\pgfpathlineto{\pgfqpoint{3.205592in}{0.647127in}}%
\pgfpathlineto{\pgfqpoint{3.208556in}{0.824788in}}%
\pgfpathlineto{\pgfqpoint{3.211519in}{0.642683in}}%
\pgfpathlineto{\pgfqpoint{3.214482in}{0.689842in}}%
\pgfpathlineto{\pgfqpoint{3.217445in}{0.663401in}}%
\pgfpathlineto{\pgfqpoint{3.220409in}{0.710582in}}%
\pgfpathlineto{\pgfqpoint{3.223372in}{0.796745in}}%
\pgfpathlineto{\pgfqpoint{3.226335in}{0.645154in}}%
\pgfpathlineto{\pgfqpoint{3.229298in}{0.718363in}}%
\pgfpathlineto{\pgfqpoint{3.232262in}{0.643396in}}%
\pgfpathlineto{\pgfqpoint{3.235225in}{0.673631in}}%
\pgfpathlineto{\pgfqpoint{3.238188in}{0.810643in}}%
\pgfpathlineto{\pgfqpoint{3.241151in}{0.682056in}}%
\pgfpathlineto{\pgfqpoint{3.247078in}{0.789676in}}%
\pgfpathlineto{\pgfqpoint{3.250041in}{0.711458in}}%
\pgfpathlineto{\pgfqpoint{3.255968in}{0.701383in}}%
\pgfpathlineto{\pgfqpoint{3.258931in}{0.684767in}}%
\pgfpathlineto{\pgfqpoint{3.261894in}{0.631355in}}%
\pgfpathlineto{\pgfqpoint{3.264857in}{0.722856in}}%
\pgfpathlineto{\pgfqpoint{3.267821in}{0.628165in}}%
\pgfpathlineto{\pgfqpoint{3.270784in}{0.660315in}}%
\pgfpathlineto{\pgfqpoint{3.273747in}{0.786223in}}%
\pgfpathlineto{\pgfqpoint{3.276710in}{0.650806in}}%
\pgfpathlineto{\pgfqpoint{3.279674in}{0.624041in}}%
\pgfpathlineto{\pgfqpoint{3.282637in}{0.709842in}}%
\pgfpathlineto{\pgfqpoint{3.285600in}{0.713610in}}%
\pgfpathlineto{\pgfqpoint{3.288563in}{0.677203in}}%
\pgfpathlineto{\pgfqpoint{3.291527in}{0.750784in}}%
\pgfpathlineto{\pgfqpoint{3.294490in}{0.723101in}}%
\pgfpathlineto{\pgfqpoint{3.297453in}{0.671848in}}%
\pgfpathlineto{\pgfqpoint{3.300416in}{0.700650in}}%
\pgfpathlineto{\pgfqpoint{3.303380in}{0.715808in}}%
\pgfpathlineto{\pgfqpoint{3.306343in}{0.770993in}}%
\pgfpathlineto{\pgfqpoint{3.312269in}{0.663872in}}%
\pgfpathlineto{\pgfqpoint{3.315233in}{0.663132in}}%
\pgfpathlineto{\pgfqpoint{3.318196in}{0.641163in}}%
\pgfpathlineto{\pgfqpoint{3.321159in}{0.664595in}}%
\pgfpathlineto{\pgfqpoint{3.324122in}{0.627913in}}%
\pgfpathlineto{\pgfqpoint{3.327086in}{0.709134in}}%
\pgfpathlineto{\pgfqpoint{3.330049in}{0.705014in}}%
\pgfpathlineto{\pgfqpoint{3.333012in}{0.634753in}}%
\pgfpathlineto{\pgfqpoint{3.335975in}{0.671089in}}%
\pgfpathlineto{\pgfqpoint{3.338939in}{0.737567in}}%
\pgfpathlineto{\pgfqpoint{3.341902in}{0.690502in}}%
\pgfpathlineto{\pgfqpoint{3.344865in}{0.780940in}}%
\pgfpathlineto{\pgfqpoint{3.347828in}{0.685465in}}%
\pgfpathlineto{\pgfqpoint{3.350792in}{0.751012in}}%
\pgfpathlineto{\pgfqpoint{3.353755in}{0.661077in}}%
\pgfpathlineto{\pgfqpoint{3.356718in}{0.710884in}}%
\pgfpathlineto{\pgfqpoint{3.359681in}{0.633451in}}%
\pgfpathlineto{\pgfqpoint{3.362645in}{0.646184in}}%
\pgfpathlineto{\pgfqpoint{3.365608in}{0.632699in}}%
\pgfpathlineto{\pgfqpoint{3.368571in}{0.703744in}}%
\pgfpathlineto{\pgfqpoint{3.371534in}{0.657215in}}%
\pgfpathlineto{\pgfqpoint{3.374498in}{0.670314in}}%
\pgfpathlineto{\pgfqpoint{3.377461in}{0.670136in}}%
\pgfpathlineto{\pgfqpoint{3.380424in}{0.731365in}}%
\pgfpathlineto{\pgfqpoint{3.383387in}{0.638373in}}%
\pgfpathlineto{\pgfqpoint{3.386351in}{0.715252in}}%
\pgfpathlineto{\pgfqpoint{3.389314in}{0.632859in}}%
\pgfpathlineto{\pgfqpoint{3.392277in}{0.663741in}}%
\pgfpathlineto{\pgfqpoint{3.395240in}{0.792472in}}%
\pgfpathlineto{\pgfqpoint{3.398204in}{0.784968in}}%
\pgfpathlineto{\pgfqpoint{3.401167in}{0.676819in}}%
\pgfpathlineto{\pgfqpoint{3.404130in}{0.683880in}}%
\pgfpathlineto{\pgfqpoint{3.407093in}{0.721827in}}%
\pgfpathlineto{\pgfqpoint{3.410057in}{0.707886in}}%
\pgfpathlineto{\pgfqpoint{3.413020in}{0.683707in}}%
\pgfpathlineto{\pgfqpoint{3.415983in}{0.696686in}}%
\pgfpathlineto{\pgfqpoint{3.418946in}{0.801696in}}%
\pgfpathlineto{\pgfqpoint{3.421910in}{0.770205in}}%
\pgfpathlineto{\pgfqpoint{3.424873in}{0.645510in}}%
\pgfpathlineto{\pgfqpoint{3.427836in}{0.689678in}}%
\pgfpathlineto{\pgfqpoint{3.430800in}{0.705114in}}%
\pgfpathlineto{\pgfqpoint{3.433763in}{0.758273in}}%
\pgfpathlineto{\pgfqpoint{3.436726in}{0.690909in}}%
\pgfpathlineto{\pgfqpoint{3.439689in}{0.685687in}}%
\pgfpathlineto{\pgfqpoint{3.442653in}{0.673547in}}%
\pgfpathlineto{\pgfqpoint{3.445616in}{0.716923in}}%
\pgfpathlineto{\pgfqpoint{3.448579in}{0.720562in}}%
\pgfpathlineto{\pgfqpoint{3.451542in}{0.709203in}}%
\pgfpathlineto{\pgfqpoint{3.454506in}{0.731457in}}%
\pgfpathlineto{\pgfqpoint{3.460432in}{0.657670in}}%
\pgfpathlineto{\pgfqpoint{3.463395in}{0.651238in}}%
\pgfpathlineto{\pgfqpoint{3.466359in}{0.796031in}}%
\pgfpathlineto{\pgfqpoint{3.469322in}{0.630832in}}%
\pgfpathlineto{\pgfqpoint{3.472285in}{0.631642in}}%
\pgfpathlineto{\pgfqpoint{3.475248in}{0.718588in}}%
\pgfpathlineto{\pgfqpoint{3.478212in}{0.707023in}}%
\pgfpathlineto{\pgfqpoint{3.481175in}{0.755040in}}%
\pgfpathlineto{\pgfqpoint{3.484138in}{0.636447in}}%
\pgfpathlineto{\pgfqpoint{3.487101in}{0.714957in}}%
\pgfpathlineto{\pgfqpoint{3.490065in}{0.718994in}}%
\pgfpathlineto{\pgfqpoint{3.493028in}{0.691060in}}%
\pgfpathlineto{\pgfqpoint{3.495991in}{0.700789in}}%
\pgfpathlineto{\pgfqpoint{3.498954in}{0.697941in}}%
\pgfpathlineto{\pgfqpoint{3.501918in}{0.673646in}}%
\pgfpathlineto{\pgfqpoint{3.504881in}{0.784596in}}%
\pgfpathlineto{\pgfqpoint{3.507844in}{0.724005in}}%
\pgfpathlineto{\pgfqpoint{3.510807in}{0.712786in}}%
\pgfpathlineto{\pgfqpoint{3.513771in}{0.710090in}}%
\pgfpathlineto{\pgfqpoint{3.516734in}{0.641490in}}%
\pgfpathlineto{\pgfqpoint{3.519697in}{0.763678in}}%
\pgfpathlineto{\pgfqpoint{3.522660in}{0.726854in}}%
\pgfpathlineto{\pgfqpoint{3.525624in}{0.788421in}}%
\pgfpathlineto{\pgfqpoint{3.528587in}{0.679902in}}%
\pgfpathlineto{\pgfqpoint{3.531550in}{0.631144in}}%
\pgfpathlineto{\pgfqpoint{3.534513in}{0.678521in}}%
\pgfpathlineto{\pgfqpoint{3.537477in}{0.687126in}}%
\pgfpathlineto{\pgfqpoint{3.540440in}{0.736183in}}%
\pgfpathlineto{\pgfqpoint{3.543403in}{0.689185in}}%
\pgfpathlineto{\pgfqpoint{3.546366in}{0.828903in}}%
\pgfpathlineto{\pgfqpoint{3.549330in}{0.677424in}}%
\pgfpathlineto{\pgfqpoint{3.552293in}{0.771916in}}%
\pgfpathlineto{\pgfqpoint{3.555256in}{0.652031in}}%
\pgfpathlineto{\pgfqpoint{3.558219in}{0.732365in}}%
\pgfpathlineto{\pgfqpoint{3.561183in}{0.645574in}}%
\pgfpathlineto{\pgfqpoint{3.564146in}{0.867976in}}%
\pgfpathlineto{\pgfqpoint{3.567109in}{0.796328in}}%
\pgfpathlineto{\pgfqpoint{3.570072in}{0.887430in}}%
\pgfpathlineto{\pgfqpoint{3.573036in}{0.720351in}}%
\pgfpathlineto{\pgfqpoint{3.575999in}{0.639796in}}%
\pgfpathlineto{\pgfqpoint{3.578962in}{0.674094in}}%
\pgfpathlineto{\pgfqpoint{3.581925in}{0.641896in}}%
\pgfpathlineto{\pgfqpoint{3.584889in}{0.690275in}}%
\pgfpathlineto{\pgfqpoint{3.587852in}{0.716524in}}%
\pgfpathlineto{\pgfqpoint{3.590815in}{0.732702in}}%
\pgfpathlineto{\pgfqpoint{3.593778in}{0.733853in}}%
\pgfpathlineto{\pgfqpoint{3.596742in}{0.704795in}}%
\pgfpathlineto{\pgfqpoint{3.599705in}{0.720341in}}%
\pgfpathlineto{\pgfqpoint{3.602668in}{0.724673in}}%
\pgfpathlineto{\pgfqpoint{3.605631in}{0.725384in}}%
\pgfpathlineto{\pgfqpoint{3.608595in}{0.787080in}}%
\pgfpathlineto{\pgfqpoint{3.611558in}{0.708569in}}%
\pgfpathlineto{\pgfqpoint{3.614521in}{0.796187in}}%
\pgfpathlineto{\pgfqpoint{3.617484in}{0.663443in}}%
\pgfpathlineto{\pgfqpoint{3.620448in}{0.779163in}}%
\pgfpathlineto{\pgfqpoint{3.623411in}{0.632278in}}%
\pgfpathlineto{\pgfqpoint{3.626374in}{0.693706in}}%
\pgfpathlineto{\pgfqpoint{3.629337in}{0.693045in}}%
\pgfpathlineto{\pgfqpoint{3.632301in}{0.645290in}}%
\pgfpathlineto{\pgfqpoint{3.635264in}{0.847084in}}%
\pgfpathlineto{\pgfqpoint{3.638227in}{0.674960in}}%
\pgfpathlineto{\pgfqpoint{3.641190in}{0.788525in}}%
\pgfpathlineto{\pgfqpoint{3.644154in}{0.641984in}}%
\pgfpathlineto{\pgfqpoint{3.647117in}{0.634818in}}%
\pgfpathlineto{\pgfqpoint{3.653044in}{0.805311in}}%
\pgfpathlineto{\pgfqpoint{3.656007in}{0.815341in}}%
\pgfpathlineto{\pgfqpoint{3.658970in}{0.623408in}}%
\pgfpathlineto{\pgfqpoint{3.661933in}{0.648162in}}%
\pgfpathlineto{\pgfqpoint{3.664897in}{0.648401in}}%
\pgfpathlineto{\pgfqpoint{3.667860in}{0.694595in}}%
\pgfpathlineto{\pgfqpoint{3.670823in}{0.761635in}}%
\pgfpathlineto{\pgfqpoint{3.673786in}{0.700794in}}%
\pgfpathlineto{\pgfqpoint{3.676750in}{0.705668in}}%
\pgfpathlineto{\pgfqpoint{3.679713in}{0.675814in}}%
\pgfpathlineto{\pgfqpoint{3.682676in}{0.747225in}}%
\pgfpathlineto{\pgfqpoint{3.685639in}{0.654504in}}%
\pgfpathlineto{\pgfqpoint{3.688603in}{0.688547in}}%
\pgfpathlineto{\pgfqpoint{3.691566in}{0.692631in}}%
\pgfpathlineto{\pgfqpoint{3.694529in}{0.646928in}}%
\pgfpathlineto{\pgfqpoint{3.697492in}{0.629653in}}%
\pgfpathlineto{\pgfqpoint{3.700456in}{0.657182in}}%
\pgfpathlineto{\pgfqpoint{3.703419in}{0.659109in}}%
\pgfpathlineto{\pgfqpoint{3.706382in}{0.694253in}}%
\pgfpathlineto{\pgfqpoint{3.709345in}{0.765489in}}%
\pgfpathlineto{\pgfqpoint{3.712309in}{0.636457in}}%
\pgfpathlineto{\pgfqpoint{3.715272in}{0.679297in}}%
\pgfpathlineto{\pgfqpoint{3.718235in}{0.666078in}}%
\pgfpathlineto{\pgfqpoint{3.721198in}{0.670932in}}%
\pgfpathlineto{\pgfqpoint{3.724162in}{0.632543in}}%
\pgfpathlineto{\pgfqpoint{3.730088in}{0.689158in}}%
\pgfpathlineto{\pgfqpoint{3.733051in}{0.670423in}}%
\pgfpathlineto{\pgfqpoint{3.736015in}{0.698538in}}%
\pgfpathlineto{\pgfqpoint{3.738978in}{0.612787in}}%
\pgfpathlineto{\pgfqpoint{3.741941in}{0.723013in}}%
\pgfpathlineto{\pgfqpoint{3.744904in}{0.656297in}}%
\pgfpathlineto{\pgfqpoint{3.747868in}{0.644344in}}%
\pgfpathlineto{\pgfqpoint{3.750831in}{0.763016in}}%
\pgfpathlineto{\pgfqpoint{3.753794in}{0.725756in}}%
\pgfpathlineto{\pgfqpoint{3.756757in}{0.824871in}}%
\pgfpathlineto{\pgfqpoint{3.759721in}{0.629005in}}%
\pgfpathlineto{\pgfqpoint{3.762684in}{0.678788in}}%
\pgfpathlineto{\pgfqpoint{3.765647in}{0.649709in}}%
\pgfpathlineto{\pgfqpoint{3.768610in}{0.665189in}}%
\pgfpathlineto{\pgfqpoint{3.771574in}{0.694872in}}%
\pgfpathlineto{\pgfqpoint{3.774537in}{0.708730in}}%
\pgfpathlineto{\pgfqpoint{3.777500in}{0.709419in}}%
\pgfpathlineto{\pgfqpoint{3.780463in}{0.738768in}}%
\pgfpathlineto{\pgfqpoint{3.783427in}{0.637751in}}%
\pgfpathlineto{\pgfqpoint{3.786390in}{0.699407in}}%
\pgfpathlineto{\pgfqpoint{3.789353in}{0.722290in}}%
\pgfpathlineto{\pgfqpoint{3.792316in}{0.663354in}}%
\pgfpathlineto{\pgfqpoint{3.795280in}{0.647330in}}%
\pgfpathlineto{\pgfqpoint{3.798243in}{0.679229in}}%
\pgfpathlineto{\pgfqpoint{3.801206in}{0.735715in}}%
\pgfpathlineto{\pgfqpoint{3.804169in}{0.661765in}}%
\pgfpathlineto{\pgfqpoint{3.807133in}{0.658946in}}%
\pgfpathlineto{\pgfqpoint{3.810096in}{0.647708in}}%
\pgfpathlineto{\pgfqpoint{3.813059in}{0.767920in}}%
\pgfpathlineto{\pgfqpoint{3.816022in}{0.801342in}}%
\pgfpathlineto{\pgfqpoint{3.818986in}{0.648387in}}%
\pgfpathlineto{\pgfqpoint{3.821949in}{0.673029in}}%
\pgfpathlineto{\pgfqpoint{3.824912in}{0.650176in}}%
\pgfpathlineto{\pgfqpoint{3.827875in}{0.732227in}}%
\pgfpathlineto{\pgfqpoint{3.830839in}{0.901759in}}%
\pgfpathlineto{\pgfqpoint{3.833802in}{0.717975in}}%
\pgfpathlineto{\pgfqpoint{3.836765in}{0.659017in}}%
\pgfpathlineto{\pgfqpoint{3.839728in}{0.640319in}}%
\pgfpathlineto{\pgfqpoint{3.842692in}{0.648068in}}%
\pgfpathlineto{\pgfqpoint{3.845655in}{0.693347in}}%
\pgfpathlineto{\pgfqpoint{3.848618in}{0.627341in}}%
\pgfpathlineto{\pgfqpoint{3.851581in}{0.817151in}}%
\pgfpathlineto{\pgfqpoint{3.854545in}{0.715727in}}%
\pgfpathlineto{\pgfqpoint{3.857508in}{0.717876in}}%
\pgfpathlineto{\pgfqpoint{3.860471in}{0.686422in}}%
\pgfpathlineto{\pgfqpoint{3.866398in}{0.771969in}}%
\pgfpathlineto{\pgfqpoint{3.869361in}{0.697706in}}%
\pgfpathlineto{\pgfqpoint{3.875287in}{0.617412in}}%
\pgfpathlineto{\pgfqpoint{3.878251in}{0.672512in}}%
\pgfpathlineto{\pgfqpoint{3.881214in}{0.692518in}}%
\pgfpathlineto{\pgfqpoint{3.884177in}{0.643547in}}%
\pgfpathlineto{\pgfqpoint{3.887141in}{0.708347in}}%
\pgfpathlineto{\pgfqpoint{3.890104in}{0.652967in}}%
\pgfpathlineto{\pgfqpoint{3.893067in}{0.669328in}}%
\pgfpathlineto{\pgfqpoint{3.896030in}{0.647228in}}%
\pgfpathlineto{\pgfqpoint{3.898994in}{0.716148in}}%
\pgfpathlineto{\pgfqpoint{3.901957in}{0.658618in}}%
\pgfpathlineto{\pgfqpoint{3.904920in}{0.669128in}}%
\pgfpathlineto{\pgfqpoint{3.907883in}{0.628859in}}%
\pgfpathlineto{\pgfqpoint{3.910847in}{0.647575in}}%
\pgfpathlineto{\pgfqpoint{3.913810in}{0.756037in}}%
\pgfpathlineto{\pgfqpoint{3.916773in}{0.626183in}}%
\pgfpathlineto{\pgfqpoint{3.919736in}{0.753243in}}%
\pgfpathlineto{\pgfqpoint{3.922700in}{0.663362in}}%
\pgfpathlineto{\pgfqpoint{3.925663in}{0.662337in}}%
\pgfpathlineto{\pgfqpoint{3.928626in}{0.741914in}}%
\pgfpathlineto{\pgfqpoint{3.931589in}{0.745221in}}%
\pgfpathlineto{\pgfqpoint{3.934553in}{0.646743in}}%
\pgfpathlineto{\pgfqpoint{3.937516in}{0.631673in}}%
\pgfpathlineto{\pgfqpoint{3.940479in}{0.686923in}}%
\pgfpathlineto{\pgfqpoint{3.943442in}{0.665781in}}%
\pgfpathlineto{\pgfqpoint{3.949369in}{0.823110in}}%
\pgfpathlineto{\pgfqpoint{3.952332in}{0.719876in}}%
\pgfpathlineto{\pgfqpoint{3.955295in}{0.707906in}}%
\pgfpathlineto{\pgfqpoint{3.958259in}{0.905133in}}%
\pgfpathlineto{\pgfqpoint{3.961222in}{0.685046in}}%
\pgfpathlineto{\pgfqpoint{3.964185in}{0.627369in}}%
\pgfpathlineto{\pgfqpoint{3.967148in}{0.667670in}}%
\pgfpathlineto{\pgfqpoint{3.970112in}{0.807011in}}%
\pgfpathlineto{\pgfqpoint{3.973075in}{0.747150in}}%
\pgfpathlineto{\pgfqpoint{3.976038in}{0.641999in}}%
\pgfpathlineto{\pgfqpoint{3.979001in}{0.665461in}}%
\pgfpathlineto{\pgfqpoint{3.981965in}{0.785283in}}%
\pgfpathlineto{\pgfqpoint{3.984928in}{0.637113in}}%
\pgfpathlineto{\pgfqpoint{3.990854in}{0.711817in}}%
\pgfpathlineto{\pgfqpoint{3.993818in}{0.705943in}}%
\pgfpathlineto{\pgfqpoint{3.996781in}{0.726152in}}%
\pgfpathlineto{\pgfqpoint{3.999744in}{0.600141in}}%
\pgfpathlineto{\pgfqpoint{4.002707in}{0.649033in}}%
\pgfpathlineto{\pgfqpoint{4.005671in}{0.847733in}}%
\pgfpathlineto{\pgfqpoint{4.008634in}{0.622229in}}%
\pgfpathlineto{\pgfqpoint{4.014560in}{0.711936in}}%
\pgfpathlineto{\pgfqpoint{4.017524in}{0.720371in}}%
\pgfpathlineto{\pgfqpoint{4.020487in}{0.743353in}}%
\pgfpathlineto{\pgfqpoint{4.023450in}{0.665400in}}%
\pgfpathlineto{\pgfqpoint{4.026413in}{0.742463in}}%
\pgfpathlineto{\pgfqpoint{4.029377in}{0.686589in}}%
\pgfpathlineto{\pgfqpoint{4.032340in}{0.677169in}}%
\pgfpathlineto{\pgfqpoint{4.035303in}{0.701299in}}%
\pgfpathlineto{\pgfqpoint{4.038266in}{0.632810in}}%
\pgfpathlineto{\pgfqpoint{4.044193in}{0.689567in}}%
\pgfpathlineto{\pgfqpoint{4.047156in}{0.698581in}}%
\pgfpathlineto{\pgfqpoint{4.050119in}{0.814039in}}%
\pgfpathlineto{\pgfqpoint{4.053083in}{0.618275in}}%
\pgfpathlineto{\pgfqpoint{4.056046in}{0.711956in}}%
\pgfpathlineto{\pgfqpoint{4.059009in}{0.698191in}}%
\pgfpathlineto{\pgfqpoint{4.061972in}{0.673582in}}%
\pgfpathlineto{\pgfqpoint{4.064936in}{0.602018in}}%
\pgfpathlineto{\pgfqpoint{4.067899in}{0.694238in}}%
\pgfpathlineto{\pgfqpoint{4.070862in}{0.747080in}}%
\pgfpathlineto{\pgfqpoint{4.073825in}{0.696607in}}%
\pgfpathlineto{\pgfqpoint{4.076789in}{0.701428in}}%
\pgfpathlineto{\pgfqpoint{4.079752in}{0.743877in}}%
\pgfpathlineto{\pgfqpoint{4.085678in}{0.655802in}}%
\pgfpathlineto{\pgfqpoint{4.091605in}{0.732975in}}%
\pgfpathlineto{\pgfqpoint{4.097531in}{0.620694in}}%
\pgfpathlineto{\pgfqpoint{4.100495in}{0.779454in}}%
\pgfpathlineto{\pgfqpoint{4.103458in}{0.729424in}}%
\pgfpathlineto{\pgfqpoint{4.106421in}{0.706529in}}%
\pgfpathlineto{\pgfqpoint{4.109385in}{0.672193in}}%
\pgfpathlineto{\pgfqpoint{4.112348in}{0.696486in}}%
\pgfpathlineto{\pgfqpoint{4.115311in}{0.736747in}}%
\pgfpathlineto{\pgfqpoint{4.124201in}{0.644106in}}%
\pgfpathlineto{\pgfqpoint{4.127164in}{0.722421in}}%
\pgfpathlineto{\pgfqpoint{4.130127in}{0.728840in}}%
\pgfpathlineto{\pgfqpoint{4.133091in}{0.673005in}}%
\pgfpathlineto{\pgfqpoint{4.136054in}{0.757642in}}%
\pgfpathlineto{\pgfqpoint{4.139017in}{0.632298in}}%
\pgfpathlineto{\pgfqpoint{4.141980in}{0.711025in}}%
\pgfpathlineto{\pgfqpoint{4.144944in}{0.658212in}}%
\pgfpathlineto{\pgfqpoint{4.147907in}{0.754196in}}%
\pgfpathlineto{\pgfqpoint{4.150870in}{0.750452in}}%
\pgfpathlineto{\pgfqpoint{4.153833in}{0.679357in}}%
\pgfpathlineto{\pgfqpoint{4.156797in}{0.656145in}}%
\pgfpathlineto{\pgfqpoint{4.159760in}{0.731350in}}%
\pgfpathlineto{\pgfqpoint{4.162723in}{0.694295in}}%
\pgfpathlineto{\pgfqpoint{4.165686in}{0.689383in}}%
\pgfpathlineto{\pgfqpoint{4.168650in}{0.665974in}}%
\pgfpathlineto{\pgfqpoint{4.171613in}{0.696332in}}%
\pgfpathlineto{\pgfqpoint{4.174576in}{0.632438in}}%
\pgfpathlineto{\pgfqpoint{4.177539in}{0.757474in}}%
\pgfpathlineto{\pgfqpoint{4.183466in}{0.650818in}}%
\pgfpathlineto{\pgfqpoint{4.186429in}{0.628953in}}%
\pgfpathlineto{\pgfqpoint{4.189392in}{0.743229in}}%
\pgfpathlineto{\pgfqpoint{4.192356in}{0.743890in}}%
\pgfpathlineto{\pgfqpoint{4.195319in}{0.662901in}}%
\pgfpathlineto{\pgfqpoint{4.198282in}{0.671698in}}%
\pgfpathlineto{\pgfqpoint{4.201245in}{0.625833in}}%
\pgfpathlineto{\pgfqpoint{4.204209in}{0.646027in}}%
\pgfpathlineto{\pgfqpoint{4.207172in}{0.648664in}}%
\pgfpathlineto{\pgfqpoint{4.210135in}{0.691527in}}%
\pgfpathlineto{\pgfqpoint{4.213098in}{0.661181in}}%
\pgfpathlineto{\pgfqpoint{4.219025in}{0.708235in}}%
\pgfpathlineto{\pgfqpoint{4.221988in}{0.637096in}}%
\pgfpathlineto{\pgfqpoint{4.224951in}{0.892745in}}%
\pgfpathlineto{\pgfqpoint{4.227915in}{0.627270in}}%
\pgfpathlineto{\pgfqpoint{4.230878in}{0.802506in}}%
\pgfpathlineto{\pgfqpoint{4.233841in}{0.599757in}}%
\pgfpathlineto{\pgfqpoint{4.236804in}{0.687057in}}%
\pgfpathlineto{\pgfqpoint{4.239768in}{0.736101in}}%
\pgfpathlineto{\pgfqpoint{4.242731in}{0.922207in}}%
\pgfpathlineto{\pgfqpoint{4.245694in}{1.009036in}}%
\pgfpathlineto{\pgfqpoint{4.248657in}{0.654987in}}%
\pgfpathlineto{\pgfqpoint{4.251621in}{0.687646in}}%
\pgfpathlineto{\pgfqpoint{4.254584in}{0.806407in}}%
\pgfpathlineto{\pgfqpoint{4.257547in}{0.673997in}}%
\pgfpathlineto{\pgfqpoint{4.260510in}{0.636989in}}%
\pgfpathlineto{\pgfqpoint{4.263474in}{0.679461in}}%
\pgfpathlineto{\pgfqpoint{4.266437in}{0.697368in}}%
\pgfpathlineto{\pgfqpoint{4.269400in}{0.613597in}}%
\pgfpathlineto{\pgfqpoint{4.272363in}{0.700308in}}%
\pgfpathlineto{\pgfqpoint{4.275327in}{0.640294in}}%
\pgfpathlineto{\pgfqpoint{4.278290in}{0.822622in}}%
\pgfpathlineto{\pgfqpoint{4.284216in}{0.718041in}}%
\pgfpathlineto{\pgfqpoint{4.287180in}{0.681120in}}%
\pgfpathlineto{\pgfqpoint{4.290143in}{0.673337in}}%
\pgfpathlineto{\pgfqpoint{4.293106in}{0.870638in}}%
\pgfpathlineto{\pgfqpoint{4.299033in}{0.644955in}}%
\pgfpathlineto{\pgfqpoint{4.301996in}{0.696416in}}%
\pgfpathlineto{\pgfqpoint{4.304959in}{0.771268in}}%
\pgfpathlineto{\pgfqpoint{4.307922in}{0.715714in}}%
\pgfpathlineto{\pgfqpoint{4.310886in}{0.599621in}}%
\pgfpathlineto{\pgfqpoint{4.313849in}{0.694664in}}%
\pgfpathlineto{\pgfqpoint{4.316812in}{0.683991in}}%
\pgfpathlineto{\pgfqpoint{4.319775in}{0.712538in}}%
\pgfpathlineto{\pgfqpoint{4.322739in}{0.638772in}}%
\pgfpathlineto{\pgfqpoint{4.325702in}{0.801151in}}%
\pgfpathlineto{\pgfqpoint{4.328665in}{0.810235in}}%
\pgfpathlineto{\pgfqpoint{4.331629in}{0.762699in}}%
\pgfpathlineto{\pgfqpoint{4.334592in}{0.939557in}}%
\pgfpathlineto{\pgfqpoint{4.337555in}{0.650125in}}%
\pgfpathlineto{\pgfqpoint{4.340518in}{0.691631in}}%
\pgfpathlineto{\pgfqpoint{4.343482in}{0.701393in}}%
\pgfpathlineto{\pgfqpoint{4.346445in}{0.692721in}}%
\pgfpathlineto{\pgfqpoint{4.349408in}{0.769498in}}%
\pgfpathlineto{\pgfqpoint{4.352371in}{0.613995in}}%
\pgfpathlineto{\pgfqpoint{4.355335in}{0.610178in}}%
\pgfpathlineto{\pgfqpoint{4.358298in}{0.636028in}}%
\pgfpathlineto{\pgfqpoint{4.361261in}{0.689886in}}%
\pgfpathlineto{\pgfqpoint{4.364224in}{0.706812in}}%
\pgfpathlineto{\pgfqpoint{4.367188in}{0.691327in}}%
\pgfpathlineto{\pgfqpoint{4.370151in}{0.620113in}}%
\pgfpathlineto{\pgfqpoint{4.373114in}{0.864847in}}%
\pgfpathlineto{\pgfqpoint{4.379041in}{0.663817in}}%
\pgfpathlineto{\pgfqpoint{4.382004in}{0.719104in}}%
\pgfpathlineto{\pgfqpoint{4.384967in}{0.687584in}}%
\pgfpathlineto{\pgfqpoint{4.387930in}{0.617142in}}%
\pgfpathlineto{\pgfqpoint{4.390894in}{0.657767in}}%
\pgfpathlineto{\pgfqpoint{4.393857in}{0.679035in}}%
\pgfpathlineto{\pgfqpoint{4.396820in}{0.711463in}}%
\pgfpathlineto{\pgfqpoint{4.402747in}{0.730321in}}%
\pgfpathlineto{\pgfqpoint{4.405710in}{0.691263in}}%
\pgfpathlineto{\pgfqpoint{4.408673in}{0.762244in}}%
\pgfpathlineto{\pgfqpoint{4.411636in}{0.763264in}}%
\pgfpathlineto{\pgfqpoint{4.414600in}{0.635702in}}%
\pgfpathlineto{\pgfqpoint{4.417563in}{0.736928in}}%
\pgfpathlineto{\pgfqpoint{4.420526in}{0.653358in}}%
\pgfpathlineto{\pgfqpoint{4.423489in}{0.635320in}}%
\pgfpathlineto{\pgfqpoint{4.426453in}{0.672135in}}%
\pgfpathlineto{\pgfqpoint{4.429416in}{0.793791in}}%
\pgfpathlineto{\pgfqpoint{4.432379in}{0.730591in}}%
\pgfpathlineto{\pgfqpoint{4.435342in}{0.795111in}}%
\pgfpathlineto{\pgfqpoint{4.438306in}{0.655808in}}%
\pgfpathlineto{\pgfqpoint{4.441269in}{0.781848in}}%
\pgfpathlineto{\pgfqpoint{4.444232in}{0.800242in}}%
\pgfpathlineto{\pgfqpoint{4.447195in}{0.656059in}}%
\pgfpathlineto{\pgfqpoint{4.450159in}{0.605833in}}%
\pgfpathlineto{\pgfqpoint{4.453122in}{0.599429in}}%
\pgfpathlineto{\pgfqpoint{4.456085in}{0.718720in}}%
\pgfpathlineto{\pgfqpoint{4.459048in}{0.634263in}}%
\pgfpathlineto{\pgfqpoint{4.462012in}{0.609848in}}%
\pgfpathlineto{\pgfqpoint{4.467938in}{0.698746in}}%
\pgfpathlineto{\pgfqpoint{4.470901in}{0.689977in}}%
\pgfpathlineto{\pgfqpoint{4.473865in}{0.668226in}}%
\pgfpathlineto{\pgfqpoint{4.476828in}{1.073568in}}%
\pgfpathlineto{\pgfqpoint{4.479791in}{0.706284in}}%
\pgfpathlineto{\pgfqpoint{4.482754in}{0.648325in}}%
\pgfpathlineto{\pgfqpoint{4.485718in}{0.683692in}}%
\pgfpathlineto{\pgfqpoint{4.488681in}{0.696137in}}%
\pgfpathlineto{\pgfqpoint{4.491644in}{0.676243in}}%
\pgfpathlineto{\pgfqpoint{4.494607in}{0.786664in}}%
\pgfpathlineto{\pgfqpoint{4.503497in}{0.620343in}}%
\pgfpathlineto{\pgfqpoint{4.506460in}{0.630060in}}%
\pgfpathlineto{\pgfqpoint{4.509424in}{0.675743in}}%
\pgfpathlineto{\pgfqpoint{4.512387in}{0.628279in}}%
\pgfpathlineto{\pgfqpoint{4.515350in}{0.605466in}}%
\pgfpathlineto{\pgfqpoint{4.518313in}{0.634181in}}%
\pgfpathlineto{\pgfqpoint{4.521277in}{0.637078in}}%
\pgfpathlineto{\pgfqpoint{4.524240in}{0.791645in}}%
\pgfpathlineto{\pgfqpoint{4.527203in}{0.723966in}}%
\pgfpathlineto{\pgfqpoint{4.530166in}{0.628878in}}%
\pgfpathlineto{\pgfqpoint{4.533130in}{0.694672in}}%
\pgfpathlineto{\pgfqpoint{4.536093in}{0.842878in}}%
\pgfpathlineto{\pgfqpoint{4.539056in}{0.718663in}}%
\pgfpathlineto{\pgfqpoint{4.542019in}{0.699442in}}%
\pgfpathlineto{\pgfqpoint{4.544983in}{0.637107in}}%
\pgfpathlineto{\pgfqpoint{4.547946in}{0.719138in}}%
\pgfpathlineto{\pgfqpoint{4.550909in}{0.727229in}}%
\pgfpathlineto{\pgfqpoint{4.556836in}{0.673319in}}%
\pgfpathlineto{\pgfqpoint{4.562762in}{0.707557in}}%
\pgfpathlineto{\pgfqpoint{4.565726in}{0.668721in}}%
\pgfpathlineto{\pgfqpoint{4.568689in}{0.655534in}}%
\pgfpathlineto{\pgfqpoint{4.574615in}{0.727781in}}%
\pgfpathlineto{\pgfqpoint{4.577579in}{0.700132in}}%
\pgfpathlineto{\pgfqpoint{4.580542in}{0.702250in}}%
\pgfpathlineto{\pgfqpoint{4.583505in}{0.677956in}}%
\pgfpathlineto{\pgfqpoint{4.586468in}{0.601946in}}%
\pgfpathlineto{\pgfqpoint{4.589432in}{0.667840in}}%
\pgfpathlineto{\pgfqpoint{4.592395in}{0.766233in}}%
\pgfpathlineto{\pgfqpoint{4.595358in}{0.650097in}}%
\pgfpathlineto{\pgfqpoint{4.598321in}{0.735020in}}%
\pgfpathlineto{\pgfqpoint{4.601285in}{0.778187in}}%
\pgfpathlineto{\pgfqpoint{4.604248in}{0.661406in}}%
\pgfpathlineto{\pgfqpoint{4.610174in}{0.675794in}}%
\pgfpathlineto{\pgfqpoint{4.613138in}{0.619954in}}%
\pgfpathlineto{\pgfqpoint{4.616101in}{0.652690in}}%
\pgfpathlineto{\pgfqpoint{4.619064in}{0.739991in}}%
\pgfpathlineto{\pgfqpoint{4.622027in}{0.632277in}}%
\pgfpathlineto{\pgfqpoint{4.624991in}{0.643576in}}%
\pgfpathlineto{\pgfqpoint{4.627954in}{0.799495in}}%
\pgfpathlineto{\pgfqpoint{4.630917in}{0.611312in}}%
\pgfpathlineto{\pgfqpoint{4.633880in}{0.606263in}}%
\pgfpathlineto{\pgfqpoint{4.636844in}{0.650216in}}%
\pgfpathlineto{\pgfqpoint{4.639807in}{0.642165in}}%
\pgfpathlineto{\pgfqpoint{4.642770in}{0.826240in}}%
\pgfpathlineto{\pgfqpoint{4.645733in}{0.615277in}}%
\pgfpathlineto{\pgfqpoint{4.648697in}{0.609323in}}%
\pgfpathlineto{\pgfqpoint{4.651660in}{0.671831in}}%
\pgfpathlineto{\pgfqpoint{4.654623in}{0.691950in}}%
\pgfpathlineto{\pgfqpoint{4.657586in}{0.846361in}}%
\pgfpathlineto{\pgfqpoint{4.660550in}{0.733660in}}%
\pgfpathlineto{\pgfqpoint{4.663513in}{0.672962in}}%
\pgfpathlineto{\pgfqpoint{4.666476in}{0.674467in}}%
\pgfpathlineto{\pgfqpoint{4.669439in}{0.728644in}}%
\pgfpathlineto{\pgfqpoint{4.672403in}{0.705775in}}%
\pgfpathlineto{\pgfqpoint{4.675366in}{0.768225in}}%
\pgfpathlineto{\pgfqpoint{4.678329in}{0.773218in}}%
\pgfpathlineto{\pgfqpoint{4.681292in}{0.622343in}}%
\pgfpathlineto{\pgfqpoint{4.684256in}{0.717136in}}%
\pgfpathlineto{\pgfqpoint{4.687219in}{0.678491in}}%
\pgfpathlineto{\pgfqpoint{4.690182in}{0.609192in}}%
\pgfpathlineto{\pgfqpoint{4.693145in}{0.665206in}}%
\pgfpathlineto{\pgfqpoint{4.696109in}{0.672041in}}%
\pgfpathlineto{\pgfqpoint{4.699072in}{0.656942in}}%
\pgfpathlineto{\pgfqpoint{4.702035in}{0.609294in}}%
\pgfpathlineto{\pgfqpoint{4.707962in}{0.650320in}}%
\pgfpathlineto{\pgfqpoint{4.710925in}{0.708671in}}%
\pgfpathlineto{\pgfqpoint{4.713888in}{0.603312in}}%
\pgfpathlineto{\pgfqpoint{4.716851in}{0.681838in}}%
\pgfpathlineto{\pgfqpoint{4.719815in}{0.617585in}}%
\pgfpathlineto{\pgfqpoint{4.722778in}{0.659787in}}%
\pgfpathlineto{\pgfqpoint{4.725741in}{0.791660in}}%
\pgfpathlineto{\pgfqpoint{4.728704in}{0.658440in}}%
\pgfpathlineto{\pgfqpoint{4.731668in}{0.693975in}}%
\pgfpathlineto{\pgfqpoint{4.734631in}{0.705686in}}%
\pgfpathlineto{\pgfqpoint{4.737594in}{0.655912in}}%
\pgfpathlineto{\pgfqpoint{4.743521in}{0.687316in}}%
\pgfpathlineto{\pgfqpoint{4.746484in}{0.721324in}}%
\pgfpathlineto{\pgfqpoint{4.749447in}{0.669553in}}%
\pgfpathlineto{\pgfqpoint{4.752410in}{0.655818in}}%
\pgfpathlineto{\pgfqpoint{4.755374in}{0.737089in}}%
\pgfpathlineto{\pgfqpoint{4.758337in}{0.651023in}}%
\pgfpathlineto{\pgfqpoint{4.761300in}{0.720398in}}%
\pgfpathlineto{\pgfqpoint{4.764263in}{0.655289in}}%
\pgfpathlineto{\pgfqpoint{4.767227in}{0.837070in}}%
\pgfpathlineto{\pgfqpoint{4.770190in}{0.688437in}}%
\pgfpathlineto{\pgfqpoint{4.776116in}{0.800579in}}%
\pgfpathlineto{\pgfqpoint{4.779080in}{0.717809in}}%
\pgfpathlineto{\pgfqpoint{4.782043in}{0.708862in}}%
\pgfpathlineto{\pgfqpoint{4.785006in}{0.770186in}}%
\pgfpathlineto{\pgfqpoint{4.787970in}{0.627589in}}%
\pgfpathlineto{\pgfqpoint{4.790933in}{0.713033in}}%
\pgfpathlineto{\pgfqpoint{4.793896in}{0.692872in}}%
\pgfpathlineto{\pgfqpoint{4.796859in}{0.712060in}}%
\pgfpathlineto{\pgfqpoint{4.802786in}{0.628520in}}%
\pgfpathlineto{\pgfqpoint{4.805749in}{0.629648in}}%
\pgfpathlineto{\pgfqpoint{4.808712in}{0.753874in}}%
\pgfpathlineto{\pgfqpoint{4.811676in}{0.626336in}}%
\pgfpathlineto{\pgfqpoint{4.814639in}{0.862322in}}%
\pgfpathlineto{\pgfqpoint{4.820565in}{0.635207in}}%
\pgfpathlineto{\pgfqpoint{4.823529in}{0.760300in}}%
\pgfpathlineto{\pgfqpoint{4.826492in}{0.697229in}}%
\pgfpathlineto{\pgfqpoint{4.829455in}{0.689319in}}%
\pgfpathlineto{\pgfqpoint{4.835382in}{0.729053in}}%
\pgfpathlineto{\pgfqpoint{4.838345in}{0.774110in}}%
\pgfpathlineto{\pgfqpoint{4.841308in}{0.638429in}}%
\pgfpathlineto{\pgfqpoint{4.847235in}{0.728038in}}%
\pgfpathlineto{\pgfqpoint{4.850198in}{0.649562in}}%
\pgfpathlineto{\pgfqpoint{4.853161in}{0.738973in}}%
\pgfpathlineto{\pgfqpoint{4.856124in}{0.914637in}}%
\pgfpathlineto{\pgfqpoint{4.859088in}{0.762823in}}%
\pgfpathlineto{\pgfqpoint{4.862051in}{0.675594in}}%
\pgfpathlineto{\pgfqpoint{4.865014in}{0.673233in}}%
\pgfpathlineto{\pgfqpoint{4.867977in}{0.720720in}}%
\pgfpathlineto{\pgfqpoint{4.870941in}{0.646401in}}%
\pgfpathlineto{\pgfqpoint{4.873904in}{0.749354in}}%
\pgfpathlineto{\pgfqpoint{4.876867in}{0.646104in}}%
\pgfpathlineto{\pgfqpoint{4.879830in}{0.688948in}}%
\pgfpathlineto{\pgfqpoint{4.882794in}{0.707866in}}%
\pgfpathlineto{\pgfqpoint{4.885757in}{0.753299in}}%
\pgfpathlineto{\pgfqpoint{4.888720in}{0.632070in}}%
\pgfpathlineto{\pgfqpoint{4.891683in}{0.708359in}}%
\pgfpathlineto{\pgfqpoint{4.894647in}{0.725312in}}%
\pgfpathlineto{\pgfqpoint{4.897610in}{0.837368in}}%
\pgfpathlineto{\pgfqpoint{4.900573in}{0.684348in}}%
\pgfpathlineto{\pgfqpoint{4.903536in}{0.702640in}}%
\pgfpathlineto{\pgfqpoint{4.906500in}{0.809165in}}%
\pgfpathlineto{\pgfqpoint{4.909463in}{0.715793in}}%
\pgfpathlineto{\pgfqpoint{4.912426in}{0.675680in}}%
\pgfpathlineto{\pgfqpoint{4.915389in}{0.686370in}}%
\pgfpathlineto{\pgfqpoint{4.918353in}{0.735207in}}%
\pgfpathlineto{\pgfqpoint{4.921316in}{0.620139in}}%
\pgfpathlineto{\pgfqpoint{4.924279in}{0.624791in}}%
\pgfpathlineto{\pgfqpoint{4.930206in}{0.731553in}}%
\pgfpathlineto{\pgfqpoint{4.933169in}{0.713714in}}%
\pgfpathlineto{\pgfqpoint{4.936132in}{0.593901in}}%
\pgfpathlineto{\pgfqpoint{4.939095in}{0.714925in}}%
\pgfpathlineto{\pgfqpoint{4.942059in}{0.764940in}}%
\pgfpathlineto{\pgfqpoint{4.945022in}{0.696921in}}%
\pgfpathlineto{\pgfqpoint{4.947985in}{0.659242in}}%
\pgfpathlineto{\pgfqpoint{4.950948in}{0.604738in}}%
\pgfpathlineto{\pgfqpoint{4.953912in}{0.698209in}}%
\pgfpathlineto{\pgfqpoint{4.956875in}{0.653699in}}%
\pgfpathlineto{\pgfqpoint{4.959838in}{0.788760in}}%
\pgfpathlineto{\pgfqpoint{4.962801in}{0.627195in}}%
\pgfpathlineto{\pgfqpoint{4.965765in}{0.738971in}}%
\pgfpathlineto{\pgfqpoint{4.968728in}{0.710359in}}%
\pgfpathlineto{\pgfqpoint{4.971691in}{0.805994in}}%
\pgfpathlineto{\pgfqpoint{4.974654in}{0.652150in}}%
\pgfpathlineto{\pgfqpoint{4.977618in}{0.694342in}}%
\pgfpathlineto{\pgfqpoint{4.980581in}{0.637356in}}%
\pgfpathlineto{\pgfqpoint{4.986507in}{0.764361in}}%
\pgfpathlineto{\pgfqpoint{4.989471in}{0.689292in}}%
\pgfpathlineto{\pgfqpoint{4.992434in}{0.695939in}}%
\pgfpathlineto{\pgfqpoint{4.995397in}{0.685878in}}%
\pgfpathlineto{\pgfqpoint{4.998360in}{0.605256in}}%
\pgfpathlineto{\pgfqpoint{5.001324in}{0.655339in}}%
\pgfpathlineto{\pgfqpoint{5.004287in}{0.603979in}}%
\pgfpathlineto{\pgfqpoint{5.010214in}{0.697283in}}%
\pgfpathlineto{\pgfqpoint{5.013177in}{0.845403in}}%
\pgfpathlineto{\pgfqpoint{5.019103in}{0.684299in}}%
\pgfpathlineto{\pgfqpoint{5.022067in}{0.627923in}}%
\pgfpathlineto{\pgfqpoint{5.025030in}{0.593308in}}%
\pgfpathlineto{\pgfqpoint{5.027993in}{0.595291in}}%
\pgfpathlineto{\pgfqpoint{5.030956in}{0.708874in}}%
\pgfpathlineto{\pgfqpoint{5.033920in}{0.640651in}}%
\pgfpathlineto{\pgfqpoint{5.036883in}{0.716321in}}%
\pgfpathlineto{\pgfqpoint{5.039846in}{0.681811in}}%
\pgfpathlineto{\pgfqpoint{5.042809in}{0.660604in}}%
\pgfpathlineto{\pgfqpoint{5.045773in}{0.620144in}}%
\pgfpathlineto{\pgfqpoint{5.051699in}{0.654148in}}%
\pgfpathlineto{\pgfqpoint{5.054662in}{0.621682in}}%
\pgfpathlineto{\pgfqpoint{5.057626in}{0.637991in}}%
\pgfpathlineto{\pgfqpoint{5.060589in}{0.679756in}}%
\pgfpathlineto{\pgfqpoint{5.063552in}{0.703849in}}%
\pgfpathlineto{\pgfqpoint{5.066515in}{0.869318in}}%
\pgfpathlineto{\pgfqpoint{5.069479in}{0.799497in}}%
\pgfpathlineto{\pgfqpoint{5.072442in}{0.670031in}}%
\pgfpathlineto{\pgfqpoint{5.075405in}{0.708689in}}%
\pgfpathlineto{\pgfqpoint{5.078368in}{0.705051in}}%
\pgfpathlineto{\pgfqpoint{5.081332in}{0.641270in}}%
\pgfpathlineto{\pgfqpoint{5.084295in}{0.677215in}}%
\pgfpathlineto{\pgfqpoint{5.087258in}{0.633266in}}%
\pgfpathlineto{\pgfqpoint{5.090221in}{0.617052in}}%
\pgfpathlineto{\pgfqpoint{5.093185in}{0.643178in}}%
\pgfpathlineto{\pgfqpoint{5.096148in}{0.624391in}}%
\pgfpathlineto{\pgfqpoint{5.099111in}{0.694604in}}%
\pgfpathlineto{\pgfqpoint{5.102074in}{0.677423in}}%
\pgfpathlineto{\pgfqpoint{5.105038in}{0.678349in}}%
\pgfpathlineto{\pgfqpoint{5.105038in}{0.678349in}}%
\pgfusepath{stroke}%
\end{pgfscope}%
\begin{pgfscope}%
\pgfsetrectcap%
\pgfsetmiterjoin%
\pgfsetlinewidth{0.803000pt}%
\definecolor{currentstroke}{rgb}{0.000000,0.000000,0.000000}%
\pgfsetstrokecolor{currentstroke}%
\pgfsetdash{}{0pt}%
\pgfpathmoveto{\pgfqpoint{0.444137in}{0.320679in}}%
\pgfpathlineto{\pgfqpoint{0.444137in}{4.170679in}}%
\pgfusepath{stroke}%
\end{pgfscope}%
\begin{pgfscope}%
\pgfsetrectcap%
\pgfsetmiterjoin%
\pgfsetlinewidth{0.803000pt}%
\definecolor{currentstroke}{rgb}{0.000000,0.000000,0.000000}%
\pgfsetstrokecolor{currentstroke}%
\pgfsetdash{}{0pt}%
\pgfpathmoveto{\pgfqpoint{5.326985in}{0.320679in}}%
\pgfpathlineto{\pgfqpoint{5.326985in}{4.170679in}}%
\pgfusepath{stroke}%
\end{pgfscope}%
\begin{pgfscope}%
\pgfsetrectcap%
\pgfsetmiterjoin%
\pgfsetlinewidth{0.803000pt}%
\definecolor{currentstroke}{rgb}{0.000000,0.000000,0.000000}%
\pgfsetstrokecolor{currentstroke}%
\pgfsetdash{}{0pt}%
\pgfpathmoveto{\pgfqpoint{0.444137in}{0.320679in}}%
\pgfpathlineto{\pgfqpoint{5.326985in}{0.320679in}}%
\pgfusepath{stroke}%
\end{pgfscope}%
\begin{pgfscope}%
\pgfsetrectcap%
\pgfsetmiterjoin%
\pgfsetlinewidth{0.803000pt}%
\definecolor{currentstroke}{rgb}{0.000000,0.000000,0.000000}%
\pgfsetstrokecolor{currentstroke}%
\pgfsetdash{}{0pt}%
\pgfpathmoveto{\pgfqpoint{0.444137in}{4.170679in}}%
\pgfpathlineto{\pgfqpoint{5.326985in}{4.170679in}}%
\pgfusepath{stroke}%
\end{pgfscope}%
\end{pgfpicture}%
\makeatother%
\endgroup%

    \caption{Loss train}
    \label{fig:loss_train}
\end{figure}

\begin{table}[H]
    \centering
    \begin{tabular}{|l|l|}
        \hline
        Accuracy & 0.5394388839910802\\
        \hline
        RMSE & 1.217445418244049 \\
        \hline
    \end{tabular}
    \caption{Train metrics}
    \label{tab:train_metrics}
\end{table}

\begin{table}[H]
    \centering
    \begin{tabular}{l|l|l|l|l|l|}
         True\backslash^{\textstyle{\textrm{Predicted}}} & 1 & 2 & 3 & 4 & 5\\ \hline
        1 & 5592 &   27 &   51 &   79 &   69 \\ \hline
        2 & 2619 &   64 &   92 &  142 &   82 \\ \hline
        3 & 1180 &   90 &  239 &  791 &  445 \\ \hline
        4 & 363 &   54 &  162 & 1956 & 1325 \\ \hline
        5 & 229 &   16 &   63 & 1002 & 2551 \\ \hline
    \end{tabular}
    \caption{Train confusion matrix}
    \label{tab:val_confusion_matrix}
\end{table}


\begin{table}[H]
    \centering
    \begin{tabular}{|l|l|}
        \hline
        Accuracy & 0.5289359054138145 \\
        \hline
        RMSE & 1.3432897739058287 \\
        \hline
    \end{tabular}
    \caption{Val metrics}
    \label{tab:val_metrics}
\end{table}


\begin{table}[H]
    \centering
    \begin{tabular}{l|l|l|l|l|l|}
         True\backslash^{\textstyle{\textrm{Predicted}}} & 1 & 2 & 3 & 4 & 5\\ \hline
         1 & 1384 & 11 & 17 & 24 & 17 \\ \hline
         2 & 606 & 16 & 22 & 50 & 23 \\ \hline
         3 & 276 & 13 & 49 & 188 & 111 \\ \hline
         4 & 121 & 11 & 54 & 472 & 367 \\ \hline
         5 & 75 &  4 & 18 & 263 & 629 \\ \hline
    \end{tabular}
    \caption{Val confusion matrix}
    \label{tab:val_confusion_matrix}
\end{table}


\end{document}
